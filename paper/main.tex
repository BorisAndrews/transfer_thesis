\documentclass[12pt, a4paper]{report}

\documentclass[12pt, a4paper]{report}

\documentclass[12pt, a4paper]{report}

\documentclass[12pt, a4paper]{report}

\input{template/main.tex}

\title{\BA{Title in Progress...}}
\author{Boris Andrews}
\affil{Mathematical Institute, University of Oxford}
\date{\today}


\begin{document}
    \pagenumbering{gobble}
    \maketitle
    
    
    \begin{abstract}
        Magnetic confinement reactors---in particular tokamaks---offer one of the most promising options for achieving practical nuclear fusion, with the potential to provide virtually limitless, clean energy. The theoretical and numerical modeling of tokamak plasmas is simultaneously an essential component of effective reactor design, and a great research barrier. Tokamak operational conditions exhibit comparatively low Knudsen numbers. Kinetic effects, including kinetic waves and instabilities, Landau damping, bump-on-tail instabilities and more, are therefore highly influential in tokamak plasma dynamics. Purely fluid models are inherently incapable of capturing these effects, whereas the high dimensionality in purely kinetic models render them practically intractable for most relevant purposes.

        We consider a $\delta\!f$ decomposition model, with a macroscopic fluid background and microscopic kinetic correction, both fully coupled to each other. A similar manner of discretization is proposed to that used in the recent \texttt{STRUPHY} code \cite{Holderied_Possanner_Wang_2021, Holderied_2022, Li_et_al_2023} with a finite-element model for the background and a pseudo-particle/PiC model for the correction.

        The fluid background satisfies the full, non-linear, resistive, compressible, Hall MHD equations. \cite{Laakmann_Hu_Farrell_2022} introduces finite-element(-in-space) implicit timesteppers for the incompressible analogue to this system with structure-preserving (SP) properties in the ideal case, alongside parameter-robust preconditioners. We show that these timesteppers can derive from a finite-element-in-time (FET) (and finite-element-in-space) interpretation. The benefits of this reformulation are discussed, including the derivation of timesteppers that are higher order in time, and the quantifiable dissipative SP properties in the non-ideal, resistive case.
        
        We discuss possible options for extending this FET approach to timesteppers for the compressible case.

        The kinetic corrections satisfy linearized Boltzmann equations. Using a Lénard--Bernstein collision operator, these take Fokker--Planck-like forms \cite{Fokker_1914, Planck_1917} wherein pseudo-particles in the numerical model obey the neoclassical transport equations, with particle-independent Brownian drift terms. This offers a rigorous methodology for incorporating collisions into the particle transport model, without coupling the equations of motions for each particle.
        
        Works by Chen, Chacón et al. \cite{Chen_Chacón_Barnes_2011, Chacón_Chen_Barnes_2013, Chen_Chacón_2014, Chen_Chacón_2015} have developed structure-preserving particle pushers for neoclassical transport in the Vlasov equations, derived from Crank--Nicolson integrators. We show these too can can derive from a FET interpretation, similarly offering potential extensions to higher-order-in-time particle pushers. The FET formulation is used also to consider how the stochastic drift terms can be incorporated into the pushers. Stochastic gyrokinetic expansions are also discussed.

        Different options for the numerical implementation of these schemes are considered.

        Due to the efficacy of FET in the development of SP timesteppers for both the fluid and kinetic component, we hope this approach will prove effective in the future for developing SP timesteppers for the full hybrid model. We hope this will give us the opportunity to incorporate previously inaccessible kinetic effects into the highly effective, modern, finite-element MHD models.
    \end{abstract}
    
    
    \newpage
    \tableofcontents
    
    
    \newpage
    \pagenumbering{arabic}
    %\linenumbers\renewcommand\thelinenumber{\color{black!50}\arabic{linenumber}}
            \input{0 - introduction/main.tex}
        \part{Research}
            \input{1 - low-noise PiC models/main.tex}
            \input{2 - kinetic component/main.tex}
            \input{3 - fluid component/main.tex}
            \input{4 - numerical implementation/main.tex}
        \part{Project Overview}
            \input{5 - research plan/main.tex}
            \input{6 - summary/main.tex}
    
    
    %\section{}
    \newpage
    \pagenumbering{gobble}
        \printbibliography


    \newpage
    \pagenumbering{roman}
    \appendix
        \part{Appendices}
            \input{8 - Hilbert complexes/main.tex}
            \input{9 - weak conservation proofs/main.tex}
\end{document}


\title{\BA{Title in Progress...}}
\author{Boris Andrews}
\affil{Mathematical Institute, University of Oxford}
\date{\today}


\begin{document}
    \pagenumbering{gobble}
    \maketitle
    
    
    \begin{abstract}
        Magnetic confinement reactors---in particular tokamaks---offer one of the most promising options for achieving practical nuclear fusion, with the potential to provide virtually limitless, clean energy. The theoretical and numerical modeling of tokamak plasmas is simultaneously an essential component of effective reactor design, and a great research barrier. Tokamak operational conditions exhibit comparatively low Knudsen numbers. Kinetic effects, including kinetic waves and instabilities, Landau damping, bump-on-tail instabilities and more, are therefore highly influential in tokamak plasma dynamics. Purely fluid models are inherently incapable of capturing these effects, whereas the high dimensionality in purely kinetic models render them practically intractable for most relevant purposes.

        We consider a $\delta\!f$ decomposition model, with a macroscopic fluid background and microscopic kinetic correction, both fully coupled to each other. A similar manner of discretization is proposed to that used in the recent \texttt{STRUPHY} code \cite{Holderied_Possanner_Wang_2021, Holderied_2022, Li_et_al_2023} with a finite-element model for the background and a pseudo-particle/PiC model for the correction.

        The fluid background satisfies the full, non-linear, resistive, compressible, Hall MHD equations. \cite{Laakmann_Hu_Farrell_2022} introduces finite-element(-in-space) implicit timesteppers for the incompressible analogue to this system with structure-preserving (SP) properties in the ideal case, alongside parameter-robust preconditioners. We show that these timesteppers can derive from a finite-element-in-time (FET) (and finite-element-in-space) interpretation. The benefits of this reformulation are discussed, including the derivation of timesteppers that are higher order in time, and the quantifiable dissipative SP properties in the non-ideal, resistive case.
        
        We discuss possible options for extending this FET approach to timesteppers for the compressible case.

        The kinetic corrections satisfy linearized Boltzmann equations. Using a Lénard--Bernstein collision operator, these take Fokker--Planck-like forms \cite{Fokker_1914, Planck_1917} wherein pseudo-particles in the numerical model obey the neoclassical transport equations, with particle-independent Brownian drift terms. This offers a rigorous methodology for incorporating collisions into the particle transport model, without coupling the equations of motions for each particle.
        
        Works by Chen, Chacón et al. \cite{Chen_Chacón_Barnes_2011, Chacón_Chen_Barnes_2013, Chen_Chacón_2014, Chen_Chacón_2015} have developed structure-preserving particle pushers for neoclassical transport in the Vlasov equations, derived from Crank--Nicolson integrators. We show these too can can derive from a FET interpretation, similarly offering potential extensions to higher-order-in-time particle pushers. The FET formulation is used also to consider how the stochastic drift terms can be incorporated into the pushers. Stochastic gyrokinetic expansions are also discussed.

        Different options for the numerical implementation of these schemes are considered.

        Due to the efficacy of FET in the development of SP timesteppers for both the fluid and kinetic component, we hope this approach will prove effective in the future for developing SP timesteppers for the full hybrid model. We hope this will give us the opportunity to incorporate previously inaccessible kinetic effects into the highly effective, modern, finite-element MHD models.
    \end{abstract}
    
    
    \newpage
    \tableofcontents
    
    
    \newpage
    \pagenumbering{arabic}
    %\linenumbers\renewcommand\thelinenumber{\color{black!50}\arabic{linenumber}}
            \documentclass[12pt, a4paper]{report}

\input{template/main.tex}

\title{\BA{Title in Progress...}}
\author{Boris Andrews}
\affil{Mathematical Institute, University of Oxford}
\date{\today}


\begin{document}
    \pagenumbering{gobble}
    \maketitle
    
    
    \begin{abstract}
        Magnetic confinement reactors---in particular tokamaks---offer one of the most promising options for achieving practical nuclear fusion, with the potential to provide virtually limitless, clean energy. The theoretical and numerical modeling of tokamak plasmas is simultaneously an essential component of effective reactor design, and a great research barrier. Tokamak operational conditions exhibit comparatively low Knudsen numbers. Kinetic effects, including kinetic waves and instabilities, Landau damping, bump-on-tail instabilities and more, are therefore highly influential in tokamak plasma dynamics. Purely fluid models are inherently incapable of capturing these effects, whereas the high dimensionality in purely kinetic models render them practically intractable for most relevant purposes.

        We consider a $\delta\!f$ decomposition model, with a macroscopic fluid background and microscopic kinetic correction, both fully coupled to each other. A similar manner of discretization is proposed to that used in the recent \texttt{STRUPHY} code \cite{Holderied_Possanner_Wang_2021, Holderied_2022, Li_et_al_2023} with a finite-element model for the background and a pseudo-particle/PiC model for the correction.

        The fluid background satisfies the full, non-linear, resistive, compressible, Hall MHD equations. \cite{Laakmann_Hu_Farrell_2022} introduces finite-element(-in-space) implicit timesteppers for the incompressible analogue to this system with structure-preserving (SP) properties in the ideal case, alongside parameter-robust preconditioners. We show that these timesteppers can derive from a finite-element-in-time (FET) (and finite-element-in-space) interpretation. The benefits of this reformulation are discussed, including the derivation of timesteppers that are higher order in time, and the quantifiable dissipative SP properties in the non-ideal, resistive case.
        
        We discuss possible options for extending this FET approach to timesteppers for the compressible case.

        The kinetic corrections satisfy linearized Boltzmann equations. Using a Lénard--Bernstein collision operator, these take Fokker--Planck-like forms \cite{Fokker_1914, Planck_1917} wherein pseudo-particles in the numerical model obey the neoclassical transport equations, with particle-independent Brownian drift terms. This offers a rigorous methodology for incorporating collisions into the particle transport model, without coupling the equations of motions for each particle.
        
        Works by Chen, Chacón et al. \cite{Chen_Chacón_Barnes_2011, Chacón_Chen_Barnes_2013, Chen_Chacón_2014, Chen_Chacón_2015} have developed structure-preserving particle pushers for neoclassical transport in the Vlasov equations, derived from Crank--Nicolson integrators. We show these too can can derive from a FET interpretation, similarly offering potential extensions to higher-order-in-time particle pushers. The FET formulation is used also to consider how the stochastic drift terms can be incorporated into the pushers. Stochastic gyrokinetic expansions are also discussed.

        Different options for the numerical implementation of these schemes are considered.

        Due to the efficacy of FET in the development of SP timesteppers for both the fluid and kinetic component, we hope this approach will prove effective in the future for developing SP timesteppers for the full hybrid model. We hope this will give us the opportunity to incorporate previously inaccessible kinetic effects into the highly effective, modern, finite-element MHD models.
    \end{abstract}
    
    
    \newpage
    \tableofcontents
    
    
    \newpage
    \pagenumbering{arabic}
    %\linenumbers\renewcommand\thelinenumber{\color{black!50}\arabic{linenumber}}
            \input{0 - introduction/main.tex}
        \part{Research}
            \input{1 - low-noise PiC models/main.tex}
            \input{2 - kinetic component/main.tex}
            \input{3 - fluid component/main.tex}
            \input{4 - numerical implementation/main.tex}
        \part{Project Overview}
            \input{5 - research plan/main.tex}
            \input{6 - summary/main.tex}
    
    
    %\section{}
    \newpage
    \pagenumbering{gobble}
        \printbibliography


    \newpage
    \pagenumbering{roman}
    \appendix
        \part{Appendices}
            \input{8 - Hilbert complexes/main.tex}
            \input{9 - weak conservation proofs/main.tex}
\end{document}

        \part{Research}
            \documentclass[12pt, a4paper]{report}

\input{template/main.tex}

\title{\BA{Title in Progress...}}
\author{Boris Andrews}
\affil{Mathematical Institute, University of Oxford}
\date{\today}


\begin{document}
    \pagenumbering{gobble}
    \maketitle
    
    
    \begin{abstract}
        Magnetic confinement reactors---in particular tokamaks---offer one of the most promising options for achieving practical nuclear fusion, with the potential to provide virtually limitless, clean energy. The theoretical and numerical modeling of tokamak plasmas is simultaneously an essential component of effective reactor design, and a great research barrier. Tokamak operational conditions exhibit comparatively low Knudsen numbers. Kinetic effects, including kinetic waves and instabilities, Landau damping, bump-on-tail instabilities and more, are therefore highly influential in tokamak plasma dynamics. Purely fluid models are inherently incapable of capturing these effects, whereas the high dimensionality in purely kinetic models render them practically intractable for most relevant purposes.

        We consider a $\delta\!f$ decomposition model, with a macroscopic fluid background and microscopic kinetic correction, both fully coupled to each other. A similar manner of discretization is proposed to that used in the recent \texttt{STRUPHY} code \cite{Holderied_Possanner_Wang_2021, Holderied_2022, Li_et_al_2023} with a finite-element model for the background and a pseudo-particle/PiC model for the correction.

        The fluid background satisfies the full, non-linear, resistive, compressible, Hall MHD equations. \cite{Laakmann_Hu_Farrell_2022} introduces finite-element(-in-space) implicit timesteppers for the incompressible analogue to this system with structure-preserving (SP) properties in the ideal case, alongside parameter-robust preconditioners. We show that these timesteppers can derive from a finite-element-in-time (FET) (and finite-element-in-space) interpretation. The benefits of this reformulation are discussed, including the derivation of timesteppers that are higher order in time, and the quantifiable dissipative SP properties in the non-ideal, resistive case.
        
        We discuss possible options for extending this FET approach to timesteppers for the compressible case.

        The kinetic corrections satisfy linearized Boltzmann equations. Using a Lénard--Bernstein collision operator, these take Fokker--Planck-like forms \cite{Fokker_1914, Planck_1917} wherein pseudo-particles in the numerical model obey the neoclassical transport equations, with particle-independent Brownian drift terms. This offers a rigorous methodology for incorporating collisions into the particle transport model, without coupling the equations of motions for each particle.
        
        Works by Chen, Chacón et al. \cite{Chen_Chacón_Barnes_2011, Chacón_Chen_Barnes_2013, Chen_Chacón_2014, Chen_Chacón_2015} have developed structure-preserving particle pushers for neoclassical transport in the Vlasov equations, derived from Crank--Nicolson integrators. We show these too can can derive from a FET interpretation, similarly offering potential extensions to higher-order-in-time particle pushers. The FET formulation is used also to consider how the stochastic drift terms can be incorporated into the pushers. Stochastic gyrokinetic expansions are also discussed.

        Different options for the numerical implementation of these schemes are considered.

        Due to the efficacy of FET in the development of SP timesteppers for both the fluid and kinetic component, we hope this approach will prove effective in the future for developing SP timesteppers for the full hybrid model. We hope this will give us the opportunity to incorporate previously inaccessible kinetic effects into the highly effective, modern, finite-element MHD models.
    \end{abstract}
    
    
    \newpage
    \tableofcontents
    
    
    \newpage
    \pagenumbering{arabic}
    %\linenumbers\renewcommand\thelinenumber{\color{black!50}\arabic{linenumber}}
            \input{0 - introduction/main.tex}
        \part{Research}
            \input{1 - low-noise PiC models/main.tex}
            \input{2 - kinetic component/main.tex}
            \input{3 - fluid component/main.tex}
            \input{4 - numerical implementation/main.tex}
        \part{Project Overview}
            \input{5 - research plan/main.tex}
            \input{6 - summary/main.tex}
    
    
    %\section{}
    \newpage
    \pagenumbering{gobble}
        \printbibliography


    \newpage
    \pagenumbering{roman}
    \appendix
        \part{Appendices}
            \input{8 - Hilbert complexes/main.tex}
            \input{9 - weak conservation proofs/main.tex}
\end{document}

            \documentclass[12pt, a4paper]{report}

\input{template/main.tex}

\title{\BA{Title in Progress...}}
\author{Boris Andrews}
\affil{Mathematical Institute, University of Oxford}
\date{\today}


\begin{document}
    \pagenumbering{gobble}
    \maketitle
    
    
    \begin{abstract}
        Magnetic confinement reactors---in particular tokamaks---offer one of the most promising options for achieving practical nuclear fusion, with the potential to provide virtually limitless, clean energy. The theoretical and numerical modeling of tokamak plasmas is simultaneously an essential component of effective reactor design, and a great research barrier. Tokamak operational conditions exhibit comparatively low Knudsen numbers. Kinetic effects, including kinetic waves and instabilities, Landau damping, bump-on-tail instabilities and more, are therefore highly influential in tokamak plasma dynamics. Purely fluid models are inherently incapable of capturing these effects, whereas the high dimensionality in purely kinetic models render them practically intractable for most relevant purposes.

        We consider a $\delta\!f$ decomposition model, with a macroscopic fluid background and microscopic kinetic correction, both fully coupled to each other. A similar manner of discretization is proposed to that used in the recent \texttt{STRUPHY} code \cite{Holderied_Possanner_Wang_2021, Holderied_2022, Li_et_al_2023} with a finite-element model for the background and a pseudo-particle/PiC model for the correction.

        The fluid background satisfies the full, non-linear, resistive, compressible, Hall MHD equations. \cite{Laakmann_Hu_Farrell_2022} introduces finite-element(-in-space) implicit timesteppers for the incompressible analogue to this system with structure-preserving (SP) properties in the ideal case, alongside parameter-robust preconditioners. We show that these timesteppers can derive from a finite-element-in-time (FET) (and finite-element-in-space) interpretation. The benefits of this reformulation are discussed, including the derivation of timesteppers that are higher order in time, and the quantifiable dissipative SP properties in the non-ideal, resistive case.
        
        We discuss possible options for extending this FET approach to timesteppers for the compressible case.

        The kinetic corrections satisfy linearized Boltzmann equations. Using a Lénard--Bernstein collision operator, these take Fokker--Planck-like forms \cite{Fokker_1914, Planck_1917} wherein pseudo-particles in the numerical model obey the neoclassical transport equations, with particle-independent Brownian drift terms. This offers a rigorous methodology for incorporating collisions into the particle transport model, without coupling the equations of motions for each particle.
        
        Works by Chen, Chacón et al. \cite{Chen_Chacón_Barnes_2011, Chacón_Chen_Barnes_2013, Chen_Chacón_2014, Chen_Chacón_2015} have developed structure-preserving particle pushers for neoclassical transport in the Vlasov equations, derived from Crank--Nicolson integrators. We show these too can can derive from a FET interpretation, similarly offering potential extensions to higher-order-in-time particle pushers. The FET formulation is used also to consider how the stochastic drift terms can be incorporated into the pushers. Stochastic gyrokinetic expansions are also discussed.

        Different options for the numerical implementation of these schemes are considered.

        Due to the efficacy of FET in the development of SP timesteppers for both the fluid and kinetic component, we hope this approach will prove effective in the future for developing SP timesteppers for the full hybrid model. We hope this will give us the opportunity to incorporate previously inaccessible kinetic effects into the highly effective, modern, finite-element MHD models.
    \end{abstract}
    
    
    \newpage
    \tableofcontents
    
    
    \newpage
    \pagenumbering{arabic}
    %\linenumbers\renewcommand\thelinenumber{\color{black!50}\arabic{linenumber}}
            \input{0 - introduction/main.tex}
        \part{Research}
            \input{1 - low-noise PiC models/main.tex}
            \input{2 - kinetic component/main.tex}
            \input{3 - fluid component/main.tex}
            \input{4 - numerical implementation/main.tex}
        \part{Project Overview}
            \input{5 - research plan/main.tex}
            \input{6 - summary/main.tex}
    
    
    %\section{}
    \newpage
    \pagenumbering{gobble}
        \printbibliography


    \newpage
    \pagenumbering{roman}
    \appendix
        \part{Appendices}
            \input{8 - Hilbert complexes/main.tex}
            \input{9 - weak conservation proofs/main.tex}
\end{document}

            \documentclass[12pt, a4paper]{report}

\input{template/main.tex}

\title{\BA{Title in Progress...}}
\author{Boris Andrews}
\affil{Mathematical Institute, University of Oxford}
\date{\today}


\begin{document}
    \pagenumbering{gobble}
    \maketitle
    
    
    \begin{abstract}
        Magnetic confinement reactors---in particular tokamaks---offer one of the most promising options for achieving practical nuclear fusion, with the potential to provide virtually limitless, clean energy. The theoretical and numerical modeling of tokamak plasmas is simultaneously an essential component of effective reactor design, and a great research barrier. Tokamak operational conditions exhibit comparatively low Knudsen numbers. Kinetic effects, including kinetic waves and instabilities, Landau damping, bump-on-tail instabilities and more, are therefore highly influential in tokamak plasma dynamics. Purely fluid models are inherently incapable of capturing these effects, whereas the high dimensionality in purely kinetic models render them practically intractable for most relevant purposes.

        We consider a $\delta\!f$ decomposition model, with a macroscopic fluid background and microscopic kinetic correction, both fully coupled to each other. A similar manner of discretization is proposed to that used in the recent \texttt{STRUPHY} code \cite{Holderied_Possanner_Wang_2021, Holderied_2022, Li_et_al_2023} with a finite-element model for the background and a pseudo-particle/PiC model for the correction.

        The fluid background satisfies the full, non-linear, resistive, compressible, Hall MHD equations. \cite{Laakmann_Hu_Farrell_2022} introduces finite-element(-in-space) implicit timesteppers for the incompressible analogue to this system with structure-preserving (SP) properties in the ideal case, alongside parameter-robust preconditioners. We show that these timesteppers can derive from a finite-element-in-time (FET) (and finite-element-in-space) interpretation. The benefits of this reformulation are discussed, including the derivation of timesteppers that are higher order in time, and the quantifiable dissipative SP properties in the non-ideal, resistive case.
        
        We discuss possible options for extending this FET approach to timesteppers for the compressible case.

        The kinetic corrections satisfy linearized Boltzmann equations. Using a Lénard--Bernstein collision operator, these take Fokker--Planck-like forms \cite{Fokker_1914, Planck_1917} wherein pseudo-particles in the numerical model obey the neoclassical transport equations, with particle-independent Brownian drift terms. This offers a rigorous methodology for incorporating collisions into the particle transport model, without coupling the equations of motions for each particle.
        
        Works by Chen, Chacón et al. \cite{Chen_Chacón_Barnes_2011, Chacón_Chen_Barnes_2013, Chen_Chacón_2014, Chen_Chacón_2015} have developed structure-preserving particle pushers for neoclassical transport in the Vlasov equations, derived from Crank--Nicolson integrators. We show these too can can derive from a FET interpretation, similarly offering potential extensions to higher-order-in-time particle pushers. The FET formulation is used also to consider how the stochastic drift terms can be incorporated into the pushers. Stochastic gyrokinetic expansions are also discussed.

        Different options for the numerical implementation of these schemes are considered.

        Due to the efficacy of FET in the development of SP timesteppers for both the fluid and kinetic component, we hope this approach will prove effective in the future for developing SP timesteppers for the full hybrid model. We hope this will give us the opportunity to incorporate previously inaccessible kinetic effects into the highly effective, modern, finite-element MHD models.
    \end{abstract}
    
    
    \newpage
    \tableofcontents
    
    
    \newpage
    \pagenumbering{arabic}
    %\linenumbers\renewcommand\thelinenumber{\color{black!50}\arabic{linenumber}}
            \input{0 - introduction/main.tex}
        \part{Research}
            \input{1 - low-noise PiC models/main.tex}
            \input{2 - kinetic component/main.tex}
            \input{3 - fluid component/main.tex}
            \input{4 - numerical implementation/main.tex}
        \part{Project Overview}
            \input{5 - research plan/main.tex}
            \input{6 - summary/main.tex}
    
    
    %\section{}
    \newpage
    \pagenumbering{gobble}
        \printbibliography


    \newpage
    \pagenumbering{roman}
    \appendix
        \part{Appendices}
            \input{8 - Hilbert complexes/main.tex}
            \input{9 - weak conservation proofs/main.tex}
\end{document}

            \documentclass[12pt, a4paper]{report}

\input{template/main.tex}

\title{\BA{Title in Progress...}}
\author{Boris Andrews}
\affil{Mathematical Institute, University of Oxford}
\date{\today}


\begin{document}
    \pagenumbering{gobble}
    \maketitle
    
    
    \begin{abstract}
        Magnetic confinement reactors---in particular tokamaks---offer one of the most promising options for achieving practical nuclear fusion, with the potential to provide virtually limitless, clean energy. The theoretical and numerical modeling of tokamak plasmas is simultaneously an essential component of effective reactor design, and a great research barrier. Tokamak operational conditions exhibit comparatively low Knudsen numbers. Kinetic effects, including kinetic waves and instabilities, Landau damping, bump-on-tail instabilities and more, are therefore highly influential in tokamak plasma dynamics. Purely fluid models are inherently incapable of capturing these effects, whereas the high dimensionality in purely kinetic models render them practically intractable for most relevant purposes.

        We consider a $\delta\!f$ decomposition model, with a macroscopic fluid background and microscopic kinetic correction, both fully coupled to each other. A similar manner of discretization is proposed to that used in the recent \texttt{STRUPHY} code \cite{Holderied_Possanner_Wang_2021, Holderied_2022, Li_et_al_2023} with a finite-element model for the background and a pseudo-particle/PiC model for the correction.

        The fluid background satisfies the full, non-linear, resistive, compressible, Hall MHD equations. \cite{Laakmann_Hu_Farrell_2022} introduces finite-element(-in-space) implicit timesteppers for the incompressible analogue to this system with structure-preserving (SP) properties in the ideal case, alongside parameter-robust preconditioners. We show that these timesteppers can derive from a finite-element-in-time (FET) (and finite-element-in-space) interpretation. The benefits of this reformulation are discussed, including the derivation of timesteppers that are higher order in time, and the quantifiable dissipative SP properties in the non-ideal, resistive case.
        
        We discuss possible options for extending this FET approach to timesteppers for the compressible case.

        The kinetic corrections satisfy linearized Boltzmann equations. Using a Lénard--Bernstein collision operator, these take Fokker--Planck-like forms \cite{Fokker_1914, Planck_1917} wherein pseudo-particles in the numerical model obey the neoclassical transport equations, with particle-independent Brownian drift terms. This offers a rigorous methodology for incorporating collisions into the particle transport model, without coupling the equations of motions for each particle.
        
        Works by Chen, Chacón et al. \cite{Chen_Chacón_Barnes_2011, Chacón_Chen_Barnes_2013, Chen_Chacón_2014, Chen_Chacón_2015} have developed structure-preserving particle pushers for neoclassical transport in the Vlasov equations, derived from Crank--Nicolson integrators. We show these too can can derive from a FET interpretation, similarly offering potential extensions to higher-order-in-time particle pushers. The FET formulation is used also to consider how the stochastic drift terms can be incorporated into the pushers. Stochastic gyrokinetic expansions are also discussed.

        Different options for the numerical implementation of these schemes are considered.

        Due to the efficacy of FET in the development of SP timesteppers for both the fluid and kinetic component, we hope this approach will prove effective in the future for developing SP timesteppers for the full hybrid model. We hope this will give us the opportunity to incorporate previously inaccessible kinetic effects into the highly effective, modern, finite-element MHD models.
    \end{abstract}
    
    
    \newpage
    \tableofcontents
    
    
    \newpage
    \pagenumbering{arabic}
    %\linenumbers\renewcommand\thelinenumber{\color{black!50}\arabic{linenumber}}
            \input{0 - introduction/main.tex}
        \part{Research}
            \input{1 - low-noise PiC models/main.tex}
            \input{2 - kinetic component/main.tex}
            \input{3 - fluid component/main.tex}
            \input{4 - numerical implementation/main.tex}
        \part{Project Overview}
            \input{5 - research plan/main.tex}
            \input{6 - summary/main.tex}
    
    
    %\section{}
    \newpage
    \pagenumbering{gobble}
        \printbibliography


    \newpage
    \pagenumbering{roman}
    \appendix
        \part{Appendices}
            \input{8 - Hilbert complexes/main.tex}
            \input{9 - weak conservation proofs/main.tex}
\end{document}

        \part{Project Overview}
            \documentclass[12pt, a4paper]{report}

\input{template/main.tex}

\title{\BA{Title in Progress...}}
\author{Boris Andrews}
\affil{Mathematical Institute, University of Oxford}
\date{\today}


\begin{document}
    \pagenumbering{gobble}
    \maketitle
    
    
    \begin{abstract}
        Magnetic confinement reactors---in particular tokamaks---offer one of the most promising options for achieving practical nuclear fusion, with the potential to provide virtually limitless, clean energy. The theoretical and numerical modeling of tokamak plasmas is simultaneously an essential component of effective reactor design, and a great research barrier. Tokamak operational conditions exhibit comparatively low Knudsen numbers. Kinetic effects, including kinetic waves and instabilities, Landau damping, bump-on-tail instabilities and more, are therefore highly influential in tokamak plasma dynamics. Purely fluid models are inherently incapable of capturing these effects, whereas the high dimensionality in purely kinetic models render them practically intractable for most relevant purposes.

        We consider a $\delta\!f$ decomposition model, with a macroscopic fluid background and microscopic kinetic correction, both fully coupled to each other. A similar manner of discretization is proposed to that used in the recent \texttt{STRUPHY} code \cite{Holderied_Possanner_Wang_2021, Holderied_2022, Li_et_al_2023} with a finite-element model for the background and a pseudo-particle/PiC model for the correction.

        The fluid background satisfies the full, non-linear, resistive, compressible, Hall MHD equations. \cite{Laakmann_Hu_Farrell_2022} introduces finite-element(-in-space) implicit timesteppers for the incompressible analogue to this system with structure-preserving (SP) properties in the ideal case, alongside parameter-robust preconditioners. We show that these timesteppers can derive from a finite-element-in-time (FET) (and finite-element-in-space) interpretation. The benefits of this reformulation are discussed, including the derivation of timesteppers that are higher order in time, and the quantifiable dissipative SP properties in the non-ideal, resistive case.
        
        We discuss possible options for extending this FET approach to timesteppers for the compressible case.

        The kinetic corrections satisfy linearized Boltzmann equations. Using a Lénard--Bernstein collision operator, these take Fokker--Planck-like forms \cite{Fokker_1914, Planck_1917} wherein pseudo-particles in the numerical model obey the neoclassical transport equations, with particle-independent Brownian drift terms. This offers a rigorous methodology for incorporating collisions into the particle transport model, without coupling the equations of motions for each particle.
        
        Works by Chen, Chacón et al. \cite{Chen_Chacón_Barnes_2011, Chacón_Chen_Barnes_2013, Chen_Chacón_2014, Chen_Chacón_2015} have developed structure-preserving particle pushers for neoclassical transport in the Vlasov equations, derived from Crank--Nicolson integrators. We show these too can can derive from a FET interpretation, similarly offering potential extensions to higher-order-in-time particle pushers. The FET formulation is used also to consider how the stochastic drift terms can be incorporated into the pushers. Stochastic gyrokinetic expansions are also discussed.

        Different options for the numerical implementation of these schemes are considered.

        Due to the efficacy of FET in the development of SP timesteppers for both the fluid and kinetic component, we hope this approach will prove effective in the future for developing SP timesteppers for the full hybrid model. We hope this will give us the opportunity to incorporate previously inaccessible kinetic effects into the highly effective, modern, finite-element MHD models.
    \end{abstract}
    
    
    \newpage
    \tableofcontents
    
    
    \newpage
    \pagenumbering{arabic}
    %\linenumbers\renewcommand\thelinenumber{\color{black!50}\arabic{linenumber}}
            \input{0 - introduction/main.tex}
        \part{Research}
            \input{1 - low-noise PiC models/main.tex}
            \input{2 - kinetic component/main.tex}
            \input{3 - fluid component/main.tex}
            \input{4 - numerical implementation/main.tex}
        \part{Project Overview}
            \input{5 - research plan/main.tex}
            \input{6 - summary/main.tex}
    
    
    %\section{}
    \newpage
    \pagenumbering{gobble}
        \printbibliography


    \newpage
    \pagenumbering{roman}
    \appendix
        \part{Appendices}
            \input{8 - Hilbert complexes/main.tex}
            \input{9 - weak conservation proofs/main.tex}
\end{document}

            \documentclass[12pt, a4paper]{report}

\input{template/main.tex}

\title{\BA{Title in Progress...}}
\author{Boris Andrews}
\affil{Mathematical Institute, University of Oxford}
\date{\today}


\begin{document}
    \pagenumbering{gobble}
    \maketitle
    
    
    \begin{abstract}
        Magnetic confinement reactors---in particular tokamaks---offer one of the most promising options for achieving practical nuclear fusion, with the potential to provide virtually limitless, clean energy. The theoretical and numerical modeling of tokamak plasmas is simultaneously an essential component of effective reactor design, and a great research barrier. Tokamak operational conditions exhibit comparatively low Knudsen numbers. Kinetic effects, including kinetic waves and instabilities, Landau damping, bump-on-tail instabilities and more, are therefore highly influential in tokamak plasma dynamics. Purely fluid models are inherently incapable of capturing these effects, whereas the high dimensionality in purely kinetic models render them practically intractable for most relevant purposes.

        We consider a $\delta\!f$ decomposition model, with a macroscopic fluid background and microscopic kinetic correction, both fully coupled to each other. A similar manner of discretization is proposed to that used in the recent \texttt{STRUPHY} code \cite{Holderied_Possanner_Wang_2021, Holderied_2022, Li_et_al_2023} with a finite-element model for the background and a pseudo-particle/PiC model for the correction.

        The fluid background satisfies the full, non-linear, resistive, compressible, Hall MHD equations. \cite{Laakmann_Hu_Farrell_2022} introduces finite-element(-in-space) implicit timesteppers for the incompressible analogue to this system with structure-preserving (SP) properties in the ideal case, alongside parameter-robust preconditioners. We show that these timesteppers can derive from a finite-element-in-time (FET) (and finite-element-in-space) interpretation. The benefits of this reformulation are discussed, including the derivation of timesteppers that are higher order in time, and the quantifiable dissipative SP properties in the non-ideal, resistive case.
        
        We discuss possible options for extending this FET approach to timesteppers for the compressible case.

        The kinetic corrections satisfy linearized Boltzmann equations. Using a Lénard--Bernstein collision operator, these take Fokker--Planck-like forms \cite{Fokker_1914, Planck_1917} wherein pseudo-particles in the numerical model obey the neoclassical transport equations, with particle-independent Brownian drift terms. This offers a rigorous methodology for incorporating collisions into the particle transport model, without coupling the equations of motions for each particle.
        
        Works by Chen, Chacón et al. \cite{Chen_Chacón_Barnes_2011, Chacón_Chen_Barnes_2013, Chen_Chacón_2014, Chen_Chacón_2015} have developed structure-preserving particle pushers for neoclassical transport in the Vlasov equations, derived from Crank--Nicolson integrators. We show these too can can derive from a FET interpretation, similarly offering potential extensions to higher-order-in-time particle pushers. The FET formulation is used also to consider how the stochastic drift terms can be incorporated into the pushers. Stochastic gyrokinetic expansions are also discussed.

        Different options for the numerical implementation of these schemes are considered.

        Due to the efficacy of FET in the development of SP timesteppers for both the fluid and kinetic component, we hope this approach will prove effective in the future for developing SP timesteppers for the full hybrid model. We hope this will give us the opportunity to incorporate previously inaccessible kinetic effects into the highly effective, modern, finite-element MHD models.
    \end{abstract}
    
    
    \newpage
    \tableofcontents
    
    
    \newpage
    \pagenumbering{arabic}
    %\linenumbers\renewcommand\thelinenumber{\color{black!50}\arabic{linenumber}}
            \input{0 - introduction/main.tex}
        \part{Research}
            \input{1 - low-noise PiC models/main.tex}
            \input{2 - kinetic component/main.tex}
            \input{3 - fluid component/main.tex}
            \input{4 - numerical implementation/main.tex}
        \part{Project Overview}
            \input{5 - research plan/main.tex}
            \input{6 - summary/main.tex}
    
    
    %\section{}
    \newpage
    \pagenumbering{gobble}
        \printbibliography


    \newpage
    \pagenumbering{roman}
    \appendix
        \part{Appendices}
            \input{8 - Hilbert complexes/main.tex}
            \input{9 - weak conservation proofs/main.tex}
\end{document}

    
    
    %\section{}
    \newpage
    \pagenumbering{gobble}
        \printbibliography


    \newpage
    \pagenumbering{roman}
    \appendix
        \part{Appendices}
            \documentclass[12pt, a4paper]{report}

\input{template/main.tex}

\title{\BA{Title in Progress...}}
\author{Boris Andrews}
\affil{Mathematical Institute, University of Oxford}
\date{\today}


\begin{document}
    \pagenumbering{gobble}
    \maketitle
    
    
    \begin{abstract}
        Magnetic confinement reactors---in particular tokamaks---offer one of the most promising options for achieving practical nuclear fusion, with the potential to provide virtually limitless, clean energy. The theoretical and numerical modeling of tokamak plasmas is simultaneously an essential component of effective reactor design, and a great research barrier. Tokamak operational conditions exhibit comparatively low Knudsen numbers. Kinetic effects, including kinetic waves and instabilities, Landau damping, bump-on-tail instabilities and more, are therefore highly influential in tokamak plasma dynamics. Purely fluid models are inherently incapable of capturing these effects, whereas the high dimensionality in purely kinetic models render them practically intractable for most relevant purposes.

        We consider a $\delta\!f$ decomposition model, with a macroscopic fluid background and microscopic kinetic correction, both fully coupled to each other. A similar manner of discretization is proposed to that used in the recent \texttt{STRUPHY} code \cite{Holderied_Possanner_Wang_2021, Holderied_2022, Li_et_al_2023} with a finite-element model for the background and a pseudo-particle/PiC model for the correction.

        The fluid background satisfies the full, non-linear, resistive, compressible, Hall MHD equations. \cite{Laakmann_Hu_Farrell_2022} introduces finite-element(-in-space) implicit timesteppers for the incompressible analogue to this system with structure-preserving (SP) properties in the ideal case, alongside parameter-robust preconditioners. We show that these timesteppers can derive from a finite-element-in-time (FET) (and finite-element-in-space) interpretation. The benefits of this reformulation are discussed, including the derivation of timesteppers that are higher order in time, and the quantifiable dissipative SP properties in the non-ideal, resistive case.
        
        We discuss possible options for extending this FET approach to timesteppers for the compressible case.

        The kinetic corrections satisfy linearized Boltzmann equations. Using a Lénard--Bernstein collision operator, these take Fokker--Planck-like forms \cite{Fokker_1914, Planck_1917} wherein pseudo-particles in the numerical model obey the neoclassical transport equations, with particle-independent Brownian drift terms. This offers a rigorous methodology for incorporating collisions into the particle transport model, without coupling the equations of motions for each particle.
        
        Works by Chen, Chacón et al. \cite{Chen_Chacón_Barnes_2011, Chacón_Chen_Barnes_2013, Chen_Chacón_2014, Chen_Chacón_2015} have developed structure-preserving particle pushers for neoclassical transport in the Vlasov equations, derived from Crank--Nicolson integrators. We show these too can can derive from a FET interpretation, similarly offering potential extensions to higher-order-in-time particle pushers. The FET formulation is used also to consider how the stochastic drift terms can be incorporated into the pushers. Stochastic gyrokinetic expansions are also discussed.

        Different options for the numerical implementation of these schemes are considered.

        Due to the efficacy of FET in the development of SP timesteppers for both the fluid and kinetic component, we hope this approach will prove effective in the future for developing SP timesteppers for the full hybrid model. We hope this will give us the opportunity to incorporate previously inaccessible kinetic effects into the highly effective, modern, finite-element MHD models.
    \end{abstract}
    
    
    \newpage
    \tableofcontents
    
    
    \newpage
    \pagenumbering{arabic}
    %\linenumbers\renewcommand\thelinenumber{\color{black!50}\arabic{linenumber}}
            \input{0 - introduction/main.tex}
        \part{Research}
            \input{1 - low-noise PiC models/main.tex}
            \input{2 - kinetic component/main.tex}
            \input{3 - fluid component/main.tex}
            \input{4 - numerical implementation/main.tex}
        \part{Project Overview}
            \input{5 - research plan/main.tex}
            \input{6 - summary/main.tex}
    
    
    %\section{}
    \newpage
    \pagenumbering{gobble}
        \printbibliography


    \newpage
    \pagenumbering{roman}
    \appendix
        \part{Appendices}
            \input{8 - Hilbert complexes/main.tex}
            \input{9 - weak conservation proofs/main.tex}
\end{document}

            \documentclass[12pt, a4paper]{report}

\input{template/main.tex}

\title{\BA{Title in Progress...}}
\author{Boris Andrews}
\affil{Mathematical Institute, University of Oxford}
\date{\today}


\begin{document}
    \pagenumbering{gobble}
    \maketitle
    
    
    \begin{abstract}
        Magnetic confinement reactors---in particular tokamaks---offer one of the most promising options for achieving practical nuclear fusion, with the potential to provide virtually limitless, clean energy. The theoretical and numerical modeling of tokamak plasmas is simultaneously an essential component of effective reactor design, and a great research barrier. Tokamak operational conditions exhibit comparatively low Knudsen numbers. Kinetic effects, including kinetic waves and instabilities, Landau damping, bump-on-tail instabilities and more, are therefore highly influential in tokamak plasma dynamics. Purely fluid models are inherently incapable of capturing these effects, whereas the high dimensionality in purely kinetic models render them practically intractable for most relevant purposes.

        We consider a $\delta\!f$ decomposition model, with a macroscopic fluid background and microscopic kinetic correction, both fully coupled to each other. A similar manner of discretization is proposed to that used in the recent \texttt{STRUPHY} code \cite{Holderied_Possanner_Wang_2021, Holderied_2022, Li_et_al_2023} with a finite-element model for the background and a pseudo-particle/PiC model for the correction.

        The fluid background satisfies the full, non-linear, resistive, compressible, Hall MHD equations. \cite{Laakmann_Hu_Farrell_2022} introduces finite-element(-in-space) implicit timesteppers for the incompressible analogue to this system with structure-preserving (SP) properties in the ideal case, alongside parameter-robust preconditioners. We show that these timesteppers can derive from a finite-element-in-time (FET) (and finite-element-in-space) interpretation. The benefits of this reformulation are discussed, including the derivation of timesteppers that are higher order in time, and the quantifiable dissipative SP properties in the non-ideal, resistive case.
        
        We discuss possible options for extending this FET approach to timesteppers for the compressible case.

        The kinetic corrections satisfy linearized Boltzmann equations. Using a Lénard--Bernstein collision operator, these take Fokker--Planck-like forms \cite{Fokker_1914, Planck_1917} wherein pseudo-particles in the numerical model obey the neoclassical transport equations, with particle-independent Brownian drift terms. This offers a rigorous methodology for incorporating collisions into the particle transport model, without coupling the equations of motions for each particle.
        
        Works by Chen, Chacón et al. \cite{Chen_Chacón_Barnes_2011, Chacón_Chen_Barnes_2013, Chen_Chacón_2014, Chen_Chacón_2015} have developed structure-preserving particle pushers for neoclassical transport in the Vlasov equations, derived from Crank--Nicolson integrators. We show these too can can derive from a FET interpretation, similarly offering potential extensions to higher-order-in-time particle pushers. The FET formulation is used also to consider how the stochastic drift terms can be incorporated into the pushers. Stochastic gyrokinetic expansions are also discussed.

        Different options for the numerical implementation of these schemes are considered.

        Due to the efficacy of FET in the development of SP timesteppers for both the fluid and kinetic component, we hope this approach will prove effective in the future for developing SP timesteppers for the full hybrid model. We hope this will give us the opportunity to incorporate previously inaccessible kinetic effects into the highly effective, modern, finite-element MHD models.
    \end{abstract}
    
    
    \newpage
    \tableofcontents
    
    
    \newpage
    \pagenumbering{arabic}
    %\linenumbers\renewcommand\thelinenumber{\color{black!50}\arabic{linenumber}}
            \input{0 - introduction/main.tex}
        \part{Research}
            \input{1 - low-noise PiC models/main.tex}
            \input{2 - kinetic component/main.tex}
            \input{3 - fluid component/main.tex}
            \input{4 - numerical implementation/main.tex}
        \part{Project Overview}
            \input{5 - research plan/main.tex}
            \input{6 - summary/main.tex}
    
    
    %\section{}
    \newpage
    \pagenumbering{gobble}
        \printbibliography


    \newpage
    \pagenumbering{roman}
    \appendix
        \part{Appendices}
            \input{8 - Hilbert complexes/main.tex}
            \input{9 - weak conservation proofs/main.tex}
\end{document}

\end{document}


\title{\BA{Title in Progress...}}
\author{Boris Andrews}
\affil{Mathematical Institute, University of Oxford}
\date{\today}


\begin{document}
    \pagenumbering{gobble}
    \maketitle
    
    
    \begin{abstract}
        Magnetic confinement reactors---in particular tokamaks---offer one of the most promising options for achieving practical nuclear fusion, with the potential to provide virtually limitless, clean energy. The theoretical and numerical modeling of tokamak plasmas is simultaneously an essential component of effective reactor design, and a great research barrier. Tokamak operational conditions exhibit comparatively low Knudsen numbers. Kinetic effects, including kinetic waves and instabilities, Landau damping, bump-on-tail instabilities and more, are therefore highly influential in tokamak plasma dynamics. Purely fluid models are inherently incapable of capturing these effects, whereas the high dimensionality in purely kinetic models render them practically intractable for most relevant purposes.

        We consider a $\delta\!f$ decomposition model, with a macroscopic fluid background and microscopic kinetic correction, both fully coupled to each other. A similar manner of discretization is proposed to that used in the recent \texttt{STRUPHY} code \cite{Holderied_Possanner_Wang_2021, Holderied_2022, Li_et_al_2023} with a finite-element model for the background and a pseudo-particle/PiC model for the correction.

        The fluid background satisfies the full, non-linear, resistive, compressible, Hall MHD equations. \cite{Laakmann_Hu_Farrell_2022} introduces finite-element(-in-space) implicit timesteppers for the incompressible analogue to this system with structure-preserving (SP) properties in the ideal case, alongside parameter-robust preconditioners. We show that these timesteppers can derive from a finite-element-in-time (FET) (and finite-element-in-space) interpretation. The benefits of this reformulation are discussed, including the derivation of timesteppers that are higher order in time, and the quantifiable dissipative SP properties in the non-ideal, resistive case.
        
        We discuss possible options for extending this FET approach to timesteppers for the compressible case.

        The kinetic corrections satisfy linearized Boltzmann equations. Using a Lénard--Bernstein collision operator, these take Fokker--Planck-like forms \cite{Fokker_1914, Planck_1917} wherein pseudo-particles in the numerical model obey the neoclassical transport equations, with particle-independent Brownian drift terms. This offers a rigorous methodology for incorporating collisions into the particle transport model, without coupling the equations of motions for each particle.
        
        Works by Chen, Chacón et al. \cite{Chen_Chacón_Barnes_2011, Chacón_Chen_Barnes_2013, Chen_Chacón_2014, Chen_Chacón_2015} have developed structure-preserving particle pushers for neoclassical transport in the Vlasov equations, derived from Crank--Nicolson integrators. We show these too can can derive from a FET interpretation, similarly offering potential extensions to higher-order-in-time particle pushers. The FET formulation is used also to consider how the stochastic drift terms can be incorporated into the pushers. Stochastic gyrokinetic expansions are also discussed.

        Different options for the numerical implementation of these schemes are considered.

        Due to the efficacy of FET in the development of SP timesteppers for both the fluid and kinetic component, we hope this approach will prove effective in the future for developing SP timesteppers for the full hybrid model. We hope this will give us the opportunity to incorporate previously inaccessible kinetic effects into the highly effective, modern, finite-element MHD models.
    \end{abstract}
    
    
    \newpage
    \tableofcontents
    
    
    \newpage
    \pagenumbering{arabic}
    %\linenumbers\renewcommand\thelinenumber{\color{black!50}\arabic{linenumber}}
            \documentclass[12pt, a4paper]{report}

\documentclass[12pt, a4paper]{report}

\input{template/main.tex}

\title{\BA{Title in Progress...}}
\author{Boris Andrews}
\affil{Mathematical Institute, University of Oxford}
\date{\today}


\begin{document}
    \pagenumbering{gobble}
    \maketitle
    
    
    \begin{abstract}
        Magnetic confinement reactors---in particular tokamaks---offer one of the most promising options for achieving practical nuclear fusion, with the potential to provide virtually limitless, clean energy. The theoretical and numerical modeling of tokamak plasmas is simultaneously an essential component of effective reactor design, and a great research barrier. Tokamak operational conditions exhibit comparatively low Knudsen numbers. Kinetic effects, including kinetic waves and instabilities, Landau damping, bump-on-tail instabilities and more, are therefore highly influential in tokamak plasma dynamics. Purely fluid models are inherently incapable of capturing these effects, whereas the high dimensionality in purely kinetic models render them practically intractable for most relevant purposes.

        We consider a $\delta\!f$ decomposition model, with a macroscopic fluid background and microscopic kinetic correction, both fully coupled to each other. A similar manner of discretization is proposed to that used in the recent \texttt{STRUPHY} code \cite{Holderied_Possanner_Wang_2021, Holderied_2022, Li_et_al_2023} with a finite-element model for the background and a pseudo-particle/PiC model for the correction.

        The fluid background satisfies the full, non-linear, resistive, compressible, Hall MHD equations. \cite{Laakmann_Hu_Farrell_2022} introduces finite-element(-in-space) implicit timesteppers for the incompressible analogue to this system with structure-preserving (SP) properties in the ideal case, alongside parameter-robust preconditioners. We show that these timesteppers can derive from a finite-element-in-time (FET) (and finite-element-in-space) interpretation. The benefits of this reformulation are discussed, including the derivation of timesteppers that are higher order in time, and the quantifiable dissipative SP properties in the non-ideal, resistive case.
        
        We discuss possible options for extending this FET approach to timesteppers for the compressible case.

        The kinetic corrections satisfy linearized Boltzmann equations. Using a Lénard--Bernstein collision operator, these take Fokker--Planck-like forms \cite{Fokker_1914, Planck_1917} wherein pseudo-particles in the numerical model obey the neoclassical transport equations, with particle-independent Brownian drift terms. This offers a rigorous methodology for incorporating collisions into the particle transport model, without coupling the equations of motions for each particle.
        
        Works by Chen, Chacón et al. \cite{Chen_Chacón_Barnes_2011, Chacón_Chen_Barnes_2013, Chen_Chacón_2014, Chen_Chacón_2015} have developed structure-preserving particle pushers for neoclassical transport in the Vlasov equations, derived from Crank--Nicolson integrators. We show these too can can derive from a FET interpretation, similarly offering potential extensions to higher-order-in-time particle pushers. The FET formulation is used also to consider how the stochastic drift terms can be incorporated into the pushers. Stochastic gyrokinetic expansions are also discussed.

        Different options for the numerical implementation of these schemes are considered.

        Due to the efficacy of FET in the development of SP timesteppers for both the fluid and kinetic component, we hope this approach will prove effective in the future for developing SP timesteppers for the full hybrid model. We hope this will give us the opportunity to incorporate previously inaccessible kinetic effects into the highly effective, modern, finite-element MHD models.
    \end{abstract}
    
    
    \newpage
    \tableofcontents
    
    
    \newpage
    \pagenumbering{arabic}
    %\linenumbers\renewcommand\thelinenumber{\color{black!50}\arabic{linenumber}}
            \input{0 - introduction/main.tex}
        \part{Research}
            \input{1 - low-noise PiC models/main.tex}
            \input{2 - kinetic component/main.tex}
            \input{3 - fluid component/main.tex}
            \input{4 - numerical implementation/main.tex}
        \part{Project Overview}
            \input{5 - research plan/main.tex}
            \input{6 - summary/main.tex}
    
    
    %\section{}
    \newpage
    \pagenumbering{gobble}
        \printbibliography


    \newpage
    \pagenumbering{roman}
    \appendix
        \part{Appendices}
            \input{8 - Hilbert complexes/main.tex}
            \input{9 - weak conservation proofs/main.tex}
\end{document}


\title{\BA{Title in Progress...}}
\author{Boris Andrews}
\affil{Mathematical Institute, University of Oxford}
\date{\today}


\begin{document}
    \pagenumbering{gobble}
    \maketitle
    
    
    \begin{abstract}
        Magnetic confinement reactors---in particular tokamaks---offer one of the most promising options for achieving practical nuclear fusion, with the potential to provide virtually limitless, clean energy. The theoretical and numerical modeling of tokamak plasmas is simultaneously an essential component of effective reactor design, and a great research barrier. Tokamak operational conditions exhibit comparatively low Knudsen numbers. Kinetic effects, including kinetic waves and instabilities, Landau damping, bump-on-tail instabilities and more, are therefore highly influential in tokamak plasma dynamics. Purely fluid models are inherently incapable of capturing these effects, whereas the high dimensionality in purely kinetic models render them practically intractable for most relevant purposes.

        We consider a $\delta\!f$ decomposition model, with a macroscopic fluid background and microscopic kinetic correction, both fully coupled to each other. A similar manner of discretization is proposed to that used in the recent \texttt{STRUPHY} code \cite{Holderied_Possanner_Wang_2021, Holderied_2022, Li_et_al_2023} with a finite-element model for the background and a pseudo-particle/PiC model for the correction.

        The fluid background satisfies the full, non-linear, resistive, compressible, Hall MHD equations. \cite{Laakmann_Hu_Farrell_2022} introduces finite-element(-in-space) implicit timesteppers for the incompressible analogue to this system with structure-preserving (SP) properties in the ideal case, alongside parameter-robust preconditioners. We show that these timesteppers can derive from a finite-element-in-time (FET) (and finite-element-in-space) interpretation. The benefits of this reformulation are discussed, including the derivation of timesteppers that are higher order in time, and the quantifiable dissipative SP properties in the non-ideal, resistive case.
        
        We discuss possible options for extending this FET approach to timesteppers for the compressible case.

        The kinetic corrections satisfy linearized Boltzmann equations. Using a Lénard--Bernstein collision operator, these take Fokker--Planck-like forms \cite{Fokker_1914, Planck_1917} wherein pseudo-particles in the numerical model obey the neoclassical transport equations, with particle-independent Brownian drift terms. This offers a rigorous methodology for incorporating collisions into the particle transport model, without coupling the equations of motions for each particle.
        
        Works by Chen, Chacón et al. \cite{Chen_Chacón_Barnes_2011, Chacón_Chen_Barnes_2013, Chen_Chacón_2014, Chen_Chacón_2015} have developed structure-preserving particle pushers for neoclassical transport in the Vlasov equations, derived from Crank--Nicolson integrators. We show these too can can derive from a FET interpretation, similarly offering potential extensions to higher-order-in-time particle pushers. The FET formulation is used also to consider how the stochastic drift terms can be incorporated into the pushers. Stochastic gyrokinetic expansions are also discussed.

        Different options for the numerical implementation of these schemes are considered.

        Due to the efficacy of FET in the development of SP timesteppers for both the fluid and kinetic component, we hope this approach will prove effective in the future for developing SP timesteppers for the full hybrid model. We hope this will give us the opportunity to incorporate previously inaccessible kinetic effects into the highly effective, modern, finite-element MHD models.
    \end{abstract}
    
    
    \newpage
    \tableofcontents
    
    
    \newpage
    \pagenumbering{arabic}
    %\linenumbers\renewcommand\thelinenumber{\color{black!50}\arabic{linenumber}}
            \documentclass[12pt, a4paper]{report}

\input{template/main.tex}

\title{\BA{Title in Progress...}}
\author{Boris Andrews}
\affil{Mathematical Institute, University of Oxford}
\date{\today}


\begin{document}
    \pagenumbering{gobble}
    \maketitle
    
    
    \begin{abstract}
        Magnetic confinement reactors---in particular tokamaks---offer one of the most promising options for achieving practical nuclear fusion, with the potential to provide virtually limitless, clean energy. The theoretical and numerical modeling of tokamak plasmas is simultaneously an essential component of effective reactor design, and a great research barrier. Tokamak operational conditions exhibit comparatively low Knudsen numbers. Kinetic effects, including kinetic waves and instabilities, Landau damping, bump-on-tail instabilities and more, are therefore highly influential in tokamak plasma dynamics. Purely fluid models are inherently incapable of capturing these effects, whereas the high dimensionality in purely kinetic models render them practically intractable for most relevant purposes.

        We consider a $\delta\!f$ decomposition model, with a macroscopic fluid background and microscopic kinetic correction, both fully coupled to each other. A similar manner of discretization is proposed to that used in the recent \texttt{STRUPHY} code \cite{Holderied_Possanner_Wang_2021, Holderied_2022, Li_et_al_2023} with a finite-element model for the background and a pseudo-particle/PiC model for the correction.

        The fluid background satisfies the full, non-linear, resistive, compressible, Hall MHD equations. \cite{Laakmann_Hu_Farrell_2022} introduces finite-element(-in-space) implicit timesteppers for the incompressible analogue to this system with structure-preserving (SP) properties in the ideal case, alongside parameter-robust preconditioners. We show that these timesteppers can derive from a finite-element-in-time (FET) (and finite-element-in-space) interpretation. The benefits of this reformulation are discussed, including the derivation of timesteppers that are higher order in time, and the quantifiable dissipative SP properties in the non-ideal, resistive case.
        
        We discuss possible options for extending this FET approach to timesteppers for the compressible case.

        The kinetic corrections satisfy linearized Boltzmann equations. Using a Lénard--Bernstein collision operator, these take Fokker--Planck-like forms \cite{Fokker_1914, Planck_1917} wherein pseudo-particles in the numerical model obey the neoclassical transport equations, with particle-independent Brownian drift terms. This offers a rigorous methodology for incorporating collisions into the particle transport model, without coupling the equations of motions for each particle.
        
        Works by Chen, Chacón et al. \cite{Chen_Chacón_Barnes_2011, Chacón_Chen_Barnes_2013, Chen_Chacón_2014, Chen_Chacón_2015} have developed structure-preserving particle pushers for neoclassical transport in the Vlasov equations, derived from Crank--Nicolson integrators. We show these too can can derive from a FET interpretation, similarly offering potential extensions to higher-order-in-time particle pushers. The FET formulation is used also to consider how the stochastic drift terms can be incorporated into the pushers. Stochastic gyrokinetic expansions are also discussed.

        Different options for the numerical implementation of these schemes are considered.

        Due to the efficacy of FET in the development of SP timesteppers for both the fluid and kinetic component, we hope this approach will prove effective in the future for developing SP timesteppers for the full hybrid model. We hope this will give us the opportunity to incorporate previously inaccessible kinetic effects into the highly effective, modern, finite-element MHD models.
    \end{abstract}
    
    
    \newpage
    \tableofcontents
    
    
    \newpage
    \pagenumbering{arabic}
    %\linenumbers\renewcommand\thelinenumber{\color{black!50}\arabic{linenumber}}
            \input{0 - introduction/main.tex}
        \part{Research}
            \input{1 - low-noise PiC models/main.tex}
            \input{2 - kinetic component/main.tex}
            \input{3 - fluid component/main.tex}
            \input{4 - numerical implementation/main.tex}
        \part{Project Overview}
            \input{5 - research plan/main.tex}
            \input{6 - summary/main.tex}
    
    
    %\section{}
    \newpage
    \pagenumbering{gobble}
        \printbibliography


    \newpage
    \pagenumbering{roman}
    \appendix
        \part{Appendices}
            \input{8 - Hilbert complexes/main.tex}
            \input{9 - weak conservation proofs/main.tex}
\end{document}

        \part{Research}
            \documentclass[12pt, a4paper]{report}

\input{template/main.tex}

\title{\BA{Title in Progress...}}
\author{Boris Andrews}
\affil{Mathematical Institute, University of Oxford}
\date{\today}


\begin{document}
    \pagenumbering{gobble}
    \maketitle
    
    
    \begin{abstract}
        Magnetic confinement reactors---in particular tokamaks---offer one of the most promising options for achieving practical nuclear fusion, with the potential to provide virtually limitless, clean energy. The theoretical and numerical modeling of tokamak plasmas is simultaneously an essential component of effective reactor design, and a great research barrier. Tokamak operational conditions exhibit comparatively low Knudsen numbers. Kinetic effects, including kinetic waves and instabilities, Landau damping, bump-on-tail instabilities and more, are therefore highly influential in tokamak plasma dynamics. Purely fluid models are inherently incapable of capturing these effects, whereas the high dimensionality in purely kinetic models render them practically intractable for most relevant purposes.

        We consider a $\delta\!f$ decomposition model, with a macroscopic fluid background and microscopic kinetic correction, both fully coupled to each other. A similar manner of discretization is proposed to that used in the recent \texttt{STRUPHY} code \cite{Holderied_Possanner_Wang_2021, Holderied_2022, Li_et_al_2023} with a finite-element model for the background and a pseudo-particle/PiC model for the correction.

        The fluid background satisfies the full, non-linear, resistive, compressible, Hall MHD equations. \cite{Laakmann_Hu_Farrell_2022} introduces finite-element(-in-space) implicit timesteppers for the incompressible analogue to this system with structure-preserving (SP) properties in the ideal case, alongside parameter-robust preconditioners. We show that these timesteppers can derive from a finite-element-in-time (FET) (and finite-element-in-space) interpretation. The benefits of this reformulation are discussed, including the derivation of timesteppers that are higher order in time, and the quantifiable dissipative SP properties in the non-ideal, resistive case.
        
        We discuss possible options for extending this FET approach to timesteppers for the compressible case.

        The kinetic corrections satisfy linearized Boltzmann equations. Using a Lénard--Bernstein collision operator, these take Fokker--Planck-like forms \cite{Fokker_1914, Planck_1917} wherein pseudo-particles in the numerical model obey the neoclassical transport equations, with particle-independent Brownian drift terms. This offers a rigorous methodology for incorporating collisions into the particle transport model, without coupling the equations of motions for each particle.
        
        Works by Chen, Chacón et al. \cite{Chen_Chacón_Barnes_2011, Chacón_Chen_Barnes_2013, Chen_Chacón_2014, Chen_Chacón_2015} have developed structure-preserving particle pushers for neoclassical transport in the Vlasov equations, derived from Crank--Nicolson integrators. We show these too can can derive from a FET interpretation, similarly offering potential extensions to higher-order-in-time particle pushers. The FET formulation is used also to consider how the stochastic drift terms can be incorporated into the pushers. Stochastic gyrokinetic expansions are also discussed.

        Different options for the numerical implementation of these schemes are considered.

        Due to the efficacy of FET in the development of SP timesteppers for both the fluid and kinetic component, we hope this approach will prove effective in the future for developing SP timesteppers for the full hybrid model. We hope this will give us the opportunity to incorporate previously inaccessible kinetic effects into the highly effective, modern, finite-element MHD models.
    \end{abstract}
    
    
    \newpage
    \tableofcontents
    
    
    \newpage
    \pagenumbering{arabic}
    %\linenumbers\renewcommand\thelinenumber{\color{black!50}\arabic{linenumber}}
            \input{0 - introduction/main.tex}
        \part{Research}
            \input{1 - low-noise PiC models/main.tex}
            \input{2 - kinetic component/main.tex}
            \input{3 - fluid component/main.tex}
            \input{4 - numerical implementation/main.tex}
        \part{Project Overview}
            \input{5 - research plan/main.tex}
            \input{6 - summary/main.tex}
    
    
    %\section{}
    \newpage
    \pagenumbering{gobble}
        \printbibliography


    \newpage
    \pagenumbering{roman}
    \appendix
        \part{Appendices}
            \input{8 - Hilbert complexes/main.tex}
            \input{9 - weak conservation proofs/main.tex}
\end{document}

            \documentclass[12pt, a4paper]{report}

\input{template/main.tex}

\title{\BA{Title in Progress...}}
\author{Boris Andrews}
\affil{Mathematical Institute, University of Oxford}
\date{\today}


\begin{document}
    \pagenumbering{gobble}
    \maketitle
    
    
    \begin{abstract}
        Magnetic confinement reactors---in particular tokamaks---offer one of the most promising options for achieving practical nuclear fusion, with the potential to provide virtually limitless, clean energy. The theoretical and numerical modeling of tokamak plasmas is simultaneously an essential component of effective reactor design, and a great research barrier. Tokamak operational conditions exhibit comparatively low Knudsen numbers. Kinetic effects, including kinetic waves and instabilities, Landau damping, bump-on-tail instabilities and more, are therefore highly influential in tokamak plasma dynamics. Purely fluid models are inherently incapable of capturing these effects, whereas the high dimensionality in purely kinetic models render them practically intractable for most relevant purposes.

        We consider a $\delta\!f$ decomposition model, with a macroscopic fluid background and microscopic kinetic correction, both fully coupled to each other. A similar manner of discretization is proposed to that used in the recent \texttt{STRUPHY} code \cite{Holderied_Possanner_Wang_2021, Holderied_2022, Li_et_al_2023} with a finite-element model for the background and a pseudo-particle/PiC model for the correction.

        The fluid background satisfies the full, non-linear, resistive, compressible, Hall MHD equations. \cite{Laakmann_Hu_Farrell_2022} introduces finite-element(-in-space) implicit timesteppers for the incompressible analogue to this system with structure-preserving (SP) properties in the ideal case, alongside parameter-robust preconditioners. We show that these timesteppers can derive from a finite-element-in-time (FET) (and finite-element-in-space) interpretation. The benefits of this reformulation are discussed, including the derivation of timesteppers that are higher order in time, and the quantifiable dissipative SP properties in the non-ideal, resistive case.
        
        We discuss possible options for extending this FET approach to timesteppers for the compressible case.

        The kinetic corrections satisfy linearized Boltzmann equations. Using a Lénard--Bernstein collision operator, these take Fokker--Planck-like forms \cite{Fokker_1914, Planck_1917} wherein pseudo-particles in the numerical model obey the neoclassical transport equations, with particle-independent Brownian drift terms. This offers a rigorous methodology for incorporating collisions into the particle transport model, without coupling the equations of motions for each particle.
        
        Works by Chen, Chacón et al. \cite{Chen_Chacón_Barnes_2011, Chacón_Chen_Barnes_2013, Chen_Chacón_2014, Chen_Chacón_2015} have developed structure-preserving particle pushers for neoclassical transport in the Vlasov equations, derived from Crank--Nicolson integrators. We show these too can can derive from a FET interpretation, similarly offering potential extensions to higher-order-in-time particle pushers. The FET formulation is used also to consider how the stochastic drift terms can be incorporated into the pushers. Stochastic gyrokinetic expansions are also discussed.

        Different options for the numerical implementation of these schemes are considered.

        Due to the efficacy of FET in the development of SP timesteppers for both the fluid and kinetic component, we hope this approach will prove effective in the future for developing SP timesteppers for the full hybrid model. We hope this will give us the opportunity to incorporate previously inaccessible kinetic effects into the highly effective, modern, finite-element MHD models.
    \end{abstract}
    
    
    \newpage
    \tableofcontents
    
    
    \newpage
    \pagenumbering{arabic}
    %\linenumbers\renewcommand\thelinenumber{\color{black!50}\arabic{linenumber}}
            \input{0 - introduction/main.tex}
        \part{Research}
            \input{1 - low-noise PiC models/main.tex}
            \input{2 - kinetic component/main.tex}
            \input{3 - fluid component/main.tex}
            \input{4 - numerical implementation/main.tex}
        \part{Project Overview}
            \input{5 - research plan/main.tex}
            \input{6 - summary/main.tex}
    
    
    %\section{}
    \newpage
    \pagenumbering{gobble}
        \printbibliography


    \newpage
    \pagenumbering{roman}
    \appendix
        \part{Appendices}
            \input{8 - Hilbert complexes/main.tex}
            \input{9 - weak conservation proofs/main.tex}
\end{document}

            \documentclass[12pt, a4paper]{report}

\input{template/main.tex}

\title{\BA{Title in Progress...}}
\author{Boris Andrews}
\affil{Mathematical Institute, University of Oxford}
\date{\today}


\begin{document}
    \pagenumbering{gobble}
    \maketitle
    
    
    \begin{abstract}
        Magnetic confinement reactors---in particular tokamaks---offer one of the most promising options for achieving practical nuclear fusion, with the potential to provide virtually limitless, clean energy. The theoretical and numerical modeling of tokamak plasmas is simultaneously an essential component of effective reactor design, and a great research barrier. Tokamak operational conditions exhibit comparatively low Knudsen numbers. Kinetic effects, including kinetic waves and instabilities, Landau damping, bump-on-tail instabilities and more, are therefore highly influential in tokamak plasma dynamics. Purely fluid models are inherently incapable of capturing these effects, whereas the high dimensionality in purely kinetic models render them practically intractable for most relevant purposes.

        We consider a $\delta\!f$ decomposition model, with a macroscopic fluid background and microscopic kinetic correction, both fully coupled to each other. A similar manner of discretization is proposed to that used in the recent \texttt{STRUPHY} code \cite{Holderied_Possanner_Wang_2021, Holderied_2022, Li_et_al_2023} with a finite-element model for the background and a pseudo-particle/PiC model for the correction.

        The fluid background satisfies the full, non-linear, resistive, compressible, Hall MHD equations. \cite{Laakmann_Hu_Farrell_2022} introduces finite-element(-in-space) implicit timesteppers for the incompressible analogue to this system with structure-preserving (SP) properties in the ideal case, alongside parameter-robust preconditioners. We show that these timesteppers can derive from a finite-element-in-time (FET) (and finite-element-in-space) interpretation. The benefits of this reformulation are discussed, including the derivation of timesteppers that are higher order in time, and the quantifiable dissipative SP properties in the non-ideal, resistive case.
        
        We discuss possible options for extending this FET approach to timesteppers for the compressible case.

        The kinetic corrections satisfy linearized Boltzmann equations. Using a Lénard--Bernstein collision operator, these take Fokker--Planck-like forms \cite{Fokker_1914, Planck_1917} wherein pseudo-particles in the numerical model obey the neoclassical transport equations, with particle-independent Brownian drift terms. This offers a rigorous methodology for incorporating collisions into the particle transport model, without coupling the equations of motions for each particle.
        
        Works by Chen, Chacón et al. \cite{Chen_Chacón_Barnes_2011, Chacón_Chen_Barnes_2013, Chen_Chacón_2014, Chen_Chacón_2015} have developed structure-preserving particle pushers for neoclassical transport in the Vlasov equations, derived from Crank--Nicolson integrators. We show these too can can derive from a FET interpretation, similarly offering potential extensions to higher-order-in-time particle pushers. The FET formulation is used also to consider how the stochastic drift terms can be incorporated into the pushers. Stochastic gyrokinetic expansions are also discussed.

        Different options for the numerical implementation of these schemes are considered.

        Due to the efficacy of FET in the development of SP timesteppers for both the fluid and kinetic component, we hope this approach will prove effective in the future for developing SP timesteppers for the full hybrid model. We hope this will give us the opportunity to incorporate previously inaccessible kinetic effects into the highly effective, modern, finite-element MHD models.
    \end{abstract}
    
    
    \newpage
    \tableofcontents
    
    
    \newpage
    \pagenumbering{arabic}
    %\linenumbers\renewcommand\thelinenumber{\color{black!50}\arabic{linenumber}}
            \input{0 - introduction/main.tex}
        \part{Research}
            \input{1 - low-noise PiC models/main.tex}
            \input{2 - kinetic component/main.tex}
            \input{3 - fluid component/main.tex}
            \input{4 - numerical implementation/main.tex}
        \part{Project Overview}
            \input{5 - research plan/main.tex}
            \input{6 - summary/main.tex}
    
    
    %\section{}
    \newpage
    \pagenumbering{gobble}
        \printbibliography


    \newpage
    \pagenumbering{roman}
    \appendix
        \part{Appendices}
            \input{8 - Hilbert complexes/main.tex}
            \input{9 - weak conservation proofs/main.tex}
\end{document}

            \documentclass[12pt, a4paper]{report}

\input{template/main.tex}

\title{\BA{Title in Progress...}}
\author{Boris Andrews}
\affil{Mathematical Institute, University of Oxford}
\date{\today}


\begin{document}
    \pagenumbering{gobble}
    \maketitle
    
    
    \begin{abstract}
        Magnetic confinement reactors---in particular tokamaks---offer one of the most promising options for achieving practical nuclear fusion, with the potential to provide virtually limitless, clean energy. The theoretical and numerical modeling of tokamak plasmas is simultaneously an essential component of effective reactor design, and a great research barrier. Tokamak operational conditions exhibit comparatively low Knudsen numbers. Kinetic effects, including kinetic waves and instabilities, Landau damping, bump-on-tail instabilities and more, are therefore highly influential in tokamak plasma dynamics. Purely fluid models are inherently incapable of capturing these effects, whereas the high dimensionality in purely kinetic models render them practically intractable for most relevant purposes.

        We consider a $\delta\!f$ decomposition model, with a macroscopic fluid background and microscopic kinetic correction, both fully coupled to each other. A similar manner of discretization is proposed to that used in the recent \texttt{STRUPHY} code \cite{Holderied_Possanner_Wang_2021, Holderied_2022, Li_et_al_2023} with a finite-element model for the background and a pseudo-particle/PiC model for the correction.

        The fluid background satisfies the full, non-linear, resistive, compressible, Hall MHD equations. \cite{Laakmann_Hu_Farrell_2022} introduces finite-element(-in-space) implicit timesteppers for the incompressible analogue to this system with structure-preserving (SP) properties in the ideal case, alongside parameter-robust preconditioners. We show that these timesteppers can derive from a finite-element-in-time (FET) (and finite-element-in-space) interpretation. The benefits of this reformulation are discussed, including the derivation of timesteppers that are higher order in time, and the quantifiable dissipative SP properties in the non-ideal, resistive case.
        
        We discuss possible options for extending this FET approach to timesteppers for the compressible case.

        The kinetic corrections satisfy linearized Boltzmann equations. Using a Lénard--Bernstein collision operator, these take Fokker--Planck-like forms \cite{Fokker_1914, Planck_1917} wherein pseudo-particles in the numerical model obey the neoclassical transport equations, with particle-independent Brownian drift terms. This offers a rigorous methodology for incorporating collisions into the particle transport model, without coupling the equations of motions for each particle.
        
        Works by Chen, Chacón et al. \cite{Chen_Chacón_Barnes_2011, Chacón_Chen_Barnes_2013, Chen_Chacón_2014, Chen_Chacón_2015} have developed structure-preserving particle pushers for neoclassical transport in the Vlasov equations, derived from Crank--Nicolson integrators. We show these too can can derive from a FET interpretation, similarly offering potential extensions to higher-order-in-time particle pushers. The FET formulation is used also to consider how the stochastic drift terms can be incorporated into the pushers. Stochastic gyrokinetic expansions are also discussed.

        Different options for the numerical implementation of these schemes are considered.

        Due to the efficacy of FET in the development of SP timesteppers for both the fluid and kinetic component, we hope this approach will prove effective in the future for developing SP timesteppers for the full hybrid model. We hope this will give us the opportunity to incorporate previously inaccessible kinetic effects into the highly effective, modern, finite-element MHD models.
    \end{abstract}
    
    
    \newpage
    \tableofcontents
    
    
    \newpage
    \pagenumbering{arabic}
    %\linenumbers\renewcommand\thelinenumber{\color{black!50}\arabic{linenumber}}
            \input{0 - introduction/main.tex}
        \part{Research}
            \input{1 - low-noise PiC models/main.tex}
            \input{2 - kinetic component/main.tex}
            \input{3 - fluid component/main.tex}
            \input{4 - numerical implementation/main.tex}
        \part{Project Overview}
            \input{5 - research plan/main.tex}
            \input{6 - summary/main.tex}
    
    
    %\section{}
    \newpage
    \pagenumbering{gobble}
        \printbibliography


    \newpage
    \pagenumbering{roman}
    \appendix
        \part{Appendices}
            \input{8 - Hilbert complexes/main.tex}
            \input{9 - weak conservation proofs/main.tex}
\end{document}

        \part{Project Overview}
            \documentclass[12pt, a4paper]{report}

\input{template/main.tex}

\title{\BA{Title in Progress...}}
\author{Boris Andrews}
\affil{Mathematical Institute, University of Oxford}
\date{\today}


\begin{document}
    \pagenumbering{gobble}
    \maketitle
    
    
    \begin{abstract}
        Magnetic confinement reactors---in particular tokamaks---offer one of the most promising options for achieving practical nuclear fusion, with the potential to provide virtually limitless, clean energy. The theoretical and numerical modeling of tokamak plasmas is simultaneously an essential component of effective reactor design, and a great research barrier. Tokamak operational conditions exhibit comparatively low Knudsen numbers. Kinetic effects, including kinetic waves and instabilities, Landau damping, bump-on-tail instabilities and more, are therefore highly influential in tokamak plasma dynamics. Purely fluid models are inherently incapable of capturing these effects, whereas the high dimensionality in purely kinetic models render them practically intractable for most relevant purposes.

        We consider a $\delta\!f$ decomposition model, with a macroscopic fluid background and microscopic kinetic correction, both fully coupled to each other. A similar manner of discretization is proposed to that used in the recent \texttt{STRUPHY} code \cite{Holderied_Possanner_Wang_2021, Holderied_2022, Li_et_al_2023} with a finite-element model for the background and a pseudo-particle/PiC model for the correction.

        The fluid background satisfies the full, non-linear, resistive, compressible, Hall MHD equations. \cite{Laakmann_Hu_Farrell_2022} introduces finite-element(-in-space) implicit timesteppers for the incompressible analogue to this system with structure-preserving (SP) properties in the ideal case, alongside parameter-robust preconditioners. We show that these timesteppers can derive from a finite-element-in-time (FET) (and finite-element-in-space) interpretation. The benefits of this reformulation are discussed, including the derivation of timesteppers that are higher order in time, and the quantifiable dissipative SP properties in the non-ideal, resistive case.
        
        We discuss possible options for extending this FET approach to timesteppers for the compressible case.

        The kinetic corrections satisfy linearized Boltzmann equations. Using a Lénard--Bernstein collision operator, these take Fokker--Planck-like forms \cite{Fokker_1914, Planck_1917} wherein pseudo-particles in the numerical model obey the neoclassical transport equations, with particle-independent Brownian drift terms. This offers a rigorous methodology for incorporating collisions into the particle transport model, without coupling the equations of motions for each particle.
        
        Works by Chen, Chacón et al. \cite{Chen_Chacón_Barnes_2011, Chacón_Chen_Barnes_2013, Chen_Chacón_2014, Chen_Chacón_2015} have developed structure-preserving particle pushers for neoclassical transport in the Vlasov equations, derived from Crank--Nicolson integrators. We show these too can can derive from a FET interpretation, similarly offering potential extensions to higher-order-in-time particle pushers. The FET formulation is used also to consider how the stochastic drift terms can be incorporated into the pushers. Stochastic gyrokinetic expansions are also discussed.

        Different options for the numerical implementation of these schemes are considered.

        Due to the efficacy of FET in the development of SP timesteppers for both the fluid and kinetic component, we hope this approach will prove effective in the future for developing SP timesteppers for the full hybrid model. We hope this will give us the opportunity to incorporate previously inaccessible kinetic effects into the highly effective, modern, finite-element MHD models.
    \end{abstract}
    
    
    \newpage
    \tableofcontents
    
    
    \newpage
    \pagenumbering{arabic}
    %\linenumbers\renewcommand\thelinenumber{\color{black!50}\arabic{linenumber}}
            \input{0 - introduction/main.tex}
        \part{Research}
            \input{1 - low-noise PiC models/main.tex}
            \input{2 - kinetic component/main.tex}
            \input{3 - fluid component/main.tex}
            \input{4 - numerical implementation/main.tex}
        \part{Project Overview}
            \input{5 - research plan/main.tex}
            \input{6 - summary/main.tex}
    
    
    %\section{}
    \newpage
    \pagenumbering{gobble}
        \printbibliography


    \newpage
    \pagenumbering{roman}
    \appendix
        \part{Appendices}
            \input{8 - Hilbert complexes/main.tex}
            \input{9 - weak conservation proofs/main.tex}
\end{document}

            \documentclass[12pt, a4paper]{report}

\input{template/main.tex}

\title{\BA{Title in Progress...}}
\author{Boris Andrews}
\affil{Mathematical Institute, University of Oxford}
\date{\today}


\begin{document}
    \pagenumbering{gobble}
    \maketitle
    
    
    \begin{abstract}
        Magnetic confinement reactors---in particular tokamaks---offer one of the most promising options for achieving practical nuclear fusion, with the potential to provide virtually limitless, clean energy. The theoretical and numerical modeling of tokamak plasmas is simultaneously an essential component of effective reactor design, and a great research barrier. Tokamak operational conditions exhibit comparatively low Knudsen numbers. Kinetic effects, including kinetic waves and instabilities, Landau damping, bump-on-tail instabilities and more, are therefore highly influential in tokamak plasma dynamics. Purely fluid models are inherently incapable of capturing these effects, whereas the high dimensionality in purely kinetic models render them practically intractable for most relevant purposes.

        We consider a $\delta\!f$ decomposition model, with a macroscopic fluid background and microscopic kinetic correction, both fully coupled to each other. A similar manner of discretization is proposed to that used in the recent \texttt{STRUPHY} code \cite{Holderied_Possanner_Wang_2021, Holderied_2022, Li_et_al_2023} with a finite-element model for the background and a pseudo-particle/PiC model for the correction.

        The fluid background satisfies the full, non-linear, resistive, compressible, Hall MHD equations. \cite{Laakmann_Hu_Farrell_2022} introduces finite-element(-in-space) implicit timesteppers for the incompressible analogue to this system with structure-preserving (SP) properties in the ideal case, alongside parameter-robust preconditioners. We show that these timesteppers can derive from a finite-element-in-time (FET) (and finite-element-in-space) interpretation. The benefits of this reformulation are discussed, including the derivation of timesteppers that are higher order in time, and the quantifiable dissipative SP properties in the non-ideal, resistive case.
        
        We discuss possible options for extending this FET approach to timesteppers for the compressible case.

        The kinetic corrections satisfy linearized Boltzmann equations. Using a Lénard--Bernstein collision operator, these take Fokker--Planck-like forms \cite{Fokker_1914, Planck_1917} wherein pseudo-particles in the numerical model obey the neoclassical transport equations, with particle-independent Brownian drift terms. This offers a rigorous methodology for incorporating collisions into the particle transport model, without coupling the equations of motions for each particle.
        
        Works by Chen, Chacón et al. \cite{Chen_Chacón_Barnes_2011, Chacón_Chen_Barnes_2013, Chen_Chacón_2014, Chen_Chacón_2015} have developed structure-preserving particle pushers for neoclassical transport in the Vlasov equations, derived from Crank--Nicolson integrators. We show these too can can derive from a FET interpretation, similarly offering potential extensions to higher-order-in-time particle pushers. The FET formulation is used also to consider how the stochastic drift terms can be incorporated into the pushers. Stochastic gyrokinetic expansions are also discussed.

        Different options for the numerical implementation of these schemes are considered.

        Due to the efficacy of FET in the development of SP timesteppers for both the fluid and kinetic component, we hope this approach will prove effective in the future for developing SP timesteppers for the full hybrid model. We hope this will give us the opportunity to incorporate previously inaccessible kinetic effects into the highly effective, modern, finite-element MHD models.
    \end{abstract}
    
    
    \newpage
    \tableofcontents
    
    
    \newpage
    \pagenumbering{arabic}
    %\linenumbers\renewcommand\thelinenumber{\color{black!50}\arabic{linenumber}}
            \input{0 - introduction/main.tex}
        \part{Research}
            \input{1 - low-noise PiC models/main.tex}
            \input{2 - kinetic component/main.tex}
            \input{3 - fluid component/main.tex}
            \input{4 - numerical implementation/main.tex}
        \part{Project Overview}
            \input{5 - research plan/main.tex}
            \input{6 - summary/main.tex}
    
    
    %\section{}
    \newpage
    \pagenumbering{gobble}
        \printbibliography


    \newpage
    \pagenumbering{roman}
    \appendix
        \part{Appendices}
            \input{8 - Hilbert complexes/main.tex}
            \input{9 - weak conservation proofs/main.tex}
\end{document}

    
    
    %\section{}
    \newpage
    \pagenumbering{gobble}
        \printbibliography


    \newpage
    \pagenumbering{roman}
    \appendix
        \part{Appendices}
            \documentclass[12pt, a4paper]{report}

\input{template/main.tex}

\title{\BA{Title in Progress...}}
\author{Boris Andrews}
\affil{Mathematical Institute, University of Oxford}
\date{\today}


\begin{document}
    \pagenumbering{gobble}
    \maketitle
    
    
    \begin{abstract}
        Magnetic confinement reactors---in particular tokamaks---offer one of the most promising options for achieving practical nuclear fusion, with the potential to provide virtually limitless, clean energy. The theoretical and numerical modeling of tokamak plasmas is simultaneously an essential component of effective reactor design, and a great research barrier. Tokamak operational conditions exhibit comparatively low Knudsen numbers. Kinetic effects, including kinetic waves and instabilities, Landau damping, bump-on-tail instabilities and more, are therefore highly influential in tokamak plasma dynamics. Purely fluid models are inherently incapable of capturing these effects, whereas the high dimensionality in purely kinetic models render them practically intractable for most relevant purposes.

        We consider a $\delta\!f$ decomposition model, with a macroscopic fluid background and microscopic kinetic correction, both fully coupled to each other. A similar manner of discretization is proposed to that used in the recent \texttt{STRUPHY} code \cite{Holderied_Possanner_Wang_2021, Holderied_2022, Li_et_al_2023} with a finite-element model for the background and a pseudo-particle/PiC model for the correction.

        The fluid background satisfies the full, non-linear, resistive, compressible, Hall MHD equations. \cite{Laakmann_Hu_Farrell_2022} introduces finite-element(-in-space) implicit timesteppers for the incompressible analogue to this system with structure-preserving (SP) properties in the ideal case, alongside parameter-robust preconditioners. We show that these timesteppers can derive from a finite-element-in-time (FET) (and finite-element-in-space) interpretation. The benefits of this reformulation are discussed, including the derivation of timesteppers that are higher order in time, and the quantifiable dissipative SP properties in the non-ideal, resistive case.
        
        We discuss possible options for extending this FET approach to timesteppers for the compressible case.

        The kinetic corrections satisfy linearized Boltzmann equations. Using a Lénard--Bernstein collision operator, these take Fokker--Planck-like forms \cite{Fokker_1914, Planck_1917} wherein pseudo-particles in the numerical model obey the neoclassical transport equations, with particle-independent Brownian drift terms. This offers a rigorous methodology for incorporating collisions into the particle transport model, without coupling the equations of motions for each particle.
        
        Works by Chen, Chacón et al. \cite{Chen_Chacón_Barnes_2011, Chacón_Chen_Barnes_2013, Chen_Chacón_2014, Chen_Chacón_2015} have developed structure-preserving particle pushers for neoclassical transport in the Vlasov equations, derived from Crank--Nicolson integrators. We show these too can can derive from a FET interpretation, similarly offering potential extensions to higher-order-in-time particle pushers. The FET formulation is used also to consider how the stochastic drift terms can be incorporated into the pushers. Stochastic gyrokinetic expansions are also discussed.

        Different options for the numerical implementation of these schemes are considered.

        Due to the efficacy of FET in the development of SP timesteppers for both the fluid and kinetic component, we hope this approach will prove effective in the future for developing SP timesteppers for the full hybrid model. We hope this will give us the opportunity to incorporate previously inaccessible kinetic effects into the highly effective, modern, finite-element MHD models.
    \end{abstract}
    
    
    \newpage
    \tableofcontents
    
    
    \newpage
    \pagenumbering{arabic}
    %\linenumbers\renewcommand\thelinenumber{\color{black!50}\arabic{linenumber}}
            \input{0 - introduction/main.tex}
        \part{Research}
            \input{1 - low-noise PiC models/main.tex}
            \input{2 - kinetic component/main.tex}
            \input{3 - fluid component/main.tex}
            \input{4 - numerical implementation/main.tex}
        \part{Project Overview}
            \input{5 - research plan/main.tex}
            \input{6 - summary/main.tex}
    
    
    %\section{}
    \newpage
    \pagenumbering{gobble}
        \printbibliography


    \newpage
    \pagenumbering{roman}
    \appendix
        \part{Appendices}
            \input{8 - Hilbert complexes/main.tex}
            \input{9 - weak conservation proofs/main.tex}
\end{document}

            \documentclass[12pt, a4paper]{report}

\input{template/main.tex}

\title{\BA{Title in Progress...}}
\author{Boris Andrews}
\affil{Mathematical Institute, University of Oxford}
\date{\today}


\begin{document}
    \pagenumbering{gobble}
    \maketitle
    
    
    \begin{abstract}
        Magnetic confinement reactors---in particular tokamaks---offer one of the most promising options for achieving practical nuclear fusion, with the potential to provide virtually limitless, clean energy. The theoretical and numerical modeling of tokamak plasmas is simultaneously an essential component of effective reactor design, and a great research barrier. Tokamak operational conditions exhibit comparatively low Knudsen numbers. Kinetic effects, including kinetic waves and instabilities, Landau damping, bump-on-tail instabilities and more, are therefore highly influential in tokamak plasma dynamics. Purely fluid models are inherently incapable of capturing these effects, whereas the high dimensionality in purely kinetic models render them practically intractable for most relevant purposes.

        We consider a $\delta\!f$ decomposition model, with a macroscopic fluid background and microscopic kinetic correction, both fully coupled to each other. A similar manner of discretization is proposed to that used in the recent \texttt{STRUPHY} code \cite{Holderied_Possanner_Wang_2021, Holderied_2022, Li_et_al_2023} with a finite-element model for the background and a pseudo-particle/PiC model for the correction.

        The fluid background satisfies the full, non-linear, resistive, compressible, Hall MHD equations. \cite{Laakmann_Hu_Farrell_2022} introduces finite-element(-in-space) implicit timesteppers for the incompressible analogue to this system with structure-preserving (SP) properties in the ideal case, alongside parameter-robust preconditioners. We show that these timesteppers can derive from a finite-element-in-time (FET) (and finite-element-in-space) interpretation. The benefits of this reformulation are discussed, including the derivation of timesteppers that are higher order in time, and the quantifiable dissipative SP properties in the non-ideal, resistive case.
        
        We discuss possible options for extending this FET approach to timesteppers for the compressible case.

        The kinetic corrections satisfy linearized Boltzmann equations. Using a Lénard--Bernstein collision operator, these take Fokker--Planck-like forms \cite{Fokker_1914, Planck_1917} wherein pseudo-particles in the numerical model obey the neoclassical transport equations, with particle-independent Brownian drift terms. This offers a rigorous methodology for incorporating collisions into the particle transport model, without coupling the equations of motions for each particle.
        
        Works by Chen, Chacón et al. \cite{Chen_Chacón_Barnes_2011, Chacón_Chen_Barnes_2013, Chen_Chacón_2014, Chen_Chacón_2015} have developed structure-preserving particle pushers for neoclassical transport in the Vlasov equations, derived from Crank--Nicolson integrators. We show these too can can derive from a FET interpretation, similarly offering potential extensions to higher-order-in-time particle pushers. The FET formulation is used also to consider how the stochastic drift terms can be incorporated into the pushers. Stochastic gyrokinetic expansions are also discussed.

        Different options for the numerical implementation of these schemes are considered.

        Due to the efficacy of FET in the development of SP timesteppers for both the fluid and kinetic component, we hope this approach will prove effective in the future for developing SP timesteppers for the full hybrid model. We hope this will give us the opportunity to incorporate previously inaccessible kinetic effects into the highly effective, modern, finite-element MHD models.
    \end{abstract}
    
    
    \newpage
    \tableofcontents
    
    
    \newpage
    \pagenumbering{arabic}
    %\linenumbers\renewcommand\thelinenumber{\color{black!50}\arabic{linenumber}}
            \input{0 - introduction/main.tex}
        \part{Research}
            \input{1 - low-noise PiC models/main.tex}
            \input{2 - kinetic component/main.tex}
            \input{3 - fluid component/main.tex}
            \input{4 - numerical implementation/main.tex}
        \part{Project Overview}
            \input{5 - research plan/main.tex}
            \input{6 - summary/main.tex}
    
    
    %\section{}
    \newpage
    \pagenumbering{gobble}
        \printbibliography


    \newpage
    \pagenumbering{roman}
    \appendix
        \part{Appendices}
            \input{8 - Hilbert complexes/main.tex}
            \input{9 - weak conservation proofs/main.tex}
\end{document}

\end{document}

        \part{Research}
            \documentclass[12pt, a4paper]{report}

\documentclass[12pt, a4paper]{report}

\input{template/main.tex}

\title{\BA{Title in Progress...}}
\author{Boris Andrews}
\affil{Mathematical Institute, University of Oxford}
\date{\today}


\begin{document}
    \pagenumbering{gobble}
    \maketitle
    
    
    \begin{abstract}
        Magnetic confinement reactors---in particular tokamaks---offer one of the most promising options for achieving practical nuclear fusion, with the potential to provide virtually limitless, clean energy. The theoretical and numerical modeling of tokamak plasmas is simultaneously an essential component of effective reactor design, and a great research barrier. Tokamak operational conditions exhibit comparatively low Knudsen numbers. Kinetic effects, including kinetic waves and instabilities, Landau damping, bump-on-tail instabilities and more, are therefore highly influential in tokamak plasma dynamics. Purely fluid models are inherently incapable of capturing these effects, whereas the high dimensionality in purely kinetic models render them practically intractable for most relevant purposes.

        We consider a $\delta\!f$ decomposition model, with a macroscopic fluid background and microscopic kinetic correction, both fully coupled to each other. A similar manner of discretization is proposed to that used in the recent \texttt{STRUPHY} code \cite{Holderied_Possanner_Wang_2021, Holderied_2022, Li_et_al_2023} with a finite-element model for the background and a pseudo-particle/PiC model for the correction.

        The fluid background satisfies the full, non-linear, resistive, compressible, Hall MHD equations. \cite{Laakmann_Hu_Farrell_2022} introduces finite-element(-in-space) implicit timesteppers for the incompressible analogue to this system with structure-preserving (SP) properties in the ideal case, alongside parameter-robust preconditioners. We show that these timesteppers can derive from a finite-element-in-time (FET) (and finite-element-in-space) interpretation. The benefits of this reformulation are discussed, including the derivation of timesteppers that are higher order in time, and the quantifiable dissipative SP properties in the non-ideal, resistive case.
        
        We discuss possible options for extending this FET approach to timesteppers for the compressible case.

        The kinetic corrections satisfy linearized Boltzmann equations. Using a Lénard--Bernstein collision operator, these take Fokker--Planck-like forms \cite{Fokker_1914, Planck_1917} wherein pseudo-particles in the numerical model obey the neoclassical transport equations, with particle-independent Brownian drift terms. This offers a rigorous methodology for incorporating collisions into the particle transport model, without coupling the equations of motions for each particle.
        
        Works by Chen, Chacón et al. \cite{Chen_Chacón_Barnes_2011, Chacón_Chen_Barnes_2013, Chen_Chacón_2014, Chen_Chacón_2015} have developed structure-preserving particle pushers for neoclassical transport in the Vlasov equations, derived from Crank--Nicolson integrators. We show these too can can derive from a FET interpretation, similarly offering potential extensions to higher-order-in-time particle pushers. The FET formulation is used also to consider how the stochastic drift terms can be incorporated into the pushers. Stochastic gyrokinetic expansions are also discussed.

        Different options for the numerical implementation of these schemes are considered.

        Due to the efficacy of FET in the development of SP timesteppers for both the fluid and kinetic component, we hope this approach will prove effective in the future for developing SP timesteppers for the full hybrid model. We hope this will give us the opportunity to incorporate previously inaccessible kinetic effects into the highly effective, modern, finite-element MHD models.
    \end{abstract}
    
    
    \newpage
    \tableofcontents
    
    
    \newpage
    \pagenumbering{arabic}
    %\linenumbers\renewcommand\thelinenumber{\color{black!50}\arabic{linenumber}}
            \input{0 - introduction/main.tex}
        \part{Research}
            \input{1 - low-noise PiC models/main.tex}
            \input{2 - kinetic component/main.tex}
            \input{3 - fluid component/main.tex}
            \input{4 - numerical implementation/main.tex}
        \part{Project Overview}
            \input{5 - research plan/main.tex}
            \input{6 - summary/main.tex}
    
    
    %\section{}
    \newpage
    \pagenumbering{gobble}
        \printbibliography


    \newpage
    \pagenumbering{roman}
    \appendix
        \part{Appendices}
            \input{8 - Hilbert complexes/main.tex}
            \input{9 - weak conservation proofs/main.tex}
\end{document}


\title{\BA{Title in Progress...}}
\author{Boris Andrews}
\affil{Mathematical Institute, University of Oxford}
\date{\today}


\begin{document}
    \pagenumbering{gobble}
    \maketitle
    
    
    \begin{abstract}
        Magnetic confinement reactors---in particular tokamaks---offer one of the most promising options for achieving practical nuclear fusion, with the potential to provide virtually limitless, clean energy. The theoretical and numerical modeling of tokamak plasmas is simultaneously an essential component of effective reactor design, and a great research barrier. Tokamak operational conditions exhibit comparatively low Knudsen numbers. Kinetic effects, including kinetic waves and instabilities, Landau damping, bump-on-tail instabilities and more, are therefore highly influential in tokamak plasma dynamics. Purely fluid models are inherently incapable of capturing these effects, whereas the high dimensionality in purely kinetic models render them practically intractable for most relevant purposes.

        We consider a $\delta\!f$ decomposition model, with a macroscopic fluid background and microscopic kinetic correction, both fully coupled to each other. A similar manner of discretization is proposed to that used in the recent \texttt{STRUPHY} code \cite{Holderied_Possanner_Wang_2021, Holderied_2022, Li_et_al_2023} with a finite-element model for the background and a pseudo-particle/PiC model for the correction.

        The fluid background satisfies the full, non-linear, resistive, compressible, Hall MHD equations. \cite{Laakmann_Hu_Farrell_2022} introduces finite-element(-in-space) implicit timesteppers for the incompressible analogue to this system with structure-preserving (SP) properties in the ideal case, alongside parameter-robust preconditioners. We show that these timesteppers can derive from a finite-element-in-time (FET) (and finite-element-in-space) interpretation. The benefits of this reformulation are discussed, including the derivation of timesteppers that are higher order in time, and the quantifiable dissipative SP properties in the non-ideal, resistive case.
        
        We discuss possible options for extending this FET approach to timesteppers for the compressible case.

        The kinetic corrections satisfy linearized Boltzmann equations. Using a Lénard--Bernstein collision operator, these take Fokker--Planck-like forms \cite{Fokker_1914, Planck_1917} wherein pseudo-particles in the numerical model obey the neoclassical transport equations, with particle-independent Brownian drift terms. This offers a rigorous methodology for incorporating collisions into the particle transport model, without coupling the equations of motions for each particle.
        
        Works by Chen, Chacón et al. \cite{Chen_Chacón_Barnes_2011, Chacón_Chen_Barnes_2013, Chen_Chacón_2014, Chen_Chacón_2015} have developed structure-preserving particle pushers for neoclassical transport in the Vlasov equations, derived from Crank--Nicolson integrators. We show these too can can derive from a FET interpretation, similarly offering potential extensions to higher-order-in-time particle pushers. The FET formulation is used also to consider how the stochastic drift terms can be incorporated into the pushers. Stochastic gyrokinetic expansions are also discussed.

        Different options for the numerical implementation of these schemes are considered.

        Due to the efficacy of FET in the development of SP timesteppers for both the fluid and kinetic component, we hope this approach will prove effective in the future for developing SP timesteppers for the full hybrid model. We hope this will give us the opportunity to incorporate previously inaccessible kinetic effects into the highly effective, modern, finite-element MHD models.
    \end{abstract}
    
    
    \newpage
    \tableofcontents
    
    
    \newpage
    \pagenumbering{arabic}
    %\linenumbers\renewcommand\thelinenumber{\color{black!50}\arabic{linenumber}}
            \documentclass[12pt, a4paper]{report}

\input{template/main.tex}

\title{\BA{Title in Progress...}}
\author{Boris Andrews}
\affil{Mathematical Institute, University of Oxford}
\date{\today}


\begin{document}
    \pagenumbering{gobble}
    \maketitle
    
    
    \begin{abstract}
        Magnetic confinement reactors---in particular tokamaks---offer one of the most promising options for achieving practical nuclear fusion, with the potential to provide virtually limitless, clean energy. The theoretical and numerical modeling of tokamak plasmas is simultaneously an essential component of effective reactor design, and a great research barrier. Tokamak operational conditions exhibit comparatively low Knudsen numbers. Kinetic effects, including kinetic waves and instabilities, Landau damping, bump-on-tail instabilities and more, are therefore highly influential in tokamak plasma dynamics. Purely fluid models are inherently incapable of capturing these effects, whereas the high dimensionality in purely kinetic models render them practically intractable for most relevant purposes.

        We consider a $\delta\!f$ decomposition model, with a macroscopic fluid background and microscopic kinetic correction, both fully coupled to each other. A similar manner of discretization is proposed to that used in the recent \texttt{STRUPHY} code \cite{Holderied_Possanner_Wang_2021, Holderied_2022, Li_et_al_2023} with a finite-element model for the background and a pseudo-particle/PiC model for the correction.

        The fluid background satisfies the full, non-linear, resistive, compressible, Hall MHD equations. \cite{Laakmann_Hu_Farrell_2022} introduces finite-element(-in-space) implicit timesteppers for the incompressible analogue to this system with structure-preserving (SP) properties in the ideal case, alongside parameter-robust preconditioners. We show that these timesteppers can derive from a finite-element-in-time (FET) (and finite-element-in-space) interpretation. The benefits of this reformulation are discussed, including the derivation of timesteppers that are higher order in time, and the quantifiable dissipative SP properties in the non-ideal, resistive case.
        
        We discuss possible options for extending this FET approach to timesteppers for the compressible case.

        The kinetic corrections satisfy linearized Boltzmann equations. Using a Lénard--Bernstein collision operator, these take Fokker--Planck-like forms \cite{Fokker_1914, Planck_1917} wherein pseudo-particles in the numerical model obey the neoclassical transport equations, with particle-independent Brownian drift terms. This offers a rigorous methodology for incorporating collisions into the particle transport model, without coupling the equations of motions for each particle.
        
        Works by Chen, Chacón et al. \cite{Chen_Chacón_Barnes_2011, Chacón_Chen_Barnes_2013, Chen_Chacón_2014, Chen_Chacón_2015} have developed structure-preserving particle pushers for neoclassical transport in the Vlasov equations, derived from Crank--Nicolson integrators. We show these too can can derive from a FET interpretation, similarly offering potential extensions to higher-order-in-time particle pushers. The FET formulation is used also to consider how the stochastic drift terms can be incorporated into the pushers. Stochastic gyrokinetic expansions are also discussed.

        Different options for the numerical implementation of these schemes are considered.

        Due to the efficacy of FET in the development of SP timesteppers for both the fluid and kinetic component, we hope this approach will prove effective in the future for developing SP timesteppers for the full hybrid model. We hope this will give us the opportunity to incorporate previously inaccessible kinetic effects into the highly effective, modern, finite-element MHD models.
    \end{abstract}
    
    
    \newpage
    \tableofcontents
    
    
    \newpage
    \pagenumbering{arabic}
    %\linenumbers\renewcommand\thelinenumber{\color{black!50}\arabic{linenumber}}
            \input{0 - introduction/main.tex}
        \part{Research}
            \input{1 - low-noise PiC models/main.tex}
            \input{2 - kinetic component/main.tex}
            \input{3 - fluid component/main.tex}
            \input{4 - numerical implementation/main.tex}
        \part{Project Overview}
            \input{5 - research plan/main.tex}
            \input{6 - summary/main.tex}
    
    
    %\section{}
    \newpage
    \pagenumbering{gobble}
        \printbibliography


    \newpage
    \pagenumbering{roman}
    \appendix
        \part{Appendices}
            \input{8 - Hilbert complexes/main.tex}
            \input{9 - weak conservation proofs/main.tex}
\end{document}

        \part{Research}
            \documentclass[12pt, a4paper]{report}

\input{template/main.tex}

\title{\BA{Title in Progress...}}
\author{Boris Andrews}
\affil{Mathematical Institute, University of Oxford}
\date{\today}


\begin{document}
    \pagenumbering{gobble}
    \maketitle
    
    
    \begin{abstract}
        Magnetic confinement reactors---in particular tokamaks---offer one of the most promising options for achieving practical nuclear fusion, with the potential to provide virtually limitless, clean energy. The theoretical and numerical modeling of tokamak plasmas is simultaneously an essential component of effective reactor design, and a great research barrier. Tokamak operational conditions exhibit comparatively low Knudsen numbers. Kinetic effects, including kinetic waves and instabilities, Landau damping, bump-on-tail instabilities and more, are therefore highly influential in tokamak plasma dynamics. Purely fluid models are inherently incapable of capturing these effects, whereas the high dimensionality in purely kinetic models render them practically intractable for most relevant purposes.

        We consider a $\delta\!f$ decomposition model, with a macroscopic fluid background and microscopic kinetic correction, both fully coupled to each other. A similar manner of discretization is proposed to that used in the recent \texttt{STRUPHY} code \cite{Holderied_Possanner_Wang_2021, Holderied_2022, Li_et_al_2023} with a finite-element model for the background and a pseudo-particle/PiC model for the correction.

        The fluid background satisfies the full, non-linear, resistive, compressible, Hall MHD equations. \cite{Laakmann_Hu_Farrell_2022} introduces finite-element(-in-space) implicit timesteppers for the incompressible analogue to this system with structure-preserving (SP) properties in the ideal case, alongside parameter-robust preconditioners. We show that these timesteppers can derive from a finite-element-in-time (FET) (and finite-element-in-space) interpretation. The benefits of this reformulation are discussed, including the derivation of timesteppers that are higher order in time, and the quantifiable dissipative SP properties in the non-ideal, resistive case.
        
        We discuss possible options for extending this FET approach to timesteppers for the compressible case.

        The kinetic corrections satisfy linearized Boltzmann equations. Using a Lénard--Bernstein collision operator, these take Fokker--Planck-like forms \cite{Fokker_1914, Planck_1917} wherein pseudo-particles in the numerical model obey the neoclassical transport equations, with particle-independent Brownian drift terms. This offers a rigorous methodology for incorporating collisions into the particle transport model, without coupling the equations of motions for each particle.
        
        Works by Chen, Chacón et al. \cite{Chen_Chacón_Barnes_2011, Chacón_Chen_Barnes_2013, Chen_Chacón_2014, Chen_Chacón_2015} have developed structure-preserving particle pushers for neoclassical transport in the Vlasov equations, derived from Crank--Nicolson integrators. We show these too can can derive from a FET interpretation, similarly offering potential extensions to higher-order-in-time particle pushers. The FET formulation is used also to consider how the stochastic drift terms can be incorporated into the pushers. Stochastic gyrokinetic expansions are also discussed.

        Different options for the numerical implementation of these schemes are considered.

        Due to the efficacy of FET in the development of SP timesteppers for both the fluid and kinetic component, we hope this approach will prove effective in the future for developing SP timesteppers for the full hybrid model. We hope this will give us the opportunity to incorporate previously inaccessible kinetic effects into the highly effective, modern, finite-element MHD models.
    \end{abstract}
    
    
    \newpage
    \tableofcontents
    
    
    \newpage
    \pagenumbering{arabic}
    %\linenumbers\renewcommand\thelinenumber{\color{black!50}\arabic{linenumber}}
            \input{0 - introduction/main.tex}
        \part{Research}
            \input{1 - low-noise PiC models/main.tex}
            \input{2 - kinetic component/main.tex}
            \input{3 - fluid component/main.tex}
            \input{4 - numerical implementation/main.tex}
        \part{Project Overview}
            \input{5 - research plan/main.tex}
            \input{6 - summary/main.tex}
    
    
    %\section{}
    \newpage
    \pagenumbering{gobble}
        \printbibliography


    \newpage
    \pagenumbering{roman}
    \appendix
        \part{Appendices}
            \input{8 - Hilbert complexes/main.tex}
            \input{9 - weak conservation proofs/main.tex}
\end{document}

            \documentclass[12pt, a4paper]{report}

\input{template/main.tex}

\title{\BA{Title in Progress...}}
\author{Boris Andrews}
\affil{Mathematical Institute, University of Oxford}
\date{\today}


\begin{document}
    \pagenumbering{gobble}
    \maketitle
    
    
    \begin{abstract}
        Magnetic confinement reactors---in particular tokamaks---offer one of the most promising options for achieving practical nuclear fusion, with the potential to provide virtually limitless, clean energy. The theoretical and numerical modeling of tokamak plasmas is simultaneously an essential component of effective reactor design, and a great research barrier. Tokamak operational conditions exhibit comparatively low Knudsen numbers. Kinetic effects, including kinetic waves and instabilities, Landau damping, bump-on-tail instabilities and more, are therefore highly influential in tokamak plasma dynamics. Purely fluid models are inherently incapable of capturing these effects, whereas the high dimensionality in purely kinetic models render them practically intractable for most relevant purposes.

        We consider a $\delta\!f$ decomposition model, with a macroscopic fluid background and microscopic kinetic correction, both fully coupled to each other. A similar manner of discretization is proposed to that used in the recent \texttt{STRUPHY} code \cite{Holderied_Possanner_Wang_2021, Holderied_2022, Li_et_al_2023} with a finite-element model for the background and a pseudo-particle/PiC model for the correction.

        The fluid background satisfies the full, non-linear, resistive, compressible, Hall MHD equations. \cite{Laakmann_Hu_Farrell_2022} introduces finite-element(-in-space) implicit timesteppers for the incompressible analogue to this system with structure-preserving (SP) properties in the ideal case, alongside parameter-robust preconditioners. We show that these timesteppers can derive from a finite-element-in-time (FET) (and finite-element-in-space) interpretation. The benefits of this reformulation are discussed, including the derivation of timesteppers that are higher order in time, and the quantifiable dissipative SP properties in the non-ideal, resistive case.
        
        We discuss possible options for extending this FET approach to timesteppers for the compressible case.

        The kinetic corrections satisfy linearized Boltzmann equations. Using a Lénard--Bernstein collision operator, these take Fokker--Planck-like forms \cite{Fokker_1914, Planck_1917} wherein pseudo-particles in the numerical model obey the neoclassical transport equations, with particle-independent Brownian drift terms. This offers a rigorous methodology for incorporating collisions into the particle transport model, without coupling the equations of motions for each particle.
        
        Works by Chen, Chacón et al. \cite{Chen_Chacón_Barnes_2011, Chacón_Chen_Barnes_2013, Chen_Chacón_2014, Chen_Chacón_2015} have developed structure-preserving particle pushers for neoclassical transport in the Vlasov equations, derived from Crank--Nicolson integrators. We show these too can can derive from a FET interpretation, similarly offering potential extensions to higher-order-in-time particle pushers. The FET formulation is used also to consider how the stochastic drift terms can be incorporated into the pushers. Stochastic gyrokinetic expansions are also discussed.

        Different options for the numerical implementation of these schemes are considered.

        Due to the efficacy of FET in the development of SP timesteppers for both the fluid and kinetic component, we hope this approach will prove effective in the future for developing SP timesteppers for the full hybrid model. We hope this will give us the opportunity to incorporate previously inaccessible kinetic effects into the highly effective, modern, finite-element MHD models.
    \end{abstract}
    
    
    \newpage
    \tableofcontents
    
    
    \newpage
    \pagenumbering{arabic}
    %\linenumbers\renewcommand\thelinenumber{\color{black!50}\arabic{linenumber}}
            \input{0 - introduction/main.tex}
        \part{Research}
            \input{1 - low-noise PiC models/main.tex}
            \input{2 - kinetic component/main.tex}
            \input{3 - fluid component/main.tex}
            \input{4 - numerical implementation/main.tex}
        \part{Project Overview}
            \input{5 - research plan/main.tex}
            \input{6 - summary/main.tex}
    
    
    %\section{}
    \newpage
    \pagenumbering{gobble}
        \printbibliography


    \newpage
    \pagenumbering{roman}
    \appendix
        \part{Appendices}
            \input{8 - Hilbert complexes/main.tex}
            \input{9 - weak conservation proofs/main.tex}
\end{document}

            \documentclass[12pt, a4paper]{report}

\input{template/main.tex}

\title{\BA{Title in Progress...}}
\author{Boris Andrews}
\affil{Mathematical Institute, University of Oxford}
\date{\today}


\begin{document}
    \pagenumbering{gobble}
    \maketitle
    
    
    \begin{abstract}
        Magnetic confinement reactors---in particular tokamaks---offer one of the most promising options for achieving practical nuclear fusion, with the potential to provide virtually limitless, clean energy. The theoretical and numerical modeling of tokamak plasmas is simultaneously an essential component of effective reactor design, and a great research barrier. Tokamak operational conditions exhibit comparatively low Knudsen numbers. Kinetic effects, including kinetic waves and instabilities, Landau damping, bump-on-tail instabilities and more, are therefore highly influential in tokamak plasma dynamics. Purely fluid models are inherently incapable of capturing these effects, whereas the high dimensionality in purely kinetic models render them practically intractable for most relevant purposes.

        We consider a $\delta\!f$ decomposition model, with a macroscopic fluid background and microscopic kinetic correction, both fully coupled to each other. A similar manner of discretization is proposed to that used in the recent \texttt{STRUPHY} code \cite{Holderied_Possanner_Wang_2021, Holderied_2022, Li_et_al_2023} with a finite-element model for the background and a pseudo-particle/PiC model for the correction.

        The fluid background satisfies the full, non-linear, resistive, compressible, Hall MHD equations. \cite{Laakmann_Hu_Farrell_2022} introduces finite-element(-in-space) implicit timesteppers for the incompressible analogue to this system with structure-preserving (SP) properties in the ideal case, alongside parameter-robust preconditioners. We show that these timesteppers can derive from a finite-element-in-time (FET) (and finite-element-in-space) interpretation. The benefits of this reformulation are discussed, including the derivation of timesteppers that are higher order in time, and the quantifiable dissipative SP properties in the non-ideal, resistive case.
        
        We discuss possible options for extending this FET approach to timesteppers for the compressible case.

        The kinetic corrections satisfy linearized Boltzmann equations. Using a Lénard--Bernstein collision operator, these take Fokker--Planck-like forms \cite{Fokker_1914, Planck_1917} wherein pseudo-particles in the numerical model obey the neoclassical transport equations, with particle-independent Brownian drift terms. This offers a rigorous methodology for incorporating collisions into the particle transport model, without coupling the equations of motions for each particle.
        
        Works by Chen, Chacón et al. \cite{Chen_Chacón_Barnes_2011, Chacón_Chen_Barnes_2013, Chen_Chacón_2014, Chen_Chacón_2015} have developed structure-preserving particle pushers for neoclassical transport in the Vlasov equations, derived from Crank--Nicolson integrators. We show these too can can derive from a FET interpretation, similarly offering potential extensions to higher-order-in-time particle pushers. The FET formulation is used also to consider how the stochastic drift terms can be incorporated into the pushers. Stochastic gyrokinetic expansions are also discussed.

        Different options for the numerical implementation of these schemes are considered.

        Due to the efficacy of FET in the development of SP timesteppers for both the fluid and kinetic component, we hope this approach will prove effective in the future for developing SP timesteppers for the full hybrid model. We hope this will give us the opportunity to incorporate previously inaccessible kinetic effects into the highly effective, modern, finite-element MHD models.
    \end{abstract}
    
    
    \newpage
    \tableofcontents
    
    
    \newpage
    \pagenumbering{arabic}
    %\linenumbers\renewcommand\thelinenumber{\color{black!50}\arabic{linenumber}}
            \input{0 - introduction/main.tex}
        \part{Research}
            \input{1 - low-noise PiC models/main.tex}
            \input{2 - kinetic component/main.tex}
            \input{3 - fluid component/main.tex}
            \input{4 - numerical implementation/main.tex}
        \part{Project Overview}
            \input{5 - research plan/main.tex}
            \input{6 - summary/main.tex}
    
    
    %\section{}
    \newpage
    \pagenumbering{gobble}
        \printbibliography


    \newpage
    \pagenumbering{roman}
    \appendix
        \part{Appendices}
            \input{8 - Hilbert complexes/main.tex}
            \input{9 - weak conservation proofs/main.tex}
\end{document}

            \documentclass[12pt, a4paper]{report}

\input{template/main.tex}

\title{\BA{Title in Progress...}}
\author{Boris Andrews}
\affil{Mathematical Institute, University of Oxford}
\date{\today}


\begin{document}
    \pagenumbering{gobble}
    \maketitle
    
    
    \begin{abstract}
        Magnetic confinement reactors---in particular tokamaks---offer one of the most promising options for achieving practical nuclear fusion, with the potential to provide virtually limitless, clean energy. The theoretical and numerical modeling of tokamak plasmas is simultaneously an essential component of effective reactor design, and a great research barrier. Tokamak operational conditions exhibit comparatively low Knudsen numbers. Kinetic effects, including kinetic waves and instabilities, Landau damping, bump-on-tail instabilities and more, are therefore highly influential in tokamak plasma dynamics. Purely fluid models are inherently incapable of capturing these effects, whereas the high dimensionality in purely kinetic models render them practically intractable for most relevant purposes.

        We consider a $\delta\!f$ decomposition model, with a macroscopic fluid background and microscopic kinetic correction, both fully coupled to each other. A similar manner of discretization is proposed to that used in the recent \texttt{STRUPHY} code \cite{Holderied_Possanner_Wang_2021, Holderied_2022, Li_et_al_2023} with a finite-element model for the background and a pseudo-particle/PiC model for the correction.

        The fluid background satisfies the full, non-linear, resistive, compressible, Hall MHD equations. \cite{Laakmann_Hu_Farrell_2022} introduces finite-element(-in-space) implicit timesteppers for the incompressible analogue to this system with structure-preserving (SP) properties in the ideal case, alongside parameter-robust preconditioners. We show that these timesteppers can derive from a finite-element-in-time (FET) (and finite-element-in-space) interpretation. The benefits of this reformulation are discussed, including the derivation of timesteppers that are higher order in time, and the quantifiable dissipative SP properties in the non-ideal, resistive case.
        
        We discuss possible options for extending this FET approach to timesteppers for the compressible case.

        The kinetic corrections satisfy linearized Boltzmann equations. Using a Lénard--Bernstein collision operator, these take Fokker--Planck-like forms \cite{Fokker_1914, Planck_1917} wherein pseudo-particles in the numerical model obey the neoclassical transport equations, with particle-independent Brownian drift terms. This offers a rigorous methodology for incorporating collisions into the particle transport model, without coupling the equations of motions for each particle.
        
        Works by Chen, Chacón et al. \cite{Chen_Chacón_Barnes_2011, Chacón_Chen_Barnes_2013, Chen_Chacón_2014, Chen_Chacón_2015} have developed structure-preserving particle pushers for neoclassical transport in the Vlasov equations, derived from Crank--Nicolson integrators. We show these too can can derive from a FET interpretation, similarly offering potential extensions to higher-order-in-time particle pushers. The FET formulation is used also to consider how the stochastic drift terms can be incorporated into the pushers. Stochastic gyrokinetic expansions are also discussed.

        Different options for the numerical implementation of these schemes are considered.

        Due to the efficacy of FET in the development of SP timesteppers for both the fluid and kinetic component, we hope this approach will prove effective in the future for developing SP timesteppers for the full hybrid model. We hope this will give us the opportunity to incorporate previously inaccessible kinetic effects into the highly effective, modern, finite-element MHD models.
    \end{abstract}
    
    
    \newpage
    \tableofcontents
    
    
    \newpage
    \pagenumbering{arabic}
    %\linenumbers\renewcommand\thelinenumber{\color{black!50}\arabic{linenumber}}
            \input{0 - introduction/main.tex}
        \part{Research}
            \input{1 - low-noise PiC models/main.tex}
            \input{2 - kinetic component/main.tex}
            \input{3 - fluid component/main.tex}
            \input{4 - numerical implementation/main.tex}
        \part{Project Overview}
            \input{5 - research plan/main.tex}
            \input{6 - summary/main.tex}
    
    
    %\section{}
    \newpage
    \pagenumbering{gobble}
        \printbibliography


    \newpage
    \pagenumbering{roman}
    \appendix
        \part{Appendices}
            \input{8 - Hilbert complexes/main.tex}
            \input{9 - weak conservation proofs/main.tex}
\end{document}

        \part{Project Overview}
            \documentclass[12pt, a4paper]{report}

\input{template/main.tex}

\title{\BA{Title in Progress...}}
\author{Boris Andrews}
\affil{Mathematical Institute, University of Oxford}
\date{\today}


\begin{document}
    \pagenumbering{gobble}
    \maketitle
    
    
    \begin{abstract}
        Magnetic confinement reactors---in particular tokamaks---offer one of the most promising options for achieving practical nuclear fusion, with the potential to provide virtually limitless, clean energy. The theoretical and numerical modeling of tokamak plasmas is simultaneously an essential component of effective reactor design, and a great research barrier. Tokamak operational conditions exhibit comparatively low Knudsen numbers. Kinetic effects, including kinetic waves and instabilities, Landau damping, bump-on-tail instabilities and more, are therefore highly influential in tokamak plasma dynamics. Purely fluid models are inherently incapable of capturing these effects, whereas the high dimensionality in purely kinetic models render them practically intractable for most relevant purposes.

        We consider a $\delta\!f$ decomposition model, with a macroscopic fluid background and microscopic kinetic correction, both fully coupled to each other. A similar manner of discretization is proposed to that used in the recent \texttt{STRUPHY} code \cite{Holderied_Possanner_Wang_2021, Holderied_2022, Li_et_al_2023} with a finite-element model for the background and a pseudo-particle/PiC model for the correction.

        The fluid background satisfies the full, non-linear, resistive, compressible, Hall MHD equations. \cite{Laakmann_Hu_Farrell_2022} introduces finite-element(-in-space) implicit timesteppers for the incompressible analogue to this system with structure-preserving (SP) properties in the ideal case, alongside parameter-robust preconditioners. We show that these timesteppers can derive from a finite-element-in-time (FET) (and finite-element-in-space) interpretation. The benefits of this reformulation are discussed, including the derivation of timesteppers that are higher order in time, and the quantifiable dissipative SP properties in the non-ideal, resistive case.
        
        We discuss possible options for extending this FET approach to timesteppers for the compressible case.

        The kinetic corrections satisfy linearized Boltzmann equations. Using a Lénard--Bernstein collision operator, these take Fokker--Planck-like forms \cite{Fokker_1914, Planck_1917} wherein pseudo-particles in the numerical model obey the neoclassical transport equations, with particle-independent Brownian drift terms. This offers a rigorous methodology for incorporating collisions into the particle transport model, without coupling the equations of motions for each particle.
        
        Works by Chen, Chacón et al. \cite{Chen_Chacón_Barnes_2011, Chacón_Chen_Barnes_2013, Chen_Chacón_2014, Chen_Chacón_2015} have developed structure-preserving particle pushers for neoclassical transport in the Vlasov equations, derived from Crank--Nicolson integrators. We show these too can can derive from a FET interpretation, similarly offering potential extensions to higher-order-in-time particle pushers. The FET formulation is used also to consider how the stochastic drift terms can be incorporated into the pushers. Stochastic gyrokinetic expansions are also discussed.

        Different options for the numerical implementation of these schemes are considered.

        Due to the efficacy of FET in the development of SP timesteppers for both the fluid and kinetic component, we hope this approach will prove effective in the future for developing SP timesteppers for the full hybrid model. We hope this will give us the opportunity to incorporate previously inaccessible kinetic effects into the highly effective, modern, finite-element MHD models.
    \end{abstract}
    
    
    \newpage
    \tableofcontents
    
    
    \newpage
    \pagenumbering{arabic}
    %\linenumbers\renewcommand\thelinenumber{\color{black!50}\arabic{linenumber}}
            \input{0 - introduction/main.tex}
        \part{Research}
            \input{1 - low-noise PiC models/main.tex}
            \input{2 - kinetic component/main.tex}
            \input{3 - fluid component/main.tex}
            \input{4 - numerical implementation/main.tex}
        \part{Project Overview}
            \input{5 - research plan/main.tex}
            \input{6 - summary/main.tex}
    
    
    %\section{}
    \newpage
    \pagenumbering{gobble}
        \printbibliography


    \newpage
    \pagenumbering{roman}
    \appendix
        \part{Appendices}
            \input{8 - Hilbert complexes/main.tex}
            \input{9 - weak conservation proofs/main.tex}
\end{document}

            \documentclass[12pt, a4paper]{report}

\input{template/main.tex}

\title{\BA{Title in Progress...}}
\author{Boris Andrews}
\affil{Mathematical Institute, University of Oxford}
\date{\today}


\begin{document}
    \pagenumbering{gobble}
    \maketitle
    
    
    \begin{abstract}
        Magnetic confinement reactors---in particular tokamaks---offer one of the most promising options for achieving practical nuclear fusion, with the potential to provide virtually limitless, clean energy. The theoretical and numerical modeling of tokamak plasmas is simultaneously an essential component of effective reactor design, and a great research barrier. Tokamak operational conditions exhibit comparatively low Knudsen numbers. Kinetic effects, including kinetic waves and instabilities, Landau damping, bump-on-tail instabilities and more, are therefore highly influential in tokamak plasma dynamics. Purely fluid models are inherently incapable of capturing these effects, whereas the high dimensionality in purely kinetic models render them practically intractable for most relevant purposes.

        We consider a $\delta\!f$ decomposition model, with a macroscopic fluid background and microscopic kinetic correction, both fully coupled to each other. A similar manner of discretization is proposed to that used in the recent \texttt{STRUPHY} code \cite{Holderied_Possanner_Wang_2021, Holderied_2022, Li_et_al_2023} with a finite-element model for the background and a pseudo-particle/PiC model for the correction.

        The fluid background satisfies the full, non-linear, resistive, compressible, Hall MHD equations. \cite{Laakmann_Hu_Farrell_2022} introduces finite-element(-in-space) implicit timesteppers for the incompressible analogue to this system with structure-preserving (SP) properties in the ideal case, alongside parameter-robust preconditioners. We show that these timesteppers can derive from a finite-element-in-time (FET) (and finite-element-in-space) interpretation. The benefits of this reformulation are discussed, including the derivation of timesteppers that are higher order in time, and the quantifiable dissipative SP properties in the non-ideal, resistive case.
        
        We discuss possible options for extending this FET approach to timesteppers for the compressible case.

        The kinetic corrections satisfy linearized Boltzmann equations. Using a Lénard--Bernstein collision operator, these take Fokker--Planck-like forms \cite{Fokker_1914, Planck_1917} wherein pseudo-particles in the numerical model obey the neoclassical transport equations, with particle-independent Brownian drift terms. This offers a rigorous methodology for incorporating collisions into the particle transport model, without coupling the equations of motions for each particle.
        
        Works by Chen, Chacón et al. \cite{Chen_Chacón_Barnes_2011, Chacón_Chen_Barnes_2013, Chen_Chacón_2014, Chen_Chacón_2015} have developed structure-preserving particle pushers for neoclassical transport in the Vlasov equations, derived from Crank--Nicolson integrators. We show these too can can derive from a FET interpretation, similarly offering potential extensions to higher-order-in-time particle pushers. The FET formulation is used also to consider how the stochastic drift terms can be incorporated into the pushers. Stochastic gyrokinetic expansions are also discussed.

        Different options for the numerical implementation of these schemes are considered.

        Due to the efficacy of FET in the development of SP timesteppers for both the fluid and kinetic component, we hope this approach will prove effective in the future for developing SP timesteppers for the full hybrid model. We hope this will give us the opportunity to incorporate previously inaccessible kinetic effects into the highly effective, modern, finite-element MHD models.
    \end{abstract}
    
    
    \newpage
    \tableofcontents
    
    
    \newpage
    \pagenumbering{arabic}
    %\linenumbers\renewcommand\thelinenumber{\color{black!50}\arabic{linenumber}}
            \input{0 - introduction/main.tex}
        \part{Research}
            \input{1 - low-noise PiC models/main.tex}
            \input{2 - kinetic component/main.tex}
            \input{3 - fluid component/main.tex}
            \input{4 - numerical implementation/main.tex}
        \part{Project Overview}
            \input{5 - research plan/main.tex}
            \input{6 - summary/main.tex}
    
    
    %\section{}
    \newpage
    \pagenumbering{gobble}
        \printbibliography


    \newpage
    \pagenumbering{roman}
    \appendix
        \part{Appendices}
            \input{8 - Hilbert complexes/main.tex}
            \input{9 - weak conservation proofs/main.tex}
\end{document}

    
    
    %\section{}
    \newpage
    \pagenumbering{gobble}
        \printbibliography


    \newpage
    \pagenumbering{roman}
    \appendix
        \part{Appendices}
            \documentclass[12pt, a4paper]{report}

\input{template/main.tex}

\title{\BA{Title in Progress...}}
\author{Boris Andrews}
\affil{Mathematical Institute, University of Oxford}
\date{\today}


\begin{document}
    \pagenumbering{gobble}
    \maketitle
    
    
    \begin{abstract}
        Magnetic confinement reactors---in particular tokamaks---offer one of the most promising options for achieving practical nuclear fusion, with the potential to provide virtually limitless, clean energy. The theoretical and numerical modeling of tokamak plasmas is simultaneously an essential component of effective reactor design, and a great research barrier. Tokamak operational conditions exhibit comparatively low Knudsen numbers. Kinetic effects, including kinetic waves and instabilities, Landau damping, bump-on-tail instabilities and more, are therefore highly influential in tokamak plasma dynamics. Purely fluid models are inherently incapable of capturing these effects, whereas the high dimensionality in purely kinetic models render them practically intractable for most relevant purposes.

        We consider a $\delta\!f$ decomposition model, with a macroscopic fluid background and microscopic kinetic correction, both fully coupled to each other. A similar manner of discretization is proposed to that used in the recent \texttt{STRUPHY} code \cite{Holderied_Possanner_Wang_2021, Holderied_2022, Li_et_al_2023} with a finite-element model for the background and a pseudo-particle/PiC model for the correction.

        The fluid background satisfies the full, non-linear, resistive, compressible, Hall MHD equations. \cite{Laakmann_Hu_Farrell_2022} introduces finite-element(-in-space) implicit timesteppers for the incompressible analogue to this system with structure-preserving (SP) properties in the ideal case, alongside parameter-robust preconditioners. We show that these timesteppers can derive from a finite-element-in-time (FET) (and finite-element-in-space) interpretation. The benefits of this reformulation are discussed, including the derivation of timesteppers that are higher order in time, and the quantifiable dissipative SP properties in the non-ideal, resistive case.
        
        We discuss possible options for extending this FET approach to timesteppers for the compressible case.

        The kinetic corrections satisfy linearized Boltzmann equations. Using a Lénard--Bernstein collision operator, these take Fokker--Planck-like forms \cite{Fokker_1914, Planck_1917} wherein pseudo-particles in the numerical model obey the neoclassical transport equations, with particle-independent Brownian drift terms. This offers a rigorous methodology for incorporating collisions into the particle transport model, without coupling the equations of motions for each particle.
        
        Works by Chen, Chacón et al. \cite{Chen_Chacón_Barnes_2011, Chacón_Chen_Barnes_2013, Chen_Chacón_2014, Chen_Chacón_2015} have developed structure-preserving particle pushers for neoclassical transport in the Vlasov equations, derived from Crank--Nicolson integrators. We show these too can can derive from a FET interpretation, similarly offering potential extensions to higher-order-in-time particle pushers. The FET formulation is used also to consider how the stochastic drift terms can be incorporated into the pushers. Stochastic gyrokinetic expansions are also discussed.

        Different options for the numerical implementation of these schemes are considered.

        Due to the efficacy of FET in the development of SP timesteppers for both the fluid and kinetic component, we hope this approach will prove effective in the future for developing SP timesteppers for the full hybrid model. We hope this will give us the opportunity to incorporate previously inaccessible kinetic effects into the highly effective, modern, finite-element MHD models.
    \end{abstract}
    
    
    \newpage
    \tableofcontents
    
    
    \newpage
    \pagenumbering{arabic}
    %\linenumbers\renewcommand\thelinenumber{\color{black!50}\arabic{linenumber}}
            \input{0 - introduction/main.tex}
        \part{Research}
            \input{1 - low-noise PiC models/main.tex}
            \input{2 - kinetic component/main.tex}
            \input{3 - fluid component/main.tex}
            \input{4 - numerical implementation/main.tex}
        \part{Project Overview}
            \input{5 - research plan/main.tex}
            \input{6 - summary/main.tex}
    
    
    %\section{}
    \newpage
    \pagenumbering{gobble}
        \printbibliography


    \newpage
    \pagenumbering{roman}
    \appendix
        \part{Appendices}
            \input{8 - Hilbert complexes/main.tex}
            \input{9 - weak conservation proofs/main.tex}
\end{document}

            \documentclass[12pt, a4paper]{report}

\input{template/main.tex}

\title{\BA{Title in Progress...}}
\author{Boris Andrews}
\affil{Mathematical Institute, University of Oxford}
\date{\today}


\begin{document}
    \pagenumbering{gobble}
    \maketitle
    
    
    \begin{abstract}
        Magnetic confinement reactors---in particular tokamaks---offer one of the most promising options for achieving practical nuclear fusion, with the potential to provide virtually limitless, clean energy. The theoretical and numerical modeling of tokamak plasmas is simultaneously an essential component of effective reactor design, and a great research barrier. Tokamak operational conditions exhibit comparatively low Knudsen numbers. Kinetic effects, including kinetic waves and instabilities, Landau damping, bump-on-tail instabilities and more, are therefore highly influential in tokamak plasma dynamics. Purely fluid models are inherently incapable of capturing these effects, whereas the high dimensionality in purely kinetic models render them practically intractable for most relevant purposes.

        We consider a $\delta\!f$ decomposition model, with a macroscopic fluid background and microscopic kinetic correction, both fully coupled to each other. A similar manner of discretization is proposed to that used in the recent \texttt{STRUPHY} code \cite{Holderied_Possanner_Wang_2021, Holderied_2022, Li_et_al_2023} with a finite-element model for the background and a pseudo-particle/PiC model for the correction.

        The fluid background satisfies the full, non-linear, resistive, compressible, Hall MHD equations. \cite{Laakmann_Hu_Farrell_2022} introduces finite-element(-in-space) implicit timesteppers for the incompressible analogue to this system with structure-preserving (SP) properties in the ideal case, alongside parameter-robust preconditioners. We show that these timesteppers can derive from a finite-element-in-time (FET) (and finite-element-in-space) interpretation. The benefits of this reformulation are discussed, including the derivation of timesteppers that are higher order in time, and the quantifiable dissipative SP properties in the non-ideal, resistive case.
        
        We discuss possible options for extending this FET approach to timesteppers for the compressible case.

        The kinetic corrections satisfy linearized Boltzmann equations. Using a Lénard--Bernstein collision operator, these take Fokker--Planck-like forms \cite{Fokker_1914, Planck_1917} wherein pseudo-particles in the numerical model obey the neoclassical transport equations, with particle-independent Brownian drift terms. This offers a rigorous methodology for incorporating collisions into the particle transport model, without coupling the equations of motions for each particle.
        
        Works by Chen, Chacón et al. \cite{Chen_Chacón_Barnes_2011, Chacón_Chen_Barnes_2013, Chen_Chacón_2014, Chen_Chacón_2015} have developed structure-preserving particle pushers for neoclassical transport in the Vlasov equations, derived from Crank--Nicolson integrators. We show these too can can derive from a FET interpretation, similarly offering potential extensions to higher-order-in-time particle pushers. The FET formulation is used also to consider how the stochastic drift terms can be incorporated into the pushers. Stochastic gyrokinetic expansions are also discussed.

        Different options for the numerical implementation of these schemes are considered.

        Due to the efficacy of FET in the development of SP timesteppers for both the fluid and kinetic component, we hope this approach will prove effective in the future for developing SP timesteppers for the full hybrid model. We hope this will give us the opportunity to incorporate previously inaccessible kinetic effects into the highly effective, modern, finite-element MHD models.
    \end{abstract}
    
    
    \newpage
    \tableofcontents
    
    
    \newpage
    \pagenumbering{arabic}
    %\linenumbers\renewcommand\thelinenumber{\color{black!50}\arabic{linenumber}}
            \input{0 - introduction/main.tex}
        \part{Research}
            \input{1 - low-noise PiC models/main.tex}
            \input{2 - kinetic component/main.tex}
            \input{3 - fluid component/main.tex}
            \input{4 - numerical implementation/main.tex}
        \part{Project Overview}
            \input{5 - research plan/main.tex}
            \input{6 - summary/main.tex}
    
    
    %\section{}
    \newpage
    \pagenumbering{gobble}
        \printbibliography


    \newpage
    \pagenumbering{roman}
    \appendix
        \part{Appendices}
            \input{8 - Hilbert complexes/main.tex}
            \input{9 - weak conservation proofs/main.tex}
\end{document}

\end{document}

            \documentclass[12pt, a4paper]{report}

\documentclass[12pt, a4paper]{report}

\input{template/main.tex}

\title{\BA{Title in Progress...}}
\author{Boris Andrews}
\affil{Mathematical Institute, University of Oxford}
\date{\today}


\begin{document}
    \pagenumbering{gobble}
    \maketitle
    
    
    \begin{abstract}
        Magnetic confinement reactors---in particular tokamaks---offer one of the most promising options for achieving practical nuclear fusion, with the potential to provide virtually limitless, clean energy. The theoretical and numerical modeling of tokamak plasmas is simultaneously an essential component of effective reactor design, and a great research barrier. Tokamak operational conditions exhibit comparatively low Knudsen numbers. Kinetic effects, including kinetic waves and instabilities, Landau damping, bump-on-tail instabilities and more, are therefore highly influential in tokamak plasma dynamics. Purely fluid models are inherently incapable of capturing these effects, whereas the high dimensionality in purely kinetic models render them practically intractable for most relevant purposes.

        We consider a $\delta\!f$ decomposition model, with a macroscopic fluid background and microscopic kinetic correction, both fully coupled to each other. A similar manner of discretization is proposed to that used in the recent \texttt{STRUPHY} code \cite{Holderied_Possanner_Wang_2021, Holderied_2022, Li_et_al_2023} with a finite-element model for the background and a pseudo-particle/PiC model for the correction.

        The fluid background satisfies the full, non-linear, resistive, compressible, Hall MHD equations. \cite{Laakmann_Hu_Farrell_2022} introduces finite-element(-in-space) implicit timesteppers for the incompressible analogue to this system with structure-preserving (SP) properties in the ideal case, alongside parameter-robust preconditioners. We show that these timesteppers can derive from a finite-element-in-time (FET) (and finite-element-in-space) interpretation. The benefits of this reformulation are discussed, including the derivation of timesteppers that are higher order in time, and the quantifiable dissipative SP properties in the non-ideal, resistive case.
        
        We discuss possible options for extending this FET approach to timesteppers for the compressible case.

        The kinetic corrections satisfy linearized Boltzmann equations. Using a Lénard--Bernstein collision operator, these take Fokker--Planck-like forms \cite{Fokker_1914, Planck_1917} wherein pseudo-particles in the numerical model obey the neoclassical transport equations, with particle-independent Brownian drift terms. This offers a rigorous methodology for incorporating collisions into the particle transport model, without coupling the equations of motions for each particle.
        
        Works by Chen, Chacón et al. \cite{Chen_Chacón_Barnes_2011, Chacón_Chen_Barnes_2013, Chen_Chacón_2014, Chen_Chacón_2015} have developed structure-preserving particle pushers for neoclassical transport in the Vlasov equations, derived from Crank--Nicolson integrators. We show these too can can derive from a FET interpretation, similarly offering potential extensions to higher-order-in-time particle pushers. The FET formulation is used also to consider how the stochastic drift terms can be incorporated into the pushers. Stochastic gyrokinetic expansions are also discussed.

        Different options for the numerical implementation of these schemes are considered.

        Due to the efficacy of FET in the development of SP timesteppers for both the fluid and kinetic component, we hope this approach will prove effective in the future for developing SP timesteppers for the full hybrid model. We hope this will give us the opportunity to incorporate previously inaccessible kinetic effects into the highly effective, modern, finite-element MHD models.
    \end{abstract}
    
    
    \newpage
    \tableofcontents
    
    
    \newpage
    \pagenumbering{arabic}
    %\linenumbers\renewcommand\thelinenumber{\color{black!50}\arabic{linenumber}}
            \input{0 - introduction/main.tex}
        \part{Research}
            \input{1 - low-noise PiC models/main.tex}
            \input{2 - kinetic component/main.tex}
            \input{3 - fluid component/main.tex}
            \input{4 - numerical implementation/main.tex}
        \part{Project Overview}
            \input{5 - research plan/main.tex}
            \input{6 - summary/main.tex}
    
    
    %\section{}
    \newpage
    \pagenumbering{gobble}
        \printbibliography


    \newpage
    \pagenumbering{roman}
    \appendix
        \part{Appendices}
            \input{8 - Hilbert complexes/main.tex}
            \input{9 - weak conservation proofs/main.tex}
\end{document}


\title{\BA{Title in Progress...}}
\author{Boris Andrews}
\affil{Mathematical Institute, University of Oxford}
\date{\today}


\begin{document}
    \pagenumbering{gobble}
    \maketitle
    
    
    \begin{abstract}
        Magnetic confinement reactors---in particular tokamaks---offer one of the most promising options for achieving practical nuclear fusion, with the potential to provide virtually limitless, clean energy. The theoretical and numerical modeling of tokamak plasmas is simultaneously an essential component of effective reactor design, and a great research barrier. Tokamak operational conditions exhibit comparatively low Knudsen numbers. Kinetic effects, including kinetic waves and instabilities, Landau damping, bump-on-tail instabilities and more, are therefore highly influential in tokamak plasma dynamics. Purely fluid models are inherently incapable of capturing these effects, whereas the high dimensionality in purely kinetic models render them practically intractable for most relevant purposes.

        We consider a $\delta\!f$ decomposition model, with a macroscopic fluid background and microscopic kinetic correction, both fully coupled to each other. A similar manner of discretization is proposed to that used in the recent \texttt{STRUPHY} code \cite{Holderied_Possanner_Wang_2021, Holderied_2022, Li_et_al_2023} with a finite-element model for the background and a pseudo-particle/PiC model for the correction.

        The fluid background satisfies the full, non-linear, resistive, compressible, Hall MHD equations. \cite{Laakmann_Hu_Farrell_2022} introduces finite-element(-in-space) implicit timesteppers for the incompressible analogue to this system with structure-preserving (SP) properties in the ideal case, alongside parameter-robust preconditioners. We show that these timesteppers can derive from a finite-element-in-time (FET) (and finite-element-in-space) interpretation. The benefits of this reformulation are discussed, including the derivation of timesteppers that are higher order in time, and the quantifiable dissipative SP properties in the non-ideal, resistive case.
        
        We discuss possible options for extending this FET approach to timesteppers for the compressible case.

        The kinetic corrections satisfy linearized Boltzmann equations. Using a Lénard--Bernstein collision operator, these take Fokker--Planck-like forms \cite{Fokker_1914, Planck_1917} wherein pseudo-particles in the numerical model obey the neoclassical transport equations, with particle-independent Brownian drift terms. This offers a rigorous methodology for incorporating collisions into the particle transport model, without coupling the equations of motions for each particle.
        
        Works by Chen, Chacón et al. \cite{Chen_Chacón_Barnes_2011, Chacón_Chen_Barnes_2013, Chen_Chacón_2014, Chen_Chacón_2015} have developed structure-preserving particle pushers for neoclassical transport in the Vlasov equations, derived from Crank--Nicolson integrators. We show these too can can derive from a FET interpretation, similarly offering potential extensions to higher-order-in-time particle pushers. The FET formulation is used also to consider how the stochastic drift terms can be incorporated into the pushers. Stochastic gyrokinetic expansions are also discussed.

        Different options for the numerical implementation of these schemes are considered.

        Due to the efficacy of FET in the development of SP timesteppers for both the fluid and kinetic component, we hope this approach will prove effective in the future for developing SP timesteppers for the full hybrid model. We hope this will give us the opportunity to incorporate previously inaccessible kinetic effects into the highly effective, modern, finite-element MHD models.
    \end{abstract}
    
    
    \newpage
    \tableofcontents
    
    
    \newpage
    \pagenumbering{arabic}
    %\linenumbers\renewcommand\thelinenumber{\color{black!50}\arabic{linenumber}}
            \documentclass[12pt, a4paper]{report}

\input{template/main.tex}

\title{\BA{Title in Progress...}}
\author{Boris Andrews}
\affil{Mathematical Institute, University of Oxford}
\date{\today}


\begin{document}
    \pagenumbering{gobble}
    \maketitle
    
    
    \begin{abstract}
        Magnetic confinement reactors---in particular tokamaks---offer one of the most promising options for achieving practical nuclear fusion, with the potential to provide virtually limitless, clean energy. The theoretical and numerical modeling of tokamak plasmas is simultaneously an essential component of effective reactor design, and a great research barrier. Tokamak operational conditions exhibit comparatively low Knudsen numbers. Kinetic effects, including kinetic waves and instabilities, Landau damping, bump-on-tail instabilities and more, are therefore highly influential in tokamak plasma dynamics. Purely fluid models are inherently incapable of capturing these effects, whereas the high dimensionality in purely kinetic models render them practically intractable for most relevant purposes.

        We consider a $\delta\!f$ decomposition model, with a macroscopic fluid background and microscopic kinetic correction, both fully coupled to each other. A similar manner of discretization is proposed to that used in the recent \texttt{STRUPHY} code \cite{Holderied_Possanner_Wang_2021, Holderied_2022, Li_et_al_2023} with a finite-element model for the background and a pseudo-particle/PiC model for the correction.

        The fluid background satisfies the full, non-linear, resistive, compressible, Hall MHD equations. \cite{Laakmann_Hu_Farrell_2022} introduces finite-element(-in-space) implicit timesteppers for the incompressible analogue to this system with structure-preserving (SP) properties in the ideal case, alongside parameter-robust preconditioners. We show that these timesteppers can derive from a finite-element-in-time (FET) (and finite-element-in-space) interpretation. The benefits of this reformulation are discussed, including the derivation of timesteppers that are higher order in time, and the quantifiable dissipative SP properties in the non-ideal, resistive case.
        
        We discuss possible options for extending this FET approach to timesteppers for the compressible case.

        The kinetic corrections satisfy linearized Boltzmann equations. Using a Lénard--Bernstein collision operator, these take Fokker--Planck-like forms \cite{Fokker_1914, Planck_1917} wherein pseudo-particles in the numerical model obey the neoclassical transport equations, with particle-independent Brownian drift terms. This offers a rigorous methodology for incorporating collisions into the particle transport model, without coupling the equations of motions for each particle.
        
        Works by Chen, Chacón et al. \cite{Chen_Chacón_Barnes_2011, Chacón_Chen_Barnes_2013, Chen_Chacón_2014, Chen_Chacón_2015} have developed structure-preserving particle pushers for neoclassical transport in the Vlasov equations, derived from Crank--Nicolson integrators. We show these too can can derive from a FET interpretation, similarly offering potential extensions to higher-order-in-time particle pushers. The FET formulation is used also to consider how the stochastic drift terms can be incorporated into the pushers. Stochastic gyrokinetic expansions are also discussed.

        Different options for the numerical implementation of these schemes are considered.

        Due to the efficacy of FET in the development of SP timesteppers for both the fluid and kinetic component, we hope this approach will prove effective in the future for developing SP timesteppers for the full hybrid model. We hope this will give us the opportunity to incorporate previously inaccessible kinetic effects into the highly effective, modern, finite-element MHD models.
    \end{abstract}
    
    
    \newpage
    \tableofcontents
    
    
    \newpage
    \pagenumbering{arabic}
    %\linenumbers\renewcommand\thelinenumber{\color{black!50}\arabic{linenumber}}
            \input{0 - introduction/main.tex}
        \part{Research}
            \input{1 - low-noise PiC models/main.tex}
            \input{2 - kinetic component/main.tex}
            \input{3 - fluid component/main.tex}
            \input{4 - numerical implementation/main.tex}
        \part{Project Overview}
            \input{5 - research plan/main.tex}
            \input{6 - summary/main.tex}
    
    
    %\section{}
    \newpage
    \pagenumbering{gobble}
        \printbibliography


    \newpage
    \pagenumbering{roman}
    \appendix
        \part{Appendices}
            \input{8 - Hilbert complexes/main.tex}
            \input{9 - weak conservation proofs/main.tex}
\end{document}

        \part{Research}
            \documentclass[12pt, a4paper]{report}

\input{template/main.tex}

\title{\BA{Title in Progress...}}
\author{Boris Andrews}
\affil{Mathematical Institute, University of Oxford}
\date{\today}


\begin{document}
    \pagenumbering{gobble}
    \maketitle
    
    
    \begin{abstract}
        Magnetic confinement reactors---in particular tokamaks---offer one of the most promising options for achieving practical nuclear fusion, with the potential to provide virtually limitless, clean energy. The theoretical and numerical modeling of tokamak plasmas is simultaneously an essential component of effective reactor design, and a great research barrier. Tokamak operational conditions exhibit comparatively low Knudsen numbers. Kinetic effects, including kinetic waves and instabilities, Landau damping, bump-on-tail instabilities and more, are therefore highly influential in tokamak plasma dynamics. Purely fluid models are inherently incapable of capturing these effects, whereas the high dimensionality in purely kinetic models render them practically intractable for most relevant purposes.

        We consider a $\delta\!f$ decomposition model, with a macroscopic fluid background and microscopic kinetic correction, both fully coupled to each other. A similar manner of discretization is proposed to that used in the recent \texttt{STRUPHY} code \cite{Holderied_Possanner_Wang_2021, Holderied_2022, Li_et_al_2023} with a finite-element model for the background and a pseudo-particle/PiC model for the correction.

        The fluid background satisfies the full, non-linear, resistive, compressible, Hall MHD equations. \cite{Laakmann_Hu_Farrell_2022} introduces finite-element(-in-space) implicit timesteppers for the incompressible analogue to this system with structure-preserving (SP) properties in the ideal case, alongside parameter-robust preconditioners. We show that these timesteppers can derive from a finite-element-in-time (FET) (and finite-element-in-space) interpretation. The benefits of this reformulation are discussed, including the derivation of timesteppers that are higher order in time, and the quantifiable dissipative SP properties in the non-ideal, resistive case.
        
        We discuss possible options for extending this FET approach to timesteppers for the compressible case.

        The kinetic corrections satisfy linearized Boltzmann equations. Using a Lénard--Bernstein collision operator, these take Fokker--Planck-like forms \cite{Fokker_1914, Planck_1917} wherein pseudo-particles in the numerical model obey the neoclassical transport equations, with particle-independent Brownian drift terms. This offers a rigorous methodology for incorporating collisions into the particle transport model, without coupling the equations of motions for each particle.
        
        Works by Chen, Chacón et al. \cite{Chen_Chacón_Barnes_2011, Chacón_Chen_Barnes_2013, Chen_Chacón_2014, Chen_Chacón_2015} have developed structure-preserving particle pushers for neoclassical transport in the Vlasov equations, derived from Crank--Nicolson integrators. We show these too can can derive from a FET interpretation, similarly offering potential extensions to higher-order-in-time particle pushers. The FET formulation is used also to consider how the stochastic drift terms can be incorporated into the pushers. Stochastic gyrokinetic expansions are also discussed.

        Different options for the numerical implementation of these schemes are considered.

        Due to the efficacy of FET in the development of SP timesteppers for both the fluid and kinetic component, we hope this approach will prove effective in the future for developing SP timesteppers for the full hybrid model. We hope this will give us the opportunity to incorporate previously inaccessible kinetic effects into the highly effective, modern, finite-element MHD models.
    \end{abstract}
    
    
    \newpage
    \tableofcontents
    
    
    \newpage
    \pagenumbering{arabic}
    %\linenumbers\renewcommand\thelinenumber{\color{black!50}\arabic{linenumber}}
            \input{0 - introduction/main.tex}
        \part{Research}
            \input{1 - low-noise PiC models/main.tex}
            \input{2 - kinetic component/main.tex}
            \input{3 - fluid component/main.tex}
            \input{4 - numerical implementation/main.tex}
        \part{Project Overview}
            \input{5 - research plan/main.tex}
            \input{6 - summary/main.tex}
    
    
    %\section{}
    \newpage
    \pagenumbering{gobble}
        \printbibliography


    \newpage
    \pagenumbering{roman}
    \appendix
        \part{Appendices}
            \input{8 - Hilbert complexes/main.tex}
            \input{9 - weak conservation proofs/main.tex}
\end{document}

            \documentclass[12pt, a4paper]{report}

\input{template/main.tex}

\title{\BA{Title in Progress...}}
\author{Boris Andrews}
\affil{Mathematical Institute, University of Oxford}
\date{\today}


\begin{document}
    \pagenumbering{gobble}
    \maketitle
    
    
    \begin{abstract}
        Magnetic confinement reactors---in particular tokamaks---offer one of the most promising options for achieving practical nuclear fusion, with the potential to provide virtually limitless, clean energy. The theoretical and numerical modeling of tokamak plasmas is simultaneously an essential component of effective reactor design, and a great research barrier. Tokamak operational conditions exhibit comparatively low Knudsen numbers. Kinetic effects, including kinetic waves and instabilities, Landau damping, bump-on-tail instabilities and more, are therefore highly influential in tokamak plasma dynamics. Purely fluid models are inherently incapable of capturing these effects, whereas the high dimensionality in purely kinetic models render them practically intractable for most relevant purposes.

        We consider a $\delta\!f$ decomposition model, with a macroscopic fluid background and microscopic kinetic correction, both fully coupled to each other. A similar manner of discretization is proposed to that used in the recent \texttt{STRUPHY} code \cite{Holderied_Possanner_Wang_2021, Holderied_2022, Li_et_al_2023} with a finite-element model for the background and a pseudo-particle/PiC model for the correction.

        The fluid background satisfies the full, non-linear, resistive, compressible, Hall MHD equations. \cite{Laakmann_Hu_Farrell_2022} introduces finite-element(-in-space) implicit timesteppers for the incompressible analogue to this system with structure-preserving (SP) properties in the ideal case, alongside parameter-robust preconditioners. We show that these timesteppers can derive from a finite-element-in-time (FET) (and finite-element-in-space) interpretation. The benefits of this reformulation are discussed, including the derivation of timesteppers that are higher order in time, and the quantifiable dissipative SP properties in the non-ideal, resistive case.
        
        We discuss possible options for extending this FET approach to timesteppers for the compressible case.

        The kinetic corrections satisfy linearized Boltzmann equations. Using a Lénard--Bernstein collision operator, these take Fokker--Planck-like forms \cite{Fokker_1914, Planck_1917} wherein pseudo-particles in the numerical model obey the neoclassical transport equations, with particle-independent Brownian drift terms. This offers a rigorous methodology for incorporating collisions into the particle transport model, without coupling the equations of motions for each particle.
        
        Works by Chen, Chacón et al. \cite{Chen_Chacón_Barnes_2011, Chacón_Chen_Barnes_2013, Chen_Chacón_2014, Chen_Chacón_2015} have developed structure-preserving particle pushers for neoclassical transport in the Vlasov equations, derived from Crank--Nicolson integrators. We show these too can can derive from a FET interpretation, similarly offering potential extensions to higher-order-in-time particle pushers. The FET formulation is used also to consider how the stochastic drift terms can be incorporated into the pushers. Stochastic gyrokinetic expansions are also discussed.

        Different options for the numerical implementation of these schemes are considered.

        Due to the efficacy of FET in the development of SP timesteppers for both the fluid and kinetic component, we hope this approach will prove effective in the future for developing SP timesteppers for the full hybrid model. We hope this will give us the opportunity to incorporate previously inaccessible kinetic effects into the highly effective, modern, finite-element MHD models.
    \end{abstract}
    
    
    \newpage
    \tableofcontents
    
    
    \newpage
    \pagenumbering{arabic}
    %\linenumbers\renewcommand\thelinenumber{\color{black!50}\arabic{linenumber}}
            \input{0 - introduction/main.tex}
        \part{Research}
            \input{1 - low-noise PiC models/main.tex}
            \input{2 - kinetic component/main.tex}
            \input{3 - fluid component/main.tex}
            \input{4 - numerical implementation/main.tex}
        \part{Project Overview}
            \input{5 - research plan/main.tex}
            \input{6 - summary/main.tex}
    
    
    %\section{}
    \newpage
    \pagenumbering{gobble}
        \printbibliography


    \newpage
    \pagenumbering{roman}
    \appendix
        \part{Appendices}
            \input{8 - Hilbert complexes/main.tex}
            \input{9 - weak conservation proofs/main.tex}
\end{document}

            \documentclass[12pt, a4paper]{report}

\input{template/main.tex}

\title{\BA{Title in Progress...}}
\author{Boris Andrews}
\affil{Mathematical Institute, University of Oxford}
\date{\today}


\begin{document}
    \pagenumbering{gobble}
    \maketitle
    
    
    \begin{abstract}
        Magnetic confinement reactors---in particular tokamaks---offer one of the most promising options for achieving practical nuclear fusion, with the potential to provide virtually limitless, clean energy. The theoretical and numerical modeling of tokamak plasmas is simultaneously an essential component of effective reactor design, and a great research barrier. Tokamak operational conditions exhibit comparatively low Knudsen numbers. Kinetic effects, including kinetic waves and instabilities, Landau damping, bump-on-tail instabilities and more, are therefore highly influential in tokamak plasma dynamics. Purely fluid models are inherently incapable of capturing these effects, whereas the high dimensionality in purely kinetic models render them practically intractable for most relevant purposes.

        We consider a $\delta\!f$ decomposition model, with a macroscopic fluid background and microscopic kinetic correction, both fully coupled to each other. A similar manner of discretization is proposed to that used in the recent \texttt{STRUPHY} code \cite{Holderied_Possanner_Wang_2021, Holderied_2022, Li_et_al_2023} with a finite-element model for the background and a pseudo-particle/PiC model for the correction.

        The fluid background satisfies the full, non-linear, resistive, compressible, Hall MHD equations. \cite{Laakmann_Hu_Farrell_2022} introduces finite-element(-in-space) implicit timesteppers for the incompressible analogue to this system with structure-preserving (SP) properties in the ideal case, alongside parameter-robust preconditioners. We show that these timesteppers can derive from a finite-element-in-time (FET) (and finite-element-in-space) interpretation. The benefits of this reformulation are discussed, including the derivation of timesteppers that are higher order in time, and the quantifiable dissipative SP properties in the non-ideal, resistive case.
        
        We discuss possible options for extending this FET approach to timesteppers for the compressible case.

        The kinetic corrections satisfy linearized Boltzmann equations. Using a Lénard--Bernstein collision operator, these take Fokker--Planck-like forms \cite{Fokker_1914, Planck_1917} wherein pseudo-particles in the numerical model obey the neoclassical transport equations, with particle-independent Brownian drift terms. This offers a rigorous methodology for incorporating collisions into the particle transport model, without coupling the equations of motions for each particle.
        
        Works by Chen, Chacón et al. \cite{Chen_Chacón_Barnes_2011, Chacón_Chen_Barnes_2013, Chen_Chacón_2014, Chen_Chacón_2015} have developed structure-preserving particle pushers for neoclassical transport in the Vlasov equations, derived from Crank--Nicolson integrators. We show these too can can derive from a FET interpretation, similarly offering potential extensions to higher-order-in-time particle pushers. The FET formulation is used also to consider how the stochastic drift terms can be incorporated into the pushers. Stochastic gyrokinetic expansions are also discussed.

        Different options for the numerical implementation of these schemes are considered.

        Due to the efficacy of FET in the development of SP timesteppers for both the fluid and kinetic component, we hope this approach will prove effective in the future for developing SP timesteppers for the full hybrid model. We hope this will give us the opportunity to incorporate previously inaccessible kinetic effects into the highly effective, modern, finite-element MHD models.
    \end{abstract}
    
    
    \newpage
    \tableofcontents
    
    
    \newpage
    \pagenumbering{arabic}
    %\linenumbers\renewcommand\thelinenumber{\color{black!50}\arabic{linenumber}}
            \input{0 - introduction/main.tex}
        \part{Research}
            \input{1 - low-noise PiC models/main.tex}
            \input{2 - kinetic component/main.tex}
            \input{3 - fluid component/main.tex}
            \input{4 - numerical implementation/main.tex}
        \part{Project Overview}
            \input{5 - research plan/main.tex}
            \input{6 - summary/main.tex}
    
    
    %\section{}
    \newpage
    \pagenumbering{gobble}
        \printbibliography


    \newpage
    \pagenumbering{roman}
    \appendix
        \part{Appendices}
            \input{8 - Hilbert complexes/main.tex}
            \input{9 - weak conservation proofs/main.tex}
\end{document}

            \documentclass[12pt, a4paper]{report}

\input{template/main.tex}

\title{\BA{Title in Progress...}}
\author{Boris Andrews}
\affil{Mathematical Institute, University of Oxford}
\date{\today}


\begin{document}
    \pagenumbering{gobble}
    \maketitle
    
    
    \begin{abstract}
        Magnetic confinement reactors---in particular tokamaks---offer one of the most promising options for achieving practical nuclear fusion, with the potential to provide virtually limitless, clean energy. The theoretical and numerical modeling of tokamak plasmas is simultaneously an essential component of effective reactor design, and a great research barrier. Tokamak operational conditions exhibit comparatively low Knudsen numbers. Kinetic effects, including kinetic waves and instabilities, Landau damping, bump-on-tail instabilities and more, are therefore highly influential in tokamak plasma dynamics. Purely fluid models are inherently incapable of capturing these effects, whereas the high dimensionality in purely kinetic models render them practically intractable for most relevant purposes.

        We consider a $\delta\!f$ decomposition model, with a macroscopic fluid background and microscopic kinetic correction, both fully coupled to each other. A similar manner of discretization is proposed to that used in the recent \texttt{STRUPHY} code \cite{Holderied_Possanner_Wang_2021, Holderied_2022, Li_et_al_2023} with a finite-element model for the background and a pseudo-particle/PiC model for the correction.

        The fluid background satisfies the full, non-linear, resistive, compressible, Hall MHD equations. \cite{Laakmann_Hu_Farrell_2022} introduces finite-element(-in-space) implicit timesteppers for the incompressible analogue to this system with structure-preserving (SP) properties in the ideal case, alongside parameter-robust preconditioners. We show that these timesteppers can derive from a finite-element-in-time (FET) (and finite-element-in-space) interpretation. The benefits of this reformulation are discussed, including the derivation of timesteppers that are higher order in time, and the quantifiable dissipative SP properties in the non-ideal, resistive case.
        
        We discuss possible options for extending this FET approach to timesteppers for the compressible case.

        The kinetic corrections satisfy linearized Boltzmann equations. Using a Lénard--Bernstein collision operator, these take Fokker--Planck-like forms \cite{Fokker_1914, Planck_1917} wherein pseudo-particles in the numerical model obey the neoclassical transport equations, with particle-independent Brownian drift terms. This offers a rigorous methodology for incorporating collisions into the particle transport model, without coupling the equations of motions for each particle.
        
        Works by Chen, Chacón et al. \cite{Chen_Chacón_Barnes_2011, Chacón_Chen_Barnes_2013, Chen_Chacón_2014, Chen_Chacón_2015} have developed structure-preserving particle pushers for neoclassical transport in the Vlasov equations, derived from Crank--Nicolson integrators. We show these too can can derive from a FET interpretation, similarly offering potential extensions to higher-order-in-time particle pushers. The FET formulation is used also to consider how the stochastic drift terms can be incorporated into the pushers. Stochastic gyrokinetic expansions are also discussed.

        Different options for the numerical implementation of these schemes are considered.

        Due to the efficacy of FET in the development of SP timesteppers for both the fluid and kinetic component, we hope this approach will prove effective in the future for developing SP timesteppers for the full hybrid model. We hope this will give us the opportunity to incorporate previously inaccessible kinetic effects into the highly effective, modern, finite-element MHD models.
    \end{abstract}
    
    
    \newpage
    \tableofcontents
    
    
    \newpage
    \pagenumbering{arabic}
    %\linenumbers\renewcommand\thelinenumber{\color{black!50}\arabic{linenumber}}
            \input{0 - introduction/main.tex}
        \part{Research}
            \input{1 - low-noise PiC models/main.tex}
            \input{2 - kinetic component/main.tex}
            \input{3 - fluid component/main.tex}
            \input{4 - numerical implementation/main.tex}
        \part{Project Overview}
            \input{5 - research plan/main.tex}
            \input{6 - summary/main.tex}
    
    
    %\section{}
    \newpage
    \pagenumbering{gobble}
        \printbibliography


    \newpage
    \pagenumbering{roman}
    \appendix
        \part{Appendices}
            \input{8 - Hilbert complexes/main.tex}
            \input{9 - weak conservation proofs/main.tex}
\end{document}

        \part{Project Overview}
            \documentclass[12pt, a4paper]{report}

\input{template/main.tex}

\title{\BA{Title in Progress...}}
\author{Boris Andrews}
\affil{Mathematical Institute, University of Oxford}
\date{\today}


\begin{document}
    \pagenumbering{gobble}
    \maketitle
    
    
    \begin{abstract}
        Magnetic confinement reactors---in particular tokamaks---offer one of the most promising options for achieving practical nuclear fusion, with the potential to provide virtually limitless, clean energy. The theoretical and numerical modeling of tokamak plasmas is simultaneously an essential component of effective reactor design, and a great research barrier. Tokamak operational conditions exhibit comparatively low Knudsen numbers. Kinetic effects, including kinetic waves and instabilities, Landau damping, bump-on-tail instabilities and more, are therefore highly influential in tokamak plasma dynamics. Purely fluid models are inherently incapable of capturing these effects, whereas the high dimensionality in purely kinetic models render them practically intractable for most relevant purposes.

        We consider a $\delta\!f$ decomposition model, with a macroscopic fluid background and microscopic kinetic correction, both fully coupled to each other. A similar manner of discretization is proposed to that used in the recent \texttt{STRUPHY} code \cite{Holderied_Possanner_Wang_2021, Holderied_2022, Li_et_al_2023} with a finite-element model for the background and a pseudo-particle/PiC model for the correction.

        The fluid background satisfies the full, non-linear, resistive, compressible, Hall MHD equations. \cite{Laakmann_Hu_Farrell_2022} introduces finite-element(-in-space) implicit timesteppers for the incompressible analogue to this system with structure-preserving (SP) properties in the ideal case, alongside parameter-robust preconditioners. We show that these timesteppers can derive from a finite-element-in-time (FET) (and finite-element-in-space) interpretation. The benefits of this reformulation are discussed, including the derivation of timesteppers that are higher order in time, and the quantifiable dissipative SP properties in the non-ideal, resistive case.
        
        We discuss possible options for extending this FET approach to timesteppers for the compressible case.

        The kinetic corrections satisfy linearized Boltzmann equations. Using a Lénard--Bernstein collision operator, these take Fokker--Planck-like forms \cite{Fokker_1914, Planck_1917} wherein pseudo-particles in the numerical model obey the neoclassical transport equations, with particle-independent Brownian drift terms. This offers a rigorous methodology for incorporating collisions into the particle transport model, without coupling the equations of motions for each particle.
        
        Works by Chen, Chacón et al. \cite{Chen_Chacón_Barnes_2011, Chacón_Chen_Barnes_2013, Chen_Chacón_2014, Chen_Chacón_2015} have developed structure-preserving particle pushers for neoclassical transport in the Vlasov equations, derived from Crank--Nicolson integrators. We show these too can can derive from a FET interpretation, similarly offering potential extensions to higher-order-in-time particle pushers. The FET formulation is used also to consider how the stochastic drift terms can be incorporated into the pushers. Stochastic gyrokinetic expansions are also discussed.

        Different options for the numerical implementation of these schemes are considered.

        Due to the efficacy of FET in the development of SP timesteppers for both the fluid and kinetic component, we hope this approach will prove effective in the future for developing SP timesteppers for the full hybrid model. We hope this will give us the opportunity to incorporate previously inaccessible kinetic effects into the highly effective, modern, finite-element MHD models.
    \end{abstract}
    
    
    \newpage
    \tableofcontents
    
    
    \newpage
    \pagenumbering{arabic}
    %\linenumbers\renewcommand\thelinenumber{\color{black!50}\arabic{linenumber}}
            \input{0 - introduction/main.tex}
        \part{Research}
            \input{1 - low-noise PiC models/main.tex}
            \input{2 - kinetic component/main.tex}
            \input{3 - fluid component/main.tex}
            \input{4 - numerical implementation/main.tex}
        \part{Project Overview}
            \input{5 - research plan/main.tex}
            \input{6 - summary/main.tex}
    
    
    %\section{}
    \newpage
    \pagenumbering{gobble}
        \printbibliography


    \newpage
    \pagenumbering{roman}
    \appendix
        \part{Appendices}
            \input{8 - Hilbert complexes/main.tex}
            \input{9 - weak conservation proofs/main.tex}
\end{document}

            \documentclass[12pt, a4paper]{report}

\input{template/main.tex}

\title{\BA{Title in Progress...}}
\author{Boris Andrews}
\affil{Mathematical Institute, University of Oxford}
\date{\today}


\begin{document}
    \pagenumbering{gobble}
    \maketitle
    
    
    \begin{abstract}
        Magnetic confinement reactors---in particular tokamaks---offer one of the most promising options for achieving practical nuclear fusion, with the potential to provide virtually limitless, clean energy. The theoretical and numerical modeling of tokamak plasmas is simultaneously an essential component of effective reactor design, and a great research barrier. Tokamak operational conditions exhibit comparatively low Knudsen numbers. Kinetic effects, including kinetic waves and instabilities, Landau damping, bump-on-tail instabilities and more, are therefore highly influential in tokamak plasma dynamics. Purely fluid models are inherently incapable of capturing these effects, whereas the high dimensionality in purely kinetic models render them practically intractable for most relevant purposes.

        We consider a $\delta\!f$ decomposition model, with a macroscopic fluid background and microscopic kinetic correction, both fully coupled to each other. A similar manner of discretization is proposed to that used in the recent \texttt{STRUPHY} code \cite{Holderied_Possanner_Wang_2021, Holderied_2022, Li_et_al_2023} with a finite-element model for the background and a pseudo-particle/PiC model for the correction.

        The fluid background satisfies the full, non-linear, resistive, compressible, Hall MHD equations. \cite{Laakmann_Hu_Farrell_2022} introduces finite-element(-in-space) implicit timesteppers for the incompressible analogue to this system with structure-preserving (SP) properties in the ideal case, alongside parameter-robust preconditioners. We show that these timesteppers can derive from a finite-element-in-time (FET) (and finite-element-in-space) interpretation. The benefits of this reformulation are discussed, including the derivation of timesteppers that are higher order in time, and the quantifiable dissipative SP properties in the non-ideal, resistive case.
        
        We discuss possible options for extending this FET approach to timesteppers for the compressible case.

        The kinetic corrections satisfy linearized Boltzmann equations. Using a Lénard--Bernstein collision operator, these take Fokker--Planck-like forms \cite{Fokker_1914, Planck_1917} wherein pseudo-particles in the numerical model obey the neoclassical transport equations, with particle-independent Brownian drift terms. This offers a rigorous methodology for incorporating collisions into the particle transport model, without coupling the equations of motions for each particle.
        
        Works by Chen, Chacón et al. \cite{Chen_Chacón_Barnes_2011, Chacón_Chen_Barnes_2013, Chen_Chacón_2014, Chen_Chacón_2015} have developed structure-preserving particle pushers for neoclassical transport in the Vlasov equations, derived from Crank--Nicolson integrators. We show these too can can derive from a FET interpretation, similarly offering potential extensions to higher-order-in-time particle pushers. The FET formulation is used also to consider how the stochastic drift terms can be incorporated into the pushers. Stochastic gyrokinetic expansions are also discussed.

        Different options for the numerical implementation of these schemes are considered.

        Due to the efficacy of FET in the development of SP timesteppers for both the fluid and kinetic component, we hope this approach will prove effective in the future for developing SP timesteppers for the full hybrid model. We hope this will give us the opportunity to incorporate previously inaccessible kinetic effects into the highly effective, modern, finite-element MHD models.
    \end{abstract}
    
    
    \newpage
    \tableofcontents
    
    
    \newpage
    \pagenumbering{arabic}
    %\linenumbers\renewcommand\thelinenumber{\color{black!50}\arabic{linenumber}}
            \input{0 - introduction/main.tex}
        \part{Research}
            \input{1 - low-noise PiC models/main.tex}
            \input{2 - kinetic component/main.tex}
            \input{3 - fluid component/main.tex}
            \input{4 - numerical implementation/main.tex}
        \part{Project Overview}
            \input{5 - research plan/main.tex}
            \input{6 - summary/main.tex}
    
    
    %\section{}
    \newpage
    \pagenumbering{gobble}
        \printbibliography


    \newpage
    \pagenumbering{roman}
    \appendix
        \part{Appendices}
            \input{8 - Hilbert complexes/main.tex}
            \input{9 - weak conservation proofs/main.tex}
\end{document}

    
    
    %\section{}
    \newpage
    \pagenumbering{gobble}
        \printbibliography


    \newpage
    \pagenumbering{roman}
    \appendix
        \part{Appendices}
            \documentclass[12pt, a4paper]{report}

\input{template/main.tex}

\title{\BA{Title in Progress...}}
\author{Boris Andrews}
\affil{Mathematical Institute, University of Oxford}
\date{\today}


\begin{document}
    \pagenumbering{gobble}
    \maketitle
    
    
    \begin{abstract}
        Magnetic confinement reactors---in particular tokamaks---offer one of the most promising options for achieving practical nuclear fusion, with the potential to provide virtually limitless, clean energy. The theoretical and numerical modeling of tokamak plasmas is simultaneously an essential component of effective reactor design, and a great research barrier. Tokamak operational conditions exhibit comparatively low Knudsen numbers. Kinetic effects, including kinetic waves and instabilities, Landau damping, bump-on-tail instabilities and more, are therefore highly influential in tokamak plasma dynamics. Purely fluid models are inherently incapable of capturing these effects, whereas the high dimensionality in purely kinetic models render them practically intractable for most relevant purposes.

        We consider a $\delta\!f$ decomposition model, with a macroscopic fluid background and microscopic kinetic correction, both fully coupled to each other. A similar manner of discretization is proposed to that used in the recent \texttt{STRUPHY} code \cite{Holderied_Possanner_Wang_2021, Holderied_2022, Li_et_al_2023} with a finite-element model for the background and a pseudo-particle/PiC model for the correction.

        The fluid background satisfies the full, non-linear, resistive, compressible, Hall MHD equations. \cite{Laakmann_Hu_Farrell_2022} introduces finite-element(-in-space) implicit timesteppers for the incompressible analogue to this system with structure-preserving (SP) properties in the ideal case, alongside parameter-robust preconditioners. We show that these timesteppers can derive from a finite-element-in-time (FET) (and finite-element-in-space) interpretation. The benefits of this reformulation are discussed, including the derivation of timesteppers that are higher order in time, and the quantifiable dissipative SP properties in the non-ideal, resistive case.
        
        We discuss possible options for extending this FET approach to timesteppers for the compressible case.

        The kinetic corrections satisfy linearized Boltzmann equations. Using a Lénard--Bernstein collision operator, these take Fokker--Planck-like forms \cite{Fokker_1914, Planck_1917} wherein pseudo-particles in the numerical model obey the neoclassical transport equations, with particle-independent Brownian drift terms. This offers a rigorous methodology for incorporating collisions into the particle transport model, without coupling the equations of motions for each particle.
        
        Works by Chen, Chacón et al. \cite{Chen_Chacón_Barnes_2011, Chacón_Chen_Barnes_2013, Chen_Chacón_2014, Chen_Chacón_2015} have developed structure-preserving particle pushers for neoclassical transport in the Vlasov equations, derived from Crank--Nicolson integrators. We show these too can can derive from a FET interpretation, similarly offering potential extensions to higher-order-in-time particle pushers. The FET formulation is used also to consider how the stochastic drift terms can be incorporated into the pushers. Stochastic gyrokinetic expansions are also discussed.

        Different options for the numerical implementation of these schemes are considered.

        Due to the efficacy of FET in the development of SP timesteppers for both the fluid and kinetic component, we hope this approach will prove effective in the future for developing SP timesteppers for the full hybrid model. We hope this will give us the opportunity to incorporate previously inaccessible kinetic effects into the highly effective, modern, finite-element MHD models.
    \end{abstract}
    
    
    \newpage
    \tableofcontents
    
    
    \newpage
    \pagenumbering{arabic}
    %\linenumbers\renewcommand\thelinenumber{\color{black!50}\arabic{linenumber}}
            \input{0 - introduction/main.tex}
        \part{Research}
            \input{1 - low-noise PiC models/main.tex}
            \input{2 - kinetic component/main.tex}
            \input{3 - fluid component/main.tex}
            \input{4 - numerical implementation/main.tex}
        \part{Project Overview}
            \input{5 - research plan/main.tex}
            \input{6 - summary/main.tex}
    
    
    %\section{}
    \newpage
    \pagenumbering{gobble}
        \printbibliography


    \newpage
    \pagenumbering{roman}
    \appendix
        \part{Appendices}
            \input{8 - Hilbert complexes/main.tex}
            \input{9 - weak conservation proofs/main.tex}
\end{document}

            \documentclass[12pt, a4paper]{report}

\input{template/main.tex}

\title{\BA{Title in Progress...}}
\author{Boris Andrews}
\affil{Mathematical Institute, University of Oxford}
\date{\today}


\begin{document}
    \pagenumbering{gobble}
    \maketitle
    
    
    \begin{abstract}
        Magnetic confinement reactors---in particular tokamaks---offer one of the most promising options for achieving practical nuclear fusion, with the potential to provide virtually limitless, clean energy. The theoretical and numerical modeling of tokamak plasmas is simultaneously an essential component of effective reactor design, and a great research barrier. Tokamak operational conditions exhibit comparatively low Knudsen numbers. Kinetic effects, including kinetic waves and instabilities, Landau damping, bump-on-tail instabilities and more, are therefore highly influential in tokamak plasma dynamics. Purely fluid models are inherently incapable of capturing these effects, whereas the high dimensionality in purely kinetic models render them practically intractable for most relevant purposes.

        We consider a $\delta\!f$ decomposition model, with a macroscopic fluid background and microscopic kinetic correction, both fully coupled to each other. A similar manner of discretization is proposed to that used in the recent \texttt{STRUPHY} code \cite{Holderied_Possanner_Wang_2021, Holderied_2022, Li_et_al_2023} with a finite-element model for the background and a pseudo-particle/PiC model for the correction.

        The fluid background satisfies the full, non-linear, resistive, compressible, Hall MHD equations. \cite{Laakmann_Hu_Farrell_2022} introduces finite-element(-in-space) implicit timesteppers for the incompressible analogue to this system with structure-preserving (SP) properties in the ideal case, alongside parameter-robust preconditioners. We show that these timesteppers can derive from a finite-element-in-time (FET) (and finite-element-in-space) interpretation. The benefits of this reformulation are discussed, including the derivation of timesteppers that are higher order in time, and the quantifiable dissipative SP properties in the non-ideal, resistive case.
        
        We discuss possible options for extending this FET approach to timesteppers for the compressible case.

        The kinetic corrections satisfy linearized Boltzmann equations. Using a Lénard--Bernstein collision operator, these take Fokker--Planck-like forms \cite{Fokker_1914, Planck_1917} wherein pseudo-particles in the numerical model obey the neoclassical transport equations, with particle-independent Brownian drift terms. This offers a rigorous methodology for incorporating collisions into the particle transport model, without coupling the equations of motions for each particle.
        
        Works by Chen, Chacón et al. \cite{Chen_Chacón_Barnes_2011, Chacón_Chen_Barnes_2013, Chen_Chacón_2014, Chen_Chacón_2015} have developed structure-preserving particle pushers for neoclassical transport in the Vlasov equations, derived from Crank--Nicolson integrators. We show these too can can derive from a FET interpretation, similarly offering potential extensions to higher-order-in-time particle pushers. The FET formulation is used also to consider how the stochastic drift terms can be incorporated into the pushers. Stochastic gyrokinetic expansions are also discussed.

        Different options for the numerical implementation of these schemes are considered.

        Due to the efficacy of FET in the development of SP timesteppers for both the fluid and kinetic component, we hope this approach will prove effective in the future for developing SP timesteppers for the full hybrid model. We hope this will give us the opportunity to incorporate previously inaccessible kinetic effects into the highly effective, modern, finite-element MHD models.
    \end{abstract}
    
    
    \newpage
    \tableofcontents
    
    
    \newpage
    \pagenumbering{arabic}
    %\linenumbers\renewcommand\thelinenumber{\color{black!50}\arabic{linenumber}}
            \input{0 - introduction/main.tex}
        \part{Research}
            \input{1 - low-noise PiC models/main.tex}
            \input{2 - kinetic component/main.tex}
            \input{3 - fluid component/main.tex}
            \input{4 - numerical implementation/main.tex}
        \part{Project Overview}
            \input{5 - research plan/main.tex}
            \input{6 - summary/main.tex}
    
    
    %\section{}
    \newpage
    \pagenumbering{gobble}
        \printbibliography


    \newpage
    \pagenumbering{roman}
    \appendix
        \part{Appendices}
            \input{8 - Hilbert complexes/main.tex}
            \input{9 - weak conservation proofs/main.tex}
\end{document}

\end{document}

            \documentclass[12pt, a4paper]{report}

\documentclass[12pt, a4paper]{report}

\input{template/main.tex}

\title{\BA{Title in Progress...}}
\author{Boris Andrews}
\affil{Mathematical Institute, University of Oxford}
\date{\today}


\begin{document}
    \pagenumbering{gobble}
    \maketitle
    
    
    \begin{abstract}
        Magnetic confinement reactors---in particular tokamaks---offer one of the most promising options for achieving practical nuclear fusion, with the potential to provide virtually limitless, clean energy. The theoretical and numerical modeling of tokamak plasmas is simultaneously an essential component of effective reactor design, and a great research barrier. Tokamak operational conditions exhibit comparatively low Knudsen numbers. Kinetic effects, including kinetic waves and instabilities, Landau damping, bump-on-tail instabilities and more, are therefore highly influential in tokamak plasma dynamics. Purely fluid models are inherently incapable of capturing these effects, whereas the high dimensionality in purely kinetic models render them practically intractable for most relevant purposes.

        We consider a $\delta\!f$ decomposition model, with a macroscopic fluid background and microscopic kinetic correction, both fully coupled to each other. A similar manner of discretization is proposed to that used in the recent \texttt{STRUPHY} code \cite{Holderied_Possanner_Wang_2021, Holderied_2022, Li_et_al_2023} with a finite-element model for the background and a pseudo-particle/PiC model for the correction.

        The fluid background satisfies the full, non-linear, resistive, compressible, Hall MHD equations. \cite{Laakmann_Hu_Farrell_2022} introduces finite-element(-in-space) implicit timesteppers for the incompressible analogue to this system with structure-preserving (SP) properties in the ideal case, alongside parameter-robust preconditioners. We show that these timesteppers can derive from a finite-element-in-time (FET) (and finite-element-in-space) interpretation. The benefits of this reformulation are discussed, including the derivation of timesteppers that are higher order in time, and the quantifiable dissipative SP properties in the non-ideal, resistive case.
        
        We discuss possible options for extending this FET approach to timesteppers for the compressible case.

        The kinetic corrections satisfy linearized Boltzmann equations. Using a Lénard--Bernstein collision operator, these take Fokker--Planck-like forms \cite{Fokker_1914, Planck_1917} wherein pseudo-particles in the numerical model obey the neoclassical transport equations, with particle-independent Brownian drift terms. This offers a rigorous methodology for incorporating collisions into the particle transport model, without coupling the equations of motions for each particle.
        
        Works by Chen, Chacón et al. \cite{Chen_Chacón_Barnes_2011, Chacón_Chen_Barnes_2013, Chen_Chacón_2014, Chen_Chacón_2015} have developed structure-preserving particle pushers for neoclassical transport in the Vlasov equations, derived from Crank--Nicolson integrators. We show these too can can derive from a FET interpretation, similarly offering potential extensions to higher-order-in-time particle pushers. The FET formulation is used also to consider how the stochastic drift terms can be incorporated into the pushers. Stochastic gyrokinetic expansions are also discussed.

        Different options for the numerical implementation of these schemes are considered.

        Due to the efficacy of FET in the development of SP timesteppers for both the fluid and kinetic component, we hope this approach will prove effective in the future for developing SP timesteppers for the full hybrid model. We hope this will give us the opportunity to incorporate previously inaccessible kinetic effects into the highly effective, modern, finite-element MHD models.
    \end{abstract}
    
    
    \newpage
    \tableofcontents
    
    
    \newpage
    \pagenumbering{arabic}
    %\linenumbers\renewcommand\thelinenumber{\color{black!50}\arabic{linenumber}}
            \input{0 - introduction/main.tex}
        \part{Research}
            \input{1 - low-noise PiC models/main.tex}
            \input{2 - kinetic component/main.tex}
            \input{3 - fluid component/main.tex}
            \input{4 - numerical implementation/main.tex}
        \part{Project Overview}
            \input{5 - research plan/main.tex}
            \input{6 - summary/main.tex}
    
    
    %\section{}
    \newpage
    \pagenumbering{gobble}
        \printbibliography


    \newpage
    \pagenumbering{roman}
    \appendix
        \part{Appendices}
            \input{8 - Hilbert complexes/main.tex}
            \input{9 - weak conservation proofs/main.tex}
\end{document}


\title{\BA{Title in Progress...}}
\author{Boris Andrews}
\affil{Mathematical Institute, University of Oxford}
\date{\today}


\begin{document}
    \pagenumbering{gobble}
    \maketitle
    
    
    \begin{abstract}
        Magnetic confinement reactors---in particular tokamaks---offer one of the most promising options for achieving practical nuclear fusion, with the potential to provide virtually limitless, clean energy. The theoretical and numerical modeling of tokamak plasmas is simultaneously an essential component of effective reactor design, and a great research barrier. Tokamak operational conditions exhibit comparatively low Knudsen numbers. Kinetic effects, including kinetic waves and instabilities, Landau damping, bump-on-tail instabilities and more, are therefore highly influential in tokamak plasma dynamics. Purely fluid models are inherently incapable of capturing these effects, whereas the high dimensionality in purely kinetic models render them practically intractable for most relevant purposes.

        We consider a $\delta\!f$ decomposition model, with a macroscopic fluid background and microscopic kinetic correction, both fully coupled to each other. A similar manner of discretization is proposed to that used in the recent \texttt{STRUPHY} code \cite{Holderied_Possanner_Wang_2021, Holderied_2022, Li_et_al_2023} with a finite-element model for the background and a pseudo-particle/PiC model for the correction.

        The fluid background satisfies the full, non-linear, resistive, compressible, Hall MHD equations. \cite{Laakmann_Hu_Farrell_2022} introduces finite-element(-in-space) implicit timesteppers for the incompressible analogue to this system with structure-preserving (SP) properties in the ideal case, alongside parameter-robust preconditioners. We show that these timesteppers can derive from a finite-element-in-time (FET) (and finite-element-in-space) interpretation. The benefits of this reformulation are discussed, including the derivation of timesteppers that are higher order in time, and the quantifiable dissipative SP properties in the non-ideal, resistive case.
        
        We discuss possible options for extending this FET approach to timesteppers for the compressible case.

        The kinetic corrections satisfy linearized Boltzmann equations. Using a Lénard--Bernstein collision operator, these take Fokker--Planck-like forms \cite{Fokker_1914, Planck_1917} wherein pseudo-particles in the numerical model obey the neoclassical transport equations, with particle-independent Brownian drift terms. This offers a rigorous methodology for incorporating collisions into the particle transport model, without coupling the equations of motions for each particle.
        
        Works by Chen, Chacón et al. \cite{Chen_Chacón_Barnes_2011, Chacón_Chen_Barnes_2013, Chen_Chacón_2014, Chen_Chacón_2015} have developed structure-preserving particle pushers for neoclassical transport in the Vlasov equations, derived from Crank--Nicolson integrators. We show these too can can derive from a FET interpretation, similarly offering potential extensions to higher-order-in-time particle pushers. The FET formulation is used also to consider how the stochastic drift terms can be incorporated into the pushers. Stochastic gyrokinetic expansions are also discussed.

        Different options for the numerical implementation of these schemes are considered.

        Due to the efficacy of FET in the development of SP timesteppers for both the fluid and kinetic component, we hope this approach will prove effective in the future for developing SP timesteppers for the full hybrid model. We hope this will give us the opportunity to incorporate previously inaccessible kinetic effects into the highly effective, modern, finite-element MHD models.
    \end{abstract}
    
    
    \newpage
    \tableofcontents
    
    
    \newpage
    \pagenumbering{arabic}
    %\linenumbers\renewcommand\thelinenumber{\color{black!50}\arabic{linenumber}}
            \documentclass[12pt, a4paper]{report}

\input{template/main.tex}

\title{\BA{Title in Progress...}}
\author{Boris Andrews}
\affil{Mathematical Institute, University of Oxford}
\date{\today}


\begin{document}
    \pagenumbering{gobble}
    \maketitle
    
    
    \begin{abstract}
        Magnetic confinement reactors---in particular tokamaks---offer one of the most promising options for achieving practical nuclear fusion, with the potential to provide virtually limitless, clean energy. The theoretical and numerical modeling of tokamak plasmas is simultaneously an essential component of effective reactor design, and a great research barrier. Tokamak operational conditions exhibit comparatively low Knudsen numbers. Kinetic effects, including kinetic waves and instabilities, Landau damping, bump-on-tail instabilities and more, are therefore highly influential in tokamak plasma dynamics. Purely fluid models are inherently incapable of capturing these effects, whereas the high dimensionality in purely kinetic models render them practically intractable for most relevant purposes.

        We consider a $\delta\!f$ decomposition model, with a macroscopic fluid background and microscopic kinetic correction, both fully coupled to each other. A similar manner of discretization is proposed to that used in the recent \texttt{STRUPHY} code \cite{Holderied_Possanner_Wang_2021, Holderied_2022, Li_et_al_2023} with a finite-element model for the background and a pseudo-particle/PiC model for the correction.

        The fluid background satisfies the full, non-linear, resistive, compressible, Hall MHD equations. \cite{Laakmann_Hu_Farrell_2022} introduces finite-element(-in-space) implicit timesteppers for the incompressible analogue to this system with structure-preserving (SP) properties in the ideal case, alongside parameter-robust preconditioners. We show that these timesteppers can derive from a finite-element-in-time (FET) (and finite-element-in-space) interpretation. The benefits of this reformulation are discussed, including the derivation of timesteppers that are higher order in time, and the quantifiable dissipative SP properties in the non-ideal, resistive case.
        
        We discuss possible options for extending this FET approach to timesteppers for the compressible case.

        The kinetic corrections satisfy linearized Boltzmann equations. Using a Lénard--Bernstein collision operator, these take Fokker--Planck-like forms \cite{Fokker_1914, Planck_1917} wherein pseudo-particles in the numerical model obey the neoclassical transport equations, with particle-independent Brownian drift terms. This offers a rigorous methodology for incorporating collisions into the particle transport model, without coupling the equations of motions for each particle.
        
        Works by Chen, Chacón et al. \cite{Chen_Chacón_Barnes_2011, Chacón_Chen_Barnes_2013, Chen_Chacón_2014, Chen_Chacón_2015} have developed structure-preserving particle pushers for neoclassical transport in the Vlasov equations, derived from Crank--Nicolson integrators. We show these too can can derive from a FET interpretation, similarly offering potential extensions to higher-order-in-time particle pushers. The FET formulation is used also to consider how the stochastic drift terms can be incorporated into the pushers. Stochastic gyrokinetic expansions are also discussed.

        Different options for the numerical implementation of these schemes are considered.

        Due to the efficacy of FET in the development of SP timesteppers for both the fluid and kinetic component, we hope this approach will prove effective in the future for developing SP timesteppers for the full hybrid model. We hope this will give us the opportunity to incorporate previously inaccessible kinetic effects into the highly effective, modern, finite-element MHD models.
    \end{abstract}
    
    
    \newpage
    \tableofcontents
    
    
    \newpage
    \pagenumbering{arabic}
    %\linenumbers\renewcommand\thelinenumber{\color{black!50}\arabic{linenumber}}
            \input{0 - introduction/main.tex}
        \part{Research}
            \input{1 - low-noise PiC models/main.tex}
            \input{2 - kinetic component/main.tex}
            \input{3 - fluid component/main.tex}
            \input{4 - numerical implementation/main.tex}
        \part{Project Overview}
            \input{5 - research plan/main.tex}
            \input{6 - summary/main.tex}
    
    
    %\section{}
    \newpage
    \pagenumbering{gobble}
        \printbibliography


    \newpage
    \pagenumbering{roman}
    \appendix
        \part{Appendices}
            \input{8 - Hilbert complexes/main.tex}
            \input{9 - weak conservation proofs/main.tex}
\end{document}

        \part{Research}
            \documentclass[12pt, a4paper]{report}

\input{template/main.tex}

\title{\BA{Title in Progress...}}
\author{Boris Andrews}
\affil{Mathematical Institute, University of Oxford}
\date{\today}


\begin{document}
    \pagenumbering{gobble}
    \maketitle
    
    
    \begin{abstract}
        Magnetic confinement reactors---in particular tokamaks---offer one of the most promising options for achieving practical nuclear fusion, with the potential to provide virtually limitless, clean energy. The theoretical and numerical modeling of tokamak plasmas is simultaneously an essential component of effective reactor design, and a great research barrier. Tokamak operational conditions exhibit comparatively low Knudsen numbers. Kinetic effects, including kinetic waves and instabilities, Landau damping, bump-on-tail instabilities and more, are therefore highly influential in tokamak plasma dynamics. Purely fluid models are inherently incapable of capturing these effects, whereas the high dimensionality in purely kinetic models render them practically intractable for most relevant purposes.

        We consider a $\delta\!f$ decomposition model, with a macroscopic fluid background and microscopic kinetic correction, both fully coupled to each other. A similar manner of discretization is proposed to that used in the recent \texttt{STRUPHY} code \cite{Holderied_Possanner_Wang_2021, Holderied_2022, Li_et_al_2023} with a finite-element model for the background and a pseudo-particle/PiC model for the correction.

        The fluid background satisfies the full, non-linear, resistive, compressible, Hall MHD equations. \cite{Laakmann_Hu_Farrell_2022} introduces finite-element(-in-space) implicit timesteppers for the incompressible analogue to this system with structure-preserving (SP) properties in the ideal case, alongside parameter-robust preconditioners. We show that these timesteppers can derive from a finite-element-in-time (FET) (and finite-element-in-space) interpretation. The benefits of this reformulation are discussed, including the derivation of timesteppers that are higher order in time, and the quantifiable dissipative SP properties in the non-ideal, resistive case.
        
        We discuss possible options for extending this FET approach to timesteppers for the compressible case.

        The kinetic corrections satisfy linearized Boltzmann equations. Using a Lénard--Bernstein collision operator, these take Fokker--Planck-like forms \cite{Fokker_1914, Planck_1917} wherein pseudo-particles in the numerical model obey the neoclassical transport equations, with particle-independent Brownian drift terms. This offers a rigorous methodology for incorporating collisions into the particle transport model, without coupling the equations of motions for each particle.
        
        Works by Chen, Chacón et al. \cite{Chen_Chacón_Barnes_2011, Chacón_Chen_Barnes_2013, Chen_Chacón_2014, Chen_Chacón_2015} have developed structure-preserving particle pushers for neoclassical transport in the Vlasov equations, derived from Crank--Nicolson integrators. We show these too can can derive from a FET interpretation, similarly offering potential extensions to higher-order-in-time particle pushers. The FET formulation is used also to consider how the stochastic drift terms can be incorporated into the pushers. Stochastic gyrokinetic expansions are also discussed.

        Different options for the numerical implementation of these schemes are considered.

        Due to the efficacy of FET in the development of SP timesteppers for both the fluid and kinetic component, we hope this approach will prove effective in the future for developing SP timesteppers for the full hybrid model. We hope this will give us the opportunity to incorporate previously inaccessible kinetic effects into the highly effective, modern, finite-element MHD models.
    \end{abstract}
    
    
    \newpage
    \tableofcontents
    
    
    \newpage
    \pagenumbering{arabic}
    %\linenumbers\renewcommand\thelinenumber{\color{black!50}\arabic{linenumber}}
            \input{0 - introduction/main.tex}
        \part{Research}
            \input{1 - low-noise PiC models/main.tex}
            \input{2 - kinetic component/main.tex}
            \input{3 - fluid component/main.tex}
            \input{4 - numerical implementation/main.tex}
        \part{Project Overview}
            \input{5 - research plan/main.tex}
            \input{6 - summary/main.tex}
    
    
    %\section{}
    \newpage
    \pagenumbering{gobble}
        \printbibliography


    \newpage
    \pagenumbering{roman}
    \appendix
        \part{Appendices}
            \input{8 - Hilbert complexes/main.tex}
            \input{9 - weak conservation proofs/main.tex}
\end{document}

            \documentclass[12pt, a4paper]{report}

\input{template/main.tex}

\title{\BA{Title in Progress...}}
\author{Boris Andrews}
\affil{Mathematical Institute, University of Oxford}
\date{\today}


\begin{document}
    \pagenumbering{gobble}
    \maketitle
    
    
    \begin{abstract}
        Magnetic confinement reactors---in particular tokamaks---offer one of the most promising options for achieving practical nuclear fusion, with the potential to provide virtually limitless, clean energy. The theoretical and numerical modeling of tokamak plasmas is simultaneously an essential component of effective reactor design, and a great research barrier. Tokamak operational conditions exhibit comparatively low Knudsen numbers. Kinetic effects, including kinetic waves and instabilities, Landau damping, bump-on-tail instabilities and more, are therefore highly influential in tokamak plasma dynamics. Purely fluid models are inherently incapable of capturing these effects, whereas the high dimensionality in purely kinetic models render them practically intractable for most relevant purposes.

        We consider a $\delta\!f$ decomposition model, with a macroscopic fluid background and microscopic kinetic correction, both fully coupled to each other. A similar manner of discretization is proposed to that used in the recent \texttt{STRUPHY} code \cite{Holderied_Possanner_Wang_2021, Holderied_2022, Li_et_al_2023} with a finite-element model for the background and a pseudo-particle/PiC model for the correction.

        The fluid background satisfies the full, non-linear, resistive, compressible, Hall MHD equations. \cite{Laakmann_Hu_Farrell_2022} introduces finite-element(-in-space) implicit timesteppers for the incompressible analogue to this system with structure-preserving (SP) properties in the ideal case, alongside parameter-robust preconditioners. We show that these timesteppers can derive from a finite-element-in-time (FET) (and finite-element-in-space) interpretation. The benefits of this reformulation are discussed, including the derivation of timesteppers that are higher order in time, and the quantifiable dissipative SP properties in the non-ideal, resistive case.
        
        We discuss possible options for extending this FET approach to timesteppers for the compressible case.

        The kinetic corrections satisfy linearized Boltzmann equations. Using a Lénard--Bernstein collision operator, these take Fokker--Planck-like forms \cite{Fokker_1914, Planck_1917} wherein pseudo-particles in the numerical model obey the neoclassical transport equations, with particle-independent Brownian drift terms. This offers a rigorous methodology for incorporating collisions into the particle transport model, without coupling the equations of motions for each particle.
        
        Works by Chen, Chacón et al. \cite{Chen_Chacón_Barnes_2011, Chacón_Chen_Barnes_2013, Chen_Chacón_2014, Chen_Chacón_2015} have developed structure-preserving particle pushers for neoclassical transport in the Vlasov equations, derived from Crank--Nicolson integrators. We show these too can can derive from a FET interpretation, similarly offering potential extensions to higher-order-in-time particle pushers. The FET formulation is used also to consider how the stochastic drift terms can be incorporated into the pushers. Stochastic gyrokinetic expansions are also discussed.

        Different options for the numerical implementation of these schemes are considered.

        Due to the efficacy of FET in the development of SP timesteppers for both the fluid and kinetic component, we hope this approach will prove effective in the future for developing SP timesteppers for the full hybrid model. We hope this will give us the opportunity to incorporate previously inaccessible kinetic effects into the highly effective, modern, finite-element MHD models.
    \end{abstract}
    
    
    \newpage
    \tableofcontents
    
    
    \newpage
    \pagenumbering{arabic}
    %\linenumbers\renewcommand\thelinenumber{\color{black!50}\arabic{linenumber}}
            \input{0 - introduction/main.tex}
        \part{Research}
            \input{1 - low-noise PiC models/main.tex}
            \input{2 - kinetic component/main.tex}
            \input{3 - fluid component/main.tex}
            \input{4 - numerical implementation/main.tex}
        \part{Project Overview}
            \input{5 - research plan/main.tex}
            \input{6 - summary/main.tex}
    
    
    %\section{}
    \newpage
    \pagenumbering{gobble}
        \printbibliography


    \newpage
    \pagenumbering{roman}
    \appendix
        \part{Appendices}
            \input{8 - Hilbert complexes/main.tex}
            \input{9 - weak conservation proofs/main.tex}
\end{document}

            \documentclass[12pt, a4paper]{report}

\input{template/main.tex}

\title{\BA{Title in Progress...}}
\author{Boris Andrews}
\affil{Mathematical Institute, University of Oxford}
\date{\today}


\begin{document}
    \pagenumbering{gobble}
    \maketitle
    
    
    \begin{abstract}
        Magnetic confinement reactors---in particular tokamaks---offer one of the most promising options for achieving practical nuclear fusion, with the potential to provide virtually limitless, clean energy. The theoretical and numerical modeling of tokamak plasmas is simultaneously an essential component of effective reactor design, and a great research barrier. Tokamak operational conditions exhibit comparatively low Knudsen numbers. Kinetic effects, including kinetic waves and instabilities, Landau damping, bump-on-tail instabilities and more, are therefore highly influential in tokamak plasma dynamics. Purely fluid models are inherently incapable of capturing these effects, whereas the high dimensionality in purely kinetic models render them practically intractable for most relevant purposes.

        We consider a $\delta\!f$ decomposition model, with a macroscopic fluid background and microscopic kinetic correction, both fully coupled to each other. A similar manner of discretization is proposed to that used in the recent \texttt{STRUPHY} code \cite{Holderied_Possanner_Wang_2021, Holderied_2022, Li_et_al_2023} with a finite-element model for the background and a pseudo-particle/PiC model for the correction.

        The fluid background satisfies the full, non-linear, resistive, compressible, Hall MHD equations. \cite{Laakmann_Hu_Farrell_2022} introduces finite-element(-in-space) implicit timesteppers for the incompressible analogue to this system with structure-preserving (SP) properties in the ideal case, alongside parameter-robust preconditioners. We show that these timesteppers can derive from a finite-element-in-time (FET) (and finite-element-in-space) interpretation. The benefits of this reformulation are discussed, including the derivation of timesteppers that are higher order in time, and the quantifiable dissipative SP properties in the non-ideal, resistive case.
        
        We discuss possible options for extending this FET approach to timesteppers for the compressible case.

        The kinetic corrections satisfy linearized Boltzmann equations. Using a Lénard--Bernstein collision operator, these take Fokker--Planck-like forms \cite{Fokker_1914, Planck_1917} wherein pseudo-particles in the numerical model obey the neoclassical transport equations, with particle-independent Brownian drift terms. This offers a rigorous methodology for incorporating collisions into the particle transport model, without coupling the equations of motions for each particle.
        
        Works by Chen, Chacón et al. \cite{Chen_Chacón_Barnes_2011, Chacón_Chen_Barnes_2013, Chen_Chacón_2014, Chen_Chacón_2015} have developed structure-preserving particle pushers for neoclassical transport in the Vlasov equations, derived from Crank--Nicolson integrators. We show these too can can derive from a FET interpretation, similarly offering potential extensions to higher-order-in-time particle pushers. The FET formulation is used also to consider how the stochastic drift terms can be incorporated into the pushers. Stochastic gyrokinetic expansions are also discussed.

        Different options for the numerical implementation of these schemes are considered.

        Due to the efficacy of FET in the development of SP timesteppers for both the fluid and kinetic component, we hope this approach will prove effective in the future for developing SP timesteppers for the full hybrid model. We hope this will give us the opportunity to incorporate previously inaccessible kinetic effects into the highly effective, modern, finite-element MHD models.
    \end{abstract}
    
    
    \newpage
    \tableofcontents
    
    
    \newpage
    \pagenumbering{arabic}
    %\linenumbers\renewcommand\thelinenumber{\color{black!50}\arabic{linenumber}}
            \input{0 - introduction/main.tex}
        \part{Research}
            \input{1 - low-noise PiC models/main.tex}
            \input{2 - kinetic component/main.tex}
            \input{3 - fluid component/main.tex}
            \input{4 - numerical implementation/main.tex}
        \part{Project Overview}
            \input{5 - research plan/main.tex}
            \input{6 - summary/main.tex}
    
    
    %\section{}
    \newpage
    \pagenumbering{gobble}
        \printbibliography


    \newpage
    \pagenumbering{roman}
    \appendix
        \part{Appendices}
            \input{8 - Hilbert complexes/main.tex}
            \input{9 - weak conservation proofs/main.tex}
\end{document}

            \documentclass[12pt, a4paper]{report}

\input{template/main.tex}

\title{\BA{Title in Progress...}}
\author{Boris Andrews}
\affil{Mathematical Institute, University of Oxford}
\date{\today}


\begin{document}
    \pagenumbering{gobble}
    \maketitle
    
    
    \begin{abstract}
        Magnetic confinement reactors---in particular tokamaks---offer one of the most promising options for achieving practical nuclear fusion, with the potential to provide virtually limitless, clean energy. The theoretical and numerical modeling of tokamak plasmas is simultaneously an essential component of effective reactor design, and a great research barrier. Tokamak operational conditions exhibit comparatively low Knudsen numbers. Kinetic effects, including kinetic waves and instabilities, Landau damping, bump-on-tail instabilities and more, are therefore highly influential in tokamak plasma dynamics. Purely fluid models are inherently incapable of capturing these effects, whereas the high dimensionality in purely kinetic models render them practically intractable for most relevant purposes.

        We consider a $\delta\!f$ decomposition model, with a macroscopic fluid background and microscopic kinetic correction, both fully coupled to each other. A similar manner of discretization is proposed to that used in the recent \texttt{STRUPHY} code \cite{Holderied_Possanner_Wang_2021, Holderied_2022, Li_et_al_2023} with a finite-element model for the background and a pseudo-particle/PiC model for the correction.

        The fluid background satisfies the full, non-linear, resistive, compressible, Hall MHD equations. \cite{Laakmann_Hu_Farrell_2022} introduces finite-element(-in-space) implicit timesteppers for the incompressible analogue to this system with structure-preserving (SP) properties in the ideal case, alongside parameter-robust preconditioners. We show that these timesteppers can derive from a finite-element-in-time (FET) (and finite-element-in-space) interpretation. The benefits of this reformulation are discussed, including the derivation of timesteppers that are higher order in time, and the quantifiable dissipative SP properties in the non-ideal, resistive case.
        
        We discuss possible options for extending this FET approach to timesteppers for the compressible case.

        The kinetic corrections satisfy linearized Boltzmann equations. Using a Lénard--Bernstein collision operator, these take Fokker--Planck-like forms \cite{Fokker_1914, Planck_1917} wherein pseudo-particles in the numerical model obey the neoclassical transport equations, with particle-independent Brownian drift terms. This offers a rigorous methodology for incorporating collisions into the particle transport model, without coupling the equations of motions for each particle.
        
        Works by Chen, Chacón et al. \cite{Chen_Chacón_Barnes_2011, Chacón_Chen_Barnes_2013, Chen_Chacón_2014, Chen_Chacón_2015} have developed structure-preserving particle pushers for neoclassical transport in the Vlasov equations, derived from Crank--Nicolson integrators. We show these too can can derive from a FET interpretation, similarly offering potential extensions to higher-order-in-time particle pushers. The FET formulation is used also to consider how the stochastic drift terms can be incorporated into the pushers. Stochastic gyrokinetic expansions are also discussed.

        Different options for the numerical implementation of these schemes are considered.

        Due to the efficacy of FET in the development of SP timesteppers for both the fluid and kinetic component, we hope this approach will prove effective in the future for developing SP timesteppers for the full hybrid model. We hope this will give us the opportunity to incorporate previously inaccessible kinetic effects into the highly effective, modern, finite-element MHD models.
    \end{abstract}
    
    
    \newpage
    \tableofcontents
    
    
    \newpage
    \pagenumbering{arabic}
    %\linenumbers\renewcommand\thelinenumber{\color{black!50}\arabic{linenumber}}
            \input{0 - introduction/main.tex}
        \part{Research}
            \input{1 - low-noise PiC models/main.tex}
            \input{2 - kinetic component/main.tex}
            \input{3 - fluid component/main.tex}
            \input{4 - numerical implementation/main.tex}
        \part{Project Overview}
            \input{5 - research plan/main.tex}
            \input{6 - summary/main.tex}
    
    
    %\section{}
    \newpage
    \pagenumbering{gobble}
        \printbibliography


    \newpage
    \pagenumbering{roman}
    \appendix
        \part{Appendices}
            \input{8 - Hilbert complexes/main.tex}
            \input{9 - weak conservation proofs/main.tex}
\end{document}

        \part{Project Overview}
            \documentclass[12pt, a4paper]{report}

\input{template/main.tex}

\title{\BA{Title in Progress...}}
\author{Boris Andrews}
\affil{Mathematical Institute, University of Oxford}
\date{\today}


\begin{document}
    \pagenumbering{gobble}
    \maketitle
    
    
    \begin{abstract}
        Magnetic confinement reactors---in particular tokamaks---offer one of the most promising options for achieving practical nuclear fusion, with the potential to provide virtually limitless, clean energy. The theoretical and numerical modeling of tokamak plasmas is simultaneously an essential component of effective reactor design, and a great research barrier. Tokamak operational conditions exhibit comparatively low Knudsen numbers. Kinetic effects, including kinetic waves and instabilities, Landau damping, bump-on-tail instabilities and more, are therefore highly influential in tokamak plasma dynamics. Purely fluid models are inherently incapable of capturing these effects, whereas the high dimensionality in purely kinetic models render them practically intractable for most relevant purposes.

        We consider a $\delta\!f$ decomposition model, with a macroscopic fluid background and microscopic kinetic correction, both fully coupled to each other. A similar manner of discretization is proposed to that used in the recent \texttt{STRUPHY} code \cite{Holderied_Possanner_Wang_2021, Holderied_2022, Li_et_al_2023} with a finite-element model for the background and a pseudo-particle/PiC model for the correction.

        The fluid background satisfies the full, non-linear, resistive, compressible, Hall MHD equations. \cite{Laakmann_Hu_Farrell_2022} introduces finite-element(-in-space) implicit timesteppers for the incompressible analogue to this system with structure-preserving (SP) properties in the ideal case, alongside parameter-robust preconditioners. We show that these timesteppers can derive from a finite-element-in-time (FET) (and finite-element-in-space) interpretation. The benefits of this reformulation are discussed, including the derivation of timesteppers that are higher order in time, and the quantifiable dissipative SP properties in the non-ideal, resistive case.
        
        We discuss possible options for extending this FET approach to timesteppers for the compressible case.

        The kinetic corrections satisfy linearized Boltzmann equations. Using a Lénard--Bernstein collision operator, these take Fokker--Planck-like forms \cite{Fokker_1914, Planck_1917} wherein pseudo-particles in the numerical model obey the neoclassical transport equations, with particle-independent Brownian drift terms. This offers a rigorous methodology for incorporating collisions into the particle transport model, without coupling the equations of motions for each particle.
        
        Works by Chen, Chacón et al. \cite{Chen_Chacón_Barnes_2011, Chacón_Chen_Barnes_2013, Chen_Chacón_2014, Chen_Chacón_2015} have developed structure-preserving particle pushers for neoclassical transport in the Vlasov equations, derived from Crank--Nicolson integrators. We show these too can can derive from a FET interpretation, similarly offering potential extensions to higher-order-in-time particle pushers. The FET formulation is used also to consider how the stochastic drift terms can be incorporated into the pushers. Stochastic gyrokinetic expansions are also discussed.

        Different options for the numerical implementation of these schemes are considered.

        Due to the efficacy of FET in the development of SP timesteppers for both the fluid and kinetic component, we hope this approach will prove effective in the future for developing SP timesteppers for the full hybrid model. We hope this will give us the opportunity to incorporate previously inaccessible kinetic effects into the highly effective, modern, finite-element MHD models.
    \end{abstract}
    
    
    \newpage
    \tableofcontents
    
    
    \newpage
    \pagenumbering{arabic}
    %\linenumbers\renewcommand\thelinenumber{\color{black!50}\arabic{linenumber}}
            \input{0 - introduction/main.tex}
        \part{Research}
            \input{1 - low-noise PiC models/main.tex}
            \input{2 - kinetic component/main.tex}
            \input{3 - fluid component/main.tex}
            \input{4 - numerical implementation/main.tex}
        \part{Project Overview}
            \input{5 - research plan/main.tex}
            \input{6 - summary/main.tex}
    
    
    %\section{}
    \newpage
    \pagenumbering{gobble}
        \printbibliography


    \newpage
    \pagenumbering{roman}
    \appendix
        \part{Appendices}
            \input{8 - Hilbert complexes/main.tex}
            \input{9 - weak conservation proofs/main.tex}
\end{document}

            \documentclass[12pt, a4paper]{report}

\input{template/main.tex}

\title{\BA{Title in Progress...}}
\author{Boris Andrews}
\affil{Mathematical Institute, University of Oxford}
\date{\today}


\begin{document}
    \pagenumbering{gobble}
    \maketitle
    
    
    \begin{abstract}
        Magnetic confinement reactors---in particular tokamaks---offer one of the most promising options for achieving practical nuclear fusion, with the potential to provide virtually limitless, clean energy. The theoretical and numerical modeling of tokamak plasmas is simultaneously an essential component of effective reactor design, and a great research barrier. Tokamak operational conditions exhibit comparatively low Knudsen numbers. Kinetic effects, including kinetic waves and instabilities, Landau damping, bump-on-tail instabilities and more, are therefore highly influential in tokamak plasma dynamics. Purely fluid models are inherently incapable of capturing these effects, whereas the high dimensionality in purely kinetic models render them practically intractable for most relevant purposes.

        We consider a $\delta\!f$ decomposition model, with a macroscopic fluid background and microscopic kinetic correction, both fully coupled to each other. A similar manner of discretization is proposed to that used in the recent \texttt{STRUPHY} code \cite{Holderied_Possanner_Wang_2021, Holderied_2022, Li_et_al_2023} with a finite-element model for the background and a pseudo-particle/PiC model for the correction.

        The fluid background satisfies the full, non-linear, resistive, compressible, Hall MHD equations. \cite{Laakmann_Hu_Farrell_2022} introduces finite-element(-in-space) implicit timesteppers for the incompressible analogue to this system with structure-preserving (SP) properties in the ideal case, alongside parameter-robust preconditioners. We show that these timesteppers can derive from a finite-element-in-time (FET) (and finite-element-in-space) interpretation. The benefits of this reformulation are discussed, including the derivation of timesteppers that are higher order in time, and the quantifiable dissipative SP properties in the non-ideal, resistive case.
        
        We discuss possible options for extending this FET approach to timesteppers for the compressible case.

        The kinetic corrections satisfy linearized Boltzmann equations. Using a Lénard--Bernstein collision operator, these take Fokker--Planck-like forms \cite{Fokker_1914, Planck_1917} wherein pseudo-particles in the numerical model obey the neoclassical transport equations, with particle-independent Brownian drift terms. This offers a rigorous methodology for incorporating collisions into the particle transport model, without coupling the equations of motions for each particle.
        
        Works by Chen, Chacón et al. \cite{Chen_Chacón_Barnes_2011, Chacón_Chen_Barnes_2013, Chen_Chacón_2014, Chen_Chacón_2015} have developed structure-preserving particle pushers for neoclassical transport in the Vlasov equations, derived from Crank--Nicolson integrators. We show these too can can derive from a FET interpretation, similarly offering potential extensions to higher-order-in-time particle pushers. The FET formulation is used also to consider how the stochastic drift terms can be incorporated into the pushers. Stochastic gyrokinetic expansions are also discussed.

        Different options for the numerical implementation of these schemes are considered.

        Due to the efficacy of FET in the development of SP timesteppers for both the fluid and kinetic component, we hope this approach will prove effective in the future for developing SP timesteppers for the full hybrid model. We hope this will give us the opportunity to incorporate previously inaccessible kinetic effects into the highly effective, modern, finite-element MHD models.
    \end{abstract}
    
    
    \newpage
    \tableofcontents
    
    
    \newpage
    \pagenumbering{arabic}
    %\linenumbers\renewcommand\thelinenumber{\color{black!50}\arabic{linenumber}}
            \input{0 - introduction/main.tex}
        \part{Research}
            \input{1 - low-noise PiC models/main.tex}
            \input{2 - kinetic component/main.tex}
            \input{3 - fluid component/main.tex}
            \input{4 - numerical implementation/main.tex}
        \part{Project Overview}
            \input{5 - research plan/main.tex}
            \input{6 - summary/main.tex}
    
    
    %\section{}
    \newpage
    \pagenumbering{gobble}
        \printbibliography


    \newpage
    \pagenumbering{roman}
    \appendix
        \part{Appendices}
            \input{8 - Hilbert complexes/main.tex}
            \input{9 - weak conservation proofs/main.tex}
\end{document}

    
    
    %\section{}
    \newpage
    \pagenumbering{gobble}
        \printbibliography


    \newpage
    \pagenumbering{roman}
    \appendix
        \part{Appendices}
            \documentclass[12pt, a4paper]{report}

\input{template/main.tex}

\title{\BA{Title in Progress...}}
\author{Boris Andrews}
\affil{Mathematical Institute, University of Oxford}
\date{\today}


\begin{document}
    \pagenumbering{gobble}
    \maketitle
    
    
    \begin{abstract}
        Magnetic confinement reactors---in particular tokamaks---offer one of the most promising options for achieving practical nuclear fusion, with the potential to provide virtually limitless, clean energy. The theoretical and numerical modeling of tokamak plasmas is simultaneously an essential component of effective reactor design, and a great research barrier. Tokamak operational conditions exhibit comparatively low Knudsen numbers. Kinetic effects, including kinetic waves and instabilities, Landau damping, bump-on-tail instabilities and more, are therefore highly influential in tokamak plasma dynamics. Purely fluid models are inherently incapable of capturing these effects, whereas the high dimensionality in purely kinetic models render them practically intractable for most relevant purposes.

        We consider a $\delta\!f$ decomposition model, with a macroscopic fluid background and microscopic kinetic correction, both fully coupled to each other. A similar manner of discretization is proposed to that used in the recent \texttt{STRUPHY} code \cite{Holderied_Possanner_Wang_2021, Holderied_2022, Li_et_al_2023} with a finite-element model for the background and a pseudo-particle/PiC model for the correction.

        The fluid background satisfies the full, non-linear, resistive, compressible, Hall MHD equations. \cite{Laakmann_Hu_Farrell_2022} introduces finite-element(-in-space) implicit timesteppers for the incompressible analogue to this system with structure-preserving (SP) properties in the ideal case, alongside parameter-robust preconditioners. We show that these timesteppers can derive from a finite-element-in-time (FET) (and finite-element-in-space) interpretation. The benefits of this reformulation are discussed, including the derivation of timesteppers that are higher order in time, and the quantifiable dissipative SP properties in the non-ideal, resistive case.
        
        We discuss possible options for extending this FET approach to timesteppers for the compressible case.

        The kinetic corrections satisfy linearized Boltzmann equations. Using a Lénard--Bernstein collision operator, these take Fokker--Planck-like forms \cite{Fokker_1914, Planck_1917} wherein pseudo-particles in the numerical model obey the neoclassical transport equations, with particle-independent Brownian drift terms. This offers a rigorous methodology for incorporating collisions into the particle transport model, without coupling the equations of motions for each particle.
        
        Works by Chen, Chacón et al. \cite{Chen_Chacón_Barnes_2011, Chacón_Chen_Barnes_2013, Chen_Chacón_2014, Chen_Chacón_2015} have developed structure-preserving particle pushers for neoclassical transport in the Vlasov equations, derived from Crank--Nicolson integrators. We show these too can can derive from a FET interpretation, similarly offering potential extensions to higher-order-in-time particle pushers. The FET formulation is used also to consider how the stochastic drift terms can be incorporated into the pushers. Stochastic gyrokinetic expansions are also discussed.

        Different options for the numerical implementation of these schemes are considered.

        Due to the efficacy of FET in the development of SP timesteppers for both the fluid and kinetic component, we hope this approach will prove effective in the future for developing SP timesteppers for the full hybrid model. We hope this will give us the opportunity to incorporate previously inaccessible kinetic effects into the highly effective, modern, finite-element MHD models.
    \end{abstract}
    
    
    \newpage
    \tableofcontents
    
    
    \newpage
    \pagenumbering{arabic}
    %\linenumbers\renewcommand\thelinenumber{\color{black!50}\arabic{linenumber}}
            \input{0 - introduction/main.tex}
        \part{Research}
            \input{1 - low-noise PiC models/main.tex}
            \input{2 - kinetic component/main.tex}
            \input{3 - fluid component/main.tex}
            \input{4 - numerical implementation/main.tex}
        \part{Project Overview}
            \input{5 - research plan/main.tex}
            \input{6 - summary/main.tex}
    
    
    %\section{}
    \newpage
    \pagenumbering{gobble}
        \printbibliography


    \newpage
    \pagenumbering{roman}
    \appendix
        \part{Appendices}
            \input{8 - Hilbert complexes/main.tex}
            \input{9 - weak conservation proofs/main.tex}
\end{document}

            \documentclass[12pt, a4paper]{report}

\input{template/main.tex}

\title{\BA{Title in Progress...}}
\author{Boris Andrews}
\affil{Mathematical Institute, University of Oxford}
\date{\today}


\begin{document}
    \pagenumbering{gobble}
    \maketitle
    
    
    \begin{abstract}
        Magnetic confinement reactors---in particular tokamaks---offer one of the most promising options for achieving practical nuclear fusion, with the potential to provide virtually limitless, clean energy. The theoretical and numerical modeling of tokamak plasmas is simultaneously an essential component of effective reactor design, and a great research barrier. Tokamak operational conditions exhibit comparatively low Knudsen numbers. Kinetic effects, including kinetic waves and instabilities, Landau damping, bump-on-tail instabilities and more, are therefore highly influential in tokamak plasma dynamics. Purely fluid models are inherently incapable of capturing these effects, whereas the high dimensionality in purely kinetic models render them practically intractable for most relevant purposes.

        We consider a $\delta\!f$ decomposition model, with a macroscopic fluid background and microscopic kinetic correction, both fully coupled to each other. A similar manner of discretization is proposed to that used in the recent \texttt{STRUPHY} code \cite{Holderied_Possanner_Wang_2021, Holderied_2022, Li_et_al_2023} with a finite-element model for the background and a pseudo-particle/PiC model for the correction.

        The fluid background satisfies the full, non-linear, resistive, compressible, Hall MHD equations. \cite{Laakmann_Hu_Farrell_2022} introduces finite-element(-in-space) implicit timesteppers for the incompressible analogue to this system with structure-preserving (SP) properties in the ideal case, alongside parameter-robust preconditioners. We show that these timesteppers can derive from a finite-element-in-time (FET) (and finite-element-in-space) interpretation. The benefits of this reformulation are discussed, including the derivation of timesteppers that are higher order in time, and the quantifiable dissipative SP properties in the non-ideal, resistive case.
        
        We discuss possible options for extending this FET approach to timesteppers for the compressible case.

        The kinetic corrections satisfy linearized Boltzmann equations. Using a Lénard--Bernstein collision operator, these take Fokker--Planck-like forms \cite{Fokker_1914, Planck_1917} wherein pseudo-particles in the numerical model obey the neoclassical transport equations, with particle-independent Brownian drift terms. This offers a rigorous methodology for incorporating collisions into the particle transport model, without coupling the equations of motions for each particle.
        
        Works by Chen, Chacón et al. \cite{Chen_Chacón_Barnes_2011, Chacón_Chen_Barnes_2013, Chen_Chacón_2014, Chen_Chacón_2015} have developed structure-preserving particle pushers for neoclassical transport in the Vlasov equations, derived from Crank--Nicolson integrators. We show these too can can derive from a FET interpretation, similarly offering potential extensions to higher-order-in-time particle pushers. The FET formulation is used also to consider how the stochastic drift terms can be incorporated into the pushers. Stochastic gyrokinetic expansions are also discussed.

        Different options for the numerical implementation of these schemes are considered.

        Due to the efficacy of FET in the development of SP timesteppers for both the fluid and kinetic component, we hope this approach will prove effective in the future for developing SP timesteppers for the full hybrid model. We hope this will give us the opportunity to incorporate previously inaccessible kinetic effects into the highly effective, modern, finite-element MHD models.
    \end{abstract}
    
    
    \newpage
    \tableofcontents
    
    
    \newpage
    \pagenumbering{arabic}
    %\linenumbers\renewcommand\thelinenumber{\color{black!50}\arabic{linenumber}}
            \input{0 - introduction/main.tex}
        \part{Research}
            \input{1 - low-noise PiC models/main.tex}
            \input{2 - kinetic component/main.tex}
            \input{3 - fluid component/main.tex}
            \input{4 - numerical implementation/main.tex}
        \part{Project Overview}
            \input{5 - research plan/main.tex}
            \input{6 - summary/main.tex}
    
    
    %\section{}
    \newpage
    \pagenumbering{gobble}
        \printbibliography


    \newpage
    \pagenumbering{roman}
    \appendix
        \part{Appendices}
            \input{8 - Hilbert complexes/main.tex}
            \input{9 - weak conservation proofs/main.tex}
\end{document}

\end{document}

            \documentclass[12pt, a4paper]{report}

\documentclass[12pt, a4paper]{report}

\input{template/main.tex}

\title{\BA{Title in Progress...}}
\author{Boris Andrews}
\affil{Mathematical Institute, University of Oxford}
\date{\today}


\begin{document}
    \pagenumbering{gobble}
    \maketitle
    
    
    \begin{abstract}
        Magnetic confinement reactors---in particular tokamaks---offer one of the most promising options for achieving practical nuclear fusion, with the potential to provide virtually limitless, clean energy. The theoretical and numerical modeling of tokamak plasmas is simultaneously an essential component of effective reactor design, and a great research barrier. Tokamak operational conditions exhibit comparatively low Knudsen numbers. Kinetic effects, including kinetic waves and instabilities, Landau damping, bump-on-tail instabilities and more, are therefore highly influential in tokamak plasma dynamics. Purely fluid models are inherently incapable of capturing these effects, whereas the high dimensionality in purely kinetic models render them practically intractable for most relevant purposes.

        We consider a $\delta\!f$ decomposition model, with a macroscopic fluid background and microscopic kinetic correction, both fully coupled to each other. A similar manner of discretization is proposed to that used in the recent \texttt{STRUPHY} code \cite{Holderied_Possanner_Wang_2021, Holderied_2022, Li_et_al_2023} with a finite-element model for the background and a pseudo-particle/PiC model for the correction.

        The fluid background satisfies the full, non-linear, resistive, compressible, Hall MHD equations. \cite{Laakmann_Hu_Farrell_2022} introduces finite-element(-in-space) implicit timesteppers for the incompressible analogue to this system with structure-preserving (SP) properties in the ideal case, alongside parameter-robust preconditioners. We show that these timesteppers can derive from a finite-element-in-time (FET) (and finite-element-in-space) interpretation. The benefits of this reformulation are discussed, including the derivation of timesteppers that are higher order in time, and the quantifiable dissipative SP properties in the non-ideal, resistive case.
        
        We discuss possible options for extending this FET approach to timesteppers for the compressible case.

        The kinetic corrections satisfy linearized Boltzmann equations. Using a Lénard--Bernstein collision operator, these take Fokker--Planck-like forms \cite{Fokker_1914, Planck_1917} wherein pseudo-particles in the numerical model obey the neoclassical transport equations, with particle-independent Brownian drift terms. This offers a rigorous methodology for incorporating collisions into the particle transport model, without coupling the equations of motions for each particle.
        
        Works by Chen, Chacón et al. \cite{Chen_Chacón_Barnes_2011, Chacón_Chen_Barnes_2013, Chen_Chacón_2014, Chen_Chacón_2015} have developed structure-preserving particle pushers for neoclassical transport in the Vlasov equations, derived from Crank--Nicolson integrators. We show these too can can derive from a FET interpretation, similarly offering potential extensions to higher-order-in-time particle pushers. The FET formulation is used also to consider how the stochastic drift terms can be incorporated into the pushers. Stochastic gyrokinetic expansions are also discussed.

        Different options for the numerical implementation of these schemes are considered.

        Due to the efficacy of FET in the development of SP timesteppers for both the fluid and kinetic component, we hope this approach will prove effective in the future for developing SP timesteppers for the full hybrid model. We hope this will give us the opportunity to incorporate previously inaccessible kinetic effects into the highly effective, modern, finite-element MHD models.
    \end{abstract}
    
    
    \newpage
    \tableofcontents
    
    
    \newpage
    \pagenumbering{arabic}
    %\linenumbers\renewcommand\thelinenumber{\color{black!50}\arabic{linenumber}}
            \input{0 - introduction/main.tex}
        \part{Research}
            \input{1 - low-noise PiC models/main.tex}
            \input{2 - kinetic component/main.tex}
            \input{3 - fluid component/main.tex}
            \input{4 - numerical implementation/main.tex}
        \part{Project Overview}
            \input{5 - research plan/main.tex}
            \input{6 - summary/main.tex}
    
    
    %\section{}
    \newpage
    \pagenumbering{gobble}
        \printbibliography


    \newpage
    \pagenumbering{roman}
    \appendix
        \part{Appendices}
            \input{8 - Hilbert complexes/main.tex}
            \input{9 - weak conservation proofs/main.tex}
\end{document}


\title{\BA{Title in Progress...}}
\author{Boris Andrews}
\affil{Mathematical Institute, University of Oxford}
\date{\today}


\begin{document}
    \pagenumbering{gobble}
    \maketitle
    
    
    \begin{abstract}
        Magnetic confinement reactors---in particular tokamaks---offer one of the most promising options for achieving practical nuclear fusion, with the potential to provide virtually limitless, clean energy. The theoretical and numerical modeling of tokamak plasmas is simultaneously an essential component of effective reactor design, and a great research barrier. Tokamak operational conditions exhibit comparatively low Knudsen numbers. Kinetic effects, including kinetic waves and instabilities, Landau damping, bump-on-tail instabilities and more, are therefore highly influential in tokamak plasma dynamics. Purely fluid models are inherently incapable of capturing these effects, whereas the high dimensionality in purely kinetic models render them practically intractable for most relevant purposes.

        We consider a $\delta\!f$ decomposition model, with a macroscopic fluid background and microscopic kinetic correction, both fully coupled to each other. A similar manner of discretization is proposed to that used in the recent \texttt{STRUPHY} code \cite{Holderied_Possanner_Wang_2021, Holderied_2022, Li_et_al_2023} with a finite-element model for the background and a pseudo-particle/PiC model for the correction.

        The fluid background satisfies the full, non-linear, resistive, compressible, Hall MHD equations. \cite{Laakmann_Hu_Farrell_2022} introduces finite-element(-in-space) implicit timesteppers for the incompressible analogue to this system with structure-preserving (SP) properties in the ideal case, alongside parameter-robust preconditioners. We show that these timesteppers can derive from a finite-element-in-time (FET) (and finite-element-in-space) interpretation. The benefits of this reformulation are discussed, including the derivation of timesteppers that are higher order in time, and the quantifiable dissipative SP properties in the non-ideal, resistive case.
        
        We discuss possible options for extending this FET approach to timesteppers for the compressible case.

        The kinetic corrections satisfy linearized Boltzmann equations. Using a Lénard--Bernstein collision operator, these take Fokker--Planck-like forms \cite{Fokker_1914, Planck_1917} wherein pseudo-particles in the numerical model obey the neoclassical transport equations, with particle-independent Brownian drift terms. This offers a rigorous methodology for incorporating collisions into the particle transport model, without coupling the equations of motions for each particle.
        
        Works by Chen, Chacón et al. \cite{Chen_Chacón_Barnes_2011, Chacón_Chen_Barnes_2013, Chen_Chacón_2014, Chen_Chacón_2015} have developed structure-preserving particle pushers for neoclassical transport in the Vlasov equations, derived from Crank--Nicolson integrators. We show these too can can derive from a FET interpretation, similarly offering potential extensions to higher-order-in-time particle pushers. The FET formulation is used also to consider how the stochastic drift terms can be incorporated into the pushers. Stochastic gyrokinetic expansions are also discussed.

        Different options for the numerical implementation of these schemes are considered.

        Due to the efficacy of FET in the development of SP timesteppers for both the fluid and kinetic component, we hope this approach will prove effective in the future for developing SP timesteppers for the full hybrid model. We hope this will give us the opportunity to incorporate previously inaccessible kinetic effects into the highly effective, modern, finite-element MHD models.
    \end{abstract}
    
    
    \newpage
    \tableofcontents
    
    
    \newpage
    \pagenumbering{arabic}
    %\linenumbers\renewcommand\thelinenumber{\color{black!50}\arabic{linenumber}}
            \documentclass[12pt, a4paper]{report}

\input{template/main.tex}

\title{\BA{Title in Progress...}}
\author{Boris Andrews}
\affil{Mathematical Institute, University of Oxford}
\date{\today}


\begin{document}
    \pagenumbering{gobble}
    \maketitle
    
    
    \begin{abstract}
        Magnetic confinement reactors---in particular tokamaks---offer one of the most promising options for achieving practical nuclear fusion, with the potential to provide virtually limitless, clean energy. The theoretical and numerical modeling of tokamak plasmas is simultaneously an essential component of effective reactor design, and a great research barrier. Tokamak operational conditions exhibit comparatively low Knudsen numbers. Kinetic effects, including kinetic waves and instabilities, Landau damping, bump-on-tail instabilities and more, are therefore highly influential in tokamak plasma dynamics. Purely fluid models are inherently incapable of capturing these effects, whereas the high dimensionality in purely kinetic models render them practically intractable for most relevant purposes.

        We consider a $\delta\!f$ decomposition model, with a macroscopic fluid background and microscopic kinetic correction, both fully coupled to each other. A similar manner of discretization is proposed to that used in the recent \texttt{STRUPHY} code \cite{Holderied_Possanner_Wang_2021, Holderied_2022, Li_et_al_2023} with a finite-element model for the background and a pseudo-particle/PiC model for the correction.

        The fluid background satisfies the full, non-linear, resistive, compressible, Hall MHD equations. \cite{Laakmann_Hu_Farrell_2022} introduces finite-element(-in-space) implicit timesteppers for the incompressible analogue to this system with structure-preserving (SP) properties in the ideal case, alongside parameter-robust preconditioners. We show that these timesteppers can derive from a finite-element-in-time (FET) (and finite-element-in-space) interpretation. The benefits of this reformulation are discussed, including the derivation of timesteppers that are higher order in time, and the quantifiable dissipative SP properties in the non-ideal, resistive case.
        
        We discuss possible options for extending this FET approach to timesteppers for the compressible case.

        The kinetic corrections satisfy linearized Boltzmann equations. Using a Lénard--Bernstein collision operator, these take Fokker--Planck-like forms \cite{Fokker_1914, Planck_1917} wherein pseudo-particles in the numerical model obey the neoclassical transport equations, with particle-independent Brownian drift terms. This offers a rigorous methodology for incorporating collisions into the particle transport model, without coupling the equations of motions for each particle.
        
        Works by Chen, Chacón et al. \cite{Chen_Chacón_Barnes_2011, Chacón_Chen_Barnes_2013, Chen_Chacón_2014, Chen_Chacón_2015} have developed structure-preserving particle pushers for neoclassical transport in the Vlasov equations, derived from Crank--Nicolson integrators. We show these too can can derive from a FET interpretation, similarly offering potential extensions to higher-order-in-time particle pushers. The FET formulation is used also to consider how the stochastic drift terms can be incorporated into the pushers. Stochastic gyrokinetic expansions are also discussed.

        Different options for the numerical implementation of these schemes are considered.

        Due to the efficacy of FET in the development of SP timesteppers for both the fluid and kinetic component, we hope this approach will prove effective in the future for developing SP timesteppers for the full hybrid model. We hope this will give us the opportunity to incorporate previously inaccessible kinetic effects into the highly effective, modern, finite-element MHD models.
    \end{abstract}
    
    
    \newpage
    \tableofcontents
    
    
    \newpage
    \pagenumbering{arabic}
    %\linenumbers\renewcommand\thelinenumber{\color{black!50}\arabic{linenumber}}
            \input{0 - introduction/main.tex}
        \part{Research}
            \input{1 - low-noise PiC models/main.tex}
            \input{2 - kinetic component/main.tex}
            \input{3 - fluid component/main.tex}
            \input{4 - numerical implementation/main.tex}
        \part{Project Overview}
            \input{5 - research plan/main.tex}
            \input{6 - summary/main.tex}
    
    
    %\section{}
    \newpage
    \pagenumbering{gobble}
        \printbibliography


    \newpage
    \pagenumbering{roman}
    \appendix
        \part{Appendices}
            \input{8 - Hilbert complexes/main.tex}
            \input{9 - weak conservation proofs/main.tex}
\end{document}

        \part{Research}
            \documentclass[12pt, a4paper]{report}

\input{template/main.tex}

\title{\BA{Title in Progress...}}
\author{Boris Andrews}
\affil{Mathematical Institute, University of Oxford}
\date{\today}


\begin{document}
    \pagenumbering{gobble}
    \maketitle
    
    
    \begin{abstract}
        Magnetic confinement reactors---in particular tokamaks---offer one of the most promising options for achieving practical nuclear fusion, with the potential to provide virtually limitless, clean energy. The theoretical and numerical modeling of tokamak plasmas is simultaneously an essential component of effective reactor design, and a great research barrier. Tokamak operational conditions exhibit comparatively low Knudsen numbers. Kinetic effects, including kinetic waves and instabilities, Landau damping, bump-on-tail instabilities and more, are therefore highly influential in tokamak plasma dynamics. Purely fluid models are inherently incapable of capturing these effects, whereas the high dimensionality in purely kinetic models render them practically intractable for most relevant purposes.

        We consider a $\delta\!f$ decomposition model, with a macroscopic fluid background and microscopic kinetic correction, both fully coupled to each other. A similar manner of discretization is proposed to that used in the recent \texttt{STRUPHY} code \cite{Holderied_Possanner_Wang_2021, Holderied_2022, Li_et_al_2023} with a finite-element model for the background and a pseudo-particle/PiC model for the correction.

        The fluid background satisfies the full, non-linear, resistive, compressible, Hall MHD equations. \cite{Laakmann_Hu_Farrell_2022} introduces finite-element(-in-space) implicit timesteppers for the incompressible analogue to this system with structure-preserving (SP) properties in the ideal case, alongside parameter-robust preconditioners. We show that these timesteppers can derive from a finite-element-in-time (FET) (and finite-element-in-space) interpretation. The benefits of this reformulation are discussed, including the derivation of timesteppers that are higher order in time, and the quantifiable dissipative SP properties in the non-ideal, resistive case.
        
        We discuss possible options for extending this FET approach to timesteppers for the compressible case.

        The kinetic corrections satisfy linearized Boltzmann equations. Using a Lénard--Bernstein collision operator, these take Fokker--Planck-like forms \cite{Fokker_1914, Planck_1917} wherein pseudo-particles in the numerical model obey the neoclassical transport equations, with particle-independent Brownian drift terms. This offers a rigorous methodology for incorporating collisions into the particle transport model, without coupling the equations of motions for each particle.
        
        Works by Chen, Chacón et al. \cite{Chen_Chacón_Barnes_2011, Chacón_Chen_Barnes_2013, Chen_Chacón_2014, Chen_Chacón_2015} have developed structure-preserving particle pushers for neoclassical transport in the Vlasov equations, derived from Crank--Nicolson integrators. We show these too can can derive from a FET interpretation, similarly offering potential extensions to higher-order-in-time particle pushers. The FET formulation is used also to consider how the stochastic drift terms can be incorporated into the pushers. Stochastic gyrokinetic expansions are also discussed.

        Different options for the numerical implementation of these schemes are considered.

        Due to the efficacy of FET in the development of SP timesteppers for both the fluid and kinetic component, we hope this approach will prove effective in the future for developing SP timesteppers for the full hybrid model. We hope this will give us the opportunity to incorporate previously inaccessible kinetic effects into the highly effective, modern, finite-element MHD models.
    \end{abstract}
    
    
    \newpage
    \tableofcontents
    
    
    \newpage
    \pagenumbering{arabic}
    %\linenumbers\renewcommand\thelinenumber{\color{black!50}\arabic{linenumber}}
            \input{0 - introduction/main.tex}
        \part{Research}
            \input{1 - low-noise PiC models/main.tex}
            \input{2 - kinetic component/main.tex}
            \input{3 - fluid component/main.tex}
            \input{4 - numerical implementation/main.tex}
        \part{Project Overview}
            \input{5 - research plan/main.tex}
            \input{6 - summary/main.tex}
    
    
    %\section{}
    \newpage
    \pagenumbering{gobble}
        \printbibliography


    \newpage
    \pagenumbering{roman}
    \appendix
        \part{Appendices}
            \input{8 - Hilbert complexes/main.tex}
            \input{9 - weak conservation proofs/main.tex}
\end{document}

            \documentclass[12pt, a4paper]{report}

\input{template/main.tex}

\title{\BA{Title in Progress...}}
\author{Boris Andrews}
\affil{Mathematical Institute, University of Oxford}
\date{\today}


\begin{document}
    \pagenumbering{gobble}
    \maketitle
    
    
    \begin{abstract}
        Magnetic confinement reactors---in particular tokamaks---offer one of the most promising options for achieving practical nuclear fusion, with the potential to provide virtually limitless, clean energy. The theoretical and numerical modeling of tokamak plasmas is simultaneously an essential component of effective reactor design, and a great research barrier. Tokamak operational conditions exhibit comparatively low Knudsen numbers. Kinetic effects, including kinetic waves and instabilities, Landau damping, bump-on-tail instabilities and more, are therefore highly influential in tokamak plasma dynamics. Purely fluid models are inherently incapable of capturing these effects, whereas the high dimensionality in purely kinetic models render them practically intractable for most relevant purposes.

        We consider a $\delta\!f$ decomposition model, with a macroscopic fluid background and microscopic kinetic correction, both fully coupled to each other. A similar manner of discretization is proposed to that used in the recent \texttt{STRUPHY} code \cite{Holderied_Possanner_Wang_2021, Holderied_2022, Li_et_al_2023} with a finite-element model for the background and a pseudo-particle/PiC model for the correction.

        The fluid background satisfies the full, non-linear, resistive, compressible, Hall MHD equations. \cite{Laakmann_Hu_Farrell_2022} introduces finite-element(-in-space) implicit timesteppers for the incompressible analogue to this system with structure-preserving (SP) properties in the ideal case, alongside parameter-robust preconditioners. We show that these timesteppers can derive from a finite-element-in-time (FET) (and finite-element-in-space) interpretation. The benefits of this reformulation are discussed, including the derivation of timesteppers that are higher order in time, and the quantifiable dissipative SP properties in the non-ideal, resistive case.
        
        We discuss possible options for extending this FET approach to timesteppers for the compressible case.

        The kinetic corrections satisfy linearized Boltzmann equations. Using a Lénard--Bernstein collision operator, these take Fokker--Planck-like forms \cite{Fokker_1914, Planck_1917} wherein pseudo-particles in the numerical model obey the neoclassical transport equations, with particle-independent Brownian drift terms. This offers a rigorous methodology for incorporating collisions into the particle transport model, without coupling the equations of motions for each particle.
        
        Works by Chen, Chacón et al. \cite{Chen_Chacón_Barnes_2011, Chacón_Chen_Barnes_2013, Chen_Chacón_2014, Chen_Chacón_2015} have developed structure-preserving particle pushers for neoclassical transport in the Vlasov equations, derived from Crank--Nicolson integrators. We show these too can can derive from a FET interpretation, similarly offering potential extensions to higher-order-in-time particle pushers. The FET formulation is used also to consider how the stochastic drift terms can be incorporated into the pushers. Stochastic gyrokinetic expansions are also discussed.

        Different options for the numerical implementation of these schemes are considered.

        Due to the efficacy of FET in the development of SP timesteppers for both the fluid and kinetic component, we hope this approach will prove effective in the future for developing SP timesteppers for the full hybrid model. We hope this will give us the opportunity to incorporate previously inaccessible kinetic effects into the highly effective, modern, finite-element MHD models.
    \end{abstract}
    
    
    \newpage
    \tableofcontents
    
    
    \newpage
    \pagenumbering{arabic}
    %\linenumbers\renewcommand\thelinenumber{\color{black!50}\arabic{linenumber}}
            \input{0 - introduction/main.tex}
        \part{Research}
            \input{1 - low-noise PiC models/main.tex}
            \input{2 - kinetic component/main.tex}
            \input{3 - fluid component/main.tex}
            \input{4 - numerical implementation/main.tex}
        \part{Project Overview}
            \input{5 - research plan/main.tex}
            \input{6 - summary/main.tex}
    
    
    %\section{}
    \newpage
    \pagenumbering{gobble}
        \printbibliography


    \newpage
    \pagenumbering{roman}
    \appendix
        \part{Appendices}
            \input{8 - Hilbert complexes/main.tex}
            \input{9 - weak conservation proofs/main.tex}
\end{document}

            \documentclass[12pt, a4paper]{report}

\input{template/main.tex}

\title{\BA{Title in Progress...}}
\author{Boris Andrews}
\affil{Mathematical Institute, University of Oxford}
\date{\today}


\begin{document}
    \pagenumbering{gobble}
    \maketitle
    
    
    \begin{abstract}
        Magnetic confinement reactors---in particular tokamaks---offer one of the most promising options for achieving practical nuclear fusion, with the potential to provide virtually limitless, clean energy. The theoretical and numerical modeling of tokamak plasmas is simultaneously an essential component of effective reactor design, and a great research barrier. Tokamak operational conditions exhibit comparatively low Knudsen numbers. Kinetic effects, including kinetic waves and instabilities, Landau damping, bump-on-tail instabilities and more, are therefore highly influential in tokamak plasma dynamics. Purely fluid models are inherently incapable of capturing these effects, whereas the high dimensionality in purely kinetic models render them practically intractable for most relevant purposes.

        We consider a $\delta\!f$ decomposition model, with a macroscopic fluid background and microscopic kinetic correction, both fully coupled to each other. A similar manner of discretization is proposed to that used in the recent \texttt{STRUPHY} code \cite{Holderied_Possanner_Wang_2021, Holderied_2022, Li_et_al_2023} with a finite-element model for the background and a pseudo-particle/PiC model for the correction.

        The fluid background satisfies the full, non-linear, resistive, compressible, Hall MHD equations. \cite{Laakmann_Hu_Farrell_2022} introduces finite-element(-in-space) implicit timesteppers for the incompressible analogue to this system with structure-preserving (SP) properties in the ideal case, alongside parameter-robust preconditioners. We show that these timesteppers can derive from a finite-element-in-time (FET) (and finite-element-in-space) interpretation. The benefits of this reformulation are discussed, including the derivation of timesteppers that are higher order in time, and the quantifiable dissipative SP properties in the non-ideal, resistive case.
        
        We discuss possible options for extending this FET approach to timesteppers for the compressible case.

        The kinetic corrections satisfy linearized Boltzmann equations. Using a Lénard--Bernstein collision operator, these take Fokker--Planck-like forms \cite{Fokker_1914, Planck_1917} wherein pseudo-particles in the numerical model obey the neoclassical transport equations, with particle-independent Brownian drift terms. This offers a rigorous methodology for incorporating collisions into the particle transport model, without coupling the equations of motions for each particle.
        
        Works by Chen, Chacón et al. \cite{Chen_Chacón_Barnes_2011, Chacón_Chen_Barnes_2013, Chen_Chacón_2014, Chen_Chacón_2015} have developed structure-preserving particle pushers for neoclassical transport in the Vlasov equations, derived from Crank--Nicolson integrators. We show these too can can derive from a FET interpretation, similarly offering potential extensions to higher-order-in-time particle pushers. The FET formulation is used also to consider how the stochastic drift terms can be incorporated into the pushers. Stochastic gyrokinetic expansions are also discussed.

        Different options for the numerical implementation of these schemes are considered.

        Due to the efficacy of FET in the development of SP timesteppers for both the fluid and kinetic component, we hope this approach will prove effective in the future for developing SP timesteppers for the full hybrid model. We hope this will give us the opportunity to incorporate previously inaccessible kinetic effects into the highly effective, modern, finite-element MHD models.
    \end{abstract}
    
    
    \newpage
    \tableofcontents
    
    
    \newpage
    \pagenumbering{arabic}
    %\linenumbers\renewcommand\thelinenumber{\color{black!50}\arabic{linenumber}}
            \input{0 - introduction/main.tex}
        \part{Research}
            \input{1 - low-noise PiC models/main.tex}
            \input{2 - kinetic component/main.tex}
            \input{3 - fluid component/main.tex}
            \input{4 - numerical implementation/main.tex}
        \part{Project Overview}
            \input{5 - research plan/main.tex}
            \input{6 - summary/main.tex}
    
    
    %\section{}
    \newpage
    \pagenumbering{gobble}
        \printbibliography


    \newpage
    \pagenumbering{roman}
    \appendix
        \part{Appendices}
            \input{8 - Hilbert complexes/main.tex}
            \input{9 - weak conservation proofs/main.tex}
\end{document}

            \documentclass[12pt, a4paper]{report}

\input{template/main.tex}

\title{\BA{Title in Progress...}}
\author{Boris Andrews}
\affil{Mathematical Institute, University of Oxford}
\date{\today}


\begin{document}
    \pagenumbering{gobble}
    \maketitle
    
    
    \begin{abstract}
        Magnetic confinement reactors---in particular tokamaks---offer one of the most promising options for achieving practical nuclear fusion, with the potential to provide virtually limitless, clean energy. The theoretical and numerical modeling of tokamak plasmas is simultaneously an essential component of effective reactor design, and a great research barrier. Tokamak operational conditions exhibit comparatively low Knudsen numbers. Kinetic effects, including kinetic waves and instabilities, Landau damping, bump-on-tail instabilities and more, are therefore highly influential in tokamak plasma dynamics. Purely fluid models are inherently incapable of capturing these effects, whereas the high dimensionality in purely kinetic models render them practically intractable for most relevant purposes.

        We consider a $\delta\!f$ decomposition model, with a macroscopic fluid background and microscopic kinetic correction, both fully coupled to each other. A similar manner of discretization is proposed to that used in the recent \texttt{STRUPHY} code \cite{Holderied_Possanner_Wang_2021, Holderied_2022, Li_et_al_2023} with a finite-element model for the background and a pseudo-particle/PiC model for the correction.

        The fluid background satisfies the full, non-linear, resistive, compressible, Hall MHD equations. \cite{Laakmann_Hu_Farrell_2022} introduces finite-element(-in-space) implicit timesteppers for the incompressible analogue to this system with structure-preserving (SP) properties in the ideal case, alongside parameter-robust preconditioners. We show that these timesteppers can derive from a finite-element-in-time (FET) (and finite-element-in-space) interpretation. The benefits of this reformulation are discussed, including the derivation of timesteppers that are higher order in time, and the quantifiable dissipative SP properties in the non-ideal, resistive case.
        
        We discuss possible options for extending this FET approach to timesteppers for the compressible case.

        The kinetic corrections satisfy linearized Boltzmann equations. Using a Lénard--Bernstein collision operator, these take Fokker--Planck-like forms \cite{Fokker_1914, Planck_1917} wherein pseudo-particles in the numerical model obey the neoclassical transport equations, with particle-independent Brownian drift terms. This offers a rigorous methodology for incorporating collisions into the particle transport model, without coupling the equations of motions for each particle.
        
        Works by Chen, Chacón et al. \cite{Chen_Chacón_Barnes_2011, Chacón_Chen_Barnes_2013, Chen_Chacón_2014, Chen_Chacón_2015} have developed structure-preserving particle pushers for neoclassical transport in the Vlasov equations, derived from Crank--Nicolson integrators. We show these too can can derive from a FET interpretation, similarly offering potential extensions to higher-order-in-time particle pushers. The FET formulation is used also to consider how the stochastic drift terms can be incorporated into the pushers. Stochastic gyrokinetic expansions are also discussed.

        Different options for the numerical implementation of these schemes are considered.

        Due to the efficacy of FET in the development of SP timesteppers for both the fluid and kinetic component, we hope this approach will prove effective in the future for developing SP timesteppers for the full hybrid model. We hope this will give us the opportunity to incorporate previously inaccessible kinetic effects into the highly effective, modern, finite-element MHD models.
    \end{abstract}
    
    
    \newpage
    \tableofcontents
    
    
    \newpage
    \pagenumbering{arabic}
    %\linenumbers\renewcommand\thelinenumber{\color{black!50}\arabic{linenumber}}
            \input{0 - introduction/main.tex}
        \part{Research}
            \input{1 - low-noise PiC models/main.tex}
            \input{2 - kinetic component/main.tex}
            \input{3 - fluid component/main.tex}
            \input{4 - numerical implementation/main.tex}
        \part{Project Overview}
            \input{5 - research plan/main.tex}
            \input{6 - summary/main.tex}
    
    
    %\section{}
    \newpage
    \pagenumbering{gobble}
        \printbibliography


    \newpage
    \pagenumbering{roman}
    \appendix
        \part{Appendices}
            \input{8 - Hilbert complexes/main.tex}
            \input{9 - weak conservation proofs/main.tex}
\end{document}

        \part{Project Overview}
            \documentclass[12pt, a4paper]{report}

\input{template/main.tex}

\title{\BA{Title in Progress...}}
\author{Boris Andrews}
\affil{Mathematical Institute, University of Oxford}
\date{\today}


\begin{document}
    \pagenumbering{gobble}
    \maketitle
    
    
    \begin{abstract}
        Magnetic confinement reactors---in particular tokamaks---offer one of the most promising options for achieving practical nuclear fusion, with the potential to provide virtually limitless, clean energy. The theoretical and numerical modeling of tokamak plasmas is simultaneously an essential component of effective reactor design, and a great research barrier. Tokamak operational conditions exhibit comparatively low Knudsen numbers. Kinetic effects, including kinetic waves and instabilities, Landau damping, bump-on-tail instabilities and more, are therefore highly influential in tokamak plasma dynamics. Purely fluid models are inherently incapable of capturing these effects, whereas the high dimensionality in purely kinetic models render them practically intractable for most relevant purposes.

        We consider a $\delta\!f$ decomposition model, with a macroscopic fluid background and microscopic kinetic correction, both fully coupled to each other. A similar manner of discretization is proposed to that used in the recent \texttt{STRUPHY} code \cite{Holderied_Possanner_Wang_2021, Holderied_2022, Li_et_al_2023} with a finite-element model for the background and a pseudo-particle/PiC model for the correction.

        The fluid background satisfies the full, non-linear, resistive, compressible, Hall MHD equations. \cite{Laakmann_Hu_Farrell_2022} introduces finite-element(-in-space) implicit timesteppers for the incompressible analogue to this system with structure-preserving (SP) properties in the ideal case, alongside parameter-robust preconditioners. We show that these timesteppers can derive from a finite-element-in-time (FET) (and finite-element-in-space) interpretation. The benefits of this reformulation are discussed, including the derivation of timesteppers that are higher order in time, and the quantifiable dissipative SP properties in the non-ideal, resistive case.
        
        We discuss possible options for extending this FET approach to timesteppers for the compressible case.

        The kinetic corrections satisfy linearized Boltzmann equations. Using a Lénard--Bernstein collision operator, these take Fokker--Planck-like forms \cite{Fokker_1914, Planck_1917} wherein pseudo-particles in the numerical model obey the neoclassical transport equations, with particle-independent Brownian drift terms. This offers a rigorous methodology for incorporating collisions into the particle transport model, without coupling the equations of motions for each particle.
        
        Works by Chen, Chacón et al. \cite{Chen_Chacón_Barnes_2011, Chacón_Chen_Barnes_2013, Chen_Chacón_2014, Chen_Chacón_2015} have developed structure-preserving particle pushers for neoclassical transport in the Vlasov equations, derived from Crank--Nicolson integrators. We show these too can can derive from a FET interpretation, similarly offering potential extensions to higher-order-in-time particle pushers. The FET formulation is used also to consider how the stochastic drift terms can be incorporated into the pushers. Stochastic gyrokinetic expansions are also discussed.

        Different options for the numerical implementation of these schemes are considered.

        Due to the efficacy of FET in the development of SP timesteppers for both the fluid and kinetic component, we hope this approach will prove effective in the future for developing SP timesteppers for the full hybrid model. We hope this will give us the opportunity to incorporate previously inaccessible kinetic effects into the highly effective, modern, finite-element MHD models.
    \end{abstract}
    
    
    \newpage
    \tableofcontents
    
    
    \newpage
    \pagenumbering{arabic}
    %\linenumbers\renewcommand\thelinenumber{\color{black!50}\arabic{linenumber}}
            \input{0 - introduction/main.tex}
        \part{Research}
            \input{1 - low-noise PiC models/main.tex}
            \input{2 - kinetic component/main.tex}
            \input{3 - fluid component/main.tex}
            \input{4 - numerical implementation/main.tex}
        \part{Project Overview}
            \input{5 - research plan/main.tex}
            \input{6 - summary/main.tex}
    
    
    %\section{}
    \newpage
    \pagenumbering{gobble}
        \printbibliography


    \newpage
    \pagenumbering{roman}
    \appendix
        \part{Appendices}
            \input{8 - Hilbert complexes/main.tex}
            \input{9 - weak conservation proofs/main.tex}
\end{document}

            \documentclass[12pt, a4paper]{report}

\input{template/main.tex}

\title{\BA{Title in Progress...}}
\author{Boris Andrews}
\affil{Mathematical Institute, University of Oxford}
\date{\today}


\begin{document}
    \pagenumbering{gobble}
    \maketitle
    
    
    \begin{abstract}
        Magnetic confinement reactors---in particular tokamaks---offer one of the most promising options for achieving practical nuclear fusion, with the potential to provide virtually limitless, clean energy. The theoretical and numerical modeling of tokamak plasmas is simultaneously an essential component of effective reactor design, and a great research barrier. Tokamak operational conditions exhibit comparatively low Knudsen numbers. Kinetic effects, including kinetic waves and instabilities, Landau damping, bump-on-tail instabilities and more, are therefore highly influential in tokamak plasma dynamics. Purely fluid models are inherently incapable of capturing these effects, whereas the high dimensionality in purely kinetic models render them practically intractable for most relevant purposes.

        We consider a $\delta\!f$ decomposition model, with a macroscopic fluid background and microscopic kinetic correction, both fully coupled to each other. A similar manner of discretization is proposed to that used in the recent \texttt{STRUPHY} code \cite{Holderied_Possanner_Wang_2021, Holderied_2022, Li_et_al_2023} with a finite-element model for the background and a pseudo-particle/PiC model for the correction.

        The fluid background satisfies the full, non-linear, resistive, compressible, Hall MHD equations. \cite{Laakmann_Hu_Farrell_2022} introduces finite-element(-in-space) implicit timesteppers for the incompressible analogue to this system with structure-preserving (SP) properties in the ideal case, alongside parameter-robust preconditioners. We show that these timesteppers can derive from a finite-element-in-time (FET) (and finite-element-in-space) interpretation. The benefits of this reformulation are discussed, including the derivation of timesteppers that are higher order in time, and the quantifiable dissipative SP properties in the non-ideal, resistive case.
        
        We discuss possible options for extending this FET approach to timesteppers for the compressible case.

        The kinetic corrections satisfy linearized Boltzmann equations. Using a Lénard--Bernstein collision operator, these take Fokker--Planck-like forms \cite{Fokker_1914, Planck_1917} wherein pseudo-particles in the numerical model obey the neoclassical transport equations, with particle-independent Brownian drift terms. This offers a rigorous methodology for incorporating collisions into the particle transport model, without coupling the equations of motions for each particle.
        
        Works by Chen, Chacón et al. \cite{Chen_Chacón_Barnes_2011, Chacón_Chen_Barnes_2013, Chen_Chacón_2014, Chen_Chacón_2015} have developed structure-preserving particle pushers for neoclassical transport in the Vlasov equations, derived from Crank--Nicolson integrators. We show these too can can derive from a FET interpretation, similarly offering potential extensions to higher-order-in-time particle pushers. The FET formulation is used also to consider how the stochastic drift terms can be incorporated into the pushers. Stochastic gyrokinetic expansions are also discussed.

        Different options for the numerical implementation of these schemes are considered.

        Due to the efficacy of FET in the development of SP timesteppers for both the fluid and kinetic component, we hope this approach will prove effective in the future for developing SP timesteppers for the full hybrid model. We hope this will give us the opportunity to incorporate previously inaccessible kinetic effects into the highly effective, modern, finite-element MHD models.
    \end{abstract}
    
    
    \newpage
    \tableofcontents
    
    
    \newpage
    \pagenumbering{arabic}
    %\linenumbers\renewcommand\thelinenumber{\color{black!50}\arabic{linenumber}}
            \input{0 - introduction/main.tex}
        \part{Research}
            \input{1 - low-noise PiC models/main.tex}
            \input{2 - kinetic component/main.tex}
            \input{3 - fluid component/main.tex}
            \input{4 - numerical implementation/main.tex}
        \part{Project Overview}
            \input{5 - research plan/main.tex}
            \input{6 - summary/main.tex}
    
    
    %\section{}
    \newpage
    \pagenumbering{gobble}
        \printbibliography


    \newpage
    \pagenumbering{roman}
    \appendix
        \part{Appendices}
            \input{8 - Hilbert complexes/main.tex}
            \input{9 - weak conservation proofs/main.tex}
\end{document}

    
    
    %\section{}
    \newpage
    \pagenumbering{gobble}
        \printbibliography


    \newpage
    \pagenumbering{roman}
    \appendix
        \part{Appendices}
            \documentclass[12pt, a4paper]{report}

\input{template/main.tex}

\title{\BA{Title in Progress...}}
\author{Boris Andrews}
\affil{Mathematical Institute, University of Oxford}
\date{\today}


\begin{document}
    \pagenumbering{gobble}
    \maketitle
    
    
    \begin{abstract}
        Magnetic confinement reactors---in particular tokamaks---offer one of the most promising options for achieving practical nuclear fusion, with the potential to provide virtually limitless, clean energy. The theoretical and numerical modeling of tokamak plasmas is simultaneously an essential component of effective reactor design, and a great research barrier. Tokamak operational conditions exhibit comparatively low Knudsen numbers. Kinetic effects, including kinetic waves and instabilities, Landau damping, bump-on-tail instabilities and more, are therefore highly influential in tokamak plasma dynamics. Purely fluid models are inherently incapable of capturing these effects, whereas the high dimensionality in purely kinetic models render them practically intractable for most relevant purposes.

        We consider a $\delta\!f$ decomposition model, with a macroscopic fluid background and microscopic kinetic correction, both fully coupled to each other. A similar manner of discretization is proposed to that used in the recent \texttt{STRUPHY} code \cite{Holderied_Possanner_Wang_2021, Holderied_2022, Li_et_al_2023} with a finite-element model for the background and a pseudo-particle/PiC model for the correction.

        The fluid background satisfies the full, non-linear, resistive, compressible, Hall MHD equations. \cite{Laakmann_Hu_Farrell_2022} introduces finite-element(-in-space) implicit timesteppers for the incompressible analogue to this system with structure-preserving (SP) properties in the ideal case, alongside parameter-robust preconditioners. We show that these timesteppers can derive from a finite-element-in-time (FET) (and finite-element-in-space) interpretation. The benefits of this reformulation are discussed, including the derivation of timesteppers that are higher order in time, and the quantifiable dissipative SP properties in the non-ideal, resistive case.
        
        We discuss possible options for extending this FET approach to timesteppers for the compressible case.

        The kinetic corrections satisfy linearized Boltzmann equations. Using a Lénard--Bernstein collision operator, these take Fokker--Planck-like forms \cite{Fokker_1914, Planck_1917} wherein pseudo-particles in the numerical model obey the neoclassical transport equations, with particle-independent Brownian drift terms. This offers a rigorous methodology for incorporating collisions into the particle transport model, without coupling the equations of motions for each particle.
        
        Works by Chen, Chacón et al. \cite{Chen_Chacón_Barnes_2011, Chacón_Chen_Barnes_2013, Chen_Chacón_2014, Chen_Chacón_2015} have developed structure-preserving particle pushers for neoclassical transport in the Vlasov equations, derived from Crank--Nicolson integrators. We show these too can can derive from a FET interpretation, similarly offering potential extensions to higher-order-in-time particle pushers. The FET formulation is used also to consider how the stochastic drift terms can be incorporated into the pushers. Stochastic gyrokinetic expansions are also discussed.

        Different options for the numerical implementation of these schemes are considered.

        Due to the efficacy of FET in the development of SP timesteppers for both the fluid and kinetic component, we hope this approach will prove effective in the future for developing SP timesteppers for the full hybrid model. We hope this will give us the opportunity to incorporate previously inaccessible kinetic effects into the highly effective, modern, finite-element MHD models.
    \end{abstract}
    
    
    \newpage
    \tableofcontents
    
    
    \newpage
    \pagenumbering{arabic}
    %\linenumbers\renewcommand\thelinenumber{\color{black!50}\arabic{linenumber}}
            \input{0 - introduction/main.tex}
        \part{Research}
            \input{1 - low-noise PiC models/main.tex}
            \input{2 - kinetic component/main.tex}
            \input{3 - fluid component/main.tex}
            \input{4 - numerical implementation/main.tex}
        \part{Project Overview}
            \input{5 - research plan/main.tex}
            \input{6 - summary/main.tex}
    
    
    %\section{}
    \newpage
    \pagenumbering{gobble}
        \printbibliography


    \newpage
    \pagenumbering{roman}
    \appendix
        \part{Appendices}
            \input{8 - Hilbert complexes/main.tex}
            \input{9 - weak conservation proofs/main.tex}
\end{document}

            \documentclass[12pt, a4paper]{report}

\input{template/main.tex}

\title{\BA{Title in Progress...}}
\author{Boris Andrews}
\affil{Mathematical Institute, University of Oxford}
\date{\today}


\begin{document}
    \pagenumbering{gobble}
    \maketitle
    
    
    \begin{abstract}
        Magnetic confinement reactors---in particular tokamaks---offer one of the most promising options for achieving practical nuclear fusion, with the potential to provide virtually limitless, clean energy. The theoretical and numerical modeling of tokamak plasmas is simultaneously an essential component of effective reactor design, and a great research barrier. Tokamak operational conditions exhibit comparatively low Knudsen numbers. Kinetic effects, including kinetic waves and instabilities, Landau damping, bump-on-tail instabilities and more, are therefore highly influential in tokamak plasma dynamics. Purely fluid models are inherently incapable of capturing these effects, whereas the high dimensionality in purely kinetic models render them practically intractable for most relevant purposes.

        We consider a $\delta\!f$ decomposition model, with a macroscopic fluid background and microscopic kinetic correction, both fully coupled to each other. A similar manner of discretization is proposed to that used in the recent \texttt{STRUPHY} code \cite{Holderied_Possanner_Wang_2021, Holderied_2022, Li_et_al_2023} with a finite-element model for the background and a pseudo-particle/PiC model for the correction.

        The fluid background satisfies the full, non-linear, resistive, compressible, Hall MHD equations. \cite{Laakmann_Hu_Farrell_2022} introduces finite-element(-in-space) implicit timesteppers for the incompressible analogue to this system with structure-preserving (SP) properties in the ideal case, alongside parameter-robust preconditioners. We show that these timesteppers can derive from a finite-element-in-time (FET) (and finite-element-in-space) interpretation. The benefits of this reformulation are discussed, including the derivation of timesteppers that are higher order in time, and the quantifiable dissipative SP properties in the non-ideal, resistive case.
        
        We discuss possible options for extending this FET approach to timesteppers for the compressible case.

        The kinetic corrections satisfy linearized Boltzmann equations. Using a Lénard--Bernstein collision operator, these take Fokker--Planck-like forms \cite{Fokker_1914, Planck_1917} wherein pseudo-particles in the numerical model obey the neoclassical transport equations, with particle-independent Brownian drift terms. This offers a rigorous methodology for incorporating collisions into the particle transport model, without coupling the equations of motions for each particle.
        
        Works by Chen, Chacón et al. \cite{Chen_Chacón_Barnes_2011, Chacón_Chen_Barnes_2013, Chen_Chacón_2014, Chen_Chacón_2015} have developed structure-preserving particle pushers for neoclassical transport in the Vlasov equations, derived from Crank--Nicolson integrators. We show these too can can derive from a FET interpretation, similarly offering potential extensions to higher-order-in-time particle pushers. The FET formulation is used also to consider how the stochastic drift terms can be incorporated into the pushers. Stochastic gyrokinetic expansions are also discussed.

        Different options for the numerical implementation of these schemes are considered.

        Due to the efficacy of FET in the development of SP timesteppers for both the fluid and kinetic component, we hope this approach will prove effective in the future for developing SP timesteppers for the full hybrid model. We hope this will give us the opportunity to incorporate previously inaccessible kinetic effects into the highly effective, modern, finite-element MHD models.
    \end{abstract}
    
    
    \newpage
    \tableofcontents
    
    
    \newpage
    \pagenumbering{arabic}
    %\linenumbers\renewcommand\thelinenumber{\color{black!50}\arabic{linenumber}}
            \input{0 - introduction/main.tex}
        \part{Research}
            \input{1 - low-noise PiC models/main.tex}
            \input{2 - kinetic component/main.tex}
            \input{3 - fluid component/main.tex}
            \input{4 - numerical implementation/main.tex}
        \part{Project Overview}
            \input{5 - research plan/main.tex}
            \input{6 - summary/main.tex}
    
    
    %\section{}
    \newpage
    \pagenumbering{gobble}
        \printbibliography


    \newpage
    \pagenumbering{roman}
    \appendix
        \part{Appendices}
            \input{8 - Hilbert complexes/main.tex}
            \input{9 - weak conservation proofs/main.tex}
\end{document}

\end{document}

        \part{Project Overview}
            \documentclass[12pt, a4paper]{report}

\documentclass[12pt, a4paper]{report}

\input{template/main.tex}

\title{\BA{Title in Progress...}}
\author{Boris Andrews}
\affil{Mathematical Institute, University of Oxford}
\date{\today}


\begin{document}
    \pagenumbering{gobble}
    \maketitle
    
    
    \begin{abstract}
        Magnetic confinement reactors---in particular tokamaks---offer one of the most promising options for achieving practical nuclear fusion, with the potential to provide virtually limitless, clean energy. The theoretical and numerical modeling of tokamak plasmas is simultaneously an essential component of effective reactor design, and a great research barrier. Tokamak operational conditions exhibit comparatively low Knudsen numbers. Kinetic effects, including kinetic waves and instabilities, Landau damping, bump-on-tail instabilities and more, are therefore highly influential in tokamak plasma dynamics. Purely fluid models are inherently incapable of capturing these effects, whereas the high dimensionality in purely kinetic models render them practically intractable for most relevant purposes.

        We consider a $\delta\!f$ decomposition model, with a macroscopic fluid background and microscopic kinetic correction, both fully coupled to each other. A similar manner of discretization is proposed to that used in the recent \texttt{STRUPHY} code \cite{Holderied_Possanner_Wang_2021, Holderied_2022, Li_et_al_2023} with a finite-element model for the background and a pseudo-particle/PiC model for the correction.

        The fluid background satisfies the full, non-linear, resistive, compressible, Hall MHD equations. \cite{Laakmann_Hu_Farrell_2022} introduces finite-element(-in-space) implicit timesteppers for the incompressible analogue to this system with structure-preserving (SP) properties in the ideal case, alongside parameter-robust preconditioners. We show that these timesteppers can derive from a finite-element-in-time (FET) (and finite-element-in-space) interpretation. The benefits of this reformulation are discussed, including the derivation of timesteppers that are higher order in time, and the quantifiable dissipative SP properties in the non-ideal, resistive case.
        
        We discuss possible options for extending this FET approach to timesteppers for the compressible case.

        The kinetic corrections satisfy linearized Boltzmann equations. Using a Lénard--Bernstein collision operator, these take Fokker--Planck-like forms \cite{Fokker_1914, Planck_1917} wherein pseudo-particles in the numerical model obey the neoclassical transport equations, with particle-independent Brownian drift terms. This offers a rigorous methodology for incorporating collisions into the particle transport model, without coupling the equations of motions for each particle.
        
        Works by Chen, Chacón et al. \cite{Chen_Chacón_Barnes_2011, Chacón_Chen_Barnes_2013, Chen_Chacón_2014, Chen_Chacón_2015} have developed structure-preserving particle pushers for neoclassical transport in the Vlasov equations, derived from Crank--Nicolson integrators. We show these too can can derive from a FET interpretation, similarly offering potential extensions to higher-order-in-time particle pushers. The FET formulation is used also to consider how the stochastic drift terms can be incorporated into the pushers. Stochastic gyrokinetic expansions are also discussed.

        Different options for the numerical implementation of these schemes are considered.

        Due to the efficacy of FET in the development of SP timesteppers for both the fluid and kinetic component, we hope this approach will prove effective in the future for developing SP timesteppers for the full hybrid model. We hope this will give us the opportunity to incorporate previously inaccessible kinetic effects into the highly effective, modern, finite-element MHD models.
    \end{abstract}
    
    
    \newpage
    \tableofcontents
    
    
    \newpage
    \pagenumbering{arabic}
    %\linenumbers\renewcommand\thelinenumber{\color{black!50}\arabic{linenumber}}
            \input{0 - introduction/main.tex}
        \part{Research}
            \input{1 - low-noise PiC models/main.tex}
            \input{2 - kinetic component/main.tex}
            \input{3 - fluid component/main.tex}
            \input{4 - numerical implementation/main.tex}
        \part{Project Overview}
            \input{5 - research plan/main.tex}
            \input{6 - summary/main.tex}
    
    
    %\section{}
    \newpage
    \pagenumbering{gobble}
        \printbibliography


    \newpage
    \pagenumbering{roman}
    \appendix
        \part{Appendices}
            \input{8 - Hilbert complexes/main.tex}
            \input{9 - weak conservation proofs/main.tex}
\end{document}


\title{\BA{Title in Progress...}}
\author{Boris Andrews}
\affil{Mathematical Institute, University of Oxford}
\date{\today}


\begin{document}
    \pagenumbering{gobble}
    \maketitle
    
    
    \begin{abstract}
        Magnetic confinement reactors---in particular tokamaks---offer one of the most promising options for achieving practical nuclear fusion, with the potential to provide virtually limitless, clean energy. The theoretical and numerical modeling of tokamak plasmas is simultaneously an essential component of effective reactor design, and a great research barrier. Tokamak operational conditions exhibit comparatively low Knudsen numbers. Kinetic effects, including kinetic waves and instabilities, Landau damping, bump-on-tail instabilities and more, are therefore highly influential in tokamak plasma dynamics. Purely fluid models are inherently incapable of capturing these effects, whereas the high dimensionality in purely kinetic models render them practically intractable for most relevant purposes.

        We consider a $\delta\!f$ decomposition model, with a macroscopic fluid background and microscopic kinetic correction, both fully coupled to each other. A similar manner of discretization is proposed to that used in the recent \texttt{STRUPHY} code \cite{Holderied_Possanner_Wang_2021, Holderied_2022, Li_et_al_2023} with a finite-element model for the background and a pseudo-particle/PiC model for the correction.

        The fluid background satisfies the full, non-linear, resistive, compressible, Hall MHD equations. \cite{Laakmann_Hu_Farrell_2022} introduces finite-element(-in-space) implicit timesteppers for the incompressible analogue to this system with structure-preserving (SP) properties in the ideal case, alongside parameter-robust preconditioners. We show that these timesteppers can derive from a finite-element-in-time (FET) (and finite-element-in-space) interpretation. The benefits of this reformulation are discussed, including the derivation of timesteppers that are higher order in time, and the quantifiable dissipative SP properties in the non-ideal, resistive case.
        
        We discuss possible options for extending this FET approach to timesteppers for the compressible case.

        The kinetic corrections satisfy linearized Boltzmann equations. Using a Lénard--Bernstein collision operator, these take Fokker--Planck-like forms \cite{Fokker_1914, Planck_1917} wherein pseudo-particles in the numerical model obey the neoclassical transport equations, with particle-independent Brownian drift terms. This offers a rigorous methodology for incorporating collisions into the particle transport model, without coupling the equations of motions for each particle.
        
        Works by Chen, Chacón et al. \cite{Chen_Chacón_Barnes_2011, Chacón_Chen_Barnes_2013, Chen_Chacón_2014, Chen_Chacón_2015} have developed structure-preserving particle pushers for neoclassical transport in the Vlasov equations, derived from Crank--Nicolson integrators. We show these too can can derive from a FET interpretation, similarly offering potential extensions to higher-order-in-time particle pushers. The FET formulation is used also to consider how the stochastic drift terms can be incorporated into the pushers. Stochastic gyrokinetic expansions are also discussed.

        Different options for the numerical implementation of these schemes are considered.

        Due to the efficacy of FET in the development of SP timesteppers for both the fluid and kinetic component, we hope this approach will prove effective in the future for developing SP timesteppers for the full hybrid model. We hope this will give us the opportunity to incorporate previously inaccessible kinetic effects into the highly effective, modern, finite-element MHD models.
    \end{abstract}
    
    
    \newpage
    \tableofcontents
    
    
    \newpage
    \pagenumbering{arabic}
    %\linenumbers\renewcommand\thelinenumber{\color{black!50}\arabic{linenumber}}
            \documentclass[12pt, a4paper]{report}

\input{template/main.tex}

\title{\BA{Title in Progress...}}
\author{Boris Andrews}
\affil{Mathematical Institute, University of Oxford}
\date{\today}


\begin{document}
    \pagenumbering{gobble}
    \maketitle
    
    
    \begin{abstract}
        Magnetic confinement reactors---in particular tokamaks---offer one of the most promising options for achieving practical nuclear fusion, with the potential to provide virtually limitless, clean energy. The theoretical and numerical modeling of tokamak plasmas is simultaneously an essential component of effective reactor design, and a great research barrier. Tokamak operational conditions exhibit comparatively low Knudsen numbers. Kinetic effects, including kinetic waves and instabilities, Landau damping, bump-on-tail instabilities and more, are therefore highly influential in tokamak plasma dynamics. Purely fluid models are inherently incapable of capturing these effects, whereas the high dimensionality in purely kinetic models render them practically intractable for most relevant purposes.

        We consider a $\delta\!f$ decomposition model, with a macroscopic fluid background and microscopic kinetic correction, both fully coupled to each other. A similar manner of discretization is proposed to that used in the recent \texttt{STRUPHY} code \cite{Holderied_Possanner_Wang_2021, Holderied_2022, Li_et_al_2023} with a finite-element model for the background and a pseudo-particle/PiC model for the correction.

        The fluid background satisfies the full, non-linear, resistive, compressible, Hall MHD equations. \cite{Laakmann_Hu_Farrell_2022} introduces finite-element(-in-space) implicit timesteppers for the incompressible analogue to this system with structure-preserving (SP) properties in the ideal case, alongside parameter-robust preconditioners. We show that these timesteppers can derive from a finite-element-in-time (FET) (and finite-element-in-space) interpretation. The benefits of this reformulation are discussed, including the derivation of timesteppers that are higher order in time, and the quantifiable dissipative SP properties in the non-ideal, resistive case.
        
        We discuss possible options for extending this FET approach to timesteppers for the compressible case.

        The kinetic corrections satisfy linearized Boltzmann equations. Using a Lénard--Bernstein collision operator, these take Fokker--Planck-like forms \cite{Fokker_1914, Planck_1917} wherein pseudo-particles in the numerical model obey the neoclassical transport equations, with particle-independent Brownian drift terms. This offers a rigorous methodology for incorporating collisions into the particle transport model, without coupling the equations of motions for each particle.
        
        Works by Chen, Chacón et al. \cite{Chen_Chacón_Barnes_2011, Chacón_Chen_Barnes_2013, Chen_Chacón_2014, Chen_Chacón_2015} have developed structure-preserving particle pushers for neoclassical transport in the Vlasov equations, derived from Crank--Nicolson integrators. We show these too can can derive from a FET interpretation, similarly offering potential extensions to higher-order-in-time particle pushers. The FET formulation is used also to consider how the stochastic drift terms can be incorporated into the pushers. Stochastic gyrokinetic expansions are also discussed.

        Different options for the numerical implementation of these schemes are considered.

        Due to the efficacy of FET in the development of SP timesteppers for both the fluid and kinetic component, we hope this approach will prove effective in the future for developing SP timesteppers for the full hybrid model. We hope this will give us the opportunity to incorporate previously inaccessible kinetic effects into the highly effective, modern, finite-element MHD models.
    \end{abstract}
    
    
    \newpage
    \tableofcontents
    
    
    \newpage
    \pagenumbering{arabic}
    %\linenumbers\renewcommand\thelinenumber{\color{black!50}\arabic{linenumber}}
            \input{0 - introduction/main.tex}
        \part{Research}
            \input{1 - low-noise PiC models/main.tex}
            \input{2 - kinetic component/main.tex}
            \input{3 - fluid component/main.tex}
            \input{4 - numerical implementation/main.tex}
        \part{Project Overview}
            \input{5 - research plan/main.tex}
            \input{6 - summary/main.tex}
    
    
    %\section{}
    \newpage
    \pagenumbering{gobble}
        \printbibliography


    \newpage
    \pagenumbering{roman}
    \appendix
        \part{Appendices}
            \input{8 - Hilbert complexes/main.tex}
            \input{9 - weak conservation proofs/main.tex}
\end{document}

        \part{Research}
            \documentclass[12pt, a4paper]{report}

\input{template/main.tex}

\title{\BA{Title in Progress...}}
\author{Boris Andrews}
\affil{Mathematical Institute, University of Oxford}
\date{\today}


\begin{document}
    \pagenumbering{gobble}
    \maketitle
    
    
    \begin{abstract}
        Magnetic confinement reactors---in particular tokamaks---offer one of the most promising options for achieving practical nuclear fusion, with the potential to provide virtually limitless, clean energy. The theoretical and numerical modeling of tokamak plasmas is simultaneously an essential component of effective reactor design, and a great research barrier. Tokamak operational conditions exhibit comparatively low Knudsen numbers. Kinetic effects, including kinetic waves and instabilities, Landau damping, bump-on-tail instabilities and more, are therefore highly influential in tokamak plasma dynamics. Purely fluid models are inherently incapable of capturing these effects, whereas the high dimensionality in purely kinetic models render them practically intractable for most relevant purposes.

        We consider a $\delta\!f$ decomposition model, with a macroscopic fluid background and microscopic kinetic correction, both fully coupled to each other. A similar manner of discretization is proposed to that used in the recent \texttt{STRUPHY} code \cite{Holderied_Possanner_Wang_2021, Holderied_2022, Li_et_al_2023} with a finite-element model for the background and a pseudo-particle/PiC model for the correction.

        The fluid background satisfies the full, non-linear, resistive, compressible, Hall MHD equations. \cite{Laakmann_Hu_Farrell_2022} introduces finite-element(-in-space) implicit timesteppers for the incompressible analogue to this system with structure-preserving (SP) properties in the ideal case, alongside parameter-robust preconditioners. We show that these timesteppers can derive from a finite-element-in-time (FET) (and finite-element-in-space) interpretation. The benefits of this reformulation are discussed, including the derivation of timesteppers that are higher order in time, and the quantifiable dissipative SP properties in the non-ideal, resistive case.
        
        We discuss possible options for extending this FET approach to timesteppers for the compressible case.

        The kinetic corrections satisfy linearized Boltzmann equations. Using a Lénard--Bernstein collision operator, these take Fokker--Planck-like forms \cite{Fokker_1914, Planck_1917} wherein pseudo-particles in the numerical model obey the neoclassical transport equations, with particle-independent Brownian drift terms. This offers a rigorous methodology for incorporating collisions into the particle transport model, without coupling the equations of motions for each particle.
        
        Works by Chen, Chacón et al. \cite{Chen_Chacón_Barnes_2011, Chacón_Chen_Barnes_2013, Chen_Chacón_2014, Chen_Chacón_2015} have developed structure-preserving particle pushers for neoclassical transport in the Vlasov equations, derived from Crank--Nicolson integrators. We show these too can can derive from a FET interpretation, similarly offering potential extensions to higher-order-in-time particle pushers. The FET formulation is used also to consider how the stochastic drift terms can be incorporated into the pushers. Stochastic gyrokinetic expansions are also discussed.

        Different options for the numerical implementation of these schemes are considered.

        Due to the efficacy of FET in the development of SP timesteppers for both the fluid and kinetic component, we hope this approach will prove effective in the future for developing SP timesteppers for the full hybrid model. We hope this will give us the opportunity to incorporate previously inaccessible kinetic effects into the highly effective, modern, finite-element MHD models.
    \end{abstract}
    
    
    \newpage
    \tableofcontents
    
    
    \newpage
    \pagenumbering{arabic}
    %\linenumbers\renewcommand\thelinenumber{\color{black!50}\arabic{linenumber}}
            \input{0 - introduction/main.tex}
        \part{Research}
            \input{1 - low-noise PiC models/main.tex}
            \input{2 - kinetic component/main.tex}
            \input{3 - fluid component/main.tex}
            \input{4 - numerical implementation/main.tex}
        \part{Project Overview}
            \input{5 - research plan/main.tex}
            \input{6 - summary/main.tex}
    
    
    %\section{}
    \newpage
    \pagenumbering{gobble}
        \printbibliography


    \newpage
    \pagenumbering{roman}
    \appendix
        \part{Appendices}
            \input{8 - Hilbert complexes/main.tex}
            \input{9 - weak conservation proofs/main.tex}
\end{document}

            \documentclass[12pt, a4paper]{report}

\input{template/main.tex}

\title{\BA{Title in Progress...}}
\author{Boris Andrews}
\affil{Mathematical Institute, University of Oxford}
\date{\today}


\begin{document}
    \pagenumbering{gobble}
    \maketitle
    
    
    \begin{abstract}
        Magnetic confinement reactors---in particular tokamaks---offer one of the most promising options for achieving practical nuclear fusion, with the potential to provide virtually limitless, clean energy. The theoretical and numerical modeling of tokamak plasmas is simultaneously an essential component of effective reactor design, and a great research barrier. Tokamak operational conditions exhibit comparatively low Knudsen numbers. Kinetic effects, including kinetic waves and instabilities, Landau damping, bump-on-tail instabilities and more, are therefore highly influential in tokamak plasma dynamics. Purely fluid models are inherently incapable of capturing these effects, whereas the high dimensionality in purely kinetic models render them practically intractable for most relevant purposes.

        We consider a $\delta\!f$ decomposition model, with a macroscopic fluid background and microscopic kinetic correction, both fully coupled to each other. A similar manner of discretization is proposed to that used in the recent \texttt{STRUPHY} code \cite{Holderied_Possanner_Wang_2021, Holderied_2022, Li_et_al_2023} with a finite-element model for the background and a pseudo-particle/PiC model for the correction.

        The fluid background satisfies the full, non-linear, resistive, compressible, Hall MHD equations. \cite{Laakmann_Hu_Farrell_2022} introduces finite-element(-in-space) implicit timesteppers for the incompressible analogue to this system with structure-preserving (SP) properties in the ideal case, alongside parameter-robust preconditioners. We show that these timesteppers can derive from a finite-element-in-time (FET) (and finite-element-in-space) interpretation. The benefits of this reformulation are discussed, including the derivation of timesteppers that are higher order in time, and the quantifiable dissipative SP properties in the non-ideal, resistive case.
        
        We discuss possible options for extending this FET approach to timesteppers for the compressible case.

        The kinetic corrections satisfy linearized Boltzmann equations. Using a Lénard--Bernstein collision operator, these take Fokker--Planck-like forms \cite{Fokker_1914, Planck_1917} wherein pseudo-particles in the numerical model obey the neoclassical transport equations, with particle-independent Brownian drift terms. This offers a rigorous methodology for incorporating collisions into the particle transport model, without coupling the equations of motions for each particle.
        
        Works by Chen, Chacón et al. \cite{Chen_Chacón_Barnes_2011, Chacón_Chen_Barnes_2013, Chen_Chacón_2014, Chen_Chacón_2015} have developed structure-preserving particle pushers for neoclassical transport in the Vlasov equations, derived from Crank--Nicolson integrators. We show these too can can derive from a FET interpretation, similarly offering potential extensions to higher-order-in-time particle pushers. The FET formulation is used also to consider how the stochastic drift terms can be incorporated into the pushers. Stochastic gyrokinetic expansions are also discussed.

        Different options for the numerical implementation of these schemes are considered.

        Due to the efficacy of FET in the development of SP timesteppers for both the fluid and kinetic component, we hope this approach will prove effective in the future for developing SP timesteppers for the full hybrid model. We hope this will give us the opportunity to incorporate previously inaccessible kinetic effects into the highly effective, modern, finite-element MHD models.
    \end{abstract}
    
    
    \newpage
    \tableofcontents
    
    
    \newpage
    \pagenumbering{arabic}
    %\linenumbers\renewcommand\thelinenumber{\color{black!50}\arabic{linenumber}}
            \input{0 - introduction/main.tex}
        \part{Research}
            \input{1 - low-noise PiC models/main.tex}
            \input{2 - kinetic component/main.tex}
            \input{3 - fluid component/main.tex}
            \input{4 - numerical implementation/main.tex}
        \part{Project Overview}
            \input{5 - research plan/main.tex}
            \input{6 - summary/main.tex}
    
    
    %\section{}
    \newpage
    \pagenumbering{gobble}
        \printbibliography


    \newpage
    \pagenumbering{roman}
    \appendix
        \part{Appendices}
            \input{8 - Hilbert complexes/main.tex}
            \input{9 - weak conservation proofs/main.tex}
\end{document}

            \documentclass[12pt, a4paper]{report}

\input{template/main.tex}

\title{\BA{Title in Progress...}}
\author{Boris Andrews}
\affil{Mathematical Institute, University of Oxford}
\date{\today}


\begin{document}
    \pagenumbering{gobble}
    \maketitle
    
    
    \begin{abstract}
        Magnetic confinement reactors---in particular tokamaks---offer one of the most promising options for achieving practical nuclear fusion, with the potential to provide virtually limitless, clean energy. The theoretical and numerical modeling of tokamak plasmas is simultaneously an essential component of effective reactor design, and a great research barrier. Tokamak operational conditions exhibit comparatively low Knudsen numbers. Kinetic effects, including kinetic waves and instabilities, Landau damping, bump-on-tail instabilities and more, are therefore highly influential in tokamak plasma dynamics. Purely fluid models are inherently incapable of capturing these effects, whereas the high dimensionality in purely kinetic models render them practically intractable for most relevant purposes.

        We consider a $\delta\!f$ decomposition model, with a macroscopic fluid background and microscopic kinetic correction, both fully coupled to each other. A similar manner of discretization is proposed to that used in the recent \texttt{STRUPHY} code \cite{Holderied_Possanner_Wang_2021, Holderied_2022, Li_et_al_2023} with a finite-element model for the background and a pseudo-particle/PiC model for the correction.

        The fluid background satisfies the full, non-linear, resistive, compressible, Hall MHD equations. \cite{Laakmann_Hu_Farrell_2022} introduces finite-element(-in-space) implicit timesteppers for the incompressible analogue to this system with structure-preserving (SP) properties in the ideal case, alongside parameter-robust preconditioners. We show that these timesteppers can derive from a finite-element-in-time (FET) (and finite-element-in-space) interpretation. The benefits of this reformulation are discussed, including the derivation of timesteppers that are higher order in time, and the quantifiable dissipative SP properties in the non-ideal, resistive case.
        
        We discuss possible options for extending this FET approach to timesteppers for the compressible case.

        The kinetic corrections satisfy linearized Boltzmann equations. Using a Lénard--Bernstein collision operator, these take Fokker--Planck-like forms \cite{Fokker_1914, Planck_1917} wherein pseudo-particles in the numerical model obey the neoclassical transport equations, with particle-independent Brownian drift terms. This offers a rigorous methodology for incorporating collisions into the particle transport model, without coupling the equations of motions for each particle.
        
        Works by Chen, Chacón et al. \cite{Chen_Chacón_Barnes_2011, Chacón_Chen_Barnes_2013, Chen_Chacón_2014, Chen_Chacón_2015} have developed structure-preserving particle pushers for neoclassical transport in the Vlasov equations, derived from Crank--Nicolson integrators. We show these too can can derive from a FET interpretation, similarly offering potential extensions to higher-order-in-time particle pushers. The FET formulation is used also to consider how the stochastic drift terms can be incorporated into the pushers. Stochastic gyrokinetic expansions are also discussed.

        Different options for the numerical implementation of these schemes are considered.

        Due to the efficacy of FET in the development of SP timesteppers for both the fluid and kinetic component, we hope this approach will prove effective in the future for developing SP timesteppers for the full hybrid model. We hope this will give us the opportunity to incorporate previously inaccessible kinetic effects into the highly effective, modern, finite-element MHD models.
    \end{abstract}
    
    
    \newpage
    \tableofcontents
    
    
    \newpage
    \pagenumbering{arabic}
    %\linenumbers\renewcommand\thelinenumber{\color{black!50}\arabic{linenumber}}
            \input{0 - introduction/main.tex}
        \part{Research}
            \input{1 - low-noise PiC models/main.tex}
            \input{2 - kinetic component/main.tex}
            \input{3 - fluid component/main.tex}
            \input{4 - numerical implementation/main.tex}
        \part{Project Overview}
            \input{5 - research plan/main.tex}
            \input{6 - summary/main.tex}
    
    
    %\section{}
    \newpage
    \pagenumbering{gobble}
        \printbibliography


    \newpage
    \pagenumbering{roman}
    \appendix
        \part{Appendices}
            \input{8 - Hilbert complexes/main.tex}
            \input{9 - weak conservation proofs/main.tex}
\end{document}

            \documentclass[12pt, a4paper]{report}

\input{template/main.tex}

\title{\BA{Title in Progress...}}
\author{Boris Andrews}
\affil{Mathematical Institute, University of Oxford}
\date{\today}


\begin{document}
    \pagenumbering{gobble}
    \maketitle
    
    
    \begin{abstract}
        Magnetic confinement reactors---in particular tokamaks---offer one of the most promising options for achieving practical nuclear fusion, with the potential to provide virtually limitless, clean energy. The theoretical and numerical modeling of tokamak plasmas is simultaneously an essential component of effective reactor design, and a great research barrier. Tokamak operational conditions exhibit comparatively low Knudsen numbers. Kinetic effects, including kinetic waves and instabilities, Landau damping, bump-on-tail instabilities and more, are therefore highly influential in tokamak plasma dynamics. Purely fluid models are inherently incapable of capturing these effects, whereas the high dimensionality in purely kinetic models render them practically intractable for most relevant purposes.

        We consider a $\delta\!f$ decomposition model, with a macroscopic fluid background and microscopic kinetic correction, both fully coupled to each other. A similar manner of discretization is proposed to that used in the recent \texttt{STRUPHY} code \cite{Holderied_Possanner_Wang_2021, Holderied_2022, Li_et_al_2023} with a finite-element model for the background and a pseudo-particle/PiC model for the correction.

        The fluid background satisfies the full, non-linear, resistive, compressible, Hall MHD equations. \cite{Laakmann_Hu_Farrell_2022} introduces finite-element(-in-space) implicit timesteppers for the incompressible analogue to this system with structure-preserving (SP) properties in the ideal case, alongside parameter-robust preconditioners. We show that these timesteppers can derive from a finite-element-in-time (FET) (and finite-element-in-space) interpretation. The benefits of this reformulation are discussed, including the derivation of timesteppers that are higher order in time, and the quantifiable dissipative SP properties in the non-ideal, resistive case.
        
        We discuss possible options for extending this FET approach to timesteppers for the compressible case.

        The kinetic corrections satisfy linearized Boltzmann equations. Using a Lénard--Bernstein collision operator, these take Fokker--Planck-like forms \cite{Fokker_1914, Planck_1917} wherein pseudo-particles in the numerical model obey the neoclassical transport equations, with particle-independent Brownian drift terms. This offers a rigorous methodology for incorporating collisions into the particle transport model, without coupling the equations of motions for each particle.
        
        Works by Chen, Chacón et al. \cite{Chen_Chacón_Barnes_2011, Chacón_Chen_Barnes_2013, Chen_Chacón_2014, Chen_Chacón_2015} have developed structure-preserving particle pushers for neoclassical transport in the Vlasov equations, derived from Crank--Nicolson integrators. We show these too can can derive from a FET interpretation, similarly offering potential extensions to higher-order-in-time particle pushers. The FET formulation is used also to consider how the stochastic drift terms can be incorporated into the pushers. Stochastic gyrokinetic expansions are also discussed.

        Different options for the numerical implementation of these schemes are considered.

        Due to the efficacy of FET in the development of SP timesteppers for both the fluid and kinetic component, we hope this approach will prove effective in the future for developing SP timesteppers for the full hybrid model. We hope this will give us the opportunity to incorporate previously inaccessible kinetic effects into the highly effective, modern, finite-element MHD models.
    \end{abstract}
    
    
    \newpage
    \tableofcontents
    
    
    \newpage
    \pagenumbering{arabic}
    %\linenumbers\renewcommand\thelinenumber{\color{black!50}\arabic{linenumber}}
            \input{0 - introduction/main.tex}
        \part{Research}
            \input{1 - low-noise PiC models/main.tex}
            \input{2 - kinetic component/main.tex}
            \input{3 - fluid component/main.tex}
            \input{4 - numerical implementation/main.tex}
        \part{Project Overview}
            \input{5 - research plan/main.tex}
            \input{6 - summary/main.tex}
    
    
    %\section{}
    \newpage
    \pagenumbering{gobble}
        \printbibliography


    \newpage
    \pagenumbering{roman}
    \appendix
        \part{Appendices}
            \input{8 - Hilbert complexes/main.tex}
            \input{9 - weak conservation proofs/main.tex}
\end{document}

        \part{Project Overview}
            \documentclass[12pt, a4paper]{report}

\input{template/main.tex}

\title{\BA{Title in Progress...}}
\author{Boris Andrews}
\affil{Mathematical Institute, University of Oxford}
\date{\today}


\begin{document}
    \pagenumbering{gobble}
    \maketitle
    
    
    \begin{abstract}
        Magnetic confinement reactors---in particular tokamaks---offer one of the most promising options for achieving practical nuclear fusion, with the potential to provide virtually limitless, clean energy. The theoretical and numerical modeling of tokamak plasmas is simultaneously an essential component of effective reactor design, and a great research barrier. Tokamak operational conditions exhibit comparatively low Knudsen numbers. Kinetic effects, including kinetic waves and instabilities, Landau damping, bump-on-tail instabilities and more, are therefore highly influential in tokamak plasma dynamics. Purely fluid models are inherently incapable of capturing these effects, whereas the high dimensionality in purely kinetic models render them practically intractable for most relevant purposes.

        We consider a $\delta\!f$ decomposition model, with a macroscopic fluid background and microscopic kinetic correction, both fully coupled to each other. A similar manner of discretization is proposed to that used in the recent \texttt{STRUPHY} code \cite{Holderied_Possanner_Wang_2021, Holderied_2022, Li_et_al_2023} with a finite-element model for the background and a pseudo-particle/PiC model for the correction.

        The fluid background satisfies the full, non-linear, resistive, compressible, Hall MHD equations. \cite{Laakmann_Hu_Farrell_2022} introduces finite-element(-in-space) implicit timesteppers for the incompressible analogue to this system with structure-preserving (SP) properties in the ideal case, alongside parameter-robust preconditioners. We show that these timesteppers can derive from a finite-element-in-time (FET) (and finite-element-in-space) interpretation. The benefits of this reformulation are discussed, including the derivation of timesteppers that are higher order in time, and the quantifiable dissipative SP properties in the non-ideal, resistive case.
        
        We discuss possible options for extending this FET approach to timesteppers for the compressible case.

        The kinetic corrections satisfy linearized Boltzmann equations. Using a Lénard--Bernstein collision operator, these take Fokker--Planck-like forms \cite{Fokker_1914, Planck_1917} wherein pseudo-particles in the numerical model obey the neoclassical transport equations, with particle-independent Brownian drift terms. This offers a rigorous methodology for incorporating collisions into the particle transport model, without coupling the equations of motions for each particle.
        
        Works by Chen, Chacón et al. \cite{Chen_Chacón_Barnes_2011, Chacón_Chen_Barnes_2013, Chen_Chacón_2014, Chen_Chacón_2015} have developed structure-preserving particle pushers for neoclassical transport in the Vlasov equations, derived from Crank--Nicolson integrators. We show these too can can derive from a FET interpretation, similarly offering potential extensions to higher-order-in-time particle pushers. The FET formulation is used also to consider how the stochastic drift terms can be incorporated into the pushers. Stochastic gyrokinetic expansions are also discussed.

        Different options for the numerical implementation of these schemes are considered.

        Due to the efficacy of FET in the development of SP timesteppers for both the fluid and kinetic component, we hope this approach will prove effective in the future for developing SP timesteppers for the full hybrid model. We hope this will give us the opportunity to incorporate previously inaccessible kinetic effects into the highly effective, modern, finite-element MHD models.
    \end{abstract}
    
    
    \newpage
    \tableofcontents
    
    
    \newpage
    \pagenumbering{arabic}
    %\linenumbers\renewcommand\thelinenumber{\color{black!50}\arabic{linenumber}}
            \input{0 - introduction/main.tex}
        \part{Research}
            \input{1 - low-noise PiC models/main.tex}
            \input{2 - kinetic component/main.tex}
            \input{3 - fluid component/main.tex}
            \input{4 - numerical implementation/main.tex}
        \part{Project Overview}
            \input{5 - research plan/main.tex}
            \input{6 - summary/main.tex}
    
    
    %\section{}
    \newpage
    \pagenumbering{gobble}
        \printbibliography


    \newpage
    \pagenumbering{roman}
    \appendix
        \part{Appendices}
            \input{8 - Hilbert complexes/main.tex}
            \input{9 - weak conservation proofs/main.tex}
\end{document}

            \documentclass[12pt, a4paper]{report}

\input{template/main.tex}

\title{\BA{Title in Progress...}}
\author{Boris Andrews}
\affil{Mathematical Institute, University of Oxford}
\date{\today}


\begin{document}
    \pagenumbering{gobble}
    \maketitle
    
    
    \begin{abstract}
        Magnetic confinement reactors---in particular tokamaks---offer one of the most promising options for achieving practical nuclear fusion, with the potential to provide virtually limitless, clean energy. The theoretical and numerical modeling of tokamak plasmas is simultaneously an essential component of effective reactor design, and a great research barrier. Tokamak operational conditions exhibit comparatively low Knudsen numbers. Kinetic effects, including kinetic waves and instabilities, Landau damping, bump-on-tail instabilities and more, are therefore highly influential in tokamak plasma dynamics. Purely fluid models are inherently incapable of capturing these effects, whereas the high dimensionality in purely kinetic models render them practically intractable for most relevant purposes.

        We consider a $\delta\!f$ decomposition model, with a macroscopic fluid background and microscopic kinetic correction, both fully coupled to each other. A similar manner of discretization is proposed to that used in the recent \texttt{STRUPHY} code \cite{Holderied_Possanner_Wang_2021, Holderied_2022, Li_et_al_2023} with a finite-element model for the background and a pseudo-particle/PiC model for the correction.

        The fluid background satisfies the full, non-linear, resistive, compressible, Hall MHD equations. \cite{Laakmann_Hu_Farrell_2022} introduces finite-element(-in-space) implicit timesteppers for the incompressible analogue to this system with structure-preserving (SP) properties in the ideal case, alongside parameter-robust preconditioners. We show that these timesteppers can derive from a finite-element-in-time (FET) (and finite-element-in-space) interpretation. The benefits of this reformulation are discussed, including the derivation of timesteppers that are higher order in time, and the quantifiable dissipative SP properties in the non-ideal, resistive case.
        
        We discuss possible options for extending this FET approach to timesteppers for the compressible case.

        The kinetic corrections satisfy linearized Boltzmann equations. Using a Lénard--Bernstein collision operator, these take Fokker--Planck-like forms \cite{Fokker_1914, Planck_1917} wherein pseudo-particles in the numerical model obey the neoclassical transport equations, with particle-independent Brownian drift terms. This offers a rigorous methodology for incorporating collisions into the particle transport model, without coupling the equations of motions for each particle.
        
        Works by Chen, Chacón et al. \cite{Chen_Chacón_Barnes_2011, Chacón_Chen_Barnes_2013, Chen_Chacón_2014, Chen_Chacón_2015} have developed structure-preserving particle pushers for neoclassical transport in the Vlasov equations, derived from Crank--Nicolson integrators. We show these too can can derive from a FET interpretation, similarly offering potential extensions to higher-order-in-time particle pushers. The FET formulation is used also to consider how the stochastic drift terms can be incorporated into the pushers. Stochastic gyrokinetic expansions are also discussed.

        Different options for the numerical implementation of these schemes are considered.

        Due to the efficacy of FET in the development of SP timesteppers for both the fluid and kinetic component, we hope this approach will prove effective in the future for developing SP timesteppers for the full hybrid model. We hope this will give us the opportunity to incorporate previously inaccessible kinetic effects into the highly effective, modern, finite-element MHD models.
    \end{abstract}
    
    
    \newpage
    \tableofcontents
    
    
    \newpage
    \pagenumbering{arabic}
    %\linenumbers\renewcommand\thelinenumber{\color{black!50}\arabic{linenumber}}
            \input{0 - introduction/main.tex}
        \part{Research}
            \input{1 - low-noise PiC models/main.tex}
            \input{2 - kinetic component/main.tex}
            \input{3 - fluid component/main.tex}
            \input{4 - numerical implementation/main.tex}
        \part{Project Overview}
            \input{5 - research plan/main.tex}
            \input{6 - summary/main.tex}
    
    
    %\section{}
    \newpage
    \pagenumbering{gobble}
        \printbibliography


    \newpage
    \pagenumbering{roman}
    \appendix
        \part{Appendices}
            \input{8 - Hilbert complexes/main.tex}
            \input{9 - weak conservation proofs/main.tex}
\end{document}

    
    
    %\section{}
    \newpage
    \pagenumbering{gobble}
        \printbibliography


    \newpage
    \pagenumbering{roman}
    \appendix
        \part{Appendices}
            \documentclass[12pt, a4paper]{report}

\input{template/main.tex}

\title{\BA{Title in Progress...}}
\author{Boris Andrews}
\affil{Mathematical Institute, University of Oxford}
\date{\today}


\begin{document}
    \pagenumbering{gobble}
    \maketitle
    
    
    \begin{abstract}
        Magnetic confinement reactors---in particular tokamaks---offer one of the most promising options for achieving practical nuclear fusion, with the potential to provide virtually limitless, clean energy. The theoretical and numerical modeling of tokamak plasmas is simultaneously an essential component of effective reactor design, and a great research barrier. Tokamak operational conditions exhibit comparatively low Knudsen numbers. Kinetic effects, including kinetic waves and instabilities, Landau damping, bump-on-tail instabilities and more, are therefore highly influential in tokamak plasma dynamics. Purely fluid models are inherently incapable of capturing these effects, whereas the high dimensionality in purely kinetic models render them practically intractable for most relevant purposes.

        We consider a $\delta\!f$ decomposition model, with a macroscopic fluid background and microscopic kinetic correction, both fully coupled to each other. A similar manner of discretization is proposed to that used in the recent \texttt{STRUPHY} code \cite{Holderied_Possanner_Wang_2021, Holderied_2022, Li_et_al_2023} with a finite-element model for the background and a pseudo-particle/PiC model for the correction.

        The fluid background satisfies the full, non-linear, resistive, compressible, Hall MHD equations. \cite{Laakmann_Hu_Farrell_2022} introduces finite-element(-in-space) implicit timesteppers for the incompressible analogue to this system with structure-preserving (SP) properties in the ideal case, alongside parameter-robust preconditioners. We show that these timesteppers can derive from a finite-element-in-time (FET) (and finite-element-in-space) interpretation. The benefits of this reformulation are discussed, including the derivation of timesteppers that are higher order in time, and the quantifiable dissipative SP properties in the non-ideal, resistive case.
        
        We discuss possible options for extending this FET approach to timesteppers for the compressible case.

        The kinetic corrections satisfy linearized Boltzmann equations. Using a Lénard--Bernstein collision operator, these take Fokker--Planck-like forms \cite{Fokker_1914, Planck_1917} wherein pseudo-particles in the numerical model obey the neoclassical transport equations, with particle-independent Brownian drift terms. This offers a rigorous methodology for incorporating collisions into the particle transport model, without coupling the equations of motions for each particle.
        
        Works by Chen, Chacón et al. \cite{Chen_Chacón_Barnes_2011, Chacón_Chen_Barnes_2013, Chen_Chacón_2014, Chen_Chacón_2015} have developed structure-preserving particle pushers for neoclassical transport in the Vlasov equations, derived from Crank--Nicolson integrators. We show these too can can derive from a FET interpretation, similarly offering potential extensions to higher-order-in-time particle pushers. The FET formulation is used also to consider how the stochastic drift terms can be incorporated into the pushers. Stochastic gyrokinetic expansions are also discussed.

        Different options for the numerical implementation of these schemes are considered.

        Due to the efficacy of FET in the development of SP timesteppers for both the fluid and kinetic component, we hope this approach will prove effective in the future for developing SP timesteppers for the full hybrid model. We hope this will give us the opportunity to incorporate previously inaccessible kinetic effects into the highly effective, modern, finite-element MHD models.
    \end{abstract}
    
    
    \newpage
    \tableofcontents
    
    
    \newpage
    \pagenumbering{arabic}
    %\linenumbers\renewcommand\thelinenumber{\color{black!50}\arabic{linenumber}}
            \input{0 - introduction/main.tex}
        \part{Research}
            \input{1 - low-noise PiC models/main.tex}
            \input{2 - kinetic component/main.tex}
            \input{3 - fluid component/main.tex}
            \input{4 - numerical implementation/main.tex}
        \part{Project Overview}
            \input{5 - research plan/main.tex}
            \input{6 - summary/main.tex}
    
    
    %\section{}
    \newpage
    \pagenumbering{gobble}
        \printbibliography


    \newpage
    \pagenumbering{roman}
    \appendix
        \part{Appendices}
            \input{8 - Hilbert complexes/main.tex}
            \input{9 - weak conservation proofs/main.tex}
\end{document}

            \documentclass[12pt, a4paper]{report}

\input{template/main.tex}

\title{\BA{Title in Progress...}}
\author{Boris Andrews}
\affil{Mathematical Institute, University of Oxford}
\date{\today}


\begin{document}
    \pagenumbering{gobble}
    \maketitle
    
    
    \begin{abstract}
        Magnetic confinement reactors---in particular tokamaks---offer one of the most promising options for achieving practical nuclear fusion, with the potential to provide virtually limitless, clean energy. The theoretical and numerical modeling of tokamak plasmas is simultaneously an essential component of effective reactor design, and a great research barrier. Tokamak operational conditions exhibit comparatively low Knudsen numbers. Kinetic effects, including kinetic waves and instabilities, Landau damping, bump-on-tail instabilities and more, are therefore highly influential in tokamak plasma dynamics. Purely fluid models are inherently incapable of capturing these effects, whereas the high dimensionality in purely kinetic models render them practically intractable for most relevant purposes.

        We consider a $\delta\!f$ decomposition model, with a macroscopic fluid background and microscopic kinetic correction, both fully coupled to each other. A similar manner of discretization is proposed to that used in the recent \texttt{STRUPHY} code \cite{Holderied_Possanner_Wang_2021, Holderied_2022, Li_et_al_2023} with a finite-element model for the background and a pseudo-particle/PiC model for the correction.

        The fluid background satisfies the full, non-linear, resistive, compressible, Hall MHD equations. \cite{Laakmann_Hu_Farrell_2022} introduces finite-element(-in-space) implicit timesteppers for the incompressible analogue to this system with structure-preserving (SP) properties in the ideal case, alongside parameter-robust preconditioners. We show that these timesteppers can derive from a finite-element-in-time (FET) (and finite-element-in-space) interpretation. The benefits of this reformulation are discussed, including the derivation of timesteppers that are higher order in time, and the quantifiable dissipative SP properties in the non-ideal, resistive case.
        
        We discuss possible options for extending this FET approach to timesteppers for the compressible case.

        The kinetic corrections satisfy linearized Boltzmann equations. Using a Lénard--Bernstein collision operator, these take Fokker--Planck-like forms \cite{Fokker_1914, Planck_1917} wherein pseudo-particles in the numerical model obey the neoclassical transport equations, with particle-independent Brownian drift terms. This offers a rigorous methodology for incorporating collisions into the particle transport model, without coupling the equations of motions for each particle.
        
        Works by Chen, Chacón et al. \cite{Chen_Chacón_Barnes_2011, Chacón_Chen_Barnes_2013, Chen_Chacón_2014, Chen_Chacón_2015} have developed structure-preserving particle pushers for neoclassical transport in the Vlasov equations, derived from Crank--Nicolson integrators. We show these too can can derive from a FET interpretation, similarly offering potential extensions to higher-order-in-time particle pushers. The FET formulation is used also to consider how the stochastic drift terms can be incorporated into the pushers. Stochastic gyrokinetic expansions are also discussed.

        Different options for the numerical implementation of these schemes are considered.

        Due to the efficacy of FET in the development of SP timesteppers for both the fluid and kinetic component, we hope this approach will prove effective in the future for developing SP timesteppers for the full hybrid model. We hope this will give us the opportunity to incorporate previously inaccessible kinetic effects into the highly effective, modern, finite-element MHD models.
    \end{abstract}
    
    
    \newpage
    \tableofcontents
    
    
    \newpage
    \pagenumbering{arabic}
    %\linenumbers\renewcommand\thelinenumber{\color{black!50}\arabic{linenumber}}
            \input{0 - introduction/main.tex}
        \part{Research}
            \input{1 - low-noise PiC models/main.tex}
            \input{2 - kinetic component/main.tex}
            \input{3 - fluid component/main.tex}
            \input{4 - numerical implementation/main.tex}
        \part{Project Overview}
            \input{5 - research plan/main.tex}
            \input{6 - summary/main.tex}
    
    
    %\section{}
    \newpage
    \pagenumbering{gobble}
        \printbibliography


    \newpage
    \pagenumbering{roman}
    \appendix
        \part{Appendices}
            \input{8 - Hilbert complexes/main.tex}
            \input{9 - weak conservation proofs/main.tex}
\end{document}

\end{document}

            \documentclass[12pt, a4paper]{report}

\documentclass[12pt, a4paper]{report}

\input{template/main.tex}

\title{\BA{Title in Progress...}}
\author{Boris Andrews}
\affil{Mathematical Institute, University of Oxford}
\date{\today}


\begin{document}
    \pagenumbering{gobble}
    \maketitle
    
    
    \begin{abstract}
        Magnetic confinement reactors---in particular tokamaks---offer one of the most promising options for achieving practical nuclear fusion, with the potential to provide virtually limitless, clean energy. The theoretical and numerical modeling of tokamak plasmas is simultaneously an essential component of effective reactor design, and a great research barrier. Tokamak operational conditions exhibit comparatively low Knudsen numbers. Kinetic effects, including kinetic waves and instabilities, Landau damping, bump-on-tail instabilities and more, are therefore highly influential in tokamak plasma dynamics. Purely fluid models are inherently incapable of capturing these effects, whereas the high dimensionality in purely kinetic models render them practically intractable for most relevant purposes.

        We consider a $\delta\!f$ decomposition model, with a macroscopic fluid background and microscopic kinetic correction, both fully coupled to each other. A similar manner of discretization is proposed to that used in the recent \texttt{STRUPHY} code \cite{Holderied_Possanner_Wang_2021, Holderied_2022, Li_et_al_2023} with a finite-element model for the background and a pseudo-particle/PiC model for the correction.

        The fluid background satisfies the full, non-linear, resistive, compressible, Hall MHD equations. \cite{Laakmann_Hu_Farrell_2022} introduces finite-element(-in-space) implicit timesteppers for the incompressible analogue to this system with structure-preserving (SP) properties in the ideal case, alongside parameter-robust preconditioners. We show that these timesteppers can derive from a finite-element-in-time (FET) (and finite-element-in-space) interpretation. The benefits of this reformulation are discussed, including the derivation of timesteppers that are higher order in time, and the quantifiable dissipative SP properties in the non-ideal, resistive case.
        
        We discuss possible options for extending this FET approach to timesteppers for the compressible case.

        The kinetic corrections satisfy linearized Boltzmann equations. Using a Lénard--Bernstein collision operator, these take Fokker--Planck-like forms \cite{Fokker_1914, Planck_1917} wherein pseudo-particles in the numerical model obey the neoclassical transport equations, with particle-independent Brownian drift terms. This offers a rigorous methodology for incorporating collisions into the particle transport model, without coupling the equations of motions for each particle.
        
        Works by Chen, Chacón et al. \cite{Chen_Chacón_Barnes_2011, Chacón_Chen_Barnes_2013, Chen_Chacón_2014, Chen_Chacón_2015} have developed structure-preserving particle pushers for neoclassical transport in the Vlasov equations, derived from Crank--Nicolson integrators. We show these too can can derive from a FET interpretation, similarly offering potential extensions to higher-order-in-time particle pushers. The FET formulation is used also to consider how the stochastic drift terms can be incorporated into the pushers. Stochastic gyrokinetic expansions are also discussed.

        Different options for the numerical implementation of these schemes are considered.

        Due to the efficacy of FET in the development of SP timesteppers for both the fluid and kinetic component, we hope this approach will prove effective in the future for developing SP timesteppers for the full hybrid model. We hope this will give us the opportunity to incorporate previously inaccessible kinetic effects into the highly effective, modern, finite-element MHD models.
    \end{abstract}
    
    
    \newpage
    \tableofcontents
    
    
    \newpage
    \pagenumbering{arabic}
    %\linenumbers\renewcommand\thelinenumber{\color{black!50}\arabic{linenumber}}
            \input{0 - introduction/main.tex}
        \part{Research}
            \input{1 - low-noise PiC models/main.tex}
            \input{2 - kinetic component/main.tex}
            \input{3 - fluid component/main.tex}
            \input{4 - numerical implementation/main.tex}
        \part{Project Overview}
            \input{5 - research plan/main.tex}
            \input{6 - summary/main.tex}
    
    
    %\section{}
    \newpage
    \pagenumbering{gobble}
        \printbibliography


    \newpage
    \pagenumbering{roman}
    \appendix
        \part{Appendices}
            \input{8 - Hilbert complexes/main.tex}
            \input{9 - weak conservation proofs/main.tex}
\end{document}


\title{\BA{Title in Progress...}}
\author{Boris Andrews}
\affil{Mathematical Institute, University of Oxford}
\date{\today}


\begin{document}
    \pagenumbering{gobble}
    \maketitle
    
    
    \begin{abstract}
        Magnetic confinement reactors---in particular tokamaks---offer one of the most promising options for achieving practical nuclear fusion, with the potential to provide virtually limitless, clean energy. The theoretical and numerical modeling of tokamak plasmas is simultaneously an essential component of effective reactor design, and a great research barrier. Tokamak operational conditions exhibit comparatively low Knudsen numbers. Kinetic effects, including kinetic waves and instabilities, Landau damping, bump-on-tail instabilities and more, are therefore highly influential in tokamak plasma dynamics. Purely fluid models are inherently incapable of capturing these effects, whereas the high dimensionality in purely kinetic models render them practically intractable for most relevant purposes.

        We consider a $\delta\!f$ decomposition model, with a macroscopic fluid background and microscopic kinetic correction, both fully coupled to each other. A similar manner of discretization is proposed to that used in the recent \texttt{STRUPHY} code \cite{Holderied_Possanner_Wang_2021, Holderied_2022, Li_et_al_2023} with a finite-element model for the background and a pseudo-particle/PiC model for the correction.

        The fluid background satisfies the full, non-linear, resistive, compressible, Hall MHD equations. \cite{Laakmann_Hu_Farrell_2022} introduces finite-element(-in-space) implicit timesteppers for the incompressible analogue to this system with structure-preserving (SP) properties in the ideal case, alongside parameter-robust preconditioners. We show that these timesteppers can derive from a finite-element-in-time (FET) (and finite-element-in-space) interpretation. The benefits of this reformulation are discussed, including the derivation of timesteppers that are higher order in time, and the quantifiable dissipative SP properties in the non-ideal, resistive case.
        
        We discuss possible options for extending this FET approach to timesteppers for the compressible case.

        The kinetic corrections satisfy linearized Boltzmann equations. Using a Lénard--Bernstein collision operator, these take Fokker--Planck-like forms \cite{Fokker_1914, Planck_1917} wherein pseudo-particles in the numerical model obey the neoclassical transport equations, with particle-independent Brownian drift terms. This offers a rigorous methodology for incorporating collisions into the particle transport model, without coupling the equations of motions for each particle.
        
        Works by Chen, Chacón et al. \cite{Chen_Chacón_Barnes_2011, Chacón_Chen_Barnes_2013, Chen_Chacón_2014, Chen_Chacón_2015} have developed structure-preserving particle pushers for neoclassical transport in the Vlasov equations, derived from Crank--Nicolson integrators. We show these too can can derive from a FET interpretation, similarly offering potential extensions to higher-order-in-time particle pushers. The FET formulation is used also to consider how the stochastic drift terms can be incorporated into the pushers. Stochastic gyrokinetic expansions are also discussed.

        Different options for the numerical implementation of these schemes are considered.

        Due to the efficacy of FET in the development of SP timesteppers for both the fluid and kinetic component, we hope this approach will prove effective in the future for developing SP timesteppers for the full hybrid model. We hope this will give us the opportunity to incorporate previously inaccessible kinetic effects into the highly effective, modern, finite-element MHD models.
    \end{abstract}
    
    
    \newpage
    \tableofcontents
    
    
    \newpage
    \pagenumbering{arabic}
    %\linenumbers\renewcommand\thelinenumber{\color{black!50}\arabic{linenumber}}
            \documentclass[12pt, a4paper]{report}

\input{template/main.tex}

\title{\BA{Title in Progress...}}
\author{Boris Andrews}
\affil{Mathematical Institute, University of Oxford}
\date{\today}


\begin{document}
    \pagenumbering{gobble}
    \maketitle
    
    
    \begin{abstract}
        Magnetic confinement reactors---in particular tokamaks---offer one of the most promising options for achieving practical nuclear fusion, with the potential to provide virtually limitless, clean energy. The theoretical and numerical modeling of tokamak plasmas is simultaneously an essential component of effective reactor design, and a great research barrier. Tokamak operational conditions exhibit comparatively low Knudsen numbers. Kinetic effects, including kinetic waves and instabilities, Landau damping, bump-on-tail instabilities and more, are therefore highly influential in tokamak plasma dynamics. Purely fluid models are inherently incapable of capturing these effects, whereas the high dimensionality in purely kinetic models render them practically intractable for most relevant purposes.

        We consider a $\delta\!f$ decomposition model, with a macroscopic fluid background and microscopic kinetic correction, both fully coupled to each other. A similar manner of discretization is proposed to that used in the recent \texttt{STRUPHY} code \cite{Holderied_Possanner_Wang_2021, Holderied_2022, Li_et_al_2023} with a finite-element model for the background and a pseudo-particle/PiC model for the correction.

        The fluid background satisfies the full, non-linear, resistive, compressible, Hall MHD equations. \cite{Laakmann_Hu_Farrell_2022} introduces finite-element(-in-space) implicit timesteppers for the incompressible analogue to this system with structure-preserving (SP) properties in the ideal case, alongside parameter-robust preconditioners. We show that these timesteppers can derive from a finite-element-in-time (FET) (and finite-element-in-space) interpretation. The benefits of this reformulation are discussed, including the derivation of timesteppers that are higher order in time, and the quantifiable dissipative SP properties in the non-ideal, resistive case.
        
        We discuss possible options for extending this FET approach to timesteppers for the compressible case.

        The kinetic corrections satisfy linearized Boltzmann equations. Using a Lénard--Bernstein collision operator, these take Fokker--Planck-like forms \cite{Fokker_1914, Planck_1917} wherein pseudo-particles in the numerical model obey the neoclassical transport equations, with particle-independent Brownian drift terms. This offers a rigorous methodology for incorporating collisions into the particle transport model, without coupling the equations of motions for each particle.
        
        Works by Chen, Chacón et al. \cite{Chen_Chacón_Barnes_2011, Chacón_Chen_Barnes_2013, Chen_Chacón_2014, Chen_Chacón_2015} have developed structure-preserving particle pushers for neoclassical transport in the Vlasov equations, derived from Crank--Nicolson integrators. We show these too can can derive from a FET interpretation, similarly offering potential extensions to higher-order-in-time particle pushers. The FET formulation is used also to consider how the stochastic drift terms can be incorporated into the pushers. Stochastic gyrokinetic expansions are also discussed.

        Different options for the numerical implementation of these schemes are considered.

        Due to the efficacy of FET in the development of SP timesteppers for both the fluid and kinetic component, we hope this approach will prove effective in the future for developing SP timesteppers for the full hybrid model. We hope this will give us the opportunity to incorporate previously inaccessible kinetic effects into the highly effective, modern, finite-element MHD models.
    \end{abstract}
    
    
    \newpage
    \tableofcontents
    
    
    \newpage
    \pagenumbering{arabic}
    %\linenumbers\renewcommand\thelinenumber{\color{black!50}\arabic{linenumber}}
            \input{0 - introduction/main.tex}
        \part{Research}
            \input{1 - low-noise PiC models/main.tex}
            \input{2 - kinetic component/main.tex}
            \input{3 - fluid component/main.tex}
            \input{4 - numerical implementation/main.tex}
        \part{Project Overview}
            \input{5 - research plan/main.tex}
            \input{6 - summary/main.tex}
    
    
    %\section{}
    \newpage
    \pagenumbering{gobble}
        \printbibliography


    \newpage
    \pagenumbering{roman}
    \appendix
        \part{Appendices}
            \input{8 - Hilbert complexes/main.tex}
            \input{9 - weak conservation proofs/main.tex}
\end{document}

        \part{Research}
            \documentclass[12pt, a4paper]{report}

\input{template/main.tex}

\title{\BA{Title in Progress...}}
\author{Boris Andrews}
\affil{Mathematical Institute, University of Oxford}
\date{\today}


\begin{document}
    \pagenumbering{gobble}
    \maketitle
    
    
    \begin{abstract}
        Magnetic confinement reactors---in particular tokamaks---offer one of the most promising options for achieving practical nuclear fusion, with the potential to provide virtually limitless, clean energy. The theoretical and numerical modeling of tokamak plasmas is simultaneously an essential component of effective reactor design, and a great research barrier. Tokamak operational conditions exhibit comparatively low Knudsen numbers. Kinetic effects, including kinetic waves and instabilities, Landau damping, bump-on-tail instabilities and more, are therefore highly influential in tokamak plasma dynamics. Purely fluid models are inherently incapable of capturing these effects, whereas the high dimensionality in purely kinetic models render them practically intractable for most relevant purposes.

        We consider a $\delta\!f$ decomposition model, with a macroscopic fluid background and microscopic kinetic correction, both fully coupled to each other. A similar manner of discretization is proposed to that used in the recent \texttt{STRUPHY} code \cite{Holderied_Possanner_Wang_2021, Holderied_2022, Li_et_al_2023} with a finite-element model for the background and a pseudo-particle/PiC model for the correction.

        The fluid background satisfies the full, non-linear, resistive, compressible, Hall MHD equations. \cite{Laakmann_Hu_Farrell_2022} introduces finite-element(-in-space) implicit timesteppers for the incompressible analogue to this system with structure-preserving (SP) properties in the ideal case, alongside parameter-robust preconditioners. We show that these timesteppers can derive from a finite-element-in-time (FET) (and finite-element-in-space) interpretation. The benefits of this reformulation are discussed, including the derivation of timesteppers that are higher order in time, and the quantifiable dissipative SP properties in the non-ideal, resistive case.
        
        We discuss possible options for extending this FET approach to timesteppers for the compressible case.

        The kinetic corrections satisfy linearized Boltzmann equations. Using a Lénard--Bernstein collision operator, these take Fokker--Planck-like forms \cite{Fokker_1914, Planck_1917} wherein pseudo-particles in the numerical model obey the neoclassical transport equations, with particle-independent Brownian drift terms. This offers a rigorous methodology for incorporating collisions into the particle transport model, without coupling the equations of motions for each particle.
        
        Works by Chen, Chacón et al. \cite{Chen_Chacón_Barnes_2011, Chacón_Chen_Barnes_2013, Chen_Chacón_2014, Chen_Chacón_2015} have developed structure-preserving particle pushers for neoclassical transport in the Vlasov equations, derived from Crank--Nicolson integrators. We show these too can can derive from a FET interpretation, similarly offering potential extensions to higher-order-in-time particle pushers. The FET formulation is used also to consider how the stochastic drift terms can be incorporated into the pushers. Stochastic gyrokinetic expansions are also discussed.

        Different options for the numerical implementation of these schemes are considered.

        Due to the efficacy of FET in the development of SP timesteppers for both the fluid and kinetic component, we hope this approach will prove effective in the future for developing SP timesteppers for the full hybrid model. We hope this will give us the opportunity to incorporate previously inaccessible kinetic effects into the highly effective, modern, finite-element MHD models.
    \end{abstract}
    
    
    \newpage
    \tableofcontents
    
    
    \newpage
    \pagenumbering{arabic}
    %\linenumbers\renewcommand\thelinenumber{\color{black!50}\arabic{linenumber}}
            \input{0 - introduction/main.tex}
        \part{Research}
            \input{1 - low-noise PiC models/main.tex}
            \input{2 - kinetic component/main.tex}
            \input{3 - fluid component/main.tex}
            \input{4 - numerical implementation/main.tex}
        \part{Project Overview}
            \input{5 - research plan/main.tex}
            \input{6 - summary/main.tex}
    
    
    %\section{}
    \newpage
    \pagenumbering{gobble}
        \printbibliography


    \newpage
    \pagenumbering{roman}
    \appendix
        \part{Appendices}
            \input{8 - Hilbert complexes/main.tex}
            \input{9 - weak conservation proofs/main.tex}
\end{document}

            \documentclass[12pt, a4paper]{report}

\input{template/main.tex}

\title{\BA{Title in Progress...}}
\author{Boris Andrews}
\affil{Mathematical Institute, University of Oxford}
\date{\today}


\begin{document}
    \pagenumbering{gobble}
    \maketitle
    
    
    \begin{abstract}
        Magnetic confinement reactors---in particular tokamaks---offer one of the most promising options for achieving practical nuclear fusion, with the potential to provide virtually limitless, clean energy. The theoretical and numerical modeling of tokamak plasmas is simultaneously an essential component of effective reactor design, and a great research barrier. Tokamak operational conditions exhibit comparatively low Knudsen numbers. Kinetic effects, including kinetic waves and instabilities, Landau damping, bump-on-tail instabilities and more, are therefore highly influential in tokamak plasma dynamics. Purely fluid models are inherently incapable of capturing these effects, whereas the high dimensionality in purely kinetic models render them practically intractable for most relevant purposes.

        We consider a $\delta\!f$ decomposition model, with a macroscopic fluid background and microscopic kinetic correction, both fully coupled to each other. A similar manner of discretization is proposed to that used in the recent \texttt{STRUPHY} code \cite{Holderied_Possanner_Wang_2021, Holderied_2022, Li_et_al_2023} with a finite-element model for the background and a pseudo-particle/PiC model for the correction.

        The fluid background satisfies the full, non-linear, resistive, compressible, Hall MHD equations. \cite{Laakmann_Hu_Farrell_2022} introduces finite-element(-in-space) implicit timesteppers for the incompressible analogue to this system with structure-preserving (SP) properties in the ideal case, alongside parameter-robust preconditioners. We show that these timesteppers can derive from a finite-element-in-time (FET) (and finite-element-in-space) interpretation. The benefits of this reformulation are discussed, including the derivation of timesteppers that are higher order in time, and the quantifiable dissipative SP properties in the non-ideal, resistive case.
        
        We discuss possible options for extending this FET approach to timesteppers for the compressible case.

        The kinetic corrections satisfy linearized Boltzmann equations. Using a Lénard--Bernstein collision operator, these take Fokker--Planck-like forms \cite{Fokker_1914, Planck_1917} wherein pseudo-particles in the numerical model obey the neoclassical transport equations, with particle-independent Brownian drift terms. This offers a rigorous methodology for incorporating collisions into the particle transport model, without coupling the equations of motions for each particle.
        
        Works by Chen, Chacón et al. \cite{Chen_Chacón_Barnes_2011, Chacón_Chen_Barnes_2013, Chen_Chacón_2014, Chen_Chacón_2015} have developed structure-preserving particle pushers for neoclassical transport in the Vlasov equations, derived from Crank--Nicolson integrators. We show these too can can derive from a FET interpretation, similarly offering potential extensions to higher-order-in-time particle pushers. The FET formulation is used also to consider how the stochastic drift terms can be incorporated into the pushers. Stochastic gyrokinetic expansions are also discussed.

        Different options for the numerical implementation of these schemes are considered.

        Due to the efficacy of FET in the development of SP timesteppers for both the fluid and kinetic component, we hope this approach will prove effective in the future for developing SP timesteppers for the full hybrid model. We hope this will give us the opportunity to incorporate previously inaccessible kinetic effects into the highly effective, modern, finite-element MHD models.
    \end{abstract}
    
    
    \newpage
    \tableofcontents
    
    
    \newpage
    \pagenumbering{arabic}
    %\linenumbers\renewcommand\thelinenumber{\color{black!50}\arabic{linenumber}}
            \input{0 - introduction/main.tex}
        \part{Research}
            \input{1 - low-noise PiC models/main.tex}
            \input{2 - kinetic component/main.tex}
            \input{3 - fluid component/main.tex}
            \input{4 - numerical implementation/main.tex}
        \part{Project Overview}
            \input{5 - research plan/main.tex}
            \input{6 - summary/main.tex}
    
    
    %\section{}
    \newpage
    \pagenumbering{gobble}
        \printbibliography


    \newpage
    \pagenumbering{roman}
    \appendix
        \part{Appendices}
            \input{8 - Hilbert complexes/main.tex}
            \input{9 - weak conservation proofs/main.tex}
\end{document}

            \documentclass[12pt, a4paper]{report}

\input{template/main.tex}

\title{\BA{Title in Progress...}}
\author{Boris Andrews}
\affil{Mathematical Institute, University of Oxford}
\date{\today}


\begin{document}
    \pagenumbering{gobble}
    \maketitle
    
    
    \begin{abstract}
        Magnetic confinement reactors---in particular tokamaks---offer one of the most promising options for achieving practical nuclear fusion, with the potential to provide virtually limitless, clean energy. The theoretical and numerical modeling of tokamak plasmas is simultaneously an essential component of effective reactor design, and a great research barrier. Tokamak operational conditions exhibit comparatively low Knudsen numbers. Kinetic effects, including kinetic waves and instabilities, Landau damping, bump-on-tail instabilities and more, are therefore highly influential in tokamak plasma dynamics. Purely fluid models are inherently incapable of capturing these effects, whereas the high dimensionality in purely kinetic models render them practically intractable for most relevant purposes.

        We consider a $\delta\!f$ decomposition model, with a macroscopic fluid background and microscopic kinetic correction, both fully coupled to each other. A similar manner of discretization is proposed to that used in the recent \texttt{STRUPHY} code \cite{Holderied_Possanner_Wang_2021, Holderied_2022, Li_et_al_2023} with a finite-element model for the background and a pseudo-particle/PiC model for the correction.

        The fluid background satisfies the full, non-linear, resistive, compressible, Hall MHD equations. \cite{Laakmann_Hu_Farrell_2022} introduces finite-element(-in-space) implicit timesteppers for the incompressible analogue to this system with structure-preserving (SP) properties in the ideal case, alongside parameter-robust preconditioners. We show that these timesteppers can derive from a finite-element-in-time (FET) (and finite-element-in-space) interpretation. The benefits of this reformulation are discussed, including the derivation of timesteppers that are higher order in time, and the quantifiable dissipative SP properties in the non-ideal, resistive case.
        
        We discuss possible options for extending this FET approach to timesteppers for the compressible case.

        The kinetic corrections satisfy linearized Boltzmann equations. Using a Lénard--Bernstein collision operator, these take Fokker--Planck-like forms \cite{Fokker_1914, Planck_1917} wherein pseudo-particles in the numerical model obey the neoclassical transport equations, with particle-independent Brownian drift terms. This offers a rigorous methodology for incorporating collisions into the particle transport model, without coupling the equations of motions for each particle.
        
        Works by Chen, Chacón et al. \cite{Chen_Chacón_Barnes_2011, Chacón_Chen_Barnes_2013, Chen_Chacón_2014, Chen_Chacón_2015} have developed structure-preserving particle pushers for neoclassical transport in the Vlasov equations, derived from Crank--Nicolson integrators. We show these too can can derive from a FET interpretation, similarly offering potential extensions to higher-order-in-time particle pushers. The FET formulation is used also to consider how the stochastic drift terms can be incorporated into the pushers. Stochastic gyrokinetic expansions are also discussed.

        Different options for the numerical implementation of these schemes are considered.

        Due to the efficacy of FET in the development of SP timesteppers for both the fluid and kinetic component, we hope this approach will prove effective in the future for developing SP timesteppers for the full hybrid model. We hope this will give us the opportunity to incorporate previously inaccessible kinetic effects into the highly effective, modern, finite-element MHD models.
    \end{abstract}
    
    
    \newpage
    \tableofcontents
    
    
    \newpage
    \pagenumbering{arabic}
    %\linenumbers\renewcommand\thelinenumber{\color{black!50}\arabic{linenumber}}
            \input{0 - introduction/main.tex}
        \part{Research}
            \input{1 - low-noise PiC models/main.tex}
            \input{2 - kinetic component/main.tex}
            \input{3 - fluid component/main.tex}
            \input{4 - numerical implementation/main.tex}
        \part{Project Overview}
            \input{5 - research plan/main.tex}
            \input{6 - summary/main.tex}
    
    
    %\section{}
    \newpage
    \pagenumbering{gobble}
        \printbibliography


    \newpage
    \pagenumbering{roman}
    \appendix
        \part{Appendices}
            \input{8 - Hilbert complexes/main.tex}
            \input{9 - weak conservation proofs/main.tex}
\end{document}

            \documentclass[12pt, a4paper]{report}

\input{template/main.tex}

\title{\BA{Title in Progress...}}
\author{Boris Andrews}
\affil{Mathematical Institute, University of Oxford}
\date{\today}


\begin{document}
    \pagenumbering{gobble}
    \maketitle
    
    
    \begin{abstract}
        Magnetic confinement reactors---in particular tokamaks---offer one of the most promising options for achieving practical nuclear fusion, with the potential to provide virtually limitless, clean energy. The theoretical and numerical modeling of tokamak plasmas is simultaneously an essential component of effective reactor design, and a great research barrier. Tokamak operational conditions exhibit comparatively low Knudsen numbers. Kinetic effects, including kinetic waves and instabilities, Landau damping, bump-on-tail instabilities and more, are therefore highly influential in tokamak plasma dynamics. Purely fluid models are inherently incapable of capturing these effects, whereas the high dimensionality in purely kinetic models render them practically intractable for most relevant purposes.

        We consider a $\delta\!f$ decomposition model, with a macroscopic fluid background and microscopic kinetic correction, both fully coupled to each other. A similar manner of discretization is proposed to that used in the recent \texttt{STRUPHY} code \cite{Holderied_Possanner_Wang_2021, Holderied_2022, Li_et_al_2023} with a finite-element model for the background and a pseudo-particle/PiC model for the correction.

        The fluid background satisfies the full, non-linear, resistive, compressible, Hall MHD equations. \cite{Laakmann_Hu_Farrell_2022} introduces finite-element(-in-space) implicit timesteppers for the incompressible analogue to this system with structure-preserving (SP) properties in the ideal case, alongside parameter-robust preconditioners. We show that these timesteppers can derive from a finite-element-in-time (FET) (and finite-element-in-space) interpretation. The benefits of this reformulation are discussed, including the derivation of timesteppers that are higher order in time, and the quantifiable dissipative SP properties in the non-ideal, resistive case.
        
        We discuss possible options for extending this FET approach to timesteppers for the compressible case.

        The kinetic corrections satisfy linearized Boltzmann equations. Using a Lénard--Bernstein collision operator, these take Fokker--Planck-like forms \cite{Fokker_1914, Planck_1917} wherein pseudo-particles in the numerical model obey the neoclassical transport equations, with particle-independent Brownian drift terms. This offers a rigorous methodology for incorporating collisions into the particle transport model, without coupling the equations of motions for each particle.
        
        Works by Chen, Chacón et al. \cite{Chen_Chacón_Barnes_2011, Chacón_Chen_Barnes_2013, Chen_Chacón_2014, Chen_Chacón_2015} have developed structure-preserving particle pushers for neoclassical transport in the Vlasov equations, derived from Crank--Nicolson integrators. We show these too can can derive from a FET interpretation, similarly offering potential extensions to higher-order-in-time particle pushers. The FET formulation is used also to consider how the stochastic drift terms can be incorporated into the pushers. Stochastic gyrokinetic expansions are also discussed.

        Different options for the numerical implementation of these schemes are considered.

        Due to the efficacy of FET in the development of SP timesteppers for both the fluid and kinetic component, we hope this approach will prove effective in the future for developing SP timesteppers for the full hybrid model. We hope this will give us the opportunity to incorporate previously inaccessible kinetic effects into the highly effective, modern, finite-element MHD models.
    \end{abstract}
    
    
    \newpage
    \tableofcontents
    
    
    \newpage
    \pagenumbering{arabic}
    %\linenumbers\renewcommand\thelinenumber{\color{black!50}\arabic{linenumber}}
            \input{0 - introduction/main.tex}
        \part{Research}
            \input{1 - low-noise PiC models/main.tex}
            \input{2 - kinetic component/main.tex}
            \input{3 - fluid component/main.tex}
            \input{4 - numerical implementation/main.tex}
        \part{Project Overview}
            \input{5 - research plan/main.tex}
            \input{6 - summary/main.tex}
    
    
    %\section{}
    \newpage
    \pagenumbering{gobble}
        \printbibliography


    \newpage
    \pagenumbering{roman}
    \appendix
        \part{Appendices}
            \input{8 - Hilbert complexes/main.tex}
            \input{9 - weak conservation proofs/main.tex}
\end{document}

        \part{Project Overview}
            \documentclass[12pt, a4paper]{report}

\input{template/main.tex}

\title{\BA{Title in Progress...}}
\author{Boris Andrews}
\affil{Mathematical Institute, University of Oxford}
\date{\today}


\begin{document}
    \pagenumbering{gobble}
    \maketitle
    
    
    \begin{abstract}
        Magnetic confinement reactors---in particular tokamaks---offer one of the most promising options for achieving practical nuclear fusion, with the potential to provide virtually limitless, clean energy. The theoretical and numerical modeling of tokamak plasmas is simultaneously an essential component of effective reactor design, and a great research barrier. Tokamak operational conditions exhibit comparatively low Knudsen numbers. Kinetic effects, including kinetic waves and instabilities, Landau damping, bump-on-tail instabilities and more, are therefore highly influential in tokamak plasma dynamics. Purely fluid models are inherently incapable of capturing these effects, whereas the high dimensionality in purely kinetic models render them practically intractable for most relevant purposes.

        We consider a $\delta\!f$ decomposition model, with a macroscopic fluid background and microscopic kinetic correction, both fully coupled to each other. A similar manner of discretization is proposed to that used in the recent \texttt{STRUPHY} code \cite{Holderied_Possanner_Wang_2021, Holderied_2022, Li_et_al_2023} with a finite-element model for the background and a pseudo-particle/PiC model for the correction.

        The fluid background satisfies the full, non-linear, resistive, compressible, Hall MHD equations. \cite{Laakmann_Hu_Farrell_2022} introduces finite-element(-in-space) implicit timesteppers for the incompressible analogue to this system with structure-preserving (SP) properties in the ideal case, alongside parameter-robust preconditioners. We show that these timesteppers can derive from a finite-element-in-time (FET) (and finite-element-in-space) interpretation. The benefits of this reformulation are discussed, including the derivation of timesteppers that are higher order in time, and the quantifiable dissipative SP properties in the non-ideal, resistive case.
        
        We discuss possible options for extending this FET approach to timesteppers for the compressible case.

        The kinetic corrections satisfy linearized Boltzmann equations. Using a Lénard--Bernstein collision operator, these take Fokker--Planck-like forms \cite{Fokker_1914, Planck_1917} wherein pseudo-particles in the numerical model obey the neoclassical transport equations, with particle-independent Brownian drift terms. This offers a rigorous methodology for incorporating collisions into the particle transport model, without coupling the equations of motions for each particle.
        
        Works by Chen, Chacón et al. \cite{Chen_Chacón_Barnes_2011, Chacón_Chen_Barnes_2013, Chen_Chacón_2014, Chen_Chacón_2015} have developed structure-preserving particle pushers for neoclassical transport in the Vlasov equations, derived from Crank--Nicolson integrators. We show these too can can derive from a FET interpretation, similarly offering potential extensions to higher-order-in-time particle pushers. The FET formulation is used also to consider how the stochastic drift terms can be incorporated into the pushers. Stochastic gyrokinetic expansions are also discussed.

        Different options for the numerical implementation of these schemes are considered.

        Due to the efficacy of FET in the development of SP timesteppers for both the fluid and kinetic component, we hope this approach will prove effective in the future for developing SP timesteppers for the full hybrid model. We hope this will give us the opportunity to incorporate previously inaccessible kinetic effects into the highly effective, modern, finite-element MHD models.
    \end{abstract}
    
    
    \newpage
    \tableofcontents
    
    
    \newpage
    \pagenumbering{arabic}
    %\linenumbers\renewcommand\thelinenumber{\color{black!50}\arabic{linenumber}}
            \input{0 - introduction/main.tex}
        \part{Research}
            \input{1 - low-noise PiC models/main.tex}
            \input{2 - kinetic component/main.tex}
            \input{3 - fluid component/main.tex}
            \input{4 - numerical implementation/main.tex}
        \part{Project Overview}
            \input{5 - research plan/main.tex}
            \input{6 - summary/main.tex}
    
    
    %\section{}
    \newpage
    \pagenumbering{gobble}
        \printbibliography


    \newpage
    \pagenumbering{roman}
    \appendix
        \part{Appendices}
            \input{8 - Hilbert complexes/main.tex}
            \input{9 - weak conservation proofs/main.tex}
\end{document}

            \documentclass[12pt, a4paper]{report}

\input{template/main.tex}

\title{\BA{Title in Progress...}}
\author{Boris Andrews}
\affil{Mathematical Institute, University of Oxford}
\date{\today}


\begin{document}
    \pagenumbering{gobble}
    \maketitle
    
    
    \begin{abstract}
        Magnetic confinement reactors---in particular tokamaks---offer one of the most promising options for achieving practical nuclear fusion, with the potential to provide virtually limitless, clean energy. The theoretical and numerical modeling of tokamak plasmas is simultaneously an essential component of effective reactor design, and a great research barrier. Tokamak operational conditions exhibit comparatively low Knudsen numbers. Kinetic effects, including kinetic waves and instabilities, Landau damping, bump-on-tail instabilities and more, are therefore highly influential in tokamak plasma dynamics. Purely fluid models are inherently incapable of capturing these effects, whereas the high dimensionality in purely kinetic models render them practically intractable for most relevant purposes.

        We consider a $\delta\!f$ decomposition model, with a macroscopic fluid background and microscopic kinetic correction, both fully coupled to each other. A similar manner of discretization is proposed to that used in the recent \texttt{STRUPHY} code \cite{Holderied_Possanner_Wang_2021, Holderied_2022, Li_et_al_2023} with a finite-element model for the background and a pseudo-particle/PiC model for the correction.

        The fluid background satisfies the full, non-linear, resistive, compressible, Hall MHD equations. \cite{Laakmann_Hu_Farrell_2022} introduces finite-element(-in-space) implicit timesteppers for the incompressible analogue to this system with structure-preserving (SP) properties in the ideal case, alongside parameter-robust preconditioners. We show that these timesteppers can derive from a finite-element-in-time (FET) (and finite-element-in-space) interpretation. The benefits of this reformulation are discussed, including the derivation of timesteppers that are higher order in time, and the quantifiable dissipative SP properties in the non-ideal, resistive case.
        
        We discuss possible options for extending this FET approach to timesteppers for the compressible case.

        The kinetic corrections satisfy linearized Boltzmann equations. Using a Lénard--Bernstein collision operator, these take Fokker--Planck-like forms \cite{Fokker_1914, Planck_1917} wherein pseudo-particles in the numerical model obey the neoclassical transport equations, with particle-independent Brownian drift terms. This offers a rigorous methodology for incorporating collisions into the particle transport model, without coupling the equations of motions for each particle.
        
        Works by Chen, Chacón et al. \cite{Chen_Chacón_Barnes_2011, Chacón_Chen_Barnes_2013, Chen_Chacón_2014, Chen_Chacón_2015} have developed structure-preserving particle pushers for neoclassical transport in the Vlasov equations, derived from Crank--Nicolson integrators. We show these too can can derive from a FET interpretation, similarly offering potential extensions to higher-order-in-time particle pushers. The FET formulation is used also to consider how the stochastic drift terms can be incorporated into the pushers. Stochastic gyrokinetic expansions are also discussed.

        Different options for the numerical implementation of these schemes are considered.

        Due to the efficacy of FET in the development of SP timesteppers for both the fluid and kinetic component, we hope this approach will prove effective in the future for developing SP timesteppers for the full hybrid model. We hope this will give us the opportunity to incorporate previously inaccessible kinetic effects into the highly effective, modern, finite-element MHD models.
    \end{abstract}
    
    
    \newpage
    \tableofcontents
    
    
    \newpage
    \pagenumbering{arabic}
    %\linenumbers\renewcommand\thelinenumber{\color{black!50}\arabic{linenumber}}
            \input{0 - introduction/main.tex}
        \part{Research}
            \input{1 - low-noise PiC models/main.tex}
            \input{2 - kinetic component/main.tex}
            \input{3 - fluid component/main.tex}
            \input{4 - numerical implementation/main.tex}
        \part{Project Overview}
            \input{5 - research plan/main.tex}
            \input{6 - summary/main.tex}
    
    
    %\section{}
    \newpage
    \pagenumbering{gobble}
        \printbibliography


    \newpage
    \pagenumbering{roman}
    \appendix
        \part{Appendices}
            \input{8 - Hilbert complexes/main.tex}
            \input{9 - weak conservation proofs/main.tex}
\end{document}

    
    
    %\section{}
    \newpage
    \pagenumbering{gobble}
        \printbibliography


    \newpage
    \pagenumbering{roman}
    \appendix
        \part{Appendices}
            \documentclass[12pt, a4paper]{report}

\input{template/main.tex}

\title{\BA{Title in Progress...}}
\author{Boris Andrews}
\affil{Mathematical Institute, University of Oxford}
\date{\today}


\begin{document}
    \pagenumbering{gobble}
    \maketitle
    
    
    \begin{abstract}
        Magnetic confinement reactors---in particular tokamaks---offer one of the most promising options for achieving practical nuclear fusion, with the potential to provide virtually limitless, clean energy. The theoretical and numerical modeling of tokamak plasmas is simultaneously an essential component of effective reactor design, and a great research barrier. Tokamak operational conditions exhibit comparatively low Knudsen numbers. Kinetic effects, including kinetic waves and instabilities, Landau damping, bump-on-tail instabilities and more, are therefore highly influential in tokamak plasma dynamics. Purely fluid models are inherently incapable of capturing these effects, whereas the high dimensionality in purely kinetic models render them practically intractable for most relevant purposes.

        We consider a $\delta\!f$ decomposition model, with a macroscopic fluid background and microscopic kinetic correction, both fully coupled to each other. A similar manner of discretization is proposed to that used in the recent \texttt{STRUPHY} code \cite{Holderied_Possanner_Wang_2021, Holderied_2022, Li_et_al_2023} with a finite-element model for the background and a pseudo-particle/PiC model for the correction.

        The fluid background satisfies the full, non-linear, resistive, compressible, Hall MHD equations. \cite{Laakmann_Hu_Farrell_2022} introduces finite-element(-in-space) implicit timesteppers for the incompressible analogue to this system with structure-preserving (SP) properties in the ideal case, alongside parameter-robust preconditioners. We show that these timesteppers can derive from a finite-element-in-time (FET) (and finite-element-in-space) interpretation. The benefits of this reformulation are discussed, including the derivation of timesteppers that are higher order in time, and the quantifiable dissipative SP properties in the non-ideal, resistive case.
        
        We discuss possible options for extending this FET approach to timesteppers for the compressible case.

        The kinetic corrections satisfy linearized Boltzmann equations. Using a Lénard--Bernstein collision operator, these take Fokker--Planck-like forms \cite{Fokker_1914, Planck_1917} wherein pseudo-particles in the numerical model obey the neoclassical transport equations, with particle-independent Brownian drift terms. This offers a rigorous methodology for incorporating collisions into the particle transport model, without coupling the equations of motions for each particle.
        
        Works by Chen, Chacón et al. \cite{Chen_Chacón_Barnes_2011, Chacón_Chen_Barnes_2013, Chen_Chacón_2014, Chen_Chacón_2015} have developed structure-preserving particle pushers for neoclassical transport in the Vlasov equations, derived from Crank--Nicolson integrators. We show these too can can derive from a FET interpretation, similarly offering potential extensions to higher-order-in-time particle pushers. The FET formulation is used also to consider how the stochastic drift terms can be incorporated into the pushers. Stochastic gyrokinetic expansions are also discussed.

        Different options for the numerical implementation of these schemes are considered.

        Due to the efficacy of FET in the development of SP timesteppers for both the fluid and kinetic component, we hope this approach will prove effective in the future for developing SP timesteppers for the full hybrid model. We hope this will give us the opportunity to incorporate previously inaccessible kinetic effects into the highly effective, modern, finite-element MHD models.
    \end{abstract}
    
    
    \newpage
    \tableofcontents
    
    
    \newpage
    \pagenumbering{arabic}
    %\linenumbers\renewcommand\thelinenumber{\color{black!50}\arabic{linenumber}}
            \input{0 - introduction/main.tex}
        \part{Research}
            \input{1 - low-noise PiC models/main.tex}
            \input{2 - kinetic component/main.tex}
            \input{3 - fluid component/main.tex}
            \input{4 - numerical implementation/main.tex}
        \part{Project Overview}
            \input{5 - research plan/main.tex}
            \input{6 - summary/main.tex}
    
    
    %\section{}
    \newpage
    \pagenumbering{gobble}
        \printbibliography


    \newpage
    \pagenumbering{roman}
    \appendix
        \part{Appendices}
            \input{8 - Hilbert complexes/main.tex}
            \input{9 - weak conservation proofs/main.tex}
\end{document}

            \documentclass[12pt, a4paper]{report}

\input{template/main.tex}

\title{\BA{Title in Progress...}}
\author{Boris Andrews}
\affil{Mathematical Institute, University of Oxford}
\date{\today}


\begin{document}
    \pagenumbering{gobble}
    \maketitle
    
    
    \begin{abstract}
        Magnetic confinement reactors---in particular tokamaks---offer one of the most promising options for achieving practical nuclear fusion, with the potential to provide virtually limitless, clean energy. The theoretical and numerical modeling of tokamak plasmas is simultaneously an essential component of effective reactor design, and a great research barrier. Tokamak operational conditions exhibit comparatively low Knudsen numbers. Kinetic effects, including kinetic waves and instabilities, Landau damping, bump-on-tail instabilities and more, are therefore highly influential in tokamak plasma dynamics. Purely fluid models are inherently incapable of capturing these effects, whereas the high dimensionality in purely kinetic models render them practically intractable for most relevant purposes.

        We consider a $\delta\!f$ decomposition model, with a macroscopic fluid background and microscopic kinetic correction, both fully coupled to each other. A similar manner of discretization is proposed to that used in the recent \texttt{STRUPHY} code \cite{Holderied_Possanner_Wang_2021, Holderied_2022, Li_et_al_2023} with a finite-element model for the background and a pseudo-particle/PiC model for the correction.

        The fluid background satisfies the full, non-linear, resistive, compressible, Hall MHD equations. \cite{Laakmann_Hu_Farrell_2022} introduces finite-element(-in-space) implicit timesteppers for the incompressible analogue to this system with structure-preserving (SP) properties in the ideal case, alongside parameter-robust preconditioners. We show that these timesteppers can derive from a finite-element-in-time (FET) (and finite-element-in-space) interpretation. The benefits of this reformulation are discussed, including the derivation of timesteppers that are higher order in time, and the quantifiable dissipative SP properties in the non-ideal, resistive case.
        
        We discuss possible options for extending this FET approach to timesteppers for the compressible case.

        The kinetic corrections satisfy linearized Boltzmann equations. Using a Lénard--Bernstein collision operator, these take Fokker--Planck-like forms \cite{Fokker_1914, Planck_1917} wherein pseudo-particles in the numerical model obey the neoclassical transport equations, with particle-independent Brownian drift terms. This offers a rigorous methodology for incorporating collisions into the particle transport model, without coupling the equations of motions for each particle.
        
        Works by Chen, Chacón et al. \cite{Chen_Chacón_Barnes_2011, Chacón_Chen_Barnes_2013, Chen_Chacón_2014, Chen_Chacón_2015} have developed structure-preserving particle pushers for neoclassical transport in the Vlasov equations, derived from Crank--Nicolson integrators. We show these too can can derive from a FET interpretation, similarly offering potential extensions to higher-order-in-time particle pushers. The FET formulation is used also to consider how the stochastic drift terms can be incorporated into the pushers. Stochastic gyrokinetic expansions are also discussed.

        Different options for the numerical implementation of these schemes are considered.

        Due to the efficacy of FET in the development of SP timesteppers for both the fluid and kinetic component, we hope this approach will prove effective in the future for developing SP timesteppers for the full hybrid model. We hope this will give us the opportunity to incorporate previously inaccessible kinetic effects into the highly effective, modern, finite-element MHD models.
    \end{abstract}
    
    
    \newpage
    \tableofcontents
    
    
    \newpage
    \pagenumbering{arabic}
    %\linenumbers\renewcommand\thelinenumber{\color{black!50}\arabic{linenumber}}
            \input{0 - introduction/main.tex}
        \part{Research}
            \input{1 - low-noise PiC models/main.tex}
            \input{2 - kinetic component/main.tex}
            \input{3 - fluid component/main.tex}
            \input{4 - numerical implementation/main.tex}
        \part{Project Overview}
            \input{5 - research plan/main.tex}
            \input{6 - summary/main.tex}
    
    
    %\section{}
    \newpage
    \pagenumbering{gobble}
        \printbibliography


    \newpage
    \pagenumbering{roman}
    \appendix
        \part{Appendices}
            \input{8 - Hilbert complexes/main.tex}
            \input{9 - weak conservation proofs/main.tex}
\end{document}

\end{document}

    
    
    %\section{}
    \newpage
    \pagenumbering{gobble}
        \printbibliography


    \newpage
    \pagenumbering{roman}
    \appendix
        \part{Appendices}
            \documentclass[12pt, a4paper]{report}

\documentclass[12pt, a4paper]{report}

\input{template/main.tex}

\title{\BA{Title in Progress...}}
\author{Boris Andrews}
\affil{Mathematical Institute, University of Oxford}
\date{\today}


\begin{document}
    \pagenumbering{gobble}
    \maketitle
    
    
    \begin{abstract}
        Magnetic confinement reactors---in particular tokamaks---offer one of the most promising options for achieving practical nuclear fusion, with the potential to provide virtually limitless, clean energy. The theoretical and numerical modeling of tokamak plasmas is simultaneously an essential component of effective reactor design, and a great research barrier. Tokamak operational conditions exhibit comparatively low Knudsen numbers. Kinetic effects, including kinetic waves and instabilities, Landau damping, bump-on-tail instabilities and more, are therefore highly influential in tokamak plasma dynamics. Purely fluid models are inherently incapable of capturing these effects, whereas the high dimensionality in purely kinetic models render them practically intractable for most relevant purposes.

        We consider a $\delta\!f$ decomposition model, with a macroscopic fluid background and microscopic kinetic correction, both fully coupled to each other. A similar manner of discretization is proposed to that used in the recent \texttt{STRUPHY} code \cite{Holderied_Possanner_Wang_2021, Holderied_2022, Li_et_al_2023} with a finite-element model for the background and a pseudo-particle/PiC model for the correction.

        The fluid background satisfies the full, non-linear, resistive, compressible, Hall MHD equations. \cite{Laakmann_Hu_Farrell_2022} introduces finite-element(-in-space) implicit timesteppers for the incompressible analogue to this system with structure-preserving (SP) properties in the ideal case, alongside parameter-robust preconditioners. We show that these timesteppers can derive from a finite-element-in-time (FET) (and finite-element-in-space) interpretation. The benefits of this reformulation are discussed, including the derivation of timesteppers that are higher order in time, and the quantifiable dissipative SP properties in the non-ideal, resistive case.
        
        We discuss possible options for extending this FET approach to timesteppers for the compressible case.

        The kinetic corrections satisfy linearized Boltzmann equations. Using a Lénard--Bernstein collision operator, these take Fokker--Planck-like forms \cite{Fokker_1914, Planck_1917} wherein pseudo-particles in the numerical model obey the neoclassical transport equations, with particle-independent Brownian drift terms. This offers a rigorous methodology for incorporating collisions into the particle transport model, without coupling the equations of motions for each particle.
        
        Works by Chen, Chacón et al. \cite{Chen_Chacón_Barnes_2011, Chacón_Chen_Barnes_2013, Chen_Chacón_2014, Chen_Chacón_2015} have developed structure-preserving particle pushers for neoclassical transport in the Vlasov equations, derived from Crank--Nicolson integrators. We show these too can can derive from a FET interpretation, similarly offering potential extensions to higher-order-in-time particle pushers. The FET formulation is used also to consider how the stochastic drift terms can be incorporated into the pushers. Stochastic gyrokinetic expansions are also discussed.

        Different options for the numerical implementation of these schemes are considered.

        Due to the efficacy of FET in the development of SP timesteppers for both the fluid and kinetic component, we hope this approach will prove effective in the future for developing SP timesteppers for the full hybrid model. We hope this will give us the opportunity to incorporate previously inaccessible kinetic effects into the highly effective, modern, finite-element MHD models.
    \end{abstract}
    
    
    \newpage
    \tableofcontents
    
    
    \newpage
    \pagenumbering{arabic}
    %\linenumbers\renewcommand\thelinenumber{\color{black!50}\arabic{linenumber}}
            \input{0 - introduction/main.tex}
        \part{Research}
            \input{1 - low-noise PiC models/main.tex}
            \input{2 - kinetic component/main.tex}
            \input{3 - fluid component/main.tex}
            \input{4 - numerical implementation/main.tex}
        \part{Project Overview}
            \input{5 - research plan/main.tex}
            \input{6 - summary/main.tex}
    
    
    %\section{}
    \newpage
    \pagenumbering{gobble}
        \printbibliography


    \newpage
    \pagenumbering{roman}
    \appendix
        \part{Appendices}
            \input{8 - Hilbert complexes/main.tex}
            \input{9 - weak conservation proofs/main.tex}
\end{document}


\title{\BA{Title in Progress...}}
\author{Boris Andrews}
\affil{Mathematical Institute, University of Oxford}
\date{\today}


\begin{document}
    \pagenumbering{gobble}
    \maketitle
    
    
    \begin{abstract}
        Magnetic confinement reactors---in particular tokamaks---offer one of the most promising options for achieving practical nuclear fusion, with the potential to provide virtually limitless, clean energy. The theoretical and numerical modeling of tokamak plasmas is simultaneously an essential component of effective reactor design, and a great research barrier. Tokamak operational conditions exhibit comparatively low Knudsen numbers. Kinetic effects, including kinetic waves and instabilities, Landau damping, bump-on-tail instabilities and more, are therefore highly influential in tokamak plasma dynamics. Purely fluid models are inherently incapable of capturing these effects, whereas the high dimensionality in purely kinetic models render them practically intractable for most relevant purposes.

        We consider a $\delta\!f$ decomposition model, with a macroscopic fluid background and microscopic kinetic correction, both fully coupled to each other. A similar manner of discretization is proposed to that used in the recent \texttt{STRUPHY} code \cite{Holderied_Possanner_Wang_2021, Holderied_2022, Li_et_al_2023} with a finite-element model for the background and a pseudo-particle/PiC model for the correction.

        The fluid background satisfies the full, non-linear, resistive, compressible, Hall MHD equations. \cite{Laakmann_Hu_Farrell_2022} introduces finite-element(-in-space) implicit timesteppers for the incompressible analogue to this system with structure-preserving (SP) properties in the ideal case, alongside parameter-robust preconditioners. We show that these timesteppers can derive from a finite-element-in-time (FET) (and finite-element-in-space) interpretation. The benefits of this reformulation are discussed, including the derivation of timesteppers that are higher order in time, and the quantifiable dissipative SP properties in the non-ideal, resistive case.
        
        We discuss possible options for extending this FET approach to timesteppers for the compressible case.

        The kinetic corrections satisfy linearized Boltzmann equations. Using a Lénard--Bernstein collision operator, these take Fokker--Planck-like forms \cite{Fokker_1914, Planck_1917} wherein pseudo-particles in the numerical model obey the neoclassical transport equations, with particle-independent Brownian drift terms. This offers a rigorous methodology for incorporating collisions into the particle transport model, without coupling the equations of motions for each particle.
        
        Works by Chen, Chacón et al. \cite{Chen_Chacón_Barnes_2011, Chacón_Chen_Barnes_2013, Chen_Chacón_2014, Chen_Chacón_2015} have developed structure-preserving particle pushers for neoclassical transport in the Vlasov equations, derived from Crank--Nicolson integrators. We show these too can can derive from a FET interpretation, similarly offering potential extensions to higher-order-in-time particle pushers. The FET formulation is used also to consider how the stochastic drift terms can be incorporated into the pushers. Stochastic gyrokinetic expansions are also discussed.

        Different options for the numerical implementation of these schemes are considered.

        Due to the efficacy of FET in the development of SP timesteppers for both the fluid and kinetic component, we hope this approach will prove effective in the future for developing SP timesteppers for the full hybrid model. We hope this will give us the opportunity to incorporate previously inaccessible kinetic effects into the highly effective, modern, finite-element MHD models.
    \end{abstract}
    
    
    \newpage
    \tableofcontents
    
    
    \newpage
    \pagenumbering{arabic}
    %\linenumbers\renewcommand\thelinenumber{\color{black!50}\arabic{linenumber}}
            \documentclass[12pt, a4paper]{report}

\input{template/main.tex}

\title{\BA{Title in Progress...}}
\author{Boris Andrews}
\affil{Mathematical Institute, University of Oxford}
\date{\today}


\begin{document}
    \pagenumbering{gobble}
    \maketitle
    
    
    \begin{abstract}
        Magnetic confinement reactors---in particular tokamaks---offer one of the most promising options for achieving practical nuclear fusion, with the potential to provide virtually limitless, clean energy. The theoretical and numerical modeling of tokamak plasmas is simultaneously an essential component of effective reactor design, and a great research barrier. Tokamak operational conditions exhibit comparatively low Knudsen numbers. Kinetic effects, including kinetic waves and instabilities, Landau damping, bump-on-tail instabilities and more, are therefore highly influential in tokamak plasma dynamics. Purely fluid models are inherently incapable of capturing these effects, whereas the high dimensionality in purely kinetic models render them practically intractable for most relevant purposes.

        We consider a $\delta\!f$ decomposition model, with a macroscopic fluid background and microscopic kinetic correction, both fully coupled to each other. A similar manner of discretization is proposed to that used in the recent \texttt{STRUPHY} code \cite{Holderied_Possanner_Wang_2021, Holderied_2022, Li_et_al_2023} with a finite-element model for the background and a pseudo-particle/PiC model for the correction.

        The fluid background satisfies the full, non-linear, resistive, compressible, Hall MHD equations. \cite{Laakmann_Hu_Farrell_2022} introduces finite-element(-in-space) implicit timesteppers for the incompressible analogue to this system with structure-preserving (SP) properties in the ideal case, alongside parameter-robust preconditioners. We show that these timesteppers can derive from a finite-element-in-time (FET) (and finite-element-in-space) interpretation. The benefits of this reformulation are discussed, including the derivation of timesteppers that are higher order in time, and the quantifiable dissipative SP properties in the non-ideal, resistive case.
        
        We discuss possible options for extending this FET approach to timesteppers for the compressible case.

        The kinetic corrections satisfy linearized Boltzmann equations. Using a Lénard--Bernstein collision operator, these take Fokker--Planck-like forms \cite{Fokker_1914, Planck_1917} wherein pseudo-particles in the numerical model obey the neoclassical transport equations, with particle-independent Brownian drift terms. This offers a rigorous methodology for incorporating collisions into the particle transport model, without coupling the equations of motions for each particle.
        
        Works by Chen, Chacón et al. \cite{Chen_Chacón_Barnes_2011, Chacón_Chen_Barnes_2013, Chen_Chacón_2014, Chen_Chacón_2015} have developed structure-preserving particle pushers for neoclassical transport in the Vlasov equations, derived from Crank--Nicolson integrators. We show these too can can derive from a FET interpretation, similarly offering potential extensions to higher-order-in-time particle pushers. The FET formulation is used also to consider how the stochastic drift terms can be incorporated into the pushers. Stochastic gyrokinetic expansions are also discussed.

        Different options for the numerical implementation of these schemes are considered.

        Due to the efficacy of FET in the development of SP timesteppers for both the fluid and kinetic component, we hope this approach will prove effective in the future for developing SP timesteppers for the full hybrid model. We hope this will give us the opportunity to incorporate previously inaccessible kinetic effects into the highly effective, modern, finite-element MHD models.
    \end{abstract}
    
    
    \newpage
    \tableofcontents
    
    
    \newpage
    \pagenumbering{arabic}
    %\linenumbers\renewcommand\thelinenumber{\color{black!50}\arabic{linenumber}}
            \input{0 - introduction/main.tex}
        \part{Research}
            \input{1 - low-noise PiC models/main.tex}
            \input{2 - kinetic component/main.tex}
            \input{3 - fluid component/main.tex}
            \input{4 - numerical implementation/main.tex}
        \part{Project Overview}
            \input{5 - research plan/main.tex}
            \input{6 - summary/main.tex}
    
    
    %\section{}
    \newpage
    \pagenumbering{gobble}
        \printbibliography


    \newpage
    \pagenumbering{roman}
    \appendix
        \part{Appendices}
            \input{8 - Hilbert complexes/main.tex}
            \input{9 - weak conservation proofs/main.tex}
\end{document}

        \part{Research}
            \documentclass[12pt, a4paper]{report}

\input{template/main.tex}

\title{\BA{Title in Progress...}}
\author{Boris Andrews}
\affil{Mathematical Institute, University of Oxford}
\date{\today}


\begin{document}
    \pagenumbering{gobble}
    \maketitle
    
    
    \begin{abstract}
        Magnetic confinement reactors---in particular tokamaks---offer one of the most promising options for achieving practical nuclear fusion, with the potential to provide virtually limitless, clean energy. The theoretical and numerical modeling of tokamak plasmas is simultaneously an essential component of effective reactor design, and a great research barrier. Tokamak operational conditions exhibit comparatively low Knudsen numbers. Kinetic effects, including kinetic waves and instabilities, Landau damping, bump-on-tail instabilities and more, are therefore highly influential in tokamak plasma dynamics. Purely fluid models are inherently incapable of capturing these effects, whereas the high dimensionality in purely kinetic models render them practically intractable for most relevant purposes.

        We consider a $\delta\!f$ decomposition model, with a macroscopic fluid background and microscopic kinetic correction, both fully coupled to each other. A similar manner of discretization is proposed to that used in the recent \texttt{STRUPHY} code \cite{Holderied_Possanner_Wang_2021, Holderied_2022, Li_et_al_2023} with a finite-element model for the background and a pseudo-particle/PiC model for the correction.

        The fluid background satisfies the full, non-linear, resistive, compressible, Hall MHD equations. \cite{Laakmann_Hu_Farrell_2022} introduces finite-element(-in-space) implicit timesteppers for the incompressible analogue to this system with structure-preserving (SP) properties in the ideal case, alongside parameter-robust preconditioners. We show that these timesteppers can derive from a finite-element-in-time (FET) (and finite-element-in-space) interpretation. The benefits of this reformulation are discussed, including the derivation of timesteppers that are higher order in time, and the quantifiable dissipative SP properties in the non-ideal, resistive case.
        
        We discuss possible options for extending this FET approach to timesteppers for the compressible case.

        The kinetic corrections satisfy linearized Boltzmann equations. Using a Lénard--Bernstein collision operator, these take Fokker--Planck-like forms \cite{Fokker_1914, Planck_1917} wherein pseudo-particles in the numerical model obey the neoclassical transport equations, with particle-independent Brownian drift terms. This offers a rigorous methodology for incorporating collisions into the particle transport model, without coupling the equations of motions for each particle.
        
        Works by Chen, Chacón et al. \cite{Chen_Chacón_Barnes_2011, Chacón_Chen_Barnes_2013, Chen_Chacón_2014, Chen_Chacón_2015} have developed structure-preserving particle pushers for neoclassical transport in the Vlasov equations, derived from Crank--Nicolson integrators. We show these too can can derive from a FET interpretation, similarly offering potential extensions to higher-order-in-time particle pushers. The FET formulation is used also to consider how the stochastic drift terms can be incorporated into the pushers. Stochastic gyrokinetic expansions are also discussed.

        Different options for the numerical implementation of these schemes are considered.

        Due to the efficacy of FET in the development of SP timesteppers for both the fluid and kinetic component, we hope this approach will prove effective in the future for developing SP timesteppers for the full hybrid model. We hope this will give us the opportunity to incorporate previously inaccessible kinetic effects into the highly effective, modern, finite-element MHD models.
    \end{abstract}
    
    
    \newpage
    \tableofcontents
    
    
    \newpage
    \pagenumbering{arabic}
    %\linenumbers\renewcommand\thelinenumber{\color{black!50}\arabic{linenumber}}
            \input{0 - introduction/main.tex}
        \part{Research}
            \input{1 - low-noise PiC models/main.tex}
            \input{2 - kinetic component/main.tex}
            \input{3 - fluid component/main.tex}
            \input{4 - numerical implementation/main.tex}
        \part{Project Overview}
            \input{5 - research plan/main.tex}
            \input{6 - summary/main.tex}
    
    
    %\section{}
    \newpage
    \pagenumbering{gobble}
        \printbibliography


    \newpage
    \pagenumbering{roman}
    \appendix
        \part{Appendices}
            \input{8 - Hilbert complexes/main.tex}
            \input{9 - weak conservation proofs/main.tex}
\end{document}

            \documentclass[12pt, a4paper]{report}

\input{template/main.tex}

\title{\BA{Title in Progress...}}
\author{Boris Andrews}
\affil{Mathematical Institute, University of Oxford}
\date{\today}


\begin{document}
    \pagenumbering{gobble}
    \maketitle
    
    
    \begin{abstract}
        Magnetic confinement reactors---in particular tokamaks---offer one of the most promising options for achieving practical nuclear fusion, with the potential to provide virtually limitless, clean energy. The theoretical and numerical modeling of tokamak plasmas is simultaneously an essential component of effective reactor design, and a great research barrier. Tokamak operational conditions exhibit comparatively low Knudsen numbers. Kinetic effects, including kinetic waves and instabilities, Landau damping, bump-on-tail instabilities and more, are therefore highly influential in tokamak plasma dynamics. Purely fluid models are inherently incapable of capturing these effects, whereas the high dimensionality in purely kinetic models render them practically intractable for most relevant purposes.

        We consider a $\delta\!f$ decomposition model, with a macroscopic fluid background and microscopic kinetic correction, both fully coupled to each other. A similar manner of discretization is proposed to that used in the recent \texttt{STRUPHY} code \cite{Holderied_Possanner_Wang_2021, Holderied_2022, Li_et_al_2023} with a finite-element model for the background and a pseudo-particle/PiC model for the correction.

        The fluid background satisfies the full, non-linear, resistive, compressible, Hall MHD equations. \cite{Laakmann_Hu_Farrell_2022} introduces finite-element(-in-space) implicit timesteppers for the incompressible analogue to this system with structure-preserving (SP) properties in the ideal case, alongside parameter-robust preconditioners. We show that these timesteppers can derive from a finite-element-in-time (FET) (and finite-element-in-space) interpretation. The benefits of this reformulation are discussed, including the derivation of timesteppers that are higher order in time, and the quantifiable dissipative SP properties in the non-ideal, resistive case.
        
        We discuss possible options for extending this FET approach to timesteppers for the compressible case.

        The kinetic corrections satisfy linearized Boltzmann equations. Using a Lénard--Bernstein collision operator, these take Fokker--Planck-like forms \cite{Fokker_1914, Planck_1917} wherein pseudo-particles in the numerical model obey the neoclassical transport equations, with particle-independent Brownian drift terms. This offers a rigorous methodology for incorporating collisions into the particle transport model, without coupling the equations of motions for each particle.
        
        Works by Chen, Chacón et al. \cite{Chen_Chacón_Barnes_2011, Chacón_Chen_Barnes_2013, Chen_Chacón_2014, Chen_Chacón_2015} have developed structure-preserving particle pushers for neoclassical transport in the Vlasov equations, derived from Crank--Nicolson integrators. We show these too can can derive from a FET interpretation, similarly offering potential extensions to higher-order-in-time particle pushers. The FET formulation is used also to consider how the stochastic drift terms can be incorporated into the pushers. Stochastic gyrokinetic expansions are also discussed.

        Different options for the numerical implementation of these schemes are considered.

        Due to the efficacy of FET in the development of SP timesteppers for both the fluid and kinetic component, we hope this approach will prove effective in the future for developing SP timesteppers for the full hybrid model. We hope this will give us the opportunity to incorporate previously inaccessible kinetic effects into the highly effective, modern, finite-element MHD models.
    \end{abstract}
    
    
    \newpage
    \tableofcontents
    
    
    \newpage
    \pagenumbering{arabic}
    %\linenumbers\renewcommand\thelinenumber{\color{black!50}\arabic{linenumber}}
            \input{0 - introduction/main.tex}
        \part{Research}
            \input{1 - low-noise PiC models/main.tex}
            \input{2 - kinetic component/main.tex}
            \input{3 - fluid component/main.tex}
            \input{4 - numerical implementation/main.tex}
        \part{Project Overview}
            \input{5 - research plan/main.tex}
            \input{6 - summary/main.tex}
    
    
    %\section{}
    \newpage
    \pagenumbering{gobble}
        \printbibliography


    \newpage
    \pagenumbering{roman}
    \appendix
        \part{Appendices}
            \input{8 - Hilbert complexes/main.tex}
            \input{9 - weak conservation proofs/main.tex}
\end{document}

            \documentclass[12pt, a4paper]{report}

\input{template/main.tex}

\title{\BA{Title in Progress...}}
\author{Boris Andrews}
\affil{Mathematical Institute, University of Oxford}
\date{\today}


\begin{document}
    \pagenumbering{gobble}
    \maketitle
    
    
    \begin{abstract}
        Magnetic confinement reactors---in particular tokamaks---offer one of the most promising options for achieving practical nuclear fusion, with the potential to provide virtually limitless, clean energy. The theoretical and numerical modeling of tokamak plasmas is simultaneously an essential component of effective reactor design, and a great research barrier. Tokamak operational conditions exhibit comparatively low Knudsen numbers. Kinetic effects, including kinetic waves and instabilities, Landau damping, bump-on-tail instabilities and more, are therefore highly influential in tokamak plasma dynamics. Purely fluid models are inherently incapable of capturing these effects, whereas the high dimensionality in purely kinetic models render them practically intractable for most relevant purposes.

        We consider a $\delta\!f$ decomposition model, with a macroscopic fluid background and microscopic kinetic correction, both fully coupled to each other. A similar manner of discretization is proposed to that used in the recent \texttt{STRUPHY} code \cite{Holderied_Possanner_Wang_2021, Holderied_2022, Li_et_al_2023} with a finite-element model for the background and a pseudo-particle/PiC model for the correction.

        The fluid background satisfies the full, non-linear, resistive, compressible, Hall MHD equations. \cite{Laakmann_Hu_Farrell_2022} introduces finite-element(-in-space) implicit timesteppers for the incompressible analogue to this system with structure-preserving (SP) properties in the ideal case, alongside parameter-robust preconditioners. We show that these timesteppers can derive from a finite-element-in-time (FET) (and finite-element-in-space) interpretation. The benefits of this reformulation are discussed, including the derivation of timesteppers that are higher order in time, and the quantifiable dissipative SP properties in the non-ideal, resistive case.
        
        We discuss possible options for extending this FET approach to timesteppers for the compressible case.

        The kinetic corrections satisfy linearized Boltzmann equations. Using a Lénard--Bernstein collision operator, these take Fokker--Planck-like forms \cite{Fokker_1914, Planck_1917} wherein pseudo-particles in the numerical model obey the neoclassical transport equations, with particle-independent Brownian drift terms. This offers a rigorous methodology for incorporating collisions into the particle transport model, without coupling the equations of motions for each particle.
        
        Works by Chen, Chacón et al. \cite{Chen_Chacón_Barnes_2011, Chacón_Chen_Barnes_2013, Chen_Chacón_2014, Chen_Chacón_2015} have developed structure-preserving particle pushers for neoclassical transport in the Vlasov equations, derived from Crank--Nicolson integrators. We show these too can can derive from a FET interpretation, similarly offering potential extensions to higher-order-in-time particle pushers. The FET formulation is used also to consider how the stochastic drift terms can be incorporated into the pushers. Stochastic gyrokinetic expansions are also discussed.

        Different options for the numerical implementation of these schemes are considered.

        Due to the efficacy of FET in the development of SP timesteppers for both the fluid and kinetic component, we hope this approach will prove effective in the future for developing SP timesteppers for the full hybrid model. We hope this will give us the opportunity to incorporate previously inaccessible kinetic effects into the highly effective, modern, finite-element MHD models.
    \end{abstract}
    
    
    \newpage
    \tableofcontents
    
    
    \newpage
    \pagenumbering{arabic}
    %\linenumbers\renewcommand\thelinenumber{\color{black!50}\arabic{linenumber}}
            \input{0 - introduction/main.tex}
        \part{Research}
            \input{1 - low-noise PiC models/main.tex}
            \input{2 - kinetic component/main.tex}
            \input{3 - fluid component/main.tex}
            \input{4 - numerical implementation/main.tex}
        \part{Project Overview}
            \input{5 - research plan/main.tex}
            \input{6 - summary/main.tex}
    
    
    %\section{}
    \newpage
    \pagenumbering{gobble}
        \printbibliography


    \newpage
    \pagenumbering{roman}
    \appendix
        \part{Appendices}
            \input{8 - Hilbert complexes/main.tex}
            \input{9 - weak conservation proofs/main.tex}
\end{document}

            \documentclass[12pt, a4paper]{report}

\input{template/main.tex}

\title{\BA{Title in Progress...}}
\author{Boris Andrews}
\affil{Mathematical Institute, University of Oxford}
\date{\today}


\begin{document}
    \pagenumbering{gobble}
    \maketitle
    
    
    \begin{abstract}
        Magnetic confinement reactors---in particular tokamaks---offer one of the most promising options for achieving practical nuclear fusion, with the potential to provide virtually limitless, clean energy. The theoretical and numerical modeling of tokamak plasmas is simultaneously an essential component of effective reactor design, and a great research barrier. Tokamak operational conditions exhibit comparatively low Knudsen numbers. Kinetic effects, including kinetic waves and instabilities, Landau damping, bump-on-tail instabilities and more, are therefore highly influential in tokamak plasma dynamics. Purely fluid models are inherently incapable of capturing these effects, whereas the high dimensionality in purely kinetic models render them practically intractable for most relevant purposes.

        We consider a $\delta\!f$ decomposition model, with a macroscopic fluid background and microscopic kinetic correction, both fully coupled to each other. A similar manner of discretization is proposed to that used in the recent \texttt{STRUPHY} code \cite{Holderied_Possanner_Wang_2021, Holderied_2022, Li_et_al_2023} with a finite-element model for the background and a pseudo-particle/PiC model for the correction.

        The fluid background satisfies the full, non-linear, resistive, compressible, Hall MHD equations. \cite{Laakmann_Hu_Farrell_2022} introduces finite-element(-in-space) implicit timesteppers for the incompressible analogue to this system with structure-preserving (SP) properties in the ideal case, alongside parameter-robust preconditioners. We show that these timesteppers can derive from a finite-element-in-time (FET) (and finite-element-in-space) interpretation. The benefits of this reformulation are discussed, including the derivation of timesteppers that are higher order in time, and the quantifiable dissipative SP properties in the non-ideal, resistive case.
        
        We discuss possible options for extending this FET approach to timesteppers for the compressible case.

        The kinetic corrections satisfy linearized Boltzmann equations. Using a Lénard--Bernstein collision operator, these take Fokker--Planck-like forms \cite{Fokker_1914, Planck_1917} wherein pseudo-particles in the numerical model obey the neoclassical transport equations, with particle-independent Brownian drift terms. This offers a rigorous methodology for incorporating collisions into the particle transport model, without coupling the equations of motions for each particle.
        
        Works by Chen, Chacón et al. \cite{Chen_Chacón_Barnes_2011, Chacón_Chen_Barnes_2013, Chen_Chacón_2014, Chen_Chacón_2015} have developed structure-preserving particle pushers for neoclassical transport in the Vlasov equations, derived from Crank--Nicolson integrators. We show these too can can derive from a FET interpretation, similarly offering potential extensions to higher-order-in-time particle pushers. The FET formulation is used also to consider how the stochastic drift terms can be incorporated into the pushers. Stochastic gyrokinetic expansions are also discussed.

        Different options for the numerical implementation of these schemes are considered.

        Due to the efficacy of FET in the development of SP timesteppers for both the fluid and kinetic component, we hope this approach will prove effective in the future for developing SP timesteppers for the full hybrid model. We hope this will give us the opportunity to incorporate previously inaccessible kinetic effects into the highly effective, modern, finite-element MHD models.
    \end{abstract}
    
    
    \newpage
    \tableofcontents
    
    
    \newpage
    \pagenumbering{arabic}
    %\linenumbers\renewcommand\thelinenumber{\color{black!50}\arabic{linenumber}}
            \input{0 - introduction/main.tex}
        \part{Research}
            \input{1 - low-noise PiC models/main.tex}
            \input{2 - kinetic component/main.tex}
            \input{3 - fluid component/main.tex}
            \input{4 - numerical implementation/main.tex}
        \part{Project Overview}
            \input{5 - research plan/main.tex}
            \input{6 - summary/main.tex}
    
    
    %\section{}
    \newpage
    \pagenumbering{gobble}
        \printbibliography


    \newpage
    \pagenumbering{roman}
    \appendix
        \part{Appendices}
            \input{8 - Hilbert complexes/main.tex}
            \input{9 - weak conservation proofs/main.tex}
\end{document}

        \part{Project Overview}
            \documentclass[12pt, a4paper]{report}

\input{template/main.tex}

\title{\BA{Title in Progress...}}
\author{Boris Andrews}
\affil{Mathematical Institute, University of Oxford}
\date{\today}


\begin{document}
    \pagenumbering{gobble}
    \maketitle
    
    
    \begin{abstract}
        Magnetic confinement reactors---in particular tokamaks---offer one of the most promising options for achieving practical nuclear fusion, with the potential to provide virtually limitless, clean energy. The theoretical and numerical modeling of tokamak plasmas is simultaneously an essential component of effective reactor design, and a great research barrier. Tokamak operational conditions exhibit comparatively low Knudsen numbers. Kinetic effects, including kinetic waves and instabilities, Landau damping, bump-on-tail instabilities and more, are therefore highly influential in tokamak plasma dynamics. Purely fluid models are inherently incapable of capturing these effects, whereas the high dimensionality in purely kinetic models render them practically intractable for most relevant purposes.

        We consider a $\delta\!f$ decomposition model, with a macroscopic fluid background and microscopic kinetic correction, both fully coupled to each other. A similar manner of discretization is proposed to that used in the recent \texttt{STRUPHY} code \cite{Holderied_Possanner_Wang_2021, Holderied_2022, Li_et_al_2023} with a finite-element model for the background and a pseudo-particle/PiC model for the correction.

        The fluid background satisfies the full, non-linear, resistive, compressible, Hall MHD equations. \cite{Laakmann_Hu_Farrell_2022} introduces finite-element(-in-space) implicit timesteppers for the incompressible analogue to this system with structure-preserving (SP) properties in the ideal case, alongside parameter-robust preconditioners. We show that these timesteppers can derive from a finite-element-in-time (FET) (and finite-element-in-space) interpretation. The benefits of this reformulation are discussed, including the derivation of timesteppers that are higher order in time, and the quantifiable dissipative SP properties in the non-ideal, resistive case.
        
        We discuss possible options for extending this FET approach to timesteppers for the compressible case.

        The kinetic corrections satisfy linearized Boltzmann equations. Using a Lénard--Bernstein collision operator, these take Fokker--Planck-like forms \cite{Fokker_1914, Planck_1917} wherein pseudo-particles in the numerical model obey the neoclassical transport equations, with particle-independent Brownian drift terms. This offers a rigorous methodology for incorporating collisions into the particle transport model, without coupling the equations of motions for each particle.
        
        Works by Chen, Chacón et al. \cite{Chen_Chacón_Barnes_2011, Chacón_Chen_Barnes_2013, Chen_Chacón_2014, Chen_Chacón_2015} have developed structure-preserving particle pushers for neoclassical transport in the Vlasov equations, derived from Crank--Nicolson integrators. We show these too can can derive from a FET interpretation, similarly offering potential extensions to higher-order-in-time particle pushers. The FET formulation is used also to consider how the stochastic drift terms can be incorporated into the pushers. Stochastic gyrokinetic expansions are also discussed.

        Different options for the numerical implementation of these schemes are considered.

        Due to the efficacy of FET in the development of SP timesteppers for both the fluid and kinetic component, we hope this approach will prove effective in the future for developing SP timesteppers for the full hybrid model. We hope this will give us the opportunity to incorporate previously inaccessible kinetic effects into the highly effective, modern, finite-element MHD models.
    \end{abstract}
    
    
    \newpage
    \tableofcontents
    
    
    \newpage
    \pagenumbering{arabic}
    %\linenumbers\renewcommand\thelinenumber{\color{black!50}\arabic{linenumber}}
            \input{0 - introduction/main.tex}
        \part{Research}
            \input{1 - low-noise PiC models/main.tex}
            \input{2 - kinetic component/main.tex}
            \input{3 - fluid component/main.tex}
            \input{4 - numerical implementation/main.tex}
        \part{Project Overview}
            \input{5 - research plan/main.tex}
            \input{6 - summary/main.tex}
    
    
    %\section{}
    \newpage
    \pagenumbering{gobble}
        \printbibliography


    \newpage
    \pagenumbering{roman}
    \appendix
        \part{Appendices}
            \input{8 - Hilbert complexes/main.tex}
            \input{9 - weak conservation proofs/main.tex}
\end{document}

            \documentclass[12pt, a4paper]{report}

\input{template/main.tex}

\title{\BA{Title in Progress...}}
\author{Boris Andrews}
\affil{Mathematical Institute, University of Oxford}
\date{\today}


\begin{document}
    \pagenumbering{gobble}
    \maketitle
    
    
    \begin{abstract}
        Magnetic confinement reactors---in particular tokamaks---offer one of the most promising options for achieving practical nuclear fusion, with the potential to provide virtually limitless, clean energy. The theoretical and numerical modeling of tokamak plasmas is simultaneously an essential component of effective reactor design, and a great research barrier. Tokamak operational conditions exhibit comparatively low Knudsen numbers. Kinetic effects, including kinetic waves and instabilities, Landau damping, bump-on-tail instabilities and more, are therefore highly influential in tokamak plasma dynamics. Purely fluid models are inherently incapable of capturing these effects, whereas the high dimensionality in purely kinetic models render them practically intractable for most relevant purposes.

        We consider a $\delta\!f$ decomposition model, with a macroscopic fluid background and microscopic kinetic correction, both fully coupled to each other. A similar manner of discretization is proposed to that used in the recent \texttt{STRUPHY} code \cite{Holderied_Possanner_Wang_2021, Holderied_2022, Li_et_al_2023} with a finite-element model for the background and a pseudo-particle/PiC model for the correction.

        The fluid background satisfies the full, non-linear, resistive, compressible, Hall MHD equations. \cite{Laakmann_Hu_Farrell_2022} introduces finite-element(-in-space) implicit timesteppers for the incompressible analogue to this system with structure-preserving (SP) properties in the ideal case, alongside parameter-robust preconditioners. We show that these timesteppers can derive from a finite-element-in-time (FET) (and finite-element-in-space) interpretation. The benefits of this reformulation are discussed, including the derivation of timesteppers that are higher order in time, and the quantifiable dissipative SP properties in the non-ideal, resistive case.
        
        We discuss possible options for extending this FET approach to timesteppers for the compressible case.

        The kinetic corrections satisfy linearized Boltzmann equations. Using a Lénard--Bernstein collision operator, these take Fokker--Planck-like forms \cite{Fokker_1914, Planck_1917} wherein pseudo-particles in the numerical model obey the neoclassical transport equations, with particle-independent Brownian drift terms. This offers a rigorous methodology for incorporating collisions into the particle transport model, without coupling the equations of motions for each particle.
        
        Works by Chen, Chacón et al. \cite{Chen_Chacón_Barnes_2011, Chacón_Chen_Barnes_2013, Chen_Chacón_2014, Chen_Chacón_2015} have developed structure-preserving particle pushers for neoclassical transport in the Vlasov equations, derived from Crank--Nicolson integrators. We show these too can can derive from a FET interpretation, similarly offering potential extensions to higher-order-in-time particle pushers. The FET formulation is used also to consider how the stochastic drift terms can be incorporated into the pushers. Stochastic gyrokinetic expansions are also discussed.

        Different options for the numerical implementation of these schemes are considered.

        Due to the efficacy of FET in the development of SP timesteppers for both the fluid and kinetic component, we hope this approach will prove effective in the future for developing SP timesteppers for the full hybrid model. We hope this will give us the opportunity to incorporate previously inaccessible kinetic effects into the highly effective, modern, finite-element MHD models.
    \end{abstract}
    
    
    \newpage
    \tableofcontents
    
    
    \newpage
    \pagenumbering{arabic}
    %\linenumbers\renewcommand\thelinenumber{\color{black!50}\arabic{linenumber}}
            \input{0 - introduction/main.tex}
        \part{Research}
            \input{1 - low-noise PiC models/main.tex}
            \input{2 - kinetic component/main.tex}
            \input{3 - fluid component/main.tex}
            \input{4 - numerical implementation/main.tex}
        \part{Project Overview}
            \input{5 - research plan/main.tex}
            \input{6 - summary/main.tex}
    
    
    %\section{}
    \newpage
    \pagenumbering{gobble}
        \printbibliography


    \newpage
    \pagenumbering{roman}
    \appendix
        \part{Appendices}
            \input{8 - Hilbert complexes/main.tex}
            \input{9 - weak conservation proofs/main.tex}
\end{document}

    
    
    %\section{}
    \newpage
    \pagenumbering{gobble}
        \printbibliography


    \newpage
    \pagenumbering{roman}
    \appendix
        \part{Appendices}
            \documentclass[12pt, a4paper]{report}

\input{template/main.tex}

\title{\BA{Title in Progress...}}
\author{Boris Andrews}
\affil{Mathematical Institute, University of Oxford}
\date{\today}


\begin{document}
    \pagenumbering{gobble}
    \maketitle
    
    
    \begin{abstract}
        Magnetic confinement reactors---in particular tokamaks---offer one of the most promising options for achieving practical nuclear fusion, with the potential to provide virtually limitless, clean energy. The theoretical and numerical modeling of tokamak plasmas is simultaneously an essential component of effective reactor design, and a great research barrier. Tokamak operational conditions exhibit comparatively low Knudsen numbers. Kinetic effects, including kinetic waves and instabilities, Landau damping, bump-on-tail instabilities and more, are therefore highly influential in tokamak plasma dynamics. Purely fluid models are inherently incapable of capturing these effects, whereas the high dimensionality in purely kinetic models render them practically intractable for most relevant purposes.

        We consider a $\delta\!f$ decomposition model, with a macroscopic fluid background and microscopic kinetic correction, both fully coupled to each other. A similar manner of discretization is proposed to that used in the recent \texttt{STRUPHY} code \cite{Holderied_Possanner_Wang_2021, Holderied_2022, Li_et_al_2023} with a finite-element model for the background and a pseudo-particle/PiC model for the correction.

        The fluid background satisfies the full, non-linear, resistive, compressible, Hall MHD equations. \cite{Laakmann_Hu_Farrell_2022} introduces finite-element(-in-space) implicit timesteppers for the incompressible analogue to this system with structure-preserving (SP) properties in the ideal case, alongside parameter-robust preconditioners. We show that these timesteppers can derive from a finite-element-in-time (FET) (and finite-element-in-space) interpretation. The benefits of this reformulation are discussed, including the derivation of timesteppers that are higher order in time, and the quantifiable dissipative SP properties in the non-ideal, resistive case.
        
        We discuss possible options for extending this FET approach to timesteppers for the compressible case.

        The kinetic corrections satisfy linearized Boltzmann equations. Using a Lénard--Bernstein collision operator, these take Fokker--Planck-like forms \cite{Fokker_1914, Planck_1917} wherein pseudo-particles in the numerical model obey the neoclassical transport equations, with particle-independent Brownian drift terms. This offers a rigorous methodology for incorporating collisions into the particle transport model, without coupling the equations of motions for each particle.
        
        Works by Chen, Chacón et al. \cite{Chen_Chacón_Barnes_2011, Chacón_Chen_Barnes_2013, Chen_Chacón_2014, Chen_Chacón_2015} have developed structure-preserving particle pushers for neoclassical transport in the Vlasov equations, derived from Crank--Nicolson integrators. We show these too can can derive from a FET interpretation, similarly offering potential extensions to higher-order-in-time particle pushers. The FET formulation is used also to consider how the stochastic drift terms can be incorporated into the pushers. Stochastic gyrokinetic expansions are also discussed.

        Different options for the numerical implementation of these schemes are considered.

        Due to the efficacy of FET in the development of SP timesteppers for both the fluid and kinetic component, we hope this approach will prove effective in the future for developing SP timesteppers for the full hybrid model. We hope this will give us the opportunity to incorporate previously inaccessible kinetic effects into the highly effective, modern, finite-element MHD models.
    \end{abstract}
    
    
    \newpage
    \tableofcontents
    
    
    \newpage
    \pagenumbering{arabic}
    %\linenumbers\renewcommand\thelinenumber{\color{black!50}\arabic{linenumber}}
            \input{0 - introduction/main.tex}
        \part{Research}
            \input{1 - low-noise PiC models/main.tex}
            \input{2 - kinetic component/main.tex}
            \input{3 - fluid component/main.tex}
            \input{4 - numerical implementation/main.tex}
        \part{Project Overview}
            \input{5 - research plan/main.tex}
            \input{6 - summary/main.tex}
    
    
    %\section{}
    \newpage
    \pagenumbering{gobble}
        \printbibliography


    \newpage
    \pagenumbering{roman}
    \appendix
        \part{Appendices}
            \input{8 - Hilbert complexes/main.tex}
            \input{9 - weak conservation proofs/main.tex}
\end{document}

            \documentclass[12pt, a4paper]{report}

\input{template/main.tex}

\title{\BA{Title in Progress...}}
\author{Boris Andrews}
\affil{Mathematical Institute, University of Oxford}
\date{\today}


\begin{document}
    \pagenumbering{gobble}
    \maketitle
    
    
    \begin{abstract}
        Magnetic confinement reactors---in particular tokamaks---offer one of the most promising options for achieving practical nuclear fusion, with the potential to provide virtually limitless, clean energy. The theoretical and numerical modeling of tokamak plasmas is simultaneously an essential component of effective reactor design, and a great research barrier. Tokamak operational conditions exhibit comparatively low Knudsen numbers. Kinetic effects, including kinetic waves and instabilities, Landau damping, bump-on-tail instabilities and more, are therefore highly influential in tokamak plasma dynamics. Purely fluid models are inherently incapable of capturing these effects, whereas the high dimensionality in purely kinetic models render them practically intractable for most relevant purposes.

        We consider a $\delta\!f$ decomposition model, with a macroscopic fluid background and microscopic kinetic correction, both fully coupled to each other. A similar manner of discretization is proposed to that used in the recent \texttt{STRUPHY} code \cite{Holderied_Possanner_Wang_2021, Holderied_2022, Li_et_al_2023} with a finite-element model for the background and a pseudo-particle/PiC model for the correction.

        The fluid background satisfies the full, non-linear, resistive, compressible, Hall MHD equations. \cite{Laakmann_Hu_Farrell_2022} introduces finite-element(-in-space) implicit timesteppers for the incompressible analogue to this system with structure-preserving (SP) properties in the ideal case, alongside parameter-robust preconditioners. We show that these timesteppers can derive from a finite-element-in-time (FET) (and finite-element-in-space) interpretation. The benefits of this reformulation are discussed, including the derivation of timesteppers that are higher order in time, and the quantifiable dissipative SP properties in the non-ideal, resistive case.
        
        We discuss possible options for extending this FET approach to timesteppers for the compressible case.

        The kinetic corrections satisfy linearized Boltzmann equations. Using a Lénard--Bernstein collision operator, these take Fokker--Planck-like forms \cite{Fokker_1914, Planck_1917} wherein pseudo-particles in the numerical model obey the neoclassical transport equations, with particle-independent Brownian drift terms. This offers a rigorous methodology for incorporating collisions into the particle transport model, without coupling the equations of motions for each particle.
        
        Works by Chen, Chacón et al. \cite{Chen_Chacón_Barnes_2011, Chacón_Chen_Barnes_2013, Chen_Chacón_2014, Chen_Chacón_2015} have developed structure-preserving particle pushers for neoclassical transport in the Vlasov equations, derived from Crank--Nicolson integrators. We show these too can can derive from a FET interpretation, similarly offering potential extensions to higher-order-in-time particle pushers. The FET formulation is used also to consider how the stochastic drift terms can be incorporated into the pushers. Stochastic gyrokinetic expansions are also discussed.

        Different options for the numerical implementation of these schemes are considered.

        Due to the efficacy of FET in the development of SP timesteppers for both the fluid and kinetic component, we hope this approach will prove effective in the future for developing SP timesteppers for the full hybrid model. We hope this will give us the opportunity to incorporate previously inaccessible kinetic effects into the highly effective, modern, finite-element MHD models.
    \end{abstract}
    
    
    \newpage
    \tableofcontents
    
    
    \newpage
    \pagenumbering{arabic}
    %\linenumbers\renewcommand\thelinenumber{\color{black!50}\arabic{linenumber}}
            \input{0 - introduction/main.tex}
        \part{Research}
            \input{1 - low-noise PiC models/main.tex}
            \input{2 - kinetic component/main.tex}
            \input{3 - fluid component/main.tex}
            \input{4 - numerical implementation/main.tex}
        \part{Project Overview}
            \input{5 - research plan/main.tex}
            \input{6 - summary/main.tex}
    
    
    %\section{}
    \newpage
    \pagenumbering{gobble}
        \printbibliography


    \newpage
    \pagenumbering{roman}
    \appendix
        \part{Appendices}
            \input{8 - Hilbert complexes/main.tex}
            \input{9 - weak conservation proofs/main.tex}
\end{document}

\end{document}

            \documentclass[12pt, a4paper]{report}

\documentclass[12pt, a4paper]{report}

\input{template/main.tex}

\title{\BA{Title in Progress...}}
\author{Boris Andrews}
\affil{Mathematical Institute, University of Oxford}
\date{\today}


\begin{document}
    \pagenumbering{gobble}
    \maketitle
    
    
    \begin{abstract}
        Magnetic confinement reactors---in particular tokamaks---offer one of the most promising options for achieving practical nuclear fusion, with the potential to provide virtually limitless, clean energy. The theoretical and numerical modeling of tokamak plasmas is simultaneously an essential component of effective reactor design, and a great research barrier. Tokamak operational conditions exhibit comparatively low Knudsen numbers. Kinetic effects, including kinetic waves and instabilities, Landau damping, bump-on-tail instabilities and more, are therefore highly influential in tokamak plasma dynamics. Purely fluid models are inherently incapable of capturing these effects, whereas the high dimensionality in purely kinetic models render them practically intractable for most relevant purposes.

        We consider a $\delta\!f$ decomposition model, with a macroscopic fluid background and microscopic kinetic correction, both fully coupled to each other. A similar manner of discretization is proposed to that used in the recent \texttt{STRUPHY} code \cite{Holderied_Possanner_Wang_2021, Holderied_2022, Li_et_al_2023} with a finite-element model for the background and a pseudo-particle/PiC model for the correction.

        The fluid background satisfies the full, non-linear, resistive, compressible, Hall MHD equations. \cite{Laakmann_Hu_Farrell_2022} introduces finite-element(-in-space) implicit timesteppers for the incompressible analogue to this system with structure-preserving (SP) properties in the ideal case, alongside parameter-robust preconditioners. We show that these timesteppers can derive from a finite-element-in-time (FET) (and finite-element-in-space) interpretation. The benefits of this reformulation are discussed, including the derivation of timesteppers that are higher order in time, and the quantifiable dissipative SP properties in the non-ideal, resistive case.
        
        We discuss possible options for extending this FET approach to timesteppers for the compressible case.

        The kinetic corrections satisfy linearized Boltzmann equations. Using a Lénard--Bernstein collision operator, these take Fokker--Planck-like forms \cite{Fokker_1914, Planck_1917} wherein pseudo-particles in the numerical model obey the neoclassical transport equations, with particle-independent Brownian drift terms. This offers a rigorous methodology for incorporating collisions into the particle transport model, without coupling the equations of motions for each particle.
        
        Works by Chen, Chacón et al. \cite{Chen_Chacón_Barnes_2011, Chacón_Chen_Barnes_2013, Chen_Chacón_2014, Chen_Chacón_2015} have developed structure-preserving particle pushers for neoclassical transport in the Vlasov equations, derived from Crank--Nicolson integrators. We show these too can can derive from a FET interpretation, similarly offering potential extensions to higher-order-in-time particle pushers. The FET formulation is used also to consider how the stochastic drift terms can be incorporated into the pushers. Stochastic gyrokinetic expansions are also discussed.

        Different options for the numerical implementation of these schemes are considered.

        Due to the efficacy of FET in the development of SP timesteppers for both the fluid and kinetic component, we hope this approach will prove effective in the future for developing SP timesteppers for the full hybrid model. We hope this will give us the opportunity to incorporate previously inaccessible kinetic effects into the highly effective, modern, finite-element MHD models.
    \end{abstract}
    
    
    \newpage
    \tableofcontents
    
    
    \newpage
    \pagenumbering{arabic}
    %\linenumbers\renewcommand\thelinenumber{\color{black!50}\arabic{linenumber}}
            \input{0 - introduction/main.tex}
        \part{Research}
            \input{1 - low-noise PiC models/main.tex}
            \input{2 - kinetic component/main.tex}
            \input{3 - fluid component/main.tex}
            \input{4 - numerical implementation/main.tex}
        \part{Project Overview}
            \input{5 - research plan/main.tex}
            \input{6 - summary/main.tex}
    
    
    %\section{}
    \newpage
    \pagenumbering{gobble}
        \printbibliography


    \newpage
    \pagenumbering{roman}
    \appendix
        \part{Appendices}
            \input{8 - Hilbert complexes/main.tex}
            \input{9 - weak conservation proofs/main.tex}
\end{document}


\title{\BA{Title in Progress...}}
\author{Boris Andrews}
\affil{Mathematical Institute, University of Oxford}
\date{\today}


\begin{document}
    \pagenumbering{gobble}
    \maketitle
    
    
    \begin{abstract}
        Magnetic confinement reactors---in particular tokamaks---offer one of the most promising options for achieving practical nuclear fusion, with the potential to provide virtually limitless, clean energy. The theoretical and numerical modeling of tokamak plasmas is simultaneously an essential component of effective reactor design, and a great research barrier. Tokamak operational conditions exhibit comparatively low Knudsen numbers. Kinetic effects, including kinetic waves and instabilities, Landau damping, bump-on-tail instabilities and more, are therefore highly influential in tokamak plasma dynamics. Purely fluid models are inherently incapable of capturing these effects, whereas the high dimensionality in purely kinetic models render them practically intractable for most relevant purposes.

        We consider a $\delta\!f$ decomposition model, with a macroscopic fluid background and microscopic kinetic correction, both fully coupled to each other. A similar manner of discretization is proposed to that used in the recent \texttt{STRUPHY} code \cite{Holderied_Possanner_Wang_2021, Holderied_2022, Li_et_al_2023} with a finite-element model for the background and a pseudo-particle/PiC model for the correction.

        The fluid background satisfies the full, non-linear, resistive, compressible, Hall MHD equations. \cite{Laakmann_Hu_Farrell_2022} introduces finite-element(-in-space) implicit timesteppers for the incompressible analogue to this system with structure-preserving (SP) properties in the ideal case, alongside parameter-robust preconditioners. We show that these timesteppers can derive from a finite-element-in-time (FET) (and finite-element-in-space) interpretation. The benefits of this reformulation are discussed, including the derivation of timesteppers that are higher order in time, and the quantifiable dissipative SP properties in the non-ideal, resistive case.
        
        We discuss possible options for extending this FET approach to timesteppers for the compressible case.

        The kinetic corrections satisfy linearized Boltzmann equations. Using a Lénard--Bernstein collision operator, these take Fokker--Planck-like forms \cite{Fokker_1914, Planck_1917} wherein pseudo-particles in the numerical model obey the neoclassical transport equations, with particle-independent Brownian drift terms. This offers a rigorous methodology for incorporating collisions into the particle transport model, without coupling the equations of motions for each particle.
        
        Works by Chen, Chacón et al. \cite{Chen_Chacón_Barnes_2011, Chacón_Chen_Barnes_2013, Chen_Chacón_2014, Chen_Chacón_2015} have developed structure-preserving particle pushers for neoclassical transport in the Vlasov equations, derived from Crank--Nicolson integrators. We show these too can can derive from a FET interpretation, similarly offering potential extensions to higher-order-in-time particle pushers. The FET formulation is used also to consider how the stochastic drift terms can be incorporated into the pushers. Stochastic gyrokinetic expansions are also discussed.

        Different options for the numerical implementation of these schemes are considered.

        Due to the efficacy of FET in the development of SP timesteppers for both the fluid and kinetic component, we hope this approach will prove effective in the future for developing SP timesteppers for the full hybrid model. We hope this will give us the opportunity to incorporate previously inaccessible kinetic effects into the highly effective, modern, finite-element MHD models.
    \end{abstract}
    
    
    \newpage
    \tableofcontents
    
    
    \newpage
    \pagenumbering{arabic}
    %\linenumbers\renewcommand\thelinenumber{\color{black!50}\arabic{linenumber}}
            \documentclass[12pt, a4paper]{report}

\input{template/main.tex}

\title{\BA{Title in Progress...}}
\author{Boris Andrews}
\affil{Mathematical Institute, University of Oxford}
\date{\today}


\begin{document}
    \pagenumbering{gobble}
    \maketitle
    
    
    \begin{abstract}
        Magnetic confinement reactors---in particular tokamaks---offer one of the most promising options for achieving practical nuclear fusion, with the potential to provide virtually limitless, clean energy. The theoretical and numerical modeling of tokamak plasmas is simultaneously an essential component of effective reactor design, and a great research barrier. Tokamak operational conditions exhibit comparatively low Knudsen numbers. Kinetic effects, including kinetic waves and instabilities, Landau damping, bump-on-tail instabilities and more, are therefore highly influential in tokamak plasma dynamics. Purely fluid models are inherently incapable of capturing these effects, whereas the high dimensionality in purely kinetic models render them practically intractable for most relevant purposes.

        We consider a $\delta\!f$ decomposition model, with a macroscopic fluid background and microscopic kinetic correction, both fully coupled to each other. A similar manner of discretization is proposed to that used in the recent \texttt{STRUPHY} code \cite{Holderied_Possanner_Wang_2021, Holderied_2022, Li_et_al_2023} with a finite-element model for the background and a pseudo-particle/PiC model for the correction.

        The fluid background satisfies the full, non-linear, resistive, compressible, Hall MHD equations. \cite{Laakmann_Hu_Farrell_2022} introduces finite-element(-in-space) implicit timesteppers for the incompressible analogue to this system with structure-preserving (SP) properties in the ideal case, alongside parameter-robust preconditioners. We show that these timesteppers can derive from a finite-element-in-time (FET) (and finite-element-in-space) interpretation. The benefits of this reformulation are discussed, including the derivation of timesteppers that are higher order in time, and the quantifiable dissipative SP properties in the non-ideal, resistive case.
        
        We discuss possible options for extending this FET approach to timesteppers for the compressible case.

        The kinetic corrections satisfy linearized Boltzmann equations. Using a Lénard--Bernstein collision operator, these take Fokker--Planck-like forms \cite{Fokker_1914, Planck_1917} wherein pseudo-particles in the numerical model obey the neoclassical transport equations, with particle-independent Brownian drift terms. This offers a rigorous methodology for incorporating collisions into the particle transport model, without coupling the equations of motions for each particle.
        
        Works by Chen, Chacón et al. \cite{Chen_Chacón_Barnes_2011, Chacón_Chen_Barnes_2013, Chen_Chacón_2014, Chen_Chacón_2015} have developed structure-preserving particle pushers for neoclassical transport in the Vlasov equations, derived from Crank--Nicolson integrators. We show these too can can derive from a FET interpretation, similarly offering potential extensions to higher-order-in-time particle pushers. The FET formulation is used also to consider how the stochastic drift terms can be incorporated into the pushers. Stochastic gyrokinetic expansions are also discussed.

        Different options for the numerical implementation of these schemes are considered.

        Due to the efficacy of FET in the development of SP timesteppers for both the fluid and kinetic component, we hope this approach will prove effective in the future for developing SP timesteppers for the full hybrid model. We hope this will give us the opportunity to incorporate previously inaccessible kinetic effects into the highly effective, modern, finite-element MHD models.
    \end{abstract}
    
    
    \newpage
    \tableofcontents
    
    
    \newpage
    \pagenumbering{arabic}
    %\linenumbers\renewcommand\thelinenumber{\color{black!50}\arabic{linenumber}}
            \input{0 - introduction/main.tex}
        \part{Research}
            \input{1 - low-noise PiC models/main.tex}
            \input{2 - kinetic component/main.tex}
            \input{3 - fluid component/main.tex}
            \input{4 - numerical implementation/main.tex}
        \part{Project Overview}
            \input{5 - research plan/main.tex}
            \input{6 - summary/main.tex}
    
    
    %\section{}
    \newpage
    \pagenumbering{gobble}
        \printbibliography


    \newpage
    \pagenumbering{roman}
    \appendix
        \part{Appendices}
            \input{8 - Hilbert complexes/main.tex}
            \input{9 - weak conservation proofs/main.tex}
\end{document}

        \part{Research}
            \documentclass[12pt, a4paper]{report}

\input{template/main.tex}

\title{\BA{Title in Progress...}}
\author{Boris Andrews}
\affil{Mathematical Institute, University of Oxford}
\date{\today}


\begin{document}
    \pagenumbering{gobble}
    \maketitle
    
    
    \begin{abstract}
        Magnetic confinement reactors---in particular tokamaks---offer one of the most promising options for achieving practical nuclear fusion, with the potential to provide virtually limitless, clean energy. The theoretical and numerical modeling of tokamak plasmas is simultaneously an essential component of effective reactor design, and a great research barrier. Tokamak operational conditions exhibit comparatively low Knudsen numbers. Kinetic effects, including kinetic waves and instabilities, Landau damping, bump-on-tail instabilities and more, are therefore highly influential in tokamak plasma dynamics. Purely fluid models are inherently incapable of capturing these effects, whereas the high dimensionality in purely kinetic models render them practically intractable for most relevant purposes.

        We consider a $\delta\!f$ decomposition model, with a macroscopic fluid background and microscopic kinetic correction, both fully coupled to each other. A similar manner of discretization is proposed to that used in the recent \texttt{STRUPHY} code \cite{Holderied_Possanner_Wang_2021, Holderied_2022, Li_et_al_2023} with a finite-element model for the background and a pseudo-particle/PiC model for the correction.

        The fluid background satisfies the full, non-linear, resistive, compressible, Hall MHD equations. \cite{Laakmann_Hu_Farrell_2022} introduces finite-element(-in-space) implicit timesteppers for the incompressible analogue to this system with structure-preserving (SP) properties in the ideal case, alongside parameter-robust preconditioners. We show that these timesteppers can derive from a finite-element-in-time (FET) (and finite-element-in-space) interpretation. The benefits of this reformulation are discussed, including the derivation of timesteppers that are higher order in time, and the quantifiable dissipative SP properties in the non-ideal, resistive case.
        
        We discuss possible options for extending this FET approach to timesteppers for the compressible case.

        The kinetic corrections satisfy linearized Boltzmann equations. Using a Lénard--Bernstein collision operator, these take Fokker--Planck-like forms \cite{Fokker_1914, Planck_1917} wherein pseudo-particles in the numerical model obey the neoclassical transport equations, with particle-independent Brownian drift terms. This offers a rigorous methodology for incorporating collisions into the particle transport model, without coupling the equations of motions for each particle.
        
        Works by Chen, Chacón et al. \cite{Chen_Chacón_Barnes_2011, Chacón_Chen_Barnes_2013, Chen_Chacón_2014, Chen_Chacón_2015} have developed structure-preserving particle pushers for neoclassical transport in the Vlasov equations, derived from Crank--Nicolson integrators. We show these too can can derive from a FET interpretation, similarly offering potential extensions to higher-order-in-time particle pushers. The FET formulation is used also to consider how the stochastic drift terms can be incorporated into the pushers. Stochastic gyrokinetic expansions are also discussed.

        Different options for the numerical implementation of these schemes are considered.

        Due to the efficacy of FET in the development of SP timesteppers for both the fluid and kinetic component, we hope this approach will prove effective in the future for developing SP timesteppers for the full hybrid model. We hope this will give us the opportunity to incorporate previously inaccessible kinetic effects into the highly effective, modern, finite-element MHD models.
    \end{abstract}
    
    
    \newpage
    \tableofcontents
    
    
    \newpage
    \pagenumbering{arabic}
    %\linenumbers\renewcommand\thelinenumber{\color{black!50}\arabic{linenumber}}
            \input{0 - introduction/main.tex}
        \part{Research}
            \input{1 - low-noise PiC models/main.tex}
            \input{2 - kinetic component/main.tex}
            \input{3 - fluid component/main.tex}
            \input{4 - numerical implementation/main.tex}
        \part{Project Overview}
            \input{5 - research plan/main.tex}
            \input{6 - summary/main.tex}
    
    
    %\section{}
    \newpage
    \pagenumbering{gobble}
        \printbibliography


    \newpage
    \pagenumbering{roman}
    \appendix
        \part{Appendices}
            \input{8 - Hilbert complexes/main.tex}
            \input{9 - weak conservation proofs/main.tex}
\end{document}

            \documentclass[12pt, a4paper]{report}

\input{template/main.tex}

\title{\BA{Title in Progress...}}
\author{Boris Andrews}
\affil{Mathematical Institute, University of Oxford}
\date{\today}


\begin{document}
    \pagenumbering{gobble}
    \maketitle
    
    
    \begin{abstract}
        Magnetic confinement reactors---in particular tokamaks---offer one of the most promising options for achieving practical nuclear fusion, with the potential to provide virtually limitless, clean energy. The theoretical and numerical modeling of tokamak plasmas is simultaneously an essential component of effective reactor design, and a great research barrier. Tokamak operational conditions exhibit comparatively low Knudsen numbers. Kinetic effects, including kinetic waves and instabilities, Landau damping, bump-on-tail instabilities and more, are therefore highly influential in tokamak plasma dynamics. Purely fluid models are inherently incapable of capturing these effects, whereas the high dimensionality in purely kinetic models render them practically intractable for most relevant purposes.

        We consider a $\delta\!f$ decomposition model, with a macroscopic fluid background and microscopic kinetic correction, both fully coupled to each other. A similar manner of discretization is proposed to that used in the recent \texttt{STRUPHY} code \cite{Holderied_Possanner_Wang_2021, Holderied_2022, Li_et_al_2023} with a finite-element model for the background and a pseudo-particle/PiC model for the correction.

        The fluid background satisfies the full, non-linear, resistive, compressible, Hall MHD equations. \cite{Laakmann_Hu_Farrell_2022} introduces finite-element(-in-space) implicit timesteppers for the incompressible analogue to this system with structure-preserving (SP) properties in the ideal case, alongside parameter-robust preconditioners. We show that these timesteppers can derive from a finite-element-in-time (FET) (and finite-element-in-space) interpretation. The benefits of this reformulation are discussed, including the derivation of timesteppers that are higher order in time, and the quantifiable dissipative SP properties in the non-ideal, resistive case.
        
        We discuss possible options for extending this FET approach to timesteppers for the compressible case.

        The kinetic corrections satisfy linearized Boltzmann equations. Using a Lénard--Bernstein collision operator, these take Fokker--Planck-like forms \cite{Fokker_1914, Planck_1917} wherein pseudo-particles in the numerical model obey the neoclassical transport equations, with particle-independent Brownian drift terms. This offers a rigorous methodology for incorporating collisions into the particle transport model, without coupling the equations of motions for each particle.
        
        Works by Chen, Chacón et al. \cite{Chen_Chacón_Barnes_2011, Chacón_Chen_Barnes_2013, Chen_Chacón_2014, Chen_Chacón_2015} have developed structure-preserving particle pushers for neoclassical transport in the Vlasov equations, derived from Crank--Nicolson integrators. We show these too can can derive from a FET interpretation, similarly offering potential extensions to higher-order-in-time particle pushers. The FET formulation is used also to consider how the stochastic drift terms can be incorporated into the pushers. Stochastic gyrokinetic expansions are also discussed.

        Different options for the numerical implementation of these schemes are considered.

        Due to the efficacy of FET in the development of SP timesteppers for both the fluid and kinetic component, we hope this approach will prove effective in the future for developing SP timesteppers for the full hybrid model. We hope this will give us the opportunity to incorporate previously inaccessible kinetic effects into the highly effective, modern, finite-element MHD models.
    \end{abstract}
    
    
    \newpage
    \tableofcontents
    
    
    \newpage
    \pagenumbering{arabic}
    %\linenumbers\renewcommand\thelinenumber{\color{black!50}\arabic{linenumber}}
            \input{0 - introduction/main.tex}
        \part{Research}
            \input{1 - low-noise PiC models/main.tex}
            \input{2 - kinetic component/main.tex}
            \input{3 - fluid component/main.tex}
            \input{4 - numerical implementation/main.tex}
        \part{Project Overview}
            \input{5 - research plan/main.tex}
            \input{6 - summary/main.tex}
    
    
    %\section{}
    \newpage
    \pagenumbering{gobble}
        \printbibliography


    \newpage
    \pagenumbering{roman}
    \appendix
        \part{Appendices}
            \input{8 - Hilbert complexes/main.tex}
            \input{9 - weak conservation proofs/main.tex}
\end{document}

            \documentclass[12pt, a4paper]{report}

\input{template/main.tex}

\title{\BA{Title in Progress...}}
\author{Boris Andrews}
\affil{Mathematical Institute, University of Oxford}
\date{\today}


\begin{document}
    \pagenumbering{gobble}
    \maketitle
    
    
    \begin{abstract}
        Magnetic confinement reactors---in particular tokamaks---offer one of the most promising options for achieving practical nuclear fusion, with the potential to provide virtually limitless, clean energy. The theoretical and numerical modeling of tokamak plasmas is simultaneously an essential component of effective reactor design, and a great research barrier. Tokamak operational conditions exhibit comparatively low Knudsen numbers. Kinetic effects, including kinetic waves and instabilities, Landau damping, bump-on-tail instabilities and more, are therefore highly influential in tokamak plasma dynamics. Purely fluid models are inherently incapable of capturing these effects, whereas the high dimensionality in purely kinetic models render them practically intractable for most relevant purposes.

        We consider a $\delta\!f$ decomposition model, with a macroscopic fluid background and microscopic kinetic correction, both fully coupled to each other. A similar manner of discretization is proposed to that used in the recent \texttt{STRUPHY} code \cite{Holderied_Possanner_Wang_2021, Holderied_2022, Li_et_al_2023} with a finite-element model for the background and a pseudo-particle/PiC model for the correction.

        The fluid background satisfies the full, non-linear, resistive, compressible, Hall MHD equations. \cite{Laakmann_Hu_Farrell_2022} introduces finite-element(-in-space) implicit timesteppers for the incompressible analogue to this system with structure-preserving (SP) properties in the ideal case, alongside parameter-robust preconditioners. We show that these timesteppers can derive from a finite-element-in-time (FET) (and finite-element-in-space) interpretation. The benefits of this reformulation are discussed, including the derivation of timesteppers that are higher order in time, and the quantifiable dissipative SP properties in the non-ideal, resistive case.
        
        We discuss possible options for extending this FET approach to timesteppers for the compressible case.

        The kinetic corrections satisfy linearized Boltzmann equations. Using a Lénard--Bernstein collision operator, these take Fokker--Planck-like forms \cite{Fokker_1914, Planck_1917} wherein pseudo-particles in the numerical model obey the neoclassical transport equations, with particle-independent Brownian drift terms. This offers a rigorous methodology for incorporating collisions into the particle transport model, without coupling the equations of motions for each particle.
        
        Works by Chen, Chacón et al. \cite{Chen_Chacón_Barnes_2011, Chacón_Chen_Barnes_2013, Chen_Chacón_2014, Chen_Chacón_2015} have developed structure-preserving particle pushers for neoclassical transport in the Vlasov equations, derived from Crank--Nicolson integrators. We show these too can can derive from a FET interpretation, similarly offering potential extensions to higher-order-in-time particle pushers. The FET formulation is used also to consider how the stochastic drift terms can be incorporated into the pushers. Stochastic gyrokinetic expansions are also discussed.

        Different options for the numerical implementation of these schemes are considered.

        Due to the efficacy of FET in the development of SP timesteppers for both the fluid and kinetic component, we hope this approach will prove effective in the future for developing SP timesteppers for the full hybrid model. We hope this will give us the opportunity to incorporate previously inaccessible kinetic effects into the highly effective, modern, finite-element MHD models.
    \end{abstract}
    
    
    \newpage
    \tableofcontents
    
    
    \newpage
    \pagenumbering{arabic}
    %\linenumbers\renewcommand\thelinenumber{\color{black!50}\arabic{linenumber}}
            \input{0 - introduction/main.tex}
        \part{Research}
            \input{1 - low-noise PiC models/main.tex}
            \input{2 - kinetic component/main.tex}
            \input{3 - fluid component/main.tex}
            \input{4 - numerical implementation/main.tex}
        \part{Project Overview}
            \input{5 - research plan/main.tex}
            \input{6 - summary/main.tex}
    
    
    %\section{}
    \newpage
    \pagenumbering{gobble}
        \printbibliography


    \newpage
    \pagenumbering{roman}
    \appendix
        \part{Appendices}
            \input{8 - Hilbert complexes/main.tex}
            \input{9 - weak conservation proofs/main.tex}
\end{document}

            \documentclass[12pt, a4paper]{report}

\input{template/main.tex}

\title{\BA{Title in Progress...}}
\author{Boris Andrews}
\affil{Mathematical Institute, University of Oxford}
\date{\today}


\begin{document}
    \pagenumbering{gobble}
    \maketitle
    
    
    \begin{abstract}
        Magnetic confinement reactors---in particular tokamaks---offer one of the most promising options for achieving practical nuclear fusion, with the potential to provide virtually limitless, clean energy. The theoretical and numerical modeling of tokamak plasmas is simultaneously an essential component of effective reactor design, and a great research barrier. Tokamak operational conditions exhibit comparatively low Knudsen numbers. Kinetic effects, including kinetic waves and instabilities, Landau damping, bump-on-tail instabilities and more, are therefore highly influential in tokamak plasma dynamics. Purely fluid models are inherently incapable of capturing these effects, whereas the high dimensionality in purely kinetic models render them practically intractable for most relevant purposes.

        We consider a $\delta\!f$ decomposition model, with a macroscopic fluid background and microscopic kinetic correction, both fully coupled to each other. A similar manner of discretization is proposed to that used in the recent \texttt{STRUPHY} code \cite{Holderied_Possanner_Wang_2021, Holderied_2022, Li_et_al_2023} with a finite-element model for the background and a pseudo-particle/PiC model for the correction.

        The fluid background satisfies the full, non-linear, resistive, compressible, Hall MHD equations. \cite{Laakmann_Hu_Farrell_2022} introduces finite-element(-in-space) implicit timesteppers for the incompressible analogue to this system with structure-preserving (SP) properties in the ideal case, alongside parameter-robust preconditioners. We show that these timesteppers can derive from a finite-element-in-time (FET) (and finite-element-in-space) interpretation. The benefits of this reformulation are discussed, including the derivation of timesteppers that are higher order in time, and the quantifiable dissipative SP properties in the non-ideal, resistive case.
        
        We discuss possible options for extending this FET approach to timesteppers for the compressible case.

        The kinetic corrections satisfy linearized Boltzmann equations. Using a Lénard--Bernstein collision operator, these take Fokker--Planck-like forms \cite{Fokker_1914, Planck_1917} wherein pseudo-particles in the numerical model obey the neoclassical transport equations, with particle-independent Brownian drift terms. This offers a rigorous methodology for incorporating collisions into the particle transport model, without coupling the equations of motions for each particle.
        
        Works by Chen, Chacón et al. \cite{Chen_Chacón_Barnes_2011, Chacón_Chen_Barnes_2013, Chen_Chacón_2014, Chen_Chacón_2015} have developed structure-preserving particle pushers for neoclassical transport in the Vlasov equations, derived from Crank--Nicolson integrators. We show these too can can derive from a FET interpretation, similarly offering potential extensions to higher-order-in-time particle pushers. The FET formulation is used also to consider how the stochastic drift terms can be incorporated into the pushers. Stochastic gyrokinetic expansions are also discussed.

        Different options for the numerical implementation of these schemes are considered.

        Due to the efficacy of FET in the development of SP timesteppers for both the fluid and kinetic component, we hope this approach will prove effective in the future for developing SP timesteppers for the full hybrid model. We hope this will give us the opportunity to incorporate previously inaccessible kinetic effects into the highly effective, modern, finite-element MHD models.
    \end{abstract}
    
    
    \newpage
    \tableofcontents
    
    
    \newpage
    \pagenumbering{arabic}
    %\linenumbers\renewcommand\thelinenumber{\color{black!50}\arabic{linenumber}}
            \input{0 - introduction/main.tex}
        \part{Research}
            \input{1 - low-noise PiC models/main.tex}
            \input{2 - kinetic component/main.tex}
            \input{3 - fluid component/main.tex}
            \input{4 - numerical implementation/main.tex}
        \part{Project Overview}
            \input{5 - research plan/main.tex}
            \input{6 - summary/main.tex}
    
    
    %\section{}
    \newpage
    \pagenumbering{gobble}
        \printbibliography


    \newpage
    \pagenumbering{roman}
    \appendix
        \part{Appendices}
            \input{8 - Hilbert complexes/main.tex}
            \input{9 - weak conservation proofs/main.tex}
\end{document}

        \part{Project Overview}
            \documentclass[12pt, a4paper]{report}

\input{template/main.tex}

\title{\BA{Title in Progress...}}
\author{Boris Andrews}
\affil{Mathematical Institute, University of Oxford}
\date{\today}


\begin{document}
    \pagenumbering{gobble}
    \maketitle
    
    
    \begin{abstract}
        Magnetic confinement reactors---in particular tokamaks---offer one of the most promising options for achieving practical nuclear fusion, with the potential to provide virtually limitless, clean energy. The theoretical and numerical modeling of tokamak plasmas is simultaneously an essential component of effective reactor design, and a great research barrier. Tokamak operational conditions exhibit comparatively low Knudsen numbers. Kinetic effects, including kinetic waves and instabilities, Landau damping, bump-on-tail instabilities and more, are therefore highly influential in tokamak plasma dynamics. Purely fluid models are inherently incapable of capturing these effects, whereas the high dimensionality in purely kinetic models render them practically intractable for most relevant purposes.

        We consider a $\delta\!f$ decomposition model, with a macroscopic fluid background and microscopic kinetic correction, both fully coupled to each other. A similar manner of discretization is proposed to that used in the recent \texttt{STRUPHY} code \cite{Holderied_Possanner_Wang_2021, Holderied_2022, Li_et_al_2023} with a finite-element model for the background and a pseudo-particle/PiC model for the correction.

        The fluid background satisfies the full, non-linear, resistive, compressible, Hall MHD equations. \cite{Laakmann_Hu_Farrell_2022} introduces finite-element(-in-space) implicit timesteppers for the incompressible analogue to this system with structure-preserving (SP) properties in the ideal case, alongside parameter-robust preconditioners. We show that these timesteppers can derive from a finite-element-in-time (FET) (and finite-element-in-space) interpretation. The benefits of this reformulation are discussed, including the derivation of timesteppers that are higher order in time, and the quantifiable dissipative SP properties in the non-ideal, resistive case.
        
        We discuss possible options for extending this FET approach to timesteppers for the compressible case.

        The kinetic corrections satisfy linearized Boltzmann equations. Using a Lénard--Bernstein collision operator, these take Fokker--Planck-like forms \cite{Fokker_1914, Planck_1917} wherein pseudo-particles in the numerical model obey the neoclassical transport equations, with particle-independent Brownian drift terms. This offers a rigorous methodology for incorporating collisions into the particle transport model, without coupling the equations of motions for each particle.
        
        Works by Chen, Chacón et al. \cite{Chen_Chacón_Barnes_2011, Chacón_Chen_Barnes_2013, Chen_Chacón_2014, Chen_Chacón_2015} have developed structure-preserving particle pushers for neoclassical transport in the Vlasov equations, derived from Crank--Nicolson integrators. We show these too can can derive from a FET interpretation, similarly offering potential extensions to higher-order-in-time particle pushers. The FET formulation is used also to consider how the stochastic drift terms can be incorporated into the pushers. Stochastic gyrokinetic expansions are also discussed.

        Different options for the numerical implementation of these schemes are considered.

        Due to the efficacy of FET in the development of SP timesteppers for both the fluid and kinetic component, we hope this approach will prove effective in the future for developing SP timesteppers for the full hybrid model. We hope this will give us the opportunity to incorporate previously inaccessible kinetic effects into the highly effective, modern, finite-element MHD models.
    \end{abstract}
    
    
    \newpage
    \tableofcontents
    
    
    \newpage
    \pagenumbering{arabic}
    %\linenumbers\renewcommand\thelinenumber{\color{black!50}\arabic{linenumber}}
            \input{0 - introduction/main.tex}
        \part{Research}
            \input{1 - low-noise PiC models/main.tex}
            \input{2 - kinetic component/main.tex}
            \input{3 - fluid component/main.tex}
            \input{4 - numerical implementation/main.tex}
        \part{Project Overview}
            \input{5 - research plan/main.tex}
            \input{6 - summary/main.tex}
    
    
    %\section{}
    \newpage
    \pagenumbering{gobble}
        \printbibliography


    \newpage
    \pagenumbering{roman}
    \appendix
        \part{Appendices}
            \input{8 - Hilbert complexes/main.tex}
            \input{9 - weak conservation proofs/main.tex}
\end{document}

            \documentclass[12pt, a4paper]{report}

\input{template/main.tex}

\title{\BA{Title in Progress...}}
\author{Boris Andrews}
\affil{Mathematical Institute, University of Oxford}
\date{\today}


\begin{document}
    \pagenumbering{gobble}
    \maketitle
    
    
    \begin{abstract}
        Magnetic confinement reactors---in particular tokamaks---offer one of the most promising options for achieving practical nuclear fusion, with the potential to provide virtually limitless, clean energy. The theoretical and numerical modeling of tokamak plasmas is simultaneously an essential component of effective reactor design, and a great research barrier. Tokamak operational conditions exhibit comparatively low Knudsen numbers. Kinetic effects, including kinetic waves and instabilities, Landau damping, bump-on-tail instabilities and more, are therefore highly influential in tokamak plasma dynamics. Purely fluid models are inherently incapable of capturing these effects, whereas the high dimensionality in purely kinetic models render them practically intractable for most relevant purposes.

        We consider a $\delta\!f$ decomposition model, with a macroscopic fluid background and microscopic kinetic correction, both fully coupled to each other. A similar manner of discretization is proposed to that used in the recent \texttt{STRUPHY} code \cite{Holderied_Possanner_Wang_2021, Holderied_2022, Li_et_al_2023} with a finite-element model for the background and a pseudo-particle/PiC model for the correction.

        The fluid background satisfies the full, non-linear, resistive, compressible, Hall MHD equations. \cite{Laakmann_Hu_Farrell_2022} introduces finite-element(-in-space) implicit timesteppers for the incompressible analogue to this system with structure-preserving (SP) properties in the ideal case, alongside parameter-robust preconditioners. We show that these timesteppers can derive from a finite-element-in-time (FET) (and finite-element-in-space) interpretation. The benefits of this reformulation are discussed, including the derivation of timesteppers that are higher order in time, and the quantifiable dissipative SP properties in the non-ideal, resistive case.
        
        We discuss possible options for extending this FET approach to timesteppers for the compressible case.

        The kinetic corrections satisfy linearized Boltzmann equations. Using a Lénard--Bernstein collision operator, these take Fokker--Planck-like forms \cite{Fokker_1914, Planck_1917} wherein pseudo-particles in the numerical model obey the neoclassical transport equations, with particle-independent Brownian drift terms. This offers a rigorous methodology for incorporating collisions into the particle transport model, without coupling the equations of motions for each particle.
        
        Works by Chen, Chacón et al. \cite{Chen_Chacón_Barnes_2011, Chacón_Chen_Barnes_2013, Chen_Chacón_2014, Chen_Chacón_2015} have developed structure-preserving particle pushers for neoclassical transport in the Vlasov equations, derived from Crank--Nicolson integrators. We show these too can can derive from a FET interpretation, similarly offering potential extensions to higher-order-in-time particle pushers. The FET formulation is used also to consider how the stochastic drift terms can be incorporated into the pushers. Stochastic gyrokinetic expansions are also discussed.

        Different options for the numerical implementation of these schemes are considered.

        Due to the efficacy of FET in the development of SP timesteppers for both the fluid and kinetic component, we hope this approach will prove effective in the future for developing SP timesteppers for the full hybrid model. We hope this will give us the opportunity to incorporate previously inaccessible kinetic effects into the highly effective, modern, finite-element MHD models.
    \end{abstract}
    
    
    \newpage
    \tableofcontents
    
    
    \newpage
    \pagenumbering{arabic}
    %\linenumbers\renewcommand\thelinenumber{\color{black!50}\arabic{linenumber}}
            \input{0 - introduction/main.tex}
        \part{Research}
            \input{1 - low-noise PiC models/main.tex}
            \input{2 - kinetic component/main.tex}
            \input{3 - fluid component/main.tex}
            \input{4 - numerical implementation/main.tex}
        \part{Project Overview}
            \input{5 - research plan/main.tex}
            \input{6 - summary/main.tex}
    
    
    %\section{}
    \newpage
    \pagenumbering{gobble}
        \printbibliography


    \newpage
    \pagenumbering{roman}
    \appendix
        \part{Appendices}
            \input{8 - Hilbert complexes/main.tex}
            \input{9 - weak conservation proofs/main.tex}
\end{document}

    
    
    %\section{}
    \newpage
    \pagenumbering{gobble}
        \printbibliography


    \newpage
    \pagenumbering{roman}
    \appendix
        \part{Appendices}
            \documentclass[12pt, a4paper]{report}

\input{template/main.tex}

\title{\BA{Title in Progress...}}
\author{Boris Andrews}
\affil{Mathematical Institute, University of Oxford}
\date{\today}


\begin{document}
    \pagenumbering{gobble}
    \maketitle
    
    
    \begin{abstract}
        Magnetic confinement reactors---in particular tokamaks---offer one of the most promising options for achieving practical nuclear fusion, with the potential to provide virtually limitless, clean energy. The theoretical and numerical modeling of tokamak plasmas is simultaneously an essential component of effective reactor design, and a great research barrier. Tokamak operational conditions exhibit comparatively low Knudsen numbers. Kinetic effects, including kinetic waves and instabilities, Landau damping, bump-on-tail instabilities and more, are therefore highly influential in tokamak plasma dynamics. Purely fluid models are inherently incapable of capturing these effects, whereas the high dimensionality in purely kinetic models render them practically intractable for most relevant purposes.

        We consider a $\delta\!f$ decomposition model, with a macroscopic fluid background and microscopic kinetic correction, both fully coupled to each other. A similar manner of discretization is proposed to that used in the recent \texttt{STRUPHY} code \cite{Holderied_Possanner_Wang_2021, Holderied_2022, Li_et_al_2023} with a finite-element model for the background and a pseudo-particle/PiC model for the correction.

        The fluid background satisfies the full, non-linear, resistive, compressible, Hall MHD equations. \cite{Laakmann_Hu_Farrell_2022} introduces finite-element(-in-space) implicit timesteppers for the incompressible analogue to this system with structure-preserving (SP) properties in the ideal case, alongside parameter-robust preconditioners. We show that these timesteppers can derive from a finite-element-in-time (FET) (and finite-element-in-space) interpretation. The benefits of this reformulation are discussed, including the derivation of timesteppers that are higher order in time, and the quantifiable dissipative SP properties in the non-ideal, resistive case.
        
        We discuss possible options for extending this FET approach to timesteppers for the compressible case.

        The kinetic corrections satisfy linearized Boltzmann equations. Using a Lénard--Bernstein collision operator, these take Fokker--Planck-like forms \cite{Fokker_1914, Planck_1917} wherein pseudo-particles in the numerical model obey the neoclassical transport equations, with particle-independent Brownian drift terms. This offers a rigorous methodology for incorporating collisions into the particle transport model, without coupling the equations of motions for each particle.
        
        Works by Chen, Chacón et al. \cite{Chen_Chacón_Barnes_2011, Chacón_Chen_Barnes_2013, Chen_Chacón_2014, Chen_Chacón_2015} have developed structure-preserving particle pushers for neoclassical transport in the Vlasov equations, derived from Crank--Nicolson integrators. We show these too can can derive from a FET interpretation, similarly offering potential extensions to higher-order-in-time particle pushers. The FET formulation is used also to consider how the stochastic drift terms can be incorporated into the pushers. Stochastic gyrokinetic expansions are also discussed.

        Different options for the numerical implementation of these schemes are considered.

        Due to the efficacy of FET in the development of SP timesteppers for both the fluid and kinetic component, we hope this approach will prove effective in the future for developing SP timesteppers for the full hybrid model. We hope this will give us the opportunity to incorporate previously inaccessible kinetic effects into the highly effective, modern, finite-element MHD models.
    \end{abstract}
    
    
    \newpage
    \tableofcontents
    
    
    \newpage
    \pagenumbering{arabic}
    %\linenumbers\renewcommand\thelinenumber{\color{black!50}\arabic{linenumber}}
            \input{0 - introduction/main.tex}
        \part{Research}
            \input{1 - low-noise PiC models/main.tex}
            \input{2 - kinetic component/main.tex}
            \input{3 - fluid component/main.tex}
            \input{4 - numerical implementation/main.tex}
        \part{Project Overview}
            \input{5 - research plan/main.tex}
            \input{6 - summary/main.tex}
    
    
    %\section{}
    \newpage
    \pagenumbering{gobble}
        \printbibliography


    \newpage
    \pagenumbering{roman}
    \appendix
        \part{Appendices}
            \input{8 - Hilbert complexes/main.tex}
            \input{9 - weak conservation proofs/main.tex}
\end{document}

            \documentclass[12pt, a4paper]{report}

\input{template/main.tex}

\title{\BA{Title in Progress...}}
\author{Boris Andrews}
\affil{Mathematical Institute, University of Oxford}
\date{\today}


\begin{document}
    \pagenumbering{gobble}
    \maketitle
    
    
    \begin{abstract}
        Magnetic confinement reactors---in particular tokamaks---offer one of the most promising options for achieving practical nuclear fusion, with the potential to provide virtually limitless, clean energy. The theoretical and numerical modeling of tokamak plasmas is simultaneously an essential component of effective reactor design, and a great research barrier. Tokamak operational conditions exhibit comparatively low Knudsen numbers. Kinetic effects, including kinetic waves and instabilities, Landau damping, bump-on-tail instabilities and more, are therefore highly influential in tokamak plasma dynamics. Purely fluid models are inherently incapable of capturing these effects, whereas the high dimensionality in purely kinetic models render them practically intractable for most relevant purposes.

        We consider a $\delta\!f$ decomposition model, with a macroscopic fluid background and microscopic kinetic correction, both fully coupled to each other. A similar manner of discretization is proposed to that used in the recent \texttt{STRUPHY} code \cite{Holderied_Possanner_Wang_2021, Holderied_2022, Li_et_al_2023} with a finite-element model for the background and a pseudo-particle/PiC model for the correction.

        The fluid background satisfies the full, non-linear, resistive, compressible, Hall MHD equations. \cite{Laakmann_Hu_Farrell_2022} introduces finite-element(-in-space) implicit timesteppers for the incompressible analogue to this system with structure-preserving (SP) properties in the ideal case, alongside parameter-robust preconditioners. We show that these timesteppers can derive from a finite-element-in-time (FET) (and finite-element-in-space) interpretation. The benefits of this reformulation are discussed, including the derivation of timesteppers that are higher order in time, and the quantifiable dissipative SP properties in the non-ideal, resistive case.
        
        We discuss possible options for extending this FET approach to timesteppers for the compressible case.

        The kinetic corrections satisfy linearized Boltzmann equations. Using a Lénard--Bernstein collision operator, these take Fokker--Planck-like forms \cite{Fokker_1914, Planck_1917} wherein pseudo-particles in the numerical model obey the neoclassical transport equations, with particle-independent Brownian drift terms. This offers a rigorous methodology for incorporating collisions into the particle transport model, without coupling the equations of motions for each particle.
        
        Works by Chen, Chacón et al. \cite{Chen_Chacón_Barnes_2011, Chacón_Chen_Barnes_2013, Chen_Chacón_2014, Chen_Chacón_2015} have developed structure-preserving particle pushers for neoclassical transport in the Vlasov equations, derived from Crank--Nicolson integrators. We show these too can can derive from a FET interpretation, similarly offering potential extensions to higher-order-in-time particle pushers. The FET formulation is used also to consider how the stochastic drift terms can be incorporated into the pushers. Stochastic gyrokinetic expansions are also discussed.

        Different options for the numerical implementation of these schemes are considered.

        Due to the efficacy of FET in the development of SP timesteppers for both the fluid and kinetic component, we hope this approach will prove effective in the future for developing SP timesteppers for the full hybrid model. We hope this will give us the opportunity to incorporate previously inaccessible kinetic effects into the highly effective, modern, finite-element MHD models.
    \end{abstract}
    
    
    \newpage
    \tableofcontents
    
    
    \newpage
    \pagenumbering{arabic}
    %\linenumbers\renewcommand\thelinenumber{\color{black!50}\arabic{linenumber}}
            \input{0 - introduction/main.tex}
        \part{Research}
            \input{1 - low-noise PiC models/main.tex}
            \input{2 - kinetic component/main.tex}
            \input{3 - fluid component/main.tex}
            \input{4 - numerical implementation/main.tex}
        \part{Project Overview}
            \input{5 - research plan/main.tex}
            \input{6 - summary/main.tex}
    
    
    %\section{}
    \newpage
    \pagenumbering{gobble}
        \printbibliography


    \newpage
    \pagenumbering{roman}
    \appendix
        \part{Appendices}
            \input{8 - Hilbert complexes/main.tex}
            \input{9 - weak conservation proofs/main.tex}
\end{document}

\end{document}

\end{document}


\title{\BA{Title in Progress...}}
\author{Boris Andrews}
\affil{Mathematical Institute, University of Oxford}
\date{\today}


\begin{document}
    \pagenumbering{gobble}
    \maketitle
    
    
    \begin{abstract}
        Recent works by Laakmann, Farrell, and others \cite{LFM22, LHF22} have been successful in developing parameter-robust preconditioners and structure-preserving timesteppers for finite-element discretizations of the incompressible Hall MHD equations. In this thesis, we discuss the shortcomings of these models, especially when applied to the modeling of tokamak-like environments, and propose some options for filling these gaps.
        
        We consider a finite-element-in-time (FET) approach to discretizing the MHD models, and deriving MHD FE timesteppers. This alternative lens is very effective in the development of structure-preserving timesteppers, with a reduced number of auxiliary fields, scalability to timesteppers of arbitrary order, and a more transparent mathematical foundation.

        We think about whether the FET technique enables us to apply the modified compressibility constraint analogously and whether separate discretizations are necessary to reflect the various dynamics caused by the compressibility. We also look into how these compressible models might be added to the FET approach for building structure-preserving timesteppers.

        The models so far have been solely for incompressible MHD models, relying heavily on the incompressibility constraint, $\nabla\cdot\bfu  =  0$. Tokamak plasma dynamics however are necessarily severely compressible. We consider how the FET approach might enable us to use the modified compressibility constraint, $\partial_{t}\rho + \nabla\cdot\bfp  =  0$, in an analogous manner, and how distinct discretizations are required to capture the different dynamics induced by the compressibility. We also look into how the FET approach for the construction of structure-preserving timesteppers can be extended to these compressible models.

        A major drawback of all fluid models is their near inability to capture kinetic effects, which are pivotal in tokamak plasma dynamics. We consider a $\delta\!f$-like model, wherein the background compressible MHD equations are discretized using the FET techniques of above, and the correction is discretized using a pseudo-particle PiC model, with comparitvely low noise when compared to a classical full PiC model due to the background fluid model. The two discretizations are fully coupled, to give an efficient discretization to the full kinetic model. We consider the computational difficulties (parallelizations/scalability/etc.) of implementing such a coupled model, and the software possibilities.
        
        Different classes of approximations to the collision operator are considered, inferring different particle pushers containing Monte-Carlo-like stochastic components to capture the collisional effects. These include Brownian motion, Lévy flights, and more general classes of SDEs. We apply the gyrokinetic expansion to these different SDEs to create SDEs for the gyro-averaged variables. The stochastic collisional model---only attainable through the MHD/low-noise PiC coupling---has the benefit of fully decoupling the particle pushers for each individual pseudo-particle from the others.

        \BA{Need to fix.}
    \end{abstract}
    
    
    \newpage
    \tableofcontents
    
    
    \newpage
    \pagenumbering{arabic}
    %\linenumbers\renewcommand\thelinenumber{\color{black!50}\arabic{linenumber}}
            \documentclass[12pt, a4paper]{report}

\documentclass[12pt, a4paper]{report}

\documentclass[12pt, a4paper]{report}

\input{template/main.tex}

\title{\BA{Title in Progress...}}
\author{Boris Andrews}
\affil{Mathematical Institute, University of Oxford}
\date{\today}


\begin{document}
    \pagenumbering{gobble}
    \maketitle
    
    
    \begin{abstract}
        Magnetic confinement reactors---in particular tokamaks---offer one of the most promising options for achieving practical nuclear fusion, with the potential to provide virtually limitless, clean energy. The theoretical and numerical modeling of tokamak plasmas is simultaneously an essential component of effective reactor design, and a great research barrier. Tokamak operational conditions exhibit comparatively low Knudsen numbers. Kinetic effects, including kinetic waves and instabilities, Landau damping, bump-on-tail instabilities and more, are therefore highly influential in tokamak plasma dynamics. Purely fluid models are inherently incapable of capturing these effects, whereas the high dimensionality in purely kinetic models render them practically intractable for most relevant purposes.

        We consider a $\delta\!f$ decomposition model, with a macroscopic fluid background and microscopic kinetic correction, both fully coupled to each other. A similar manner of discretization is proposed to that used in the recent \texttt{STRUPHY} code \cite{Holderied_Possanner_Wang_2021, Holderied_2022, Li_et_al_2023} with a finite-element model for the background and a pseudo-particle/PiC model for the correction.

        The fluid background satisfies the full, non-linear, resistive, compressible, Hall MHD equations. \cite{Laakmann_Hu_Farrell_2022} introduces finite-element(-in-space) implicit timesteppers for the incompressible analogue to this system with structure-preserving (SP) properties in the ideal case, alongside parameter-robust preconditioners. We show that these timesteppers can derive from a finite-element-in-time (FET) (and finite-element-in-space) interpretation. The benefits of this reformulation are discussed, including the derivation of timesteppers that are higher order in time, and the quantifiable dissipative SP properties in the non-ideal, resistive case.
        
        We discuss possible options for extending this FET approach to timesteppers for the compressible case.

        The kinetic corrections satisfy linearized Boltzmann equations. Using a Lénard--Bernstein collision operator, these take Fokker--Planck-like forms \cite{Fokker_1914, Planck_1917} wherein pseudo-particles in the numerical model obey the neoclassical transport equations, with particle-independent Brownian drift terms. This offers a rigorous methodology for incorporating collisions into the particle transport model, without coupling the equations of motions for each particle.
        
        Works by Chen, Chacón et al. \cite{Chen_Chacón_Barnes_2011, Chacón_Chen_Barnes_2013, Chen_Chacón_2014, Chen_Chacón_2015} have developed structure-preserving particle pushers for neoclassical transport in the Vlasov equations, derived from Crank--Nicolson integrators. We show these too can can derive from a FET interpretation, similarly offering potential extensions to higher-order-in-time particle pushers. The FET formulation is used also to consider how the stochastic drift terms can be incorporated into the pushers. Stochastic gyrokinetic expansions are also discussed.

        Different options for the numerical implementation of these schemes are considered.

        Due to the efficacy of FET in the development of SP timesteppers for both the fluid and kinetic component, we hope this approach will prove effective in the future for developing SP timesteppers for the full hybrid model. We hope this will give us the opportunity to incorporate previously inaccessible kinetic effects into the highly effective, modern, finite-element MHD models.
    \end{abstract}
    
    
    \newpage
    \tableofcontents
    
    
    \newpage
    \pagenumbering{arabic}
    %\linenumbers\renewcommand\thelinenumber{\color{black!50}\arabic{linenumber}}
            \input{0 - introduction/main.tex}
        \part{Research}
            \input{1 - low-noise PiC models/main.tex}
            \input{2 - kinetic component/main.tex}
            \input{3 - fluid component/main.tex}
            \input{4 - numerical implementation/main.tex}
        \part{Project Overview}
            \input{5 - research plan/main.tex}
            \input{6 - summary/main.tex}
    
    
    %\section{}
    \newpage
    \pagenumbering{gobble}
        \printbibliography


    \newpage
    \pagenumbering{roman}
    \appendix
        \part{Appendices}
            \input{8 - Hilbert complexes/main.tex}
            \input{9 - weak conservation proofs/main.tex}
\end{document}


\title{\BA{Title in Progress...}}
\author{Boris Andrews}
\affil{Mathematical Institute, University of Oxford}
\date{\today}


\begin{document}
    \pagenumbering{gobble}
    \maketitle
    
    
    \begin{abstract}
        Magnetic confinement reactors---in particular tokamaks---offer one of the most promising options for achieving practical nuclear fusion, with the potential to provide virtually limitless, clean energy. The theoretical and numerical modeling of tokamak plasmas is simultaneously an essential component of effective reactor design, and a great research barrier. Tokamak operational conditions exhibit comparatively low Knudsen numbers. Kinetic effects, including kinetic waves and instabilities, Landau damping, bump-on-tail instabilities and more, are therefore highly influential in tokamak plasma dynamics. Purely fluid models are inherently incapable of capturing these effects, whereas the high dimensionality in purely kinetic models render them practically intractable for most relevant purposes.

        We consider a $\delta\!f$ decomposition model, with a macroscopic fluid background and microscopic kinetic correction, both fully coupled to each other. A similar manner of discretization is proposed to that used in the recent \texttt{STRUPHY} code \cite{Holderied_Possanner_Wang_2021, Holderied_2022, Li_et_al_2023} with a finite-element model for the background and a pseudo-particle/PiC model for the correction.

        The fluid background satisfies the full, non-linear, resistive, compressible, Hall MHD equations. \cite{Laakmann_Hu_Farrell_2022} introduces finite-element(-in-space) implicit timesteppers for the incompressible analogue to this system with structure-preserving (SP) properties in the ideal case, alongside parameter-robust preconditioners. We show that these timesteppers can derive from a finite-element-in-time (FET) (and finite-element-in-space) interpretation. The benefits of this reformulation are discussed, including the derivation of timesteppers that are higher order in time, and the quantifiable dissipative SP properties in the non-ideal, resistive case.
        
        We discuss possible options for extending this FET approach to timesteppers for the compressible case.

        The kinetic corrections satisfy linearized Boltzmann equations. Using a Lénard--Bernstein collision operator, these take Fokker--Planck-like forms \cite{Fokker_1914, Planck_1917} wherein pseudo-particles in the numerical model obey the neoclassical transport equations, with particle-independent Brownian drift terms. This offers a rigorous methodology for incorporating collisions into the particle transport model, without coupling the equations of motions for each particle.
        
        Works by Chen, Chacón et al. \cite{Chen_Chacón_Barnes_2011, Chacón_Chen_Barnes_2013, Chen_Chacón_2014, Chen_Chacón_2015} have developed structure-preserving particle pushers for neoclassical transport in the Vlasov equations, derived from Crank--Nicolson integrators. We show these too can can derive from a FET interpretation, similarly offering potential extensions to higher-order-in-time particle pushers. The FET formulation is used also to consider how the stochastic drift terms can be incorporated into the pushers. Stochastic gyrokinetic expansions are also discussed.

        Different options for the numerical implementation of these schemes are considered.

        Due to the efficacy of FET in the development of SP timesteppers for both the fluid and kinetic component, we hope this approach will prove effective in the future for developing SP timesteppers for the full hybrid model. We hope this will give us the opportunity to incorporate previously inaccessible kinetic effects into the highly effective, modern, finite-element MHD models.
    \end{abstract}
    
    
    \newpage
    \tableofcontents
    
    
    \newpage
    \pagenumbering{arabic}
    %\linenumbers\renewcommand\thelinenumber{\color{black!50}\arabic{linenumber}}
            \documentclass[12pt, a4paper]{report}

\input{template/main.tex}

\title{\BA{Title in Progress...}}
\author{Boris Andrews}
\affil{Mathematical Institute, University of Oxford}
\date{\today}


\begin{document}
    \pagenumbering{gobble}
    \maketitle
    
    
    \begin{abstract}
        Magnetic confinement reactors---in particular tokamaks---offer one of the most promising options for achieving practical nuclear fusion, with the potential to provide virtually limitless, clean energy. The theoretical and numerical modeling of tokamak plasmas is simultaneously an essential component of effective reactor design, and a great research barrier. Tokamak operational conditions exhibit comparatively low Knudsen numbers. Kinetic effects, including kinetic waves and instabilities, Landau damping, bump-on-tail instabilities and more, are therefore highly influential in tokamak plasma dynamics. Purely fluid models are inherently incapable of capturing these effects, whereas the high dimensionality in purely kinetic models render them practically intractable for most relevant purposes.

        We consider a $\delta\!f$ decomposition model, with a macroscopic fluid background and microscopic kinetic correction, both fully coupled to each other. A similar manner of discretization is proposed to that used in the recent \texttt{STRUPHY} code \cite{Holderied_Possanner_Wang_2021, Holderied_2022, Li_et_al_2023} with a finite-element model for the background and a pseudo-particle/PiC model for the correction.

        The fluid background satisfies the full, non-linear, resistive, compressible, Hall MHD equations. \cite{Laakmann_Hu_Farrell_2022} introduces finite-element(-in-space) implicit timesteppers for the incompressible analogue to this system with structure-preserving (SP) properties in the ideal case, alongside parameter-robust preconditioners. We show that these timesteppers can derive from a finite-element-in-time (FET) (and finite-element-in-space) interpretation. The benefits of this reformulation are discussed, including the derivation of timesteppers that are higher order in time, and the quantifiable dissipative SP properties in the non-ideal, resistive case.
        
        We discuss possible options for extending this FET approach to timesteppers for the compressible case.

        The kinetic corrections satisfy linearized Boltzmann equations. Using a Lénard--Bernstein collision operator, these take Fokker--Planck-like forms \cite{Fokker_1914, Planck_1917} wherein pseudo-particles in the numerical model obey the neoclassical transport equations, with particle-independent Brownian drift terms. This offers a rigorous methodology for incorporating collisions into the particle transport model, without coupling the equations of motions for each particle.
        
        Works by Chen, Chacón et al. \cite{Chen_Chacón_Barnes_2011, Chacón_Chen_Barnes_2013, Chen_Chacón_2014, Chen_Chacón_2015} have developed structure-preserving particle pushers for neoclassical transport in the Vlasov equations, derived from Crank--Nicolson integrators. We show these too can can derive from a FET interpretation, similarly offering potential extensions to higher-order-in-time particle pushers. The FET formulation is used also to consider how the stochastic drift terms can be incorporated into the pushers. Stochastic gyrokinetic expansions are also discussed.

        Different options for the numerical implementation of these schemes are considered.

        Due to the efficacy of FET in the development of SP timesteppers for both the fluid and kinetic component, we hope this approach will prove effective in the future for developing SP timesteppers for the full hybrid model. We hope this will give us the opportunity to incorporate previously inaccessible kinetic effects into the highly effective, modern, finite-element MHD models.
    \end{abstract}
    
    
    \newpage
    \tableofcontents
    
    
    \newpage
    \pagenumbering{arabic}
    %\linenumbers\renewcommand\thelinenumber{\color{black!50}\arabic{linenumber}}
            \input{0 - introduction/main.tex}
        \part{Research}
            \input{1 - low-noise PiC models/main.tex}
            \input{2 - kinetic component/main.tex}
            \input{3 - fluid component/main.tex}
            \input{4 - numerical implementation/main.tex}
        \part{Project Overview}
            \input{5 - research plan/main.tex}
            \input{6 - summary/main.tex}
    
    
    %\section{}
    \newpage
    \pagenumbering{gobble}
        \printbibliography


    \newpage
    \pagenumbering{roman}
    \appendix
        \part{Appendices}
            \input{8 - Hilbert complexes/main.tex}
            \input{9 - weak conservation proofs/main.tex}
\end{document}

        \part{Research}
            \documentclass[12pt, a4paper]{report}

\input{template/main.tex}

\title{\BA{Title in Progress...}}
\author{Boris Andrews}
\affil{Mathematical Institute, University of Oxford}
\date{\today}


\begin{document}
    \pagenumbering{gobble}
    \maketitle
    
    
    \begin{abstract}
        Magnetic confinement reactors---in particular tokamaks---offer one of the most promising options for achieving practical nuclear fusion, with the potential to provide virtually limitless, clean energy. The theoretical and numerical modeling of tokamak plasmas is simultaneously an essential component of effective reactor design, and a great research barrier. Tokamak operational conditions exhibit comparatively low Knudsen numbers. Kinetic effects, including kinetic waves and instabilities, Landau damping, bump-on-tail instabilities and more, are therefore highly influential in tokamak plasma dynamics. Purely fluid models are inherently incapable of capturing these effects, whereas the high dimensionality in purely kinetic models render them practically intractable for most relevant purposes.

        We consider a $\delta\!f$ decomposition model, with a macroscopic fluid background and microscopic kinetic correction, both fully coupled to each other. A similar manner of discretization is proposed to that used in the recent \texttt{STRUPHY} code \cite{Holderied_Possanner_Wang_2021, Holderied_2022, Li_et_al_2023} with a finite-element model for the background and a pseudo-particle/PiC model for the correction.

        The fluid background satisfies the full, non-linear, resistive, compressible, Hall MHD equations. \cite{Laakmann_Hu_Farrell_2022} introduces finite-element(-in-space) implicit timesteppers for the incompressible analogue to this system with structure-preserving (SP) properties in the ideal case, alongside parameter-robust preconditioners. We show that these timesteppers can derive from a finite-element-in-time (FET) (and finite-element-in-space) interpretation. The benefits of this reformulation are discussed, including the derivation of timesteppers that are higher order in time, and the quantifiable dissipative SP properties in the non-ideal, resistive case.
        
        We discuss possible options for extending this FET approach to timesteppers for the compressible case.

        The kinetic corrections satisfy linearized Boltzmann equations. Using a Lénard--Bernstein collision operator, these take Fokker--Planck-like forms \cite{Fokker_1914, Planck_1917} wherein pseudo-particles in the numerical model obey the neoclassical transport equations, with particle-independent Brownian drift terms. This offers a rigorous methodology for incorporating collisions into the particle transport model, without coupling the equations of motions for each particle.
        
        Works by Chen, Chacón et al. \cite{Chen_Chacón_Barnes_2011, Chacón_Chen_Barnes_2013, Chen_Chacón_2014, Chen_Chacón_2015} have developed structure-preserving particle pushers for neoclassical transport in the Vlasov equations, derived from Crank--Nicolson integrators. We show these too can can derive from a FET interpretation, similarly offering potential extensions to higher-order-in-time particle pushers. The FET formulation is used also to consider how the stochastic drift terms can be incorporated into the pushers. Stochastic gyrokinetic expansions are also discussed.

        Different options for the numerical implementation of these schemes are considered.

        Due to the efficacy of FET in the development of SP timesteppers for both the fluid and kinetic component, we hope this approach will prove effective in the future for developing SP timesteppers for the full hybrid model. We hope this will give us the opportunity to incorporate previously inaccessible kinetic effects into the highly effective, modern, finite-element MHD models.
    \end{abstract}
    
    
    \newpage
    \tableofcontents
    
    
    \newpage
    \pagenumbering{arabic}
    %\linenumbers\renewcommand\thelinenumber{\color{black!50}\arabic{linenumber}}
            \input{0 - introduction/main.tex}
        \part{Research}
            \input{1 - low-noise PiC models/main.tex}
            \input{2 - kinetic component/main.tex}
            \input{3 - fluid component/main.tex}
            \input{4 - numerical implementation/main.tex}
        \part{Project Overview}
            \input{5 - research plan/main.tex}
            \input{6 - summary/main.tex}
    
    
    %\section{}
    \newpage
    \pagenumbering{gobble}
        \printbibliography


    \newpage
    \pagenumbering{roman}
    \appendix
        \part{Appendices}
            \input{8 - Hilbert complexes/main.tex}
            \input{9 - weak conservation proofs/main.tex}
\end{document}

            \documentclass[12pt, a4paper]{report}

\input{template/main.tex}

\title{\BA{Title in Progress...}}
\author{Boris Andrews}
\affil{Mathematical Institute, University of Oxford}
\date{\today}


\begin{document}
    \pagenumbering{gobble}
    \maketitle
    
    
    \begin{abstract}
        Magnetic confinement reactors---in particular tokamaks---offer one of the most promising options for achieving practical nuclear fusion, with the potential to provide virtually limitless, clean energy. The theoretical and numerical modeling of tokamak plasmas is simultaneously an essential component of effective reactor design, and a great research barrier. Tokamak operational conditions exhibit comparatively low Knudsen numbers. Kinetic effects, including kinetic waves and instabilities, Landau damping, bump-on-tail instabilities and more, are therefore highly influential in tokamak plasma dynamics. Purely fluid models are inherently incapable of capturing these effects, whereas the high dimensionality in purely kinetic models render them practically intractable for most relevant purposes.

        We consider a $\delta\!f$ decomposition model, with a macroscopic fluid background and microscopic kinetic correction, both fully coupled to each other. A similar manner of discretization is proposed to that used in the recent \texttt{STRUPHY} code \cite{Holderied_Possanner_Wang_2021, Holderied_2022, Li_et_al_2023} with a finite-element model for the background and a pseudo-particle/PiC model for the correction.

        The fluid background satisfies the full, non-linear, resistive, compressible, Hall MHD equations. \cite{Laakmann_Hu_Farrell_2022} introduces finite-element(-in-space) implicit timesteppers for the incompressible analogue to this system with structure-preserving (SP) properties in the ideal case, alongside parameter-robust preconditioners. We show that these timesteppers can derive from a finite-element-in-time (FET) (and finite-element-in-space) interpretation. The benefits of this reformulation are discussed, including the derivation of timesteppers that are higher order in time, and the quantifiable dissipative SP properties in the non-ideal, resistive case.
        
        We discuss possible options for extending this FET approach to timesteppers for the compressible case.

        The kinetic corrections satisfy linearized Boltzmann equations. Using a Lénard--Bernstein collision operator, these take Fokker--Planck-like forms \cite{Fokker_1914, Planck_1917} wherein pseudo-particles in the numerical model obey the neoclassical transport equations, with particle-independent Brownian drift terms. This offers a rigorous methodology for incorporating collisions into the particle transport model, without coupling the equations of motions for each particle.
        
        Works by Chen, Chacón et al. \cite{Chen_Chacón_Barnes_2011, Chacón_Chen_Barnes_2013, Chen_Chacón_2014, Chen_Chacón_2015} have developed structure-preserving particle pushers for neoclassical transport in the Vlasov equations, derived from Crank--Nicolson integrators. We show these too can can derive from a FET interpretation, similarly offering potential extensions to higher-order-in-time particle pushers. The FET formulation is used also to consider how the stochastic drift terms can be incorporated into the pushers. Stochastic gyrokinetic expansions are also discussed.

        Different options for the numerical implementation of these schemes are considered.

        Due to the efficacy of FET in the development of SP timesteppers for both the fluid and kinetic component, we hope this approach will prove effective in the future for developing SP timesteppers for the full hybrid model. We hope this will give us the opportunity to incorporate previously inaccessible kinetic effects into the highly effective, modern, finite-element MHD models.
    \end{abstract}
    
    
    \newpage
    \tableofcontents
    
    
    \newpage
    \pagenumbering{arabic}
    %\linenumbers\renewcommand\thelinenumber{\color{black!50}\arabic{linenumber}}
            \input{0 - introduction/main.tex}
        \part{Research}
            \input{1 - low-noise PiC models/main.tex}
            \input{2 - kinetic component/main.tex}
            \input{3 - fluid component/main.tex}
            \input{4 - numerical implementation/main.tex}
        \part{Project Overview}
            \input{5 - research plan/main.tex}
            \input{6 - summary/main.tex}
    
    
    %\section{}
    \newpage
    \pagenumbering{gobble}
        \printbibliography


    \newpage
    \pagenumbering{roman}
    \appendix
        \part{Appendices}
            \input{8 - Hilbert complexes/main.tex}
            \input{9 - weak conservation proofs/main.tex}
\end{document}

            \documentclass[12pt, a4paper]{report}

\input{template/main.tex}

\title{\BA{Title in Progress...}}
\author{Boris Andrews}
\affil{Mathematical Institute, University of Oxford}
\date{\today}


\begin{document}
    \pagenumbering{gobble}
    \maketitle
    
    
    \begin{abstract}
        Magnetic confinement reactors---in particular tokamaks---offer one of the most promising options for achieving practical nuclear fusion, with the potential to provide virtually limitless, clean energy. The theoretical and numerical modeling of tokamak plasmas is simultaneously an essential component of effective reactor design, and a great research barrier. Tokamak operational conditions exhibit comparatively low Knudsen numbers. Kinetic effects, including kinetic waves and instabilities, Landau damping, bump-on-tail instabilities and more, are therefore highly influential in tokamak plasma dynamics. Purely fluid models are inherently incapable of capturing these effects, whereas the high dimensionality in purely kinetic models render them practically intractable for most relevant purposes.

        We consider a $\delta\!f$ decomposition model, with a macroscopic fluid background and microscopic kinetic correction, both fully coupled to each other. A similar manner of discretization is proposed to that used in the recent \texttt{STRUPHY} code \cite{Holderied_Possanner_Wang_2021, Holderied_2022, Li_et_al_2023} with a finite-element model for the background and a pseudo-particle/PiC model for the correction.

        The fluid background satisfies the full, non-linear, resistive, compressible, Hall MHD equations. \cite{Laakmann_Hu_Farrell_2022} introduces finite-element(-in-space) implicit timesteppers for the incompressible analogue to this system with structure-preserving (SP) properties in the ideal case, alongside parameter-robust preconditioners. We show that these timesteppers can derive from a finite-element-in-time (FET) (and finite-element-in-space) interpretation. The benefits of this reformulation are discussed, including the derivation of timesteppers that are higher order in time, and the quantifiable dissipative SP properties in the non-ideal, resistive case.
        
        We discuss possible options for extending this FET approach to timesteppers for the compressible case.

        The kinetic corrections satisfy linearized Boltzmann equations. Using a Lénard--Bernstein collision operator, these take Fokker--Planck-like forms \cite{Fokker_1914, Planck_1917} wherein pseudo-particles in the numerical model obey the neoclassical transport equations, with particle-independent Brownian drift terms. This offers a rigorous methodology for incorporating collisions into the particle transport model, without coupling the equations of motions for each particle.
        
        Works by Chen, Chacón et al. \cite{Chen_Chacón_Barnes_2011, Chacón_Chen_Barnes_2013, Chen_Chacón_2014, Chen_Chacón_2015} have developed structure-preserving particle pushers for neoclassical transport in the Vlasov equations, derived from Crank--Nicolson integrators. We show these too can can derive from a FET interpretation, similarly offering potential extensions to higher-order-in-time particle pushers. The FET formulation is used also to consider how the stochastic drift terms can be incorporated into the pushers. Stochastic gyrokinetic expansions are also discussed.

        Different options for the numerical implementation of these schemes are considered.

        Due to the efficacy of FET in the development of SP timesteppers for both the fluid and kinetic component, we hope this approach will prove effective in the future for developing SP timesteppers for the full hybrid model. We hope this will give us the opportunity to incorporate previously inaccessible kinetic effects into the highly effective, modern, finite-element MHD models.
    \end{abstract}
    
    
    \newpage
    \tableofcontents
    
    
    \newpage
    \pagenumbering{arabic}
    %\linenumbers\renewcommand\thelinenumber{\color{black!50}\arabic{linenumber}}
            \input{0 - introduction/main.tex}
        \part{Research}
            \input{1 - low-noise PiC models/main.tex}
            \input{2 - kinetic component/main.tex}
            \input{3 - fluid component/main.tex}
            \input{4 - numerical implementation/main.tex}
        \part{Project Overview}
            \input{5 - research plan/main.tex}
            \input{6 - summary/main.tex}
    
    
    %\section{}
    \newpage
    \pagenumbering{gobble}
        \printbibliography


    \newpage
    \pagenumbering{roman}
    \appendix
        \part{Appendices}
            \input{8 - Hilbert complexes/main.tex}
            \input{9 - weak conservation proofs/main.tex}
\end{document}

            \documentclass[12pt, a4paper]{report}

\input{template/main.tex}

\title{\BA{Title in Progress...}}
\author{Boris Andrews}
\affil{Mathematical Institute, University of Oxford}
\date{\today}


\begin{document}
    \pagenumbering{gobble}
    \maketitle
    
    
    \begin{abstract}
        Magnetic confinement reactors---in particular tokamaks---offer one of the most promising options for achieving practical nuclear fusion, with the potential to provide virtually limitless, clean energy. The theoretical and numerical modeling of tokamak plasmas is simultaneously an essential component of effective reactor design, and a great research barrier. Tokamak operational conditions exhibit comparatively low Knudsen numbers. Kinetic effects, including kinetic waves and instabilities, Landau damping, bump-on-tail instabilities and more, are therefore highly influential in tokamak plasma dynamics. Purely fluid models are inherently incapable of capturing these effects, whereas the high dimensionality in purely kinetic models render them practically intractable for most relevant purposes.

        We consider a $\delta\!f$ decomposition model, with a macroscopic fluid background and microscopic kinetic correction, both fully coupled to each other. A similar manner of discretization is proposed to that used in the recent \texttt{STRUPHY} code \cite{Holderied_Possanner_Wang_2021, Holderied_2022, Li_et_al_2023} with a finite-element model for the background and a pseudo-particle/PiC model for the correction.

        The fluid background satisfies the full, non-linear, resistive, compressible, Hall MHD equations. \cite{Laakmann_Hu_Farrell_2022} introduces finite-element(-in-space) implicit timesteppers for the incompressible analogue to this system with structure-preserving (SP) properties in the ideal case, alongside parameter-robust preconditioners. We show that these timesteppers can derive from a finite-element-in-time (FET) (and finite-element-in-space) interpretation. The benefits of this reformulation are discussed, including the derivation of timesteppers that are higher order in time, and the quantifiable dissipative SP properties in the non-ideal, resistive case.
        
        We discuss possible options for extending this FET approach to timesteppers for the compressible case.

        The kinetic corrections satisfy linearized Boltzmann equations. Using a Lénard--Bernstein collision operator, these take Fokker--Planck-like forms \cite{Fokker_1914, Planck_1917} wherein pseudo-particles in the numerical model obey the neoclassical transport equations, with particle-independent Brownian drift terms. This offers a rigorous methodology for incorporating collisions into the particle transport model, without coupling the equations of motions for each particle.
        
        Works by Chen, Chacón et al. \cite{Chen_Chacón_Barnes_2011, Chacón_Chen_Barnes_2013, Chen_Chacón_2014, Chen_Chacón_2015} have developed structure-preserving particle pushers for neoclassical transport in the Vlasov equations, derived from Crank--Nicolson integrators. We show these too can can derive from a FET interpretation, similarly offering potential extensions to higher-order-in-time particle pushers. The FET formulation is used also to consider how the stochastic drift terms can be incorporated into the pushers. Stochastic gyrokinetic expansions are also discussed.

        Different options for the numerical implementation of these schemes are considered.

        Due to the efficacy of FET in the development of SP timesteppers for both the fluid and kinetic component, we hope this approach will prove effective in the future for developing SP timesteppers for the full hybrid model. We hope this will give us the opportunity to incorporate previously inaccessible kinetic effects into the highly effective, modern, finite-element MHD models.
    \end{abstract}
    
    
    \newpage
    \tableofcontents
    
    
    \newpage
    \pagenumbering{arabic}
    %\linenumbers\renewcommand\thelinenumber{\color{black!50}\arabic{linenumber}}
            \input{0 - introduction/main.tex}
        \part{Research}
            \input{1 - low-noise PiC models/main.tex}
            \input{2 - kinetic component/main.tex}
            \input{3 - fluid component/main.tex}
            \input{4 - numerical implementation/main.tex}
        \part{Project Overview}
            \input{5 - research plan/main.tex}
            \input{6 - summary/main.tex}
    
    
    %\section{}
    \newpage
    \pagenumbering{gobble}
        \printbibliography


    \newpage
    \pagenumbering{roman}
    \appendix
        \part{Appendices}
            \input{8 - Hilbert complexes/main.tex}
            \input{9 - weak conservation proofs/main.tex}
\end{document}

        \part{Project Overview}
            \documentclass[12pt, a4paper]{report}

\input{template/main.tex}

\title{\BA{Title in Progress...}}
\author{Boris Andrews}
\affil{Mathematical Institute, University of Oxford}
\date{\today}


\begin{document}
    \pagenumbering{gobble}
    \maketitle
    
    
    \begin{abstract}
        Magnetic confinement reactors---in particular tokamaks---offer one of the most promising options for achieving practical nuclear fusion, with the potential to provide virtually limitless, clean energy. The theoretical and numerical modeling of tokamak plasmas is simultaneously an essential component of effective reactor design, and a great research barrier. Tokamak operational conditions exhibit comparatively low Knudsen numbers. Kinetic effects, including kinetic waves and instabilities, Landau damping, bump-on-tail instabilities and more, are therefore highly influential in tokamak plasma dynamics. Purely fluid models are inherently incapable of capturing these effects, whereas the high dimensionality in purely kinetic models render them practically intractable for most relevant purposes.

        We consider a $\delta\!f$ decomposition model, with a macroscopic fluid background and microscopic kinetic correction, both fully coupled to each other. A similar manner of discretization is proposed to that used in the recent \texttt{STRUPHY} code \cite{Holderied_Possanner_Wang_2021, Holderied_2022, Li_et_al_2023} with a finite-element model for the background and a pseudo-particle/PiC model for the correction.

        The fluid background satisfies the full, non-linear, resistive, compressible, Hall MHD equations. \cite{Laakmann_Hu_Farrell_2022} introduces finite-element(-in-space) implicit timesteppers for the incompressible analogue to this system with structure-preserving (SP) properties in the ideal case, alongside parameter-robust preconditioners. We show that these timesteppers can derive from a finite-element-in-time (FET) (and finite-element-in-space) interpretation. The benefits of this reformulation are discussed, including the derivation of timesteppers that are higher order in time, and the quantifiable dissipative SP properties in the non-ideal, resistive case.
        
        We discuss possible options for extending this FET approach to timesteppers for the compressible case.

        The kinetic corrections satisfy linearized Boltzmann equations. Using a Lénard--Bernstein collision operator, these take Fokker--Planck-like forms \cite{Fokker_1914, Planck_1917} wherein pseudo-particles in the numerical model obey the neoclassical transport equations, with particle-independent Brownian drift terms. This offers a rigorous methodology for incorporating collisions into the particle transport model, without coupling the equations of motions for each particle.
        
        Works by Chen, Chacón et al. \cite{Chen_Chacón_Barnes_2011, Chacón_Chen_Barnes_2013, Chen_Chacón_2014, Chen_Chacón_2015} have developed structure-preserving particle pushers for neoclassical transport in the Vlasov equations, derived from Crank--Nicolson integrators. We show these too can can derive from a FET interpretation, similarly offering potential extensions to higher-order-in-time particle pushers. The FET formulation is used also to consider how the stochastic drift terms can be incorporated into the pushers. Stochastic gyrokinetic expansions are also discussed.

        Different options for the numerical implementation of these schemes are considered.

        Due to the efficacy of FET in the development of SP timesteppers for both the fluid and kinetic component, we hope this approach will prove effective in the future for developing SP timesteppers for the full hybrid model. We hope this will give us the opportunity to incorporate previously inaccessible kinetic effects into the highly effective, modern, finite-element MHD models.
    \end{abstract}
    
    
    \newpage
    \tableofcontents
    
    
    \newpage
    \pagenumbering{arabic}
    %\linenumbers\renewcommand\thelinenumber{\color{black!50}\arabic{linenumber}}
            \input{0 - introduction/main.tex}
        \part{Research}
            \input{1 - low-noise PiC models/main.tex}
            \input{2 - kinetic component/main.tex}
            \input{3 - fluid component/main.tex}
            \input{4 - numerical implementation/main.tex}
        \part{Project Overview}
            \input{5 - research plan/main.tex}
            \input{6 - summary/main.tex}
    
    
    %\section{}
    \newpage
    \pagenumbering{gobble}
        \printbibliography


    \newpage
    \pagenumbering{roman}
    \appendix
        \part{Appendices}
            \input{8 - Hilbert complexes/main.tex}
            \input{9 - weak conservation proofs/main.tex}
\end{document}

            \documentclass[12pt, a4paper]{report}

\input{template/main.tex}

\title{\BA{Title in Progress...}}
\author{Boris Andrews}
\affil{Mathematical Institute, University of Oxford}
\date{\today}


\begin{document}
    \pagenumbering{gobble}
    \maketitle
    
    
    \begin{abstract}
        Magnetic confinement reactors---in particular tokamaks---offer one of the most promising options for achieving practical nuclear fusion, with the potential to provide virtually limitless, clean energy. The theoretical and numerical modeling of tokamak plasmas is simultaneously an essential component of effective reactor design, and a great research barrier. Tokamak operational conditions exhibit comparatively low Knudsen numbers. Kinetic effects, including kinetic waves and instabilities, Landau damping, bump-on-tail instabilities and more, are therefore highly influential in tokamak plasma dynamics. Purely fluid models are inherently incapable of capturing these effects, whereas the high dimensionality in purely kinetic models render them practically intractable for most relevant purposes.

        We consider a $\delta\!f$ decomposition model, with a macroscopic fluid background and microscopic kinetic correction, both fully coupled to each other. A similar manner of discretization is proposed to that used in the recent \texttt{STRUPHY} code \cite{Holderied_Possanner_Wang_2021, Holderied_2022, Li_et_al_2023} with a finite-element model for the background and a pseudo-particle/PiC model for the correction.

        The fluid background satisfies the full, non-linear, resistive, compressible, Hall MHD equations. \cite{Laakmann_Hu_Farrell_2022} introduces finite-element(-in-space) implicit timesteppers for the incompressible analogue to this system with structure-preserving (SP) properties in the ideal case, alongside parameter-robust preconditioners. We show that these timesteppers can derive from a finite-element-in-time (FET) (and finite-element-in-space) interpretation. The benefits of this reformulation are discussed, including the derivation of timesteppers that are higher order in time, and the quantifiable dissipative SP properties in the non-ideal, resistive case.
        
        We discuss possible options for extending this FET approach to timesteppers for the compressible case.

        The kinetic corrections satisfy linearized Boltzmann equations. Using a Lénard--Bernstein collision operator, these take Fokker--Planck-like forms \cite{Fokker_1914, Planck_1917} wherein pseudo-particles in the numerical model obey the neoclassical transport equations, with particle-independent Brownian drift terms. This offers a rigorous methodology for incorporating collisions into the particle transport model, without coupling the equations of motions for each particle.
        
        Works by Chen, Chacón et al. \cite{Chen_Chacón_Barnes_2011, Chacón_Chen_Barnes_2013, Chen_Chacón_2014, Chen_Chacón_2015} have developed structure-preserving particle pushers for neoclassical transport in the Vlasov equations, derived from Crank--Nicolson integrators. We show these too can can derive from a FET interpretation, similarly offering potential extensions to higher-order-in-time particle pushers. The FET formulation is used also to consider how the stochastic drift terms can be incorporated into the pushers. Stochastic gyrokinetic expansions are also discussed.

        Different options for the numerical implementation of these schemes are considered.

        Due to the efficacy of FET in the development of SP timesteppers for both the fluid and kinetic component, we hope this approach will prove effective in the future for developing SP timesteppers for the full hybrid model. We hope this will give us the opportunity to incorporate previously inaccessible kinetic effects into the highly effective, modern, finite-element MHD models.
    \end{abstract}
    
    
    \newpage
    \tableofcontents
    
    
    \newpage
    \pagenumbering{arabic}
    %\linenumbers\renewcommand\thelinenumber{\color{black!50}\arabic{linenumber}}
            \input{0 - introduction/main.tex}
        \part{Research}
            \input{1 - low-noise PiC models/main.tex}
            \input{2 - kinetic component/main.tex}
            \input{3 - fluid component/main.tex}
            \input{4 - numerical implementation/main.tex}
        \part{Project Overview}
            \input{5 - research plan/main.tex}
            \input{6 - summary/main.tex}
    
    
    %\section{}
    \newpage
    \pagenumbering{gobble}
        \printbibliography


    \newpage
    \pagenumbering{roman}
    \appendix
        \part{Appendices}
            \input{8 - Hilbert complexes/main.tex}
            \input{9 - weak conservation proofs/main.tex}
\end{document}

    
    
    %\section{}
    \newpage
    \pagenumbering{gobble}
        \printbibliography


    \newpage
    \pagenumbering{roman}
    \appendix
        \part{Appendices}
            \documentclass[12pt, a4paper]{report}

\input{template/main.tex}

\title{\BA{Title in Progress...}}
\author{Boris Andrews}
\affil{Mathematical Institute, University of Oxford}
\date{\today}


\begin{document}
    \pagenumbering{gobble}
    \maketitle
    
    
    \begin{abstract}
        Magnetic confinement reactors---in particular tokamaks---offer one of the most promising options for achieving practical nuclear fusion, with the potential to provide virtually limitless, clean energy. The theoretical and numerical modeling of tokamak plasmas is simultaneously an essential component of effective reactor design, and a great research barrier. Tokamak operational conditions exhibit comparatively low Knudsen numbers. Kinetic effects, including kinetic waves and instabilities, Landau damping, bump-on-tail instabilities and more, are therefore highly influential in tokamak plasma dynamics. Purely fluid models are inherently incapable of capturing these effects, whereas the high dimensionality in purely kinetic models render them practically intractable for most relevant purposes.

        We consider a $\delta\!f$ decomposition model, with a macroscopic fluid background and microscopic kinetic correction, both fully coupled to each other. A similar manner of discretization is proposed to that used in the recent \texttt{STRUPHY} code \cite{Holderied_Possanner_Wang_2021, Holderied_2022, Li_et_al_2023} with a finite-element model for the background and a pseudo-particle/PiC model for the correction.

        The fluid background satisfies the full, non-linear, resistive, compressible, Hall MHD equations. \cite{Laakmann_Hu_Farrell_2022} introduces finite-element(-in-space) implicit timesteppers for the incompressible analogue to this system with structure-preserving (SP) properties in the ideal case, alongside parameter-robust preconditioners. We show that these timesteppers can derive from a finite-element-in-time (FET) (and finite-element-in-space) interpretation. The benefits of this reformulation are discussed, including the derivation of timesteppers that are higher order in time, and the quantifiable dissipative SP properties in the non-ideal, resistive case.
        
        We discuss possible options for extending this FET approach to timesteppers for the compressible case.

        The kinetic corrections satisfy linearized Boltzmann equations. Using a Lénard--Bernstein collision operator, these take Fokker--Planck-like forms \cite{Fokker_1914, Planck_1917} wherein pseudo-particles in the numerical model obey the neoclassical transport equations, with particle-independent Brownian drift terms. This offers a rigorous methodology for incorporating collisions into the particle transport model, without coupling the equations of motions for each particle.
        
        Works by Chen, Chacón et al. \cite{Chen_Chacón_Barnes_2011, Chacón_Chen_Barnes_2013, Chen_Chacón_2014, Chen_Chacón_2015} have developed structure-preserving particle pushers for neoclassical transport in the Vlasov equations, derived from Crank--Nicolson integrators. We show these too can can derive from a FET interpretation, similarly offering potential extensions to higher-order-in-time particle pushers. The FET formulation is used also to consider how the stochastic drift terms can be incorporated into the pushers. Stochastic gyrokinetic expansions are also discussed.

        Different options for the numerical implementation of these schemes are considered.

        Due to the efficacy of FET in the development of SP timesteppers for both the fluid and kinetic component, we hope this approach will prove effective in the future for developing SP timesteppers for the full hybrid model. We hope this will give us the opportunity to incorporate previously inaccessible kinetic effects into the highly effective, modern, finite-element MHD models.
    \end{abstract}
    
    
    \newpage
    \tableofcontents
    
    
    \newpage
    \pagenumbering{arabic}
    %\linenumbers\renewcommand\thelinenumber{\color{black!50}\arabic{linenumber}}
            \input{0 - introduction/main.tex}
        \part{Research}
            \input{1 - low-noise PiC models/main.tex}
            \input{2 - kinetic component/main.tex}
            \input{3 - fluid component/main.tex}
            \input{4 - numerical implementation/main.tex}
        \part{Project Overview}
            \input{5 - research plan/main.tex}
            \input{6 - summary/main.tex}
    
    
    %\section{}
    \newpage
    \pagenumbering{gobble}
        \printbibliography


    \newpage
    \pagenumbering{roman}
    \appendix
        \part{Appendices}
            \input{8 - Hilbert complexes/main.tex}
            \input{9 - weak conservation proofs/main.tex}
\end{document}

            \documentclass[12pt, a4paper]{report}

\input{template/main.tex}

\title{\BA{Title in Progress...}}
\author{Boris Andrews}
\affil{Mathematical Institute, University of Oxford}
\date{\today}


\begin{document}
    \pagenumbering{gobble}
    \maketitle
    
    
    \begin{abstract}
        Magnetic confinement reactors---in particular tokamaks---offer one of the most promising options for achieving practical nuclear fusion, with the potential to provide virtually limitless, clean energy. The theoretical and numerical modeling of tokamak plasmas is simultaneously an essential component of effective reactor design, and a great research barrier. Tokamak operational conditions exhibit comparatively low Knudsen numbers. Kinetic effects, including kinetic waves and instabilities, Landau damping, bump-on-tail instabilities and more, are therefore highly influential in tokamak plasma dynamics. Purely fluid models are inherently incapable of capturing these effects, whereas the high dimensionality in purely kinetic models render them practically intractable for most relevant purposes.

        We consider a $\delta\!f$ decomposition model, with a macroscopic fluid background and microscopic kinetic correction, both fully coupled to each other. A similar manner of discretization is proposed to that used in the recent \texttt{STRUPHY} code \cite{Holderied_Possanner_Wang_2021, Holderied_2022, Li_et_al_2023} with a finite-element model for the background and a pseudo-particle/PiC model for the correction.

        The fluid background satisfies the full, non-linear, resistive, compressible, Hall MHD equations. \cite{Laakmann_Hu_Farrell_2022} introduces finite-element(-in-space) implicit timesteppers for the incompressible analogue to this system with structure-preserving (SP) properties in the ideal case, alongside parameter-robust preconditioners. We show that these timesteppers can derive from a finite-element-in-time (FET) (and finite-element-in-space) interpretation. The benefits of this reformulation are discussed, including the derivation of timesteppers that are higher order in time, and the quantifiable dissipative SP properties in the non-ideal, resistive case.
        
        We discuss possible options for extending this FET approach to timesteppers for the compressible case.

        The kinetic corrections satisfy linearized Boltzmann equations. Using a Lénard--Bernstein collision operator, these take Fokker--Planck-like forms \cite{Fokker_1914, Planck_1917} wherein pseudo-particles in the numerical model obey the neoclassical transport equations, with particle-independent Brownian drift terms. This offers a rigorous methodology for incorporating collisions into the particle transport model, without coupling the equations of motions for each particle.
        
        Works by Chen, Chacón et al. \cite{Chen_Chacón_Barnes_2011, Chacón_Chen_Barnes_2013, Chen_Chacón_2014, Chen_Chacón_2015} have developed structure-preserving particle pushers for neoclassical transport in the Vlasov equations, derived from Crank--Nicolson integrators. We show these too can can derive from a FET interpretation, similarly offering potential extensions to higher-order-in-time particle pushers. The FET formulation is used also to consider how the stochastic drift terms can be incorporated into the pushers. Stochastic gyrokinetic expansions are also discussed.

        Different options for the numerical implementation of these schemes are considered.

        Due to the efficacy of FET in the development of SP timesteppers for both the fluid and kinetic component, we hope this approach will prove effective in the future for developing SP timesteppers for the full hybrid model. We hope this will give us the opportunity to incorporate previously inaccessible kinetic effects into the highly effective, modern, finite-element MHD models.
    \end{abstract}
    
    
    \newpage
    \tableofcontents
    
    
    \newpage
    \pagenumbering{arabic}
    %\linenumbers\renewcommand\thelinenumber{\color{black!50}\arabic{linenumber}}
            \input{0 - introduction/main.tex}
        \part{Research}
            \input{1 - low-noise PiC models/main.tex}
            \input{2 - kinetic component/main.tex}
            \input{3 - fluid component/main.tex}
            \input{4 - numerical implementation/main.tex}
        \part{Project Overview}
            \input{5 - research plan/main.tex}
            \input{6 - summary/main.tex}
    
    
    %\section{}
    \newpage
    \pagenumbering{gobble}
        \printbibliography


    \newpage
    \pagenumbering{roman}
    \appendix
        \part{Appendices}
            \input{8 - Hilbert complexes/main.tex}
            \input{9 - weak conservation proofs/main.tex}
\end{document}

\end{document}


\title{\BA{Title in Progress...}}
\author{Boris Andrews}
\affil{Mathematical Institute, University of Oxford}
\date{\today}


\begin{document}
    \pagenumbering{gobble}
    \maketitle
    
    
    \begin{abstract}
        Magnetic confinement reactors---in particular tokamaks---offer one of the most promising options for achieving practical nuclear fusion, with the potential to provide virtually limitless, clean energy. The theoretical and numerical modeling of tokamak plasmas is simultaneously an essential component of effective reactor design, and a great research barrier. Tokamak operational conditions exhibit comparatively low Knudsen numbers. Kinetic effects, including kinetic waves and instabilities, Landau damping, bump-on-tail instabilities and more, are therefore highly influential in tokamak plasma dynamics. Purely fluid models are inherently incapable of capturing these effects, whereas the high dimensionality in purely kinetic models render them practically intractable for most relevant purposes.

        We consider a $\delta\!f$ decomposition model, with a macroscopic fluid background and microscopic kinetic correction, both fully coupled to each other. A similar manner of discretization is proposed to that used in the recent \texttt{STRUPHY} code \cite{Holderied_Possanner_Wang_2021, Holderied_2022, Li_et_al_2023} with a finite-element model for the background and a pseudo-particle/PiC model for the correction.

        The fluid background satisfies the full, non-linear, resistive, compressible, Hall MHD equations. \cite{Laakmann_Hu_Farrell_2022} introduces finite-element(-in-space) implicit timesteppers for the incompressible analogue to this system with structure-preserving (SP) properties in the ideal case, alongside parameter-robust preconditioners. We show that these timesteppers can derive from a finite-element-in-time (FET) (and finite-element-in-space) interpretation. The benefits of this reformulation are discussed, including the derivation of timesteppers that are higher order in time, and the quantifiable dissipative SP properties in the non-ideal, resistive case.
        
        We discuss possible options for extending this FET approach to timesteppers for the compressible case.

        The kinetic corrections satisfy linearized Boltzmann equations. Using a Lénard--Bernstein collision operator, these take Fokker--Planck-like forms \cite{Fokker_1914, Planck_1917} wherein pseudo-particles in the numerical model obey the neoclassical transport equations, with particle-independent Brownian drift terms. This offers a rigorous methodology for incorporating collisions into the particle transport model, without coupling the equations of motions for each particle.
        
        Works by Chen, Chacón et al. \cite{Chen_Chacón_Barnes_2011, Chacón_Chen_Barnes_2013, Chen_Chacón_2014, Chen_Chacón_2015} have developed structure-preserving particle pushers for neoclassical transport in the Vlasov equations, derived from Crank--Nicolson integrators. We show these too can can derive from a FET interpretation, similarly offering potential extensions to higher-order-in-time particle pushers. The FET formulation is used also to consider how the stochastic drift terms can be incorporated into the pushers. Stochastic gyrokinetic expansions are also discussed.

        Different options for the numerical implementation of these schemes are considered.

        Due to the efficacy of FET in the development of SP timesteppers for both the fluid and kinetic component, we hope this approach will prove effective in the future for developing SP timesteppers for the full hybrid model. We hope this will give us the opportunity to incorporate previously inaccessible kinetic effects into the highly effective, modern, finite-element MHD models.
    \end{abstract}
    
    
    \newpage
    \tableofcontents
    
    
    \newpage
    \pagenumbering{arabic}
    %\linenumbers\renewcommand\thelinenumber{\color{black!50}\arabic{linenumber}}
            \documentclass[12pt, a4paper]{report}

\documentclass[12pt, a4paper]{report}

\input{template/main.tex}

\title{\BA{Title in Progress...}}
\author{Boris Andrews}
\affil{Mathematical Institute, University of Oxford}
\date{\today}


\begin{document}
    \pagenumbering{gobble}
    \maketitle
    
    
    \begin{abstract}
        Magnetic confinement reactors---in particular tokamaks---offer one of the most promising options for achieving practical nuclear fusion, with the potential to provide virtually limitless, clean energy. The theoretical and numerical modeling of tokamak plasmas is simultaneously an essential component of effective reactor design, and a great research barrier. Tokamak operational conditions exhibit comparatively low Knudsen numbers. Kinetic effects, including kinetic waves and instabilities, Landau damping, bump-on-tail instabilities and more, are therefore highly influential in tokamak plasma dynamics. Purely fluid models are inherently incapable of capturing these effects, whereas the high dimensionality in purely kinetic models render them practically intractable for most relevant purposes.

        We consider a $\delta\!f$ decomposition model, with a macroscopic fluid background and microscopic kinetic correction, both fully coupled to each other. A similar manner of discretization is proposed to that used in the recent \texttt{STRUPHY} code \cite{Holderied_Possanner_Wang_2021, Holderied_2022, Li_et_al_2023} with a finite-element model for the background and a pseudo-particle/PiC model for the correction.

        The fluid background satisfies the full, non-linear, resistive, compressible, Hall MHD equations. \cite{Laakmann_Hu_Farrell_2022} introduces finite-element(-in-space) implicit timesteppers for the incompressible analogue to this system with structure-preserving (SP) properties in the ideal case, alongside parameter-robust preconditioners. We show that these timesteppers can derive from a finite-element-in-time (FET) (and finite-element-in-space) interpretation. The benefits of this reformulation are discussed, including the derivation of timesteppers that are higher order in time, and the quantifiable dissipative SP properties in the non-ideal, resistive case.
        
        We discuss possible options for extending this FET approach to timesteppers for the compressible case.

        The kinetic corrections satisfy linearized Boltzmann equations. Using a Lénard--Bernstein collision operator, these take Fokker--Planck-like forms \cite{Fokker_1914, Planck_1917} wherein pseudo-particles in the numerical model obey the neoclassical transport equations, with particle-independent Brownian drift terms. This offers a rigorous methodology for incorporating collisions into the particle transport model, without coupling the equations of motions for each particle.
        
        Works by Chen, Chacón et al. \cite{Chen_Chacón_Barnes_2011, Chacón_Chen_Barnes_2013, Chen_Chacón_2014, Chen_Chacón_2015} have developed structure-preserving particle pushers for neoclassical transport in the Vlasov equations, derived from Crank--Nicolson integrators. We show these too can can derive from a FET interpretation, similarly offering potential extensions to higher-order-in-time particle pushers. The FET formulation is used also to consider how the stochastic drift terms can be incorporated into the pushers. Stochastic gyrokinetic expansions are also discussed.

        Different options for the numerical implementation of these schemes are considered.

        Due to the efficacy of FET in the development of SP timesteppers for both the fluid and kinetic component, we hope this approach will prove effective in the future for developing SP timesteppers for the full hybrid model. We hope this will give us the opportunity to incorporate previously inaccessible kinetic effects into the highly effective, modern, finite-element MHD models.
    \end{abstract}
    
    
    \newpage
    \tableofcontents
    
    
    \newpage
    \pagenumbering{arabic}
    %\linenumbers\renewcommand\thelinenumber{\color{black!50}\arabic{linenumber}}
            \input{0 - introduction/main.tex}
        \part{Research}
            \input{1 - low-noise PiC models/main.tex}
            \input{2 - kinetic component/main.tex}
            \input{3 - fluid component/main.tex}
            \input{4 - numerical implementation/main.tex}
        \part{Project Overview}
            \input{5 - research plan/main.tex}
            \input{6 - summary/main.tex}
    
    
    %\section{}
    \newpage
    \pagenumbering{gobble}
        \printbibliography


    \newpage
    \pagenumbering{roman}
    \appendix
        \part{Appendices}
            \input{8 - Hilbert complexes/main.tex}
            \input{9 - weak conservation proofs/main.tex}
\end{document}


\title{\BA{Title in Progress...}}
\author{Boris Andrews}
\affil{Mathematical Institute, University of Oxford}
\date{\today}


\begin{document}
    \pagenumbering{gobble}
    \maketitle
    
    
    \begin{abstract}
        Magnetic confinement reactors---in particular tokamaks---offer one of the most promising options for achieving practical nuclear fusion, with the potential to provide virtually limitless, clean energy. The theoretical and numerical modeling of tokamak plasmas is simultaneously an essential component of effective reactor design, and a great research barrier. Tokamak operational conditions exhibit comparatively low Knudsen numbers. Kinetic effects, including kinetic waves and instabilities, Landau damping, bump-on-tail instabilities and more, are therefore highly influential in tokamak plasma dynamics. Purely fluid models are inherently incapable of capturing these effects, whereas the high dimensionality in purely kinetic models render them practically intractable for most relevant purposes.

        We consider a $\delta\!f$ decomposition model, with a macroscopic fluid background and microscopic kinetic correction, both fully coupled to each other. A similar manner of discretization is proposed to that used in the recent \texttt{STRUPHY} code \cite{Holderied_Possanner_Wang_2021, Holderied_2022, Li_et_al_2023} with a finite-element model for the background and a pseudo-particle/PiC model for the correction.

        The fluid background satisfies the full, non-linear, resistive, compressible, Hall MHD equations. \cite{Laakmann_Hu_Farrell_2022} introduces finite-element(-in-space) implicit timesteppers for the incompressible analogue to this system with structure-preserving (SP) properties in the ideal case, alongside parameter-robust preconditioners. We show that these timesteppers can derive from a finite-element-in-time (FET) (and finite-element-in-space) interpretation. The benefits of this reformulation are discussed, including the derivation of timesteppers that are higher order in time, and the quantifiable dissipative SP properties in the non-ideal, resistive case.
        
        We discuss possible options for extending this FET approach to timesteppers for the compressible case.

        The kinetic corrections satisfy linearized Boltzmann equations. Using a Lénard--Bernstein collision operator, these take Fokker--Planck-like forms \cite{Fokker_1914, Planck_1917} wherein pseudo-particles in the numerical model obey the neoclassical transport equations, with particle-independent Brownian drift terms. This offers a rigorous methodology for incorporating collisions into the particle transport model, without coupling the equations of motions for each particle.
        
        Works by Chen, Chacón et al. \cite{Chen_Chacón_Barnes_2011, Chacón_Chen_Barnes_2013, Chen_Chacón_2014, Chen_Chacón_2015} have developed structure-preserving particle pushers for neoclassical transport in the Vlasov equations, derived from Crank--Nicolson integrators. We show these too can can derive from a FET interpretation, similarly offering potential extensions to higher-order-in-time particle pushers. The FET formulation is used also to consider how the stochastic drift terms can be incorporated into the pushers. Stochastic gyrokinetic expansions are also discussed.

        Different options for the numerical implementation of these schemes are considered.

        Due to the efficacy of FET in the development of SP timesteppers for both the fluid and kinetic component, we hope this approach will prove effective in the future for developing SP timesteppers for the full hybrid model. We hope this will give us the opportunity to incorporate previously inaccessible kinetic effects into the highly effective, modern, finite-element MHD models.
    \end{abstract}
    
    
    \newpage
    \tableofcontents
    
    
    \newpage
    \pagenumbering{arabic}
    %\linenumbers\renewcommand\thelinenumber{\color{black!50}\arabic{linenumber}}
            \documentclass[12pt, a4paper]{report}

\input{template/main.tex}

\title{\BA{Title in Progress...}}
\author{Boris Andrews}
\affil{Mathematical Institute, University of Oxford}
\date{\today}


\begin{document}
    \pagenumbering{gobble}
    \maketitle
    
    
    \begin{abstract}
        Magnetic confinement reactors---in particular tokamaks---offer one of the most promising options for achieving practical nuclear fusion, with the potential to provide virtually limitless, clean energy. The theoretical and numerical modeling of tokamak plasmas is simultaneously an essential component of effective reactor design, and a great research barrier. Tokamak operational conditions exhibit comparatively low Knudsen numbers. Kinetic effects, including kinetic waves and instabilities, Landau damping, bump-on-tail instabilities and more, are therefore highly influential in tokamak plasma dynamics. Purely fluid models are inherently incapable of capturing these effects, whereas the high dimensionality in purely kinetic models render them practically intractable for most relevant purposes.

        We consider a $\delta\!f$ decomposition model, with a macroscopic fluid background and microscopic kinetic correction, both fully coupled to each other. A similar manner of discretization is proposed to that used in the recent \texttt{STRUPHY} code \cite{Holderied_Possanner_Wang_2021, Holderied_2022, Li_et_al_2023} with a finite-element model for the background and a pseudo-particle/PiC model for the correction.

        The fluid background satisfies the full, non-linear, resistive, compressible, Hall MHD equations. \cite{Laakmann_Hu_Farrell_2022} introduces finite-element(-in-space) implicit timesteppers for the incompressible analogue to this system with structure-preserving (SP) properties in the ideal case, alongside parameter-robust preconditioners. We show that these timesteppers can derive from a finite-element-in-time (FET) (and finite-element-in-space) interpretation. The benefits of this reformulation are discussed, including the derivation of timesteppers that are higher order in time, and the quantifiable dissipative SP properties in the non-ideal, resistive case.
        
        We discuss possible options for extending this FET approach to timesteppers for the compressible case.

        The kinetic corrections satisfy linearized Boltzmann equations. Using a Lénard--Bernstein collision operator, these take Fokker--Planck-like forms \cite{Fokker_1914, Planck_1917} wherein pseudo-particles in the numerical model obey the neoclassical transport equations, with particle-independent Brownian drift terms. This offers a rigorous methodology for incorporating collisions into the particle transport model, without coupling the equations of motions for each particle.
        
        Works by Chen, Chacón et al. \cite{Chen_Chacón_Barnes_2011, Chacón_Chen_Barnes_2013, Chen_Chacón_2014, Chen_Chacón_2015} have developed structure-preserving particle pushers for neoclassical transport in the Vlasov equations, derived from Crank--Nicolson integrators. We show these too can can derive from a FET interpretation, similarly offering potential extensions to higher-order-in-time particle pushers. The FET formulation is used also to consider how the stochastic drift terms can be incorporated into the pushers. Stochastic gyrokinetic expansions are also discussed.

        Different options for the numerical implementation of these schemes are considered.

        Due to the efficacy of FET in the development of SP timesteppers for both the fluid and kinetic component, we hope this approach will prove effective in the future for developing SP timesteppers for the full hybrid model. We hope this will give us the opportunity to incorporate previously inaccessible kinetic effects into the highly effective, modern, finite-element MHD models.
    \end{abstract}
    
    
    \newpage
    \tableofcontents
    
    
    \newpage
    \pagenumbering{arabic}
    %\linenumbers\renewcommand\thelinenumber{\color{black!50}\arabic{linenumber}}
            \input{0 - introduction/main.tex}
        \part{Research}
            \input{1 - low-noise PiC models/main.tex}
            \input{2 - kinetic component/main.tex}
            \input{3 - fluid component/main.tex}
            \input{4 - numerical implementation/main.tex}
        \part{Project Overview}
            \input{5 - research plan/main.tex}
            \input{6 - summary/main.tex}
    
    
    %\section{}
    \newpage
    \pagenumbering{gobble}
        \printbibliography


    \newpage
    \pagenumbering{roman}
    \appendix
        \part{Appendices}
            \input{8 - Hilbert complexes/main.tex}
            \input{9 - weak conservation proofs/main.tex}
\end{document}

        \part{Research}
            \documentclass[12pt, a4paper]{report}

\input{template/main.tex}

\title{\BA{Title in Progress...}}
\author{Boris Andrews}
\affil{Mathematical Institute, University of Oxford}
\date{\today}


\begin{document}
    \pagenumbering{gobble}
    \maketitle
    
    
    \begin{abstract}
        Magnetic confinement reactors---in particular tokamaks---offer one of the most promising options for achieving practical nuclear fusion, with the potential to provide virtually limitless, clean energy. The theoretical and numerical modeling of tokamak plasmas is simultaneously an essential component of effective reactor design, and a great research barrier. Tokamak operational conditions exhibit comparatively low Knudsen numbers. Kinetic effects, including kinetic waves and instabilities, Landau damping, bump-on-tail instabilities and more, are therefore highly influential in tokamak plasma dynamics. Purely fluid models are inherently incapable of capturing these effects, whereas the high dimensionality in purely kinetic models render them practically intractable for most relevant purposes.

        We consider a $\delta\!f$ decomposition model, with a macroscopic fluid background and microscopic kinetic correction, both fully coupled to each other. A similar manner of discretization is proposed to that used in the recent \texttt{STRUPHY} code \cite{Holderied_Possanner_Wang_2021, Holderied_2022, Li_et_al_2023} with a finite-element model for the background and a pseudo-particle/PiC model for the correction.

        The fluid background satisfies the full, non-linear, resistive, compressible, Hall MHD equations. \cite{Laakmann_Hu_Farrell_2022} introduces finite-element(-in-space) implicit timesteppers for the incompressible analogue to this system with structure-preserving (SP) properties in the ideal case, alongside parameter-robust preconditioners. We show that these timesteppers can derive from a finite-element-in-time (FET) (and finite-element-in-space) interpretation. The benefits of this reformulation are discussed, including the derivation of timesteppers that are higher order in time, and the quantifiable dissipative SP properties in the non-ideal, resistive case.
        
        We discuss possible options for extending this FET approach to timesteppers for the compressible case.

        The kinetic corrections satisfy linearized Boltzmann equations. Using a Lénard--Bernstein collision operator, these take Fokker--Planck-like forms \cite{Fokker_1914, Planck_1917} wherein pseudo-particles in the numerical model obey the neoclassical transport equations, with particle-independent Brownian drift terms. This offers a rigorous methodology for incorporating collisions into the particle transport model, without coupling the equations of motions for each particle.
        
        Works by Chen, Chacón et al. \cite{Chen_Chacón_Barnes_2011, Chacón_Chen_Barnes_2013, Chen_Chacón_2014, Chen_Chacón_2015} have developed structure-preserving particle pushers for neoclassical transport in the Vlasov equations, derived from Crank--Nicolson integrators. We show these too can can derive from a FET interpretation, similarly offering potential extensions to higher-order-in-time particle pushers. The FET formulation is used also to consider how the stochastic drift terms can be incorporated into the pushers. Stochastic gyrokinetic expansions are also discussed.

        Different options for the numerical implementation of these schemes are considered.

        Due to the efficacy of FET in the development of SP timesteppers for both the fluid and kinetic component, we hope this approach will prove effective in the future for developing SP timesteppers for the full hybrid model. We hope this will give us the opportunity to incorporate previously inaccessible kinetic effects into the highly effective, modern, finite-element MHD models.
    \end{abstract}
    
    
    \newpage
    \tableofcontents
    
    
    \newpage
    \pagenumbering{arabic}
    %\linenumbers\renewcommand\thelinenumber{\color{black!50}\arabic{linenumber}}
            \input{0 - introduction/main.tex}
        \part{Research}
            \input{1 - low-noise PiC models/main.tex}
            \input{2 - kinetic component/main.tex}
            \input{3 - fluid component/main.tex}
            \input{4 - numerical implementation/main.tex}
        \part{Project Overview}
            \input{5 - research plan/main.tex}
            \input{6 - summary/main.tex}
    
    
    %\section{}
    \newpage
    \pagenumbering{gobble}
        \printbibliography


    \newpage
    \pagenumbering{roman}
    \appendix
        \part{Appendices}
            \input{8 - Hilbert complexes/main.tex}
            \input{9 - weak conservation proofs/main.tex}
\end{document}

            \documentclass[12pt, a4paper]{report}

\input{template/main.tex}

\title{\BA{Title in Progress...}}
\author{Boris Andrews}
\affil{Mathematical Institute, University of Oxford}
\date{\today}


\begin{document}
    \pagenumbering{gobble}
    \maketitle
    
    
    \begin{abstract}
        Magnetic confinement reactors---in particular tokamaks---offer one of the most promising options for achieving practical nuclear fusion, with the potential to provide virtually limitless, clean energy. The theoretical and numerical modeling of tokamak plasmas is simultaneously an essential component of effective reactor design, and a great research barrier. Tokamak operational conditions exhibit comparatively low Knudsen numbers. Kinetic effects, including kinetic waves and instabilities, Landau damping, bump-on-tail instabilities and more, are therefore highly influential in tokamak plasma dynamics. Purely fluid models are inherently incapable of capturing these effects, whereas the high dimensionality in purely kinetic models render them practically intractable for most relevant purposes.

        We consider a $\delta\!f$ decomposition model, with a macroscopic fluid background and microscopic kinetic correction, both fully coupled to each other. A similar manner of discretization is proposed to that used in the recent \texttt{STRUPHY} code \cite{Holderied_Possanner_Wang_2021, Holderied_2022, Li_et_al_2023} with a finite-element model for the background and a pseudo-particle/PiC model for the correction.

        The fluid background satisfies the full, non-linear, resistive, compressible, Hall MHD equations. \cite{Laakmann_Hu_Farrell_2022} introduces finite-element(-in-space) implicit timesteppers for the incompressible analogue to this system with structure-preserving (SP) properties in the ideal case, alongside parameter-robust preconditioners. We show that these timesteppers can derive from a finite-element-in-time (FET) (and finite-element-in-space) interpretation. The benefits of this reformulation are discussed, including the derivation of timesteppers that are higher order in time, and the quantifiable dissipative SP properties in the non-ideal, resistive case.
        
        We discuss possible options for extending this FET approach to timesteppers for the compressible case.

        The kinetic corrections satisfy linearized Boltzmann equations. Using a Lénard--Bernstein collision operator, these take Fokker--Planck-like forms \cite{Fokker_1914, Planck_1917} wherein pseudo-particles in the numerical model obey the neoclassical transport equations, with particle-independent Brownian drift terms. This offers a rigorous methodology for incorporating collisions into the particle transport model, without coupling the equations of motions for each particle.
        
        Works by Chen, Chacón et al. \cite{Chen_Chacón_Barnes_2011, Chacón_Chen_Barnes_2013, Chen_Chacón_2014, Chen_Chacón_2015} have developed structure-preserving particle pushers for neoclassical transport in the Vlasov equations, derived from Crank--Nicolson integrators. We show these too can can derive from a FET interpretation, similarly offering potential extensions to higher-order-in-time particle pushers. The FET formulation is used also to consider how the stochastic drift terms can be incorporated into the pushers. Stochastic gyrokinetic expansions are also discussed.

        Different options for the numerical implementation of these schemes are considered.

        Due to the efficacy of FET in the development of SP timesteppers for both the fluid and kinetic component, we hope this approach will prove effective in the future for developing SP timesteppers for the full hybrid model. We hope this will give us the opportunity to incorporate previously inaccessible kinetic effects into the highly effective, modern, finite-element MHD models.
    \end{abstract}
    
    
    \newpage
    \tableofcontents
    
    
    \newpage
    \pagenumbering{arabic}
    %\linenumbers\renewcommand\thelinenumber{\color{black!50}\arabic{linenumber}}
            \input{0 - introduction/main.tex}
        \part{Research}
            \input{1 - low-noise PiC models/main.tex}
            \input{2 - kinetic component/main.tex}
            \input{3 - fluid component/main.tex}
            \input{4 - numerical implementation/main.tex}
        \part{Project Overview}
            \input{5 - research plan/main.tex}
            \input{6 - summary/main.tex}
    
    
    %\section{}
    \newpage
    \pagenumbering{gobble}
        \printbibliography


    \newpage
    \pagenumbering{roman}
    \appendix
        \part{Appendices}
            \input{8 - Hilbert complexes/main.tex}
            \input{9 - weak conservation proofs/main.tex}
\end{document}

            \documentclass[12pt, a4paper]{report}

\input{template/main.tex}

\title{\BA{Title in Progress...}}
\author{Boris Andrews}
\affil{Mathematical Institute, University of Oxford}
\date{\today}


\begin{document}
    \pagenumbering{gobble}
    \maketitle
    
    
    \begin{abstract}
        Magnetic confinement reactors---in particular tokamaks---offer one of the most promising options for achieving practical nuclear fusion, with the potential to provide virtually limitless, clean energy. The theoretical and numerical modeling of tokamak plasmas is simultaneously an essential component of effective reactor design, and a great research barrier. Tokamak operational conditions exhibit comparatively low Knudsen numbers. Kinetic effects, including kinetic waves and instabilities, Landau damping, bump-on-tail instabilities and more, are therefore highly influential in tokamak plasma dynamics. Purely fluid models are inherently incapable of capturing these effects, whereas the high dimensionality in purely kinetic models render them practically intractable for most relevant purposes.

        We consider a $\delta\!f$ decomposition model, with a macroscopic fluid background and microscopic kinetic correction, both fully coupled to each other. A similar manner of discretization is proposed to that used in the recent \texttt{STRUPHY} code \cite{Holderied_Possanner_Wang_2021, Holderied_2022, Li_et_al_2023} with a finite-element model for the background and a pseudo-particle/PiC model for the correction.

        The fluid background satisfies the full, non-linear, resistive, compressible, Hall MHD equations. \cite{Laakmann_Hu_Farrell_2022} introduces finite-element(-in-space) implicit timesteppers for the incompressible analogue to this system with structure-preserving (SP) properties in the ideal case, alongside parameter-robust preconditioners. We show that these timesteppers can derive from a finite-element-in-time (FET) (and finite-element-in-space) interpretation. The benefits of this reformulation are discussed, including the derivation of timesteppers that are higher order in time, and the quantifiable dissipative SP properties in the non-ideal, resistive case.
        
        We discuss possible options for extending this FET approach to timesteppers for the compressible case.

        The kinetic corrections satisfy linearized Boltzmann equations. Using a Lénard--Bernstein collision operator, these take Fokker--Planck-like forms \cite{Fokker_1914, Planck_1917} wherein pseudo-particles in the numerical model obey the neoclassical transport equations, with particle-independent Brownian drift terms. This offers a rigorous methodology for incorporating collisions into the particle transport model, without coupling the equations of motions for each particle.
        
        Works by Chen, Chacón et al. \cite{Chen_Chacón_Barnes_2011, Chacón_Chen_Barnes_2013, Chen_Chacón_2014, Chen_Chacón_2015} have developed structure-preserving particle pushers for neoclassical transport in the Vlasov equations, derived from Crank--Nicolson integrators. We show these too can can derive from a FET interpretation, similarly offering potential extensions to higher-order-in-time particle pushers. The FET formulation is used also to consider how the stochastic drift terms can be incorporated into the pushers. Stochastic gyrokinetic expansions are also discussed.

        Different options for the numerical implementation of these schemes are considered.

        Due to the efficacy of FET in the development of SP timesteppers for both the fluid and kinetic component, we hope this approach will prove effective in the future for developing SP timesteppers for the full hybrid model. We hope this will give us the opportunity to incorporate previously inaccessible kinetic effects into the highly effective, modern, finite-element MHD models.
    \end{abstract}
    
    
    \newpage
    \tableofcontents
    
    
    \newpage
    \pagenumbering{arabic}
    %\linenumbers\renewcommand\thelinenumber{\color{black!50}\arabic{linenumber}}
            \input{0 - introduction/main.tex}
        \part{Research}
            \input{1 - low-noise PiC models/main.tex}
            \input{2 - kinetic component/main.tex}
            \input{3 - fluid component/main.tex}
            \input{4 - numerical implementation/main.tex}
        \part{Project Overview}
            \input{5 - research plan/main.tex}
            \input{6 - summary/main.tex}
    
    
    %\section{}
    \newpage
    \pagenumbering{gobble}
        \printbibliography


    \newpage
    \pagenumbering{roman}
    \appendix
        \part{Appendices}
            \input{8 - Hilbert complexes/main.tex}
            \input{9 - weak conservation proofs/main.tex}
\end{document}

            \documentclass[12pt, a4paper]{report}

\input{template/main.tex}

\title{\BA{Title in Progress...}}
\author{Boris Andrews}
\affil{Mathematical Institute, University of Oxford}
\date{\today}


\begin{document}
    \pagenumbering{gobble}
    \maketitle
    
    
    \begin{abstract}
        Magnetic confinement reactors---in particular tokamaks---offer one of the most promising options for achieving practical nuclear fusion, with the potential to provide virtually limitless, clean energy. The theoretical and numerical modeling of tokamak plasmas is simultaneously an essential component of effective reactor design, and a great research barrier. Tokamak operational conditions exhibit comparatively low Knudsen numbers. Kinetic effects, including kinetic waves and instabilities, Landau damping, bump-on-tail instabilities and more, are therefore highly influential in tokamak plasma dynamics. Purely fluid models are inherently incapable of capturing these effects, whereas the high dimensionality in purely kinetic models render them practically intractable for most relevant purposes.

        We consider a $\delta\!f$ decomposition model, with a macroscopic fluid background and microscopic kinetic correction, both fully coupled to each other. A similar manner of discretization is proposed to that used in the recent \texttt{STRUPHY} code \cite{Holderied_Possanner_Wang_2021, Holderied_2022, Li_et_al_2023} with a finite-element model for the background and a pseudo-particle/PiC model for the correction.

        The fluid background satisfies the full, non-linear, resistive, compressible, Hall MHD equations. \cite{Laakmann_Hu_Farrell_2022} introduces finite-element(-in-space) implicit timesteppers for the incompressible analogue to this system with structure-preserving (SP) properties in the ideal case, alongside parameter-robust preconditioners. We show that these timesteppers can derive from a finite-element-in-time (FET) (and finite-element-in-space) interpretation. The benefits of this reformulation are discussed, including the derivation of timesteppers that are higher order in time, and the quantifiable dissipative SP properties in the non-ideal, resistive case.
        
        We discuss possible options for extending this FET approach to timesteppers for the compressible case.

        The kinetic corrections satisfy linearized Boltzmann equations. Using a Lénard--Bernstein collision operator, these take Fokker--Planck-like forms \cite{Fokker_1914, Planck_1917} wherein pseudo-particles in the numerical model obey the neoclassical transport equations, with particle-independent Brownian drift terms. This offers a rigorous methodology for incorporating collisions into the particle transport model, without coupling the equations of motions for each particle.
        
        Works by Chen, Chacón et al. \cite{Chen_Chacón_Barnes_2011, Chacón_Chen_Barnes_2013, Chen_Chacón_2014, Chen_Chacón_2015} have developed structure-preserving particle pushers for neoclassical transport in the Vlasov equations, derived from Crank--Nicolson integrators. We show these too can can derive from a FET interpretation, similarly offering potential extensions to higher-order-in-time particle pushers. The FET formulation is used also to consider how the stochastic drift terms can be incorporated into the pushers. Stochastic gyrokinetic expansions are also discussed.

        Different options for the numerical implementation of these schemes are considered.

        Due to the efficacy of FET in the development of SP timesteppers for both the fluid and kinetic component, we hope this approach will prove effective in the future for developing SP timesteppers for the full hybrid model. We hope this will give us the opportunity to incorporate previously inaccessible kinetic effects into the highly effective, modern, finite-element MHD models.
    \end{abstract}
    
    
    \newpage
    \tableofcontents
    
    
    \newpage
    \pagenumbering{arabic}
    %\linenumbers\renewcommand\thelinenumber{\color{black!50}\arabic{linenumber}}
            \input{0 - introduction/main.tex}
        \part{Research}
            \input{1 - low-noise PiC models/main.tex}
            \input{2 - kinetic component/main.tex}
            \input{3 - fluid component/main.tex}
            \input{4 - numerical implementation/main.tex}
        \part{Project Overview}
            \input{5 - research plan/main.tex}
            \input{6 - summary/main.tex}
    
    
    %\section{}
    \newpage
    \pagenumbering{gobble}
        \printbibliography


    \newpage
    \pagenumbering{roman}
    \appendix
        \part{Appendices}
            \input{8 - Hilbert complexes/main.tex}
            \input{9 - weak conservation proofs/main.tex}
\end{document}

        \part{Project Overview}
            \documentclass[12pt, a4paper]{report}

\input{template/main.tex}

\title{\BA{Title in Progress...}}
\author{Boris Andrews}
\affil{Mathematical Institute, University of Oxford}
\date{\today}


\begin{document}
    \pagenumbering{gobble}
    \maketitle
    
    
    \begin{abstract}
        Magnetic confinement reactors---in particular tokamaks---offer one of the most promising options for achieving practical nuclear fusion, with the potential to provide virtually limitless, clean energy. The theoretical and numerical modeling of tokamak plasmas is simultaneously an essential component of effective reactor design, and a great research barrier. Tokamak operational conditions exhibit comparatively low Knudsen numbers. Kinetic effects, including kinetic waves and instabilities, Landau damping, bump-on-tail instabilities and more, are therefore highly influential in tokamak plasma dynamics. Purely fluid models are inherently incapable of capturing these effects, whereas the high dimensionality in purely kinetic models render them practically intractable for most relevant purposes.

        We consider a $\delta\!f$ decomposition model, with a macroscopic fluid background and microscopic kinetic correction, both fully coupled to each other. A similar manner of discretization is proposed to that used in the recent \texttt{STRUPHY} code \cite{Holderied_Possanner_Wang_2021, Holderied_2022, Li_et_al_2023} with a finite-element model for the background and a pseudo-particle/PiC model for the correction.

        The fluid background satisfies the full, non-linear, resistive, compressible, Hall MHD equations. \cite{Laakmann_Hu_Farrell_2022} introduces finite-element(-in-space) implicit timesteppers for the incompressible analogue to this system with structure-preserving (SP) properties in the ideal case, alongside parameter-robust preconditioners. We show that these timesteppers can derive from a finite-element-in-time (FET) (and finite-element-in-space) interpretation. The benefits of this reformulation are discussed, including the derivation of timesteppers that are higher order in time, and the quantifiable dissipative SP properties in the non-ideal, resistive case.
        
        We discuss possible options for extending this FET approach to timesteppers for the compressible case.

        The kinetic corrections satisfy linearized Boltzmann equations. Using a Lénard--Bernstein collision operator, these take Fokker--Planck-like forms \cite{Fokker_1914, Planck_1917} wherein pseudo-particles in the numerical model obey the neoclassical transport equations, with particle-independent Brownian drift terms. This offers a rigorous methodology for incorporating collisions into the particle transport model, without coupling the equations of motions for each particle.
        
        Works by Chen, Chacón et al. \cite{Chen_Chacón_Barnes_2011, Chacón_Chen_Barnes_2013, Chen_Chacón_2014, Chen_Chacón_2015} have developed structure-preserving particle pushers for neoclassical transport in the Vlasov equations, derived from Crank--Nicolson integrators. We show these too can can derive from a FET interpretation, similarly offering potential extensions to higher-order-in-time particle pushers. The FET formulation is used also to consider how the stochastic drift terms can be incorporated into the pushers. Stochastic gyrokinetic expansions are also discussed.

        Different options for the numerical implementation of these schemes are considered.

        Due to the efficacy of FET in the development of SP timesteppers for both the fluid and kinetic component, we hope this approach will prove effective in the future for developing SP timesteppers for the full hybrid model. We hope this will give us the opportunity to incorporate previously inaccessible kinetic effects into the highly effective, modern, finite-element MHD models.
    \end{abstract}
    
    
    \newpage
    \tableofcontents
    
    
    \newpage
    \pagenumbering{arabic}
    %\linenumbers\renewcommand\thelinenumber{\color{black!50}\arabic{linenumber}}
            \input{0 - introduction/main.tex}
        \part{Research}
            \input{1 - low-noise PiC models/main.tex}
            \input{2 - kinetic component/main.tex}
            \input{3 - fluid component/main.tex}
            \input{4 - numerical implementation/main.tex}
        \part{Project Overview}
            \input{5 - research plan/main.tex}
            \input{6 - summary/main.tex}
    
    
    %\section{}
    \newpage
    \pagenumbering{gobble}
        \printbibliography


    \newpage
    \pagenumbering{roman}
    \appendix
        \part{Appendices}
            \input{8 - Hilbert complexes/main.tex}
            \input{9 - weak conservation proofs/main.tex}
\end{document}

            \documentclass[12pt, a4paper]{report}

\input{template/main.tex}

\title{\BA{Title in Progress...}}
\author{Boris Andrews}
\affil{Mathematical Institute, University of Oxford}
\date{\today}


\begin{document}
    \pagenumbering{gobble}
    \maketitle
    
    
    \begin{abstract}
        Magnetic confinement reactors---in particular tokamaks---offer one of the most promising options for achieving practical nuclear fusion, with the potential to provide virtually limitless, clean energy. The theoretical and numerical modeling of tokamak plasmas is simultaneously an essential component of effective reactor design, and a great research barrier. Tokamak operational conditions exhibit comparatively low Knudsen numbers. Kinetic effects, including kinetic waves and instabilities, Landau damping, bump-on-tail instabilities and more, are therefore highly influential in tokamak plasma dynamics. Purely fluid models are inherently incapable of capturing these effects, whereas the high dimensionality in purely kinetic models render them practically intractable for most relevant purposes.

        We consider a $\delta\!f$ decomposition model, with a macroscopic fluid background and microscopic kinetic correction, both fully coupled to each other. A similar manner of discretization is proposed to that used in the recent \texttt{STRUPHY} code \cite{Holderied_Possanner_Wang_2021, Holderied_2022, Li_et_al_2023} with a finite-element model for the background and a pseudo-particle/PiC model for the correction.

        The fluid background satisfies the full, non-linear, resistive, compressible, Hall MHD equations. \cite{Laakmann_Hu_Farrell_2022} introduces finite-element(-in-space) implicit timesteppers for the incompressible analogue to this system with structure-preserving (SP) properties in the ideal case, alongside parameter-robust preconditioners. We show that these timesteppers can derive from a finite-element-in-time (FET) (and finite-element-in-space) interpretation. The benefits of this reformulation are discussed, including the derivation of timesteppers that are higher order in time, and the quantifiable dissipative SP properties in the non-ideal, resistive case.
        
        We discuss possible options for extending this FET approach to timesteppers for the compressible case.

        The kinetic corrections satisfy linearized Boltzmann equations. Using a Lénard--Bernstein collision operator, these take Fokker--Planck-like forms \cite{Fokker_1914, Planck_1917} wherein pseudo-particles in the numerical model obey the neoclassical transport equations, with particle-independent Brownian drift terms. This offers a rigorous methodology for incorporating collisions into the particle transport model, without coupling the equations of motions for each particle.
        
        Works by Chen, Chacón et al. \cite{Chen_Chacón_Barnes_2011, Chacón_Chen_Barnes_2013, Chen_Chacón_2014, Chen_Chacón_2015} have developed structure-preserving particle pushers for neoclassical transport in the Vlasov equations, derived from Crank--Nicolson integrators. We show these too can can derive from a FET interpretation, similarly offering potential extensions to higher-order-in-time particle pushers. The FET formulation is used also to consider how the stochastic drift terms can be incorporated into the pushers. Stochastic gyrokinetic expansions are also discussed.

        Different options for the numerical implementation of these schemes are considered.

        Due to the efficacy of FET in the development of SP timesteppers for both the fluid and kinetic component, we hope this approach will prove effective in the future for developing SP timesteppers for the full hybrid model. We hope this will give us the opportunity to incorporate previously inaccessible kinetic effects into the highly effective, modern, finite-element MHD models.
    \end{abstract}
    
    
    \newpage
    \tableofcontents
    
    
    \newpage
    \pagenumbering{arabic}
    %\linenumbers\renewcommand\thelinenumber{\color{black!50}\arabic{linenumber}}
            \input{0 - introduction/main.tex}
        \part{Research}
            \input{1 - low-noise PiC models/main.tex}
            \input{2 - kinetic component/main.tex}
            \input{3 - fluid component/main.tex}
            \input{4 - numerical implementation/main.tex}
        \part{Project Overview}
            \input{5 - research plan/main.tex}
            \input{6 - summary/main.tex}
    
    
    %\section{}
    \newpage
    \pagenumbering{gobble}
        \printbibliography


    \newpage
    \pagenumbering{roman}
    \appendix
        \part{Appendices}
            \input{8 - Hilbert complexes/main.tex}
            \input{9 - weak conservation proofs/main.tex}
\end{document}

    
    
    %\section{}
    \newpage
    \pagenumbering{gobble}
        \printbibliography


    \newpage
    \pagenumbering{roman}
    \appendix
        \part{Appendices}
            \documentclass[12pt, a4paper]{report}

\input{template/main.tex}

\title{\BA{Title in Progress...}}
\author{Boris Andrews}
\affil{Mathematical Institute, University of Oxford}
\date{\today}


\begin{document}
    \pagenumbering{gobble}
    \maketitle
    
    
    \begin{abstract}
        Magnetic confinement reactors---in particular tokamaks---offer one of the most promising options for achieving practical nuclear fusion, with the potential to provide virtually limitless, clean energy. The theoretical and numerical modeling of tokamak plasmas is simultaneously an essential component of effective reactor design, and a great research barrier. Tokamak operational conditions exhibit comparatively low Knudsen numbers. Kinetic effects, including kinetic waves and instabilities, Landau damping, bump-on-tail instabilities and more, are therefore highly influential in tokamak plasma dynamics. Purely fluid models are inherently incapable of capturing these effects, whereas the high dimensionality in purely kinetic models render them practically intractable for most relevant purposes.

        We consider a $\delta\!f$ decomposition model, with a macroscopic fluid background and microscopic kinetic correction, both fully coupled to each other. A similar manner of discretization is proposed to that used in the recent \texttt{STRUPHY} code \cite{Holderied_Possanner_Wang_2021, Holderied_2022, Li_et_al_2023} with a finite-element model for the background and a pseudo-particle/PiC model for the correction.

        The fluid background satisfies the full, non-linear, resistive, compressible, Hall MHD equations. \cite{Laakmann_Hu_Farrell_2022} introduces finite-element(-in-space) implicit timesteppers for the incompressible analogue to this system with structure-preserving (SP) properties in the ideal case, alongside parameter-robust preconditioners. We show that these timesteppers can derive from a finite-element-in-time (FET) (and finite-element-in-space) interpretation. The benefits of this reformulation are discussed, including the derivation of timesteppers that are higher order in time, and the quantifiable dissipative SP properties in the non-ideal, resistive case.
        
        We discuss possible options for extending this FET approach to timesteppers for the compressible case.

        The kinetic corrections satisfy linearized Boltzmann equations. Using a Lénard--Bernstein collision operator, these take Fokker--Planck-like forms \cite{Fokker_1914, Planck_1917} wherein pseudo-particles in the numerical model obey the neoclassical transport equations, with particle-independent Brownian drift terms. This offers a rigorous methodology for incorporating collisions into the particle transport model, without coupling the equations of motions for each particle.
        
        Works by Chen, Chacón et al. \cite{Chen_Chacón_Barnes_2011, Chacón_Chen_Barnes_2013, Chen_Chacón_2014, Chen_Chacón_2015} have developed structure-preserving particle pushers for neoclassical transport in the Vlasov equations, derived from Crank--Nicolson integrators. We show these too can can derive from a FET interpretation, similarly offering potential extensions to higher-order-in-time particle pushers. The FET formulation is used also to consider how the stochastic drift terms can be incorporated into the pushers. Stochastic gyrokinetic expansions are also discussed.

        Different options for the numerical implementation of these schemes are considered.

        Due to the efficacy of FET in the development of SP timesteppers for both the fluid and kinetic component, we hope this approach will prove effective in the future for developing SP timesteppers for the full hybrid model. We hope this will give us the opportunity to incorporate previously inaccessible kinetic effects into the highly effective, modern, finite-element MHD models.
    \end{abstract}
    
    
    \newpage
    \tableofcontents
    
    
    \newpage
    \pagenumbering{arabic}
    %\linenumbers\renewcommand\thelinenumber{\color{black!50}\arabic{linenumber}}
            \input{0 - introduction/main.tex}
        \part{Research}
            \input{1 - low-noise PiC models/main.tex}
            \input{2 - kinetic component/main.tex}
            \input{3 - fluid component/main.tex}
            \input{4 - numerical implementation/main.tex}
        \part{Project Overview}
            \input{5 - research plan/main.tex}
            \input{6 - summary/main.tex}
    
    
    %\section{}
    \newpage
    \pagenumbering{gobble}
        \printbibliography


    \newpage
    \pagenumbering{roman}
    \appendix
        \part{Appendices}
            \input{8 - Hilbert complexes/main.tex}
            \input{9 - weak conservation proofs/main.tex}
\end{document}

            \documentclass[12pt, a4paper]{report}

\input{template/main.tex}

\title{\BA{Title in Progress...}}
\author{Boris Andrews}
\affil{Mathematical Institute, University of Oxford}
\date{\today}


\begin{document}
    \pagenumbering{gobble}
    \maketitle
    
    
    \begin{abstract}
        Magnetic confinement reactors---in particular tokamaks---offer one of the most promising options for achieving practical nuclear fusion, with the potential to provide virtually limitless, clean energy. The theoretical and numerical modeling of tokamak plasmas is simultaneously an essential component of effective reactor design, and a great research barrier. Tokamak operational conditions exhibit comparatively low Knudsen numbers. Kinetic effects, including kinetic waves and instabilities, Landau damping, bump-on-tail instabilities and more, are therefore highly influential in tokamak plasma dynamics. Purely fluid models are inherently incapable of capturing these effects, whereas the high dimensionality in purely kinetic models render them practically intractable for most relevant purposes.

        We consider a $\delta\!f$ decomposition model, with a macroscopic fluid background and microscopic kinetic correction, both fully coupled to each other. A similar manner of discretization is proposed to that used in the recent \texttt{STRUPHY} code \cite{Holderied_Possanner_Wang_2021, Holderied_2022, Li_et_al_2023} with a finite-element model for the background and a pseudo-particle/PiC model for the correction.

        The fluid background satisfies the full, non-linear, resistive, compressible, Hall MHD equations. \cite{Laakmann_Hu_Farrell_2022} introduces finite-element(-in-space) implicit timesteppers for the incompressible analogue to this system with structure-preserving (SP) properties in the ideal case, alongside parameter-robust preconditioners. We show that these timesteppers can derive from a finite-element-in-time (FET) (and finite-element-in-space) interpretation. The benefits of this reformulation are discussed, including the derivation of timesteppers that are higher order in time, and the quantifiable dissipative SP properties in the non-ideal, resistive case.
        
        We discuss possible options for extending this FET approach to timesteppers for the compressible case.

        The kinetic corrections satisfy linearized Boltzmann equations. Using a Lénard--Bernstein collision operator, these take Fokker--Planck-like forms \cite{Fokker_1914, Planck_1917} wherein pseudo-particles in the numerical model obey the neoclassical transport equations, with particle-independent Brownian drift terms. This offers a rigorous methodology for incorporating collisions into the particle transport model, without coupling the equations of motions for each particle.
        
        Works by Chen, Chacón et al. \cite{Chen_Chacón_Barnes_2011, Chacón_Chen_Barnes_2013, Chen_Chacón_2014, Chen_Chacón_2015} have developed structure-preserving particle pushers for neoclassical transport in the Vlasov equations, derived from Crank--Nicolson integrators. We show these too can can derive from a FET interpretation, similarly offering potential extensions to higher-order-in-time particle pushers. The FET formulation is used also to consider how the stochastic drift terms can be incorporated into the pushers. Stochastic gyrokinetic expansions are also discussed.

        Different options for the numerical implementation of these schemes are considered.

        Due to the efficacy of FET in the development of SP timesteppers for both the fluid and kinetic component, we hope this approach will prove effective in the future for developing SP timesteppers for the full hybrid model. We hope this will give us the opportunity to incorporate previously inaccessible kinetic effects into the highly effective, modern, finite-element MHD models.
    \end{abstract}
    
    
    \newpage
    \tableofcontents
    
    
    \newpage
    \pagenumbering{arabic}
    %\linenumbers\renewcommand\thelinenumber{\color{black!50}\arabic{linenumber}}
            \input{0 - introduction/main.tex}
        \part{Research}
            \input{1 - low-noise PiC models/main.tex}
            \input{2 - kinetic component/main.tex}
            \input{3 - fluid component/main.tex}
            \input{4 - numerical implementation/main.tex}
        \part{Project Overview}
            \input{5 - research plan/main.tex}
            \input{6 - summary/main.tex}
    
    
    %\section{}
    \newpage
    \pagenumbering{gobble}
        \printbibliography


    \newpage
    \pagenumbering{roman}
    \appendix
        \part{Appendices}
            \input{8 - Hilbert complexes/main.tex}
            \input{9 - weak conservation proofs/main.tex}
\end{document}

\end{document}

        \part{Research}
            \documentclass[12pt, a4paper]{report}

\documentclass[12pt, a4paper]{report}

\input{template/main.tex}

\title{\BA{Title in Progress...}}
\author{Boris Andrews}
\affil{Mathematical Institute, University of Oxford}
\date{\today}


\begin{document}
    \pagenumbering{gobble}
    \maketitle
    
    
    \begin{abstract}
        Magnetic confinement reactors---in particular tokamaks---offer one of the most promising options for achieving practical nuclear fusion, with the potential to provide virtually limitless, clean energy. The theoretical and numerical modeling of tokamak plasmas is simultaneously an essential component of effective reactor design, and a great research barrier. Tokamak operational conditions exhibit comparatively low Knudsen numbers. Kinetic effects, including kinetic waves and instabilities, Landau damping, bump-on-tail instabilities and more, are therefore highly influential in tokamak plasma dynamics. Purely fluid models are inherently incapable of capturing these effects, whereas the high dimensionality in purely kinetic models render them practically intractable for most relevant purposes.

        We consider a $\delta\!f$ decomposition model, with a macroscopic fluid background and microscopic kinetic correction, both fully coupled to each other. A similar manner of discretization is proposed to that used in the recent \texttt{STRUPHY} code \cite{Holderied_Possanner_Wang_2021, Holderied_2022, Li_et_al_2023} with a finite-element model for the background and a pseudo-particle/PiC model for the correction.

        The fluid background satisfies the full, non-linear, resistive, compressible, Hall MHD equations. \cite{Laakmann_Hu_Farrell_2022} introduces finite-element(-in-space) implicit timesteppers for the incompressible analogue to this system with structure-preserving (SP) properties in the ideal case, alongside parameter-robust preconditioners. We show that these timesteppers can derive from a finite-element-in-time (FET) (and finite-element-in-space) interpretation. The benefits of this reformulation are discussed, including the derivation of timesteppers that are higher order in time, and the quantifiable dissipative SP properties in the non-ideal, resistive case.
        
        We discuss possible options for extending this FET approach to timesteppers for the compressible case.

        The kinetic corrections satisfy linearized Boltzmann equations. Using a Lénard--Bernstein collision operator, these take Fokker--Planck-like forms \cite{Fokker_1914, Planck_1917} wherein pseudo-particles in the numerical model obey the neoclassical transport equations, with particle-independent Brownian drift terms. This offers a rigorous methodology for incorporating collisions into the particle transport model, without coupling the equations of motions for each particle.
        
        Works by Chen, Chacón et al. \cite{Chen_Chacón_Barnes_2011, Chacón_Chen_Barnes_2013, Chen_Chacón_2014, Chen_Chacón_2015} have developed structure-preserving particle pushers for neoclassical transport in the Vlasov equations, derived from Crank--Nicolson integrators. We show these too can can derive from a FET interpretation, similarly offering potential extensions to higher-order-in-time particle pushers. The FET formulation is used also to consider how the stochastic drift terms can be incorporated into the pushers. Stochastic gyrokinetic expansions are also discussed.

        Different options for the numerical implementation of these schemes are considered.

        Due to the efficacy of FET in the development of SP timesteppers for both the fluid and kinetic component, we hope this approach will prove effective in the future for developing SP timesteppers for the full hybrid model. We hope this will give us the opportunity to incorporate previously inaccessible kinetic effects into the highly effective, modern, finite-element MHD models.
    \end{abstract}
    
    
    \newpage
    \tableofcontents
    
    
    \newpage
    \pagenumbering{arabic}
    %\linenumbers\renewcommand\thelinenumber{\color{black!50}\arabic{linenumber}}
            \input{0 - introduction/main.tex}
        \part{Research}
            \input{1 - low-noise PiC models/main.tex}
            \input{2 - kinetic component/main.tex}
            \input{3 - fluid component/main.tex}
            \input{4 - numerical implementation/main.tex}
        \part{Project Overview}
            \input{5 - research plan/main.tex}
            \input{6 - summary/main.tex}
    
    
    %\section{}
    \newpage
    \pagenumbering{gobble}
        \printbibliography


    \newpage
    \pagenumbering{roman}
    \appendix
        \part{Appendices}
            \input{8 - Hilbert complexes/main.tex}
            \input{9 - weak conservation proofs/main.tex}
\end{document}


\title{\BA{Title in Progress...}}
\author{Boris Andrews}
\affil{Mathematical Institute, University of Oxford}
\date{\today}


\begin{document}
    \pagenumbering{gobble}
    \maketitle
    
    
    \begin{abstract}
        Magnetic confinement reactors---in particular tokamaks---offer one of the most promising options for achieving practical nuclear fusion, with the potential to provide virtually limitless, clean energy. The theoretical and numerical modeling of tokamak plasmas is simultaneously an essential component of effective reactor design, and a great research barrier. Tokamak operational conditions exhibit comparatively low Knudsen numbers. Kinetic effects, including kinetic waves and instabilities, Landau damping, bump-on-tail instabilities and more, are therefore highly influential in tokamak plasma dynamics. Purely fluid models are inherently incapable of capturing these effects, whereas the high dimensionality in purely kinetic models render them practically intractable for most relevant purposes.

        We consider a $\delta\!f$ decomposition model, with a macroscopic fluid background and microscopic kinetic correction, both fully coupled to each other. A similar manner of discretization is proposed to that used in the recent \texttt{STRUPHY} code \cite{Holderied_Possanner_Wang_2021, Holderied_2022, Li_et_al_2023} with a finite-element model for the background and a pseudo-particle/PiC model for the correction.

        The fluid background satisfies the full, non-linear, resistive, compressible, Hall MHD equations. \cite{Laakmann_Hu_Farrell_2022} introduces finite-element(-in-space) implicit timesteppers for the incompressible analogue to this system with structure-preserving (SP) properties in the ideal case, alongside parameter-robust preconditioners. We show that these timesteppers can derive from a finite-element-in-time (FET) (and finite-element-in-space) interpretation. The benefits of this reformulation are discussed, including the derivation of timesteppers that are higher order in time, and the quantifiable dissipative SP properties in the non-ideal, resistive case.
        
        We discuss possible options for extending this FET approach to timesteppers for the compressible case.

        The kinetic corrections satisfy linearized Boltzmann equations. Using a Lénard--Bernstein collision operator, these take Fokker--Planck-like forms \cite{Fokker_1914, Planck_1917} wherein pseudo-particles in the numerical model obey the neoclassical transport equations, with particle-independent Brownian drift terms. This offers a rigorous methodology for incorporating collisions into the particle transport model, without coupling the equations of motions for each particle.
        
        Works by Chen, Chacón et al. \cite{Chen_Chacón_Barnes_2011, Chacón_Chen_Barnes_2013, Chen_Chacón_2014, Chen_Chacón_2015} have developed structure-preserving particle pushers for neoclassical transport in the Vlasov equations, derived from Crank--Nicolson integrators. We show these too can can derive from a FET interpretation, similarly offering potential extensions to higher-order-in-time particle pushers. The FET formulation is used also to consider how the stochastic drift terms can be incorporated into the pushers. Stochastic gyrokinetic expansions are also discussed.

        Different options for the numerical implementation of these schemes are considered.

        Due to the efficacy of FET in the development of SP timesteppers for both the fluid and kinetic component, we hope this approach will prove effective in the future for developing SP timesteppers for the full hybrid model. We hope this will give us the opportunity to incorporate previously inaccessible kinetic effects into the highly effective, modern, finite-element MHD models.
    \end{abstract}
    
    
    \newpage
    \tableofcontents
    
    
    \newpage
    \pagenumbering{arabic}
    %\linenumbers\renewcommand\thelinenumber{\color{black!50}\arabic{linenumber}}
            \documentclass[12pt, a4paper]{report}

\input{template/main.tex}

\title{\BA{Title in Progress...}}
\author{Boris Andrews}
\affil{Mathematical Institute, University of Oxford}
\date{\today}


\begin{document}
    \pagenumbering{gobble}
    \maketitle
    
    
    \begin{abstract}
        Magnetic confinement reactors---in particular tokamaks---offer one of the most promising options for achieving practical nuclear fusion, with the potential to provide virtually limitless, clean energy. The theoretical and numerical modeling of tokamak plasmas is simultaneously an essential component of effective reactor design, and a great research barrier. Tokamak operational conditions exhibit comparatively low Knudsen numbers. Kinetic effects, including kinetic waves and instabilities, Landau damping, bump-on-tail instabilities and more, are therefore highly influential in tokamak plasma dynamics. Purely fluid models are inherently incapable of capturing these effects, whereas the high dimensionality in purely kinetic models render them practically intractable for most relevant purposes.

        We consider a $\delta\!f$ decomposition model, with a macroscopic fluid background and microscopic kinetic correction, both fully coupled to each other. A similar manner of discretization is proposed to that used in the recent \texttt{STRUPHY} code \cite{Holderied_Possanner_Wang_2021, Holderied_2022, Li_et_al_2023} with a finite-element model for the background and a pseudo-particle/PiC model for the correction.

        The fluid background satisfies the full, non-linear, resistive, compressible, Hall MHD equations. \cite{Laakmann_Hu_Farrell_2022} introduces finite-element(-in-space) implicit timesteppers for the incompressible analogue to this system with structure-preserving (SP) properties in the ideal case, alongside parameter-robust preconditioners. We show that these timesteppers can derive from a finite-element-in-time (FET) (and finite-element-in-space) interpretation. The benefits of this reformulation are discussed, including the derivation of timesteppers that are higher order in time, and the quantifiable dissipative SP properties in the non-ideal, resistive case.
        
        We discuss possible options for extending this FET approach to timesteppers for the compressible case.

        The kinetic corrections satisfy linearized Boltzmann equations. Using a Lénard--Bernstein collision operator, these take Fokker--Planck-like forms \cite{Fokker_1914, Planck_1917} wherein pseudo-particles in the numerical model obey the neoclassical transport equations, with particle-independent Brownian drift terms. This offers a rigorous methodology for incorporating collisions into the particle transport model, without coupling the equations of motions for each particle.
        
        Works by Chen, Chacón et al. \cite{Chen_Chacón_Barnes_2011, Chacón_Chen_Barnes_2013, Chen_Chacón_2014, Chen_Chacón_2015} have developed structure-preserving particle pushers for neoclassical transport in the Vlasov equations, derived from Crank--Nicolson integrators. We show these too can can derive from a FET interpretation, similarly offering potential extensions to higher-order-in-time particle pushers. The FET formulation is used also to consider how the stochastic drift terms can be incorporated into the pushers. Stochastic gyrokinetic expansions are also discussed.

        Different options for the numerical implementation of these schemes are considered.

        Due to the efficacy of FET in the development of SP timesteppers for both the fluid and kinetic component, we hope this approach will prove effective in the future for developing SP timesteppers for the full hybrid model. We hope this will give us the opportunity to incorporate previously inaccessible kinetic effects into the highly effective, modern, finite-element MHD models.
    \end{abstract}
    
    
    \newpage
    \tableofcontents
    
    
    \newpage
    \pagenumbering{arabic}
    %\linenumbers\renewcommand\thelinenumber{\color{black!50}\arabic{linenumber}}
            \input{0 - introduction/main.tex}
        \part{Research}
            \input{1 - low-noise PiC models/main.tex}
            \input{2 - kinetic component/main.tex}
            \input{3 - fluid component/main.tex}
            \input{4 - numerical implementation/main.tex}
        \part{Project Overview}
            \input{5 - research plan/main.tex}
            \input{6 - summary/main.tex}
    
    
    %\section{}
    \newpage
    \pagenumbering{gobble}
        \printbibliography


    \newpage
    \pagenumbering{roman}
    \appendix
        \part{Appendices}
            \input{8 - Hilbert complexes/main.tex}
            \input{9 - weak conservation proofs/main.tex}
\end{document}

        \part{Research}
            \documentclass[12pt, a4paper]{report}

\input{template/main.tex}

\title{\BA{Title in Progress...}}
\author{Boris Andrews}
\affil{Mathematical Institute, University of Oxford}
\date{\today}


\begin{document}
    \pagenumbering{gobble}
    \maketitle
    
    
    \begin{abstract}
        Magnetic confinement reactors---in particular tokamaks---offer one of the most promising options for achieving practical nuclear fusion, with the potential to provide virtually limitless, clean energy. The theoretical and numerical modeling of tokamak plasmas is simultaneously an essential component of effective reactor design, and a great research barrier. Tokamak operational conditions exhibit comparatively low Knudsen numbers. Kinetic effects, including kinetic waves and instabilities, Landau damping, bump-on-tail instabilities and more, are therefore highly influential in tokamak plasma dynamics. Purely fluid models are inherently incapable of capturing these effects, whereas the high dimensionality in purely kinetic models render them practically intractable for most relevant purposes.

        We consider a $\delta\!f$ decomposition model, with a macroscopic fluid background and microscopic kinetic correction, both fully coupled to each other. A similar manner of discretization is proposed to that used in the recent \texttt{STRUPHY} code \cite{Holderied_Possanner_Wang_2021, Holderied_2022, Li_et_al_2023} with a finite-element model for the background and a pseudo-particle/PiC model for the correction.

        The fluid background satisfies the full, non-linear, resistive, compressible, Hall MHD equations. \cite{Laakmann_Hu_Farrell_2022} introduces finite-element(-in-space) implicit timesteppers for the incompressible analogue to this system with structure-preserving (SP) properties in the ideal case, alongside parameter-robust preconditioners. We show that these timesteppers can derive from a finite-element-in-time (FET) (and finite-element-in-space) interpretation. The benefits of this reformulation are discussed, including the derivation of timesteppers that are higher order in time, and the quantifiable dissipative SP properties in the non-ideal, resistive case.
        
        We discuss possible options for extending this FET approach to timesteppers for the compressible case.

        The kinetic corrections satisfy linearized Boltzmann equations. Using a Lénard--Bernstein collision operator, these take Fokker--Planck-like forms \cite{Fokker_1914, Planck_1917} wherein pseudo-particles in the numerical model obey the neoclassical transport equations, with particle-independent Brownian drift terms. This offers a rigorous methodology for incorporating collisions into the particle transport model, without coupling the equations of motions for each particle.
        
        Works by Chen, Chacón et al. \cite{Chen_Chacón_Barnes_2011, Chacón_Chen_Barnes_2013, Chen_Chacón_2014, Chen_Chacón_2015} have developed structure-preserving particle pushers for neoclassical transport in the Vlasov equations, derived from Crank--Nicolson integrators. We show these too can can derive from a FET interpretation, similarly offering potential extensions to higher-order-in-time particle pushers. The FET formulation is used also to consider how the stochastic drift terms can be incorporated into the pushers. Stochastic gyrokinetic expansions are also discussed.

        Different options for the numerical implementation of these schemes are considered.

        Due to the efficacy of FET in the development of SP timesteppers for both the fluid and kinetic component, we hope this approach will prove effective in the future for developing SP timesteppers for the full hybrid model. We hope this will give us the opportunity to incorporate previously inaccessible kinetic effects into the highly effective, modern, finite-element MHD models.
    \end{abstract}
    
    
    \newpage
    \tableofcontents
    
    
    \newpage
    \pagenumbering{arabic}
    %\linenumbers\renewcommand\thelinenumber{\color{black!50}\arabic{linenumber}}
            \input{0 - introduction/main.tex}
        \part{Research}
            \input{1 - low-noise PiC models/main.tex}
            \input{2 - kinetic component/main.tex}
            \input{3 - fluid component/main.tex}
            \input{4 - numerical implementation/main.tex}
        \part{Project Overview}
            \input{5 - research plan/main.tex}
            \input{6 - summary/main.tex}
    
    
    %\section{}
    \newpage
    \pagenumbering{gobble}
        \printbibliography


    \newpage
    \pagenumbering{roman}
    \appendix
        \part{Appendices}
            \input{8 - Hilbert complexes/main.tex}
            \input{9 - weak conservation proofs/main.tex}
\end{document}

            \documentclass[12pt, a4paper]{report}

\input{template/main.tex}

\title{\BA{Title in Progress...}}
\author{Boris Andrews}
\affil{Mathematical Institute, University of Oxford}
\date{\today}


\begin{document}
    \pagenumbering{gobble}
    \maketitle
    
    
    \begin{abstract}
        Magnetic confinement reactors---in particular tokamaks---offer one of the most promising options for achieving practical nuclear fusion, with the potential to provide virtually limitless, clean energy. The theoretical and numerical modeling of tokamak plasmas is simultaneously an essential component of effective reactor design, and a great research barrier. Tokamak operational conditions exhibit comparatively low Knudsen numbers. Kinetic effects, including kinetic waves and instabilities, Landau damping, bump-on-tail instabilities and more, are therefore highly influential in tokamak plasma dynamics. Purely fluid models are inherently incapable of capturing these effects, whereas the high dimensionality in purely kinetic models render them practically intractable for most relevant purposes.

        We consider a $\delta\!f$ decomposition model, with a macroscopic fluid background and microscopic kinetic correction, both fully coupled to each other. A similar manner of discretization is proposed to that used in the recent \texttt{STRUPHY} code \cite{Holderied_Possanner_Wang_2021, Holderied_2022, Li_et_al_2023} with a finite-element model for the background and a pseudo-particle/PiC model for the correction.

        The fluid background satisfies the full, non-linear, resistive, compressible, Hall MHD equations. \cite{Laakmann_Hu_Farrell_2022} introduces finite-element(-in-space) implicit timesteppers for the incompressible analogue to this system with structure-preserving (SP) properties in the ideal case, alongside parameter-robust preconditioners. We show that these timesteppers can derive from a finite-element-in-time (FET) (and finite-element-in-space) interpretation. The benefits of this reformulation are discussed, including the derivation of timesteppers that are higher order in time, and the quantifiable dissipative SP properties in the non-ideal, resistive case.
        
        We discuss possible options for extending this FET approach to timesteppers for the compressible case.

        The kinetic corrections satisfy linearized Boltzmann equations. Using a Lénard--Bernstein collision operator, these take Fokker--Planck-like forms \cite{Fokker_1914, Planck_1917} wherein pseudo-particles in the numerical model obey the neoclassical transport equations, with particle-independent Brownian drift terms. This offers a rigorous methodology for incorporating collisions into the particle transport model, without coupling the equations of motions for each particle.
        
        Works by Chen, Chacón et al. \cite{Chen_Chacón_Barnes_2011, Chacón_Chen_Barnes_2013, Chen_Chacón_2014, Chen_Chacón_2015} have developed structure-preserving particle pushers for neoclassical transport in the Vlasov equations, derived from Crank--Nicolson integrators. We show these too can can derive from a FET interpretation, similarly offering potential extensions to higher-order-in-time particle pushers. The FET formulation is used also to consider how the stochastic drift terms can be incorporated into the pushers. Stochastic gyrokinetic expansions are also discussed.

        Different options for the numerical implementation of these schemes are considered.

        Due to the efficacy of FET in the development of SP timesteppers for both the fluid and kinetic component, we hope this approach will prove effective in the future for developing SP timesteppers for the full hybrid model. We hope this will give us the opportunity to incorporate previously inaccessible kinetic effects into the highly effective, modern, finite-element MHD models.
    \end{abstract}
    
    
    \newpage
    \tableofcontents
    
    
    \newpage
    \pagenumbering{arabic}
    %\linenumbers\renewcommand\thelinenumber{\color{black!50}\arabic{linenumber}}
            \input{0 - introduction/main.tex}
        \part{Research}
            \input{1 - low-noise PiC models/main.tex}
            \input{2 - kinetic component/main.tex}
            \input{3 - fluid component/main.tex}
            \input{4 - numerical implementation/main.tex}
        \part{Project Overview}
            \input{5 - research plan/main.tex}
            \input{6 - summary/main.tex}
    
    
    %\section{}
    \newpage
    \pagenumbering{gobble}
        \printbibliography


    \newpage
    \pagenumbering{roman}
    \appendix
        \part{Appendices}
            \input{8 - Hilbert complexes/main.tex}
            \input{9 - weak conservation proofs/main.tex}
\end{document}

            \documentclass[12pt, a4paper]{report}

\input{template/main.tex}

\title{\BA{Title in Progress...}}
\author{Boris Andrews}
\affil{Mathematical Institute, University of Oxford}
\date{\today}


\begin{document}
    \pagenumbering{gobble}
    \maketitle
    
    
    \begin{abstract}
        Magnetic confinement reactors---in particular tokamaks---offer one of the most promising options for achieving practical nuclear fusion, with the potential to provide virtually limitless, clean energy. The theoretical and numerical modeling of tokamak plasmas is simultaneously an essential component of effective reactor design, and a great research barrier. Tokamak operational conditions exhibit comparatively low Knudsen numbers. Kinetic effects, including kinetic waves and instabilities, Landau damping, bump-on-tail instabilities and more, are therefore highly influential in tokamak plasma dynamics. Purely fluid models are inherently incapable of capturing these effects, whereas the high dimensionality in purely kinetic models render them practically intractable for most relevant purposes.

        We consider a $\delta\!f$ decomposition model, with a macroscopic fluid background and microscopic kinetic correction, both fully coupled to each other. A similar manner of discretization is proposed to that used in the recent \texttt{STRUPHY} code \cite{Holderied_Possanner_Wang_2021, Holderied_2022, Li_et_al_2023} with a finite-element model for the background and a pseudo-particle/PiC model for the correction.

        The fluid background satisfies the full, non-linear, resistive, compressible, Hall MHD equations. \cite{Laakmann_Hu_Farrell_2022} introduces finite-element(-in-space) implicit timesteppers for the incompressible analogue to this system with structure-preserving (SP) properties in the ideal case, alongside parameter-robust preconditioners. We show that these timesteppers can derive from a finite-element-in-time (FET) (and finite-element-in-space) interpretation. The benefits of this reformulation are discussed, including the derivation of timesteppers that are higher order in time, and the quantifiable dissipative SP properties in the non-ideal, resistive case.
        
        We discuss possible options for extending this FET approach to timesteppers for the compressible case.

        The kinetic corrections satisfy linearized Boltzmann equations. Using a Lénard--Bernstein collision operator, these take Fokker--Planck-like forms \cite{Fokker_1914, Planck_1917} wherein pseudo-particles in the numerical model obey the neoclassical transport equations, with particle-independent Brownian drift terms. This offers a rigorous methodology for incorporating collisions into the particle transport model, without coupling the equations of motions for each particle.
        
        Works by Chen, Chacón et al. \cite{Chen_Chacón_Barnes_2011, Chacón_Chen_Barnes_2013, Chen_Chacón_2014, Chen_Chacón_2015} have developed structure-preserving particle pushers for neoclassical transport in the Vlasov equations, derived from Crank--Nicolson integrators. We show these too can can derive from a FET interpretation, similarly offering potential extensions to higher-order-in-time particle pushers. The FET formulation is used also to consider how the stochastic drift terms can be incorporated into the pushers. Stochastic gyrokinetic expansions are also discussed.

        Different options for the numerical implementation of these schemes are considered.

        Due to the efficacy of FET in the development of SP timesteppers for both the fluid and kinetic component, we hope this approach will prove effective in the future for developing SP timesteppers for the full hybrid model. We hope this will give us the opportunity to incorporate previously inaccessible kinetic effects into the highly effective, modern, finite-element MHD models.
    \end{abstract}
    
    
    \newpage
    \tableofcontents
    
    
    \newpage
    \pagenumbering{arabic}
    %\linenumbers\renewcommand\thelinenumber{\color{black!50}\arabic{linenumber}}
            \input{0 - introduction/main.tex}
        \part{Research}
            \input{1 - low-noise PiC models/main.tex}
            \input{2 - kinetic component/main.tex}
            \input{3 - fluid component/main.tex}
            \input{4 - numerical implementation/main.tex}
        \part{Project Overview}
            \input{5 - research plan/main.tex}
            \input{6 - summary/main.tex}
    
    
    %\section{}
    \newpage
    \pagenumbering{gobble}
        \printbibliography


    \newpage
    \pagenumbering{roman}
    \appendix
        \part{Appendices}
            \input{8 - Hilbert complexes/main.tex}
            \input{9 - weak conservation proofs/main.tex}
\end{document}

            \documentclass[12pt, a4paper]{report}

\input{template/main.tex}

\title{\BA{Title in Progress...}}
\author{Boris Andrews}
\affil{Mathematical Institute, University of Oxford}
\date{\today}


\begin{document}
    \pagenumbering{gobble}
    \maketitle
    
    
    \begin{abstract}
        Magnetic confinement reactors---in particular tokamaks---offer one of the most promising options for achieving practical nuclear fusion, with the potential to provide virtually limitless, clean energy. The theoretical and numerical modeling of tokamak plasmas is simultaneously an essential component of effective reactor design, and a great research barrier. Tokamak operational conditions exhibit comparatively low Knudsen numbers. Kinetic effects, including kinetic waves and instabilities, Landau damping, bump-on-tail instabilities and more, are therefore highly influential in tokamak plasma dynamics. Purely fluid models are inherently incapable of capturing these effects, whereas the high dimensionality in purely kinetic models render them practically intractable for most relevant purposes.

        We consider a $\delta\!f$ decomposition model, with a macroscopic fluid background and microscopic kinetic correction, both fully coupled to each other. A similar manner of discretization is proposed to that used in the recent \texttt{STRUPHY} code \cite{Holderied_Possanner_Wang_2021, Holderied_2022, Li_et_al_2023} with a finite-element model for the background and a pseudo-particle/PiC model for the correction.

        The fluid background satisfies the full, non-linear, resistive, compressible, Hall MHD equations. \cite{Laakmann_Hu_Farrell_2022} introduces finite-element(-in-space) implicit timesteppers for the incompressible analogue to this system with structure-preserving (SP) properties in the ideal case, alongside parameter-robust preconditioners. We show that these timesteppers can derive from a finite-element-in-time (FET) (and finite-element-in-space) interpretation. The benefits of this reformulation are discussed, including the derivation of timesteppers that are higher order in time, and the quantifiable dissipative SP properties in the non-ideal, resistive case.
        
        We discuss possible options for extending this FET approach to timesteppers for the compressible case.

        The kinetic corrections satisfy linearized Boltzmann equations. Using a Lénard--Bernstein collision operator, these take Fokker--Planck-like forms \cite{Fokker_1914, Planck_1917} wherein pseudo-particles in the numerical model obey the neoclassical transport equations, with particle-independent Brownian drift terms. This offers a rigorous methodology for incorporating collisions into the particle transport model, without coupling the equations of motions for each particle.
        
        Works by Chen, Chacón et al. \cite{Chen_Chacón_Barnes_2011, Chacón_Chen_Barnes_2013, Chen_Chacón_2014, Chen_Chacón_2015} have developed structure-preserving particle pushers for neoclassical transport in the Vlasov equations, derived from Crank--Nicolson integrators. We show these too can can derive from a FET interpretation, similarly offering potential extensions to higher-order-in-time particle pushers. The FET formulation is used also to consider how the stochastic drift terms can be incorporated into the pushers. Stochastic gyrokinetic expansions are also discussed.

        Different options for the numerical implementation of these schemes are considered.

        Due to the efficacy of FET in the development of SP timesteppers for both the fluid and kinetic component, we hope this approach will prove effective in the future for developing SP timesteppers for the full hybrid model. We hope this will give us the opportunity to incorporate previously inaccessible kinetic effects into the highly effective, modern, finite-element MHD models.
    \end{abstract}
    
    
    \newpage
    \tableofcontents
    
    
    \newpage
    \pagenumbering{arabic}
    %\linenumbers\renewcommand\thelinenumber{\color{black!50}\arabic{linenumber}}
            \input{0 - introduction/main.tex}
        \part{Research}
            \input{1 - low-noise PiC models/main.tex}
            \input{2 - kinetic component/main.tex}
            \input{3 - fluid component/main.tex}
            \input{4 - numerical implementation/main.tex}
        \part{Project Overview}
            \input{5 - research plan/main.tex}
            \input{6 - summary/main.tex}
    
    
    %\section{}
    \newpage
    \pagenumbering{gobble}
        \printbibliography


    \newpage
    \pagenumbering{roman}
    \appendix
        \part{Appendices}
            \input{8 - Hilbert complexes/main.tex}
            \input{9 - weak conservation proofs/main.tex}
\end{document}

        \part{Project Overview}
            \documentclass[12pt, a4paper]{report}

\input{template/main.tex}

\title{\BA{Title in Progress...}}
\author{Boris Andrews}
\affil{Mathematical Institute, University of Oxford}
\date{\today}


\begin{document}
    \pagenumbering{gobble}
    \maketitle
    
    
    \begin{abstract}
        Magnetic confinement reactors---in particular tokamaks---offer one of the most promising options for achieving practical nuclear fusion, with the potential to provide virtually limitless, clean energy. The theoretical and numerical modeling of tokamak plasmas is simultaneously an essential component of effective reactor design, and a great research barrier. Tokamak operational conditions exhibit comparatively low Knudsen numbers. Kinetic effects, including kinetic waves and instabilities, Landau damping, bump-on-tail instabilities and more, are therefore highly influential in tokamak plasma dynamics. Purely fluid models are inherently incapable of capturing these effects, whereas the high dimensionality in purely kinetic models render them practically intractable for most relevant purposes.

        We consider a $\delta\!f$ decomposition model, with a macroscopic fluid background and microscopic kinetic correction, both fully coupled to each other. A similar manner of discretization is proposed to that used in the recent \texttt{STRUPHY} code \cite{Holderied_Possanner_Wang_2021, Holderied_2022, Li_et_al_2023} with a finite-element model for the background and a pseudo-particle/PiC model for the correction.

        The fluid background satisfies the full, non-linear, resistive, compressible, Hall MHD equations. \cite{Laakmann_Hu_Farrell_2022} introduces finite-element(-in-space) implicit timesteppers for the incompressible analogue to this system with structure-preserving (SP) properties in the ideal case, alongside parameter-robust preconditioners. We show that these timesteppers can derive from a finite-element-in-time (FET) (and finite-element-in-space) interpretation. The benefits of this reformulation are discussed, including the derivation of timesteppers that are higher order in time, and the quantifiable dissipative SP properties in the non-ideal, resistive case.
        
        We discuss possible options for extending this FET approach to timesteppers for the compressible case.

        The kinetic corrections satisfy linearized Boltzmann equations. Using a Lénard--Bernstein collision operator, these take Fokker--Planck-like forms \cite{Fokker_1914, Planck_1917} wherein pseudo-particles in the numerical model obey the neoclassical transport equations, with particle-independent Brownian drift terms. This offers a rigorous methodology for incorporating collisions into the particle transport model, without coupling the equations of motions for each particle.
        
        Works by Chen, Chacón et al. \cite{Chen_Chacón_Barnes_2011, Chacón_Chen_Barnes_2013, Chen_Chacón_2014, Chen_Chacón_2015} have developed structure-preserving particle pushers for neoclassical transport in the Vlasov equations, derived from Crank--Nicolson integrators. We show these too can can derive from a FET interpretation, similarly offering potential extensions to higher-order-in-time particle pushers. The FET formulation is used also to consider how the stochastic drift terms can be incorporated into the pushers. Stochastic gyrokinetic expansions are also discussed.

        Different options for the numerical implementation of these schemes are considered.

        Due to the efficacy of FET in the development of SP timesteppers for both the fluid and kinetic component, we hope this approach will prove effective in the future for developing SP timesteppers for the full hybrid model. We hope this will give us the opportunity to incorporate previously inaccessible kinetic effects into the highly effective, modern, finite-element MHD models.
    \end{abstract}
    
    
    \newpage
    \tableofcontents
    
    
    \newpage
    \pagenumbering{arabic}
    %\linenumbers\renewcommand\thelinenumber{\color{black!50}\arabic{linenumber}}
            \input{0 - introduction/main.tex}
        \part{Research}
            \input{1 - low-noise PiC models/main.tex}
            \input{2 - kinetic component/main.tex}
            \input{3 - fluid component/main.tex}
            \input{4 - numerical implementation/main.tex}
        \part{Project Overview}
            \input{5 - research plan/main.tex}
            \input{6 - summary/main.tex}
    
    
    %\section{}
    \newpage
    \pagenumbering{gobble}
        \printbibliography


    \newpage
    \pagenumbering{roman}
    \appendix
        \part{Appendices}
            \input{8 - Hilbert complexes/main.tex}
            \input{9 - weak conservation proofs/main.tex}
\end{document}

            \documentclass[12pt, a4paper]{report}

\input{template/main.tex}

\title{\BA{Title in Progress...}}
\author{Boris Andrews}
\affil{Mathematical Institute, University of Oxford}
\date{\today}


\begin{document}
    \pagenumbering{gobble}
    \maketitle
    
    
    \begin{abstract}
        Magnetic confinement reactors---in particular tokamaks---offer one of the most promising options for achieving practical nuclear fusion, with the potential to provide virtually limitless, clean energy. The theoretical and numerical modeling of tokamak plasmas is simultaneously an essential component of effective reactor design, and a great research barrier. Tokamak operational conditions exhibit comparatively low Knudsen numbers. Kinetic effects, including kinetic waves and instabilities, Landau damping, bump-on-tail instabilities and more, are therefore highly influential in tokamak plasma dynamics. Purely fluid models are inherently incapable of capturing these effects, whereas the high dimensionality in purely kinetic models render them practically intractable for most relevant purposes.

        We consider a $\delta\!f$ decomposition model, with a macroscopic fluid background and microscopic kinetic correction, both fully coupled to each other. A similar manner of discretization is proposed to that used in the recent \texttt{STRUPHY} code \cite{Holderied_Possanner_Wang_2021, Holderied_2022, Li_et_al_2023} with a finite-element model for the background and a pseudo-particle/PiC model for the correction.

        The fluid background satisfies the full, non-linear, resistive, compressible, Hall MHD equations. \cite{Laakmann_Hu_Farrell_2022} introduces finite-element(-in-space) implicit timesteppers for the incompressible analogue to this system with structure-preserving (SP) properties in the ideal case, alongside parameter-robust preconditioners. We show that these timesteppers can derive from a finite-element-in-time (FET) (and finite-element-in-space) interpretation. The benefits of this reformulation are discussed, including the derivation of timesteppers that are higher order in time, and the quantifiable dissipative SP properties in the non-ideal, resistive case.
        
        We discuss possible options for extending this FET approach to timesteppers for the compressible case.

        The kinetic corrections satisfy linearized Boltzmann equations. Using a Lénard--Bernstein collision operator, these take Fokker--Planck-like forms \cite{Fokker_1914, Planck_1917} wherein pseudo-particles in the numerical model obey the neoclassical transport equations, with particle-independent Brownian drift terms. This offers a rigorous methodology for incorporating collisions into the particle transport model, without coupling the equations of motions for each particle.
        
        Works by Chen, Chacón et al. \cite{Chen_Chacón_Barnes_2011, Chacón_Chen_Barnes_2013, Chen_Chacón_2014, Chen_Chacón_2015} have developed structure-preserving particle pushers for neoclassical transport in the Vlasov equations, derived from Crank--Nicolson integrators. We show these too can can derive from a FET interpretation, similarly offering potential extensions to higher-order-in-time particle pushers. The FET formulation is used also to consider how the stochastic drift terms can be incorporated into the pushers. Stochastic gyrokinetic expansions are also discussed.

        Different options for the numerical implementation of these schemes are considered.

        Due to the efficacy of FET in the development of SP timesteppers for both the fluid and kinetic component, we hope this approach will prove effective in the future for developing SP timesteppers for the full hybrid model. We hope this will give us the opportunity to incorporate previously inaccessible kinetic effects into the highly effective, modern, finite-element MHD models.
    \end{abstract}
    
    
    \newpage
    \tableofcontents
    
    
    \newpage
    \pagenumbering{arabic}
    %\linenumbers\renewcommand\thelinenumber{\color{black!50}\arabic{linenumber}}
            \input{0 - introduction/main.tex}
        \part{Research}
            \input{1 - low-noise PiC models/main.tex}
            \input{2 - kinetic component/main.tex}
            \input{3 - fluid component/main.tex}
            \input{4 - numerical implementation/main.tex}
        \part{Project Overview}
            \input{5 - research plan/main.tex}
            \input{6 - summary/main.tex}
    
    
    %\section{}
    \newpage
    \pagenumbering{gobble}
        \printbibliography


    \newpage
    \pagenumbering{roman}
    \appendix
        \part{Appendices}
            \input{8 - Hilbert complexes/main.tex}
            \input{9 - weak conservation proofs/main.tex}
\end{document}

    
    
    %\section{}
    \newpage
    \pagenumbering{gobble}
        \printbibliography


    \newpage
    \pagenumbering{roman}
    \appendix
        \part{Appendices}
            \documentclass[12pt, a4paper]{report}

\input{template/main.tex}

\title{\BA{Title in Progress...}}
\author{Boris Andrews}
\affil{Mathematical Institute, University of Oxford}
\date{\today}


\begin{document}
    \pagenumbering{gobble}
    \maketitle
    
    
    \begin{abstract}
        Magnetic confinement reactors---in particular tokamaks---offer one of the most promising options for achieving practical nuclear fusion, with the potential to provide virtually limitless, clean energy. The theoretical and numerical modeling of tokamak plasmas is simultaneously an essential component of effective reactor design, and a great research barrier. Tokamak operational conditions exhibit comparatively low Knudsen numbers. Kinetic effects, including kinetic waves and instabilities, Landau damping, bump-on-tail instabilities and more, are therefore highly influential in tokamak plasma dynamics. Purely fluid models are inherently incapable of capturing these effects, whereas the high dimensionality in purely kinetic models render them practically intractable for most relevant purposes.

        We consider a $\delta\!f$ decomposition model, with a macroscopic fluid background and microscopic kinetic correction, both fully coupled to each other. A similar manner of discretization is proposed to that used in the recent \texttt{STRUPHY} code \cite{Holderied_Possanner_Wang_2021, Holderied_2022, Li_et_al_2023} with a finite-element model for the background and a pseudo-particle/PiC model for the correction.

        The fluid background satisfies the full, non-linear, resistive, compressible, Hall MHD equations. \cite{Laakmann_Hu_Farrell_2022} introduces finite-element(-in-space) implicit timesteppers for the incompressible analogue to this system with structure-preserving (SP) properties in the ideal case, alongside parameter-robust preconditioners. We show that these timesteppers can derive from a finite-element-in-time (FET) (and finite-element-in-space) interpretation. The benefits of this reformulation are discussed, including the derivation of timesteppers that are higher order in time, and the quantifiable dissipative SP properties in the non-ideal, resistive case.
        
        We discuss possible options for extending this FET approach to timesteppers for the compressible case.

        The kinetic corrections satisfy linearized Boltzmann equations. Using a Lénard--Bernstein collision operator, these take Fokker--Planck-like forms \cite{Fokker_1914, Planck_1917} wherein pseudo-particles in the numerical model obey the neoclassical transport equations, with particle-independent Brownian drift terms. This offers a rigorous methodology for incorporating collisions into the particle transport model, without coupling the equations of motions for each particle.
        
        Works by Chen, Chacón et al. \cite{Chen_Chacón_Barnes_2011, Chacón_Chen_Barnes_2013, Chen_Chacón_2014, Chen_Chacón_2015} have developed structure-preserving particle pushers for neoclassical transport in the Vlasov equations, derived from Crank--Nicolson integrators. We show these too can can derive from a FET interpretation, similarly offering potential extensions to higher-order-in-time particle pushers. The FET formulation is used also to consider how the stochastic drift terms can be incorporated into the pushers. Stochastic gyrokinetic expansions are also discussed.

        Different options for the numerical implementation of these schemes are considered.

        Due to the efficacy of FET in the development of SP timesteppers for both the fluid and kinetic component, we hope this approach will prove effective in the future for developing SP timesteppers for the full hybrid model. We hope this will give us the opportunity to incorporate previously inaccessible kinetic effects into the highly effective, modern, finite-element MHD models.
    \end{abstract}
    
    
    \newpage
    \tableofcontents
    
    
    \newpage
    \pagenumbering{arabic}
    %\linenumbers\renewcommand\thelinenumber{\color{black!50}\arabic{linenumber}}
            \input{0 - introduction/main.tex}
        \part{Research}
            \input{1 - low-noise PiC models/main.tex}
            \input{2 - kinetic component/main.tex}
            \input{3 - fluid component/main.tex}
            \input{4 - numerical implementation/main.tex}
        \part{Project Overview}
            \input{5 - research plan/main.tex}
            \input{6 - summary/main.tex}
    
    
    %\section{}
    \newpage
    \pagenumbering{gobble}
        \printbibliography


    \newpage
    \pagenumbering{roman}
    \appendix
        \part{Appendices}
            \input{8 - Hilbert complexes/main.tex}
            \input{9 - weak conservation proofs/main.tex}
\end{document}

            \documentclass[12pt, a4paper]{report}

\input{template/main.tex}

\title{\BA{Title in Progress...}}
\author{Boris Andrews}
\affil{Mathematical Institute, University of Oxford}
\date{\today}


\begin{document}
    \pagenumbering{gobble}
    \maketitle
    
    
    \begin{abstract}
        Magnetic confinement reactors---in particular tokamaks---offer one of the most promising options for achieving practical nuclear fusion, with the potential to provide virtually limitless, clean energy. The theoretical and numerical modeling of tokamak plasmas is simultaneously an essential component of effective reactor design, and a great research barrier. Tokamak operational conditions exhibit comparatively low Knudsen numbers. Kinetic effects, including kinetic waves and instabilities, Landau damping, bump-on-tail instabilities and more, are therefore highly influential in tokamak plasma dynamics. Purely fluid models are inherently incapable of capturing these effects, whereas the high dimensionality in purely kinetic models render them practically intractable for most relevant purposes.

        We consider a $\delta\!f$ decomposition model, with a macroscopic fluid background and microscopic kinetic correction, both fully coupled to each other. A similar manner of discretization is proposed to that used in the recent \texttt{STRUPHY} code \cite{Holderied_Possanner_Wang_2021, Holderied_2022, Li_et_al_2023} with a finite-element model for the background and a pseudo-particle/PiC model for the correction.

        The fluid background satisfies the full, non-linear, resistive, compressible, Hall MHD equations. \cite{Laakmann_Hu_Farrell_2022} introduces finite-element(-in-space) implicit timesteppers for the incompressible analogue to this system with structure-preserving (SP) properties in the ideal case, alongside parameter-robust preconditioners. We show that these timesteppers can derive from a finite-element-in-time (FET) (and finite-element-in-space) interpretation. The benefits of this reformulation are discussed, including the derivation of timesteppers that are higher order in time, and the quantifiable dissipative SP properties in the non-ideal, resistive case.
        
        We discuss possible options for extending this FET approach to timesteppers for the compressible case.

        The kinetic corrections satisfy linearized Boltzmann equations. Using a Lénard--Bernstein collision operator, these take Fokker--Planck-like forms \cite{Fokker_1914, Planck_1917} wherein pseudo-particles in the numerical model obey the neoclassical transport equations, with particle-independent Brownian drift terms. This offers a rigorous methodology for incorporating collisions into the particle transport model, without coupling the equations of motions for each particle.
        
        Works by Chen, Chacón et al. \cite{Chen_Chacón_Barnes_2011, Chacón_Chen_Barnes_2013, Chen_Chacón_2014, Chen_Chacón_2015} have developed structure-preserving particle pushers for neoclassical transport in the Vlasov equations, derived from Crank--Nicolson integrators. We show these too can can derive from a FET interpretation, similarly offering potential extensions to higher-order-in-time particle pushers. The FET formulation is used also to consider how the stochastic drift terms can be incorporated into the pushers. Stochastic gyrokinetic expansions are also discussed.

        Different options for the numerical implementation of these schemes are considered.

        Due to the efficacy of FET in the development of SP timesteppers for both the fluid and kinetic component, we hope this approach will prove effective in the future for developing SP timesteppers for the full hybrid model. We hope this will give us the opportunity to incorporate previously inaccessible kinetic effects into the highly effective, modern, finite-element MHD models.
    \end{abstract}
    
    
    \newpage
    \tableofcontents
    
    
    \newpage
    \pagenumbering{arabic}
    %\linenumbers\renewcommand\thelinenumber{\color{black!50}\arabic{linenumber}}
            \input{0 - introduction/main.tex}
        \part{Research}
            \input{1 - low-noise PiC models/main.tex}
            \input{2 - kinetic component/main.tex}
            \input{3 - fluid component/main.tex}
            \input{4 - numerical implementation/main.tex}
        \part{Project Overview}
            \input{5 - research plan/main.tex}
            \input{6 - summary/main.tex}
    
    
    %\section{}
    \newpage
    \pagenumbering{gobble}
        \printbibliography


    \newpage
    \pagenumbering{roman}
    \appendix
        \part{Appendices}
            \input{8 - Hilbert complexes/main.tex}
            \input{9 - weak conservation proofs/main.tex}
\end{document}

\end{document}

            \documentclass[12pt, a4paper]{report}

\documentclass[12pt, a4paper]{report}

\input{template/main.tex}

\title{\BA{Title in Progress...}}
\author{Boris Andrews}
\affil{Mathematical Institute, University of Oxford}
\date{\today}


\begin{document}
    \pagenumbering{gobble}
    \maketitle
    
    
    \begin{abstract}
        Magnetic confinement reactors---in particular tokamaks---offer one of the most promising options for achieving practical nuclear fusion, with the potential to provide virtually limitless, clean energy. The theoretical and numerical modeling of tokamak plasmas is simultaneously an essential component of effective reactor design, and a great research barrier. Tokamak operational conditions exhibit comparatively low Knudsen numbers. Kinetic effects, including kinetic waves and instabilities, Landau damping, bump-on-tail instabilities and more, are therefore highly influential in tokamak plasma dynamics. Purely fluid models are inherently incapable of capturing these effects, whereas the high dimensionality in purely kinetic models render them practically intractable for most relevant purposes.

        We consider a $\delta\!f$ decomposition model, with a macroscopic fluid background and microscopic kinetic correction, both fully coupled to each other. A similar manner of discretization is proposed to that used in the recent \texttt{STRUPHY} code \cite{Holderied_Possanner_Wang_2021, Holderied_2022, Li_et_al_2023} with a finite-element model for the background and a pseudo-particle/PiC model for the correction.

        The fluid background satisfies the full, non-linear, resistive, compressible, Hall MHD equations. \cite{Laakmann_Hu_Farrell_2022} introduces finite-element(-in-space) implicit timesteppers for the incompressible analogue to this system with structure-preserving (SP) properties in the ideal case, alongside parameter-robust preconditioners. We show that these timesteppers can derive from a finite-element-in-time (FET) (and finite-element-in-space) interpretation. The benefits of this reformulation are discussed, including the derivation of timesteppers that are higher order in time, and the quantifiable dissipative SP properties in the non-ideal, resistive case.
        
        We discuss possible options for extending this FET approach to timesteppers for the compressible case.

        The kinetic corrections satisfy linearized Boltzmann equations. Using a Lénard--Bernstein collision operator, these take Fokker--Planck-like forms \cite{Fokker_1914, Planck_1917} wherein pseudo-particles in the numerical model obey the neoclassical transport equations, with particle-independent Brownian drift terms. This offers a rigorous methodology for incorporating collisions into the particle transport model, without coupling the equations of motions for each particle.
        
        Works by Chen, Chacón et al. \cite{Chen_Chacón_Barnes_2011, Chacón_Chen_Barnes_2013, Chen_Chacón_2014, Chen_Chacón_2015} have developed structure-preserving particle pushers for neoclassical transport in the Vlasov equations, derived from Crank--Nicolson integrators. We show these too can can derive from a FET interpretation, similarly offering potential extensions to higher-order-in-time particle pushers. The FET formulation is used also to consider how the stochastic drift terms can be incorporated into the pushers. Stochastic gyrokinetic expansions are also discussed.

        Different options for the numerical implementation of these schemes are considered.

        Due to the efficacy of FET in the development of SP timesteppers for both the fluid and kinetic component, we hope this approach will prove effective in the future for developing SP timesteppers for the full hybrid model. We hope this will give us the opportunity to incorporate previously inaccessible kinetic effects into the highly effective, modern, finite-element MHD models.
    \end{abstract}
    
    
    \newpage
    \tableofcontents
    
    
    \newpage
    \pagenumbering{arabic}
    %\linenumbers\renewcommand\thelinenumber{\color{black!50}\arabic{linenumber}}
            \input{0 - introduction/main.tex}
        \part{Research}
            \input{1 - low-noise PiC models/main.tex}
            \input{2 - kinetic component/main.tex}
            \input{3 - fluid component/main.tex}
            \input{4 - numerical implementation/main.tex}
        \part{Project Overview}
            \input{5 - research plan/main.tex}
            \input{6 - summary/main.tex}
    
    
    %\section{}
    \newpage
    \pagenumbering{gobble}
        \printbibliography


    \newpage
    \pagenumbering{roman}
    \appendix
        \part{Appendices}
            \input{8 - Hilbert complexes/main.tex}
            \input{9 - weak conservation proofs/main.tex}
\end{document}


\title{\BA{Title in Progress...}}
\author{Boris Andrews}
\affil{Mathematical Institute, University of Oxford}
\date{\today}


\begin{document}
    \pagenumbering{gobble}
    \maketitle
    
    
    \begin{abstract}
        Magnetic confinement reactors---in particular tokamaks---offer one of the most promising options for achieving practical nuclear fusion, with the potential to provide virtually limitless, clean energy. The theoretical and numerical modeling of tokamak plasmas is simultaneously an essential component of effective reactor design, and a great research barrier. Tokamak operational conditions exhibit comparatively low Knudsen numbers. Kinetic effects, including kinetic waves and instabilities, Landau damping, bump-on-tail instabilities and more, are therefore highly influential in tokamak plasma dynamics. Purely fluid models are inherently incapable of capturing these effects, whereas the high dimensionality in purely kinetic models render them practically intractable for most relevant purposes.

        We consider a $\delta\!f$ decomposition model, with a macroscopic fluid background and microscopic kinetic correction, both fully coupled to each other. A similar manner of discretization is proposed to that used in the recent \texttt{STRUPHY} code \cite{Holderied_Possanner_Wang_2021, Holderied_2022, Li_et_al_2023} with a finite-element model for the background and a pseudo-particle/PiC model for the correction.

        The fluid background satisfies the full, non-linear, resistive, compressible, Hall MHD equations. \cite{Laakmann_Hu_Farrell_2022} introduces finite-element(-in-space) implicit timesteppers for the incompressible analogue to this system with structure-preserving (SP) properties in the ideal case, alongside parameter-robust preconditioners. We show that these timesteppers can derive from a finite-element-in-time (FET) (and finite-element-in-space) interpretation. The benefits of this reformulation are discussed, including the derivation of timesteppers that are higher order in time, and the quantifiable dissipative SP properties in the non-ideal, resistive case.
        
        We discuss possible options for extending this FET approach to timesteppers for the compressible case.

        The kinetic corrections satisfy linearized Boltzmann equations. Using a Lénard--Bernstein collision operator, these take Fokker--Planck-like forms \cite{Fokker_1914, Planck_1917} wherein pseudo-particles in the numerical model obey the neoclassical transport equations, with particle-independent Brownian drift terms. This offers a rigorous methodology for incorporating collisions into the particle transport model, without coupling the equations of motions for each particle.
        
        Works by Chen, Chacón et al. \cite{Chen_Chacón_Barnes_2011, Chacón_Chen_Barnes_2013, Chen_Chacón_2014, Chen_Chacón_2015} have developed structure-preserving particle pushers for neoclassical transport in the Vlasov equations, derived from Crank--Nicolson integrators. We show these too can can derive from a FET interpretation, similarly offering potential extensions to higher-order-in-time particle pushers. The FET formulation is used also to consider how the stochastic drift terms can be incorporated into the pushers. Stochastic gyrokinetic expansions are also discussed.

        Different options for the numerical implementation of these schemes are considered.

        Due to the efficacy of FET in the development of SP timesteppers for both the fluid and kinetic component, we hope this approach will prove effective in the future for developing SP timesteppers for the full hybrid model. We hope this will give us the opportunity to incorporate previously inaccessible kinetic effects into the highly effective, modern, finite-element MHD models.
    \end{abstract}
    
    
    \newpage
    \tableofcontents
    
    
    \newpage
    \pagenumbering{arabic}
    %\linenumbers\renewcommand\thelinenumber{\color{black!50}\arabic{linenumber}}
            \documentclass[12pt, a4paper]{report}

\input{template/main.tex}

\title{\BA{Title in Progress...}}
\author{Boris Andrews}
\affil{Mathematical Institute, University of Oxford}
\date{\today}


\begin{document}
    \pagenumbering{gobble}
    \maketitle
    
    
    \begin{abstract}
        Magnetic confinement reactors---in particular tokamaks---offer one of the most promising options for achieving practical nuclear fusion, with the potential to provide virtually limitless, clean energy. The theoretical and numerical modeling of tokamak plasmas is simultaneously an essential component of effective reactor design, and a great research barrier. Tokamak operational conditions exhibit comparatively low Knudsen numbers. Kinetic effects, including kinetic waves and instabilities, Landau damping, bump-on-tail instabilities and more, are therefore highly influential in tokamak plasma dynamics. Purely fluid models are inherently incapable of capturing these effects, whereas the high dimensionality in purely kinetic models render them practically intractable for most relevant purposes.

        We consider a $\delta\!f$ decomposition model, with a macroscopic fluid background and microscopic kinetic correction, both fully coupled to each other. A similar manner of discretization is proposed to that used in the recent \texttt{STRUPHY} code \cite{Holderied_Possanner_Wang_2021, Holderied_2022, Li_et_al_2023} with a finite-element model for the background and a pseudo-particle/PiC model for the correction.

        The fluid background satisfies the full, non-linear, resistive, compressible, Hall MHD equations. \cite{Laakmann_Hu_Farrell_2022} introduces finite-element(-in-space) implicit timesteppers for the incompressible analogue to this system with structure-preserving (SP) properties in the ideal case, alongside parameter-robust preconditioners. We show that these timesteppers can derive from a finite-element-in-time (FET) (and finite-element-in-space) interpretation. The benefits of this reformulation are discussed, including the derivation of timesteppers that are higher order in time, and the quantifiable dissipative SP properties in the non-ideal, resistive case.
        
        We discuss possible options for extending this FET approach to timesteppers for the compressible case.

        The kinetic corrections satisfy linearized Boltzmann equations. Using a Lénard--Bernstein collision operator, these take Fokker--Planck-like forms \cite{Fokker_1914, Planck_1917} wherein pseudo-particles in the numerical model obey the neoclassical transport equations, with particle-independent Brownian drift terms. This offers a rigorous methodology for incorporating collisions into the particle transport model, without coupling the equations of motions for each particle.
        
        Works by Chen, Chacón et al. \cite{Chen_Chacón_Barnes_2011, Chacón_Chen_Barnes_2013, Chen_Chacón_2014, Chen_Chacón_2015} have developed structure-preserving particle pushers for neoclassical transport in the Vlasov equations, derived from Crank--Nicolson integrators. We show these too can can derive from a FET interpretation, similarly offering potential extensions to higher-order-in-time particle pushers. The FET formulation is used also to consider how the stochastic drift terms can be incorporated into the pushers. Stochastic gyrokinetic expansions are also discussed.

        Different options for the numerical implementation of these schemes are considered.

        Due to the efficacy of FET in the development of SP timesteppers for both the fluid and kinetic component, we hope this approach will prove effective in the future for developing SP timesteppers for the full hybrid model. We hope this will give us the opportunity to incorporate previously inaccessible kinetic effects into the highly effective, modern, finite-element MHD models.
    \end{abstract}
    
    
    \newpage
    \tableofcontents
    
    
    \newpage
    \pagenumbering{arabic}
    %\linenumbers\renewcommand\thelinenumber{\color{black!50}\arabic{linenumber}}
            \input{0 - introduction/main.tex}
        \part{Research}
            \input{1 - low-noise PiC models/main.tex}
            \input{2 - kinetic component/main.tex}
            \input{3 - fluid component/main.tex}
            \input{4 - numerical implementation/main.tex}
        \part{Project Overview}
            \input{5 - research plan/main.tex}
            \input{6 - summary/main.tex}
    
    
    %\section{}
    \newpage
    \pagenumbering{gobble}
        \printbibliography


    \newpage
    \pagenumbering{roman}
    \appendix
        \part{Appendices}
            \input{8 - Hilbert complexes/main.tex}
            \input{9 - weak conservation proofs/main.tex}
\end{document}

        \part{Research}
            \documentclass[12pt, a4paper]{report}

\input{template/main.tex}

\title{\BA{Title in Progress...}}
\author{Boris Andrews}
\affil{Mathematical Institute, University of Oxford}
\date{\today}


\begin{document}
    \pagenumbering{gobble}
    \maketitle
    
    
    \begin{abstract}
        Magnetic confinement reactors---in particular tokamaks---offer one of the most promising options for achieving practical nuclear fusion, with the potential to provide virtually limitless, clean energy. The theoretical and numerical modeling of tokamak plasmas is simultaneously an essential component of effective reactor design, and a great research barrier. Tokamak operational conditions exhibit comparatively low Knudsen numbers. Kinetic effects, including kinetic waves and instabilities, Landau damping, bump-on-tail instabilities and more, are therefore highly influential in tokamak plasma dynamics. Purely fluid models are inherently incapable of capturing these effects, whereas the high dimensionality in purely kinetic models render them practically intractable for most relevant purposes.

        We consider a $\delta\!f$ decomposition model, with a macroscopic fluid background and microscopic kinetic correction, both fully coupled to each other. A similar manner of discretization is proposed to that used in the recent \texttt{STRUPHY} code \cite{Holderied_Possanner_Wang_2021, Holderied_2022, Li_et_al_2023} with a finite-element model for the background and a pseudo-particle/PiC model for the correction.

        The fluid background satisfies the full, non-linear, resistive, compressible, Hall MHD equations. \cite{Laakmann_Hu_Farrell_2022} introduces finite-element(-in-space) implicit timesteppers for the incompressible analogue to this system with structure-preserving (SP) properties in the ideal case, alongside parameter-robust preconditioners. We show that these timesteppers can derive from a finite-element-in-time (FET) (and finite-element-in-space) interpretation. The benefits of this reformulation are discussed, including the derivation of timesteppers that are higher order in time, and the quantifiable dissipative SP properties in the non-ideal, resistive case.
        
        We discuss possible options for extending this FET approach to timesteppers for the compressible case.

        The kinetic corrections satisfy linearized Boltzmann equations. Using a Lénard--Bernstein collision operator, these take Fokker--Planck-like forms \cite{Fokker_1914, Planck_1917} wherein pseudo-particles in the numerical model obey the neoclassical transport equations, with particle-independent Brownian drift terms. This offers a rigorous methodology for incorporating collisions into the particle transport model, without coupling the equations of motions for each particle.
        
        Works by Chen, Chacón et al. \cite{Chen_Chacón_Barnes_2011, Chacón_Chen_Barnes_2013, Chen_Chacón_2014, Chen_Chacón_2015} have developed structure-preserving particle pushers for neoclassical transport in the Vlasov equations, derived from Crank--Nicolson integrators. We show these too can can derive from a FET interpretation, similarly offering potential extensions to higher-order-in-time particle pushers. The FET formulation is used also to consider how the stochastic drift terms can be incorporated into the pushers. Stochastic gyrokinetic expansions are also discussed.

        Different options for the numerical implementation of these schemes are considered.

        Due to the efficacy of FET in the development of SP timesteppers for both the fluid and kinetic component, we hope this approach will prove effective in the future for developing SP timesteppers for the full hybrid model. We hope this will give us the opportunity to incorporate previously inaccessible kinetic effects into the highly effective, modern, finite-element MHD models.
    \end{abstract}
    
    
    \newpage
    \tableofcontents
    
    
    \newpage
    \pagenumbering{arabic}
    %\linenumbers\renewcommand\thelinenumber{\color{black!50}\arabic{linenumber}}
            \input{0 - introduction/main.tex}
        \part{Research}
            \input{1 - low-noise PiC models/main.tex}
            \input{2 - kinetic component/main.tex}
            \input{3 - fluid component/main.tex}
            \input{4 - numerical implementation/main.tex}
        \part{Project Overview}
            \input{5 - research plan/main.tex}
            \input{6 - summary/main.tex}
    
    
    %\section{}
    \newpage
    \pagenumbering{gobble}
        \printbibliography


    \newpage
    \pagenumbering{roman}
    \appendix
        \part{Appendices}
            \input{8 - Hilbert complexes/main.tex}
            \input{9 - weak conservation proofs/main.tex}
\end{document}

            \documentclass[12pt, a4paper]{report}

\input{template/main.tex}

\title{\BA{Title in Progress...}}
\author{Boris Andrews}
\affil{Mathematical Institute, University of Oxford}
\date{\today}


\begin{document}
    \pagenumbering{gobble}
    \maketitle
    
    
    \begin{abstract}
        Magnetic confinement reactors---in particular tokamaks---offer one of the most promising options for achieving practical nuclear fusion, with the potential to provide virtually limitless, clean energy. The theoretical and numerical modeling of tokamak plasmas is simultaneously an essential component of effective reactor design, and a great research barrier. Tokamak operational conditions exhibit comparatively low Knudsen numbers. Kinetic effects, including kinetic waves and instabilities, Landau damping, bump-on-tail instabilities and more, are therefore highly influential in tokamak plasma dynamics. Purely fluid models are inherently incapable of capturing these effects, whereas the high dimensionality in purely kinetic models render them practically intractable for most relevant purposes.

        We consider a $\delta\!f$ decomposition model, with a macroscopic fluid background and microscopic kinetic correction, both fully coupled to each other. A similar manner of discretization is proposed to that used in the recent \texttt{STRUPHY} code \cite{Holderied_Possanner_Wang_2021, Holderied_2022, Li_et_al_2023} with a finite-element model for the background and a pseudo-particle/PiC model for the correction.

        The fluid background satisfies the full, non-linear, resistive, compressible, Hall MHD equations. \cite{Laakmann_Hu_Farrell_2022} introduces finite-element(-in-space) implicit timesteppers for the incompressible analogue to this system with structure-preserving (SP) properties in the ideal case, alongside parameter-robust preconditioners. We show that these timesteppers can derive from a finite-element-in-time (FET) (and finite-element-in-space) interpretation. The benefits of this reformulation are discussed, including the derivation of timesteppers that are higher order in time, and the quantifiable dissipative SP properties in the non-ideal, resistive case.
        
        We discuss possible options for extending this FET approach to timesteppers for the compressible case.

        The kinetic corrections satisfy linearized Boltzmann equations. Using a Lénard--Bernstein collision operator, these take Fokker--Planck-like forms \cite{Fokker_1914, Planck_1917} wherein pseudo-particles in the numerical model obey the neoclassical transport equations, with particle-independent Brownian drift terms. This offers a rigorous methodology for incorporating collisions into the particle transport model, without coupling the equations of motions for each particle.
        
        Works by Chen, Chacón et al. \cite{Chen_Chacón_Barnes_2011, Chacón_Chen_Barnes_2013, Chen_Chacón_2014, Chen_Chacón_2015} have developed structure-preserving particle pushers for neoclassical transport in the Vlasov equations, derived from Crank--Nicolson integrators. We show these too can can derive from a FET interpretation, similarly offering potential extensions to higher-order-in-time particle pushers. The FET formulation is used also to consider how the stochastic drift terms can be incorporated into the pushers. Stochastic gyrokinetic expansions are also discussed.

        Different options for the numerical implementation of these schemes are considered.

        Due to the efficacy of FET in the development of SP timesteppers for both the fluid and kinetic component, we hope this approach will prove effective in the future for developing SP timesteppers for the full hybrid model. We hope this will give us the opportunity to incorporate previously inaccessible kinetic effects into the highly effective, modern, finite-element MHD models.
    \end{abstract}
    
    
    \newpage
    \tableofcontents
    
    
    \newpage
    \pagenumbering{arabic}
    %\linenumbers\renewcommand\thelinenumber{\color{black!50}\arabic{linenumber}}
            \input{0 - introduction/main.tex}
        \part{Research}
            \input{1 - low-noise PiC models/main.tex}
            \input{2 - kinetic component/main.tex}
            \input{3 - fluid component/main.tex}
            \input{4 - numerical implementation/main.tex}
        \part{Project Overview}
            \input{5 - research plan/main.tex}
            \input{6 - summary/main.tex}
    
    
    %\section{}
    \newpage
    \pagenumbering{gobble}
        \printbibliography


    \newpage
    \pagenumbering{roman}
    \appendix
        \part{Appendices}
            \input{8 - Hilbert complexes/main.tex}
            \input{9 - weak conservation proofs/main.tex}
\end{document}

            \documentclass[12pt, a4paper]{report}

\input{template/main.tex}

\title{\BA{Title in Progress...}}
\author{Boris Andrews}
\affil{Mathematical Institute, University of Oxford}
\date{\today}


\begin{document}
    \pagenumbering{gobble}
    \maketitle
    
    
    \begin{abstract}
        Magnetic confinement reactors---in particular tokamaks---offer one of the most promising options for achieving practical nuclear fusion, with the potential to provide virtually limitless, clean energy. The theoretical and numerical modeling of tokamak plasmas is simultaneously an essential component of effective reactor design, and a great research barrier. Tokamak operational conditions exhibit comparatively low Knudsen numbers. Kinetic effects, including kinetic waves and instabilities, Landau damping, bump-on-tail instabilities and more, are therefore highly influential in tokamak plasma dynamics. Purely fluid models are inherently incapable of capturing these effects, whereas the high dimensionality in purely kinetic models render them practically intractable for most relevant purposes.

        We consider a $\delta\!f$ decomposition model, with a macroscopic fluid background and microscopic kinetic correction, both fully coupled to each other. A similar manner of discretization is proposed to that used in the recent \texttt{STRUPHY} code \cite{Holderied_Possanner_Wang_2021, Holderied_2022, Li_et_al_2023} with a finite-element model for the background and a pseudo-particle/PiC model for the correction.

        The fluid background satisfies the full, non-linear, resistive, compressible, Hall MHD equations. \cite{Laakmann_Hu_Farrell_2022} introduces finite-element(-in-space) implicit timesteppers for the incompressible analogue to this system with structure-preserving (SP) properties in the ideal case, alongside parameter-robust preconditioners. We show that these timesteppers can derive from a finite-element-in-time (FET) (and finite-element-in-space) interpretation. The benefits of this reformulation are discussed, including the derivation of timesteppers that are higher order in time, and the quantifiable dissipative SP properties in the non-ideal, resistive case.
        
        We discuss possible options for extending this FET approach to timesteppers for the compressible case.

        The kinetic corrections satisfy linearized Boltzmann equations. Using a Lénard--Bernstein collision operator, these take Fokker--Planck-like forms \cite{Fokker_1914, Planck_1917} wherein pseudo-particles in the numerical model obey the neoclassical transport equations, with particle-independent Brownian drift terms. This offers a rigorous methodology for incorporating collisions into the particle transport model, without coupling the equations of motions for each particle.
        
        Works by Chen, Chacón et al. \cite{Chen_Chacón_Barnes_2011, Chacón_Chen_Barnes_2013, Chen_Chacón_2014, Chen_Chacón_2015} have developed structure-preserving particle pushers for neoclassical transport in the Vlasov equations, derived from Crank--Nicolson integrators. We show these too can can derive from a FET interpretation, similarly offering potential extensions to higher-order-in-time particle pushers. The FET formulation is used also to consider how the stochastic drift terms can be incorporated into the pushers. Stochastic gyrokinetic expansions are also discussed.

        Different options for the numerical implementation of these schemes are considered.

        Due to the efficacy of FET in the development of SP timesteppers for both the fluid and kinetic component, we hope this approach will prove effective in the future for developing SP timesteppers for the full hybrid model. We hope this will give us the opportunity to incorporate previously inaccessible kinetic effects into the highly effective, modern, finite-element MHD models.
    \end{abstract}
    
    
    \newpage
    \tableofcontents
    
    
    \newpage
    \pagenumbering{arabic}
    %\linenumbers\renewcommand\thelinenumber{\color{black!50}\arabic{linenumber}}
            \input{0 - introduction/main.tex}
        \part{Research}
            \input{1 - low-noise PiC models/main.tex}
            \input{2 - kinetic component/main.tex}
            \input{3 - fluid component/main.tex}
            \input{4 - numerical implementation/main.tex}
        \part{Project Overview}
            \input{5 - research plan/main.tex}
            \input{6 - summary/main.tex}
    
    
    %\section{}
    \newpage
    \pagenumbering{gobble}
        \printbibliography


    \newpage
    \pagenumbering{roman}
    \appendix
        \part{Appendices}
            \input{8 - Hilbert complexes/main.tex}
            \input{9 - weak conservation proofs/main.tex}
\end{document}

            \documentclass[12pt, a4paper]{report}

\input{template/main.tex}

\title{\BA{Title in Progress...}}
\author{Boris Andrews}
\affil{Mathematical Institute, University of Oxford}
\date{\today}


\begin{document}
    \pagenumbering{gobble}
    \maketitle
    
    
    \begin{abstract}
        Magnetic confinement reactors---in particular tokamaks---offer one of the most promising options for achieving practical nuclear fusion, with the potential to provide virtually limitless, clean energy. The theoretical and numerical modeling of tokamak plasmas is simultaneously an essential component of effective reactor design, and a great research barrier. Tokamak operational conditions exhibit comparatively low Knudsen numbers. Kinetic effects, including kinetic waves and instabilities, Landau damping, bump-on-tail instabilities and more, are therefore highly influential in tokamak plasma dynamics. Purely fluid models are inherently incapable of capturing these effects, whereas the high dimensionality in purely kinetic models render them practically intractable for most relevant purposes.

        We consider a $\delta\!f$ decomposition model, with a macroscopic fluid background and microscopic kinetic correction, both fully coupled to each other. A similar manner of discretization is proposed to that used in the recent \texttt{STRUPHY} code \cite{Holderied_Possanner_Wang_2021, Holderied_2022, Li_et_al_2023} with a finite-element model for the background and a pseudo-particle/PiC model for the correction.

        The fluid background satisfies the full, non-linear, resistive, compressible, Hall MHD equations. \cite{Laakmann_Hu_Farrell_2022} introduces finite-element(-in-space) implicit timesteppers for the incompressible analogue to this system with structure-preserving (SP) properties in the ideal case, alongside parameter-robust preconditioners. We show that these timesteppers can derive from a finite-element-in-time (FET) (and finite-element-in-space) interpretation. The benefits of this reformulation are discussed, including the derivation of timesteppers that are higher order in time, and the quantifiable dissipative SP properties in the non-ideal, resistive case.
        
        We discuss possible options for extending this FET approach to timesteppers for the compressible case.

        The kinetic corrections satisfy linearized Boltzmann equations. Using a Lénard--Bernstein collision operator, these take Fokker--Planck-like forms \cite{Fokker_1914, Planck_1917} wherein pseudo-particles in the numerical model obey the neoclassical transport equations, with particle-independent Brownian drift terms. This offers a rigorous methodology for incorporating collisions into the particle transport model, without coupling the equations of motions for each particle.
        
        Works by Chen, Chacón et al. \cite{Chen_Chacón_Barnes_2011, Chacón_Chen_Barnes_2013, Chen_Chacón_2014, Chen_Chacón_2015} have developed structure-preserving particle pushers for neoclassical transport in the Vlasov equations, derived from Crank--Nicolson integrators. We show these too can can derive from a FET interpretation, similarly offering potential extensions to higher-order-in-time particle pushers. The FET formulation is used also to consider how the stochastic drift terms can be incorporated into the pushers. Stochastic gyrokinetic expansions are also discussed.

        Different options for the numerical implementation of these schemes are considered.

        Due to the efficacy of FET in the development of SP timesteppers for both the fluid and kinetic component, we hope this approach will prove effective in the future for developing SP timesteppers for the full hybrid model. We hope this will give us the opportunity to incorporate previously inaccessible kinetic effects into the highly effective, modern, finite-element MHD models.
    \end{abstract}
    
    
    \newpage
    \tableofcontents
    
    
    \newpage
    \pagenumbering{arabic}
    %\linenumbers\renewcommand\thelinenumber{\color{black!50}\arabic{linenumber}}
            \input{0 - introduction/main.tex}
        \part{Research}
            \input{1 - low-noise PiC models/main.tex}
            \input{2 - kinetic component/main.tex}
            \input{3 - fluid component/main.tex}
            \input{4 - numerical implementation/main.tex}
        \part{Project Overview}
            \input{5 - research plan/main.tex}
            \input{6 - summary/main.tex}
    
    
    %\section{}
    \newpage
    \pagenumbering{gobble}
        \printbibliography


    \newpage
    \pagenumbering{roman}
    \appendix
        \part{Appendices}
            \input{8 - Hilbert complexes/main.tex}
            \input{9 - weak conservation proofs/main.tex}
\end{document}

        \part{Project Overview}
            \documentclass[12pt, a4paper]{report}

\input{template/main.tex}

\title{\BA{Title in Progress...}}
\author{Boris Andrews}
\affil{Mathematical Institute, University of Oxford}
\date{\today}


\begin{document}
    \pagenumbering{gobble}
    \maketitle
    
    
    \begin{abstract}
        Magnetic confinement reactors---in particular tokamaks---offer one of the most promising options for achieving practical nuclear fusion, with the potential to provide virtually limitless, clean energy. The theoretical and numerical modeling of tokamak plasmas is simultaneously an essential component of effective reactor design, and a great research barrier. Tokamak operational conditions exhibit comparatively low Knudsen numbers. Kinetic effects, including kinetic waves and instabilities, Landau damping, bump-on-tail instabilities and more, are therefore highly influential in tokamak plasma dynamics. Purely fluid models are inherently incapable of capturing these effects, whereas the high dimensionality in purely kinetic models render them practically intractable for most relevant purposes.

        We consider a $\delta\!f$ decomposition model, with a macroscopic fluid background and microscopic kinetic correction, both fully coupled to each other. A similar manner of discretization is proposed to that used in the recent \texttt{STRUPHY} code \cite{Holderied_Possanner_Wang_2021, Holderied_2022, Li_et_al_2023} with a finite-element model for the background and a pseudo-particle/PiC model for the correction.

        The fluid background satisfies the full, non-linear, resistive, compressible, Hall MHD equations. \cite{Laakmann_Hu_Farrell_2022} introduces finite-element(-in-space) implicit timesteppers for the incompressible analogue to this system with structure-preserving (SP) properties in the ideal case, alongside parameter-robust preconditioners. We show that these timesteppers can derive from a finite-element-in-time (FET) (and finite-element-in-space) interpretation. The benefits of this reformulation are discussed, including the derivation of timesteppers that are higher order in time, and the quantifiable dissipative SP properties in the non-ideal, resistive case.
        
        We discuss possible options for extending this FET approach to timesteppers for the compressible case.

        The kinetic corrections satisfy linearized Boltzmann equations. Using a Lénard--Bernstein collision operator, these take Fokker--Planck-like forms \cite{Fokker_1914, Planck_1917} wherein pseudo-particles in the numerical model obey the neoclassical transport equations, with particle-independent Brownian drift terms. This offers a rigorous methodology for incorporating collisions into the particle transport model, without coupling the equations of motions for each particle.
        
        Works by Chen, Chacón et al. \cite{Chen_Chacón_Barnes_2011, Chacón_Chen_Barnes_2013, Chen_Chacón_2014, Chen_Chacón_2015} have developed structure-preserving particle pushers for neoclassical transport in the Vlasov equations, derived from Crank--Nicolson integrators. We show these too can can derive from a FET interpretation, similarly offering potential extensions to higher-order-in-time particle pushers. The FET formulation is used also to consider how the stochastic drift terms can be incorporated into the pushers. Stochastic gyrokinetic expansions are also discussed.

        Different options for the numerical implementation of these schemes are considered.

        Due to the efficacy of FET in the development of SP timesteppers for both the fluid and kinetic component, we hope this approach will prove effective in the future for developing SP timesteppers for the full hybrid model. We hope this will give us the opportunity to incorporate previously inaccessible kinetic effects into the highly effective, modern, finite-element MHD models.
    \end{abstract}
    
    
    \newpage
    \tableofcontents
    
    
    \newpage
    \pagenumbering{arabic}
    %\linenumbers\renewcommand\thelinenumber{\color{black!50}\arabic{linenumber}}
            \input{0 - introduction/main.tex}
        \part{Research}
            \input{1 - low-noise PiC models/main.tex}
            \input{2 - kinetic component/main.tex}
            \input{3 - fluid component/main.tex}
            \input{4 - numerical implementation/main.tex}
        \part{Project Overview}
            \input{5 - research plan/main.tex}
            \input{6 - summary/main.tex}
    
    
    %\section{}
    \newpage
    \pagenumbering{gobble}
        \printbibliography


    \newpage
    \pagenumbering{roman}
    \appendix
        \part{Appendices}
            \input{8 - Hilbert complexes/main.tex}
            \input{9 - weak conservation proofs/main.tex}
\end{document}

            \documentclass[12pt, a4paper]{report}

\input{template/main.tex}

\title{\BA{Title in Progress...}}
\author{Boris Andrews}
\affil{Mathematical Institute, University of Oxford}
\date{\today}


\begin{document}
    \pagenumbering{gobble}
    \maketitle
    
    
    \begin{abstract}
        Magnetic confinement reactors---in particular tokamaks---offer one of the most promising options for achieving practical nuclear fusion, with the potential to provide virtually limitless, clean energy. The theoretical and numerical modeling of tokamak plasmas is simultaneously an essential component of effective reactor design, and a great research barrier. Tokamak operational conditions exhibit comparatively low Knudsen numbers. Kinetic effects, including kinetic waves and instabilities, Landau damping, bump-on-tail instabilities and more, are therefore highly influential in tokamak plasma dynamics. Purely fluid models are inherently incapable of capturing these effects, whereas the high dimensionality in purely kinetic models render them practically intractable for most relevant purposes.

        We consider a $\delta\!f$ decomposition model, with a macroscopic fluid background and microscopic kinetic correction, both fully coupled to each other. A similar manner of discretization is proposed to that used in the recent \texttt{STRUPHY} code \cite{Holderied_Possanner_Wang_2021, Holderied_2022, Li_et_al_2023} with a finite-element model for the background and a pseudo-particle/PiC model for the correction.

        The fluid background satisfies the full, non-linear, resistive, compressible, Hall MHD equations. \cite{Laakmann_Hu_Farrell_2022} introduces finite-element(-in-space) implicit timesteppers for the incompressible analogue to this system with structure-preserving (SP) properties in the ideal case, alongside parameter-robust preconditioners. We show that these timesteppers can derive from a finite-element-in-time (FET) (and finite-element-in-space) interpretation. The benefits of this reformulation are discussed, including the derivation of timesteppers that are higher order in time, and the quantifiable dissipative SP properties in the non-ideal, resistive case.
        
        We discuss possible options for extending this FET approach to timesteppers for the compressible case.

        The kinetic corrections satisfy linearized Boltzmann equations. Using a Lénard--Bernstein collision operator, these take Fokker--Planck-like forms \cite{Fokker_1914, Planck_1917} wherein pseudo-particles in the numerical model obey the neoclassical transport equations, with particle-independent Brownian drift terms. This offers a rigorous methodology for incorporating collisions into the particle transport model, without coupling the equations of motions for each particle.
        
        Works by Chen, Chacón et al. \cite{Chen_Chacón_Barnes_2011, Chacón_Chen_Barnes_2013, Chen_Chacón_2014, Chen_Chacón_2015} have developed structure-preserving particle pushers for neoclassical transport in the Vlasov equations, derived from Crank--Nicolson integrators. We show these too can can derive from a FET interpretation, similarly offering potential extensions to higher-order-in-time particle pushers. The FET formulation is used also to consider how the stochastic drift terms can be incorporated into the pushers. Stochastic gyrokinetic expansions are also discussed.

        Different options for the numerical implementation of these schemes are considered.

        Due to the efficacy of FET in the development of SP timesteppers for both the fluid and kinetic component, we hope this approach will prove effective in the future for developing SP timesteppers for the full hybrid model. We hope this will give us the opportunity to incorporate previously inaccessible kinetic effects into the highly effective, modern, finite-element MHD models.
    \end{abstract}
    
    
    \newpage
    \tableofcontents
    
    
    \newpage
    \pagenumbering{arabic}
    %\linenumbers\renewcommand\thelinenumber{\color{black!50}\arabic{linenumber}}
            \input{0 - introduction/main.tex}
        \part{Research}
            \input{1 - low-noise PiC models/main.tex}
            \input{2 - kinetic component/main.tex}
            \input{3 - fluid component/main.tex}
            \input{4 - numerical implementation/main.tex}
        \part{Project Overview}
            \input{5 - research plan/main.tex}
            \input{6 - summary/main.tex}
    
    
    %\section{}
    \newpage
    \pagenumbering{gobble}
        \printbibliography


    \newpage
    \pagenumbering{roman}
    \appendix
        \part{Appendices}
            \input{8 - Hilbert complexes/main.tex}
            \input{9 - weak conservation proofs/main.tex}
\end{document}

    
    
    %\section{}
    \newpage
    \pagenumbering{gobble}
        \printbibliography


    \newpage
    \pagenumbering{roman}
    \appendix
        \part{Appendices}
            \documentclass[12pt, a4paper]{report}

\input{template/main.tex}

\title{\BA{Title in Progress...}}
\author{Boris Andrews}
\affil{Mathematical Institute, University of Oxford}
\date{\today}


\begin{document}
    \pagenumbering{gobble}
    \maketitle
    
    
    \begin{abstract}
        Magnetic confinement reactors---in particular tokamaks---offer one of the most promising options for achieving practical nuclear fusion, with the potential to provide virtually limitless, clean energy. The theoretical and numerical modeling of tokamak plasmas is simultaneously an essential component of effective reactor design, and a great research barrier. Tokamak operational conditions exhibit comparatively low Knudsen numbers. Kinetic effects, including kinetic waves and instabilities, Landau damping, bump-on-tail instabilities and more, are therefore highly influential in tokamak plasma dynamics. Purely fluid models are inherently incapable of capturing these effects, whereas the high dimensionality in purely kinetic models render them practically intractable for most relevant purposes.

        We consider a $\delta\!f$ decomposition model, with a macroscopic fluid background and microscopic kinetic correction, both fully coupled to each other. A similar manner of discretization is proposed to that used in the recent \texttt{STRUPHY} code \cite{Holderied_Possanner_Wang_2021, Holderied_2022, Li_et_al_2023} with a finite-element model for the background and a pseudo-particle/PiC model for the correction.

        The fluid background satisfies the full, non-linear, resistive, compressible, Hall MHD equations. \cite{Laakmann_Hu_Farrell_2022} introduces finite-element(-in-space) implicit timesteppers for the incompressible analogue to this system with structure-preserving (SP) properties in the ideal case, alongside parameter-robust preconditioners. We show that these timesteppers can derive from a finite-element-in-time (FET) (and finite-element-in-space) interpretation. The benefits of this reformulation are discussed, including the derivation of timesteppers that are higher order in time, and the quantifiable dissipative SP properties in the non-ideal, resistive case.
        
        We discuss possible options for extending this FET approach to timesteppers for the compressible case.

        The kinetic corrections satisfy linearized Boltzmann equations. Using a Lénard--Bernstein collision operator, these take Fokker--Planck-like forms \cite{Fokker_1914, Planck_1917} wherein pseudo-particles in the numerical model obey the neoclassical transport equations, with particle-independent Brownian drift terms. This offers a rigorous methodology for incorporating collisions into the particle transport model, without coupling the equations of motions for each particle.
        
        Works by Chen, Chacón et al. \cite{Chen_Chacón_Barnes_2011, Chacón_Chen_Barnes_2013, Chen_Chacón_2014, Chen_Chacón_2015} have developed structure-preserving particle pushers for neoclassical transport in the Vlasov equations, derived from Crank--Nicolson integrators. We show these too can can derive from a FET interpretation, similarly offering potential extensions to higher-order-in-time particle pushers. The FET formulation is used also to consider how the stochastic drift terms can be incorporated into the pushers. Stochastic gyrokinetic expansions are also discussed.

        Different options for the numerical implementation of these schemes are considered.

        Due to the efficacy of FET in the development of SP timesteppers for both the fluid and kinetic component, we hope this approach will prove effective in the future for developing SP timesteppers for the full hybrid model. We hope this will give us the opportunity to incorporate previously inaccessible kinetic effects into the highly effective, modern, finite-element MHD models.
    \end{abstract}
    
    
    \newpage
    \tableofcontents
    
    
    \newpage
    \pagenumbering{arabic}
    %\linenumbers\renewcommand\thelinenumber{\color{black!50}\arabic{linenumber}}
            \input{0 - introduction/main.tex}
        \part{Research}
            \input{1 - low-noise PiC models/main.tex}
            \input{2 - kinetic component/main.tex}
            \input{3 - fluid component/main.tex}
            \input{4 - numerical implementation/main.tex}
        \part{Project Overview}
            \input{5 - research plan/main.tex}
            \input{6 - summary/main.tex}
    
    
    %\section{}
    \newpage
    \pagenumbering{gobble}
        \printbibliography


    \newpage
    \pagenumbering{roman}
    \appendix
        \part{Appendices}
            \input{8 - Hilbert complexes/main.tex}
            \input{9 - weak conservation proofs/main.tex}
\end{document}

            \documentclass[12pt, a4paper]{report}

\input{template/main.tex}

\title{\BA{Title in Progress...}}
\author{Boris Andrews}
\affil{Mathematical Institute, University of Oxford}
\date{\today}


\begin{document}
    \pagenumbering{gobble}
    \maketitle
    
    
    \begin{abstract}
        Magnetic confinement reactors---in particular tokamaks---offer one of the most promising options for achieving practical nuclear fusion, with the potential to provide virtually limitless, clean energy. The theoretical and numerical modeling of tokamak plasmas is simultaneously an essential component of effective reactor design, and a great research barrier. Tokamak operational conditions exhibit comparatively low Knudsen numbers. Kinetic effects, including kinetic waves and instabilities, Landau damping, bump-on-tail instabilities and more, are therefore highly influential in tokamak plasma dynamics. Purely fluid models are inherently incapable of capturing these effects, whereas the high dimensionality in purely kinetic models render them practically intractable for most relevant purposes.

        We consider a $\delta\!f$ decomposition model, with a macroscopic fluid background and microscopic kinetic correction, both fully coupled to each other. A similar manner of discretization is proposed to that used in the recent \texttt{STRUPHY} code \cite{Holderied_Possanner_Wang_2021, Holderied_2022, Li_et_al_2023} with a finite-element model for the background and a pseudo-particle/PiC model for the correction.

        The fluid background satisfies the full, non-linear, resistive, compressible, Hall MHD equations. \cite{Laakmann_Hu_Farrell_2022} introduces finite-element(-in-space) implicit timesteppers for the incompressible analogue to this system with structure-preserving (SP) properties in the ideal case, alongside parameter-robust preconditioners. We show that these timesteppers can derive from a finite-element-in-time (FET) (and finite-element-in-space) interpretation. The benefits of this reformulation are discussed, including the derivation of timesteppers that are higher order in time, and the quantifiable dissipative SP properties in the non-ideal, resistive case.
        
        We discuss possible options for extending this FET approach to timesteppers for the compressible case.

        The kinetic corrections satisfy linearized Boltzmann equations. Using a Lénard--Bernstein collision operator, these take Fokker--Planck-like forms \cite{Fokker_1914, Planck_1917} wherein pseudo-particles in the numerical model obey the neoclassical transport equations, with particle-independent Brownian drift terms. This offers a rigorous methodology for incorporating collisions into the particle transport model, without coupling the equations of motions for each particle.
        
        Works by Chen, Chacón et al. \cite{Chen_Chacón_Barnes_2011, Chacón_Chen_Barnes_2013, Chen_Chacón_2014, Chen_Chacón_2015} have developed structure-preserving particle pushers for neoclassical transport in the Vlasov equations, derived from Crank--Nicolson integrators. We show these too can can derive from a FET interpretation, similarly offering potential extensions to higher-order-in-time particle pushers. The FET formulation is used also to consider how the stochastic drift terms can be incorporated into the pushers. Stochastic gyrokinetic expansions are also discussed.

        Different options for the numerical implementation of these schemes are considered.

        Due to the efficacy of FET in the development of SP timesteppers for both the fluid and kinetic component, we hope this approach will prove effective in the future for developing SP timesteppers for the full hybrid model. We hope this will give us the opportunity to incorporate previously inaccessible kinetic effects into the highly effective, modern, finite-element MHD models.
    \end{abstract}
    
    
    \newpage
    \tableofcontents
    
    
    \newpage
    \pagenumbering{arabic}
    %\linenumbers\renewcommand\thelinenumber{\color{black!50}\arabic{linenumber}}
            \input{0 - introduction/main.tex}
        \part{Research}
            \input{1 - low-noise PiC models/main.tex}
            \input{2 - kinetic component/main.tex}
            \input{3 - fluid component/main.tex}
            \input{4 - numerical implementation/main.tex}
        \part{Project Overview}
            \input{5 - research plan/main.tex}
            \input{6 - summary/main.tex}
    
    
    %\section{}
    \newpage
    \pagenumbering{gobble}
        \printbibliography


    \newpage
    \pagenumbering{roman}
    \appendix
        \part{Appendices}
            \input{8 - Hilbert complexes/main.tex}
            \input{9 - weak conservation proofs/main.tex}
\end{document}

\end{document}

            \documentclass[12pt, a4paper]{report}

\documentclass[12pt, a4paper]{report}

\input{template/main.tex}

\title{\BA{Title in Progress...}}
\author{Boris Andrews}
\affil{Mathematical Institute, University of Oxford}
\date{\today}


\begin{document}
    \pagenumbering{gobble}
    \maketitle
    
    
    \begin{abstract}
        Magnetic confinement reactors---in particular tokamaks---offer one of the most promising options for achieving practical nuclear fusion, with the potential to provide virtually limitless, clean energy. The theoretical and numerical modeling of tokamak plasmas is simultaneously an essential component of effective reactor design, and a great research barrier. Tokamak operational conditions exhibit comparatively low Knudsen numbers. Kinetic effects, including kinetic waves and instabilities, Landau damping, bump-on-tail instabilities and more, are therefore highly influential in tokamak plasma dynamics. Purely fluid models are inherently incapable of capturing these effects, whereas the high dimensionality in purely kinetic models render them practically intractable for most relevant purposes.

        We consider a $\delta\!f$ decomposition model, with a macroscopic fluid background and microscopic kinetic correction, both fully coupled to each other. A similar manner of discretization is proposed to that used in the recent \texttt{STRUPHY} code \cite{Holderied_Possanner_Wang_2021, Holderied_2022, Li_et_al_2023} with a finite-element model for the background and a pseudo-particle/PiC model for the correction.

        The fluid background satisfies the full, non-linear, resistive, compressible, Hall MHD equations. \cite{Laakmann_Hu_Farrell_2022} introduces finite-element(-in-space) implicit timesteppers for the incompressible analogue to this system with structure-preserving (SP) properties in the ideal case, alongside parameter-robust preconditioners. We show that these timesteppers can derive from a finite-element-in-time (FET) (and finite-element-in-space) interpretation. The benefits of this reformulation are discussed, including the derivation of timesteppers that are higher order in time, and the quantifiable dissipative SP properties in the non-ideal, resistive case.
        
        We discuss possible options for extending this FET approach to timesteppers for the compressible case.

        The kinetic corrections satisfy linearized Boltzmann equations. Using a Lénard--Bernstein collision operator, these take Fokker--Planck-like forms \cite{Fokker_1914, Planck_1917} wherein pseudo-particles in the numerical model obey the neoclassical transport equations, with particle-independent Brownian drift terms. This offers a rigorous methodology for incorporating collisions into the particle transport model, without coupling the equations of motions for each particle.
        
        Works by Chen, Chacón et al. \cite{Chen_Chacón_Barnes_2011, Chacón_Chen_Barnes_2013, Chen_Chacón_2014, Chen_Chacón_2015} have developed structure-preserving particle pushers for neoclassical transport in the Vlasov equations, derived from Crank--Nicolson integrators. We show these too can can derive from a FET interpretation, similarly offering potential extensions to higher-order-in-time particle pushers. The FET formulation is used also to consider how the stochastic drift terms can be incorporated into the pushers. Stochastic gyrokinetic expansions are also discussed.

        Different options for the numerical implementation of these schemes are considered.

        Due to the efficacy of FET in the development of SP timesteppers for both the fluid and kinetic component, we hope this approach will prove effective in the future for developing SP timesteppers for the full hybrid model. We hope this will give us the opportunity to incorporate previously inaccessible kinetic effects into the highly effective, modern, finite-element MHD models.
    \end{abstract}
    
    
    \newpage
    \tableofcontents
    
    
    \newpage
    \pagenumbering{arabic}
    %\linenumbers\renewcommand\thelinenumber{\color{black!50}\arabic{linenumber}}
            \input{0 - introduction/main.tex}
        \part{Research}
            \input{1 - low-noise PiC models/main.tex}
            \input{2 - kinetic component/main.tex}
            \input{3 - fluid component/main.tex}
            \input{4 - numerical implementation/main.tex}
        \part{Project Overview}
            \input{5 - research plan/main.tex}
            \input{6 - summary/main.tex}
    
    
    %\section{}
    \newpage
    \pagenumbering{gobble}
        \printbibliography


    \newpage
    \pagenumbering{roman}
    \appendix
        \part{Appendices}
            \input{8 - Hilbert complexes/main.tex}
            \input{9 - weak conservation proofs/main.tex}
\end{document}


\title{\BA{Title in Progress...}}
\author{Boris Andrews}
\affil{Mathematical Institute, University of Oxford}
\date{\today}


\begin{document}
    \pagenumbering{gobble}
    \maketitle
    
    
    \begin{abstract}
        Magnetic confinement reactors---in particular tokamaks---offer one of the most promising options for achieving practical nuclear fusion, with the potential to provide virtually limitless, clean energy. The theoretical and numerical modeling of tokamak plasmas is simultaneously an essential component of effective reactor design, and a great research barrier. Tokamak operational conditions exhibit comparatively low Knudsen numbers. Kinetic effects, including kinetic waves and instabilities, Landau damping, bump-on-tail instabilities and more, are therefore highly influential in tokamak plasma dynamics. Purely fluid models are inherently incapable of capturing these effects, whereas the high dimensionality in purely kinetic models render them practically intractable for most relevant purposes.

        We consider a $\delta\!f$ decomposition model, with a macroscopic fluid background and microscopic kinetic correction, both fully coupled to each other. A similar manner of discretization is proposed to that used in the recent \texttt{STRUPHY} code \cite{Holderied_Possanner_Wang_2021, Holderied_2022, Li_et_al_2023} with a finite-element model for the background and a pseudo-particle/PiC model for the correction.

        The fluid background satisfies the full, non-linear, resistive, compressible, Hall MHD equations. \cite{Laakmann_Hu_Farrell_2022} introduces finite-element(-in-space) implicit timesteppers for the incompressible analogue to this system with structure-preserving (SP) properties in the ideal case, alongside parameter-robust preconditioners. We show that these timesteppers can derive from a finite-element-in-time (FET) (and finite-element-in-space) interpretation. The benefits of this reformulation are discussed, including the derivation of timesteppers that are higher order in time, and the quantifiable dissipative SP properties in the non-ideal, resistive case.
        
        We discuss possible options for extending this FET approach to timesteppers for the compressible case.

        The kinetic corrections satisfy linearized Boltzmann equations. Using a Lénard--Bernstein collision operator, these take Fokker--Planck-like forms \cite{Fokker_1914, Planck_1917} wherein pseudo-particles in the numerical model obey the neoclassical transport equations, with particle-independent Brownian drift terms. This offers a rigorous methodology for incorporating collisions into the particle transport model, without coupling the equations of motions for each particle.
        
        Works by Chen, Chacón et al. \cite{Chen_Chacón_Barnes_2011, Chacón_Chen_Barnes_2013, Chen_Chacón_2014, Chen_Chacón_2015} have developed structure-preserving particle pushers for neoclassical transport in the Vlasov equations, derived from Crank--Nicolson integrators. We show these too can can derive from a FET interpretation, similarly offering potential extensions to higher-order-in-time particle pushers. The FET formulation is used also to consider how the stochastic drift terms can be incorporated into the pushers. Stochastic gyrokinetic expansions are also discussed.

        Different options for the numerical implementation of these schemes are considered.

        Due to the efficacy of FET in the development of SP timesteppers for both the fluid and kinetic component, we hope this approach will prove effective in the future for developing SP timesteppers for the full hybrid model. We hope this will give us the opportunity to incorporate previously inaccessible kinetic effects into the highly effective, modern, finite-element MHD models.
    \end{abstract}
    
    
    \newpage
    \tableofcontents
    
    
    \newpage
    \pagenumbering{arabic}
    %\linenumbers\renewcommand\thelinenumber{\color{black!50}\arabic{linenumber}}
            \documentclass[12pt, a4paper]{report}

\input{template/main.tex}

\title{\BA{Title in Progress...}}
\author{Boris Andrews}
\affil{Mathematical Institute, University of Oxford}
\date{\today}


\begin{document}
    \pagenumbering{gobble}
    \maketitle
    
    
    \begin{abstract}
        Magnetic confinement reactors---in particular tokamaks---offer one of the most promising options for achieving practical nuclear fusion, with the potential to provide virtually limitless, clean energy. The theoretical and numerical modeling of tokamak plasmas is simultaneously an essential component of effective reactor design, and a great research barrier. Tokamak operational conditions exhibit comparatively low Knudsen numbers. Kinetic effects, including kinetic waves and instabilities, Landau damping, bump-on-tail instabilities and more, are therefore highly influential in tokamak plasma dynamics. Purely fluid models are inherently incapable of capturing these effects, whereas the high dimensionality in purely kinetic models render them practically intractable for most relevant purposes.

        We consider a $\delta\!f$ decomposition model, with a macroscopic fluid background and microscopic kinetic correction, both fully coupled to each other. A similar manner of discretization is proposed to that used in the recent \texttt{STRUPHY} code \cite{Holderied_Possanner_Wang_2021, Holderied_2022, Li_et_al_2023} with a finite-element model for the background and a pseudo-particle/PiC model for the correction.

        The fluid background satisfies the full, non-linear, resistive, compressible, Hall MHD equations. \cite{Laakmann_Hu_Farrell_2022} introduces finite-element(-in-space) implicit timesteppers for the incompressible analogue to this system with structure-preserving (SP) properties in the ideal case, alongside parameter-robust preconditioners. We show that these timesteppers can derive from a finite-element-in-time (FET) (and finite-element-in-space) interpretation. The benefits of this reformulation are discussed, including the derivation of timesteppers that are higher order in time, and the quantifiable dissipative SP properties in the non-ideal, resistive case.
        
        We discuss possible options for extending this FET approach to timesteppers for the compressible case.

        The kinetic corrections satisfy linearized Boltzmann equations. Using a Lénard--Bernstein collision operator, these take Fokker--Planck-like forms \cite{Fokker_1914, Planck_1917} wherein pseudo-particles in the numerical model obey the neoclassical transport equations, with particle-independent Brownian drift terms. This offers a rigorous methodology for incorporating collisions into the particle transport model, without coupling the equations of motions for each particle.
        
        Works by Chen, Chacón et al. \cite{Chen_Chacón_Barnes_2011, Chacón_Chen_Barnes_2013, Chen_Chacón_2014, Chen_Chacón_2015} have developed structure-preserving particle pushers for neoclassical transport in the Vlasov equations, derived from Crank--Nicolson integrators. We show these too can can derive from a FET interpretation, similarly offering potential extensions to higher-order-in-time particle pushers. The FET formulation is used also to consider how the stochastic drift terms can be incorporated into the pushers. Stochastic gyrokinetic expansions are also discussed.

        Different options for the numerical implementation of these schemes are considered.

        Due to the efficacy of FET in the development of SP timesteppers for both the fluid and kinetic component, we hope this approach will prove effective in the future for developing SP timesteppers for the full hybrid model. We hope this will give us the opportunity to incorporate previously inaccessible kinetic effects into the highly effective, modern, finite-element MHD models.
    \end{abstract}
    
    
    \newpage
    \tableofcontents
    
    
    \newpage
    \pagenumbering{arabic}
    %\linenumbers\renewcommand\thelinenumber{\color{black!50}\arabic{linenumber}}
            \input{0 - introduction/main.tex}
        \part{Research}
            \input{1 - low-noise PiC models/main.tex}
            \input{2 - kinetic component/main.tex}
            \input{3 - fluid component/main.tex}
            \input{4 - numerical implementation/main.tex}
        \part{Project Overview}
            \input{5 - research plan/main.tex}
            \input{6 - summary/main.tex}
    
    
    %\section{}
    \newpage
    \pagenumbering{gobble}
        \printbibliography


    \newpage
    \pagenumbering{roman}
    \appendix
        \part{Appendices}
            \input{8 - Hilbert complexes/main.tex}
            \input{9 - weak conservation proofs/main.tex}
\end{document}

        \part{Research}
            \documentclass[12pt, a4paper]{report}

\input{template/main.tex}

\title{\BA{Title in Progress...}}
\author{Boris Andrews}
\affil{Mathematical Institute, University of Oxford}
\date{\today}


\begin{document}
    \pagenumbering{gobble}
    \maketitle
    
    
    \begin{abstract}
        Magnetic confinement reactors---in particular tokamaks---offer one of the most promising options for achieving practical nuclear fusion, with the potential to provide virtually limitless, clean energy. The theoretical and numerical modeling of tokamak plasmas is simultaneously an essential component of effective reactor design, and a great research barrier. Tokamak operational conditions exhibit comparatively low Knudsen numbers. Kinetic effects, including kinetic waves and instabilities, Landau damping, bump-on-tail instabilities and more, are therefore highly influential in tokamak plasma dynamics. Purely fluid models are inherently incapable of capturing these effects, whereas the high dimensionality in purely kinetic models render them practically intractable for most relevant purposes.

        We consider a $\delta\!f$ decomposition model, with a macroscopic fluid background and microscopic kinetic correction, both fully coupled to each other. A similar manner of discretization is proposed to that used in the recent \texttt{STRUPHY} code \cite{Holderied_Possanner_Wang_2021, Holderied_2022, Li_et_al_2023} with a finite-element model for the background and a pseudo-particle/PiC model for the correction.

        The fluid background satisfies the full, non-linear, resistive, compressible, Hall MHD equations. \cite{Laakmann_Hu_Farrell_2022} introduces finite-element(-in-space) implicit timesteppers for the incompressible analogue to this system with structure-preserving (SP) properties in the ideal case, alongside parameter-robust preconditioners. We show that these timesteppers can derive from a finite-element-in-time (FET) (and finite-element-in-space) interpretation. The benefits of this reformulation are discussed, including the derivation of timesteppers that are higher order in time, and the quantifiable dissipative SP properties in the non-ideal, resistive case.
        
        We discuss possible options for extending this FET approach to timesteppers for the compressible case.

        The kinetic corrections satisfy linearized Boltzmann equations. Using a Lénard--Bernstein collision operator, these take Fokker--Planck-like forms \cite{Fokker_1914, Planck_1917} wherein pseudo-particles in the numerical model obey the neoclassical transport equations, with particle-independent Brownian drift terms. This offers a rigorous methodology for incorporating collisions into the particle transport model, without coupling the equations of motions for each particle.
        
        Works by Chen, Chacón et al. \cite{Chen_Chacón_Barnes_2011, Chacón_Chen_Barnes_2013, Chen_Chacón_2014, Chen_Chacón_2015} have developed structure-preserving particle pushers for neoclassical transport in the Vlasov equations, derived from Crank--Nicolson integrators. We show these too can can derive from a FET interpretation, similarly offering potential extensions to higher-order-in-time particle pushers. The FET formulation is used also to consider how the stochastic drift terms can be incorporated into the pushers. Stochastic gyrokinetic expansions are also discussed.

        Different options for the numerical implementation of these schemes are considered.

        Due to the efficacy of FET in the development of SP timesteppers for both the fluid and kinetic component, we hope this approach will prove effective in the future for developing SP timesteppers for the full hybrid model. We hope this will give us the opportunity to incorporate previously inaccessible kinetic effects into the highly effective, modern, finite-element MHD models.
    \end{abstract}
    
    
    \newpage
    \tableofcontents
    
    
    \newpage
    \pagenumbering{arabic}
    %\linenumbers\renewcommand\thelinenumber{\color{black!50}\arabic{linenumber}}
            \input{0 - introduction/main.tex}
        \part{Research}
            \input{1 - low-noise PiC models/main.tex}
            \input{2 - kinetic component/main.tex}
            \input{3 - fluid component/main.tex}
            \input{4 - numerical implementation/main.tex}
        \part{Project Overview}
            \input{5 - research plan/main.tex}
            \input{6 - summary/main.tex}
    
    
    %\section{}
    \newpage
    \pagenumbering{gobble}
        \printbibliography


    \newpage
    \pagenumbering{roman}
    \appendix
        \part{Appendices}
            \input{8 - Hilbert complexes/main.tex}
            \input{9 - weak conservation proofs/main.tex}
\end{document}

            \documentclass[12pt, a4paper]{report}

\input{template/main.tex}

\title{\BA{Title in Progress...}}
\author{Boris Andrews}
\affil{Mathematical Institute, University of Oxford}
\date{\today}


\begin{document}
    \pagenumbering{gobble}
    \maketitle
    
    
    \begin{abstract}
        Magnetic confinement reactors---in particular tokamaks---offer one of the most promising options for achieving practical nuclear fusion, with the potential to provide virtually limitless, clean energy. The theoretical and numerical modeling of tokamak plasmas is simultaneously an essential component of effective reactor design, and a great research barrier. Tokamak operational conditions exhibit comparatively low Knudsen numbers. Kinetic effects, including kinetic waves and instabilities, Landau damping, bump-on-tail instabilities and more, are therefore highly influential in tokamak plasma dynamics. Purely fluid models are inherently incapable of capturing these effects, whereas the high dimensionality in purely kinetic models render them practically intractable for most relevant purposes.

        We consider a $\delta\!f$ decomposition model, with a macroscopic fluid background and microscopic kinetic correction, both fully coupled to each other. A similar manner of discretization is proposed to that used in the recent \texttt{STRUPHY} code \cite{Holderied_Possanner_Wang_2021, Holderied_2022, Li_et_al_2023} with a finite-element model for the background and a pseudo-particle/PiC model for the correction.

        The fluid background satisfies the full, non-linear, resistive, compressible, Hall MHD equations. \cite{Laakmann_Hu_Farrell_2022} introduces finite-element(-in-space) implicit timesteppers for the incompressible analogue to this system with structure-preserving (SP) properties in the ideal case, alongside parameter-robust preconditioners. We show that these timesteppers can derive from a finite-element-in-time (FET) (and finite-element-in-space) interpretation. The benefits of this reformulation are discussed, including the derivation of timesteppers that are higher order in time, and the quantifiable dissipative SP properties in the non-ideal, resistive case.
        
        We discuss possible options for extending this FET approach to timesteppers for the compressible case.

        The kinetic corrections satisfy linearized Boltzmann equations. Using a Lénard--Bernstein collision operator, these take Fokker--Planck-like forms \cite{Fokker_1914, Planck_1917} wherein pseudo-particles in the numerical model obey the neoclassical transport equations, with particle-independent Brownian drift terms. This offers a rigorous methodology for incorporating collisions into the particle transport model, without coupling the equations of motions for each particle.
        
        Works by Chen, Chacón et al. \cite{Chen_Chacón_Barnes_2011, Chacón_Chen_Barnes_2013, Chen_Chacón_2014, Chen_Chacón_2015} have developed structure-preserving particle pushers for neoclassical transport in the Vlasov equations, derived from Crank--Nicolson integrators. We show these too can can derive from a FET interpretation, similarly offering potential extensions to higher-order-in-time particle pushers. The FET formulation is used also to consider how the stochastic drift terms can be incorporated into the pushers. Stochastic gyrokinetic expansions are also discussed.

        Different options for the numerical implementation of these schemes are considered.

        Due to the efficacy of FET in the development of SP timesteppers for both the fluid and kinetic component, we hope this approach will prove effective in the future for developing SP timesteppers for the full hybrid model. We hope this will give us the opportunity to incorporate previously inaccessible kinetic effects into the highly effective, modern, finite-element MHD models.
    \end{abstract}
    
    
    \newpage
    \tableofcontents
    
    
    \newpage
    \pagenumbering{arabic}
    %\linenumbers\renewcommand\thelinenumber{\color{black!50}\arabic{linenumber}}
            \input{0 - introduction/main.tex}
        \part{Research}
            \input{1 - low-noise PiC models/main.tex}
            \input{2 - kinetic component/main.tex}
            \input{3 - fluid component/main.tex}
            \input{4 - numerical implementation/main.tex}
        \part{Project Overview}
            \input{5 - research plan/main.tex}
            \input{6 - summary/main.tex}
    
    
    %\section{}
    \newpage
    \pagenumbering{gobble}
        \printbibliography


    \newpage
    \pagenumbering{roman}
    \appendix
        \part{Appendices}
            \input{8 - Hilbert complexes/main.tex}
            \input{9 - weak conservation proofs/main.tex}
\end{document}

            \documentclass[12pt, a4paper]{report}

\input{template/main.tex}

\title{\BA{Title in Progress...}}
\author{Boris Andrews}
\affil{Mathematical Institute, University of Oxford}
\date{\today}


\begin{document}
    \pagenumbering{gobble}
    \maketitle
    
    
    \begin{abstract}
        Magnetic confinement reactors---in particular tokamaks---offer one of the most promising options for achieving practical nuclear fusion, with the potential to provide virtually limitless, clean energy. The theoretical and numerical modeling of tokamak plasmas is simultaneously an essential component of effective reactor design, and a great research barrier. Tokamak operational conditions exhibit comparatively low Knudsen numbers. Kinetic effects, including kinetic waves and instabilities, Landau damping, bump-on-tail instabilities and more, are therefore highly influential in tokamak plasma dynamics. Purely fluid models are inherently incapable of capturing these effects, whereas the high dimensionality in purely kinetic models render them practically intractable for most relevant purposes.

        We consider a $\delta\!f$ decomposition model, with a macroscopic fluid background and microscopic kinetic correction, both fully coupled to each other. A similar manner of discretization is proposed to that used in the recent \texttt{STRUPHY} code \cite{Holderied_Possanner_Wang_2021, Holderied_2022, Li_et_al_2023} with a finite-element model for the background and a pseudo-particle/PiC model for the correction.

        The fluid background satisfies the full, non-linear, resistive, compressible, Hall MHD equations. \cite{Laakmann_Hu_Farrell_2022} introduces finite-element(-in-space) implicit timesteppers for the incompressible analogue to this system with structure-preserving (SP) properties in the ideal case, alongside parameter-robust preconditioners. We show that these timesteppers can derive from a finite-element-in-time (FET) (and finite-element-in-space) interpretation. The benefits of this reformulation are discussed, including the derivation of timesteppers that are higher order in time, and the quantifiable dissipative SP properties in the non-ideal, resistive case.
        
        We discuss possible options for extending this FET approach to timesteppers for the compressible case.

        The kinetic corrections satisfy linearized Boltzmann equations. Using a Lénard--Bernstein collision operator, these take Fokker--Planck-like forms \cite{Fokker_1914, Planck_1917} wherein pseudo-particles in the numerical model obey the neoclassical transport equations, with particle-independent Brownian drift terms. This offers a rigorous methodology for incorporating collisions into the particle transport model, without coupling the equations of motions for each particle.
        
        Works by Chen, Chacón et al. \cite{Chen_Chacón_Barnes_2011, Chacón_Chen_Barnes_2013, Chen_Chacón_2014, Chen_Chacón_2015} have developed structure-preserving particle pushers for neoclassical transport in the Vlasov equations, derived from Crank--Nicolson integrators. We show these too can can derive from a FET interpretation, similarly offering potential extensions to higher-order-in-time particle pushers. The FET formulation is used also to consider how the stochastic drift terms can be incorporated into the pushers. Stochastic gyrokinetic expansions are also discussed.

        Different options for the numerical implementation of these schemes are considered.

        Due to the efficacy of FET in the development of SP timesteppers for both the fluid and kinetic component, we hope this approach will prove effective in the future for developing SP timesteppers for the full hybrid model. We hope this will give us the opportunity to incorporate previously inaccessible kinetic effects into the highly effective, modern, finite-element MHD models.
    \end{abstract}
    
    
    \newpage
    \tableofcontents
    
    
    \newpage
    \pagenumbering{arabic}
    %\linenumbers\renewcommand\thelinenumber{\color{black!50}\arabic{linenumber}}
            \input{0 - introduction/main.tex}
        \part{Research}
            \input{1 - low-noise PiC models/main.tex}
            \input{2 - kinetic component/main.tex}
            \input{3 - fluid component/main.tex}
            \input{4 - numerical implementation/main.tex}
        \part{Project Overview}
            \input{5 - research plan/main.tex}
            \input{6 - summary/main.tex}
    
    
    %\section{}
    \newpage
    \pagenumbering{gobble}
        \printbibliography


    \newpage
    \pagenumbering{roman}
    \appendix
        \part{Appendices}
            \input{8 - Hilbert complexes/main.tex}
            \input{9 - weak conservation proofs/main.tex}
\end{document}

            \documentclass[12pt, a4paper]{report}

\input{template/main.tex}

\title{\BA{Title in Progress...}}
\author{Boris Andrews}
\affil{Mathematical Institute, University of Oxford}
\date{\today}


\begin{document}
    \pagenumbering{gobble}
    \maketitle
    
    
    \begin{abstract}
        Magnetic confinement reactors---in particular tokamaks---offer one of the most promising options for achieving practical nuclear fusion, with the potential to provide virtually limitless, clean energy. The theoretical and numerical modeling of tokamak plasmas is simultaneously an essential component of effective reactor design, and a great research barrier. Tokamak operational conditions exhibit comparatively low Knudsen numbers. Kinetic effects, including kinetic waves and instabilities, Landau damping, bump-on-tail instabilities and more, are therefore highly influential in tokamak plasma dynamics. Purely fluid models are inherently incapable of capturing these effects, whereas the high dimensionality in purely kinetic models render them practically intractable for most relevant purposes.

        We consider a $\delta\!f$ decomposition model, with a macroscopic fluid background and microscopic kinetic correction, both fully coupled to each other. A similar manner of discretization is proposed to that used in the recent \texttt{STRUPHY} code \cite{Holderied_Possanner_Wang_2021, Holderied_2022, Li_et_al_2023} with a finite-element model for the background and a pseudo-particle/PiC model for the correction.

        The fluid background satisfies the full, non-linear, resistive, compressible, Hall MHD equations. \cite{Laakmann_Hu_Farrell_2022} introduces finite-element(-in-space) implicit timesteppers for the incompressible analogue to this system with structure-preserving (SP) properties in the ideal case, alongside parameter-robust preconditioners. We show that these timesteppers can derive from a finite-element-in-time (FET) (and finite-element-in-space) interpretation. The benefits of this reformulation are discussed, including the derivation of timesteppers that are higher order in time, and the quantifiable dissipative SP properties in the non-ideal, resistive case.
        
        We discuss possible options for extending this FET approach to timesteppers for the compressible case.

        The kinetic corrections satisfy linearized Boltzmann equations. Using a Lénard--Bernstein collision operator, these take Fokker--Planck-like forms \cite{Fokker_1914, Planck_1917} wherein pseudo-particles in the numerical model obey the neoclassical transport equations, with particle-independent Brownian drift terms. This offers a rigorous methodology for incorporating collisions into the particle transport model, without coupling the equations of motions for each particle.
        
        Works by Chen, Chacón et al. \cite{Chen_Chacón_Barnes_2011, Chacón_Chen_Barnes_2013, Chen_Chacón_2014, Chen_Chacón_2015} have developed structure-preserving particle pushers for neoclassical transport in the Vlasov equations, derived from Crank--Nicolson integrators. We show these too can can derive from a FET interpretation, similarly offering potential extensions to higher-order-in-time particle pushers. The FET formulation is used also to consider how the stochastic drift terms can be incorporated into the pushers. Stochastic gyrokinetic expansions are also discussed.

        Different options for the numerical implementation of these schemes are considered.

        Due to the efficacy of FET in the development of SP timesteppers for both the fluid and kinetic component, we hope this approach will prove effective in the future for developing SP timesteppers for the full hybrid model. We hope this will give us the opportunity to incorporate previously inaccessible kinetic effects into the highly effective, modern, finite-element MHD models.
    \end{abstract}
    
    
    \newpage
    \tableofcontents
    
    
    \newpage
    \pagenumbering{arabic}
    %\linenumbers\renewcommand\thelinenumber{\color{black!50}\arabic{linenumber}}
            \input{0 - introduction/main.tex}
        \part{Research}
            \input{1 - low-noise PiC models/main.tex}
            \input{2 - kinetic component/main.tex}
            \input{3 - fluid component/main.tex}
            \input{4 - numerical implementation/main.tex}
        \part{Project Overview}
            \input{5 - research plan/main.tex}
            \input{6 - summary/main.tex}
    
    
    %\section{}
    \newpage
    \pagenumbering{gobble}
        \printbibliography


    \newpage
    \pagenumbering{roman}
    \appendix
        \part{Appendices}
            \input{8 - Hilbert complexes/main.tex}
            \input{9 - weak conservation proofs/main.tex}
\end{document}

        \part{Project Overview}
            \documentclass[12pt, a4paper]{report}

\input{template/main.tex}

\title{\BA{Title in Progress...}}
\author{Boris Andrews}
\affil{Mathematical Institute, University of Oxford}
\date{\today}


\begin{document}
    \pagenumbering{gobble}
    \maketitle
    
    
    \begin{abstract}
        Magnetic confinement reactors---in particular tokamaks---offer one of the most promising options for achieving practical nuclear fusion, with the potential to provide virtually limitless, clean energy. The theoretical and numerical modeling of tokamak plasmas is simultaneously an essential component of effective reactor design, and a great research barrier. Tokamak operational conditions exhibit comparatively low Knudsen numbers. Kinetic effects, including kinetic waves and instabilities, Landau damping, bump-on-tail instabilities and more, are therefore highly influential in tokamak plasma dynamics. Purely fluid models are inherently incapable of capturing these effects, whereas the high dimensionality in purely kinetic models render them practically intractable for most relevant purposes.

        We consider a $\delta\!f$ decomposition model, with a macroscopic fluid background and microscopic kinetic correction, both fully coupled to each other. A similar manner of discretization is proposed to that used in the recent \texttt{STRUPHY} code \cite{Holderied_Possanner_Wang_2021, Holderied_2022, Li_et_al_2023} with a finite-element model for the background and a pseudo-particle/PiC model for the correction.

        The fluid background satisfies the full, non-linear, resistive, compressible, Hall MHD equations. \cite{Laakmann_Hu_Farrell_2022} introduces finite-element(-in-space) implicit timesteppers for the incompressible analogue to this system with structure-preserving (SP) properties in the ideal case, alongside parameter-robust preconditioners. We show that these timesteppers can derive from a finite-element-in-time (FET) (and finite-element-in-space) interpretation. The benefits of this reformulation are discussed, including the derivation of timesteppers that are higher order in time, and the quantifiable dissipative SP properties in the non-ideal, resistive case.
        
        We discuss possible options for extending this FET approach to timesteppers for the compressible case.

        The kinetic corrections satisfy linearized Boltzmann equations. Using a Lénard--Bernstein collision operator, these take Fokker--Planck-like forms \cite{Fokker_1914, Planck_1917} wherein pseudo-particles in the numerical model obey the neoclassical transport equations, with particle-independent Brownian drift terms. This offers a rigorous methodology for incorporating collisions into the particle transport model, without coupling the equations of motions for each particle.
        
        Works by Chen, Chacón et al. \cite{Chen_Chacón_Barnes_2011, Chacón_Chen_Barnes_2013, Chen_Chacón_2014, Chen_Chacón_2015} have developed structure-preserving particle pushers for neoclassical transport in the Vlasov equations, derived from Crank--Nicolson integrators. We show these too can can derive from a FET interpretation, similarly offering potential extensions to higher-order-in-time particle pushers. The FET formulation is used also to consider how the stochastic drift terms can be incorporated into the pushers. Stochastic gyrokinetic expansions are also discussed.

        Different options for the numerical implementation of these schemes are considered.

        Due to the efficacy of FET in the development of SP timesteppers for both the fluid and kinetic component, we hope this approach will prove effective in the future for developing SP timesteppers for the full hybrid model. We hope this will give us the opportunity to incorporate previously inaccessible kinetic effects into the highly effective, modern, finite-element MHD models.
    \end{abstract}
    
    
    \newpage
    \tableofcontents
    
    
    \newpage
    \pagenumbering{arabic}
    %\linenumbers\renewcommand\thelinenumber{\color{black!50}\arabic{linenumber}}
            \input{0 - introduction/main.tex}
        \part{Research}
            \input{1 - low-noise PiC models/main.tex}
            \input{2 - kinetic component/main.tex}
            \input{3 - fluid component/main.tex}
            \input{4 - numerical implementation/main.tex}
        \part{Project Overview}
            \input{5 - research plan/main.tex}
            \input{6 - summary/main.tex}
    
    
    %\section{}
    \newpage
    \pagenumbering{gobble}
        \printbibliography


    \newpage
    \pagenumbering{roman}
    \appendix
        \part{Appendices}
            \input{8 - Hilbert complexes/main.tex}
            \input{9 - weak conservation proofs/main.tex}
\end{document}

            \documentclass[12pt, a4paper]{report}

\input{template/main.tex}

\title{\BA{Title in Progress...}}
\author{Boris Andrews}
\affil{Mathematical Institute, University of Oxford}
\date{\today}


\begin{document}
    \pagenumbering{gobble}
    \maketitle
    
    
    \begin{abstract}
        Magnetic confinement reactors---in particular tokamaks---offer one of the most promising options for achieving practical nuclear fusion, with the potential to provide virtually limitless, clean energy. The theoretical and numerical modeling of tokamak plasmas is simultaneously an essential component of effective reactor design, and a great research barrier. Tokamak operational conditions exhibit comparatively low Knudsen numbers. Kinetic effects, including kinetic waves and instabilities, Landau damping, bump-on-tail instabilities and more, are therefore highly influential in tokamak plasma dynamics. Purely fluid models are inherently incapable of capturing these effects, whereas the high dimensionality in purely kinetic models render them practically intractable for most relevant purposes.

        We consider a $\delta\!f$ decomposition model, with a macroscopic fluid background and microscopic kinetic correction, both fully coupled to each other. A similar manner of discretization is proposed to that used in the recent \texttt{STRUPHY} code \cite{Holderied_Possanner_Wang_2021, Holderied_2022, Li_et_al_2023} with a finite-element model for the background and a pseudo-particle/PiC model for the correction.

        The fluid background satisfies the full, non-linear, resistive, compressible, Hall MHD equations. \cite{Laakmann_Hu_Farrell_2022} introduces finite-element(-in-space) implicit timesteppers for the incompressible analogue to this system with structure-preserving (SP) properties in the ideal case, alongside parameter-robust preconditioners. We show that these timesteppers can derive from a finite-element-in-time (FET) (and finite-element-in-space) interpretation. The benefits of this reformulation are discussed, including the derivation of timesteppers that are higher order in time, and the quantifiable dissipative SP properties in the non-ideal, resistive case.
        
        We discuss possible options for extending this FET approach to timesteppers for the compressible case.

        The kinetic corrections satisfy linearized Boltzmann equations. Using a Lénard--Bernstein collision operator, these take Fokker--Planck-like forms \cite{Fokker_1914, Planck_1917} wherein pseudo-particles in the numerical model obey the neoclassical transport equations, with particle-independent Brownian drift terms. This offers a rigorous methodology for incorporating collisions into the particle transport model, without coupling the equations of motions for each particle.
        
        Works by Chen, Chacón et al. \cite{Chen_Chacón_Barnes_2011, Chacón_Chen_Barnes_2013, Chen_Chacón_2014, Chen_Chacón_2015} have developed structure-preserving particle pushers for neoclassical transport in the Vlasov equations, derived from Crank--Nicolson integrators. We show these too can can derive from a FET interpretation, similarly offering potential extensions to higher-order-in-time particle pushers. The FET formulation is used also to consider how the stochastic drift terms can be incorporated into the pushers. Stochastic gyrokinetic expansions are also discussed.

        Different options for the numerical implementation of these schemes are considered.

        Due to the efficacy of FET in the development of SP timesteppers for both the fluid and kinetic component, we hope this approach will prove effective in the future for developing SP timesteppers for the full hybrid model. We hope this will give us the opportunity to incorporate previously inaccessible kinetic effects into the highly effective, modern, finite-element MHD models.
    \end{abstract}
    
    
    \newpage
    \tableofcontents
    
    
    \newpage
    \pagenumbering{arabic}
    %\linenumbers\renewcommand\thelinenumber{\color{black!50}\arabic{linenumber}}
            \input{0 - introduction/main.tex}
        \part{Research}
            \input{1 - low-noise PiC models/main.tex}
            \input{2 - kinetic component/main.tex}
            \input{3 - fluid component/main.tex}
            \input{4 - numerical implementation/main.tex}
        \part{Project Overview}
            \input{5 - research plan/main.tex}
            \input{6 - summary/main.tex}
    
    
    %\section{}
    \newpage
    \pagenumbering{gobble}
        \printbibliography


    \newpage
    \pagenumbering{roman}
    \appendix
        \part{Appendices}
            \input{8 - Hilbert complexes/main.tex}
            \input{9 - weak conservation proofs/main.tex}
\end{document}

    
    
    %\section{}
    \newpage
    \pagenumbering{gobble}
        \printbibliography


    \newpage
    \pagenumbering{roman}
    \appendix
        \part{Appendices}
            \documentclass[12pt, a4paper]{report}

\input{template/main.tex}

\title{\BA{Title in Progress...}}
\author{Boris Andrews}
\affil{Mathematical Institute, University of Oxford}
\date{\today}


\begin{document}
    \pagenumbering{gobble}
    \maketitle
    
    
    \begin{abstract}
        Magnetic confinement reactors---in particular tokamaks---offer one of the most promising options for achieving practical nuclear fusion, with the potential to provide virtually limitless, clean energy. The theoretical and numerical modeling of tokamak plasmas is simultaneously an essential component of effective reactor design, and a great research barrier. Tokamak operational conditions exhibit comparatively low Knudsen numbers. Kinetic effects, including kinetic waves and instabilities, Landau damping, bump-on-tail instabilities and more, are therefore highly influential in tokamak plasma dynamics. Purely fluid models are inherently incapable of capturing these effects, whereas the high dimensionality in purely kinetic models render them practically intractable for most relevant purposes.

        We consider a $\delta\!f$ decomposition model, with a macroscopic fluid background and microscopic kinetic correction, both fully coupled to each other. A similar manner of discretization is proposed to that used in the recent \texttt{STRUPHY} code \cite{Holderied_Possanner_Wang_2021, Holderied_2022, Li_et_al_2023} with a finite-element model for the background and a pseudo-particle/PiC model for the correction.

        The fluid background satisfies the full, non-linear, resistive, compressible, Hall MHD equations. \cite{Laakmann_Hu_Farrell_2022} introduces finite-element(-in-space) implicit timesteppers for the incompressible analogue to this system with structure-preserving (SP) properties in the ideal case, alongside parameter-robust preconditioners. We show that these timesteppers can derive from a finite-element-in-time (FET) (and finite-element-in-space) interpretation. The benefits of this reformulation are discussed, including the derivation of timesteppers that are higher order in time, and the quantifiable dissipative SP properties in the non-ideal, resistive case.
        
        We discuss possible options for extending this FET approach to timesteppers for the compressible case.

        The kinetic corrections satisfy linearized Boltzmann equations. Using a Lénard--Bernstein collision operator, these take Fokker--Planck-like forms \cite{Fokker_1914, Planck_1917} wherein pseudo-particles in the numerical model obey the neoclassical transport equations, with particle-independent Brownian drift terms. This offers a rigorous methodology for incorporating collisions into the particle transport model, without coupling the equations of motions for each particle.
        
        Works by Chen, Chacón et al. \cite{Chen_Chacón_Barnes_2011, Chacón_Chen_Barnes_2013, Chen_Chacón_2014, Chen_Chacón_2015} have developed structure-preserving particle pushers for neoclassical transport in the Vlasov equations, derived from Crank--Nicolson integrators. We show these too can can derive from a FET interpretation, similarly offering potential extensions to higher-order-in-time particle pushers. The FET formulation is used also to consider how the stochastic drift terms can be incorporated into the pushers. Stochastic gyrokinetic expansions are also discussed.

        Different options for the numerical implementation of these schemes are considered.

        Due to the efficacy of FET in the development of SP timesteppers for both the fluid and kinetic component, we hope this approach will prove effective in the future for developing SP timesteppers for the full hybrid model. We hope this will give us the opportunity to incorporate previously inaccessible kinetic effects into the highly effective, modern, finite-element MHD models.
    \end{abstract}
    
    
    \newpage
    \tableofcontents
    
    
    \newpage
    \pagenumbering{arabic}
    %\linenumbers\renewcommand\thelinenumber{\color{black!50}\arabic{linenumber}}
            \input{0 - introduction/main.tex}
        \part{Research}
            \input{1 - low-noise PiC models/main.tex}
            \input{2 - kinetic component/main.tex}
            \input{3 - fluid component/main.tex}
            \input{4 - numerical implementation/main.tex}
        \part{Project Overview}
            \input{5 - research plan/main.tex}
            \input{6 - summary/main.tex}
    
    
    %\section{}
    \newpage
    \pagenumbering{gobble}
        \printbibliography


    \newpage
    \pagenumbering{roman}
    \appendix
        \part{Appendices}
            \input{8 - Hilbert complexes/main.tex}
            \input{9 - weak conservation proofs/main.tex}
\end{document}

            \documentclass[12pt, a4paper]{report}

\input{template/main.tex}

\title{\BA{Title in Progress...}}
\author{Boris Andrews}
\affil{Mathematical Institute, University of Oxford}
\date{\today}


\begin{document}
    \pagenumbering{gobble}
    \maketitle
    
    
    \begin{abstract}
        Magnetic confinement reactors---in particular tokamaks---offer one of the most promising options for achieving practical nuclear fusion, with the potential to provide virtually limitless, clean energy. The theoretical and numerical modeling of tokamak plasmas is simultaneously an essential component of effective reactor design, and a great research barrier. Tokamak operational conditions exhibit comparatively low Knudsen numbers. Kinetic effects, including kinetic waves and instabilities, Landau damping, bump-on-tail instabilities and more, are therefore highly influential in tokamak plasma dynamics. Purely fluid models are inherently incapable of capturing these effects, whereas the high dimensionality in purely kinetic models render them practically intractable for most relevant purposes.

        We consider a $\delta\!f$ decomposition model, with a macroscopic fluid background and microscopic kinetic correction, both fully coupled to each other. A similar manner of discretization is proposed to that used in the recent \texttt{STRUPHY} code \cite{Holderied_Possanner_Wang_2021, Holderied_2022, Li_et_al_2023} with a finite-element model for the background and a pseudo-particle/PiC model for the correction.

        The fluid background satisfies the full, non-linear, resistive, compressible, Hall MHD equations. \cite{Laakmann_Hu_Farrell_2022} introduces finite-element(-in-space) implicit timesteppers for the incompressible analogue to this system with structure-preserving (SP) properties in the ideal case, alongside parameter-robust preconditioners. We show that these timesteppers can derive from a finite-element-in-time (FET) (and finite-element-in-space) interpretation. The benefits of this reformulation are discussed, including the derivation of timesteppers that are higher order in time, and the quantifiable dissipative SP properties in the non-ideal, resistive case.
        
        We discuss possible options for extending this FET approach to timesteppers for the compressible case.

        The kinetic corrections satisfy linearized Boltzmann equations. Using a Lénard--Bernstein collision operator, these take Fokker--Planck-like forms \cite{Fokker_1914, Planck_1917} wherein pseudo-particles in the numerical model obey the neoclassical transport equations, with particle-independent Brownian drift terms. This offers a rigorous methodology for incorporating collisions into the particle transport model, without coupling the equations of motions for each particle.
        
        Works by Chen, Chacón et al. \cite{Chen_Chacón_Barnes_2011, Chacón_Chen_Barnes_2013, Chen_Chacón_2014, Chen_Chacón_2015} have developed structure-preserving particle pushers for neoclassical transport in the Vlasov equations, derived from Crank--Nicolson integrators. We show these too can can derive from a FET interpretation, similarly offering potential extensions to higher-order-in-time particle pushers. The FET formulation is used also to consider how the stochastic drift terms can be incorporated into the pushers. Stochastic gyrokinetic expansions are also discussed.

        Different options for the numerical implementation of these schemes are considered.

        Due to the efficacy of FET in the development of SP timesteppers for both the fluid and kinetic component, we hope this approach will prove effective in the future for developing SP timesteppers for the full hybrid model. We hope this will give us the opportunity to incorporate previously inaccessible kinetic effects into the highly effective, modern, finite-element MHD models.
    \end{abstract}
    
    
    \newpage
    \tableofcontents
    
    
    \newpage
    \pagenumbering{arabic}
    %\linenumbers\renewcommand\thelinenumber{\color{black!50}\arabic{linenumber}}
            \input{0 - introduction/main.tex}
        \part{Research}
            \input{1 - low-noise PiC models/main.tex}
            \input{2 - kinetic component/main.tex}
            \input{3 - fluid component/main.tex}
            \input{4 - numerical implementation/main.tex}
        \part{Project Overview}
            \input{5 - research plan/main.tex}
            \input{6 - summary/main.tex}
    
    
    %\section{}
    \newpage
    \pagenumbering{gobble}
        \printbibliography


    \newpage
    \pagenumbering{roman}
    \appendix
        \part{Appendices}
            \input{8 - Hilbert complexes/main.tex}
            \input{9 - weak conservation proofs/main.tex}
\end{document}

\end{document}

            \documentclass[12pt, a4paper]{report}

\documentclass[12pt, a4paper]{report}

\input{template/main.tex}

\title{\BA{Title in Progress...}}
\author{Boris Andrews}
\affil{Mathematical Institute, University of Oxford}
\date{\today}


\begin{document}
    \pagenumbering{gobble}
    \maketitle
    
    
    \begin{abstract}
        Magnetic confinement reactors---in particular tokamaks---offer one of the most promising options for achieving practical nuclear fusion, with the potential to provide virtually limitless, clean energy. The theoretical and numerical modeling of tokamak plasmas is simultaneously an essential component of effective reactor design, and a great research barrier. Tokamak operational conditions exhibit comparatively low Knudsen numbers. Kinetic effects, including kinetic waves and instabilities, Landau damping, bump-on-tail instabilities and more, are therefore highly influential in tokamak plasma dynamics. Purely fluid models are inherently incapable of capturing these effects, whereas the high dimensionality in purely kinetic models render them practically intractable for most relevant purposes.

        We consider a $\delta\!f$ decomposition model, with a macroscopic fluid background and microscopic kinetic correction, both fully coupled to each other. A similar manner of discretization is proposed to that used in the recent \texttt{STRUPHY} code \cite{Holderied_Possanner_Wang_2021, Holderied_2022, Li_et_al_2023} with a finite-element model for the background and a pseudo-particle/PiC model for the correction.

        The fluid background satisfies the full, non-linear, resistive, compressible, Hall MHD equations. \cite{Laakmann_Hu_Farrell_2022} introduces finite-element(-in-space) implicit timesteppers for the incompressible analogue to this system with structure-preserving (SP) properties in the ideal case, alongside parameter-robust preconditioners. We show that these timesteppers can derive from a finite-element-in-time (FET) (and finite-element-in-space) interpretation. The benefits of this reformulation are discussed, including the derivation of timesteppers that are higher order in time, and the quantifiable dissipative SP properties in the non-ideal, resistive case.
        
        We discuss possible options for extending this FET approach to timesteppers for the compressible case.

        The kinetic corrections satisfy linearized Boltzmann equations. Using a Lénard--Bernstein collision operator, these take Fokker--Planck-like forms \cite{Fokker_1914, Planck_1917} wherein pseudo-particles in the numerical model obey the neoclassical transport equations, with particle-independent Brownian drift terms. This offers a rigorous methodology for incorporating collisions into the particle transport model, without coupling the equations of motions for each particle.
        
        Works by Chen, Chacón et al. \cite{Chen_Chacón_Barnes_2011, Chacón_Chen_Barnes_2013, Chen_Chacón_2014, Chen_Chacón_2015} have developed structure-preserving particle pushers for neoclassical transport in the Vlasov equations, derived from Crank--Nicolson integrators. We show these too can can derive from a FET interpretation, similarly offering potential extensions to higher-order-in-time particle pushers. The FET formulation is used also to consider how the stochastic drift terms can be incorporated into the pushers. Stochastic gyrokinetic expansions are also discussed.

        Different options for the numerical implementation of these schemes are considered.

        Due to the efficacy of FET in the development of SP timesteppers for both the fluid and kinetic component, we hope this approach will prove effective in the future for developing SP timesteppers for the full hybrid model. We hope this will give us the opportunity to incorporate previously inaccessible kinetic effects into the highly effective, modern, finite-element MHD models.
    \end{abstract}
    
    
    \newpage
    \tableofcontents
    
    
    \newpage
    \pagenumbering{arabic}
    %\linenumbers\renewcommand\thelinenumber{\color{black!50}\arabic{linenumber}}
            \input{0 - introduction/main.tex}
        \part{Research}
            \input{1 - low-noise PiC models/main.tex}
            \input{2 - kinetic component/main.tex}
            \input{3 - fluid component/main.tex}
            \input{4 - numerical implementation/main.tex}
        \part{Project Overview}
            \input{5 - research plan/main.tex}
            \input{6 - summary/main.tex}
    
    
    %\section{}
    \newpage
    \pagenumbering{gobble}
        \printbibliography


    \newpage
    \pagenumbering{roman}
    \appendix
        \part{Appendices}
            \input{8 - Hilbert complexes/main.tex}
            \input{9 - weak conservation proofs/main.tex}
\end{document}


\title{\BA{Title in Progress...}}
\author{Boris Andrews}
\affil{Mathematical Institute, University of Oxford}
\date{\today}


\begin{document}
    \pagenumbering{gobble}
    \maketitle
    
    
    \begin{abstract}
        Magnetic confinement reactors---in particular tokamaks---offer one of the most promising options for achieving practical nuclear fusion, with the potential to provide virtually limitless, clean energy. The theoretical and numerical modeling of tokamak plasmas is simultaneously an essential component of effective reactor design, and a great research barrier. Tokamak operational conditions exhibit comparatively low Knudsen numbers. Kinetic effects, including kinetic waves and instabilities, Landau damping, bump-on-tail instabilities and more, are therefore highly influential in tokamak plasma dynamics. Purely fluid models are inherently incapable of capturing these effects, whereas the high dimensionality in purely kinetic models render them practically intractable for most relevant purposes.

        We consider a $\delta\!f$ decomposition model, with a macroscopic fluid background and microscopic kinetic correction, both fully coupled to each other. A similar manner of discretization is proposed to that used in the recent \texttt{STRUPHY} code \cite{Holderied_Possanner_Wang_2021, Holderied_2022, Li_et_al_2023} with a finite-element model for the background and a pseudo-particle/PiC model for the correction.

        The fluid background satisfies the full, non-linear, resistive, compressible, Hall MHD equations. \cite{Laakmann_Hu_Farrell_2022} introduces finite-element(-in-space) implicit timesteppers for the incompressible analogue to this system with structure-preserving (SP) properties in the ideal case, alongside parameter-robust preconditioners. We show that these timesteppers can derive from a finite-element-in-time (FET) (and finite-element-in-space) interpretation. The benefits of this reformulation are discussed, including the derivation of timesteppers that are higher order in time, and the quantifiable dissipative SP properties in the non-ideal, resistive case.
        
        We discuss possible options for extending this FET approach to timesteppers for the compressible case.

        The kinetic corrections satisfy linearized Boltzmann equations. Using a Lénard--Bernstein collision operator, these take Fokker--Planck-like forms \cite{Fokker_1914, Planck_1917} wherein pseudo-particles in the numerical model obey the neoclassical transport equations, with particle-independent Brownian drift terms. This offers a rigorous methodology for incorporating collisions into the particle transport model, without coupling the equations of motions for each particle.
        
        Works by Chen, Chacón et al. \cite{Chen_Chacón_Barnes_2011, Chacón_Chen_Barnes_2013, Chen_Chacón_2014, Chen_Chacón_2015} have developed structure-preserving particle pushers for neoclassical transport in the Vlasov equations, derived from Crank--Nicolson integrators. We show these too can can derive from a FET interpretation, similarly offering potential extensions to higher-order-in-time particle pushers. The FET formulation is used also to consider how the stochastic drift terms can be incorporated into the pushers. Stochastic gyrokinetic expansions are also discussed.

        Different options for the numerical implementation of these schemes are considered.

        Due to the efficacy of FET in the development of SP timesteppers for both the fluid and kinetic component, we hope this approach will prove effective in the future for developing SP timesteppers for the full hybrid model. We hope this will give us the opportunity to incorporate previously inaccessible kinetic effects into the highly effective, modern, finite-element MHD models.
    \end{abstract}
    
    
    \newpage
    \tableofcontents
    
    
    \newpage
    \pagenumbering{arabic}
    %\linenumbers\renewcommand\thelinenumber{\color{black!50}\arabic{linenumber}}
            \documentclass[12pt, a4paper]{report}

\input{template/main.tex}

\title{\BA{Title in Progress...}}
\author{Boris Andrews}
\affil{Mathematical Institute, University of Oxford}
\date{\today}


\begin{document}
    \pagenumbering{gobble}
    \maketitle
    
    
    \begin{abstract}
        Magnetic confinement reactors---in particular tokamaks---offer one of the most promising options for achieving practical nuclear fusion, with the potential to provide virtually limitless, clean energy. The theoretical and numerical modeling of tokamak plasmas is simultaneously an essential component of effective reactor design, and a great research barrier. Tokamak operational conditions exhibit comparatively low Knudsen numbers. Kinetic effects, including kinetic waves and instabilities, Landau damping, bump-on-tail instabilities and more, are therefore highly influential in tokamak plasma dynamics. Purely fluid models are inherently incapable of capturing these effects, whereas the high dimensionality in purely kinetic models render them practically intractable for most relevant purposes.

        We consider a $\delta\!f$ decomposition model, with a macroscopic fluid background and microscopic kinetic correction, both fully coupled to each other. A similar manner of discretization is proposed to that used in the recent \texttt{STRUPHY} code \cite{Holderied_Possanner_Wang_2021, Holderied_2022, Li_et_al_2023} with a finite-element model for the background and a pseudo-particle/PiC model for the correction.

        The fluid background satisfies the full, non-linear, resistive, compressible, Hall MHD equations. \cite{Laakmann_Hu_Farrell_2022} introduces finite-element(-in-space) implicit timesteppers for the incompressible analogue to this system with structure-preserving (SP) properties in the ideal case, alongside parameter-robust preconditioners. We show that these timesteppers can derive from a finite-element-in-time (FET) (and finite-element-in-space) interpretation. The benefits of this reformulation are discussed, including the derivation of timesteppers that are higher order in time, and the quantifiable dissipative SP properties in the non-ideal, resistive case.
        
        We discuss possible options for extending this FET approach to timesteppers for the compressible case.

        The kinetic corrections satisfy linearized Boltzmann equations. Using a Lénard--Bernstein collision operator, these take Fokker--Planck-like forms \cite{Fokker_1914, Planck_1917} wherein pseudo-particles in the numerical model obey the neoclassical transport equations, with particle-independent Brownian drift terms. This offers a rigorous methodology for incorporating collisions into the particle transport model, without coupling the equations of motions for each particle.
        
        Works by Chen, Chacón et al. \cite{Chen_Chacón_Barnes_2011, Chacón_Chen_Barnes_2013, Chen_Chacón_2014, Chen_Chacón_2015} have developed structure-preserving particle pushers for neoclassical transport in the Vlasov equations, derived from Crank--Nicolson integrators. We show these too can can derive from a FET interpretation, similarly offering potential extensions to higher-order-in-time particle pushers. The FET formulation is used also to consider how the stochastic drift terms can be incorporated into the pushers. Stochastic gyrokinetic expansions are also discussed.

        Different options for the numerical implementation of these schemes are considered.

        Due to the efficacy of FET in the development of SP timesteppers for both the fluid and kinetic component, we hope this approach will prove effective in the future for developing SP timesteppers for the full hybrid model. We hope this will give us the opportunity to incorporate previously inaccessible kinetic effects into the highly effective, modern, finite-element MHD models.
    \end{abstract}
    
    
    \newpage
    \tableofcontents
    
    
    \newpage
    \pagenumbering{arabic}
    %\linenumbers\renewcommand\thelinenumber{\color{black!50}\arabic{linenumber}}
            \input{0 - introduction/main.tex}
        \part{Research}
            \input{1 - low-noise PiC models/main.tex}
            \input{2 - kinetic component/main.tex}
            \input{3 - fluid component/main.tex}
            \input{4 - numerical implementation/main.tex}
        \part{Project Overview}
            \input{5 - research plan/main.tex}
            \input{6 - summary/main.tex}
    
    
    %\section{}
    \newpage
    \pagenumbering{gobble}
        \printbibliography


    \newpage
    \pagenumbering{roman}
    \appendix
        \part{Appendices}
            \input{8 - Hilbert complexes/main.tex}
            \input{9 - weak conservation proofs/main.tex}
\end{document}

        \part{Research}
            \documentclass[12pt, a4paper]{report}

\input{template/main.tex}

\title{\BA{Title in Progress...}}
\author{Boris Andrews}
\affil{Mathematical Institute, University of Oxford}
\date{\today}


\begin{document}
    \pagenumbering{gobble}
    \maketitle
    
    
    \begin{abstract}
        Magnetic confinement reactors---in particular tokamaks---offer one of the most promising options for achieving practical nuclear fusion, with the potential to provide virtually limitless, clean energy. The theoretical and numerical modeling of tokamak plasmas is simultaneously an essential component of effective reactor design, and a great research barrier. Tokamak operational conditions exhibit comparatively low Knudsen numbers. Kinetic effects, including kinetic waves and instabilities, Landau damping, bump-on-tail instabilities and more, are therefore highly influential in tokamak plasma dynamics. Purely fluid models are inherently incapable of capturing these effects, whereas the high dimensionality in purely kinetic models render them practically intractable for most relevant purposes.

        We consider a $\delta\!f$ decomposition model, with a macroscopic fluid background and microscopic kinetic correction, both fully coupled to each other. A similar manner of discretization is proposed to that used in the recent \texttt{STRUPHY} code \cite{Holderied_Possanner_Wang_2021, Holderied_2022, Li_et_al_2023} with a finite-element model for the background and a pseudo-particle/PiC model for the correction.

        The fluid background satisfies the full, non-linear, resistive, compressible, Hall MHD equations. \cite{Laakmann_Hu_Farrell_2022} introduces finite-element(-in-space) implicit timesteppers for the incompressible analogue to this system with structure-preserving (SP) properties in the ideal case, alongside parameter-robust preconditioners. We show that these timesteppers can derive from a finite-element-in-time (FET) (and finite-element-in-space) interpretation. The benefits of this reformulation are discussed, including the derivation of timesteppers that are higher order in time, and the quantifiable dissipative SP properties in the non-ideal, resistive case.
        
        We discuss possible options for extending this FET approach to timesteppers for the compressible case.

        The kinetic corrections satisfy linearized Boltzmann equations. Using a Lénard--Bernstein collision operator, these take Fokker--Planck-like forms \cite{Fokker_1914, Planck_1917} wherein pseudo-particles in the numerical model obey the neoclassical transport equations, with particle-independent Brownian drift terms. This offers a rigorous methodology for incorporating collisions into the particle transport model, without coupling the equations of motions for each particle.
        
        Works by Chen, Chacón et al. \cite{Chen_Chacón_Barnes_2011, Chacón_Chen_Barnes_2013, Chen_Chacón_2014, Chen_Chacón_2015} have developed structure-preserving particle pushers for neoclassical transport in the Vlasov equations, derived from Crank--Nicolson integrators. We show these too can can derive from a FET interpretation, similarly offering potential extensions to higher-order-in-time particle pushers. The FET formulation is used also to consider how the stochastic drift terms can be incorporated into the pushers. Stochastic gyrokinetic expansions are also discussed.

        Different options for the numerical implementation of these schemes are considered.

        Due to the efficacy of FET in the development of SP timesteppers for both the fluid and kinetic component, we hope this approach will prove effective in the future for developing SP timesteppers for the full hybrid model. We hope this will give us the opportunity to incorporate previously inaccessible kinetic effects into the highly effective, modern, finite-element MHD models.
    \end{abstract}
    
    
    \newpage
    \tableofcontents
    
    
    \newpage
    \pagenumbering{arabic}
    %\linenumbers\renewcommand\thelinenumber{\color{black!50}\arabic{linenumber}}
            \input{0 - introduction/main.tex}
        \part{Research}
            \input{1 - low-noise PiC models/main.tex}
            \input{2 - kinetic component/main.tex}
            \input{3 - fluid component/main.tex}
            \input{4 - numerical implementation/main.tex}
        \part{Project Overview}
            \input{5 - research plan/main.tex}
            \input{6 - summary/main.tex}
    
    
    %\section{}
    \newpage
    \pagenumbering{gobble}
        \printbibliography


    \newpage
    \pagenumbering{roman}
    \appendix
        \part{Appendices}
            \input{8 - Hilbert complexes/main.tex}
            \input{9 - weak conservation proofs/main.tex}
\end{document}

            \documentclass[12pt, a4paper]{report}

\input{template/main.tex}

\title{\BA{Title in Progress...}}
\author{Boris Andrews}
\affil{Mathematical Institute, University of Oxford}
\date{\today}


\begin{document}
    \pagenumbering{gobble}
    \maketitle
    
    
    \begin{abstract}
        Magnetic confinement reactors---in particular tokamaks---offer one of the most promising options for achieving practical nuclear fusion, with the potential to provide virtually limitless, clean energy. The theoretical and numerical modeling of tokamak plasmas is simultaneously an essential component of effective reactor design, and a great research barrier. Tokamak operational conditions exhibit comparatively low Knudsen numbers. Kinetic effects, including kinetic waves and instabilities, Landau damping, bump-on-tail instabilities and more, are therefore highly influential in tokamak plasma dynamics. Purely fluid models are inherently incapable of capturing these effects, whereas the high dimensionality in purely kinetic models render them practically intractable for most relevant purposes.

        We consider a $\delta\!f$ decomposition model, with a macroscopic fluid background and microscopic kinetic correction, both fully coupled to each other. A similar manner of discretization is proposed to that used in the recent \texttt{STRUPHY} code \cite{Holderied_Possanner_Wang_2021, Holderied_2022, Li_et_al_2023} with a finite-element model for the background and a pseudo-particle/PiC model for the correction.

        The fluid background satisfies the full, non-linear, resistive, compressible, Hall MHD equations. \cite{Laakmann_Hu_Farrell_2022} introduces finite-element(-in-space) implicit timesteppers for the incompressible analogue to this system with structure-preserving (SP) properties in the ideal case, alongside parameter-robust preconditioners. We show that these timesteppers can derive from a finite-element-in-time (FET) (and finite-element-in-space) interpretation. The benefits of this reformulation are discussed, including the derivation of timesteppers that are higher order in time, and the quantifiable dissipative SP properties in the non-ideal, resistive case.
        
        We discuss possible options for extending this FET approach to timesteppers for the compressible case.

        The kinetic corrections satisfy linearized Boltzmann equations. Using a Lénard--Bernstein collision operator, these take Fokker--Planck-like forms \cite{Fokker_1914, Planck_1917} wherein pseudo-particles in the numerical model obey the neoclassical transport equations, with particle-independent Brownian drift terms. This offers a rigorous methodology for incorporating collisions into the particle transport model, without coupling the equations of motions for each particle.
        
        Works by Chen, Chacón et al. \cite{Chen_Chacón_Barnes_2011, Chacón_Chen_Barnes_2013, Chen_Chacón_2014, Chen_Chacón_2015} have developed structure-preserving particle pushers for neoclassical transport in the Vlasov equations, derived from Crank--Nicolson integrators. We show these too can can derive from a FET interpretation, similarly offering potential extensions to higher-order-in-time particle pushers. The FET formulation is used also to consider how the stochastic drift terms can be incorporated into the pushers. Stochastic gyrokinetic expansions are also discussed.

        Different options for the numerical implementation of these schemes are considered.

        Due to the efficacy of FET in the development of SP timesteppers for both the fluid and kinetic component, we hope this approach will prove effective in the future for developing SP timesteppers for the full hybrid model. We hope this will give us the opportunity to incorporate previously inaccessible kinetic effects into the highly effective, modern, finite-element MHD models.
    \end{abstract}
    
    
    \newpage
    \tableofcontents
    
    
    \newpage
    \pagenumbering{arabic}
    %\linenumbers\renewcommand\thelinenumber{\color{black!50}\arabic{linenumber}}
            \input{0 - introduction/main.tex}
        \part{Research}
            \input{1 - low-noise PiC models/main.tex}
            \input{2 - kinetic component/main.tex}
            \input{3 - fluid component/main.tex}
            \input{4 - numerical implementation/main.tex}
        \part{Project Overview}
            \input{5 - research plan/main.tex}
            \input{6 - summary/main.tex}
    
    
    %\section{}
    \newpage
    \pagenumbering{gobble}
        \printbibliography


    \newpage
    \pagenumbering{roman}
    \appendix
        \part{Appendices}
            \input{8 - Hilbert complexes/main.tex}
            \input{9 - weak conservation proofs/main.tex}
\end{document}

            \documentclass[12pt, a4paper]{report}

\input{template/main.tex}

\title{\BA{Title in Progress...}}
\author{Boris Andrews}
\affil{Mathematical Institute, University of Oxford}
\date{\today}


\begin{document}
    \pagenumbering{gobble}
    \maketitle
    
    
    \begin{abstract}
        Magnetic confinement reactors---in particular tokamaks---offer one of the most promising options for achieving practical nuclear fusion, with the potential to provide virtually limitless, clean energy. The theoretical and numerical modeling of tokamak plasmas is simultaneously an essential component of effective reactor design, and a great research barrier. Tokamak operational conditions exhibit comparatively low Knudsen numbers. Kinetic effects, including kinetic waves and instabilities, Landau damping, bump-on-tail instabilities and more, are therefore highly influential in tokamak plasma dynamics. Purely fluid models are inherently incapable of capturing these effects, whereas the high dimensionality in purely kinetic models render them practically intractable for most relevant purposes.

        We consider a $\delta\!f$ decomposition model, with a macroscopic fluid background and microscopic kinetic correction, both fully coupled to each other. A similar manner of discretization is proposed to that used in the recent \texttt{STRUPHY} code \cite{Holderied_Possanner_Wang_2021, Holderied_2022, Li_et_al_2023} with a finite-element model for the background and a pseudo-particle/PiC model for the correction.

        The fluid background satisfies the full, non-linear, resistive, compressible, Hall MHD equations. \cite{Laakmann_Hu_Farrell_2022} introduces finite-element(-in-space) implicit timesteppers for the incompressible analogue to this system with structure-preserving (SP) properties in the ideal case, alongside parameter-robust preconditioners. We show that these timesteppers can derive from a finite-element-in-time (FET) (and finite-element-in-space) interpretation. The benefits of this reformulation are discussed, including the derivation of timesteppers that are higher order in time, and the quantifiable dissipative SP properties in the non-ideal, resistive case.
        
        We discuss possible options for extending this FET approach to timesteppers for the compressible case.

        The kinetic corrections satisfy linearized Boltzmann equations. Using a Lénard--Bernstein collision operator, these take Fokker--Planck-like forms \cite{Fokker_1914, Planck_1917} wherein pseudo-particles in the numerical model obey the neoclassical transport equations, with particle-independent Brownian drift terms. This offers a rigorous methodology for incorporating collisions into the particle transport model, without coupling the equations of motions for each particle.
        
        Works by Chen, Chacón et al. \cite{Chen_Chacón_Barnes_2011, Chacón_Chen_Barnes_2013, Chen_Chacón_2014, Chen_Chacón_2015} have developed structure-preserving particle pushers for neoclassical transport in the Vlasov equations, derived from Crank--Nicolson integrators. We show these too can can derive from a FET interpretation, similarly offering potential extensions to higher-order-in-time particle pushers. The FET formulation is used also to consider how the stochastic drift terms can be incorporated into the pushers. Stochastic gyrokinetic expansions are also discussed.

        Different options for the numerical implementation of these schemes are considered.

        Due to the efficacy of FET in the development of SP timesteppers for both the fluid and kinetic component, we hope this approach will prove effective in the future for developing SP timesteppers for the full hybrid model. We hope this will give us the opportunity to incorporate previously inaccessible kinetic effects into the highly effective, modern, finite-element MHD models.
    \end{abstract}
    
    
    \newpage
    \tableofcontents
    
    
    \newpage
    \pagenumbering{arabic}
    %\linenumbers\renewcommand\thelinenumber{\color{black!50}\arabic{linenumber}}
            \input{0 - introduction/main.tex}
        \part{Research}
            \input{1 - low-noise PiC models/main.tex}
            \input{2 - kinetic component/main.tex}
            \input{3 - fluid component/main.tex}
            \input{4 - numerical implementation/main.tex}
        \part{Project Overview}
            \input{5 - research plan/main.tex}
            \input{6 - summary/main.tex}
    
    
    %\section{}
    \newpage
    \pagenumbering{gobble}
        \printbibliography


    \newpage
    \pagenumbering{roman}
    \appendix
        \part{Appendices}
            \input{8 - Hilbert complexes/main.tex}
            \input{9 - weak conservation proofs/main.tex}
\end{document}

            \documentclass[12pt, a4paper]{report}

\input{template/main.tex}

\title{\BA{Title in Progress...}}
\author{Boris Andrews}
\affil{Mathematical Institute, University of Oxford}
\date{\today}


\begin{document}
    \pagenumbering{gobble}
    \maketitle
    
    
    \begin{abstract}
        Magnetic confinement reactors---in particular tokamaks---offer one of the most promising options for achieving practical nuclear fusion, with the potential to provide virtually limitless, clean energy. The theoretical and numerical modeling of tokamak plasmas is simultaneously an essential component of effective reactor design, and a great research barrier. Tokamak operational conditions exhibit comparatively low Knudsen numbers. Kinetic effects, including kinetic waves and instabilities, Landau damping, bump-on-tail instabilities and more, are therefore highly influential in tokamak plasma dynamics. Purely fluid models are inherently incapable of capturing these effects, whereas the high dimensionality in purely kinetic models render them practically intractable for most relevant purposes.

        We consider a $\delta\!f$ decomposition model, with a macroscopic fluid background and microscopic kinetic correction, both fully coupled to each other. A similar manner of discretization is proposed to that used in the recent \texttt{STRUPHY} code \cite{Holderied_Possanner_Wang_2021, Holderied_2022, Li_et_al_2023} with a finite-element model for the background and a pseudo-particle/PiC model for the correction.

        The fluid background satisfies the full, non-linear, resistive, compressible, Hall MHD equations. \cite{Laakmann_Hu_Farrell_2022} introduces finite-element(-in-space) implicit timesteppers for the incompressible analogue to this system with structure-preserving (SP) properties in the ideal case, alongside parameter-robust preconditioners. We show that these timesteppers can derive from a finite-element-in-time (FET) (and finite-element-in-space) interpretation. The benefits of this reformulation are discussed, including the derivation of timesteppers that are higher order in time, and the quantifiable dissipative SP properties in the non-ideal, resistive case.
        
        We discuss possible options for extending this FET approach to timesteppers for the compressible case.

        The kinetic corrections satisfy linearized Boltzmann equations. Using a Lénard--Bernstein collision operator, these take Fokker--Planck-like forms \cite{Fokker_1914, Planck_1917} wherein pseudo-particles in the numerical model obey the neoclassical transport equations, with particle-independent Brownian drift terms. This offers a rigorous methodology for incorporating collisions into the particle transport model, without coupling the equations of motions for each particle.
        
        Works by Chen, Chacón et al. \cite{Chen_Chacón_Barnes_2011, Chacón_Chen_Barnes_2013, Chen_Chacón_2014, Chen_Chacón_2015} have developed structure-preserving particle pushers for neoclassical transport in the Vlasov equations, derived from Crank--Nicolson integrators. We show these too can can derive from a FET interpretation, similarly offering potential extensions to higher-order-in-time particle pushers. The FET formulation is used also to consider how the stochastic drift terms can be incorporated into the pushers. Stochastic gyrokinetic expansions are also discussed.

        Different options for the numerical implementation of these schemes are considered.

        Due to the efficacy of FET in the development of SP timesteppers for both the fluid and kinetic component, we hope this approach will prove effective in the future for developing SP timesteppers for the full hybrid model. We hope this will give us the opportunity to incorporate previously inaccessible kinetic effects into the highly effective, modern, finite-element MHD models.
    \end{abstract}
    
    
    \newpage
    \tableofcontents
    
    
    \newpage
    \pagenumbering{arabic}
    %\linenumbers\renewcommand\thelinenumber{\color{black!50}\arabic{linenumber}}
            \input{0 - introduction/main.tex}
        \part{Research}
            \input{1 - low-noise PiC models/main.tex}
            \input{2 - kinetic component/main.tex}
            \input{3 - fluid component/main.tex}
            \input{4 - numerical implementation/main.tex}
        \part{Project Overview}
            \input{5 - research plan/main.tex}
            \input{6 - summary/main.tex}
    
    
    %\section{}
    \newpage
    \pagenumbering{gobble}
        \printbibliography


    \newpage
    \pagenumbering{roman}
    \appendix
        \part{Appendices}
            \input{8 - Hilbert complexes/main.tex}
            \input{9 - weak conservation proofs/main.tex}
\end{document}

        \part{Project Overview}
            \documentclass[12pt, a4paper]{report}

\input{template/main.tex}

\title{\BA{Title in Progress...}}
\author{Boris Andrews}
\affil{Mathematical Institute, University of Oxford}
\date{\today}


\begin{document}
    \pagenumbering{gobble}
    \maketitle
    
    
    \begin{abstract}
        Magnetic confinement reactors---in particular tokamaks---offer one of the most promising options for achieving practical nuclear fusion, with the potential to provide virtually limitless, clean energy. The theoretical and numerical modeling of tokamak plasmas is simultaneously an essential component of effective reactor design, and a great research barrier. Tokamak operational conditions exhibit comparatively low Knudsen numbers. Kinetic effects, including kinetic waves and instabilities, Landau damping, bump-on-tail instabilities and more, are therefore highly influential in tokamak plasma dynamics. Purely fluid models are inherently incapable of capturing these effects, whereas the high dimensionality in purely kinetic models render them practically intractable for most relevant purposes.

        We consider a $\delta\!f$ decomposition model, with a macroscopic fluid background and microscopic kinetic correction, both fully coupled to each other. A similar manner of discretization is proposed to that used in the recent \texttt{STRUPHY} code \cite{Holderied_Possanner_Wang_2021, Holderied_2022, Li_et_al_2023} with a finite-element model for the background and a pseudo-particle/PiC model for the correction.

        The fluid background satisfies the full, non-linear, resistive, compressible, Hall MHD equations. \cite{Laakmann_Hu_Farrell_2022} introduces finite-element(-in-space) implicit timesteppers for the incompressible analogue to this system with structure-preserving (SP) properties in the ideal case, alongside parameter-robust preconditioners. We show that these timesteppers can derive from a finite-element-in-time (FET) (and finite-element-in-space) interpretation. The benefits of this reformulation are discussed, including the derivation of timesteppers that are higher order in time, and the quantifiable dissipative SP properties in the non-ideal, resistive case.
        
        We discuss possible options for extending this FET approach to timesteppers for the compressible case.

        The kinetic corrections satisfy linearized Boltzmann equations. Using a Lénard--Bernstein collision operator, these take Fokker--Planck-like forms \cite{Fokker_1914, Planck_1917} wherein pseudo-particles in the numerical model obey the neoclassical transport equations, with particle-independent Brownian drift terms. This offers a rigorous methodology for incorporating collisions into the particle transport model, without coupling the equations of motions for each particle.
        
        Works by Chen, Chacón et al. \cite{Chen_Chacón_Barnes_2011, Chacón_Chen_Barnes_2013, Chen_Chacón_2014, Chen_Chacón_2015} have developed structure-preserving particle pushers for neoclassical transport in the Vlasov equations, derived from Crank--Nicolson integrators. We show these too can can derive from a FET interpretation, similarly offering potential extensions to higher-order-in-time particle pushers. The FET formulation is used also to consider how the stochastic drift terms can be incorporated into the pushers. Stochastic gyrokinetic expansions are also discussed.

        Different options for the numerical implementation of these schemes are considered.

        Due to the efficacy of FET in the development of SP timesteppers for both the fluid and kinetic component, we hope this approach will prove effective in the future for developing SP timesteppers for the full hybrid model. We hope this will give us the opportunity to incorporate previously inaccessible kinetic effects into the highly effective, modern, finite-element MHD models.
    \end{abstract}
    
    
    \newpage
    \tableofcontents
    
    
    \newpage
    \pagenumbering{arabic}
    %\linenumbers\renewcommand\thelinenumber{\color{black!50}\arabic{linenumber}}
            \input{0 - introduction/main.tex}
        \part{Research}
            \input{1 - low-noise PiC models/main.tex}
            \input{2 - kinetic component/main.tex}
            \input{3 - fluid component/main.tex}
            \input{4 - numerical implementation/main.tex}
        \part{Project Overview}
            \input{5 - research plan/main.tex}
            \input{6 - summary/main.tex}
    
    
    %\section{}
    \newpage
    \pagenumbering{gobble}
        \printbibliography


    \newpage
    \pagenumbering{roman}
    \appendix
        \part{Appendices}
            \input{8 - Hilbert complexes/main.tex}
            \input{9 - weak conservation proofs/main.tex}
\end{document}

            \documentclass[12pt, a4paper]{report}

\input{template/main.tex}

\title{\BA{Title in Progress...}}
\author{Boris Andrews}
\affil{Mathematical Institute, University of Oxford}
\date{\today}


\begin{document}
    \pagenumbering{gobble}
    \maketitle
    
    
    \begin{abstract}
        Magnetic confinement reactors---in particular tokamaks---offer one of the most promising options for achieving practical nuclear fusion, with the potential to provide virtually limitless, clean energy. The theoretical and numerical modeling of tokamak plasmas is simultaneously an essential component of effective reactor design, and a great research barrier. Tokamak operational conditions exhibit comparatively low Knudsen numbers. Kinetic effects, including kinetic waves and instabilities, Landau damping, bump-on-tail instabilities and more, are therefore highly influential in tokamak plasma dynamics. Purely fluid models are inherently incapable of capturing these effects, whereas the high dimensionality in purely kinetic models render them practically intractable for most relevant purposes.

        We consider a $\delta\!f$ decomposition model, with a macroscopic fluid background and microscopic kinetic correction, both fully coupled to each other. A similar manner of discretization is proposed to that used in the recent \texttt{STRUPHY} code \cite{Holderied_Possanner_Wang_2021, Holderied_2022, Li_et_al_2023} with a finite-element model for the background and a pseudo-particle/PiC model for the correction.

        The fluid background satisfies the full, non-linear, resistive, compressible, Hall MHD equations. \cite{Laakmann_Hu_Farrell_2022} introduces finite-element(-in-space) implicit timesteppers for the incompressible analogue to this system with structure-preserving (SP) properties in the ideal case, alongside parameter-robust preconditioners. We show that these timesteppers can derive from a finite-element-in-time (FET) (and finite-element-in-space) interpretation. The benefits of this reformulation are discussed, including the derivation of timesteppers that are higher order in time, and the quantifiable dissipative SP properties in the non-ideal, resistive case.
        
        We discuss possible options for extending this FET approach to timesteppers for the compressible case.

        The kinetic corrections satisfy linearized Boltzmann equations. Using a Lénard--Bernstein collision operator, these take Fokker--Planck-like forms \cite{Fokker_1914, Planck_1917} wherein pseudo-particles in the numerical model obey the neoclassical transport equations, with particle-independent Brownian drift terms. This offers a rigorous methodology for incorporating collisions into the particle transport model, without coupling the equations of motions for each particle.
        
        Works by Chen, Chacón et al. \cite{Chen_Chacón_Barnes_2011, Chacón_Chen_Barnes_2013, Chen_Chacón_2014, Chen_Chacón_2015} have developed structure-preserving particle pushers for neoclassical transport in the Vlasov equations, derived from Crank--Nicolson integrators. We show these too can can derive from a FET interpretation, similarly offering potential extensions to higher-order-in-time particle pushers. The FET formulation is used also to consider how the stochastic drift terms can be incorporated into the pushers. Stochastic gyrokinetic expansions are also discussed.

        Different options for the numerical implementation of these schemes are considered.

        Due to the efficacy of FET in the development of SP timesteppers for both the fluid and kinetic component, we hope this approach will prove effective in the future for developing SP timesteppers for the full hybrid model. We hope this will give us the opportunity to incorporate previously inaccessible kinetic effects into the highly effective, modern, finite-element MHD models.
    \end{abstract}
    
    
    \newpage
    \tableofcontents
    
    
    \newpage
    \pagenumbering{arabic}
    %\linenumbers\renewcommand\thelinenumber{\color{black!50}\arabic{linenumber}}
            \input{0 - introduction/main.tex}
        \part{Research}
            \input{1 - low-noise PiC models/main.tex}
            \input{2 - kinetic component/main.tex}
            \input{3 - fluid component/main.tex}
            \input{4 - numerical implementation/main.tex}
        \part{Project Overview}
            \input{5 - research plan/main.tex}
            \input{6 - summary/main.tex}
    
    
    %\section{}
    \newpage
    \pagenumbering{gobble}
        \printbibliography


    \newpage
    \pagenumbering{roman}
    \appendix
        \part{Appendices}
            \input{8 - Hilbert complexes/main.tex}
            \input{9 - weak conservation proofs/main.tex}
\end{document}

    
    
    %\section{}
    \newpage
    \pagenumbering{gobble}
        \printbibliography


    \newpage
    \pagenumbering{roman}
    \appendix
        \part{Appendices}
            \documentclass[12pt, a4paper]{report}

\input{template/main.tex}

\title{\BA{Title in Progress...}}
\author{Boris Andrews}
\affil{Mathematical Institute, University of Oxford}
\date{\today}


\begin{document}
    \pagenumbering{gobble}
    \maketitle
    
    
    \begin{abstract}
        Magnetic confinement reactors---in particular tokamaks---offer one of the most promising options for achieving practical nuclear fusion, with the potential to provide virtually limitless, clean energy. The theoretical and numerical modeling of tokamak plasmas is simultaneously an essential component of effective reactor design, and a great research barrier. Tokamak operational conditions exhibit comparatively low Knudsen numbers. Kinetic effects, including kinetic waves and instabilities, Landau damping, bump-on-tail instabilities and more, are therefore highly influential in tokamak plasma dynamics. Purely fluid models are inherently incapable of capturing these effects, whereas the high dimensionality in purely kinetic models render them practically intractable for most relevant purposes.

        We consider a $\delta\!f$ decomposition model, with a macroscopic fluid background and microscopic kinetic correction, both fully coupled to each other. A similar manner of discretization is proposed to that used in the recent \texttt{STRUPHY} code \cite{Holderied_Possanner_Wang_2021, Holderied_2022, Li_et_al_2023} with a finite-element model for the background and a pseudo-particle/PiC model for the correction.

        The fluid background satisfies the full, non-linear, resistive, compressible, Hall MHD equations. \cite{Laakmann_Hu_Farrell_2022} introduces finite-element(-in-space) implicit timesteppers for the incompressible analogue to this system with structure-preserving (SP) properties in the ideal case, alongside parameter-robust preconditioners. We show that these timesteppers can derive from a finite-element-in-time (FET) (and finite-element-in-space) interpretation. The benefits of this reformulation are discussed, including the derivation of timesteppers that are higher order in time, and the quantifiable dissipative SP properties in the non-ideal, resistive case.
        
        We discuss possible options for extending this FET approach to timesteppers for the compressible case.

        The kinetic corrections satisfy linearized Boltzmann equations. Using a Lénard--Bernstein collision operator, these take Fokker--Planck-like forms \cite{Fokker_1914, Planck_1917} wherein pseudo-particles in the numerical model obey the neoclassical transport equations, with particle-independent Brownian drift terms. This offers a rigorous methodology for incorporating collisions into the particle transport model, without coupling the equations of motions for each particle.
        
        Works by Chen, Chacón et al. \cite{Chen_Chacón_Barnes_2011, Chacón_Chen_Barnes_2013, Chen_Chacón_2014, Chen_Chacón_2015} have developed structure-preserving particle pushers for neoclassical transport in the Vlasov equations, derived from Crank--Nicolson integrators. We show these too can can derive from a FET interpretation, similarly offering potential extensions to higher-order-in-time particle pushers. The FET formulation is used also to consider how the stochastic drift terms can be incorporated into the pushers. Stochastic gyrokinetic expansions are also discussed.

        Different options for the numerical implementation of these schemes are considered.

        Due to the efficacy of FET in the development of SP timesteppers for both the fluid and kinetic component, we hope this approach will prove effective in the future for developing SP timesteppers for the full hybrid model. We hope this will give us the opportunity to incorporate previously inaccessible kinetic effects into the highly effective, modern, finite-element MHD models.
    \end{abstract}
    
    
    \newpage
    \tableofcontents
    
    
    \newpage
    \pagenumbering{arabic}
    %\linenumbers\renewcommand\thelinenumber{\color{black!50}\arabic{linenumber}}
            \input{0 - introduction/main.tex}
        \part{Research}
            \input{1 - low-noise PiC models/main.tex}
            \input{2 - kinetic component/main.tex}
            \input{3 - fluid component/main.tex}
            \input{4 - numerical implementation/main.tex}
        \part{Project Overview}
            \input{5 - research plan/main.tex}
            \input{6 - summary/main.tex}
    
    
    %\section{}
    \newpage
    \pagenumbering{gobble}
        \printbibliography


    \newpage
    \pagenumbering{roman}
    \appendix
        \part{Appendices}
            \input{8 - Hilbert complexes/main.tex}
            \input{9 - weak conservation proofs/main.tex}
\end{document}

            \documentclass[12pt, a4paper]{report}

\input{template/main.tex}

\title{\BA{Title in Progress...}}
\author{Boris Andrews}
\affil{Mathematical Institute, University of Oxford}
\date{\today}


\begin{document}
    \pagenumbering{gobble}
    \maketitle
    
    
    \begin{abstract}
        Magnetic confinement reactors---in particular tokamaks---offer one of the most promising options for achieving practical nuclear fusion, with the potential to provide virtually limitless, clean energy. The theoretical and numerical modeling of tokamak plasmas is simultaneously an essential component of effective reactor design, and a great research barrier. Tokamak operational conditions exhibit comparatively low Knudsen numbers. Kinetic effects, including kinetic waves and instabilities, Landau damping, bump-on-tail instabilities and more, are therefore highly influential in tokamak plasma dynamics. Purely fluid models are inherently incapable of capturing these effects, whereas the high dimensionality in purely kinetic models render them practically intractable for most relevant purposes.

        We consider a $\delta\!f$ decomposition model, with a macroscopic fluid background and microscopic kinetic correction, both fully coupled to each other. A similar manner of discretization is proposed to that used in the recent \texttt{STRUPHY} code \cite{Holderied_Possanner_Wang_2021, Holderied_2022, Li_et_al_2023} with a finite-element model for the background and a pseudo-particle/PiC model for the correction.

        The fluid background satisfies the full, non-linear, resistive, compressible, Hall MHD equations. \cite{Laakmann_Hu_Farrell_2022} introduces finite-element(-in-space) implicit timesteppers for the incompressible analogue to this system with structure-preserving (SP) properties in the ideal case, alongside parameter-robust preconditioners. We show that these timesteppers can derive from a finite-element-in-time (FET) (and finite-element-in-space) interpretation. The benefits of this reformulation are discussed, including the derivation of timesteppers that are higher order in time, and the quantifiable dissipative SP properties in the non-ideal, resistive case.
        
        We discuss possible options for extending this FET approach to timesteppers for the compressible case.

        The kinetic corrections satisfy linearized Boltzmann equations. Using a Lénard--Bernstein collision operator, these take Fokker--Planck-like forms \cite{Fokker_1914, Planck_1917} wherein pseudo-particles in the numerical model obey the neoclassical transport equations, with particle-independent Brownian drift terms. This offers a rigorous methodology for incorporating collisions into the particle transport model, without coupling the equations of motions for each particle.
        
        Works by Chen, Chacón et al. \cite{Chen_Chacón_Barnes_2011, Chacón_Chen_Barnes_2013, Chen_Chacón_2014, Chen_Chacón_2015} have developed structure-preserving particle pushers for neoclassical transport in the Vlasov equations, derived from Crank--Nicolson integrators. We show these too can can derive from a FET interpretation, similarly offering potential extensions to higher-order-in-time particle pushers. The FET formulation is used also to consider how the stochastic drift terms can be incorporated into the pushers. Stochastic gyrokinetic expansions are also discussed.

        Different options for the numerical implementation of these schemes are considered.

        Due to the efficacy of FET in the development of SP timesteppers for both the fluid and kinetic component, we hope this approach will prove effective in the future for developing SP timesteppers for the full hybrid model. We hope this will give us the opportunity to incorporate previously inaccessible kinetic effects into the highly effective, modern, finite-element MHD models.
    \end{abstract}
    
    
    \newpage
    \tableofcontents
    
    
    \newpage
    \pagenumbering{arabic}
    %\linenumbers\renewcommand\thelinenumber{\color{black!50}\arabic{linenumber}}
            \input{0 - introduction/main.tex}
        \part{Research}
            \input{1 - low-noise PiC models/main.tex}
            \input{2 - kinetic component/main.tex}
            \input{3 - fluid component/main.tex}
            \input{4 - numerical implementation/main.tex}
        \part{Project Overview}
            \input{5 - research plan/main.tex}
            \input{6 - summary/main.tex}
    
    
    %\section{}
    \newpage
    \pagenumbering{gobble}
        \printbibliography


    \newpage
    \pagenumbering{roman}
    \appendix
        \part{Appendices}
            \input{8 - Hilbert complexes/main.tex}
            \input{9 - weak conservation proofs/main.tex}
\end{document}

\end{document}

        \part{Project Overview}
            \documentclass[12pt, a4paper]{report}

\documentclass[12pt, a4paper]{report}

\input{template/main.tex}

\title{\BA{Title in Progress...}}
\author{Boris Andrews}
\affil{Mathematical Institute, University of Oxford}
\date{\today}


\begin{document}
    \pagenumbering{gobble}
    \maketitle
    
    
    \begin{abstract}
        Magnetic confinement reactors---in particular tokamaks---offer one of the most promising options for achieving practical nuclear fusion, with the potential to provide virtually limitless, clean energy. The theoretical and numerical modeling of tokamak plasmas is simultaneously an essential component of effective reactor design, and a great research barrier. Tokamak operational conditions exhibit comparatively low Knudsen numbers. Kinetic effects, including kinetic waves and instabilities, Landau damping, bump-on-tail instabilities and more, are therefore highly influential in tokamak plasma dynamics. Purely fluid models are inherently incapable of capturing these effects, whereas the high dimensionality in purely kinetic models render them practically intractable for most relevant purposes.

        We consider a $\delta\!f$ decomposition model, with a macroscopic fluid background and microscopic kinetic correction, both fully coupled to each other. A similar manner of discretization is proposed to that used in the recent \texttt{STRUPHY} code \cite{Holderied_Possanner_Wang_2021, Holderied_2022, Li_et_al_2023} with a finite-element model for the background and a pseudo-particle/PiC model for the correction.

        The fluid background satisfies the full, non-linear, resistive, compressible, Hall MHD equations. \cite{Laakmann_Hu_Farrell_2022} introduces finite-element(-in-space) implicit timesteppers for the incompressible analogue to this system with structure-preserving (SP) properties in the ideal case, alongside parameter-robust preconditioners. We show that these timesteppers can derive from a finite-element-in-time (FET) (and finite-element-in-space) interpretation. The benefits of this reformulation are discussed, including the derivation of timesteppers that are higher order in time, and the quantifiable dissipative SP properties in the non-ideal, resistive case.
        
        We discuss possible options for extending this FET approach to timesteppers for the compressible case.

        The kinetic corrections satisfy linearized Boltzmann equations. Using a Lénard--Bernstein collision operator, these take Fokker--Planck-like forms \cite{Fokker_1914, Planck_1917} wherein pseudo-particles in the numerical model obey the neoclassical transport equations, with particle-independent Brownian drift terms. This offers a rigorous methodology for incorporating collisions into the particle transport model, without coupling the equations of motions for each particle.
        
        Works by Chen, Chacón et al. \cite{Chen_Chacón_Barnes_2011, Chacón_Chen_Barnes_2013, Chen_Chacón_2014, Chen_Chacón_2015} have developed structure-preserving particle pushers for neoclassical transport in the Vlasov equations, derived from Crank--Nicolson integrators. We show these too can can derive from a FET interpretation, similarly offering potential extensions to higher-order-in-time particle pushers. The FET formulation is used also to consider how the stochastic drift terms can be incorporated into the pushers. Stochastic gyrokinetic expansions are also discussed.

        Different options for the numerical implementation of these schemes are considered.

        Due to the efficacy of FET in the development of SP timesteppers for both the fluid and kinetic component, we hope this approach will prove effective in the future for developing SP timesteppers for the full hybrid model. We hope this will give us the opportunity to incorporate previously inaccessible kinetic effects into the highly effective, modern, finite-element MHD models.
    \end{abstract}
    
    
    \newpage
    \tableofcontents
    
    
    \newpage
    \pagenumbering{arabic}
    %\linenumbers\renewcommand\thelinenumber{\color{black!50}\arabic{linenumber}}
            \input{0 - introduction/main.tex}
        \part{Research}
            \input{1 - low-noise PiC models/main.tex}
            \input{2 - kinetic component/main.tex}
            \input{3 - fluid component/main.tex}
            \input{4 - numerical implementation/main.tex}
        \part{Project Overview}
            \input{5 - research plan/main.tex}
            \input{6 - summary/main.tex}
    
    
    %\section{}
    \newpage
    \pagenumbering{gobble}
        \printbibliography


    \newpage
    \pagenumbering{roman}
    \appendix
        \part{Appendices}
            \input{8 - Hilbert complexes/main.tex}
            \input{9 - weak conservation proofs/main.tex}
\end{document}


\title{\BA{Title in Progress...}}
\author{Boris Andrews}
\affil{Mathematical Institute, University of Oxford}
\date{\today}


\begin{document}
    \pagenumbering{gobble}
    \maketitle
    
    
    \begin{abstract}
        Magnetic confinement reactors---in particular tokamaks---offer one of the most promising options for achieving practical nuclear fusion, with the potential to provide virtually limitless, clean energy. The theoretical and numerical modeling of tokamak plasmas is simultaneously an essential component of effective reactor design, and a great research barrier. Tokamak operational conditions exhibit comparatively low Knudsen numbers. Kinetic effects, including kinetic waves and instabilities, Landau damping, bump-on-tail instabilities and more, are therefore highly influential in tokamak plasma dynamics. Purely fluid models are inherently incapable of capturing these effects, whereas the high dimensionality in purely kinetic models render them practically intractable for most relevant purposes.

        We consider a $\delta\!f$ decomposition model, with a macroscopic fluid background and microscopic kinetic correction, both fully coupled to each other. A similar manner of discretization is proposed to that used in the recent \texttt{STRUPHY} code \cite{Holderied_Possanner_Wang_2021, Holderied_2022, Li_et_al_2023} with a finite-element model for the background and a pseudo-particle/PiC model for the correction.

        The fluid background satisfies the full, non-linear, resistive, compressible, Hall MHD equations. \cite{Laakmann_Hu_Farrell_2022} introduces finite-element(-in-space) implicit timesteppers for the incompressible analogue to this system with structure-preserving (SP) properties in the ideal case, alongside parameter-robust preconditioners. We show that these timesteppers can derive from a finite-element-in-time (FET) (and finite-element-in-space) interpretation. The benefits of this reformulation are discussed, including the derivation of timesteppers that are higher order in time, and the quantifiable dissipative SP properties in the non-ideal, resistive case.
        
        We discuss possible options for extending this FET approach to timesteppers for the compressible case.

        The kinetic corrections satisfy linearized Boltzmann equations. Using a Lénard--Bernstein collision operator, these take Fokker--Planck-like forms \cite{Fokker_1914, Planck_1917} wherein pseudo-particles in the numerical model obey the neoclassical transport equations, with particle-independent Brownian drift terms. This offers a rigorous methodology for incorporating collisions into the particle transport model, without coupling the equations of motions for each particle.
        
        Works by Chen, Chacón et al. \cite{Chen_Chacón_Barnes_2011, Chacón_Chen_Barnes_2013, Chen_Chacón_2014, Chen_Chacón_2015} have developed structure-preserving particle pushers for neoclassical transport in the Vlasov equations, derived from Crank--Nicolson integrators. We show these too can can derive from a FET interpretation, similarly offering potential extensions to higher-order-in-time particle pushers. The FET formulation is used also to consider how the stochastic drift terms can be incorporated into the pushers. Stochastic gyrokinetic expansions are also discussed.

        Different options for the numerical implementation of these schemes are considered.

        Due to the efficacy of FET in the development of SP timesteppers for both the fluid and kinetic component, we hope this approach will prove effective in the future for developing SP timesteppers for the full hybrid model. We hope this will give us the opportunity to incorporate previously inaccessible kinetic effects into the highly effective, modern, finite-element MHD models.
    \end{abstract}
    
    
    \newpage
    \tableofcontents
    
    
    \newpage
    \pagenumbering{arabic}
    %\linenumbers\renewcommand\thelinenumber{\color{black!50}\arabic{linenumber}}
            \documentclass[12pt, a4paper]{report}

\input{template/main.tex}

\title{\BA{Title in Progress...}}
\author{Boris Andrews}
\affil{Mathematical Institute, University of Oxford}
\date{\today}


\begin{document}
    \pagenumbering{gobble}
    \maketitle
    
    
    \begin{abstract}
        Magnetic confinement reactors---in particular tokamaks---offer one of the most promising options for achieving practical nuclear fusion, with the potential to provide virtually limitless, clean energy. The theoretical and numerical modeling of tokamak plasmas is simultaneously an essential component of effective reactor design, and a great research barrier. Tokamak operational conditions exhibit comparatively low Knudsen numbers. Kinetic effects, including kinetic waves and instabilities, Landau damping, bump-on-tail instabilities and more, are therefore highly influential in tokamak plasma dynamics. Purely fluid models are inherently incapable of capturing these effects, whereas the high dimensionality in purely kinetic models render them practically intractable for most relevant purposes.

        We consider a $\delta\!f$ decomposition model, with a macroscopic fluid background and microscopic kinetic correction, both fully coupled to each other. A similar manner of discretization is proposed to that used in the recent \texttt{STRUPHY} code \cite{Holderied_Possanner_Wang_2021, Holderied_2022, Li_et_al_2023} with a finite-element model for the background and a pseudo-particle/PiC model for the correction.

        The fluid background satisfies the full, non-linear, resistive, compressible, Hall MHD equations. \cite{Laakmann_Hu_Farrell_2022} introduces finite-element(-in-space) implicit timesteppers for the incompressible analogue to this system with structure-preserving (SP) properties in the ideal case, alongside parameter-robust preconditioners. We show that these timesteppers can derive from a finite-element-in-time (FET) (and finite-element-in-space) interpretation. The benefits of this reformulation are discussed, including the derivation of timesteppers that are higher order in time, and the quantifiable dissipative SP properties in the non-ideal, resistive case.
        
        We discuss possible options for extending this FET approach to timesteppers for the compressible case.

        The kinetic corrections satisfy linearized Boltzmann equations. Using a Lénard--Bernstein collision operator, these take Fokker--Planck-like forms \cite{Fokker_1914, Planck_1917} wherein pseudo-particles in the numerical model obey the neoclassical transport equations, with particle-independent Brownian drift terms. This offers a rigorous methodology for incorporating collisions into the particle transport model, without coupling the equations of motions for each particle.
        
        Works by Chen, Chacón et al. \cite{Chen_Chacón_Barnes_2011, Chacón_Chen_Barnes_2013, Chen_Chacón_2014, Chen_Chacón_2015} have developed structure-preserving particle pushers for neoclassical transport in the Vlasov equations, derived from Crank--Nicolson integrators. We show these too can can derive from a FET interpretation, similarly offering potential extensions to higher-order-in-time particle pushers. The FET formulation is used also to consider how the stochastic drift terms can be incorporated into the pushers. Stochastic gyrokinetic expansions are also discussed.

        Different options for the numerical implementation of these schemes are considered.

        Due to the efficacy of FET in the development of SP timesteppers for both the fluid and kinetic component, we hope this approach will prove effective in the future for developing SP timesteppers for the full hybrid model. We hope this will give us the opportunity to incorporate previously inaccessible kinetic effects into the highly effective, modern, finite-element MHD models.
    \end{abstract}
    
    
    \newpage
    \tableofcontents
    
    
    \newpage
    \pagenumbering{arabic}
    %\linenumbers\renewcommand\thelinenumber{\color{black!50}\arabic{linenumber}}
            \input{0 - introduction/main.tex}
        \part{Research}
            \input{1 - low-noise PiC models/main.tex}
            \input{2 - kinetic component/main.tex}
            \input{3 - fluid component/main.tex}
            \input{4 - numerical implementation/main.tex}
        \part{Project Overview}
            \input{5 - research plan/main.tex}
            \input{6 - summary/main.tex}
    
    
    %\section{}
    \newpage
    \pagenumbering{gobble}
        \printbibliography


    \newpage
    \pagenumbering{roman}
    \appendix
        \part{Appendices}
            \input{8 - Hilbert complexes/main.tex}
            \input{9 - weak conservation proofs/main.tex}
\end{document}

        \part{Research}
            \documentclass[12pt, a4paper]{report}

\input{template/main.tex}

\title{\BA{Title in Progress...}}
\author{Boris Andrews}
\affil{Mathematical Institute, University of Oxford}
\date{\today}


\begin{document}
    \pagenumbering{gobble}
    \maketitle
    
    
    \begin{abstract}
        Magnetic confinement reactors---in particular tokamaks---offer one of the most promising options for achieving practical nuclear fusion, with the potential to provide virtually limitless, clean energy. The theoretical and numerical modeling of tokamak plasmas is simultaneously an essential component of effective reactor design, and a great research barrier. Tokamak operational conditions exhibit comparatively low Knudsen numbers. Kinetic effects, including kinetic waves and instabilities, Landau damping, bump-on-tail instabilities and more, are therefore highly influential in tokamak plasma dynamics. Purely fluid models are inherently incapable of capturing these effects, whereas the high dimensionality in purely kinetic models render them practically intractable for most relevant purposes.

        We consider a $\delta\!f$ decomposition model, with a macroscopic fluid background and microscopic kinetic correction, both fully coupled to each other. A similar manner of discretization is proposed to that used in the recent \texttt{STRUPHY} code \cite{Holderied_Possanner_Wang_2021, Holderied_2022, Li_et_al_2023} with a finite-element model for the background and a pseudo-particle/PiC model for the correction.

        The fluid background satisfies the full, non-linear, resistive, compressible, Hall MHD equations. \cite{Laakmann_Hu_Farrell_2022} introduces finite-element(-in-space) implicit timesteppers for the incompressible analogue to this system with structure-preserving (SP) properties in the ideal case, alongside parameter-robust preconditioners. We show that these timesteppers can derive from a finite-element-in-time (FET) (and finite-element-in-space) interpretation. The benefits of this reformulation are discussed, including the derivation of timesteppers that are higher order in time, and the quantifiable dissipative SP properties in the non-ideal, resistive case.
        
        We discuss possible options for extending this FET approach to timesteppers for the compressible case.

        The kinetic corrections satisfy linearized Boltzmann equations. Using a Lénard--Bernstein collision operator, these take Fokker--Planck-like forms \cite{Fokker_1914, Planck_1917} wherein pseudo-particles in the numerical model obey the neoclassical transport equations, with particle-independent Brownian drift terms. This offers a rigorous methodology for incorporating collisions into the particle transport model, without coupling the equations of motions for each particle.
        
        Works by Chen, Chacón et al. \cite{Chen_Chacón_Barnes_2011, Chacón_Chen_Barnes_2013, Chen_Chacón_2014, Chen_Chacón_2015} have developed structure-preserving particle pushers for neoclassical transport in the Vlasov equations, derived from Crank--Nicolson integrators. We show these too can can derive from a FET interpretation, similarly offering potential extensions to higher-order-in-time particle pushers. The FET formulation is used also to consider how the stochastic drift terms can be incorporated into the pushers. Stochastic gyrokinetic expansions are also discussed.

        Different options for the numerical implementation of these schemes are considered.

        Due to the efficacy of FET in the development of SP timesteppers for both the fluid and kinetic component, we hope this approach will prove effective in the future for developing SP timesteppers for the full hybrid model. We hope this will give us the opportunity to incorporate previously inaccessible kinetic effects into the highly effective, modern, finite-element MHD models.
    \end{abstract}
    
    
    \newpage
    \tableofcontents
    
    
    \newpage
    \pagenumbering{arabic}
    %\linenumbers\renewcommand\thelinenumber{\color{black!50}\arabic{linenumber}}
            \input{0 - introduction/main.tex}
        \part{Research}
            \input{1 - low-noise PiC models/main.tex}
            \input{2 - kinetic component/main.tex}
            \input{3 - fluid component/main.tex}
            \input{4 - numerical implementation/main.tex}
        \part{Project Overview}
            \input{5 - research plan/main.tex}
            \input{6 - summary/main.tex}
    
    
    %\section{}
    \newpage
    \pagenumbering{gobble}
        \printbibliography


    \newpage
    \pagenumbering{roman}
    \appendix
        \part{Appendices}
            \input{8 - Hilbert complexes/main.tex}
            \input{9 - weak conservation proofs/main.tex}
\end{document}

            \documentclass[12pt, a4paper]{report}

\input{template/main.tex}

\title{\BA{Title in Progress...}}
\author{Boris Andrews}
\affil{Mathematical Institute, University of Oxford}
\date{\today}


\begin{document}
    \pagenumbering{gobble}
    \maketitle
    
    
    \begin{abstract}
        Magnetic confinement reactors---in particular tokamaks---offer one of the most promising options for achieving practical nuclear fusion, with the potential to provide virtually limitless, clean energy. The theoretical and numerical modeling of tokamak plasmas is simultaneously an essential component of effective reactor design, and a great research barrier. Tokamak operational conditions exhibit comparatively low Knudsen numbers. Kinetic effects, including kinetic waves and instabilities, Landau damping, bump-on-tail instabilities and more, are therefore highly influential in tokamak plasma dynamics. Purely fluid models are inherently incapable of capturing these effects, whereas the high dimensionality in purely kinetic models render them practically intractable for most relevant purposes.

        We consider a $\delta\!f$ decomposition model, with a macroscopic fluid background and microscopic kinetic correction, both fully coupled to each other. A similar manner of discretization is proposed to that used in the recent \texttt{STRUPHY} code \cite{Holderied_Possanner_Wang_2021, Holderied_2022, Li_et_al_2023} with a finite-element model for the background and a pseudo-particle/PiC model for the correction.

        The fluid background satisfies the full, non-linear, resistive, compressible, Hall MHD equations. \cite{Laakmann_Hu_Farrell_2022} introduces finite-element(-in-space) implicit timesteppers for the incompressible analogue to this system with structure-preserving (SP) properties in the ideal case, alongside parameter-robust preconditioners. We show that these timesteppers can derive from a finite-element-in-time (FET) (and finite-element-in-space) interpretation. The benefits of this reformulation are discussed, including the derivation of timesteppers that are higher order in time, and the quantifiable dissipative SP properties in the non-ideal, resistive case.
        
        We discuss possible options for extending this FET approach to timesteppers for the compressible case.

        The kinetic corrections satisfy linearized Boltzmann equations. Using a Lénard--Bernstein collision operator, these take Fokker--Planck-like forms \cite{Fokker_1914, Planck_1917} wherein pseudo-particles in the numerical model obey the neoclassical transport equations, with particle-independent Brownian drift terms. This offers a rigorous methodology for incorporating collisions into the particle transport model, without coupling the equations of motions for each particle.
        
        Works by Chen, Chacón et al. \cite{Chen_Chacón_Barnes_2011, Chacón_Chen_Barnes_2013, Chen_Chacón_2014, Chen_Chacón_2015} have developed structure-preserving particle pushers for neoclassical transport in the Vlasov equations, derived from Crank--Nicolson integrators. We show these too can can derive from a FET interpretation, similarly offering potential extensions to higher-order-in-time particle pushers. The FET formulation is used also to consider how the stochastic drift terms can be incorporated into the pushers. Stochastic gyrokinetic expansions are also discussed.

        Different options for the numerical implementation of these schemes are considered.

        Due to the efficacy of FET in the development of SP timesteppers for both the fluid and kinetic component, we hope this approach will prove effective in the future for developing SP timesteppers for the full hybrid model. We hope this will give us the opportunity to incorporate previously inaccessible kinetic effects into the highly effective, modern, finite-element MHD models.
    \end{abstract}
    
    
    \newpage
    \tableofcontents
    
    
    \newpage
    \pagenumbering{arabic}
    %\linenumbers\renewcommand\thelinenumber{\color{black!50}\arabic{linenumber}}
            \input{0 - introduction/main.tex}
        \part{Research}
            \input{1 - low-noise PiC models/main.tex}
            \input{2 - kinetic component/main.tex}
            \input{3 - fluid component/main.tex}
            \input{4 - numerical implementation/main.tex}
        \part{Project Overview}
            \input{5 - research plan/main.tex}
            \input{6 - summary/main.tex}
    
    
    %\section{}
    \newpage
    \pagenumbering{gobble}
        \printbibliography


    \newpage
    \pagenumbering{roman}
    \appendix
        \part{Appendices}
            \input{8 - Hilbert complexes/main.tex}
            \input{9 - weak conservation proofs/main.tex}
\end{document}

            \documentclass[12pt, a4paper]{report}

\input{template/main.tex}

\title{\BA{Title in Progress...}}
\author{Boris Andrews}
\affil{Mathematical Institute, University of Oxford}
\date{\today}


\begin{document}
    \pagenumbering{gobble}
    \maketitle
    
    
    \begin{abstract}
        Magnetic confinement reactors---in particular tokamaks---offer one of the most promising options for achieving practical nuclear fusion, with the potential to provide virtually limitless, clean energy. The theoretical and numerical modeling of tokamak plasmas is simultaneously an essential component of effective reactor design, and a great research barrier. Tokamak operational conditions exhibit comparatively low Knudsen numbers. Kinetic effects, including kinetic waves and instabilities, Landau damping, bump-on-tail instabilities and more, are therefore highly influential in tokamak plasma dynamics. Purely fluid models are inherently incapable of capturing these effects, whereas the high dimensionality in purely kinetic models render them practically intractable for most relevant purposes.

        We consider a $\delta\!f$ decomposition model, with a macroscopic fluid background and microscopic kinetic correction, both fully coupled to each other. A similar manner of discretization is proposed to that used in the recent \texttt{STRUPHY} code \cite{Holderied_Possanner_Wang_2021, Holderied_2022, Li_et_al_2023} with a finite-element model for the background and a pseudo-particle/PiC model for the correction.

        The fluid background satisfies the full, non-linear, resistive, compressible, Hall MHD equations. \cite{Laakmann_Hu_Farrell_2022} introduces finite-element(-in-space) implicit timesteppers for the incompressible analogue to this system with structure-preserving (SP) properties in the ideal case, alongside parameter-robust preconditioners. We show that these timesteppers can derive from a finite-element-in-time (FET) (and finite-element-in-space) interpretation. The benefits of this reformulation are discussed, including the derivation of timesteppers that are higher order in time, and the quantifiable dissipative SP properties in the non-ideal, resistive case.
        
        We discuss possible options for extending this FET approach to timesteppers for the compressible case.

        The kinetic corrections satisfy linearized Boltzmann equations. Using a Lénard--Bernstein collision operator, these take Fokker--Planck-like forms \cite{Fokker_1914, Planck_1917} wherein pseudo-particles in the numerical model obey the neoclassical transport equations, with particle-independent Brownian drift terms. This offers a rigorous methodology for incorporating collisions into the particle transport model, without coupling the equations of motions for each particle.
        
        Works by Chen, Chacón et al. \cite{Chen_Chacón_Barnes_2011, Chacón_Chen_Barnes_2013, Chen_Chacón_2014, Chen_Chacón_2015} have developed structure-preserving particle pushers for neoclassical transport in the Vlasov equations, derived from Crank--Nicolson integrators. We show these too can can derive from a FET interpretation, similarly offering potential extensions to higher-order-in-time particle pushers. The FET formulation is used also to consider how the stochastic drift terms can be incorporated into the pushers. Stochastic gyrokinetic expansions are also discussed.

        Different options for the numerical implementation of these schemes are considered.

        Due to the efficacy of FET in the development of SP timesteppers for both the fluid and kinetic component, we hope this approach will prove effective in the future for developing SP timesteppers for the full hybrid model. We hope this will give us the opportunity to incorporate previously inaccessible kinetic effects into the highly effective, modern, finite-element MHD models.
    \end{abstract}
    
    
    \newpage
    \tableofcontents
    
    
    \newpage
    \pagenumbering{arabic}
    %\linenumbers\renewcommand\thelinenumber{\color{black!50}\arabic{linenumber}}
            \input{0 - introduction/main.tex}
        \part{Research}
            \input{1 - low-noise PiC models/main.tex}
            \input{2 - kinetic component/main.tex}
            \input{3 - fluid component/main.tex}
            \input{4 - numerical implementation/main.tex}
        \part{Project Overview}
            \input{5 - research plan/main.tex}
            \input{6 - summary/main.tex}
    
    
    %\section{}
    \newpage
    \pagenumbering{gobble}
        \printbibliography


    \newpage
    \pagenumbering{roman}
    \appendix
        \part{Appendices}
            \input{8 - Hilbert complexes/main.tex}
            \input{9 - weak conservation proofs/main.tex}
\end{document}

            \documentclass[12pt, a4paper]{report}

\input{template/main.tex}

\title{\BA{Title in Progress...}}
\author{Boris Andrews}
\affil{Mathematical Institute, University of Oxford}
\date{\today}


\begin{document}
    \pagenumbering{gobble}
    \maketitle
    
    
    \begin{abstract}
        Magnetic confinement reactors---in particular tokamaks---offer one of the most promising options for achieving practical nuclear fusion, with the potential to provide virtually limitless, clean energy. The theoretical and numerical modeling of tokamak plasmas is simultaneously an essential component of effective reactor design, and a great research barrier. Tokamak operational conditions exhibit comparatively low Knudsen numbers. Kinetic effects, including kinetic waves and instabilities, Landau damping, bump-on-tail instabilities and more, are therefore highly influential in tokamak plasma dynamics. Purely fluid models are inherently incapable of capturing these effects, whereas the high dimensionality in purely kinetic models render them practically intractable for most relevant purposes.

        We consider a $\delta\!f$ decomposition model, with a macroscopic fluid background and microscopic kinetic correction, both fully coupled to each other. A similar manner of discretization is proposed to that used in the recent \texttt{STRUPHY} code \cite{Holderied_Possanner_Wang_2021, Holderied_2022, Li_et_al_2023} with a finite-element model for the background and a pseudo-particle/PiC model for the correction.

        The fluid background satisfies the full, non-linear, resistive, compressible, Hall MHD equations. \cite{Laakmann_Hu_Farrell_2022} introduces finite-element(-in-space) implicit timesteppers for the incompressible analogue to this system with structure-preserving (SP) properties in the ideal case, alongside parameter-robust preconditioners. We show that these timesteppers can derive from a finite-element-in-time (FET) (and finite-element-in-space) interpretation. The benefits of this reformulation are discussed, including the derivation of timesteppers that are higher order in time, and the quantifiable dissipative SP properties in the non-ideal, resistive case.
        
        We discuss possible options for extending this FET approach to timesteppers for the compressible case.

        The kinetic corrections satisfy linearized Boltzmann equations. Using a Lénard--Bernstein collision operator, these take Fokker--Planck-like forms \cite{Fokker_1914, Planck_1917} wherein pseudo-particles in the numerical model obey the neoclassical transport equations, with particle-independent Brownian drift terms. This offers a rigorous methodology for incorporating collisions into the particle transport model, without coupling the equations of motions for each particle.
        
        Works by Chen, Chacón et al. \cite{Chen_Chacón_Barnes_2011, Chacón_Chen_Barnes_2013, Chen_Chacón_2014, Chen_Chacón_2015} have developed structure-preserving particle pushers for neoclassical transport in the Vlasov equations, derived from Crank--Nicolson integrators. We show these too can can derive from a FET interpretation, similarly offering potential extensions to higher-order-in-time particle pushers. The FET formulation is used also to consider how the stochastic drift terms can be incorporated into the pushers. Stochastic gyrokinetic expansions are also discussed.

        Different options for the numerical implementation of these schemes are considered.

        Due to the efficacy of FET in the development of SP timesteppers for both the fluid and kinetic component, we hope this approach will prove effective in the future for developing SP timesteppers for the full hybrid model. We hope this will give us the opportunity to incorporate previously inaccessible kinetic effects into the highly effective, modern, finite-element MHD models.
    \end{abstract}
    
    
    \newpage
    \tableofcontents
    
    
    \newpage
    \pagenumbering{arabic}
    %\linenumbers\renewcommand\thelinenumber{\color{black!50}\arabic{linenumber}}
            \input{0 - introduction/main.tex}
        \part{Research}
            \input{1 - low-noise PiC models/main.tex}
            \input{2 - kinetic component/main.tex}
            \input{3 - fluid component/main.tex}
            \input{4 - numerical implementation/main.tex}
        \part{Project Overview}
            \input{5 - research plan/main.tex}
            \input{6 - summary/main.tex}
    
    
    %\section{}
    \newpage
    \pagenumbering{gobble}
        \printbibliography


    \newpage
    \pagenumbering{roman}
    \appendix
        \part{Appendices}
            \input{8 - Hilbert complexes/main.tex}
            \input{9 - weak conservation proofs/main.tex}
\end{document}

        \part{Project Overview}
            \documentclass[12pt, a4paper]{report}

\input{template/main.tex}

\title{\BA{Title in Progress...}}
\author{Boris Andrews}
\affil{Mathematical Institute, University of Oxford}
\date{\today}


\begin{document}
    \pagenumbering{gobble}
    \maketitle
    
    
    \begin{abstract}
        Magnetic confinement reactors---in particular tokamaks---offer one of the most promising options for achieving practical nuclear fusion, with the potential to provide virtually limitless, clean energy. The theoretical and numerical modeling of tokamak plasmas is simultaneously an essential component of effective reactor design, and a great research barrier. Tokamak operational conditions exhibit comparatively low Knudsen numbers. Kinetic effects, including kinetic waves and instabilities, Landau damping, bump-on-tail instabilities and more, are therefore highly influential in tokamak plasma dynamics. Purely fluid models are inherently incapable of capturing these effects, whereas the high dimensionality in purely kinetic models render them practically intractable for most relevant purposes.

        We consider a $\delta\!f$ decomposition model, with a macroscopic fluid background and microscopic kinetic correction, both fully coupled to each other. A similar manner of discretization is proposed to that used in the recent \texttt{STRUPHY} code \cite{Holderied_Possanner_Wang_2021, Holderied_2022, Li_et_al_2023} with a finite-element model for the background and a pseudo-particle/PiC model for the correction.

        The fluid background satisfies the full, non-linear, resistive, compressible, Hall MHD equations. \cite{Laakmann_Hu_Farrell_2022} introduces finite-element(-in-space) implicit timesteppers for the incompressible analogue to this system with structure-preserving (SP) properties in the ideal case, alongside parameter-robust preconditioners. We show that these timesteppers can derive from a finite-element-in-time (FET) (and finite-element-in-space) interpretation. The benefits of this reformulation are discussed, including the derivation of timesteppers that are higher order in time, and the quantifiable dissipative SP properties in the non-ideal, resistive case.
        
        We discuss possible options for extending this FET approach to timesteppers for the compressible case.

        The kinetic corrections satisfy linearized Boltzmann equations. Using a Lénard--Bernstein collision operator, these take Fokker--Planck-like forms \cite{Fokker_1914, Planck_1917} wherein pseudo-particles in the numerical model obey the neoclassical transport equations, with particle-independent Brownian drift terms. This offers a rigorous methodology for incorporating collisions into the particle transport model, without coupling the equations of motions for each particle.
        
        Works by Chen, Chacón et al. \cite{Chen_Chacón_Barnes_2011, Chacón_Chen_Barnes_2013, Chen_Chacón_2014, Chen_Chacón_2015} have developed structure-preserving particle pushers for neoclassical transport in the Vlasov equations, derived from Crank--Nicolson integrators. We show these too can can derive from a FET interpretation, similarly offering potential extensions to higher-order-in-time particle pushers. The FET formulation is used also to consider how the stochastic drift terms can be incorporated into the pushers. Stochastic gyrokinetic expansions are also discussed.

        Different options for the numerical implementation of these schemes are considered.

        Due to the efficacy of FET in the development of SP timesteppers for both the fluid and kinetic component, we hope this approach will prove effective in the future for developing SP timesteppers for the full hybrid model. We hope this will give us the opportunity to incorporate previously inaccessible kinetic effects into the highly effective, modern, finite-element MHD models.
    \end{abstract}
    
    
    \newpage
    \tableofcontents
    
    
    \newpage
    \pagenumbering{arabic}
    %\linenumbers\renewcommand\thelinenumber{\color{black!50}\arabic{linenumber}}
            \input{0 - introduction/main.tex}
        \part{Research}
            \input{1 - low-noise PiC models/main.tex}
            \input{2 - kinetic component/main.tex}
            \input{3 - fluid component/main.tex}
            \input{4 - numerical implementation/main.tex}
        \part{Project Overview}
            \input{5 - research plan/main.tex}
            \input{6 - summary/main.tex}
    
    
    %\section{}
    \newpage
    \pagenumbering{gobble}
        \printbibliography


    \newpage
    \pagenumbering{roman}
    \appendix
        \part{Appendices}
            \input{8 - Hilbert complexes/main.tex}
            \input{9 - weak conservation proofs/main.tex}
\end{document}

            \documentclass[12pt, a4paper]{report}

\input{template/main.tex}

\title{\BA{Title in Progress...}}
\author{Boris Andrews}
\affil{Mathematical Institute, University of Oxford}
\date{\today}


\begin{document}
    \pagenumbering{gobble}
    \maketitle
    
    
    \begin{abstract}
        Magnetic confinement reactors---in particular tokamaks---offer one of the most promising options for achieving practical nuclear fusion, with the potential to provide virtually limitless, clean energy. The theoretical and numerical modeling of tokamak plasmas is simultaneously an essential component of effective reactor design, and a great research barrier. Tokamak operational conditions exhibit comparatively low Knudsen numbers. Kinetic effects, including kinetic waves and instabilities, Landau damping, bump-on-tail instabilities and more, are therefore highly influential in tokamak plasma dynamics. Purely fluid models are inherently incapable of capturing these effects, whereas the high dimensionality in purely kinetic models render them practically intractable for most relevant purposes.

        We consider a $\delta\!f$ decomposition model, with a macroscopic fluid background and microscopic kinetic correction, both fully coupled to each other. A similar manner of discretization is proposed to that used in the recent \texttt{STRUPHY} code \cite{Holderied_Possanner_Wang_2021, Holderied_2022, Li_et_al_2023} with a finite-element model for the background and a pseudo-particle/PiC model for the correction.

        The fluid background satisfies the full, non-linear, resistive, compressible, Hall MHD equations. \cite{Laakmann_Hu_Farrell_2022} introduces finite-element(-in-space) implicit timesteppers for the incompressible analogue to this system with structure-preserving (SP) properties in the ideal case, alongside parameter-robust preconditioners. We show that these timesteppers can derive from a finite-element-in-time (FET) (and finite-element-in-space) interpretation. The benefits of this reformulation are discussed, including the derivation of timesteppers that are higher order in time, and the quantifiable dissipative SP properties in the non-ideal, resistive case.
        
        We discuss possible options for extending this FET approach to timesteppers for the compressible case.

        The kinetic corrections satisfy linearized Boltzmann equations. Using a Lénard--Bernstein collision operator, these take Fokker--Planck-like forms \cite{Fokker_1914, Planck_1917} wherein pseudo-particles in the numerical model obey the neoclassical transport equations, with particle-independent Brownian drift terms. This offers a rigorous methodology for incorporating collisions into the particle transport model, without coupling the equations of motions for each particle.
        
        Works by Chen, Chacón et al. \cite{Chen_Chacón_Barnes_2011, Chacón_Chen_Barnes_2013, Chen_Chacón_2014, Chen_Chacón_2015} have developed structure-preserving particle pushers for neoclassical transport in the Vlasov equations, derived from Crank--Nicolson integrators. We show these too can can derive from a FET interpretation, similarly offering potential extensions to higher-order-in-time particle pushers. The FET formulation is used also to consider how the stochastic drift terms can be incorporated into the pushers. Stochastic gyrokinetic expansions are also discussed.

        Different options for the numerical implementation of these schemes are considered.

        Due to the efficacy of FET in the development of SP timesteppers for both the fluid and kinetic component, we hope this approach will prove effective in the future for developing SP timesteppers for the full hybrid model. We hope this will give us the opportunity to incorporate previously inaccessible kinetic effects into the highly effective, modern, finite-element MHD models.
    \end{abstract}
    
    
    \newpage
    \tableofcontents
    
    
    \newpage
    \pagenumbering{arabic}
    %\linenumbers\renewcommand\thelinenumber{\color{black!50}\arabic{linenumber}}
            \input{0 - introduction/main.tex}
        \part{Research}
            \input{1 - low-noise PiC models/main.tex}
            \input{2 - kinetic component/main.tex}
            \input{3 - fluid component/main.tex}
            \input{4 - numerical implementation/main.tex}
        \part{Project Overview}
            \input{5 - research plan/main.tex}
            \input{6 - summary/main.tex}
    
    
    %\section{}
    \newpage
    \pagenumbering{gobble}
        \printbibliography


    \newpage
    \pagenumbering{roman}
    \appendix
        \part{Appendices}
            \input{8 - Hilbert complexes/main.tex}
            \input{9 - weak conservation proofs/main.tex}
\end{document}

    
    
    %\section{}
    \newpage
    \pagenumbering{gobble}
        \printbibliography


    \newpage
    \pagenumbering{roman}
    \appendix
        \part{Appendices}
            \documentclass[12pt, a4paper]{report}

\input{template/main.tex}

\title{\BA{Title in Progress...}}
\author{Boris Andrews}
\affil{Mathematical Institute, University of Oxford}
\date{\today}


\begin{document}
    \pagenumbering{gobble}
    \maketitle
    
    
    \begin{abstract}
        Magnetic confinement reactors---in particular tokamaks---offer one of the most promising options for achieving practical nuclear fusion, with the potential to provide virtually limitless, clean energy. The theoretical and numerical modeling of tokamak plasmas is simultaneously an essential component of effective reactor design, and a great research barrier. Tokamak operational conditions exhibit comparatively low Knudsen numbers. Kinetic effects, including kinetic waves and instabilities, Landau damping, bump-on-tail instabilities and more, are therefore highly influential in tokamak plasma dynamics. Purely fluid models are inherently incapable of capturing these effects, whereas the high dimensionality in purely kinetic models render them practically intractable for most relevant purposes.

        We consider a $\delta\!f$ decomposition model, with a macroscopic fluid background and microscopic kinetic correction, both fully coupled to each other. A similar manner of discretization is proposed to that used in the recent \texttt{STRUPHY} code \cite{Holderied_Possanner_Wang_2021, Holderied_2022, Li_et_al_2023} with a finite-element model for the background and a pseudo-particle/PiC model for the correction.

        The fluid background satisfies the full, non-linear, resistive, compressible, Hall MHD equations. \cite{Laakmann_Hu_Farrell_2022} introduces finite-element(-in-space) implicit timesteppers for the incompressible analogue to this system with structure-preserving (SP) properties in the ideal case, alongside parameter-robust preconditioners. We show that these timesteppers can derive from a finite-element-in-time (FET) (and finite-element-in-space) interpretation. The benefits of this reformulation are discussed, including the derivation of timesteppers that are higher order in time, and the quantifiable dissipative SP properties in the non-ideal, resistive case.
        
        We discuss possible options for extending this FET approach to timesteppers for the compressible case.

        The kinetic corrections satisfy linearized Boltzmann equations. Using a Lénard--Bernstein collision operator, these take Fokker--Planck-like forms \cite{Fokker_1914, Planck_1917} wherein pseudo-particles in the numerical model obey the neoclassical transport equations, with particle-independent Brownian drift terms. This offers a rigorous methodology for incorporating collisions into the particle transport model, without coupling the equations of motions for each particle.
        
        Works by Chen, Chacón et al. \cite{Chen_Chacón_Barnes_2011, Chacón_Chen_Barnes_2013, Chen_Chacón_2014, Chen_Chacón_2015} have developed structure-preserving particle pushers for neoclassical transport in the Vlasov equations, derived from Crank--Nicolson integrators. We show these too can can derive from a FET interpretation, similarly offering potential extensions to higher-order-in-time particle pushers. The FET formulation is used also to consider how the stochastic drift terms can be incorporated into the pushers. Stochastic gyrokinetic expansions are also discussed.

        Different options for the numerical implementation of these schemes are considered.

        Due to the efficacy of FET in the development of SP timesteppers for both the fluid and kinetic component, we hope this approach will prove effective in the future for developing SP timesteppers for the full hybrid model. We hope this will give us the opportunity to incorporate previously inaccessible kinetic effects into the highly effective, modern, finite-element MHD models.
    \end{abstract}
    
    
    \newpage
    \tableofcontents
    
    
    \newpage
    \pagenumbering{arabic}
    %\linenumbers\renewcommand\thelinenumber{\color{black!50}\arabic{linenumber}}
            \input{0 - introduction/main.tex}
        \part{Research}
            \input{1 - low-noise PiC models/main.tex}
            \input{2 - kinetic component/main.tex}
            \input{3 - fluid component/main.tex}
            \input{4 - numerical implementation/main.tex}
        \part{Project Overview}
            \input{5 - research plan/main.tex}
            \input{6 - summary/main.tex}
    
    
    %\section{}
    \newpage
    \pagenumbering{gobble}
        \printbibliography


    \newpage
    \pagenumbering{roman}
    \appendix
        \part{Appendices}
            \input{8 - Hilbert complexes/main.tex}
            \input{9 - weak conservation proofs/main.tex}
\end{document}

            \documentclass[12pt, a4paper]{report}

\input{template/main.tex}

\title{\BA{Title in Progress...}}
\author{Boris Andrews}
\affil{Mathematical Institute, University of Oxford}
\date{\today}


\begin{document}
    \pagenumbering{gobble}
    \maketitle
    
    
    \begin{abstract}
        Magnetic confinement reactors---in particular tokamaks---offer one of the most promising options for achieving practical nuclear fusion, with the potential to provide virtually limitless, clean energy. The theoretical and numerical modeling of tokamak plasmas is simultaneously an essential component of effective reactor design, and a great research barrier. Tokamak operational conditions exhibit comparatively low Knudsen numbers. Kinetic effects, including kinetic waves and instabilities, Landau damping, bump-on-tail instabilities and more, are therefore highly influential in tokamak plasma dynamics. Purely fluid models are inherently incapable of capturing these effects, whereas the high dimensionality in purely kinetic models render them practically intractable for most relevant purposes.

        We consider a $\delta\!f$ decomposition model, with a macroscopic fluid background and microscopic kinetic correction, both fully coupled to each other. A similar manner of discretization is proposed to that used in the recent \texttt{STRUPHY} code \cite{Holderied_Possanner_Wang_2021, Holderied_2022, Li_et_al_2023} with a finite-element model for the background and a pseudo-particle/PiC model for the correction.

        The fluid background satisfies the full, non-linear, resistive, compressible, Hall MHD equations. \cite{Laakmann_Hu_Farrell_2022} introduces finite-element(-in-space) implicit timesteppers for the incompressible analogue to this system with structure-preserving (SP) properties in the ideal case, alongside parameter-robust preconditioners. We show that these timesteppers can derive from a finite-element-in-time (FET) (and finite-element-in-space) interpretation. The benefits of this reformulation are discussed, including the derivation of timesteppers that are higher order in time, and the quantifiable dissipative SP properties in the non-ideal, resistive case.
        
        We discuss possible options for extending this FET approach to timesteppers for the compressible case.

        The kinetic corrections satisfy linearized Boltzmann equations. Using a Lénard--Bernstein collision operator, these take Fokker--Planck-like forms \cite{Fokker_1914, Planck_1917} wherein pseudo-particles in the numerical model obey the neoclassical transport equations, with particle-independent Brownian drift terms. This offers a rigorous methodology for incorporating collisions into the particle transport model, without coupling the equations of motions for each particle.
        
        Works by Chen, Chacón et al. \cite{Chen_Chacón_Barnes_2011, Chacón_Chen_Barnes_2013, Chen_Chacón_2014, Chen_Chacón_2015} have developed structure-preserving particle pushers for neoclassical transport in the Vlasov equations, derived from Crank--Nicolson integrators. We show these too can can derive from a FET interpretation, similarly offering potential extensions to higher-order-in-time particle pushers. The FET formulation is used also to consider how the stochastic drift terms can be incorporated into the pushers. Stochastic gyrokinetic expansions are also discussed.

        Different options for the numerical implementation of these schemes are considered.

        Due to the efficacy of FET in the development of SP timesteppers for both the fluid and kinetic component, we hope this approach will prove effective in the future for developing SP timesteppers for the full hybrid model. We hope this will give us the opportunity to incorporate previously inaccessible kinetic effects into the highly effective, modern, finite-element MHD models.
    \end{abstract}
    
    
    \newpage
    \tableofcontents
    
    
    \newpage
    \pagenumbering{arabic}
    %\linenumbers\renewcommand\thelinenumber{\color{black!50}\arabic{linenumber}}
            \input{0 - introduction/main.tex}
        \part{Research}
            \input{1 - low-noise PiC models/main.tex}
            \input{2 - kinetic component/main.tex}
            \input{3 - fluid component/main.tex}
            \input{4 - numerical implementation/main.tex}
        \part{Project Overview}
            \input{5 - research plan/main.tex}
            \input{6 - summary/main.tex}
    
    
    %\section{}
    \newpage
    \pagenumbering{gobble}
        \printbibliography


    \newpage
    \pagenumbering{roman}
    \appendix
        \part{Appendices}
            \input{8 - Hilbert complexes/main.tex}
            \input{9 - weak conservation proofs/main.tex}
\end{document}

\end{document}

            \documentclass[12pt, a4paper]{report}

\documentclass[12pt, a4paper]{report}

\input{template/main.tex}

\title{\BA{Title in Progress...}}
\author{Boris Andrews}
\affil{Mathematical Institute, University of Oxford}
\date{\today}


\begin{document}
    \pagenumbering{gobble}
    \maketitle
    
    
    \begin{abstract}
        Magnetic confinement reactors---in particular tokamaks---offer one of the most promising options for achieving practical nuclear fusion, with the potential to provide virtually limitless, clean energy. The theoretical and numerical modeling of tokamak plasmas is simultaneously an essential component of effective reactor design, and a great research barrier. Tokamak operational conditions exhibit comparatively low Knudsen numbers. Kinetic effects, including kinetic waves and instabilities, Landau damping, bump-on-tail instabilities and more, are therefore highly influential in tokamak plasma dynamics. Purely fluid models are inherently incapable of capturing these effects, whereas the high dimensionality in purely kinetic models render them practically intractable for most relevant purposes.

        We consider a $\delta\!f$ decomposition model, with a macroscopic fluid background and microscopic kinetic correction, both fully coupled to each other. A similar manner of discretization is proposed to that used in the recent \texttt{STRUPHY} code \cite{Holderied_Possanner_Wang_2021, Holderied_2022, Li_et_al_2023} with a finite-element model for the background and a pseudo-particle/PiC model for the correction.

        The fluid background satisfies the full, non-linear, resistive, compressible, Hall MHD equations. \cite{Laakmann_Hu_Farrell_2022} introduces finite-element(-in-space) implicit timesteppers for the incompressible analogue to this system with structure-preserving (SP) properties in the ideal case, alongside parameter-robust preconditioners. We show that these timesteppers can derive from a finite-element-in-time (FET) (and finite-element-in-space) interpretation. The benefits of this reformulation are discussed, including the derivation of timesteppers that are higher order in time, and the quantifiable dissipative SP properties in the non-ideal, resistive case.
        
        We discuss possible options for extending this FET approach to timesteppers for the compressible case.

        The kinetic corrections satisfy linearized Boltzmann equations. Using a Lénard--Bernstein collision operator, these take Fokker--Planck-like forms \cite{Fokker_1914, Planck_1917} wherein pseudo-particles in the numerical model obey the neoclassical transport equations, with particle-independent Brownian drift terms. This offers a rigorous methodology for incorporating collisions into the particle transport model, without coupling the equations of motions for each particle.
        
        Works by Chen, Chacón et al. \cite{Chen_Chacón_Barnes_2011, Chacón_Chen_Barnes_2013, Chen_Chacón_2014, Chen_Chacón_2015} have developed structure-preserving particle pushers for neoclassical transport in the Vlasov equations, derived from Crank--Nicolson integrators. We show these too can can derive from a FET interpretation, similarly offering potential extensions to higher-order-in-time particle pushers. The FET formulation is used also to consider how the stochastic drift terms can be incorporated into the pushers. Stochastic gyrokinetic expansions are also discussed.

        Different options for the numerical implementation of these schemes are considered.

        Due to the efficacy of FET in the development of SP timesteppers for both the fluid and kinetic component, we hope this approach will prove effective in the future for developing SP timesteppers for the full hybrid model. We hope this will give us the opportunity to incorporate previously inaccessible kinetic effects into the highly effective, modern, finite-element MHD models.
    \end{abstract}
    
    
    \newpage
    \tableofcontents
    
    
    \newpage
    \pagenumbering{arabic}
    %\linenumbers\renewcommand\thelinenumber{\color{black!50}\arabic{linenumber}}
            \input{0 - introduction/main.tex}
        \part{Research}
            \input{1 - low-noise PiC models/main.tex}
            \input{2 - kinetic component/main.tex}
            \input{3 - fluid component/main.tex}
            \input{4 - numerical implementation/main.tex}
        \part{Project Overview}
            \input{5 - research plan/main.tex}
            \input{6 - summary/main.tex}
    
    
    %\section{}
    \newpage
    \pagenumbering{gobble}
        \printbibliography


    \newpage
    \pagenumbering{roman}
    \appendix
        \part{Appendices}
            \input{8 - Hilbert complexes/main.tex}
            \input{9 - weak conservation proofs/main.tex}
\end{document}


\title{\BA{Title in Progress...}}
\author{Boris Andrews}
\affil{Mathematical Institute, University of Oxford}
\date{\today}


\begin{document}
    \pagenumbering{gobble}
    \maketitle
    
    
    \begin{abstract}
        Magnetic confinement reactors---in particular tokamaks---offer one of the most promising options for achieving practical nuclear fusion, with the potential to provide virtually limitless, clean energy. The theoretical and numerical modeling of tokamak plasmas is simultaneously an essential component of effective reactor design, and a great research barrier. Tokamak operational conditions exhibit comparatively low Knudsen numbers. Kinetic effects, including kinetic waves and instabilities, Landau damping, bump-on-tail instabilities and more, are therefore highly influential in tokamak plasma dynamics. Purely fluid models are inherently incapable of capturing these effects, whereas the high dimensionality in purely kinetic models render them practically intractable for most relevant purposes.

        We consider a $\delta\!f$ decomposition model, with a macroscopic fluid background and microscopic kinetic correction, both fully coupled to each other. A similar manner of discretization is proposed to that used in the recent \texttt{STRUPHY} code \cite{Holderied_Possanner_Wang_2021, Holderied_2022, Li_et_al_2023} with a finite-element model for the background and a pseudo-particle/PiC model for the correction.

        The fluid background satisfies the full, non-linear, resistive, compressible, Hall MHD equations. \cite{Laakmann_Hu_Farrell_2022} introduces finite-element(-in-space) implicit timesteppers for the incompressible analogue to this system with structure-preserving (SP) properties in the ideal case, alongside parameter-robust preconditioners. We show that these timesteppers can derive from a finite-element-in-time (FET) (and finite-element-in-space) interpretation. The benefits of this reformulation are discussed, including the derivation of timesteppers that are higher order in time, and the quantifiable dissipative SP properties in the non-ideal, resistive case.
        
        We discuss possible options for extending this FET approach to timesteppers for the compressible case.

        The kinetic corrections satisfy linearized Boltzmann equations. Using a Lénard--Bernstein collision operator, these take Fokker--Planck-like forms \cite{Fokker_1914, Planck_1917} wherein pseudo-particles in the numerical model obey the neoclassical transport equations, with particle-independent Brownian drift terms. This offers a rigorous methodology for incorporating collisions into the particle transport model, without coupling the equations of motions for each particle.
        
        Works by Chen, Chacón et al. \cite{Chen_Chacón_Barnes_2011, Chacón_Chen_Barnes_2013, Chen_Chacón_2014, Chen_Chacón_2015} have developed structure-preserving particle pushers for neoclassical transport in the Vlasov equations, derived from Crank--Nicolson integrators. We show these too can can derive from a FET interpretation, similarly offering potential extensions to higher-order-in-time particle pushers. The FET formulation is used also to consider how the stochastic drift terms can be incorporated into the pushers. Stochastic gyrokinetic expansions are also discussed.

        Different options for the numerical implementation of these schemes are considered.

        Due to the efficacy of FET in the development of SP timesteppers for both the fluid and kinetic component, we hope this approach will prove effective in the future for developing SP timesteppers for the full hybrid model. We hope this will give us the opportunity to incorporate previously inaccessible kinetic effects into the highly effective, modern, finite-element MHD models.
    \end{abstract}
    
    
    \newpage
    \tableofcontents
    
    
    \newpage
    \pagenumbering{arabic}
    %\linenumbers\renewcommand\thelinenumber{\color{black!50}\arabic{linenumber}}
            \documentclass[12pt, a4paper]{report}

\input{template/main.tex}

\title{\BA{Title in Progress...}}
\author{Boris Andrews}
\affil{Mathematical Institute, University of Oxford}
\date{\today}


\begin{document}
    \pagenumbering{gobble}
    \maketitle
    
    
    \begin{abstract}
        Magnetic confinement reactors---in particular tokamaks---offer one of the most promising options for achieving practical nuclear fusion, with the potential to provide virtually limitless, clean energy. The theoretical and numerical modeling of tokamak plasmas is simultaneously an essential component of effective reactor design, and a great research barrier. Tokamak operational conditions exhibit comparatively low Knudsen numbers. Kinetic effects, including kinetic waves and instabilities, Landau damping, bump-on-tail instabilities and more, are therefore highly influential in tokamak plasma dynamics. Purely fluid models are inherently incapable of capturing these effects, whereas the high dimensionality in purely kinetic models render them practically intractable for most relevant purposes.

        We consider a $\delta\!f$ decomposition model, with a macroscopic fluid background and microscopic kinetic correction, both fully coupled to each other. A similar manner of discretization is proposed to that used in the recent \texttt{STRUPHY} code \cite{Holderied_Possanner_Wang_2021, Holderied_2022, Li_et_al_2023} with a finite-element model for the background and a pseudo-particle/PiC model for the correction.

        The fluid background satisfies the full, non-linear, resistive, compressible, Hall MHD equations. \cite{Laakmann_Hu_Farrell_2022} introduces finite-element(-in-space) implicit timesteppers for the incompressible analogue to this system with structure-preserving (SP) properties in the ideal case, alongside parameter-robust preconditioners. We show that these timesteppers can derive from a finite-element-in-time (FET) (and finite-element-in-space) interpretation. The benefits of this reformulation are discussed, including the derivation of timesteppers that are higher order in time, and the quantifiable dissipative SP properties in the non-ideal, resistive case.
        
        We discuss possible options for extending this FET approach to timesteppers for the compressible case.

        The kinetic corrections satisfy linearized Boltzmann equations. Using a Lénard--Bernstein collision operator, these take Fokker--Planck-like forms \cite{Fokker_1914, Planck_1917} wherein pseudo-particles in the numerical model obey the neoclassical transport equations, with particle-independent Brownian drift terms. This offers a rigorous methodology for incorporating collisions into the particle transport model, without coupling the equations of motions for each particle.
        
        Works by Chen, Chacón et al. \cite{Chen_Chacón_Barnes_2011, Chacón_Chen_Barnes_2013, Chen_Chacón_2014, Chen_Chacón_2015} have developed structure-preserving particle pushers for neoclassical transport in the Vlasov equations, derived from Crank--Nicolson integrators. We show these too can can derive from a FET interpretation, similarly offering potential extensions to higher-order-in-time particle pushers. The FET formulation is used also to consider how the stochastic drift terms can be incorporated into the pushers. Stochastic gyrokinetic expansions are also discussed.

        Different options for the numerical implementation of these schemes are considered.

        Due to the efficacy of FET in the development of SP timesteppers for both the fluid and kinetic component, we hope this approach will prove effective in the future for developing SP timesteppers for the full hybrid model. We hope this will give us the opportunity to incorporate previously inaccessible kinetic effects into the highly effective, modern, finite-element MHD models.
    \end{abstract}
    
    
    \newpage
    \tableofcontents
    
    
    \newpage
    \pagenumbering{arabic}
    %\linenumbers\renewcommand\thelinenumber{\color{black!50}\arabic{linenumber}}
            \input{0 - introduction/main.tex}
        \part{Research}
            \input{1 - low-noise PiC models/main.tex}
            \input{2 - kinetic component/main.tex}
            \input{3 - fluid component/main.tex}
            \input{4 - numerical implementation/main.tex}
        \part{Project Overview}
            \input{5 - research plan/main.tex}
            \input{6 - summary/main.tex}
    
    
    %\section{}
    \newpage
    \pagenumbering{gobble}
        \printbibliography


    \newpage
    \pagenumbering{roman}
    \appendix
        \part{Appendices}
            \input{8 - Hilbert complexes/main.tex}
            \input{9 - weak conservation proofs/main.tex}
\end{document}

        \part{Research}
            \documentclass[12pt, a4paper]{report}

\input{template/main.tex}

\title{\BA{Title in Progress...}}
\author{Boris Andrews}
\affil{Mathematical Institute, University of Oxford}
\date{\today}


\begin{document}
    \pagenumbering{gobble}
    \maketitle
    
    
    \begin{abstract}
        Magnetic confinement reactors---in particular tokamaks---offer one of the most promising options for achieving practical nuclear fusion, with the potential to provide virtually limitless, clean energy. The theoretical and numerical modeling of tokamak plasmas is simultaneously an essential component of effective reactor design, and a great research barrier. Tokamak operational conditions exhibit comparatively low Knudsen numbers. Kinetic effects, including kinetic waves and instabilities, Landau damping, bump-on-tail instabilities and more, are therefore highly influential in tokamak plasma dynamics. Purely fluid models are inherently incapable of capturing these effects, whereas the high dimensionality in purely kinetic models render them practically intractable for most relevant purposes.

        We consider a $\delta\!f$ decomposition model, with a macroscopic fluid background and microscopic kinetic correction, both fully coupled to each other. A similar manner of discretization is proposed to that used in the recent \texttt{STRUPHY} code \cite{Holderied_Possanner_Wang_2021, Holderied_2022, Li_et_al_2023} with a finite-element model for the background and a pseudo-particle/PiC model for the correction.

        The fluid background satisfies the full, non-linear, resistive, compressible, Hall MHD equations. \cite{Laakmann_Hu_Farrell_2022} introduces finite-element(-in-space) implicit timesteppers for the incompressible analogue to this system with structure-preserving (SP) properties in the ideal case, alongside parameter-robust preconditioners. We show that these timesteppers can derive from a finite-element-in-time (FET) (and finite-element-in-space) interpretation. The benefits of this reformulation are discussed, including the derivation of timesteppers that are higher order in time, and the quantifiable dissipative SP properties in the non-ideal, resistive case.
        
        We discuss possible options for extending this FET approach to timesteppers for the compressible case.

        The kinetic corrections satisfy linearized Boltzmann equations. Using a Lénard--Bernstein collision operator, these take Fokker--Planck-like forms \cite{Fokker_1914, Planck_1917} wherein pseudo-particles in the numerical model obey the neoclassical transport equations, with particle-independent Brownian drift terms. This offers a rigorous methodology for incorporating collisions into the particle transport model, without coupling the equations of motions for each particle.
        
        Works by Chen, Chacón et al. \cite{Chen_Chacón_Barnes_2011, Chacón_Chen_Barnes_2013, Chen_Chacón_2014, Chen_Chacón_2015} have developed structure-preserving particle pushers for neoclassical transport in the Vlasov equations, derived from Crank--Nicolson integrators. We show these too can can derive from a FET interpretation, similarly offering potential extensions to higher-order-in-time particle pushers. The FET formulation is used also to consider how the stochastic drift terms can be incorporated into the pushers. Stochastic gyrokinetic expansions are also discussed.

        Different options for the numerical implementation of these schemes are considered.

        Due to the efficacy of FET in the development of SP timesteppers for both the fluid and kinetic component, we hope this approach will prove effective in the future for developing SP timesteppers for the full hybrid model. We hope this will give us the opportunity to incorporate previously inaccessible kinetic effects into the highly effective, modern, finite-element MHD models.
    \end{abstract}
    
    
    \newpage
    \tableofcontents
    
    
    \newpage
    \pagenumbering{arabic}
    %\linenumbers\renewcommand\thelinenumber{\color{black!50}\arabic{linenumber}}
            \input{0 - introduction/main.tex}
        \part{Research}
            \input{1 - low-noise PiC models/main.tex}
            \input{2 - kinetic component/main.tex}
            \input{3 - fluid component/main.tex}
            \input{4 - numerical implementation/main.tex}
        \part{Project Overview}
            \input{5 - research plan/main.tex}
            \input{6 - summary/main.tex}
    
    
    %\section{}
    \newpage
    \pagenumbering{gobble}
        \printbibliography


    \newpage
    \pagenumbering{roman}
    \appendix
        \part{Appendices}
            \input{8 - Hilbert complexes/main.tex}
            \input{9 - weak conservation proofs/main.tex}
\end{document}

            \documentclass[12pt, a4paper]{report}

\input{template/main.tex}

\title{\BA{Title in Progress...}}
\author{Boris Andrews}
\affil{Mathematical Institute, University of Oxford}
\date{\today}


\begin{document}
    \pagenumbering{gobble}
    \maketitle
    
    
    \begin{abstract}
        Magnetic confinement reactors---in particular tokamaks---offer one of the most promising options for achieving practical nuclear fusion, with the potential to provide virtually limitless, clean energy. The theoretical and numerical modeling of tokamak plasmas is simultaneously an essential component of effective reactor design, and a great research barrier. Tokamak operational conditions exhibit comparatively low Knudsen numbers. Kinetic effects, including kinetic waves and instabilities, Landau damping, bump-on-tail instabilities and more, are therefore highly influential in tokamak plasma dynamics. Purely fluid models are inherently incapable of capturing these effects, whereas the high dimensionality in purely kinetic models render them practically intractable for most relevant purposes.

        We consider a $\delta\!f$ decomposition model, with a macroscopic fluid background and microscopic kinetic correction, both fully coupled to each other. A similar manner of discretization is proposed to that used in the recent \texttt{STRUPHY} code \cite{Holderied_Possanner_Wang_2021, Holderied_2022, Li_et_al_2023} with a finite-element model for the background and a pseudo-particle/PiC model for the correction.

        The fluid background satisfies the full, non-linear, resistive, compressible, Hall MHD equations. \cite{Laakmann_Hu_Farrell_2022} introduces finite-element(-in-space) implicit timesteppers for the incompressible analogue to this system with structure-preserving (SP) properties in the ideal case, alongside parameter-robust preconditioners. We show that these timesteppers can derive from a finite-element-in-time (FET) (and finite-element-in-space) interpretation. The benefits of this reformulation are discussed, including the derivation of timesteppers that are higher order in time, and the quantifiable dissipative SP properties in the non-ideal, resistive case.
        
        We discuss possible options for extending this FET approach to timesteppers for the compressible case.

        The kinetic corrections satisfy linearized Boltzmann equations. Using a Lénard--Bernstein collision operator, these take Fokker--Planck-like forms \cite{Fokker_1914, Planck_1917} wherein pseudo-particles in the numerical model obey the neoclassical transport equations, with particle-independent Brownian drift terms. This offers a rigorous methodology for incorporating collisions into the particle transport model, without coupling the equations of motions for each particle.
        
        Works by Chen, Chacón et al. \cite{Chen_Chacón_Barnes_2011, Chacón_Chen_Barnes_2013, Chen_Chacón_2014, Chen_Chacón_2015} have developed structure-preserving particle pushers for neoclassical transport in the Vlasov equations, derived from Crank--Nicolson integrators. We show these too can can derive from a FET interpretation, similarly offering potential extensions to higher-order-in-time particle pushers. The FET formulation is used also to consider how the stochastic drift terms can be incorporated into the pushers. Stochastic gyrokinetic expansions are also discussed.

        Different options for the numerical implementation of these schemes are considered.

        Due to the efficacy of FET in the development of SP timesteppers for both the fluid and kinetic component, we hope this approach will prove effective in the future for developing SP timesteppers for the full hybrid model. We hope this will give us the opportunity to incorporate previously inaccessible kinetic effects into the highly effective, modern, finite-element MHD models.
    \end{abstract}
    
    
    \newpage
    \tableofcontents
    
    
    \newpage
    \pagenumbering{arabic}
    %\linenumbers\renewcommand\thelinenumber{\color{black!50}\arabic{linenumber}}
            \input{0 - introduction/main.tex}
        \part{Research}
            \input{1 - low-noise PiC models/main.tex}
            \input{2 - kinetic component/main.tex}
            \input{3 - fluid component/main.tex}
            \input{4 - numerical implementation/main.tex}
        \part{Project Overview}
            \input{5 - research plan/main.tex}
            \input{6 - summary/main.tex}
    
    
    %\section{}
    \newpage
    \pagenumbering{gobble}
        \printbibliography


    \newpage
    \pagenumbering{roman}
    \appendix
        \part{Appendices}
            \input{8 - Hilbert complexes/main.tex}
            \input{9 - weak conservation proofs/main.tex}
\end{document}

            \documentclass[12pt, a4paper]{report}

\input{template/main.tex}

\title{\BA{Title in Progress...}}
\author{Boris Andrews}
\affil{Mathematical Institute, University of Oxford}
\date{\today}


\begin{document}
    \pagenumbering{gobble}
    \maketitle
    
    
    \begin{abstract}
        Magnetic confinement reactors---in particular tokamaks---offer one of the most promising options for achieving practical nuclear fusion, with the potential to provide virtually limitless, clean energy. The theoretical and numerical modeling of tokamak plasmas is simultaneously an essential component of effective reactor design, and a great research barrier. Tokamak operational conditions exhibit comparatively low Knudsen numbers. Kinetic effects, including kinetic waves and instabilities, Landau damping, bump-on-tail instabilities and more, are therefore highly influential in tokamak plasma dynamics. Purely fluid models are inherently incapable of capturing these effects, whereas the high dimensionality in purely kinetic models render them practically intractable for most relevant purposes.

        We consider a $\delta\!f$ decomposition model, with a macroscopic fluid background and microscopic kinetic correction, both fully coupled to each other. A similar manner of discretization is proposed to that used in the recent \texttt{STRUPHY} code \cite{Holderied_Possanner_Wang_2021, Holderied_2022, Li_et_al_2023} with a finite-element model for the background and a pseudo-particle/PiC model for the correction.

        The fluid background satisfies the full, non-linear, resistive, compressible, Hall MHD equations. \cite{Laakmann_Hu_Farrell_2022} introduces finite-element(-in-space) implicit timesteppers for the incompressible analogue to this system with structure-preserving (SP) properties in the ideal case, alongside parameter-robust preconditioners. We show that these timesteppers can derive from a finite-element-in-time (FET) (and finite-element-in-space) interpretation. The benefits of this reformulation are discussed, including the derivation of timesteppers that are higher order in time, and the quantifiable dissipative SP properties in the non-ideal, resistive case.
        
        We discuss possible options for extending this FET approach to timesteppers for the compressible case.

        The kinetic corrections satisfy linearized Boltzmann equations. Using a Lénard--Bernstein collision operator, these take Fokker--Planck-like forms \cite{Fokker_1914, Planck_1917} wherein pseudo-particles in the numerical model obey the neoclassical transport equations, with particle-independent Brownian drift terms. This offers a rigorous methodology for incorporating collisions into the particle transport model, without coupling the equations of motions for each particle.
        
        Works by Chen, Chacón et al. \cite{Chen_Chacón_Barnes_2011, Chacón_Chen_Barnes_2013, Chen_Chacón_2014, Chen_Chacón_2015} have developed structure-preserving particle pushers for neoclassical transport in the Vlasov equations, derived from Crank--Nicolson integrators. We show these too can can derive from a FET interpretation, similarly offering potential extensions to higher-order-in-time particle pushers. The FET formulation is used also to consider how the stochastic drift terms can be incorporated into the pushers. Stochastic gyrokinetic expansions are also discussed.

        Different options for the numerical implementation of these schemes are considered.

        Due to the efficacy of FET in the development of SP timesteppers for both the fluid and kinetic component, we hope this approach will prove effective in the future for developing SP timesteppers for the full hybrid model. We hope this will give us the opportunity to incorporate previously inaccessible kinetic effects into the highly effective, modern, finite-element MHD models.
    \end{abstract}
    
    
    \newpage
    \tableofcontents
    
    
    \newpage
    \pagenumbering{arabic}
    %\linenumbers\renewcommand\thelinenumber{\color{black!50}\arabic{linenumber}}
            \input{0 - introduction/main.tex}
        \part{Research}
            \input{1 - low-noise PiC models/main.tex}
            \input{2 - kinetic component/main.tex}
            \input{3 - fluid component/main.tex}
            \input{4 - numerical implementation/main.tex}
        \part{Project Overview}
            \input{5 - research plan/main.tex}
            \input{6 - summary/main.tex}
    
    
    %\section{}
    \newpage
    \pagenumbering{gobble}
        \printbibliography


    \newpage
    \pagenumbering{roman}
    \appendix
        \part{Appendices}
            \input{8 - Hilbert complexes/main.tex}
            \input{9 - weak conservation proofs/main.tex}
\end{document}

            \documentclass[12pt, a4paper]{report}

\input{template/main.tex}

\title{\BA{Title in Progress...}}
\author{Boris Andrews}
\affil{Mathematical Institute, University of Oxford}
\date{\today}


\begin{document}
    \pagenumbering{gobble}
    \maketitle
    
    
    \begin{abstract}
        Magnetic confinement reactors---in particular tokamaks---offer one of the most promising options for achieving practical nuclear fusion, with the potential to provide virtually limitless, clean energy. The theoretical and numerical modeling of tokamak plasmas is simultaneously an essential component of effective reactor design, and a great research barrier. Tokamak operational conditions exhibit comparatively low Knudsen numbers. Kinetic effects, including kinetic waves and instabilities, Landau damping, bump-on-tail instabilities and more, are therefore highly influential in tokamak plasma dynamics. Purely fluid models are inherently incapable of capturing these effects, whereas the high dimensionality in purely kinetic models render them practically intractable for most relevant purposes.

        We consider a $\delta\!f$ decomposition model, with a macroscopic fluid background and microscopic kinetic correction, both fully coupled to each other. A similar manner of discretization is proposed to that used in the recent \texttt{STRUPHY} code \cite{Holderied_Possanner_Wang_2021, Holderied_2022, Li_et_al_2023} with a finite-element model for the background and a pseudo-particle/PiC model for the correction.

        The fluid background satisfies the full, non-linear, resistive, compressible, Hall MHD equations. \cite{Laakmann_Hu_Farrell_2022} introduces finite-element(-in-space) implicit timesteppers for the incompressible analogue to this system with structure-preserving (SP) properties in the ideal case, alongside parameter-robust preconditioners. We show that these timesteppers can derive from a finite-element-in-time (FET) (and finite-element-in-space) interpretation. The benefits of this reformulation are discussed, including the derivation of timesteppers that are higher order in time, and the quantifiable dissipative SP properties in the non-ideal, resistive case.
        
        We discuss possible options for extending this FET approach to timesteppers for the compressible case.

        The kinetic corrections satisfy linearized Boltzmann equations. Using a Lénard--Bernstein collision operator, these take Fokker--Planck-like forms \cite{Fokker_1914, Planck_1917} wherein pseudo-particles in the numerical model obey the neoclassical transport equations, with particle-independent Brownian drift terms. This offers a rigorous methodology for incorporating collisions into the particle transport model, without coupling the equations of motions for each particle.
        
        Works by Chen, Chacón et al. \cite{Chen_Chacón_Barnes_2011, Chacón_Chen_Barnes_2013, Chen_Chacón_2014, Chen_Chacón_2015} have developed structure-preserving particle pushers for neoclassical transport in the Vlasov equations, derived from Crank--Nicolson integrators. We show these too can can derive from a FET interpretation, similarly offering potential extensions to higher-order-in-time particle pushers. The FET formulation is used also to consider how the stochastic drift terms can be incorporated into the pushers. Stochastic gyrokinetic expansions are also discussed.

        Different options for the numerical implementation of these schemes are considered.

        Due to the efficacy of FET in the development of SP timesteppers for both the fluid and kinetic component, we hope this approach will prove effective in the future for developing SP timesteppers for the full hybrid model. We hope this will give us the opportunity to incorporate previously inaccessible kinetic effects into the highly effective, modern, finite-element MHD models.
    \end{abstract}
    
    
    \newpage
    \tableofcontents
    
    
    \newpage
    \pagenumbering{arabic}
    %\linenumbers\renewcommand\thelinenumber{\color{black!50}\arabic{linenumber}}
            \input{0 - introduction/main.tex}
        \part{Research}
            \input{1 - low-noise PiC models/main.tex}
            \input{2 - kinetic component/main.tex}
            \input{3 - fluid component/main.tex}
            \input{4 - numerical implementation/main.tex}
        \part{Project Overview}
            \input{5 - research plan/main.tex}
            \input{6 - summary/main.tex}
    
    
    %\section{}
    \newpage
    \pagenumbering{gobble}
        \printbibliography


    \newpage
    \pagenumbering{roman}
    \appendix
        \part{Appendices}
            \input{8 - Hilbert complexes/main.tex}
            \input{9 - weak conservation proofs/main.tex}
\end{document}

        \part{Project Overview}
            \documentclass[12pt, a4paper]{report}

\input{template/main.tex}

\title{\BA{Title in Progress...}}
\author{Boris Andrews}
\affil{Mathematical Institute, University of Oxford}
\date{\today}


\begin{document}
    \pagenumbering{gobble}
    \maketitle
    
    
    \begin{abstract}
        Magnetic confinement reactors---in particular tokamaks---offer one of the most promising options for achieving practical nuclear fusion, with the potential to provide virtually limitless, clean energy. The theoretical and numerical modeling of tokamak plasmas is simultaneously an essential component of effective reactor design, and a great research barrier. Tokamak operational conditions exhibit comparatively low Knudsen numbers. Kinetic effects, including kinetic waves and instabilities, Landau damping, bump-on-tail instabilities and more, are therefore highly influential in tokamak plasma dynamics. Purely fluid models are inherently incapable of capturing these effects, whereas the high dimensionality in purely kinetic models render them practically intractable for most relevant purposes.

        We consider a $\delta\!f$ decomposition model, with a macroscopic fluid background and microscopic kinetic correction, both fully coupled to each other. A similar manner of discretization is proposed to that used in the recent \texttt{STRUPHY} code \cite{Holderied_Possanner_Wang_2021, Holderied_2022, Li_et_al_2023} with a finite-element model for the background and a pseudo-particle/PiC model for the correction.

        The fluid background satisfies the full, non-linear, resistive, compressible, Hall MHD equations. \cite{Laakmann_Hu_Farrell_2022} introduces finite-element(-in-space) implicit timesteppers for the incompressible analogue to this system with structure-preserving (SP) properties in the ideal case, alongside parameter-robust preconditioners. We show that these timesteppers can derive from a finite-element-in-time (FET) (and finite-element-in-space) interpretation. The benefits of this reformulation are discussed, including the derivation of timesteppers that are higher order in time, and the quantifiable dissipative SP properties in the non-ideal, resistive case.
        
        We discuss possible options for extending this FET approach to timesteppers for the compressible case.

        The kinetic corrections satisfy linearized Boltzmann equations. Using a Lénard--Bernstein collision operator, these take Fokker--Planck-like forms \cite{Fokker_1914, Planck_1917} wherein pseudo-particles in the numerical model obey the neoclassical transport equations, with particle-independent Brownian drift terms. This offers a rigorous methodology for incorporating collisions into the particle transport model, without coupling the equations of motions for each particle.
        
        Works by Chen, Chacón et al. \cite{Chen_Chacón_Barnes_2011, Chacón_Chen_Barnes_2013, Chen_Chacón_2014, Chen_Chacón_2015} have developed structure-preserving particle pushers for neoclassical transport in the Vlasov equations, derived from Crank--Nicolson integrators. We show these too can can derive from a FET interpretation, similarly offering potential extensions to higher-order-in-time particle pushers. The FET formulation is used also to consider how the stochastic drift terms can be incorporated into the pushers. Stochastic gyrokinetic expansions are also discussed.

        Different options for the numerical implementation of these schemes are considered.

        Due to the efficacy of FET in the development of SP timesteppers for both the fluid and kinetic component, we hope this approach will prove effective in the future for developing SP timesteppers for the full hybrid model. We hope this will give us the opportunity to incorporate previously inaccessible kinetic effects into the highly effective, modern, finite-element MHD models.
    \end{abstract}
    
    
    \newpage
    \tableofcontents
    
    
    \newpage
    \pagenumbering{arabic}
    %\linenumbers\renewcommand\thelinenumber{\color{black!50}\arabic{linenumber}}
            \input{0 - introduction/main.tex}
        \part{Research}
            \input{1 - low-noise PiC models/main.tex}
            \input{2 - kinetic component/main.tex}
            \input{3 - fluid component/main.tex}
            \input{4 - numerical implementation/main.tex}
        \part{Project Overview}
            \input{5 - research plan/main.tex}
            \input{6 - summary/main.tex}
    
    
    %\section{}
    \newpage
    \pagenumbering{gobble}
        \printbibliography


    \newpage
    \pagenumbering{roman}
    \appendix
        \part{Appendices}
            \input{8 - Hilbert complexes/main.tex}
            \input{9 - weak conservation proofs/main.tex}
\end{document}

            \documentclass[12pt, a4paper]{report}

\input{template/main.tex}

\title{\BA{Title in Progress...}}
\author{Boris Andrews}
\affil{Mathematical Institute, University of Oxford}
\date{\today}


\begin{document}
    \pagenumbering{gobble}
    \maketitle
    
    
    \begin{abstract}
        Magnetic confinement reactors---in particular tokamaks---offer one of the most promising options for achieving practical nuclear fusion, with the potential to provide virtually limitless, clean energy. The theoretical and numerical modeling of tokamak plasmas is simultaneously an essential component of effective reactor design, and a great research barrier. Tokamak operational conditions exhibit comparatively low Knudsen numbers. Kinetic effects, including kinetic waves and instabilities, Landau damping, bump-on-tail instabilities and more, are therefore highly influential in tokamak plasma dynamics. Purely fluid models are inherently incapable of capturing these effects, whereas the high dimensionality in purely kinetic models render them practically intractable for most relevant purposes.

        We consider a $\delta\!f$ decomposition model, with a macroscopic fluid background and microscopic kinetic correction, both fully coupled to each other. A similar manner of discretization is proposed to that used in the recent \texttt{STRUPHY} code \cite{Holderied_Possanner_Wang_2021, Holderied_2022, Li_et_al_2023} with a finite-element model for the background and a pseudo-particle/PiC model for the correction.

        The fluid background satisfies the full, non-linear, resistive, compressible, Hall MHD equations. \cite{Laakmann_Hu_Farrell_2022} introduces finite-element(-in-space) implicit timesteppers for the incompressible analogue to this system with structure-preserving (SP) properties in the ideal case, alongside parameter-robust preconditioners. We show that these timesteppers can derive from a finite-element-in-time (FET) (and finite-element-in-space) interpretation. The benefits of this reformulation are discussed, including the derivation of timesteppers that are higher order in time, and the quantifiable dissipative SP properties in the non-ideal, resistive case.
        
        We discuss possible options for extending this FET approach to timesteppers for the compressible case.

        The kinetic corrections satisfy linearized Boltzmann equations. Using a Lénard--Bernstein collision operator, these take Fokker--Planck-like forms \cite{Fokker_1914, Planck_1917} wherein pseudo-particles in the numerical model obey the neoclassical transport equations, with particle-independent Brownian drift terms. This offers a rigorous methodology for incorporating collisions into the particle transport model, without coupling the equations of motions for each particle.
        
        Works by Chen, Chacón et al. \cite{Chen_Chacón_Barnes_2011, Chacón_Chen_Barnes_2013, Chen_Chacón_2014, Chen_Chacón_2015} have developed structure-preserving particle pushers for neoclassical transport in the Vlasov equations, derived from Crank--Nicolson integrators. We show these too can can derive from a FET interpretation, similarly offering potential extensions to higher-order-in-time particle pushers. The FET formulation is used also to consider how the stochastic drift terms can be incorporated into the pushers. Stochastic gyrokinetic expansions are also discussed.

        Different options for the numerical implementation of these schemes are considered.

        Due to the efficacy of FET in the development of SP timesteppers for both the fluid and kinetic component, we hope this approach will prove effective in the future for developing SP timesteppers for the full hybrid model. We hope this will give us the opportunity to incorporate previously inaccessible kinetic effects into the highly effective, modern, finite-element MHD models.
    \end{abstract}
    
    
    \newpage
    \tableofcontents
    
    
    \newpage
    \pagenumbering{arabic}
    %\linenumbers\renewcommand\thelinenumber{\color{black!50}\arabic{linenumber}}
            \input{0 - introduction/main.tex}
        \part{Research}
            \input{1 - low-noise PiC models/main.tex}
            \input{2 - kinetic component/main.tex}
            \input{3 - fluid component/main.tex}
            \input{4 - numerical implementation/main.tex}
        \part{Project Overview}
            \input{5 - research plan/main.tex}
            \input{6 - summary/main.tex}
    
    
    %\section{}
    \newpage
    \pagenumbering{gobble}
        \printbibliography


    \newpage
    \pagenumbering{roman}
    \appendix
        \part{Appendices}
            \input{8 - Hilbert complexes/main.tex}
            \input{9 - weak conservation proofs/main.tex}
\end{document}

    
    
    %\section{}
    \newpage
    \pagenumbering{gobble}
        \printbibliography


    \newpage
    \pagenumbering{roman}
    \appendix
        \part{Appendices}
            \documentclass[12pt, a4paper]{report}

\input{template/main.tex}

\title{\BA{Title in Progress...}}
\author{Boris Andrews}
\affil{Mathematical Institute, University of Oxford}
\date{\today}


\begin{document}
    \pagenumbering{gobble}
    \maketitle
    
    
    \begin{abstract}
        Magnetic confinement reactors---in particular tokamaks---offer one of the most promising options for achieving practical nuclear fusion, with the potential to provide virtually limitless, clean energy. The theoretical and numerical modeling of tokamak plasmas is simultaneously an essential component of effective reactor design, and a great research barrier. Tokamak operational conditions exhibit comparatively low Knudsen numbers. Kinetic effects, including kinetic waves and instabilities, Landau damping, bump-on-tail instabilities and more, are therefore highly influential in tokamak plasma dynamics. Purely fluid models are inherently incapable of capturing these effects, whereas the high dimensionality in purely kinetic models render them practically intractable for most relevant purposes.

        We consider a $\delta\!f$ decomposition model, with a macroscopic fluid background and microscopic kinetic correction, both fully coupled to each other. A similar manner of discretization is proposed to that used in the recent \texttt{STRUPHY} code \cite{Holderied_Possanner_Wang_2021, Holderied_2022, Li_et_al_2023} with a finite-element model for the background and a pseudo-particle/PiC model for the correction.

        The fluid background satisfies the full, non-linear, resistive, compressible, Hall MHD equations. \cite{Laakmann_Hu_Farrell_2022} introduces finite-element(-in-space) implicit timesteppers for the incompressible analogue to this system with structure-preserving (SP) properties in the ideal case, alongside parameter-robust preconditioners. We show that these timesteppers can derive from a finite-element-in-time (FET) (and finite-element-in-space) interpretation. The benefits of this reformulation are discussed, including the derivation of timesteppers that are higher order in time, and the quantifiable dissipative SP properties in the non-ideal, resistive case.
        
        We discuss possible options for extending this FET approach to timesteppers for the compressible case.

        The kinetic corrections satisfy linearized Boltzmann equations. Using a Lénard--Bernstein collision operator, these take Fokker--Planck-like forms \cite{Fokker_1914, Planck_1917} wherein pseudo-particles in the numerical model obey the neoclassical transport equations, with particle-independent Brownian drift terms. This offers a rigorous methodology for incorporating collisions into the particle transport model, without coupling the equations of motions for each particle.
        
        Works by Chen, Chacón et al. \cite{Chen_Chacón_Barnes_2011, Chacón_Chen_Barnes_2013, Chen_Chacón_2014, Chen_Chacón_2015} have developed structure-preserving particle pushers for neoclassical transport in the Vlasov equations, derived from Crank--Nicolson integrators. We show these too can can derive from a FET interpretation, similarly offering potential extensions to higher-order-in-time particle pushers. The FET formulation is used also to consider how the stochastic drift terms can be incorporated into the pushers. Stochastic gyrokinetic expansions are also discussed.

        Different options for the numerical implementation of these schemes are considered.

        Due to the efficacy of FET in the development of SP timesteppers for both the fluid and kinetic component, we hope this approach will prove effective in the future for developing SP timesteppers for the full hybrid model. We hope this will give us the opportunity to incorporate previously inaccessible kinetic effects into the highly effective, modern, finite-element MHD models.
    \end{abstract}
    
    
    \newpage
    \tableofcontents
    
    
    \newpage
    \pagenumbering{arabic}
    %\linenumbers\renewcommand\thelinenumber{\color{black!50}\arabic{linenumber}}
            \input{0 - introduction/main.tex}
        \part{Research}
            \input{1 - low-noise PiC models/main.tex}
            \input{2 - kinetic component/main.tex}
            \input{3 - fluid component/main.tex}
            \input{4 - numerical implementation/main.tex}
        \part{Project Overview}
            \input{5 - research plan/main.tex}
            \input{6 - summary/main.tex}
    
    
    %\section{}
    \newpage
    \pagenumbering{gobble}
        \printbibliography


    \newpage
    \pagenumbering{roman}
    \appendix
        \part{Appendices}
            \input{8 - Hilbert complexes/main.tex}
            \input{9 - weak conservation proofs/main.tex}
\end{document}

            \documentclass[12pt, a4paper]{report}

\input{template/main.tex}

\title{\BA{Title in Progress...}}
\author{Boris Andrews}
\affil{Mathematical Institute, University of Oxford}
\date{\today}


\begin{document}
    \pagenumbering{gobble}
    \maketitle
    
    
    \begin{abstract}
        Magnetic confinement reactors---in particular tokamaks---offer one of the most promising options for achieving practical nuclear fusion, with the potential to provide virtually limitless, clean energy. The theoretical and numerical modeling of tokamak plasmas is simultaneously an essential component of effective reactor design, and a great research barrier. Tokamak operational conditions exhibit comparatively low Knudsen numbers. Kinetic effects, including kinetic waves and instabilities, Landau damping, bump-on-tail instabilities and more, are therefore highly influential in tokamak plasma dynamics. Purely fluid models are inherently incapable of capturing these effects, whereas the high dimensionality in purely kinetic models render them practically intractable for most relevant purposes.

        We consider a $\delta\!f$ decomposition model, with a macroscopic fluid background and microscopic kinetic correction, both fully coupled to each other. A similar manner of discretization is proposed to that used in the recent \texttt{STRUPHY} code \cite{Holderied_Possanner_Wang_2021, Holderied_2022, Li_et_al_2023} with a finite-element model for the background and a pseudo-particle/PiC model for the correction.

        The fluid background satisfies the full, non-linear, resistive, compressible, Hall MHD equations. \cite{Laakmann_Hu_Farrell_2022} introduces finite-element(-in-space) implicit timesteppers for the incompressible analogue to this system with structure-preserving (SP) properties in the ideal case, alongside parameter-robust preconditioners. We show that these timesteppers can derive from a finite-element-in-time (FET) (and finite-element-in-space) interpretation. The benefits of this reformulation are discussed, including the derivation of timesteppers that are higher order in time, and the quantifiable dissipative SP properties in the non-ideal, resistive case.
        
        We discuss possible options for extending this FET approach to timesteppers for the compressible case.

        The kinetic corrections satisfy linearized Boltzmann equations. Using a Lénard--Bernstein collision operator, these take Fokker--Planck-like forms \cite{Fokker_1914, Planck_1917} wherein pseudo-particles in the numerical model obey the neoclassical transport equations, with particle-independent Brownian drift terms. This offers a rigorous methodology for incorporating collisions into the particle transport model, without coupling the equations of motions for each particle.
        
        Works by Chen, Chacón et al. \cite{Chen_Chacón_Barnes_2011, Chacón_Chen_Barnes_2013, Chen_Chacón_2014, Chen_Chacón_2015} have developed structure-preserving particle pushers for neoclassical transport in the Vlasov equations, derived from Crank--Nicolson integrators. We show these too can can derive from a FET interpretation, similarly offering potential extensions to higher-order-in-time particle pushers. The FET formulation is used also to consider how the stochastic drift terms can be incorporated into the pushers. Stochastic gyrokinetic expansions are also discussed.

        Different options for the numerical implementation of these schemes are considered.

        Due to the efficacy of FET in the development of SP timesteppers for both the fluid and kinetic component, we hope this approach will prove effective in the future for developing SP timesteppers for the full hybrid model. We hope this will give us the opportunity to incorporate previously inaccessible kinetic effects into the highly effective, modern, finite-element MHD models.
    \end{abstract}
    
    
    \newpage
    \tableofcontents
    
    
    \newpage
    \pagenumbering{arabic}
    %\linenumbers\renewcommand\thelinenumber{\color{black!50}\arabic{linenumber}}
            \input{0 - introduction/main.tex}
        \part{Research}
            \input{1 - low-noise PiC models/main.tex}
            \input{2 - kinetic component/main.tex}
            \input{3 - fluid component/main.tex}
            \input{4 - numerical implementation/main.tex}
        \part{Project Overview}
            \input{5 - research plan/main.tex}
            \input{6 - summary/main.tex}
    
    
    %\section{}
    \newpage
    \pagenumbering{gobble}
        \printbibliography


    \newpage
    \pagenumbering{roman}
    \appendix
        \part{Appendices}
            \input{8 - Hilbert complexes/main.tex}
            \input{9 - weak conservation proofs/main.tex}
\end{document}

\end{document}

    
    
    %\section{}
    \newpage
    \pagenumbering{gobble}
        \printbibliography


    \newpage
    \pagenumbering{roman}
    \appendix
        \part{Appendices}
            \documentclass[12pt, a4paper]{report}

\documentclass[12pt, a4paper]{report}

\input{template/main.tex}

\title{\BA{Title in Progress...}}
\author{Boris Andrews}
\affil{Mathematical Institute, University of Oxford}
\date{\today}


\begin{document}
    \pagenumbering{gobble}
    \maketitle
    
    
    \begin{abstract}
        Magnetic confinement reactors---in particular tokamaks---offer one of the most promising options for achieving practical nuclear fusion, with the potential to provide virtually limitless, clean energy. The theoretical and numerical modeling of tokamak plasmas is simultaneously an essential component of effective reactor design, and a great research barrier. Tokamak operational conditions exhibit comparatively low Knudsen numbers. Kinetic effects, including kinetic waves and instabilities, Landau damping, bump-on-tail instabilities and more, are therefore highly influential in tokamak plasma dynamics. Purely fluid models are inherently incapable of capturing these effects, whereas the high dimensionality in purely kinetic models render them practically intractable for most relevant purposes.

        We consider a $\delta\!f$ decomposition model, with a macroscopic fluid background and microscopic kinetic correction, both fully coupled to each other. A similar manner of discretization is proposed to that used in the recent \texttt{STRUPHY} code \cite{Holderied_Possanner_Wang_2021, Holderied_2022, Li_et_al_2023} with a finite-element model for the background and a pseudo-particle/PiC model for the correction.

        The fluid background satisfies the full, non-linear, resistive, compressible, Hall MHD equations. \cite{Laakmann_Hu_Farrell_2022} introduces finite-element(-in-space) implicit timesteppers for the incompressible analogue to this system with structure-preserving (SP) properties in the ideal case, alongside parameter-robust preconditioners. We show that these timesteppers can derive from a finite-element-in-time (FET) (and finite-element-in-space) interpretation. The benefits of this reformulation are discussed, including the derivation of timesteppers that are higher order in time, and the quantifiable dissipative SP properties in the non-ideal, resistive case.
        
        We discuss possible options for extending this FET approach to timesteppers for the compressible case.

        The kinetic corrections satisfy linearized Boltzmann equations. Using a Lénard--Bernstein collision operator, these take Fokker--Planck-like forms \cite{Fokker_1914, Planck_1917} wherein pseudo-particles in the numerical model obey the neoclassical transport equations, with particle-independent Brownian drift terms. This offers a rigorous methodology for incorporating collisions into the particle transport model, without coupling the equations of motions for each particle.
        
        Works by Chen, Chacón et al. \cite{Chen_Chacón_Barnes_2011, Chacón_Chen_Barnes_2013, Chen_Chacón_2014, Chen_Chacón_2015} have developed structure-preserving particle pushers for neoclassical transport in the Vlasov equations, derived from Crank--Nicolson integrators. We show these too can can derive from a FET interpretation, similarly offering potential extensions to higher-order-in-time particle pushers. The FET formulation is used also to consider how the stochastic drift terms can be incorporated into the pushers. Stochastic gyrokinetic expansions are also discussed.

        Different options for the numerical implementation of these schemes are considered.

        Due to the efficacy of FET in the development of SP timesteppers for both the fluid and kinetic component, we hope this approach will prove effective in the future for developing SP timesteppers for the full hybrid model. We hope this will give us the opportunity to incorporate previously inaccessible kinetic effects into the highly effective, modern, finite-element MHD models.
    \end{abstract}
    
    
    \newpage
    \tableofcontents
    
    
    \newpage
    \pagenumbering{arabic}
    %\linenumbers\renewcommand\thelinenumber{\color{black!50}\arabic{linenumber}}
            \input{0 - introduction/main.tex}
        \part{Research}
            \input{1 - low-noise PiC models/main.tex}
            \input{2 - kinetic component/main.tex}
            \input{3 - fluid component/main.tex}
            \input{4 - numerical implementation/main.tex}
        \part{Project Overview}
            \input{5 - research plan/main.tex}
            \input{6 - summary/main.tex}
    
    
    %\section{}
    \newpage
    \pagenumbering{gobble}
        \printbibliography


    \newpage
    \pagenumbering{roman}
    \appendix
        \part{Appendices}
            \input{8 - Hilbert complexes/main.tex}
            \input{9 - weak conservation proofs/main.tex}
\end{document}


\title{\BA{Title in Progress...}}
\author{Boris Andrews}
\affil{Mathematical Institute, University of Oxford}
\date{\today}


\begin{document}
    \pagenumbering{gobble}
    \maketitle
    
    
    \begin{abstract}
        Magnetic confinement reactors---in particular tokamaks---offer one of the most promising options for achieving practical nuclear fusion, with the potential to provide virtually limitless, clean energy. The theoretical and numerical modeling of tokamak plasmas is simultaneously an essential component of effective reactor design, and a great research barrier. Tokamak operational conditions exhibit comparatively low Knudsen numbers. Kinetic effects, including kinetic waves and instabilities, Landau damping, bump-on-tail instabilities and more, are therefore highly influential in tokamak plasma dynamics. Purely fluid models are inherently incapable of capturing these effects, whereas the high dimensionality in purely kinetic models render them practically intractable for most relevant purposes.

        We consider a $\delta\!f$ decomposition model, with a macroscopic fluid background and microscopic kinetic correction, both fully coupled to each other. A similar manner of discretization is proposed to that used in the recent \texttt{STRUPHY} code \cite{Holderied_Possanner_Wang_2021, Holderied_2022, Li_et_al_2023} with a finite-element model for the background and a pseudo-particle/PiC model for the correction.

        The fluid background satisfies the full, non-linear, resistive, compressible, Hall MHD equations. \cite{Laakmann_Hu_Farrell_2022} introduces finite-element(-in-space) implicit timesteppers for the incompressible analogue to this system with structure-preserving (SP) properties in the ideal case, alongside parameter-robust preconditioners. We show that these timesteppers can derive from a finite-element-in-time (FET) (and finite-element-in-space) interpretation. The benefits of this reformulation are discussed, including the derivation of timesteppers that are higher order in time, and the quantifiable dissipative SP properties in the non-ideal, resistive case.
        
        We discuss possible options for extending this FET approach to timesteppers for the compressible case.

        The kinetic corrections satisfy linearized Boltzmann equations. Using a Lénard--Bernstein collision operator, these take Fokker--Planck-like forms \cite{Fokker_1914, Planck_1917} wherein pseudo-particles in the numerical model obey the neoclassical transport equations, with particle-independent Brownian drift terms. This offers a rigorous methodology for incorporating collisions into the particle transport model, without coupling the equations of motions for each particle.
        
        Works by Chen, Chacón et al. \cite{Chen_Chacón_Barnes_2011, Chacón_Chen_Barnes_2013, Chen_Chacón_2014, Chen_Chacón_2015} have developed structure-preserving particle pushers for neoclassical transport in the Vlasov equations, derived from Crank--Nicolson integrators. We show these too can can derive from a FET interpretation, similarly offering potential extensions to higher-order-in-time particle pushers. The FET formulation is used also to consider how the stochastic drift terms can be incorporated into the pushers. Stochastic gyrokinetic expansions are also discussed.

        Different options for the numerical implementation of these schemes are considered.

        Due to the efficacy of FET in the development of SP timesteppers for both the fluid and kinetic component, we hope this approach will prove effective in the future for developing SP timesteppers for the full hybrid model. We hope this will give us the opportunity to incorporate previously inaccessible kinetic effects into the highly effective, modern, finite-element MHD models.
    \end{abstract}
    
    
    \newpage
    \tableofcontents
    
    
    \newpage
    \pagenumbering{arabic}
    %\linenumbers\renewcommand\thelinenumber{\color{black!50}\arabic{linenumber}}
            \documentclass[12pt, a4paper]{report}

\input{template/main.tex}

\title{\BA{Title in Progress...}}
\author{Boris Andrews}
\affil{Mathematical Institute, University of Oxford}
\date{\today}


\begin{document}
    \pagenumbering{gobble}
    \maketitle
    
    
    \begin{abstract}
        Magnetic confinement reactors---in particular tokamaks---offer one of the most promising options for achieving practical nuclear fusion, with the potential to provide virtually limitless, clean energy. The theoretical and numerical modeling of tokamak plasmas is simultaneously an essential component of effective reactor design, and a great research barrier. Tokamak operational conditions exhibit comparatively low Knudsen numbers. Kinetic effects, including kinetic waves and instabilities, Landau damping, bump-on-tail instabilities and more, are therefore highly influential in tokamak plasma dynamics. Purely fluid models are inherently incapable of capturing these effects, whereas the high dimensionality in purely kinetic models render them practically intractable for most relevant purposes.

        We consider a $\delta\!f$ decomposition model, with a macroscopic fluid background and microscopic kinetic correction, both fully coupled to each other. A similar manner of discretization is proposed to that used in the recent \texttt{STRUPHY} code \cite{Holderied_Possanner_Wang_2021, Holderied_2022, Li_et_al_2023} with a finite-element model for the background and a pseudo-particle/PiC model for the correction.

        The fluid background satisfies the full, non-linear, resistive, compressible, Hall MHD equations. \cite{Laakmann_Hu_Farrell_2022} introduces finite-element(-in-space) implicit timesteppers for the incompressible analogue to this system with structure-preserving (SP) properties in the ideal case, alongside parameter-robust preconditioners. We show that these timesteppers can derive from a finite-element-in-time (FET) (and finite-element-in-space) interpretation. The benefits of this reformulation are discussed, including the derivation of timesteppers that are higher order in time, and the quantifiable dissipative SP properties in the non-ideal, resistive case.
        
        We discuss possible options for extending this FET approach to timesteppers for the compressible case.

        The kinetic corrections satisfy linearized Boltzmann equations. Using a Lénard--Bernstein collision operator, these take Fokker--Planck-like forms \cite{Fokker_1914, Planck_1917} wherein pseudo-particles in the numerical model obey the neoclassical transport equations, with particle-independent Brownian drift terms. This offers a rigorous methodology for incorporating collisions into the particle transport model, without coupling the equations of motions for each particle.
        
        Works by Chen, Chacón et al. \cite{Chen_Chacón_Barnes_2011, Chacón_Chen_Barnes_2013, Chen_Chacón_2014, Chen_Chacón_2015} have developed structure-preserving particle pushers for neoclassical transport in the Vlasov equations, derived from Crank--Nicolson integrators. We show these too can can derive from a FET interpretation, similarly offering potential extensions to higher-order-in-time particle pushers. The FET formulation is used also to consider how the stochastic drift terms can be incorporated into the pushers. Stochastic gyrokinetic expansions are also discussed.

        Different options for the numerical implementation of these schemes are considered.

        Due to the efficacy of FET in the development of SP timesteppers for both the fluid and kinetic component, we hope this approach will prove effective in the future for developing SP timesteppers for the full hybrid model. We hope this will give us the opportunity to incorporate previously inaccessible kinetic effects into the highly effective, modern, finite-element MHD models.
    \end{abstract}
    
    
    \newpage
    \tableofcontents
    
    
    \newpage
    \pagenumbering{arabic}
    %\linenumbers\renewcommand\thelinenumber{\color{black!50}\arabic{linenumber}}
            \input{0 - introduction/main.tex}
        \part{Research}
            \input{1 - low-noise PiC models/main.tex}
            \input{2 - kinetic component/main.tex}
            \input{3 - fluid component/main.tex}
            \input{4 - numerical implementation/main.tex}
        \part{Project Overview}
            \input{5 - research plan/main.tex}
            \input{6 - summary/main.tex}
    
    
    %\section{}
    \newpage
    \pagenumbering{gobble}
        \printbibliography


    \newpage
    \pagenumbering{roman}
    \appendix
        \part{Appendices}
            \input{8 - Hilbert complexes/main.tex}
            \input{9 - weak conservation proofs/main.tex}
\end{document}

        \part{Research}
            \documentclass[12pt, a4paper]{report}

\input{template/main.tex}

\title{\BA{Title in Progress...}}
\author{Boris Andrews}
\affil{Mathematical Institute, University of Oxford}
\date{\today}


\begin{document}
    \pagenumbering{gobble}
    \maketitle
    
    
    \begin{abstract}
        Magnetic confinement reactors---in particular tokamaks---offer one of the most promising options for achieving practical nuclear fusion, with the potential to provide virtually limitless, clean energy. The theoretical and numerical modeling of tokamak plasmas is simultaneously an essential component of effective reactor design, and a great research barrier. Tokamak operational conditions exhibit comparatively low Knudsen numbers. Kinetic effects, including kinetic waves and instabilities, Landau damping, bump-on-tail instabilities and more, are therefore highly influential in tokamak plasma dynamics. Purely fluid models are inherently incapable of capturing these effects, whereas the high dimensionality in purely kinetic models render them practically intractable for most relevant purposes.

        We consider a $\delta\!f$ decomposition model, with a macroscopic fluid background and microscopic kinetic correction, both fully coupled to each other. A similar manner of discretization is proposed to that used in the recent \texttt{STRUPHY} code \cite{Holderied_Possanner_Wang_2021, Holderied_2022, Li_et_al_2023} with a finite-element model for the background and a pseudo-particle/PiC model for the correction.

        The fluid background satisfies the full, non-linear, resistive, compressible, Hall MHD equations. \cite{Laakmann_Hu_Farrell_2022} introduces finite-element(-in-space) implicit timesteppers for the incompressible analogue to this system with structure-preserving (SP) properties in the ideal case, alongside parameter-robust preconditioners. We show that these timesteppers can derive from a finite-element-in-time (FET) (and finite-element-in-space) interpretation. The benefits of this reformulation are discussed, including the derivation of timesteppers that are higher order in time, and the quantifiable dissipative SP properties in the non-ideal, resistive case.
        
        We discuss possible options for extending this FET approach to timesteppers for the compressible case.

        The kinetic corrections satisfy linearized Boltzmann equations. Using a Lénard--Bernstein collision operator, these take Fokker--Planck-like forms \cite{Fokker_1914, Planck_1917} wherein pseudo-particles in the numerical model obey the neoclassical transport equations, with particle-independent Brownian drift terms. This offers a rigorous methodology for incorporating collisions into the particle transport model, without coupling the equations of motions for each particle.
        
        Works by Chen, Chacón et al. \cite{Chen_Chacón_Barnes_2011, Chacón_Chen_Barnes_2013, Chen_Chacón_2014, Chen_Chacón_2015} have developed structure-preserving particle pushers for neoclassical transport in the Vlasov equations, derived from Crank--Nicolson integrators. We show these too can can derive from a FET interpretation, similarly offering potential extensions to higher-order-in-time particle pushers. The FET formulation is used also to consider how the stochastic drift terms can be incorporated into the pushers. Stochastic gyrokinetic expansions are also discussed.

        Different options for the numerical implementation of these schemes are considered.

        Due to the efficacy of FET in the development of SP timesteppers for both the fluid and kinetic component, we hope this approach will prove effective in the future for developing SP timesteppers for the full hybrid model. We hope this will give us the opportunity to incorporate previously inaccessible kinetic effects into the highly effective, modern, finite-element MHD models.
    \end{abstract}
    
    
    \newpage
    \tableofcontents
    
    
    \newpage
    \pagenumbering{arabic}
    %\linenumbers\renewcommand\thelinenumber{\color{black!50}\arabic{linenumber}}
            \input{0 - introduction/main.tex}
        \part{Research}
            \input{1 - low-noise PiC models/main.tex}
            \input{2 - kinetic component/main.tex}
            \input{3 - fluid component/main.tex}
            \input{4 - numerical implementation/main.tex}
        \part{Project Overview}
            \input{5 - research plan/main.tex}
            \input{6 - summary/main.tex}
    
    
    %\section{}
    \newpage
    \pagenumbering{gobble}
        \printbibliography


    \newpage
    \pagenumbering{roman}
    \appendix
        \part{Appendices}
            \input{8 - Hilbert complexes/main.tex}
            \input{9 - weak conservation proofs/main.tex}
\end{document}

            \documentclass[12pt, a4paper]{report}

\input{template/main.tex}

\title{\BA{Title in Progress...}}
\author{Boris Andrews}
\affil{Mathematical Institute, University of Oxford}
\date{\today}


\begin{document}
    \pagenumbering{gobble}
    \maketitle
    
    
    \begin{abstract}
        Magnetic confinement reactors---in particular tokamaks---offer one of the most promising options for achieving practical nuclear fusion, with the potential to provide virtually limitless, clean energy. The theoretical and numerical modeling of tokamak plasmas is simultaneously an essential component of effective reactor design, and a great research barrier. Tokamak operational conditions exhibit comparatively low Knudsen numbers. Kinetic effects, including kinetic waves and instabilities, Landau damping, bump-on-tail instabilities and more, are therefore highly influential in tokamak plasma dynamics. Purely fluid models are inherently incapable of capturing these effects, whereas the high dimensionality in purely kinetic models render them practically intractable for most relevant purposes.

        We consider a $\delta\!f$ decomposition model, with a macroscopic fluid background and microscopic kinetic correction, both fully coupled to each other. A similar manner of discretization is proposed to that used in the recent \texttt{STRUPHY} code \cite{Holderied_Possanner_Wang_2021, Holderied_2022, Li_et_al_2023} with a finite-element model for the background and a pseudo-particle/PiC model for the correction.

        The fluid background satisfies the full, non-linear, resistive, compressible, Hall MHD equations. \cite{Laakmann_Hu_Farrell_2022} introduces finite-element(-in-space) implicit timesteppers for the incompressible analogue to this system with structure-preserving (SP) properties in the ideal case, alongside parameter-robust preconditioners. We show that these timesteppers can derive from a finite-element-in-time (FET) (and finite-element-in-space) interpretation. The benefits of this reformulation are discussed, including the derivation of timesteppers that are higher order in time, and the quantifiable dissipative SP properties in the non-ideal, resistive case.
        
        We discuss possible options for extending this FET approach to timesteppers for the compressible case.

        The kinetic corrections satisfy linearized Boltzmann equations. Using a Lénard--Bernstein collision operator, these take Fokker--Planck-like forms \cite{Fokker_1914, Planck_1917} wherein pseudo-particles in the numerical model obey the neoclassical transport equations, with particle-independent Brownian drift terms. This offers a rigorous methodology for incorporating collisions into the particle transport model, without coupling the equations of motions for each particle.
        
        Works by Chen, Chacón et al. \cite{Chen_Chacón_Barnes_2011, Chacón_Chen_Barnes_2013, Chen_Chacón_2014, Chen_Chacón_2015} have developed structure-preserving particle pushers for neoclassical transport in the Vlasov equations, derived from Crank--Nicolson integrators. We show these too can can derive from a FET interpretation, similarly offering potential extensions to higher-order-in-time particle pushers. The FET formulation is used also to consider how the stochastic drift terms can be incorporated into the pushers. Stochastic gyrokinetic expansions are also discussed.

        Different options for the numerical implementation of these schemes are considered.

        Due to the efficacy of FET in the development of SP timesteppers for both the fluid and kinetic component, we hope this approach will prove effective in the future for developing SP timesteppers for the full hybrid model. We hope this will give us the opportunity to incorporate previously inaccessible kinetic effects into the highly effective, modern, finite-element MHD models.
    \end{abstract}
    
    
    \newpage
    \tableofcontents
    
    
    \newpage
    \pagenumbering{arabic}
    %\linenumbers\renewcommand\thelinenumber{\color{black!50}\arabic{linenumber}}
            \input{0 - introduction/main.tex}
        \part{Research}
            \input{1 - low-noise PiC models/main.tex}
            \input{2 - kinetic component/main.tex}
            \input{3 - fluid component/main.tex}
            \input{4 - numerical implementation/main.tex}
        \part{Project Overview}
            \input{5 - research plan/main.tex}
            \input{6 - summary/main.tex}
    
    
    %\section{}
    \newpage
    \pagenumbering{gobble}
        \printbibliography


    \newpage
    \pagenumbering{roman}
    \appendix
        \part{Appendices}
            \input{8 - Hilbert complexes/main.tex}
            \input{9 - weak conservation proofs/main.tex}
\end{document}

            \documentclass[12pt, a4paper]{report}

\input{template/main.tex}

\title{\BA{Title in Progress...}}
\author{Boris Andrews}
\affil{Mathematical Institute, University of Oxford}
\date{\today}


\begin{document}
    \pagenumbering{gobble}
    \maketitle
    
    
    \begin{abstract}
        Magnetic confinement reactors---in particular tokamaks---offer one of the most promising options for achieving practical nuclear fusion, with the potential to provide virtually limitless, clean energy. The theoretical and numerical modeling of tokamak plasmas is simultaneously an essential component of effective reactor design, and a great research barrier. Tokamak operational conditions exhibit comparatively low Knudsen numbers. Kinetic effects, including kinetic waves and instabilities, Landau damping, bump-on-tail instabilities and more, are therefore highly influential in tokamak plasma dynamics. Purely fluid models are inherently incapable of capturing these effects, whereas the high dimensionality in purely kinetic models render them practically intractable for most relevant purposes.

        We consider a $\delta\!f$ decomposition model, with a macroscopic fluid background and microscopic kinetic correction, both fully coupled to each other. A similar manner of discretization is proposed to that used in the recent \texttt{STRUPHY} code \cite{Holderied_Possanner_Wang_2021, Holderied_2022, Li_et_al_2023} with a finite-element model for the background and a pseudo-particle/PiC model for the correction.

        The fluid background satisfies the full, non-linear, resistive, compressible, Hall MHD equations. \cite{Laakmann_Hu_Farrell_2022} introduces finite-element(-in-space) implicit timesteppers for the incompressible analogue to this system with structure-preserving (SP) properties in the ideal case, alongside parameter-robust preconditioners. We show that these timesteppers can derive from a finite-element-in-time (FET) (and finite-element-in-space) interpretation. The benefits of this reformulation are discussed, including the derivation of timesteppers that are higher order in time, and the quantifiable dissipative SP properties in the non-ideal, resistive case.
        
        We discuss possible options for extending this FET approach to timesteppers for the compressible case.

        The kinetic corrections satisfy linearized Boltzmann equations. Using a Lénard--Bernstein collision operator, these take Fokker--Planck-like forms \cite{Fokker_1914, Planck_1917} wherein pseudo-particles in the numerical model obey the neoclassical transport equations, with particle-independent Brownian drift terms. This offers a rigorous methodology for incorporating collisions into the particle transport model, without coupling the equations of motions for each particle.
        
        Works by Chen, Chacón et al. \cite{Chen_Chacón_Barnes_2011, Chacón_Chen_Barnes_2013, Chen_Chacón_2014, Chen_Chacón_2015} have developed structure-preserving particle pushers for neoclassical transport in the Vlasov equations, derived from Crank--Nicolson integrators. We show these too can can derive from a FET interpretation, similarly offering potential extensions to higher-order-in-time particle pushers. The FET formulation is used also to consider how the stochastic drift terms can be incorporated into the pushers. Stochastic gyrokinetic expansions are also discussed.

        Different options for the numerical implementation of these schemes are considered.

        Due to the efficacy of FET in the development of SP timesteppers for both the fluid and kinetic component, we hope this approach will prove effective in the future for developing SP timesteppers for the full hybrid model. We hope this will give us the opportunity to incorporate previously inaccessible kinetic effects into the highly effective, modern, finite-element MHD models.
    \end{abstract}
    
    
    \newpage
    \tableofcontents
    
    
    \newpage
    \pagenumbering{arabic}
    %\linenumbers\renewcommand\thelinenumber{\color{black!50}\arabic{linenumber}}
            \input{0 - introduction/main.tex}
        \part{Research}
            \input{1 - low-noise PiC models/main.tex}
            \input{2 - kinetic component/main.tex}
            \input{3 - fluid component/main.tex}
            \input{4 - numerical implementation/main.tex}
        \part{Project Overview}
            \input{5 - research plan/main.tex}
            \input{6 - summary/main.tex}
    
    
    %\section{}
    \newpage
    \pagenumbering{gobble}
        \printbibliography


    \newpage
    \pagenumbering{roman}
    \appendix
        \part{Appendices}
            \input{8 - Hilbert complexes/main.tex}
            \input{9 - weak conservation proofs/main.tex}
\end{document}

            \documentclass[12pt, a4paper]{report}

\input{template/main.tex}

\title{\BA{Title in Progress...}}
\author{Boris Andrews}
\affil{Mathematical Institute, University of Oxford}
\date{\today}


\begin{document}
    \pagenumbering{gobble}
    \maketitle
    
    
    \begin{abstract}
        Magnetic confinement reactors---in particular tokamaks---offer one of the most promising options for achieving practical nuclear fusion, with the potential to provide virtually limitless, clean energy. The theoretical and numerical modeling of tokamak plasmas is simultaneously an essential component of effective reactor design, and a great research barrier. Tokamak operational conditions exhibit comparatively low Knudsen numbers. Kinetic effects, including kinetic waves and instabilities, Landau damping, bump-on-tail instabilities and more, are therefore highly influential in tokamak plasma dynamics. Purely fluid models are inherently incapable of capturing these effects, whereas the high dimensionality in purely kinetic models render them practically intractable for most relevant purposes.

        We consider a $\delta\!f$ decomposition model, with a macroscopic fluid background and microscopic kinetic correction, both fully coupled to each other. A similar manner of discretization is proposed to that used in the recent \texttt{STRUPHY} code \cite{Holderied_Possanner_Wang_2021, Holderied_2022, Li_et_al_2023} with a finite-element model for the background and a pseudo-particle/PiC model for the correction.

        The fluid background satisfies the full, non-linear, resistive, compressible, Hall MHD equations. \cite{Laakmann_Hu_Farrell_2022} introduces finite-element(-in-space) implicit timesteppers for the incompressible analogue to this system with structure-preserving (SP) properties in the ideal case, alongside parameter-robust preconditioners. We show that these timesteppers can derive from a finite-element-in-time (FET) (and finite-element-in-space) interpretation. The benefits of this reformulation are discussed, including the derivation of timesteppers that are higher order in time, and the quantifiable dissipative SP properties in the non-ideal, resistive case.
        
        We discuss possible options for extending this FET approach to timesteppers for the compressible case.

        The kinetic corrections satisfy linearized Boltzmann equations. Using a Lénard--Bernstein collision operator, these take Fokker--Planck-like forms \cite{Fokker_1914, Planck_1917} wherein pseudo-particles in the numerical model obey the neoclassical transport equations, with particle-independent Brownian drift terms. This offers a rigorous methodology for incorporating collisions into the particle transport model, without coupling the equations of motions for each particle.
        
        Works by Chen, Chacón et al. \cite{Chen_Chacón_Barnes_2011, Chacón_Chen_Barnes_2013, Chen_Chacón_2014, Chen_Chacón_2015} have developed structure-preserving particle pushers for neoclassical transport in the Vlasov equations, derived from Crank--Nicolson integrators. We show these too can can derive from a FET interpretation, similarly offering potential extensions to higher-order-in-time particle pushers. The FET formulation is used also to consider how the stochastic drift terms can be incorporated into the pushers. Stochastic gyrokinetic expansions are also discussed.

        Different options for the numerical implementation of these schemes are considered.

        Due to the efficacy of FET in the development of SP timesteppers for both the fluid and kinetic component, we hope this approach will prove effective in the future for developing SP timesteppers for the full hybrid model. We hope this will give us the opportunity to incorporate previously inaccessible kinetic effects into the highly effective, modern, finite-element MHD models.
    \end{abstract}
    
    
    \newpage
    \tableofcontents
    
    
    \newpage
    \pagenumbering{arabic}
    %\linenumbers\renewcommand\thelinenumber{\color{black!50}\arabic{linenumber}}
            \input{0 - introduction/main.tex}
        \part{Research}
            \input{1 - low-noise PiC models/main.tex}
            \input{2 - kinetic component/main.tex}
            \input{3 - fluid component/main.tex}
            \input{4 - numerical implementation/main.tex}
        \part{Project Overview}
            \input{5 - research plan/main.tex}
            \input{6 - summary/main.tex}
    
    
    %\section{}
    \newpage
    \pagenumbering{gobble}
        \printbibliography


    \newpage
    \pagenumbering{roman}
    \appendix
        \part{Appendices}
            \input{8 - Hilbert complexes/main.tex}
            \input{9 - weak conservation proofs/main.tex}
\end{document}

        \part{Project Overview}
            \documentclass[12pt, a4paper]{report}

\input{template/main.tex}

\title{\BA{Title in Progress...}}
\author{Boris Andrews}
\affil{Mathematical Institute, University of Oxford}
\date{\today}


\begin{document}
    \pagenumbering{gobble}
    \maketitle
    
    
    \begin{abstract}
        Magnetic confinement reactors---in particular tokamaks---offer one of the most promising options for achieving practical nuclear fusion, with the potential to provide virtually limitless, clean energy. The theoretical and numerical modeling of tokamak plasmas is simultaneously an essential component of effective reactor design, and a great research barrier. Tokamak operational conditions exhibit comparatively low Knudsen numbers. Kinetic effects, including kinetic waves and instabilities, Landau damping, bump-on-tail instabilities and more, are therefore highly influential in tokamak plasma dynamics. Purely fluid models are inherently incapable of capturing these effects, whereas the high dimensionality in purely kinetic models render them practically intractable for most relevant purposes.

        We consider a $\delta\!f$ decomposition model, with a macroscopic fluid background and microscopic kinetic correction, both fully coupled to each other. A similar manner of discretization is proposed to that used in the recent \texttt{STRUPHY} code \cite{Holderied_Possanner_Wang_2021, Holderied_2022, Li_et_al_2023} with a finite-element model for the background and a pseudo-particle/PiC model for the correction.

        The fluid background satisfies the full, non-linear, resistive, compressible, Hall MHD equations. \cite{Laakmann_Hu_Farrell_2022} introduces finite-element(-in-space) implicit timesteppers for the incompressible analogue to this system with structure-preserving (SP) properties in the ideal case, alongside parameter-robust preconditioners. We show that these timesteppers can derive from a finite-element-in-time (FET) (and finite-element-in-space) interpretation. The benefits of this reformulation are discussed, including the derivation of timesteppers that are higher order in time, and the quantifiable dissipative SP properties in the non-ideal, resistive case.
        
        We discuss possible options for extending this FET approach to timesteppers for the compressible case.

        The kinetic corrections satisfy linearized Boltzmann equations. Using a Lénard--Bernstein collision operator, these take Fokker--Planck-like forms \cite{Fokker_1914, Planck_1917} wherein pseudo-particles in the numerical model obey the neoclassical transport equations, with particle-independent Brownian drift terms. This offers a rigorous methodology for incorporating collisions into the particle transport model, without coupling the equations of motions for each particle.
        
        Works by Chen, Chacón et al. \cite{Chen_Chacón_Barnes_2011, Chacón_Chen_Barnes_2013, Chen_Chacón_2014, Chen_Chacón_2015} have developed structure-preserving particle pushers for neoclassical transport in the Vlasov equations, derived from Crank--Nicolson integrators. We show these too can can derive from a FET interpretation, similarly offering potential extensions to higher-order-in-time particle pushers. The FET formulation is used also to consider how the stochastic drift terms can be incorporated into the pushers. Stochastic gyrokinetic expansions are also discussed.

        Different options for the numerical implementation of these schemes are considered.

        Due to the efficacy of FET in the development of SP timesteppers for both the fluid and kinetic component, we hope this approach will prove effective in the future for developing SP timesteppers for the full hybrid model. We hope this will give us the opportunity to incorporate previously inaccessible kinetic effects into the highly effective, modern, finite-element MHD models.
    \end{abstract}
    
    
    \newpage
    \tableofcontents
    
    
    \newpage
    \pagenumbering{arabic}
    %\linenumbers\renewcommand\thelinenumber{\color{black!50}\arabic{linenumber}}
            \input{0 - introduction/main.tex}
        \part{Research}
            \input{1 - low-noise PiC models/main.tex}
            \input{2 - kinetic component/main.tex}
            \input{3 - fluid component/main.tex}
            \input{4 - numerical implementation/main.tex}
        \part{Project Overview}
            \input{5 - research plan/main.tex}
            \input{6 - summary/main.tex}
    
    
    %\section{}
    \newpage
    \pagenumbering{gobble}
        \printbibliography


    \newpage
    \pagenumbering{roman}
    \appendix
        \part{Appendices}
            \input{8 - Hilbert complexes/main.tex}
            \input{9 - weak conservation proofs/main.tex}
\end{document}

            \documentclass[12pt, a4paper]{report}

\input{template/main.tex}

\title{\BA{Title in Progress...}}
\author{Boris Andrews}
\affil{Mathematical Institute, University of Oxford}
\date{\today}


\begin{document}
    \pagenumbering{gobble}
    \maketitle
    
    
    \begin{abstract}
        Magnetic confinement reactors---in particular tokamaks---offer one of the most promising options for achieving practical nuclear fusion, with the potential to provide virtually limitless, clean energy. The theoretical and numerical modeling of tokamak plasmas is simultaneously an essential component of effective reactor design, and a great research barrier. Tokamak operational conditions exhibit comparatively low Knudsen numbers. Kinetic effects, including kinetic waves and instabilities, Landau damping, bump-on-tail instabilities and more, are therefore highly influential in tokamak plasma dynamics. Purely fluid models are inherently incapable of capturing these effects, whereas the high dimensionality in purely kinetic models render them practically intractable for most relevant purposes.

        We consider a $\delta\!f$ decomposition model, with a macroscopic fluid background and microscopic kinetic correction, both fully coupled to each other. A similar manner of discretization is proposed to that used in the recent \texttt{STRUPHY} code \cite{Holderied_Possanner_Wang_2021, Holderied_2022, Li_et_al_2023} with a finite-element model for the background and a pseudo-particle/PiC model for the correction.

        The fluid background satisfies the full, non-linear, resistive, compressible, Hall MHD equations. \cite{Laakmann_Hu_Farrell_2022} introduces finite-element(-in-space) implicit timesteppers for the incompressible analogue to this system with structure-preserving (SP) properties in the ideal case, alongside parameter-robust preconditioners. We show that these timesteppers can derive from a finite-element-in-time (FET) (and finite-element-in-space) interpretation. The benefits of this reformulation are discussed, including the derivation of timesteppers that are higher order in time, and the quantifiable dissipative SP properties in the non-ideal, resistive case.
        
        We discuss possible options for extending this FET approach to timesteppers for the compressible case.

        The kinetic corrections satisfy linearized Boltzmann equations. Using a Lénard--Bernstein collision operator, these take Fokker--Planck-like forms \cite{Fokker_1914, Planck_1917} wherein pseudo-particles in the numerical model obey the neoclassical transport equations, with particle-independent Brownian drift terms. This offers a rigorous methodology for incorporating collisions into the particle transport model, without coupling the equations of motions for each particle.
        
        Works by Chen, Chacón et al. \cite{Chen_Chacón_Barnes_2011, Chacón_Chen_Barnes_2013, Chen_Chacón_2014, Chen_Chacón_2015} have developed structure-preserving particle pushers for neoclassical transport in the Vlasov equations, derived from Crank--Nicolson integrators. We show these too can can derive from a FET interpretation, similarly offering potential extensions to higher-order-in-time particle pushers. The FET formulation is used also to consider how the stochastic drift terms can be incorporated into the pushers. Stochastic gyrokinetic expansions are also discussed.

        Different options for the numerical implementation of these schemes are considered.

        Due to the efficacy of FET in the development of SP timesteppers for both the fluid and kinetic component, we hope this approach will prove effective in the future for developing SP timesteppers for the full hybrid model. We hope this will give us the opportunity to incorporate previously inaccessible kinetic effects into the highly effective, modern, finite-element MHD models.
    \end{abstract}
    
    
    \newpage
    \tableofcontents
    
    
    \newpage
    \pagenumbering{arabic}
    %\linenumbers\renewcommand\thelinenumber{\color{black!50}\arabic{linenumber}}
            \input{0 - introduction/main.tex}
        \part{Research}
            \input{1 - low-noise PiC models/main.tex}
            \input{2 - kinetic component/main.tex}
            \input{3 - fluid component/main.tex}
            \input{4 - numerical implementation/main.tex}
        \part{Project Overview}
            \input{5 - research plan/main.tex}
            \input{6 - summary/main.tex}
    
    
    %\section{}
    \newpage
    \pagenumbering{gobble}
        \printbibliography


    \newpage
    \pagenumbering{roman}
    \appendix
        \part{Appendices}
            \input{8 - Hilbert complexes/main.tex}
            \input{9 - weak conservation proofs/main.tex}
\end{document}

    
    
    %\section{}
    \newpage
    \pagenumbering{gobble}
        \printbibliography


    \newpage
    \pagenumbering{roman}
    \appendix
        \part{Appendices}
            \documentclass[12pt, a4paper]{report}

\input{template/main.tex}

\title{\BA{Title in Progress...}}
\author{Boris Andrews}
\affil{Mathematical Institute, University of Oxford}
\date{\today}


\begin{document}
    \pagenumbering{gobble}
    \maketitle
    
    
    \begin{abstract}
        Magnetic confinement reactors---in particular tokamaks---offer one of the most promising options for achieving practical nuclear fusion, with the potential to provide virtually limitless, clean energy. The theoretical and numerical modeling of tokamak plasmas is simultaneously an essential component of effective reactor design, and a great research barrier. Tokamak operational conditions exhibit comparatively low Knudsen numbers. Kinetic effects, including kinetic waves and instabilities, Landau damping, bump-on-tail instabilities and more, are therefore highly influential in tokamak plasma dynamics. Purely fluid models are inherently incapable of capturing these effects, whereas the high dimensionality in purely kinetic models render them practically intractable for most relevant purposes.

        We consider a $\delta\!f$ decomposition model, with a macroscopic fluid background and microscopic kinetic correction, both fully coupled to each other. A similar manner of discretization is proposed to that used in the recent \texttt{STRUPHY} code \cite{Holderied_Possanner_Wang_2021, Holderied_2022, Li_et_al_2023} with a finite-element model for the background and a pseudo-particle/PiC model for the correction.

        The fluid background satisfies the full, non-linear, resistive, compressible, Hall MHD equations. \cite{Laakmann_Hu_Farrell_2022} introduces finite-element(-in-space) implicit timesteppers for the incompressible analogue to this system with structure-preserving (SP) properties in the ideal case, alongside parameter-robust preconditioners. We show that these timesteppers can derive from a finite-element-in-time (FET) (and finite-element-in-space) interpretation. The benefits of this reformulation are discussed, including the derivation of timesteppers that are higher order in time, and the quantifiable dissipative SP properties in the non-ideal, resistive case.
        
        We discuss possible options for extending this FET approach to timesteppers for the compressible case.

        The kinetic corrections satisfy linearized Boltzmann equations. Using a Lénard--Bernstein collision operator, these take Fokker--Planck-like forms \cite{Fokker_1914, Planck_1917} wherein pseudo-particles in the numerical model obey the neoclassical transport equations, with particle-independent Brownian drift terms. This offers a rigorous methodology for incorporating collisions into the particle transport model, without coupling the equations of motions for each particle.
        
        Works by Chen, Chacón et al. \cite{Chen_Chacón_Barnes_2011, Chacón_Chen_Barnes_2013, Chen_Chacón_2014, Chen_Chacón_2015} have developed structure-preserving particle pushers for neoclassical transport in the Vlasov equations, derived from Crank--Nicolson integrators. We show these too can can derive from a FET interpretation, similarly offering potential extensions to higher-order-in-time particle pushers. The FET formulation is used also to consider how the stochastic drift terms can be incorporated into the pushers. Stochastic gyrokinetic expansions are also discussed.

        Different options for the numerical implementation of these schemes are considered.

        Due to the efficacy of FET in the development of SP timesteppers for both the fluid and kinetic component, we hope this approach will prove effective in the future for developing SP timesteppers for the full hybrid model. We hope this will give us the opportunity to incorporate previously inaccessible kinetic effects into the highly effective, modern, finite-element MHD models.
    \end{abstract}
    
    
    \newpage
    \tableofcontents
    
    
    \newpage
    \pagenumbering{arabic}
    %\linenumbers\renewcommand\thelinenumber{\color{black!50}\arabic{linenumber}}
            \input{0 - introduction/main.tex}
        \part{Research}
            \input{1 - low-noise PiC models/main.tex}
            \input{2 - kinetic component/main.tex}
            \input{3 - fluid component/main.tex}
            \input{4 - numerical implementation/main.tex}
        \part{Project Overview}
            \input{5 - research plan/main.tex}
            \input{6 - summary/main.tex}
    
    
    %\section{}
    \newpage
    \pagenumbering{gobble}
        \printbibliography


    \newpage
    \pagenumbering{roman}
    \appendix
        \part{Appendices}
            \input{8 - Hilbert complexes/main.tex}
            \input{9 - weak conservation proofs/main.tex}
\end{document}

            \documentclass[12pt, a4paper]{report}

\input{template/main.tex}

\title{\BA{Title in Progress...}}
\author{Boris Andrews}
\affil{Mathematical Institute, University of Oxford}
\date{\today}


\begin{document}
    \pagenumbering{gobble}
    \maketitle
    
    
    \begin{abstract}
        Magnetic confinement reactors---in particular tokamaks---offer one of the most promising options for achieving practical nuclear fusion, with the potential to provide virtually limitless, clean energy. The theoretical and numerical modeling of tokamak plasmas is simultaneously an essential component of effective reactor design, and a great research barrier. Tokamak operational conditions exhibit comparatively low Knudsen numbers. Kinetic effects, including kinetic waves and instabilities, Landau damping, bump-on-tail instabilities and more, are therefore highly influential in tokamak plasma dynamics. Purely fluid models are inherently incapable of capturing these effects, whereas the high dimensionality in purely kinetic models render them practically intractable for most relevant purposes.

        We consider a $\delta\!f$ decomposition model, with a macroscopic fluid background and microscopic kinetic correction, both fully coupled to each other. A similar manner of discretization is proposed to that used in the recent \texttt{STRUPHY} code \cite{Holderied_Possanner_Wang_2021, Holderied_2022, Li_et_al_2023} with a finite-element model for the background and a pseudo-particle/PiC model for the correction.

        The fluid background satisfies the full, non-linear, resistive, compressible, Hall MHD equations. \cite{Laakmann_Hu_Farrell_2022} introduces finite-element(-in-space) implicit timesteppers for the incompressible analogue to this system with structure-preserving (SP) properties in the ideal case, alongside parameter-robust preconditioners. We show that these timesteppers can derive from a finite-element-in-time (FET) (and finite-element-in-space) interpretation. The benefits of this reformulation are discussed, including the derivation of timesteppers that are higher order in time, and the quantifiable dissipative SP properties in the non-ideal, resistive case.
        
        We discuss possible options for extending this FET approach to timesteppers for the compressible case.

        The kinetic corrections satisfy linearized Boltzmann equations. Using a Lénard--Bernstein collision operator, these take Fokker--Planck-like forms \cite{Fokker_1914, Planck_1917} wherein pseudo-particles in the numerical model obey the neoclassical transport equations, with particle-independent Brownian drift terms. This offers a rigorous methodology for incorporating collisions into the particle transport model, without coupling the equations of motions for each particle.
        
        Works by Chen, Chacón et al. \cite{Chen_Chacón_Barnes_2011, Chacón_Chen_Barnes_2013, Chen_Chacón_2014, Chen_Chacón_2015} have developed structure-preserving particle pushers for neoclassical transport in the Vlasov equations, derived from Crank--Nicolson integrators. We show these too can can derive from a FET interpretation, similarly offering potential extensions to higher-order-in-time particle pushers. The FET formulation is used also to consider how the stochastic drift terms can be incorporated into the pushers. Stochastic gyrokinetic expansions are also discussed.

        Different options for the numerical implementation of these schemes are considered.

        Due to the efficacy of FET in the development of SP timesteppers for both the fluid and kinetic component, we hope this approach will prove effective in the future for developing SP timesteppers for the full hybrid model. We hope this will give us the opportunity to incorporate previously inaccessible kinetic effects into the highly effective, modern, finite-element MHD models.
    \end{abstract}
    
    
    \newpage
    \tableofcontents
    
    
    \newpage
    \pagenumbering{arabic}
    %\linenumbers\renewcommand\thelinenumber{\color{black!50}\arabic{linenumber}}
            \input{0 - introduction/main.tex}
        \part{Research}
            \input{1 - low-noise PiC models/main.tex}
            \input{2 - kinetic component/main.tex}
            \input{3 - fluid component/main.tex}
            \input{4 - numerical implementation/main.tex}
        \part{Project Overview}
            \input{5 - research plan/main.tex}
            \input{6 - summary/main.tex}
    
    
    %\section{}
    \newpage
    \pagenumbering{gobble}
        \printbibliography


    \newpage
    \pagenumbering{roman}
    \appendix
        \part{Appendices}
            \input{8 - Hilbert complexes/main.tex}
            \input{9 - weak conservation proofs/main.tex}
\end{document}

\end{document}

            \documentclass[12pt, a4paper]{report}

\documentclass[12pt, a4paper]{report}

\input{template/main.tex}

\title{\BA{Title in Progress...}}
\author{Boris Andrews}
\affil{Mathematical Institute, University of Oxford}
\date{\today}


\begin{document}
    \pagenumbering{gobble}
    \maketitle
    
    
    \begin{abstract}
        Magnetic confinement reactors---in particular tokamaks---offer one of the most promising options for achieving practical nuclear fusion, with the potential to provide virtually limitless, clean energy. The theoretical and numerical modeling of tokamak plasmas is simultaneously an essential component of effective reactor design, and a great research barrier. Tokamak operational conditions exhibit comparatively low Knudsen numbers. Kinetic effects, including kinetic waves and instabilities, Landau damping, bump-on-tail instabilities and more, are therefore highly influential in tokamak plasma dynamics. Purely fluid models are inherently incapable of capturing these effects, whereas the high dimensionality in purely kinetic models render them practically intractable for most relevant purposes.

        We consider a $\delta\!f$ decomposition model, with a macroscopic fluid background and microscopic kinetic correction, both fully coupled to each other. A similar manner of discretization is proposed to that used in the recent \texttt{STRUPHY} code \cite{Holderied_Possanner_Wang_2021, Holderied_2022, Li_et_al_2023} with a finite-element model for the background and a pseudo-particle/PiC model for the correction.

        The fluid background satisfies the full, non-linear, resistive, compressible, Hall MHD equations. \cite{Laakmann_Hu_Farrell_2022} introduces finite-element(-in-space) implicit timesteppers for the incompressible analogue to this system with structure-preserving (SP) properties in the ideal case, alongside parameter-robust preconditioners. We show that these timesteppers can derive from a finite-element-in-time (FET) (and finite-element-in-space) interpretation. The benefits of this reformulation are discussed, including the derivation of timesteppers that are higher order in time, and the quantifiable dissipative SP properties in the non-ideal, resistive case.
        
        We discuss possible options for extending this FET approach to timesteppers for the compressible case.

        The kinetic corrections satisfy linearized Boltzmann equations. Using a Lénard--Bernstein collision operator, these take Fokker--Planck-like forms \cite{Fokker_1914, Planck_1917} wherein pseudo-particles in the numerical model obey the neoclassical transport equations, with particle-independent Brownian drift terms. This offers a rigorous methodology for incorporating collisions into the particle transport model, without coupling the equations of motions for each particle.
        
        Works by Chen, Chacón et al. \cite{Chen_Chacón_Barnes_2011, Chacón_Chen_Barnes_2013, Chen_Chacón_2014, Chen_Chacón_2015} have developed structure-preserving particle pushers for neoclassical transport in the Vlasov equations, derived from Crank--Nicolson integrators. We show these too can can derive from a FET interpretation, similarly offering potential extensions to higher-order-in-time particle pushers. The FET formulation is used also to consider how the stochastic drift terms can be incorporated into the pushers. Stochastic gyrokinetic expansions are also discussed.

        Different options for the numerical implementation of these schemes are considered.

        Due to the efficacy of FET in the development of SP timesteppers for both the fluid and kinetic component, we hope this approach will prove effective in the future for developing SP timesteppers for the full hybrid model. We hope this will give us the opportunity to incorporate previously inaccessible kinetic effects into the highly effective, modern, finite-element MHD models.
    \end{abstract}
    
    
    \newpage
    \tableofcontents
    
    
    \newpage
    \pagenumbering{arabic}
    %\linenumbers\renewcommand\thelinenumber{\color{black!50}\arabic{linenumber}}
            \input{0 - introduction/main.tex}
        \part{Research}
            \input{1 - low-noise PiC models/main.tex}
            \input{2 - kinetic component/main.tex}
            \input{3 - fluid component/main.tex}
            \input{4 - numerical implementation/main.tex}
        \part{Project Overview}
            \input{5 - research plan/main.tex}
            \input{6 - summary/main.tex}
    
    
    %\section{}
    \newpage
    \pagenumbering{gobble}
        \printbibliography


    \newpage
    \pagenumbering{roman}
    \appendix
        \part{Appendices}
            \input{8 - Hilbert complexes/main.tex}
            \input{9 - weak conservation proofs/main.tex}
\end{document}


\title{\BA{Title in Progress...}}
\author{Boris Andrews}
\affil{Mathematical Institute, University of Oxford}
\date{\today}


\begin{document}
    \pagenumbering{gobble}
    \maketitle
    
    
    \begin{abstract}
        Magnetic confinement reactors---in particular tokamaks---offer one of the most promising options for achieving practical nuclear fusion, with the potential to provide virtually limitless, clean energy. The theoretical and numerical modeling of tokamak plasmas is simultaneously an essential component of effective reactor design, and a great research barrier. Tokamak operational conditions exhibit comparatively low Knudsen numbers. Kinetic effects, including kinetic waves and instabilities, Landau damping, bump-on-tail instabilities and more, are therefore highly influential in tokamak plasma dynamics. Purely fluid models are inherently incapable of capturing these effects, whereas the high dimensionality in purely kinetic models render them practically intractable for most relevant purposes.

        We consider a $\delta\!f$ decomposition model, with a macroscopic fluid background and microscopic kinetic correction, both fully coupled to each other. A similar manner of discretization is proposed to that used in the recent \texttt{STRUPHY} code \cite{Holderied_Possanner_Wang_2021, Holderied_2022, Li_et_al_2023} with a finite-element model for the background and a pseudo-particle/PiC model for the correction.

        The fluid background satisfies the full, non-linear, resistive, compressible, Hall MHD equations. \cite{Laakmann_Hu_Farrell_2022} introduces finite-element(-in-space) implicit timesteppers for the incompressible analogue to this system with structure-preserving (SP) properties in the ideal case, alongside parameter-robust preconditioners. We show that these timesteppers can derive from a finite-element-in-time (FET) (and finite-element-in-space) interpretation. The benefits of this reformulation are discussed, including the derivation of timesteppers that are higher order in time, and the quantifiable dissipative SP properties in the non-ideal, resistive case.
        
        We discuss possible options for extending this FET approach to timesteppers for the compressible case.

        The kinetic corrections satisfy linearized Boltzmann equations. Using a Lénard--Bernstein collision operator, these take Fokker--Planck-like forms \cite{Fokker_1914, Planck_1917} wherein pseudo-particles in the numerical model obey the neoclassical transport equations, with particle-independent Brownian drift terms. This offers a rigorous methodology for incorporating collisions into the particle transport model, without coupling the equations of motions for each particle.
        
        Works by Chen, Chacón et al. \cite{Chen_Chacón_Barnes_2011, Chacón_Chen_Barnes_2013, Chen_Chacón_2014, Chen_Chacón_2015} have developed structure-preserving particle pushers for neoclassical transport in the Vlasov equations, derived from Crank--Nicolson integrators. We show these too can can derive from a FET interpretation, similarly offering potential extensions to higher-order-in-time particle pushers. The FET formulation is used also to consider how the stochastic drift terms can be incorporated into the pushers. Stochastic gyrokinetic expansions are also discussed.

        Different options for the numerical implementation of these schemes are considered.

        Due to the efficacy of FET in the development of SP timesteppers for both the fluid and kinetic component, we hope this approach will prove effective in the future for developing SP timesteppers for the full hybrid model. We hope this will give us the opportunity to incorporate previously inaccessible kinetic effects into the highly effective, modern, finite-element MHD models.
    \end{abstract}
    
    
    \newpage
    \tableofcontents
    
    
    \newpage
    \pagenumbering{arabic}
    %\linenumbers\renewcommand\thelinenumber{\color{black!50}\arabic{linenumber}}
            \documentclass[12pt, a4paper]{report}

\input{template/main.tex}

\title{\BA{Title in Progress...}}
\author{Boris Andrews}
\affil{Mathematical Institute, University of Oxford}
\date{\today}


\begin{document}
    \pagenumbering{gobble}
    \maketitle
    
    
    \begin{abstract}
        Magnetic confinement reactors---in particular tokamaks---offer one of the most promising options for achieving practical nuclear fusion, with the potential to provide virtually limitless, clean energy. The theoretical and numerical modeling of tokamak plasmas is simultaneously an essential component of effective reactor design, and a great research barrier. Tokamak operational conditions exhibit comparatively low Knudsen numbers. Kinetic effects, including kinetic waves and instabilities, Landau damping, bump-on-tail instabilities and more, are therefore highly influential in tokamak plasma dynamics. Purely fluid models are inherently incapable of capturing these effects, whereas the high dimensionality in purely kinetic models render them practically intractable for most relevant purposes.

        We consider a $\delta\!f$ decomposition model, with a macroscopic fluid background and microscopic kinetic correction, both fully coupled to each other. A similar manner of discretization is proposed to that used in the recent \texttt{STRUPHY} code \cite{Holderied_Possanner_Wang_2021, Holderied_2022, Li_et_al_2023} with a finite-element model for the background and a pseudo-particle/PiC model for the correction.

        The fluid background satisfies the full, non-linear, resistive, compressible, Hall MHD equations. \cite{Laakmann_Hu_Farrell_2022} introduces finite-element(-in-space) implicit timesteppers for the incompressible analogue to this system with structure-preserving (SP) properties in the ideal case, alongside parameter-robust preconditioners. We show that these timesteppers can derive from a finite-element-in-time (FET) (and finite-element-in-space) interpretation. The benefits of this reformulation are discussed, including the derivation of timesteppers that are higher order in time, and the quantifiable dissipative SP properties in the non-ideal, resistive case.
        
        We discuss possible options for extending this FET approach to timesteppers for the compressible case.

        The kinetic corrections satisfy linearized Boltzmann equations. Using a Lénard--Bernstein collision operator, these take Fokker--Planck-like forms \cite{Fokker_1914, Planck_1917} wherein pseudo-particles in the numerical model obey the neoclassical transport equations, with particle-independent Brownian drift terms. This offers a rigorous methodology for incorporating collisions into the particle transport model, without coupling the equations of motions for each particle.
        
        Works by Chen, Chacón et al. \cite{Chen_Chacón_Barnes_2011, Chacón_Chen_Barnes_2013, Chen_Chacón_2014, Chen_Chacón_2015} have developed structure-preserving particle pushers for neoclassical transport in the Vlasov equations, derived from Crank--Nicolson integrators. We show these too can can derive from a FET interpretation, similarly offering potential extensions to higher-order-in-time particle pushers. The FET formulation is used also to consider how the stochastic drift terms can be incorporated into the pushers. Stochastic gyrokinetic expansions are also discussed.

        Different options for the numerical implementation of these schemes are considered.

        Due to the efficacy of FET in the development of SP timesteppers for both the fluid and kinetic component, we hope this approach will prove effective in the future for developing SP timesteppers for the full hybrid model. We hope this will give us the opportunity to incorporate previously inaccessible kinetic effects into the highly effective, modern, finite-element MHD models.
    \end{abstract}
    
    
    \newpage
    \tableofcontents
    
    
    \newpage
    \pagenumbering{arabic}
    %\linenumbers\renewcommand\thelinenumber{\color{black!50}\arabic{linenumber}}
            \input{0 - introduction/main.tex}
        \part{Research}
            \input{1 - low-noise PiC models/main.tex}
            \input{2 - kinetic component/main.tex}
            \input{3 - fluid component/main.tex}
            \input{4 - numerical implementation/main.tex}
        \part{Project Overview}
            \input{5 - research plan/main.tex}
            \input{6 - summary/main.tex}
    
    
    %\section{}
    \newpage
    \pagenumbering{gobble}
        \printbibliography


    \newpage
    \pagenumbering{roman}
    \appendix
        \part{Appendices}
            \input{8 - Hilbert complexes/main.tex}
            \input{9 - weak conservation proofs/main.tex}
\end{document}

        \part{Research}
            \documentclass[12pt, a4paper]{report}

\input{template/main.tex}

\title{\BA{Title in Progress...}}
\author{Boris Andrews}
\affil{Mathematical Institute, University of Oxford}
\date{\today}


\begin{document}
    \pagenumbering{gobble}
    \maketitle
    
    
    \begin{abstract}
        Magnetic confinement reactors---in particular tokamaks---offer one of the most promising options for achieving practical nuclear fusion, with the potential to provide virtually limitless, clean energy. The theoretical and numerical modeling of tokamak plasmas is simultaneously an essential component of effective reactor design, and a great research barrier. Tokamak operational conditions exhibit comparatively low Knudsen numbers. Kinetic effects, including kinetic waves and instabilities, Landau damping, bump-on-tail instabilities and more, are therefore highly influential in tokamak plasma dynamics. Purely fluid models are inherently incapable of capturing these effects, whereas the high dimensionality in purely kinetic models render them practically intractable for most relevant purposes.

        We consider a $\delta\!f$ decomposition model, with a macroscopic fluid background and microscopic kinetic correction, both fully coupled to each other. A similar manner of discretization is proposed to that used in the recent \texttt{STRUPHY} code \cite{Holderied_Possanner_Wang_2021, Holderied_2022, Li_et_al_2023} with a finite-element model for the background and a pseudo-particle/PiC model for the correction.

        The fluid background satisfies the full, non-linear, resistive, compressible, Hall MHD equations. \cite{Laakmann_Hu_Farrell_2022} introduces finite-element(-in-space) implicit timesteppers for the incompressible analogue to this system with structure-preserving (SP) properties in the ideal case, alongside parameter-robust preconditioners. We show that these timesteppers can derive from a finite-element-in-time (FET) (and finite-element-in-space) interpretation. The benefits of this reformulation are discussed, including the derivation of timesteppers that are higher order in time, and the quantifiable dissipative SP properties in the non-ideal, resistive case.
        
        We discuss possible options for extending this FET approach to timesteppers for the compressible case.

        The kinetic corrections satisfy linearized Boltzmann equations. Using a Lénard--Bernstein collision operator, these take Fokker--Planck-like forms \cite{Fokker_1914, Planck_1917} wherein pseudo-particles in the numerical model obey the neoclassical transport equations, with particle-independent Brownian drift terms. This offers a rigorous methodology for incorporating collisions into the particle transport model, without coupling the equations of motions for each particle.
        
        Works by Chen, Chacón et al. \cite{Chen_Chacón_Barnes_2011, Chacón_Chen_Barnes_2013, Chen_Chacón_2014, Chen_Chacón_2015} have developed structure-preserving particle pushers for neoclassical transport in the Vlasov equations, derived from Crank--Nicolson integrators. We show these too can can derive from a FET interpretation, similarly offering potential extensions to higher-order-in-time particle pushers. The FET formulation is used also to consider how the stochastic drift terms can be incorporated into the pushers. Stochastic gyrokinetic expansions are also discussed.

        Different options for the numerical implementation of these schemes are considered.

        Due to the efficacy of FET in the development of SP timesteppers for both the fluid and kinetic component, we hope this approach will prove effective in the future for developing SP timesteppers for the full hybrid model. We hope this will give us the opportunity to incorporate previously inaccessible kinetic effects into the highly effective, modern, finite-element MHD models.
    \end{abstract}
    
    
    \newpage
    \tableofcontents
    
    
    \newpage
    \pagenumbering{arabic}
    %\linenumbers\renewcommand\thelinenumber{\color{black!50}\arabic{linenumber}}
            \input{0 - introduction/main.tex}
        \part{Research}
            \input{1 - low-noise PiC models/main.tex}
            \input{2 - kinetic component/main.tex}
            \input{3 - fluid component/main.tex}
            \input{4 - numerical implementation/main.tex}
        \part{Project Overview}
            \input{5 - research plan/main.tex}
            \input{6 - summary/main.tex}
    
    
    %\section{}
    \newpage
    \pagenumbering{gobble}
        \printbibliography


    \newpage
    \pagenumbering{roman}
    \appendix
        \part{Appendices}
            \input{8 - Hilbert complexes/main.tex}
            \input{9 - weak conservation proofs/main.tex}
\end{document}

            \documentclass[12pt, a4paper]{report}

\input{template/main.tex}

\title{\BA{Title in Progress...}}
\author{Boris Andrews}
\affil{Mathematical Institute, University of Oxford}
\date{\today}


\begin{document}
    \pagenumbering{gobble}
    \maketitle
    
    
    \begin{abstract}
        Magnetic confinement reactors---in particular tokamaks---offer one of the most promising options for achieving practical nuclear fusion, with the potential to provide virtually limitless, clean energy. The theoretical and numerical modeling of tokamak plasmas is simultaneously an essential component of effective reactor design, and a great research barrier. Tokamak operational conditions exhibit comparatively low Knudsen numbers. Kinetic effects, including kinetic waves and instabilities, Landau damping, bump-on-tail instabilities and more, are therefore highly influential in tokamak plasma dynamics. Purely fluid models are inherently incapable of capturing these effects, whereas the high dimensionality in purely kinetic models render them practically intractable for most relevant purposes.

        We consider a $\delta\!f$ decomposition model, with a macroscopic fluid background and microscopic kinetic correction, both fully coupled to each other. A similar manner of discretization is proposed to that used in the recent \texttt{STRUPHY} code \cite{Holderied_Possanner_Wang_2021, Holderied_2022, Li_et_al_2023} with a finite-element model for the background and a pseudo-particle/PiC model for the correction.

        The fluid background satisfies the full, non-linear, resistive, compressible, Hall MHD equations. \cite{Laakmann_Hu_Farrell_2022} introduces finite-element(-in-space) implicit timesteppers for the incompressible analogue to this system with structure-preserving (SP) properties in the ideal case, alongside parameter-robust preconditioners. We show that these timesteppers can derive from a finite-element-in-time (FET) (and finite-element-in-space) interpretation. The benefits of this reformulation are discussed, including the derivation of timesteppers that are higher order in time, and the quantifiable dissipative SP properties in the non-ideal, resistive case.
        
        We discuss possible options for extending this FET approach to timesteppers for the compressible case.

        The kinetic corrections satisfy linearized Boltzmann equations. Using a Lénard--Bernstein collision operator, these take Fokker--Planck-like forms \cite{Fokker_1914, Planck_1917} wherein pseudo-particles in the numerical model obey the neoclassical transport equations, with particle-independent Brownian drift terms. This offers a rigorous methodology for incorporating collisions into the particle transport model, without coupling the equations of motions for each particle.
        
        Works by Chen, Chacón et al. \cite{Chen_Chacón_Barnes_2011, Chacón_Chen_Barnes_2013, Chen_Chacón_2014, Chen_Chacón_2015} have developed structure-preserving particle pushers for neoclassical transport in the Vlasov equations, derived from Crank--Nicolson integrators. We show these too can can derive from a FET interpretation, similarly offering potential extensions to higher-order-in-time particle pushers. The FET formulation is used also to consider how the stochastic drift terms can be incorporated into the pushers. Stochastic gyrokinetic expansions are also discussed.

        Different options for the numerical implementation of these schemes are considered.

        Due to the efficacy of FET in the development of SP timesteppers for both the fluid and kinetic component, we hope this approach will prove effective in the future for developing SP timesteppers for the full hybrid model. We hope this will give us the opportunity to incorporate previously inaccessible kinetic effects into the highly effective, modern, finite-element MHD models.
    \end{abstract}
    
    
    \newpage
    \tableofcontents
    
    
    \newpage
    \pagenumbering{arabic}
    %\linenumbers\renewcommand\thelinenumber{\color{black!50}\arabic{linenumber}}
            \input{0 - introduction/main.tex}
        \part{Research}
            \input{1 - low-noise PiC models/main.tex}
            \input{2 - kinetic component/main.tex}
            \input{3 - fluid component/main.tex}
            \input{4 - numerical implementation/main.tex}
        \part{Project Overview}
            \input{5 - research plan/main.tex}
            \input{6 - summary/main.tex}
    
    
    %\section{}
    \newpage
    \pagenumbering{gobble}
        \printbibliography


    \newpage
    \pagenumbering{roman}
    \appendix
        \part{Appendices}
            \input{8 - Hilbert complexes/main.tex}
            \input{9 - weak conservation proofs/main.tex}
\end{document}

            \documentclass[12pt, a4paper]{report}

\input{template/main.tex}

\title{\BA{Title in Progress...}}
\author{Boris Andrews}
\affil{Mathematical Institute, University of Oxford}
\date{\today}


\begin{document}
    \pagenumbering{gobble}
    \maketitle
    
    
    \begin{abstract}
        Magnetic confinement reactors---in particular tokamaks---offer one of the most promising options for achieving practical nuclear fusion, with the potential to provide virtually limitless, clean energy. The theoretical and numerical modeling of tokamak plasmas is simultaneously an essential component of effective reactor design, and a great research barrier. Tokamak operational conditions exhibit comparatively low Knudsen numbers. Kinetic effects, including kinetic waves and instabilities, Landau damping, bump-on-tail instabilities and more, are therefore highly influential in tokamak plasma dynamics. Purely fluid models are inherently incapable of capturing these effects, whereas the high dimensionality in purely kinetic models render them practically intractable for most relevant purposes.

        We consider a $\delta\!f$ decomposition model, with a macroscopic fluid background and microscopic kinetic correction, both fully coupled to each other. A similar manner of discretization is proposed to that used in the recent \texttt{STRUPHY} code \cite{Holderied_Possanner_Wang_2021, Holderied_2022, Li_et_al_2023} with a finite-element model for the background and a pseudo-particle/PiC model for the correction.

        The fluid background satisfies the full, non-linear, resistive, compressible, Hall MHD equations. \cite{Laakmann_Hu_Farrell_2022} introduces finite-element(-in-space) implicit timesteppers for the incompressible analogue to this system with structure-preserving (SP) properties in the ideal case, alongside parameter-robust preconditioners. We show that these timesteppers can derive from a finite-element-in-time (FET) (and finite-element-in-space) interpretation. The benefits of this reformulation are discussed, including the derivation of timesteppers that are higher order in time, and the quantifiable dissipative SP properties in the non-ideal, resistive case.
        
        We discuss possible options for extending this FET approach to timesteppers for the compressible case.

        The kinetic corrections satisfy linearized Boltzmann equations. Using a Lénard--Bernstein collision operator, these take Fokker--Planck-like forms \cite{Fokker_1914, Planck_1917} wherein pseudo-particles in the numerical model obey the neoclassical transport equations, with particle-independent Brownian drift terms. This offers a rigorous methodology for incorporating collisions into the particle transport model, without coupling the equations of motions for each particle.
        
        Works by Chen, Chacón et al. \cite{Chen_Chacón_Barnes_2011, Chacón_Chen_Barnes_2013, Chen_Chacón_2014, Chen_Chacón_2015} have developed structure-preserving particle pushers for neoclassical transport in the Vlasov equations, derived from Crank--Nicolson integrators. We show these too can can derive from a FET interpretation, similarly offering potential extensions to higher-order-in-time particle pushers. The FET formulation is used also to consider how the stochastic drift terms can be incorporated into the pushers. Stochastic gyrokinetic expansions are also discussed.

        Different options for the numerical implementation of these schemes are considered.

        Due to the efficacy of FET in the development of SP timesteppers for both the fluid and kinetic component, we hope this approach will prove effective in the future for developing SP timesteppers for the full hybrid model. We hope this will give us the opportunity to incorporate previously inaccessible kinetic effects into the highly effective, modern, finite-element MHD models.
    \end{abstract}
    
    
    \newpage
    \tableofcontents
    
    
    \newpage
    \pagenumbering{arabic}
    %\linenumbers\renewcommand\thelinenumber{\color{black!50}\arabic{linenumber}}
            \input{0 - introduction/main.tex}
        \part{Research}
            \input{1 - low-noise PiC models/main.tex}
            \input{2 - kinetic component/main.tex}
            \input{3 - fluid component/main.tex}
            \input{4 - numerical implementation/main.tex}
        \part{Project Overview}
            \input{5 - research plan/main.tex}
            \input{6 - summary/main.tex}
    
    
    %\section{}
    \newpage
    \pagenumbering{gobble}
        \printbibliography


    \newpage
    \pagenumbering{roman}
    \appendix
        \part{Appendices}
            \input{8 - Hilbert complexes/main.tex}
            \input{9 - weak conservation proofs/main.tex}
\end{document}

            \documentclass[12pt, a4paper]{report}

\input{template/main.tex}

\title{\BA{Title in Progress...}}
\author{Boris Andrews}
\affil{Mathematical Institute, University of Oxford}
\date{\today}


\begin{document}
    \pagenumbering{gobble}
    \maketitle
    
    
    \begin{abstract}
        Magnetic confinement reactors---in particular tokamaks---offer one of the most promising options for achieving practical nuclear fusion, with the potential to provide virtually limitless, clean energy. The theoretical and numerical modeling of tokamak plasmas is simultaneously an essential component of effective reactor design, and a great research barrier. Tokamak operational conditions exhibit comparatively low Knudsen numbers. Kinetic effects, including kinetic waves and instabilities, Landau damping, bump-on-tail instabilities and more, are therefore highly influential in tokamak plasma dynamics. Purely fluid models are inherently incapable of capturing these effects, whereas the high dimensionality in purely kinetic models render them practically intractable for most relevant purposes.

        We consider a $\delta\!f$ decomposition model, with a macroscopic fluid background and microscopic kinetic correction, both fully coupled to each other. A similar manner of discretization is proposed to that used in the recent \texttt{STRUPHY} code \cite{Holderied_Possanner_Wang_2021, Holderied_2022, Li_et_al_2023} with a finite-element model for the background and a pseudo-particle/PiC model for the correction.

        The fluid background satisfies the full, non-linear, resistive, compressible, Hall MHD equations. \cite{Laakmann_Hu_Farrell_2022} introduces finite-element(-in-space) implicit timesteppers for the incompressible analogue to this system with structure-preserving (SP) properties in the ideal case, alongside parameter-robust preconditioners. We show that these timesteppers can derive from a finite-element-in-time (FET) (and finite-element-in-space) interpretation. The benefits of this reformulation are discussed, including the derivation of timesteppers that are higher order in time, and the quantifiable dissipative SP properties in the non-ideal, resistive case.
        
        We discuss possible options for extending this FET approach to timesteppers for the compressible case.

        The kinetic corrections satisfy linearized Boltzmann equations. Using a Lénard--Bernstein collision operator, these take Fokker--Planck-like forms \cite{Fokker_1914, Planck_1917} wherein pseudo-particles in the numerical model obey the neoclassical transport equations, with particle-independent Brownian drift terms. This offers a rigorous methodology for incorporating collisions into the particle transport model, without coupling the equations of motions for each particle.
        
        Works by Chen, Chacón et al. \cite{Chen_Chacón_Barnes_2011, Chacón_Chen_Barnes_2013, Chen_Chacón_2014, Chen_Chacón_2015} have developed structure-preserving particle pushers for neoclassical transport in the Vlasov equations, derived from Crank--Nicolson integrators. We show these too can can derive from a FET interpretation, similarly offering potential extensions to higher-order-in-time particle pushers. The FET formulation is used also to consider how the stochastic drift terms can be incorporated into the pushers. Stochastic gyrokinetic expansions are also discussed.

        Different options for the numerical implementation of these schemes are considered.

        Due to the efficacy of FET in the development of SP timesteppers for both the fluid and kinetic component, we hope this approach will prove effective in the future for developing SP timesteppers for the full hybrid model. We hope this will give us the opportunity to incorporate previously inaccessible kinetic effects into the highly effective, modern, finite-element MHD models.
    \end{abstract}
    
    
    \newpage
    \tableofcontents
    
    
    \newpage
    \pagenumbering{arabic}
    %\linenumbers\renewcommand\thelinenumber{\color{black!50}\arabic{linenumber}}
            \input{0 - introduction/main.tex}
        \part{Research}
            \input{1 - low-noise PiC models/main.tex}
            \input{2 - kinetic component/main.tex}
            \input{3 - fluid component/main.tex}
            \input{4 - numerical implementation/main.tex}
        \part{Project Overview}
            \input{5 - research plan/main.tex}
            \input{6 - summary/main.tex}
    
    
    %\section{}
    \newpage
    \pagenumbering{gobble}
        \printbibliography


    \newpage
    \pagenumbering{roman}
    \appendix
        \part{Appendices}
            \input{8 - Hilbert complexes/main.tex}
            \input{9 - weak conservation proofs/main.tex}
\end{document}

        \part{Project Overview}
            \documentclass[12pt, a4paper]{report}

\input{template/main.tex}

\title{\BA{Title in Progress...}}
\author{Boris Andrews}
\affil{Mathematical Institute, University of Oxford}
\date{\today}


\begin{document}
    \pagenumbering{gobble}
    \maketitle
    
    
    \begin{abstract}
        Magnetic confinement reactors---in particular tokamaks---offer one of the most promising options for achieving practical nuclear fusion, with the potential to provide virtually limitless, clean energy. The theoretical and numerical modeling of tokamak plasmas is simultaneously an essential component of effective reactor design, and a great research barrier. Tokamak operational conditions exhibit comparatively low Knudsen numbers. Kinetic effects, including kinetic waves and instabilities, Landau damping, bump-on-tail instabilities and more, are therefore highly influential in tokamak plasma dynamics. Purely fluid models are inherently incapable of capturing these effects, whereas the high dimensionality in purely kinetic models render them practically intractable for most relevant purposes.

        We consider a $\delta\!f$ decomposition model, with a macroscopic fluid background and microscopic kinetic correction, both fully coupled to each other. A similar manner of discretization is proposed to that used in the recent \texttt{STRUPHY} code \cite{Holderied_Possanner_Wang_2021, Holderied_2022, Li_et_al_2023} with a finite-element model for the background and a pseudo-particle/PiC model for the correction.

        The fluid background satisfies the full, non-linear, resistive, compressible, Hall MHD equations. \cite{Laakmann_Hu_Farrell_2022} introduces finite-element(-in-space) implicit timesteppers for the incompressible analogue to this system with structure-preserving (SP) properties in the ideal case, alongside parameter-robust preconditioners. We show that these timesteppers can derive from a finite-element-in-time (FET) (and finite-element-in-space) interpretation. The benefits of this reformulation are discussed, including the derivation of timesteppers that are higher order in time, and the quantifiable dissipative SP properties in the non-ideal, resistive case.
        
        We discuss possible options for extending this FET approach to timesteppers for the compressible case.

        The kinetic corrections satisfy linearized Boltzmann equations. Using a Lénard--Bernstein collision operator, these take Fokker--Planck-like forms \cite{Fokker_1914, Planck_1917} wherein pseudo-particles in the numerical model obey the neoclassical transport equations, with particle-independent Brownian drift terms. This offers a rigorous methodology for incorporating collisions into the particle transport model, without coupling the equations of motions for each particle.
        
        Works by Chen, Chacón et al. \cite{Chen_Chacón_Barnes_2011, Chacón_Chen_Barnes_2013, Chen_Chacón_2014, Chen_Chacón_2015} have developed structure-preserving particle pushers for neoclassical transport in the Vlasov equations, derived from Crank--Nicolson integrators. We show these too can can derive from a FET interpretation, similarly offering potential extensions to higher-order-in-time particle pushers. The FET formulation is used also to consider how the stochastic drift terms can be incorporated into the pushers. Stochastic gyrokinetic expansions are also discussed.

        Different options for the numerical implementation of these schemes are considered.

        Due to the efficacy of FET in the development of SP timesteppers for both the fluid and kinetic component, we hope this approach will prove effective in the future for developing SP timesteppers for the full hybrid model. We hope this will give us the opportunity to incorporate previously inaccessible kinetic effects into the highly effective, modern, finite-element MHD models.
    \end{abstract}
    
    
    \newpage
    \tableofcontents
    
    
    \newpage
    \pagenumbering{arabic}
    %\linenumbers\renewcommand\thelinenumber{\color{black!50}\arabic{linenumber}}
            \input{0 - introduction/main.tex}
        \part{Research}
            \input{1 - low-noise PiC models/main.tex}
            \input{2 - kinetic component/main.tex}
            \input{3 - fluid component/main.tex}
            \input{4 - numerical implementation/main.tex}
        \part{Project Overview}
            \input{5 - research plan/main.tex}
            \input{6 - summary/main.tex}
    
    
    %\section{}
    \newpage
    \pagenumbering{gobble}
        \printbibliography


    \newpage
    \pagenumbering{roman}
    \appendix
        \part{Appendices}
            \input{8 - Hilbert complexes/main.tex}
            \input{9 - weak conservation proofs/main.tex}
\end{document}

            \documentclass[12pt, a4paper]{report}

\input{template/main.tex}

\title{\BA{Title in Progress...}}
\author{Boris Andrews}
\affil{Mathematical Institute, University of Oxford}
\date{\today}


\begin{document}
    \pagenumbering{gobble}
    \maketitle
    
    
    \begin{abstract}
        Magnetic confinement reactors---in particular tokamaks---offer one of the most promising options for achieving practical nuclear fusion, with the potential to provide virtually limitless, clean energy. The theoretical and numerical modeling of tokamak plasmas is simultaneously an essential component of effective reactor design, and a great research barrier. Tokamak operational conditions exhibit comparatively low Knudsen numbers. Kinetic effects, including kinetic waves and instabilities, Landau damping, bump-on-tail instabilities and more, are therefore highly influential in tokamak plasma dynamics. Purely fluid models are inherently incapable of capturing these effects, whereas the high dimensionality in purely kinetic models render them practically intractable for most relevant purposes.

        We consider a $\delta\!f$ decomposition model, with a macroscopic fluid background and microscopic kinetic correction, both fully coupled to each other. A similar manner of discretization is proposed to that used in the recent \texttt{STRUPHY} code \cite{Holderied_Possanner_Wang_2021, Holderied_2022, Li_et_al_2023} with a finite-element model for the background and a pseudo-particle/PiC model for the correction.

        The fluid background satisfies the full, non-linear, resistive, compressible, Hall MHD equations. \cite{Laakmann_Hu_Farrell_2022} introduces finite-element(-in-space) implicit timesteppers for the incompressible analogue to this system with structure-preserving (SP) properties in the ideal case, alongside parameter-robust preconditioners. We show that these timesteppers can derive from a finite-element-in-time (FET) (and finite-element-in-space) interpretation. The benefits of this reformulation are discussed, including the derivation of timesteppers that are higher order in time, and the quantifiable dissipative SP properties in the non-ideal, resistive case.
        
        We discuss possible options for extending this FET approach to timesteppers for the compressible case.

        The kinetic corrections satisfy linearized Boltzmann equations. Using a Lénard--Bernstein collision operator, these take Fokker--Planck-like forms \cite{Fokker_1914, Planck_1917} wherein pseudo-particles in the numerical model obey the neoclassical transport equations, with particle-independent Brownian drift terms. This offers a rigorous methodology for incorporating collisions into the particle transport model, without coupling the equations of motions for each particle.
        
        Works by Chen, Chacón et al. \cite{Chen_Chacón_Barnes_2011, Chacón_Chen_Barnes_2013, Chen_Chacón_2014, Chen_Chacón_2015} have developed structure-preserving particle pushers for neoclassical transport in the Vlasov equations, derived from Crank--Nicolson integrators. We show these too can can derive from a FET interpretation, similarly offering potential extensions to higher-order-in-time particle pushers. The FET formulation is used also to consider how the stochastic drift terms can be incorporated into the pushers. Stochastic gyrokinetic expansions are also discussed.

        Different options for the numerical implementation of these schemes are considered.

        Due to the efficacy of FET in the development of SP timesteppers for both the fluid and kinetic component, we hope this approach will prove effective in the future for developing SP timesteppers for the full hybrid model. We hope this will give us the opportunity to incorporate previously inaccessible kinetic effects into the highly effective, modern, finite-element MHD models.
    \end{abstract}
    
    
    \newpage
    \tableofcontents
    
    
    \newpage
    \pagenumbering{arabic}
    %\linenumbers\renewcommand\thelinenumber{\color{black!50}\arabic{linenumber}}
            \input{0 - introduction/main.tex}
        \part{Research}
            \input{1 - low-noise PiC models/main.tex}
            \input{2 - kinetic component/main.tex}
            \input{3 - fluid component/main.tex}
            \input{4 - numerical implementation/main.tex}
        \part{Project Overview}
            \input{5 - research plan/main.tex}
            \input{6 - summary/main.tex}
    
    
    %\section{}
    \newpage
    \pagenumbering{gobble}
        \printbibliography


    \newpage
    \pagenumbering{roman}
    \appendix
        \part{Appendices}
            \input{8 - Hilbert complexes/main.tex}
            \input{9 - weak conservation proofs/main.tex}
\end{document}

    
    
    %\section{}
    \newpage
    \pagenumbering{gobble}
        \printbibliography


    \newpage
    \pagenumbering{roman}
    \appendix
        \part{Appendices}
            \documentclass[12pt, a4paper]{report}

\input{template/main.tex}

\title{\BA{Title in Progress...}}
\author{Boris Andrews}
\affil{Mathematical Institute, University of Oxford}
\date{\today}


\begin{document}
    \pagenumbering{gobble}
    \maketitle
    
    
    \begin{abstract}
        Magnetic confinement reactors---in particular tokamaks---offer one of the most promising options for achieving practical nuclear fusion, with the potential to provide virtually limitless, clean energy. The theoretical and numerical modeling of tokamak plasmas is simultaneously an essential component of effective reactor design, and a great research barrier. Tokamak operational conditions exhibit comparatively low Knudsen numbers. Kinetic effects, including kinetic waves and instabilities, Landau damping, bump-on-tail instabilities and more, are therefore highly influential in tokamak plasma dynamics. Purely fluid models are inherently incapable of capturing these effects, whereas the high dimensionality in purely kinetic models render them practically intractable for most relevant purposes.

        We consider a $\delta\!f$ decomposition model, with a macroscopic fluid background and microscopic kinetic correction, both fully coupled to each other. A similar manner of discretization is proposed to that used in the recent \texttt{STRUPHY} code \cite{Holderied_Possanner_Wang_2021, Holderied_2022, Li_et_al_2023} with a finite-element model for the background and a pseudo-particle/PiC model for the correction.

        The fluid background satisfies the full, non-linear, resistive, compressible, Hall MHD equations. \cite{Laakmann_Hu_Farrell_2022} introduces finite-element(-in-space) implicit timesteppers for the incompressible analogue to this system with structure-preserving (SP) properties in the ideal case, alongside parameter-robust preconditioners. We show that these timesteppers can derive from a finite-element-in-time (FET) (and finite-element-in-space) interpretation. The benefits of this reformulation are discussed, including the derivation of timesteppers that are higher order in time, and the quantifiable dissipative SP properties in the non-ideal, resistive case.
        
        We discuss possible options for extending this FET approach to timesteppers for the compressible case.

        The kinetic corrections satisfy linearized Boltzmann equations. Using a Lénard--Bernstein collision operator, these take Fokker--Planck-like forms \cite{Fokker_1914, Planck_1917} wherein pseudo-particles in the numerical model obey the neoclassical transport equations, with particle-independent Brownian drift terms. This offers a rigorous methodology for incorporating collisions into the particle transport model, without coupling the equations of motions for each particle.
        
        Works by Chen, Chacón et al. \cite{Chen_Chacón_Barnes_2011, Chacón_Chen_Barnes_2013, Chen_Chacón_2014, Chen_Chacón_2015} have developed structure-preserving particle pushers for neoclassical transport in the Vlasov equations, derived from Crank--Nicolson integrators. We show these too can can derive from a FET interpretation, similarly offering potential extensions to higher-order-in-time particle pushers. The FET formulation is used also to consider how the stochastic drift terms can be incorporated into the pushers. Stochastic gyrokinetic expansions are also discussed.

        Different options for the numerical implementation of these schemes are considered.

        Due to the efficacy of FET in the development of SP timesteppers for both the fluid and kinetic component, we hope this approach will prove effective in the future for developing SP timesteppers for the full hybrid model. We hope this will give us the opportunity to incorporate previously inaccessible kinetic effects into the highly effective, modern, finite-element MHD models.
    \end{abstract}
    
    
    \newpage
    \tableofcontents
    
    
    \newpage
    \pagenumbering{arabic}
    %\linenumbers\renewcommand\thelinenumber{\color{black!50}\arabic{linenumber}}
            \input{0 - introduction/main.tex}
        \part{Research}
            \input{1 - low-noise PiC models/main.tex}
            \input{2 - kinetic component/main.tex}
            \input{3 - fluid component/main.tex}
            \input{4 - numerical implementation/main.tex}
        \part{Project Overview}
            \input{5 - research plan/main.tex}
            \input{6 - summary/main.tex}
    
    
    %\section{}
    \newpage
    \pagenumbering{gobble}
        \printbibliography


    \newpage
    \pagenumbering{roman}
    \appendix
        \part{Appendices}
            \input{8 - Hilbert complexes/main.tex}
            \input{9 - weak conservation proofs/main.tex}
\end{document}

            \documentclass[12pt, a4paper]{report}

\input{template/main.tex}

\title{\BA{Title in Progress...}}
\author{Boris Andrews}
\affil{Mathematical Institute, University of Oxford}
\date{\today}


\begin{document}
    \pagenumbering{gobble}
    \maketitle
    
    
    \begin{abstract}
        Magnetic confinement reactors---in particular tokamaks---offer one of the most promising options for achieving practical nuclear fusion, with the potential to provide virtually limitless, clean energy. The theoretical and numerical modeling of tokamak plasmas is simultaneously an essential component of effective reactor design, and a great research barrier. Tokamak operational conditions exhibit comparatively low Knudsen numbers. Kinetic effects, including kinetic waves and instabilities, Landau damping, bump-on-tail instabilities and more, are therefore highly influential in tokamak plasma dynamics. Purely fluid models are inherently incapable of capturing these effects, whereas the high dimensionality in purely kinetic models render them practically intractable for most relevant purposes.

        We consider a $\delta\!f$ decomposition model, with a macroscopic fluid background and microscopic kinetic correction, both fully coupled to each other. A similar manner of discretization is proposed to that used in the recent \texttt{STRUPHY} code \cite{Holderied_Possanner_Wang_2021, Holderied_2022, Li_et_al_2023} with a finite-element model for the background and a pseudo-particle/PiC model for the correction.

        The fluid background satisfies the full, non-linear, resistive, compressible, Hall MHD equations. \cite{Laakmann_Hu_Farrell_2022} introduces finite-element(-in-space) implicit timesteppers for the incompressible analogue to this system with structure-preserving (SP) properties in the ideal case, alongside parameter-robust preconditioners. We show that these timesteppers can derive from a finite-element-in-time (FET) (and finite-element-in-space) interpretation. The benefits of this reformulation are discussed, including the derivation of timesteppers that are higher order in time, and the quantifiable dissipative SP properties in the non-ideal, resistive case.
        
        We discuss possible options for extending this FET approach to timesteppers for the compressible case.

        The kinetic corrections satisfy linearized Boltzmann equations. Using a Lénard--Bernstein collision operator, these take Fokker--Planck-like forms \cite{Fokker_1914, Planck_1917} wherein pseudo-particles in the numerical model obey the neoclassical transport equations, with particle-independent Brownian drift terms. This offers a rigorous methodology for incorporating collisions into the particle transport model, without coupling the equations of motions for each particle.
        
        Works by Chen, Chacón et al. \cite{Chen_Chacón_Barnes_2011, Chacón_Chen_Barnes_2013, Chen_Chacón_2014, Chen_Chacón_2015} have developed structure-preserving particle pushers for neoclassical transport in the Vlasov equations, derived from Crank--Nicolson integrators. We show these too can can derive from a FET interpretation, similarly offering potential extensions to higher-order-in-time particle pushers. The FET formulation is used also to consider how the stochastic drift terms can be incorporated into the pushers. Stochastic gyrokinetic expansions are also discussed.

        Different options for the numerical implementation of these schemes are considered.

        Due to the efficacy of FET in the development of SP timesteppers for both the fluid and kinetic component, we hope this approach will prove effective in the future for developing SP timesteppers for the full hybrid model. We hope this will give us the opportunity to incorporate previously inaccessible kinetic effects into the highly effective, modern, finite-element MHD models.
    \end{abstract}
    
    
    \newpage
    \tableofcontents
    
    
    \newpage
    \pagenumbering{arabic}
    %\linenumbers\renewcommand\thelinenumber{\color{black!50}\arabic{linenumber}}
            \input{0 - introduction/main.tex}
        \part{Research}
            \input{1 - low-noise PiC models/main.tex}
            \input{2 - kinetic component/main.tex}
            \input{3 - fluid component/main.tex}
            \input{4 - numerical implementation/main.tex}
        \part{Project Overview}
            \input{5 - research plan/main.tex}
            \input{6 - summary/main.tex}
    
    
    %\section{}
    \newpage
    \pagenumbering{gobble}
        \printbibliography


    \newpage
    \pagenumbering{roman}
    \appendix
        \part{Appendices}
            \input{8 - Hilbert complexes/main.tex}
            \input{9 - weak conservation proofs/main.tex}
\end{document}

\end{document}

\end{document}

        \part{Research}
            \documentclass[12pt, a4paper]{report}

\documentclass[12pt, a4paper]{report}

\documentclass[12pt, a4paper]{report}

\input{template/main.tex}

\title{\BA{Title in Progress...}}
\author{Boris Andrews}
\affil{Mathematical Institute, University of Oxford}
\date{\today}


\begin{document}
    \pagenumbering{gobble}
    \maketitle
    
    
    \begin{abstract}
        Magnetic confinement reactors---in particular tokamaks---offer one of the most promising options for achieving practical nuclear fusion, with the potential to provide virtually limitless, clean energy. The theoretical and numerical modeling of tokamak plasmas is simultaneously an essential component of effective reactor design, and a great research barrier. Tokamak operational conditions exhibit comparatively low Knudsen numbers. Kinetic effects, including kinetic waves and instabilities, Landau damping, bump-on-tail instabilities and more, are therefore highly influential in tokamak plasma dynamics. Purely fluid models are inherently incapable of capturing these effects, whereas the high dimensionality in purely kinetic models render them practically intractable for most relevant purposes.

        We consider a $\delta\!f$ decomposition model, with a macroscopic fluid background and microscopic kinetic correction, both fully coupled to each other. A similar manner of discretization is proposed to that used in the recent \texttt{STRUPHY} code \cite{Holderied_Possanner_Wang_2021, Holderied_2022, Li_et_al_2023} with a finite-element model for the background and a pseudo-particle/PiC model for the correction.

        The fluid background satisfies the full, non-linear, resistive, compressible, Hall MHD equations. \cite{Laakmann_Hu_Farrell_2022} introduces finite-element(-in-space) implicit timesteppers for the incompressible analogue to this system with structure-preserving (SP) properties in the ideal case, alongside parameter-robust preconditioners. We show that these timesteppers can derive from a finite-element-in-time (FET) (and finite-element-in-space) interpretation. The benefits of this reformulation are discussed, including the derivation of timesteppers that are higher order in time, and the quantifiable dissipative SP properties in the non-ideal, resistive case.
        
        We discuss possible options for extending this FET approach to timesteppers for the compressible case.

        The kinetic corrections satisfy linearized Boltzmann equations. Using a Lénard--Bernstein collision operator, these take Fokker--Planck-like forms \cite{Fokker_1914, Planck_1917} wherein pseudo-particles in the numerical model obey the neoclassical transport equations, with particle-independent Brownian drift terms. This offers a rigorous methodology for incorporating collisions into the particle transport model, without coupling the equations of motions for each particle.
        
        Works by Chen, Chacón et al. \cite{Chen_Chacón_Barnes_2011, Chacón_Chen_Barnes_2013, Chen_Chacón_2014, Chen_Chacón_2015} have developed structure-preserving particle pushers for neoclassical transport in the Vlasov equations, derived from Crank--Nicolson integrators. We show these too can can derive from a FET interpretation, similarly offering potential extensions to higher-order-in-time particle pushers. The FET formulation is used also to consider how the stochastic drift terms can be incorporated into the pushers. Stochastic gyrokinetic expansions are also discussed.

        Different options for the numerical implementation of these schemes are considered.

        Due to the efficacy of FET in the development of SP timesteppers for both the fluid and kinetic component, we hope this approach will prove effective in the future for developing SP timesteppers for the full hybrid model. We hope this will give us the opportunity to incorporate previously inaccessible kinetic effects into the highly effective, modern, finite-element MHD models.
    \end{abstract}
    
    
    \newpage
    \tableofcontents
    
    
    \newpage
    \pagenumbering{arabic}
    %\linenumbers\renewcommand\thelinenumber{\color{black!50}\arabic{linenumber}}
            \input{0 - introduction/main.tex}
        \part{Research}
            \input{1 - low-noise PiC models/main.tex}
            \input{2 - kinetic component/main.tex}
            \input{3 - fluid component/main.tex}
            \input{4 - numerical implementation/main.tex}
        \part{Project Overview}
            \input{5 - research plan/main.tex}
            \input{6 - summary/main.tex}
    
    
    %\section{}
    \newpage
    \pagenumbering{gobble}
        \printbibliography


    \newpage
    \pagenumbering{roman}
    \appendix
        \part{Appendices}
            \input{8 - Hilbert complexes/main.tex}
            \input{9 - weak conservation proofs/main.tex}
\end{document}


\title{\BA{Title in Progress...}}
\author{Boris Andrews}
\affil{Mathematical Institute, University of Oxford}
\date{\today}


\begin{document}
    \pagenumbering{gobble}
    \maketitle
    
    
    \begin{abstract}
        Magnetic confinement reactors---in particular tokamaks---offer one of the most promising options for achieving practical nuclear fusion, with the potential to provide virtually limitless, clean energy. The theoretical and numerical modeling of tokamak plasmas is simultaneously an essential component of effective reactor design, and a great research barrier. Tokamak operational conditions exhibit comparatively low Knudsen numbers. Kinetic effects, including kinetic waves and instabilities, Landau damping, bump-on-tail instabilities and more, are therefore highly influential in tokamak plasma dynamics. Purely fluid models are inherently incapable of capturing these effects, whereas the high dimensionality in purely kinetic models render them practically intractable for most relevant purposes.

        We consider a $\delta\!f$ decomposition model, with a macroscopic fluid background and microscopic kinetic correction, both fully coupled to each other. A similar manner of discretization is proposed to that used in the recent \texttt{STRUPHY} code \cite{Holderied_Possanner_Wang_2021, Holderied_2022, Li_et_al_2023} with a finite-element model for the background and a pseudo-particle/PiC model for the correction.

        The fluid background satisfies the full, non-linear, resistive, compressible, Hall MHD equations. \cite{Laakmann_Hu_Farrell_2022} introduces finite-element(-in-space) implicit timesteppers for the incompressible analogue to this system with structure-preserving (SP) properties in the ideal case, alongside parameter-robust preconditioners. We show that these timesteppers can derive from a finite-element-in-time (FET) (and finite-element-in-space) interpretation. The benefits of this reformulation are discussed, including the derivation of timesteppers that are higher order in time, and the quantifiable dissipative SP properties in the non-ideal, resistive case.
        
        We discuss possible options for extending this FET approach to timesteppers for the compressible case.

        The kinetic corrections satisfy linearized Boltzmann equations. Using a Lénard--Bernstein collision operator, these take Fokker--Planck-like forms \cite{Fokker_1914, Planck_1917} wherein pseudo-particles in the numerical model obey the neoclassical transport equations, with particle-independent Brownian drift terms. This offers a rigorous methodology for incorporating collisions into the particle transport model, without coupling the equations of motions for each particle.
        
        Works by Chen, Chacón et al. \cite{Chen_Chacón_Barnes_2011, Chacón_Chen_Barnes_2013, Chen_Chacón_2014, Chen_Chacón_2015} have developed structure-preserving particle pushers for neoclassical transport in the Vlasov equations, derived from Crank--Nicolson integrators. We show these too can can derive from a FET interpretation, similarly offering potential extensions to higher-order-in-time particle pushers. The FET formulation is used also to consider how the stochastic drift terms can be incorporated into the pushers. Stochastic gyrokinetic expansions are also discussed.

        Different options for the numerical implementation of these schemes are considered.

        Due to the efficacy of FET in the development of SP timesteppers for both the fluid and kinetic component, we hope this approach will prove effective in the future for developing SP timesteppers for the full hybrid model. We hope this will give us the opportunity to incorporate previously inaccessible kinetic effects into the highly effective, modern, finite-element MHD models.
    \end{abstract}
    
    
    \newpage
    \tableofcontents
    
    
    \newpage
    \pagenumbering{arabic}
    %\linenumbers\renewcommand\thelinenumber{\color{black!50}\arabic{linenumber}}
            \documentclass[12pt, a4paper]{report}

\input{template/main.tex}

\title{\BA{Title in Progress...}}
\author{Boris Andrews}
\affil{Mathematical Institute, University of Oxford}
\date{\today}


\begin{document}
    \pagenumbering{gobble}
    \maketitle
    
    
    \begin{abstract}
        Magnetic confinement reactors---in particular tokamaks---offer one of the most promising options for achieving practical nuclear fusion, with the potential to provide virtually limitless, clean energy. The theoretical and numerical modeling of tokamak plasmas is simultaneously an essential component of effective reactor design, and a great research barrier. Tokamak operational conditions exhibit comparatively low Knudsen numbers. Kinetic effects, including kinetic waves and instabilities, Landau damping, bump-on-tail instabilities and more, are therefore highly influential in tokamak plasma dynamics. Purely fluid models are inherently incapable of capturing these effects, whereas the high dimensionality in purely kinetic models render them practically intractable for most relevant purposes.

        We consider a $\delta\!f$ decomposition model, with a macroscopic fluid background and microscopic kinetic correction, both fully coupled to each other. A similar manner of discretization is proposed to that used in the recent \texttt{STRUPHY} code \cite{Holderied_Possanner_Wang_2021, Holderied_2022, Li_et_al_2023} with a finite-element model for the background and a pseudo-particle/PiC model for the correction.

        The fluid background satisfies the full, non-linear, resistive, compressible, Hall MHD equations. \cite{Laakmann_Hu_Farrell_2022} introduces finite-element(-in-space) implicit timesteppers for the incompressible analogue to this system with structure-preserving (SP) properties in the ideal case, alongside parameter-robust preconditioners. We show that these timesteppers can derive from a finite-element-in-time (FET) (and finite-element-in-space) interpretation. The benefits of this reformulation are discussed, including the derivation of timesteppers that are higher order in time, and the quantifiable dissipative SP properties in the non-ideal, resistive case.
        
        We discuss possible options for extending this FET approach to timesteppers for the compressible case.

        The kinetic corrections satisfy linearized Boltzmann equations. Using a Lénard--Bernstein collision operator, these take Fokker--Planck-like forms \cite{Fokker_1914, Planck_1917} wherein pseudo-particles in the numerical model obey the neoclassical transport equations, with particle-independent Brownian drift terms. This offers a rigorous methodology for incorporating collisions into the particle transport model, without coupling the equations of motions for each particle.
        
        Works by Chen, Chacón et al. \cite{Chen_Chacón_Barnes_2011, Chacón_Chen_Barnes_2013, Chen_Chacón_2014, Chen_Chacón_2015} have developed structure-preserving particle pushers for neoclassical transport in the Vlasov equations, derived from Crank--Nicolson integrators. We show these too can can derive from a FET interpretation, similarly offering potential extensions to higher-order-in-time particle pushers. The FET formulation is used also to consider how the stochastic drift terms can be incorporated into the pushers. Stochastic gyrokinetic expansions are also discussed.

        Different options for the numerical implementation of these schemes are considered.

        Due to the efficacy of FET in the development of SP timesteppers for both the fluid and kinetic component, we hope this approach will prove effective in the future for developing SP timesteppers for the full hybrid model. We hope this will give us the opportunity to incorporate previously inaccessible kinetic effects into the highly effective, modern, finite-element MHD models.
    \end{abstract}
    
    
    \newpage
    \tableofcontents
    
    
    \newpage
    \pagenumbering{arabic}
    %\linenumbers\renewcommand\thelinenumber{\color{black!50}\arabic{linenumber}}
            \input{0 - introduction/main.tex}
        \part{Research}
            \input{1 - low-noise PiC models/main.tex}
            \input{2 - kinetic component/main.tex}
            \input{3 - fluid component/main.tex}
            \input{4 - numerical implementation/main.tex}
        \part{Project Overview}
            \input{5 - research plan/main.tex}
            \input{6 - summary/main.tex}
    
    
    %\section{}
    \newpage
    \pagenumbering{gobble}
        \printbibliography


    \newpage
    \pagenumbering{roman}
    \appendix
        \part{Appendices}
            \input{8 - Hilbert complexes/main.tex}
            \input{9 - weak conservation proofs/main.tex}
\end{document}

        \part{Research}
            \documentclass[12pt, a4paper]{report}

\input{template/main.tex}

\title{\BA{Title in Progress...}}
\author{Boris Andrews}
\affil{Mathematical Institute, University of Oxford}
\date{\today}


\begin{document}
    \pagenumbering{gobble}
    \maketitle
    
    
    \begin{abstract}
        Magnetic confinement reactors---in particular tokamaks---offer one of the most promising options for achieving practical nuclear fusion, with the potential to provide virtually limitless, clean energy. The theoretical and numerical modeling of tokamak plasmas is simultaneously an essential component of effective reactor design, and a great research barrier. Tokamak operational conditions exhibit comparatively low Knudsen numbers. Kinetic effects, including kinetic waves and instabilities, Landau damping, bump-on-tail instabilities and more, are therefore highly influential in tokamak plasma dynamics. Purely fluid models are inherently incapable of capturing these effects, whereas the high dimensionality in purely kinetic models render them practically intractable for most relevant purposes.

        We consider a $\delta\!f$ decomposition model, with a macroscopic fluid background and microscopic kinetic correction, both fully coupled to each other. A similar manner of discretization is proposed to that used in the recent \texttt{STRUPHY} code \cite{Holderied_Possanner_Wang_2021, Holderied_2022, Li_et_al_2023} with a finite-element model for the background and a pseudo-particle/PiC model for the correction.

        The fluid background satisfies the full, non-linear, resistive, compressible, Hall MHD equations. \cite{Laakmann_Hu_Farrell_2022} introduces finite-element(-in-space) implicit timesteppers for the incompressible analogue to this system with structure-preserving (SP) properties in the ideal case, alongside parameter-robust preconditioners. We show that these timesteppers can derive from a finite-element-in-time (FET) (and finite-element-in-space) interpretation. The benefits of this reformulation are discussed, including the derivation of timesteppers that are higher order in time, and the quantifiable dissipative SP properties in the non-ideal, resistive case.
        
        We discuss possible options for extending this FET approach to timesteppers for the compressible case.

        The kinetic corrections satisfy linearized Boltzmann equations. Using a Lénard--Bernstein collision operator, these take Fokker--Planck-like forms \cite{Fokker_1914, Planck_1917} wherein pseudo-particles in the numerical model obey the neoclassical transport equations, with particle-independent Brownian drift terms. This offers a rigorous methodology for incorporating collisions into the particle transport model, without coupling the equations of motions for each particle.
        
        Works by Chen, Chacón et al. \cite{Chen_Chacón_Barnes_2011, Chacón_Chen_Barnes_2013, Chen_Chacón_2014, Chen_Chacón_2015} have developed structure-preserving particle pushers for neoclassical transport in the Vlasov equations, derived from Crank--Nicolson integrators. We show these too can can derive from a FET interpretation, similarly offering potential extensions to higher-order-in-time particle pushers. The FET formulation is used also to consider how the stochastic drift terms can be incorporated into the pushers. Stochastic gyrokinetic expansions are also discussed.

        Different options for the numerical implementation of these schemes are considered.

        Due to the efficacy of FET in the development of SP timesteppers for both the fluid and kinetic component, we hope this approach will prove effective in the future for developing SP timesteppers for the full hybrid model. We hope this will give us the opportunity to incorporate previously inaccessible kinetic effects into the highly effective, modern, finite-element MHD models.
    \end{abstract}
    
    
    \newpage
    \tableofcontents
    
    
    \newpage
    \pagenumbering{arabic}
    %\linenumbers\renewcommand\thelinenumber{\color{black!50}\arabic{linenumber}}
            \input{0 - introduction/main.tex}
        \part{Research}
            \input{1 - low-noise PiC models/main.tex}
            \input{2 - kinetic component/main.tex}
            \input{3 - fluid component/main.tex}
            \input{4 - numerical implementation/main.tex}
        \part{Project Overview}
            \input{5 - research plan/main.tex}
            \input{6 - summary/main.tex}
    
    
    %\section{}
    \newpage
    \pagenumbering{gobble}
        \printbibliography


    \newpage
    \pagenumbering{roman}
    \appendix
        \part{Appendices}
            \input{8 - Hilbert complexes/main.tex}
            \input{9 - weak conservation proofs/main.tex}
\end{document}

            \documentclass[12pt, a4paper]{report}

\input{template/main.tex}

\title{\BA{Title in Progress...}}
\author{Boris Andrews}
\affil{Mathematical Institute, University of Oxford}
\date{\today}


\begin{document}
    \pagenumbering{gobble}
    \maketitle
    
    
    \begin{abstract}
        Magnetic confinement reactors---in particular tokamaks---offer one of the most promising options for achieving practical nuclear fusion, with the potential to provide virtually limitless, clean energy. The theoretical and numerical modeling of tokamak plasmas is simultaneously an essential component of effective reactor design, and a great research barrier. Tokamak operational conditions exhibit comparatively low Knudsen numbers. Kinetic effects, including kinetic waves and instabilities, Landau damping, bump-on-tail instabilities and more, are therefore highly influential in tokamak plasma dynamics. Purely fluid models are inherently incapable of capturing these effects, whereas the high dimensionality in purely kinetic models render them practically intractable for most relevant purposes.

        We consider a $\delta\!f$ decomposition model, with a macroscopic fluid background and microscopic kinetic correction, both fully coupled to each other. A similar manner of discretization is proposed to that used in the recent \texttt{STRUPHY} code \cite{Holderied_Possanner_Wang_2021, Holderied_2022, Li_et_al_2023} with a finite-element model for the background and a pseudo-particle/PiC model for the correction.

        The fluid background satisfies the full, non-linear, resistive, compressible, Hall MHD equations. \cite{Laakmann_Hu_Farrell_2022} introduces finite-element(-in-space) implicit timesteppers for the incompressible analogue to this system with structure-preserving (SP) properties in the ideal case, alongside parameter-robust preconditioners. We show that these timesteppers can derive from a finite-element-in-time (FET) (and finite-element-in-space) interpretation. The benefits of this reformulation are discussed, including the derivation of timesteppers that are higher order in time, and the quantifiable dissipative SP properties in the non-ideal, resistive case.
        
        We discuss possible options for extending this FET approach to timesteppers for the compressible case.

        The kinetic corrections satisfy linearized Boltzmann equations. Using a Lénard--Bernstein collision operator, these take Fokker--Planck-like forms \cite{Fokker_1914, Planck_1917} wherein pseudo-particles in the numerical model obey the neoclassical transport equations, with particle-independent Brownian drift terms. This offers a rigorous methodology for incorporating collisions into the particle transport model, without coupling the equations of motions for each particle.
        
        Works by Chen, Chacón et al. \cite{Chen_Chacón_Barnes_2011, Chacón_Chen_Barnes_2013, Chen_Chacón_2014, Chen_Chacón_2015} have developed structure-preserving particle pushers for neoclassical transport in the Vlasov equations, derived from Crank--Nicolson integrators. We show these too can can derive from a FET interpretation, similarly offering potential extensions to higher-order-in-time particle pushers. The FET formulation is used also to consider how the stochastic drift terms can be incorporated into the pushers. Stochastic gyrokinetic expansions are also discussed.

        Different options for the numerical implementation of these schemes are considered.

        Due to the efficacy of FET in the development of SP timesteppers for both the fluid and kinetic component, we hope this approach will prove effective in the future for developing SP timesteppers for the full hybrid model. We hope this will give us the opportunity to incorporate previously inaccessible kinetic effects into the highly effective, modern, finite-element MHD models.
    \end{abstract}
    
    
    \newpage
    \tableofcontents
    
    
    \newpage
    \pagenumbering{arabic}
    %\linenumbers\renewcommand\thelinenumber{\color{black!50}\arabic{linenumber}}
            \input{0 - introduction/main.tex}
        \part{Research}
            \input{1 - low-noise PiC models/main.tex}
            \input{2 - kinetic component/main.tex}
            \input{3 - fluid component/main.tex}
            \input{4 - numerical implementation/main.tex}
        \part{Project Overview}
            \input{5 - research plan/main.tex}
            \input{6 - summary/main.tex}
    
    
    %\section{}
    \newpage
    \pagenumbering{gobble}
        \printbibliography


    \newpage
    \pagenumbering{roman}
    \appendix
        \part{Appendices}
            \input{8 - Hilbert complexes/main.tex}
            \input{9 - weak conservation proofs/main.tex}
\end{document}

            \documentclass[12pt, a4paper]{report}

\input{template/main.tex}

\title{\BA{Title in Progress...}}
\author{Boris Andrews}
\affil{Mathematical Institute, University of Oxford}
\date{\today}


\begin{document}
    \pagenumbering{gobble}
    \maketitle
    
    
    \begin{abstract}
        Magnetic confinement reactors---in particular tokamaks---offer one of the most promising options for achieving practical nuclear fusion, with the potential to provide virtually limitless, clean energy. The theoretical and numerical modeling of tokamak plasmas is simultaneously an essential component of effective reactor design, and a great research barrier. Tokamak operational conditions exhibit comparatively low Knudsen numbers. Kinetic effects, including kinetic waves and instabilities, Landau damping, bump-on-tail instabilities and more, are therefore highly influential in tokamak plasma dynamics. Purely fluid models are inherently incapable of capturing these effects, whereas the high dimensionality in purely kinetic models render them practically intractable for most relevant purposes.

        We consider a $\delta\!f$ decomposition model, with a macroscopic fluid background and microscopic kinetic correction, both fully coupled to each other. A similar manner of discretization is proposed to that used in the recent \texttt{STRUPHY} code \cite{Holderied_Possanner_Wang_2021, Holderied_2022, Li_et_al_2023} with a finite-element model for the background and a pseudo-particle/PiC model for the correction.

        The fluid background satisfies the full, non-linear, resistive, compressible, Hall MHD equations. \cite{Laakmann_Hu_Farrell_2022} introduces finite-element(-in-space) implicit timesteppers for the incompressible analogue to this system with structure-preserving (SP) properties in the ideal case, alongside parameter-robust preconditioners. We show that these timesteppers can derive from a finite-element-in-time (FET) (and finite-element-in-space) interpretation. The benefits of this reformulation are discussed, including the derivation of timesteppers that are higher order in time, and the quantifiable dissipative SP properties in the non-ideal, resistive case.
        
        We discuss possible options for extending this FET approach to timesteppers for the compressible case.

        The kinetic corrections satisfy linearized Boltzmann equations. Using a Lénard--Bernstein collision operator, these take Fokker--Planck-like forms \cite{Fokker_1914, Planck_1917} wherein pseudo-particles in the numerical model obey the neoclassical transport equations, with particle-independent Brownian drift terms. This offers a rigorous methodology for incorporating collisions into the particle transport model, without coupling the equations of motions for each particle.
        
        Works by Chen, Chacón et al. \cite{Chen_Chacón_Barnes_2011, Chacón_Chen_Barnes_2013, Chen_Chacón_2014, Chen_Chacón_2015} have developed structure-preserving particle pushers for neoclassical transport in the Vlasov equations, derived from Crank--Nicolson integrators. We show these too can can derive from a FET interpretation, similarly offering potential extensions to higher-order-in-time particle pushers. The FET formulation is used also to consider how the stochastic drift terms can be incorporated into the pushers. Stochastic gyrokinetic expansions are also discussed.

        Different options for the numerical implementation of these schemes are considered.

        Due to the efficacy of FET in the development of SP timesteppers for both the fluid and kinetic component, we hope this approach will prove effective in the future for developing SP timesteppers for the full hybrid model. We hope this will give us the opportunity to incorporate previously inaccessible kinetic effects into the highly effective, modern, finite-element MHD models.
    \end{abstract}
    
    
    \newpage
    \tableofcontents
    
    
    \newpage
    \pagenumbering{arabic}
    %\linenumbers\renewcommand\thelinenumber{\color{black!50}\arabic{linenumber}}
            \input{0 - introduction/main.tex}
        \part{Research}
            \input{1 - low-noise PiC models/main.tex}
            \input{2 - kinetic component/main.tex}
            \input{3 - fluid component/main.tex}
            \input{4 - numerical implementation/main.tex}
        \part{Project Overview}
            \input{5 - research plan/main.tex}
            \input{6 - summary/main.tex}
    
    
    %\section{}
    \newpage
    \pagenumbering{gobble}
        \printbibliography


    \newpage
    \pagenumbering{roman}
    \appendix
        \part{Appendices}
            \input{8 - Hilbert complexes/main.tex}
            \input{9 - weak conservation proofs/main.tex}
\end{document}

            \documentclass[12pt, a4paper]{report}

\input{template/main.tex}

\title{\BA{Title in Progress...}}
\author{Boris Andrews}
\affil{Mathematical Institute, University of Oxford}
\date{\today}


\begin{document}
    \pagenumbering{gobble}
    \maketitle
    
    
    \begin{abstract}
        Magnetic confinement reactors---in particular tokamaks---offer one of the most promising options for achieving practical nuclear fusion, with the potential to provide virtually limitless, clean energy. The theoretical and numerical modeling of tokamak plasmas is simultaneously an essential component of effective reactor design, and a great research barrier. Tokamak operational conditions exhibit comparatively low Knudsen numbers. Kinetic effects, including kinetic waves and instabilities, Landau damping, bump-on-tail instabilities and more, are therefore highly influential in tokamak plasma dynamics. Purely fluid models are inherently incapable of capturing these effects, whereas the high dimensionality in purely kinetic models render them practically intractable for most relevant purposes.

        We consider a $\delta\!f$ decomposition model, with a macroscopic fluid background and microscopic kinetic correction, both fully coupled to each other. A similar manner of discretization is proposed to that used in the recent \texttt{STRUPHY} code \cite{Holderied_Possanner_Wang_2021, Holderied_2022, Li_et_al_2023} with a finite-element model for the background and a pseudo-particle/PiC model for the correction.

        The fluid background satisfies the full, non-linear, resistive, compressible, Hall MHD equations. \cite{Laakmann_Hu_Farrell_2022} introduces finite-element(-in-space) implicit timesteppers for the incompressible analogue to this system with structure-preserving (SP) properties in the ideal case, alongside parameter-robust preconditioners. We show that these timesteppers can derive from a finite-element-in-time (FET) (and finite-element-in-space) interpretation. The benefits of this reformulation are discussed, including the derivation of timesteppers that are higher order in time, and the quantifiable dissipative SP properties in the non-ideal, resistive case.
        
        We discuss possible options for extending this FET approach to timesteppers for the compressible case.

        The kinetic corrections satisfy linearized Boltzmann equations. Using a Lénard--Bernstein collision operator, these take Fokker--Planck-like forms \cite{Fokker_1914, Planck_1917} wherein pseudo-particles in the numerical model obey the neoclassical transport equations, with particle-independent Brownian drift terms. This offers a rigorous methodology for incorporating collisions into the particle transport model, without coupling the equations of motions for each particle.
        
        Works by Chen, Chacón et al. \cite{Chen_Chacón_Barnes_2011, Chacón_Chen_Barnes_2013, Chen_Chacón_2014, Chen_Chacón_2015} have developed structure-preserving particle pushers for neoclassical transport in the Vlasov equations, derived from Crank--Nicolson integrators. We show these too can can derive from a FET interpretation, similarly offering potential extensions to higher-order-in-time particle pushers. The FET formulation is used also to consider how the stochastic drift terms can be incorporated into the pushers. Stochastic gyrokinetic expansions are also discussed.

        Different options for the numerical implementation of these schemes are considered.

        Due to the efficacy of FET in the development of SP timesteppers for both the fluid and kinetic component, we hope this approach will prove effective in the future for developing SP timesteppers for the full hybrid model. We hope this will give us the opportunity to incorporate previously inaccessible kinetic effects into the highly effective, modern, finite-element MHD models.
    \end{abstract}
    
    
    \newpage
    \tableofcontents
    
    
    \newpage
    \pagenumbering{arabic}
    %\linenumbers\renewcommand\thelinenumber{\color{black!50}\arabic{linenumber}}
            \input{0 - introduction/main.tex}
        \part{Research}
            \input{1 - low-noise PiC models/main.tex}
            \input{2 - kinetic component/main.tex}
            \input{3 - fluid component/main.tex}
            \input{4 - numerical implementation/main.tex}
        \part{Project Overview}
            \input{5 - research plan/main.tex}
            \input{6 - summary/main.tex}
    
    
    %\section{}
    \newpage
    \pagenumbering{gobble}
        \printbibliography


    \newpage
    \pagenumbering{roman}
    \appendix
        \part{Appendices}
            \input{8 - Hilbert complexes/main.tex}
            \input{9 - weak conservation proofs/main.tex}
\end{document}

        \part{Project Overview}
            \documentclass[12pt, a4paper]{report}

\input{template/main.tex}

\title{\BA{Title in Progress...}}
\author{Boris Andrews}
\affil{Mathematical Institute, University of Oxford}
\date{\today}


\begin{document}
    \pagenumbering{gobble}
    \maketitle
    
    
    \begin{abstract}
        Magnetic confinement reactors---in particular tokamaks---offer one of the most promising options for achieving practical nuclear fusion, with the potential to provide virtually limitless, clean energy. The theoretical and numerical modeling of tokamak plasmas is simultaneously an essential component of effective reactor design, and a great research barrier. Tokamak operational conditions exhibit comparatively low Knudsen numbers. Kinetic effects, including kinetic waves and instabilities, Landau damping, bump-on-tail instabilities and more, are therefore highly influential in tokamak plasma dynamics. Purely fluid models are inherently incapable of capturing these effects, whereas the high dimensionality in purely kinetic models render them practically intractable for most relevant purposes.

        We consider a $\delta\!f$ decomposition model, with a macroscopic fluid background and microscopic kinetic correction, both fully coupled to each other. A similar manner of discretization is proposed to that used in the recent \texttt{STRUPHY} code \cite{Holderied_Possanner_Wang_2021, Holderied_2022, Li_et_al_2023} with a finite-element model for the background and a pseudo-particle/PiC model for the correction.

        The fluid background satisfies the full, non-linear, resistive, compressible, Hall MHD equations. \cite{Laakmann_Hu_Farrell_2022} introduces finite-element(-in-space) implicit timesteppers for the incompressible analogue to this system with structure-preserving (SP) properties in the ideal case, alongside parameter-robust preconditioners. We show that these timesteppers can derive from a finite-element-in-time (FET) (and finite-element-in-space) interpretation. The benefits of this reformulation are discussed, including the derivation of timesteppers that are higher order in time, and the quantifiable dissipative SP properties in the non-ideal, resistive case.
        
        We discuss possible options for extending this FET approach to timesteppers for the compressible case.

        The kinetic corrections satisfy linearized Boltzmann equations. Using a Lénard--Bernstein collision operator, these take Fokker--Planck-like forms \cite{Fokker_1914, Planck_1917} wherein pseudo-particles in the numerical model obey the neoclassical transport equations, with particle-independent Brownian drift terms. This offers a rigorous methodology for incorporating collisions into the particle transport model, without coupling the equations of motions for each particle.
        
        Works by Chen, Chacón et al. \cite{Chen_Chacón_Barnes_2011, Chacón_Chen_Barnes_2013, Chen_Chacón_2014, Chen_Chacón_2015} have developed structure-preserving particle pushers for neoclassical transport in the Vlasov equations, derived from Crank--Nicolson integrators. We show these too can can derive from a FET interpretation, similarly offering potential extensions to higher-order-in-time particle pushers. The FET formulation is used also to consider how the stochastic drift terms can be incorporated into the pushers. Stochastic gyrokinetic expansions are also discussed.

        Different options for the numerical implementation of these schemes are considered.

        Due to the efficacy of FET in the development of SP timesteppers for both the fluid and kinetic component, we hope this approach will prove effective in the future for developing SP timesteppers for the full hybrid model. We hope this will give us the opportunity to incorporate previously inaccessible kinetic effects into the highly effective, modern, finite-element MHD models.
    \end{abstract}
    
    
    \newpage
    \tableofcontents
    
    
    \newpage
    \pagenumbering{arabic}
    %\linenumbers\renewcommand\thelinenumber{\color{black!50}\arabic{linenumber}}
            \input{0 - introduction/main.tex}
        \part{Research}
            \input{1 - low-noise PiC models/main.tex}
            \input{2 - kinetic component/main.tex}
            \input{3 - fluid component/main.tex}
            \input{4 - numerical implementation/main.tex}
        \part{Project Overview}
            \input{5 - research plan/main.tex}
            \input{6 - summary/main.tex}
    
    
    %\section{}
    \newpage
    \pagenumbering{gobble}
        \printbibliography


    \newpage
    \pagenumbering{roman}
    \appendix
        \part{Appendices}
            \input{8 - Hilbert complexes/main.tex}
            \input{9 - weak conservation proofs/main.tex}
\end{document}

            \documentclass[12pt, a4paper]{report}

\input{template/main.tex}

\title{\BA{Title in Progress...}}
\author{Boris Andrews}
\affil{Mathematical Institute, University of Oxford}
\date{\today}


\begin{document}
    \pagenumbering{gobble}
    \maketitle
    
    
    \begin{abstract}
        Magnetic confinement reactors---in particular tokamaks---offer one of the most promising options for achieving practical nuclear fusion, with the potential to provide virtually limitless, clean energy. The theoretical and numerical modeling of tokamak plasmas is simultaneously an essential component of effective reactor design, and a great research barrier. Tokamak operational conditions exhibit comparatively low Knudsen numbers. Kinetic effects, including kinetic waves and instabilities, Landau damping, bump-on-tail instabilities and more, are therefore highly influential in tokamak plasma dynamics. Purely fluid models are inherently incapable of capturing these effects, whereas the high dimensionality in purely kinetic models render them practically intractable for most relevant purposes.

        We consider a $\delta\!f$ decomposition model, with a macroscopic fluid background and microscopic kinetic correction, both fully coupled to each other. A similar manner of discretization is proposed to that used in the recent \texttt{STRUPHY} code \cite{Holderied_Possanner_Wang_2021, Holderied_2022, Li_et_al_2023} with a finite-element model for the background and a pseudo-particle/PiC model for the correction.

        The fluid background satisfies the full, non-linear, resistive, compressible, Hall MHD equations. \cite{Laakmann_Hu_Farrell_2022} introduces finite-element(-in-space) implicit timesteppers for the incompressible analogue to this system with structure-preserving (SP) properties in the ideal case, alongside parameter-robust preconditioners. We show that these timesteppers can derive from a finite-element-in-time (FET) (and finite-element-in-space) interpretation. The benefits of this reformulation are discussed, including the derivation of timesteppers that are higher order in time, and the quantifiable dissipative SP properties in the non-ideal, resistive case.
        
        We discuss possible options for extending this FET approach to timesteppers for the compressible case.

        The kinetic corrections satisfy linearized Boltzmann equations. Using a Lénard--Bernstein collision operator, these take Fokker--Planck-like forms \cite{Fokker_1914, Planck_1917} wherein pseudo-particles in the numerical model obey the neoclassical transport equations, with particle-independent Brownian drift terms. This offers a rigorous methodology for incorporating collisions into the particle transport model, without coupling the equations of motions for each particle.
        
        Works by Chen, Chacón et al. \cite{Chen_Chacón_Barnes_2011, Chacón_Chen_Barnes_2013, Chen_Chacón_2014, Chen_Chacón_2015} have developed structure-preserving particle pushers for neoclassical transport in the Vlasov equations, derived from Crank--Nicolson integrators. We show these too can can derive from a FET interpretation, similarly offering potential extensions to higher-order-in-time particle pushers. The FET formulation is used also to consider how the stochastic drift terms can be incorporated into the pushers. Stochastic gyrokinetic expansions are also discussed.

        Different options for the numerical implementation of these schemes are considered.

        Due to the efficacy of FET in the development of SP timesteppers for both the fluid and kinetic component, we hope this approach will prove effective in the future for developing SP timesteppers for the full hybrid model. We hope this will give us the opportunity to incorporate previously inaccessible kinetic effects into the highly effective, modern, finite-element MHD models.
    \end{abstract}
    
    
    \newpage
    \tableofcontents
    
    
    \newpage
    \pagenumbering{arabic}
    %\linenumbers\renewcommand\thelinenumber{\color{black!50}\arabic{linenumber}}
            \input{0 - introduction/main.tex}
        \part{Research}
            \input{1 - low-noise PiC models/main.tex}
            \input{2 - kinetic component/main.tex}
            \input{3 - fluid component/main.tex}
            \input{4 - numerical implementation/main.tex}
        \part{Project Overview}
            \input{5 - research plan/main.tex}
            \input{6 - summary/main.tex}
    
    
    %\section{}
    \newpage
    \pagenumbering{gobble}
        \printbibliography


    \newpage
    \pagenumbering{roman}
    \appendix
        \part{Appendices}
            \input{8 - Hilbert complexes/main.tex}
            \input{9 - weak conservation proofs/main.tex}
\end{document}

    
    
    %\section{}
    \newpage
    \pagenumbering{gobble}
        \printbibliography


    \newpage
    \pagenumbering{roman}
    \appendix
        \part{Appendices}
            \documentclass[12pt, a4paper]{report}

\input{template/main.tex}

\title{\BA{Title in Progress...}}
\author{Boris Andrews}
\affil{Mathematical Institute, University of Oxford}
\date{\today}


\begin{document}
    \pagenumbering{gobble}
    \maketitle
    
    
    \begin{abstract}
        Magnetic confinement reactors---in particular tokamaks---offer one of the most promising options for achieving practical nuclear fusion, with the potential to provide virtually limitless, clean energy. The theoretical and numerical modeling of tokamak plasmas is simultaneously an essential component of effective reactor design, and a great research barrier. Tokamak operational conditions exhibit comparatively low Knudsen numbers. Kinetic effects, including kinetic waves and instabilities, Landau damping, bump-on-tail instabilities and more, are therefore highly influential in tokamak plasma dynamics. Purely fluid models are inherently incapable of capturing these effects, whereas the high dimensionality in purely kinetic models render them practically intractable for most relevant purposes.

        We consider a $\delta\!f$ decomposition model, with a macroscopic fluid background and microscopic kinetic correction, both fully coupled to each other. A similar manner of discretization is proposed to that used in the recent \texttt{STRUPHY} code \cite{Holderied_Possanner_Wang_2021, Holderied_2022, Li_et_al_2023} with a finite-element model for the background and a pseudo-particle/PiC model for the correction.

        The fluid background satisfies the full, non-linear, resistive, compressible, Hall MHD equations. \cite{Laakmann_Hu_Farrell_2022} introduces finite-element(-in-space) implicit timesteppers for the incompressible analogue to this system with structure-preserving (SP) properties in the ideal case, alongside parameter-robust preconditioners. We show that these timesteppers can derive from a finite-element-in-time (FET) (and finite-element-in-space) interpretation. The benefits of this reformulation are discussed, including the derivation of timesteppers that are higher order in time, and the quantifiable dissipative SP properties in the non-ideal, resistive case.
        
        We discuss possible options for extending this FET approach to timesteppers for the compressible case.

        The kinetic corrections satisfy linearized Boltzmann equations. Using a Lénard--Bernstein collision operator, these take Fokker--Planck-like forms \cite{Fokker_1914, Planck_1917} wherein pseudo-particles in the numerical model obey the neoclassical transport equations, with particle-independent Brownian drift terms. This offers a rigorous methodology for incorporating collisions into the particle transport model, without coupling the equations of motions for each particle.
        
        Works by Chen, Chacón et al. \cite{Chen_Chacón_Barnes_2011, Chacón_Chen_Barnes_2013, Chen_Chacón_2014, Chen_Chacón_2015} have developed structure-preserving particle pushers for neoclassical transport in the Vlasov equations, derived from Crank--Nicolson integrators. We show these too can can derive from a FET interpretation, similarly offering potential extensions to higher-order-in-time particle pushers. The FET formulation is used also to consider how the stochastic drift terms can be incorporated into the pushers. Stochastic gyrokinetic expansions are also discussed.

        Different options for the numerical implementation of these schemes are considered.

        Due to the efficacy of FET in the development of SP timesteppers for both the fluid and kinetic component, we hope this approach will prove effective in the future for developing SP timesteppers for the full hybrid model. We hope this will give us the opportunity to incorporate previously inaccessible kinetic effects into the highly effective, modern, finite-element MHD models.
    \end{abstract}
    
    
    \newpage
    \tableofcontents
    
    
    \newpage
    \pagenumbering{arabic}
    %\linenumbers\renewcommand\thelinenumber{\color{black!50}\arabic{linenumber}}
            \input{0 - introduction/main.tex}
        \part{Research}
            \input{1 - low-noise PiC models/main.tex}
            \input{2 - kinetic component/main.tex}
            \input{3 - fluid component/main.tex}
            \input{4 - numerical implementation/main.tex}
        \part{Project Overview}
            \input{5 - research plan/main.tex}
            \input{6 - summary/main.tex}
    
    
    %\section{}
    \newpage
    \pagenumbering{gobble}
        \printbibliography


    \newpage
    \pagenumbering{roman}
    \appendix
        \part{Appendices}
            \input{8 - Hilbert complexes/main.tex}
            \input{9 - weak conservation proofs/main.tex}
\end{document}

            \documentclass[12pt, a4paper]{report}

\input{template/main.tex}

\title{\BA{Title in Progress...}}
\author{Boris Andrews}
\affil{Mathematical Institute, University of Oxford}
\date{\today}


\begin{document}
    \pagenumbering{gobble}
    \maketitle
    
    
    \begin{abstract}
        Magnetic confinement reactors---in particular tokamaks---offer one of the most promising options for achieving practical nuclear fusion, with the potential to provide virtually limitless, clean energy. The theoretical and numerical modeling of tokamak plasmas is simultaneously an essential component of effective reactor design, and a great research barrier. Tokamak operational conditions exhibit comparatively low Knudsen numbers. Kinetic effects, including kinetic waves and instabilities, Landau damping, bump-on-tail instabilities and more, are therefore highly influential in tokamak plasma dynamics. Purely fluid models are inherently incapable of capturing these effects, whereas the high dimensionality in purely kinetic models render them practically intractable for most relevant purposes.

        We consider a $\delta\!f$ decomposition model, with a macroscopic fluid background and microscopic kinetic correction, both fully coupled to each other. A similar manner of discretization is proposed to that used in the recent \texttt{STRUPHY} code \cite{Holderied_Possanner_Wang_2021, Holderied_2022, Li_et_al_2023} with a finite-element model for the background and a pseudo-particle/PiC model for the correction.

        The fluid background satisfies the full, non-linear, resistive, compressible, Hall MHD equations. \cite{Laakmann_Hu_Farrell_2022} introduces finite-element(-in-space) implicit timesteppers for the incompressible analogue to this system with structure-preserving (SP) properties in the ideal case, alongside parameter-robust preconditioners. We show that these timesteppers can derive from a finite-element-in-time (FET) (and finite-element-in-space) interpretation. The benefits of this reformulation are discussed, including the derivation of timesteppers that are higher order in time, and the quantifiable dissipative SP properties in the non-ideal, resistive case.
        
        We discuss possible options for extending this FET approach to timesteppers for the compressible case.

        The kinetic corrections satisfy linearized Boltzmann equations. Using a Lénard--Bernstein collision operator, these take Fokker--Planck-like forms \cite{Fokker_1914, Planck_1917} wherein pseudo-particles in the numerical model obey the neoclassical transport equations, with particle-independent Brownian drift terms. This offers a rigorous methodology for incorporating collisions into the particle transport model, without coupling the equations of motions for each particle.
        
        Works by Chen, Chacón et al. \cite{Chen_Chacón_Barnes_2011, Chacón_Chen_Barnes_2013, Chen_Chacón_2014, Chen_Chacón_2015} have developed structure-preserving particle pushers for neoclassical transport in the Vlasov equations, derived from Crank--Nicolson integrators. We show these too can can derive from a FET interpretation, similarly offering potential extensions to higher-order-in-time particle pushers. The FET formulation is used also to consider how the stochastic drift terms can be incorporated into the pushers. Stochastic gyrokinetic expansions are also discussed.

        Different options for the numerical implementation of these schemes are considered.

        Due to the efficacy of FET in the development of SP timesteppers for both the fluid and kinetic component, we hope this approach will prove effective in the future for developing SP timesteppers for the full hybrid model. We hope this will give us the opportunity to incorporate previously inaccessible kinetic effects into the highly effective, modern, finite-element MHD models.
    \end{abstract}
    
    
    \newpage
    \tableofcontents
    
    
    \newpage
    \pagenumbering{arabic}
    %\linenumbers\renewcommand\thelinenumber{\color{black!50}\arabic{linenumber}}
            \input{0 - introduction/main.tex}
        \part{Research}
            \input{1 - low-noise PiC models/main.tex}
            \input{2 - kinetic component/main.tex}
            \input{3 - fluid component/main.tex}
            \input{4 - numerical implementation/main.tex}
        \part{Project Overview}
            \input{5 - research plan/main.tex}
            \input{6 - summary/main.tex}
    
    
    %\section{}
    \newpage
    \pagenumbering{gobble}
        \printbibliography


    \newpage
    \pagenumbering{roman}
    \appendix
        \part{Appendices}
            \input{8 - Hilbert complexes/main.tex}
            \input{9 - weak conservation proofs/main.tex}
\end{document}

\end{document}


\title{\BA{Title in Progress...}}
\author{Boris Andrews}
\affil{Mathematical Institute, University of Oxford}
\date{\today}


\begin{document}
    \pagenumbering{gobble}
    \maketitle
    
    
    \begin{abstract}
        Magnetic confinement reactors---in particular tokamaks---offer one of the most promising options for achieving practical nuclear fusion, with the potential to provide virtually limitless, clean energy. The theoretical and numerical modeling of tokamak plasmas is simultaneously an essential component of effective reactor design, and a great research barrier. Tokamak operational conditions exhibit comparatively low Knudsen numbers. Kinetic effects, including kinetic waves and instabilities, Landau damping, bump-on-tail instabilities and more, are therefore highly influential in tokamak plasma dynamics. Purely fluid models are inherently incapable of capturing these effects, whereas the high dimensionality in purely kinetic models render them practically intractable for most relevant purposes.

        We consider a $\delta\!f$ decomposition model, with a macroscopic fluid background and microscopic kinetic correction, both fully coupled to each other. A similar manner of discretization is proposed to that used in the recent \texttt{STRUPHY} code \cite{Holderied_Possanner_Wang_2021, Holderied_2022, Li_et_al_2023} with a finite-element model for the background and a pseudo-particle/PiC model for the correction.

        The fluid background satisfies the full, non-linear, resistive, compressible, Hall MHD equations. \cite{Laakmann_Hu_Farrell_2022} introduces finite-element(-in-space) implicit timesteppers for the incompressible analogue to this system with structure-preserving (SP) properties in the ideal case, alongside parameter-robust preconditioners. We show that these timesteppers can derive from a finite-element-in-time (FET) (and finite-element-in-space) interpretation. The benefits of this reformulation are discussed, including the derivation of timesteppers that are higher order in time, and the quantifiable dissipative SP properties in the non-ideal, resistive case.
        
        We discuss possible options for extending this FET approach to timesteppers for the compressible case.

        The kinetic corrections satisfy linearized Boltzmann equations. Using a Lénard--Bernstein collision operator, these take Fokker--Planck-like forms \cite{Fokker_1914, Planck_1917} wherein pseudo-particles in the numerical model obey the neoclassical transport equations, with particle-independent Brownian drift terms. This offers a rigorous methodology for incorporating collisions into the particle transport model, without coupling the equations of motions for each particle.
        
        Works by Chen, Chacón et al. \cite{Chen_Chacón_Barnes_2011, Chacón_Chen_Barnes_2013, Chen_Chacón_2014, Chen_Chacón_2015} have developed structure-preserving particle pushers for neoclassical transport in the Vlasov equations, derived from Crank--Nicolson integrators. We show these too can can derive from a FET interpretation, similarly offering potential extensions to higher-order-in-time particle pushers. The FET formulation is used also to consider how the stochastic drift terms can be incorporated into the pushers. Stochastic gyrokinetic expansions are also discussed.

        Different options for the numerical implementation of these schemes are considered.

        Due to the efficacy of FET in the development of SP timesteppers for both the fluid and kinetic component, we hope this approach will prove effective in the future for developing SP timesteppers for the full hybrid model. We hope this will give us the opportunity to incorporate previously inaccessible kinetic effects into the highly effective, modern, finite-element MHD models.
    \end{abstract}
    
    
    \newpage
    \tableofcontents
    
    
    \newpage
    \pagenumbering{arabic}
    %\linenumbers\renewcommand\thelinenumber{\color{black!50}\arabic{linenumber}}
            \documentclass[12pt, a4paper]{report}

\documentclass[12pt, a4paper]{report}

\input{template/main.tex}

\title{\BA{Title in Progress...}}
\author{Boris Andrews}
\affil{Mathematical Institute, University of Oxford}
\date{\today}


\begin{document}
    \pagenumbering{gobble}
    \maketitle
    
    
    \begin{abstract}
        Magnetic confinement reactors---in particular tokamaks---offer one of the most promising options for achieving practical nuclear fusion, with the potential to provide virtually limitless, clean energy. The theoretical and numerical modeling of tokamak plasmas is simultaneously an essential component of effective reactor design, and a great research barrier. Tokamak operational conditions exhibit comparatively low Knudsen numbers. Kinetic effects, including kinetic waves and instabilities, Landau damping, bump-on-tail instabilities and more, are therefore highly influential in tokamak plasma dynamics. Purely fluid models are inherently incapable of capturing these effects, whereas the high dimensionality in purely kinetic models render them practically intractable for most relevant purposes.

        We consider a $\delta\!f$ decomposition model, with a macroscopic fluid background and microscopic kinetic correction, both fully coupled to each other. A similar manner of discretization is proposed to that used in the recent \texttt{STRUPHY} code \cite{Holderied_Possanner_Wang_2021, Holderied_2022, Li_et_al_2023} with a finite-element model for the background and a pseudo-particle/PiC model for the correction.

        The fluid background satisfies the full, non-linear, resistive, compressible, Hall MHD equations. \cite{Laakmann_Hu_Farrell_2022} introduces finite-element(-in-space) implicit timesteppers for the incompressible analogue to this system with structure-preserving (SP) properties in the ideal case, alongside parameter-robust preconditioners. We show that these timesteppers can derive from a finite-element-in-time (FET) (and finite-element-in-space) interpretation. The benefits of this reformulation are discussed, including the derivation of timesteppers that are higher order in time, and the quantifiable dissipative SP properties in the non-ideal, resistive case.
        
        We discuss possible options for extending this FET approach to timesteppers for the compressible case.

        The kinetic corrections satisfy linearized Boltzmann equations. Using a Lénard--Bernstein collision operator, these take Fokker--Planck-like forms \cite{Fokker_1914, Planck_1917} wherein pseudo-particles in the numerical model obey the neoclassical transport equations, with particle-independent Brownian drift terms. This offers a rigorous methodology for incorporating collisions into the particle transport model, without coupling the equations of motions for each particle.
        
        Works by Chen, Chacón et al. \cite{Chen_Chacón_Barnes_2011, Chacón_Chen_Barnes_2013, Chen_Chacón_2014, Chen_Chacón_2015} have developed structure-preserving particle pushers for neoclassical transport in the Vlasov equations, derived from Crank--Nicolson integrators. We show these too can can derive from a FET interpretation, similarly offering potential extensions to higher-order-in-time particle pushers. The FET formulation is used also to consider how the stochastic drift terms can be incorporated into the pushers. Stochastic gyrokinetic expansions are also discussed.

        Different options for the numerical implementation of these schemes are considered.

        Due to the efficacy of FET in the development of SP timesteppers for both the fluid and kinetic component, we hope this approach will prove effective in the future for developing SP timesteppers for the full hybrid model. We hope this will give us the opportunity to incorporate previously inaccessible kinetic effects into the highly effective, modern, finite-element MHD models.
    \end{abstract}
    
    
    \newpage
    \tableofcontents
    
    
    \newpage
    \pagenumbering{arabic}
    %\linenumbers\renewcommand\thelinenumber{\color{black!50}\arabic{linenumber}}
            \input{0 - introduction/main.tex}
        \part{Research}
            \input{1 - low-noise PiC models/main.tex}
            \input{2 - kinetic component/main.tex}
            \input{3 - fluid component/main.tex}
            \input{4 - numerical implementation/main.tex}
        \part{Project Overview}
            \input{5 - research plan/main.tex}
            \input{6 - summary/main.tex}
    
    
    %\section{}
    \newpage
    \pagenumbering{gobble}
        \printbibliography


    \newpage
    \pagenumbering{roman}
    \appendix
        \part{Appendices}
            \input{8 - Hilbert complexes/main.tex}
            \input{9 - weak conservation proofs/main.tex}
\end{document}


\title{\BA{Title in Progress...}}
\author{Boris Andrews}
\affil{Mathematical Institute, University of Oxford}
\date{\today}


\begin{document}
    \pagenumbering{gobble}
    \maketitle
    
    
    \begin{abstract}
        Magnetic confinement reactors---in particular tokamaks---offer one of the most promising options for achieving practical nuclear fusion, with the potential to provide virtually limitless, clean energy. The theoretical and numerical modeling of tokamak plasmas is simultaneously an essential component of effective reactor design, and a great research barrier. Tokamak operational conditions exhibit comparatively low Knudsen numbers. Kinetic effects, including kinetic waves and instabilities, Landau damping, bump-on-tail instabilities and more, are therefore highly influential in tokamak plasma dynamics. Purely fluid models are inherently incapable of capturing these effects, whereas the high dimensionality in purely kinetic models render them practically intractable for most relevant purposes.

        We consider a $\delta\!f$ decomposition model, with a macroscopic fluid background and microscopic kinetic correction, both fully coupled to each other. A similar manner of discretization is proposed to that used in the recent \texttt{STRUPHY} code \cite{Holderied_Possanner_Wang_2021, Holderied_2022, Li_et_al_2023} with a finite-element model for the background and a pseudo-particle/PiC model for the correction.

        The fluid background satisfies the full, non-linear, resistive, compressible, Hall MHD equations. \cite{Laakmann_Hu_Farrell_2022} introduces finite-element(-in-space) implicit timesteppers for the incompressible analogue to this system with structure-preserving (SP) properties in the ideal case, alongside parameter-robust preconditioners. We show that these timesteppers can derive from a finite-element-in-time (FET) (and finite-element-in-space) interpretation. The benefits of this reformulation are discussed, including the derivation of timesteppers that are higher order in time, and the quantifiable dissipative SP properties in the non-ideal, resistive case.
        
        We discuss possible options for extending this FET approach to timesteppers for the compressible case.

        The kinetic corrections satisfy linearized Boltzmann equations. Using a Lénard--Bernstein collision operator, these take Fokker--Planck-like forms \cite{Fokker_1914, Planck_1917} wherein pseudo-particles in the numerical model obey the neoclassical transport equations, with particle-independent Brownian drift terms. This offers a rigorous methodology for incorporating collisions into the particle transport model, without coupling the equations of motions for each particle.
        
        Works by Chen, Chacón et al. \cite{Chen_Chacón_Barnes_2011, Chacón_Chen_Barnes_2013, Chen_Chacón_2014, Chen_Chacón_2015} have developed structure-preserving particle pushers for neoclassical transport in the Vlasov equations, derived from Crank--Nicolson integrators. We show these too can can derive from a FET interpretation, similarly offering potential extensions to higher-order-in-time particle pushers. The FET formulation is used also to consider how the stochastic drift terms can be incorporated into the pushers. Stochastic gyrokinetic expansions are also discussed.

        Different options for the numerical implementation of these schemes are considered.

        Due to the efficacy of FET in the development of SP timesteppers for both the fluid and kinetic component, we hope this approach will prove effective in the future for developing SP timesteppers for the full hybrid model. We hope this will give us the opportunity to incorporate previously inaccessible kinetic effects into the highly effective, modern, finite-element MHD models.
    \end{abstract}
    
    
    \newpage
    \tableofcontents
    
    
    \newpage
    \pagenumbering{arabic}
    %\linenumbers\renewcommand\thelinenumber{\color{black!50}\arabic{linenumber}}
            \documentclass[12pt, a4paper]{report}

\input{template/main.tex}

\title{\BA{Title in Progress...}}
\author{Boris Andrews}
\affil{Mathematical Institute, University of Oxford}
\date{\today}


\begin{document}
    \pagenumbering{gobble}
    \maketitle
    
    
    \begin{abstract}
        Magnetic confinement reactors---in particular tokamaks---offer one of the most promising options for achieving practical nuclear fusion, with the potential to provide virtually limitless, clean energy. The theoretical and numerical modeling of tokamak plasmas is simultaneously an essential component of effective reactor design, and a great research barrier. Tokamak operational conditions exhibit comparatively low Knudsen numbers. Kinetic effects, including kinetic waves and instabilities, Landau damping, bump-on-tail instabilities and more, are therefore highly influential in tokamak plasma dynamics. Purely fluid models are inherently incapable of capturing these effects, whereas the high dimensionality in purely kinetic models render them practically intractable for most relevant purposes.

        We consider a $\delta\!f$ decomposition model, with a macroscopic fluid background and microscopic kinetic correction, both fully coupled to each other. A similar manner of discretization is proposed to that used in the recent \texttt{STRUPHY} code \cite{Holderied_Possanner_Wang_2021, Holderied_2022, Li_et_al_2023} with a finite-element model for the background and a pseudo-particle/PiC model for the correction.

        The fluid background satisfies the full, non-linear, resistive, compressible, Hall MHD equations. \cite{Laakmann_Hu_Farrell_2022} introduces finite-element(-in-space) implicit timesteppers for the incompressible analogue to this system with structure-preserving (SP) properties in the ideal case, alongside parameter-robust preconditioners. We show that these timesteppers can derive from a finite-element-in-time (FET) (and finite-element-in-space) interpretation. The benefits of this reformulation are discussed, including the derivation of timesteppers that are higher order in time, and the quantifiable dissipative SP properties in the non-ideal, resistive case.
        
        We discuss possible options for extending this FET approach to timesteppers for the compressible case.

        The kinetic corrections satisfy linearized Boltzmann equations. Using a Lénard--Bernstein collision operator, these take Fokker--Planck-like forms \cite{Fokker_1914, Planck_1917} wherein pseudo-particles in the numerical model obey the neoclassical transport equations, with particle-independent Brownian drift terms. This offers a rigorous methodology for incorporating collisions into the particle transport model, without coupling the equations of motions for each particle.
        
        Works by Chen, Chacón et al. \cite{Chen_Chacón_Barnes_2011, Chacón_Chen_Barnes_2013, Chen_Chacón_2014, Chen_Chacón_2015} have developed structure-preserving particle pushers for neoclassical transport in the Vlasov equations, derived from Crank--Nicolson integrators. We show these too can can derive from a FET interpretation, similarly offering potential extensions to higher-order-in-time particle pushers. The FET formulation is used also to consider how the stochastic drift terms can be incorporated into the pushers. Stochastic gyrokinetic expansions are also discussed.

        Different options for the numerical implementation of these schemes are considered.

        Due to the efficacy of FET in the development of SP timesteppers for both the fluid and kinetic component, we hope this approach will prove effective in the future for developing SP timesteppers for the full hybrid model. We hope this will give us the opportunity to incorporate previously inaccessible kinetic effects into the highly effective, modern, finite-element MHD models.
    \end{abstract}
    
    
    \newpage
    \tableofcontents
    
    
    \newpage
    \pagenumbering{arabic}
    %\linenumbers\renewcommand\thelinenumber{\color{black!50}\arabic{linenumber}}
            \input{0 - introduction/main.tex}
        \part{Research}
            \input{1 - low-noise PiC models/main.tex}
            \input{2 - kinetic component/main.tex}
            \input{3 - fluid component/main.tex}
            \input{4 - numerical implementation/main.tex}
        \part{Project Overview}
            \input{5 - research plan/main.tex}
            \input{6 - summary/main.tex}
    
    
    %\section{}
    \newpage
    \pagenumbering{gobble}
        \printbibliography


    \newpage
    \pagenumbering{roman}
    \appendix
        \part{Appendices}
            \input{8 - Hilbert complexes/main.tex}
            \input{9 - weak conservation proofs/main.tex}
\end{document}

        \part{Research}
            \documentclass[12pt, a4paper]{report}

\input{template/main.tex}

\title{\BA{Title in Progress...}}
\author{Boris Andrews}
\affil{Mathematical Institute, University of Oxford}
\date{\today}


\begin{document}
    \pagenumbering{gobble}
    \maketitle
    
    
    \begin{abstract}
        Magnetic confinement reactors---in particular tokamaks---offer one of the most promising options for achieving practical nuclear fusion, with the potential to provide virtually limitless, clean energy. The theoretical and numerical modeling of tokamak plasmas is simultaneously an essential component of effective reactor design, and a great research barrier. Tokamak operational conditions exhibit comparatively low Knudsen numbers. Kinetic effects, including kinetic waves and instabilities, Landau damping, bump-on-tail instabilities and more, are therefore highly influential in tokamak plasma dynamics. Purely fluid models are inherently incapable of capturing these effects, whereas the high dimensionality in purely kinetic models render them practically intractable for most relevant purposes.

        We consider a $\delta\!f$ decomposition model, with a macroscopic fluid background and microscopic kinetic correction, both fully coupled to each other. A similar manner of discretization is proposed to that used in the recent \texttt{STRUPHY} code \cite{Holderied_Possanner_Wang_2021, Holderied_2022, Li_et_al_2023} with a finite-element model for the background and a pseudo-particle/PiC model for the correction.

        The fluid background satisfies the full, non-linear, resistive, compressible, Hall MHD equations. \cite{Laakmann_Hu_Farrell_2022} introduces finite-element(-in-space) implicit timesteppers for the incompressible analogue to this system with structure-preserving (SP) properties in the ideal case, alongside parameter-robust preconditioners. We show that these timesteppers can derive from a finite-element-in-time (FET) (and finite-element-in-space) interpretation. The benefits of this reformulation are discussed, including the derivation of timesteppers that are higher order in time, and the quantifiable dissipative SP properties in the non-ideal, resistive case.
        
        We discuss possible options for extending this FET approach to timesteppers for the compressible case.

        The kinetic corrections satisfy linearized Boltzmann equations. Using a Lénard--Bernstein collision operator, these take Fokker--Planck-like forms \cite{Fokker_1914, Planck_1917} wherein pseudo-particles in the numerical model obey the neoclassical transport equations, with particle-independent Brownian drift terms. This offers a rigorous methodology for incorporating collisions into the particle transport model, without coupling the equations of motions for each particle.
        
        Works by Chen, Chacón et al. \cite{Chen_Chacón_Barnes_2011, Chacón_Chen_Barnes_2013, Chen_Chacón_2014, Chen_Chacón_2015} have developed structure-preserving particle pushers for neoclassical transport in the Vlasov equations, derived from Crank--Nicolson integrators. We show these too can can derive from a FET interpretation, similarly offering potential extensions to higher-order-in-time particle pushers. The FET formulation is used also to consider how the stochastic drift terms can be incorporated into the pushers. Stochastic gyrokinetic expansions are also discussed.

        Different options for the numerical implementation of these schemes are considered.

        Due to the efficacy of FET in the development of SP timesteppers for both the fluid and kinetic component, we hope this approach will prove effective in the future for developing SP timesteppers for the full hybrid model. We hope this will give us the opportunity to incorporate previously inaccessible kinetic effects into the highly effective, modern, finite-element MHD models.
    \end{abstract}
    
    
    \newpage
    \tableofcontents
    
    
    \newpage
    \pagenumbering{arabic}
    %\linenumbers\renewcommand\thelinenumber{\color{black!50}\arabic{linenumber}}
            \input{0 - introduction/main.tex}
        \part{Research}
            \input{1 - low-noise PiC models/main.tex}
            \input{2 - kinetic component/main.tex}
            \input{3 - fluid component/main.tex}
            \input{4 - numerical implementation/main.tex}
        \part{Project Overview}
            \input{5 - research plan/main.tex}
            \input{6 - summary/main.tex}
    
    
    %\section{}
    \newpage
    \pagenumbering{gobble}
        \printbibliography


    \newpage
    \pagenumbering{roman}
    \appendix
        \part{Appendices}
            \input{8 - Hilbert complexes/main.tex}
            \input{9 - weak conservation proofs/main.tex}
\end{document}

            \documentclass[12pt, a4paper]{report}

\input{template/main.tex}

\title{\BA{Title in Progress...}}
\author{Boris Andrews}
\affil{Mathematical Institute, University of Oxford}
\date{\today}


\begin{document}
    \pagenumbering{gobble}
    \maketitle
    
    
    \begin{abstract}
        Magnetic confinement reactors---in particular tokamaks---offer one of the most promising options for achieving practical nuclear fusion, with the potential to provide virtually limitless, clean energy. The theoretical and numerical modeling of tokamak plasmas is simultaneously an essential component of effective reactor design, and a great research barrier. Tokamak operational conditions exhibit comparatively low Knudsen numbers. Kinetic effects, including kinetic waves and instabilities, Landau damping, bump-on-tail instabilities and more, are therefore highly influential in tokamak plasma dynamics. Purely fluid models are inherently incapable of capturing these effects, whereas the high dimensionality in purely kinetic models render them practically intractable for most relevant purposes.

        We consider a $\delta\!f$ decomposition model, with a macroscopic fluid background and microscopic kinetic correction, both fully coupled to each other. A similar manner of discretization is proposed to that used in the recent \texttt{STRUPHY} code \cite{Holderied_Possanner_Wang_2021, Holderied_2022, Li_et_al_2023} with a finite-element model for the background and a pseudo-particle/PiC model for the correction.

        The fluid background satisfies the full, non-linear, resistive, compressible, Hall MHD equations. \cite{Laakmann_Hu_Farrell_2022} introduces finite-element(-in-space) implicit timesteppers for the incompressible analogue to this system with structure-preserving (SP) properties in the ideal case, alongside parameter-robust preconditioners. We show that these timesteppers can derive from a finite-element-in-time (FET) (and finite-element-in-space) interpretation. The benefits of this reformulation are discussed, including the derivation of timesteppers that are higher order in time, and the quantifiable dissipative SP properties in the non-ideal, resistive case.
        
        We discuss possible options for extending this FET approach to timesteppers for the compressible case.

        The kinetic corrections satisfy linearized Boltzmann equations. Using a Lénard--Bernstein collision operator, these take Fokker--Planck-like forms \cite{Fokker_1914, Planck_1917} wherein pseudo-particles in the numerical model obey the neoclassical transport equations, with particle-independent Brownian drift terms. This offers a rigorous methodology for incorporating collisions into the particle transport model, without coupling the equations of motions for each particle.
        
        Works by Chen, Chacón et al. \cite{Chen_Chacón_Barnes_2011, Chacón_Chen_Barnes_2013, Chen_Chacón_2014, Chen_Chacón_2015} have developed structure-preserving particle pushers for neoclassical transport in the Vlasov equations, derived from Crank--Nicolson integrators. We show these too can can derive from a FET interpretation, similarly offering potential extensions to higher-order-in-time particle pushers. The FET formulation is used also to consider how the stochastic drift terms can be incorporated into the pushers. Stochastic gyrokinetic expansions are also discussed.

        Different options for the numerical implementation of these schemes are considered.

        Due to the efficacy of FET in the development of SP timesteppers for both the fluid and kinetic component, we hope this approach will prove effective in the future for developing SP timesteppers for the full hybrid model. We hope this will give us the opportunity to incorporate previously inaccessible kinetic effects into the highly effective, modern, finite-element MHD models.
    \end{abstract}
    
    
    \newpage
    \tableofcontents
    
    
    \newpage
    \pagenumbering{arabic}
    %\linenumbers\renewcommand\thelinenumber{\color{black!50}\arabic{linenumber}}
            \input{0 - introduction/main.tex}
        \part{Research}
            \input{1 - low-noise PiC models/main.tex}
            \input{2 - kinetic component/main.tex}
            \input{3 - fluid component/main.tex}
            \input{4 - numerical implementation/main.tex}
        \part{Project Overview}
            \input{5 - research plan/main.tex}
            \input{6 - summary/main.tex}
    
    
    %\section{}
    \newpage
    \pagenumbering{gobble}
        \printbibliography


    \newpage
    \pagenumbering{roman}
    \appendix
        \part{Appendices}
            \input{8 - Hilbert complexes/main.tex}
            \input{9 - weak conservation proofs/main.tex}
\end{document}

            \documentclass[12pt, a4paper]{report}

\input{template/main.tex}

\title{\BA{Title in Progress...}}
\author{Boris Andrews}
\affil{Mathematical Institute, University of Oxford}
\date{\today}


\begin{document}
    \pagenumbering{gobble}
    \maketitle
    
    
    \begin{abstract}
        Magnetic confinement reactors---in particular tokamaks---offer one of the most promising options for achieving practical nuclear fusion, with the potential to provide virtually limitless, clean energy. The theoretical and numerical modeling of tokamak plasmas is simultaneously an essential component of effective reactor design, and a great research barrier. Tokamak operational conditions exhibit comparatively low Knudsen numbers. Kinetic effects, including kinetic waves and instabilities, Landau damping, bump-on-tail instabilities and more, are therefore highly influential in tokamak plasma dynamics. Purely fluid models are inherently incapable of capturing these effects, whereas the high dimensionality in purely kinetic models render them practically intractable for most relevant purposes.

        We consider a $\delta\!f$ decomposition model, with a macroscopic fluid background and microscopic kinetic correction, both fully coupled to each other. A similar manner of discretization is proposed to that used in the recent \texttt{STRUPHY} code \cite{Holderied_Possanner_Wang_2021, Holderied_2022, Li_et_al_2023} with a finite-element model for the background and a pseudo-particle/PiC model for the correction.

        The fluid background satisfies the full, non-linear, resistive, compressible, Hall MHD equations. \cite{Laakmann_Hu_Farrell_2022} introduces finite-element(-in-space) implicit timesteppers for the incompressible analogue to this system with structure-preserving (SP) properties in the ideal case, alongside parameter-robust preconditioners. We show that these timesteppers can derive from a finite-element-in-time (FET) (and finite-element-in-space) interpretation. The benefits of this reformulation are discussed, including the derivation of timesteppers that are higher order in time, and the quantifiable dissipative SP properties in the non-ideal, resistive case.
        
        We discuss possible options for extending this FET approach to timesteppers for the compressible case.

        The kinetic corrections satisfy linearized Boltzmann equations. Using a Lénard--Bernstein collision operator, these take Fokker--Planck-like forms \cite{Fokker_1914, Planck_1917} wherein pseudo-particles in the numerical model obey the neoclassical transport equations, with particle-independent Brownian drift terms. This offers a rigorous methodology for incorporating collisions into the particle transport model, without coupling the equations of motions for each particle.
        
        Works by Chen, Chacón et al. \cite{Chen_Chacón_Barnes_2011, Chacón_Chen_Barnes_2013, Chen_Chacón_2014, Chen_Chacón_2015} have developed structure-preserving particle pushers for neoclassical transport in the Vlasov equations, derived from Crank--Nicolson integrators. We show these too can can derive from a FET interpretation, similarly offering potential extensions to higher-order-in-time particle pushers. The FET formulation is used also to consider how the stochastic drift terms can be incorporated into the pushers. Stochastic gyrokinetic expansions are also discussed.

        Different options for the numerical implementation of these schemes are considered.

        Due to the efficacy of FET in the development of SP timesteppers for both the fluid and kinetic component, we hope this approach will prove effective in the future for developing SP timesteppers for the full hybrid model. We hope this will give us the opportunity to incorporate previously inaccessible kinetic effects into the highly effective, modern, finite-element MHD models.
    \end{abstract}
    
    
    \newpage
    \tableofcontents
    
    
    \newpage
    \pagenumbering{arabic}
    %\linenumbers\renewcommand\thelinenumber{\color{black!50}\arabic{linenumber}}
            \input{0 - introduction/main.tex}
        \part{Research}
            \input{1 - low-noise PiC models/main.tex}
            \input{2 - kinetic component/main.tex}
            \input{3 - fluid component/main.tex}
            \input{4 - numerical implementation/main.tex}
        \part{Project Overview}
            \input{5 - research plan/main.tex}
            \input{6 - summary/main.tex}
    
    
    %\section{}
    \newpage
    \pagenumbering{gobble}
        \printbibliography


    \newpage
    \pagenumbering{roman}
    \appendix
        \part{Appendices}
            \input{8 - Hilbert complexes/main.tex}
            \input{9 - weak conservation proofs/main.tex}
\end{document}

            \documentclass[12pt, a4paper]{report}

\input{template/main.tex}

\title{\BA{Title in Progress...}}
\author{Boris Andrews}
\affil{Mathematical Institute, University of Oxford}
\date{\today}


\begin{document}
    \pagenumbering{gobble}
    \maketitle
    
    
    \begin{abstract}
        Magnetic confinement reactors---in particular tokamaks---offer one of the most promising options for achieving practical nuclear fusion, with the potential to provide virtually limitless, clean energy. The theoretical and numerical modeling of tokamak plasmas is simultaneously an essential component of effective reactor design, and a great research barrier. Tokamak operational conditions exhibit comparatively low Knudsen numbers. Kinetic effects, including kinetic waves and instabilities, Landau damping, bump-on-tail instabilities and more, are therefore highly influential in tokamak plasma dynamics. Purely fluid models are inherently incapable of capturing these effects, whereas the high dimensionality in purely kinetic models render them practically intractable for most relevant purposes.

        We consider a $\delta\!f$ decomposition model, with a macroscopic fluid background and microscopic kinetic correction, both fully coupled to each other. A similar manner of discretization is proposed to that used in the recent \texttt{STRUPHY} code \cite{Holderied_Possanner_Wang_2021, Holderied_2022, Li_et_al_2023} with a finite-element model for the background and a pseudo-particle/PiC model for the correction.

        The fluid background satisfies the full, non-linear, resistive, compressible, Hall MHD equations. \cite{Laakmann_Hu_Farrell_2022} introduces finite-element(-in-space) implicit timesteppers for the incompressible analogue to this system with structure-preserving (SP) properties in the ideal case, alongside parameter-robust preconditioners. We show that these timesteppers can derive from a finite-element-in-time (FET) (and finite-element-in-space) interpretation. The benefits of this reformulation are discussed, including the derivation of timesteppers that are higher order in time, and the quantifiable dissipative SP properties in the non-ideal, resistive case.
        
        We discuss possible options for extending this FET approach to timesteppers for the compressible case.

        The kinetic corrections satisfy linearized Boltzmann equations. Using a Lénard--Bernstein collision operator, these take Fokker--Planck-like forms \cite{Fokker_1914, Planck_1917} wherein pseudo-particles in the numerical model obey the neoclassical transport equations, with particle-independent Brownian drift terms. This offers a rigorous methodology for incorporating collisions into the particle transport model, without coupling the equations of motions for each particle.
        
        Works by Chen, Chacón et al. \cite{Chen_Chacón_Barnes_2011, Chacón_Chen_Barnes_2013, Chen_Chacón_2014, Chen_Chacón_2015} have developed structure-preserving particle pushers for neoclassical transport in the Vlasov equations, derived from Crank--Nicolson integrators. We show these too can can derive from a FET interpretation, similarly offering potential extensions to higher-order-in-time particle pushers. The FET formulation is used also to consider how the stochastic drift terms can be incorporated into the pushers. Stochastic gyrokinetic expansions are also discussed.

        Different options for the numerical implementation of these schemes are considered.

        Due to the efficacy of FET in the development of SP timesteppers for both the fluid and kinetic component, we hope this approach will prove effective in the future for developing SP timesteppers for the full hybrid model. We hope this will give us the opportunity to incorporate previously inaccessible kinetic effects into the highly effective, modern, finite-element MHD models.
    \end{abstract}
    
    
    \newpage
    \tableofcontents
    
    
    \newpage
    \pagenumbering{arabic}
    %\linenumbers\renewcommand\thelinenumber{\color{black!50}\arabic{linenumber}}
            \input{0 - introduction/main.tex}
        \part{Research}
            \input{1 - low-noise PiC models/main.tex}
            \input{2 - kinetic component/main.tex}
            \input{3 - fluid component/main.tex}
            \input{4 - numerical implementation/main.tex}
        \part{Project Overview}
            \input{5 - research plan/main.tex}
            \input{6 - summary/main.tex}
    
    
    %\section{}
    \newpage
    \pagenumbering{gobble}
        \printbibliography


    \newpage
    \pagenumbering{roman}
    \appendix
        \part{Appendices}
            \input{8 - Hilbert complexes/main.tex}
            \input{9 - weak conservation proofs/main.tex}
\end{document}

        \part{Project Overview}
            \documentclass[12pt, a4paper]{report}

\input{template/main.tex}

\title{\BA{Title in Progress...}}
\author{Boris Andrews}
\affil{Mathematical Institute, University of Oxford}
\date{\today}


\begin{document}
    \pagenumbering{gobble}
    \maketitle
    
    
    \begin{abstract}
        Magnetic confinement reactors---in particular tokamaks---offer one of the most promising options for achieving practical nuclear fusion, with the potential to provide virtually limitless, clean energy. The theoretical and numerical modeling of tokamak plasmas is simultaneously an essential component of effective reactor design, and a great research barrier. Tokamak operational conditions exhibit comparatively low Knudsen numbers. Kinetic effects, including kinetic waves and instabilities, Landau damping, bump-on-tail instabilities and more, are therefore highly influential in tokamak plasma dynamics. Purely fluid models are inherently incapable of capturing these effects, whereas the high dimensionality in purely kinetic models render them practically intractable for most relevant purposes.

        We consider a $\delta\!f$ decomposition model, with a macroscopic fluid background and microscopic kinetic correction, both fully coupled to each other. A similar manner of discretization is proposed to that used in the recent \texttt{STRUPHY} code \cite{Holderied_Possanner_Wang_2021, Holderied_2022, Li_et_al_2023} with a finite-element model for the background and a pseudo-particle/PiC model for the correction.

        The fluid background satisfies the full, non-linear, resistive, compressible, Hall MHD equations. \cite{Laakmann_Hu_Farrell_2022} introduces finite-element(-in-space) implicit timesteppers for the incompressible analogue to this system with structure-preserving (SP) properties in the ideal case, alongside parameter-robust preconditioners. We show that these timesteppers can derive from a finite-element-in-time (FET) (and finite-element-in-space) interpretation. The benefits of this reformulation are discussed, including the derivation of timesteppers that are higher order in time, and the quantifiable dissipative SP properties in the non-ideal, resistive case.
        
        We discuss possible options for extending this FET approach to timesteppers for the compressible case.

        The kinetic corrections satisfy linearized Boltzmann equations. Using a Lénard--Bernstein collision operator, these take Fokker--Planck-like forms \cite{Fokker_1914, Planck_1917} wherein pseudo-particles in the numerical model obey the neoclassical transport equations, with particle-independent Brownian drift terms. This offers a rigorous methodology for incorporating collisions into the particle transport model, without coupling the equations of motions for each particle.
        
        Works by Chen, Chacón et al. \cite{Chen_Chacón_Barnes_2011, Chacón_Chen_Barnes_2013, Chen_Chacón_2014, Chen_Chacón_2015} have developed structure-preserving particle pushers for neoclassical transport in the Vlasov equations, derived from Crank--Nicolson integrators. We show these too can can derive from a FET interpretation, similarly offering potential extensions to higher-order-in-time particle pushers. The FET formulation is used also to consider how the stochastic drift terms can be incorporated into the pushers. Stochastic gyrokinetic expansions are also discussed.

        Different options for the numerical implementation of these schemes are considered.

        Due to the efficacy of FET in the development of SP timesteppers for both the fluid and kinetic component, we hope this approach will prove effective in the future for developing SP timesteppers for the full hybrid model. We hope this will give us the opportunity to incorporate previously inaccessible kinetic effects into the highly effective, modern, finite-element MHD models.
    \end{abstract}
    
    
    \newpage
    \tableofcontents
    
    
    \newpage
    \pagenumbering{arabic}
    %\linenumbers\renewcommand\thelinenumber{\color{black!50}\arabic{linenumber}}
            \input{0 - introduction/main.tex}
        \part{Research}
            \input{1 - low-noise PiC models/main.tex}
            \input{2 - kinetic component/main.tex}
            \input{3 - fluid component/main.tex}
            \input{4 - numerical implementation/main.tex}
        \part{Project Overview}
            \input{5 - research plan/main.tex}
            \input{6 - summary/main.tex}
    
    
    %\section{}
    \newpage
    \pagenumbering{gobble}
        \printbibliography


    \newpage
    \pagenumbering{roman}
    \appendix
        \part{Appendices}
            \input{8 - Hilbert complexes/main.tex}
            \input{9 - weak conservation proofs/main.tex}
\end{document}

            \documentclass[12pt, a4paper]{report}

\input{template/main.tex}

\title{\BA{Title in Progress...}}
\author{Boris Andrews}
\affil{Mathematical Institute, University of Oxford}
\date{\today}


\begin{document}
    \pagenumbering{gobble}
    \maketitle
    
    
    \begin{abstract}
        Magnetic confinement reactors---in particular tokamaks---offer one of the most promising options for achieving practical nuclear fusion, with the potential to provide virtually limitless, clean energy. The theoretical and numerical modeling of tokamak plasmas is simultaneously an essential component of effective reactor design, and a great research barrier. Tokamak operational conditions exhibit comparatively low Knudsen numbers. Kinetic effects, including kinetic waves and instabilities, Landau damping, bump-on-tail instabilities and more, are therefore highly influential in tokamak plasma dynamics. Purely fluid models are inherently incapable of capturing these effects, whereas the high dimensionality in purely kinetic models render them practically intractable for most relevant purposes.

        We consider a $\delta\!f$ decomposition model, with a macroscopic fluid background and microscopic kinetic correction, both fully coupled to each other. A similar manner of discretization is proposed to that used in the recent \texttt{STRUPHY} code \cite{Holderied_Possanner_Wang_2021, Holderied_2022, Li_et_al_2023} with a finite-element model for the background and a pseudo-particle/PiC model for the correction.

        The fluid background satisfies the full, non-linear, resistive, compressible, Hall MHD equations. \cite{Laakmann_Hu_Farrell_2022} introduces finite-element(-in-space) implicit timesteppers for the incompressible analogue to this system with structure-preserving (SP) properties in the ideal case, alongside parameter-robust preconditioners. We show that these timesteppers can derive from a finite-element-in-time (FET) (and finite-element-in-space) interpretation. The benefits of this reformulation are discussed, including the derivation of timesteppers that are higher order in time, and the quantifiable dissipative SP properties in the non-ideal, resistive case.
        
        We discuss possible options for extending this FET approach to timesteppers for the compressible case.

        The kinetic corrections satisfy linearized Boltzmann equations. Using a Lénard--Bernstein collision operator, these take Fokker--Planck-like forms \cite{Fokker_1914, Planck_1917} wherein pseudo-particles in the numerical model obey the neoclassical transport equations, with particle-independent Brownian drift terms. This offers a rigorous methodology for incorporating collisions into the particle transport model, without coupling the equations of motions for each particle.
        
        Works by Chen, Chacón et al. \cite{Chen_Chacón_Barnes_2011, Chacón_Chen_Barnes_2013, Chen_Chacón_2014, Chen_Chacón_2015} have developed structure-preserving particle pushers for neoclassical transport in the Vlasov equations, derived from Crank--Nicolson integrators. We show these too can can derive from a FET interpretation, similarly offering potential extensions to higher-order-in-time particle pushers. The FET formulation is used also to consider how the stochastic drift terms can be incorporated into the pushers. Stochastic gyrokinetic expansions are also discussed.

        Different options for the numerical implementation of these schemes are considered.

        Due to the efficacy of FET in the development of SP timesteppers for both the fluid and kinetic component, we hope this approach will prove effective in the future for developing SP timesteppers for the full hybrid model. We hope this will give us the opportunity to incorporate previously inaccessible kinetic effects into the highly effective, modern, finite-element MHD models.
    \end{abstract}
    
    
    \newpage
    \tableofcontents
    
    
    \newpage
    \pagenumbering{arabic}
    %\linenumbers\renewcommand\thelinenumber{\color{black!50}\arabic{linenumber}}
            \input{0 - introduction/main.tex}
        \part{Research}
            \input{1 - low-noise PiC models/main.tex}
            \input{2 - kinetic component/main.tex}
            \input{3 - fluid component/main.tex}
            \input{4 - numerical implementation/main.tex}
        \part{Project Overview}
            \input{5 - research plan/main.tex}
            \input{6 - summary/main.tex}
    
    
    %\section{}
    \newpage
    \pagenumbering{gobble}
        \printbibliography


    \newpage
    \pagenumbering{roman}
    \appendix
        \part{Appendices}
            \input{8 - Hilbert complexes/main.tex}
            \input{9 - weak conservation proofs/main.tex}
\end{document}

    
    
    %\section{}
    \newpage
    \pagenumbering{gobble}
        \printbibliography


    \newpage
    \pagenumbering{roman}
    \appendix
        \part{Appendices}
            \documentclass[12pt, a4paper]{report}

\input{template/main.tex}

\title{\BA{Title in Progress...}}
\author{Boris Andrews}
\affil{Mathematical Institute, University of Oxford}
\date{\today}


\begin{document}
    \pagenumbering{gobble}
    \maketitle
    
    
    \begin{abstract}
        Magnetic confinement reactors---in particular tokamaks---offer one of the most promising options for achieving practical nuclear fusion, with the potential to provide virtually limitless, clean energy. The theoretical and numerical modeling of tokamak plasmas is simultaneously an essential component of effective reactor design, and a great research barrier. Tokamak operational conditions exhibit comparatively low Knudsen numbers. Kinetic effects, including kinetic waves and instabilities, Landau damping, bump-on-tail instabilities and more, are therefore highly influential in tokamak plasma dynamics. Purely fluid models are inherently incapable of capturing these effects, whereas the high dimensionality in purely kinetic models render them practically intractable for most relevant purposes.

        We consider a $\delta\!f$ decomposition model, with a macroscopic fluid background and microscopic kinetic correction, both fully coupled to each other. A similar manner of discretization is proposed to that used in the recent \texttt{STRUPHY} code \cite{Holderied_Possanner_Wang_2021, Holderied_2022, Li_et_al_2023} with a finite-element model for the background and a pseudo-particle/PiC model for the correction.

        The fluid background satisfies the full, non-linear, resistive, compressible, Hall MHD equations. \cite{Laakmann_Hu_Farrell_2022} introduces finite-element(-in-space) implicit timesteppers for the incompressible analogue to this system with structure-preserving (SP) properties in the ideal case, alongside parameter-robust preconditioners. We show that these timesteppers can derive from a finite-element-in-time (FET) (and finite-element-in-space) interpretation. The benefits of this reformulation are discussed, including the derivation of timesteppers that are higher order in time, and the quantifiable dissipative SP properties in the non-ideal, resistive case.
        
        We discuss possible options for extending this FET approach to timesteppers for the compressible case.

        The kinetic corrections satisfy linearized Boltzmann equations. Using a Lénard--Bernstein collision operator, these take Fokker--Planck-like forms \cite{Fokker_1914, Planck_1917} wherein pseudo-particles in the numerical model obey the neoclassical transport equations, with particle-independent Brownian drift terms. This offers a rigorous methodology for incorporating collisions into the particle transport model, without coupling the equations of motions for each particle.
        
        Works by Chen, Chacón et al. \cite{Chen_Chacón_Barnes_2011, Chacón_Chen_Barnes_2013, Chen_Chacón_2014, Chen_Chacón_2015} have developed structure-preserving particle pushers for neoclassical transport in the Vlasov equations, derived from Crank--Nicolson integrators. We show these too can can derive from a FET interpretation, similarly offering potential extensions to higher-order-in-time particle pushers. The FET formulation is used also to consider how the stochastic drift terms can be incorporated into the pushers. Stochastic gyrokinetic expansions are also discussed.

        Different options for the numerical implementation of these schemes are considered.

        Due to the efficacy of FET in the development of SP timesteppers for both the fluid and kinetic component, we hope this approach will prove effective in the future for developing SP timesteppers for the full hybrid model. We hope this will give us the opportunity to incorporate previously inaccessible kinetic effects into the highly effective, modern, finite-element MHD models.
    \end{abstract}
    
    
    \newpage
    \tableofcontents
    
    
    \newpage
    \pagenumbering{arabic}
    %\linenumbers\renewcommand\thelinenumber{\color{black!50}\arabic{linenumber}}
            \input{0 - introduction/main.tex}
        \part{Research}
            \input{1 - low-noise PiC models/main.tex}
            \input{2 - kinetic component/main.tex}
            \input{3 - fluid component/main.tex}
            \input{4 - numerical implementation/main.tex}
        \part{Project Overview}
            \input{5 - research plan/main.tex}
            \input{6 - summary/main.tex}
    
    
    %\section{}
    \newpage
    \pagenumbering{gobble}
        \printbibliography


    \newpage
    \pagenumbering{roman}
    \appendix
        \part{Appendices}
            \input{8 - Hilbert complexes/main.tex}
            \input{9 - weak conservation proofs/main.tex}
\end{document}

            \documentclass[12pt, a4paper]{report}

\input{template/main.tex}

\title{\BA{Title in Progress...}}
\author{Boris Andrews}
\affil{Mathematical Institute, University of Oxford}
\date{\today}


\begin{document}
    \pagenumbering{gobble}
    \maketitle
    
    
    \begin{abstract}
        Magnetic confinement reactors---in particular tokamaks---offer one of the most promising options for achieving practical nuclear fusion, with the potential to provide virtually limitless, clean energy. The theoretical and numerical modeling of tokamak plasmas is simultaneously an essential component of effective reactor design, and a great research barrier. Tokamak operational conditions exhibit comparatively low Knudsen numbers. Kinetic effects, including kinetic waves and instabilities, Landau damping, bump-on-tail instabilities and more, are therefore highly influential in tokamak plasma dynamics. Purely fluid models are inherently incapable of capturing these effects, whereas the high dimensionality in purely kinetic models render them practically intractable for most relevant purposes.

        We consider a $\delta\!f$ decomposition model, with a macroscopic fluid background and microscopic kinetic correction, both fully coupled to each other. A similar manner of discretization is proposed to that used in the recent \texttt{STRUPHY} code \cite{Holderied_Possanner_Wang_2021, Holderied_2022, Li_et_al_2023} with a finite-element model for the background and a pseudo-particle/PiC model for the correction.

        The fluid background satisfies the full, non-linear, resistive, compressible, Hall MHD equations. \cite{Laakmann_Hu_Farrell_2022} introduces finite-element(-in-space) implicit timesteppers for the incompressible analogue to this system with structure-preserving (SP) properties in the ideal case, alongside parameter-robust preconditioners. We show that these timesteppers can derive from a finite-element-in-time (FET) (and finite-element-in-space) interpretation. The benefits of this reformulation are discussed, including the derivation of timesteppers that are higher order in time, and the quantifiable dissipative SP properties in the non-ideal, resistive case.
        
        We discuss possible options for extending this FET approach to timesteppers for the compressible case.

        The kinetic corrections satisfy linearized Boltzmann equations. Using a Lénard--Bernstein collision operator, these take Fokker--Planck-like forms \cite{Fokker_1914, Planck_1917} wherein pseudo-particles in the numerical model obey the neoclassical transport equations, with particle-independent Brownian drift terms. This offers a rigorous methodology for incorporating collisions into the particle transport model, without coupling the equations of motions for each particle.
        
        Works by Chen, Chacón et al. \cite{Chen_Chacón_Barnes_2011, Chacón_Chen_Barnes_2013, Chen_Chacón_2014, Chen_Chacón_2015} have developed structure-preserving particle pushers for neoclassical transport in the Vlasov equations, derived from Crank--Nicolson integrators. We show these too can can derive from a FET interpretation, similarly offering potential extensions to higher-order-in-time particle pushers. The FET formulation is used also to consider how the stochastic drift terms can be incorporated into the pushers. Stochastic gyrokinetic expansions are also discussed.

        Different options for the numerical implementation of these schemes are considered.

        Due to the efficacy of FET in the development of SP timesteppers for both the fluid and kinetic component, we hope this approach will prove effective in the future for developing SP timesteppers for the full hybrid model. We hope this will give us the opportunity to incorporate previously inaccessible kinetic effects into the highly effective, modern, finite-element MHD models.
    \end{abstract}
    
    
    \newpage
    \tableofcontents
    
    
    \newpage
    \pagenumbering{arabic}
    %\linenumbers\renewcommand\thelinenumber{\color{black!50}\arabic{linenumber}}
            \input{0 - introduction/main.tex}
        \part{Research}
            \input{1 - low-noise PiC models/main.tex}
            \input{2 - kinetic component/main.tex}
            \input{3 - fluid component/main.tex}
            \input{4 - numerical implementation/main.tex}
        \part{Project Overview}
            \input{5 - research plan/main.tex}
            \input{6 - summary/main.tex}
    
    
    %\section{}
    \newpage
    \pagenumbering{gobble}
        \printbibliography


    \newpage
    \pagenumbering{roman}
    \appendix
        \part{Appendices}
            \input{8 - Hilbert complexes/main.tex}
            \input{9 - weak conservation proofs/main.tex}
\end{document}

\end{document}

        \part{Research}
            \documentclass[12pt, a4paper]{report}

\documentclass[12pt, a4paper]{report}

\input{template/main.tex}

\title{\BA{Title in Progress...}}
\author{Boris Andrews}
\affil{Mathematical Institute, University of Oxford}
\date{\today}


\begin{document}
    \pagenumbering{gobble}
    \maketitle
    
    
    \begin{abstract}
        Magnetic confinement reactors---in particular tokamaks---offer one of the most promising options for achieving practical nuclear fusion, with the potential to provide virtually limitless, clean energy. The theoretical and numerical modeling of tokamak plasmas is simultaneously an essential component of effective reactor design, and a great research barrier. Tokamak operational conditions exhibit comparatively low Knudsen numbers. Kinetic effects, including kinetic waves and instabilities, Landau damping, bump-on-tail instabilities and more, are therefore highly influential in tokamak plasma dynamics. Purely fluid models are inherently incapable of capturing these effects, whereas the high dimensionality in purely kinetic models render them practically intractable for most relevant purposes.

        We consider a $\delta\!f$ decomposition model, with a macroscopic fluid background and microscopic kinetic correction, both fully coupled to each other. A similar manner of discretization is proposed to that used in the recent \texttt{STRUPHY} code \cite{Holderied_Possanner_Wang_2021, Holderied_2022, Li_et_al_2023} with a finite-element model for the background and a pseudo-particle/PiC model for the correction.

        The fluid background satisfies the full, non-linear, resistive, compressible, Hall MHD equations. \cite{Laakmann_Hu_Farrell_2022} introduces finite-element(-in-space) implicit timesteppers for the incompressible analogue to this system with structure-preserving (SP) properties in the ideal case, alongside parameter-robust preconditioners. We show that these timesteppers can derive from a finite-element-in-time (FET) (and finite-element-in-space) interpretation. The benefits of this reformulation are discussed, including the derivation of timesteppers that are higher order in time, and the quantifiable dissipative SP properties in the non-ideal, resistive case.
        
        We discuss possible options for extending this FET approach to timesteppers for the compressible case.

        The kinetic corrections satisfy linearized Boltzmann equations. Using a Lénard--Bernstein collision operator, these take Fokker--Planck-like forms \cite{Fokker_1914, Planck_1917} wherein pseudo-particles in the numerical model obey the neoclassical transport equations, with particle-independent Brownian drift terms. This offers a rigorous methodology for incorporating collisions into the particle transport model, without coupling the equations of motions for each particle.
        
        Works by Chen, Chacón et al. \cite{Chen_Chacón_Barnes_2011, Chacón_Chen_Barnes_2013, Chen_Chacón_2014, Chen_Chacón_2015} have developed structure-preserving particle pushers for neoclassical transport in the Vlasov equations, derived from Crank--Nicolson integrators. We show these too can can derive from a FET interpretation, similarly offering potential extensions to higher-order-in-time particle pushers. The FET formulation is used also to consider how the stochastic drift terms can be incorporated into the pushers. Stochastic gyrokinetic expansions are also discussed.

        Different options for the numerical implementation of these schemes are considered.

        Due to the efficacy of FET in the development of SP timesteppers for both the fluid and kinetic component, we hope this approach will prove effective in the future for developing SP timesteppers for the full hybrid model. We hope this will give us the opportunity to incorporate previously inaccessible kinetic effects into the highly effective, modern, finite-element MHD models.
    \end{abstract}
    
    
    \newpage
    \tableofcontents
    
    
    \newpage
    \pagenumbering{arabic}
    %\linenumbers\renewcommand\thelinenumber{\color{black!50}\arabic{linenumber}}
            \input{0 - introduction/main.tex}
        \part{Research}
            \input{1 - low-noise PiC models/main.tex}
            \input{2 - kinetic component/main.tex}
            \input{3 - fluid component/main.tex}
            \input{4 - numerical implementation/main.tex}
        \part{Project Overview}
            \input{5 - research plan/main.tex}
            \input{6 - summary/main.tex}
    
    
    %\section{}
    \newpage
    \pagenumbering{gobble}
        \printbibliography


    \newpage
    \pagenumbering{roman}
    \appendix
        \part{Appendices}
            \input{8 - Hilbert complexes/main.tex}
            \input{9 - weak conservation proofs/main.tex}
\end{document}


\title{\BA{Title in Progress...}}
\author{Boris Andrews}
\affil{Mathematical Institute, University of Oxford}
\date{\today}


\begin{document}
    \pagenumbering{gobble}
    \maketitle
    
    
    \begin{abstract}
        Magnetic confinement reactors---in particular tokamaks---offer one of the most promising options for achieving practical nuclear fusion, with the potential to provide virtually limitless, clean energy. The theoretical and numerical modeling of tokamak plasmas is simultaneously an essential component of effective reactor design, and a great research barrier. Tokamak operational conditions exhibit comparatively low Knudsen numbers. Kinetic effects, including kinetic waves and instabilities, Landau damping, bump-on-tail instabilities and more, are therefore highly influential in tokamak plasma dynamics. Purely fluid models are inherently incapable of capturing these effects, whereas the high dimensionality in purely kinetic models render them practically intractable for most relevant purposes.

        We consider a $\delta\!f$ decomposition model, with a macroscopic fluid background and microscopic kinetic correction, both fully coupled to each other. A similar manner of discretization is proposed to that used in the recent \texttt{STRUPHY} code \cite{Holderied_Possanner_Wang_2021, Holderied_2022, Li_et_al_2023} with a finite-element model for the background and a pseudo-particle/PiC model for the correction.

        The fluid background satisfies the full, non-linear, resistive, compressible, Hall MHD equations. \cite{Laakmann_Hu_Farrell_2022} introduces finite-element(-in-space) implicit timesteppers for the incompressible analogue to this system with structure-preserving (SP) properties in the ideal case, alongside parameter-robust preconditioners. We show that these timesteppers can derive from a finite-element-in-time (FET) (and finite-element-in-space) interpretation. The benefits of this reformulation are discussed, including the derivation of timesteppers that are higher order in time, and the quantifiable dissipative SP properties in the non-ideal, resistive case.
        
        We discuss possible options for extending this FET approach to timesteppers for the compressible case.

        The kinetic corrections satisfy linearized Boltzmann equations. Using a Lénard--Bernstein collision operator, these take Fokker--Planck-like forms \cite{Fokker_1914, Planck_1917} wherein pseudo-particles in the numerical model obey the neoclassical transport equations, with particle-independent Brownian drift terms. This offers a rigorous methodology for incorporating collisions into the particle transport model, without coupling the equations of motions for each particle.
        
        Works by Chen, Chacón et al. \cite{Chen_Chacón_Barnes_2011, Chacón_Chen_Barnes_2013, Chen_Chacón_2014, Chen_Chacón_2015} have developed structure-preserving particle pushers for neoclassical transport in the Vlasov equations, derived from Crank--Nicolson integrators. We show these too can can derive from a FET interpretation, similarly offering potential extensions to higher-order-in-time particle pushers. The FET formulation is used also to consider how the stochastic drift terms can be incorporated into the pushers. Stochastic gyrokinetic expansions are also discussed.

        Different options for the numerical implementation of these schemes are considered.

        Due to the efficacy of FET in the development of SP timesteppers for both the fluid and kinetic component, we hope this approach will prove effective in the future for developing SP timesteppers for the full hybrid model. We hope this will give us the opportunity to incorporate previously inaccessible kinetic effects into the highly effective, modern, finite-element MHD models.
    \end{abstract}
    
    
    \newpage
    \tableofcontents
    
    
    \newpage
    \pagenumbering{arabic}
    %\linenumbers\renewcommand\thelinenumber{\color{black!50}\arabic{linenumber}}
            \documentclass[12pt, a4paper]{report}

\input{template/main.tex}

\title{\BA{Title in Progress...}}
\author{Boris Andrews}
\affil{Mathematical Institute, University of Oxford}
\date{\today}


\begin{document}
    \pagenumbering{gobble}
    \maketitle
    
    
    \begin{abstract}
        Magnetic confinement reactors---in particular tokamaks---offer one of the most promising options for achieving practical nuclear fusion, with the potential to provide virtually limitless, clean energy. The theoretical and numerical modeling of tokamak plasmas is simultaneously an essential component of effective reactor design, and a great research barrier. Tokamak operational conditions exhibit comparatively low Knudsen numbers. Kinetic effects, including kinetic waves and instabilities, Landau damping, bump-on-tail instabilities and more, are therefore highly influential in tokamak plasma dynamics. Purely fluid models are inherently incapable of capturing these effects, whereas the high dimensionality in purely kinetic models render them practically intractable for most relevant purposes.

        We consider a $\delta\!f$ decomposition model, with a macroscopic fluid background and microscopic kinetic correction, both fully coupled to each other. A similar manner of discretization is proposed to that used in the recent \texttt{STRUPHY} code \cite{Holderied_Possanner_Wang_2021, Holderied_2022, Li_et_al_2023} with a finite-element model for the background and a pseudo-particle/PiC model for the correction.

        The fluid background satisfies the full, non-linear, resistive, compressible, Hall MHD equations. \cite{Laakmann_Hu_Farrell_2022} introduces finite-element(-in-space) implicit timesteppers for the incompressible analogue to this system with structure-preserving (SP) properties in the ideal case, alongside parameter-robust preconditioners. We show that these timesteppers can derive from a finite-element-in-time (FET) (and finite-element-in-space) interpretation. The benefits of this reformulation are discussed, including the derivation of timesteppers that are higher order in time, and the quantifiable dissipative SP properties in the non-ideal, resistive case.
        
        We discuss possible options for extending this FET approach to timesteppers for the compressible case.

        The kinetic corrections satisfy linearized Boltzmann equations. Using a Lénard--Bernstein collision operator, these take Fokker--Planck-like forms \cite{Fokker_1914, Planck_1917} wherein pseudo-particles in the numerical model obey the neoclassical transport equations, with particle-independent Brownian drift terms. This offers a rigorous methodology for incorporating collisions into the particle transport model, without coupling the equations of motions for each particle.
        
        Works by Chen, Chacón et al. \cite{Chen_Chacón_Barnes_2011, Chacón_Chen_Barnes_2013, Chen_Chacón_2014, Chen_Chacón_2015} have developed structure-preserving particle pushers for neoclassical transport in the Vlasov equations, derived from Crank--Nicolson integrators. We show these too can can derive from a FET interpretation, similarly offering potential extensions to higher-order-in-time particle pushers. The FET formulation is used also to consider how the stochastic drift terms can be incorporated into the pushers. Stochastic gyrokinetic expansions are also discussed.

        Different options for the numerical implementation of these schemes are considered.

        Due to the efficacy of FET in the development of SP timesteppers for both the fluid and kinetic component, we hope this approach will prove effective in the future for developing SP timesteppers for the full hybrid model. We hope this will give us the opportunity to incorporate previously inaccessible kinetic effects into the highly effective, modern, finite-element MHD models.
    \end{abstract}
    
    
    \newpage
    \tableofcontents
    
    
    \newpage
    \pagenumbering{arabic}
    %\linenumbers\renewcommand\thelinenumber{\color{black!50}\arabic{linenumber}}
            \input{0 - introduction/main.tex}
        \part{Research}
            \input{1 - low-noise PiC models/main.tex}
            \input{2 - kinetic component/main.tex}
            \input{3 - fluid component/main.tex}
            \input{4 - numerical implementation/main.tex}
        \part{Project Overview}
            \input{5 - research plan/main.tex}
            \input{6 - summary/main.tex}
    
    
    %\section{}
    \newpage
    \pagenumbering{gobble}
        \printbibliography


    \newpage
    \pagenumbering{roman}
    \appendix
        \part{Appendices}
            \input{8 - Hilbert complexes/main.tex}
            \input{9 - weak conservation proofs/main.tex}
\end{document}

        \part{Research}
            \documentclass[12pt, a4paper]{report}

\input{template/main.tex}

\title{\BA{Title in Progress...}}
\author{Boris Andrews}
\affil{Mathematical Institute, University of Oxford}
\date{\today}


\begin{document}
    \pagenumbering{gobble}
    \maketitle
    
    
    \begin{abstract}
        Magnetic confinement reactors---in particular tokamaks---offer one of the most promising options for achieving practical nuclear fusion, with the potential to provide virtually limitless, clean energy. The theoretical and numerical modeling of tokamak plasmas is simultaneously an essential component of effective reactor design, and a great research barrier. Tokamak operational conditions exhibit comparatively low Knudsen numbers. Kinetic effects, including kinetic waves and instabilities, Landau damping, bump-on-tail instabilities and more, are therefore highly influential in tokamak plasma dynamics. Purely fluid models are inherently incapable of capturing these effects, whereas the high dimensionality in purely kinetic models render them practically intractable for most relevant purposes.

        We consider a $\delta\!f$ decomposition model, with a macroscopic fluid background and microscopic kinetic correction, both fully coupled to each other. A similar manner of discretization is proposed to that used in the recent \texttt{STRUPHY} code \cite{Holderied_Possanner_Wang_2021, Holderied_2022, Li_et_al_2023} with a finite-element model for the background and a pseudo-particle/PiC model for the correction.

        The fluid background satisfies the full, non-linear, resistive, compressible, Hall MHD equations. \cite{Laakmann_Hu_Farrell_2022} introduces finite-element(-in-space) implicit timesteppers for the incompressible analogue to this system with structure-preserving (SP) properties in the ideal case, alongside parameter-robust preconditioners. We show that these timesteppers can derive from a finite-element-in-time (FET) (and finite-element-in-space) interpretation. The benefits of this reformulation are discussed, including the derivation of timesteppers that are higher order in time, and the quantifiable dissipative SP properties in the non-ideal, resistive case.
        
        We discuss possible options for extending this FET approach to timesteppers for the compressible case.

        The kinetic corrections satisfy linearized Boltzmann equations. Using a Lénard--Bernstein collision operator, these take Fokker--Planck-like forms \cite{Fokker_1914, Planck_1917} wherein pseudo-particles in the numerical model obey the neoclassical transport equations, with particle-independent Brownian drift terms. This offers a rigorous methodology for incorporating collisions into the particle transport model, without coupling the equations of motions for each particle.
        
        Works by Chen, Chacón et al. \cite{Chen_Chacón_Barnes_2011, Chacón_Chen_Barnes_2013, Chen_Chacón_2014, Chen_Chacón_2015} have developed structure-preserving particle pushers for neoclassical transport in the Vlasov equations, derived from Crank--Nicolson integrators. We show these too can can derive from a FET interpretation, similarly offering potential extensions to higher-order-in-time particle pushers. The FET formulation is used also to consider how the stochastic drift terms can be incorporated into the pushers. Stochastic gyrokinetic expansions are also discussed.

        Different options for the numerical implementation of these schemes are considered.

        Due to the efficacy of FET in the development of SP timesteppers for both the fluid and kinetic component, we hope this approach will prove effective in the future for developing SP timesteppers for the full hybrid model. We hope this will give us the opportunity to incorporate previously inaccessible kinetic effects into the highly effective, modern, finite-element MHD models.
    \end{abstract}
    
    
    \newpage
    \tableofcontents
    
    
    \newpage
    \pagenumbering{arabic}
    %\linenumbers\renewcommand\thelinenumber{\color{black!50}\arabic{linenumber}}
            \input{0 - introduction/main.tex}
        \part{Research}
            \input{1 - low-noise PiC models/main.tex}
            \input{2 - kinetic component/main.tex}
            \input{3 - fluid component/main.tex}
            \input{4 - numerical implementation/main.tex}
        \part{Project Overview}
            \input{5 - research plan/main.tex}
            \input{6 - summary/main.tex}
    
    
    %\section{}
    \newpage
    \pagenumbering{gobble}
        \printbibliography


    \newpage
    \pagenumbering{roman}
    \appendix
        \part{Appendices}
            \input{8 - Hilbert complexes/main.tex}
            \input{9 - weak conservation proofs/main.tex}
\end{document}

            \documentclass[12pt, a4paper]{report}

\input{template/main.tex}

\title{\BA{Title in Progress...}}
\author{Boris Andrews}
\affil{Mathematical Institute, University of Oxford}
\date{\today}


\begin{document}
    \pagenumbering{gobble}
    \maketitle
    
    
    \begin{abstract}
        Magnetic confinement reactors---in particular tokamaks---offer one of the most promising options for achieving practical nuclear fusion, with the potential to provide virtually limitless, clean energy. The theoretical and numerical modeling of tokamak plasmas is simultaneously an essential component of effective reactor design, and a great research barrier. Tokamak operational conditions exhibit comparatively low Knudsen numbers. Kinetic effects, including kinetic waves and instabilities, Landau damping, bump-on-tail instabilities and more, are therefore highly influential in tokamak plasma dynamics. Purely fluid models are inherently incapable of capturing these effects, whereas the high dimensionality in purely kinetic models render them practically intractable for most relevant purposes.

        We consider a $\delta\!f$ decomposition model, with a macroscopic fluid background and microscopic kinetic correction, both fully coupled to each other. A similar manner of discretization is proposed to that used in the recent \texttt{STRUPHY} code \cite{Holderied_Possanner_Wang_2021, Holderied_2022, Li_et_al_2023} with a finite-element model for the background and a pseudo-particle/PiC model for the correction.

        The fluid background satisfies the full, non-linear, resistive, compressible, Hall MHD equations. \cite{Laakmann_Hu_Farrell_2022} introduces finite-element(-in-space) implicit timesteppers for the incompressible analogue to this system with structure-preserving (SP) properties in the ideal case, alongside parameter-robust preconditioners. We show that these timesteppers can derive from a finite-element-in-time (FET) (and finite-element-in-space) interpretation. The benefits of this reformulation are discussed, including the derivation of timesteppers that are higher order in time, and the quantifiable dissipative SP properties in the non-ideal, resistive case.
        
        We discuss possible options for extending this FET approach to timesteppers for the compressible case.

        The kinetic corrections satisfy linearized Boltzmann equations. Using a Lénard--Bernstein collision operator, these take Fokker--Planck-like forms \cite{Fokker_1914, Planck_1917} wherein pseudo-particles in the numerical model obey the neoclassical transport equations, with particle-independent Brownian drift terms. This offers a rigorous methodology for incorporating collisions into the particle transport model, without coupling the equations of motions for each particle.
        
        Works by Chen, Chacón et al. \cite{Chen_Chacón_Barnes_2011, Chacón_Chen_Barnes_2013, Chen_Chacón_2014, Chen_Chacón_2015} have developed structure-preserving particle pushers for neoclassical transport in the Vlasov equations, derived from Crank--Nicolson integrators. We show these too can can derive from a FET interpretation, similarly offering potential extensions to higher-order-in-time particle pushers. The FET formulation is used also to consider how the stochastic drift terms can be incorporated into the pushers. Stochastic gyrokinetic expansions are also discussed.

        Different options for the numerical implementation of these schemes are considered.

        Due to the efficacy of FET in the development of SP timesteppers for both the fluid and kinetic component, we hope this approach will prove effective in the future for developing SP timesteppers for the full hybrid model. We hope this will give us the opportunity to incorporate previously inaccessible kinetic effects into the highly effective, modern, finite-element MHD models.
    \end{abstract}
    
    
    \newpage
    \tableofcontents
    
    
    \newpage
    \pagenumbering{arabic}
    %\linenumbers\renewcommand\thelinenumber{\color{black!50}\arabic{linenumber}}
            \input{0 - introduction/main.tex}
        \part{Research}
            \input{1 - low-noise PiC models/main.tex}
            \input{2 - kinetic component/main.tex}
            \input{3 - fluid component/main.tex}
            \input{4 - numerical implementation/main.tex}
        \part{Project Overview}
            \input{5 - research plan/main.tex}
            \input{6 - summary/main.tex}
    
    
    %\section{}
    \newpage
    \pagenumbering{gobble}
        \printbibliography


    \newpage
    \pagenumbering{roman}
    \appendix
        \part{Appendices}
            \input{8 - Hilbert complexes/main.tex}
            \input{9 - weak conservation proofs/main.tex}
\end{document}

            \documentclass[12pt, a4paper]{report}

\input{template/main.tex}

\title{\BA{Title in Progress...}}
\author{Boris Andrews}
\affil{Mathematical Institute, University of Oxford}
\date{\today}


\begin{document}
    \pagenumbering{gobble}
    \maketitle
    
    
    \begin{abstract}
        Magnetic confinement reactors---in particular tokamaks---offer one of the most promising options for achieving practical nuclear fusion, with the potential to provide virtually limitless, clean energy. The theoretical and numerical modeling of tokamak plasmas is simultaneously an essential component of effective reactor design, and a great research barrier. Tokamak operational conditions exhibit comparatively low Knudsen numbers. Kinetic effects, including kinetic waves and instabilities, Landau damping, bump-on-tail instabilities and more, are therefore highly influential in tokamak plasma dynamics. Purely fluid models are inherently incapable of capturing these effects, whereas the high dimensionality in purely kinetic models render them practically intractable for most relevant purposes.

        We consider a $\delta\!f$ decomposition model, with a macroscopic fluid background and microscopic kinetic correction, both fully coupled to each other. A similar manner of discretization is proposed to that used in the recent \texttt{STRUPHY} code \cite{Holderied_Possanner_Wang_2021, Holderied_2022, Li_et_al_2023} with a finite-element model for the background and a pseudo-particle/PiC model for the correction.

        The fluid background satisfies the full, non-linear, resistive, compressible, Hall MHD equations. \cite{Laakmann_Hu_Farrell_2022} introduces finite-element(-in-space) implicit timesteppers for the incompressible analogue to this system with structure-preserving (SP) properties in the ideal case, alongside parameter-robust preconditioners. We show that these timesteppers can derive from a finite-element-in-time (FET) (and finite-element-in-space) interpretation. The benefits of this reformulation are discussed, including the derivation of timesteppers that are higher order in time, and the quantifiable dissipative SP properties in the non-ideal, resistive case.
        
        We discuss possible options for extending this FET approach to timesteppers for the compressible case.

        The kinetic corrections satisfy linearized Boltzmann equations. Using a Lénard--Bernstein collision operator, these take Fokker--Planck-like forms \cite{Fokker_1914, Planck_1917} wherein pseudo-particles in the numerical model obey the neoclassical transport equations, with particle-independent Brownian drift terms. This offers a rigorous methodology for incorporating collisions into the particle transport model, without coupling the equations of motions for each particle.
        
        Works by Chen, Chacón et al. \cite{Chen_Chacón_Barnes_2011, Chacón_Chen_Barnes_2013, Chen_Chacón_2014, Chen_Chacón_2015} have developed structure-preserving particle pushers for neoclassical transport in the Vlasov equations, derived from Crank--Nicolson integrators. We show these too can can derive from a FET interpretation, similarly offering potential extensions to higher-order-in-time particle pushers. The FET formulation is used also to consider how the stochastic drift terms can be incorporated into the pushers. Stochastic gyrokinetic expansions are also discussed.

        Different options for the numerical implementation of these schemes are considered.

        Due to the efficacy of FET in the development of SP timesteppers for both the fluid and kinetic component, we hope this approach will prove effective in the future for developing SP timesteppers for the full hybrid model. We hope this will give us the opportunity to incorporate previously inaccessible kinetic effects into the highly effective, modern, finite-element MHD models.
    \end{abstract}
    
    
    \newpage
    \tableofcontents
    
    
    \newpage
    \pagenumbering{arabic}
    %\linenumbers\renewcommand\thelinenumber{\color{black!50}\arabic{linenumber}}
            \input{0 - introduction/main.tex}
        \part{Research}
            \input{1 - low-noise PiC models/main.tex}
            \input{2 - kinetic component/main.tex}
            \input{3 - fluid component/main.tex}
            \input{4 - numerical implementation/main.tex}
        \part{Project Overview}
            \input{5 - research plan/main.tex}
            \input{6 - summary/main.tex}
    
    
    %\section{}
    \newpage
    \pagenumbering{gobble}
        \printbibliography


    \newpage
    \pagenumbering{roman}
    \appendix
        \part{Appendices}
            \input{8 - Hilbert complexes/main.tex}
            \input{9 - weak conservation proofs/main.tex}
\end{document}

            \documentclass[12pt, a4paper]{report}

\input{template/main.tex}

\title{\BA{Title in Progress...}}
\author{Boris Andrews}
\affil{Mathematical Institute, University of Oxford}
\date{\today}


\begin{document}
    \pagenumbering{gobble}
    \maketitle
    
    
    \begin{abstract}
        Magnetic confinement reactors---in particular tokamaks---offer one of the most promising options for achieving practical nuclear fusion, with the potential to provide virtually limitless, clean energy. The theoretical and numerical modeling of tokamak plasmas is simultaneously an essential component of effective reactor design, and a great research barrier. Tokamak operational conditions exhibit comparatively low Knudsen numbers. Kinetic effects, including kinetic waves and instabilities, Landau damping, bump-on-tail instabilities and more, are therefore highly influential in tokamak plasma dynamics. Purely fluid models are inherently incapable of capturing these effects, whereas the high dimensionality in purely kinetic models render them practically intractable for most relevant purposes.

        We consider a $\delta\!f$ decomposition model, with a macroscopic fluid background and microscopic kinetic correction, both fully coupled to each other. A similar manner of discretization is proposed to that used in the recent \texttt{STRUPHY} code \cite{Holderied_Possanner_Wang_2021, Holderied_2022, Li_et_al_2023} with a finite-element model for the background and a pseudo-particle/PiC model for the correction.

        The fluid background satisfies the full, non-linear, resistive, compressible, Hall MHD equations. \cite{Laakmann_Hu_Farrell_2022} introduces finite-element(-in-space) implicit timesteppers for the incompressible analogue to this system with structure-preserving (SP) properties in the ideal case, alongside parameter-robust preconditioners. We show that these timesteppers can derive from a finite-element-in-time (FET) (and finite-element-in-space) interpretation. The benefits of this reformulation are discussed, including the derivation of timesteppers that are higher order in time, and the quantifiable dissipative SP properties in the non-ideal, resistive case.
        
        We discuss possible options for extending this FET approach to timesteppers for the compressible case.

        The kinetic corrections satisfy linearized Boltzmann equations. Using a Lénard--Bernstein collision operator, these take Fokker--Planck-like forms \cite{Fokker_1914, Planck_1917} wherein pseudo-particles in the numerical model obey the neoclassical transport equations, with particle-independent Brownian drift terms. This offers a rigorous methodology for incorporating collisions into the particle transport model, without coupling the equations of motions for each particle.
        
        Works by Chen, Chacón et al. \cite{Chen_Chacón_Barnes_2011, Chacón_Chen_Barnes_2013, Chen_Chacón_2014, Chen_Chacón_2015} have developed structure-preserving particle pushers for neoclassical transport in the Vlasov equations, derived from Crank--Nicolson integrators. We show these too can can derive from a FET interpretation, similarly offering potential extensions to higher-order-in-time particle pushers. The FET formulation is used also to consider how the stochastic drift terms can be incorporated into the pushers. Stochastic gyrokinetic expansions are also discussed.

        Different options for the numerical implementation of these schemes are considered.

        Due to the efficacy of FET in the development of SP timesteppers for both the fluid and kinetic component, we hope this approach will prove effective in the future for developing SP timesteppers for the full hybrid model. We hope this will give us the opportunity to incorporate previously inaccessible kinetic effects into the highly effective, modern, finite-element MHD models.
    \end{abstract}
    
    
    \newpage
    \tableofcontents
    
    
    \newpage
    \pagenumbering{arabic}
    %\linenumbers\renewcommand\thelinenumber{\color{black!50}\arabic{linenumber}}
            \input{0 - introduction/main.tex}
        \part{Research}
            \input{1 - low-noise PiC models/main.tex}
            \input{2 - kinetic component/main.tex}
            \input{3 - fluid component/main.tex}
            \input{4 - numerical implementation/main.tex}
        \part{Project Overview}
            \input{5 - research plan/main.tex}
            \input{6 - summary/main.tex}
    
    
    %\section{}
    \newpage
    \pagenumbering{gobble}
        \printbibliography


    \newpage
    \pagenumbering{roman}
    \appendix
        \part{Appendices}
            \input{8 - Hilbert complexes/main.tex}
            \input{9 - weak conservation proofs/main.tex}
\end{document}

        \part{Project Overview}
            \documentclass[12pt, a4paper]{report}

\input{template/main.tex}

\title{\BA{Title in Progress...}}
\author{Boris Andrews}
\affil{Mathematical Institute, University of Oxford}
\date{\today}


\begin{document}
    \pagenumbering{gobble}
    \maketitle
    
    
    \begin{abstract}
        Magnetic confinement reactors---in particular tokamaks---offer one of the most promising options for achieving practical nuclear fusion, with the potential to provide virtually limitless, clean energy. The theoretical and numerical modeling of tokamak plasmas is simultaneously an essential component of effective reactor design, and a great research barrier. Tokamak operational conditions exhibit comparatively low Knudsen numbers. Kinetic effects, including kinetic waves and instabilities, Landau damping, bump-on-tail instabilities and more, are therefore highly influential in tokamak plasma dynamics. Purely fluid models are inherently incapable of capturing these effects, whereas the high dimensionality in purely kinetic models render them practically intractable for most relevant purposes.

        We consider a $\delta\!f$ decomposition model, with a macroscopic fluid background and microscopic kinetic correction, both fully coupled to each other. A similar manner of discretization is proposed to that used in the recent \texttt{STRUPHY} code \cite{Holderied_Possanner_Wang_2021, Holderied_2022, Li_et_al_2023} with a finite-element model for the background and a pseudo-particle/PiC model for the correction.

        The fluid background satisfies the full, non-linear, resistive, compressible, Hall MHD equations. \cite{Laakmann_Hu_Farrell_2022} introduces finite-element(-in-space) implicit timesteppers for the incompressible analogue to this system with structure-preserving (SP) properties in the ideal case, alongside parameter-robust preconditioners. We show that these timesteppers can derive from a finite-element-in-time (FET) (and finite-element-in-space) interpretation. The benefits of this reformulation are discussed, including the derivation of timesteppers that are higher order in time, and the quantifiable dissipative SP properties in the non-ideal, resistive case.
        
        We discuss possible options for extending this FET approach to timesteppers for the compressible case.

        The kinetic corrections satisfy linearized Boltzmann equations. Using a Lénard--Bernstein collision operator, these take Fokker--Planck-like forms \cite{Fokker_1914, Planck_1917} wherein pseudo-particles in the numerical model obey the neoclassical transport equations, with particle-independent Brownian drift terms. This offers a rigorous methodology for incorporating collisions into the particle transport model, without coupling the equations of motions for each particle.
        
        Works by Chen, Chacón et al. \cite{Chen_Chacón_Barnes_2011, Chacón_Chen_Barnes_2013, Chen_Chacón_2014, Chen_Chacón_2015} have developed structure-preserving particle pushers for neoclassical transport in the Vlasov equations, derived from Crank--Nicolson integrators. We show these too can can derive from a FET interpretation, similarly offering potential extensions to higher-order-in-time particle pushers. The FET formulation is used also to consider how the stochastic drift terms can be incorporated into the pushers. Stochastic gyrokinetic expansions are also discussed.

        Different options for the numerical implementation of these schemes are considered.

        Due to the efficacy of FET in the development of SP timesteppers for both the fluid and kinetic component, we hope this approach will prove effective in the future for developing SP timesteppers for the full hybrid model. We hope this will give us the opportunity to incorporate previously inaccessible kinetic effects into the highly effective, modern, finite-element MHD models.
    \end{abstract}
    
    
    \newpage
    \tableofcontents
    
    
    \newpage
    \pagenumbering{arabic}
    %\linenumbers\renewcommand\thelinenumber{\color{black!50}\arabic{linenumber}}
            \input{0 - introduction/main.tex}
        \part{Research}
            \input{1 - low-noise PiC models/main.tex}
            \input{2 - kinetic component/main.tex}
            \input{3 - fluid component/main.tex}
            \input{4 - numerical implementation/main.tex}
        \part{Project Overview}
            \input{5 - research plan/main.tex}
            \input{6 - summary/main.tex}
    
    
    %\section{}
    \newpage
    \pagenumbering{gobble}
        \printbibliography


    \newpage
    \pagenumbering{roman}
    \appendix
        \part{Appendices}
            \input{8 - Hilbert complexes/main.tex}
            \input{9 - weak conservation proofs/main.tex}
\end{document}

            \documentclass[12pt, a4paper]{report}

\input{template/main.tex}

\title{\BA{Title in Progress...}}
\author{Boris Andrews}
\affil{Mathematical Institute, University of Oxford}
\date{\today}


\begin{document}
    \pagenumbering{gobble}
    \maketitle
    
    
    \begin{abstract}
        Magnetic confinement reactors---in particular tokamaks---offer one of the most promising options for achieving practical nuclear fusion, with the potential to provide virtually limitless, clean energy. The theoretical and numerical modeling of tokamak plasmas is simultaneously an essential component of effective reactor design, and a great research barrier. Tokamak operational conditions exhibit comparatively low Knudsen numbers. Kinetic effects, including kinetic waves and instabilities, Landau damping, bump-on-tail instabilities and more, are therefore highly influential in tokamak plasma dynamics. Purely fluid models are inherently incapable of capturing these effects, whereas the high dimensionality in purely kinetic models render them practically intractable for most relevant purposes.

        We consider a $\delta\!f$ decomposition model, with a macroscopic fluid background and microscopic kinetic correction, both fully coupled to each other. A similar manner of discretization is proposed to that used in the recent \texttt{STRUPHY} code \cite{Holderied_Possanner_Wang_2021, Holderied_2022, Li_et_al_2023} with a finite-element model for the background and a pseudo-particle/PiC model for the correction.

        The fluid background satisfies the full, non-linear, resistive, compressible, Hall MHD equations. \cite{Laakmann_Hu_Farrell_2022} introduces finite-element(-in-space) implicit timesteppers for the incompressible analogue to this system with structure-preserving (SP) properties in the ideal case, alongside parameter-robust preconditioners. We show that these timesteppers can derive from a finite-element-in-time (FET) (and finite-element-in-space) interpretation. The benefits of this reformulation are discussed, including the derivation of timesteppers that are higher order in time, and the quantifiable dissipative SP properties in the non-ideal, resistive case.
        
        We discuss possible options for extending this FET approach to timesteppers for the compressible case.

        The kinetic corrections satisfy linearized Boltzmann equations. Using a Lénard--Bernstein collision operator, these take Fokker--Planck-like forms \cite{Fokker_1914, Planck_1917} wherein pseudo-particles in the numerical model obey the neoclassical transport equations, with particle-independent Brownian drift terms. This offers a rigorous methodology for incorporating collisions into the particle transport model, without coupling the equations of motions for each particle.
        
        Works by Chen, Chacón et al. \cite{Chen_Chacón_Barnes_2011, Chacón_Chen_Barnes_2013, Chen_Chacón_2014, Chen_Chacón_2015} have developed structure-preserving particle pushers for neoclassical transport in the Vlasov equations, derived from Crank--Nicolson integrators. We show these too can can derive from a FET interpretation, similarly offering potential extensions to higher-order-in-time particle pushers. The FET formulation is used also to consider how the stochastic drift terms can be incorporated into the pushers. Stochastic gyrokinetic expansions are also discussed.

        Different options for the numerical implementation of these schemes are considered.

        Due to the efficacy of FET in the development of SP timesteppers for both the fluid and kinetic component, we hope this approach will prove effective in the future for developing SP timesteppers for the full hybrid model. We hope this will give us the opportunity to incorporate previously inaccessible kinetic effects into the highly effective, modern, finite-element MHD models.
    \end{abstract}
    
    
    \newpage
    \tableofcontents
    
    
    \newpage
    \pagenumbering{arabic}
    %\linenumbers\renewcommand\thelinenumber{\color{black!50}\arabic{linenumber}}
            \input{0 - introduction/main.tex}
        \part{Research}
            \input{1 - low-noise PiC models/main.tex}
            \input{2 - kinetic component/main.tex}
            \input{3 - fluid component/main.tex}
            \input{4 - numerical implementation/main.tex}
        \part{Project Overview}
            \input{5 - research plan/main.tex}
            \input{6 - summary/main.tex}
    
    
    %\section{}
    \newpage
    \pagenumbering{gobble}
        \printbibliography


    \newpage
    \pagenumbering{roman}
    \appendix
        \part{Appendices}
            \input{8 - Hilbert complexes/main.tex}
            \input{9 - weak conservation proofs/main.tex}
\end{document}

    
    
    %\section{}
    \newpage
    \pagenumbering{gobble}
        \printbibliography


    \newpage
    \pagenumbering{roman}
    \appendix
        \part{Appendices}
            \documentclass[12pt, a4paper]{report}

\input{template/main.tex}

\title{\BA{Title in Progress...}}
\author{Boris Andrews}
\affil{Mathematical Institute, University of Oxford}
\date{\today}


\begin{document}
    \pagenumbering{gobble}
    \maketitle
    
    
    \begin{abstract}
        Magnetic confinement reactors---in particular tokamaks---offer one of the most promising options for achieving practical nuclear fusion, with the potential to provide virtually limitless, clean energy. The theoretical and numerical modeling of tokamak plasmas is simultaneously an essential component of effective reactor design, and a great research barrier. Tokamak operational conditions exhibit comparatively low Knudsen numbers. Kinetic effects, including kinetic waves and instabilities, Landau damping, bump-on-tail instabilities and more, are therefore highly influential in tokamak plasma dynamics. Purely fluid models are inherently incapable of capturing these effects, whereas the high dimensionality in purely kinetic models render them practically intractable for most relevant purposes.

        We consider a $\delta\!f$ decomposition model, with a macroscopic fluid background and microscopic kinetic correction, both fully coupled to each other. A similar manner of discretization is proposed to that used in the recent \texttt{STRUPHY} code \cite{Holderied_Possanner_Wang_2021, Holderied_2022, Li_et_al_2023} with a finite-element model for the background and a pseudo-particle/PiC model for the correction.

        The fluid background satisfies the full, non-linear, resistive, compressible, Hall MHD equations. \cite{Laakmann_Hu_Farrell_2022} introduces finite-element(-in-space) implicit timesteppers for the incompressible analogue to this system with structure-preserving (SP) properties in the ideal case, alongside parameter-robust preconditioners. We show that these timesteppers can derive from a finite-element-in-time (FET) (and finite-element-in-space) interpretation. The benefits of this reformulation are discussed, including the derivation of timesteppers that are higher order in time, and the quantifiable dissipative SP properties in the non-ideal, resistive case.
        
        We discuss possible options for extending this FET approach to timesteppers for the compressible case.

        The kinetic corrections satisfy linearized Boltzmann equations. Using a Lénard--Bernstein collision operator, these take Fokker--Planck-like forms \cite{Fokker_1914, Planck_1917} wherein pseudo-particles in the numerical model obey the neoclassical transport equations, with particle-independent Brownian drift terms. This offers a rigorous methodology for incorporating collisions into the particle transport model, without coupling the equations of motions for each particle.
        
        Works by Chen, Chacón et al. \cite{Chen_Chacón_Barnes_2011, Chacón_Chen_Barnes_2013, Chen_Chacón_2014, Chen_Chacón_2015} have developed structure-preserving particle pushers for neoclassical transport in the Vlasov equations, derived from Crank--Nicolson integrators. We show these too can can derive from a FET interpretation, similarly offering potential extensions to higher-order-in-time particle pushers. The FET formulation is used also to consider how the stochastic drift terms can be incorporated into the pushers. Stochastic gyrokinetic expansions are also discussed.

        Different options for the numerical implementation of these schemes are considered.

        Due to the efficacy of FET in the development of SP timesteppers for both the fluid and kinetic component, we hope this approach will prove effective in the future for developing SP timesteppers for the full hybrid model. We hope this will give us the opportunity to incorporate previously inaccessible kinetic effects into the highly effective, modern, finite-element MHD models.
    \end{abstract}
    
    
    \newpage
    \tableofcontents
    
    
    \newpage
    \pagenumbering{arabic}
    %\linenumbers\renewcommand\thelinenumber{\color{black!50}\arabic{linenumber}}
            \input{0 - introduction/main.tex}
        \part{Research}
            \input{1 - low-noise PiC models/main.tex}
            \input{2 - kinetic component/main.tex}
            \input{3 - fluid component/main.tex}
            \input{4 - numerical implementation/main.tex}
        \part{Project Overview}
            \input{5 - research plan/main.tex}
            \input{6 - summary/main.tex}
    
    
    %\section{}
    \newpage
    \pagenumbering{gobble}
        \printbibliography


    \newpage
    \pagenumbering{roman}
    \appendix
        \part{Appendices}
            \input{8 - Hilbert complexes/main.tex}
            \input{9 - weak conservation proofs/main.tex}
\end{document}

            \documentclass[12pt, a4paper]{report}

\input{template/main.tex}

\title{\BA{Title in Progress...}}
\author{Boris Andrews}
\affil{Mathematical Institute, University of Oxford}
\date{\today}


\begin{document}
    \pagenumbering{gobble}
    \maketitle
    
    
    \begin{abstract}
        Magnetic confinement reactors---in particular tokamaks---offer one of the most promising options for achieving practical nuclear fusion, with the potential to provide virtually limitless, clean energy. The theoretical and numerical modeling of tokamak plasmas is simultaneously an essential component of effective reactor design, and a great research barrier. Tokamak operational conditions exhibit comparatively low Knudsen numbers. Kinetic effects, including kinetic waves and instabilities, Landau damping, bump-on-tail instabilities and more, are therefore highly influential in tokamak plasma dynamics. Purely fluid models are inherently incapable of capturing these effects, whereas the high dimensionality in purely kinetic models render them practically intractable for most relevant purposes.

        We consider a $\delta\!f$ decomposition model, with a macroscopic fluid background and microscopic kinetic correction, both fully coupled to each other. A similar manner of discretization is proposed to that used in the recent \texttt{STRUPHY} code \cite{Holderied_Possanner_Wang_2021, Holderied_2022, Li_et_al_2023} with a finite-element model for the background and a pseudo-particle/PiC model for the correction.

        The fluid background satisfies the full, non-linear, resistive, compressible, Hall MHD equations. \cite{Laakmann_Hu_Farrell_2022} introduces finite-element(-in-space) implicit timesteppers for the incompressible analogue to this system with structure-preserving (SP) properties in the ideal case, alongside parameter-robust preconditioners. We show that these timesteppers can derive from a finite-element-in-time (FET) (and finite-element-in-space) interpretation. The benefits of this reformulation are discussed, including the derivation of timesteppers that are higher order in time, and the quantifiable dissipative SP properties in the non-ideal, resistive case.
        
        We discuss possible options for extending this FET approach to timesteppers for the compressible case.

        The kinetic corrections satisfy linearized Boltzmann equations. Using a Lénard--Bernstein collision operator, these take Fokker--Planck-like forms \cite{Fokker_1914, Planck_1917} wherein pseudo-particles in the numerical model obey the neoclassical transport equations, with particle-independent Brownian drift terms. This offers a rigorous methodology for incorporating collisions into the particle transport model, without coupling the equations of motions for each particle.
        
        Works by Chen, Chacón et al. \cite{Chen_Chacón_Barnes_2011, Chacón_Chen_Barnes_2013, Chen_Chacón_2014, Chen_Chacón_2015} have developed structure-preserving particle pushers for neoclassical transport in the Vlasov equations, derived from Crank--Nicolson integrators. We show these too can can derive from a FET interpretation, similarly offering potential extensions to higher-order-in-time particle pushers. The FET formulation is used also to consider how the stochastic drift terms can be incorporated into the pushers. Stochastic gyrokinetic expansions are also discussed.

        Different options for the numerical implementation of these schemes are considered.

        Due to the efficacy of FET in the development of SP timesteppers for both the fluid and kinetic component, we hope this approach will prove effective in the future for developing SP timesteppers for the full hybrid model. We hope this will give us the opportunity to incorporate previously inaccessible kinetic effects into the highly effective, modern, finite-element MHD models.
    \end{abstract}
    
    
    \newpage
    \tableofcontents
    
    
    \newpage
    \pagenumbering{arabic}
    %\linenumbers\renewcommand\thelinenumber{\color{black!50}\arabic{linenumber}}
            \input{0 - introduction/main.tex}
        \part{Research}
            \input{1 - low-noise PiC models/main.tex}
            \input{2 - kinetic component/main.tex}
            \input{3 - fluid component/main.tex}
            \input{4 - numerical implementation/main.tex}
        \part{Project Overview}
            \input{5 - research plan/main.tex}
            \input{6 - summary/main.tex}
    
    
    %\section{}
    \newpage
    \pagenumbering{gobble}
        \printbibliography


    \newpage
    \pagenumbering{roman}
    \appendix
        \part{Appendices}
            \input{8 - Hilbert complexes/main.tex}
            \input{9 - weak conservation proofs/main.tex}
\end{document}

\end{document}

            \documentclass[12pt, a4paper]{report}

\documentclass[12pt, a4paper]{report}

\input{template/main.tex}

\title{\BA{Title in Progress...}}
\author{Boris Andrews}
\affil{Mathematical Institute, University of Oxford}
\date{\today}


\begin{document}
    \pagenumbering{gobble}
    \maketitle
    
    
    \begin{abstract}
        Magnetic confinement reactors---in particular tokamaks---offer one of the most promising options for achieving practical nuclear fusion, with the potential to provide virtually limitless, clean energy. The theoretical and numerical modeling of tokamak plasmas is simultaneously an essential component of effective reactor design, and a great research barrier. Tokamak operational conditions exhibit comparatively low Knudsen numbers. Kinetic effects, including kinetic waves and instabilities, Landau damping, bump-on-tail instabilities and more, are therefore highly influential in tokamak plasma dynamics. Purely fluid models are inherently incapable of capturing these effects, whereas the high dimensionality in purely kinetic models render them practically intractable for most relevant purposes.

        We consider a $\delta\!f$ decomposition model, with a macroscopic fluid background and microscopic kinetic correction, both fully coupled to each other. A similar manner of discretization is proposed to that used in the recent \texttt{STRUPHY} code \cite{Holderied_Possanner_Wang_2021, Holderied_2022, Li_et_al_2023} with a finite-element model for the background and a pseudo-particle/PiC model for the correction.

        The fluid background satisfies the full, non-linear, resistive, compressible, Hall MHD equations. \cite{Laakmann_Hu_Farrell_2022} introduces finite-element(-in-space) implicit timesteppers for the incompressible analogue to this system with structure-preserving (SP) properties in the ideal case, alongside parameter-robust preconditioners. We show that these timesteppers can derive from a finite-element-in-time (FET) (and finite-element-in-space) interpretation. The benefits of this reformulation are discussed, including the derivation of timesteppers that are higher order in time, and the quantifiable dissipative SP properties in the non-ideal, resistive case.
        
        We discuss possible options for extending this FET approach to timesteppers for the compressible case.

        The kinetic corrections satisfy linearized Boltzmann equations. Using a Lénard--Bernstein collision operator, these take Fokker--Planck-like forms \cite{Fokker_1914, Planck_1917} wherein pseudo-particles in the numerical model obey the neoclassical transport equations, with particle-independent Brownian drift terms. This offers a rigorous methodology for incorporating collisions into the particle transport model, without coupling the equations of motions for each particle.
        
        Works by Chen, Chacón et al. \cite{Chen_Chacón_Barnes_2011, Chacón_Chen_Barnes_2013, Chen_Chacón_2014, Chen_Chacón_2015} have developed structure-preserving particle pushers for neoclassical transport in the Vlasov equations, derived from Crank--Nicolson integrators. We show these too can can derive from a FET interpretation, similarly offering potential extensions to higher-order-in-time particle pushers. The FET formulation is used also to consider how the stochastic drift terms can be incorporated into the pushers. Stochastic gyrokinetic expansions are also discussed.

        Different options for the numerical implementation of these schemes are considered.

        Due to the efficacy of FET in the development of SP timesteppers for both the fluid and kinetic component, we hope this approach will prove effective in the future for developing SP timesteppers for the full hybrid model. We hope this will give us the opportunity to incorporate previously inaccessible kinetic effects into the highly effective, modern, finite-element MHD models.
    \end{abstract}
    
    
    \newpage
    \tableofcontents
    
    
    \newpage
    \pagenumbering{arabic}
    %\linenumbers\renewcommand\thelinenumber{\color{black!50}\arabic{linenumber}}
            \input{0 - introduction/main.tex}
        \part{Research}
            \input{1 - low-noise PiC models/main.tex}
            \input{2 - kinetic component/main.tex}
            \input{3 - fluid component/main.tex}
            \input{4 - numerical implementation/main.tex}
        \part{Project Overview}
            \input{5 - research plan/main.tex}
            \input{6 - summary/main.tex}
    
    
    %\section{}
    \newpage
    \pagenumbering{gobble}
        \printbibliography


    \newpage
    \pagenumbering{roman}
    \appendix
        \part{Appendices}
            \input{8 - Hilbert complexes/main.tex}
            \input{9 - weak conservation proofs/main.tex}
\end{document}


\title{\BA{Title in Progress...}}
\author{Boris Andrews}
\affil{Mathematical Institute, University of Oxford}
\date{\today}


\begin{document}
    \pagenumbering{gobble}
    \maketitle
    
    
    \begin{abstract}
        Magnetic confinement reactors---in particular tokamaks---offer one of the most promising options for achieving practical nuclear fusion, with the potential to provide virtually limitless, clean energy. The theoretical and numerical modeling of tokamak plasmas is simultaneously an essential component of effective reactor design, and a great research barrier. Tokamak operational conditions exhibit comparatively low Knudsen numbers. Kinetic effects, including kinetic waves and instabilities, Landau damping, bump-on-tail instabilities and more, are therefore highly influential in tokamak plasma dynamics. Purely fluid models are inherently incapable of capturing these effects, whereas the high dimensionality in purely kinetic models render them practically intractable for most relevant purposes.

        We consider a $\delta\!f$ decomposition model, with a macroscopic fluid background and microscopic kinetic correction, both fully coupled to each other. A similar manner of discretization is proposed to that used in the recent \texttt{STRUPHY} code \cite{Holderied_Possanner_Wang_2021, Holderied_2022, Li_et_al_2023} with a finite-element model for the background and a pseudo-particle/PiC model for the correction.

        The fluid background satisfies the full, non-linear, resistive, compressible, Hall MHD equations. \cite{Laakmann_Hu_Farrell_2022} introduces finite-element(-in-space) implicit timesteppers for the incompressible analogue to this system with structure-preserving (SP) properties in the ideal case, alongside parameter-robust preconditioners. We show that these timesteppers can derive from a finite-element-in-time (FET) (and finite-element-in-space) interpretation. The benefits of this reformulation are discussed, including the derivation of timesteppers that are higher order in time, and the quantifiable dissipative SP properties in the non-ideal, resistive case.
        
        We discuss possible options for extending this FET approach to timesteppers for the compressible case.

        The kinetic corrections satisfy linearized Boltzmann equations. Using a Lénard--Bernstein collision operator, these take Fokker--Planck-like forms \cite{Fokker_1914, Planck_1917} wherein pseudo-particles in the numerical model obey the neoclassical transport equations, with particle-independent Brownian drift terms. This offers a rigorous methodology for incorporating collisions into the particle transport model, without coupling the equations of motions for each particle.
        
        Works by Chen, Chacón et al. \cite{Chen_Chacón_Barnes_2011, Chacón_Chen_Barnes_2013, Chen_Chacón_2014, Chen_Chacón_2015} have developed structure-preserving particle pushers for neoclassical transport in the Vlasov equations, derived from Crank--Nicolson integrators. We show these too can can derive from a FET interpretation, similarly offering potential extensions to higher-order-in-time particle pushers. The FET formulation is used also to consider how the stochastic drift terms can be incorporated into the pushers. Stochastic gyrokinetic expansions are also discussed.

        Different options for the numerical implementation of these schemes are considered.

        Due to the efficacy of FET in the development of SP timesteppers for both the fluid and kinetic component, we hope this approach will prove effective in the future for developing SP timesteppers for the full hybrid model. We hope this will give us the opportunity to incorporate previously inaccessible kinetic effects into the highly effective, modern, finite-element MHD models.
    \end{abstract}
    
    
    \newpage
    \tableofcontents
    
    
    \newpage
    \pagenumbering{arabic}
    %\linenumbers\renewcommand\thelinenumber{\color{black!50}\arabic{linenumber}}
            \documentclass[12pt, a4paper]{report}

\input{template/main.tex}

\title{\BA{Title in Progress...}}
\author{Boris Andrews}
\affil{Mathematical Institute, University of Oxford}
\date{\today}


\begin{document}
    \pagenumbering{gobble}
    \maketitle
    
    
    \begin{abstract}
        Magnetic confinement reactors---in particular tokamaks---offer one of the most promising options for achieving practical nuclear fusion, with the potential to provide virtually limitless, clean energy. The theoretical and numerical modeling of tokamak plasmas is simultaneously an essential component of effective reactor design, and a great research barrier. Tokamak operational conditions exhibit comparatively low Knudsen numbers. Kinetic effects, including kinetic waves and instabilities, Landau damping, bump-on-tail instabilities and more, are therefore highly influential in tokamak plasma dynamics. Purely fluid models are inherently incapable of capturing these effects, whereas the high dimensionality in purely kinetic models render them practically intractable for most relevant purposes.

        We consider a $\delta\!f$ decomposition model, with a macroscopic fluid background and microscopic kinetic correction, both fully coupled to each other. A similar manner of discretization is proposed to that used in the recent \texttt{STRUPHY} code \cite{Holderied_Possanner_Wang_2021, Holderied_2022, Li_et_al_2023} with a finite-element model for the background and a pseudo-particle/PiC model for the correction.

        The fluid background satisfies the full, non-linear, resistive, compressible, Hall MHD equations. \cite{Laakmann_Hu_Farrell_2022} introduces finite-element(-in-space) implicit timesteppers for the incompressible analogue to this system with structure-preserving (SP) properties in the ideal case, alongside parameter-robust preconditioners. We show that these timesteppers can derive from a finite-element-in-time (FET) (and finite-element-in-space) interpretation. The benefits of this reformulation are discussed, including the derivation of timesteppers that are higher order in time, and the quantifiable dissipative SP properties in the non-ideal, resistive case.
        
        We discuss possible options for extending this FET approach to timesteppers for the compressible case.

        The kinetic corrections satisfy linearized Boltzmann equations. Using a Lénard--Bernstein collision operator, these take Fokker--Planck-like forms \cite{Fokker_1914, Planck_1917} wherein pseudo-particles in the numerical model obey the neoclassical transport equations, with particle-independent Brownian drift terms. This offers a rigorous methodology for incorporating collisions into the particle transport model, without coupling the equations of motions for each particle.
        
        Works by Chen, Chacón et al. \cite{Chen_Chacón_Barnes_2011, Chacón_Chen_Barnes_2013, Chen_Chacón_2014, Chen_Chacón_2015} have developed structure-preserving particle pushers for neoclassical transport in the Vlasov equations, derived from Crank--Nicolson integrators. We show these too can can derive from a FET interpretation, similarly offering potential extensions to higher-order-in-time particle pushers. The FET formulation is used also to consider how the stochastic drift terms can be incorporated into the pushers. Stochastic gyrokinetic expansions are also discussed.

        Different options for the numerical implementation of these schemes are considered.

        Due to the efficacy of FET in the development of SP timesteppers for both the fluid and kinetic component, we hope this approach will prove effective in the future for developing SP timesteppers for the full hybrid model. We hope this will give us the opportunity to incorporate previously inaccessible kinetic effects into the highly effective, modern, finite-element MHD models.
    \end{abstract}
    
    
    \newpage
    \tableofcontents
    
    
    \newpage
    \pagenumbering{arabic}
    %\linenumbers\renewcommand\thelinenumber{\color{black!50}\arabic{linenumber}}
            \input{0 - introduction/main.tex}
        \part{Research}
            \input{1 - low-noise PiC models/main.tex}
            \input{2 - kinetic component/main.tex}
            \input{3 - fluid component/main.tex}
            \input{4 - numerical implementation/main.tex}
        \part{Project Overview}
            \input{5 - research plan/main.tex}
            \input{6 - summary/main.tex}
    
    
    %\section{}
    \newpage
    \pagenumbering{gobble}
        \printbibliography


    \newpage
    \pagenumbering{roman}
    \appendix
        \part{Appendices}
            \input{8 - Hilbert complexes/main.tex}
            \input{9 - weak conservation proofs/main.tex}
\end{document}

        \part{Research}
            \documentclass[12pt, a4paper]{report}

\input{template/main.tex}

\title{\BA{Title in Progress...}}
\author{Boris Andrews}
\affil{Mathematical Institute, University of Oxford}
\date{\today}


\begin{document}
    \pagenumbering{gobble}
    \maketitle
    
    
    \begin{abstract}
        Magnetic confinement reactors---in particular tokamaks---offer one of the most promising options for achieving practical nuclear fusion, with the potential to provide virtually limitless, clean energy. The theoretical and numerical modeling of tokamak plasmas is simultaneously an essential component of effective reactor design, and a great research barrier. Tokamak operational conditions exhibit comparatively low Knudsen numbers. Kinetic effects, including kinetic waves and instabilities, Landau damping, bump-on-tail instabilities and more, are therefore highly influential in tokamak plasma dynamics. Purely fluid models are inherently incapable of capturing these effects, whereas the high dimensionality in purely kinetic models render them practically intractable for most relevant purposes.

        We consider a $\delta\!f$ decomposition model, with a macroscopic fluid background and microscopic kinetic correction, both fully coupled to each other. A similar manner of discretization is proposed to that used in the recent \texttt{STRUPHY} code \cite{Holderied_Possanner_Wang_2021, Holderied_2022, Li_et_al_2023} with a finite-element model for the background and a pseudo-particle/PiC model for the correction.

        The fluid background satisfies the full, non-linear, resistive, compressible, Hall MHD equations. \cite{Laakmann_Hu_Farrell_2022} introduces finite-element(-in-space) implicit timesteppers for the incompressible analogue to this system with structure-preserving (SP) properties in the ideal case, alongside parameter-robust preconditioners. We show that these timesteppers can derive from a finite-element-in-time (FET) (and finite-element-in-space) interpretation. The benefits of this reformulation are discussed, including the derivation of timesteppers that are higher order in time, and the quantifiable dissipative SP properties in the non-ideal, resistive case.
        
        We discuss possible options for extending this FET approach to timesteppers for the compressible case.

        The kinetic corrections satisfy linearized Boltzmann equations. Using a Lénard--Bernstein collision operator, these take Fokker--Planck-like forms \cite{Fokker_1914, Planck_1917} wherein pseudo-particles in the numerical model obey the neoclassical transport equations, with particle-independent Brownian drift terms. This offers a rigorous methodology for incorporating collisions into the particle transport model, without coupling the equations of motions for each particle.
        
        Works by Chen, Chacón et al. \cite{Chen_Chacón_Barnes_2011, Chacón_Chen_Barnes_2013, Chen_Chacón_2014, Chen_Chacón_2015} have developed structure-preserving particle pushers for neoclassical transport in the Vlasov equations, derived from Crank--Nicolson integrators. We show these too can can derive from a FET interpretation, similarly offering potential extensions to higher-order-in-time particle pushers. The FET formulation is used also to consider how the stochastic drift terms can be incorporated into the pushers. Stochastic gyrokinetic expansions are also discussed.

        Different options for the numerical implementation of these schemes are considered.

        Due to the efficacy of FET in the development of SP timesteppers for both the fluid and kinetic component, we hope this approach will prove effective in the future for developing SP timesteppers for the full hybrid model. We hope this will give us the opportunity to incorporate previously inaccessible kinetic effects into the highly effective, modern, finite-element MHD models.
    \end{abstract}
    
    
    \newpage
    \tableofcontents
    
    
    \newpage
    \pagenumbering{arabic}
    %\linenumbers\renewcommand\thelinenumber{\color{black!50}\arabic{linenumber}}
            \input{0 - introduction/main.tex}
        \part{Research}
            \input{1 - low-noise PiC models/main.tex}
            \input{2 - kinetic component/main.tex}
            \input{3 - fluid component/main.tex}
            \input{4 - numerical implementation/main.tex}
        \part{Project Overview}
            \input{5 - research plan/main.tex}
            \input{6 - summary/main.tex}
    
    
    %\section{}
    \newpage
    \pagenumbering{gobble}
        \printbibliography


    \newpage
    \pagenumbering{roman}
    \appendix
        \part{Appendices}
            \input{8 - Hilbert complexes/main.tex}
            \input{9 - weak conservation proofs/main.tex}
\end{document}

            \documentclass[12pt, a4paper]{report}

\input{template/main.tex}

\title{\BA{Title in Progress...}}
\author{Boris Andrews}
\affil{Mathematical Institute, University of Oxford}
\date{\today}


\begin{document}
    \pagenumbering{gobble}
    \maketitle
    
    
    \begin{abstract}
        Magnetic confinement reactors---in particular tokamaks---offer one of the most promising options for achieving practical nuclear fusion, with the potential to provide virtually limitless, clean energy. The theoretical and numerical modeling of tokamak plasmas is simultaneously an essential component of effective reactor design, and a great research barrier. Tokamak operational conditions exhibit comparatively low Knudsen numbers. Kinetic effects, including kinetic waves and instabilities, Landau damping, bump-on-tail instabilities and more, are therefore highly influential in tokamak plasma dynamics. Purely fluid models are inherently incapable of capturing these effects, whereas the high dimensionality in purely kinetic models render them practically intractable for most relevant purposes.

        We consider a $\delta\!f$ decomposition model, with a macroscopic fluid background and microscopic kinetic correction, both fully coupled to each other. A similar manner of discretization is proposed to that used in the recent \texttt{STRUPHY} code \cite{Holderied_Possanner_Wang_2021, Holderied_2022, Li_et_al_2023} with a finite-element model for the background and a pseudo-particle/PiC model for the correction.

        The fluid background satisfies the full, non-linear, resistive, compressible, Hall MHD equations. \cite{Laakmann_Hu_Farrell_2022} introduces finite-element(-in-space) implicit timesteppers for the incompressible analogue to this system with structure-preserving (SP) properties in the ideal case, alongside parameter-robust preconditioners. We show that these timesteppers can derive from a finite-element-in-time (FET) (and finite-element-in-space) interpretation. The benefits of this reformulation are discussed, including the derivation of timesteppers that are higher order in time, and the quantifiable dissipative SP properties in the non-ideal, resistive case.
        
        We discuss possible options for extending this FET approach to timesteppers for the compressible case.

        The kinetic corrections satisfy linearized Boltzmann equations. Using a Lénard--Bernstein collision operator, these take Fokker--Planck-like forms \cite{Fokker_1914, Planck_1917} wherein pseudo-particles in the numerical model obey the neoclassical transport equations, with particle-independent Brownian drift terms. This offers a rigorous methodology for incorporating collisions into the particle transport model, without coupling the equations of motions for each particle.
        
        Works by Chen, Chacón et al. \cite{Chen_Chacón_Barnes_2011, Chacón_Chen_Barnes_2013, Chen_Chacón_2014, Chen_Chacón_2015} have developed structure-preserving particle pushers for neoclassical transport in the Vlasov equations, derived from Crank--Nicolson integrators. We show these too can can derive from a FET interpretation, similarly offering potential extensions to higher-order-in-time particle pushers. The FET formulation is used also to consider how the stochastic drift terms can be incorporated into the pushers. Stochastic gyrokinetic expansions are also discussed.

        Different options for the numerical implementation of these schemes are considered.

        Due to the efficacy of FET in the development of SP timesteppers for both the fluid and kinetic component, we hope this approach will prove effective in the future for developing SP timesteppers for the full hybrid model. We hope this will give us the opportunity to incorporate previously inaccessible kinetic effects into the highly effective, modern, finite-element MHD models.
    \end{abstract}
    
    
    \newpage
    \tableofcontents
    
    
    \newpage
    \pagenumbering{arabic}
    %\linenumbers\renewcommand\thelinenumber{\color{black!50}\arabic{linenumber}}
            \input{0 - introduction/main.tex}
        \part{Research}
            \input{1 - low-noise PiC models/main.tex}
            \input{2 - kinetic component/main.tex}
            \input{3 - fluid component/main.tex}
            \input{4 - numerical implementation/main.tex}
        \part{Project Overview}
            \input{5 - research plan/main.tex}
            \input{6 - summary/main.tex}
    
    
    %\section{}
    \newpage
    \pagenumbering{gobble}
        \printbibliography


    \newpage
    \pagenumbering{roman}
    \appendix
        \part{Appendices}
            \input{8 - Hilbert complexes/main.tex}
            \input{9 - weak conservation proofs/main.tex}
\end{document}

            \documentclass[12pt, a4paper]{report}

\input{template/main.tex}

\title{\BA{Title in Progress...}}
\author{Boris Andrews}
\affil{Mathematical Institute, University of Oxford}
\date{\today}


\begin{document}
    \pagenumbering{gobble}
    \maketitle
    
    
    \begin{abstract}
        Magnetic confinement reactors---in particular tokamaks---offer one of the most promising options for achieving practical nuclear fusion, with the potential to provide virtually limitless, clean energy. The theoretical and numerical modeling of tokamak plasmas is simultaneously an essential component of effective reactor design, and a great research barrier. Tokamak operational conditions exhibit comparatively low Knudsen numbers. Kinetic effects, including kinetic waves and instabilities, Landau damping, bump-on-tail instabilities and more, are therefore highly influential in tokamak plasma dynamics. Purely fluid models are inherently incapable of capturing these effects, whereas the high dimensionality in purely kinetic models render them practically intractable for most relevant purposes.

        We consider a $\delta\!f$ decomposition model, with a macroscopic fluid background and microscopic kinetic correction, both fully coupled to each other. A similar manner of discretization is proposed to that used in the recent \texttt{STRUPHY} code \cite{Holderied_Possanner_Wang_2021, Holderied_2022, Li_et_al_2023} with a finite-element model for the background and a pseudo-particle/PiC model for the correction.

        The fluid background satisfies the full, non-linear, resistive, compressible, Hall MHD equations. \cite{Laakmann_Hu_Farrell_2022} introduces finite-element(-in-space) implicit timesteppers for the incompressible analogue to this system with structure-preserving (SP) properties in the ideal case, alongside parameter-robust preconditioners. We show that these timesteppers can derive from a finite-element-in-time (FET) (and finite-element-in-space) interpretation. The benefits of this reformulation are discussed, including the derivation of timesteppers that are higher order in time, and the quantifiable dissipative SP properties in the non-ideal, resistive case.
        
        We discuss possible options for extending this FET approach to timesteppers for the compressible case.

        The kinetic corrections satisfy linearized Boltzmann equations. Using a Lénard--Bernstein collision operator, these take Fokker--Planck-like forms \cite{Fokker_1914, Planck_1917} wherein pseudo-particles in the numerical model obey the neoclassical transport equations, with particle-independent Brownian drift terms. This offers a rigorous methodology for incorporating collisions into the particle transport model, without coupling the equations of motions for each particle.
        
        Works by Chen, Chacón et al. \cite{Chen_Chacón_Barnes_2011, Chacón_Chen_Barnes_2013, Chen_Chacón_2014, Chen_Chacón_2015} have developed structure-preserving particle pushers for neoclassical transport in the Vlasov equations, derived from Crank--Nicolson integrators. We show these too can can derive from a FET interpretation, similarly offering potential extensions to higher-order-in-time particle pushers. The FET formulation is used also to consider how the stochastic drift terms can be incorporated into the pushers. Stochastic gyrokinetic expansions are also discussed.

        Different options for the numerical implementation of these schemes are considered.

        Due to the efficacy of FET in the development of SP timesteppers for both the fluid and kinetic component, we hope this approach will prove effective in the future for developing SP timesteppers for the full hybrid model. We hope this will give us the opportunity to incorporate previously inaccessible kinetic effects into the highly effective, modern, finite-element MHD models.
    \end{abstract}
    
    
    \newpage
    \tableofcontents
    
    
    \newpage
    \pagenumbering{arabic}
    %\linenumbers\renewcommand\thelinenumber{\color{black!50}\arabic{linenumber}}
            \input{0 - introduction/main.tex}
        \part{Research}
            \input{1 - low-noise PiC models/main.tex}
            \input{2 - kinetic component/main.tex}
            \input{3 - fluid component/main.tex}
            \input{4 - numerical implementation/main.tex}
        \part{Project Overview}
            \input{5 - research plan/main.tex}
            \input{6 - summary/main.tex}
    
    
    %\section{}
    \newpage
    \pagenumbering{gobble}
        \printbibliography


    \newpage
    \pagenumbering{roman}
    \appendix
        \part{Appendices}
            \input{8 - Hilbert complexes/main.tex}
            \input{9 - weak conservation proofs/main.tex}
\end{document}

            \documentclass[12pt, a4paper]{report}

\input{template/main.tex}

\title{\BA{Title in Progress...}}
\author{Boris Andrews}
\affil{Mathematical Institute, University of Oxford}
\date{\today}


\begin{document}
    \pagenumbering{gobble}
    \maketitle
    
    
    \begin{abstract}
        Magnetic confinement reactors---in particular tokamaks---offer one of the most promising options for achieving practical nuclear fusion, with the potential to provide virtually limitless, clean energy. The theoretical and numerical modeling of tokamak plasmas is simultaneously an essential component of effective reactor design, and a great research barrier. Tokamak operational conditions exhibit comparatively low Knudsen numbers. Kinetic effects, including kinetic waves and instabilities, Landau damping, bump-on-tail instabilities and more, are therefore highly influential in tokamak plasma dynamics. Purely fluid models are inherently incapable of capturing these effects, whereas the high dimensionality in purely kinetic models render them practically intractable for most relevant purposes.

        We consider a $\delta\!f$ decomposition model, with a macroscopic fluid background and microscopic kinetic correction, both fully coupled to each other. A similar manner of discretization is proposed to that used in the recent \texttt{STRUPHY} code \cite{Holderied_Possanner_Wang_2021, Holderied_2022, Li_et_al_2023} with a finite-element model for the background and a pseudo-particle/PiC model for the correction.

        The fluid background satisfies the full, non-linear, resistive, compressible, Hall MHD equations. \cite{Laakmann_Hu_Farrell_2022} introduces finite-element(-in-space) implicit timesteppers for the incompressible analogue to this system with structure-preserving (SP) properties in the ideal case, alongside parameter-robust preconditioners. We show that these timesteppers can derive from a finite-element-in-time (FET) (and finite-element-in-space) interpretation. The benefits of this reformulation are discussed, including the derivation of timesteppers that are higher order in time, and the quantifiable dissipative SP properties in the non-ideal, resistive case.
        
        We discuss possible options for extending this FET approach to timesteppers for the compressible case.

        The kinetic corrections satisfy linearized Boltzmann equations. Using a Lénard--Bernstein collision operator, these take Fokker--Planck-like forms \cite{Fokker_1914, Planck_1917} wherein pseudo-particles in the numerical model obey the neoclassical transport equations, with particle-independent Brownian drift terms. This offers a rigorous methodology for incorporating collisions into the particle transport model, without coupling the equations of motions for each particle.
        
        Works by Chen, Chacón et al. \cite{Chen_Chacón_Barnes_2011, Chacón_Chen_Barnes_2013, Chen_Chacón_2014, Chen_Chacón_2015} have developed structure-preserving particle pushers for neoclassical transport in the Vlasov equations, derived from Crank--Nicolson integrators. We show these too can can derive from a FET interpretation, similarly offering potential extensions to higher-order-in-time particle pushers. The FET formulation is used also to consider how the stochastic drift terms can be incorporated into the pushers. Stochastic gyrokinetic expansions are also discussed.

        Different options for the numerical implementation of these schemes are considered.

        Due to the efficacy of FET in the development of SP timesteppers for both the fluid and kinetic component, we hope this approach will prove effective in the future for developing SP timesteppers for the full hybrid model. We hope this will give us the opportunity to incorporate previously inaccessible kinetic effects into the highly effective, modern, finite-element MHD models.
    \end{abstract}
    
    
    \newpage
    \tableofcontents
    
    
    \newpage
    \pagenumbering{arabic}
    %\linenumbers\renewcommand\thelinenumber{\color{black!50}\arabic{linenumber}}
            \input{0 - introduction/main.tex}
        \part{Research}
            \input{1 - low-noise PiC models/main.tex}
            \input{2 - kinetic component/main.tex}
            \input{3 - fluid component/main.tex}
            \input{4 - numerical implementation/main.tex}
        \part{Project Overview}
            \input{5 - research plan/main.tex}
            \input{6 - summary/main.tex}
    
    
    %\section{}
    \newpage
    \pagenumbering{gobble}
        \printbibliography


    \newpage
    \pagenumbering{roman}
    \appendix
        \part{Appendices}
            \input{8 - Hilbert complexes/main.tex}
            \input{9 - weak conservation proofs/main.tex}
\end{document}

        \part{Project Overview}
            \documentclass[12pt, a4paper]{report}

\input{template/main.tex}

\title{\BA{Title in Progress...}}
\author{Boris Andrews}
\affil{Mathematical Institute, University of Oxford}
\date{\today}


\begin{document}
    \pagenumbering{gobble}
    \maketitle
    
    
    \begin{abstract}
        Magnetic confinement reactors---in particular tokamaks---offer one of the most promising options for achieving practical nuclear fusion, with the potential to provide virtually limitless, clean energy. The theoretical and numerical modeling of tokamak plasmas is simultaneously an essential component of effective reactor design, and a great research barrier. Tokamak operational conditions exhibit comparatively low Knudsen numbers. Kinetic effects, including kinetic waves and instabilities, Landau damping, bump-on-tail instabilities and more, are therefore highly influential in tokamak plasma dynamics. Purely fluid models are inherently incapable of capturing these effects, whereas the high dimensionality in purely kinetic models render them practically intractable for most relevant purposes.

        We consider a $\delta\!f$ decomposition model, with a macroscopic fluid background and microscopic kinetic correction, both fully coupled to each other. A similar manner of discretization is proposed to that used in the recent \texttt{STRUPHY} code \cite{Holderied_Possanner_Wang_2021, Holderied_2022, Li_et_al_2023} with a finite-element model for the background and a pseudo-particle/PiC model for the correction.

        The fluid background satisfies the full, non-linear, resistive, compressible, Hall MHD equations. \cite{Laakmann_Hu_Farrell_2022} introduces finite-element(-in-space) implicit timesteppers for the incompressible analogue to this system with structure-preserving (SP) properties in the ideal case, alongside parameter-robust preconditioners. We show that these timesteppers can derive from a finite-element-in-time (FET) (and finite-element-in-space) interpretation. The benefits of this reformulation are discussed, including the derivation of timesteppers that are higher order in time, and the quantifiable dissipative SP properties in the non-ideal, resistive case.
        
        We discuss possible options for extending this FET approach to timesteppers for the compressible case.

        The kinetic corrections satisfy linearized Boltzmann equations. Using a Lénard--Bernstein collision operator, these take Fokker--Planck-like forms \cite{Fokker_1914, Planck_1917} wherein pseudo-particles in the numerical model obey the neoclassical transport equations, with particle-independent Brownian drift terms. This offers a rigorous methodology for incorporating collisions into the particle transport model, without coupling the equations of motions for each particle.
        
        Works by Chen, Chacón et al. \cite{Chen_Chacón_Barnes_2011, Chacón_Chen_Barnes_2013, Chen_Chacón_2014, Chen_Chacón_2015} have developed structure-preserving particle pushers for neoclassical transport in the Vlasov equations, derived from Crank--Nicolson integrators. We show these too can can derive from a FET interpretation, similarly offering potential extensions to higher-order-in-time particle pushers. The FET formulation is used also to consider how the stochastic drift terms can be incorporated into the pushers. Stochastic gyrokinetic expansions are also discussed.

        Different options for the numerical implementation of these schemes are considered.

        Due to the efficacy of FET in the development of SP timesteppers for both the fluid and kinetic component, we hope this approach will prove effective in the future for developing SP timesteppers for the full hybrid model. We hope this will give us the opportunity to incorporate previously inaccessible kinetic effects into the highly effective, modern, finite-element MHD models.
    \end{abstract}
    
    
    \newpage
    \tableofcontents
    
    
    \newpage
    \pagenumbering{arabic}
    %\linenumbers\renewcommand\thelinenumber{\color{black!50}\arabic{linenumber}}
            \input{0 - introduction/main.tex}
        \part{Research}
            \input{1 - low-noise PiC models/main.tex}
            \input{2 - kinetic component/main.tex}
            \input{3 - fluid component/main.tex}
            \input{4 - numerical implementation/main.tex}
        \part{Project Overview}
            \input{5 - research plan/main.tex}
            \input{6 - summary/main.tex}
    
    
    %\section{}
    \newpage
    \pagenumbering{gobble}
        \printbibliography


    \newpage
    \pagenumbering{roman}
    \appendix
        \part{Appendices}
            \input{8 - Hilbert complexes/main.tex}
            \input{9 - weak conservation proofs/main.tex}
\end{document}

            \documentclass[12pt, a4paper]{report}

\input{template/main.tex}

\title{\BA{Title in Progress...}}
\author{Boris Andrews}
\affil{Mathematical Institute, University of Oxford}
\date{\today}


\begin{document}
    \pagenumbering{gobble}
    \maketitle
    
    
    \begin{abstract}
        Magnetic confinement reactors---in particular tokamaks---offer one of the most promising options for achieving practical nuclear fusion, with the potential to provide virtually limitless, clean energy. The theoretical and numerical modeling of tokamak plasmas is simultaneously an essential component of effective reactor design, and a great research barrier. Tokamak operational conditions exhibit comparatively low Knudsen numbers. Kinetic effects, including kinetic waves and instabilities, Landau damping, bump-on-tail instabilities and more, are therefore highly influential in tokamak plasma dynamics. Purely fluid models are inherently incapable of capturing these effects, whereas the high dimensionality in purely kinetic models render them practically intractable for most relevant purposes.

        We consider a $\delta\!f$ decomposition model, with a macroscopic fluid background and microscopic kinetic correction, both fully coupled to each other. A similar manner of discretization is proposed to that used in the recent \texttt{STRUPHY} code \cite{Holderied_Possanner_Wang_2021, Holderied_2022, Li_et_al_2023} with a finite-element model for the background and a pseudo-particle/PiC model for the correction.

        The fluid background satisfies the full, non-linear, resistive, compressible, Hall MHD equations. \cite{Laakmann_Hu_Farrell_2022} introduces finite-element(-in-space) implicit timesteppers for the incompressible analogue to this system with structure-preserving (SP) properties in the ideal case, alongside parameter-robust preconditioners. We show that these timesteppers can derive from a finite-element-in-time (FET) (and finite-element-in-space) interpretation. The benefits of this reformulation are discussed, including the derivation of timesteppers that are higher order in time, and the quantifiable dissipative SP properties in the non-ideal, resistive case.
        
        We discuss possible options for extending this FET approach to timesteppers for the compressible case.

        The kinetic corrections satisfy linearized Boltzmann equations. Using a Lénard--Bernstein collision operator, these take Fokker--Planck-like forms \cite{Fokker_1914, Planck_1917} wherein pseudo-particles in the numerical model obey the neoclassical transport equations, with particle-independent Brownian drift terms. This offers a rigorous methodology for incorporating collisions into the particle transport model, without coupling the equations of motions for each particle.
        
        Works by Chen, Chacón et al. \cite{Chen_Chacón_Barnes_2011, Chacón_Chen_Barnes_2013, Chen_Chacón_2014, Chen_Chacón_2015} have developed structure-preserving particle pushers for neoclassical transport in the Vlasov equations, derived from Crank--Nicolson integrators. We show these too can can derive from a FET interpretation, similarly offering potential extensions to higher-order-in-time particle pushers. The FET formulation is used also to consider how the stochastic drift terms can be incorporated into the pushers. Stochastic gyrokinetic expansions are also discussed.

        Different options for the numerical implementation of these schemes are considered.

        Due to the efficacy of FET in the development of SP timesteppers for both the fluid and kinetic component, we hope this approach will prove effective in the future for developing SP timesteppers for the full hybrid model. We hope this will give us the opportunity to incorporate previously inaccessible kinetic effects into the highly effective, modern, finite-element MHD models.
    \end{abstract}
    
    
    \newpage
    \tableofcontents
    
    
    \newpage
    \pagenumbering{arabic}
    %\linenumbers\renewcommand\thelinenumber{\color{black!50}\arabic{linenumber}}
            \input{0 - introduction/main.tex}
        \part{Research}
            \input{1 - low-noise PiC models/main.tex}
            \input{2 - kinetic component/main.tex}
            \input{3 - fluid component/main.tex}
            \input{4 - numerical implementation/main.tex}
        \part{Project Overview}
            \input{5 - research plan/main.tex}
            \input{6 - summary/main.tex}
    
    
    %\section{}
    \newpage
    \pagenumbering{gobble}
        \printbibliography


    \newpage
    \pagenumbering{roman}
    \appendix
        \part{Appendices}
            \input{8 - Hilbert complexes/main.tex}
            \input{9 - weak conservation proofs/main.tex}
\end{document}

    
    
    %\section{}
    \newpage
    \pagenumbering{gobble}
        \printbibliography


    \newpage
    \pagenumbering{roman}
    \appendix
        \part{Appendices}
            \documentclass[12pt, a4paper]{report}

\input{template/main.tex}

\title{\BA{Title in Progress...}}
\author{Boris Andrews}
\affil{Mathematical Institute, University of Oxford}
\date{\today}


\begin{document}
    \pagenumbering{gobble}
    \maketitle
    
    
    \begin{abstract}
        Magnetic confinement reactors---in particular tokamaks---offer one of the most promising options for achieving practical nuclear fusion, with the potential to provide virtually limitless, clean energy. The theoretical and numerical modeling of tokamak plasmas is simultaneously an essential component of effective reactor design, and a great research barrier. Tokamak operational conditions exhibit comparatively low Knudsen numbers. Kinetic effects, including kinetic waves and instabilities, Landau damping, bump-on-tail instabilities and more, are therefore highly influential in tokamak plasma dynamics. Purely fluid models are inherently incapable of capturing these effects, whereas the high dimensionality in purely kinetic models render them practically intractable for most relevant purposes.

        We consider a $\delta\!f$ decomposition model, with a macroscopic fluid background and microscopic kinetic correction, both fully coupled to each other. A similar manner of discretization is proposed to that used in the recent \texttt{STRUPHY} code \cite{Holderied_Possanner_Wang_2021, Holderied_2022, Li_et_al_2023} with a finite-element model for the background and a pseudo-particle/PiC model for the correction.

        The fluid background satisfies the full, non-linear, resistive, compressible, Hall MHD equations. \cite{Laakmann_Hu_Farrell_2022} introduces finite-element(-in-space) implicit timesteppers for the incompressible analogue to this system with structure-preserving (SP) properties in the ideal case, alongside parameter-robust preconditioners. We show that these timesteppers can derive from a finite-element-in-time (FET) (and finite-element-in-space) interpretation. The benefits of this reformulation are discussed, including the derivation of timesteppers that are higher order in time, and the quantifiable dissipative SP properties in the non-ideal, resistive case.
        
        We discuss possible options for extending this FET approach to timesteppers for the compressible case.

        The kinetic corrections satisfy linearized Boltzmann equations. Using a Lénard--Bernstein collision operator, these take Fokker--Planck-like forms \cite{Fokker_1914, Planck_1917} wherein pseudo-particles in the numerical model obey the neoclassical transport equations, with particle-independent Brownian drift terms. This offers a rigorous methodology for incorporating collisions into the particle transport model, without coupling the equations of motions for each particle.
        
        Works by Chen, Chacón et al. \cite{Chen_Chacón_Barnes_2011, Chacón_Chen_Barnes_2013, Chen_Chacón_2014, Chen_Chacón_2015} have developed structure-preserving particle pushers for neoclassical transport in the Vlasov equations, derived from Crank--Nicolson integrators. We show these too can can derive from a FET interpretation, similarly offering potential extensions to higher-order-in-time particle pushers. The FET formulation is used also to consider how the stochastic drift terms can be incorporated into the pushers. Stochastic gyrokinetic expansions are also discussed.

        Different options for the numerical implementation of these schemes are considered.

        Due to the efficacy of FET in the development of SP timesteppers for both the fluid and kinetic component, we hope this approach will prove effective in the future for developing SP timesteppers for the full hybrid model. We hope this will give us the opportunity to incorporate previously inaccessible kinetic effects into the highly effective, modern, finite-element MHD models.
    \end{abstract}
    
    
    \newpage
    \tableofcontents
    
    
    \newpage
    \pagenumbering{arabic}
    %\linenumbers\renewcommand\thelinenumber{\color{black!50}\arabic{linenumber}}
            \input{0 - introduction/main.tex}
        \part{Research}
            \input{1 - low-noise PiC models/main.tex}
            \input{2 - kinetic component/main.tex}
            \input{3 - fluid component/main.tex}
            \input{4 - numerical implementation/main.tex}
        \part{Project Overview}
            \input{5 - research plan/main.tex}
            \input{6 - summary/main.tex}
    
    
    %\section{}
    \newpage
    \pagenumbering{gobble}
        \printbibliography


    \newpage
    \pagenumbering{roman}
    \appendix
        \part{Appendices}
            \input{8 - Hilbert complexes/main.tex}
            \input{9 - weak conservation proofs/main.tex}
\end{document}

            \documentclass[12pt, a4paper]{report}

\input{template/main.tex}

\title{\BA{Title in Progress...}}
\author{Boris Andrews}
\affil{Mathematical Institute, University of Oxford}
\date{\today}


\begin{document}
    \pagenumbering{gobble}
    \maketitle
    
    
    \begin{abstract}
        Magnetic confinement reactors---in particular tokamaks---offer one of the most promising options for achieving practical nuclear fusion, with the potential to provide virtually limitless, clean energy. The theoretical and numerical modeling of tokamak plasmas is simultaneously an essential component of effective reactor design, and a great research barrier. Tokamak operational conditions exhibit comparatively low Knudsen numbers. Kinetic effects, including kinetic waves and instabilities, Landau damping, bump-on-tail instabilities and more, are therefore highly influential in tokamak plasma dynamics. Purely fluid models are inherently incapable of capturing these effects, whereas the high dimensionality in purely kinetic models render them practically intractable for most relevant purposes.

        We consider a $\delta\!f$ decomposition model, with a macroscopic fluid background and microscopic kinetic correction, both fully coupled to each other. A similar manner of discretization is proposed to that used in the recent \texttt{STRUPHY} code \cite{Holderied_Possanner_Wang_2021, Holderied_2022, Li_et_al_2023} with a finite-element model for the background and a pseudo-particle/PiC model for the correction.

        The fluid background satisfies the full, non-linear, resistive, compressible, Hall MHD equations. \cite{Laakmann_Hu_Farrell_2022} introduces finite-element(-in-space) implicit timesteppers for the incompressible analogue to this system with structure-preserving (SP) properties in the ideal case, alongside parameter-robust preconditioners. We show that these timesteppers can derive from a finite-element-in-time (FET) (and finite-element-in-space) interpretation. The benefits of this reformulation are discussed, including the derivation of timesteppers that are higher order in time, and the quantifiable dissipative SP properties in the non-ideal, resistive case.
        
        We discuss possible options for extending this FET approach to timesteppers for the compressible case.

        The kinetic corrections satisfy linearized Boltzmann equations. Using a Lénard--Bernstein collision operator, these take Fokker--Planck-like forms \cite{Fokker_1914, Planck_1917} wherein pseudo-particles in the numerical model obey the neoclassical transport equations, with particle-independent Brownian drift terms. This offers a rigorous methodology for incorporating collisions into the particle transport model, without coupling the equations of motions for each particle.
        
        Works by Chen, Chacón et al. \cite{Chen_Chacón_Barnes_2011, Chacón_Chen_Barnes_2013, Chen_Chacón_2014, Chen_Chacón_2015} have developed structure-preserving particle pushers for neoclassical transport in the Vlasov equations, derived from Crank--Nicolson integrators. We show these too can can derive from a FET interpretation, similarly offering potential extensions to higher-order-in-time particle pushers. The FET formulation is used also to consider how the stochastic drift terms can be incorporated into the pushers. Stochastic gyrokinetic expansions are also discussed.

        Different options for the numerical implementation of these schemes are considered.

        Due to the efficacy of FET in the development of SP timesteppers for both the fluid and kinetic component, we hope this approach will prove effective in the future for developing SP timesteppers for the full hybrid model. We hope this will give us the opportunity to incorporate previously inaccessible kinetic effects into the highly effective, modern, finite-element MHD models.
    \end{abstract}
    
    
    \newpage
    \tableofcontents
    
    
    \newpage
    \pagenumbering{arabic}
    %\linenumbers\renewcommand\thelinenumber{\color{black!50}\arabic{linenumber}}
            \input{0 - introduction/main.tex}
        \part{Research}
            \input{1 - low-noise PiC models/main.tex}
            \input{2 - kinetic component/main.tex}
            \input{3 - fluid component/main.tex}
            \input{4 - numerical implementation/main.tex}
        \part{Project Overview}
            \input{5 - research plan/main.tex}
            \input{6 - summary/main.tex}
    
    
    %\section{}
    \newpage
    \pagenumbering{gobble}
        \printbibliography


    \newpage
    \pagenumbering{roman}
    \appendix
        \part{Appendices}
            \input{8 - Hilbert complexes/main.tex}
            \input{9 - weak conservation proofs/main.tex}
\end{document}

\end{document}

            \documentclass[12pt, a4paper]{report}

\documentclass[12pt, a4paper]{report}

\input{template/main.tex}

\title{\BA{Title in Progress...}}
\author{Boris Andrews}
\affil{Mathematical Institute, University of Oxford}
\date{\today}


\begin{document}
    \pagenumbering{gobble}
    \maketitle
    
    
    \begin{abstract}
        Magnetic confinement reactors---in particular tokamaks---offer one of the most promising options for achieving practical nuclear fusion, with the potential to provide virtually limitless, clean energy. The theoretical and numerical modeling of tokamak plasmas is simultaneously an essential component of effective reactor design, and a great research barrier. Tokamak operational conditions exhibit comparatively low Knudsen numbers. Kinetic effects, including kinetic waves and instabilities, Landau damping, bump-on-tail instabilities and more, are therefore highly influential in tokamak plasma dynamics. Purely fluid models are inherently incapable of capturing these effects, whereas the high dimensionality in purely kinetic models render them practically intractable for most relevant purposes.

        We consider a $\delta\!f$ decomposition model, with a macroscopic fluid background and microscopic kinetic correction, both fully coupled to each other. A similar manner of discretization is proposed to that used in the recent \texttt{STRUPHY} code \cite{Holderied_Possanner_Wang_2021, Holderied_2022, Li_et_al_2023} with a finite-element model for the background and a pseudo-particle/PiC model for the correction.

        The fluid background satisfies the full, non-linear, resistive, compressible, Hall MHD equations. \cite{Laakmann_Hu_Farrell_2022} introduces finite-element(-in-space) implicit timesteppers for the incompressible analogue to this system with structure-preserving (SP) properties in the ideal case, alongside parameter-robust preconditioners. We show that these timesteppers can derive from a finite-element-in-time (FET) (and finite-element-in-space) interpretation. The benefits of this reformulation are discussed, including the derivation of timesteppers that are higher order in time, and the quantifiable dissipative SP properties in the non-ideal, resistive case.
        
        We discuss possible options for extending this FET approach to timesteppers for the compressible case.

        The kinetic corrections satisfy linearized Boltzmann equations. Using a Lénard--Bernstein collision operator, these take Fokker--Planck-like forms \cite{Fokker_1914, Planck_1917} wherein pseudo-particles in the numerical model obey the neoclassical transport equations, with particle-independent Brownian drift terms. This offers a rigorous methodology for incorporating collisions into the particle transport model, without coupling the equations of motions for each particle.
        
        Works by Chen, Chacón et al. \cite{Chen_Chacón_Barnes_2011, Chacón_Chen_Barnes_2013, Chen_Chacón_2014, Chen_Chacón_2015} have developed structure-preserving particle pushers for neoclassical transport in the Vlasov equations, derived from Crank--Nicolson integrators. We show these too can can derive from a FET interpretation, similarly offering potential extensions to higher-order-in-time particle pushers. The FET formulation is used also to consider how the stochastic drift terms can be incorporated into the pushers. Stochastic gyrokinetic expansions are also discussed.

        Different options for the numerical implementation of these schemes are considered.

        Due to the efficacy of FET in the development of SP timesteppers for both the fluid and kinetic component, we hope this approach will prove effective in the future for developing SP timesteppers for the full hybrid model. We hope this will give us the opportunity to incorporate previously inaccessible kinetic effects into the highly effective, modern, finite-element MHD models.
    \end{abstract}
    
    
    \newpage
    \tableofcontents
    
    
    \newpage
    \pagenumbering{arabic}
    %\linenumbers\renewcommand\thelinenumber{\color{black!50}\arabic{linenumber}}
            \input{0 - introduction/main.tex}
        \part{Research}
            \input{1 - low-noise PiC models/main.tex}
            \input{2 - kinetic component/main.tex}
            \input{3 - fluid component/main.tex}
            \input{4 - numerical implementation/main.tex}
        \part{Project Overview}
            \input{5 - research plan/main.tex}
            \input{6 - summary/main.tex}
    
    
    %\section{}
    \newpage
    \pagenumbering{gobble}
        \printbibliography


    \newpage
    \pagenumbering{roman}
    \appendix
        \part{Appendices}
            \input{8 - Hilbert complexes/main.tex}
            \input{9 - weak conservation proofs/main.tex}
\end{document}


\title{\BA{Title in Progress...}}
\author{Boris Andrews}
\affil{Mathematical Institute, University of Oxford}
\date{\today}


\begin{document}
    \pagenumbering{gobble}
    \maketitle
    
    
    \begin{abstract}
        Magnetic confinement reactors---in particular tokamaks---offer one of the most promising options for achieving practical nuclear fusion, with the potential to provide virtually limitless, clean energy. The theoretical and numerical modeling of tokamak plasmas is simultaneously an essential component of effective reactor design, and a great research barrier. Tokamak operational conditions exhibit comparatively low Knudsen numbers. Kinetic effects, including kinetic waves and instabilities, Landau damping, bump-on-tail instabilities and more, are therefore highly influential in tokamak plasma dynamics. Purely fluid models are inherently incapable of capturing these effects, whereas the high dimensionality in purely kinetic models render them practically intractable for most relevant purposes.

        We consider a $\delta\!f$ decomposition model, with a macroscopic fluid background and microscopic kinetic correction, both fully coupled to each other. A similar manner of discretization is proposed to that used in the recent \texttt{STRUPHY} code \cite{Holderied_Possanner_Wang_2021, Holderied_2022, Li_et_al_2023} with a finite-element model for the background and a pseudo-particle/PiC model for the correction.

        The fluid background satisfies the full, non-linear, resistive, compressible, Hall MHD equations. \cite{Laakmann_Hu_Farrell_2022} introduces finite-element(-in-space) implicit timesteppers for the incompressible analogue to this system with structure-preserving (SP) properties in the ideal case, alongside parameter-robust preconditioners. We show that these timesteppers can derive from a finite-element-in-time (FET) (and finite-element-in-space) interpretation. The benefits of this reformulation are discussed, including the derivation of timesteppers that are higher order in time, and the quantifiable dissipative SP properties in the non-ideal, resistive case.
        
        We discuss possible options for extending this FET approach to timesteppers for the compressible case.

        The kinetic corrections satisfy linearized Boltzmann equations. Using a Lénard--Bernstein collision operator, these take Fokker--Planck-like forms \cite{Fokker_1914, Planck_1917} wherein pseudo-particles in the numerical model obey the neoclassical transport equations, with particle-independent Brownian drift terms. This offers a rigorous methodology for incorporating collisions into the particle transport model, without coupling the equations of motions for each particle.
        
        Works by Chen, Chacón et al. \cite{Chen_Chacón_Barnes_2011, Chacón_Chen_Barnes_2013, Chen_Chacón_2014, Chen_Chacón_2015} have developed structure-preserving particle pushers for neoclassical transport in the Vlasov equations, derived from Crank--Nicolson integrators. We show these too can can derive from a FET interpretation, similarly offering potential extensions to higher-order-in-time particle pushers. The FET formulation is used also to consider how the stochastic drift terms can be incorporated into the pushers. Stochastic gyrokinetic expansions are also discussed.

        Different options for the numerical implementation of these schemes are considered.

        Due to the efficacy of FET in the development of SP timesteppers for both the fluid and kinetic component, we hope this approach will prove effective in the future for developing SP timesteppers for the full hybrid model. We hope this will give us the opportunity to incorporate previously inaccessible kinetic effects into the highly effective, modern, finite-element MHD models.
    \end{abstract}
    
    
    \newpage
    \tableofcontents
    
    
    \newpage
    \pagenumbering{arabic}
    %\linenumbers\renewcommand\thelinenumber{\color{black!50}\arabic{linenumber}}
            \documentclass[12pt, a4paper]{report}

\input{template/main.tex}

\title{\BA{Title in Progress...}}
\author{Boris Andrews}
\affil{Mathematical Institute, University of Oxford}
\date{\today}


\begin{document}
    \pagenumbering{gobble}
    \maketitle
    
    
    \begin{abstract}
        Magnetic confinement reactors---in particular tokamaks---offer one of the most promising options for achieving practical nuclear fusion, with the potential to provide virtually limitless, clean energy. The theoretical and numerical modeling of tokamak plasmas is simultaneously an essential component of effective reactor design, and a great research barrier. Tokamak operational conditions exhibit comparatively low Knudsen numbers. Kinetic effects, including kinetic waves and instabilities, Landau damping, bump-on-tail instabilities and more, are therefore highly influential in tokamak plasma dynamics. Purely fluid models are inherently incapable of capturing these effects, whereas the high dimensionality in purely kinetic models render them practically intractable for most relevant purposes.

        We consider a $\delta\!f$ decomposition model, with a macroscopic fluid background and microscopic kinetic correction, both fully coupled to each other. A similar manner of discretization is proposed to that used in the recent \texttt{STRUPHY} code \cite{Holderied_Possanner_Wang_2021, Holderied_2022, Li_et_al_2023} with a finite-element model for the background and a pseudo-particle/PiC model for the correction.

        The fluid background satisfies the full, non-linear, resistive, compressible, Hall MHD equations. \cite{Laakmann_Hu_Farrell_2022} introduces finite-element(-in-space) implicit timesteppers for the incompressible analogue to this system with structure-preserving (SP) properties in the ideal case, alongside parameter-robust preconditioners. We show that these timesteppers can derive from a finite-element-in-time (FET) (and finite-element-in-space) interpretation. The benefits of this reformulation are discussed, including the derivation of timesteppers that are higher order in time, and the quantifiable dissipative SP properties in the non-ideal, resistive case.
        
        We discuss possible options for extending this FET approach to timesteppers for the compressible case.

        The kinetic corrections satisfy linearized Boltzmann equations. Using a Lénard--Bernstein collision operator, these take Fokker--Planck-like forms \cite{Fokker_1914, Planck_1917} wherein pseudo-particles in the numerical model obey the neoclassical transport equations, with particle-independent Brownian drift terms. This offers a rigorous methodology for incorporating collisions into the particle transport model, without coupling the equations of motions for each particle.
        
        Works by Chen, Chacón et al. \cite{Chen_Chacón_Barnes_2011, Chacón_Chen_Barnes_2013, Chen_Chacón_2014, Chen_Chacón_2015} have developed structure-preserving particle pushers for neoclassical transport in the Vlasov equations, derived from Crank--Nicolson integrators. We show these too can can derive from a FET interpretation, similarly offering potential extensions to higher-order-in-time particle pushers. The FET formulation is used also to consider how the stochastic drift terms can be incorporated into the pushers. Stochastic gyrokinetic expansions are also discussed.

        Different options for the numerical implementation of these schemes are considered.

        Due to the efficacy of FET in the development of SP timesteppers for both the fluid and kinetic component, we hope this approach will prove effective in the future for developing SP timesteppers for the full hybrid model. We hope this will give us the opportunity to incorporate previously inaccessible kinetic effects into the highly effective, modern, finite-element MHD models.
    \end{abstract}
    
    
    \newpage
    \tableofcontents
    
    
    \newpage
    \pagenumbering{arabic}
    %\linenumbers\renewcommand\thelinenumber{\color{black!50}\arabic{linenumber}}
            \input{0 - introduction/main.tex}
        \part{Research}
            \input{1 - low-noise PiC models/main.tex}
            \input{2 - kinetic component/main.tex}
            \input{3 - fluid component/main.tex}
            \input{4 - numerical implementation/main.tex}
        \part{Project Overview}
            \input{5 - research plan/main.tex}
            \input{6 - summary/main.tex}
    
    
    %\section{}
    \newpage
    \pagenumbering{gobble}
        \printbibliography


    \newpage
    \pagenumbering{roman}
    \appendix
        \part{Appendices}
            \input{8 - Hilbert complexes/main.tex}
            \input{9 - weak conservation proofs/main.tex}
\end{document}

        \part{Research}
            \documentclass[12pt, a4paper]{report}

\input{template/main.tex}

\title{\BA{Title in Progress...}}
\author{Boris Andrews}
\affil{Mathematical Institute, University of Oxford}
\date{\today}


\begin{document}
    \pagenumbering{gobble}
    \maketitle
    
    
    \begin{abstract}
        Magnetic confinement reactors---in particular tokamaks---offer one of the most promising options for achieving practical nuclear fusion, with the potential to provide virtually limitless, clean energy. The theoretical and numerical modeling of tokamak plasmas is simultaneously an essential component of effective reactor design, and a great research barrier. Tokamak operational conditions exhibit comparatively low Knudsen numbers. Kinetic effects, including kinetic waves and instabilities, Landau damping, bump-on-tail instabilities and more, are therefore highly influential in tokamak plasma dynamics. Purely fluid models are inherently incapable of capturing these effects, whereas the high dimensionality in purely kinetic models render them practically intractable for most relevant purposes.

        We consider a $\delta\!f$ decomposition model, with a macroscopic fluid background and microscopic kinetic correction, both fully coupled to each other. A similar manner of discretization is proposed to that used in the recent \texttt{STRUPHY} code \cite{Holderied_Possanner_Wang_2021, Holderied_2022, Li_et_al_2023} with a finite-element model for the background and a pseudo-particle/PiC model for the correction.

        The fluid background satisfies the full, non-linear, resistive, compressible, Hall MHD equations. \cite{Laakmann_Hu_Farrell_2022} introduces finite-element(-in-space) implicit timesteppers for the incompressible analogue to this system with structure-preserving (SP) properties in the ideal case, alongside parameter-robust preconditioners. We show that these timesteppers can derive from a finite-element-in-time (FET) (and finite-element-in-space) interpretation. The benefits of this reformulation are discussed, including the derivation of timesteppers that are higher order in time, and the quantifiable dissipative SP properties in the non-ideal, resistive case.
        
        We discuss possible options for extending this FET approach to timesteppers for the compressible case.

        The kinetic corrections satisfy linearized Boltzmann equations. Using a Lénard--Bernstein collision operator, these take Fokker--Planck-like forms \cite{Fokker_1914, Planck_1917} wherein pseudo-particles in the numerical model obey the neoclassical transport equations, with particle-independent Brownian drift terms. This offers a rigorous methodology for incorporating collisions into the particle transport model, without coupling the equations of motions for each particle.
        
        Works by Chen, Chacón et al. \cite{Chen_Chacón_Barnes_2011, Chacón_Chen_Barnes_2013, Chen_Chacón_2014, Chen_Chacón_2015} have developed structure-preserving particle pushers for neoclassical transport in the Vlasov equations, derived from Crank--Nicolson integrators. We show these too can can derive from a FET interpretation, similarly offering potential extensions to higher-order-in-time particle pushers. The FET formulation is used also to consider how the stochastic drift terms can be incorporated into the pushers. Stochastic gyrokinetic expansions are also discussed.

        Different options for the numerical implementation of these schemes are considered.

        Due to the efficacy of FET in the development of SP timesteppers for both the fluid and kinetic component, we hope this approach will prove effective in the future for developing SP timesteppers for the full hybrid model. We hope this will give us the opportunity to incorporate previously inaccessible kinetic effects into the highly effective, modern, finite-element MHD models.
    \end{abstract}
    
    
    \newpage
    \tableofcontents
    
    
    \newpage
    \pagenumbering{arabic}
    %\linenumbers\renewcommand\thelinenumber{\color{black!50}\arabic{linenumber}}
            \input{0 - introduction/main.tex}
        \part{Research}
            \input{1 - low-noise PiC models/main.tex}
            \input{2 - kinetic component/main.tex}
            \input{3 - fluid component/main.tex}
            \input{4 - numerical implementation/main.tex}
        \part{Project Overview}
            \input{5 - research plan/main.tex}
            \input{6 - summary/main.tex}
    
    
    %\section{}
    \newpage
    \pagenumbering{gobble}
        \printbibliography


    \newpage
    \pagenumbering{roman}
    \appendix
        \part{Appendices}
            \input{8 - Hilbert complexes/main.tex}
            \input{9 - weak conservation proofs/main.tex}
\end{document}

            \documentclass[12pt, a4paper]{report}

\input{template/main.tex}

\title{\BA{Title in Progress...}}
\author{Boris Andrews}
\affil{Mathematical Institute, University of Oxford}
\date{\today}


\begin{document}
    \pagenumbering{gobble}
    \maketitle
    
    
    \begin{abstract}
        Magnetic confinement reactors---in particular tokamaks---offer one of the most promising options for achieving practical nuclear fusion, with the potential to provide virtually limitless, clean energy. The theoretical and numerical modeling of tokamak plasmas is simultaneously an essential component of effective reactor design, and a great research barrier. Tokamak operational conditions exhibit comparatively low Knudsen numbers. Kinetic effects, including kinetic waves and instabilities, Landau damping, bump-on-tail instabilities and more, are therefore highly influential in tokamak plasma dynamics. Purely fluid models are inherently incapable of capturing these effects, whereas the high dimensionality in purely kinetic models render them practically intractable for most relevant purposes.

        We consider a $\delta\!f$ decomposition model, with a macroscopic fluid background and microscopic kinetic correction, both fully coupled to each other. A similar manner of discretization is proposed to that used in the recent \texttt{STRUPHY} code \cite{Holderied_Possanner_Wang_2021, Holderied_2022, Li_et_al_2023} with a finite-element model for the background and a pseudo-particle/PiC model for the correction.

        The fluid background satisfies the full, non-linear, resistive, compressible, Hall MHD equations. \cite{Laakmann_Hu_Farrell_2022} introduces finite-element(-in-space) implicit timesteppers for the incompressible analogue to this system with structure-preserving (SP) properties in the ideal case, alongside parameter-robust preconditioners. We show that these timesteppers can derive from a finite-element-in-time (FET) (and finite-element-in-space) interpretation. The benefits of this reformulation are discussed, including the derivation of timesteppers that are higher order in time, and the quantifiable dissipative SP properties in the non-ideal, resistive case.
        
        We discuss possible options for extending this FET approach to timesteppers for the compressible case.

        The kinetic corrections satisfy linearized Boltzmann equations. Using a Lénard--Bernstein collision operator, these take Fokker--Planck-like forms \cite{Fokker_1914, Planck_1917} wherein pseudo-particles in the numerical model obey the neoclassical transport equations, with particle-independent Brownian drift terms. This offers a rigorous methodology for incorporating collisions into the particle transport model, without coupling the equations of motions for each particle.
        
        Works by Chen, Chacón et al. \cite{Chen_Chacón_Barnes_2011, Chacón_Chen_Barnes_2013, Chen_Chacón_2014, Chen_Chacón_2015} have developed structure-preserving particle pushers for neoclassical transport in the Vlasov equations, derived from Crank--Nicolson integrators. We show these too can can derive from a FET interpretation, similarly offering potential extensions to higher-order-in-time particle pushers. The FET formulation is used also to consider how the stochastic drift terms can be incorporated into the pushers. Stochastic gyrokinetic expansions are also discussed.

        Different options for the numerical implementation of these schemes are considered.

        Due to the efficacy of FET in the development of SP timesteppers for both the fluid and kinetic component, we hope this approach will prove effective in the future for developing SP timesteppers for the full hybrid model. We hope this will give us the opportunity to incorporate previously inaccessible kinetic effects into the highly effective, modern, finite-element MHD models.
    \end{abstract}
    
    
    \newpage
    \tableofcontents
    
    
    \newpage
    \pagenumbering{arabic}
    %\linenumbers\renewcommand\thelinenumber{\color{black!50}\arabic{linenumber}}
            \input{0 - introduction/main.tex}
        \part{Research}
            \input{1 - low-noise PiC models/main.tex}
            \input{2 - kinetic component/main.tex}
            \input{3 - fluid component/main.tex}
            \input{4 - numerical implementation/main.tex}
        \part{Project Overview}
            \input{5 - research plan/main.tex}
            \input{6 - summary/main.tex}
    
    
    %\section{}
    \newpage
    \pagenumbering{gobble}
        \printbibliography


    \newpage
    \pagenumbering{roman}
    \appendix
        \part{Appendices}
            \input{8 - Hilbert complexes/main.tex}
            \input{9 - weak conservation proofs/main.tex}
\end{document}

            \documentclass[12pt, a4paper]{report}

\input{template/main.tex}

\title{\BA{Title in Progress...}}
\author{Boris Andrews}
\affil{Mathematical Institute, University of Oxford}
\date{\today}


\begin{document}
    \pagenumbering{gobble}
    \maketitle
    
    
    \begin{abstract}
        Magnetic confinement reactors---in particular tokamaks---offer one of the most promising options for achieving practical nuclear fusion, with the potential to provide virtually limitless, clean energy. The theoretical and numerical modeling of tokamak plasmas is simultaneously an essential component of effective reactor design, and a great research barrier. Tokamak operational conditions exhibit comparatively low Knudsen numbers. Kinetic effects, including kinetic waves and instabilities, Landau damping, bump-on-tail instabilities and more, are therefore highly influential in tokamak plasma dynamics. Purely fluid models are inherently incapable of capturing these effects, whereas the high dimensionality in purely kinetic models render them practically intractable for most relevant purposes.

        We consider a $\delta\!f$ decomposition model, with a macroscopic fluid background and microscopic kinetic correction, both fully coupled to each other. A similar manner of discretization is proposed to that used in the recent \texttt{STRUPHY} code \cite{Holderied_Possanner_Wang_2021, Holderied_2022, Li_et_al_2023} with a finite-element model for the background and a pseudo-particle/PiC model for the correction.

        The fluid background satisfies the full, non-linear, resistive, compressible, Hall MHD equations. \cite{Laakmann_Hu_Farrell_2022} introduces finite-element(-in-space) implicit timesteppers for the incompressible analogue to this system with structure-preserving (SP) properties in the ideal case, alongside parameter-robust preconditioners. We show that these timesteppers can derive from a finite-element-in-time (FET) (and finite-element-in-space) interpretation. The benefits of this reformulation are discussed, including the derivation of timesteppers that are higher order in time, and the quantifiable dissipative SP properties in the non-ideal, resistive case.
        
        We discuss possible options for extending this FET approach to timesteppers for the compressible case.

        The kinetic corrections satisfy linearized Boltzmann equations. Using a Lénard--Bernstein collision operator, these take Fokker--Planck-like forms \cite{Fokker_1914, Planck_1917} wherein pseudo-particles in the numerical model obey the neoclassical transport equations, with particle-independent Brownian drift terms. This offers a rigorous methodology for incorporating collisions into the particle transport model, without coupling the equations of motions for each particle.
        
        Works by Chen, Chacón et al. \cite{Chen_Chacón_Barnes_2011, Chacón_Chen_Barnes_2013, Chen_Chacón_2014, Chen_Chacón_2015} have developed structure-preserving particle pushers for neoclassical transport in the Vlasov equations, derived from Crank--Nicolson integrators. We show these too can can derive from a FET interpretation, similarly offering potential extensions to higher-order-in-time particle pushers. The FET formulation is used also to consider how the stochastic drift terms can be incorporated into the pushers. Stochastic gyrokinetic expansions are also discussed.

        Different options for the numerical implementation of these schemes are considered.

        Due to the efficacy of FET in the development of SP timesteppers for both the fluid and kinetic component, we hope this approach will prove effective in the future for developing SP timesteppers for the full hybrid model. We hope this will give us the opportunity to incorporate previously inaccessible kinetic effects into the highly effective, modern, finite-element MHD models.
    \end{abstract}
    
    
    \newpage
    \tableofcontents
    
    
    \newpage
    \pagenumbering{arabic}
    %\linenumbers\renewcommand\thelinenumber{\color{black!50}\arabic{linenumber}}
            \input{0 - introduction/main.tex}
        \part{Research}
            \input{1 - low-noise PiC models/main.tex}
            \input{2 - kinetic component/main.tex}
            \input{3 - fluid component/main.tex}
            \input{4 - numerical implementation/main.tex}
        \part{Project Overview}
            \input{5 - research plan/main.tex}
            \input{6 - summary/main.tex}
    
    
    %\section{}
    \newpage
    \pagenumbering{gobble}
        \printbibliography


    \newpage
    \pagenumbering{roman}
    \appendix
        \part{Appendices}
            \input{8 - Hilbert complexes/main.tex}
            \input{9 - weak conservation proofs/main.tex}
\end{document}

            \documentclass[12pt, a4paper]{report}

\input{template/main.tex}

\title{\BA{Title in Progress...}}
\author{Boris Andrews}
\affil{Mathematical Institute, University of Oxford}
\date{\today}


\begin{document}
    \pagenumbering{gobble}
    \maketitle
    
    
    \begin{abstract}
        Magnetic confinement reactors---in particular tokamaks---offer one of the most promising options for achieving practical nuclear fusion, with the potential to provide virtually limitless, clean energy. The theoretical and numerical modeling of tokamak plasmas is simultaneously an essential component of effective reactor design, and a great research barrier. Tokamak operational conditions exhibit comparatively low Knudsen numbers. Kinetic effects, including kinetic waves and instabilities, Landau damping, bump-on-tail instabilities and more, are therefore highly influential in tokamak plasma dynamics. Purely fluid models are inherently incapable of capturing these effects, whereas the high dimensionality in purely kinetic models render them practically intractable for most relevant purposes.

        We consider a $\delta\!f$ decomposition model, with a macroscopic fluid background and microscopic kinetic correction, both fully coupled to each other. A similar manner of discretization is proposed to that used in the recent \texttt{STRUPHY} code \cite{Holderied_Possanner_Wang_2021, Holderied_2022, Li_et_al_2023} with a finite-element model for the background and a pseudo-particle/PiC model for the correction.

        The fluid background satisfies the full, non-linear, resistive, compressible, Hall MHD equations. \cite{Laakmann_Hu_Farrell_2022} introduces finite-element(-in-space) implicit timesteppers for the incompressible analogue to this system with structure-preserving (SP) properties in the ideal case, alongside parameter-robust preconditioners. We show that these timesteppers can derive from a finite-element-in-time (FET) (and finite-element-in-space) interpretation. The benefits of this reformulation are discussed, including the derivation of timesteppers that are higher order in time, and the quantifiable dissipative SP properties in the non-ideal, resistive case.
        
        We discuss possible options for extending this FET approach to timesteppers for the compressible case.

        The kinetic corrections satisfy linearized Boltzmann equations. Using a Lénard--Bernstein collision operator, these take Fokker--Planck-like forms \cite{Fokker_1914, Planck_1917} wherein pseudo-particles in the numerical model obey the neoclassical transport equations, with particle-independent Brownian drift terms. This offers a rigorous methodology for incorporating collisions into the particle transport model, without coupling the equations of motions for each particle.
        
        Works by Chen, Chacón et al. \cite{Chen_Chacón_Barnes_2011, Chacón_Chen_Barnes_2013, Chen_Chacón_2014, Chen_Chacón_2015} have developed structure-preserving particle pushers for neoclassical transport in the Vlasov equations, derived from Crank--Nicolson integrators. We show these too can can derive from a FET interpretation, similarly offering potential extensions to higher-order-in-time particle pushers. The FET formulation is used also to consider how the stochastic drift terms can be incorporated into the pushers. Stochastic gyrokinetic expansions are also discussed.

        Different options for the numerical implementation of these schemes are considered.

        Due to the efficacy of FET in the development of SP timesteppers for both the fluid and kinetic component, we hope this approach will prove effective in the future for developing SP timesteppers for the full hybrid model. We hope this will give us the opportunity to incorporate previously inaccessible kinetic effects into the highly effective, modern, finite-element MHD models.
    \end{abstract}
    
    
    \newpage
    \tableofcontents
    
    
    \newpage
    \pagenumbering{arabic}
    %\linenumbers\renewcommand\thelinenumber{\color{black!50}\arabic{linenumber}}
            \input{0 - introduction/main.tex}
        \part{Research}
            \input{1 - low-noise PiC models/main.tex}
            \input{2 - kinetic component/main.tex}
            \input{3 - fluid component/main.tex}
            \input{4 - numerical implementation/main.tex}
        \part{Project Overview}
            \input{5 - research plan/main.tex}
            \input{6 - summary/main.tex}
    
    
    %\section{}
    \newpage
    \pagenumbering{gobble}
        \printbibliography


    \newpage
    \pagenumbering{roman}
    \appendix
        \part{Appendices}
            \input{8 - Hilbert complexes/main.tex}
            \input{9 - weak conservation proofs/main.tex}
\end{document}

        \part{Project Overview}
            \documentclass[12pt, a4paper]{report}

\input{template/main.tex}

\title{\BA{Title in Progress...}}
\author{Boris Andrews}
\affil{Mathematical Institute, University of Oxford}
\date{\today}


\begin{document}
    \pagenumbering{gobble}
    \maketitle
    
    
    \begin{abstract}
        Magnetic confinement reactors---in particular tokamaks---offer one of the most promising options for achieving practical nuclear fusion, with the potential to provide virtually limitless, clean energy. The theoretical and numerical modeling of tokamak plasmas is simultaneously an essential component of effective reactor design, and a great research barrier. Tokamak operational conditions exhibit comparatively low Knudsen numbers. Kinetic effects, including kinetic waves and instabilities, Landau damping, bump-on-tail instabilities and more, are therefore highly influential in tokamak plasma dynamics. Purely fluid models are inherently incapable of capturing these effects, whereas the high dimensionality in purely kinetic models render them practically intractable for most relevant purposes.

        We consider a $\delta\!f$ decomposition model, with a macroscopic fluid background and microscopic kinetic correction, both fully coupled to each other. A similar manner of discretization is proposed to that used in the recent \texttt{STRUPHY} code \cite{Holderied_Possanner_Wang_2021, Holderied_2022, Li_et_al_2023} with a finite-element model for the background and a pseudo-particle/PiC model for the correction.

        The fluid background satisfies the full, non-linear, resistive, compressible, Hall MHD equations. \cite{Laakmann_Hu_Farrell_2022} introduces finite-element(-in-space) implicit timesteppers for the incompressible analogue to this system with structure-preserving (SP) properties in the ideal case, alongside parameter-robust preconditioners. We show that these timesteppers can derive from a finite-element-in-time (FET) (and finite-element-in-space) interpretation. The benefits of this reformulation are discussed, including the derivation of timesteppers that are higher order in time, and the quantifiable dissipative SP properties in the non-ideal, resistive case.
        
        We discuss possible options for extending this FET approach to timesteppers for the compressible case.

        The kinetic corrections satisfy linearized Boltzmann equations. Using a Lénard--Bernstein collision operator, these take Fokker--Planck-like forms \cite{Fokker_1914, Planck_1917} wherein pseudo-particles in the numerical model obey the neoclassical transport equations, with particle-independent Brownian drift terms. This offers a rigorous methodology for incorporating collisions into the particle transport model, without coupling the equations of motions for each particle.
        
        Works by Chen, Chacón et al. \cite{Chen_Chacón_Barnes_2011, Chacón_Chen_Barnes_2013, Chen_Chacón_2014, Chen_Chacón_2015} have developed structure-preserving particle pushers for neoclassical transport in the Vlasov equations, derived from Crank--Nicolson integrators. We show these too can can derive from a FET interpretation, similarly offering potential extensions to higher-order-in-time particle pushers. The FET formulation is used also to consider how the stochastic drift terms can be incorporated into the pushers. Stochastic gyrokinetic expansions are also discussed.

        Different options for the numerical implementation of these schemes are considered.

        Due to the efficacy of FET in the development of SP timesteppers for both the fluid and kinetic component, we hope this approach will prove effective in the future for developing SP timesteppers for the full hybrid model. We hope this will give us the opportunity to incorporate previously inaccessible kinetic effects into the highly effective, modern, finite-element MHD models.
    \end{abstract}
    
    
    \newpage
    \tableofcontents
    
    
    \newpage
    \pagenumbering{arabic}
    %\linenumbers\renewcommand\thelinenumber{\color{black!50}\arabic{linenumber}}
            \input{0 - introduction/main.tex}
        \part{Research}
            \input{1 - low-noise PiC models/main.tex}
            \input{2 - kinetic component/main.tex}
            \input{3 - fluid component/main.tex}
            \input{4 - numerical implementation/main.tex}
        \part{Project Overview}
            \input{5 - research plan/main.tex}
            \input{6 - summary/main.tex}
    
    
    %\section{}
    \newpage
    \pagenumbering{gobble}
        \printbibliography


    \newpage
    \pagenumbering{roman}
    \appendix
        \part{Appendices}
            \input{8 - Hilbert complexes/main.tex}
            \input{9 - weak conservation proofs/main.tex}
\end{document}

            \documentclass[12pt, a4paper]{report}

\input{template/main.tex}

\title{\BA{Title in Progress...}}
\author{Boris Andrews}
\affil{Mathematical Institute, University of Oxford}
\date{\today}


\begin{document}
    \pagenumbering{gobble}
    \maketitle
    
    
    \begin{abstract}
        Magnetic confinement reactors---in particular tokamaks---offer one of the most promising options for achieving practical nuclear fusion, with the potential to provide virtually limitless, clean energy. The theoretical and numerical modeling of tokamak plasmas is simultaneously an essential component of effective reactor design, and a great research barrier. Tokamak operational conditions exhibit comparatively low Knudsen numbers. Kinetic effects, including kinetic waves and instabilities, Landau damping, bump-on-tail instabilities and more, are therefore highly influential in tokamak plasma dynamics. Purely fluid models are inherently incapable of capturing these effects, whereas the high dimensionality in purely kinetic models render them practically intractable for most relevant purposes.

        We consider a $\delta\!f$ decomposition model, with a macroscopic fluid background and microscopic kinetic correction, both fully coupled to each other. A similar manner of discretization is proposed to that used in the recent \texttt{STRUPHY} code \cite{Holderied_Possanner_Wang_2021, Holderied_2022, Li_et_al_2023} with a finite-element model for the background and a pseudo-particle/PiC model for the correction.

        The fluid background satisfies the full, non-linear, resistive, compressible, Hall MHD equations. \cite{Laakmann_Hu_Farrell_2022} introduces finite-element(-in-space) implicit timesteppers for the incompressible analogue to this system with structure-preserving (SP) properties in the ideal case, alongside parameter-robust preconditioners. We show that these timesteppers can derive from a finite-element-in-time (FET) (and finite-element-in-space) interpretation. The benefits of this reformulation are discussed, including the derivation of timesteppers that are higher order in time, and the quantifiable dissipative SP properties in the non-ideal, resistive case.
        
        We discuss possible options for extending this FET approach to timesteppers for the compressible case.

        The kinetic corrections satisfy linearized Boltzmann equations. Using a Lénard--Bernstein collision operator, these take Fokker--Planck-like forms \cite{Fokker_1914, Planck_1917} wherein pseudo-particles in the numerical model obey the neoclassical transport equations, with particle-independent Brownian drift terms. This offers a rigorous methodology for incorporating collisions into the particle transport model, without coupling the equations of motions for each particle.
        
        Works by Chen, Chacón et al. \cite{Chen_Chacón_Barnes_2011, Chacón_Chen_Barnes_2013, Chen_Chacón_2014, Chen_Chacón_2015} have developed structure-preserving particle pushers for neoclassical transport in the Vlasov equations, derived from Crank--Nicolson integrators. We show these too can can derive from a FET interpretation, similarly offering potential extensions to higher-order-in-time particle pushers. The FET formulation is used also to consider how the stochastic drift terms can be incorporated into the pushers. Stochastic gyrokinetic expansions are also discussed.

        Different options for the numerical implementation of these schemes are considered.

        Due to the efficacy of FET in the development of SP timesteppers for both the fluid and kinetic component, we hope this approach will prove effective in the future for developing SP timesteppers for the full hybrid model. We hope this will give us the opportunity to incorporate previously inaccessible kinetic effects into the highly effective, modern, finite-element MHD models.
    \end{abstract}
    
    
    \newpage
    \tableofcontents
    
    
    \newpage
    \pagenumbering{arabic}
    %\linenumbers\renewcommand\thelinenumber{\color{black!50}\arabic{linenumber}}
            \input{0 - introduction/main.tex}
        \part{Research}
            \input{1 - low-noise PiC models/main.tex}
            \input{2 - kinetic component/main.tex}
            \input{3 - fluid component/main.tex}
            \input{4 - numerical implementation/main.tex}
        \part{Project Overview}
            \input{5 - research plan/main.tex}
            \input{6 - summary/main.tex}
    
    
    %\section{}
    \newpage
    \pagenumbering{gobble}
        \printbibliography


    \newpage
    \pagenumbering{roman}
    \appendix
        \part{Appendices}
            \input{8 - Hilbert complexes/main.tex}
            \input{9 - weak conservation proofs/main.tex}
\end{document}

    
    
    %\section{}
    \newpage
    \pagenumbering{gobble}
        \printbibliography


    \newpage
    \pagenumbering{roman}
    \appendix
        \part{Appendices}
            \documentclass[12pt, a4paper]{report}

\input{template/main.tex}

\title{\BA{Title in Progress...}}
\author{Boris Andrews}
\affil{Mathematical Institute, University of Oxford}
\date{\today}


\begin{document}
    \pagenumbering{gobble}
    \maketitle
    
    
    \begin{abstract}
        Magnetic confinement reactors---in particular tokamaks---offer one of the most promising options for achieving practical nuclear fusion, with the potential to provide virtually limitless, clean energy. The theoretical and numerical modeling of tokamak plasmas is simultaneously an essential component of effective reactor design, and a great research barrier. Tokamak operational conditions exhibit comparatively low Knudsen numbers. Kinetic effects, including kinetic waves and instabilities, Landau damping, bump-on-tail instabilities and more, are therefore highly influential in tokamak plasma dynamics. Purely fluid models are inherently incapable of capturing these effects, whereas the high dimensionality in purely kinetic models render them practically intractable for most relevant purposes.

        We consider a $\delta\!f$ decomposition model, with a macroscopic fluid background and microscopic kinetic correction, both fully coupled to each other. A similar manner of discretization is proposed to that used in the recent \texttt{STRUPHY} code \cite{Holderied_Possanner_Wang_2021, Holderied_2022, Li_et_al_2023} with a finite-element model for the background and a pseudo-particle/PiC model for the correction.

        The fluid background satisfies the full, non-linear, resistive, compressible, Hall MHD equations. \cite{Laakmann_Hu_Farrell_2022} introduces finite-element(-in-space) implicit timesteppers for the incompressible analogue to this system with structure-preserving (SP) properties in the ideal case, alongside parameter-robust preconditioners. We show that these timesteppers can derive from a finite-element-in-time (FET) (and finite-element-in-space) interpretation. The benefits of this reformulation are discussed, including the derivation of timesteppers that are higher order in time, and the quantifiable dissipative SP properties in the non-ideal, resistive case.
        
        We discuss possible options for extending this FET approach to timesteppers for the compressible case.

        The kinetic corrections satisfy linearized Boltzmann equations. Using a Lénard--Bernstein collision operator, these take Fokker--Planck-like forms \cite{Fokker_1914, Planck_1917} wherein pseudo-particles in the numerical model obey the neoclassical transport equations, with particle-independent Brownian drift terms. This offers a rigorous methodology for incorporating collisions into the particle transport model, without coupling the equations of motions for each particle.
        
        Works by Chen, Chacón et al. \cite{Chen_Chacón_Barnes_2011, Chacón_Chen_Barnes_2013, Chen_Chacón_2014, Chen_Chacón_2015} have developed structure-preserving particle pushers for neoclassical transport in the Vlasov equations, derived from Crank--Nicolson integrators. We show these too can can derive from a FET interpretation, similarly offering potential extensions to higher-order-in-time particle pushers. The FET formulation is used also to consider how the stochastic drift terms can be incorporated into the pushers. Stochastic gyrokinetic expansions are also discussed.

        Different options for the numerical implementation of these schemes are considered.

        Due to the efficacy of FET in the development of SP timesteppers for both the fluid and kinetic component, we hope this approach will prove effective in the future for developing SP timesteppers for the full hybrid model. We hope this will give us the opportunity to incorporate previously inaccessible kinetic effects into the highly effective, modern, finite-element MHD models.
    \end{abstract}
    
    
    \newpage
    \tableofcontents
    
    
    \newpage
    \pagenumbering{arabic}
    %\linenumbers\renewcommand\thelinenumber{\color{black!50}\arabic{linenumber}}
            \input{0 - introduction/main.tex}
        \part{Research}
            \input{1 - low-noise PiC models/main.tex}
            \input{2 - kinetic component/main.tex}
            \input{3 - fluid component/main.tex}
            \input{4 - numerical implementation/main.tex}
        \part{Project Overview}
            \input{5 - research plan/main.tex}
            \input{6 - summary/main.tex}
    
    
    %\section{}
    \newpage
    \pagenumbering{gobble}
        \printbibliography


    \newpage
    \pagenumbering{roman}
    \appendix
        \part{Appendices}
            \input{8 - Hilbert complexes/main.tex}
            \input{9 - weak conservation proofs/main.tex}
\end{document}

            \documentclass[12pt, a4paper]{report}

\input{template/main.tex}

\title{\BA{Title in Progress...}}
\author{Boris Andrews}
\affil{Mathematical Institute, University of Oxford}
\date{\today}


\begin{document}
    \pagenumbering{gobble}
    \maketitle
    
    
    \begin{abstract}
        Magnetic confinement reactors---in particular tokamaks---offer one of the most promising options for achieving practical nuclear fusion, with the potential to provide virtually limitless, clean energy. The theoretical and numerical modeling of tokamak plasmas is simultaneously an essential component of effective reactor design, and a great research barrier. Tokamak operational conditions exhibit comparatively low Knudsen numbers. Kinetic effects, including kinetic waves and instabilities, Landau damping, bump-on-tail instabilities and more, are therefore highly influential in tokamak plasma dynamics. Purely fluid models are inherently incapable of capturing these effects, whereas the high dimensionality in purely kinetic models render them practically intractable for most relevant purposes.

        We consider a $\delta\!f$ decomposition model, with a macroscopic fluid background and microscopic kinetic correction, both fully coupled to each other. A similar manner of discretization is proposed to that used in the recent \texttt{STRUPHY} code \cite{Holderied_Possanner_Wang_2021, Holderied_2022, Li_et_al_2023} with a finite-element model for the background and a pseudo-particle/PiC model for the correction.

        The fluid background satisfies the full, non-linear, resistive, compressible, Hall MHD equations. \cite{Laakmann_Hu_Farrell_2022} introduces finite-element(-in-space) implicit timesteppers for the incompressible analogue to this system with structure-preserving (SP) properties in the ideal case, alongside parameter-robust preconditioners. We show that these timesteppers can derive from a finite-element-in-time (FET) (and finite-element-in-space) interpretation. The benefits of this reformulation are discussed, including the derivation of timesteppers that are higher order in time, and the quantifiable dissipative SP properties in the non-ideal, resistive case.
        
        We discuss possible options for extending this FET approach to timesteppers for the compressible case.

        The kinetic corrections satisfy linearized Boltzmann equations. Using a Lénard--Bernstein collision operator, these take Fokker--Planck-like forms \cite{Fokker_1914, Planck_1917} wherein pseudo-particles in the numerical model obey the neoclassical transport equations, with particle-independent Brownian drift terms. This offers a rigorous methodology for incorporating collisions into the particle transport model, without coupling the equations of motions for each particle.
        
        Works by Chen, Chacón et al. \cite{Chen_Chacón_Barnes_2011, Chacón_Chen_Barnes_2013, Chen_Chacón_2014, Chen_Chacón_2015} have developed structure-preserving particle pushers for neoclassical transport in the Vlasov equations, derived from Crank--Nicolson integrators. We show these too can can derive from a FET interpretation, similarly offering potential extensions to higher-order-in-time particle pushers. The FET formulation is used also to consider how the stochastic drift terms can be incorporated into the pushers. Stochastic gyrokinetic expansions are also discussed.

        Different options for the numerical implementation of these schemes are considered.

        Due to the efficacy of FET in the development of SP timesteppers for both the fluid and kinetic component, we hope this approach will prove effective in the future for developing SP timesteppers for the full hybrid model. We hope this will give us the opportunity to incorporate previously inaccessible kinetic effects into the highly effective, modern, finite-element MHD models.
    \end{abstract}
    
    
    \newpage
    \tableofcontents
    
    
    \newpage
    \pagenumbering{arabic}
    %\linenumbers\renewcommand\thelinenumber{\color{black!50}\arabic{linenumber}}
            \input{0 - introduction/main.tex}
        \part{Research}
            \input{1 - low-noise PiC models/main.tex}
            \input{2 - kinetic component/main.tex}
            \input{3 - fluid component/main.tex}
            \input{4 - numerical implementation/main.tex}
        \part{Project Overview}
            \input{5 - research plan/main.tex}
            \input{6 - summary/main.tex}
    
    
    %\section{}
    \newpage
    \pagenumbering{gobble}
        \printbibliography


    \newpage
    \pagenumbering{roman}
    \appendix
        \part{Appendices}
            \input{8 - Hilbert complexes/main.tex}
            \input{9 - weak conservation proofs/main.tex}
\end{document}

\end{document}

            \documentclass[12pt, a4paper]{report}

\documentclass[12pt, a4paper]{report}

\input{template/main.tex}

\title{\BA{Title in Progress...}}
\author{Boris Andrews}
\affil{Mathematical Institute, University of Oxford}
\date{\today}


\begin{document}
    \pagenumbering{gobble}
    \maketitle
    
    
    \begin{abstract}
        Magnetic confinement reactors---in particular tokamaks---offer one of the most promising options for achieving practical nuclear fusion, with the potential to provide virtually limitless, clean energy. The theoretical and numerical modeling of tokamak plasmas is simultaneously an essential component of effective reactor design, and a great research barrier. Tokamak operational conditions exhibit comparatively low Knudsen numbers. Kinetic effects, including kinetic waves and instabilities, Landau damping, bump-on-tail instabilities and more, are therefore highly influential in tokamak plasma dynamics. Purely fluid models are inherently incapable of capturing these effects, whereas the high dimensionality in purely kinetic models render them practically intractable for most relevant purposes.

        We consider a $\delta\!f$ decomposition model, with a macroscopic fluid background and microscopic kinetic correction, both fully coupled to each other. A similar manner of discretization is proposed to that used in the recent \texttt{STRUPHY} code \cite{Holderied_Possanner_Wang_2021, Holderied_2022, Li_et_al_2023} with a finite-element model for the background and a pseudo-particle/PiC model for the correction.

        The fluid background satisfies the full, non-linear, resistive, compressible, Hall MHD equations. \cite{Laakmann_Hu_Farrell_2022} introduces finite-element(-in-space) implicit timesteppers for the incompressible analogue to this system with structure-preserving (SP) properties in the ideal case, alongside parameter-robust preconditioners. We show that these timesteppers can derive from a finite-element-in-time (FET) (and finite-element-in-space) interpretation. The benefits of this reformulation are discussed, including the derivation of timesteppers that are higher order in time, and the quantifiable dissipative SP properties in the non-ideal, resistive case.
        
        We discuss possible options for extending this FET approach to timesteppers for the compressible case.

        The kinetic corrections satisfy linearized Boltzmann equations. Using a Lénard--Bernstein collision operator, these take Fokker--Planck-like forms \cite{Fokker_1914, Planck_1917} wherein pseudo-particles in the numerical model obey the neoclassical transport equations, with particle-independent Brownian drift terms. This offers a rigorous methodology for incorporating collisions into the particle transport model, without coupling the equations of motions for each particle.
        
        Works by Chen, Chacón et al. \cite{Chen_Chacón_Barnes_2011, Chacón_Chen_Barnes_2013, Chen_Chacón_2014, Chen_Chacón_2015} have developed structure-preserving particle pushers for neoclassical transport in the Vlasov equations, derived from Crank--Nicolson integrators. We show these too can can derive from a FET interpretation, similarly offering potential extensions to higher-order-in-time particle pushers. The FET formulation is used also to consider how the stochastic drift terms can be incorporated into the pushers. Stochastic gyrokinetic expansions are also discussed.

        Different options for the numerical implementation of these schemes are considered.

        Due to the efficacy of FET in the development of SP timesteppers for both the fluid and kinetic component, we hope this approach will prove effective in the future for developing SP timesteppers for the full hybrid model. We hope this will give us the opportunity to incorporate previously inaccessible kinetic effects into the highly effective, modern, finite-element MHD models.
    \end{abstract}
    
    
    \newpage
    \tableofcontents
    
    
    \newpage
    \pagenumbering{arabic}
    %\linenumbers\renewcommand\thelinenumber{\color{black!50}\arabic{linenumber}}
            \input{0 - introduction/main.tex}
        \part{Research}
            \input{1 - low-noise PiC models/main.tex}
            \input{2 - kinetic component/main.tex}
            \input{3 - fluid component/main.tex}
            \input{4 - numerical implementation/main.tex}
        \part{Project Overview}
            \input{5 - research plan/main.tex}
            \input{6 - summary/main.tex}
    
    
    %\section{}
    \newpage
    \pagenumbering{gobble}
        \printbibliography


    \newpage
    \pagenumbering{roman}
    \appendix
        \part{Appendices}
            \input{8 - Hilbert complexes/main.tex}
            \input{9 - weak conservation proofs/main.tex}
\end{document}


\title{\BA{Title in Progress...}}
\author{Boris Andrews}
\affil{Mathematical Institute, University of Oxford}
\date{\today}


\begin{document}
    \pagenumbering{gobble}
    \maketitle
    
    
    \begin{abstract}
        Magnetic confinement reactors---in particular tokamaks---offer one of the most promising options for achieving practical nuclear fusion, with the potential to provide virtually limitless, clean energy. The theoretical and numerical modeling of tokamak plasmas is simultaneously an essential component of effective reactor design, and a great research barrier. Tokamak operational conditions exhibit comparatively low Knudsen numbers. Kinetic effects, including kinetic waves and instabilities, Landau damping, bump-on-tail instabilities and more, are therefore highly influential in tokamak plasma dynamics. Purely fluid models are inherently incapable of capturing these effects, whereas the high dimensionality in purely kinetic models render them practically intractable for most relevant purposes.

        We consider a $\delta\!f$ decomposition model, with a macroscopic fluid background and microscopic kinetic correction, both fully coupled to each other. A similar manner of discretization is proposed to that used in the recent \texttt{STRUPHY} code \cite{Holderied_Possanner_Wang_2021, Holderied_2022, Li_et_al_2023} with a finite-element model for the background and a pseudo-particle/PiC model for the correction.

        The fluid background satisfies the full, non-linear, resistive, compressible, Hall MHD equations. \cite{Laakmann_Hu_Farrell_2022} introduces finite-element(-in-space) implicit timesteppers for the incompressible analogue to this system with structure-preserving (SP) properties in the ideal case, alongside parameter-robust preconditioners. We show that these timesteppers can derive from a finite-element-in-time (FET) (and finite-element-in-space) interpretation. The benefits of this reformulation are discussed, including the derivation of timesteppers that are higher order in time, and the quantifiable dissipative SP properties in the non-ideal, resistive case.
        
        We discuss possible options for extending this FET approach to timesteppers for the compressible case.

        The kinetic corrections satisfy linearized Boltzmann equations. Using a Lénard--Bernstein collision operator, these take Fokker--Planck-like forms \cite{Fokker_1914, Planck_1917} wherein pseudo-particles in the numerical model obey the neoclassical transport equations, with particle-independent Brownian drift terms. This offers a rigorous methodology for incorporating collisions into the particle transport model, without coupling the equations of motions for each particle.
        
        Works by Chen, Chacón et al. \cite{Chen_Chacón_Barnes_2011, Chacón_Chen_Barnes_2013, Chen_Chacón_2014, Chen_Chacón_2015} have developed structure-preserving particle pushers for neoclassical transport in the Vlasov equations, derived from Crank--Nicolson integrators. We show these too can can derive from a FET interpretation, similarly offering potential extensions to higher-order-in-time particle pushers. The FET formulation is used also to consider how the stochastic drift terms can be incorporated into the pushers. Stochastic gyrokinetic expansions are also discussed.

        Different options for the numerical implementation of these schemes are considered.

        Due to the efficacy of FET in the development of SP timesteppers for both the fluid and kinetic component, we hope this approach will prove effective in the future for developing SP timesteppers for the full hybrid model. We hope this will give us the opportunity to incorporate previously inaccessible kinetic effects into the highly effective, modern, finite-element MHD models.
    \end{abstract}
    
    
    \newpage
    \tableofcontents
    
    
    \newpage
    \pagenumbering{arabic}
    %\linenumbers\renewcommand\thelinenumber{\color{black!50}\arabic{linenumber}}
            \documentclass[12pt, a4paper]{report}

\input{template/main.tex}

\title{\BA{Title in Progress...}}
\author{Boris Andrews}
\affil{Mathematical Institute, University of Oxford}
\date{\today}


\begin{document}
    \pagenumbering{gobble}
    \maketitle
    
    
    \begin{abstract}
        Magnetic confinement reactors---in particular tokamaks---offer one of the most promising options for achieving practical nuclear fusion, with the potential to provide virtually limitless, clean energy. The theoretical and numerical modeling of tokamak plasmas is simultaneously an essential component of effective reactor design, and a great research barrier. Tokamak operational conditions exhibit comparatively low Knudsen numbers. Kinetic effects, including kinetic waves and instabilities, Landau damping, bump-on-tail instabilities and more, are therefore highly influential in tokamak plasma dynamics. Purely fluid models are inherently incapable of capturing these effects, whereas the high dimensionality in purely kinetic models render them practically intractable for most relevant purposes.

        We consider a $\delta\!f$ decomposition model, with a macroscopic fluid background and microscopic kinetic correction, both fully coupled to each other. A similar manner of discretization is proposed to that used in the recent \texttt{STRUPHY} code \cite{Holderied_Possanner_Wang_2021, Holderied_2022, Li_et_al_2023} with a finite-element model for the background and a pseudo-particle/PiC model for the correction.

        The fluid background satisfies the full, non-linear, resistive, compressible, Hall MHD equations. \cite{Laakmann_Hu_Farrell_2022} introduces finite-element(-in-space) implicit timesteppers for the incompressible analogue to this system with structure-preserving (SP) properties in the ideal case, alongside parameter-robust preconditioners. We show that these timesteppers can derive from a finite-element-in-time (FET) (and finite-element-in-space) interpretation. The benefits of this reformulation are discussed, including the derivation of timesteppers that are higher order in time, and the quantifiable dissipative SP properties in the non-ideal, resistive case.
        
        We discuss possible options for extending this FET approach to timesteppers for the compressible case.

        The kinetic corrections satisfy linearized Boltzmann equations. Using a Lénard--Bernstein collision operator, these take Fokker--Planck-like forms \cite{Fokker_1914, Planck_1917} wherein pseudo-particles in the numerical model obey the neoclassical transport equations, with particle-independent Brownian drift terms. This offers a rigorous methodology for incorporating collisions into the particle transport model, without coupling the equations of motions for each particle.
        
        Works by Chen, Chacón et al. \cite{Chen_Chacón_Barnes_2011, Chacón_Chen_Barnes_2013, Chen_Chacón_2014, Chen_Chacón_2015} have developed structure-preserving particle pushers for neoclassical transport in the Vlasov equations, derived from Crank--Nicolson integrators. We show these too can can derive from a FET interpretation, similarly offering potential extensions to higher-order-in-time particle pushers. The FET formulation is used also to consider how the stochastic drift terms can be incorporated into the pushers. Stochastic gyrokinetic expansions are also discussed.

        Different options for the numerical implementation of these schemes are considered.

        Due to the efficacy of FET in the development of SP timesteppers for both the fluid and kinetic component, we hope this approach will prove effective in the future for developing SP timesteppers for the full hybrid model. We hope this will give us the opportunity to incorporate previously inaccessible kinetic effects into the highly effective, modern, finite-element MHD models.
    \end{abstract}
    
    
    \newpage
    \tableofcontents
    
    
    \newpage
    \pagenumbering{arabic}
    %\linenumbers\renewcommand\thelinenumber{\color{black!50}\arabic{linenumber}}
            \input{0 - introduction/main.tex}
        \part{Research}
            \input{1 - low-noise PiC models/main.tex}
            \input{2 - kinetic component/main.tex}
            \input{3 - fluid component/main.tex}
            \input{4 - numerical implementation/main.tex}
        \part{Project Overview}
            \input{5 - research plan/main.tex}
            \input{6 - summary/main.tex}
    
    
    %\section{}
    \newpage
    \pagenumbering{gobble}
        \printbibliography


    \newpage
    \pagenumbering{roman}
    \appendix
        \part{Appendices}
            \input{8 - Hilbert complexes/main.tex}
            \input{9 - weak conservation proofs/main.tex}
\end{document}

        \part{Research}
            \documentclass[12pt, a4paper]{report}

\input{template/main.tex}

\title{\BA{Title in Progress...}}
\author{Boris Andrews}
\affil{Mathematical Institute, University of Oxford}
\date{\today}


\begin{document}
    \pagenumbering{gobble}
    \maketitle
    
    
    \begin{abstract}
        Magnetic confinement reactors---in particular tokamaks---offer one of the most promising options for achieving practical nuclear fusion, with the potential to provide virtually limitless, clean energy. The theoretical and numerical modeling of tokamak plasmas is simultaneously an essential component of effective reactor design, and a great research barrier. Tokamak operational conditions exhibit comparatively low Knudsen numbers. Kinetic effects, including kinetic waves and instabilities, Landau damping, bump-on-tail instabilities and more, are therefore highly influential in tokamak plasma dynamics. Purely fluid models are inherently incapable of capturing these effects, whereas the high dimensionality in purely kinetic models render them practically intractable for most relevant purposes.

        We consider a $\delta\!f$ decomposition model, with a macroscopic fluid background and microscopic kinetic correction, both fully coupled to each other. A similar manner of discretization is proposed to that used in the recent \texttt{STRUPHY} code \cite{Holderied_Possanner_Wang_2021, Holderied_2022, Li_et_al_2023} with a finite-element model for the background and a pseudo-particle/PiC model for the correction.

        The fluid background satisfies the full, non-linear, resistive, compressible, Hall MHD equations. \cite{Laakmann_Hu_Farrell_2022} introduces finite-element(-in-space) implicit timesteppers for the incompressible analogue to this system with structure-preserving (SP) properties in the ideal case, alongside parameter-robust preconditioners. We show that these timesteppers can derive from a finite-element-in-time (FET) (and finite-element-in-space) interpretation. The benefits of this reformulation are discussed, including the derivation of timesteppers that are higher order in time, and the quantifiable dissipative SP properties in the non-ideal, resistive case.
        
        We discuss possible options for extending this FET approach to timesteppers for the compressible case.

        The kinetic corrections satisfy linearized Boltzmann equations. Using a Lénard--Bernstein collision operator, these take Fokker--Planck-like forms \cite{Fokker_1914, Planck_1917} wherein pseudo-particles in the numerical model obey the neoclassical transport equations, with particle-independent Brownian drift terms. This offers a rigorous methodology for incorporating collisions into the particle transport model, without coupling the equations of motions for each particle.
        
        Works by Chen, Chacón et al. \cite{Chen_Chacón_Barnes_2011, Chacón_Chen_Barnes_2013, Chen_Chacón_2014, Chen_Chacón_2015} have developed structure-preserving particle pushers for neoclassical transport in the Vlasov equations, derived from Crank--Nicolson integrators. We show these too can can derive from a FET interpretation, similarly offering potential extensions to higher-order-in-time particle pushers. The FET formulation is used also to consider how the stochastic drift terms can be incorporated into the pushers. Stochastic gyrokinetic expansions are also discussed.

        Different options for the numerical implementation of these schemes are considered.

        Due to the efficacy of FET in the development of SP timesteppers for both the fluid and kinetic component, we hope this approach will prove effective in the future for developing SP timesteppers for the full hybrid model. We hope this will give us the opportunity to incorporate previously inaccessible kinetic effects into the highly effective, modern, finite-element MHD models.
    \end{abstract}
    
    
    \newpage
    \tableofcontents
    
    
    \newpage
    \pagenumbering{arabic}
    %\linenumbers\renewcommand\thelinenumber{\color{black!50}\arabic{linenumber}}
            \input{0 - introduction/main.tex}
        \part{Research}
            \input{1 - low-noise PiC models/main.tex}
            \input{2 - kinetic component/main.tex}
            \input{3 - fluid component/main.tex}
            \input{4 - numerical implementation/main.tex}
        \part{Project Overview}
            \input{5 - research plan/main.tex}
            \input{6 - summary/main.tex}
    
    
    %\section{}
    \newpage
    \pagenumbering{gobble}
        \printbibliography


    \newpage
    \pagenumbering{roman}
    \appendix
        \part{Appendices}
            \input{8 - Hilbert complexes/main.tex}
            \input{9 - weak conservation proofs/main.tex}
\end{document}

            \documentclass[12pt, a4paper]{report}

\input{template/main.tex}

\title{\BA{Title in Progress...}}
\author{Boris Andrews}
\affil{Mathematical Institute, University of Oxford}
\date{\today}


\begin{document}
    \pagenumbering{gobble}
    \maketitle
    
    
    \begin{abstract}
        Magnetic confinement reactors---in particular tokamaks---offer one of the most promising options for achieving practical nuclear fusion, with the potential to provide virtually limitless, clean energy. The theoretical and numerical modeling of tokamak plasmas is simultaneously an essential component of effective reactor design, and a great research barrier. Tokamak operational conditions exhibit comparatively low Knudsen numbers. Kinetic effects, including kinetic waves and instabilities, Landau damping, bump-on-tail instabilities and more, are therefore highly influential in tokamak plasma dynamics. Purely fluid models are inherently incapable of capturing these effects, whereas the high dimensionality in purely kinetic models render them practically intractable for most relevant purposes.

        We consider a $\delta\!f$ decomposition model, with a macroscopic fluid background and microscopic kinetic correction, both fully coupled to each other. A similar manner of discretization is proposed to that used in the recent \texttt{STRUPHY} code \cite{Holderied_Possanner_Wang_2021, Holderied_2022, Li_et_al_2023} with a finite-element model for the background and a pseudo-particle/PiC model for the correction.

        The fluid background satisfies the full, non-linear, resistive, compressible, Hall MHD equations. \cite{Laakmann_Hu_Farrell_2022} introduces finite-element(-in-space) implicit timesteppers for the incompressible analogue to this system with structure-preserving (SP) properties in the ideal case, alongside parameter-robust preconditioners. We show that these timesteppers can derive from a finite-element-in-time (FET) (and finite-element-in-space) interpretation. The benefits of this reformulation are discussed, including the derivation of timesteppers that are higher order in time, and the quantifiable dissipative SP properties in the non-ideal, resistive case.
        
        We discuss possible options for extending this FET approach to timesteppers for the compressible case.

        The kinetic corrections satisfy linearized Boltzmann equations. Using a Lénard--Bernstein collision operator, these take Fokker--Planck-like forms \cite{Fokker_1914, Planck_1917} wherein pseudo-particles in the numerical model obey the neoclassical transport equations, with particle-independent Brownian drift terms. This offers a rigorous methodology for incorporating collisions into the particle transport model, without coupling the equations of motions for each particle.
        
        Works by Chen, Chacón et al. \cite{Chen_Chacón_Barnes_2011, Chacón_Chen_Barnes_2013, Chen_Chacón_2014, Chen_Chacón_2015} have developed structure-preserving particle pushers for neoclassical transport in the Vlasov equations, derived from Crank--Nicolson integrators. We show these too can can derive from a FET interpretation, similarly offering potential extensions to higher-order-in-time particle pushers. The FET formulation is used also to consider how the stochastic drift terms can be incorporated into the pushers. Stochastic gyrokinetic expansions are also discussed.

        Different options for the numerical implementation of these schemes are considered.

        Due to the efficacy of FET in the development of SP timesteppers for both the fluid and kinetic component, we hope this approach will prove effective in the future for developing SP timesteppers for the full hybrid model. We hope this will give us the opportunity to incorporate previously inaccessible kinetic effects into the highly effective, modern, finite-element MHD models.
    \end{abstract}
    
    
    \newpage
    \tableofcontents
    
    
    \newpage
    \pagenumbering{arabic}
    %\linenumbers\renewcommand\thelinenumber{\color{black!50}\arabic{linenumber}}
            \input{0 - introduction/main.tex}
        \part{Research}
            \input{1 - low-noise PiC models/main.tex}
            \input{2 - kinetic component/main.tex}
            \input{3 - fluid component/main.tex}
            \input{4 - numerical implementation/main.tex}
        \part{Project Overview}
            \input{5 - research plan/main.tex}
            \input{6 - summary/main.tex}
    
    
    %\section{}
    \newpage
    \pagenumbering{gobble}
        \printbibliography


    \newpage
    \pagenumbering{roman}
    \appendix
        \part{Appendices}
            \input{8 - Hilbert complexes/main.tex}
            \input{9 - weak conservation proofs/main.tex}
\end{document}

            \documentclass[12pt, a4paper]{report}

\input{template/main.tex}

\title{\BA{Title in Progress...}}
\author{Boris Andrews}
\affil{Mathematical Institute, University of Oxford}
\date{\today}


\begin{document}
    \pagenumbering{gobble}
    \maketitle
    
    
    \begin{abstract}
        Magnetic confinement reactors---in particular tokamaks---offer one of the most promising options for achieving practical nuclear fusion, with the potential to provide virtually limitless, clean energy. The theoretical and numerical modeling of tokamak plasmas is simultaneously an essential component of effective reactor design, and a great research barrier. Tokamak operational conditions exhibit comparatively low Knudsen numbers. Kinetic effects, including kinetic waves and instabilities, Landau damping, bump-on-tail instabilities and more, are therefore highly influential in tokamak plasma dynamics. Purely fluid models are inherently incapable of capturing these effects, whereas the high dimensionality in purely kinetic models render them practically intractable for most relevant purposes.

        We consider a $\delta\!f$ decomposition model, with a macroscopic fluid background and microscopic kinetic correction, both fully coupled to each other. A similar manner of discretization is proposed to that used in the recent \texttt{STRUPHY} code \cite{Holderied_Possanner_Wang_2021, Holderied_2022, Li_et_al_2023} with a finite-element model for the background and a pseudo-particle/PiC model for the correction.

        The fluid background satisfies the full, non-linear, resistive, compressible, Hall MHD equations. \cite{Laakmann_Hu_Farrell_2022} introduces finite-element(-in-space) implicit timesteppers for the incompressible analogue to this system with structure-preserving (SP) properties in the ideal case, alongside parameter-robust preconditioners. We show that these timesteppers can derive from a finite-element-in-time (FET) (and finite-element-in-space) interpretation. The benefits of this reformulation are discussed, including the derivation of timesteppers that are higher order in time, and the quantifiable dissipative SP properties in the non-ideal, resistive case.
        
        We discuss possible options for extending this FET approach to timesteppers for the compressible case.

        The kinetic corrections satisfy linearized Boltzmann equations. Using a Lénard--Bernstein collision operator, these take Fokker--Planck-like forms \cite{Fokker_1914, Planck_1917} wherein pseudo-particles in the numerical model obey the neoclassical transport equations, with particle-independent Brownian drift terms. This offers a rigorous methodology for incorporating collisions into the particle transport model, without coupling the equations of motions for each particle.
        
        Works by Chen, Chacón et al. \cite{Chen_Chacón_Barnes_2011, Chacón_Chen_Barnes_2013, Chen_Chacón_2014, Chen_Chacón_2015} have developed structure-preserving particle pushers for neoclassical transport in the Vlasov equations, derived from Crank--Nicolson integrators. We show these too can can derive from a FET interpretation, similarly offering potential extensions to higher-order-in-time particle pushers. The FET formulation is used also to consider how the stochastic drift terms can be incorporated into the pushers. Stochastic gyrokinetic expansions are also discussed.

        Different options for the numerical implementation of these schemes are considered.

        Due to the efficacy of FET in the development of SP timesteppers for both the fluid and kinetic component, we hope this approach will prove effective in the future for developing SP timesteppers for the full hybrid model. We hope this will give us the opportunity to incorporate previously inaccessible kinetic effects into the highly effective, modern, finite-element MHD models.
    \end{abstract}
    
    
    \newpage
    \tableofcontents
    
    
    \newpage
    \pagenumbering{arabic}
    %\linenumbers\renewcommand\thelinenumber{\color{black!50}\arabic{linenumber}}
            \input{0 - introduction/main.tex}
        \part{Research}
            \input{1 - low-noise PiC models/main.tex}
            \input{2 - kinetic component/main.tex}
            \input{3 - fluid component/main.tex}
            \input{4 - numerical implementation/main.tex}
        \part{Project Overview}
            \input{5 - research plan/main.tex}
            \input{6 - summary/main.tex}
    
    
    %\section{}
    \newpage
    \pagenumbering{gobble}
        \printbibliography


    \newpage
    \pagenumbering{roman}
    \appendix
        \part{Appendices}
            \input{8 - Hilbert complexes/main.tex}
            \input{9 - weak conservation proofs/main.tex}
\end{document}

            \documentclass[12pt, a4paper]{report}

\input{template/main.tex}

\title{\BA{Title in Progress...}}
\author{Boris Andrews}
\affil{Mathematical Institute, University of Oxford}
\date{\today}


\begin{document}
    \pagenumbering{gobble}
    \maketitle
    
    
    \begin{abstract}
        Magnetic confinement reactors---in particular tokamaks---offer one of the most promising options for achieving practical nuclear fusion, with the potential to provide virtually limitless, clean energy. The theoretical and numerical modeling of tokamak plasmas is simultaneously an essential component of effective reactor design, and a great research barrier. Tokamak operational conditions exhibit comparatively low Knudsen numbers. Kinetic effects, including kinetic waves and instabilities, Landau damping, bump-on-tail instabilities and more, are therefore highly influential in tokamak plasma dynamics. Purely fluid models are inherently incapable of capturing these effects, whereas the high dimensionality in purely kinetic models render them practically intractable for most relevant purposes.

        We consider a $\delta\!f$ decomposition model, with a macroscopic fluid background and microscopic kinetic correction, both fully coupled to each other. A similar manner of discretization is proposed to that used in the recent \texttt{STRUPHY} code \cite{Holderied_Possanner_Wang_2021, Holderied_2022, Li_et_al_2023} with a finite-element model for the background and a pseudo-particle/PiC model for the correction.

        The fluid background satisfies the full, non-linear, resistive, compressible, Hall MHD equations. \cite{Laakmann_Hu_Farrell_2022} introduces finite-element(-in-space) implicit timesteppers for the incompressible analogue to this system with structure-preserving (SP) properties in the ideal case, alongside parameter-robust preconditioners. We show that these timesteppers can derive from a finite-element-in-time (FET) (and finite-element-in-space) interpretation. The benefits of this reformulation are discussed, including the derivation of timesteppers that are higher order in time, and the quantifiable dissipative SP properties in the non-ideal, resistive case.
        
        We discuss possible options for extending this FET approach to timesteppers for the compressible case.

        The kinetic corrections satisfy linearized Boltzmann equations. Using a Lénard--Bernstein collision operator, these take Fokker--Planck-like forms \cite{Fokker_1914, Planck_1917} wherein pseudo-particles in the numerical model obey the neoclassical transport equations, with particle-independent Brownian drift terms. This offers a rigorous methodology for incorporating collisions into the particle transport model, without coupling the equations of motions for each particle.
        
        Works by Chen, Chacón et al. \cite{Chen_Chacón_Barnes_2011, Chacón_Chen_Barnes_2013, Chen_Chacón_2014, Chen_Chacón_2015} have developed structure-preserving particle pushers for neoclassical transport in the Vlasov equations, derived from Crank--Nicolson integrators. We show these too can can derive from a FET interpretation, similarly offering potential extensions to higher-order-in-time particle pushers. The FET formulation is used also to consider how the stochastic drift terms can be incorporated into the pushers. Stochastic gyrokinetic expansions are also discussed.

        Different options for the numerical implementation of these schemes are considered.

        Due to the efficacy of FET in the development of SP timesteppers for both the fluid and kinetic component, we hope this approach will prove effective in the future for developing SP timesteppers for the full hybrid model. We hope this will give us the opportunity to incorporate previously inaccessible kinetic effects into the highly effective, modern, finite-element MHD models.
    \end{abstract}
    
    
    \newpage
    \tableofcontents
    
    
    \newpage
    \pagenumbering{arabic}
    %\linenumbers\renewcommand\thelinenumber{\color{black!50}\arabic{linenumber}}
            \input{0 - introduction/main.tex}
        \part{Research}
            \input{1 - low-noise PiC models/main.tex}
            \input{2 - kinetic component/main.tex}
            \input{3 - fluid component/main.tex}
            \input{4 - numerical implementation/main.tex}
        \part{Project Overview}
            \input{5 - research plan/main.tex}
            \input{6 - summary/main.tex}
    
    
    %\section{}
    \newpage
    \pagenumbering{gobble}
        \printbibliography


    \newpage
    \pagenumbering{roman}
    \appendix
        \part{Appendices}
            \input{8 - Hilbert complexes/main.tex}
            \input{9 - weak conservation proofs/main.tex}
\end{document}

        \part{Project Overview}
            \documentclass[12pt, a4paper]{report}

\input{template/main.tex}

\title{\BA{Title in Progress...}}
\author{Boris Andrews}
\affil{Mathematical Institute, University of Oxford}
\date{\today}


\begin{document}
    \pagenumbering{gobble}
    \maketitle
    
    
    \begin{abstract}
        Magnetic confinement reactors---in particular tokamaks---offer one of the most promising options for achieving practical nuclear fusion, with the potential to provide virtually limitless, clean energy. The theoretical and numerical modeling of tokamak plasmas is simultaneously an essential component of effective reactor design, and a great research barrier. Tokamak operational conditions exhibit comparatively low Knudsen numbers. Kinetic effects, including kinetic waves and instabilities, Landau damping, bump-on-tail instabilities and more, are therefore highly influential in tokamak plasma dynamics. Purely fluid models are inherently incapable of capturing these effects, whereas the high dimensionality in purely kinetic models render them practically intractable for most relevant purposes.

        We consider a $\delta\!f$ decomposition model, with a macroscopic fluid background and microscopic kinetic correction, both fully coupled to each other. A similar manner of discretization is proposed to that used in the recent \texttt{STRUPHY} code \cite{Holderied_Possanner_Wang_2021, Holderied_2022, Li_et_al_2023} with a finite-element model for the background and a pseudo-particle/PiC model for the correction.

        The fluid background satisfies the full, non-linear, resistive, compressible, Hall MHD equations. \cite{Laakmann_Hu_Farrell_2022} introduces finite-element(-in-space) implicit timesteppers for the incompressible analogue to this system with structure-preserving (SP) properties in the ideal case, alongside parameter-robust preconditioners. We show that these timesteppers can derive from a finite-element-in-time (FET) (and finite-element-in-space) interpretation. The benefits of this reformulation are discussed, including the derivation of timesteppers that are higher order in time, and the quantifiable dissipative SP properties in the non-ideal, resistive case.
        
        We discuss possible options for extending this FET approach to timesteppers for the compressible case.

        The kinetic corrections satisfy linearized Boltzmann equations. Using a Lénard--Bernstein collision operator, these take Fokker--Planck-like forms \cite{Fokker_1914, Planck_1917} wherein pseudo-particles in the numerical model obey the neoclassical transport equations, with particle-independent Brownian drift terms. This offers a rigorous methodology for incorporating collisions into the particle transport model, without coupling the equations of motions for each particle.
        
        Works by Chen, Chacón et al. \cite{Chen_Chacón_Barnes_2011, Chacón_Chen_Barnes_2013, Chen_Chacón_2014, Chen_Chacón_2015} have developed structure-preserving particle pushers for neoclassical transport in the Vlasov equations, derived from Crank--Nicolson integrators. We show these too can can derive from a FET interpretation, similarly offering potential extensions to higher-order-in-time particle pushers. The FET formulation is used also to consider how the stochastic drift terms can be incorporated into the pushers. Stochastic gyrokinetic expansions are also discussed.

        Different options for the numerical implementation of these schemes are considered.

        Due to the efficacy of FET in the development of SP timesteppers for both the fluid and kinetic component, we hope this approach will prove effective in the future for developing SP timesteppers for the full hybrid model. We hope this will give us the opportunity to incorporate previously inaccessible kinetic effects into the highly effective, modern, finite-element MHD models.
    \end{abstract}
    
    
    \newpage
    \tableofcontents
    
    
    \newpage
    \pagenumbering{arabic}
    %\linenumbers\renewcommand\thelinenumber{\color{black!50}\arabic{linenumber}}
            \input{0 - introduction/main.tex}
        \part{Research}
            \input{1 - low-noise PiC models/main.tex}
            \input{2 - kinetic component/main.tex}
            \input{3 - fluid component/main.tex}
            \input{4 - numerical implementation/main.tex}
        \part{Project Overview}
            \input{5 - research plan/main.tex}
            \input{6 - summary/main.tex}
    
    
    %\section{}
    \newpage
    \pagenumbering{gobble}
        \printbibliography


    \newpage
    \pagenumbering{roman}
    \appendix
        \part{Appendices}
            \input{8 - Hilbert complexes/main.tex}
            \input{9 - weak conservation proofs/main.tex}
\end{document}

            \documentclass[12pt, a4paper]{report}

\input{template/main.tex}

\title{\BA{Title in Progress...}}
\author{Boris Andrews}
\affil{Mathematical Institute, University of Oxford}
\date{\today}


\begin{document}
    \pagenumbering{gobble}
    \maketitle
    
    
    \begin{abstract}
        Magnetic confinement reactors---in particular tokamaks---offer one of the most promising options for achieving practical nuclear fusion, with the potential to provide virtually limitless, clean energy. The theoretical and numerical modeling of tokamak plasmas is simultaneously an essential component of effective reactor design, and a great research barrier. Tokamak operational conditions exhibit comparatively low Knudsen numbers. Kinetic effects, including kinetic waves and instabilities, Landau damping, bump-on-tail instabilities and more, are therefore highly influential in tokamak plasma dynamics. Purely fluid models are inherently incapable of capturing these effects, whereas the high dimensionality in purely kinetic models render them practically intractable for most relevant purposes.

        We consider a $\delta\!f$ decomposition model, with a macroscopic fluid background and microscopic kinetic correction, both fully coupled to each other. A similar manner of discretization is proposed to that used in the recent \texttt{STRUPHY} code \cite{Holderied_Possanner_Wang_2021, Holderied_2022, Li_et_al_2023} with a finite-element model for the background and a pseudo-particle/PiC model for the correction.

        The fluid background satisfies the full, non-linear, resistive, compressible, Hall MHD equations. \cite{Laakmann_Hu_Farrell_2022} introduces finite-element(-in-space) implicit timesteppers for the incompressible analogue to this system with structure-preserving (SP) properties in the ideal case, alongside parameter-robust preconditioners. We show that these timesteppers can derive from a finite-element-in-time (FET) (and finite-element-in-space) interpretation. The benefits of this reformulation are discussed, including the derivation of timesteppers that are higher order in time, and the quantifiable dissipative SP properties in the non-ideal, resistive case.
        
        We discuss possible options for extending this FET approach to timesteppers for the compressible case.

        The kinetic corrections satisfy linearized Boltzmann equations. Using a Lénard--Bernstein collision operator, these take Fokker--Planck-like forms \cite{Fokker_1914, Planck_1917} wherein pseudo-particles in the numerical model obey the neoclassical transport equations, with particle-independent Brownian drift terms. This offers a rigorous methodology for incorporating collisions into the particle transport model, without coupling the equations of motions for each particle.
        
        Works by Chen, Chacón et al. \cite{Chen_Chacón_Barnes_2011, Chacón_Chen_Barnes_2013, Chen_Chacón_2014, Chen_Chacón_2015} have developed structure-preserving particle pushers for neoclassical transport in the Vlasov equations, derived from Crank--Nicolson integrators. We show these too can can derive from a FET interpretation, similarly offering potential extensions to higher-order-in-time particle pushers. The FET formulation is used also to consider how the stochastic drift terms can be incorporated into the pushers. Stochastic gyrokinetic expansions are also discussed.

        Different options for the numerical implementation of these schemes are considered.

        Due to the efficacy of FET in the development of SP timesteppers for both the fluid and kinetic component, we hope this approach will prove effective in the future for developing SP timesteppers for the full hybrid model. We hope this will give us the opportunity to incorporate previously inaccessible kinetic effects into the highly effective, modern, finite-element MHD models.
    \end{abstract}
    
    
    \newpage
    \tableofcontents
    
    
    \newpage
    \pagenumbering{arabic}
    %\linenumbers\renewcommand\thelinenumber{\color{black!50}\arabic{linenumber}}
            \input{0 - introduction/main.tex}
        \part{Research}
            \input{1 - low-noise PiC models/main.tex}
            \input{2 - kinetic component/main.tex}
            \input{3 - fluid component/main.tex}
            \input{4 - numerical implementation/main.tex}
        \part{Project Overview}
            \input{5 - research plan/main.tex}
            \input{6 - summary/main.tex}
    
    
    %\section{}
    \newpage
    \pagenumbering{gobble}
        \printbibliography


    \newpage
    \pagenumbering{roman}
    \appendix
        \part{Appendices}
            \input{8 - Hilbert complexes/main.tex}
            \input{9 - weak conservation proofs/main.tex}
\end{document}

    
    
    %\section{}
    \newpage
    \pagenumbering{gobble}
        \printbibliography


    \newpage
    \pagenumbering{roman}
    \appendix
        \part{Appendices}
            \documentclass[12pt, a4paper]{report}

\input{template/main.tex}

\title{\BA{Title in Progress...}}
\author{Boris Andrews}
\affil{Mathematical Institute, University of Oxford}
\date{\today}


\begin{document}
    \pagenumbering{gobble}
    \maketitle
    
    
    \begin{abstract}
        Magnetic confinement reactors---in particular tokamaks---offer one of the most promising options for achieving practical nuclear fusion, with the potential to provide virtually limitless, clean energy. The theoretical and numerical modeling of tokamak plasmas is simultaneously an essential component of effective reactor design, and a great research barrier. Tokamak operational conditions exhibit comparatively low Knudsen numbers. Kinetic effects, including kinetic waves and instabilities, Landau damping, bump-on-tail instabilities and more, are therefore highly influential in tokamak plasma dynamics. Purely fluid models are inherently incapable of capturing these effects, whereas the high dimensionality in purely kinetic models render them practically intractable for most relevant purposes.

        We consider a $\delta\!f$ decomposition model, with a macroscopic fluid background and microscopic kinetic correction, both fully coupled to each other. A similar manner of discretization is proposed to that used in the recent \texttt{STRUPHY} code \cite{Holderied_Possanner_Wang_2021, Holderied_2022, Li_et_al_2023} with a finite-element model for the background and a pseudo-particle/PiC model for the correction.

        The fluid background satisfies the full, non-linear, resistive, compressible, Hall MHD equations. \cite{Laakmann_Hu_Farrell_2022} introduces finite-element(-in-space) implicit timesteppers for the incompressible analogue to this system with structure-preserving (SP) properties in the ideal case, alongside parameter-robust preconditioners. We show that these timesteppers can derive from a finite-element-in-time (FET) (and finite-element-in-space) interpretation. The benefits of this reformulation are discussed, including the derivation of timesteppers that are higher order in time, and the quantifiable dissipative SP properties in the non-ideal, resistive case.
        
        We discuss possible options for extending this FET approach to timesteppers for the compressible case.

        The kinetic corrections satisfy linearized Boltzmann equations. Using a Lénard--Bernstein collision operator, these take Fokker--Planck-like forms \cite{Fokker_1914, Planck_1917} wherein pseudo-particles in the numerical model obey the neoclassical transport equations, with particle-independent Brownian drift terms. This offers a rigorous methodology for incorporating collisions into the particle transport model, without coupling the equations of motions for each particle.
        
        Works by Chen, Chacón et al. \cite{Chen_Chacón_Barnes_2011, Chacón_Chen_Barnes_2013, Chen_Chacón_2014, Chen_Chacón_2015} have developed structure-preserving particle pushers for neoclassical transport in the Vlasov equations, derived from Crank--Nicolson integrators. We show these too can can derive from a FET interpretation, similarly offering potential extensions to higher-order-in-time particle pushers. The FET formulation is used also to consider how the stochastic drift terms can be incorporated into the pushers. Stochastic gyrokinetic expansions are also discussed.

        Different options for the numerical implementation of these schemes are considered.

        Due to the efficacy of FET in the development of SP timesteppers for both the fluid and kinetic component, we hope this approach will prove effective in the future for developing SP timesteppers for the full hybrid model. We hope this will give us the opportunity to incorporate previously inaccessible kinetic effects into the highly effective, modern, finite-element MHD models.
    \end{abstract}
    
    
    \newpage
    \tableofcontents
    
    
    \newpage
    \pagenumbering{arabic}
    %\linenumbers\renewcommand\thelinenumber{\color{black!50}\arabic{linenumber}}
            \input{0 - introduction/main.tex}
        \part{Research}
            \input{1 - low-noise PiC models/main.tex}
            \input{2 - kinetic component/main.tex}
            \input{3 - fluid component/main.tex}
            \input{4 - numerical implementation/main.tex}
        \part{Project Overview}
            \input{5 - research plan/main.tex}
            \input{6 - summary/main.tex}
    
    
    %\section{}
    \newpage
    \pagenumbering{gobble}
        \printbibliography


    \newpage
    \pagenumbering{roman}
    \appendix
        \part{Appendices}
            \input{8 - Hilbert complexes/main.tex}
            \input{9 - weak conservation proofs/main.tex}
\end{document}

            \documentclass[12pt, a4paper]{report}

\input{template/main.tex}

\title{\BA{Title in Progress...}}
\author{Boris Andrews}
\affil{Mathematical Institute, University of Oxford}
\date{\today}


\begin{document}
    \pagenumbering{gobble}
    \maketitle
    
    
    \begin{abstract}
        Magnetic confinement reactors---in particular tokamaks---offer one of the most promising options for achieving practical nuclear fusion, with the potential to provide virtually limitless, clean energy. The theoretical and numerical modeling of tokamak plasmas is simultaneously an essential component of effective reactor design, and a great research barrier. Tokamak operational conditions exhibit comparatively low Knudsen numbers. Kinetic effects, including kinetic waves and instabilities, Landau damping, bump-on-tail instabilities and more, are therefore highly influential in tokamak plasma dynamics. Purely fluid models are inherently incapable of capturing these effects, whereas the high dimensionality in purely kinetic models render them practically intractable for most relevant purposes.

        We consider a $\delta\!f$ decomposition model, with a macroscopic fluid background and microscopic kinetic correction, both fully coupled to each other. A similar manner of discretization is proposed to that used in the recent \texttt{STRUPHY} code \cite{Holderied_Possanner_Wang_2021, Holderied_2022, Li_et_al_2023} with a finite-element model for the background and a pseudo-particle/PiC model for the correction.

        The fluid background satisfies the full, non-linear, resistive, compressible, Hall MHD equations. \cite{Laakmann_Hu_Farrell_2022} introduces finite-element(-in-space) implicit timesteppers for the incompressible analogue to this system with structure-preserving (SP) properties in the ideal case, alongside parameter-robust preconditioners. We show that these timesteppers can derive from a finite-element-in-time (FET) (and finite-element-in-space) interpretation. The benefits of this reformulation are discussed, including the derivation of timesteppers that are higher order in time, and the quantifiable dissipative SP properties in the non-ideal, resistive case.
        
        We discuss possible options for extending this FET approach to timesteppers for the compressible case.

        The kinetic corrections satisfy linearized Boltzmann equations. Using a Lénard--Bernstein collision operator, these take Fokker--Planck-like forms \cite{Fokker_1914, Planck_1917} wherein pseudo-particles in the numerical model obey the neoclassical transport equations, with particle-independent Brownian drift terms. This offers a rigorous methodology for incorporating collisions into the particle transport model, without coupling the equations of motions for each particle.
        
        Works by Chen, Chacón et al. \cite{Chen_Chacón_Barnes_2011, Chacón_Chen_Barnes_2013, Chen_Chacón_2014, Chen_Chacón_2015} have developed structure-preserving particle pushers for neoclassical transport in the Vlasov equations, derived from Crank--Nicolson integrators. We show these too can can derive from a FET interpretation, similarly offering potential extensions to higher-order-in-time particle pushers. The FET formulation is used also to consider how the stochastic drift terms can be incorporated into the pushers. Stochastic gyrokinetic expansions are also discussed.

        Different options for the numerical implementation of these schemes are considered.

        Due to the efficacy of FET in the development of SP timesteppers for both the fluid and kinetic component, we hope this approach will prove effective in the future for developing SP timesteppers for the full hybrid model. We hope this will give us the opportunity to incorporate previously inaccessible kinetic effects into the highly effective, modern, finite-element MHD models.
    \end{abstract}
    
    
    \newpage
    \tableofcontents
    
    
    \newpage
    \pagenumbering{arabic}
    %\linenumbers\renewcommand\thelinenumber{\color{black!50}\arabic{linenumber}}
            \input{0 - introduction/main.tex}
        \part{Research}
            \input{1 - low-noise PiC models/main.tex}
            \input{2 - kinetic component/main.tex}
            \input{3 - fluid component/main.tex}
            \input{4 - numerical implementation/main.tex}
        \part{Project Overview}
            \input{5 - research plan/main.tex}
            \input{6 - summary/main.tex}
    
    
    %\section{}
    \newpage
    \pagenumbering{gobble}
        \printbibliography


    \newpage
    \pagenumbering{roman}
    \appendix
        \part{Appendices}
            \input{8 - Hilbert complexes/main.tex}
            \input{9 - weak conservation proofs/main.tex}
\end{document}

\end{document}

        \part{Project Overview}
            \documentclass[12pt, a4paper]{report}

\documentclass[12pt, a4paper]{report}

\input{template/main.tex}

\title{\BA{Title in Progress...}}
\author{Boris Andrews}
\affil{Mathematical Institute, University of Oxford}
\date{\today}


\begin{document}
    \pagenumbering{gobble}
    \maketitle
    
    
    \begin{abstract}
        Magnetic confinement reactors---in particular tokamaks---offer one of the most promising options for achieving practical nuclear fusion, with the potential to provide virtually limitless, clean energy. The theoretical and numerical modeling of tokamak plasmas is simultaneously an essential component of effective reactor design, and a great research barrier. Tokamak operational conditions exhibit comparatively low Knudsen numbers. Kinetic effects, including kinetic waves and instabilities, Landau damping, bump-on-tail instabilities and more, are therefore highly influential in tokamak plasma dynamics. Purely fluid models are inherently incapable of capturing these effects, whereas the high dimensionality in purely kinetic models render them practically intractable for most relevant purposes.

        We consider a $\delta\!f$ decomposition model, with a macroscopic fluid background and microscopic kinetic correction, both fully coupled to each other. A similar manner of discretization is proposed to that used in the recent \texttt{STRUPHY} code \cite{Holderied_Possanner_Wang_2021, Holderied_2022, Li_et_al_2023} with a finite-element model for the background and a pseudo-particle/PiC model for the correction.

        The fluid background satisfies the full, non-linear, resistive, compressible, Hall MHD equations. \cite{Laakmann_Hu_Farrell_2022} introduces finite-element(-in-space) implicit timesteppers for the incompressible analogue to this system with structure-preserving (SP) properties in the ideal case, alongside parameter-robust preconditioners. We show that these timesteppers can derive from a finite-element-in-time (FET) (and finite-element-in-space) interpretation. The benefits of this reformulation are discussed, including the derivation of timesteppers that are higher order in time, and the quantifiable dissipative SP properties in the non-ideal, resistive case.
        
        We discuss possible options for extending this FET approach to timesteppers for the compressible case.

        The kinetic corrections satisfy linearized Boltzmann equations. Using a Lénard--Bernstein collision operator, these take Fokker--Planck-like forms \cite{Fokker_1914, Planck_1917} wherein pseudo-particles in the numerical model obey the neoclassical transport equations, with particle-independent Brownian drift terms. This offers a rigorous methodology for incorporating collisions into the particle transport model, without coupling the equations of motions for each particle.
        
        Works by Chen, Chacón et al. \cite{Chen_Chacón_Barnes_2011, Chacón_Chen_Barnes_2013, Chen_Chacón_2014, Chen_Chacón_2015} have developed structure-preserving particle pushers for neoclassical transport in the Vlasov equations, derived from Crank--Nicolson integrators. We show these too can can derive from a FET interpretation, similarly offering potential extensions to higher-order-in-time particle pushers. The FET formulation is used also to consider how the stochastic drift terms can be incorporated into the pushers. Stochastic gyrokinetic expansions are also discussed.

        Different options for the numerical implementation of these schemes are considered.

        Due to the efficacy of FET in the development of SP timesteppers for both the fluid and kinetic component, we hope this approach will prove effective in the future for developing SP timesteppers for the full hybrid model. We hope this will give us the opportunity to incorporate previously inaccessible kinetic effects into the highly effective, modern, finite-element MHD models.
    \end{abstract}
    
    
    \newpage
    \tableofcontents
    
    
    \newpage
    \pagenumbering{arabic}
    %\linenumbers\renewcommand\thelinenumber{\color{black!50}\arabic{linenumber}}
            \input{0 - introduction/main.tex}
        \part{Research}
            \input{1 - low-noise PiC models/main.tex}
            \input{2 - kinetic component/main.tex}
            \input{3 - fluid component/main.tex}
            \input{4 - numerical implementation/main.tex}
        \part{Project Overview}
            \input{5 - research plan/main.tex}
            \input{6 - summary/main.tex}
    
    
    %\section{}
    \newpage
    \pagenumbering{gobble}
        \printbibliography


    \newpage
    \pagenumbering{roman}
    \appendix
        \part{Appendices}
            \input{8 - Hilbert complexes/main.tex}
            \input{9 - weak conservation proofs/main.tex}
\end{document}


\title{\BA{Title in Progress...}}
\author{Boris Andrews}
\affil{Mathematical Institute, University of Oxford}
\date{\today}


\begin{document}
    \pagenumbering{gobble}
    \maketitle
    
    
    \begin{abstract}
        Magnetic confinement reactors---in particular tokamaks---offer one of the most promising options for achieving practical nuclear fusion, with the potential to provide virtually limitless, clean energy. The theoretical and numerical modeling of tokamak plasmas is simultaneously an essential component of effective reactor design, and a great research barrier. Tokamak operational conditions exhibit comparatively low Knudsen numbers. Kinetic effects, including kinetic waves and instabilities, Landau damping, bump-on-tail instabilities and more, are therefore highly influential in tokamak plasma dynamics. Purely fluid models are inherently incapable of capturing these effects, whereas the high dimensionality in purely kinetic models render them practically intractable for most relevant purposes.

        We consider a $\delta\!f$ decomposition model, with a macroscopic fluid background and microscopic kinetic correction, both fully coupled to each other. A similar manner of discretization is proposed to that used in the recent \texttt{STRUPHY} code \cite{Holderied_Possanner_Wang_2021, Holderied_2022, Li_et_al_2023} with a finite-element model for the background and a pseudo-particle/PiC model for the correction.

        The fluid background satisfies the full, non-linear, resistive, compressible, Hall MHD equations. \cite{Laakmann_Hu_Farrell_2022} introduces finite-element(-in-space) implicit timesteppers for the incompressible analogue to this system with structure-preserving (SP) properties in the ideal case, alongside parameter-robust preconditioners. We show that these timesteppers can derive from a finite-element-in-time (FET) (and finite-element-in-space) interpretation. The benefits of this reformulation are discussed, including the derivation of timesteppers that are higher order in time, and the quantifiable dissipative SP properties in the non-ideal, resistive case.
        
        We discuss possible options for extending this FET approach to timesteppers for the compressible case.

        The kinetic corrections satisfy linearized Boltzmann equations. Using a Lénard--Bernstein collision operator, these take Fokker--Planck-like forms \cite{Fokker_1914, Planck_1917} wherein pseudo-particles in the numerical model obey the neoclassical transport equations, with particle-independent Brownian drift terms. This offers a rigorous methodology for incorporating collisions into the particle transport model, without coupling the equations of motions for each particle.
        
        Works by Chen, Chacón et al. \cite{Chen_Chacón_Barnes_2011, Chacón_Chen_Barnes_2013, Chen_Chacón_2014, Chen_Chacón_2015} have developed structure-preserving particle pushers for neoclassical transport in the Vlasov equations, derived from Crank--Nicolson integrators. We show these too can can derive from a FET interpretation, similarly offering potential extensions to higher-order-in-time particle pushers. The FET formulation is used also to consider how the stochastic drift terms can be incorporated into the pushers. Stochastic gyrokinetic expansions are also discussed.

        Different options for the numerical implementation of these schemes are considered.

        Due to the efficacy of FET in the development of SP timesteppers for both the fluid and kinetic component, we hope this approach will prove effective in the future for developing SP timesteppers for the full hybrid model. We hope this will give us the opportunity to incorporate previously inaccessible kinetic effects into the highly effective, modern, finite-element MHD models.
    \end{abstract}
    
    
    \newpage
    \tableofcontents
    
    
    \newpage
    \pagenumbering{arabic}
    %\linenumbers\renewcommand\thelinenumber{\color{black!50}\arabic{linenumber}}
            \documentclass[12pt, a4paper]{report}

\input{template/main.tex}

\title{\BA{Title in Progress...}}
\author{Boris Andrews}
\affil{Mathematical Institute, University of Oxford}
\date{\today}


\begin{document}
    \pagenumbering{gobble}
    \maketitle
    
    
    \begin{abstract}
        Magnetic confinement reactors---in particular tokamaks---offer one of the most promising options for achieving practical nuclear fusion, with the potential to provide virtually limitless, clean energy. The theoretical and numerical modeling of tokamak plasmas is simultaneously an essential component of effective reactor design, and a great research barrier. Tokamak operational conditions exhibit comparatively low Knudsen numbers. Kinetic effects, including kinetic waves and instabilities, Landau damping, bump-on-tail instabilities and more, are therefore highly influential in tokamak plasma dynamics. Purely fluid models are inherently incapable of capturing these effects, whereas the high dimensionality in purely kinetic models render them practically intractable for most relevant purposes.

        We consider a $\delta\!f$ decomposition model, with a macroscopic fluid background and microscopic kinetic correction, both fully coupled to each other. A similar manner of discretization is proposed to that used in the recent \texttt{STRUPHY} code \cite{Holderied_Possanner_Wang_2021, Holderied_2022, Li_et_al_2023} with a finite-element model for the background and a pseudo-particle/PiC model for the correction.

        The fluid background satisfies the full, non-linear, resistive, compressible, Hall MHD equations. \cite{Laakmann_Hu_Farrell_2022} introduces finite-element(-in-space) implicit timesteppers for the incompressible analogue to this system with structure-preserving (SP) properties in the ideal case, alongside parameter-robust preconditioners. We show that these timesteppers can derive from a finite-element-in-time (FET) (and finite-element-in-space) interpretation. The benefits of this reformulation are discussed, including the derivation of timesteppers that are higher order in time, and the quantifiable dissipative SP properties in the non-ideal, resistive case.
        
        We discuss possible options for extending this FET approach to timesteppers for the compressible case.

        The kinetic corrections satisfy linearized Boltzmann equations. Using a Lénard--Bernstein collision operator, these take Fokker--Planck-like forms \cite{Fokker_1914, Planck_1917} wherein pseudo-particles in the numerical model obey the neoclassical transport equations, with particle-independent Brownian drift terms. This offers a rigorous methodology for incorporating collisions into the particle transport model, without coupling the equations of motions for each particle.
        
        Works by Chen, Chacón et al. \cite{Chen_Chacón_Barnes_2011, Chacón_Chen_Barnes_2013, Chen_Chacón_2014, Chen_Chacón_2015} have developed structure-preserving particle pushers for neoclassical transport in the Vlasov equations, derived from Crank--Nicolson integrators. We show these too can can derive from a FET interpretation, similarly offering potential extensions to higher-order-in-time particle pushers. The FET formulation is used also to consider how the stochastic drift terms can be incorporated into the pushers. Stochastic gyrokinetic expansions are also discussed.

        Different options for the numerical implementation of these schemes are considered.

        Due to the efficacy of FET in the development of SP timesteppers for both the fluid and kinetic component, we hope this approach will prove effective in the future for developing SP timesteppers for the full hybrid model. We hope this will give us the opportunity to incorporate previously inaccessible kinetic effects into the highly effective, modern, finite-element MHD models.
    \end{abstract}
    
    
    \newpage
    \tableofcontents
    
    
    \newpage
    \pagenumbering{arabic}
    %\linenumbers\renewcommand\thelinenumber{\color{black!50}\arabic{linenumber}}
            \input{0 - introduction/main.tex}
        \part{Research}
            \input{1 - low-noise PiC models/main.tex}
            \input{2 - kinetic component/main.tex}
            \input{3 - fluid component/main.tex}
            \input{4 - numerical implementation/main.tex}
        \part{Project Overview}
            \input{5 - research plan/main.tex}
            \input{6 - summary/main.tex}
    
    
    %\section{}
    \newpage
    \pagenumbering{gobble}
        \printbibliography


    \newpage
    \pagenumbering{roman}
    \appendix
        \part{Appendices}
            \input{8 - Hilbert complexes/main.tex}
            \input{9 - weak conservation proofs/main.tex}
\end{document}

        \part{Research}
            \documentclass[12pt, a4paper]{report}

\input{template/main.tex}

\title{\BA{Title in Progress...}}
\author{Boris Andrews}
\affil{Mathematical Institute, University of Oxford}
\date{\today}


\begin{document}
    \pagenumbering{gobble}
    \maketitle
    
    
    \begin{abstract}
        Magnetic confinement reactors---in particular tokamaks---offer one of the most promising options for achieving practical nuclear fusion, with the potential to provide virtually limitless, clean energy. The theoretical and numerical modeling of tokamak plasmas is simultaneously an essential component of effective reactor design, and a great research barrier. Tokamak operational conditions exhibit comparatively low Knudsen numbers. Kinetic effects, including kinetic waves and instabilities, Landau damping, bump-on-tail instabilities and more, are therefore highly influential in tokamak plasma dynamics. Purely fluid models are inherently incapable of capturing these effects, whereas the high dimensionality in purely kinetic models render them practically intractable for most relevant purposes.

        We consider a $\delta\!f$ decomposition model, with a macroscopic fluid background and microscopic kinetic correction, both fully coupled to each other. A similar manner of discretization is proposed to that used in the recent \texttt{STRUPHY} code \cite{Holderied_Possanner_Wang_2021, Holderied_2022, Li_et_al_2023} with a finite-element model for the background and a pseudo-particle/PiC model for the correction.

        The fluid background satisfies the full, non-linear, resistive, compressible, Hall MHD equations. \cite{Laakmann_Hu_Farrell_2022} introduces finite-element(-in-space) implicit timesteppers for the incompressible analogue to this system with structure-preserving (SP) properties in the ideal case, alongside parameter-robust preconditioners. We show that these timesteppers can derive from a finite-element-in-time (FET) (and finite-element-in-space) interpretation. The benefits of this reformulation are discussed, including the derivation of timesteppers that are higher order in time, and the quantifiable dissipative SP properties in the non-ideal, resistive case.
        
        We discuss possible options for extending this FET approach to timesteppers for the compressible case.

        The kinetic corrections satisfy linearized Boltzmann equations. Using a Lénard--Bernstein collision operator, these take Fokker--Planck-like forms \cite{Fokker_1914, Planck_1917} wherein pseudo-particles in the numerical model obey the neoclassical transport equations, with particle-independent Brownian drift terms. This offers a rigorous methodology for incorporating collisions into the particle transport model, without coupling the equations of motions for each particle.
        
        Works by Chen, Chacón et al. \cite{Chen_Chacón_Barnes_2011, Chacón_Chen_Barnes_2013, Chen_Chacón_2014, Chen_Chacón_2015} have developed structure-preserving particle pushers for neoclassical transport in the Vlasov equations, derived from Crank--Nicolson integrators. We show these too can can derive from a FET interpretation, similarly offering potential extensions to higher-order-in-time particle pushers. The FET formulation is used also to consider how the stochastic drift terms can be incorporated into the pushers. Stochastic gyrokinetic expansions are also discussed.

        Different options for the numerical implementation of these schemes are considered.

        Due to the efficacy of FET in the development of SP timesteppers for both the fluid and kinetic component, we hope this approach will prove effective in the future for developing SP timesteppers for the full hybrid model. We hope this will give us the opportunity to incorporate previously inaccessible kinetic effects into the highly effective, modern, finite-element MHD models.
    \end{abstract}
    
    
    \newpage
    \tableofcontents
    
    
    \newpage
    \pagenumbering{arabic}
    %\linenumbers\renewcommand\thelinenumber{\color{black!50}\arabic{linenumber}}
            \input{0 - introduction/main.tex}
        \part{Research}
            \input{1 - low-noise PiC models/main.tex}
            \input{2 - kinetic component/main.tex}
            \input{3 - fluid component/main.tex}
            \input{4 - numerical implementation/main.tex}
        \part{Project Overview}
            \input{5 - research plan/main.tex}
            \input{6 - summary/main.tex}
    
    
    %\section{}
    \newpage
    \pagenumbering{gobble}
        \printbibliography


    \newpage
    \pagenumbering{roman}
    \appendix
        \part{Appendices}
            \input{8 - Hilbert complexes/main.tex}
            \input{9 - weak conservation proofs/main.tex}
\end{document}

            \documentclass[12pt, a4paper]{report}

\input{template/main.tex}

\title{\BA{Title in Progress...}}
\author{Boris Andrews}
\affil{Mathematical Institute, University of Oxford}
\date{\today}


\begin{document}
    \pagenumbering{gobble}
    \maketitle
    
    
    \begin{abstract}
        Magnetic confinement reactors---in particular tokamaks---offer one of the most promising options for achieving practical nuclear fusion, with the potential to provide virtually limitless, clean energy. The theoretical and numerical modeling of tokamak plasmas is simultaneously an essential component of effective reactor design, and a great research barrier. Tokamak operational conditions exhibit comparatively low Knudsen numbers. Kinetic effects, including kinetic waves and instabilities, Landau damping, bump-on-tail instabilities and more, are therefore highly influential in tokamak plasma dynamics. Purely fluid models are inherently incapable of capturing these effects, whereas the high dimensionality in purely kinetic models render them practically intractable for most relevant purposes.

        We consider a $\delta\!f$ decomposition model, with a macroscopic fluid background and microscopic kinetic correction, both fully coupled to each other. A similar manner of discretization is proposed to that used in the recent \texttt{STRUPHY} code \cite{Holderied_Possanner_Wang_2021, Holderied_2022, Li_et_al_2023} with a finite-element model for the background and a pseudo-particle/PiC model for the correction.

        The fluid background satisfies the full, non-linear, resistive, compressible, Hall MHD equations. \cite{Laakmann_Hu_Farrell_2022} introduces finite-element(-in-space) implicit timesteppers for the incompressible analogue to this system with structure-preserving (SP) properties in the ideal case, alongside parameter-robust preconditioners. We show that these timesteppers can derive from a finite-element-in-time (FET) (and finite-element-in-space) interpretation. The benefits of this reformulation are discussed, including the derivation of timesteppers that are higher order in time, and the quantifiable dissipative SP properties in the non-ideal, resistive case.
        
        We discuss possible options for extending this FET approach to timesteppers for the compressible case.

        The kinetic corrections satisfy linearized Boltzmann equations. Using a Lénard--Bernstein collision operator, these take Fokker--Planck-like forms \cite{Fokker_1914, Planck_1917} wherein pseudo-particles in the numerical model obey the neoclassical transport equations, with particle-independent Brownian drift terms. This offers a rigorous methodology for incorporating collisions into the particle transport model, without coupling the equations of motions for each particle.
        
        Works by Chen, Chacón et al. \cite{Chen_Chacón_Barnes_2011, Chacón_Chen_Barnes_2013, Chen_Chacón_2014, Chen_Chacón_2015} have developed structure-preserving particle pushers for neoclassical transport in the Vlasov equations, derived from Crank--Nicolson integrators. We show these too can can derive from a FET interpretation, similarly offering potential extensions to higher-order-in-time particle pushers. The FET formulation is used also to consider how the stochastic drift terms can be incorporated into the pushers. Stochastic gyrokinetic expansions are also discussed.

        Different options for the numerical implementation of these schemes are considered.

        Due to the efficacy of FET in the development of SP timesteppers for both the fluid and kinetic component, we hope this approach will prove effective in the future for developing SP timesteppers for the full hybrid model. We hope this will give us the opportunity to incorporate previously inaccessible kinetic effects into the highly effective, modern, finite-element MHD models.
    \end{abstract}
    
    
    \newpage
    \tableofcontents
    
    
    \newpage
    \pagenumbering{arabic}
    %\linenumbers\renewcommand\thelinenumber{\color{black!50}\arabic{linenumber}}
            \input{0 - introduction/main.tex}
        \part{Research}
            \input{1 - low-noise PiC models/main.tex}
            \input{2 - kinetic component/main.tex}
            \input{3 - fluid component/main.tex}
            \input{4 - numerical implementation/main.tex}
        \part{Project Overview}
            \input{5 - research plan/main.tex}
            \input{6 - summary/main.tex}
    
    
    %\section{}
    \newpage
    \pagenumbering{gobble}
        \printbibliography


    \newpage
    \pagenumbering{roman}
    \appendix
        \part{Appendices}
            \input{8 - Hilbert complexes/main.tex}
            \input{9 - weak conservation proofs/main.tex}
\end{document}

            \documentclass[12pt, a4paper]{report}

\input{template/main.tex}

\title{\BA{Title in Progress...}}
\author{Boris Andrews}
\affil{Mathematical Institute, University of Oxford}
\date{\today}


\begin{document}
    \pagenumbering{gobble}
    \maketitle
    
    
    \begin{abstract}
        Magnetic confinement reactors---in particular tokamaks---offer one of the most promising options for achieving practical nuclear fusion, with the potential to provide virtually limitless, clean energy. The theoretical and numerical modeling of tokamak plasmas is simultaneously an essential component of effective reactor design, and a great research barrier. Tokamak operational conditions exhibit comparatively low Knudsen numbers. Kinetic effects, including kinetic waves and instabilities, Landau damping, bump-on-tail instabilities and more, are therefore highly influential in tokamak plasma dynamics. Purely fluid models are inherently incapable of capturing these effects, whereas the high dimensionality in purely kinetic models render them practically intractable for most relevant purposes.

        We consider a $\delta\!f$ decomposition model, with a macroscopic fluid background and microscopic kinetic correction, both fully coupled to each other. A similar manner of discretization is proposed to that used in the recent \texttt{STRUPHY} code \cite{Holderied_Possanner_Wang_2021, Holderied_2022, Li_et_al_2023} with a finite-element model for the background and a pseudo-particle/PiC model for the correction.

        The fluid background satisfies the full, non-linear, resistive, compressible, Hall MHD equations. \cite{Laakmann_Hu_Farrell_2022} introduces finite-element(-in-space) implicit timesteppers for the incompressible analogue to this system with structure-preserving (SP) properties in the ideal case, alongside parameter-robust preconditioners. We show that these timesteppers can derive from a finite-element-in-time (FET) (and finite-element-in-space) interpretation. The benefits of this reformulation are discussed, including the derivation of timesteppers that are higher order in time, and the quantifiable dissipative SP properties in the non-ideal, resistive case.
        
        We discuss possible options for extending this FET approach to timesteppers for the compressible case.

        The kinetic corrections satisfy linearized Boltzmann equations. Using a Lénard--Bernstein collision operator, these take Fokker--Planck-like forms \cite{Fokker_1914, Planck_1917} wherein pseudo-particles in the numerical model obey the neoclassical transport equations, with particle-independent Brownian drift terms. This offers a rigorous methodology for incorporating collisions into the particle transport model, without coupling the equations of motions for each particle.
        
        Works by Chen, Chacón et al. \cite{Chen_Chacón_Barnes_2011, Chacón_Chen_Barnes_2013, Chen_Chacón_2014, Chen_Chacón_2015} have developed structure-preserving particle pushers for neoclassical transport in the Vlasov equations, derived from Crank--Nicolson integrators. We show these too can can derive from a FET interpretation, similarly offering potential extensions to higher-order-in-time particle pushers. The FET formulation is used also to consider how the stochastic drift terms can be incorporated into the pushers. Stochastic gyrokinetic expansions are also discussed.

        Different options for the numerical implementation of these schemes are considered.

        Due to the efficacy of FET in the development of SP timesteppers for both the fluid and kinetic component, we hope this approach will prove effective in the future for developing SP timesteppers for the full hybrid model. We hope this will give us the opportunity to incorporate previously inaccessible kinetic effects into the highly effective, modern, finite-element MHD models.
    \end{abstract}
    
    
    \newpage
    \tableofcontents
    
    
    \newpage
    \pagenumbering{arabic}
    %\linenumbers\renewcommand\thelinenumber{\color{black!50}\arabic{linenumber}}
            \input{0 - introduction/main.tex}
        \part{Research}
            \input{1 - low-noise PiC models/main.tex}
            \input{2 - kinetic component/main.tex}
            \input{3 - fluid component/main.tex}
            \input{4 - numerical implementation/main.tex}
        \part{Project Overview}
            \input{5 - research plan/main.tex}
            \input{6 - summary/main.tex}
    
    
    %\section{}
    \newpage
    \pagenumbering{gobble}
        \printbibliography


    \newpage
    \pagenumbering{roman}
    \appendix
        \part{Appendices}
            \input{8 - Hilbert complexes/main.tex}
            \input{9 - weak conservation proofs/main.tex}
\end{document}

            \documentclass[12pt, a4paper]{report}

\input{template/main.tex}

\title{\BA{Title in Progress...}}
\author{Boris Andrews}
\affil{Mathematical Institute, University of Oxford}
\date{\today}


\begin{document}
    \pagenumbering{gobble}
    \maketitle
    
    
    \begin{abstract}
        Magnetic confinement reactors---in particular tokamaks---offer one of the most promising options for achieving practical nuclear fusion, with the potential to provide virtually limitless, clean energy. The theoretical and numerical modeling of tokamak plasmas is simultaneously an essential component of effective reactor design, and a great research barrier. Tokamak operational conditions exhibit comparatively low Knudsen numbers. Kinetic effects, including kinetic waves and instabilities, Landau damping, bump-on-tail instabilities and more, are therefore highly influential in tokamak plasma dynamics. Purely fluid models are inherently incapable of capturing these effects, whereas the high dimensionality in purely kinetic models render them practically intractable for most relevant purposes.

        We consider a $\delta\!f$ decomposition model, with a macroscopic fluid background and microscopic kinetic correction, both fully coupled to each other. A similar manner of discretization is proposed to that used in the recent \texttt{STRUPHY} code \cite{Holderied_Possanner_Wang_2021, Holderied_2022, Li_et_al_2023} with a finite-element model for the background and a pseudo-particle/PiC model for the correction.

        The fluid background satisfies the full, non-linear, resistive, compressible, Hall MHD equations. \cite{Laakmann_Hu_Farrell_2022} introduces finite-element(-in-space) implicit timesteppers for the incompressible analogue to this system with structure-preserving (SP) properties in the ideal case, alongside parameter-robust preconditioners. We show that these timesteppers can derive from a finite-element-in-time (FET) (and finite-element-in-space) interpretation. The benefits of this reformulation are discussed, including the derivation of timesteppers that are higher order in time, and the quantifiable dissipative SP properties in the non-ideal, resistive case.
        
        We discuss possible options for extending this FET approach to timesteppers for the compressible case.

        The kinetic corrections satisfy linearized Boltzmann equations. Using a Lénard--Bernstein collision operator, these take Fokker--Planck-like forms \cite{Fokker_1914, Planck_1917} wherein pseudo-particles in the numerical model obey the neoclassical transport equations, with particle-independent Brownian drift terms. This offers a rigorous methodology for incorporating collisions into the particle transport model, without coupling the equations of motions for each particle.
        
        Works by Chen, Chacón et al. \cite{Chen_Chacón_Barnes_2011, Chacón_Chen_Barnes_2013, Chen_Chacón_2014, Chen_Chacón_2015} have developed structure-preserving particle pushers for neoclassical transport in the Vlasov equations, derived from Crank--Nicolson integrators. We show these too can can derive from a FET interpretation, similarly offering potential extensions to higher-order-in-time particle pushers. The FET formulation is used also to consider how the stochastic drift terms can be incorporated into the pushers. Stochastic gyrokinetic expansions are also discussed.

        Different options for the numerical implementation of these schemes are considered.

        Due to the efficacy of FET in the development of SP timesteppers for both the fluid and kinetic component, we hope this approach will prove effective in the future for developing SP timesteppers for the full hybrid model. We hope this will give us the opportunity to incorporate previously inaccessible kinetic effects into the highly effective, modern, finite-element MHD models.
    \end{abstract}
    
    
    \newpage
    \tableofcontents
    
    
    \newpage
    \pagenumbering{arabic}
    %\linenumbers\renewcommand\thelinenumber{\color{black!50}\arabic{linenumber}}
            \input{0 - introduction/main.tex}
        \part{Research}
            \input{1 - low-noise PiC models/main.tex}
            \input{2 - kinetic component/main.tex}
            \input{3 - fluid component/main.tex}
            \input{4 - numerical implementation/main.tex}
        \part{Project Overview}
            \input{5 - research plan/main.tex}
            \input{6 - summary/main.tex}
    
    
    %\section{}
    \newpage
    \pagenumbering{gobble}
        \printbibliography


    \newpage
    \pagenumbering{roman}
    \appendix
        \part{Appendices}
            \input{8 - Hilbert complexes/main.tex}
            \input{9 - weak conservation proofs/main.tex}
\end{document}

        \part{Project Overview}
            \documentclass[12pt, a4paper]{report}

\input{template/main.tex}

\title{\BA{Title in Progress...}}
\author{Boris Andrews}
\affil{Mathematical Institute, University of Oxford}
\date{\today}


\begin{document}
    \pagenumbering{gobble}
    \maketitle
    
    
    \begin{abstract}
        Magnetic confinement reactors---in particular tokamaks---offer one of the most promising options for achieving practical nuclear fusion, with the potential to provide virtually limitless, clean energy. The theoretical and numerical modeling of tokamak plasmas is simultaneously an essential component of effective reactor design, and a great research barrier. Tokamak operational conditions exhibit comparatively low Knudsen numbers. Kinetic effects, including kinetic waves and instabilities, Landau damping, bump-on-tail instabilities and more, are therefore highly influential in tokamak plasma dynamics. Purely fluid models are inherently incapable of capturing these effects, whereas the high dimensionality in purely kinetic models render them practically intractable for most relevant purposes.

        We consider a $\delta\!f$ decomposition model, with a macroscopic fluid background and microscopic kinetic correction, both fully coupled to each other. A similar manner of discretization is proposed to that used in the recent \texttt{STRUPHY} code \cite{Holderied_Possanner_Wang_2021, Holderied_2022, Li_et_al_2023} with a finite-element model for the background and a pseudo-particle/PiC model for the correction.

        The fluid background satisfies the full, non-linear, resistive, compressible, Hall MHD equations. \cite{Laakmann_Hu_Farrell_2022} introduces finite-element(-in-space) implicit timesteppers for the incompressible analogue to this system with structure-preserving (SP) properties in the ideal case, alongside parameter-robust preconditioners. We show that these timesteppers can derive from a finite-element-in-time (FET) (and finite-element-in-space) interpretation. The benefits of this reformulation are discussed, including the derivation of timesteppers that are higher order in time, and the quantifiable dissipative SP properties in the non-ideal, resistive case.
        
        We discuss possible options for extending this FET approach to timesteppers for the compressible case.

        The kinetic corrections satisfy linearized Boltzmann equations. Using a Lénard--Bernstein collision operator, these take Fokker--Planck-like forms \cite{Fokker_1914, Planck_1917} wherein pseudo-particles in the numerical model obey the neoclassical transport equations, with particle-independent Brownian drift terms. This offers a rigorous methodology for incorporating collisions into the particle transport model, without coupling the equations of motions for each particle.
        
        Works by Chen, Chacón et al. \cite{Chen_Chacón_Barnes_2011, Chacón_Chen_Barnes_2013, Chen_Chacón_2014, Chen_Chacón_2015} have developed structure-preserving particle pushers for neoclassical transport in the Vlasov equations, derived from Crank--Nicolson integrators. We show these too can can derive from a FET interpretation, similarly offering potential extensions to higher-order-in-time particle pushers. The FET formulation is used also to consider how the stochastic drift terms can be incorporated into the pushers. Stochastic gyrokinetic expansions are also discussed.

        Different options for the numerical implementation of these schemes are considered.

        Due to the efficacy of FET in the development of SP timesteppers for both the fluid and kinetic component, we hope this approach will prove effective in the future for developing SP timesteppers for the full hybrid model. We hope this will give us the opportunity to incorporate previously inaccessible kinetic effects into the highly effective, modern, finite-element MHD models.
    \end{abstract}
    
    
    \newpage
    \tableofcontents
    
    
    \newpage
    \pagenumbering{arabic}
    %\linenumbers\renewcommand\thelinenumber{\color{black!50}\arabic{linenumber}}
            \input{0 - introduction/main.tex}
        \part{Research}
            \input{1 - low-noise PiC models/main.tex}
            \input{2 - kinetic component/main.tex}
            \input{3 - fluid component/main.tex}
            \input{4 - numerical implementation/main.tex}
        \part{Project Overview}
            \input{5 - research plan/main.tex}
            \input{6 - summary/main.tex}
    
    
    %\section{}
    \newpage
    \pagenumbering{gobble}
        \printbibliography


    \newpage
    \pagenumbering{roman}
    \appendix
        \part{Appendices}
            \input{8 - Hilbert complexes/main.tex}
            \input{9 - weak conservation proofs/main.tex}
\end{document}

            \documentclass[12pt, a4paper]{report}

\input{template/main.tex}

\title{\BA{Title in Progress...}}
\author{Boris Andrews}
\affil{Mathematical Institute, University of Oxford}
\date{\today}


\begin{document}
    \pagenumbering{gobble}
    \maketitle
    
    
    \begin{abstract}
        Magnetic confinement reactors---in particular tokamaks---offer one of the most promising options for achieving practical nuclear fusion, with the potential to provide virtually limitless, clean energy. The theoretical and numerical modeling of tokamak plasmas is simultaneously an essential component of effective reactor design, and a great research barrier. Tokamak operational conditions exhibit comparatively low Knudsen numbers. Kinetic effects, including kinetic waves and instabilities, Landau damping, bump-on-tail instabilities and more, are therefore highly influential in tokamak plasma dynamics. Purely fluid models are inherently incapable of capturing these effects, whereas the high dimensionality in purely kinetic models render them practically intractable for most relevant purposes.

        We consider a $\delta\!f$ decomposition model, with a macroscopic fluid background and microscopic kinetic correction, both fully coupled to each other. A similar manner of discretization is proposed to that used in the recent \texttt{STRUPHY} code \cite{Holderied_Possanner_Wang_2021, Holderied_2022, Li_et_al_2023} with a finite-element model for the background and a pseudo-particle/PiC model for the correction.

        The fluid background satisfies the full, non-linear, resistive, compressible, Hall MHD equations. \cite{Laakmann_Hu_Farrell_2022} introduces finite-element(-in-space) implicit timesteppers for the incompressible analogue to this system with structure-preserving (SP) properties in the ideal case, alongside parameter-robust preconditioners. We show that these timesteppers can derive from a finite-element-in-time (FET) (and finite-element-in-space) interpretation. The benefits of this reformulation are discussed, including the derivation of timesteppers that are higher order in time, and the quantifiable dissipative SP properties in the non-ideal, resistive case.
        
        We discuss possible options for extending this FET approach to timesteppers for the compressible case.

        The kinetic corrections satisfy linearized Boltzmann equations. Using a Lénard--Bernstein collision operator, these take Fokker--Planck-like forms \cite{Fokker_1914, Planck_1917} wherein pseudo-particles in the numerical model obey the neoclassical transport equations, with particle-independent Brownian drift terms. This offers a rigorous methodology for incorporating collisions into the particle transport model, without coupling the equations of motions for each particle.
        
        Works by Chen, Chacón et al. \cite{Chen_Chacón_Barnes_2011, Chacón_Chen_Barnes_2013, Chen_Chacón_2014, Chen_Chacón_2015} have developed structure-preserving particle pushers for neoclassical transport in the Vlasov equations, derived from Crank--Nicolson integrators. We show these too can can derive from a FET interpretation, similarly offering potential extensions to higher-order-in-time particle pushers. The FET formulation is used also to consider how the stochastic drift terms can be incorporated into the pushers. Stochastic gyrokinetic expansions are also discussed.

        Different options for the numerical implementation of these schemes are considered.

        Due to the efficacy of FET in the development of SP timesteppers for both the fluid and kinetic component, we hope this approach will prove effective in the future for developing SP timesteppers for the full hybrid model. We hope this will give us the opportunity to incorporate previously inaccessible kinetic effects into the highly effective, modern, finite-element MHD models.
    \end{abstract}
    
    
    \newpage
    \tableofcontents
    
    
    \newpage
    \pagenumbering{arabic}
    %\linenumbers\renewcommand\thelinenumber{\color{black!50}\arabic{linenumber}}
            \input{0 - introduction/main.tex}
        \part{Research}
            \input{1 - low-noise PiC models/main.tex}
            \input{2 - kinetic component/main.tex}
            \input{3 - fluid component/main.tex}
            \input{4 - numerical implementation/main.tex}
        \part{Project Overview}
            \input{5 - research plan/main.tex}
            \input{6 - summary/main.tex}
    
    
    %\section{}
    \newpage
    \pagenumbering{gobble}
        \printbibliography


    \newpage
    \pagenumbering{roman}
    \appendix
        \part{Appendices}
            \input{8 - Hilbert complexes/main.tex}
            \input{9 - weak conservation proofs/main.tex}
\end{document}

    
    
    %\section{}
    \newpage
    \pagenumbering{gobble}
        \printbibliography


    \newpage
    \pagenumbering{roman}
    \appendix
        \part{Appendices}
            \documentclass[12pt, a4paper]{report}

\input{template/main.tex}

\title{\BA{Title in Progress...}}
\author{Boris Andrews}
\affil{Mathematical Institute, University of Oxford}
\date{\today}


\begin{document}
    \pagenumbering{gobble}
    \maketitle
    
    
    \begin{abstract}
        Magnetic confinement reactors---in particular tokamaks---offer one of the most promising options for achieving practical nuclear fusion, with the potential to provide virtually limitless, clean energy. The theoretical and numerical modeling of tokamak plasmas is simultaneously an essential component of effective reactor design, and a great research barrier. Tokamak operational conditions exhibit comparatively low Knudsen numbers. Kinetic effects, including kinetic waves and instabilities, Landau damping, bump-on-tail instabilities and more, are therefore highly influential in tokamak plasma dynamics. Purely fluid models are inherently incapable of capturing these effects, whereas the high dimensionality in purely kinetic models render them practically intractable for most relevant purposes.

        We consider a $\delta\!f$ decomposition model, with a macroscopic fluid background and microscopic kinetic correction, both fully coupled to each other. A similar manner of discretization is proposed to that used in the recent \texttt{STRUPHY} code \cite{Holderied_Possanner_Wang_2021, Holderied_2022, Li_et_al_2023} with a finite-element model for the background and a pseudo-particle/PiC model for the correction.

        The fluid background satisfies the full, non-linear, resistive, compressible, Hall MHD equations. \cite{Laakmann_Hu_Farrell_2022} introduces finite-element(-in-space) implicit timesteppers for the incompressible analogue to this system with structure-preserving (SP) properties in the ideal case, alongside parameter-robust preconditioners. We show that these timesteppers can derive from a finite-element-in-time (FET) (and finite-element-in-space) interpretation. The benefits of this reformulation are discussed, including the derivation of timesteppers that are higher order in time, and the quantifiable dissipative SP properties in the non-ideal, resistive case.
        
        We discuss possible options for extending this FET approach to timesteppers for the compressible case.

        The kinetic corrections satisfy linearized Boltzmann equations. Using a Lénard--Bernstein collision operator, these take Fokker--Planck-like forms \cite{Fokker_1914, Planck_1917} wherein pseudo-particles in the numerical model obey the neoclassical transport equations, with particle-independent Brownian drift terms. This offers a rigorous methodology for incorporating collisions into the particle transport model, without coupling the equations of motions for each particle.
        
        Works by Chen, Chacón et al. \cite{Chen_Chacón_Barnes_2011, Chacón_Chen_Barnes_2013, Chen_Chacón_2014, Chen_Chacón_2015} have developed structure-preserving particle pushers for neoclassical transport in the Vlasov equations, derived from Crank--Nicolson integrators. We show these too can can derive from a FET interpretation, similarly offering potential extensions to higher-order-in-time particle pushers. The FET formulation is used also to consider how the stochastic drift terms can be incorporated into the pushers. Stochastic gyrokinetic expansions are also discussed.

        Different options for the numerical implementation of these schemes are considered.

        Due to the efficacy of FET in the development of SP timesteppers for both the fluid and kinetic component, we hope this approach will prove effective in the future for developing SP timesteppers for the full hybrid model. We hope this will give us the opportunity to incorporate previously inaccessible kinetic effects into the highly effective, modern, finite-element MHD models.
    \end{abstract}
    
    
    \newpage
    \tableofcontents
    
    
    \newpage
    \pagenumbering{arabic}
    %\linenumbers\renewcommand\thelinenumber{\color{black!50}\arabic{linenumber}}
            \input{0 - introduction/main.tex}
        \part{Research}
            \input{1 - low-noise PiC models/main.tex}
            \input{2 - kinetic component/main.tex}
            \input{3 - fluid component/main.tex}
            \input{4 - numerical implementation/main.tex}
        \part{Project Overview}
            \input{5 - research plan/main.tex}
            \input{6 - summary/main.tex}
    
    
    %\section{}
    \newpage
    \pagenumbering{gobble}
        \printbibliography


    \newpage
    \pagenumbering{roman}
    \appendix
        \part{Appendices}
            \input{8 - Hilbert complexes/main.tex}
            \input{9 - weak conservation proofs/main.tex}
\end{document}

            \documentclass[12pt, a4paper]{report}

\input{template/main.tex}

\title{\BA{Title in Progress...}}
\author{Boris Andrews}
\affil{Mathematical Institute, University of Oxford}
\date{\today}


\begin{document}
    \pagenumbering{gobble}
    \maketitle
    
    
    \begin{abstract}
        Magnetic confinement reactors---in particular tokamaks---offer one of the most promising options for achieving practical nuclear fusion, with the potential to provide virtually limitless, clean energy. The theoretical and numerical modeling of tokamak plasmas is simultaneously an essential component of effective reactor design, and a great research barrier. Tokamak operational conditions exhibit comparatively low Knudsen numbers. Kinetic effects, including kinetic waves and instabilities, Landau damping, bump-on-tail instabilities and more, are therefore highly influential in tokamak plasma dynamics. Purely fluid models are inherently incapable of capturing these effects, whereas the high dimensionality in purely kinetic models render them practically intractable for most relevant purposes.

        We consider a $\delta\!f$ decomposition model, with a macroscopic fluid background and microscopic kinetic correction, both fully coupled to each other. A similar manner of discretization is proposed to that used in the recent \texttt{STRUPHY} code \cite{Holderied_Possanner_Wang_2021, Holderied_2022, Li_et_al_2023} with a finite-element model for the background and a pseudo-particle/PiC model for the correction.

        The fluid background satisfies the full, non-linear, resistive, compressible, Hall MHD equations. \cite{Laakmann_Hu_Farrell_2022} introduces finite-element(-in-space) implicit timesteppers for the incompressible analogue to this system with structure-preserving (SP) properties in the ideal case, alongside parameter-robust preconditioners. We show that these timesteppers can derive from a finite-element-in-time (FET) (and finite-element-in-space) interpretation. The benefits of this reformulation are discussed, including the derivation of timesteppers that are higher order in time, and the quantifiable dissipative SP properties in the non-ideal, resistive case.
        
        We discuss possible options for extending this FET approach to timesteppers for the compressible case.

        The kinetic corrections satisfy linearized Boltzmann equations. Using a Lénard--Bernstein collision operator, these take Fokker--Planck-like forms \cite{Fokker_1914, Planck_1917} wherein pseudo-particles in the numerical model obey the neoclassical transport equations, with particle-independent Brownian drift terms. This offers a rigorous methodology for incorporating collisions into the particle transport model, without coupling the equations of motions for each particle.
        
        Works by Chen, Chacón et al. \cite{Chen_Chacón_Barnes_2011, Chacón_Chen_Barnes_2013, Chen_Chacón_2014, Chen_Chacón_2015} have developed structure-preserving particle pushers for neoclassical transport in the Vlasov equations, derived from Crank--Nicolson integrators. We show these too can can derive from a FET interpretation, similarly offering potential extensions to higher-order-in-time particle pushers. The FET formulation is used also to consider how the stochastic drift terms can be incorporated into the pushers. Stochastic gyrokinetic expansions are also discussed.

        Different options for the numerical implementation of these schemes are considered.

        Due to the efficacy of FET in the development of SP timesteppers for both the fluid and kinetic component, we hope this approach will prove effective in the future for developing SP timesteppers for the full hybrid model. We hope this will give us the opportunity to incorporate previously inaccessible kinetic effects into the highly effective, modern, finite-element MHD models.
    \end{abstract}
    
    
    \newpage
    \tableofcontents
    
    
    \newpage
    \pagenumbering{arabic}
    %\linenumbers\renewcommand\thelinenumber{\color{black!50}\arabic{linenumber}}
            \input{0 - introduction/main.tex}
        \part{Research}
            \input{1 - low-noise PiC models/main.tex}
            \input{2 - kinetic component/main.tex}
            \input{3 - fluid component/main.tex}
            \input{4 - numerical implementation/main.tex}
        \part{Project Overview}
            \input{5 - research plan/main.tex}
            \input{6 - summary/main.tex}
    
    
    %\section{}
    \newpage
    \pagenumbering{gobble}
        \printbibliography


    \newpage
    \pagenumbering{roman}
    \appendix
        \part{Appendices}
            \input{8 - Hilbert complexes/main.tex}
            \input{9 - weak conservation proofs/main.tex}
\end{document}

\end{document}

            \documentclass[12pt, a4paper]{report}

\documentclass[12pt, a4paper]{report}

\input{template/main.tex}

\title{\BA{Title in Progress...}}
\author{Boris Andrews}
\affil{Mathematical Institute, University of Oxford}
\date{\today}


\begin{document}
    \pagenumbering{gobble}
    \maketitle
    
    
    \begin{abstract}
        Magnetic confinement reactors---in particular tokamaks---offer one of the most promising options for achieving practical nuclear fusion, with the potential to provide virtually limitless, clean energy. The theoretical and numerical modeling of tokamak plasmas is simultaneously an essential component of effective reactor design, and a great research barrier. Tokamak operational conditions exhibit comparatively low Knudsen numbers. Kinetic effects, including kinetic waves and instabilities, Landau damping, bump-on-tail instabilities and more, are therefore highly influential in tokamak plasma dynamics. Purely fluid models are inherently incapable of capturing these effects, whereas the high dimensionality in purely kinetic models render them practically intractable for most relevant purposes.

        We consider a $\delta\!f$ decomposition model, with a macroscopic fluid background and microscopic kinetic correction, both fully coupled to each other. A similar manner of discretization is proposed to that used in the recent \texttt{STRUPHY} code \cite{Holderied_Possanner_Wang_2021, Holderied_2022, Li_et_al_2023} with a finite-element model for the background and a pseudo-particle/PiC model for the correction.

        The fluid background satisfies the full, non-linear, resistive, compressible, Hall MHD equations. \cite{Laakmann_Hu_Farrell_2022} introduces finite-element(-in-space) implicit timesteppers for the incompressible analogue to this system with structure-preserving (SP) properties in the ideal case, alongside parameter-robust preconditioners. We show that these timesteppers can derive from a finite-element-in-time (FET) (and finite-element-in-space) interpretation. The benefits of this reformulation are discussed, including the derivation of timesteppers that are higher order in time, and the quantifiable dissipative SP properties in the non-ideal, resistive case.
        
        We discuss possible options for extending this FET approach to timesteppers for the compressible case.

        The kinetic corrections satisfy linearized Boltzmann equations. Using a Lénard--Bernstein collision operator, these take Fokker--Planck-like forms \cite{Fokker_1914, Planck_1917} wherein pseudo-particles in the numerical model obey the neoclassical transport equations, with particle-independent Brownian drift terms. This offers a rigorous methodology for incorporating collisions into the particle transport model, without coupling the equations of motions for each particle.
        
        Works by Chen, Chacón et al. \cite{Chen_Chacón_Barnes_2011, Chacón_Chen_Barnes_2013, Chen_Chacón_2014, Chen_Chacón_2015} have developed structure-preserving particle pushers for neoclassical transport in the Vlasov equations, derived from Crank--Nicolson integrators. We show these too can can derive from a FET interpretation, similarly offering potential extensions to higher-order-in-time particle pushers. The FET formulation is used also to consider how the stochastic drift terms can be incorporated into the pushers. Stochastic gyrokinetic expansions are also discussed.

        Different options for the numerical implementation of these schemes are considered.

        Due to the efficacy of FET in the development of SP timesteppers for both the fluid and kinetic component, we hope this approach will prove effective in the future for developing SP timesteppers for the full hybrid model. We hope this will give us the opportunity to incorporate previously inaccessible kinetic effects into the highly effective, modern, finite-element MHD models.
    \end{abstract}
    
    
    \newpage
    \tableofcontents
    
    
    \newpage
    \pagenumbering{arabic}
    %\linenumbers\renewcommand\thelinenumber{\color{black!50}\arabic{linenumber}}
            \input{0 - introduction/main.tex}
        \part{Research}
            \input{1 - low-noise PiC models/main.tex}
            \input{2 - kinetic component/main.tex}
            \input{3 - fluid component/main.tex}
            \input{4 - numerical implementation/main.tex}
        \part{Project Overview}
            \input{5 - research plan/main.tex}
            \input{6 - summary/main.tex}
    
    
    %\section{}
    \newpage
    \pagenumbering{gobble}
        \printbibliography


    \newpage
    \pagenumbering{roman}
    \appendix
        \part{Appendices}
            \input{8 - Hilbert complexes/main.tex}
            \input{9 - weak conservation proofs/main.tex}
\end{document}


\title{\BA{Title in Progress...}}
\author{Boris Andrews}
\affil{Mathematical Institute, University of Oxford}
\date{\today}


\begin{document}
    \pagenumbering{gobble}
    \maketitle
    
    
    \begin{abstract}
        Magnetic confinement reactors---in particular tokamaks---offer one of the most promising options for achieving practical nuclear fusion, with the potential to provide virtually limitless, clean energy. The theoretical and numerical modeling of tokamak plasmas is simultaneously an essential component of effective reactor design, and a great research barrier. Tokamak operational conditions exhibit comparatively low Knudsen numbers. Kinetic effects, including kinetic waves and instabilities, Landau damping, bump-on-tail instabilities and more, are therefore highly influential in tokamak plasma dynamics. Purely fluid models are inherently incapable of capturing these effects, whereas the high dimensionality in purely kinetic models render them practically intractable for most relevant purposes.

        We consider a $\delta\!f$ decomposition model, with a macroscopic fluid background and microscopic kinetic correction, both fully coupled to each other. A similar manner of discretization is proposed to that used in the recent \texttt{STRUPHY} code \cite{Holderied_Possanner_Wang_2021, Holderied_2022, Li_et_al_2023} with a finite-element model for the background and a pseudo-particle/PiC model for the correction.

        The fluid background satisfies the full, non-linear, resistive, compressible, Hall MHD equations. \cite{Laakmann_Hu_Farrell_2022} introduces finite-element(-in-space) implicit timesteppers for the incompressible analogue to this system with structure-preserving (SP) properties in the ideal case, alongside parameter-robust preconditioners. We show that these timesteppers can derive from a finite-element-in-time (FET) (and finite-element-in-space) interpretation. The benefits of this reformulation are discussed, including the derivation of timesteppers that are higher order in time, and the quantifiable dissipative SP properties in the non-ideal, resistive case.
        
        We discuss possible options for extending this FET approach to timesteppers for the compressible case.

        The kinetic corrections satisfy linearized Boltzmann equations. Using a Lénard--Bernstein collision operator, these take Fokker--Planck-like forms \cite{Fokker_1914, Planck_1917} wherein pseudo-particles in the numerical model obey the neoclassical transport equations, with particle-independent Brownian drift terms. This offers a rigorous methodology for incorporating collisions into the particle transport model, without coupling the equations of motions for each particle.
        
        Works by Chen, Chacón et al. \cite{Chen_Chacón_Barnes_2011, Chacón_Chen_Barnes_2013, Chen_Chacón_2014, Chen_Chacón_2015} have developed structure-preserving particle pushers for neoclassical transport in the Vlasov equations, derived from Crank--Nicolson integrators. We show these too can can derive from a FET interpretation, similarly offering potential extensions to higher-order-in-time particle pushers. The FET formulation is used also to consider how the stochastic drift terms can be incorporated into the pushers. Stochastic gyrokinetic expansions are also discussed.

        Different options for the numerical implementation of these schemes are considered.

        Due to the efficacy of FET in the development of SP timesteppers for both the fluid and kinetic component, we hope this approach will prove effective in the future for developing SP timesteppers for the full hybrid model. We hope this will give us the opportunity to incorporate previously inaccessible kinetic effects into the highly effective, modern, finite-element MHD models.
    \end{abstract}
    
    
    \newpage
    \tableofcontents
    
    
    \newpage
    \pagenumbering{arabic}
    %\linenumbers\renewcommand\thelinenumber{\color{black!50}\arabic{linenumber}}
            \documentclass[12pt, a4paper]{report}

\input{template/main.tex}

\title{\BA{Title in Progress...}}
\author{Boris Andrews}
\affil{Mathematical Institute, University of Oxford}
\date{\today}


\begin{document}
    \pagenumbering{gobble}
    \maketitle
    
    
    \begin{abstract}
        Magnetic confinement reactors---in particular tokamaks---offer one of the most promising options for achieving practical nuclear fusion, with the potential to provide virtually limitless, clean energy. The theoretical and numerical modeling of tokamak plasmas is simultaneously an essential component of effective reactor design, and a great research barrier. Tokamak operational conditions exhibit comparatively low Knudsen numbers. Kinetic effects, including kinetic waves and instabilities, Landau damping, bump-on-tail instabilities and more, are therefore highly influential in tokamak plasma dynamics. Purely fluid models are inherently incapable of capturing these effects, whereas the high dimensionality in purely kinetic models render them practically intractable for most relevant purposes.

        We consider a $\delta\!f$ decomposition model, with a macroscopic fluid background and microscopic kinetic correction, both fully coupled to each other. A similar manner of discretization is proposed to that used in the recent \texttt{STRUPHY} code \cite{Holderied_Possanner_Wang_2021, Holderied_2022, Li_et_al_2023} with a finite-element model for the background and a pseudo-particle/PiC model for the correction.

        The fluid background satisfies the full, non-linear, resistive, compressible, Hall MHD equations. \cite{Laakmann_Hu_Farrell_2022} introduces finite-element(-in-space) implicit timesteppers for the incompressible analogue to this system with structure-preserving (SP) properties in the ideal case, alongside parameter-robust preconditioners. We show that these timesteppers can derive from a finite-element-in-time (FET) (and finite-element-in-space) interpretation. The benefits of this reformulation are discussed, including the derivation of timesteppers that are higher order in time, and the quantifiable dissipative SP properties in the non-ideal, resistive case.
        
        We discuss possible options for extending this FET approach to timesteppers for the compressible case.

        The kinetic corrections satisfy linearized Boltzmann equations. Using a Lénard--Bernstein collision operator, these take Fokker--Planck-like forms \cite{Fokker_1914, Planck_1917} wherein pseudo-particles in the numerical model obey the neoclassical transport equations, with particle-independent Brownian drift terms. This offers a rigorous methodology for incorporating collisions into the particle transport model, without coupling the equations of motions for each particle.
        
        Works by Chen, Chacón et al. \cite{Chen_Chacón_Barnes_2011, Chacón_Chen_Barnes_2013, Chen_Chacón_2014, Chen_Chacón_2015} have developed structure-preserving particle pushers for neoclassical transport in the Vlasov equations, derived from Crank--Nicolson integrators. We show these too can can derive from a FET interpretation, similarly offering potential extensions to higher-order-in-time particle pushers. The FET formulation is used also to consider how the stochastic drift terms can be incorporated into the pushers. Stochastic gyrokinetic expansions are also discussed.

        Different options for the numerical implementation of these schemes are considered.

        Due to the efficacy of FET in the development of SP timesteppers for both the fluid and kinetic component, we hope this approach will prove effective in the future for developing SP timesteppers for the full hybrid model. We hope this will give us the opportunity to incorporate previously inaccessible kinetic effects into the highly effective, modern, finite-element MHD models.
    \end{abstract}
    
    
    \newpage
    \tableofcontents
    
    
    \newpage
    \pagenumbering{arabic}
    %\linenumbers\renewcommand\thelinenumber{\color{black!50}\arabic{linenumber}}
            \input{0 - introduction/main.tex}
        \part{Research}
            \input{1 - low-noise PiC models/main.tex}
            \input{2 - kinetic component/main.tex}
            \input{3 - fluid component/main.tex}
            \input{4 - numerical implementation/main.tex}
        \part{Project Overview}
            \input{5 - research plan/main.tex}
            \input{6 - summary/main.tex}
    
    
    %\section{}
    \newpage
    \pagenumbering{gobble}
        \printbibliography


    \newpage
    \pagenumbering{roman}
    \appendix
        \part{Appendices}
            \input{8 - Hilbert complexes/main.tex}
            \input{9 - weak conservation proofs/main.tex}
\end{document}

        \part{Research}
            \documentclass[12pt, a4paper]{report}

\input{template/main.tex}

\title{\BA{Title in Progress...}}
\author{Boris Andrews}
\affil{Mathematical Institute, University of Oxford}
\date{\today}


\begin{document}
    \pagenumbering{gobble}
    \maketitle
    
    
    \begin{abstract}
        Magnetic confinement reactors---in particular tokamaks---offer one of the most promising options for achieving practical nuclear fusion, with the potential to provide virtually limitless, clean energy. The theoretical and numerical modeling of tokamak plasmas is simultaneously an essential component of effective reactor design, and a great research barrier. Tokamak operational conditions exhibit comparatively low Knudsen numbers. Kinetic effects, including kinetic waves and instabilities, Landau damping, bump-on-tail instabilities and more, are therefore highly influential in tokamak plasma dynamics. Purely fluid models are inherently incapable of capturing these effects, whereas the high dimensionality in purely kinetic models render them practically intractable for most relevant purposes.

        We consider a $\delta\!f$ decomposition model, with a macroscopic fluid background and microscopic kinetic correction, both fully coupled to each other. A similar manner of discretization is proposed to that used in the recent \texttt{STRUPHY} code \cite{Holderied_Possanner_Wang_2021, Holderied_2022, Li_et_al_2023} with a finite-element model for the background and a pseudo-particle/PiC model for the correction.

        The fluid background satisfies the full, non-linear, resistive, compressible, Hall MHD equations. \cite{Laakmann_Hu_Farrell_2022} introduces finite-element(-in-space) implicit timesteppers for the incompressible analogue to this system with structure-preserving (SP) properties in the ideal case, alongside parameter-robust preconditioners. We show that these timesteppers can derive from a finite-element-in-time (FET) (and finite-element-in-space) interpretation. The benefits of this reformulation are discussed, including the derivation of timesteppers that are higher order in time, and the quantifiable dissipative SP properties in the non-ideal, resistive case.
        
        We discuss possible options for extending this FET approach to timesteppers for the compressible case.

        The kinetic corrections satisfy linearized Boltzmann equations. Using a Lénard--Bernstein collision operator, these take Fokker--Planck-like forms \cite{Fokker_1914, Planck_1917} wherein pseudo-particles in the numerical model obey the neoclassical transport equations, with particle-independent Brownian drift terms. This offers a rigorous methodology for incorporating collisions into the particle transport model, without coupling the equations of motions for each particle.
        
        Works by Chen, Chacón et al. \cite{Chen_Chacón_Barnes_2011, Chacón_Chen_Barnes_2013, Chen_Chacón_2014, Chen_Chacón_2015} have developed structure-preserving particle pushers for neoclassical transport in the Vlasov equations, derived from Crank--Nicolson integrators. We show these too can can derive from a FET interpretation, similarly offering potential extensions to higher-order-in-time particle pushers. The FET formulation is used also to consider how the stochastic drift terms can be incorporated into the pushers. Stochastic gyrokinetic expansions are also discussed.

        Different options for the numerical implementation of these schemes are considered.

        Due to the efficacy of FET in the development of SP timesteppers for both the fluid and kinetic component, we hope this approach will prove effective in the future for developing SP timesteppers for the full hybrid model. We hope this will give us the opportunity to incorporate previously inaccessible kinetic effects into the highly effective, modern, finite-element MHD models.
    \end{abstract}
    
    
    \newpage
    \tableofcontents
    
    
    \newpage
    \pagenumbering{arabic}
    %\linenumbers\renewcommand\thelinenumber{\color{black!50}\arabic{linenumber}}
            \input{0 - introduction/main.tex}
        \part{Research}
            \input{1 - low-noise PiC models/main.tex}
            \input{2 - kinetic component/main.tex}
            \input{3 - fluid component/main.tex}
            \input{4 - numerical implementation/main.tex}
        \part{Project Overview}
            \input{5 - research plan/main.tex}
            \input{6 - summary/main.tex}
    
    
    %\section{}
    \newpage
    \pagenumbering{gobble}
        \printbibliography


    \newpage
    \pagenumbering{roman}
    \appendix
        \part{Appendices}
            \input{8 - Hilbert complexes/main.tex}
            \input{9 - weak conservation proofs/main.tex}
\end{document}

            \documentclass[12pt, a4paper]{report}

\input{template/main.tex}

\title{\BA{Title in Progress...}}
\author{Boris Andrews}
\affil{Mathematical Institute, University of Oxford}
\date{\today}


\begin{document}
    \pagenumbering{gobble}
    \maketitle
    
    
    \begin{abstract}
        Magnetic confinement reactors---in particular tokamaks---offer one of the most promising options for achieving practical nuclear fusion, with the potential to provide virtually limitless, clean energy. The theoretical and numerical modeling of tokamak plasmas is simultaneously an essential component of effective reactor design, and a great research barrier. Tokamak operational conditions exhibit comparatively low Knudsen numbers. Kinetic effects, including kinetic waves and instabilities, Landau damping, bump-on-tail instabilities and more, are therefore highly influential in tokamak plasma dynamics. Purely fluid models are inherently incapable of capturing these effects, whereas the high dimensionality in purely kinetic models render them practically intractable for most relevant purposes.

        We consider a $\delta\!f$ decomposition model, with a macroscopic fluid background and microscopic kinetic correction, both fully coupled to each other. A similar manner of discretization is proposed to that used in the recent \texttt{STRUPHY} code \cite{Holderied_Possanner_Wang_2021, Holderied_2022, Li_et_al_2023} with a finite-element model for the background and a pseudo-particle/PiC model for the correction.

        The fluid background satisfies the full, non-linear, resistive, compressible, Hall MHD equations. \cite{Laakmann_Hu_Farrell_2022} introduces finite-element(-in-space) implicit timesteppers for the incompressible analogue to this system with structure-preserving (SP) properties in the ideal case, alongside parameter-robust preconditioners. We show that these timesteppers can derive from a finite-element-in-time (FET) (and finite-element-in-space) interpretation. The benefits of this reformulation are discussed, including the derivation of timesteppers that are higher order in time, and the quantifiable dissipative SP properties in the non-ideal, resistive case.
        
        We discuss possible options for extending this FET approach to timesteppers for the compressible case.

        The kinetic corrections satisfy linearized Boltzmann equations. Using a Lénard--Bernstein collision operator, these take Fokker--Planck-like forms \cite{Fokker_1914, Planck_1917} wherein pseudo-particles in the numerical model obey the neoclassical transport equations, with particle-independent Brownian drift terms. This offers a rigorous methodology for incorporating collisions into the particle transport model, without coupling the equations of motions for each particle.
        
        Works by Chen, Chacón et al. \cite{Chen_Chacón_Barnes_2011, Chacón_Chen_Barnes_2013, Chen_Chacón_2014, Chen_Chacón_2015} have developed structure-preserving particle pushers for neoclassical transport in the Vlasov equations, derived from Crank--Nicolson integrators. We show these too can can derive from a FET interpretation, similarly offering potential extensions to higher-order-in-time particle pushers. The FET formulation is used also to consider how the stochastic drift terms can be incorporated into the pushers. Stochastic gyrokinetic expansions are also discussed.

        Different options for the numerical implementation of these schemes are considered.

        Due to the efficacy of FET in the development of SP timesteppers for both the fluid and kinetic component, we hope this approach will prove effective in the future for developing SP timesteppers for the full hybrid model. We hope this will give us the opportunity to incorporate previously inaccessible kinetic effects into the highly effective, modern, finite-element MHD models.
    \end{abstract}
    
    
    \newpage
    \tableofcontents
    
    
    \newpage
    \pagenumbering{arabic}
    %\linenumbers\renewcommand\thelinenumber{\color{black!50}\arabic{linenumber}}
            \input{0 - introduction/main.tex}
        \part{Research}
            \input{1 - low-noise PiC models/main.tex}
            \input{2 - kinetic component/main.tex}
            \input{3 - fluid component/main.tex}
            \input{4 - numerical implementation/main.tex}
        \part{Project Overview}
            \input{5 - research plan/main.tex}
            \input{6 - summary/main.tex}
    
    
    %\section{}
    \newpage
    \pagenumbering{gobble}
        \printbibliography


    \newpage
    \pagenumbering{roman}
    \appendix
        \part{Appendices}
            \input{8 - Hilbert complexes/main.tex}
            \input{9 - weak conservation proofs/main.tex}
\end{document}

            \documentclass[12pt, a4paper]{report}

\input{template/main.tex}

\title{\BA{Title in Progress...}}
\author{Boris Andrews}
\affil{Mathematical Institute, University of Oxford}
\date{\today}


\begin{document}
    \pagenumbering{gobble}
    \maketitle
    
    
    \begin{abstract}
        Magnetic confinement reactors---in particular tokamaks---offer one of the most promising options for achieving practical nuclear fusion, with the potential to provide virtually limitless, clean energy. The theoretical and numerical modeling of tokamak plasmas is simultaneously an essential component of effective reactor design, and a great research barrier. Tokamak operational conditions exhibit comparatively low Knudsen numbers. Kinetic effects, including kinetic waves and instabilities, Landau damping, bump-on-tail instabilities and more, are therefore highly influential in tokamak plasma dynamics. Purely fluid models are inherently incapable of capturing these effects, whereas the high dimensionality in purely kinetic models render them practically intractable for most relevant purposes.

        We consider a $\delta\!f$ decomposition model, with a macroscopic fluid background and microscopic kinetic correction, both fully coupled to each other. A similar manner of discretization is proposed to that used in the recent \texttt{STRUPHY} code \cite{Holderied_Possanner_Wang_2021, Holderied_2022, Li_et_al_2023} with a finite-element model for the background and a pseudo-particle/PiC model for the correction.

        The fluid background satisfies the full, non-linear, resistive, compressible, Hall MHD equations. \cite{Laakmann_Hu_Farrell_2022} introduces finite-element(-in-space) implicit timesteppers for the incompressible analogue to this system with structure-preserving (SP) properties in the ideal case, alongside parameter-robust preconditioners. We show that these timesteppers can derive from a finite-element-in-time (FET) (and finite-element-in-space) interpretation. The benefits of this reformulation are discussed, including the derivation of timesteppers that are higher order in time, and the quantifiable dissipative SP properties in the non-ideal, resistive case.
        
        We discuss possible options for extending this FET approach to timesteppers for the compressible case.

        The kinetic corrections satisfy linearized Boltzmann equations. Using a Lénard--Bernstein collision operator, these take Fokker--Planck-like forms \cite{Fokker_1914, Planck_1917} wherein pseudo-particles in the numerical model obey the neoclassical transport equations, with particle-independent Brownian drift terms. This offers a rigorous methodology for incorporating collisions into the particle transport model, without coupling the equations of motions for each particle.
        
        Works by Chen, Chacón et al. \cite{Chen_Chacón_Barnes_2011, Chacón_Chen_Barnes_2013, Chen_Chacón_2014, Chen_Chacón_2015} have developed structure-preserving particle pushers for neoclassical transport in the Vlasov equations, derived from Crank--Nicolson integrators. We show these too can can derive from a FET interpretation, similarly offering potential extensions to higher-order-in-time particle pushers. The FET formulation is used also to consider how the stochastic drift terms can be incorporated into the pushers. Stochastic gyrokinetic expansions are also discussed.

        Different options for the numerical implementation of these schemes are considered.

        Due to the efficacy of FET in the development of SP timesteppers for both the fluid and kinetic component, we hope this approach will prove effective in the future for developing SP timesteppers for the full hybrid model. We hope this will give us the opportunity to incorporate previously inaccessible kinetic effects into the highly effective, modern, finite-element MHD models.
    \end{abstract}
    
    
    \newpage
    \tableofcontents
    
    
    \newpage
    \pagenumbering{arabic}
    %\linenumbers\renewcommand\thelinenumber{\color{black!50}\arabic{linenumber}}
            \input{0 - introduction/main.tex}
        \part{Research}
            \input{1 - low-noise PiC models/main.tex}
            \input{2 - kinetic component/main.tex}
            \input{3 - fluid component/main.tex}
            \input{4 - numerical implementation/main.tex}
        \part{Project Overview}
            \input{5 - research plan/main.tex}
            \input{6 - summary/main.tex}
    
    
    %\section{}
    \newpage
    \pagenumbering{gobble}
        \printbibliography


    \newpage
    \pagenumbering{roman}
    \appendix
        \part{Appendices}
            \input{8 - Hilbert complexes/main.tex}
            \input{9 - weak conservation proofs/main.tex}
\end{document}

            \documentclass[12pt, a4paper]{report}

\input{template/main.tex}

\title{\BA{Title in Progress...}}
\author{Boris Andrews}
\affil{Mathematical Institute, University of Oxford}
\date{\today}


\begin{document}
    \pagenumbering{gobble}
    \maketitle
    
    
    \begin{abstract}
        Magnetic confinement reactors---in particular tokamaks---offer one of the most promising options for achieving practical nuclear fusion, with the potential to provide virtually limitless, clean energy. The theoretical and numerical modeling of tokamak plasmas is simultaneously an essential component of effective reactor design, and a great research barrier. Tokamak operational conditions exhibit comparatively low Knudsen numbers. Kinetic effects, including kinetic waves and instabilities, Landau damping, bump-on-tail instabilities and more, are therefore highly influential in tokamak plasma dynamics. Purely fluid models are inherently incapable of capturing these effects, whereas the high dimensionality in purely kinetic models render them practically intractable for most relevant purposes.

        We consider a $\delta\!f$ decomposition model, with a macroscopic fluid background and microscopic kinetic correction, both fully coupled to each other. A similar manner of discretization is proposed to that used in the recent \texttt{STRUPHY} code \cite{Holderied_Possanner_Wang_2021, Holderied_2022, Li_et_al_2023} with a finite-element model for the background and a pseudo-particle/PiC model for the correction.

        The fluid background satisfies the full, non-linear, resistive, compressible, Hall MHD equations. \cite{Laakmann_Hu_Farrell_2022} introduces finite-element(-in-space) implicit timesteppers for the incompressible analogue to this system with structure-preserving (SP) properties in the ideal case, alongside parameter-robust preconditioners. We show that these timesteppers can derive from a finite-element-in-time (FET) (and finite-element-in-space) interpretation. The benefits of this reformulation are discussed, including the derivation of timesteppers that are higher order in time, and the quantifiable dissipative SP properties in the non-ideal, resistive case.
        
        We discuss possible options for extending this FET approach to timesteppers for the compressible case.

        The kinetic corrections satisfy linearized Boltzmann equations. Using a Lénard--Bernstein collision operator, these take Fokker--Planck-like forms \cite{Fokker_1914, Planck_1917} wherein pseudo-particles in the numerical model obey the neoclassical transport equations, with particle-independent Brownian drift terms. This offers a rigorous methodology for incorporating collisions into the particle transport model, without coupling the equations of motions for each particle.
        
        Works by Chen, Chacón et al. \cite{Chen_Chacón_Barnes_2011, Chacón_Chen_Barnes_2013, Chen_Chacón_2014, Chen_Chacón_2015} have developed structure-preserving particle pushers for neoclassical transport in the Vlasov equations, derived from Crank--Nicolson integrators. We show these too can can derive from a FET interpretation, similarly offering potential extensions to higher-order-in-time particle pushers. The FET formulation is used also to consider how the stochastic drift terms can be incorporated into the pushers. Stochastic gyrokinetic expansions are also discussed.

        Different options for the numerical implementation of these schemes are considered.

        Due to the efficacy of FET in the development of SP timesteppers for both the fluid and kinetic component, we hope this approach will prove effective in the future for developing SP timesteppers for the full hybrid model. We hope this will give us the opportunity to incorporate previously inaccessible kinetic effects into the highly effective, modern, finite-element MHD models.
    \end{abstract}
    
    
    \newpage
    \tableofcontents
    
    
    \newpage
    \pagenumbering{arabic}
    %\linenumbers\renewcommand\thelinenumber{\color{black!50}\arabic{linenumber}}
            \input{0 - introduction/main.tex}
        \part{Research}
            \input{1 - low-noise PiC models/main.tex}
            \input{2 - kinetic component/main.tex}
            \input{3 - fluid component/main.tex}
            \input{4 - numerical implementation/main.tex}
        \part{Project Overview}
            \input{5 - research plan/main.tex}
            \input{6 - summary/main.tex}
    
    
    %\section{}
    \newpage
    \pagenumbering{gobble}
        \printbibliography


    \newpage
    \pagenumbering{roman}
    \appendix
        \part{Appendices}
            \input{8 - Hilbert complexes/main.tex}
            \input{9 - weak conservation proofs/main.tex}
\end{document}

        \part{Project Overview}
            \documentclass[12pt, a4paper]{report}

\input{template/main.tex}

\title{\BA{Title in Progress...}}
\author{Boris Andrews}
\affil{Mathematical Institute, University of Oxford}
\date{\today}


\begin{document}
    \pagenumbering{gobble}
    \maketitle
    
    
    \begin{abstract}
        Magnetic confinement reactors---in particular tokamaks---offer one of the most promising options for achieving practical nuclear fusion, with the potential to provide virtually limitless, clean energy. The theoretical and numerical modeling of tokamak plasmas is simultaneously an essential component of effective reactor design, and a great research barrier. Tokamak operational conditions exhibit comparatively low Knudsen numbers. Kinetic effects, including kinetic waves and instabilities, Landau damping, bump-on-tail instabilities and more, are therefore highly influential in tokamak plasma dynamics. Purely fluid models are inherently incapable of capturing these effects, whereas the high dimensionality in purely kinetic models render them practically intractable for most relevant purposes.

        We consider a $\delta\!f$ decomposition model, with a macroscopic fluid background and microscopic kinetic correction, both fully coupled to each other. A similar manner of discretization is proposed to that used in the recent \texttt{STRUPHY} code \cite{Holderied_Possanner_Wang_2021, Holderied_2022, Li_et_al_2023} with a finite-element model for the background and a pseudo-particle/PiC model for the correction.

        The fluid background satisfies the full, non-linear, resistive, compressible, Hall MHD equations. \cite{Laakmann_Hu_Farrell_2022} introduces finite-element(-in-space) implicit timesteppers for the incompressible analogue to this system with structure-preserving (SP) properties in the ideal case, alongside parameter-robust preconditioners. We show that these timesteppers can derive from a finite-element-in-time (FET) (and finite-element-in-space) interpretation. The benefits of this reformulation are discussed, including the derivation of timesteppers that are higher order in time, and the quantifiable dissipative SP properties in the non-ideal, resistive case.
        
        We discuss possible options for extending this FET approach to timesteppers for the compressible case.

        The kinetic corrections satisfy linearized Boltzmann equations. Using a Lénard--Bernstein collision operator, these take Fokker--Planck-like forms \cite{Fokker_1914, Planck_1917} wherein pseudo-particles in the numerical model obey the neoclassical transport equations, with particle-independent Brownian drift terms. This offers a rigorous methodology for incorporating collisions into the particle transport model, without coupling the equations of motions for each particle.
        
        Works by Chen, Chacón et al. \cite{Chen_Chacón_Barnes_2011, Chacón_Chen_Barnes_2013, Chen_Chacón_2014, Chen_Chacón_2015} have developed structure-preserving particle pushers for neoclassical transport in the Vlasov equations, derived from Crank--Nicolson integrators. We show these too can can derive from a FET interpretation, similarly offering potential extensions to higher-order-in-time particle pushers. The FET formulation is used also to consider how the stochastic drift terms can be incorporated into the pushers. Stochastic gyrokinetic expansions are also discussed.

        Different options for the numerical implementation of these schemes are considered.

        Due to the efficacy of FET in the development of SP timesteppers for both the fluid and kinetic component, we hope this approach will prove effective in the future for developing SP timesteppers for the full hybrid model. We hope this will give us the opportunity to incorporate previously inaccessible kinetic effects into the highly effective, modern, finite-element MHD models.
    \end{abstract}
    
    
    \newpage
    \tableofcontents
    
    
    \newpage
    \pagenumbering{arabic}
    %\linenumbers\renewcommand\thelinenumber{\color{black!50}\arabic{linenumber}}
            \input{0 - introduction/main.tex}
        \part{Research}
            \input{1 - low-noise PiC models/main.tex}
            \input{2 - kinetic component/main.tex}
            \input{3 - fluid component/main.tex}
            \input{4 - numerical implementation/main.tex}
        \part{Project Overview}
            \input{5 - research plan/main.tex}
            \input{6 - summary/main.tex}
    
    
    %\section{}
    \newpage
    \pagenumbering{gobble}
        \printbibliography


    \newpage
    \pagenumbering{roman}
    \appendix
        \part{Appendices}
            \input{8 - Hilbert complexes/main.tex}
            \input{9 - weak conservation proofs/main.tex}
\end{document}

            \documentclass[12pt, a4paper]{report}

\input{template/main.tex}

\title{\BA{Title in Progress...}}
\author{Boris Andrews}
\affil{Mathematical Institute, University of Oxford}
\date{\today}


\begin{document}
    \pagenumbering{gobble}
    \maketitle
    
    
    \begin{abstract}
        Magnetic confinement reactors---in particular tokamaks---offer one of the most promising options for achieving practical nuclear fusion, with the potential to provide virtually limitless, clean energy. The theoretical and numerical modeling of tokamak plasmas is simultaneously an essential component of effective reactor design, and a great research barrier. Tokamak operational conditions exhibit comparatively low Knudsen numbers. Kinetic effects, including kinetic waves and instabilities, Landau damping, bump-on-tail instabilities and more, are therefore highly influential in tokamak plasma dynamics. Purely fluid models are inherently incapable of capturing these effects, whereas the high dimensionality in purely kinetic models render them practically intractable for most relevant purposes.

        We consider a $\delta\!f$ decomposition model, with a macroscopic fluid background and microscopic kinetic correction, both fully coupled to each other. A similar manner of discretization is proposed to that used in the recent \texttt{STRUPHY} code \cite{Holderied_Possanner_Wang_2021, Holderied_2022, Li_et_al_2023} with a finite-element model for the background and a pseudo-particle/PiC model for the correction.

        The fluid background satisfies the full, non-linear, resistive, compressible, Hall MHD equations. \cite{Laakmann_Hu_Farrell_2022} introduces finite-element(-in-space) implicit timesteppers for the incompressible analogue to this system with structure-preserving (SP) properties in the ideal case, alongside parameter-robust preconditioners. We show that these timesteppers can derive from a finite-element-in-time (FET) (and finite-element-in-space) interpretation. The benefits of this reformulation are discussed, including the derivation of timesteppers that are higher order in time, and the quantifiable dissipative SP properties in the non-ideal, resistive case.
        
        We discuss possible options for extending this FET approach to timesteppers for the compressible case.

        The kinetic corrections satisfy linearized Boltzmann equations. Using a Lénard--Bernstein collision operator, these take Fokker--Planck-like forms \cite{Fokker_1914, Planck_1917} wherein pseudo-particles in the numerical model obey the neoclassical transport equations, with particle-independent Brownian drift terms. This offers a rigorous methodology for incorporating collisions into the particle transport model, without coupling the equations of motions for each particle.
        
        Works by Chen, Chacón et al. \cite{Chen_Chacón_Barnes_2011, Chacón_Chen_Barnes_2013, Chen_Chacón_2014, Chen_Chacón_2015} have developed structure-preserving particle pushers for neoclassical transport in the Vlasov equations, derived from Crank--Nicolson integrators. We show these too can can derive from a FET interpretation, similarly offering potential extensions to higher-order-in-time particle pushers. The FET formulation is used also to consider how the stochastic drift terms can be incorporated into the pushers. Stochastic gyrokinetic expansions are also discussed.

        Different options for the numerical implementation of these schemes are considered.

        Due to the efficacy of FET in the development of SP timesteppers for both the fluid and kinetic component, we hope this approach will prove effective in the future for developing SP timesteppers for the full hybrid model. We hope this will give us the opportunity to incorporate previously inaccessible kinetic effects into the highly effective, modern, finite-element MHD models.
    \end{abstract}
    
    
    \newpage
    \tableofcontents
    
    
    \newpage
    \pagenumbering{arabic}
    %\linenumbers\renewcommand\thelinenumber{\color{black!50}\arabic{linenumber}}
            \input{0 - introduction/main.tex}
        \part{Research}
            \input{1 - low-noise PiC models/main.tex}
            \input{2 - kinetic component/main.tex}
            \input{3 - fluid component/main.tex}
            \input{4 - numerical implementation/main.tex}
        \part{Project Overview}
            \input{5 - research plan/main.tex}
            \input{6 - summary/main.tex}
    
    
    %\section{}
    \newpage
    \pagenumbering{gobble}
        \printbibliography


    \newpage
    \pagenumbering{roman}
    \appendix
        \part{Appendices}
            \input{8 - Hilbert complexes/main.tex}
            \input{9 - weak conservation proofs/main.tex}
\end{document}

    
    
    %\section{}
    \newpage
    \pagenumbering{gobble}
        \printbibliography


    \newpage
    \pagenumbering{roman}
    \appendix
        \part{Appendices}
            \documentclass[12pt, a4paper]{report}

\input{template/main.tex}

\title{\BA{Title in Progress...}}
\author{Boris Andrews}
\affil{Mathematical Institute, University of Oxford}
\date{\today}


\begin{document}
    \pagenumbering{gobble}
    \maketitle
    
    
    \begin{abstract}
        Magnetic confinement reactors---in particular tokamaks---offer one of the most promising options for achieving practical nuclear fusion, with the potential to provide virtually limitless, clean energy. The theoretical and numerical modeling of tokamak plasmas is simultaneously an essential component of effective reactor design, and a great research barrier. Tokamak operational conditions exhibit comparatively low Knudsen numbers. Kinetic effects, including kinetic waves and instabilities, Landau damping, bump-on-tail instabilities and more, are therefore highly influential in tokamak plasma dynamics. Purely fluid models are inherently incapable of capturing these effects, whereas the high dimensionality in purely kinetic models render them practically intractable for most relevant purposes.

        We consider a $\delta\!f$ decomposition model, with a macroscopic fluid background and microscopic kinetic correction, both fully coupled to each other. A similar manner of discretization is proposed to that used in the recent \texttt{STRUPHY} code \cite{Holderied_Possanner_Wang_2021, Holderied_2022, Li_et_al_2023} with a finite-element model for the background and a pseudo-particle/PiC model for the correction.

        The fluid background satisfies the full, non-linear, resistive, compressible, Hall MHD equations. \cite{Laakmann_Hu_Farrell_2022} introduces finite-element(-in-space) implicit timesteppers for the incompressible analogue to this system with structure-preserving (SP) properties in the ideal case, alongside parameter-robust preconditioners. We show that these timesteppers can derive from a finite-element-in-time (FET) (and finite-element-in-space) interpretation. The benefits of this reformulation are discussed, including the derivation of timesteppers that are higher order in time, and the quantifiable dissipative SP properties in the non-ideal, resistive case.
        
        We discuss possible options for extending this FET approach to timesteppers for the compressible case.

        The kinetic corrections satisfy linearized Boltzmann equations. Using a Lénard--Bernstein collision operator, these take Fokker--Planck-like forms \cite{Fokker_1914, Planck_1917} wherein pseudo-particles in the numerical model obey the neoclassical transport equations, with particle-independent Brownian drift terms. This offers a rigorous methodology for incorporating collisions into the particle transport model, without coupling the equations of motions for each particle.
        
        Works by Chen, Chacón et al. \cite{Chen_Chacón_Barnes_2011, Chacón_Chen_Barnes_2013, Chen_Chacón_2014, Chen_Chacón_2015} have developed structure-preserving particle pushers for neoclassical transport in the Vlasov equations, derived from Crank--Nicolson integrators. We show these too can can derive from a FET interpretation, similarly offering potential extensions to higher-order-in-time particle pushers. The FET formulation is used also to consider how the stochastic drift terms can be incorporated into the pushers. Stochastic gyrokinetic expansions are also discussed.

        Different options for the numerical implementation of these schemes are considered.

        Due to the efficacy of FET in the development of SP timesteppers for both the fluid and kinetic component, we hope this approach will prove effective in the future for developing SP timesteppers for the full hybrid model. We hope this will give us the opportunity to incorporate previously inaccessible kinetic effects into the highly effective, modern, finite-element MHD models.
    \end{abstract}
    
    
    \newpage
    \tableofcontents
    
    
    \newpage
    \pagenumbering{arabic}
    %\linenumbers\renewcommand\thelinenumber{\color{black!50}\arabic{linenumber}}
            \input{0 - introduction/main.tex}
        \part{Research}
            \input{1 - low-noise PiC models/main.tex}
            \input{2 - kinetic component/main.tex}
            \input{3 - fluid component/main.tex}
            \input{4 - numerical implementation/main.tex}
        \part{Project Overview}
            \input{5 - research plan/main.tex}
            \input{6 - summary/main.tex}
    
    
    %\section{}
    \newpage
    \pagenumbering{gobble}
        \printbibliography


    \newpage
    \pagenumbering{roman}
    \appendix
        \part{Appendices}
            \input{8 - Hilbert complexes/main.tex}
            \input{9 - weak conservation proofs/main.tex}
\end{document}

            \documentclass[12pt, a4paper]{report}

\input{template/main.tex}

\title{\BA{Title in Progress...}}
\author{Boris Andrews}
\affil{Mathematical Institute, University of Oxford}
\date{\today}


\begin{document}
    \pagenumbering{gobble}
    \maketitle
    
    
    \begin{abstract}
        Magnetic confinement reactors---in particular tokamaks---offer one of the most promising options for achieving practical nuclear fusion, with the potential to provide virtually limitless, clean energy. The theoretical and numerical modeling of tokamak plasmas is simultaneously an essential component of effective reactor design, and a great research barrier. Tokamak operational conditions exhibit comparatively low Knudsen numbers. Kinetic effects, including kinetic waves and instabilities, Landau damping, bump-on-tail instabilities and more, are therefore highly influential in tokamak plasma dynamics. Purely fluid models are inherently incapable of capturing these effects, whereas the high dimensionality in purely kinetic models render them practically intractable for most relevant purposes.

        We consider a $\delta\!f$ decomposition model, with a macroscopic fluid background and microscopic kinetic correction, both fully coupled to each other. A similar manner of discretization is proposed to that used in the recent \texttt{STRUPHY} code \cite{Holderied_Possanner_Wang_2021, Holderied_2022, Li_et_al_2023} with a finite-element model for the background and a pseudo-particle/PiC model for the correction.

        The fluid background satisfies the full, non-linear, resistive, compressible, Hall MHD equations. \cite{Laakmann_Hu_Farrell_2022} introduces finite-element(-in-space) implicit timesteppers for the incompressible analogue to this system with structure-preserving (SP) properties in the ideal case, alongside parameter-robust preconditioners. We show that these timesteppers can derive from a finite-element-in-time (FET) (and finite-element-in-space) interpretation. The benefits of this reformulation are discussed, including the derivation of timesteppers that are higher order in time, and the quantifiable dissipative SP properties in the non-ideal, resistive case.
        
        We discuss possible options for extending this FET approach to timesteppers for the compressible case.

        The kinetic corrections satisfy linearized Boltzmann equations. Using a Lénard--Bernstein collision operator, these take Fokker--Planck-like forms \cite{Fokker_1914, Planck_1917} wherein pseudo-particles in the numerical model obey the neoclassical transport equations, with particle-independent Brownian drift terms. This offers a rigorous methodology for incorporating collisions into the particle transport model, without coupling the equations of motions for each particle.
        
        Works by Chen, Chacón et al. \cite{Chen_Chacón_Barnes_2011, Chacón_Chen_Barnes_2013, Chen_Chacón_2014, Chen_Chacón_2015} have developed structure-preserving particle pushers for neoclassical transport in the Vlasov equations, derived from Crank--Nicolson integrators. We show these too can can derive from a FET interpretation, similarly offering potential extensions to higher-order-in-time particle pushers. The FET formulation is used also to consider how the stochastic drift terms can be incorporated into the pushers. Stochastic gyrokinetic expansions are also discussed.

        Different options for the numerical implementation of these schemes are considered.

        Due to the efficacy of FET in the development of SP timesteppers for both the fluid and kinetic component, we hope this approach will prove effective in the future for developing SP timesteppers for the full hybrid model. We hope this will give us the opportunity to incorporate previously inaccessible kinetic effects into the highly effective, modern, finite-element MHD models.
    \end{abstract}
    
    
    \newpage
    \tableofcontents
    
    
    \newpage
    \pagenumbering{arabic}
    %\linenumbers\renewcommand\thelinenumber{\color{black!50}\arabic{linenumber}}
            \input{0 - introduction/main.tex}
        \part{Research}
            \input{1 - low-noise PiC models/main.tex}
            \input{2 - kinetic component/main.tex}
            \input{3 - fluid component/main.tex}
            \input{4 - numerical implementation/main.tex}
        \part{Project Overview}
            \input{5 - research plan/main.tex}
            \input{6 - summary/main.tex}
    
    
    %\section{}
    \newpage
    \pagenumbering{gobble}
        \printbibliography


    \newpage
    \pagenumbering{roman}
    \appendix
        \part{Appendices}
            \input{8 - Hilbert complexes/main.tex}
            \input{9 - weak conservation proofs/main.tex}
\end{document}

\end{document}

    
    
    %\section{}
    \newpage
    \pagenumbering{gobble}
        \printbibliography


    \newpage
    \pagenumbering{roman}
    \appendix
        \part{Appendices}
            \documentclass[12pt, a4paper]{report}

\documentclass[12pt, a4paper]{report}

\input{template/main.tex}

\title{\BA{Title in Progress...}}
\author{Boris Andrews}
\affil{Mathematical Institute, University of Oxford}
\date{\today}


\begin{document}
    \pagenumbering{gobble}
    \maketitle
    
    
    \begin{abstract}
        Magnetic confinement reactors---in particular tokamaks---offer one of the most promising options for achieving practical nuclear fusion, with the potential to provide virtually limitless, clean energy. The theoretical and numerical modeling of tokamak plasmas is simultaneously an essential component of effective reactor design, and a great research barrier. Tokamak operational conditions exhibit comparatively low Knudsen numbers. Kinetic effects, including kinetic waves and instabilities, Landau damping, bump-on-tail instabilities and more, are therefore highly influential in tokamak plasma dynamics. Purely fluid models are inherently incapable of capturing these effects, whereas the high dimensionality in purely kinetic models render them practically intractable for most relevant purposes.

        We consider a $\delta\!f$ decomposition model, with a macroscopic fluid background and microscopic kinetic correction, both fully coupled to each other. A similar manner of discretization is proposed to that used in the recent \texttt{STRUPHY} code \cite{Holderied_Possanner_Wang_2021, Holderied_2022, Li_et_al_2023} with a finite-element model for the background and a pseudo-particle/PiC model for the correction.

        The fluid background satisfies the full, non-linear, resistive, compressible, Hall MHD equations. \cite{Laakmann_Hu_Farrell_2022} introduces finite-element(-in-space) implicit timesteppers for the incompressible analogue to this system with structure-preserving (SP) properties in the ideal case, alongside parameter-robust preconditioners. We show that these timesteppers can derive from a finite-element-in-time (FET) (and finite-element-in-space) interpretation. The benefits of this reformulation are discussed, including the derivation of timesteppers that are higher order in time, and the quantifiable dissipative SP properties in the non-ideal, resistive case.
        
        We discuss possible options for extending this FET approach to timesteppers for the compressible case.

        The kinetic corrections satisfy linearized Boltzmann equations. Using a Lénard--Bernstein collision operator, these take Fokker--Planck-like forms \cite{Fokker_1914, Planck_1917} wherein pseudo-particles in the numerical model obey the neoclassical transport equations, with particle-independent Brownian drift terms. This offers a rigorous methodology for incorporating collisions into the particle transport model, without coupling the equations of motions for each particle.
        
        Works by Chen, Chacón et al. \cite{Chen_Chacón_Barnes_2011, Chacón_Chen_Barnes_2013, Chen_Chacón_2014, Chen_Chacón_2015} have developed structure-preserving particle pushers for neoclassical transport in the Vlasov equations, derived from Crank--Nicolson integrators. We show these too can can derive from a FET interpretation, similarly offering potential extensions to higher-order-in-time particle pushers. The FET formulation is used also to consider how the stochastic drift terms can be incorporated into the pushers. Stochastic gyrokinetic expansions are also discussed.

        Different options for the numerical implementation of these schemes are considered.

        Due to the efficacy of FET in the development of SP timesteppers for both the fluid and kinetic component, we hope this approach will prove effective in the future for developing SP timesteppers for the full hybrid model. We hope this will give us the opportunity to incorporate previously inaccessible kinetic effects into the highly effective, modern, finite-element MHD models.
    \end{abstract}
    
    
    \newpage
    \tableofcontents
    
    
    \newpage
    \pagenumbering{arabic}
    %\linenumbers\renewcommand\thelinenumber{\color{black!50}\arabic{linenumber}}
            \input{0 - introduction/main.tex}
        \part{Research}
            \input{1 - low-noise PiC models/main.tex}
            \input{2 - kinetic component/main.tex}
            \input{3 - fluid component/main.tex}
            \input{4 - numerical implementation/main.tex}
        \part{Project Overview}
            \input{5 - research plan/main.tex}
            \input{6 - summary/main.tex}
    
    
    %\section{}
    \newpage
    \pagenumbering{gobble}
        \printbibliography


    \newpage
    \pagenumbering{roman}
    \appendix
        \part{Appendices}
            \input{8 - Hilbert complexes/main.tex}
            \input{9 - weak conservation proofs/main.tex}
\end{document}


\title{\BA{Title in Progress...}}
\author{Boris Andrews}
\affil{Mathematical Institute, University of Oxford}
\date{\today}


\begin{document}
    \pagenumbering{gobble}
    \maketitle
    
    
    \begin{abstract}
        Magnetic confinement reactors---in particular tokamaks---offer one of the most promising options for achieving practical nuclear fusion, with the potential to provide virtually limitless, clean energy. The theoretical and numerical modeling of tokamak plasmas is simultaneously an essential component of effective reactor design, and a great research barrier. Tokamak operational conditions exhibit comparatively low Knudsen numbers. Kinetic effects, including kinetic waves and instabilities, Landau damping, bump-on-tail instabilities and more, are therefore highly influential in tokamak plasma dynamics. Purely fluid models are inherently incapable of capturing these effects, whereas the high dimensionality in purely kinetic models render them practically intractable for most relevant purposes.

        We consider a $\delta\!f$ decomposition model, with a macroscopic fluid background and microscopic kinetic correction, both fully coupled to each other. A similar manner of discretization is proposed to that used in the recent \texttt{STRUPHY} code \cite{Holderied_Possanner_Wang_2021, Holderied_2022, Li_et_al_2023} with a finite-element model for the background and a pseudo-particle/PiC model for the correction.

        The fluid background satisfies the full, non-linear, resistive, compressible, Hall MHD equations. \cite{Laakmann_Hu_Farrell_2022} introduces finite-element(-in-space) implicit timesteppers for the incompressible analogue to this system with structure-preserving (SP) properties in the ideal case, alongside parameter-robust preconditioners. We show that these timesteppers can derive from a finite-element-in-time (FET) (and finite-element-in-space) interpretation. The benefits of this reformulation are discussed, including the derivation of timesteppers that are higher order in time, and the quantifiable dissipative SP properties in the non-ideal, resistive case.
        
        We discuss possible options for extending this FET approach to timesteppers for the compressible case.

        The kinetic corrections satisfy linearized Boltzmann equations. Using a Lénard--Bernstein collision operator, these take Fokker--Planck-like forms \cite{Fokker_1914, Planck_1917} wherein pseudo-particles in the numerical model obey the neoclassical transport equations, with particle-independent Brownian drift terms. This offers a rigorous methodology for incorporating collisions into the particle transport model, without coupling the equations of motions for each particle.
        
        Works by Chen, Chacón et al. \cite{Chen_Chacón_Barnes_2011, Chacón_Chen_Barnes_2013, Chen_Chacón_2014, Chen_Chacón_2015} have developed structure-preserving particle pushers for neoclassical transport in the Vlasov equations, derived from Crank--Nicolson integrators. We show these too can can derive from a FET interpretation, similarly offering potential extensions to higher-order-in-time particle pushers. The FET formulation is used also to consider how the stochastic drift terms can be incorporated into the pushers. Stochastic gyrokinetic expansions are also discussed.

        Different options for the numerical implementation of these schemes are considered.

        Due to the efficacy of FET in the development of SP timesteppers for both the fluid and kinetic component, we hope this approach will prove effective in the future for developing SP timesteppers for the full hybrid model. We hope this will give us the opportunity to incorporate previously inaccessible kinetic effects into the highly effective, modern, finite-element MHD models.
    \end{abstract}
    
    
    \newpage
    \tableofcontents
    
    
    \newpage
    \pagenumbering{arabic}
    %\linenumbers\renewcommand\thelinenumber{\color{black!50}\arabic{linenumber}}
            \documentclass[12pt, a4paper]{report}

\input{template/main.tex}

\title{\BA{Title in Progress...}}
\author{Boris Andrews}
\affil{Mathematical Institute, University of Oxford}
\date{\today}


\begin{document}
    \pagenumbering{gobble}
    \maketitle
    
    
    \begin{abstract}
        Magnetic confinement reactors---in particular tokamaks---offer one of the most promising options for achieving practical nuclear fusion, with the potential to provide virtually limitless, clean energy. The theoretical and numerical modeling of tokamak plasmas is simultaneously an essential component of effective reactor design, and a great research barrier. Tokamak operational conditions exhibit comparatively low Knudsen numbers. Kinetic effects, including kinetic waves and instabilities, Landau damping, bump-on-tail instabilities and more, are therefore highly influential in tokamak plasma dynamics. Purely fluid models are inherently incapable of capturing these effects, whereas the high dimensionality in purely kinetic models render them practically intractable for most relevant purposes.

        We consider a $\delta\!f$ decomposition model, with a macroscopic fluid background and microscopic kinetic correction, both fully coupled to each other. A similar manner of discretization is proposed to that used in the recent \texttt{STRUPHY} code \cite{Holderied_Possanner_Wang_2021, Holderied_2022, Li_et_al_2023} with a finite-element model for the background and a pseudo-particle/PiC model for the correction.

        The fluid background satisfies the full, non-linear, resistive, compressible, Hall MHD equations. \cite{Laakmann_Hu_Farrell_2022} introduces finite-element(-in-space) implicit timesteppers for the incompressible analogue to this system with structure-preserving (SP) properties in the ideal case, alongside parameter-robust preconditioners. We show that these timesteppers can derive from a finite-element-in-time (FET) (and finite-element-in-space) interpretation. The benefits of this reformulation are discussed, including the derivation of timesteppers that are higher order in time, and the quantifiable dissipative SP properties in the non-ideal, resistive case.
        
        We discuss possible options for extending this FET approach to timesteppers for the compressible case.

        The kinetic corrections satisfy linearized Boltzmann equations. Using a Lénard--Bernstein collision operator, these take Fokker--Planck-like forms \cite{Fokker_1914, Planck_1917} wherein pseudo-particles in the numerical model obey the neoclassical transport equations, with particle-independent Brownian drift terms. This offers a rigorous methodology for incorporating collisions into the particle transport model, without coupling the equations of motions for each particle.
        
        Works by Chen, Chacón et al. \cite{Chen_Chacón_Barnes_2011, Chacón_Chen_Barnes_2013, Chen_Chacón_2014, Chen_Chacón_2015} have developed structure-preserving particle pushers for neoclassical transport in the Vlasov equations, derived from Crank--Nicolson integrators. We show these too can can derive from a FET interpretation, similarly offering potential extensions to higher-order-in-time particle pushers. The FET formulation is used also to consider how the stochastic drift terms can be incorporated into the pushers. Stochastic gyrokinetic expansions are also discussed.

        Different options for the numerical implementation of these schemes are considered.

        Due to the efficacy of FET in the development of SP timesteppers for both the fluid and kinetic component, we hope this approach will prove effective in the future for developing SP timesteppers for the full hybrid model. We hope this will give us the opportunity to incorporate previously inaccessible kinetic effects into the highly effective, modern, finite-element MHD models.
    \end{abstract}
    
    
    \newpage
    \tableofcontents
    
    
    \newpage
    \pagenumbering{arabic}
    %\linenumbers\renewcommand\thelinenumber{\color{black!50}\arabic{linenumber}}
            \input{0 - introduction/main.tex}
        \part{Research}
            \input{1 - low-noise PiC models/main.tex}
            \input{2 - kinetic component/main.tex}
            \input{3 - fluid component/main.tex}
            \input{4 - numerical implementation/main.tex}
        \part{Project Overview}
            \input{5 - research plan/main.tex}
            \input{6 - summary/main.tex}
    
    
    %\section{}
    \newpage
    \pagenumbering{gobble}
        \printbibliography


    \newpage
    \pagenumbering{roman}
    \appendix
        \part{Appendices}
            \input{8 - Hilbert complexes/main.tex}
            \input{9 - weak conservation proofs/main.tex}
\end{document}

        \part{Research}
            \documentclass[12pt, a4paper]{report}

\input{template/main.tex}

\title{\BA{Title in Progress...}}
\author{Boris Andrews}
\affil{Mathematical Institute, University of Oxford}
\date{\today}


\begin{document}
    \pagenumbering{gobble}
    \maketitle
    
    
    \begin{abstract}
        Magnetic confinement reactors---in particular tokamaks---offer one of the most promising options for achieving practical nuclear fusion, with the potential to provide virtually limitless, clean energy. The theoretical and numerical modeling of tokamak plasmas is simultaneously an essential component of effective reactor design, and a great research barrier. Tokamak operational conditions exhibit comparatively low Knudsen numbers. Kinetic effects, including kinetic waves and instabilities, Landau damping, bump-on-tail instabilities and more, are therefore highly influential in tokamak plasma dynamics. Purely fluid models are inherently incapable of capturing these effects, whereas the high dimensionality in purely kinetic models render them practically intractable for most relevant purposes.

        We consider a $\delta\!f$ decomposition model, with a macroscopic fluid background and microscopic kinetic correction, both fully coupled to each other. A similar manner of discretization is proposed to that used in the recent \texttt{STRUPHY} code \cite{Holderied_Possanner_Wang_2021, Holderied_2022, Li_et_al_2023} with a finite-element model for the background and a pseudo-particle/PiC model for the correction.

        The fluid background satisfies the full, non-linear, resistive, compressible, Hall MHD equations. \cite{Laakmann_Hu_Farrell_2022} introduces finite-element(-in-space) implicit timesteppers for the incompressible analogue to this system with structure-preserving (SP) properties in the ideal case, alongside parameter-robust preconditioners. We show that these timesteppers can derive from a finite-element-in-time (FET) (and finite-element-in-space) interpretation. The benefits of this reformulation are discussed, including the derivation of timesteppers that are higher order in time, and the quantifiable dissipative SP properties in the non-ideal, resistive case.
        
        We discuss possible options for extending this FET approach to timesteppers for the compressible case.

        The kinetic corrections satisfy linearized Boltzmann equations. Using a Lénard--Bernstein collision operator, these take Fokker--Planck-like forms \cite{Fokker_1914, Planck_1917} wherein pseudo-particles in the numerical model obey the neoclassical transport equations, with particle-independent Brownian drift terms. This offers a rigorous methodology for incorporating collisions into the particle transport model, without coupling the equations of motions for each particle.
        
        Works by Chen, Chacón et al. \cite{Chen_Chacón_Barnes_2011, Chacón_Chen_Barnes_2013, Chen_Chacón_2014, Chen_Chacón_2015} have developed structure-preserving particle pushers for neoclassical transport in the Vlasov equations, derived from Crank--Nicolson integrators. We show these too can can derive from a FET interpretation, similarly offering potential extensions to higher-order-in-time particle pushers. The FET formulation is used also to consider how the stochastic drift terms can be incorporated into the pushers. Stochastic gyrokinetic expansions are also discussed.

        Different options for the numerical implementation of these schemes are considered.

        Due to the efficacy of FET in the development of SP timesteppers for both the fluid and kinetic component, we hope this approach will prove effective in the future for developing SP timesteppers for the full hybrid model. We hope this will give us the opportunity to incorporate previously inaccessible kinetic effects into the highly effective, modern, finite-element MHD models.
    \end{abstract}
    
    
    \newpage
    \tableofcontents
    
    
    \newpage
    \pagenumbering{arabic}
    %\linenumbers\renewcommand\thelinenumber{\color{black!50}\arabic{linenumber}}
            \input{0 - introduction/main.tex}
        \part{Research}
            \input{1 - low-noise PiC models/main.tex}
            \input{2 - kinetic component/main.tex}
            \input{3 - fluid component/main.tex}
            \input{4 - numerical implementation/main.tex}
        \part{Project Overview}
            \input{5 - research plan/main.tex}
            \input{6 - summary/main.tex}
    
    
    %\section{}
    \newpage
    \pagenumbering{gobble}
        \printbibliography


    \newpage
    \pagenumbering{roman}
    \appendix
        \part{Appendices}
            \input{8 - Hilbert complexes/main.tex}
            \input{9 - weak conservation proofs/main.tex}
\end{document}

            \documentclass[12pt, a4paper]{report}

\input{template/main.tex}

\title{\BA{Title in Progress...}}
\author{Boris Andrews}
\affil{Mathematical Institute, University of Oxford}
\date{\today}


\begin{document}
    \pagenumbering{gobble}
    \maketitle
    
    
    \begin{abstract}
        Magnetic confinement reactors---in particular tokamaks---offer one of the most promising options for achieving practical nuclear fusion, with the potential to provide virtually limitless, clean energy. The theoretical and numerical modeling of tokamak plasmas is simultaneously an essential component of effective reactor design, and a great research barrier. Tokamak operational conditions exhibit comparatively low Knudsen numbers. Kinetic effects, including kinetic waves and instabilities, Landau damping, bump-on-tail instabilities and more, are therefore highly influential in tokamak plasma dynamics. Purely fluid models are inherently incapable of capturing these effects, whereas the high dimensionality in purely kinetic models render them practically intractable for most relevant purposes.

        We consider a $\delta\!f$ decomposition model, with a macroscopic fluid background and microscopic kinetic correction, both fully coupled to each other. A similar manner of discretization is proposed to that used in the recent \texttt{STRUPHY} code \cite{Holderied_Possanner_Wang_2021, Holderied_2022, Li_et_al_2023} with a finite-element model for the background and a pseudo-particle/PiC model for the correction.

        The fluid background satisfies the full, non-linear, resistive, compressible, Hall MHD equations. \cite{Laakmann_Hu_Farrell_2022} introduces finite-element(-in-space) implicit timesteppers for the incompressible analogue to this system with structure-preserving (SP) properties in the ideal case, alongside parameter-robust preconditioners. We show that these timesteppers can derive from a finite-element-in-time (FET) (and finite-element-in-space) interpretation. The benefits of this reformulation are discussed, including the derivation of timesteppers that are higher order in time, and the quantifiable dissipative SP properties in the non-ideal, resistive case.
        
        We discuss possible options for extending this FET approach to timesteppers for the compressible case.

        The kinetic corrections satisfy linearized Boltzmann equations. Using a Lénard--Bernstein collision operator, these take Fokker--Planck-like forms \cite{Fokker_1914, Planck_1917} wherein pseudo-particles in the numerical model obey the neoclassical transport equations, with particle-independent Brownian drift terms. This offers a rigorous methodology for incorporating collisions into the particle transport model, without coupling the equations of motions for each particle.
        
        Works by Chen, Chacón et al. \cite{Chen_Chacón_Barnes_2011, Chacón_Chen_Barnes_2013, Chen_Chacón_2014, Chen_Chacón_2015} have developed structure-preserving particle pushers for neoclassical transport in the Vlasov equations, derived from Crank--Nicolson integrators. We show these too can can derive from a FET interpretation, similarly offering potential extensions to higher-order-in-time particle pushers. The FET formulation is used also to consider how the stochastic drift terms can be incorporated into the pushers. Stochastic gyrokinetic expansions are also discussed.

        Different options for the numerical implementation of these schemes are considered.

        Due to the efficacy of FET in the development of SP timesteppers for both the fluid and kinetic component, we hope this approach will prove effective in the future for developing SP timesteppers for the full hybrid model. We hope this will give us the opportunity to incorporate previously inaccessible kinetic effects into the highly effective, modern, finite-element MHD models.
    \end{abstract}
    
    
    \newpage
    \tableofcontents
    
    
    \newpage
    \pagenumbering{arabic}
    %\linenumbers\renewcommand\thelinenumber{\color{black!50}\arabic{linenumber}}
            \input{0 - introduction/main.tex}
        \part{Research}
            \input{1 - low-noise PiC models/main.tex}
            \input{2 - kinetic component/main.tex}
            \input{3 - fluid component/main.tex}
            \input{4 - numerical implementation/main.tex}
        \part{Project Overview}
            \input{5 - research plan/main.tex}
            \input{6 - summary/main.tex}
    
    
    %\section{}
    \newpage
    \pagenumbering{gobble}
        \printbibliography


    \newpage
    \pagenumbering{roman}
    \appendix
        \part{Appendices}
            \input{8 - Hilbert complexes/main.tex}
            \input{9 - weak conservation proofs/main.tex}
\end{document}

            \documentclass[12pt, a4paper]{report}

\input{template/main.tex}

\title{\BA{Title in Progress...}}
\author{Boris Andrews}
\affil{Mathematical Institute, University of Oxford}
\date{\today}


\begin{document}
    \pagenumbering{gobble}
    \maketitle
    
    
    \begin{abstract}
        Magnetic confinement reactors---in particular tokamaks---offer one of the most promising options for achieving practical nuclear fusion, with the potential to provide virtually limitless, clean energy. The theoretical and numerical modeling of tokamak plasmas is simultaneously an essential component of effective reactor design, and a great research barrier. Tokamak operational conditions exhibit comparatively low Knudsen numbers. Kinetic effects, including kinetic waves and instabilities, Landau damping, bump-on-tail instabilities and more, are therefore highly influential in tokamak plasma dynamics. Purely fluid models are inherently incapable of capturing these effects, whereas the high dimensionality in purely kinetic models render them practically intractable for most relevant purposes.

        We consider a $\delta\!f$ decomposition model, with a macroscopic fluid background and microscopic kinetic correction, both fully coupled to each other. A similar manner of discretization is proposed to that used in the recent \texttt{STRUPHY} code \cite{Holderied_Possanner_Wang_2021, Holderied_2022, Li_et_al_2023} with a finite-element model for the background and a pseudo-particle/PiC model for the correction.

        The fluid background satisfies the full, non-linear, resistive, compressible, Hall MHD equations. \cite{Laakmann_Hu_Farrell_2022} introduces finite-element(-in-space) implicit timesteppers for the incompressible analogue to this system with structure-preserving (SP) properties in the ideal case, alongside parameter-robust preconditioners. We show that these timesteppers can derive from a finite-element-in-time (FET) (and finite-element-in-space) interpretation. The benefits of this reformulation are discussed, including the derivation of timesteppers that are higher order in time, and the quantifiable dissipative SP properties in the non-ideal, resistive case.
        
        We discuss possible options for extending this FET approach to timesteppers for the compressible case.

        The kinetic corrections satisfy linearized Boltzmann equations. Using a Lénard--Bernstein collision operator, these take Fokker--Planck-like forms \cite{Fokker_1914, Planck_1917} wherein pseudo-particles in the numerical model obey the neoclassical transport equations, with particle-independent Brownian drift terms. This offers a rigorous methodology for incorporating collisions into the particle transport model, without coupling the equations of motions for each particle.
        
        Works by Chen, Chacón et al. \cite{Chen_Chacón_Barnes_2011, Chacón_Chen_Barnes_2013, Chen_Chacón_2014, Chen_Chacón_2015} have developed structure-preserving particle pushers for neoclassical transport in the Vlasov equations, derived from Crank--Nicolson integrators. We show these too can can derive from a FET interpretation, similarly offering potential extensions to higher-order-in-time particle pushers. The FET formulation is used also to consider how the stochastic drift terms can be incorporated into the pushers. Stochastic gyrokinetic expansions are also discussed.

        Different options for the numerical implementation of these schemes are considered.

        Due to the efficacy of FET in the development of SP timesteppers for both the fluid and kinetic component, we hope this approach will prove effective in the future for developing SP timesteppers for the full hybrid model. We hope this will give us the opportunity to incorporate previously inaccessible kinetic effects into the highly effective, modern, finite-element MHD models.
    \end{abstract}
    
    
    \newpage
    \tableofcontents
    
    
    \newpage
    \pagenumbering{arabic}
    %\linenumbers\renewcommand\thelinenumber{\color{black!50}\arabic{linenumber}}
            \input{0 - introduction/main.tex}
        \part{Research}
            \input{1 - low-noise PiC models/main.tex}
            \input{2 - kinetic component/main.tex}
            \input{3 - fluid component/main.tex}
            \input{4 - numerical implementation/main.tex}
        \part{Project Overview}
            \input{5 - research plan/main.tex}
            \input{6 - summary/main.tex}
    
    
    %\section{}
    \newpage
    \pagenumbering{gobble}
        \printbibliography


    \newpage
    \pagenumbering{roman}
    \appendix
        \part{Appendices}
            \input{8 - Hilbert complexes/main.tex}
            \input{9 - weak conservation proofs/main.tex}
\end{document}

            \documentclass[12pt, a4paper]{report}

\input{template/main.tex}

\title{\BA{Title in Progress...}}
\author{Boris Andrews}
\affil{Mathematical Institute, University of Oxford}
\date{\today}


\begin{document}
    \pagenumbering{gobble}
    \maketitle
    
    
    \begin{abstract}
        Magnetic confinement reactors---in particular tokamaks---offer one of the most promising options for achieving practical nuclear fusion, with the potential to provide virtually limitless, clean energy. The theoretical and numerical modeling of tokamak plasmas is simultaneously an essential component of effective reactor design, and a great research barrier. Tokamak operational conditions exhibit comparatively low Knudsen numbers. Kinetic effects, including kinetic waves and instabilities, Landau damping, bump-on-tail instabilities and more, are therefore highly influential in tokamak plasma dynamics. Purely fluid models are inherently incapable of capturing these effects, whereas the high dimensionality in purely kinetic models render them practically intractable for most relevant purposes.

        We consider a $\delta\!f$ decomposition model, with a macroscopic fluid background and microscopic kinetic correction, both fully coupled to each other. A similar manner of discretization is proposed to that used in the recent \texttt{STRUPHY} code \cite{Holderied_Possanner_Wang_2021, Holderied_2022, Li_et_al_2023} with a finite-element model for the background and a pseudo-particle/PiC model for the correction.

        The fluid background satisfies the full, non-linear, resistive, compressible, Hall MHD equations. \cite{Laakmann_Hu_Farrell_2022} introduces finite-element(-in-space) implicit timesteppers for the incompressible analogue to this system with structure-preserving (SP) properties in the ideal case, alongside parameter-robust preconditioners. We show that these timesteppers can derive from a finite-element-in-time (FET) (and finite-element-in-space) interpretation. The benefits of this reformulation are discussed, including the derivation of timesteppers that are higher order in time, and the quantifiable dissipative SP properties in the non-ideal, resistive case.
        
        We discuss possible options for extending this FET approach to timesteppers for the compressible case.

        The kinetic corrections satisfy linearized Boltzmann equations. Using a Lénard--Bernstein collision operator, these take Fokker--Planck-like forms \cite{Fokker_1914, Planck_1917} wherein pseudo-particles in the numerical model obey the neoclassical transport equations, with particle-independent Brownian drift terms. This offers a rigorous methodology for incorporating collisions into the particle transport model, without coupling the equations of motions for each particle.
        
        Works by Chen, Chacón et al. \cite{Chen_Chacón_Barnes_2011, Chacón_Chen_Barnes_2013, Chen_Chacón_2014, Chen_Chacón_2015} have developed structure-preserving particle pushers for neoclassical transport in the Vlasov equations, derived from Crank--Nicolson integrators. We show these too can can derive from a FET interpretation, similarly offering potential extensions to higher-order-in-time particle pushers. The FET formulation is used also to consider how the stochastic drift terms can be incorporated into the pushers. Stochastic gyrokinetic expansions are also discussed.

        Different options for the numerical implementation of these schemes are considered.

        Due to the efficacy of FET in the development of SP timesteppers for both the fluid and kinetic component, we hope this approach will prove effective in the future for developing SP timesteppers for the full hybrid model. We hope this will give us the opportunity to incorporate previously inaccessible kinetic effects into the highly effective, modern, finite-element MHD models.
    \end{abstract}
    
    
    \newpage
    \tableofcontents
    
    
    \newpage
    \pagenumbering{arabic}
    %\linenumbers\renewcommand\thelinenumber{\color{black!50}\arabic{linenumber}}
            \input{0 - introduction/main.tex}
        \part{Research}
            \input{1 - low-noise PiC models/main.tex}
            \input{2 - kinetic component/main.tex}
            \input{3 - fluid component/main.tex}
            \input{4 - numerical implementation/main.tex}
        \part{Project Overview}
            \input{5 - research plan/main.tex}
            \input{6 - summary/main.tex}
    
    
    %\section{}
    \newpage
    \pagenumbering{gobble}
        \printbibliography


    \newpage
    \pagenumbering{roman}
    \appendix
        \part{Appendices}
            \input{8 - Hilbert complexes/main.tex}
            \input{9 - weak conservation proofs/main.tex}
\end{document}

        \part{Project Overview}
            \documentclass[12pt, a4paper]{report}

\input{template/main.tex}

\title{\BA{Title in Progress...}}
\author{Boris Andrews}
\affil{Mathematical Institute, University of Oxford}
\date{\today}


\begin{document}
    \pagenumbering{gobble}
    \maketitle
    
    
    \begin{abstract}
        Magnetic confinement reactors---in particular tokamaks---offer one of the most promising options for achieving practical nuclear fusion, with the potential to provide virtually limitless, clean energy. The theoretical and numerical modeling of tokamak plasmas is simultaneously an essential component of effective reactor design, and a great research barrier. Tokamak operational conditions exhibit comparatively low Knudsen numbers. Kinetic effects, including kinetic waves and instabilities, Landau damping, bump-on-tail instabilities and more, are therefore highly influential in tokamak plasma dynamics. Purely fluid models are inherently incapable of capturing these effects, whereas the high dimensionality in purely kinetic models render them practically intractable for most relevant purposes.

        We consider a $\delta\!f$ decomposition model, with a macroscopic fluid background and microscopic kinetic correction, both fully coupled to each other. A similar manner of discretization is proposed to that used in the recent \texttt{STRUPHY} code \cite{Holderied_Possanner_Wang_2021, Holderied_2022, Li_et_al_2023} with a finite-element model for the background and a pseudo-particle/PiC model for the correction.

        The fluid background satisfies the full, non-linear, resistive, compressible, Hall MHD equations. \cite{Laakmann_Hu_Farrell_2022} introduces finite-element(-in-space) implicit timesteppers for the incompressible analogue to this system with structure-preserving (SP) properties in the ideal case, alongside parameter-robust preconditioners. We show that these timesteppers can derive from a finite-element-in-time (FET) (and finite-element-in-space) interpretation. The benefits of this reformulation are discussed, including the derivation of timesteppers that are higher order in time, and the quantifiable dissipative SP properties in the non-ideal, resistive case.
        
        We discuss possible options for extending this FET approach to timesteppers for the compressible case.

        The kinetic corrections satisfy linearized Boltzmann equations. Using a Lénard--Bernstein collision operator, these take Fokker--Planck-like forms \cite{Fokker_1914, Planck_1917} wherein pseudo-particles in the numerical model obey the neoclassical transport equations, with particle-independent Brownian drift terms. This offers a rigorous methodology for incorporating collisions into the particle transport model, without coupling the equations of motions for each particle.
        
        Works by Chen, Chacón et al. \cite{Chen_Chacón_Barnes_2011, Chacón_Chen_Barnes_2013, Chen_Chacón_2014, Chen_Chacón_2015} have developed structure-preserving particle pushers for neoclassical transport in the Vlasov equations, derived from Crank--Nicolson integrators. We show these too can can derive from a FET interpretation, similarly offering potential extensions to higher-order-in-time particle pushers. The FET formulation is used also to consider how the stochastic drift terms can be incorporated into the pushers. Stochastic gyrokinetic expansions are also discussed.

        Different options for the numerical implementation of these schemes are considered.

        Due to the efficacy of FET in the development of SP timesteppers for both the fluid and kinetic component, we hope this approach will prove effective in the future for developing SP timesteppers for the full hybrid model. We hope this will give us the opportunity to incorporate previously inaccessible kinetic effects into the highly effective, modern, finite-element MHD models.
    \end{abstract}
    
    
    \newpage
    \tableofcontents
    
    
    \newpage
    \pagenumbering{arabic}
    %\linenumbers\renewcommand\thelinenumber{\color{black!50}\arabic{linenumber}}
            \input{0 - introduction/main.tex}
        \part{Research}
            \input{1 - low-noise PiC models/main.tex}
            \input{2 - kinetic component/main.tex}
            \input{3 - fluid component/main.tex}
            \input{4 - numerical implementation/main.tex}
        \part{Project Overview}
            \input{5 - research plan/main.tex}
            \input{6 - summary/main.tex}
    
    
    %\section{}
    \newpage
    \pagenumbering{gobble}
        \printbibliography


    \newpage
    \pagenumbering{roman}
    \appendix
        \part{Appendices}
            \input{8 - Hilbert complexes/main.tex}
            \input{9 - weak conservation proofs/main.tex}
\end{document}

            \documentclass[12pt, a4paper]{report}

\input{template/main.tex}

\title{\BA{Title in Progress...}}
\author{Boris Andrews}
\affil{Mathematical Institute, University of Oxford}
\date{\today}


\begin{document}
    \pagenumbering{gobble}
    \maketitle
    
    
    \begin{abstract}
        Magnetic confinement reactors---in particular tokamaks---offer one of the most promising options for achieving practical nuclear fusion, with the potential to provide virtually limitless, clean energy. The theoretical and numerical modeling of tokamak plasmas is simultaneously an essential component of effective reactor design, and a great research barrier. Tokamak operational conditions exhibit comparatively low Knudsen numbers. Kinetic effects, including kinetic waves and instabilities, Landau damping, bump-on-tail instabilities and more, are therefore highly influential in tokamak plasma dynamics. Purely fluid models are inherently incapable of capturing these effects, whereas the high dimensionality in purely kinetic models render them practically intractable for most relevant purposes.

        We consider a $\delta\!f$ decomposition model, with a macroscopic fluid background and microscopic kinetic correction, both fully coupled to each other. A similar manner of discretization is proposed to that used in the recent \texttt{STRUPHY} code \cite{Holderied_Possanner_Wang_2021, Holderied_2022, Li_et_al_2023} with a finite-element model for the background and a pseudo-particle/PiC model for the correction.

        The fluid background satisfies the full, non-linear, resistive, compressible, Hall MHD equations. \cite{Laakmann_Hu_Farrell_2022} introduces finite-element(-in-space) implicit timesteppers for the incompressible analogue to this system with structure-preserving (SP) properties in the ideal case, alongside parameter-robust preconditioners. We show that these timesteppers can derive from a finite-element-in-time (FET) (and finite-element-in-space) interpretation. The benefits of this reformulation are discussed, including the derivation of timesteppers that are higher order in time, and the quantifiable dissipative SP properties in the non-ideal, resistive case.
        
        We discuss possible options for extending this FET approach to timesteppers for the compressible case.

        The kinetic corrections satisfy linearized Boltzmann equations. Using a Lénard--Bernstein collision operator, these take Fokker--Planck-like forms \cite{Fokker_1914, Planck_1917} wherein pseudo-particles in the numerical model obey the neoclassical transport equations, with particle-independent Brownian drift terms. This offers a rigorous methodology for incorporating collisions into the particle transport model, without coupling the equations of motions for each particle.
        
        Works by Chen, Chacón et al. \cite{Chen_Chacón_Barnes_2011, Chacón_Chen_Barnes_2013, Chen_Chacón_2014, Chen_Chacón_2015} have developed structure-preserving particle pushers for neoclassical transport in the Vlasov equations, derived from Crank--Nicolson integrators. We show these too can can derive from a FET interpretation, similarly offering potential extensions to higher-order-in-time particle pushers. The FET formulation is used also to consider how the stochastic drift terms can be incorporated into the pushers. Stochastic gyrokinetic expansions are also discussed.

        Different options for the numerical implementation of these schemes are considered.

        Due to the efficacy of FET in the development of SP timesteppers for both the fluid and kinetic component, we hope this approach will prove effective in the future for developing SP timesteppers for the full hybrid model. We hope this will give us the opportunity to incorporate previously inaccessible kinetic effects into the highly effective, modern, finite-element MHD models.
    \end{abstract}
    
    
    \newpage
    \tableofcontents
    
    
    \newpage
    \pagenumbering{arabic}
    %\linenumbers\renewcommand\thelinenumber{\color{black!50}\arabic{linenumber}}
            \input{0 - introduction/main.tex}
        \part{Research}
            \input{1 - low-noise PiC models/main.tex}
            \input{2 - kinetic component/main.tex}
            \input{3 - fluid component/main.tex}
            \input{4 - numerical implementation/main.tex}
        \part{Project Overview}
            \input{5 - research plan/main.tex}
            \input{6 - summary/main.tex}
    
    
    %\section{}
    \newpage
    \pagenumbering{gobble}
        \printbibliography


    \newpage
    \pagenumbering{roman}
    \appendix
        \part{Appendices}
            \input{8 - Hilbert complexes/main.tex}
            \input{9 - weak conservation proofs/main.tex}
\end{document}

    
    
    %\section{}
    \newpage
    \pagenumbering{gobble}
        \printbibliography


    \newpage
    \pagenumbering{roman}
    \appendix
        \part{Appendices}
            \documentclass[12pt, a4paper]{report}

\input{template/main.tex}

\title{\BA{Title in Progress...}}
\author{Boris Andrews}
\affil{Mathematical Institute, University of Oxford}
\date{\today}


\begin{document}
    \pagenumbering{gobble}
    \maketitle
    
    
    \begin{abstract}
        Magnetic confinement reactors---in particular tokamaks---offer one of the most promising options for achieving practical nuclear fusion, with the potential to provide virtually limitless, clean energy. The theoretical and numerical modeling of tokamak plasmas is simultaneously an essential component of effective reactor design, and a great research barrier. Tokamak operational conditions exhibit comparatively low Knudsen numbers. Kinetic effects, including kinetic waves and instabilities, Landau damping, bump-on-tail instabilities and more, are therefore highly influential in tokamak plasma dynamics. Purely fluid models are inherently incapable of capturing these effects, whereas the high dimensionality in purely kinetic models render them practically intractable for most relevant purposes.

        We consider a $\delta\!f$ decomposition model, with a macroscopic fluid background and microscopic kinetic correction, both fully coupled to each other. A similar manner of discretization is proposed to that used in the recent \texttt{STRUPHY} code \cite{Holderied_Possanner_Wang_2021, Holderied_2022, Li_et_al_2023} with a finite-element model for the background and a pseudo-particle/PiC model for the correction.

        The fluid background satisfies the full, non-linear, resistive, compressible, Hall MHD equations. \cite{Laakmann_Hu_Farrell_2022} introduces finite-element(-in-space) implicit timesteppers for the incompressible analogue to this system with structure-preserving (SP) properties in the ideal case, alongside parameter-robust preconditioners. We show that these timesteppers can derive from a finite-element-in-time (FET) (and finite-element-in-space) interpretation. The benefits of this reformulation are discussed, including the derivation of timesteppers that are higher order in time, and the quantifiable dissipative SP properties in the non-ideal, resistive case.
        
        We discuss possible options for extending this FET approach to timesteppers for the compressible case.

        The kinetic corrections satisfy linearized Boltzmann equations. Using a Lénard--Bernstein collision operator, these take Fokker--Planck-like forms \cite{Fokker_1914, Planck_1917} wherein pseudo-particles in the numerical model obey the neoclassical transport equations, with particle-independent Brownian drift terms. This offers a rigorous methodology for incorporating collisions into the particle transport model, without coupling the equations of motions for each particle.
        
        Works by Chen, Chacón et al. \cite{Chen_Chacón_Barnes_2011, Chacón_Chen_Barnes_2013, Chen_Chacón_2014, Chen_Chacón_2015} have developed structure-preserving particle pushers for neoclassical transport in the Vlasov equations, derived from Crank--Nicolson integrators. We show these too can can derive from a FET interpretation, similarly offering potential extensions to higher-order-in-time particle pushers. The FET formulation is used also to consider how the stochastic drift terms can be incorporated into the pushers. Stochastic gyrokinetic expansions are also discussed.

        Different options for the numerical implementation of these schemes are considered.

        Due to the efficacy of FET in the development of SP timesteppers for both the fluid and kinetic component, we hope this approach will prove effective in the future for developing SP timesteppers for the full hybrid model. We hope this will give us the opportunity to incorporate previously inaccessible kinetic effects into the highly effective, modern, finite-element MHD models.
    \end{abstract}
    
    
    \newpage
    \tableofcontents
    
    
    \newpage
    \pagenumbering{arabic}
    %\linenumbers\renewcommand\thelinenumber{\color{black!50}\arabic{linenumber}}
            \input{0 - introduction/main.tex}
        \part{Research}
            \input{1 - low-noise PiC models/main.tex}
            \input{2 - kinetic component/main.tex}
            \input{3 - fluid component/main.tex}
            \input{4 - numerical implementation/main.tex}
        \part{Project Overview}
            \input{5 - research plan/main.tex}
            \input{6 - summary/main.tex}
    
    
    %\section{}
    \newpage
    \pagenumbering{gobble}
        \printbibliography


    \newpage
    \pagenumbering{roman}
    \appendix
        \part{Appendices}
            \input{8 - Hilbert complexes/main.tex}
            \input{9 - weak conservation proofs/main.tex}
\end{document}

            \documentclass[12pt, a4paper]{report}

\input{template/main.tex}

\title{\BA{Title in Progress...}}
\author{Boris Andrews}
\affil{Mathematical Institute, University of Oxford}
\date{\today}


\begin{document}
    \pagenumbering{gobble}
    \maketitle
    
    
    \begin{abstract}
        Magnetic confinement reactors---in particular tokamaks---offer one of the most promising options for achieving practical nuclear fusion, with the potential to provide virtually limitless, clean energy. The theoretical and numerical modeling of tokamak plasmas is simultaneously an essential component of effective reactor design, and a great research barrier. Tokamak operational conditions exhibit comparatively low Knudsen numbers. Kinetic effects, including kinetic waves and instabilities, Landau damping, bump-on-tail instabilities and more, are therefore highly influential in tokamak plasma dynamics. Purely fluid models are inherently incapable of capturing these effects, whereas the high dimensionality in purely kinetic models render them practically intractable for most relevant purposes.

        We consider a $\delta\!f$ decomposition model, with a macroscopic fluid background and microscopic kinetic correction, both fully coupled to each other. A similar manner of discretization is proposed to that used in the recent \texttt{STRUPHY} code \cite{Holderied_Possanner_Wang_2021, Holderied_2022, Li_et_al_2023} with a finite-element model for the background and a pseudo-particle/PiC model for the correction.

        The fluid background satisfies the full, non-linear, resistive, compressible, Hall MHD equations. \cite{Laakmann_Hu_Farrell_2022} introduces finite-element(-in-space) implicit timesteppers for the incompressible analogue to this system with structure-preserving (SP) properties in the ideal case, alongside parameter-robust preconditioners. We show that these timesteppers can derive from a finite-element-in-time (FET) (and finite-element-in-space) interpretation. The benefits of this reformulation are discussed, including the derivation of timesteppers that are higher order in time, and the quantifiable dissipative SP properties in the non-ideal, resistive case.
        
        We discuss possible options for extending this FET approach to timesteppers for the compressible case.

        The kinetic corrections satisfy linearized Boltzmann equations. Using a Lénard--Bernstein collision operator, these take Fokker--Planck-like forms \cite{Fokker_1914, Planck_1917} wherein pseudo-particles in the numerical model obey the neoclassical transport equations, with particle-independent Brownian drift terms. This offers a rigorous methodology for incorporating collisions into the particle transport model, without coupling the equations of motions for each particle.
        
        Works by Chen, Chacón et al. \cite{Chen_Chacón_Barnes_2011, Chacón_Chen_Barnes_2013, Chen_Chacón_2014, Chen_Chacón_2015} have developed structure-preserving particle pushers for neoclassical transport in the Vlasov equations, derived from Crank--Nicolson integrators. We show these too can can derive from a FET interpretation, similarly offering potential extensions to higher-order-in-time particle pushers. The FET formulation is used also to consider how the stochastic drift terms can be incorporated into the pushers. Stochastic gyrokinetic expansions are also discussed.

        Different options for the numerical implementation of these schemes are considered.

        Due to the efficacy of FET in the development of SP timesteppers for both the fluid and kinetic component, we hope this approach will prove effective in the future for developing SP timesteppers for the full hybrid model. We hope this will give us the opportunity to incorporate previously inaccessible kinetic effects into the highly effective, modern, finite-element MHD models.
    \end{abstract}
    
    
    \newpage
    \tableofcontents
    
    
    \newpage
    \pagenumbering{arabic}
    %\linenumbers\renewcommand\thelinenumber{\color{black!50}\arabic{linenumber}}
            \input{0 - introduction/main.tex}
        \part{Research}
            \input{1 - low-noise PiC models/main.tex}
            \input{2 - kinetic component/main.tex}
            \input{3 - fluid component/main.tex}
            \input{4 - numerical implementation/main.tex}
        \part{Project Overview}
            \input{5 - research plan/main.tex}
            \input{6 - summary/main.tex}
    
    
    %\section{}
    \newpage
    \pagenumbering{gobble}
        \printbibliography


    \newpage
    \pagenumbering{roman}
    \appendix
        \part{Appendices}
            \input{8 - Hilbert complexes/main.tex}
            \input{9 - weak conservation proofs/main.tex}
\end{document}

\end{document}

            \documentclass[12pt, a4paper]{report}

\documentclass[12pt, a4paper]{report}

\input{template/main.tex}

\title{\BA{Title in Progress...}}
\author{Boris Andrews}
\affil{Mathematical Institute, University of Oxford}
\date{\today}


\begin{document}
    \pagenumbering{gobble}
    \maketitle
    
    
    \begin{abstract}
        Magnetic confinement reactors---in particular tokamaks---offer one of the most promising options for achieving practical nuclear fusion, with the potential to provide virtually limitless, clean energy. The theoretical and numerical modeling of tokamak plasmas is simultaneously an essential component of effective reactor design, and a great research barrier. Tokamak operational conditions exhibit comparatively low Knudsen numbers. Kinetic effects, including kinetic waves and instabilities, Landau damping, bump-on-tail instabilities and more, are therefore highly influential in tokamak plasma dynamics. Purely fluid models are inherently incapable of capturing these effects, whereas the high dimensionality in purely kinetic models render them practically intractable for most relevant purposes.

        We consider a $\delta\!f$ decomposition model, with a macroscopic fluid background and microscopic kinetic correction, both fully coupled to each other. A similar manner of discretization is proposed to that used in the recent \texttt{STRUPHY} code \cite{Holderied_Possanner_Wang_2021, Holderied_2022, Li_et_al_2023} with a finite-element model for the background and a pseudo-particle/PiC model for the correction.

        The fluid background satisfies the full, non-linear, resistive, compressible, Hall MHD equations. \cite{Laakmann_Hu_Farrell_2022} introduces finite-element(-in-space) implicit timesteppers for the incompressible analogue to this system with structure-preserving (SP) properties in the ideal case, alongside parameter-robust preconditioners. We show that these timesteppers can derive from a finite-element-in-time (FET) (and finite-element-in-space) interpretation. The benefits of this reformulation are discussed, including the derivation of timesteppers that are higher order in time, and the quantifiable dissipative SP properties in the non-ideal, resistive case.
        
        We discuss possible options for extending this FET approach to timesteppers for the compressible case.

        The kinetic corrections satisfy linearized Boltzmann equations. Using a Lénard--Bernstein collision operator, these take Fokker--Planck-like forms \cite{Fokker_1914, Planck_1917} wherein pseudo-particles in the numerical model obey the neoclassical transport equations, with particle-independent Brownian drift terms. This offers a rigorous methodology for incorporating collisions into the particle transport model, without coupling the equations of motions for each particle.
        
        Works by Chen, Chacón et al. \cite{Chen_Chacón_Barnes_2011, Chacón_Chen_Barnes_2013, Chen_Chacón_2014, Chen_Chacón_2015} have developed structure-preserving particle pushers for neoclassical transport in the Vlasov equations, derived from Crank--Nicolson integrators. We show these too can can derive from a FET interpretation, similarly offering potential extensions to higher-order-in-time particle pushers. The FET formulation is used also to consider how the stochastic drift terms can be incorporated into the pushers. Stochastic gyrokinetic expansions are also discussed.

        Different options for the numerical implementation of these schemes are considered.

        Due to the efficacy of FET in the development of SP timesteppers for both the fluid and kinetic component, we hope this approach will prove effective in the future for developing SP timesteppers for the full hybrid model. We hope this will give us the opportunity to incorporate previously inaccessible kinetic effects into the highly effective, modern, finite-element MHD models.
    \end{abstract}
    
    
    \newpage
    \tableofcontents
    
    
    \newpage
    \pagenumbering{arabic}
    %\linenumbers\renewcommand\thelinenumber{\color{black!50}\arabic{linenumber}}
            \input{0 - introduction/main.tex}
        \part{Research}
            \input{1 - low-noise PiC models/main.tex}
            \input{2 - kinetic component/main.tex}
            \input{3 - fluid component/main.tex}
            \input{4 - numerical implementation/main.tex}
        \part{Project Overview}
            \input{5 - research plan/main.tex}
            \input{6 - summary/main.tex}
    
    
    %\section{}
    \newpage
    \pagenumbering{gobble}
        \printbibliography


    \newpage
    \pagenumbering{roman}
    \appendix
        \part{Appendices}
            \input{8 - Hilbert complexes/main.tex}
            \input{9 - weak conservation proofs/main.tex}
\end{document}


\title{\BA{Title in Progress...}}
\author{Boris Andrews}
\affil{Mathematical Institute, University of Oxford}
\date{\today}


\begin{document}
    \pagenumbering{gobble}
    \maketitle
    
    
    \begin{abstract}
        Magnetic confinement reactors---in particular tokamaks---offer one of the most promising options for achieving practical nuclear fusion, with the potential to provide virtually limitless, clean energy. The theoretical and numerical modeling of tokamak plasmas is simultaneously an essential component of effective reactor design, and a great research barrier. Tokamak operational conditions exhibit comparatively low Knudsen numbers. Kinetic effects, including kinetic waves and instabilities, Landau damping, bump-on-tail instabilities and more, are therefore highly influential in tokamak plasma dynamics. Purely fluid models are inherently incapable of capturing these effects, whereas the high dimensionality in purely kinetic models render them practically intractable for most relevant purposes.

        We consider a $\delta\!f$ decomposition model, with a macroscopic fluid background and microscopic kinetic correction, both fully coupled to each other. A similar manner of discretization is proposed to that used in the recent \texttt{STRUPHY} code \cite{Holderied_Possanner_Wang_2021, Holderied_2022, Li_et_al_2023} with a finite-element model for the background and a pseudo-particle/PiC model for the correction.

        The fluid background satisfies the full, non-linear, resistive, compressible, Hall MHD equations. \cite{Laakmann_Hu_Farrell_2022} introduces finite-element(-in-space) implicit timesteppers for the incompressible analogue to this system with structure-preserving (SP) properties in the ideal case, alongside parameter-robust preconditioners. We show that these timesteppers can derive from a finite-element-in-time (FET) (and finite-element-in-space) interpretation. The benefits of this reformulation are discussed, including the derivation of timesteppers that are higher order in time, and the quantifiable dissipative SP properties in the non-ideal, resistive case.
        
        We discuss possible options for extending this FET approach to timesteppers for the compressible case.

        The kinetic corrections satisfy linearized Boltzmann equations. Using a Lénard--Bernstein collision operator, these take Fokker--Planck-like forms \cite{Fokker_1914, Planck_1917} wherein pseudo-particles in the numerical model obey the neoclassical transport equations, with particle-independent Brownian drift terms. This offers a rigorous methodology for incorporating collisions into the particle transport model, without coupling the equations of motions for each particle.
        
        Works by Chen, Chacón et al. \cite{Chen_Chacón_Barnes_2011, Chacón_Chen_Barnes_2013, Chen_Chacón_2014, Chen_Chacón_2015} have developed structure-preserving particle pushers for neoclassical transport in the Vlasov equations, derived from Crank--Nicolson integrators. We show these too can can derive from a FET interpretation, similarly offering potential extensions to higher-order-in-time particle pushers. The FET formulation is used also to consider how the stochastic drift terms can be incorporated into the pushers. Stochastic gyrokinetic expansions are also discussed.

        Different options for the numerical implementation of these schemes are considered.

        Due to the efficacy of FET in the development of SP timesteppers for both the fluid and kinetic component, we hope this approach will prove effective in the future for developing SP timesteppers for the full hybrid model. We hope this will give us the opportunity to incorporate previously inaccessible kinetic effects into the highly effective, modern, finite-element MHD models.
    \end{abstract}
    
    
    \newpage
    \tableofcontents
    
    
    \newpage
    \pagenumbering{arabic}
    %\linenumbers\renewcommand\thelinenumber{\color{black!50}\arabic{linenumber}}
            \documentclass[12pt, a4paper]{report}

\input{template/main.tex}

\title{\BA{Title in Progress...}}
\author{Boris Andrews}
\affil{Mathematical Institute, University of Oxford}
\date{\today}


\begin{document}
    \pagenumbering{gobble}
    \maketitle
    
    
    \begin{abstract}
        Magnetic confinement reactors---in particular tokamaks---offer one of the most promising options for achieving practical nuclear fusion, with the potential to provide virtually limitless, clean energy. The theoretical and numerical modeling of tokamak plasmas is simultaneously an essential component of effective reactor design, and a great research barrier. Tokamak operational conditions exhibit comparatively low Knudsen numbers. Kinetic effects, including kinetic waves and instabilities, Landau damping, bump-on-tail instabilities and more, are therefore highly influential in tokamak plasma dynamics. Purely fluid models are inherently incapable of capturing these effects, whereas the high dimensionality in purely kinetic models render them practically intractable for most relevant purposes.

        We consider a $\delta\!f$ decomposition model, with a macroscopic fluid background and microscopic kinetic correction, both fully coupled to each other. A similar manner of discretization is proposed to that used in the recent \texttt{STRUPHY} code \cite{Holderied_Possanner_Wang_2021, Holderied_2022, Li_et_al_2023} with a finite-element model for the background and a pseudo-particle/PiC model for the correction.

        The fluid background satisfies the full, non-linear, resistive, compressible, Hall MHD equations. \cite{Laakmann_Hu_Farrell_2022} introduces finite-element(-in-space) implicit timesteppers for the incompressible analogue to this system with structure-preserving (SP) properties in the ideal case, alongside parameter-robust preconditioners. We show that these timesteppers can derive from a finite-element-in-time (FET) (and finite-element-in-space) interpretation. The benefits of this reformulation are discussed, including the derivation of timesteppers that are higher order in time, and the quantifiable dissipative SP properties in the non-ideal, resistive case.
        
        We discuss possible options for extending this FET approach to timesteppers for the compressible case.

        The kinetic corrections satisfy linearized Boltzmann equations. Using a Lénard--Bernstein collision operator, these take Fokker--Planck-like forms \cite{Fokker_1914, Planck_1917} wherein pseudo-particles in the numerical model obey the neoclassical transport equations, with particle-independent Brownian drift terms. This offers a rigorous methodology for incorporating collisions into the particle transport model, without coupling the equations of motions for each particle.
        
        Works by Chen, Chacón et al. \cite{Chen_Chacón_Barnes_2011, Chacón_Chen_Barnes_2013, Chen_Chacón_2014, Chen_Chacón_2015} have developed structure-preserving particle pushers for neoclassical transport in the Vlasov equations, derived from Crank--Nicolson integrators. We show these too can can derive from a FET interpretation, similarly offering potential extensions to higher-order-in-time particle pushers. The FET formulation is used also to consider how the stochastic drift terms can be incorporated into the pushers. Stochastic gyrokinetic expansions are also discussed.

        Different options for the numerical implementation of these schemes are considered.

        Due to the efficacy of FET in the development of SP timesteppers for both the fluid and kinetic component, we hope this approach will prove effective in the future for developing SP timesteppers for the full hybrid model. We hope this will give us the opportunity to incorporate previously inaccessible kinetic effects into the highly effective, modern, finite-element MHD models.
    \end{abstract}
    
    
    \newpage
    \tableofcontents
    
    
    \newpage
    \pagenumbering{arabic}
    %\linenumbers\renewcommand\thelinenumber{\color{black!50}\arabic{linenumber}}
            \input{0 - introduction/main.tex}
        \part{Research}
            \input{1 - low-noise PiC models/main.tex}
            \input{2 - kinetic component/main.tex}
            \input{3 - fluid component/main.tex}
            \input{4 - numerical implementation/main.tex}
        \part{Project Overview}
            \input{5 - research plan/main.tex}
            \input{6 - summary/main.tex}
    
    
    %\section{}
    \newpage
    \pagenumbering{gobble}
        \printbibliography


    \newpage
    \pagenumbering{roman}
    \appendix
        \part{Appendices}
            \input{8 - Hilbert complexes/main.tex}
            \input{9 - weak conservation proofs/main.tex}
\end{document}

        \part{Research}
            \documentclass[12pt, a4paper]{report}

\input{template/main.tex}

\title{\BA{Title in Progress...}}
\author{Boris Andrews}
\affil{Mathematical Institute, University of Oxford}
\date{\today}


\begin{document}
    \pagenumbering{gobble}
    \maketitle
    
    
    \begin{abstract}
        Magnetic confinement reactors---in particular tokamaks---offer one of the most promising options for achieving practical nuclear fusion, with the potential to provide virtually limitless, clean energy. The theoretical and numerical modeling of tokamak plasmas is simultaneously an essential component of effective reactor design, and a great research barrier. Tokamak operational conditions exhibit comparatively low Knudsen numbers. Kinetic effects, including kinetic waves and instabilities, Landau damping, bump-on-tail instabilities and more, are therefore highly influential in tokamak plasma dynamics. Purely fluid models are inherently incapable of capturing these effects, whereas the high dimensionality in purely kinetic models render them practically intractable for most relevant purposes.

        We consider a $\delta\!f$ decomposition model, with a macroscopic fluid background and microscopic kinetic correction, both fully coupled to each other. A similar manner of discretization is proposed to that used in the recent \texttt{STRUPHY} code \cite{Holderied_Possanner_Wang_2021, Holderied_2022, Li_et_al_2023} with a finite-element model for the background and a pseudo-particle/PiC model for the correction.

        The fluid background satisfies the full, non-linear, resistive, compressible, Hall MHD equations. \cite{Laakmann_Hu_Farrell_2022} introduces finite-element(-in-space) implicit timesteppers for the incompressible analogue to this system with structure-preserving (SP) properties in the ideal case, alongside parameter-robust preconditioners. We show that these timesteppers can derive from a finite-element-in-time (FET) (and finite-element-in-space) interpretation. The benefits of this reformulation are discussed, including the derivation of timesteppers that are higher order in time, and the quantifiable dissipative SP properties in the non-ideal, resistive case.
        
        We discuss possible options for extending this FET approach to timesteppers for the compressible case.

        The kinetic corrections satisfy linearized Boltzmann equations. Using a Lénard--Bernstein collision operator, these take Fokker--Planck-like forms \cite{Fokker_1914, Planck_1917} wherein pseudo-particles in the numerical model obey the neoclassical transport equations, with particle-independent Brownian drift terms. This offers a rigorous methodology for incorporating collisions into the particle transport model, without coupling the equations of motions for each particle.
        
        Works by Chen, Chacón et al. \cite{Chen_Chacón_Barnes_2011, Chacón_Chen_Barnes_2013, Chen_Chacón_2014, Chen_Chacón_2015} have developed structure-preserving particle pushers for neoclassical transport in the Vlasov equations, derived from Crank--Nicolson integrators. We show these too can can derive from a FET interpretation, similarly offering potential extensions to higher-order-in-time particle pushers. The FET formulation is used also to consider how the stochastic drift terms can be incorporated into the pushers. Stochastic gyrokinetic expansions are also discussed.

        Different options for the numerical implementation of these schemes are considered.

        Due to the efficacy of FET in the development of SP timesteppers for both the fluid and kinetic component, we hope this approach will prove effective in the future for developing SP timesteppers for the full hybrid model. We hope this will give us the opportunity to incorporate previously inaccessible kinetic effects into the highly effective, modern, finite-element MHD models.
    \end{abstract}
    
    
    \newpage
    \tableofcontents
    
    
    \newpage
    \pagenumbering{arabic}
    %\linenumbers\renewcommand\thelinenumber{\color{black!50}\arabic{linenumber}}
            \input{0 - introduction/main.tex}
        \part{Research}
            \input{1 - low-noise PiC models/main.tex}
            \input{2 - kinetic component/main.tex}
            \input{3 - fluid component/main.tex}
            \input{4 - numerical implementation/main.tex}
        \part{Project Overview}
            \input{5 - research plan/main.tex}
            \input{6 - summary/main.tex}
    
    
    %\section{}
    \newpage
    \pagenumbering{gobble}
        \printbibliography


    \newpage
    \pagenumbering{roman}
    \appendix
        \part{Appendices}
            \input{8 - Hilbert complexes/main.tex}
            \input{9 - weak conservation proofs/main.tex}
\end{document}

            \documentclass[12pt, a4paper]{report}

\input{template/main.tex}

\title{\BA{Title in Progress...}}
\author{Boris Andrews}
\affil{Mathematical Institute, University of Oxford}
\date{\today}


\begin{document}
    \pagenumbering{gobble}
    \maketitle
    
    
    \begin{abstract}
        Magnetic confinement reactors---in particular tokamaks---offer one of the most promising options for achieving practical nuclear fusion, with the potential to provide virtually limitless, clean energy. The theoretical and numerical modeling of tokamak plasmas is simultaneously an essential component of effective reactor design, and a great research barrier. Tokamak operational conditions exhibit comparatively low Knudsen numbers. Kinetic effects, including kinetic waves and instabilities, Landau damping, bump-on-tail instabilities and more, are therefore highly influential in tokamak plasma dynamics. Purely fluid models are inherently incapable of capturing these effects, whereas the high dimensionality in purely kinetic models render them practically intractable for most relevant purposes.

        We consider a $\delta\!f$ decomposition model, with a macroscopic fluid background and microscopic kinetic correction, both fully coupled to each other. A similar manner of discretization is proposed to that used in the recent \texttt{STRUPHY} code \cite{Holderied_Possanner_Wang_2021, Holderied_2022, Li_et_al_2023} with a finite-element model for the background and a pseudo-particle/PiC model for the correction.

        The fluid background satisfies the full, non-linear, resistive, compressible, Hall MHD equations. \cite{Laakmann_Hu_Farrell_2022} introduces finite-element(-in-space) implicit timesteppers for the incompressible analogue to this system with structure-preserving (SP) properties in the ideal case, alongside parameter-robust preconditioners. We show that these timesteppers can derive from a finite-element-in-time (FET) (and finite-element-in-space) interpretation. The benefits of this reformulation are discussed, including the derivation of timesteppers that are higher order in time, and the quantifiable dissipative SP properties in the non-ideal, resistive case.
        
        We discuss possible options for extending this FET approach to timesteppers for the compressible case.

        The kinetic corrections satisfy linearized Boltzmann equations. Using a Lénard--Bernstein collision operator, these take Fokker--Planck-like forms \cite{Fokker_1914, Planck_1917} wherein pseudo-particles in the numerical model obey the neoclassical transport equations, with particle-independent Brownian drift terms. This offers a rigorous methodology for incorporating collisions into the particle transport model, without coupling the equations of motions for each particle.
        
        Works by Chen, Chacón et al. \cite{Chen_Chacón_Barnes_2011, Chacón_Chen_Barnes_2013, Chen_Chacón_2014, Chen_Chacón_2015} have developed structure-preserving particle pushers for neoclassical transport in the Vlasov equations, derived from Crank--Nicolson integrators. We show these too can can derive from a FET interpretation, similarly offering potential extensions to higher-order-in-time particle pushers. The FET formulation is used also to consider how the stochastic drift terms can be incorporated into the pushers. Stochastic gyrokinetic expansions are also discussed.

        Different options for the numerical implementation of these schemes are considered.

        Due to the efficacy of FET in the development of SP timesteppers for both the fluid and kinetic component, we hope this approach will prove effective in the future for developing SP timesteppers for the full hybrid model. We hope this will give us the opportunity to incorporate previously inaccessible kinetic effects into the highly effective, modern, finite-element MHD models.
    \end{abstract}
    
    
    \newpage
    \tableofcontents
    
    
    \newpage
    \pagenumbering{arabic}
    %\linenumbers\renewcommand\thelinenumber{\color{black!50}\arabic{linenumber}}
            \input{0 - introduction/main.tex}
        \part{Research}
            \input{1 - low-noise PiC models/main.tex}
            \input{2 - kinetic component/main.tex}
            \input{3 - fluid component/main.tex}
            \input{4 - numerical implementation/main.tex}
        \part{Project Overview}
            \input{5 - research plan/main.tex}
            \input{6 - summary/main.tex}
    
    
    %\section{}
    \newpage
    \pagenumbering{gobble}
        \printbibliography


    \newpage
    \pagenumbering{roman}
    \appendix
        \part{Appendices}
            \input{8 - Hilbert complexes/main.tex}
            \input{9 - weak conservation proofs/main.tex}
\end{document}

            \documentclass[12pt, a4paper]{report}

\input{template/main.tex}

\title{\BA{Title in Progress...}}
\author{Boris Andrews}
\affil{Mathematical Institute, University of Oxford}
\date{\today}


\begin{document}
    \pagenumbering{gobble}
    \maketitle
    
    
    \begin{abstract}
        Magnetic confinement reactors---in particular tokamaks---offer one of the most promising options for achieving practical nuclear fusion, with the potential to provide virtually limitless, clean energy. The theoretical and numerical modeling of tokamak plasmas is simultaneously an essential component of effective reactor design, and a great research barrier. Tokamak operational conditions exhibit comparatively low Knudsen numbers. Kinetic effects, including kinetic waves and instabilities, Landau damping, bump-on-tail instabilities and more, are therefore highly influential in tokamak plasma dynamics. Purely fluid models are inherently incapable of capturing these effects, whereas the high dimensionality in purely kinetic models render them practically intractable for most relevant purposes.

        We consider a $\delta\!f$ decomposition model, with a macroscopic fluid background and microscopic kinetic correction, both fully coupled to each other. A similar manner of discretization is proposed to that used in the recent \texttt{STRUPHY} code \cite{Holderied_Possanner_Wang_2021, Holderied_2022, Li_et_al_2023} with a finite-element model for the background and a pseudo-particle/PiC model for the correction.

        The fluid background satisfies the full, non-linear, resistive, compressible, Hall MHD equations. \cite{Laakmann_Hu_Farrell_2022} introduces finite-element(-in-space) implicit timesteppers for the incompressible analogue to this system with structure-preserving (SP) properties in the ideal case, alongside parameter-robust preconditioners. We show that these timesteppers can derive from a finite-element-in-time (FET) (and finite-element-in-space) interpretation. The benefits of this reformulation are discussed, including the derivation of timesteppers that are higher order in time, and the quantifiable dissipative SP properties in the non-ideal, resistive case.
        
        We discuss possible options for extending this FET approach to timesteppers for the compressible case.

        The kinetic corrections satisfy linearized Boltzmann equations. Using a Lénard--Bernstein collision operator, these take Fokker--Planck-like forms \cite{Fokker_1914, Planck_1917} wherein pseudo-particles in the numerical model obey the neoclassical transport equations, with particle-independent Brownian drift terms. This offers a rigorous methodology for incorporating collisions into the particle transport model, without coupling the equations of motions for each particle.
        
        Works by Chen, Chacón et al. \cite{Chen_Chacón_Barnes_2011, Chacón_Chen_Barnes_2013, Chen_Chacón_2014, Chen_Chacón_2015} have developed structure-preserving particle pushers for neoclassical transport in the Vlasov equations, derived from Crank--Nicolson integrators. We show these too can can derive from a FET interpretation, similarly offering potential extensions to higher-order-in-time particle pushers. The FET formulation is used also to consider how the stochastic drift terms can be incorporated into the pushers. Stochastic gyrokinetic expansions are also discussed.

        Different options for the numerical implementation of these schemes are considered.

        Due to the efficacy of FET in the development of SP timesteppers for both the fluid and kinetic component, we hope this approach will prove effective in the future for developing SP timesteppers for the full hybrid model. We hope this will give us the opportunity to incorporate previously inaccessible kinetic effects into the highly effective, modern, finite-element MHD models.
    \end{abstract}
    
    
    \newpage
    \tableofcontents
    
    
    \newpage
    \pagenumbering{arabic}
    %\linenumbers\renewcommand\thelinenumber{\color{black!50}\arabic{linenumber}}
            \input{0 - introduction/main.tex}
        \part{Research}
            \input{1 - low-noise PiC models/main.tex}
            \input{2 - kinetic component/main.tex}
            \input{3 - fluid component/main.tex}
            \input{4 - numerical implementation/main.tex}
        \part{Project Overview}
            \input{5 - research plan/main.tex}
            \input{6 - summary/main.tex}
    
    
    %\section{}
    \newpage
    \pagenumbering{gobble}
        \printbibliography


    \newpage
    \pagenumbering{roman}
    \appendix
        \part{Appendices}
            \input{8 - Hilbert complexes/main.tex}
            \input{9 - weak conservation proofs/main.tex}
\end{document}

            \documentclass[12pt, a4paper]{report}

\input{template/main.tex}

\title{\BA{Title in Progress...}}
\author{Boris Andrews}
\affil{Mathematical Institute, University of Oxford}
\date{\today}


\begin{document}
    \pagenumbering{gobble}
    \maketitle
    
    
    \begin{abstract}
        Magnetic confinement reactors---in particular tokamaks---offer one of the most promising options for achieving practical nuclear fusion, with the potential to provide virtually limitless, clean energy. The theoretical and numerical modeling of tokamak plasmas is simultaneously an essential component of effective reactor design, and a great research barrier. Tokamak operational conditions exhibit comparatively low Knudsen numbers. Kinetic effects, including kinetic waves and instabilities, Landau damping, bump-on-tail instabilities and more, are therefore highly influential in tokamak plasma dynamics. Purely fluid models are inherently incapable of capturing these effects, whereas the high dimensionality in purely kinetic models render them practically intractable for most relevant purposes.

        We consider a $\delta\!f$ decomposition model, with a macroscopic fluid background and microscopic kinetic correction, both fully coupled to each other. A similar manner of discretization is proposed to that used in the recent \texttt{STRUPHY} code \cite{Holderied_Possanner_Wang_2021, Holderied_2022, Li_et_al_2023} with a finite-element model for the background and a pseudo-particle/PiC model for the correction.

        The fluid background satisfies the full, non-linear, resistive, compressible, Hall MHD equations. \cite{Laakmann_Hu_Farrell_2022} introduces finite-element(-in-space) implicit timesteppers for the incompressible analogue to this system with structure-preserving (SP) properties in the ideal case, alongside parameter-robust preconditioners. We show that these timesteppers can derive from a finite-element-in-time (FET) (and finite-element-in-space) interpretation. The benefits of this reformulation are discussed, including the derivation of timesteppers that are higher order in time, and the quantifiable dissipative SP properties in the non-ideal, resistive case.
        
        We discuss possible options for extending this FET approach to timesteppers for the compressible case.

        The kinetic corrections satisfy linearized Boltzmann equations. Using a Lénard--Bernstein collision operator, these take Fokker--Planck-like forms \cite{Fokker_1914, Planck_1917} wherein pseudo-particles in the numerical model obey the neoclassical transport equations, with particle-independent Brownian drift terms. This offers a rigorous methodology for incorporating collisions into the particle transport model, without coupling the equations of motions for each particle.
        
        Works by Chen, Chacón et al. \cite{Chen_Chacón_Barnes_2011, Chacón_Chen_Barnes_2013, Chen_Chacón_2014, Chen_Chacón_2015} have developed structure-preserving particle pushers for neoclassical transport in the Vlasov equations, derived from Crank--Nicolson integrators. We show these too can can derive from a FET interpretation, similarly offering potential extensions to higher-order-in-time particle pushers. The FET formulation is used also to consider how the stochastic drift terms can be incorporated into the pushers. Stochastic gyrokinetic expansions are also discussed.

        Different options for the numerical implementation of these schemes are considered.

        Due to the efficacy of FET in the development of SP timesteppers for both the fluid and kinetic component, we hope this approach will prove effective in the future for developing SP timesteppers for the full hybrid model. We hope this will give us the opportunity to incorporate previously inaccessible kinetic effects into the highly effective, modern, finite-element MHD models.
    \end{abstract}
    
    
    \newpage
    \tableofcontents
    
    
    \newpage
    \pagenumbering{arabic}
    %\linenumbers\renewcommand\thelinenumber{\color{black!50}\arabic{linenumber}}
            \input{0 - introduction/main.tex}
        \part{Research}
            \input{1 - low-noise PiC models/main.tex}
            \input{2 - kinetic component/main.tex}
            \input{3 - fluid component/main.tex}
            \input{4 - numerical implementation/main.tex}
        \part{Project Overview}
            \input{5 - research plan/main.tex}
            \input{6 - summary/main.tex}
    
    
    %\section{}
    \newpage
    \pagenumbering{gobble}
        \printbibliography


    \newpage
    \pagenumbering{roman}
    \appendix
        \part{Appendices}
            \input{8 - Hilbert complexes/main.tex}
            \input{9 - weak conservation proofs/main.tex}
\end{document}

        \part{Project Overview}
            \documentclass[12pt, a4paper]{report}

\input{template/main.tex}

\title{\BA{Title in Progress...}}
\author{Boris Andrews}
\affil{Mathematical Institute, University of Oxford}
\date{\today}


\begin{document}
    \pagenumbering{gobble}
    \maketitle
    
    
    \begin{abstract}
        Magnetic confinement reactors---in particular tokamaks---offer one of the most promising options for achieving practical nuclear fusion, with the potential to provide virtually limitless, clean energy. The theoretical and numerical modeling of tokamak plasmas is simultaneously an essential component of effective reactor design, and a great research barrier. Tokamak operational conditions exhibit comparatively low Knudsen numbers. Kinetic effects, including kinetic waves and instabilities, Landau damping, bump-on-tail instabilities and more, are therefore highly influential in tokamak plasma dynamics. Purely fluid models are inherently incapable of capturing these effects, whereas the high dimensionality in purely kinetic models render them practically intractable for most relevant purposes.

        We consider a $\delta\!f$ decomposition model, with a macroscopic fluid background and microscopic kinetic correction, both fully coupled to each other. A similar manner of discretization is proposed to that used in the recent \texttt{STRUPHY} code \cite{Holderied_Possanner_Wang_2021, Holderied_2022, Li_et_al_2023} with a finite-element model for the background and a pseudo-particle/PiC model for the correction.

        The fluid background satisfies the full, non-linear, resistive, compressible, Hall MHD equations. \cite{Laakmann_Hu_Farrell_2022} introduces finite-element(-in-space) implicit timesteppers for the incompressible analogue to this system with structure-preserving (SP) properties in the ideal case, alongside parameter-robust preconditioners. We show that these timesteppers can derive from a finite-element-in-time (FET) (and finite-element-in-space) interpretation. The benefits of this reformulation are discussed, including the derivation of timesteppers that are higher order in time, and the quantifiable dissipative SP properties in the non-ideal, resistive case.
        
        We discuss possible options for extending this FET approach to timesteppers for the compressible case.

        The kinetic corrections satisfy linearized Boltzmann equations. Using a Lénard--Bernstein collision operator, these take Fokker--Planck-like forms \cite{Fokker_1914, Planck_1917} wherein pseudo-particles in the numerical model obey the neoclassical transport equations, with particle-independent Brownian drift terms. This offers a rigorous methodology for incorporating collisions into the particle transport model, without coupling the equations of motions for each particle.
        
        Works by Chen, Chacón et al. \cite{Chen_Chacón_Barnes_2011, Chacón_Chen_Barnes_2013, Chen_Chacón_2014, Chen_Chacón_2015} have developed structure-preserving particle pushers for neoclassical transport in the Vlasov equations, derived from Crank--Nicolson integrators. We show these too can can derive from a FET interpretation, similarly offering potential extensions to higher-order-in-time particle pushers. The FET formulation is used also to consider how the stochastic drift terms can be incorporated into the pushers. Stochastic gyrokinetic expansions are also discussed.

        Different options for the numerical implementation of these schemes are considered.

        Due to the efficacy of FET in the development of SP timesteppers for both the fluid and kinetic component, we hope this approach will prove effective in the future for developing SP timesteppers for the full hybrid model. We hope this will give us the opportunity to incorporate previously inaccessible kinetic effects into the highly effective, modern, finite-element MHD models.
    \end{abstract}
    
    
    \newpage
    \tableofcontents
    
    
    \newpage
    \pagenumbering{arabic}
    %\linenumbers\renewcommand\thelinenumber{\color{black!50}\arabic{linenumber}}
            \input{0 - introduction/main.tex}
        \part{Research}
            \input{1 - low-noise PiC models/main.tex}
            \input{2 - kinetic component/main.tex}
            \input{3 - fluid component/main.tex}
            \input{4 - numerical implementation/main.tex}
        \part{Project Overview}
            \input{5 - research plan/main.tex}
            \input{6 - summary/main.tex}
    
    
    %\section{}
    \newpage
    \pagenumbering{gobble}
        \printbibliography


    \newpage
    \pagenumbering{roman}
    \appendix
        \part{Appendices}
            \input{8 - Hilbert complexes/main.tex}
            \input{9 - weak conservation proofs/main.tex}
\end{document}

            \documentclass[12pt, a4paper]{report}

\input{template/main.tex}

\title{\BA{Title in Progress...}}
\author{Boris Andrews}
\affil{Mathematical Institute, University of Oxford}
\date{\today}


\begin{document}
    \pagenumbering{gobble}
    \maketitle
    
    
    \begin{abstract}
        Magnetic confinement reactors---in particular tokamaks---offer one of the most promising options for achieving practical nuclear fusion, with the potential to provide virtually limitless, clean energy. The theoretical and numerical modeling of tokamak plasmas is simultaneously an essential component of effective reactor design, and a great research barrier. Tokamak operational conditions exhibit comparatively low Knudsen numbers. Kinetic effects, including kinetic waves and instabilities, Landau damping, bump-on-tail instabilities and more, are therefore highly influential in tokamak plasma dynamics. Purely fluid models are inherently incapable of capturing these effects, whereas the high dimensionality in purely kinetic models render them practically intractable for most relevant purposes.

        We consider a $\delta\!f$ decomposition model, with a macroscopic fluid background and microscopic kinetic correction, both fully coupled to each other. A similar manner of discretization is proposed to that used in the recent \texttt{STRUPHY} code \cite{Holderied_Possanner_Wang_2021, Holderied_2022, Li_et_al_2023} with a finite-element model for the background and a pseudo-particle/PiC model for the correction.

        The fluid background satisfies the full, non-linear, resistive, compressible, Hall MHD equations. \cite{Laakmann_Hu_Farrell_2022} introduces finite-element(-in-space) implicit timesteppers for the incompressible analogue to this system with structure-preserving (SP) properties in the ideal case, alongside parameter-robust preconditioners. We show that these timesteppers can derive from a finite-element-in-time (FET) (and finite-element-in-space) interpretation. The benefits of this reformulation are discussed, including the derivation of timesteppers that are higher order in time, and the quantifiable dissipative SP properties in the non-ideal, resistive case.
        
        We discuss possible options for extending this FET approach to timesteppers for the compressible case.

        The kinetic corrections satisfy linearized Boltzmann equations. Using a Lénard--Bernstein collision operator, these take Fokker--Planck-like forms \cite{Fokker_1914, Planck_1917} wherein pseudo-particles in the numerical model obey the neoclassical transport equations, with particle-independent Brownian drift terms. This offers a rigorous methodology for incorporating collisions into the particle transport model, without coupling the equations of motions for each particle.
        
        Works by Chen, Chacón et al. \cite{Chen_Chacón_Barnes_2011, Chacón_Chen_Barnes_2013, Chen_Chacón_2014, Chen_Chacón_2015} have developed structure-preserving particle pushers for neoclassical transport in the Vlasov equations, derived from Crank--Nicolson integrators. We show these too can can derive from a FET interpretation, similarly offering potential extensions to higher-order-in-time particle pushers. The FET formulation is used also to consider how the stochastic drift terms can be incorporated into the pushers. Stochastic gyrokinetic expansions are also discussed.

        Different options for the numerical implementation of these schemes are considered.

        Due to the efficacy of FET in the development of SP timesteppers for both the fluid and kinetic component, we hope this approach will prove effective in the future for developing SP timesteppers for the full hybrid model. We hope this will give us the opportunity to incorporate previously inaccessible kinetic effects into the highly effective, modern, finite-element MHD models.
    \end{abstract}
    
    
    \newpage
    \tableofcontents
    
    
    \newpage
    \pagenumbering{arabic}
    %\linenumbers\renewcommand\thelinenumber{\color{black!50}\arabic{linenumber}}
            \input{0 - introduction/main.tex}
        \part{Research}
            \input{1 - low-noise PiC models/main.tex}
            \input{2 - kinetic component/main.tex}
            \input{3 - fluid component/main.tex}
            \input{4 - numerical implementation/main.tex}
        \part{Project Overview}
            \input{5 - research plan/main.tex}
            \input{6 - summary/main.tex}
    
    
    %\section{}
    \newpage
    \pagenumbering{gobble}
        \printbibliography


    \newpage
    \pagenumbering{roman}
    \appendix
        \part{Appendices}
            \input{8 - Hilbert complexes/main.tex}
            \input{9 - weak conservation proofs/main.tex}
\end{document}

    
    
    %\section{}
    \newpage
    \pagenumbering{gobble}
        \printbibliography


    \newpage
    \pagenumbering{roman}
    \appendix
        \part{Appendices}
            \documentclass[12pt, a4paper]{report}

\input{template/main.tex}

\title{\BA{Title in Progress...}}
\author{Boris Andrews}
\affil{Mathematical Institute, University of Oxford}
\date{\today}


\begin{document}
    \pagenumbering{gobble}
    \maketitle
    
    
    \begin{abstract}
        Magnetic confinement reactors---in particular tokamaks---offer one of the most promising options for achieving practical nuclear fusion, with the potential to provide virtually limitless, clean energy. The theoretical and numerical modeling of tokamak plasmas is simultaneously an essential component of effective reactor design, and a great research barrier. Tokamak operational conditions exhibit comparatively low Knudsen numbers. Kinetic effects, including kinetic waves and instabilities, Landau damping, bump-on-tail instabilities and more, are therefore highly influential in tokamak plasma dynamics. Purely fluid models are inherently incapable of capturing these effects, whereas the high dimensionality in purely kinetic models render them practically intractable for most relevant purposes.

        We consider a $\delta\!f$ decomposition model, with a macroscopic fluid background and microscopic kinetic correction, both fully coupled to each other. A similar manner of discretization is proposed to that used in the recent \texttt{STRUPHY} code \cite{Holderied_Possanner_Wang_2021, Holderied_2022, Li_et_al_2023} with a finite-element model for the background and a pseudo-particle/PiC model for the correction.

        The fluid background satisfies the full, non-linear, resistive, compressible, Hall MHD equations. \cite{Laakmann_Hu_Farrell_2022} introduces finite-element(-in-space) implicit timesteppers for the incompressible analogue to this system with structure-preserving (SP) properties in the ideal case, alongside parameter-robust preconditioners. We show that these timesteppers can derive from a finite-element-in-time (FET) (and finite-element-in-space) interpretation. The benefits of this reformulation are discussed, including the derivation of timesteppers that are higher order in time, and the quantifiable dissipative SP properties in the non-ideal, resistive case.
        
        We discuss possible options for extending this FET approach to timesteppers for the compressible case.

        The kinetic corrections satisfy linearized Boltzmann equations. Using a Lénard--Bernstein collision operator, these take Fokker--Planck-like forms \cite{Fokker_1914, Planck_1917} wherein pseudo-particles in the numerical model obey the neoclassical transport equations, with particle-independent Brownian drift terms. This offers a rigorous methodology for incorporating collisions into the particle transport model, without coupling the equations of motions for each particle.
        
        Works by Chen, Chacón et al. \cite{Chen_Chacón_Barnes_2011, Chacón_Chen_Barnes_2013, Chen_Chacón_2014, Chen_Chacón_2015} have developed structure-preserving particle pushers for neoclassical transport in the Vlasov equations, derived from Crank--Nicolson integrators. We show these too can can derive from a FET interpretation, similarly offering potential extensions to higher-order-in-time particle pushers. The FET formulation is used also to consider how the stochastic drift terms can be incorporated into the pushers. Stochastic gyrokinetic expansions are also discussed.

        Different options for the numerical implementation of these schemes are considered.

        Due to the efficacy of FET in the development of SP timesteppers for both the fluid and kinetic component, we hope this approach will prove effective in the future for developing SP timesteppers for the full hybrid model. We hope this will give us the opportunity to incorporate previously inaccessible kinetic effects into the highly effective, modern, finite-element MHD models.
    \end{abstract}
    
    
    \newpage
    \tableofcontents
    
    
    \newpage
    \pagenumbering{arabic}
    %\linenumbers\renewcommand\thelinenumber{\color{black!50}\arabic{linenumber}}
            \input{0 - introduction/main.tex}
        \part{Research}
            \input{1 - low-noise PiC models/main.tex}
            \input{2 - kinetic component/main.tex}
            \input{3 - fluid component/main.tex}
            \input{4 - numerical implementation/main.tex}
        \part{Project Overview}
            \input{5 - research plan/main.tex}
            \input{6 - summary/main.tex}
    
    
    %\section{}
    \newpage
    \pagenumbering{gobble}
        \printbibliography


    \newpage
    \pagenumbering{roman}
    \appendix
        \part{Appendices}
            \input{8 - Hilbert complexes/main.tex}
            \input{9 - weak conservation proofs/main.tex}
\end{document}

            \documentclass[12pt, a4paper]{report}

\input{template/main.tex}

\title{\BA{Title in Progress...}}
\author{Boris Andrews}
\affil{Mathematical Institute, University of Oxford}
\date{\today}


\begin{document}
    \pagenumbering{gobble}
    \maketitle
    
    
    \begin{abstract}
        Magnetic confinement reactors---in particular tokamaks---offer one of the most promising options for achieving practical nuclear fusion, with the potential to provide virtually limitless, clean energy. The theoretical and numerical modeling of tokamak plasmas is simultaneously an essential component of effective reactor design, and a great research barrier. Tokamak operational conditions exhibit comparatively low Knudsen numbers. Kinetic effects, including kinetic waves and instabilities, Landau damping, bump-on-tail instabilities and more, are therefore highly influential in tokamak plasma dynamics. Purely fluid models are inherently incapable of capturing these effects, whereas the high dimensionality in purely kinetic models render them practically intractable for most relevant purposes.

        We consider a $\delta\!f$ decomposition model, with a macroscopic fluid background and microscopic kinetic correction, both fully coupled to each other. A similar manner of discretization is proposed to that used in the recent \texttt{STRUPHY} code \cite{Holderied_Possanner_Wang_2021, Holderied_2022, Li_et_al_2023} with a finite-element model for the background and a pseudo-particle/PiC model for the correction.

        The fluid background satisfies the full, non-linear, resistive, compressible, Hall MHD equations. \cite{Laakmann_Hu_Farrell_2022} introduces finite-element(-in-space) implicit timesteppers for the incompressible analogue to this system with structure-preserving (SP) properties in the ideal case, alongside parameter-robust preconditioners. We show that these timesteppers can derive from a finite-element-in-time (FET) (and finite-element-in-space) interpretation. The benefits of this reformulation are discussed, including the derivation of timesteppers that are higher order in time, and the quantifiable dissipative SP properties in the non-ideal, resistive case.
        
        We discuss possible options for extending this FET approach to timesteppers for the compressible case.

        The kinetic corrections satisfy linearized Boltzmann equations. Using a Lénard--Bernstein collision operator, these take Fokker--Planck-like forms \cite{Fokker_1914, Planck_1917} wherein pseudo-particles in the numerical model obey the neoclassical transport equations, with particle-independent Brownian drift terms. This offers a rigorous methodology for incorporating collisions into the particle transport model, without coupling the equations of motions for each particle.
        
        Works by Chen, Chacón et al. \cite{Chen_Chacón_Barnes_2011, Chacón_Chen_Barnes_2013, Chen_Chacón_2014, Chen_Chacón_2015} have developed structure-preserving particle pushers for neoclassical transport in the Vlasov equations, derived from Crank--Nicolson integrators. We show these too can can derive from a FET interpretation, similarly offering potential extensions to higher-order-in-time particle pushers. The FET formulation is used also to consider how the stochastic drift terms can be incorporated into the pushers. Stochastic gyrokinetic expansions are also discussed.

        Different options for the numerical implementation of these schemes are considered.

        Due to the efficacy of FET in the development of SP timesteppers for both the fluid and kinetic component, we hope this approach will prove effective in the future for developing SP timesteppers for the full hybrid model. We hope this will give us the opportunity to incorporate previously inaccessible kinetic effects into the highly effective, modern, finite-element MHD models.
    \end{abstract}
    
    
    \newpage
    \tableofcontents
    
    
    \newpage
    \pagenumbering{arabic}
    %\linenumbers\renewcommand\thelinenumber{\color{black!50}\arabic{linenumber}}
            \input{0 - introduction/main.tex}
        \part{Research}
            \input{1 - low-noise PiC models/main.tex}
            \input{2 - kinetic component/main.tex}
            \input{3 - fluid component/main.tex}
            \input{4 - numerical implementation/main.tex}
        \part{Project Overview}
            \input{5 - research plan/main.tex}
            \input{6 - summary/main.tex}
    
    
    %\section{}
    \newpage
    \pagenumbering{gobble}
        \printbibliography


    \newpage
    \pagenumbering{roman}
    \appendix
        \part{Appendices}
            \input{8 - Hilbert complexes/main.tex}
            \input{9 - weak conservation proofs/main.tex}
\end{document}

\end{document}

\end{document}

            \documentclass[12pt, a4paper]{report}

\documentclass[12pt, a4paper]{report}

\documentclass[12pt, a4paper]{report}

\input{template/main.tex}

\title{\BA{Title in Progress...}}
\author{Boris Andrews}
\affil{Mathematical Institute, University of Oxford}
\date{\today}


\begin{document}
    \pagenumbering{gobble}
    \maketitle
    
    
    \begin{abstract}
        Magnetic confinement reactors---in particular tokamaks---offer one of the most promising options for achieving practical nuclear fusion, with the potential to provide virtually limitless, clean energy. The theoretical and numerical modeling of tokamak plasmas is simultaneously an essential component of effective reactor design, and a great research barrier. Tokamak operational conditions exhibit comparatively low Knudsen numbers. Kinetic effects, including kinetic waves and instabilities, Landau damping, bump-on-tail instabilities and more, are therefore highly influential in tokamak plasma dynamics. Purely fluid models are inherently incapable of capturing these effects, whereas the high dimensionality in purely kinetic models render them practically intractable for most relevant purposes.

        We consider a $\delta\!f$ decomposition model, with a macroscopic fluid background and microscopic kinetic correction, both fully coupled to each other. A similar manner of discretization is proposed to that used in the recent \texttt{STRUPHY} code \cite{Holderied_Possanner_Wang_2021, Holderied_2022, Li_et_al_2023} with a finite-element model for the background and a pseudo-particle/PiC model for the correction.

        The fluid background satisfies the full, non-linear, resistive, compressible, Hall MHD equations. \cite{Laakmann_Hu_Farrell_2022} introduces finite-element(-in-space) implicit timesteppers for the incompressible analogue to this system with structure-preserving (SP) properties in the ideal case, alongside parameter-robust preconditioners. We show that these timesteppers can derive from a finite-element-in-time (FET) (and finite-element-in-space) interpretation. The benefits of this reformulation are discussed, including the derivation of timesteppers that are higher order in time, and the quantifiable dissipative SP properties in the non-ideal, resistive case.
        
        We discuss possible options for extending this FET approach to timesteppers for the compressible case.

        The kinetic corrections satisfy linearized Boltzmann equations. Using a Lénard--Bernstein collision operator, these take Fokker--Planck-like forms \cite{Fokker_1914, Planck_1917} wherein pseudo-particles in the numerical model obey the neoclassical transport equations, with particle-independent Brownian drift terms. This offers a rigorous methodology for incorporating collisions into the particle transport model, without coupling the equations of motions for each particle.
        
        Works by Chen, Chacón et al. \cite{Chen_Chacón_Barnes_2011, Chacón_Chen_Barnes_2013, Chen_Chacón_2014, Chen_Chacón_2015} have developed structure-preserving particle pushers for neoclassical transport in the Vlasov equations, derived from Crank--Nicolson integrators. We show these too can can derive from a FET interpretation, similarly offering potential extensions to higher-order-in-time particle pushers. The FET formulation is used also to consider how the stochastic drift terms can be incorporated into the pushers. Stochastic gyrokinetic expansions are also discussed.

        Different options for the numerical implementation of these schemes are considered.

        Due to the efficacy of FET in the development of SP timesteppers for both the fluid and kinetic component, we hope this approach will prove effective in the future for developing SP timesteppers for the full hybrid model. We hope this will give us the opportunity to incorporate previously inaccessible kinetic effects into the highly effective, modern, finite-element MHD models.
    \end{abstract}
    
    
    \newpage
    \tableofcontents
    
    
    \newpage
    \pagenumbering{arabic}
    %\linenumbers\renewcommand\thelinenumber{\color{black!50}\arabic{linenumber}}
            \input{0 - introduction/main.tex}
        \part{Research}
            \input{1 - low-noise PiC models/main.tex}
            \input{2 - kinetic component/main.tex}
            \input{3 - fluid component/main.tex}
            \input{4 - numerical implementation/main.tex}
        \part{Project Overview}
            \input{5 - research plan/main.tex}
            \input{6 - summary/main.tex}
    
    
    %\section{}
    \newpage
    \pagenumbering{gobble}
        \printbibliography


    \newpage
    \pagenumbering{roman}
    \appendix
        \part{Appendices}
            \input{8 - Hilbert complexes/main.tex}
            \input{9 - weak conservation proofs/main.tex}
\end{document}


\title{\BA{Title in Progress...}}
\author{Boris Andrews}
\affil{Mathematical Institute, University of Oxford}
\date{\today}


\begin{document}
    \pagenumbering{gobble}
    \maketitle
    
    
    \begin{abstract}
        Magnetic confinement reactors---in particular tokamaks---offer one of the most promising options for achieving practical nuclear fusion, with the potential to provide virtually limitless, clean energy. The theoretical and numerical modeling of tokamak plasmas is simultaneously an essential component of effective reactor design, and a great research barrier. Tokamak operational conditions exhibit comparatively low Knudsen numbers. Kinetic effects, including kinetic waves and instabilities, Landau damping, bump-on-tail instabilities and more, are therefore highly influential in tokamak plasma dynamics. Purely fluid models are inherently incapable of capturing these effects, whereas the high dimensionality in purely kinetic models render them practically intractable for most relevant purposes.

        We consider a $\delta\!f$ decomposition model, with a macroscopic fluid background and microscopic kinetic correction, both fully coupled to each other. A similar manner of discretization is proposed to that used in the recent \texttt{STRUPHY} code \cite{Holderied_Possanner_Wang_2021, Holderied_2022, Li_et_al_2023} with a finite-element model for the background and a pseudo-particle/PiC model for the correction.

        The fluid background satisfies the full, non-linear, resistive, compressible, Hall MHD equations. \cite{Laakmann_Hu_Farrell_2022} introduces finite-element(-in-space) implicit timesteppers for the incompressible analogue to this system with structure-preserving (SP) properties in the ideal case, alongside parameter-robust preconditioners. We show that these timesteppers can derive from a finite-element-in-time (FET) (and finite-element-in-space) interpretation. The benefits of this reformulation are discussed, including the derivation of timesteppers that are higher order in time, and the quantifiable dissipative SP properties in the non-ideal, resistive case.
        
        We discuss possible options for extending this FET approach to timesteppers for the compressible case.

        The kinetic corrections satisfy linearized Boltzmann equations. Using a Lénard--Bernstein collision operator, these take Fokker--Planck-like forms \cite{Fokker_1914, Planck_1917} wherein pseudo-particles in the numerical model obey the neoclassical transport equations, with particle-independent Brownian drift terms. This offers a rigorous methodology for incorporating collisions into the particle transport model, without coupling the equations of motions for each particle.
        
        Works by Chen, Chacón et al. \cite{Chen_Chacón_Barnes_2011, Chacón_Chen_Barnes_2013, Chen_Chacón_2014, Chen_Chacón_2015} have developed structure-preserving particle pushers for neoclassical transport in the Vlasov equations, derived from Crank--Nicolson integrators. We show these too can can derive from a FET interpretation, similarly offering potential extensions to higher-order-in-time particle pushers. The FET formulation is used also to consider how the stochastic drift terms can be incorporated into the pushers. Stochastic gyrokinetic expansions are also discussed.

        Different options for the numerical implementation of these schemes are considered.

        Due to the efficacy of FET in the development of SP timesteppers for both the fluid and kinetic component, we hope this approach will prove effective in the future for developing SP timesteppers for the full hybrid model. We hope this will give us the opportunity to incorporate previously inaccessible kinetic effects into the highly effective, modern, finite-element MHD models.
    \end{abstract}
    
    
    \newpage
    \tableofcontents
    
    
    \newpage
    \pagenumbering{arabic}
    %\linenumbers\renewcommand\thelinenumber{\color{black!50}\arabic{linenumber}}
            \documentclass[12pt, a4paper]{report}

\input{template/main.tex}

\title{\BA{Title in Progress...}}
\author{Boris Andrews}
\affil{Mathematical Institute, University of Oxford}
\date{\today}


\begin{document}
    \pagenumbering{gobble}
    \maketitle
    
    
    \begin{abstract}
        Magnetic confinement reactors---in particular tokamaks---offer one of the most promising options for achieving practical nuclear fusion, with the potential to provide virtually limitless, clean energy. The theoretical and numerical modeling of tokamak plasmas is simultaneously an essential component of effective reactor design, and a great research barrier. Tokamak operational conditions exhibit comparatively low Knudsen numbers. Kinetic effects, including kinetic waves and instabilities, Landau damping, bump-on-tail instabilities and more, are therefore highly influential in tokamak plasma dynamics. Purely fluid models are inherently incapable of capturing these effects, whereas the high dimensionality in purely kinetic models render them practically intractable for most relevant purposes.

        We consider a $\delta\!f$ decomposition model, with a macroscopic fluid background and microscopic kinetic correction, both fully coupled to each other. A similar manner of discretization is proposed to that used in the recent \texttt{STRUPHY} code \cite{Holderied_Possanner_Wang_2021, Holderied_2022, Li_et_al_2023} with a finite-element model for the background and a pseudo-particle/PiC model for the correction.

        The fluid background satisfies the full, non-linear, resistive, compressible, Hall MHD equations. \cite{Laakmann_Hu_Farrell_2022} introduces finite-element(-in-space) implicit timesteppers for the incompressible analogue to this system with structure-preserving (SP) properties in the ideal case, alongside parameter-robust preconditioners. We show that these timesteppers can derive from a finite-element-in-time (FET) (and finite-element-in-space) interpretation. The benefits of this reformulation are discussed, including the derivation of timesteppers that are higher order in time, and the quantifiable dissipative SP properties in the non-ideal, resistive case.
        
        We discuss possible options for extending this FET approach to timesteppers for the compressible case.

        The kinetic corrections satisfy linearized Boltzmann equations. Using a Lénard--Bernstein collision operator, these take Fokker--Planck-like forms \cite{Fokker_1914, Planck_1917} wherein pseudo-particles in the numerical model obey the neoclassical transport equations, with particle-independent Brownian drift terms. This offers a rigorous methodology for incorporating collisions into the particle transport model, without coupling the equations of motions for each particle.
        
        Works by Chen, Chacón et al. \cite{Chen_Chacón_Barnes_2011, Chacón_Chen_Barnes_2013, Chen_Chacón_2014, Chen_Chacón_2015} have developed structure-preserving particle pushers for neoclassical transport in the Vlasov equations, derived from Crank--Nicolson integrators. We show these too can can derive from a FET interpretation, similarly offering potential extensions to higher-order-in-time particle pushers. The FET formulation is used also to consider how the stochastic drift terms can be incorporated into the pushers. Stochastic gyrokinetic expansions are also discussed.

        Different options for the numerical implementation of these schemes are considered.

        Due to the efficacy of FET in the development of SP timesteppers for both the fluid and kinetic component, we hope this approach will prove effective in the future for developing SP timesteppers for the full hybrid model. We hope this will give us the opportunity to incorporate previously inaccessible kinetic effects into the highly effective, modern, finite-element MHD models.
    \end{abstract}
    
    
    \newpage
    \tableofcontents
    
    
    \newpage
    \pagenumbering{arabic}
    %\linenumbers\renewcommand\thelinenumber{\color{black!50}\arabic{linenumber}}
            \input{0 - introduction/main.tex}
        \part{Research}
            \input{1 - low-noise PiC models/main.tex}
            \input{2 - kinetic component/main.tex}
            \input{3 - fluid component/main.tex}
            \input{4 - numerical implementation/main.tex}
        \part{Project Overview}
            \input{5 - research plan/main.tex}
            \input{6 - summary/main.tex}
    
    
    %\section{}
    \newpage
    \pagenumbering{gobble}
        \printbibliography


    \newpage
    \pagenumbering{roman}
    \appendix
        \part{Appendices}
            \input{8 - Hilbert complexes/main.tex}
            \input{9 - weak conservation proofs/main.tex}
\end{document}

        \part{Research}
            \documentclass[12pt, a4paper]{report}

\input{template/main.tex}

\title{\BA{Title in Progress...}}
\author{Boris Andrews}
\affil{Mathematical Institute, University of Oxford}
\date{\today}


\begin{document}
    \pagenumbering{gobble}
    \maketitle
    
    
    \begin{abstract}
        Magnetic confinement reactors---in particular tokamaks---offer one of the most promising options for achieving practical nuclear fusion, with the potential to provide virtually limitless, clean energy. The theoretical and numerical modeling of tokamak plasmas is simultaneously an essential component of effective reactor design, and a great research barrier. Tokamak operational conditions exhibit comparatively low Knudsen numbers. Kinetic effects, including kinetic waves and instabilities, Landau damping, bump-on-tail instabilities and more, are therefore highly influential in tokamak plasma dynamics. Purely fluid models are inherently incapable of capturing these effects, whereas the high dimensionality in purely kinetic models render them practically intractable for most relevant purposes.

        We consider a $\delta\!f$ decomposition model, with a macroscopic fluid background and microscopic kinetic correction, both fully coupled to each other. A similar manner of discretization is proposed to that used in the recent \texttt{STRUPHY} code \cite{Holderied_Possanner_Wang_2021, Holderied_2022, Li_et_al_2023} with a finite-element model for the background and a pseudo-particle/PiC model for the correction.

        The fluid background satisfies the full, non-linear, resistive, compressible, Hall MHD equations. \cite{Laakmann_Hu_Farrell_2022} introduces finite-element(-in-space) implicit timesteppers for the incompressible analogue to this system with structure-preserving (SP) properties in the ideal case, alongside parameter-robust preconditioners. We show that these timesteppers can derive from a finite-element-in-time (FET) (and finite-element-in-space) interpretation. The benefits of this reformulation are discussed, including the derivation of timesteppers that are higher order in time, and the quantifiable dissipative SP properties in the non-ideal, resistive case.
        
        We discuss possible options for extending this FET approach to timesteppers for the compressible case.

        The kinetic corrections satisfy linearized Boltzmann equations. Using a Lénard--Bernstein collision operator, these take Fokker--Planck-like forms \cite{Fokker_1914, Planck_1917} wherein pseudo-particles in the numerical model obey the neoclassical transport equations, with particle-independent Brownian drift terms. This offers a rigorous methodology for incorporating collisions into the particle transport model, without coupling the equations of motions for each particle.
        
        Works by Chen, Chacón et al. \cite{Chen_Chacón_Barnes_2011, Chacón_Chen_Barnes_2013, Chen_Chacón_2014, Chen_Chacón_2015} have developed structure-preserving particle pushers for neoclassical transport in the Vlasov equations, derived from Crank--Nicolson integrators. We show these too can can derive from a FET interpretation, similarly offering potential extensions to higher-order-in-time particle pushers. The FET formulation is used also to consider how the stochastic drift terms can be incorporated into the pushers. Stochastic gyrokinetic expansions are also discussed.

        Different options for the numerical implementation of these schemes are considered.

        Due to the efficacy of FET in the development of SP timesteppers for both the fluid and kinetic component, we hope this approach will prove effective in the future for developing SP timesteppers for the full hybrid model. We hope this will give us the opportunity to incorporate previously inaccessible kinetic effects into the highly effective, modern, finite-element MHD models.
    \end{abstract}
    
    
    \newpage
    \tableofcontents
    
    
    \newpage
    \pagenumbering{arabic}
    %\linenumbers\renewcommand\thelinenumber{\color{black!50}\arabic{linenumber}}
            \input{0 - introduction/main.tex}
        \part{Research}
            \input{1 - low-noise PiC models/main.tex}
            \input{2 - kinetic component/main.tex}
            \input{3 - fluid component/main.tex}
            \input{4 - numerical implementation/main.tex}
        \part{Project Overview}
            \input{5 - research plan/main.tex}
            \input{6 - summary/main.tex}
    
    
    %\section{}
    \newpage
    \pagenumbering{gobble}
        \printbibliography


    \newpage
    \pagenumbering{roman}
    \appendix
        \part{Appendices}
            \input{8 - Hilbert complexes/main.tex}
            \input{9 - weak conservation proofs/main.tex}
\end{document}

            \documentclass[12pt, a4paper]{report}

\input{template/main.tex}

\title{\BA{Title in Progress...}}
\author{Boris Andrews}
\affil{Mathematical Institute, University of Oxford}
\date{\today}


\begin{document}
    \pagenumbering{gobble}
    \maketitle
    
    
    \begin{abstract}
        Magnetic confinement reactors---in particular tokamaks---offer one of the most promising options for achieving practical nuclear fusion, with the potential to provide virtually limitless, clean energy. The theoretical and numerical modeling of tokamak plasmas is simultaneously an essential component of effective reactor design, and a great research barrier. Tokamak operational conditions exhibit comparatively low Knudsen numbers. Kinetic effects, including kinetic waves and instabilities, Landau damping, bump-on-tail instabilities and more, are therefore highly influential in tokamak plasma dynamics. Purely fluid models are inherently incapable of capturing these effects, whereas the high dimensionality in purely kinetic models render them practically intractable for most relevant purposes.

        We consider a $\delta\!f$ decomposition model, with a macroscopic fluid background and microscopic kinetic correction, both fully coupled to each other. A similar manner of discretization is proposed to that used in the recent \texttt{STRUPHY} code \cite{Holderied_Possanner_Wang_2021, Holderied_2022, Li_et_al_2023} with a finite-element model for the background and a pseudo-particle/PiC model for the correction.

        The fluid background satisfies the full, non-linear, resistive, compressible, Hall MHD equations. \cite{Laakmann_Hu_Farrell_2022} introduces finite-element(-in-space) implicit timesteppers for the incompressible analogue to this system with structure-preserving (SP) properties in the ideal case, alongside parameter-robust preconditioners. We show that these timesteppers can derive from a finite-element-in-time (FET) (and finite-element-in-space) interpretation. The benefits of this reformulation are discussed, including the derivation of timesteppers that are higher order in time, and the quantifiable dissipative SP properties in the non-ideal, resistive case.
        
        We discuss possible options for extending this FET approach to timesteppers for the compressible case.

        The kinetic corrections satisfy linearized Boltzmann equations. Using a Lénard--Bernstein collision operator, these take Fokker--Planck-like forms \cite{Fokker_1914, Planck_1917} wherein pseudo-particles in the numerical model obey the neoclassical transport equations, with particle-independent Brownian drift terms. This offers a rigorous methodology for incorporating collisions into the particle transport model, without coupling the equations of motions for each particle.
        
        Works by Chen, Chacón et al. \cite{Chen_Chacón_Barnes_2011, Chacón_Chen_Barnes_2013, Chen_Chacón_2014, Chen_Chacón_2015} have developed structure-preserving particle pushers for neoclassical transport in the Vlasov equations, derived from Crank--Nicolson integrators. We show these too can can derive from a FET interpretation, similarly offering potential extensions to higher-order-in-time particle pushers. The FET formulation is used also to consider how the stochastic drift terms can be incorporated into the pushers. Stochastic gyrokinetic expansions are also discussed.

        Different options for the numerical implementation of these schemes are considered.

        Due to the efficacy of FET in the development of SP timesteppers for both the fluid and kinetic component, we hope this approach will prove effective in the future for developing SP timesteppers for the full hybrid model. We hope this will give us the opportunity to incorporate previously inaccessible kinetic effects into the highly effective, modern, finite-element MHD models.
    \end{abstract}
    
    
    \newpage
    \tableofcontents
    
    
    \newpage
    \pagenumbering{arabic}
    %\linenumbers\renewcommand\thelinenumber{\color{black!50}\arabic{linenumber}}
            \input{0 - introduction/main.tex}
        \part{Research}
            \input{1 - low-noise PiC models/main.tex}
            \input{2 - kinetic component/main.tex}
            \input{3 - fluid component/main.tex}
            \input{4 - numerical implementation/main.tex}
        \part{Project Overview}
            \input{5 - research plan/main.tex}
            \input{6 - summary/main.tex}
    
    
    %\section{}
    \newpage
    \pagenumbering{gobble}
        \printbibliography


    \newpage
    \pagenumbering{roman}
    \appendix
        \part{Appendices}
            \input{8 - Hilbert complexes/main.tex}
            \input{9 - weak conservation proofs/main.tex}
\end{document}

            \documentclass[12pt, a4paper]{report}

\input{template/main.tex}

\title{\BA{Title in Progress...}}
\author{Boris Andrews}
\affil{Mathematical Institute, University of Oxford}
\date{\today}


\begin{document}
    \pagenumbering{gobble}
    \maketitle
    
    
    \begin{abstract}
        Magnetic confinement reactors---in particular tokamaks---offer one of the most promising options for achieving practical nuclear fusion, with the potential to provide virtually limitless, clean energy. The theoretical and numerical modeling of tokamak plasmas is simultaneously an essential component of effective reactor design, and a great research barrier. Tokamak operational conditions exhibit comparatively low Knudsen numbers. Kinetic effects, including kinetic waves and instabilities, Landau damping, bump-on-tail instabilities and more, are therefore highly influential in tokamak plasma dynamics. Purely fluid models are inherently incapable of capturing these effects, whereas the high dimensionality in purely kinetic models render them practically intractable for most relevant purposes.

        We consider a $\delta\!f$ decomposition model, with a macroscopic fluid background and microscopic kinetic correction, both fully coupled to each other. A similar manner of discretization is proposed to that used in the recent \texttt{STRUPHY} code \cite{Holderied_Possanner_Wang_2021, Holderied_2022, Li_et_al_2023} with a finite-element model for the background and a pseudo-particle/PiC model for the correction.

        The fluid background satisfies the full, non-linear, resistive, compressible, Hall MHD equations. \cite{Laakmann_Hu_Farrell_2022} introduces finite-element(-in-space) implicit timesteppers for the incompressible analogue to this system with structure-preserving (SP) properties in the ideal case, alongside parameter-robust preconditioners. We show that these timesteppers can derive from a finite-element-in-time (FET) (and finite-element-in-space) interpretation. The benefits of this reformulation are discussed, including the derivation of timesteppers that are higher order in time, and the quantifiable dissipative SP properties in the non-ideal, resistive case.
        
        We discuss possible options for extending this FET approach to timesteppers for the compressible case.

        The kinetic corrections satisfy linearized Boltzmann equations. Using a Lénard--Bernstein collision operator, these take Fokker--Planck-like forms \cite{Fokker_1914, Planck_1917} wherein pseudo-particles in the numerical model obey the neoclassical transport equations, with particle-independent Brownian drift terms. This offers a rigorous methodology for incorporating collisions into the particle transport model, without coupling the equations of motions for each particle.
        
        Works by Chen, Chacón et al. \cite{Chen_Chacón_Barnes_2011, Chacón_Chen_Barnes_2013, Chen_Chacón_2014, Chen_Chacón_2015} have developed structure-preserving particle pushers for neoclassical transport in the Vlasov equations, derived from Crank--Nicolson integrators. We show these too can can derive from a FET interpretation, similarly offering potential extensions to higher-order-in-time particle pushers. The FET formulation is used also to consider how the stochastic drift terms can be incorporated into the pushers. Stochastic gyrokinetic expansions are also discussed.

        Different options for the numerical implementation of these schemes are considered.

        Due to the efficacy of FET in the development of SP timesteppers for both the fluid and kinetic component, we hope this approach will prove effective in the future for developing SP timesteppers for the full hybrid model. We hope this will give us the opportunity to incorporate previously inaccessible kinetic effects into the highly effective, modern, finite-element MHD models.
    \end{abstract}
    
    
    \newpage
    \tableofcontents
    
    
    \newpage
    \pagenumbering{arabic}
    %\linenumbers\renewcommand\thelinenumber{\color{black!50}\arabic{linenumber}}
            \input{0 - introduction/main.tex}
        \part{Research}
            \input{1 - low-noise PiC models/main.tex}
            \input{2 - kinetic component/main.tex}
            \input{3 - fluid component/main.tex}
            \input{4 - numerical implementation/main.tex}
        \part{Project Overview}
            \input{5 - research plan/main.tex}
            \input{6 - summary/main.tex}
    
    
    %\section{}
    \newpage
    \pagenumbering{gobble}
        \printbibliography


    \newpage
    \pagenumbering{roman}
    \appendix
        \part{Appendices}
            \input{8 - Hilbert complexes/main.tex}
            \input{9 - weak conservation proofs/main.tex}
\end{document}

            \documentclass[12pt, a4paper]{report}

\input{template/main.tex}

\title{\BA{Title in Progress...}}
\author{Boris Andrews}
\affil{Mathematical Institute, University of Oxford}
\date{\today}


\begin{document}
    \pagenumbering{gobble}
    \maketitle
    
    
    \begin{abstract}
        Magnetic confinement reactors---in particular tokamaks---offer one of the most promising options for achieving practical nuclear fusion, with the potential to provide virtually limitless, clean energy. The theoretical and numerical modeling of tokamak plasmas is simultaneously an essential component of effective reactor design, and a great research barrier. Tokamak operational conditions exhibit comparatively low Knudsen numbers. Kinetic effects, including kinetic waves and instabilities, Landau damping, bump-on-tail instabilities and more, are therefore highly influential in tokamak plasma dynamics. Purely fluid models are inherently incapable of capturing these effects, whereas the high dimensionality in purely kinetic models render them practically intractable for most relevant purposes.

        We consider a $\delta\!f$ decomposition model, with a macroscopic fluid background and microscopic kinetic correction, both fully coupled to each other. A similar manner of discretization is proposed to that used in the recent \texttt{STRUPHY} code \cite{Holderied_Possanner_Wang_2021, Holderied_2022, Li_et_al_2023} with a finite-element model for the background and a pseudo-particle/PiC model for the correction.

        The fluid background satisfies the full, non-linear, resistive, compressible, Hall MHD equations. \cite{Laakmann_Hu_Farrell_2022} introduces finite-element(-in-space) implicit timesteppers for the incompressible analogue to this system with structure-preserving (SP) properties in the ideal case, alongside parameter-robust preconditioners. We show that these timesteppers can derive from a finite-element-in-time (FET) (and finite-element-in-space) interpretation. The benefits of this reformulation are discussed, including the derivation of timesteppers that are higher order in time, and the quantifiable dissipative SP properties in the non-ideal, resistive case.
        
        We discuss possible options for extending this FET approach to timesteppers for the compressible case.

        The kinetic corrections satisfy linearized Boltzmann equations. Using a Lénard--Bernstein collision operator, these take Fokker--Planck-like forms \cite{Fokker_1914, Planck_1917} wherein pseudo-particles in the numerical model obey the neoclassical transport equations, with particle-independent Brownian drift terms. This offers a rigorous methodology for incorporating collisions into the particle transport model, without coupling the equations of motions for each particle.
        
        Works by Chen, Chacón et al. \cite{Chen_Chacón_Barnes_2011, Chacón_Chen_Barnes_2013, Chen_Chacón_2014, Chen_Chacón_2015} have developed structure-preserving particle pushers for neoclassical transport in the Vlasov equations, derived from Crank--Nicolson integrators. We show these too can can derive from a FET interpretation, similarly offering potential extensions to higher-order-in-time particle pushers. The FET formulation is used also to consider how the stochastic drift terms can be incorporated into the pushers. Stochastic gyrokinetic expansions are also discussed.

        Different options for the numerical implementation of these schemes are considered.

        Due to the efficacy of FET in the development of SP timesteppers for both the fluid and kinetic component, we hope this approach will prove effective in the future for developing SP timesteppers for the full hybrid model. We hope this will give us the opportunity to incorporate previously inaccessible kinetic effects into the highly effective, modern, finite-element MHD models.
    \end{abstract}
    
    
    \newpage
    \tableofcontents
    
    
    \newpage
    \pagenumbering{arabic}
    %\linenumbers\renewcommand\thelinenumber{\color{black!50}\arabic{linenumber}}
            \input{0 - introduction/main.tex}
        \part{Research}
            \input{1 - low-noise PiC models/main.tex}
            \input{2 - kinetic component/main.tex}
            \input{3 - fluid component/main.tex}
            \input{4 - numerical implementation/main.tex}
        \part{Project Overview}
            \input{5 - research plan/main.tex}
            \input{6 - summary/main.tex}
    
    
    %\section{}
    \newpage
    \pagenumbering{gobble}
        \printbibliography


    \newpage
    \pagenumbering{roman}
    \appendix
        \part{Appendices}
            \input{8 - Hilbert complexes/main.tex}
            \input{9 - weak conservation proofs/main.tex}
\end{document}

        \part{Project Overview}
            \documentclass[12pt, a4paper]{report}

\input{template/main.tex}

\title{\BA{Title in Progress...}}
\author{Boris Andrews}
\affil{Mathematical Institute, University of Oxford}
\date{\today}


\begin{document}
    \pagenumbering{gobble}
    \maketitle
    
    
    \begin{abstract}
        Magnetic confinement reactors---in particular tokamaks---offer one of the most promising options for achieving practical nuclear fusion, with the potential to provide virtually limitless, clean energy. The theoretical and numerical modeling of tokamak plasmas is simultaneously an essential component of effective reactor design, and a great research barrier. Tokamak operational conditions exhibit comparatively low Knudsen numbers. Kinetic effects, including kinetic waves and instabilities, Landau damping, bump-on-tail instabilities and more, are therefore highly influential in tokamak plasma dynamics. Purely fluid models are inherently incapable of capturing these effects, whereas the high dimensionality in purely kinetic models render them practically intractable for most relevant purposes.

        We consider a $\delta\!f$ decomposition model, with a macroscopic fluid background and microscopic kinetic correction, both fully coupled to each other. A similar manner of discretization is proposed to that used in the recent \texttt{STRUPHY} code \cite{Holderied_Possanner_Wang_2021, Holderied_2022, Li_et_al_2023} with a finite-element model for the background and a pseudo-particle/PiC model for the correction.

        The fluid background satisfies the full, non-linear, resistive, compressible, Hall MHD equations. \cite{Laakmann_Hu_Farrell_2022} introduces finite-element(-in-space) implicit timesteppers for the incompressible analogue to this system with structure-preserving (SP) properties in the ideal case, alongside parameter-robust preconditioners. We show that these timesteppers can derive from a finite-element-in-time (FET) (and finite-element-in-space) interpretation. The benefits of this reformulation are discussed, including the derivation of timesteppers that are higher order in time, and the quantifiable dissipative SP properties in the non-ideal, resistive case.
        
        We discuss possible options for extending this FET approach to timesteppers for the compressible case.

        The kinetic corrections satisfy linearized Boltzmann equations. Using a Lénard--Bernstein collision operator, these take Fokker--Planck-like forms \cite{Fokker_1914, Planck_1917} wherein pseudo-particles in the numerical model obey the neoclassical transport equations, with particle-independent Brownian drift terms. This offers a rigorous methodology for incorporating collisions into the particle transport model, without coupling the equations of motions for each particle.
        
        Works by Chen, Chacón et al. \cite{Chen_Chacón_Barnes_2011, Chacón_Chen_Barnes_2013, Chen_Chacón_2014, Chen_Chacón_2015} have developed structure-preserving particle pushers for neoclassical transport in the Vlasov equations, derived from Crank--Nicolson integrators. We show these too can can derive from a FET interpretation, similarly offering potential extensions to higher-order-in-time particle pushers. The FET formulation is used also to consider how the stochastic drift terms can be incorporated into the pushers. Stochastic gyrokinetic expansions are also discussed.

        Different options for the numerical implementation of these schemes are considered.

        Due to the efficacy of FET in the development of SP timesteppers for both the fluid and kinetic component, we hope this approach will prove effective in the future for developing SP timesteppers for the full hybrid model. We hope this will give us the opportunity to incorporate previously inaccessible kinetic effects into the highly effective, modern, finite-element MHD models.
    \end{abstract}
    
    
    \newpage
    \tableofcontents
    
    
    \newpage
    \pagenumbering{arabic}
    %\linenumbers\renewcommand\thelinenumber{\color{black!50}\arabic{linenumber}}
            \input{0 - introduction/main.tex}
        \part{Research}
            \input{1 - low-noise PiC models/main.tex}
            \input{2 - kinetic component/main.tex}
            \input{3 - fluid component/main.tex}
            \input{4 - numerical implementation/main.tex}
        \part{Project Overview}
            \input{5 - research plan/main.tex}
            \input{6 - summary/main.tex}
    
    
    %\section{}
    \newpage
    \pagenumbering{gobble}
        \printbibliography


    \newpage
    \pagenumbering{roman}
    \appendix
        \part{Appendices}
            \input{8 - Hilbert complexes/main.tex}
            \input{9 - weak conservation proofs/main.tex}
\end{document}

            \documentclass[12pt, a4paper]{report}

\input{template/main.tex}

\title{\BA{Title in Progress...}}
\author{Boris Andrews}
\affil{Mathematical Institute, University of Oxford}
\date{\today}


\begin{document}
    \pagenumbering{gobble}
    \maketitle
    
    
    \begin{abstract}
        Magnetic confinement reactors---in particular tokamaks---offer one of the most promising options for achieving practical nuclear fusion, with the potential to provide virtually limitless, clean energy. The theoretical and numerical modeling of tokamak plasmas is simultaneously an essential component of effective reactor design, and a great research barrier. Tokamak operational conditions exhibit comparatively low Knudsen numbers. Kinetic effects, including kinetic waves and instabilities, Landau damping, bump-on-tail instabilities and more, are therefore highly influential in tokamak plasma dynamics. Purely fluid models are inherently incapable of capturing these effects, whereas the high dimensionality in purely kinetic models render them practically intractable for most relevant purposes.

        We consider a $\delta\!f$ decomposition model, with a macroscopic fluid background and microscopic kinetic correction, both fully coupled to each other. A similar manner of discretization is proposed to that used in the recent \texttt{STRUPHY} code \cite{Holderied_Possanner_Wang_2021, Holderied_2022, Li_et_al_2023} with a finite-element model for the background and a pseudo-particle/PiC model for the correction.

        The fluid background satisfies the full, non-linear, resistive, compressible, Hall MHD equations. \cite{Laakmann_Hu_Farrell_2022} introduces finite-element(-in-space) implicit timesteppers for the incompressible analogue to this system with structure-preserving (SP) properties in the ideal case, alongside parameter-robust preconditioners. We show that these timesteppers can derive from a finite-element-in-time (FET) (and finite-element-in-space) interpretation. The benefits of this reformulation are discussed, including the derivation of timesteppers that are higher order in time, and the quantifiable dissipative SP properties in the non-ideal, resistive case.
        
        We discuss possible options for extending this FET approach to timesteppers for the compressible case.

        The kinetic corrections satisfy linearized Boltzmann equations. Using a Lénard--Bernstein collision operator, these take Fokker--Planck-like forms \cite{Fokker_1914, Planck_1917} wherein pseudo-particles in the numerical model obey the neoclassical transport equations, with particle-independent Brownian drift terms. This offers a rigorous methodology for incorporating collisions into the particle transport model, without coupling the equations of motions for each particle.
        
        Works by Chen, Chacón et al. \cite{Chen_Chacón_Barnes_2011, Chacón_Chen_Barnes_2013, Chen_Chacón_2014, Chen_Chacón_2015} have developed structure-preserving particle pushers for neoclassical transport in the Vlasov equations, derived from Crank--Nicolson integrators. We show these too can can derive from a FET interpretation, similarly offering potential extensions to higher-order-in-time particle pushers. The FET formulation is used also to consider how the stochastic drift terms can be incorporated into the pushers. Stochastic gyrokinetic expansions are also discussed.

        Different options for the numerical implementation of these schemes are considered.

        Due to the efficacy of FET in the development of SP timesteppers for both the fluid and kinetic component, we hope this approach will prove effective in the future for developing SP timesteppers for the full hybrid model. We hope this will give us the opportunity to incorporate previously inaccessible kinetic effects into the highly effective, modern, finite-element MHD models.
    \end{abstract}
    
    
    \newpage
    \tableofcontents
    
    
    \newpage
    \pagenumbering{arabic}
    %\linenumbers\renewcommand\thelinenumber{\color{black!50}\arabic{linenumber}}
            \input{0 - introduction/main.tex}
        \part{Research}
            \input{1 - low-noise PiC models/main.tex}
            \input{2 - kinetic component/main.tex}
            \input{3 - fluid component/main.tex}
            \input{4 - numerical implementation/main.tex}
        \part{Project Overview}
            \input{5 - research plan/main.tex}
            \input{6 - summary/main.tex}
    
    
    %\section{}
    \newpage
    \pagenumbering{gobble}
        \printbibliography


    \newpage
    \pagenumbering{roman}
    \appendix
        \part{Appendices}
            \input{8 - Hilbert complexes/main.tex}
            \input{9 - weak conservation proofs/main.tex}
\end{document}

    
    
    %\section{}
    \newpage
    \pagenumbering{gobble}
        \printbibliography


    \newpage
    \pagenumbering{roman}
    \appendix
        \part{Appendices}
            \documentclass[12pt, a4paper]{report}

\input{template/main.tex}

\title{\BA{Title in Progress...}}
\author{Boris Andrews}
\affil{Mathematical Institute, University of Oxford}
\date{\today}


\begin{document}
    \pagenumbering{gobble}
    \maketitle
    
    
    \begin{abstract}
        Magnetic confinement reactors---in particular tokamaks---offer one of the most promising options for achieving practical nuclear fusion, with the potential to provide virtually limitless, clean energy. The theoretical and numerical modeling of tokamak plasmas is simultaneously an essential component of effective reactor design, and a great research barrier. Tokamak operational conditions exhibit comparatively low Knudsen numbers. Kinetic effects, including kinetic waves and instabilities, Landau damping, bump-on-tail instabilities and more, are therefore highly influential in tokamak plasma dynamics. Purely fluid models are inherently incapable of capturing these effects, whereas the high dimensionality in purely kinetic models render them practically intractable for most relevant purposes.

        We consider a $\delta\!f$ decomposition model, with a macroscopic fluid background and microscopic kinetic correction, both fully coupled to each other. A similar manner of discretization is proposed to that used in the recent \texttt{STRUPHY} code \cite{Holderied_Possanner_Wang_2021, Holderied_2022, Li_et_al_2023} with a finite-element model for the background and a pseudo-particle/PiC model for the correction.

        The fluid background satisfies the full, non-linear, resistive, compressible, Hall MHD equations. \cite{Laakmann_Hu_Farrell_2022} introduces finite-element(-in-space) implicit timesteppers for the incompressible analogue to this system with structure-preserving (SP) properties in the ideal case, alongside parameter-robust preconditioners. We show that these timesteppers can derive from a finite-element-in-time (FET) (and finite-element-in-space) interpretation. The benefits of this reformulation are discussed, including the derivation of timesteppers that are higher order in time, and the quantifiable dissipative SP properties in the non-ideal, resistive case.
        
        We discuss possible options for extending this FET approach to timesteppers for the compressible case.

        The kinetic corrections satisfy linearized Boltzmann equations. Using a Lénard--Bernstein collision operator, these take Fokker--Planck-like forms \cite{Fokker_1914, Planck_1917} wherein pseudo-particles in the numerical model obey the neoclassical transport equations, with particle-independent Brownian drift terms. This offers a rigorous methodology for incorporating collisions into the particle transport model, without coupling the equations of motions for each particle.
        
        Works by Chen, Chacón et al. \cite{Chen_Chacón_Barnes_2011, Chacón_Chen_Barnes_2013, Chen_Chacón_2014, Chen_Chacón_2015} have developed structure-preserving particle pushers for neoclassical transport in the Vlasov equations, derived from Crank--Nicolson integrators. We show these too can can derive from a FET interpretation, similarly offering potential extensions to higher-order-in-time particle pushers. The FET formulation is used also to consider how the stochastic drift terms can be incorporated into the pushers. Stochastic gyrokinetic expansions are also discussed.

        Different options for the numerical implementation of these schemes are considered.

        Due to the efficacy of FET in the development of SP timesteppers for both the fluid and kinetic component, we hope this approach will prove effective in the future for developing SP timesteppers for the full hybrid model. We hope this will give us the opportunity to incorporate previously inaccessible kinetic effects into the highly effective, modern, finite-element MHD models.
    \end{abstract}
    
    
    \newpage
    \tableofcontents
    
    
    \newpage
    \pagenumbering{arabic}
    %\linenumbers\renewcommand\thelinenumber{\color{black!50}\arabic{linenumber}}
            \input{0 - introduction/main.tex}
        \part{Research}
            \input{1 - low-noise PiC models/main.tex}
            \input{2 - kinetic component/main.tex}
            \input{3 - fluid component/main.tex}
            \input{4 - numerical implementation/main.tex}
        \part{Project Overview}
            \input{5 - research plan/main.tex}
            \input{6 - summary/main.tex}
    
    
    %\section{}
    \newpage
    \pagenumbering{gobble}
        \printbibliography


    \newpage
    \pagenumbering{roman}
    \appendix
        \part{Appendices}
            \input{8 - Hilbert complexes/main.tex}
            \input{9 - weak conservation proofs/main.tex}
\end{document}

            \documentclass[12pt, a4paper]{report}

\input{template/main.tex}

\title{\BA{Title in Progress...}}
\author{Boris Andrews}
\affil{Mathematical Institute, University of Oxford}
\date{\today}


\begin{document}
    \pagenumbering{gobble}
    \maketitle
    
    
    \begin{abstract}
        Magnetic confinement reactors---in particular tokamaks---offer one of the most promising options for achieving practical nuclear fusion, with the potential to provide virtually limitless, clean energy. The theoretical and numerical modeling of tokamak plasmas is simultaneously an essential component of effective reactor design, and a great research barrier. Tokamak operational conditions exhibit comparatively low Knudsen numbers. Kinetic effects, including kinetic waves and instabilities, Landau damping, bump-on-tail instabilities and more, are therefore highly influential in tokamak plasma dynamics. Purely fluid models are inherently incapable of capturing these effects, whereas the high dimensionality in purely kinetic models render them practically intractable for most relevant purposes.

        We consider a $\delta\!f$ decomposition model, with a macroscopic fluid background and microscopic kinetic correction, both fully coupled to each other. A similar manner of discretization is proposed to that used in the recent \texttt{STRUPHY} code \cite{Holderied_Possanner_Wang_2021, Holderied_2022, Li_et_al_2023} with a finite-element model for the background and a pseudo-particle/PiC model for the correction.

        The fluid background satisfies the full, non-linear, resistive, compressible, Hall MHD equations. \cite{Laakmann_Hu_Farrell_2022} introduces finite-element(-in-space) implicit timesteppers for the incompressible analogue to this system with structure-preserving (SP) properties in the ideal case, alongside parameter-robust preconditioners. We show that these timesteppers can derive from a finite-element-in-time (FET) (and finite-element-in-space) interpretation. The benefits of this reformulation are discussed, including the derivation of timesteppers that are higher order in time, and the quantifiable dissipative SP properties in the non-ideal, resistive case.
        
        We discuss possible options for extending this FET approach to timesteppers for the compressible case.

        The kinetic corrections satisfy linearized Boltzmann equations. Using a Lénard--Bernstein collision operator, these take Fokker--Planck-like forms \cite{Fokker_1914, Planck_1917} wherein pseudo-particles in the numerical model obey the neoclassical transport equations, with particle-independent Brownian drift terms. This offers a rigorous methodology for incorporating collisions into the particle transport model, without coupling the equations of motions for each particle.
        
        Works by Chen, Chacón et al. \cite{Chen_Chacón_Barnes_2011, Chacón_Chen_Barnes_2013, Chen_Chacón_2014, Chen_Chacón_2015} have developed structure-preserving particle pushers for neoclassical transport in the Vlasov equations, derived from Crank--Nicolson integrators. We show these too can can derive from a FET interpretation, similarly offering potential extensions to higher-order-in-time particle pushers. The FET formulation is used also to consider how the stochastic drift terms can be incorporated into the pushers. Stochastic gyrokinetic expansions are also discussed.

        Different options for the numerical implementation of these schemes are considered.

        Due to the efficacy of FET in the development of SP timesteppers for both the fluid and kinetic component, we hope this approach will prove effective in the future for developing SP timesteppers for the full hybrid model. We hope this will give us the opportunity to incorporate previously inaccessible kinetic effects into the highly effective, modern, finite-element MHD models.
    \end{abstract}
    
    
    \newpage
    \tableofcontents
    
    
    \newpage
    \pagenumbering{arabic}
    %\linenumbers\renewcommand\thelinenumber{\color{black!50}\arabic{linenumber}}
            \input{0 - introduction/main.tex}
        \part{Research}
            \input{1 - low-noise PiC models/main.tex}
            \input{2 - kinetic component/main.tex}
            \input{3 - fluid component/main.tex}
            \input{4 - numerical implementation/main.tex}
        \part{Project Overview}
            \input{5 - research plan/main.tex}
            \input{6 - summary/main.tex}
    
    
    %\section{}
    \newpage
    \pagenumbering{gobble}
        \printbibliography


    \newpage
    \pagenumbering{roman}
    \appendix
        \part{Appendices}
            \input{8 - Hilbert complexes/main.tex}
            \input{9 - weak conservation proofs/main.tex}
\end{document}

\end{document}


\title{\BA{Title in Progress...}}
\author{Boris Andrews}
\affil{Mathematical Institute, University of Oxford}
\date{\today}


\begin{document}
    \pagenumbering{gobble}
    \maketitle
    
    
    \begin{abstract}
        Magnetic confinement reactors---in particular tokamaks---offer one of the most promising options for achieving practical nuclear fusion, with the potential to provide virtually limitless, clean energy. The theoretical and numerical modeling of tokamak plasmas is simultaneously an essential component of effective reactor design, and a great research barrier. Tokamak operational conditions exhibit comparatively low Knudsen numbers. Kinetic effects, including kinetic waves and instabilities, Landau damping, bump-on-tail instabilities and more, are therefore highly influential in tokamak plasma dynamics. Purely fluid models are inherently incapable of capturing these effects, whereas the high dimensionality in purely kinetic models render them practically intractable for most relevant purposes.

        We consider a $\delta\!f$ decomposition model, with a macroscopic fluid background and microscopic kinetic correction, both fully coupled to each other. A similar manner of discretization is proposed to that used in the recent \texttt{STRUPHY} code \cite{Holderied_Possanner_Wang_2021, Holderied_2022, Li_et_al_2023} with a finite-element model for the background and a pseudo-particle/PiC model for the correction.

        The fluid background satisfies the full, non-linear, resistive, compressible, Hall MHD equations. \cite{Laakmann_Hu_Farrell_2022} introduces finite-element(-in-space) implicit timesteppers for the incompressible analogue to this system with structure-preserving (SP) properties in the ideal case, alongside parameter-robust preconditioners. We show that these timesteppers can derive from a finite-element-in-time (FET) (and finite-element-in-space) interpretation. The benefits of this reformulation are discussed, including the derivation of timesteppers that are higher order in time, and the quantifiable dissipative SP properties in the non-ideal, resistive case.
        
        We discuss possible options for extending this FET approach to timesteppers for the compressible case.

        The kinetic corrections satisfy linearized Boltzmann equations. Using a Lénard--Bernstein collision operator, these take Fokker--Planck-like forms \cite{Fokker_1914, Planck_1917} wherein pseudo-particles in the numerical model obey the neoclassical transport equations, with particle-independent Brownian drift terms. This offers a rigorous methodology for incorporating collisions into the particle transport model, without coupling the equations of motions for each particle.
        
        Works by Chen, Chacón et al. \cite{Chen_Chacón_Barnes_2011, Chacón_Chen_Barnes_2013, Chen_Chacón_2014, Chen_Chacón_2015} have developed structure-preserving particle pushers for neoclassical transport in the Vlasov equations, derived from Crank--Nicolson integrators. We show these too can can derive from a FET interpretation, similarly offering potential extensions to higher-order-in-time particle pushers. The FET formulation is used also to consider how the stochastic drift terms can be incorporated into the pushers. Stochastic gyrokinetic expansions are also discussed.

        Different options for the numerical implementation of these schemes are considered.

        Due to the efficacy of FET in the development of SP timesteppers for both the fluid and kinetic component, we hope this approach will prove effective in the future for developing SP timesteppers for the full hybrid model. We hope this will give us the opportunity to incorporate previously inaccessible kinetic effects into the highly effective, modern, finite-element MHD models.
    \end{abstract}
    
    
    \newpage
    \tableofcontents
    
    
    \newpage
    \pagenumbering{arabic}
    %\linenumbers\renewcommand\thelinenumber{\color{black!50}\arabic{linenumber}}
            \documentclass[12pt, a4paper]{report}

\documentclass[12pt, a4paper]{report}

\input{template/main.tex}

\title{\BA{Title in Progress...}}
\author{Boris Andrews}
\affil{Mathematical Institute, University of Oxford}
\date{\today}


\begin{document}
    \pagenumbering{gobble}
    \maketitle
    
    
    \begin{abstract}
        Magnetic confinement reactors---in particular tokamaks---offer one of the most promising options for achieving practical nuclear fusion, with the potential to provide virtually limitless, clean energy. The theoretical and numerical modeling of tokamak plasmas is simultaneously an essential component of effective reactor design, and a great research barrier. Tokamak operational conditions exhibit comparatively low Knudsen numbers. Kinetic effects, including kinetic waves and instabilities, Landau damping, bump-on-tail instabilities and more, are therefore highly influential in tokamak plasma dynamics. Purely fluid models are inherently incapable of capturing these effects, whereas the high dimensionality in purely kinetic models render them practically intractable for most relevant purposes.

        We consider a $\delta\!f$ decomposition model, with a macroscopic fluid background and microscopic kinetic correction, both fully coupled to each other. A similar manner of discretization is proposed to that used in the recent \texttt{STRUPHY} code \cite{Holderied_Possanner_Wang_2021, Holderied_2022, Li_et_al_2023} with a finite-element model for the background and a pseudo-particle/PiC model for the correction.

        The fluid background satisfies the full, non-linear, resistive, compressible, Hall MHD equations. \cite{Laakmann_Hu_Farrell_2022} introduces finite-element(-in-space) implicit timesteppers for the incompressible analogue to this system with structure-preserving (SP) properties in the ideal case, alongside parameter-robust preconditioners. We show that these timesteppers can derive from a finite-element-in-time (FET) (and finite-element-in-space) interpretation. The benefits of this reformulation are discussed, including the derivation of timesteppers that are higher order in time, and the quantifiable dissipative SP properties in the non-ideal, resistive case.
        
        We discuss possible options for extending this FET approach to timesteppers for the compressible case.

        The kinetic corrections satisfy linearized Boltzmann equations. Using a Lénard--Bernstein collision operator, these take Fokker--Planck-like forms \cite{Fokker_1914, Planck_1917} wherein pseudo-particles in the numerical model obey the neoclassical transport equations, with particle-independent Brownian drift terms. This offers a rigorous methodology for incorporating collisions into the particle transport model, without coupling the equations of motions for each particle.
        
        Works by Chen, Chacón et al. \cite{Chen_Chacón_Barnes_2011, Chacón_Chen_Barnes_2013, Chen_Chacón_2014, Chen_Chacón_2015} have developed structure-preserving particle pushers for neoclassical transport in the Vlasov equations, derived from Crank--Nicolson integrators. We show these too can can derive from a FET interpretation, similarly offering potential extensions to higher-order-in-time particle pushers. The FET formulation is used also to consider how the stochastic drift terms can be incorporated into the pushers. Stochastic gyrokinetic expansions are also discussed.

        Different options for the numerical implementation of these schemes are considered.

        Due to the efficacy of FET in the development of SP timesteppers for both the fluid and kinetic component, we hope this approach will prove effective in the future for developing SP timesteppers for the full hybrid model. We hope this will give us the opportunity to incorporate previously inaccessible kinetic effects into the highly effective, modern, finite-element MHD models.
    \end{abstract}
    
    
    \newpage
    \tableofcontents
    
    
    \newpage
    \pagenumbering{arabic}
    %\linenumbers\renewcommand\thelinenumber{\color{black!50}\arabic{linenumber}}
            \input{0 - introduction/main.tex}
        \part{Research}
            \input{1 - low-noise PiC models/main.tex}
            \input{2 - kinetic component/main.tex}
            \input{3 - fluid component/main.tex}
            \input{4 - numerical implementation/main.tex}
        \part{Project Overview}
            \input{5 - research plan/main.tex}
            \input{6 - summary/main.tex}
    
    
    %\section{}
    \newpage
    \pagenumbering{gobble}
        \printbibliography


    \newpage
    \pagenumbering{roman}
    \appendix
        \part{Appendices}
            \input{8 - Hilbert complexes/main.tex}
            \input{9 - weak conservation proofs/main.tex}
\end{document}


\title{\BA{Title in Progress...}}
\author{Boris Andrews}
\affil{Mathematical Institute, University of Oxford}
\date{\today}


\begin{document}
    \pagenumbering{gobble}
    \maketitle
    
    
    \begin{abstract}
        Magnetic confinement reactors---in particular tokamaks---offer one of the most promising options for achieving practical nuclear fusion, with the potential to provide virtually limitless, clean energy. The theoretical and numerical modeling of tokamak plasmas is simultaneously an essential component of effective reactor design, and a great research barrier. Tokamak operational conditions exhibit comparatively low Knudsen numbers. Kinetic effects, including kinetic waves and instabilities, Landau damping, bump-on-tail instabilities and more, are therefore highly influential in tokamak plasma dynamics. Purely fluid models are inherently incapable of capturing these effects, whereas the high dimensionality in purely kinetic models render them practically intractable for most relevant purposes.

        We consider a $\delta\!f$ decomposition model, with a macroscopic fluid background and microscopic kinetic correction, both fully coupled to each other. A similar manner of discretization is proposed to that used in the recent \texttt{STRUPHY} code \cite{Holderied_Possanner_Wang_2021, Holderied_2022, Li_et_al_2023} with a finite-element model for the background and a pseudo-particle/PiC model for the correction.

        The fluid background satisfies the full, non-linear, resistive, compressible, Hall MHD equations. \cite{Laakmann_Hu_Farrell_2022} introduces finite-element(-in-space) implicit timesteppers for the incompressible analogue to this system with structure-preserving (SP) properties in the ideal case, alongside parameter-robust preconditioners. We show that these timesteppers can derive from a finite-element-in-time (FET) (and finite-element-in-space) interpretation. The benefits of this reformulation are discussed, including the derivation of timesteppers that are higher order in time, and the quantifiable dissipative SP properties in the non-ideal, resistive case.
        
        We discuss possible options for extending this FET approach to timesteppers for the compressible case.

        The kinetic corrections satisfy linearized Boltzmann equations. Using a Lénard--Bernstein collision operator, these take Fokker--Planck-like forms \cite{Fokker_1914, Planck_1917} wherein pseudo-particles in the numerical model obey the neoclassical transport equations, with particle-independent Brownian drift terms. This offers a rigorous methodology for incorporating collisions into the particle transport model, without coupling the equations of motions for each particle.
        
        Works by Chen, Chacón et al. \cite{Chen_Chacón_Barnes_2011, Chacón_Chen_Barnes_2013, Chen_Chacón_2014, Chen_Chacón_2015} have developed structure-preserving particle pushers for neoclassical transport in the Vlasov equations, derived from Crank--Nicolson integrators. We show these too can can derive from a FET interpretation, similarly offering potential extensions to higher-order-in-time particle pushers. The FET formulation is used also to consider how the stochastic drift terms can be incorporated into the pushers. Stochastic gyrokinetic expansions are also discussed.

        Different options for the numerical implementation of these schemes are considered.

        Due to the efficacy of FET in the development of SP timesteppers for both the fluid and kinetic component, we hope this approach will prove effective in the future for developing SP timesteppers for the full hybrid model. We hope this will give us the opportunity to incorporate previously inaccessible kinetic effects into the highly effective, modern, finite-element MHD models.
    \end{abstract}
    
    
    \newpage
    \tableofcontents
    
    
    \newpage
    \pagenumbering{arabic}
    %\linenumbers\renewcommand\thelinenumber{\color{black!50}\arabic{linenumber}}
            \documentclass[12pt, a4paper]{report}

\input{template/main.tex}

\title{\BA{Title in Progress...}}
\author{Boris Andrews}
\affil{Mathematical Institute, University of Oxford}
\date{\today}


\begin{document}
    \pagenumbering{gobble}
    \maketitle
    
    
    \begin{abstract}
        Magnetic confinement reactors---in particular tokamaks---offer one of the most promising options for achieving practical nuclear fusion, with the potential to provide virtually limitless, clean energy. The theoretical and numerical modeling of tokamak plasmas is simultaneously an essential component of effective reactor design, and a great research barrier. Tokamak operational conditions exhibit comparatively low Knudsen numbers. Kinetic effects, including kinetic waves and instabilities, Landau damping, bump-on-tail instabilities and more, are therefore highly influential in tokamak plasma dynamics. Purely fluid models are inherently incapable of capturing these effects, whereas the high dimensionality in purely kinetic models render them practically intractable for most relevant purposes.

        We consider a $\delta\!f$ decomposition model, with a macroscopic fluid background and microscopic kinetic correction, both fully coupled to each other. A similar manner of discretization is proposed to that used in the recent \texttt{STRUPHY} code \cite{Holderied_Possanner_Wang_2021, Holderied_2022, Li_et_al_2023} with a finite-element model for the background and a pseudo-particle/PiC model for the correction.

        The fluid background satisfies the full, non-linear, resistive, compressible, Hall MHD equations. \cite{Laakmann_Hu_Farrell_2022} introduces finite-element(-in-space) implicit timesteppers for the incompressible analogue to this system with structure-preserving (SP) properties in the ideal case, alongside parameter-robust preconditioners. We show that these timesteppers can derive from a finite-element-in-time (FET) (and finite-element-in-space) interpretation. The benefits of this reformulation are discussed, including the derivation of timesteppers that are higher order in time, and the quantifiable dissipative SP properties in the non-ideal, resistive case.
        
        We discuss possible options for extending this FET approach to timesteppers for the compressible case.

        The kinetic corrections satisfy linearized Boltzmann equations. Using a Lénard--Bernstein collision operator, these take Fokker--Planck-like forms \cite{Fokker_1914, Planck_1917} wherein pseudo-particles in the numerical model obey the neoclassical transport equations, with particle-independent Brownian drift terms. This offers a rigorous methodology for incorporating collisions into the particle transport model, without coupling the equations of motions for each particle.
        
        Works by Chen, Chacón et al. \cite{Chen_Chacón_Barnes_2011, Chacón_Chen_Barnes_2013, Chen_Chacón_2014, Chen_Chacón_2015} have developed structure-preserving particle pushers for neoclassical transport in the Vlasov equations, derived from Crank--Nicolson integrators. We show these too can can derive from a FET interpretation, similarly offering potential extensions to higher-order-in-time particle pushers. The FET formulation is used also to consider how the stochastic drift terms can be incorporated into the pushers. Stochastic gyrokinetic expansions are also discussed.

        Different options for the numerical implementation of these schemes are considered.

        Due to the efficacy of FET in the development of SP timesteppers for both the fluid and kinetic component, we hope this approach will prove effective in the future for developing SP timesteppers for the full hybrid model. We hope this will give us the opportunity to incorporate previously inaccessible kinetic effects into the highly effective, modern, finite-element MHD models.
    \end{abstract}
    
    
    \newpage
    \tableofcontents
    
    
    \newpage
    \pagenumbering{arabic}
    %\linenumbers\renewcommand\thelinenumber{\color{black!50}\arabic{linenumber}}
            \input{0 - introduction/main.tex}
        \part{Research}
            \input{1 - low-noise PiC models/main.tex}
            \input{2 - kinetic component/main.tex}
            \input{3 - fluid component/main.tex}
            \input{4 - numerical implementation/main.tex}
        \part{Project Overview}
            \input{5 - research plan/main.tex}
            \input{6 - summary/main.tex}
    
    
    %\section{}
    \newpage
    \pagenumbering{gobble}
        \printbibliography


    \newpage
    \pagenumbering{roman}
    \appendix
        \part{Appendices}
            \input{8 - Hilbert complexes/main.tex}
            \input{9 - weak conservation proofs/main.tex}
\end{document}

        \part{Research}
            \documentclass[12pt, a4paper]{report}

\input{template/main.tex}

\title{\BA{Title in Progress...}}
\author{Boris Andrews}
\affil{Mathematical Institute, University of Oxford}
\date{\today}


\begin{document}
    \pagenumbering{gobble}
    \maketitle
    
    
    \begin{abstract}
        Magnetic confinement reactors---in particular tokamaks---offer one of the most promising options for achieving practical nuclear fusion, with the potential to provide virtually limitless, clean energy. The theoretical and numerical modeling of tokamak plasmas is simultaneously an essential component of effective reactor design, and a great research barrier. Tokamak operational conditions exhibit comparatively low Knudsen numbers. Kinetic effects, including kinetic waves and instabilities, Landau damping, bump-on-tail instabilities and more, are therefore highly influential in tokamak plasma dynamics. Purely fluid models are inherently incapable of capturing these effects, whereas the high dimensionality in purely kinetic models render them practically intractable for most relevant purposes.

        We consider a $\delta\!f$ decomposition model, with a macroscopic fluid background and microscopic kinetic correction, both fully coupled to each other. A similar manner of discretization is proposed to that used in the recent \texttt{STRUPHY} code \cite{Holderied_Possanner_Wang_2021, Holderied_2022, Li_et_al_2023} with a finite-element model for the background and a pseudo-particle/PiC model for the correction.

        The fluid background satisfies the full, non-linear, resistive, compressible, Hall MHD equations. \cite{Laakmann_Hu_Farrell_2022} introduces finite-element(-in-space) implicit timesteppers for the incompressible analogue to this system with structure-preserving (SP) properties in the ideal case, alongside parameter-robust preconditioners. We show that these timesteppers can derive from a finite-element-in-time (FET) (and finite-element-in-space) interpretation. The benefits of this reformulation are discussed, including the derivation of timesteppers that are higher order in time, and the quantifiable dissipative SP properties in the non-ideal, resistive case.
        
        We discuss possible options for extending this FET approach to timesteppers for the compressible case.

        The kinetic corrections satisfy linearized Boltzmann equations. Using a Lénard--Bernstein collision operator, these take Fokker--Planck-like forms \cite{Fokker_1914, Planck_1917} wherein pseudo-particles in the numerical model obey the neoclassical transport equations, with particle-independent Brownian drift terms. This offers a rigorous methodology for incorporating collisions into the particle transport model, without coupling the equations of motions for each particle.
        
        Works by Chen, Chacón et al. \cite{Chen_Chacón_Barnes_2011, Chacón_Chen_Barnes_2013, Chen_Chacón_2014, Chen_Chacón_2015} have developed structure-preserving particle pushers for neoclassical transport in the Vlasov equations, derived from Crank--Nicolson integrators. We show these too can can derive from a FET interpretation, similarly offering potential extensions to higher-order-in-time particle pushers. The FET formulation is used also to consider how the stochastic drift terms can be incorporated into the pushers. Stochastic gyrokinetic expansions are also discussed.

        Different options for the numerical implementation of these schemes are considered.

        Due to the efficacy of FET in the development of SP timesteppers for both the fluid and kinetic component, we hope this approach will prove effective in the future for developing SP timesteppers for the full hybrid model. We hope this will give us the opportunity to incorporate previously inaccessible kinetic effects into the highly effective, modern, finite-element MHD models.
    \end{abstract}
    
    
    \newpage
    \tableofcontents
    
    
    \newpage
    \pagenumbering{arabic}
    %\linenumbers\renewcommand\thelinenumber{\color{black!50}\arabic{linenumber}}
            \input{0 - introduction/main.tex}
        \part{Research}
            \input{1 - low-noise PiC models/main.tex}
            \input{2 - kinetic component/main.tex}
            \input{3 - fluid component/main.tex}
            \input{4 - numerical implementation/main.tex}
        \part{Project Overview}
            \input{5 - research plan/main.tex}
            \input{6 - summary/main.tex}
    
    
    %\section{}
    \newpage
    \pagenumbering{gobble}
        \printbibliography


    \newpage
    \pagenumbering{roman}
    \appendix
        \part{Appendices}
            \input{8 - Hilbert complexes/main.tex}
            \input{9 - weak conservation proofs/main.tex}
\end{document}

            \documentclass[12pt, a4paper]{report}

\input{template/main.tex}

\title{\BA{Title in Progress...}}
\author{Boris Andrews}
\affil{Mathematical Institute, University of Oxford}
\date{\today}


\begin{document}
    \pagenumbering{gobble}
    \maketitle
    
    
    \begin{abstract}
        Magnetic confinement reactors---in particular tokamaks---offer one of the most promising options for achieving practical nuclear fusion, with the potential to provide virtually limitless, clean energy. The theoretical and numerical modeling of tokamak plasmas is simultaneously an essential component of effective reactor design, and a great research barrier. Tokamak operational conditions exhibit comparatively low Knudsen numbers. Kinetic effects, including kinetic waves and instabilities, Landau damping, bump-on-tail instabilities and more, are therefore highly influential in tokamak plasma dynamics. Purely fluid models are inherently incapable of capturing these effects, whereas the high dimensionality in purely kinetic models render them practically intractable for most relevant purposes.

        We consider a $\delta\!f$ decomposition model, with a macroscopic fluid background and microscopic kinetic correction, both fully coupled to each other. A similar manner of discretization is proposed to that used in the recent \texttt{STRUPHY} code \cite{Holderied_Possanner_Wang_2021, Holderied_2022, Li_et_al_2023} with a finite-element model for the background and a pseudo-particle/PiC model for the correction.

        The fluid background satisfies the full, non-linear, resistive, compressible, Hall MHD equations. \cite{Laakmann_Hu_Farrell_2022} introduces finite-element(-in-space) implicit timesteppers for the incompressible analogue to this system with structure-preserving (SP) properties in the ideal case, alongside parameter-robust preconditioners. We show that these timesteppers can derive from a finite-element-in-time (FET) (and finite-element-in-space) interpretation. The benefits of this reformulation are discussed, including the derivation of timesteppers that are higher order in time, and the quantifiable dissipative SP properties in the non-ideal, resistive case.
        
        We discuss possible options for extending this FET approach to timesteppers for the compressible case.

        The kinetic corrections satisfy linearized Boltzmann equations. Using a Lénard--Bernstein collision operator, these take Fokker--Planck-like forms \cite{Fokker_1914, Planck_1917} wherein pseudo-particles in the numerical model obey the neoclassical transport equations, with particle-independent Brownian drift terms. This offers a rigorous methodology for incorporating collisions into the particle transport model, without coupling the equations of motions for each particle.
        
        Works by Chen, Chacón et al. \cite{Chen_Chacón_Barnes_2011, Chacón_Chen_Barnes_2013, Chen_Chacón_2014, Chen_Chacón_2015} have developed structure-preserving particle pushers for neoclassical transport in the Vlasov equations, derived from Crank--Nicolson integrators. We show these too can can derive from a FET interpretation, similarly offering potential extensions to higher-order-in-time particle pushers. The FET formulation is used also to consider how the stochastic drift terms can be incorporated into the pushers. Stochastic gyrokinetic expansions are also discussed.

        Different options for the numerical implementation of these schemes are considered.

        Due to the efficacy of FET in the development of SP timesteppers for both the fluid and kinetic component, we hope this approach will prove effective in the future for developing SP timesteppers for the full hybrid model. We hope this will give us the opportunity to incorporate previously inaccessible kinetic effects into the highly effective, modern, finite-element MHD models.
    \end{abstract}
    
    
    \newpage
    \tableofcontents
    
    
    \newpage
    \pagenumbering{arabic}
    %\linenumbers\renewcommand\thelinenumber{\color{black!50}\arabic{linenumber}}
            \input{0 - introduction/main.tex}
        \part{Research}
            \input{1 - low-noise PiC models/main.tex}
            \input{2 - kinetic component/main.tex}
            \input{3 - fluid component/main.tex}
            \input{4 - numerical implementation/main.tex}
        \part{Project Overview}
            \input{5 - research plan/main.tex}
            \input{6 - summary/main.tex}
    
    
    %\section{}
    \newpage
    \pagenumbering{gobble}
        \printbibliography


    \newpage
    \pagenumbering{roman}
    \appendix
        \part{Appendices}
            \input{8 - Hilbert complexes/main.tex}
            \input{9 - weak conservation proofs/main.tex}
\end{document}

            \documentclass[12pt, a4paper]{report}

\input{template/main.tex}

\title{\BA{Title in Progress...}}
\author{Boris Andrews}
\affil{Mathematical Institute, University of Oxford}
\date{\today}


\begin{document}
    \pagenumbering{gobble}
    \maketitle
    
    
    \begin{abstract}
        Magnetic confinement reactors---in particular tokamaks---offer one of the most promising options for achieving practical nuclear fusion, with the potential to provide virtually limitless, clean energy. The theoretical and numerical modeling of tokamak plasmas is simultaneously an essential component of effective reactor design, and a great research barrier. Tokamak operational conditions exhibit comparatively low Knudsen numbers. Kinetic effects, including kinetic waves and instabilities, Landau damping, bump-on-tail instabilities and more, are therefore highly influential in tokamak plasma dynamics. Purely fluid models are inherently incapable of capturing these effects, whereas the high dimensionality in purely kinetic models render them practically intractable for most relevant purposes.

        We consider a $\delta\!f$ decomposition model, with a macroscopic fluid background and microscopic kinetic correction, both fully coupled to each other. A similar manner of discretization is proposed to that used in the recent \texttt{STRUPHY} code \cite{Holderied_Possanner_Wang_2021, Holderied_2022, Li_et_al_2023} with a finite-element model for the background and a pseudo-particle/PiC model for the correction.

        The fluid background satisfies the full, non-linear, resistive, compressible, Hall MHD equations. \cite{Laakmann_Hu_Farrell_2022} introduces finite-element(-in-space) implicit timesteppers for the incompressible analogue to this system with structure-preserving (SP) properties in the ideal case, alongside parameter-robust preconditioners. We show that these timesteppers can derive from a finite-element-in-time (FET) (and finite-element-in-space) interpretation. The benefits of this reformulation are discussed, including the derivation of timesteppers that are higher order in time, and the quantifiable dissipative SP properties in the non-ideal, resistive case.
        
        We discuss possible options for extending this FET approach to timesteppers for the compressible case.

        The kinetic corrections satisfy linearized Boltzmann equations. Using a Lénard--Bernstein collision operator, these take Fokker--Planck-like forms \cite{Fokker_1914, Planck_1917} wherein pseudo-particles in the numerical model obey the neoclassical transport equations, with particle-independent Brownian drift terms. This offers a rigorous methodology for incorporating collisions into the particle transport model, without coupling the equations of motions for each particle.
        
        Works by Chen, Chacón et al. \cite{Chen_Chacón_Barnes_2011, Chacón_Chen_Barnes_2013, Chen_Chacón_2014, Chen_Chacón_2015} have developed structure-preserving particle pushers for neoclassical transport in the Vlasov equations, derived from Crank--Nicolson integrators. We show these too can can derive from a FET interpretation, similarly offering potential extensions to higher-order-in-time particle pushers. The FET formulation is used also to consider how the stochastic drift terms can be incorporated into the pushers. Stochastic gyrokinetic expansions are also discussed.

        Different options for the numerical implementation of these schemes are considered.

        Due to the efficacy of FET in the development of SP timesteppers for both the fluid and kinetic component, we hope this approach will prove effective in the future for developing SP timesteppers for the full hybrid model. We hope this will give us the opportunity to incorporate previously inaccessible kinetic effects into the highly effective, modern, finite-element MHD models.
    \end{abstract}
    
    
    \newpage
    \tableofcontents
    
    
    \newpage
    \pagenumbering{arabic}
    %\linenumbers\renewcommand\thelinenumber{\color{black!50}\arabic{linenumber}}
            \input{0 - introduction/main.tex}
        \part{Research}
            \input{1 - low-noise PiC models/main.tex}
            \input{2 - kinetic component/main.tex}
            \input{3 - fluid component/main.tex}
            \input{4 - numerical implementation/main.tex}
        \part{Project Overview}
            \input{5 - research plan/main.tex}
            \input{6 - summary/main.tex}
    
    
    %\section{}
    \newpage
    \pagenumbering{gobble}
        \printbibliography


    \newpage
    \pagenumbering{roman}
    \appendix
        \part{Appendices}
            \input{8 - Hilbert complexes/main.tex}
            \input{9 - weak conservation proofs/main.tex}
\end{document}

            \documentclass[12pt, a4paper]{report}

\input{template/main.tex}

\title{\BA{Title in Progress...}}
\author{Boris Andrews}
\affil{Mathematical Institute, University of Oxford}
\date{\today}


\begin{document}
    \pagenumbering{gobble}
    \maketitle
    
    
    \begin{abstract}
        Magnetic confinement reactors---in particular tokamaks---offer one of the most promising options for achieving practical nuclear fusion, with the potential to provide virtually limitless, clean energy. The theoretical and numerical modeling of tokamak plasmas is simultaneously an essential component of effective reactor design, and a great research barrier. Tokamak operational conditions exhibit comparatively low Knudsen numbers. Kinetic effects, including kinetic waves and instabilities, Landau damping, bump-on-tail instabilities and more, are therefore highly influential in tokamak plasma dynamics. Purely fluid models are inherently incapable of capturing these effects, whereas the high dimensionality in purely kinetic models render them practically intractable for most relevant purposes.

        We consider a $\delta\!f$ decomposition model, with a macroscopic fluid background and microscopic kinetic correction, both fully coupled to each other. A similar manner of discretization is proposed to that used in the recent \texttt{STRUPHY} code \cite{Holderied_Possanner_Wang_2021, Holderied_2022, Li_et_al_2023} with a finite-element model for the background and a pseudo-particle/PiC model for the correction.

        The fluid background satisfies the full, non-linear, resistive, compressible, Hall MHD equations. \cite{Laakmann_Hu_Farrell_2022} introduces finite-element(-in-space) implicit timesteppers for the incompressible analogue to this system with structure-preserving (SP) properties in the ideal case, alongside parameter-robust preconditioners. We show that these timesteppers can derive from a finite-element-in-time (FET) (and finite-element-in-space) interpretation. The benefits of this reformulation are discussed, including the derivation of timesteppers that are higher order in time, and the quantifiable dissipative SP properties in the non-ideal, resistive case.
        
        We discuss possible options for extending this FET approach to timesteppers for the compressible case.

        The kinetic corrections satisfy linearized Boltzmann equations. Using a Lénard--Bernstein collision operator, these take Fokker--Planck-like forms \cite{Fokker_1914, Planck_1917} wherein pseudo-particles in the numerical model obey the neoclassical transport equations, with particle-independent Brownian drift terms. This offers a rigorous methodology for incorporating collisions into the particle transport model, without coupling the equations of motions for each particle.
        
        Works by Chen, Chacón et al. \cite{Chen_Chacón_Barnes_2011, Chacón_Chen_Barnes_2013, Chen_Chacón_2014, Chen_Chacón_2015} have developed structure-preserving particle pushers for neoclassical transport in the Vlasov equations, derived from Crank--Nicolson integrators. We show these too can can derive from a FET interpretation, similarly offering potential extensions to higher-order-in-time particle pushers. The FET formulation is used also to consider how the stochastic drift terms can be incorporated into the pushers. Stochastic gyrokinetic expansions are also discussed.

        Different options for the numerical implementation of these schemes are considered.

        Due to the efficacy of FET in the development of SP timesteppers for both the fluid and kinetic component, we hope this approach will prove effective in the future for developing SP timesteppers for the full hybrid model. We hope this will give us the opportunity to incorporate previously inaccessible kinetic effects into the highly effective, modern, finite-element MHD models.
    \end{abstract}
    
    
    \newpage
    \tableofcontents
    
    
    \newpage
    \pagenumbering{arabic}
    %\linenumbers\renewcommand\thelinenumber{\color{black!50}\arabic{linenumber}}
            \input{0 - introduction/main.tex}
        \part{Research}
            \input{1 - low-noise PiC models/main.tex}
            \input{2 - kinetic component/main.tex}
            \input{3 - fluid component/main.tex}
            \input{4 - numerical implementation/main.tex}
        \part{Project Overview}
            \input{5 - research plan/main.tex}
            \input{6 - summary/main.tex}
    
    
    %\section{}
    \newpage
    \pagenumbering{gobble}
        \printbibliography


    \newpage
    \pagenumbering{roman}
    \appendix
        \part{Appendices}
            \input{8 - Hilbert complexes/main.tex}
            \input{9 - weak conservation proofs/main.tex}
\end{document}

        \part{Project Overview}
            \documentclass[12pt, a4paper]{report}

\input{template/main.tex}

\title{\BA{Title in Progress...}}
\author{Boris Andrews}
\affil{Mathematical Institute, University of Oxford}
\date{\today}


\begin{document}
    \pagenumbering{gobble}
    \maketitle
    
    
    \begin{abstract}
        Magnetic confinement reactors---in particular tokamaks---offer one of the most promising options for achieving practical nuclear fusion, with the potential to provide virtually limitless, clean energy. The theoretical and numerical modeling of tokamak plasmas is simultaneously an essential component of effective reactor design, and a great research barrier. Tokamak operational conditions exhibit comparatively low Knudsen numbers. Kinetic effects, including kinetic waves and instabilities, Landau damping, bump-on-tail instabilities and more, are therefore highly influential in tokamak plasma dynamics. Purely fluid models are inherently incapable of capturing these effects, whereas the high dimensionality in purely kinetic models render them practically intractable for most relevant purposes.

        We consider a $\delta\!f$ decomposition model, with a macroscopic fluid background and microscopic kinetic correction, both fully coupled to each other. A similar manner of discretization is proposed to that used in the recent \texttt{STRUPHY} code \cite{Holderied_Possanner_Wang_2021, Holderied_2022, Li_et_al_2023} with a finite-element model for the background and a pseudo-particle/PiC model for the correction.

        The fluid background satisfies the full, non-linear, resistive, compressible, Hall MHD equations. \cite{Laakmann_Hu_Farrell_2022} introduces finite-element(-in-space) implicit timesteppers for the incompressible analogue to this system with structure-preserving (SP) properties in the ideal case, alongside parameter-robust preconditioners. We show that these timesteppers can derive from a finite-element-in-time (FET) (and finite-element-in-space) interpretation. The benefits of this reformulation are discussed, including the derivation of timesteppers that are higher order in time, and the quantifiable dissipative SP properties in the non-ideal, resistive case.
        
        We discuss possible options for extending this FET approach to timesteppers for the compressible case.

        The kinetic corrections satisfy linearized Boltzmann equations. Using a Lénard--Bernstein collision operator, these take Fokker--Planck-like forms \cite{Fokker_1914, Planck_1917} wherein pseudo-particles in the numerical model obey the neoclassical transport equations, with particle-independent Brownian drift terms. This offers a rigorous methodology for incorporating collisions into the particle transport model, without coupling the equations of motions for each particle.
        
        Works by Chen, Chacón et al. \cite{Chen_Chacón_Barnes_2011, Chacón_Chen_Barnes_2013, Chen_Chacón_2014, Chen_Chacón_2015} have developed structure-preserving particle pushers for neoclassical transport in the Vlasov equations, derived from Crank--Nicolson integrators. We show these too can can derive from a FET interpretation, similarly offering potential extensions to higher-order-in-time particle pushers. The FET formulation is used also to consider how the stochastic drift terms can be incorporated into the pushers. Stochastic gyrokinetic expansions are also discussed.

        Different options for the numerical implementation of these schemes are considered.

        Due to the efficacy of FET in the development of SP timesteppers for both the fluid and kinetic component, we hope this approach will prove effective in the future for developing SP timesteppers for the full hybrid model. We hope this will give us the opportunity to incorporate previously inaccessible kinetic effects into the highly effective, modern, finite-element MHD models.
    \end{abstract}
    
    
    \newpage
    \tableofcontents
    
    
    \newpage
    \pagenumbering{arabic}
    %\linenumbers\renewcommand\thelinenumber{\color{black!50}\arabic{linenumber}}
            \input{0 - introduction/main.tex}
        \part{Research}
            \input{1 - low-noise PiC models/main.tex}
            \input{2 - kinetic component/main.tex}
            \input{3 - fluid component/main.tex}
            \input{4 - numerical implementation/main.tex}
        \part{Project Overview}
            \input{5 - research plan/main.tex}
            \input{6 - summary/main.tex}
    
    
    %\section{}
    \newpage
    \pagenumbering{gobble}
        \printbibliography


    \newpage
    \pagenumbering{roman}
    \appendix
        \part{Appendices}
            \input{8 - Hilbert complexes/main.tex}
            \input{9 - weak conservation proofs/main.tex}
\end{document}

            \documentclass[12pt, a4paper]{report}

\input{template/main.tex}

\title{\BA{Title in Progress...}}
\author{Boris Andrews}
\affil{Mathematical Institute, University of Oxford}
\date{\today}


\begin{document}
    \pagenumbering{gobble}
    \maketitle
    
    
    \begin{abstract}
        Magnetic confinement reactors---in particular tokamaks---offer one of the most promising options for achieving practical nuclear fusion, with the potential to provide virtually limitless, clean energy. The theoretical and numerical modeling of tokamak plasmas is simultaneously an essential component of effective reactor design, and a great research barrier. Tokamak operational conditions exhibit comparatively low Knudsen numbers. Kinetic effects, including kinetic waves and instabilities, Landau damping, bump-on-tail instabilities and more, are therefore highly influential in tokamak plasma dynamics. Purely fluid models are inherently incapable of capturing these effects, whereas the high dimensionality in purely kinetic models render them practically intractable for most relevant purposes.

        We consider a $\delta\!f$ decomposition model, with a macroscopic fluid background and microscopic kinetic correction, both fully coupled to each other. A similar manner of discretization is proposed to that used in the recent \texttt{STRUPHY} code \cite{Holderied_Possanner_Wang_2021, Holderied_2022, Li_et_al_2023} with a finite-element model for the background and a pseudo-particle/PiC model for the correction.

        The fluid background satisfies the full, non-linear, resistive, compressible, Hall MHD equations. \cite{Laakmann_Hu_Farrell_2022} introduces finite-element(-in-space) implicit timesteppers for the incompressible analogue to this system with structure-preserving (SP) properties in the ideal case, alongside parameter-robust preconditioners. We show that these timesteppers can derive from a finite-element-in-time (FET) (and finite-element-in-space) interpretation. The benefits of this reformulation are discussed, including the derivation of timesteppers that are higher order in time, and the quantifiable dissipative SP properties in the non-ideal, resistive case.
        
        We discuss possible options for extending this FET approach to timesteppers for the compressible case.

        The kinetic corrections satisfy linearized Boltzmann equations. Using a Lénard--Bernstein collision operator, these take Fokker--Planck-like forms \cite{Fokker_1914, Planck_1917} wherein pseudo-particles in the numerical model obey the neoclassical transport equations, with particle-independent Brownian drift terms. This offers a rigorous methodology for incorporating collisions into the particle transport model, without coupling the equations of motions for each particle.
        
        Works by Chen, Chacón et al. \cite{Chen_Chacón_Barnes_2011, Chacón_Chen_Barnes_2013, Chen_Chacón_2014, Chen_Chacón_2015} have developed structure-preserving particle pushers for neoclassical transport in the Vlasov equations, derived from Crank--Nicolson integrators. We show these too can can derive from a FET interpretation, similarly offering potential extensions to higher-order-in-time particle pushers. The FET formulation is used also to consider how the stochastic drift terms can be incorporated into the pushers. Stochastic gyrokinetic expansions are also discussed.

        Different options for the numerical implementation of these schemes are considered.

        Due to the efficacy of FET in the development of SP timesteppers for both the fluid and kinetic component, we hope this approach will prove effective in the future for developing SP timesteppers for the full hybrid model. We hope this will give us the opportunity to incorporate previously inaccessible kinetic effects into the highly effective, modern, finite-element MHD models.
    \end{abstract}
    
    
    \newpage
    \tableofcontents
    
    
    \newpage
    \pagenumbering{arabic}
    %\linenumbers\renewcommand\thelinenumber{\color{black!50}\arabic{linenumber}}
            \input{0 - introduction/main.tex}
        \part{Research}
            \input{1 - low-noise PiC models/main.tex}
            \input{2 - kinetic component/main.tex}
            \input{3 - fluid component/main.tex}
            \input{4 - numerical implementation/main.tex}
        \part{Project Overview}
            \input{5 - research plan/main.tex}
            \input{6 - summary/main.tex}
    
    
    %\section{}
    \newpage
    \pagenumbering{gobble}
        \printbibliography


    \newpage
    \pagenumbering{roman}
    \appendix
        \part{Appendices}
            \input{8 - Hilbert complexes/main.tex}
            \input{9 - weak conservation proofs/main.tex}
\end{document}

    
    
    %\section{}
    \newpage
    \pagenumbering{gobble}
        \printbibliography


    \newpage
    \pagenumbering{roman}
    \appendix
        \part{Appendices}
            \documentclass[12pt, a4paper]{report}

\input{template/main.tex}

\title{\BA{Title in Progress...}}
\author{Boris Andrews}
\affil{Mathematical Institute, University of Oxford}
\date{\today}


\begin{document}
    \pagenumbering{gobble}
    \maketitle
    
    
    \begin{abstract}
        Magnetic confinement reactors---in particular tokamaks---offer one of the most promising options for achieving practical nuclear fusion, with the potential to provide virtually limitless, clean energy. The theoretical and numerical modeling of tokamak plasmas is simultaneously an essential component of effective reactor design, and a great research barrier. Tokamak operational conditions exhibit comparatively low Knudsen numbers. Kinetic effects, including kinetic waves and instabilities, Landau damping, bump-on-tail instabilities and more, are therefore highly influential in tokamak plasma dynamics. Purely fluid models are inherently incapable of capturing these effects, whereas the high dimensionality in purely kinetic models render them practically intractable for most relevant purposes.

        We consider a $\delta\!f$ decomposition model, with a macroscopic fluid background and microscopic kinetic correction, both fully coupled to each other. A similar manner of discretization is proposed to that used in the recent \texttt{STRUPHY} code \cite{Holderied_Possanner_Wang_2021, Holderied_2022, Li_et_al_2023} with a finite-element model for the background and a pseudo-particle/PiC model for the correction.

        The fluid background satisfies the full, non-linear, resistive, compressible, Hall MHD equations. \cite{Laakmann_Hu_Farrell_2022} introduces finite-element(-in-space) implicit timesteppers for the incompressible analogue to this system with structure-preserving (SP) properties in the ideal case, alongside parameter-robust preconditioners. We show that these timesteppers can derive from a finite-element-in-time (FET) (and finite-element-in-space) interpretation. The benefits of this reformulation are discussed, including the derivation of timesteppers that are higher order in time, and the quantifiable dissipative SP properties in the non-ideal, resistive case.
        
        We discuss possible options for extending this FET approach to timesteppers for the compressible case.

        The kinetic corrections satisfy linearized Boltzmann equations. Using a Lénard--Bernstein collision operator, these take Fokker--Planck-like forms \cite{Fokker_1914, Planck_1917} wherein pseudo-particles in the numerical model obey the neoclassical transport equations, with particle-independent Brownian drift terms. This offers a rigorous methodology for incorporating collisions into the particle transport model, without coupling the equations of motions for each particle.
        
        Works by Chen, Chacón et al. \cite{Chen_Chacón_Barnes_2011, Chacón_Chen_Barnes_2013, Chen_Chacón_2014, Chen_Chacón_2015} have developed structure-preserving particle pushers for neoclassical transport in the Vlasov equations, derived from Crank--Nicolson integrators. We show these too can can derive from a FET interpretation, similarly offering potential extensions to higher-order-in-time particle pushers. The FET formulation is used also to consider how the stochastic drift terms can be incorporated into the pushers. Stochastic gyrokinetic expansions are also discussed.

        Different options for the numerical implementation of these schemes are considered.

        Due to the efficacy of FET in the development of SP timesteppers for both the fluid and kinetic component, we hope this approach will prove effective in the future for developing SP timesteppers for the full hybrid model. We hope this will give us the opportunity to incorporate previously inaccessible kinetic effects into the highly effective, modern, finite-element MHD models.
    \end{abstract}
    
    
    \newpage
    \tableofcontents
    
    
    \newpage
    \pagenumbering{arabic}
    %\linenumbers\renewcommand\thelinenumber{\color{black!50}\arabic{linenumber}}
            \input{0 - introduction/main.tex}
        \part{Research}
            \input{1 - low-noise PiC models/main.tex}
            \input{2 - kinetic component/main.tex}
            \input{3 - fluid component/main.tex}
            \input{4 - numerical implementation/main.tex}
        \part{Project Overview}
            \input{5 - research plan/main.tex}
            \input{6 - summary/main.tex}
    
    
    %\section{}
    \newpage
    \pagenumbering{gobble}
        \printbibliography


    \newpage
    \pagenumbering{roman}
    \appendix
        \part{Appendices}
            \input{8 - Hilbert complexes/main.tex}
            \input{9 - weak conservation proofs/main.tex}
\end{document}

            \documentclass[12pt, a4paper]{report}

\input{template/main.tex}

\title{\BA{Title in Progress...}}
\author{Boris Andrews}
\affil{Mathematical Institute, University of Oxford}
\date{\today}


\begin{document}
    \pagenumbering{gobble}
    \maketitle
    
    
    \begin{abstract}
        Magnetic confinement reactors---in particular tokamaks---offer one of the most promising options for achieving practical nuclear fusion, with the potential to provide virtually limitless, clean energy. The theoretical and numerical modeling of tokamak plasmas is simultaneously an essential component of effective reactor design, and a great research barrier. Tokamak operational conditions exhibit comparatively low Knudsen numbers. Kinetic effects, including kinetic waves and instabilities, Landau damping, bump-on-tail instabilities and more, are therefore highly influential in tokamak plasma dynamics. Purely fluid models are inherently incapable of capturing these effects, whereas the high dimensionality in purely kinetic models render them practically intractable for most relevant purposes.

        We consider a $\delta\!f$ decomposition model, with a macroscopic fluid background and microscopic kinetic correction, both fully coupled to each other. A similar manner of discretization is proposed to that used in the recent \texttt{STRUPHY} code \cite{Holderied_Possanner_Wang_2021, Holderied_2022, Li_et_al_2023} with a finite-element model for the background and a pseudo-particle/PiC model for the correction.

        The fluid background satisfies the full, non-linear, resistive, compressible, Hall MHD equations. \cite{Laakmann_Hu_Farrell_2022} introduces finite-element(-in-space) implicit timesteppers for the incompressible analogue to this system with structure-preserving (SP) properties in the ideal case, alongside parameter-robust preconditioners. We show that these timesteppers can derive from a finite-element-in-time (FET) (and finite-element-in-space) interpretation. The benefits of this reformulation are discussed, including the derivation of timesteppers that are higher order in time, and the quantifiable dissipative SP properties in the non-ideal, resistive case.
        
        We discuss possible options for extending this FET approach to timesteppers for the compressible case.

        The kinetic corrections satisfy linearized Boltzmann equations. Using a Lénard--Bernstein collision operator, these take Fokker--Planck-like forms \cite{Fokker_1914, Planck_1917} wherein pseudo-particles in the numerical model obey the neoclassical transport equations, with particle-independent Brownian drift terms. This offers a rigorous methodology for incorporating collisions into the particle transport model, without coupling the equations of motions for each particle.
        
        Works by Chen, Chacón et al. \cite{Chen_Chacón_Barnes_2011, Chacón_Chen_Barnes_2013, Chen_Chacón_2014, Chen_Chacón_2015} have developed structure-preserving particle pushers for neoclassical transport in the Vlasov equations, derived from Crank--Nicolson integrators. We show these too can can derive from a FET interpretation, similarly offering potential extensions to higher-order-in-time particle pushers. The FET formulation is used also to consider how the stochastic drift terms can be incorporated into the pushers. Stochastic gyrokinetic expansions are also discussed.

        Different options for the numerical implementation of these schemes are considered.

        Due to the efficacy of FET in the development of SP timesteppers for both the fluid and kinetic component, we hope this approach will prove effective in the future for developing SP timesteppers for the full hybrid model. We hope this will give us the opportunity to incorporate previously inaccessible kinetic effects into the highly effective, modern, finite-element MHD models.
    \end{abstract}
    
    
    \newpage
    \tableofcontents
    
    
    \newpage
    \pagenumbering{arabic}
    %\linenumbers\renewcommand\thelinenumber{\color{black!50}\arabic{linenumber}}
            \input{0 - introduction/main.tex}
        \part{Research}
            \input{1 - low-noise PiC models/main.tex}
            \input{2 - kinetic component/main.tex}
            \input{3 - fluid component/main.tex}
            \input{4 - numerical implementation/main.tex}
        \part{Project Overview}
            \input{5 - research plan/main.tex}
            \input{6 - summary/main.tex}
    
    
    %\section{}
    \newpage
    \pagenumbering{gobble}
        \printbibliography


    \newpage
    \pagenumbering{roman}
    \appendix
        \part{Appendices}
            \input{8 - Hilbert complexes/main.tex}
            \input{9 - weak conservation proofs/main.tex}
\end{document}

\end{document}

        \part{Research}
            \documentclass[12pt, a4paper]{report}

\documentclass[12pt, a4paper]{report}

\input{template/main.tex}

\title{\BA{Title in Progress...}}
\author{Boris Andrews}
\affil{Mathematical Institute, University of Oxford}
\date{\today}


\begin{document}
    \pagenumbering{gobble}
    \maketitle
    
    
    \begin{abstract}
        Magnetic confinement reactors---in particular tokamaks---offer one of the most promising options for achieving practical nuclear fusion, with the potential to provide virtually limitless, clean energy. The theoretical and numerical modeling of tokamak plasmas is simultaneously an essential component of effective reactor design, and a great research barrier. Tokamak operational conditions exhibit comparatively low Knudsen numbers. Kinetic effects, including kinetic waves and instabilities, Landau damping, bump-on-tail instabilities and more, are therefore highly influential in tokamak plasma dynamics. Purely fluid models are inherently incapable of capturing these effects, whereas the high dimensionality in purely kinetic models render them practically intractable for most relevant purposes.

        We consider a $\delta\!f$ decomposition model, with a macroscopic fluid background and microscopic kinetic correction, both fully coupled to each other. A similar manner of discretization is proposed to that used in the recent \texttt{STRUPHY} code \cite{Holderied_Possanner_Wang_2021, Holderied_2022, Li_et_al_2023} with a finite-element model for the background and a pseudo-particle/PiC model for the correction.

        The fluid background satisfies the full, non-linear, resistive, compressible, Hall MHD equations. \cite{Laakmann_Hu_Farrell_2022} introduces finite-element(-in-space) implicit timesteppers for the incompressible analogue to this system with structure-preserving (SP) properties in the ideal case, alongside parameter-robust preconditioners. We show that these timesteppers can derive from a finite-element-in-time (FET) (and finite-element-in-space) interpretation. The benefits of this reformulation are discussed, including the derivation of timesteppers that are higher order in time, and the quantifiable dissipative SP properties in the non-ideal, resistive case.
        
        We discuss possible options for extending this FET approach to timesteppers for the compressible case.

        The kinetic corrections satisfy linearized Boltzmann equations. Using a Lénard--Bernstein collision operator, these take Fokker--Planck-like forms \cite{Fokker_1914, Planck_1917} wherein pseudo-particles in the numerical model obey the neoclassical transport equations, with particle-independent Brownian drift terms. This offers a rigorous methodology for incorporating collisions into the particle transport model, without coupling the equations of motions for each particle.
        
        Works by Chen, Chacón et al. \cite{Chen_Chacón_Barnes_2011, Chacón_Chen_Barnes_2013, Chen_Chacón_2014, Chen_Chacón_2015} have developed structure-preserving particle pushers for neoclassical transport in the Vlasov equations, derived from Crank--Nicolson integrators. We show these too can can derive from a FET interpretation, similarly offering potential extensions to higher-order-in-time particle pushers. The FET formulation is used also to consider how the stochastic drift terms can be incorporated into the pushers. Stochastic gyrokinetic expansions are also discussed.

        Different options for the numerical implementation of these schemes are considered.

        Due to the efficacy of FET in the development of SP timesteppers for both the fluid and kinetic component, we hope this approach will prove effective in the future for developing SP timesteppers for the full hybrid model. We hope this will give us the opportunity to incorporate previously inaccessible kinetic effects into the highly effective, modern, finite-element MHD models.
    \end{abstract}
    
    
    \newpage
    \tableofcontents
    
    
    \newpage
    \pagenumbering{arabic}
    %\linenumbers\renewcommand\thelinenumber{\color{black!50}\arabic{linenumber}}
            \input{0 - introduction/main.tex}
        \part{Research}
            \input{1 - low-noise PiC models/main.tex}
            \input{2 - kinetic component/main.tex}
            \input{3 - fluid component/main.tex}
            \input{4 - numerical implementation/main.tex}
        \part{Project Overview}
            \input{5 - research plan/main.tex}
            \input{6 - summary/main.tex}
    
    
    %\section{}
    \newpage
    \pagenumbering{gobble}
        \printbibliography


    \newpage
    \pagenumbering{roman}
    \appendix
        \part{Appendices}
            \input{8 - Hilbert complexes/main.tex}
            \input{9 - weak conservation proofs/main.tex}
\end{document}


\title{\BA{Title in Progress...}}
\author{Boris Andrews}
\affil{Mathematical Institute, University of Oxford}
\date{\today}


\begin{document}
    \pagenumbering{gobble}
    \maketitle
    
    
    \begin{abstract}
        Magnetic confinement reactors---in particular tokamaks---offer one of the most promising options for achieving practical nuclear fusion, with the potential to provide virtually limitless, clean energy. The theoretical and numerical modeling of tokamak plasmas is simultaneously an essential component of effective reactor design, and a great research barrier. Tokamak operational conditions exhibit comparatively low Knudsen numbers. Kinetic effects, including kinetic waves and instabilities, Landau damping, bump-on-tail instabilities and more, are therefore highly influential in tokamak plasma dynamics. Purely fluid models are inherently incapable of capturing these effects, whereas the high dimensionality in purely kinetic models render them practically intractable for most relevant purposes.

        We consider a $\delta\!f$ decomposition model, with a macroscopic fluid background and microscopic kinetic correction, both fully coupled to each other. A similar manner of discretization is proposed to that used in the recent \texttt{STRUPHY} code \cite{Holderied_Possanner_Wang_2021, Holderied_2022, Li_et_al_2023} with a finite-element model for the background and a pseudo-particle/PiC model for the correction.

        The fluid background satisfies the full, non-linear, resistive, compressible, Hall MHD equations. \cite{Laakmann_Hu_Farrell_2022} introduces finite-element(-in-space) implicit timesteppers for the incompressible analogue to this system with structure-preserving (SP) properties in the ideal case, alongside parameter-robust preconditioners. We show that these timesteppers can derive from a finite-element-in-time (FET) (and finite-element-in-space) interpretation. The benefits of this reformulation are discussed, including the derivation of timesteppers that are higher order in time, and the quantifiable dissipative SP properties in the non-ideal, resistive case.
        
        We discuss possible options for extending this FET approach to timesteppers for the compressible case.

        The kinetic corrections satisfy linearized Boltzmann equations. Using a Lénard--Bernstein collision operator, these take Fokker--Planck-like forms \cite{Fokker_1914, Planck_1917} wherein pseudo-particles in the numerical model obey the neoclassical transport equations, with particle-independent Brownian drift terms. This offers a rigorous methodology for incorporating collisions into the particle transport model, without coupling the equations of motions for each particle.
        
        Works by Chen, Chacón et al. \cite{Chen_Chacón_Barnes_2011, Chacón_Chen_Barnes_2013, Chen_Chacón_2014, Chen_Chacón_2015} have developed structure-preserving particle pushers for neoclassical transport in the Vlasov equations, derived from Crank--Nicolson integrators. We show these too can can derive from a FET interpretation, similarly offering potential extensions to higher-order-in-time particle pushers. The FET formulation is used also to consider how the stochastic drift terms can be incorporated into the pushers. Stochastic gyrokinetic expansions are also discussed.

        Different options for the numerical implementation of these schemes are considered.

        Due to the efficacy of FET in the development of SP timesteppers for both the fluid and kinetic component, we hope this approach will prove effective in the future for developing SP timesteppers for the full hybrid model. We hope this will give us the opportunity to incorporate previously inaccessible kinetic effects into the highly effective, modern, finite-element MHD models.
    \end{abstract}
    
    
    \newpage
    \tableofcontents
    
    
    \newpage
    \pagenumbering{arabic}
    %\linenumbers\renewcommand\thelinenumber{\color{black!50}\arabic{linenumber}}
            \documentclass[12pt, a4paper]{report}

\input{template/main.tex}

\title{\BA{Title in Progress...}}
\author{Boris Andrews}
\affil{Mathematical Institute, University of Oxford}
\date{\today}


\begin{document}
    \pagenumbering{gobble}
    \maketitle
    
    
    \begin{abstract}
        Magnetic confinement reactors---in particular tokamaks---offer one of the most promising options for achieving practical nuclear fusion, with the potential to provide virtually limitless, clean energy. The theoretical and numerical modeling of tokamak plasmas is simultaneously an essential component of effective reactor design, and a great research barrier. Tokamak operational conditions exhibit comparatively low Knudsen numbers. Kinetic effects, including kinetic waves and instabilities, Landau damping, bump-on-tail instabilities and more, are therefore highly influential in tokamak plasma dynamics. Purely fluid models are inherently incapable of capturing these effects, whereas the high dimensionality in purely kinetic models render them practically intractable for most relevant purposes.

        We consider a $\delta\!f$ decomposition model, with a macroscopic fluid background and microscopic kinetic correction, both fully coupled to each other. A similar manner of discretization is proposed to that used in the recent \texttt{STRUPHY} code \cite{Holderied_Possanner_Wang_2021, Holderied_2022, Li_et_al_2023} with a finite-element model for the background and a pseudo-particle/PiC model for the correction.

        The fluid background satisfies the full, non-linear, resistive, compressible, Hall MHD equations. \cite{Laakmann_Hu_Farrell_2022} introduces finite-element(-in-space) implicit timesteppers for the incompressible analogue to this system with structure-preserving (SP) properties in the ideal case, alongside parameter-robust preconditioners. We show that these timesteppers can derive from a finite-element-in-time (FET) (and finite-element-in-space) interpretation. The benefits of this reformulation are discussed, including the derivation of timesteppers that are higher order in time, and the quantifiable dissipative SP properties in the non-ideal, resistive case.
        
        We discuss possible options for extending this FET approach to timesteppers for the compressible case.

        The kinetic corrections satisfy linearized Boltzmann equations. Using a Lénard--Bernstein collision operator, these take Fokker--Planck-like forms \cite{Fokker_1914, Planck_1917} wherein pseudo-particles in the numerical model obey the neoclassical transport equations, with particle-independent Brownian drift terms. This offers a rigorous methodology for incorporating collisions into the particle transport model, without coupling the equations of motions for each particle.
        
        Works by Chen, Chacón et al. \cite{Chen_Chacón_Barnes_2011, Chacón_Chen_Barnes_2013, Chen_Chacón_2014, Chen_Chacón_2015} have developed structure-preserving particle pushers for neoclassical transport in the Vlasov equations, derived from Crank--Nicolson integrators. We show these too can can derive from a FET interpretation, similarly offering potential extensions to higher-order-in-time particle pushers. The FET formulation is used also to consider how the stochastic drift terms can be incorporated into the pushers. Stochastic gyrokinetic expansions are also discussed.

        Different options for the numerical implementation of these schemes are considered.

        Due to the efficacy of FET in the development of SP timesteppers for both the fluid and kinetic component, we hope this approach will prove effective in the future for developing SP timesteppers for the full hybrid model. We hope this will give us the opportunity to incorporate previously inaccessible kinetic effects into the highly effective, modern, finite-element MHD models.
    \end{abstract}
    
    
    \newpage
    \tableofcontents
    
    
    \newpage
    \pagenumbering{arabic}
    %\linenumbers\renewcommand\thelinenumber{\color{black!50}\arabic{linenumber}}
            \input{0 - introduction/main.tex}
        \part{Research}
            \input{1 - low-noise PiC models/main.tex}
            \input{2 - kinetic component/main.tex}
            \input{3 - fluid component/main.tex}
            \input{4 - numerical implementation/main.tex}
        \part{Project Overview}
            \input{5 - research plan/main.tex}
            \input{6 - summary/main.tex}
    
    
    %\section{}
    \newpage
    \pagenumbering{gobble}
        \printbibliography


    \newpage
    \pagenumbering{roman}
    \appendix
        \part{Appendices}
            \input{8 - Hilbert complexes/main.tex}
            \input{9 - weak conservation proofs/main.tex}
\end{document}

        \part{Research}
            \documentclass[12pt, a4paper]{report}

\input{template/main.tex}

\title{\BA{Title in Progress...}}
\author{Boris Andrews}
\affil{Mathematical Institute, University of Oxford}
\date{\today}


\begin{document}
    \pagenumbering{gobble}
    \maketitle
    
    
    \begin{abstract}
        Magnetic confinement reactors---in particular tokamaks---offer one of the most promising options for achieving practical nuclear fusion, with the potential to provide virtually limitless, clean energy. The theoretical and numerical modeling of tokamak plasmas is simultaneously an essential component of effective reactor design, and a great research barrier. Tokamak operational conditions exhibit comparatively low Knudsen numbers. Kinetic effects, including kinetic waves and instabilities, Landau damping, bump-on-tail instabilities and more, are therefore highly influential in tokamak plasma dynamics. Purely fluid models are inherently incapable of capturing these effects, whereas the high dimensionality in purely kinetic models render them practically intractable for most relevant purposes.

        We consider a $\delta\!f$ decomposition model, with a macroscopic fluid background and microscopic kinetic correction, both fully coupled to each other. A similar manner of discretization is proposed to that used in the recent \texttt{STRUPHY} code \cite{Holderied_Possanner_Wang_2021, Holderied_2022, Li_et_al_2023} with a finite-element model for the background and a pseudo-particle/PiC model for the correction.

        The fluid background satisfies the full, non-linear, resistive, compressible, Hall MHD equations. \cite{Laakmann_Hu_Farrell_2022} introduces finite-element(-in-space) implicit timesteppers for the incompressible analogue to this system with structure-preserving (SP) properties in the ideal case, alongside parameter-robust preconditioners. We show that these timesteppers can derive from a finite-element-in-time (FET) (and finite-element-in-space) interpretation. The benefits of this reformulation are discussed, including the derivation of timesteppers that are higher order in time, and the quantifiable dissipative SP properties in the non-ideal, resistive case.
        
        We discuss possible options for extending this FET approach to timesteppers for the compressible case.

        The kinetic corrections satisfy linearized Boltzmann equations. Using a Lénard--Bernstein collision operator, these take Fokker--Planck-like forms \cite{Fokker_1914, Planck_1917} wherein pseudo-particles in the numerical model obey the neoclassical transport equations, with particle-independent Brownian drift terms. This offers a rigorous methodology for incorporating collisions into the particle transport model, without coupling the equations of motions for each particle.
        
        Works by Chen, Chacón et al. \cite{Chen_Chacón_Barnes_2011, Chacón_Chen_Barnes_2013, Chen_Chacón_2014, Chen_Chacón_2015} have developed structure-preserving particle pushers for neoclassical transport in the Vlasov equations, derived from Crank--Nicolson integrators. We show these too can can derive from a FET interpretation, similarly offering potential extensions to higher-order-in-time particle pushers. The FET formulation is used also to consider how the stochastic drift terms can be incorporated into the pushers. Stochastic gyrokinetic expansions are also discussed.

        Different options for the numerical implementation of these schemes are considered.

        Due to the efficacy of FET in the development of SP timesteppers for both the fluid and kinetic component, we hope this approach will prove effective in the future for developing SP timesteppers for the full hybrid model. We hope this will give us the opportunity to incorporate previously inaccessible kinetic effects into the highly effective, modern, finite-element MHD models.
    \end{abstract}
    
    
    \newpage
    \tableofcontents
    
    
    \newpage
    \pagenumbering{arabic}
    %\linenumbers\renewcommand\thelinenumber{\color{black!50}\arabic{linenumber}}
            \input{0 - introduction/main.tex}
        \part{Research}
            \input{1 - low-noise PiC models/main.tex}
            \input{2 - kinetic component/main.tex}
            \input{3 - fluid component/main.tex}
            \input{4 - numerical implementation/main.tex}
        \part{Project Overview}
            \input{5 - research plan/main.tex}
            \input{6 - summary/main.tex}
    
    
    %\section{}
    \newpage
    \pagenumbering{gobble}
        \printbibliography


    \newpage
    \pagenumbering{roman}
    \appendix
        \part{Appendices}
            \input{8 - Hilbert complexes/main.tex}
            \input{9 - weak conservation proofs/main.tex}
\end{document}

            \documentclass[12pt, a4paper]{report}

\input{template/main.tex}

\title{\BA{Title in Progress...}}
\author{Boris Andrews}
\affil{Mathematical Institute, University of Oxford}
\date{\today}


\begin{document}
    \pagenumbering{gobble}
    \maketitle
    
    
    \begin{abstract}
        Magnetic confinement reactors---in particular tokamaks---offer one of the most promising options for achieving practical nuclear fusion, with the potential to provide virtually limitless, clean energy. The theoretical and numerical modeling of tokamak plasmas is simultaneously an essential component of effective reactor design, and a great research barrier. Tokamak operational conditions exhibit comparatively low Knudsen numbers. Kinetic effects, including kinetic waves and instabilities, Landau damping, bump-on-tail instabilities and more, are therefore highly influential in tokamak plasma dynamics. Purely fluid models are inherently incapable of capturing these effects, whereas the high dimensionality in purely kinetic models render them practically intractable for most relevant purposes.

        We consider a $\delta\!f$ decomposition model, with a macroscopic fluid background and microscopic kinetic correction, both fully coupled to each other. A similar manner of discretization is proposed to that used in the recent \texttt{STRUPHY} code \cite{Holderied_Possanner_Wang_2021, Holderied_2022, Li_et_al_2023} with a finite-element model for the background and a pseudo-particle/PiC model for the correction.

        The fluid background satisfies the full, non-linear, resistive, compressible, Hall MHD equations. \cite{Laakmann_Hu_Farrell_2022} introduces finite-element(-in-space) implicit timesteppers for the incompressible analogue to this system with structure-preserving (SP) properties in the ideal case, alongside parameter-robust preconditioners. We show that these timesteppers can derive from a finite-element-in-time (FET) (and finite-element-in-space) interpretation. The benefits of this reformulation are discussed, including the derivation of timesteppers that are higher order in time, and the quantifiable dissipative SP properties in the non-ideal, resistive case.
        
        We discuss possible options for extending this FET approach to timesteppers for the compressible case.

        The kinetic corrections satisfy linearized Boltzmann equations. Using a Lénard--Bernstein collision operator, these take Fokker--Planck-like forms \cite{Fokker_1914, Planck_1917} wherein pseudo-particles in the numerical model obey the neoclassical transport equations, with particle-independent Brownian drift terms. This offers a rigorous methodology for incorporating collisions into the particle transport model, without coupling the equations of motions for each particle.
        
        Works by Chen, Chacón et al. \cite{Chen_Chacón_Barnes_2011, Chacón_Chen_Barnes_2013, Chen_Chacón_2014, Chen_Chacón_2015} have developed structure-preserving particle pushers for neoclassical transport in the Vlasov equations, derived from Crank--Nicolson integrators. We show these too can can derive from a FET interpretation, similarly offering potential extensions to higher-order-in-time particle pushers. The FET formulation is used also to consider how the stochastic drift terms can be incorporated into the pushers. Stochastic gyrokinetic expansions are also discussed.

        Different options for the numerical implementation of these schemes are considered.

        Due to the efficacy of FET in the development of SP timesteppers for both the fluid and kinetic component, we hope this approach will prove effective in the future for developing SP timesteppers for the full hybrid model. We hope this will give us the opportunity to incorporate previously inaccessible kinetic effects into the highly effective, modern, finite-element MHD models.
    \end{abstract}
    
    
    \newpage
    \tableofcontents
    
    
    \newpage
    \pagenumbering{arabic}
    %\linenumbers\renewcommand\thelinenumber{\color{black!50}\arabic{linenumber}}
            \input{0 - introduction/main.tex}
        \part{Research}
            \input{1 - low-noise PiC models/main.tex}
            \input{2 - kinetic component/main.tex}
            \input{3 - fluid component/main.tex}
            \input{4 - numerical implementation/main.tex}
        \part{Project Overview}
            \input{5 - research plan/main.tex}
            \input{6 - summary/main.tex}
    
    
    %\section{}
    \newpage
    \pagenumbering{gobble}
        \printbibliography


    \newpage
    \pagenumbering{roman}
    \appendix
        \part{Appendices}
            \input{8 - Hilbert complexes/main.tex}
            \input{9 - weak conservation proofs/main.tex}
\end{document}

            \documentclass[12pt, a4paper]{report}

\input{template/main.tex}

\title{\BA{Title in Progress...}}
\author{Boris Andrews}
\affil{Mathematical Institute, University of Oxford}
\date{\today}


\begin{document}
    \pagenumbering{gobble}
    \maketitle
    
    
    \begin{abstract}
        Magnetic confinement reactors---in particular tokamaks---offer one of the most promising options for achieving practical nuclear fusion, with the potential to provide virtually limitless, clean energy. The theoretical and numerical modeling of tokamak plasmas is simultaneously an essential component of effective reactor design, and a great research barrier. Tokamak operational conditions exhibit comparatively low Knudsen numbers. Kinetic effects, including kinetic waves and instabilities, Landau damping, bump-on-tail instabilities and more, are therefore highly influential in tokamak plasma dynamics. Purely fluid models are inherently incapable of capturing these effects, whereas the high dimensionality in purely kinetic models render them practically intractable for most relevant purposes.

        We consider a $\delta\!f$ decomposition model, with a macroscopic fluid background and microscopic kinetic correction, both fully coupled to each other. A similar manner of discretization is proposed to that used in the recent \texttt{STRUPHY} code \cite{Holderied_Possanner_Wang_2021, Holderied_2022, Li_et_al_2023} with a finite-element model for the background and a pseudo-particle/PiC model for the correction.

        The fluid background satisfies the full, non-linear, resistive, compressible, Hall MHD equations. \cite{Laakmann_Hu_Farrell_2022} introduces finite-element(-in-space) implicit timesteppers for the incompressible analogue to this system with structure-preserving (SP) properties in the ideal case, alongside parameter-robust preconditioners. We show that these timesteppers can derive from a finite-element-in-time (FET) (and finite-element-in-space) interpretation. The benefits of this reformulation are discussed, including the derivation of timesteppers that are higher order in time, and the quantifiable dissipative SP properties in the non-ideal, resistive case.
        
        We discuss possible options for extending this FET approach to timesteppers for the compressible case.

        The kinetic corrections satisfy linearized Boltzmann equations. Using a Lénard--Bernstein collision operator, these take Fokker--Planck-like forms \cite{Fokker_1914, Planck_1917} wherein pseudo-particles in the numerical model obey the neoclassical transport equations, with particle-independent Brownian drift terms. This offers a rigorous methodology for incorporating collisions into the particle transport model, without coupling the equations of motions for each particle.
        
        Works by Chen, Chacón et al. \cite{Chen_Chacón_Barnes_2011, Chacón_Chen_Barnes_2013, Chen_Chacón_2014, Chen_Chacón_2015} have developed structure-preserving particle pushers for neoclassical transport in the Vlasov equations, derived from Crank--Nicolson integrators. We show these too can can derive from a FET interpretation, similarly offering potential extensions to higher-order-in-time particle pushers. The FET formulation is used also to consider how the stochastic drift terms can be incorporated into the pushers. Stochastic gyrokinetic expansions are also discussed.

        Different options for the numerical implementation of these schemes are considered.

        Due to the efficacy of FET in the development of SP timesteppers for both the fluid and kinetic component, we hope this approach will prove effective in the future for developing SP timesteppers for the full hybrid model. We hope this will give us the opportunity to incorporate previously inaccessible kinetic effects into the highly effective, modern, finite-element MHD models.
    \end{abstract}
    
    
    \newpage
    \tableofcontents
    
    
    \newpage
    \pagenumbering{arabic}
    %\linenumbers\renewcommand\thelinenumber{\color{black!50}\arabic{linenumber}}
            \input{0 - introduction/main.tex}
        \part{Research}
            \input{1 - low-noise PiC models/main.tex}
            \input{2 - kinetic component/main.tex}
            \input{3 - fluid component/main.tex}
            \input{4 - numerical implementation/main.tex}
        \part{Project Overview}
            \input{5 - research plan/main.tex}
            \input{6 - summary/main.tex}
    
    
    %\section{}
    \newpage
    \pagenumbering{gobble}
        \printbibliography


    \newpage
    \pagenumbering{roman}
    \appendix
        \part{Appendices}
            \input{8 - Hilbert complexes/main.tex}
            \input{9 - weak conservation proofs/main.tex}
\end{document}

            \documentclass[12pt, a4paper]{report}

\input{template/main.tex}

\title{\BA{Title in Progress...}}
\author{Boris Andrews}
\affil{Mathematical Institute, University of Oxford}
\date{\today}


\begin{document}
    \pagenumbering{gobble}
    \maketitle
    
    
    \begin{abstract}
        Magnetic confinement reactors---in particular tokamaks---offer one of the most promising options for achieving practical nuclear fusion, with the potential to provide virtually limitless, clean energy. The theoretical and numerical modeling of tokamak plasmas is simultaneously an essential component of effective reactor design, and a great research barrier. Tokamak operational conditions exhibit comparatively low Knudsen numbers. Kinetic effects, including kinetic waves and instabilities, Landau damping, bump-on-tail instabilities and more, are therefore highly influential in tokamak plasma dynamics. Purely fluid models are inherently incapable of capturing these effects, whereas the high dimensionality in purely kinetic models render them practically intractable for most relevant purposes.

        We consider a $\delta\!f$ decomposition model, with a macroscopic fluid background and microscopic kinetic correction, both fully coupled to each other. A similar manner of discretization is proposed to that used in the recent \texttt{STRUPHY} code \cite{Holderied_Possanner_Wang_2021, Holderied_2022, Li_et_al_2023} with a finite-element model for the background and a pseudo-particle/PiC model for the correction.

        The fluid background satisfies the full, non-linear, resistive, compressible, Hall MHD equations. \cite{Laakmann_Hu_Farrell_2022} introduces finite-element(-in-space) implicit timesteppers for the incompressible analogue to this system with structure-preserving (SP) properties in the ideal case, alongside parameter-robust preconditioners. We show that these timesteppers can derive from a finite-element-in-time (FET) (and finite-element-in-space) interpretation. The benefits of this reformulation are discussed, including the derivation of timesteppers that are higher order in time, and the quantifiable dissipative SP properties in the non-ideal, resistive case.
        
        We discuss possible options for extending this FET approach to timesteppers for the compressible case.

        The kinetic corrections satisfy linearized Boltzmann equations. Using a Lénard--Bernstein collision operator, these take Fokker--Planck-like forms \cite{Fokker_1914, Planck_1917} wherein pseudo-particles in the numerical model obey the neoclassical transport equations, with particle-independent Brownian drift terms. This offers a rigorous methodology for incorporating collisions into the particle transport model, without coupling the equations of motions for each particle.
        
        Works by Chen, Chacón et al. \cite{Chen_Chacón_Barnes_2011, Chacón_Chen_Barnes_2013, Chen_Chacón_2014, Chen_Chacón_2015} have developed structure-preserving particle pushers for neoclassical transport in the Vlasov equations, derived from Crank--Nicolson integrators. We show these too can can derive from a FET interpretation, similarly offering potential extensions to higher-order-in-time particle pushers. The FET formulation is used also to consider how the stochastic drift terms can be incorporated into the pushers. Stochastic gyrokinetic expansions are also discussed.

        Different options for the numerical implementation of these schemes are considered.

        Due to the efficacy of FET in the development of SP timesteppers for both the fluid and kinetic component, we hope this approach will prove effective in the future for developing SP timesteppers for the full hybrid model. We hope this will give us the opportunity to incorporate previously inaccessible kinetic effects into the highly effective, modern, finite-element MHD models.
    \end{abstract}
    
    
    \newpage
    \tableofcontents
    
    
    \newpage
    \pagenumbering{arabic}
    %\linenumbers\renewcommand\thelinenumber{\color{black!50}\arabic{linenumber}}
            \input{0 - introduction/main.tex}
        \part{Research}
            \input{1 - low-noise PiC models/main.tex}
            \input{2 - kinetic component/main.tex}
            \input{3 - fluid component/main.tex}
            \input{4 - numerical implementation/main.tex}
        \part{Project Overview}
            \input{5 - research plan/main.tex}
            \input{6 - summary/main.tex}
    
    
    %\section{}
    \newpage
    \pagenumbering{gobble}
        \printbibliography


    \newpage
    \pagenumbering{roman}
    \appendix
        \part{Appendices}
            \input{8 - Hilbert complexes/main.tex}
            \input{9 - weak conservation proofs/main.tex}
\end{document}

        \part{Project Overview}
            \documentclass[12pt, a4paper]{report}

\input{template/main.tex}

\title{\BA{Title in Progress...}}
\author{Boris Andrews}
\affil{Mathematical Institute, University of Oxford}
\date{\today}


\begin{document}
    \pagenumbering{gobble}
    \maketitle
    
    
    \begin{abstract}
        Magnetic confinement reactors---in particular tokamaks---offer one of the most promising options for achieving practical nuclear fusion, with the potential to provide virtually limitless, clean energy. The theoretical and numerical modeling of tokamak plasmas is simultaneously an essential component of effective reactor design, and a great research barrier. Tokamak operational conditions exhibit comparatively low Knudsen numbers. Kinetic effects, including kinetic waves and instabilities, Landau damping, bump-on-tail instabilities and more, are therefore highly influential in tokamak plasma dynamics. Purely fluid models are inherently incapable of capturing these effects, whereas the high dimensionality in purely kinetic models render them practically intractable for most relevant purposes.

        We consider a $\delta\!f$ decomposition model, with a macroscopic fluid background and microscopic kinetic correction, both fully coupled to each other. A similar manner of discretization is proposed to that used in the recent \texttt{STRUPHY} code \cite{Holderied_Possanner_Wang_2021, Holderied_2022, Li_et_al_2023} with a finite-element model for the background and a pseudo-particle/PiC model for the correction.

        The fluid background satisfies the full, non-linear, resistive, compressible, Hall MHD equations. \cite{Laakmann_Hu_Farrell_2022} introduces finite-element(-in-space) implicit timesteppers for the incompressible analogue to this system with structure-preserving (SP) properties in the ideal case, alongside parameter-robust preconditioners. We show that these timesteppers can derive from a finite-element-in-time (FET) (and finite-element-in-space) interpretation. The benefits of this reformulation are discussed, including the derivation of timesteppers that are higher order in time, and the quantifiable dissipative SP properties in the non-ideal, resistive case.
        
        We discuss possible options for extending this FET approach to timesteppers for the compressible case.

        The kinetic corrections satisfy linearized Boltzmann equations. Using a Lénard--Bernstein collision operator, these take Fokker--Planck-like forms \cite{Fokker_1914, Planck_1917} wherein pseudo-particles in the numerical model obey the neoclassical transport equations, with particle-independent Brownian drift terms. This offers a rigorous methodology for incorporating collisions into the particle transport model, without coupling the equations of motions for each particle.
        
        Works by Chen, Chacón et al. \cite{Chen_Chacón_Barnes_2011, Chacón_Chen_Barnes_2013, Chen_Chacón_2014, Chen_Chacón_2015} have developed structure-preserving particle pushers for neoclassical transport in the Vlasov equations, derived from Crank--Nicolson integrators. We show these too can can derive from a FET interpretation, similarly offering potential extensions to higher-order-in-time particle pushers. The FET formulation is used also to consider how the stochastic drift terms can be incorporated into the pushers. Stochastic gyrokinetic expansions are also discussed.

        Different options for the numerical implementation of these schemes are considered.

        Due to the efficacy of FET in the development of SP timesteppers for both the fluid and kinetic component, we hope this approach will prove effective in the future for developing SP timesteppers for the full hybrid model. We hope this will give us the opportunity to incorporate previously inaccessible kinetic effects into the highly effective, modern, finite-element MHD models.
    \end{abstract}
    
    
    \newpage
    \tableofcontents
    
    
    \newpage
    \pagenumbering{arabic}
    %\linenumbers\renewcommand\thelinenumber{\color{black!50}\arabic{linenumber}}
            \input{0 - introduction/main.tex}
        \part{Research}
            \input{1 - low-noise PiC models/main.tex}
            \input{2 - kinetic component/main.tex}
            \input{3 - fluid component/main.tex}
            \input{4 - numerical implementation/main.tex}
        \part{Project Overview}
            \input{5 - research plan/main.tex}
            \input{6 - summary/main.tex}
    
    
    %\section{}
    \newpage
    \pagenumbering{gobble}
        \printbibliography


    \newpage
    \pagenumbering{roman}
    \appendix
        \part{Appendices}
            \input{8 - Hilbert complexes/main.tex}
            \input{9 - weak conservation proofs/main.tex}
\end{document}

            \documentclass[12pt, a4paper]{report}

\input{template/main.tex}

\title{\BA{Title in Progress...}}
\author{Boris Andrews}
\affil{Mathematical Institute, University of Oxford}
\date{\today}


\begin{document}
    \pagenumbering{gobble}
    \maketitle
    
    
    \begin{abstract}
        Magnetic confinement reactors---in particular tokamaks---offer one of the most promising options for achieving practical nuclear fusion, with the potential to provide virtually limitless, clean energy. The theoretical and numerical modeling of tokamak plasmas is simultaneously an essential component of effective reactor design, and a great research barrier. Tokamak operational conditions exhibit comparatively low Knudsen numbers. Kinetic effects, including kinetic waves and instabilities, Landau damping, bump-on-tail instabilities and more, are therefore highly influential in tokamak plasma dynamics. Purely fluid models are inherently incapable of capturing these effects, whereas the high dimensionality in purely kinetic models render them practically intractable for most relevant purposes.

        We consider a $\delta\!f$ decomposition model, with a macroscopic fluid background and microscopic kinetic correction, both fully coupled to each other. A similar manner of discretization is proposed to that used in the recent \texttt{STRUPHY} code \cite{Holderied_Possanner_Wang_2021, Holderied_2022, Li_et_al_2023} with a finite-element model for the background and a pseudo-particle/PiC model for the correction.

        The fluid background satisfies the full, non-linear, resistive, compressible, Hall MHD equations. \cite{Laakmann_Hu_Farrell_2022} introduces finite-element(-in-space) implicit timesteppers for the incompressible analogue to this system with structure-preserving (SP) properties in the ideal case, alongside parameter-robust preconditioners. We show that these timesteppers can derive from a finite-element-in-time (FET) (and finite-element-in-space) interpretation. The benefits of this reformulation are discussed, including the derivation of timesteppers that are higher order in time, and the quantifiable dissipative SP properties in the non-ideal, resistive case.
        
        We discuss possible options for extending this FET approach to timesteppers for the compressible case.

        The kinetic corrections satisfy linearized Boltzmann equations. Using a Lénard--Bernstein collision operator, these take Fokker--Planck-like forms \cite{Fokker_1914, Planck_1917} wherein pseudo-particles in the numerical model obey the neoclassical transport equations, with particle-independent Brownian drift terms. This offers a rigorous methodology for incorporating collisions into the particle transport model, without coupling the equations of motions for each particle.
        
        Works by Chen, Chacón et al. \cite{Chen_Chacón_Barnes_2011, Chacón_Chen_Barnes_2013, Chen_Chacón_2014, Chen_Chacón_2015} have developed structure-preserving particle pushers for neoclassical transport in the Vlasov equations, derived from Crank--Nicolson integrators. We show these too can can derive from a FET interpretation, similarly offering potential extensions to higher-order-in-time particle pushers. The FET formulation is used also to consider how the stochastic drift terms can be incorporated into the pushers. Stochastic gyrokinetic expansions are also discussed.

        Different options for the numerical implementation of these schemes are considered.

        Due to the efficacy of FET in the development of SP timesteppers for both the fluid and kinetic component, we hope this approach will prove effective in the future for developing SP timesteppers for the full hybrid model. We hope this will give us the opportunity to incorporate previously inaccessible kinetic effects into the highly effective, modern, finite-element MHD models.
    \end{abstract}
    
    
    \newpage
    \tableofcontents
    
    
    \newpage
    \pagenumbering{arabic}
    %\linenumbers\renewcommand\thelinenumber{\color{black!50}\arabic{linenumber}}
            \input{0 - introduction/main.tex}
        \part{Research}
            \input{1 - low-noise PiC models/main.tex}
            \input{2 - kinetic component/main.tex}
            \input{3 - fluid component/main.tex}
            \input{4 - numerical implementation/main.tex}
        \part{Project Overview}
            \input{5 - research plan/main.tex}
            \input{6 - summary/main.tex}
    
    
    %\section{}
    \newpage
    \pagenumbering{gobble}
        \printbibliography


    \newpage
    \pagenumbering{roman}
    \appendix
        \part{Appendices}
            \input{8 - Hilbert complexes/main.tex}
            \input{9 - weak conservation proofs/main.tex}
\end{document}

    
    
    %\section{}
    \newpage
    \pagenumbering{gobble}
        \printbibliography


    \newpage
    \pagenumbering{roman}
    \appendix
        \part{Appendices}
            \documentclass[12pt, a4paper]{report}

\input{template/main.tex}

\title{\BA{Title in Progress...}}
\author{Boris Andrews}
\affil{Mathematical Institute, University of Oxford}
\date{\today}


\begin{document}
    \pagenumbering{gobble}
    \maketitle
    
    
    \begin{abstract}
        Magnetic confinement reactors---in particular tokamaks---offer one of the most promising options for achieving practical nuclear fusion, with the potential to provide virtually limitless, clean energy. The theoretical and numerical modeling of tokamak plasmas is simultaneously an essential component of effective reactor design, and a great research barrier. Tokamak operational conditions exhibit comparatively low Knudsen numbers. Kinetic effects, including kinetic waves and instabilities, Landau damping, bump-on-tail instabilities and more, are therefore highly influential in tokamak plasma dynamics. Purely fluid models are inherently incapable of capturing these effects, whereas the high dimensionality in purely kinetic models render them practically intractable for most relevant purposes.

        We consider a $\delta\!f$ decomposition model, with a macroscopic fluid background and microscopic kinetic correction, both fully coupled to each other. A similar manner of discretization is proposed to that used in the recent \texttt{STRUPHY} code \cite{Holderied_Possanner_Wang_2021, Holderied_2022, Li_et_al_2023} with a finite-element model for the background and a pseudo-particle/PiC model for the correction.

        The fluid background satisfies the full, non-linear, resistive, compressible, Hall MHD equations. \cite{Laakmann_Hu_Farrell_2022} introduces finite-element(-in-space) implicit timesteppers for the incompressible analogue to this system with structure-preserving (SP) properties in the ideal case, alongside parameter-robust preconditioners. We show that these timesteppers can derive from a finite-element-in-time (FET) (and finite-element-in-space) interpretation. The benefits of this reformulation are discussed, including the derivation of timesteppers that are higher order in time, and the quantifiable dissipative SP properties in the non-ideal, resistive case.
        
        We discuss possible options for extending this FET approach to timesteppers for the compressible case.

        The kinetic corrections satisfy linearized Boltzmann equations. Using a Lénard--Bernstein collision operator, these take Fokker--Planck-like forms \cite{Fokker_1914, Planck_1917} wherein pseudo-particles in the numerical model obey the neoclassical transport equations, with particle-independent Brownian drift terms. This offers a rigorous methodology for incorporating collisions into the particle transport model, without coupling the equations of motions for each particle.
        
        Works by Chen, Chacón et al. \cite{Chen_Chacón_Barnes_2011, Chacón_Chen_Barnes_2013, Chen_Chacón_2014, Chen_Chacón_2015} have developed structure-preserving particle pushers for neoclassical transport in the Vlasov equations, derived from Crank--Nicolson integrators. We show these too can can derive from a FET interpretation, similarly offering potential extensions to higher-order-in-time particle pushers. The FET formulation is used also to consider how the stochastic drift terms can be incorporated into the pushers. Stochastic gyrokinetic expansions are also discussed.

        Different options for the numerical implementation of these schemes are considered.

        Due to the efficacy of FET in the development of SP timesteppers for both the fluid and kinetic component, we hope this approach will prove effective in the future for developing SP timesteppers for the full hybrid model. We hope this will give us the opportunity to incorporate previously inaccessible kinetic effects into the highly effective, modern, finite-element MHD models.
    \end{abstract}
    
    
    \newpage
    \tableofcontents
    
    
    \newpage
    \pagenumbering{arabic}
    %\linenumbers\renewcommand\thelinenumber{\color{black!50}\arabic{linenumber}}
            \input{0 - introduction/main.tex}
        \part{Research}
            \input{1 - low-noise PiC models/main.tex}
            \input{2 - kinetic component/main.tex}
            \input{3 - fluid component/main.tex}
            \input{4 - numerical implementation/main.tex}
        \part{Project Overview}
            \input{5 - research plan/main.tex}
            \input{6 - summary/main.tex}
    
    
    %\section{}
    \newpage
    \pagenumbering{gobble}
        \printbibliography


    \newpage
    \pagenumbering{roman}
    \appendix
        \part{Appendices}
            \input{8 - Hilbert complexes/main.tex}
            \input{9 - weak conservation proofs/main.tex}
\end{document}

            \documentclass[12pt, a4paper]{report}

\input{template/main.tex}

\title{\BA{Title in Progress...}}
\author{Boris Andrews}
\affil{Mathematical Institute, University of Oxford}
\date{\today}


\begin{document}
    \pagenumbering{gobble}
    \maketitle
    
    
    \begin{abstract}
        Magnetic confinement reactors---in particular tokamaks---offer one of the most promising options for achieving practical nuclear fusion, with the potential to provide virtually limitless, clean energy. The theoretical and numerical modeling of tokamak plasmas is simultaneously an essential component of effective reactor design, and a great research barrier. Tokamak operational conditions exhibit comparatively low Knudsen numbers. Kinetic effects, including kinetic waves and instabilities, Landau damping, bump-on-tail instabilities and more, are therefore highly influential in tokamak plasma dynamics. Purely fluid models are inherently incapable of capturing these effects, whereas the high dimensionality in purely kinetic models render them practically intractable for most relevant purposes.

        We consider a $\delta\!f$ decomposition model, with a macroscopic fluid background and microscopic kinetic correction, both fully coupled to each other. A similar manner of discretization is proposed to that used in the recent \texttt{STRUPHY} code \cite{Holderied_Possanner_Wang_2021, Holderied_2022, Li_et_al_2023} with a finite-element model for the background and a pseudo-particle/PiC model for the correction.

        The fluid background satisfies the full, non-linear, resistive, compressible, Hall MHD equations. \cite{Laakmann_Hu_Farrell_2022} introduces finite-element(-in-space) implicit timesteppers for the incompressible analogue to this system with structure-preserving (SP) properties in the ideal case, alongside parameter-robust preconditioners. We show that these timesteppers can derive from a finite-element-in-time (FET) (and finite-element-in-space) interpretation. The benefits of this reformulation are discussed, including the derivation of timesteppers that are higher order in time, and the quantifiable dissipative SP properties in the non-ideal, resistive case.
        
        We discuss possible options for extending this FET approach to timesteppers for the compressible case.

        The kinetic corrections satisfy linearized Boltzmann equations. Using a Lénard--Bernstein collision operator, these take Fokker--Planck-like forms \cite{Fokker_1914, Planck_1917} wherein pseudo-particles in the numerical model obey the neoclassical transport equations, with particle-independent Brownian drift terms. This offers a rigorous methodology for incorporating collisions into the particle transport model, without coupling the equations of motions for each particle.
        
        Works by Chen, Chacón et al. \cite{Chen_Chacón_Barnes_2011, Chacón_Chen_Barnes_2013, Chen_Chacón_2014, Chen_Chacón_2015} have developed structure-preserving particle pushers for neoclassical transport in the Vlasov equations, derived from Crank--Nicolson integrators. We show these too can can derive from a FET interpretation, similarly offering potential extensions to higher-order-in-time particle pushers. The FET formulation is used also to consider how the stochastic drift terms can be incorporated into the pushers. Stochastic gyrokinetic expansions are also discussed.

        Different options for the numerical implementation of these schemes are considered.

        Due to the efficacy of FET in the development of SP timesteppers for both the fluid and kinetic component, we hope this approach will prove effective in the future for developing SP timesteppers for the full hybrid model. We hope this will give us the opportunity to incorporate previously inaccessible kinetic effects into the highly effective, modern, finite-element MHD models.
    \end{abstract}
    
    
    \newpage
    \tableofcontents
    
    
    \newpage
    \pagenumbering{arabic}
    %\linenumbers\renewcommand\thelinenumber{\color{black!50}\arabic{linenumber}}
            \input{0 - introduction/main.tex}
        \part{Research}
            \input{1 - low-noise PiC models/main.tex}
            \input{2 - kinetic component/main.tex}
            \input{3 - fluid component/main.tex}
            \input{4 - numerical implementation/main.tex}
        \part{Project Overview}
            \input{5 - research plan/main.tex}
            \input{6 - summary/main.tex}
    
    
    %\section{}
    \newpage
    \pagenumbering{gobble}
        \printbibliography


    \newpage
    \pagenumbering{roman}
    \appendix
        \part{Appendices}
            \input{8 - Hilbert complexes/main.tex}
            \input{9 - weak conservation proofs/main.tex}
\end{document}

\end{document}

            \documentclass[12pt, a4paper]{report}

\documentclass[12pt, a4paper]{report}

\input{template/main.tex}

\title{\BA{Title in Progress...}}
\author{Boris Andrews}
\affil{Mathematical Institute, University of Oxford}
\date{\today}


\begin{document}
    \pagenumbering{gobble}
    \maketitle
    
    
    \begin{abstract}
        Magnetic confinement reactors---in particular tokamaks---offer one of the most promising options for achieving practical nuclear fusion, with the potential to provide virtually limitless, clean energy. The theoretical and numerical modeling of tokamak plasmas is simultaneously an essential component of effective reactor design, and a great research barrier. Tokamak operational conditions exhibit comparatively low Knudsen numbers. Kinetic effects, including kinetic waves and instabilities, Landau damping, bump-on-tail instabilities and more, are therefore highly influential in tokamak plasma dynamics. Purely fluid models are inherently incapable of capturing these effects, whereas the high dimensionality in purely kinetic models render them practically intractable for most relevant purposes.

        We consider a $\delta\!f$ decomposition model, with a macroscopic fluid background and microscopic kinetic correction, both fully coupled to each other. A similar manner of discretization is proposed to that used in the recent \texttt{STRUPHY} code \cite{Holderied_Possanner_Wang_2021, Holderied_2022, Li_et_al_2023} with a finite-element model for the background and a pseudo-particle/PiC model for the correction.

        The fluid background satisfies the full, non-linear, resistive, compressible, Hall MHD equations. \cite{Laakmann_Hu_Farrell_2022} introduces finite-element(-in-space) implicit timesteppers for the incompressible analogue to this system with structure-preserving (SP) properties in the ideal case, alongside parameter-robust preconditioners. We show that these timesteppers can derive from a finite-element-in-time (FET) (and finite-element-in-space) interpretation. The benefits of this reformulation are discussed, including the derivation of timesteppers that are higher order in time, and the quantifiable dissipative SP properties in the non-ideal, resistive case.
        
        We discuss possible options for extending this FET approach to timesteppers for the compressible case.

        The kinetic corrections satisfy linearized Boltzmann equations. Using a Lénard--Bernstein collision operator, these take Fokker--Planck-like forms \cite{Fokker_1914, Planck_1917} wherein pseudo-particles in the numerical model obey the neoclassical transport equations, with particle-independent Brownian drift terms. This offers a rigorous methodology for incorporating collisions into the particle transport model, without coupling the equations of motions for each particle.
        
        Works by Chen, Chacón et al. \cite{Chen_Chacón_Barnes_2011, Chacón_Chen_Barnes_2013, Chen_Chacón_2014, Chen_Chacón_2015} have developed structure-preserving particle pushers for neoclassical transport in the Vlasov equations, derived from Crank--Nicolson integrators. We show these too can can derive from a FET interpretation, similarly offering potential extensions to higher-order-in-time particle pushers. The FET formulation is used also to consider how the stochastic drift terms can be incorporated into the pushers. Stochastic gyrokinetic expansions are also discussed.

        Different options for the numerical implementation of these schemes are considered.

        Due to the efficacy of FET in the development of SP timesteppers for both the fluid and kinetic component, we hope this approach will prove effective in the future for developing SP timesteppers for the full hybrid model. We hope this will give us the opportunity to incorporate previously inaccessible kinetic effects into the highly effective, modern, finite-element MHD models.
    \end{abstract}
    
    
    \newpage
    \tableofcontents
    
    
    \newpage
    \pagenumbering{arabic}
    %\linenumbers\renewcommand\thelinenumber{\color{black!50}\arabic{linenumber}}
            \input{0 - introduction/main.tex}
        \part{Research}
            \input{1 - low-noise PiC models/main.tex}
            \input{2 - kinetic component/main.tex}
            \input{3 - fluid component/main.tex}
            \input{4 - numerical implementation/main.tex}
        \part{Project Overview}
            \input{5 - research plan/main.tex}
            \input{6 - summary/main.tex}
    
    
    %\section{}
    \newpage
    \pagenumbering{gobble}
        \printbibliography


    \newpage
    \pagenumbering{roman}
    \appendix
        \part{Appendices}
            \input{8 - Hilbert complexes/main.tex}
            \input{9 - weak conservation proofs/main.tex}
\end{document}


\title{\BA{Title in Progress...}}
\author{Boris Andrews}
\affil{Mathematical Institute, University of Oxford}
\date{\today}


\begin{document}
    \pagenumbering{gobble}
    \maketitle
    
    
    \begin{abstract}
        Magnetic confinement reactors---in particular tokamaks---offer one of the most promising options for achieving practical nuclear fusion, with the potential to provide virtually limitless, clean energy. The theoretical and numerical modeling of tokamak plasmas is simultaneously an essential component of effective reactor design, and a great research barrier. Tokamak operational conditions exhibit comparatively low Knudsen numbers. Kinetic effects, including kinetic waves and instabilities, Landau damping, bump-on-tail instabilities and more, are therefore highly influential in tokamak plasma dynamics. Purely fluid models are inherently incapable of capturing these effects, whereas the high dimensionality in purely kinetic models render them practically intractable for most relevant purposes.

        We consider a $\delta\!f$ decomposition model, with a macroscopic fluid background and microscopic kinetic correction, both fully coupled to each other. A similar manner of discretization is proposed to that used in the recent \texttt{STRUPHY} code \cite{Holderied_Possanner_Wang_2021, Holderied_2022, Li_et_al_2023} with a finite-element model for the background and a pseudo-particle/PiC model for the correction.

        The fluid background satisfies the full, non-linear, resistive, compressible, Hall MHD equations. \cite{Laakmann_Hu_Farrell_2022} introduces finite-element(-in-space) implicit timesteppers for the incompressible analogue to this system with structure-preserving (SP) properties in the ideal case, alongside parameter-robust preconditioners. We show that these timesteppers can derive from a finite-element-in-time (FET) (and finite-element-in-space) interpretation. The benefits of this reformulation are discussed, including the derivation of timesteppers that are higher order in time, and the quantifiable dissipative SP properties in the non-ideal, resistive case.
        
        We discuss possible options for extending this FET approach to timesteppers for the compressible case.

        The kinetic corrections satisfy linearized Boltzmann equations. Using a Lénard--Bernstein collision operator, these take Fokker--Planck-like forms \cite{Fokker_1914, Planck_1917} wherein pseudo-particles in the numerical model obey the neoclassical transport equations, with particle-independent Brownian drift terms. This offers a rigorous methodology for incorporating collisions into the particle transport model, without coupling the equations of motions for each particle.
        
        Works by Chen, Chacón et al. \cite{Chen_Chacón_Barnes_2011, Chacón_Chen_Barnes_2013, Chen_Chacón_2014, Chen_Chacón_2015} have developed structure-preserving particle pushers for neoclassical transport in the Vlasov equations, derived from Crank--Nicolson integrators. We show these too can can derive from a FET interpretation, similarly offering potential extensions to higher-order-in-time particle pushers. The FET formulation is used also to consider how the stochastic drift terms can be incorporated into the pushers. Stochastic gyrokinetic expansions are also discussed.

        Different options for the numerical implementation of these schemes are considered.

        Due to the efficacy of FET in the development of SP timesteppers for both the fluid and kinetic component, we hope this approach will prove effective in the future for developing SP timesteppers for the full hybrid model. We hope this will give us the opportunity to incorporate previously inaccessible kinetic effects into the highly effective, modern, finite-element MHD models.
    \end{abstract}
    
    
    \newpage
    \tableofcontents
    
    
    \newpage
    \pagenumbering{arabic}
    %\linenumbers\renewcommand\thelinenumber{\color{black!50}\arabic{linenumber}}
            \documentclass[12pt, a4paper]{report}

\input{template/main.tex}

\title{\BA{Title in Progress...}}
\author{Boris Andrews}
\affil{Mathematical Institute, University of Oxford}
\date{\today}


\begin{document}
    \pagenumbering{gobble}
    \maketitle
    
    
    \begin{abstract}
        Magnetic confinement reactors---in particular tokamaks---offer one of the most promising options for achieving practical nuclear fusion, with the potential to provide virtually limitless, clean energy. The theoretical and numerical modeling of tokamak plasmas is simultaneously an essential component of effective reactor design, and a great research barrier. Tokamak operational conditions exhibit comparatively low Knudsen numbers. Kinetic effects, including kinetic waves and instabilities, Landau damping, bump-on-tail instabilities and more, are therefore highly influential in tokamak plasma dynamics. Purely fluid models are inherently incapable of capturing these effects, whereas the high dimensionality in purely kinetic models render them practically intractable for most relevant purposes.

        We consider a $\delta\!f$ decomposition model, with a macroscopic fluid background and microscopic kinetic correction, both fully coupled to each other. A similar manner of discretization is proposed to that used in the recent \texttt{STRUPHY} code \cite{Holderied_Possanner_Wang_2021, Holderied_2022, Li_et_al_2023} with a finite-element model for the background and a pseudo-particle/PiC model for the correction.

        The fluid background satisfies the full, non-linear, resistive, compressible, Hall MHD equations. \cite{Laakmann_Hu_Farrell_2022} introduces finite-element(-in-space) implicit timesteppers for the incompressible analogue to this system with structure-preserving (SP) properties in the ideal case, alongside parameter-robust preconditioners. We show that these timesteppers can derive from a finite-element-in-time (FET) (and finite-element-in-space) interpretation. The benefits of this reformulation are discussed, including the derivation of timesteppers that are higher order in time, and the quantifiable dissipative SP properties in the non-ideal, resistive case.
        
        We discuss possible options for extending this FET approach to timesteppers for the compressible case.

        The kinetic corrections satisfy linearized Boltzmann equations. Using a Lénard--Bernstein collision operator, these take Fokker--Planck-like forms \cite{Fokker_1914, Planck_1917} wherein pseudo-particles in the numerical model obey the neoclassical transport equations, with particle-independent Brownian drift terms. This offers a rigorous methodology for incorporating collisions into the particle transport model, without coupling the equations of motions for each particle.
        
        Works by Chen, Chacón et al. \cite{Chen_Chacón_Barnes_2011, Chacón_Chen_Barnes_2013, Chen_Chacón_2014, Chen_Chacón_2015} have developed structure-preserving particle pushers for neoclassical transport in the Vlasov equations, derived from Crank--Nicolson integrators. We show these too can can derive from a FET interpretation, similarly offering potential extensions to higher-order-in-time particle pushers. The FET formulation is used also to consider how the stochastic drift terms can be incorporated into the pushers. Stochastic gyrokinetic expansions are also discussed.

        Different options for the numerical implementation of these schemes are considered.

        Due to the efficacy of FET in the development of SP timesteppers for both the fluid and kinetic component, we hope this approach will prove effective in the future for developing SP timesteppers for the full hybrid model. We hope this will give us the opportunity to incorporate previously inaccessible kinetic effects into the highly effective, modern, finite-element MHD models.
    \end{abstract}
    
    
    \newpage
    \tableofcontents
    
    
    \newpage
    \pagenumbering{arabic}
    %\linenumbers\renewcommand\thelinenumber{\color{black!50}\arabic{linenumber}}
            \input{0 - introduction/main.tex}
        \part{Research}
            \input{1 - low-noise PiC models/main.tex}
            \input{2 - kinetic component/main.tex}
            \input{3 - fluid component/main.tex}
            \input{4 - numerical implementation/main.tex}
        \part{Project Overview}
            \input{5 - research plan/main.tex}
            \input{6 - summary/main.tex}
    
    
    %\section{}
    \newpage
    \pagenumbering{gobble}
        \printbibliography


    \newpage
    \pagenumbering{roman}
    \appendix
        \part{Appendices}
            \input{8 - Hilbert complexes/main.tex}
            \input{9 - weak conservation proofs/main.tex}
\end{document}

        \part{Research}
            \documentclass[12pt, a4paper]{report}

\input{template/main.tex}

\title{\BA{Title in Progress...}}
\author{Boris Andrews}
\affil{Mathematical Institute, University of Oxford}
\date{\today}


\begin{document}
    \pagenumbering{gobble}
    \maketitle
    
    
    \begin{abstract}
        Magnetic confinement reactors---in particular tokamaks---offer one of the most promising options for achieving practical nuclear fusion, with the potential to provide virtually limitless, clean energy. The theoretical and numerical modeling of tokamak plasmas is simultaneously an essential component of effective reactor design, and a great research barrier. Tokamak operational conditions exhibit comparatively low Knudsen numbers. Kinetic effects, including kinetic waves and instabilities, Landau damping, bump-on-tail instabilities and more, are therefore highly influential in tokamak plasma dynamics. Purely fluid models are inherently incapable of capturing these effects, whereas the high dimensionality in purely kinetic models render them practically intractable for most relevant purposes.

        We consider a $\delta\!f$ decomposition model, with a macroscopic fluid background and microscopic kinetic correction, both fully coupled to each other. A similar manner of discretization is proposed to that used in the recent \texttt{STRUPHY} code \cite{Holderied_Possanner_Wang_2021, Holderied_2022, Li_et_al_2023} with a finite-element model for the background and a pseudo-particle/PiC model for the correction.

        The fluid background satisfies the full, non-linear, resistive, compressible, Hall MHD equations. \cite{Laakmann_Hu_Farrell_2022} introduces finite-element(-in-space) implicit timesteppers for the incompressible analogue to this system with structure-preserving (SP) properties in the ideal case, alongside parameter-robust preconditioners. We show that these timesteppers can derive from a finite-element-in-time (FET) (and finite-element-in-space) interpretation. The benefits of this reformulation are discussed, including the derivation of timesteppers that are higher order in time, and the quantifiable dissipative SP properties in the non-ideal, resistive case.
        
        We discuss possible options for extending this FET approach to timesteppers for the compressible case.

        The kinetic corrections satisfy linearized Boltzmann equations. Using a Lénard--Bernstein collision operator, these take Fokker--Planck-like forms \cite{Fokker_1914, Planck_1917} wherein pseudo-particles in the numerical model obey the neoclassical transport equations, with particle-independent Brownian drift terms. This offers a rigorous methodology for incorporating collisions into the particle transport model, without coupling the equations of motions for each particle.
        
        Works by Chen, Chacón et al. \cite{Chen_Chacón_Barnes_2011, Chacón_Chen_Barnes_2013, Chen_Chacón_2014, Chen_Chacón_2015} have developed structure-preserving particle pushers for neoclassical transport in the Vlasov equations, derived from Crank--Nicolson integrators. We show these too can can derive from a FET interpretation, similarly offering potential extensions to higher-order-in-time particle pushers. The FET formulation is used also to consider how the stochastic drift terms can be incorporated into the pushers. Stochastic gyrokinetic expansions are also discussed.

        Different options for the numerical implementation of these schemes are considered.

        Due to the efficacy of FET in the development of SP timesteppers for both the fluid and kinetic component, we hope this approach will prove effective in the future for developing SP timesteppers for the full hybrid model. We hope this will give us the opportunity to incorporate previously inaccessible kinetic effects into the highly effective, modern, finite-element MHD models.
    \end{abstract}
    
    
    \newpage
    \tableofcontents
    
    
    \newpage
    \pagenumbering{arabic}
    %\linenumbers\renewcommand\thelinenumber{\color{black!50}\arabic{linenumber}}
            \input{0 - introduction/main.tex}
        \part{Research}
            \input{1 - low-noise PiC models/main.tex}
            \input{2 - kinetic component/main.tex}
            \input{3 - fluid component/main.tex}
            \input{4 - numerical implementation/main.tex}
        \part{Project Overview}
            \input{5 - research plan/main.tex}
            \input{6 - summary/main.tex}
    
    
    %\section{}
    \newpage
    \pagenumbering{gobble}
        \printbibliography


    \newpage
    \pagenumbering{roman}
    \appendix
        \part{Appendices}
            \input{8 - Hilbert complexes/main.tex}
            \input{9 - weak conservation proofs/main.tex}
\end{document}

            \documentclass[12pt, a4paper]{report}

\input{template/main.tex}

\title{\BA{Title in Progress...}}
\author{Boris Andrews}
\affil{Mathematical Institute, University of Oxford}
\date{\today}


\begin{document}
    \pagenumbering{gobble}
    \maketitle
    
    
    \begin{abstract}
        Magnetic confinement reactors---in particular tokamaks---offer one of the most promising options for achieving practical nuclear fusion, with the potential to provide virtually limitless, clean energy. The theoretical and numerical modeling of tokamak plasmas is simultaneously an essential component of effective reactor design, and a great research barrier. Tokamak operational conditions exhibit comparatively low Knudsen numbers. Kinetic effects, including kinetic waves and instabilities, Landau damping, bump-on-tail instabilities and more, are therefore highly influential in tokamak plasma dynamics. Purely fluid models are inherently incapable of capturing these effects, whereas the high dimensionality in purely kinetic models render them practically intractable for most relevant purposes.

        We consider a $\delta\!f$ decomposition model, with a macroscopic fluid background and microscopic kinetic correction, both fully coupled to each other. A similar manner of discretization is proposed to that used in the recent \texttt{STRUPHY} code \cite{Holderied_Possanner_Wang_2021, Holderied_2022, Li_et_al_2023} with a finite-element model for the background and a pseudo-particle/PiC model for the correction.

        The fluid background satisfies the full, non-linear, resistive, compressible, Hall MHD equations. \cite{Laakmann_Hu_Farrell_2022} introduces finite-element(-in-space) implicit timesteppers for the incompressible analogue to this system with structure-preserving (SP) properties in the ideal case, alongside parameter-robust preconditioners. We show that these timesteppers can derive from a finite-element-in-time (FET) (and finite-element-in-space) interpretation. The benefits of this reformulation are discussed, including the derivation of timesteppers that are higher order in time, and the quantifiable dissipative SP properties in the non-ideal, resistive case.
        
        We discuss possible options for extending this FET approach to timesteppers for the compressible case.

        The kinetic corrections satisfy linearized Boltzmann equations. Using a Lénard--Bernstein collision operator, these take Fokker--Planck-like forms \cite{Fokker_1914, Planck_1917} wherein pseudo-particles in the numerical model obey the neoclassical transport equations, with particle-independent Brownian drift terms. This offers a rigorous methodology for incorporating collisions into the particle transport model, without coupling the equations of motions for each particle.
        
        Works by Chen, Chacón et al. \cite{Chen_Chacón_Barnes_2011, Chacón_Chen_Barnes_2013, Chen_Chacón_2014, Chen_Chacón_2015} have developed structure-preserving particle pushers for neoclassical transport in the Vlasov equations, derived from Crank--Nicolson integrators. We show these too can can derive from a FET interpretation, similarly offering potential extensions to higher-order-in-time particle pushers. The FET formulation is used also to consider how the stochastic drift terms can be incorporated into the pushers. Stochastic gyrokinetic expansions are also discussed.

        Different options for the numerical implementation of these schemes are considered.

        Due to the efficacy of FET in the development of SP timesteppers for both the fluid and kinetic component, we hope this approach will prove effective in the future for developing SP timesteppers for the full hybrid model. We hope this will give us the opportunity to incorporate previously inaccessible kinetic effects into the highly effective, modern, finite-element MHD models.
    \end{abstract}
    
    
    \newpage
    \tableofcontents
    
    
    \newpage
    \pagenumbering{arabic}
    %\linenumbers\renewcommand\thelinenumber{\color{black!50}\arabic{linenumber}}
            \input{0 - introduction/main.tex}
        \part{Research}
            \input{1 - low-noise PiC models/main.tex}
            \input{2 - kinetic component/main.tex}
            \input{3 - fluid component/main.tex}
            \input{4 - numerical implementation/main.tex}
        \part{Project Overview}
            \input{5 - research plan/main.tex}
            \input{6 - summary/main.tex}
    
    
    %\section{}
    \newpage
    \pagenumbering{gobble}
        \printbibliography


    \newpage
    \pagenumbering{roman}
    \appendix
        \part{Appendices}
            \input{8 - Hilbert complexes/main.tex}
            \input{9 - weak conservation proofs/main.tex}
\end{document}

            \documentclass[12pt, a4paper]{report}

\input{template/main.tex}

\title{\BA{Title in Progress...}}
\author{Boris Andrews}
\affil{Mathematical Institute, University of Oxford}
\date{\today}


\begin{document}
    \pagenumbering{gobble}
    \maketitle
    
    
    \begin{abstract}
        Magnetic confinement reactors---in particular tokamaks---offer one of the most promising options for achieving practical nuclear fusion, with the potential to provide virtually limitless, clean energy. The theoretical and numerical modeling of tokamak plasmas is simultaneously an essential component of effective reactor design, and a great research barrier. Tokamak operational conditions exhibit comparatively low Knudsen numbers. Kinetic effects, including kinetic waves and instabilities, Landau damping, bump-on-tail instabilities and more, are therefore highly influential in tokamak plasma dynamics. Purely fluid models are inherently incapable of capturing these effects, whereas the high dimensionality in purely kinetic models render them practically intractable for most relevant purposes.

        We consider a $\delta\!f$ decomposition model, with a macroscopic fluid background and microscopic kinetic correction, both fully coupled to each other. A similar manner of discretization is proposed to that used in the recent \texttt{STRUPHY} code \cite{Holderied_Possanner_Wang_2021, Holderied_2022, Li_et_al_2023} with a finite-element model for the background and a pseudo-particle/PiC model for the correction.

        The fluid background satisfies the full, non-linear, resistive, compressible, Hall MHD equations. \cite{Laakmann_Hu_Farrell_2022} introduces finite-element(-in-space) implicit timesteppers for the incompressible analogue to this system with structure-preserving (SP) properties in the ideal case, alongside parameter-robust preconditioners. We show that these timesteppers can derive from a finite-element-in-time (FET) (and finite-element-in-space) interpretation. The benefits of this reformulation are discussed, including the derivation of timesteppers that are higher order in time, and the quantifiable dissipative SP properties in the non-ideal, resistive case.
        
        We discuss possible options for extending this FET approach to timesteppers for the compressible case.

        The kinetic corrections satisfy linearized Boltzmann equations. Using a Lénard--Bernstein collision operator, these take Fokker--Planck-like forms \cite{Fokker_1914, Planck_1917} wherein pseudo-particles in the numerical model obey the neoclassical transport equations, with particle-independent Brownian drift terms. This offers a rigorous methodology for incorporating collisions into the particle transport model, without coupling the equations of motions for each particle.
        
        Works by Chen, Chacón et al. \cite{Chen_Chacón_Barnes_2011, Chacón_Chen_Barnes_2013, Chen_Chacón_2014, Chen_Chacón_2015} have developed structure-preserving particle pushers for neoclassical transport in the Vlasov equations, derived from Crank--Nicolson integrators. We show these too can can derive from a FET interpretation, similarly offering potential extensions to higher-order-in-time particle pushers. The FET formulation is used also to consider how the stochastic drift terms can be incorporated into the pushers. Stochastic gyrokinetic expansions are also discussed.

        Different options for the numerical implementation of these schemes are considered.

        Due to the efficacy of FET in the development of SP timesteppers for both the fluid and kinetic component, we hope this approach will prove effective in the future for developing SP timesteppers for the full hybrid model. We hope this will give us the opportunity to incorporate previously inaccessible kinetic effects into the highly effective, modern, finite-element MHD models.
    \end{abstract}
    
    
    \newpage
    \tableofcontents
    
    
    \newpage
    \pagenumbering{arabic}
    %\linenumbers\renewcommand\thelinenumber{\color{black!50}\arabic{linenumber}}
            \input{0 - introduction/main.tex}
        \part{Research}
            \input{1 - low-noise PiC models/main.tex}
            \input{2 - kinetic component/main.tex}
            \input{3 - fluid component/main.tex}
            \input{4 - numerical implementation/main.tex}
        \part{Project Overview}
            \input{5 - research plan/main.tex}
            \input{6 - summary/main.tex}
    
    
    %\section{}
    \newpage
    \pagenumbering{gobble}
        \printbibliography


    \newpage
    \pagenumbering{roman}
    \appendix
        \part{Appendices}
            \input{8 - Hilbert complexes/main.tex}
            \input{9 - weak conservation proofs/main.tex}
\end{document}

            \documentclass[12pt, a4paper]{report}

\input{template/main.tex}

\title{\BA{Title in Progress...}}
\author{Boris Andrews}
\affil{Mathematical Institute, University of Oxford}
\date{\today}


\begin{document}
    \pagenumbering{gobble}
    \maketitle
    
    
    \begin{abstract}
        Magnetic confinement reactors---in particular tokamaks---offer one of the most promising options for achieving practical nuclear fusion, with the potential to provide virtually limitless, clean energy. The theoretical and numerical modeling of tokamak plasmas is simultaneously an essential component of effective reactor design, and a great research barrier. Tokamak operational conditions exhibit comparatively low Knudsen numbers. Kinetic effects, including kinetic waves and instabilities, Landau damping, bump-on-tail instabilities and more, are therefore highly influential in tokamak plasma dynamics. Purely fluid models are inherently incapable of capturing these effects, whereas the high dimensionality in purely kinetic models render them practically intractable for most relevant purposes.

        We consider a $\delta\!f$ decomposition model, with a macroscopic fluid background and microscopic kinetic correction, both fully coupled to each other. A similar manner of discretization is proposed to that used in the recent \texttt{STRUPHY} code \cite{Holderied_Possanner_Wang_2021, Holderied_2022, Li_et_al_2023} with a finite-element model for the background and a pseudo-particle/PiC model for the correction.

        The fluid background satisfies the full, non-linear, resistive, compressible, Hall MHD equations. \cite{Laakmann_Hu_Farrell_2022} introduces finite-element(-in-space) implicit timesteppers for the incompressible analogue to this system with structure-preserving (SP) properties in the ideal case, alongside parameter-robust preconditioners. We show that these timesteppers can derive from a finite-element-in-time (FET) (and finite-element-in-space) interpretation. The benefits of this reformulation are discussed, including the derivation of timesteppers that are higher order in time, and the quantifiable dissipative SP properties in the non-ideal, resistive case.
        
        We discuss possible options for extending this FET approach to timesteppers for the compressible case.

        The kinetic corrections satisfy linearized Boltzmann equations. Using a Lénard--Bernstein collision operator, these take Fokker--Planck-like forms \cite{Fokker_1914, Planck_1917} wherein pseudo-particles in the numerical model obey the neoclassical transport equations, with particle-independent Brownian drift terms. This offers a rigorous methodology for incorporating collisions into the particle transport model, without coupling the equations of motions for each particle.
        
        Works by Chen, Chacón et al. \cite{Chen_Chacón_Barnes_2011, Chacón_Chen_Barnes_2013, Chen_Chacón_2014, Chen_Chacón_2015} have developed structure-preserving particle pushers for neoclassical transport in the Vlasov equations, derived from Crank--Nicolson integrators. We show these too can can derive from a FET interpretation, similarly offering potential extensions to higher-order-in-time particle pushers. The FET formulation is used also to consider how the stochastic drift terms can be incorporated into the pushers. Stochastic gyrokinetic expansions are also discussed.

        Different options for the numerical implementation of these schemes are considered.

        Due to the efficacy of FET in the development of SP timesteppers for both the fluid and kinetic component, we hope this approach will prove effective in the future for developing SP timesteppers for the full hybrid model. We hope this will give us the opportunity to incorporate previously inaccessible kinetic effects into the highly effective, modern, finite-element MHD models.
    \end{abstract}
    
    
    \newpage
    \tableofcontents
    
    
    \newpage
    \pagenumbering{arabic}
    %\linenumbers\renewcommand\thelinenumber{\color{black!50}\arabic{linenumber}}
            \input{0 - introduction/main.tex}
        \part{Research}
            \input{1 - low-noise PiC models/main.tex}
            \input{2 - kinetic component/main.tex}
            \input{3 - fluid component/main.tex}
            \input{4 - numerical implementation/main.tex}
        \part{Project Overview}
            \input{5 - research plan/main.tex}
            \input{6 - summary/main.tex}
    
    
    %\section{}
    \newpage
    \pagenumbering{gobble}
        \printbibliography


    \newpage
    \pagenumbering{roman}
    \appendix
        \part{Appendices}
            \input{8 - Hilbert complexes/main.tex}
            \input{9 - weak conservation proofs/main.tex}
\end{document}

        \part{Project Overview}
            \documentclass[12pt, a4paper]{report}

\input{template/main.tex}

\title{\BA{Title in Progress...}}
\author{Boris Andrews}
\affil{Mathematical Institute, University of Oxford}
\date{\today}


\begin{document}
    \pagenumbering{gobble}
    \maketitle
    
    
    \begin{abstract}
        Magnetic confinement reactors---in particular tokamaks---offer one of the most promising options for achieving practical nuclear fusion, with the potential to provide virtually limitless, clean energy. The theoretical and numerical modeling of tokamak plasmas is simultaneously an essential component of effective reactor design, and a great research barrier. Tokamak operational conditions exhibit comparatively low Knudsen numbers. Kinetic effects, including kinetic waves and instabilities, Landau damping, bump-on-tail instabilities and more, are therefore highly influential in tokamak plasma dynamics. Purely fluid models are inherently incapable of capturing these effects, whereas the high dimensionality in purely kinetic models render them practically intractable for most relevant purposes.

        We consider a $\delta\!f$ decomposition model, with a macroscopic fluid background and microscopic kinetic correction, both fully coupled to each other. A similar manner of discretization is proposed to that used in the recent \texttt{STRUPHY} code \cite{Holderied_Possanner_Wang_2021, Holderied_2022, Li_et_al_2023} with a finite-element model for the background and a pseudo-particle/PiC model for the correction.

        The fluid background satisfies the full, non-linear, resistive, compressible, Hall MHD equations. \cite{Laakmann_Hu_Farrell_2022} introduces finite-element(-in-space) implicit timesteppers for the incompressible analogue to this system with structure-preserving (SP) properties in the ideal case, alongside parameter-robust preconditioners. We show that these timesteppers can derive from a finite-element-in-time (FET) (and finite-element-in-space) interpretation. The benefits of this reformulation are discussed, including the derivation of timesteppers that are higher order in time, and the quantifiable dissipative SP properties in the non-ideal, resistive case.
        
        We discuss possible options for extending this FET approach to timesteppers for the compressible case.

        The kinetic corrections satisfy linearized Boltzmann equations. Using a Lénard--Bernstein collision operator, these take Fokker--Planck-like forms \cite{Fokker_1914, Planck_1917} wherein pseudo-particles in the numerical model obey the neoclassical transport equations, with particle-independent Brownian drift terms. This offers a rigorous methodology for incorporating collisions into the particle transport model, without coupling the equations of motions for each particle.
        
        Works by Chen, Chacón et al. \cite{Chen_Chacón_Barnes_2011, Chacón_Chen_Barnes_2013, Chen_Chacón_2014, Chen_Chacón_2015} have developed structure-preserving particle pushers for neoclassical transport in the Vlasov equations, derived from Crank--Nicolson integrators. We show these too can can derive from a FET interpretation, similarly offering potential extensions to higher-order-in-time particle pushers. The FET formulation is used also to consider how the stochastic drift terms can be incorporated into the pushers. Stochastic gyrokinetic expansions are also discussed.

        Different options for the numerical implementation of these schemes are considered.

        Due to the efficacy of FET in the development of SP timesteppers for both the fluid and kinetic component, we hope this approach will prove effective in the future for developing SP timesteppers for the full hybrid model. We hope this will give us the opportunity to incorporate previously inaccessible kinetic effects into the highly effective, modern, finite-element MHD models.
    \end{abstract}
    
    
    \newpage
    \tableofcontents
    
    
    \newpage
    \pagenumbering{arabic}
    %\linenumbers\renewcommand\thelinenumber{\color{black!50}\arabic{linenumber}}
            \input{0 - introduction/main.tex}
        \part{Research}
            \input{1 - low-noise PiC models/main.tex}
            \input{2 - kinetic component/main.tex}
            \input{3 - fluid component/main.tex}
            \input{4 - numerical implementation/main.tex}
        \part{Project Overview}
            \input{5 - research plan/main.tex}
            \input{6 - summary/main.tex}
    
    
    %\section{}
    \newpage
    \pagenumbering{gobble}
        \printbibliography


    \newpage
    \pagenumbering{roman}
    \appendix
        \part{Appendices}
            \input{8 - Hilbert complexes/main.tex}
            \input{9 - weak conservation proofs/main.tex}
\end{document}

            \documentclass[12pt, a4paper]{report}

\input{template/main.tex}

\title{\BA{Title in Progress...}}
\author{Boris Andrews}
\affil{Mathematical Institute, University of Oxford}
\date{\today}


\begin{document}
    \pagenumbering{gobble}
    \maketitle
    
    
    \begin{abstract}
        Magnetic confinement reactors---in particular tokamaks---offer one of the most promising options for achieving practical nuclear fusion, with the potential to provide virtually limitless, clean energy. The theoretical and numerical modeling of tokamak plasmas is simultaneously an essential component of effective reactor design, and a great research barrier. Tokamak operational conditions exhibit comparatively low Knudsen numbers. Kinetic effects, including kinetic waves and instabilities, Landau damping, bump-on-tail instabilities and more, are therefore highly influential in tokamak plasma dynamics. Purely fluid models are inherently incapable of capturing these effects, whereas the high dimensionality in purely kinetic models render them practically intractable for most relevant purposes.

        We consider a $\delta\!f$ decomposition model, with a macroscopic fluid background and microscopic kinetic correction, both fully coupled to each other. A similar manner of discretization is proposed to that used in the recent \texttt{STRUPHY} code \cite{Holderied_Possanner_Wang_2021, Holderied_2022, Li_et_al_2023} with a finite-element model for the background and a pseudo-particle/PiC model for the correction.

        The fluid background satisfies the full, non-linear, resistive, compressible, Hall MHD equations. \cite{Laakmann_Hu_Farrell_2022} introduces finite-element(-in-space) implicit timesteppers for the incompressible analogue to this system with structure-preserving (SP) properties in the ideal case, alongside parameter-robust preconditioners. We show that these timesteppers can derive from a finite-element-in-time (FET) (and finite-element-in-space) interpretation. The benefits of this reformulation are discussed, including the derivation of timesteppers that are higher order in time, and the quantifiable dissipative SP properties in the non-ideal, resistive case.
        
        We discuss possible options for extending this FET approach to timesteppers for the compressible case.

        The kinetic corrections satisfy linearized Boltzmann equations. Using a Lénard--Bernstein collision operator, these take Fokker--Planck-like forms \cite{Fokker_1914, Planck_1917} wherein pseudo-particles in the numerical model obey the neoclassical transport equations, with particle-independent Brownian drift terms. This offers a rigorous methodology for incorporating collisions into the particle transport model, without coupling the equations of motions for each particle.
        
        Works by Chen, Chacón et al. \cite{Chen_Chacón_Barnes_2011, Chacón_Chen_Barnes_2013, Chen_Chacón_2014, Chen_Chacón_2015} have developed structure-preserving particle pushers for neoclassical transport in the Vlasov equations, derived from Crank--Nicolson integrators. We show these too can can derive from a FET interpretation, similarly offering potential extensions to higher-order-in-time particle pushers. The FET formulation is used also to consider how the stochastic drift terms can be incorporated into the pushers. Stochastic gyrokinetic expansions are also discussed.

        Different options for the numerical implementation of these schemes are considered.

        Due to the efficacy of FET in the development of SP timesteppers for both the fluid and kinetic component, we hope this approach will prove effective in the future for developing SP timesteppers for the full hybrid model. We hope this will give us the opportunity to incorporate previously inaccessible kinetic effects into the highly effective, modern, finite-element MHD models.
    \end{abstract}
    
    
    \newpage
    \tableofcontents
    
    
    \newpage
    \pagenumbering{arabic}
    %\linenumbers\renewcommand\thelinenumber{\color{black!50}\arabic{linenumber}}
            \input{0 - introduction/main.tex}
        \part{Research}
            \input{1 - low-noise PiC models/main.tex}
            \input{2 - kinetic component/main.tex}
            \input{3 - fluid component/main.tex}
            \input{4 - numerical implementation/main.tex}
        \part{Project Overview}
            \input{5 - research plan/main.tex}
            \input{6 - summary/main.tex}
    
    
    %\section{}
    \newpage
    \pagenumbering{gobble}
        \printbibliography


    \newpage
    \pagenumbering{roman}
    \appendix
        \part{Appendices}
            \input{8 - Hilbert complexes/main.tex}
            \input{9 - weak conservation proofs/main.tex}
\end{document}

    
    
    %\section{}
    \newpage
    \pagenumbering{gobble}
        \printbibliography


    \newpage
    \pagenumbering{roman}
    \appendix
        \part{Appendices}
            \documentclass[12pt, a4paper]{report}

\input{template/main.tex}

\title{\BA{Title in Progress...}}
\author{Boris Andrews}
\affil{Mathematical Institute, University of Oxford}
\date{\today}


\begin{document}
    \pagenumbering{gobble}
    \maketitle
    
    
    \begin{abstract}
        Magnetic confinement reactors---in particular tokamaks---offer one of the most promising options for achieving practical nuclear fusion, with the potential to provide virtually limitless, clean energy. The theoretical and numerical modeling of tokamak plasmas is simultaneously an essential component of effective reactor design, and a great research barrier. Tokamak operational conditions exhibit comparatively low Knudsen numbers. Kinetic effects, including kinetic waves and instabilities, Landau damping, bump-on-tail instabilities and more, are therefore highly influential in tokamak plasma dynamics. Purely fluid models are inherently incapable of capturing these effects, whereas the high dimensionality in purely kinetic models render them practically intractable for most relevant purposes.

        We consider a $\delta\!f$ decomposition model, with a macroscopic fluid background and microscopic kinetic correction, both fully coupled to each other. A similar manner of discretization is proposed to that used in the recent \texttt{STRUPHY} code \cite{Holderied_Possanner_Wang_2021, Holderied_2022, Li_et_al_2023} with a finite-element model for the background and a pseudo-particle/PiC model for the correction.

        The fluid background satisfies the full, non-linear, resistive, compressible, Hall MHD equations. \cite{Laakmann_Hu_Farrell_2022} introduces finite-element(-in-space) implicit timesteppers for the incompressible analogue to this system with structure-preserving (SP) properties in the ideal case, alongside parameter-robust preconditioners. We show that these timesteppers can derive from a finite-element-in-time (FET) (and finite-element-in-space) interpretation. The benefits of this reformulation are discussed, including the derivation of timesteppers that are higher order in time, and the quantifiable dissipative SP properties in the non-ideal, resistive case.
        
        We discuss possible options for extending this FET approach to timesteppers for the compressible case.

        The kinetic corrections satisfy linearized Boltzmann equations. Using a Lénard--Bernstein collision operator, these take Fokker--Planck-like forms \cite{Fokker_1914, Planck_1917} wherein pseudo-particles in the numerical model obey the neoclassical transport equations, with particle-independent Brownian drift terms. This offers a rigorous methodology for incorporating collisions into the particle transport model, without coupling the equations of motions for each particle.
        
        Works by Chen, Chacón et al. \cite{Chen_Chacón_Barnes_2011, Chacón_Chen_Barnes_2013, Chen_Chacón_2014, Chen_Chacón_2015} have developed structure-preserving particle pushers for neoclassical transport in the Vlasov equations, derived from Crank--Nicolson integrators. We show these too can can derive from a FET interpretation, similarly offering potential extensions to higher-order-in-time particle pushers. The FET formulation is used also to consider how the stochastic drift terms can be incorporated into the pushers. Stochastic gyrokinetic expansions are also discussed.

        Different options for the numerical implementation of these schemes are considered.

        Due to the efficacy of FET in the development of SP timesteppers for both the fluid and kinetic component, we hope this approach will prove effective in the future for developing SP timesteppers for the full hybrid model. We hope this will give us the opportunity to incorporate previously inaccessible kinetic effects into the highly effective, modern, finite-element MHD models.
    \end{abstract}
    
    
    \newpage
    \tableofcontents
    
    
    \newpage
    \pagenumbering{arabic}
    %\linenumbers\renewcommand\thelinenumber{\color{black!50}\arabic{linenumber}}
            \input{0 - introduction/main.tex}
        \part{Research}
            \input{1 - low-noise PiC models/main.tex}
            \input{2 - kinetic component/main.tex}
            \input{3 - fluid component/main.tex}
            \input{4 - numerical implementation/main.tex}
        \part{Project Overview}
            \input{5 - research plan/main.tex}
            \input{6 - summary/main.tex}
    
    
    %\section{}
    \newpage
    \pagenumbering{gobble}
        \printbibliography


    \newpage
    \pagenumbering{roman}
    \appendix
        \part{Appendices}
            \input{8 - Hilbert complexes/main.tex}
            \input{9 - weak conservation proofs/main.tex}
\end{document}

            \documentclass[12pt, a4paper]{report}

\input{template/main.tex}

\title{\BA{Title in Progress...}}
\author{Boris Andrews}
\affil{Mathematical Institute, University of Oxford}
\date{\today}


\begin{document}
    \pagenumbering{gobble}
    \maketitle
    
    
    \begin{abstract}
        Magnetic confinement reactors---in particular tokamaks---offer one of the most promising options for achieving practical nuclear fusion, with the potential to provide virtually limitless, clean energy. The theoretical and numerical modeling of tokamak plasmas is simultaneously an essential component of effective reactor design, and a great research barrier. Tokamak operational conditions exhibit comparatively low Knudsen numbers. Kinetic effects, including kinetic waves and instabilities, Landau damping, bump-on-tail instabilities and more, are therefore highly influential in tokamak plasma dynamics. Purely fluid models are inherently incapable of capturing these effects, whereas the high dimensionality in purely kinetic models render them practically intractable for most relevant purposes.

        We consider a $\delta\!f$ decomposition model, with a macroscopic fluid background and microscopic kinetic correction, both fully coupled to each other. A similar manner of discretization is proposed to that used in the recent \texttt{STRUPHY} code \cite{Holderied_Possanner_Wang_2021, Holderied_2022, Li_et_al_2023} with a finite-element model for the background and a pseudo-particle/PiC model for the correction.

        The fluid background satisfies the full, non-linear, resistive, compressible, Hall MHD equations. \cite{Laakmann_Hu_Farrell_2022} introduces finite-element(-in-space) implicit timesteppers for the incompressible analogue to this system with structure-preserving (SP) properties in the ideal case, alongside parameter-robust preconditioners. We show that these timesteppers can derive from a finite-element-in-time (FET) (and finite-element-in-space) interpretation. The benefits of this reformulation are discussed, including the derivation of timesteppers that are higher order in time, and the quantifiable dissipative SP properties in the non-ideal, resistive case.
        
        We discuss possible options for extending this FET approach to timesteppers for the compressible case.

        The kinetic corrections satisfy linearized Boltzmann equations. Using a Lénard--Bernstein collision operator, these take Fokker--Planck-like forms \cite{Fokker_1914, Planck_1917} wherein pseudo-particles in the numerical model obey the neoclassical transport equations, with particle-independent Brownian drift terms. This offers a rigorous methodology for incorporating collisions into the particle transport model, without coupling the equations of motions for each particle.
        
        Works by Chen, Chacón et al. \cite{Chen_Chacón_Barnes_2011, Chacón_Chen_Barnes_2013, Chen_Chacón_2014, Chen_Chacón_2015} have developed structure-preserving particle pushers for neoclassical transport in the Vlasov equations, derived from Crank--Nicolson integrators. We show these too can can derive from a FET interpretation, similarly offering potential extensions to higher-order-in-time particle pushers. The FET formulation is used also to consider how the stochastic drift terms can be incorporated into the pushers. Stochastic gyrokinetic expansions are also discussed.

        Different options for the numerical implementation of these schemes are considered.

        Due to the efficacy of FET in the development of SP timesteppers for both the fluid and kinetic component, we hope this approach will prove effective in the future for developing SP timesteppers for the full hybrid model. We hope this will give us the opportunity to incorporate previously inaccessible kinetic effects into the highly effective, modern, finite-element MHD models.
    \end{abstract}
    
    
    \newpage
    \tableofcontents
    
    
    \newpage
    \pagenumbering{arabic}
    %\linenumbers\renewcommand\thelinenumber{\color{black!50}\arabic{linenumber}}
            \input{0 - introduction/main.tex}
        \part{Research}
            \input{1 - low-noise PiC models/main.tex}
            \input{2 - kinetic component/main.tex}
            \input{3 - fluid component/main.tex}
            \input{4 - numerical implementation/main.tex}
        \part{Project Overview}
            \input{5 - research plan/main.tex}
            \input{6 - summary/main.tex}
    
    
    %\section{}
    \newpage
    \pagenumbering{gobble}
        \printbibliography


    \newpage
    \pagenumbering{roman}
    \appendix
        \part{Appendices}
            \input{8 - Hilbert complexes/main.tex}
            \input{9 - weak conservation proofs/main.tex}
\end{document}

\end{document}

            \documentclass[12pt, a4paper]{report}

\documentclass[12pt, a4paper]{report}

\input{template/main.tex}

\title{\BA{Title in Progress...}}
\author{Boris Andrews}
\affil{Mathematical Institute, University of Oxford}
\date{\today}


\begin{document}
    \pagenumbering{gobble}
    \maketitle
    
    
    \begin{abstract}
        Magnetic confinement reactors---in particular tokamaks---offer one of the most promising options for achieving practical nuclear fusion, with the potential to provide virtually limitless, clean energy. The theoretical and numerical modeling of tokamak plasmas is simultaneously an essential component of effective reactor design, and a great research barrier. Tokamak operational conditions exhibit comparatively low Knudsen numbers. Kinetic effects, including kinetic waves and instabilities, Landau damping, bump-on-tail instabilities and more, are therefore highly influential in tokamak plasma dynamics. Purely fluid models are inherently incapable of capturing these effects, whereas the high dimensionality in purely kinetic models render them practically intractable for most relevant purposes.

        We consider a $\delta\!f$ decomposition model, with a macroscopic fluid background and microscopic kinetic correction, both fully coupled to each other. A similar manner of discretization is proposed to that used in the recent \texttt{STRUPHY} code \cite{Holderied_Possanner_Wang_2021, Holderied_2022, Li_et_al_2023} with a finite-element model for the background and a pseudo-particle/PiC model for the correction.

        The fluid background satisfies the full, non-linear, resistive, compressible, Hall MHD equations. \cite{Laakmann_Hu_Farrell_2022} introduces finite-element(-in-space) implicit timesteppers for the incompressible analogue to this system with structure-preserving (SP) properties in the ideal case, alongside parameter-robust preconditioners. We show that these timesteppers can derive from a finite-element-in-time (FET) (and finite-element-in-space) interpretation. The benefits of this reformulation are discussed, including the derivation of timesteppers that are higher order in time, and the quantifiable dissipative SP properties in the non-ideal, resistive case.
        
        We discuss possible options for extending this FET approach to timesteppers for the compressible case.

        The kinetic corrections satisfy linearized Boltzmann equations. Using a Lénard--Bernstein collision operator, these take Fokker--Planck-like forms \cite{Fokker_1914, Planck_1917} wherein pseudo-particles in the numerical model obey the neoclassical transport equations, with particle-independent Brownian drift terms. This offers a rigorous methodology for incorporating collisions into the particle transport model, without coupling the equations of motions for each particle.
        
        Works by Chen, Chacón et al. \cite{Chen_Chacón_Barnes_2011, Chacón_Chen_Barnes_2013, Chen_Chacón_2014, Chen_Chacón_2015} have developed structure-preserving particle pushers for neoclassical transport in the Vlasov equations, derived from Crank--Nicolson integrators. We show these too can can derive from a FET interpretation, similarly offering potential extensions to higher-order-in-time particle pushers. The FET formulation is used also to consider how the stochastic drift terms can be incorporated into the pushers. Stochastic gyrokinetic expansions are also discussed.

        Different options for the numerical implementation of these schemes are considered.

        Due to the efficacy of FET in the development of SP timesteppers for both the fluid and kinetic component, we hope this approach will prove effective in the future for developing SP timesteppers for the full hybrid model. We hope this will give us the opportunity to incorporate previously inaccessible kinetic effects into the highly effective, modern, finite-element MHD models.
    \end{abstract}
    
    
    \newpage
    \tableofcontents
    
    
    \newpage
    \pagenumbering{arabic}
    %\linenumbers\renewcommand\thelinenumber{\color{black!50}\arabic{linenumber}}
            \input{0 - introduction/main.tex}
        \part{Research}
            \input{1 - low-noise PiC models/main.tex}
            \input{2 - kinetic component/main.tex}
            \input{3 - fluid component/main.tex}
            \input{4 - numerical implementation/main.tex}
        \part{Project Overview}
            \input{5 - research plan/main.tex}
            \input{6 - summary/main.tex}
    
    
    %\section{}
    \newpage
    \pagenumbering{gobble}
        \printbibliography


    \newpage
    \pagenumbering{roman}
    \appendix
        \part{Appendices}
            \input{8 - Hilbert complexes/main.tex}
            \input{9 - weak conservation proofs/main.tex}
\end{document}


\title{\BA{Title in Progress...}}
\author{Boris Andrews}
\affil{Mathematical Institute, University of Oxford}
\date{\today}


\begin{document}
    \pagenumbering{gobble}
    \maketitle
    
    
    \begin{abstract}
        Magnetic confinement reactors---in particular tokamaks---offer one of the most promising options for achieving practical nuclear fusion, with the potential to provide virtually limitless, clean energy. The theoretical and numerical modeling of tokamak plasmas is simultaneously an essential component of effective reactor design, and a great research barrier. Tokamak operational conditions exhibit comparatively low Knudsen numbers. Kinetic effects, including kinetic waves and instabilities, Landau damping, bump-on-tail instabilities and more, are therefore highly influential in tokamak plasma dynamics. Purely fluid models are inherently incapable of capturing these effects, whereas the high dimensionality in purely kinetic models render them practically intractable for most relevant purposes.

        We consider a $\delta\!f$ decomposition model, with a macroscopic fluid background and microscopic kinetic correction, both fully coupled to each other. A similar manner of discretization is proposed to that used in the recent \texttt{STRUPHY} code \cite{Holderied_Possanner_Wang_2021, Holderied_2022, Li_et_al_2023} with a finite-element model for the background and a pseudo-particle/PiC model for the correction.

        The fluid background satisfies the full, non-linear, resistive, compressible, Hall MHD equations. \cite{Laakmann_Hu_Farrell_2022} introduces finite-element(-in-space) implicit timesteppers for the incompressible analogue to this system with structure-preserving (SP) properties in the ideal case, alongside parameter-robust preconditioners. We show that these timesteppers can derive from a finite-element-in-time (FET) (and finite-element-in-space) interpretation. The benefits of this reformulation are discussed, including the derivation of timesteppers that are higher order in time, and the quantifiable dissipative SP properties in the non-ideal, resistive case.
        
        We discuss possible options for extending this FET approach to timesteppers for the compressible case.

        The kinetic corrections satisfy linearized Boltzmann equations. Using a Lénard--Bernstein collision operator, these take Fokker--Planck-like forms \cite{Fokker_1914, Planck_1917} wherein pseudo-particles in the numerical model obey the neoclassical transport equations, with particle-independent Brownian drift terms. This offers a rigorous methodology for incorporating collisions into the particle transport model, without coupling the equations of motions for each particle.
        
        Works by Chen, Chacón et al. \cite{Chen_Chacón_Barnes_2011, Chacón_Chen_Barnes_2013, Chen_Chacón_2014, Chen_Chacón_2015} have developed structure-preserving particle pushers for neoclassical transport in the Vlasov equations, derived from Crank--Nicolson integrators. We show these too can can derive from a FET interpretation, similarly offering potential extensions to higher-order-in-time particle pushers. The FET formulation is used also to consider how the stochastic drift terms can be incorporated into the pushers. Stochastic gyrokinetic expansions are also discussed.

        Different options for the numerical implementation of these schemes are considered.

        Due to the efficacy of FET in the development of SP timesteppers for both the fluid and kinetic component, we hope this approach will prove effective in the future for developing SP timesteppers for the full hybrid model. We hope this will give us the opportunity to incorporate previously inaccessible kinetic effects into the highly effective, modern, finite-element MHD models.
    \end{abstract}
    
    
    \newpage
    \tableofcontents
    
    
    \newpage
    \pagenumbering{arabic}
    %\linenumbers\renewcommand\thelinenumber{\color{black!50}\arabic{linenumber}}
            \documentclass[12pt, a4paper]{report}

\input{template/main.tex}

\title{\BA{Title in Progress...}}
\author{Boris Andrews}
\affil{Mathematical Institute, University of Oxford}
\date{\today}


\begin{document}
    \pagenumbering{gobble}
    \maketitle
    
    
    \begin{abstract}
        Magnetic confinement reactors---in particular tokamaks---offer one of the most promising options for achieving practical nuclear fusion, with the potential to provide virtually limitless, clean energy. The theoretical and numerical modeling of tokamak plasmas is simultaneously an essential component of effective reactor design, and a great research barrier. Tokamak operational conditions exhibit comparatively low Knudsen numbers. Kinetic effects, including kinetic waves and instabilities, Landau damping, bump-on-tail instabilities and more, are therefore highly influential in tokamak plasma dynamics. Purely fluid models are inherently incapable of capturing these effects, whereas the high dimensionality in purely kinetic models render them practically intractable for most relevant purposes.

        We consider a $\delta\!f$ decomposition model, with a macroscopic fluid background and microscopic kinetic correction, both fully coupled to each other. A similar manner of discretization is proposed to that used in the recent \texttt{STRUPHY} code \cite{Holderied_Possanner_Wang_2021, Holderied_2022, Li_et_al_2023} with a finite-element model for the background and a pseudo-particle/PiC model for the correction.

        The fluid background satisfies the full, non-linear, resistive, compressible, Hall MHD equations. \cite{Laakmann_Hu_Farrell_2022} introduces finite-element(-in-space) implicit timesteppers for the incompressible analogue to this system with structure-preserving (SP) properties in the ideal case, alongside parameter-robust preconditioners. We show that these timesteppers can derive from a finite-element-in-time (FET) (and finite-element-in-space) interpretation. The benefits of this reformulation are discussed, including the derivation of timesteppers that are higher order in time, and the quantifiable dissipative SP properties in the non-ideal, resistive case.
        
        We discuss possible options for extending this FET approach to timesteppers for the compressible case.

        The kinetic corrections satisfy linearized Boltzmann equations. Using a Lénard--Bernstein collision operator, these take Fokker--Planck-like forms \cite{Fokker_1914, Planck_1917} wherein pseudo-particles in the numerical model obey the neoclassical transport equations, with particle-independent Brownian drift terms. This offers a rigorous methodology for incorporating collisions into the particle transport model, without coupling the equations of motions for each particle.
        
        Works by Chen, Chacón et al. \cite{Chen_Chacón_Barnes_2011, Chacón_Chen_Barnes_2013, Chen_Chacón_2014, Chen_Chacón_2015} have developed structure-preserving particle pushers for neoclassical transport in the Vlasov equations, derived from Crank--Nicolson integrators. We show these too can can derive from a FET interpretation, similarly offering potential extensions to higher-order-in-time particle pushers. The FET formulation is used also to consider how the stochastic drift terms can be incorporated into the pushers. Stochastic gyrokinetic expansions are also discussed.

        Different options for the numerical implementation of these schemes are considered.

        Due to the efficacy of FET in the development of SP timesteppers for both the fluid and kinetic component, we hope this approach will prove effective in the future for developing SP timesteppers for the full hybrid model. We hope this will give us the opportunity to incorporate previously inaccessible kinetic effects into the highly effective, modern, finite-element MHD models.
    \end{abstract}
    
    
    \newpage
    \tableofcontents
    
    
    \newpage
    \pagenumbering{arabic}
    %\linenumbers\renewcommand\thelinenumber{\color{black!50}\arabic{linenumber}}
            \input{0 - introduction/main.tex}
        \part{Research}
            \input{1 - low-noise PiC models/main.tex}
            \input{2 - kinetic component/main.tex}
            \input{3 - fluid component/main.tex}
            \input{4 - numerical implementation/main.tex}
        \part{Project Overview}
            \input{5 - research plan/main.tex}
            \input{6 - summary/main.tex}
    
    
    %\section{}
    \newpage
    \pagenumbering{gobble}
        \printbibliography


    \newpage
    \pagenumbering{roman}
    \appendix
        \part{Appendices}
            \input{8 - Hilbert complexes/main.tex}
            \input{9 - weak conservation proofs/main.tex}
\end{document}

        \part{Research}
            \documentclass[12pt, a4paper]{report}

\input{template/main.tex}

\title{\BA{Title in Progress...}}
\author{Boris Andrews}
\affil{Mathematical Institute, University of Oxford}
\date{\today}


\begin{document}
    \pagenumbering{gobble}
    \maketitle
    
    
    \begin{abstract}
        Magnetic confinement reactors---in particular tokamaks---offer one of the most promising options for achieving practical nuclear fusion, with the potential to provide virtually limitless, clean energy. The theoretical and numerical modeling of tokamak plasmas is simultaneously an essential component of effective reactor design, and a great research barrier. Tokamak operational conditions exhibit comparatively low Knudsen numbers. Kinetic effects, including kinetic waves and instabilities, Landau damping, bump-on-tail instabilities and more, are therefore highly influential in tokamak plasma dynamics. Purely fluid models are inherently incapable of capturing these effects, whereas the high dimensionality in purely kinetic models render them practically intractable for most relevant purposes.

        We consider a $\delta\!f$ decomposition model, with a macroscopic fluid background and microscopic kinetic correction, both fully coupled to each other. A similar manner of discretization is proposed to that used in the recent \texttt{STRUPHY} code \cite{Holderied_Possanner_Wang_2021, Holderied_2022, Li_et_al_2023} with a finite-element model for the background and a pseudo-particle/PiC model for the correction.

        The fluid background satisfies the full, non-linear, resistive, compressible, Hall MHD equations. \cite{Laakmann_Hu_Farrell_2022} introduces finite-element(-in-space) implicit timesteppers for the incompressible analogue to this system with structure-preserving (SP) properties in the ideal case, alongside parameter-robust preconditioners. We show that these timesteppers can derive from a finite-element-in-time (FET) (and finite-element-in-space) interpretation. The benefits of this reformulation are discussed, including the derivation of timesteppers that are higher order in time, and the quantifiable dissipative SP properties in the non-ideal, resistive case.
        
        We discuss possible options for extending this FET approach to timesteppers for the compressible case.

        The kinetic corrections satisfy linearized Boltzmann equations. Using a Lénard--Bernstein collision operator, these take Fokker--Planck-like forms \cite{Fokker_1914, Planck_1917} wherein pseudo-particles in the numerical model obey the neoclassical transport equations, with particle-independent Brownian drift terms. This offers a rigorous methodology for incorporating collisions into the particle transport model, without coupling the equations of motions for each particle.
        
        Works by Chen, Chacón et al. \cite{Chen_Chacón_Barnes_2011, Chacón_Chen_Barnes_2013, Chen_Chacón_2014, Chen_Chacón_2015} have developed structure-preserving particle pushers for neoclassical transport in the Vlasov equations, derived from Crank--Nicolson integrators. We show these too can can derive from a FET interpretation, similarly offering potential extensions to higher-order-in-time particle pushers. The FET formulation is used also to consider how the stochastic drift terms can be incorporated into the pushers. Stochastic gyrokinetic expansions are also discussed.

        Different options for the numerical implementation of these schemes are considered.

        Due to the efficacy of FET in the development of SP timesteppers for both the fluid and kinetic component, we hope this approach will prove effective in the future for developing SP timesteppers for the full hybrid model. We hope this will give us the opportunity to incorporate previously inaccessible kinetic effects into the highly effective, modern, finite-element MHD models.
    \end{abstract}
    
    
    \newpage
    \tableofcontents
    
    
    \newpage
    \pagenumbering{arabic}
    %\linenumbers\renewcommand\thelinenumber{\color{black!50}\arabic{linenumber}}
            \input{0 - introduction/main.tex}
        \part{Research}
            \input{1 - low-noise PiC models/main.tex}
            \input{2 - kinetic component/main.tex}
            \input{3 - fluid component/main.tex}
            \input{4 - numerical implementation/main.tex}
        \part{Project Overview}
            \input{5 - research plan/main.tex}
            \input{6 - summary/main.tex}
    
    
    %\section{}
    \newpage
    \pagenumbering{gobble}
        \printbibliography


    \newpage
    \pagenumbering{roman}
    \appendix
        \part{Appendices}
            \input{8 - Hilbert complexes/main.tex}
            \input{9 - weak conservation proofs/main.tex}
\end{document}

            \documentclass[12pt, a4paper]{report}

\input{template/main.tex}

\title{\BA{Title in Progress...}}
\author{Boris Andrews}
\affil{Mathematical Institute, University of Oxford}
\date{\today}


\begin{document}
    \pagenumbering{gobble}
    \maketitle
    
    
    \begin{abstract}
        Magnetic confinement reactors---in particular tokamaks---offer one of the most promising options for achieving practical nuclear fusion, with the potential to provide virtually limitless, clean energy. The theoretical and numerical modeling of tokamak plasmas is simultaneously an essential component of effective reactor design, and a great research barrier. Tokamak operational conditions exhibit comparatively low Knudsen numbers. Kinetic effects, including kinetic waves and instabilities, Landau damping, bump-on-tail instabilities and more, are therefore highly influential in tokamak plasma dynamics. Purely fluid models are inherently incapable of capturing these effects, whereas the high dimensionality in purely kinetic models render them practically intractable for most relevant purposes.

        We consider a $\delta\!f$ decomposition model, with a macroscopic fluid background and microscopic kinetic correction, both fully coupled to each other. A similar manner of discretization is proposed to that used in the recent \texttt{STRUPHY} code \cite{Holderied_Possanner_Wang_2021, Holderied_2022, Li_et_al_2023} with a finite-element model for the background and a pseudo-particle/PiC model for the correction.

        The fluid background satisfies the full, non-linear, resistive, compressible, Hall MHD equations. \cite{Laakmann_Hu_Farrell_2022} introduces finite-element(-in-space) implicit timesteppers for the incompressible analogue to this system with structure-preserving (SP) properties in the ideal case, alongside parameter-robust preconditioners. We show that these timesteppers can derive from a finite-element-in-time (FET) (and finite-element-in-space) interpretation. The benefits of this reformulation are discussed, including the derivation of timesteppers that are higher order in time, and the quantifiable dissipative SP properties in the non-ideal, resistive case.
        
        We discuss possible options for extending this FET approach to timesteppers for the compressible case.

        The kinetic corrections satisfy linearized Boltzmann equations. Using a Lénard--Bernstein collision operator, these take Fokker--Planck-like forms \cite{Fokker_1914, Planck_1917} wherein pseudo-particles in the numerical model obey the neoclassical transport equations, with particle-independent Brownian drift terms. This offers a rigorous methodology for incorporating collisions into the particle transport model, without coupling the equations of motions for each particle.
        
        Works by Chen, Chacón et al. \cite{Chen_Chacón_Barnes_2011, Chacón_Chen_Barnes_2013, Chen_Chacón_2014, Chen_Chacón_2015} have developed structure-preserving particle pushers for neoclassical transport in the Vlasov equations, derived from Crank--Nicolson integrators. We show these too can can derive from a FET interpretation, similarly offering potential extensions to higher-order-in-time particle pushers. The FET formulation is used also to consider how the stochastic drift terms can be incorporated into the pushers. Stochastic gyrokinetic expansions are also discussed.

        Different options for the numerical implementation of these schemes are considered.

        Due to the efficacy of FET in the development of SP timesteppers for both the fluid and kinetic component, we hope this approach will prove effective in the future for developing SP timesteppers for the full hybrid model. We hope this will give us the opportunity to incorporate previously inaccessible kinetic effects into the highly effective, modern, finite-element MHD models.
    \end{abstract}
    
    
    \newpage
    \tableofcontents
    
    
    \newpage
    \pagenumbering{arabic}
    %\linenumbers\renewcommand\thelinenumber{\color{black!50}\arabic{linenumber}}
            \input{0 - introduction/main.tex}
        \part{Research}
            \input{1 - low-noise PiC models/main.tex}
            \input{2 - kinetic component/main.tex}
            \input{3 - fluid component/main.tex}
            \input{4 - numerical implementation/main.tex}
        \part{Project Overview}
            \input{5 - research plan/main.tex}
            \input{6 - summary/main.tex}
    
    
    %\section{}
    \newpage
    \pagenumbering{gobble}
        \printbibliography


    \newpage
    \pagenumbering{roman}
    \appendix
        \part{Appendices}
            \input{8 - Hilbert complexes/main.tex}
            \input{9 - weak conservation proofs/main.tex}
\end{document}

            \documentclass[12pt, a4paper]{report}

\input{template/main.tex}

\title{\BA{Title in Progress...}}
\author{Boris Andrews}
\affil{Mathematical Institute, University of Oxford}
\date{\today}


\begin{document}
    \pagenumbering{gobble}
    \maketitle
    
    
    \begin{abstract}
        Magnetic confinement reactors---in particular tokamaks---offer one of the most promising options for achieving practical nuclear fusion, with the potential to provide virtually limitless, clean energy. The theoretical and numerical modeling of tokamak plasmas is simultaneously an essential component of effective reactor design, and a great research barrier. Tokamak operational conditions exhibit comparatively low Knudsen numbers. Kinetic effects, including kinetic waves and instabilities, Landau damping, bump-on-tail instabilities and more, are therefore highly influential in tokamak plasma dynamics. Purely fluid models are inherently incapable of capturing these effects, whereas the high dimensionality in purely kinetic models render them practically intractable for most relevant purposes.

        We consider a $\delta\!f$ decomposition model, with a macroscopic fluid background and microscopic kinetic correction, both fully coupled to each other. A similar manner of discretization is proposed to that used in the recent \texttt{STRUPHY} code \cite{Holderied_Possanner_Wang_2021, Holderied_2022, Li_et_al_2023} with a finite-element model for the background and a pseudo-particle/PiC model for the correction.

        The fluid background satisfies the full, non-linear, resistive, compressible, Hall MHD equations. \cite{Laakmann_Hu_Farrell_2022} introduces finite-element(-in-space) implicit timesteppers for the incompressible analogue to this system with structure-preserving (SP) properties in the ideal case, alongside parameter-robust preconditioners. We show that these timesteppers can derive from a finite-element-in-time (FET) (and finite-element-in-space) interpretation. The benefits of this reformulation are discussed, including the derivation of timesteppers that are higher order in time, and the quantifiable dissipative SP properties in the non-ideal, resistive case.
        
        We discuss possible options for extending this FET approach to timesteppers for the compressible case.

        The kinetic corrections satisfy linearized Boltzmann equations. Using a Lénard--Bernstein collision operator, these take Fokker--Planck-like forms \cite{Fokker_1914, Planck_1917} wherein pseudo-particles in the numerical model obey the neoclassical transport equations, with particle-independent Brownian drift terms. This offers a rigorous methodology for incorporating collisions into the particle transport model, without coupling the equations of motions for each particle.
        
        Works by Chen, Chacón et al. \cite{Chen_Chacón_Barnes_2011, Chacón_Chen_Barnes_2013, Chen_Chacón_2014, Chen_Chacón_2015} have developed structure-preserving particle pushers for neoclassical transport in the Vlasov equations, derived from Crank--Nicolson integrators. We show these too can can derive from a FET interpretation, similarly offering potential extensions to higher-order-in-time particle pushers. The FET formulation is used also to consider how the stochastic drift terms can be incorporated into the pushers. Stochastic gyrokinetic expansions are also discussed.

        Different options for the numerical implementation of these schemes are considered.

        Due to the efficacy of FET in the development of SP timesteppers for both the fluid and kinetic component, we hope this approach will prove effective in the future for developing SP timesteppers for the full hybrid model. We hope this will give us the opportunity to incorporate previously inaccessible kinetic effects into the highly effective, modern, finite-element MHD models.
    \end{abstract}
    
    
    \newpage
    \tableofcontents
    
    
    \newpage
    \pagenumbering{arabic}
    %\linenumbers\renewcommand\thelinenumber{\color{black!50}\arabic{linenumber}}
            \input{0 - introduction/main.tex}
        \part{Research}
            \input{1 - low-noise PiC models/main.tex}
            \input{2 - kinetic component/main.tex}
            \input{3 - fluid component/main.tex}
            \input{4 - numerical implementation/main.tex}
        \part{Project Overview}
            \input{5 - research plan/main.tex}
            \input{6 - summary/main.tex}
    
    
    %\section{}
    \newpage
    \pagenumbering{gobble}
        \printbibliography


    \newpage
    \pagenumbering{roman}
    \appendix
        \part{Appendices}
            \input{8 - Hilbert complexes/main.tex}
            \input{9 - weak conservation proofs/main.tex}
\end{document}

            \documentclass[12pt, a4paper]{report}

\input{template/main.tex}

\title{\BA{Title in Progress...}}
\author{Boris Andrews}
\affil{Mathematical Institute, University of Oxford}
\date{\today}


\begin{document}
    \pagenumbering{gobble}
    \maketitle
    
    
    \begin{abstract}
        Magnetic confinement reactors---in particular tokamaks---offer one of the most promising options for achieving practical nuclear fusion, with the potential to provide virtually limitless, clean energy. The theoretical and numerical modeling of tokamak plasmas is simultaneously an essential component of effective reactor design, and a great research barrier. Tokamak operational conditions exhibit comparatively low Knudsen numbers. Kinetic effects, including kinetic waves and instabilities, Landau damping, bump-on-tail instabilities and more, are therefore highly influential in tokamak plasma dynamics. Purely fluid models are inherently incapable of capturing these effects, whereas the high dimensionality in purely kinetic models render them practically intractable for most relevant purposes.

        We consider a $\delta\!f$ decomposition model, with a macroscopic fluid background and microscopic kinetic correction, both fully coupled to each other. A similar manner of discretization is proposed to that used in the recent \texttt{STRUPHY} code \cite{Holderied_Possanner_Wang_2021, Holderied_2022, Li_et_al_2023} with a finite-element model for the background and a pseudo-particle/PiC model for the correction.

        The fluid background satisfies the full, non-linear, resistive, compressible, Hall MHD equations. \cite{Laakmann_Hu_Farrell_2022} introduces finite-element(-in-space) implicit timesteppers for the incompressible analogue to this system with structure-preserving (SP) properties in the ideal case, alongside parameter-robust preconditioners. We show that these timesteppers can derive from a finite-element-in-time (FET) (and finite-element-in-space) interpretation. The benefits of this reformulation are discussed, including the derivation of timesteppers that are higher order in time, and the quantifiable dissipative SP properties in the non-ideal, resistive case.
        
        We discuss possible options for extending this FET approach to timesteppers for the compressible case.

        The kinetic corrections satisfy linearized Boltzmann equations. Using a Lénard--Bernstein collision operator, these take Fokker--Planck-like forms \cite{Fokker_1914, Planck_1917} wherein pseudo-particles in the numerical model obey the neoclassical transport equations, with particle-independent Brownian drift terms. This offers a rigorous methodology for incorporating collisions into the particle transport model, without coupling the equations of motions for each particle.
        
        Works by Chen, Chacón et al. \cite{Chen_Chacón_Barnes_2011, Chacón_Chen_Barnes_2013, Chen_Chacón_2014, Chen_Chacón_2015} have developed structure-preserving particle pushers for neoclassical transport in the Vlasov equations, derived from Crank--Nicolson integrators. We show these too can can derive from a FET interpretation, similarly offering potential extensions to higher-order-in-time particle pushers. The FET formulation is used also to consider how the stochastic drift terms can be incorporated into the pushers. Stochastic gyrokinetic expansions are also discussed.

        Different options for the numerical implementation of these schemes are considered.

        Due to the efficacy of FET in the development of SP timesteppers for both the fluid and kinetic component, we hope this approach will prove effective in the future for developing SP timesteppers for the full hybrid model. We hope this will give us the opportunity to incorporate previously inaccessible kinetic effects into the highly effective, modern, finite-element MHD models.
    \end{abstract}
    
    
    \newpage
    \tableofcontents
    
    
    \newpage
    \pagenumbering{arabic}
    %\linenumbers\renewcommand\thelinenumber{\color{black!50}\arabic{linenumber}}
            \input{0 - introduction/main.tex}
        \part{Research}
            \input{1 - low-noise PiC models/main.tex}
            \input{2 - kinetic component/main.tex}
            \input{3 - fluid component/main.tex}
            \input{4 - numerical implementation/main.tex}
        \part{Project Overview}
            \input{5 - research plan/main.tex}
            \input{6 - summary/main.tex}
    
    
    %\section{}
    \newpage
    \pagenumbering{gobble}
        \printbibliography


    \newpage
    \pagenumbering{roman}
    \appendix
        \part{Appendices}
            \input{8 - Hilbert complexes/main.tex}
            \input{9 - weak conservation proofs/main.tex}
\end{document}

        \part{Project Overview}
            \documentclass[12pt, a4paper]{report}

\input{template/main.tex}

\title{\BA{Title in Progress...}}
\author{Boris Andrews}
\affil{Mathematical Institute, University of Oxford}
\date{\today}


\begin{document}
    \pagenumbering{gobble}
    \maketitle
    
    
    \begin{abstract}
        Magnetic confinement reactors---in particular tokamaks---offer one of the most promising options for achieving practical nuclear fusion, with the potential to provide virtually limitless, clean energy. The theoretical and numerical modeling of tokamak plasmas is simultaneously an essential component of effective reactor design, and a great research barrier. Tokamak operational conditions exhibit comparatively low Knudsen numbers. Kinetic effects, including kinetic waves and instabilities, Landau damping, bump-on-tail instabilities and more, are therefore highly influential in tokamak plasma dynamics. Purely fluid models are inherently incapable of capturing these effects, whereas the high dimensionality in purely kinetic models render them practically intractable for most relevant purposes.

        We consider a $\delta\!f$ decomposition model, with a macroscopic fluid background and microscopic kinetic correction, both fully coupled to each other. A similar manner of discretization is proposed to that used in the recent \texttt{STRUPHY} code \cite{Holderied_Possanner_Wang_2021, Holderied_2022, Li_et_al_2023} with a finite-element model for the background and a pseudo-particle/PiC model for the correction.

        The fluid background satisfies the full, non-linear, resistive, compressible, Hall MHD equations. \cite{Laakmann_Hu_Farrell_2022} introduces finite-element(-in-space) implicit timesteppers for the incompressible analogue to this system with structure-preserving (SP) properties in the ideal case, alongside parameter-robust preconditioners. We show that these timesteppers can derive from a finite-element-in-time (FET) (and finite-element-in-space) interpretation. The benefits of this reformulation are discussed, including the derivation of timesteppers that are higher order in time, and the quantifiable dissipative SP properties in the non-ideal, resistive case.
        
        We discuss possible options for extending this FET approach to timesteppers for the compressible case.

        The kinetic corrections satisfy linearized Boltzmann equations. Using a Lénard--Bernstein collision operator, these take Fokker--Planck-like forms \cite{Fokker_1914, Planck_1917} wherein pseudo-particles in the numerical model obey the neoclassical transport equations, with particle-independent Brownian drift terms. This offers a rigorous methodology for incorporating collisions into the particle transport model, without coupling the equations of motions for each particle.
        
        Works by Chen, Chacón et al. \cite{Chen_Chacón_Barnes_2011, Chacón_Chen_Barnes_2013, Chen_Chacón_2014, Chen_Chacón_2015} have developed structure-preserving particle pushers for neoclassical transport in the Vlasov equations, derived from Crank--Nicolson integrators. We show these too can can derive from a FET interpretation, similarly offering potential extensions to higher-order-in-time particle pushers. The FET formulation is used also to consider how the stochastic drift terms can be incorporated into the pushers. Stochastic gyrokinetic expansions are also discussed.

        Different options for the numerical implementation of these schemes are considered.

        Due to the efficacy of FET in the development of SP timesteppers for both the fluid and kinetic component, we hope this approach will prove effective in the future for developing SP timesteppers for the full hybrid model. We hope this will give us the opportunity to incorporate previously inaccessible kinetic effects into the highly effective, modern, finite-element MHD models.
    \end{abstract}
    
    
    \newpage
    \tableofcontents
    
    
    \newpage
    \pagenumbering{arabic}
    %\linenumbers\renewcommand\thelinenumber{\color{black!50}\arabic{linenumber}}
            \input{0 - introduction/main.tex}
        \part{Research}
            \input{1 - low-noise PiC models/main.tex}
            \input{2 - kinetic component/main.tex}
            \input{3 - fluid component/main.tex}
            \input{4 - numerical implementation/main.tex}
        \part{Project Overview}
            \input{5 - research plan/main.tex}
            \input{6 - summary/main.tex}
    
    
    %\section{}
    \newpage
    \pagenumbering{gobble}
        \printbibliography


    \newpage
    \pagenumbering{roman}
    \appendix
        \part{Appendices}
            \input{8 - Hilbert complexes/main.tex}
            \input{9 - weak conservation proofs/main.tex}
\end{document}

            \documentclass[12pt, a4paper]{report}

\input{template/main.tex}

\title{\BA{Title in Progress...}}
\author{Boris Andrews}
\affil{Mathematical Institute, University of Oxford}
\date{\today}


\begin{document}
    \pagenumbering{gobble}
    \maketitle
    
    
    \begin{abstract}
        Magnetic confinement reactors---in particular tokamaks---offer one of the most promising options for achieving practical nuclear fusion, with the potential to provide virtually limitless, clean energy. The theoretical and numerical modeling of tokamak plasmas is simultaneously an essential component of effective reactor design, and a great research barrier. Tokamak operational conditions exhibit comparatively low Knudsen numbers. Kinetic effects, including kinetic waves and instabilities, Landau damping, bump-on-tail instabilities and more, are therefore highly influential in tokamak plasma dynamics. Purely fluid models are inherently incapable of capturing these effects, whereas the high dimensionality in purely kinetic models render them practically intractable for most relevant purposes.

        We consider a $\delta\!f$ decomposition model, with a macroscopic fluid background and microscopic kinetic correction, both fully coupled to each other. A similar manner of discretization is proposed to that used in the recent \texttt{STRUPHY} code \cite{Holderied_Possanner_Wang_2021, Holderied_2022, Li_et_al_2023} with a finite-element model for the background and a pseudo-particle/PiC model for the correction.

        The fluid background satisfies the full, non-linear, resistive, compressible, Hall MHD equations. \cite{Laakmann_Hu_Farrell_2022} introduces finite-element(-in-space) implicit timesteppers for the incompressible analogue to this system with structure-preserving (SP) properties in the ideal case, alongside parameter-robust preconditioners. We show that these timesteppers can derive from a finite-element-in-time (FET) (and finite-element-in-space) interpretation. The benefits of this reformulation are discussed, including the derivation of timesteppers that are higher order in time, and the quantifiable dissipative SP properties in the non-ideal, resistive case.
        
        We discuss possible options for extending this FET approach to timesteppers for the compressible case.

        The kinetic corrections satisfy linearized Boltzmann equations. Using a Lénard--Bernstein collision operator, these take Fokker--Planck-like forms \cite{Fokker_1914, Planck_1917} wherein pseudo-particles in the numerical model obey the neoclassical transport equations, with particle-independent Brownian drift terms. This offers a rigorous methodology for incorporating collisions into the particle transport model, without coupling the equations of motions for each particle.
        
        Works by Chen, Chacón et al. \cite{Chen_Chacón_Barnes_2011, Chacón_Chen_Barnes_2013, Chen_Chacón_2014, Chen_Chacón_2015} have developed structure-preserving particle pushers for neoclassical transport in the Vlasov equations, derived from Crank--Nicolson integrators. We show these too can can derive from a FET interpretation, similarly offering potential extensions to higher-order-in-time particle pushers. The FET formulation is used also to consider how the stochastic drift terms can be incorporated into the pushers. Stochastic gyrokinetic expansions are also discussed.

        Different options for the numerical implementation of these schemes are considered.

        Due to the efficacy of FET in the development of SP timesteppers for both the fluid and kinetic component, we hope this approach will prove effective in the future for developing SP timesteppers for the full hybrid model. We hope this will give us the opportunity to incorporate previously inaccessible kinetic effects into the highly effective, modern, finite-element MHD models.
    \end{abstract}
    
    
    \newpage
    \tableofcontents
    
    
    \newpage
    \pagenumbering{arabic}
    %\linenumbers\renewcommand\thelinenumber{\color{black!50}\arabic{linenumber}}
            \input{0 - introduction/main.tex}
        \part{Research}
            \input{1 - low-noise PiC models/main.tex}
            \input{2 - kinetic component/main.tex}
            \input{3 - fluid component/main.tex}
            \input{4 - numerical implementation/main.tex}
        \part{Project Overview}
            \input{5 - research plan/main.tex}
            \input{6 - summary/main.tex}
    
    
    %\section{}
    \newpage
    \pagenumbering{gobble}
        \printbibliography


    \newpage
    \pagenumbering{roman}
    \appendix
        \part{Appendices}
            \input{8 - Hilbert complexes/main.tex}
            \input{9 - weak conservation proofs/main.tex}
\end{document}

    
    
    %\section{}
    \newpage
    \pagenumbering{gobble}
        \printbibliography


    \newpage
    \pagenumbering{roman}
    \appendix
        \part{Appendices}
            \documentclass[12pt, a4paper]{report}

\input{template/main.tex}

\title{\BA{Title in Progress...}}
\author{Boris Andrews}
\affil{Mathematical Institute, University of Oxford}
\date{\today}


\begin{document}
    \pagenumbering{gobble}
    \maketitle
    
    
    \begin{abstract}
        Magnetic confinement reactors---in particular tokamaks---offer one of the most promising options for achieving practical nuclear fusion, with the potential to provide virtually limitless, clean energy. The theoretical and numerical modeling of tokamak plasmas is simultaneously an essential component of effective reactor design, and a great research barrier. Tokamak operational conditions exhibit comparatively low Knudsen numbers. Kinetic effects, including kinetic waves and instabilities, Landau damping, bump-on-tail instabilities and more, are therefore highly influential in tokamak plasma dynamics. Purely fluid models are inherently incapable of capturing these effects, whereas the high dimensionality in purely kinetic models render them practically intractable for most relevant purposes.

        We consider a $\delta\!f$ decomposition model, with a macroscopic fluid background and microscopic kinetic correction, both fully coupled to each other. A similar manner of discretization is proposed to that used in the recent \texttt{STRUPHY} code \cite{Holderied_Possanner_Wang_2021, Holderied_2022, Li_et_al_2023} with a finite-element model for the background and a pseudo-particle/PiC model for the correction.

        The fluid background satisfies the full, non-linear, resistive, compressible, Hall MHD equations. \cite{Laakmann_Hu_Farrell_2022} introduces finite-element(-in-space) implicit timesteppers for the incompressible analogue to this system with structure-preserving (SP) properties in the ideal case, alongside parameter-robust preconditioners. We show that these timesteppers can derive from a finite-element-in-time (FET) (and finite-element-in-space) interpretation. The benefits of this reformulation are discussed, including the derivation of timesteppers that are higher order in time, and the quantifiable dissipative SP properties in the non-ideal, resistive case.
        
        We discuss possible options for extending this FET approach to timesteppers for the compressible case.

        The kinetic corrections satisfy linearized Boltzmann equations. Using a Lénard--Bernstein collision operator, these take Fokker--Planck-like forms \cite{Fokker_1914, Planck_1917} wherein pseudo-particles in the numerical model obey the neoclassical transport equations, with particle-independent Brownian drift terms. This offers a rigorous methodology for incorporating collisions into the particle transport model, without coupling the equations of motions for each particle.
        
        Works by Chen, Chacón et al. \cite{Chen_Chacón_Barnes_2011, Chacón_Chen_Barnes_2013, Chen_Chacón_2014, Chen_Chacón_2015} have developed structure-preserving particle pushers for neoclassical transport in the Vlasov equations, derived from Crank--Nicolson integrators. We show these too can can derive from a FET interpretation, similarly offering potential extensions to higher-order-in-time particle pushers. The FET formulation is used also to consider how the stochastic drift terms can be incorporated into the pushers. Stochastic gyrokinetic expansions are also discussed.

        Different options for the numerical implementation of these schemes are considered.

        Due to the efficacy of FET in the development of SP timesteppers for both the fluid and kinetic component, we hope this approach will prove effective in the future for developing SP timesteppers for the full hybrid model. We hope this will give us the opportunity to incorporate previously inaccessible kinetic effects into the highly effective, modern, finite-element MHD models.
    \end{abstract}
    
    
    \newpage
    \tableofcontents
    
    
    \newpage
    \pagenumbering{arabic}
    %\linenumbers\renewcommand\thelinenumber{\color{black!50}\arabic{linenumber}}
            \input{0 - introduction/main.tex}
        \part{Research}
            \input{1 - low-noise PiC models/main.tex}
            \input{2 - kinetic component/main.tex}
            \input{3 - fluid component/main.tex}
            \input{4 - numerical implementation/main.tex}
        \part{Project Overview}
            \input{5 - research plan/main.tex}
            \input{6 - summary/main.tex}
    
    
    %\section{}
    \newpage
    \pagenumbering{gobble}
        \printbibliography


    \newpage
    \pagenumbering{roman}
    \appendix
        \part{Appendices}
            \input{8 - Hilbert complexes/main.tex}
            \input{9 - weak conservation proofs/main.tex}
\end{document}

            \documentclass[12pt, a4paper]{report}

\input{template/main.tex}

\title{\BA{Title in Progress...}}
\author{Boris Andrews}
\affil{Mathematical Institute, University of Oxford}
\date{\today}


\begin{document}
    \pagenumbering{gobble}
    \maketitle
    
    
    \begin{abstract}
        Magnetic confinement reactors---in particular tokamaks---offer one of the most promising options for achieving practical nuclear fusion, with the potential to provide virtually limitless, clean energy. The theoretical and numerical modeling of tokamak plasmas is simultaneously an essential component of effective reactor design, and a great research barrier. Tokamak operational conditions exhibit comparatively low Knudsen numbers. Kinetic effects, including kinetic waves and instabilities, Landau damping, bump-on-tail instabilities and more, are therefore highly influential in tokamak plasma dynamics. Purely fluid models are inherently incapable of capturing these effects, whereas the high dimensionality in purely kinetic models render them practically intractable for most relevant purposes.

        We consider a $\delta\!f$ decomposition model, with a macroscopic fluid background and microscopic kinetic correction, both fully coupled to each other. A similar manner of discretization is proposed to that used in the recent \texttt{STRUPHY} code \cite{Holderied_Possanner_Wang_2021, Holderied_2022, Li_et_al_2023} with a finite-element model for the background and a pseudo-particle/PiC model for the correction.

        The fluid background satisfies the full, non-linear, resistive, compressible, Hall MHD equations. \cite{Laakmann_Hu_Farrell_2022} introduces finite-element(-in-space) implicit timesteppers for the incompressible analogue to this system with structure-preserving (SP) properties in the ideal case, alongside parameter-robust preconditioners. We show that these timesteppers can derive from a finite-element-in-time (FET) (and finite-element-in-space) interpretation. The benefits of this reformulation are discussed, including the derivation of timesteppers that are higher order in time, and the quantifiable dissipative SP properties in the non-ideal, resistive case.
        
        We discuss possible options for extending this FET approach to timesteppers for the compressible case.

        The kinetic corrections satisfy linearized Boltzmann equations. Using a Lénard--Bernstein collision operator, these take Fokker--Planck-like forms \cite{Fokker_1914, Planck_1917} wherein pseudo-particles in the numerical model obey the neoclassical transport equations, with particle-independent Brownian drift terms. This offers a rigorous methodology for incorporating collisions into the particle transport model, without coupling the equations of motions for each particle.
        
        Works by Chen, Chacón et al. \cite{Chen_Chacón_Barnes_2011, Chacón_Chen_Barnes_2013, Chen_Chacón_2014, Chen_Chacón_2015} have developed structure-preserving particle pushers for neoclassical transport in the Vlasov equations, derived from Crank--Nicolson integrators. We show these too can can derive from a FET interpretation, similarly offering potential extensions to higher-order-in-time particle pushers. The FET formulation is used also to consider how the stochastic drift terms can be incorporated into the pushers. Stochastic gyrokinetic expansions are also discussed.

        Different options for the numerical implementation of these schemes are considered.

        Due to the efficacy of FET in the development of SP timesteppers for both the fluid and kinetic component, we hope this approach will prove effective in the future for developing SP timesteppers for the full hybrid model. We hope this will give us the opportunity to incorporate previously inaccessible kinetic effects into the highly effective, modern, finite-element MHD models.
    \end{abstract}
    
    
    \newpage
    \tableofcontents
    
    
    \newpage
    \pagenumbering{arabic}
    %\linenumbers\renewcommand\thelinenumber{\color{black!50}\arabic{linenumber}}
            \input{0 - introduction/main.tex}
        \part{Research}
            \input{1 - low-noise PiC models/main.tex}
            \input{2 - kinetic component/main.tex}
            \input{3 - fluid component/main.tex}
            \input{4 - numerical implementation/main.tex}
        \part{Project Overview}
            \input{5 - research plan/main.tex}
            \input{6 - summary/main.tex}
    
    
    %\section{}
    \newpage
    \pagenumbering{gobble}
        \printbibliography


    \newpage
    \pagenumbering{roman}
    \appendix
        \part{Appendices}
            \input{8 - Hilbert complexes/main.tex}
            \input{9 - weak conservation proofs/main.tex}
\end{document}

\end{document}

            \documentclass[12pt, a4paper]{report}

\documentclass[12pt, a4paper]{report}

\input{template/main.tex}

\title{\BA{Title in Progress...}}
\author{Boris Andrews}
\affil{Mathematical Institute, University of Oxford}
\date{\today}


\begin{document}
    \pagenumbering{gobble}
    \maketitle
    
    
    \begin{abstract}
        Magnetic confinement reactors---in particular tokamaks---offer one of the most promising options for achieving practical nuclear fusion, with the potential to provide virtually limitless, clean energy. The theoretical and numerical modeling of tokamak plasmas is simultaneously an essential component of effective reactor design, and a great research barrier. Tokamak operational conditions exhibit comparatively low Knudsen numbers. Kinetic effects, including kinetic waves and instabilities, Landau damping, bump-on-tail instabilities and more, are therefore highly influential in tokamak plasma dynamics. Purely fluid models are inherently incapable of capturing these effects, whereas the high dimensionality in purely kinetic models render them practically intractable for most relevant purposes.

        We consider a $\delta\!f$ decomposition model, with a macroscopic fluid background and microscopic kinetic correction, both fully coupled to each other. A similar manner of discretization is proposed to that used in the recent \texttt{STRUPHY} code \cite{Holderied_Possanner_Wang_2021, Holderied_2022, Li_et_al_2023} with a finite-element model for the background and a pseudo-particle/PiC model for the correction.

        The fluid background satisfies the full, non-linear, resistive, compressible, Hall MHD equations. \cite{Laakmann_Hu_Farrell_2022} introduces finite-element(-in-space) implicit timesteppers for the incompressible analogue to this system with structure-preserving (SP) properties in the ideal case, alongside parameter-robust preconditioners. We show that these timesteppers can derive from a finite-element-in-time (FET) (and finite-element-in-space) interpretation. The benefits of this reformulation are discussed, including the derivation of timesteppers that are higher order in time, and the quantifiable dissipative SP properties in the non-ideal, resistive case.
        
        We discuss possible options for extending this FET approach to timesteppers for the compressible case.

        The kinetic corrections satisfy linearized Boltzmann equations. Using a Lénard--Bernstein collision operator, these take Fokker--Planck-like forms \cite{Fokker_1914, Planck_1917} wherein pseudo-particles in the numerical model obey the neoclassical transport equations, with particle-independent Brownian drift terms. This offers a rigorous methodology for incorporating collisions into the particle transport model, without coupling the equations of motions for each particle.
        
        Works by Chen, Chacón et al. \cite{Chen_Chacón_Barnes_2011, Chacón_Chen_Barnes_2013, Chen_Chacón_2014, Chen_Chacón_2015} have developed structure-preserving particle pushers for neoclassical transport in the Vlasov equations, derived from Crank--Nicolson integrators. We show these too can can derive from a FET interpretation, similarly offering potential extensions to higher-order-in-time particle pushers. The FET formulation is used also to consider how the stochastic drift terms can be incorporated into the pushers. Stochastic gyrokinetic expansions are also discussed.

        Different options for the numerical implementation of these schemes are considered.

        Due to the efficacy of FET in the development of SP timesteppers for both the fluid and kinetic component, we hope this approach will prove effective in the future for developing SP timesteppers for the full hybrid model. We hope this will give us the opportunity to incorporate previously inaccessible kinetic effects into the highly effective, modern, finite-element MHD models.
    \end{abstract}
    
    
    \newpage
    \tableofcontents
    
    
    \newpage
    \pagenumbering{arabic}
    %\linenumbers\renewcommand\thelinenumber{\color{black!50}\arabic{linenumber}}
            \input{0 - introduction/main.tex}
        \part{Research}
            \input{1 - low-noise PiC models/main.tex}
            \input{2 - kinetic component/main.tex}
            \input{3 - fluid component/main.tex}
            \input{4 - numerical implementation/main.tex}
        \part{Project Overview}
            \input{5 - research plan/main.tex}
            \input{6 - summary/main.tex}
    
    
    %\section{}
    \newpage
    \pagenumbering{gobble}
        \printbibliography


    \newpage
    \pagenumbering{roman}
    \appendix
        \part{Appendices}
            \input{8 - Hilbert complexes/main.tex}
            \input{9 - weak conservation proofs/main.tex}
\end{document}


\title{\BA{Title in Progress...}}
\author{Boris Andrews}
\affil{Mathematical Institute, University of Oxford}
\date{\today}


\begin{document}
    \pagenumbering{gobble}
    \maketitle
    
    
    \begin{abstract}
        Magnetic confinement reactors---in particular tokamaks---offer one of the most promising options for achieving practical nuclear fusion, with the potential to provide virtually limitless, clean energy. The theoretical and numerical modeling of tokamak plasmas is simultaneously an essential component of effective reactor design, and a great research barrier. Tokamak operational conditions exhibit comparatively low Knudsen numbers. Kinetic effects, including kinetic waves and instabilities, Landau damping, bump-on-tail instabilities and more, are therefore highly influential in tokamak plasma dynamics. Purely fluid models are inherently incapable of capturing these effects, whereas the high dimensionality in purely kinetic models render them practically intractable for most relevant purposes.

        We consider a $\delta\!f$ decomposition model, with a macroscopic fluid background and microscopic kinetic correction, both fully coupled to each other. A similar manner of discretization is proposed to that used in the recent \texttt{STRUPHY} code \cite{Holderied_Possanner_Wang_2021, Holderied_2022, Li_et_al_2023} with a finite-element model for the background and a pseudo-particle/PiC model for the correction.

        The fluid background satisfies the full, non-linear, resistive, compressible, Hall MHD equations. \cite{Laakmann_Hu_Farrell_2022} introduces finite-element(-in-space) implicit timesteppers for the incompressible analogue to this system with structure-preserving (SP) properties in the ideal case, alongside parameter-robust preconditioners. We show that these timesteppers can derive from a finite-element-in-time (FET) (and finite-element-in-space) interpretation. The benefits of this reformulation are discussed, including the derivation of timesteppers that are higher order in time, and the quantifiable dissipative SP properties in the non-ideal, resistive case.
        
        We discuss possible options for extending this FET approach to timesteppers for the compressible case.

        The kinetic corrections satisfy linearized Boltzmann equations. Using a Lénard--Bernstein collision operator, these take Fokker--Planck-like forms \cite{Fokker_1914, Planck_1917} wherein pseudo-particles in the numerical model obey the neoclassical transport equations, with particle-independent Brownian drift terms. This offers a rigorous methodology for incorporating collisions into the particle transport model, without coupling the equations of motions for each particle.
        
        Works by Chen, Chacón et al. \cite{Chen_Chacón_Barnes_2011, Chacón_Chen_Barnes_2013, Chen_Chacón_2014, Chen_Chacón_2015} have developed structure-preserving particle pushers for neoclassical transport in the Vlasov equations, derived from Crank--Nicolson integrators. We show these too can can derive from a FET interpretation, similarly offering potential extensions to higher-order-in-time particle pushers. The FET formulation is used also to consider how the stochastic drift terms can be incorporated into the pushers. Stochastic gyrokinetic expansions are also discussed.

        Different options for the numerical implementation of these schemes are considered.

        Due to the efficacy of FET in the development of SP timesteppers for both the fluid and kinetic component, we hope this approach will prove effective in the future for developing SP timesteppers for the full hybrid model. We hope this will give us the opportunity to incorporate previously inaccessible kinetic effects into the highly effective, modern, finite-element MHD models.
    \end{abstract}
    
    
    \newpage
    \tableofcontents
    
    
    \newpage
    \pagenumbering{arabic}
    %\linenumbers\renewcommand\thelinenumber{\color{black!50}\arabic{linenumber}}
            \documentclass[12pt, a4paper]{report}

\input{template/main.tex}

\title{\BA{Title in Progress...}}
\author{Boris Andrews}
\affil{Mathematical Institute, University of Oxford}
\date{\today}


\begin{document}
    \pagenumbering{gobble}
    \maketitle
    
    
    \begin{abstract}
        Magnetic confinement reactors---in particular tokamaks---offer one of the most promising options for achieving practical nuclear fusion, with the potential to provide virtually limitless, clean energy. The theoretical and numerical modeling of tokamak plasmas is simultaneously an essential component of effective reactor design, and a great research barrier. Tokamak operational conditions exhibit comparatively low Knudsen numbers. Kinetic effects, including kinetic waves and instabilities, Landau damping, bump-on-tail instabilities and more, are therefore highly influential in tokamak plasma dynamics. Purely fluid models are inherently incapable of capturing these effects, whereas the high dimensionality in purely kinetic models render them practically intractable for most relevant purposes.

        We consider a $\delta\!f$ decomposition model, with a macroscopic fluid background and microscopic kinetic correction, both fully coupled to each other. A similar manner of discretization is proposed to that used in the recent \texttt{STRUPHY} code \cite{Holderied_Possanner_Wang_2021, Holderied_2022, Li_et_al_2023} with a finite-element model for the background and a pseudo-particle/PiC model for the correction.

        The fluid background satisfies the full, non-linear, resistive, compressible, Hall MHD equations. \cite{Laakmann_Hu_Farrell_2022} introduces finite-element(-in-space) implicit timesteppers for the incompressible analogue to this system with structure-preserving (SP) properties in the ideal case, alongside parameter-robust preconditioners. We show that these timesteppers can derive from a finite-element-in-time (FET) (and finite-element-in-space) interpretation. The benefits of this reformulation are discussed, including the derivation of timesteppers that are higher order in time, and the quantifiable dissipative SP properties in the non-ideal, resistive case.
        
        We discuss possible options for extending this FET approach to timesteppers for the compressible case.

        The kinetic corrections satisfy linearized Boltzmann equations. Using a Lénard--Bernstein collision operator, these take Fokker--Planck-like forms \cite{Fokker_1914, Planck_1917} wherein pseudo-particles in the numerical model obey the neoclassical transport equations, with particle-independent Brownian drift terms. This offers a rigorous methodology for incorporating collisions into the particle transport model, without coupling the equations of motions for each particle.
        
        Works by Chen, Chacón et al. \cite{Chen_Chacón_Barnes_2011, Chacón_Chen_Barnes_2013, Chen_Chacón_2014, Chen_Chacón_2015} have developed structure-preserving particle pushers for neoclassical transport in the Vlasov equations, derived from Crank--Nicolson integrators. We show these too can can derive from a FET interpretation, similarly offering potential extensions to higher-order-in-time particle pushers. The FET formulation is used also to consider how the stochastic drift terms can be incorporated into the pushers. Stochastic gyrokinetic expansions are also discussed.

        Different options for the numerical implementation of these schemes are considered.

        Due to the efficacy of FET in the development of SP timesteppers for both the fluid and kinetic component, we hope this approach will prove effective in the future for developing SP timesteppers for the full hybrid model. We hope this will give us the opportunity to incorporate previously inaccessible kinetic effects into the highly effective, modern, finite-element MHD models.
    \end{abstract}
    
    
    \newpage
    \tableofcontents
    
    
    \newpage
    \pagenumbering{arabic}
    %\linenumbers\renewcommand\thelinenumber{\color{black!50}\arabic{linenumber}}
            \input{0 - introduction/main.tex}
        \part{Research}
            \input{1 - low-noise PiC models/main.tex}
            \input{2 - kinetic component/main.tex}
            \input{3 - fluid component/main.tex}
            \input{4 - numerical implementation/main.tex}
        \part{Project Overview}
            \input{5 - research plan/main.tex}
            \input{6 - summary/main.tex}
    
    
    %\section{}
    \newpage
    \pagenumbering{gobble}
        \printbibliography


    \newpage
    \pagenumbering{roman}
    \appendix
        \part{Appendices}
            \input{8 - Hilbert complexes/main.tex}
            \input{9 - weak conservation proofs/main.tex}
\end{document}

        \part{Research}
            \documentclass[12pt, a4paper]{report}

\input{template/main.tex}

\title{\BA{Title in Progress...}}
\author{Boris Andrews}
\affil{Mathematical Institute, University of Oxford}
\date{\today}


\begin{document}
    \pagenumbering{gobble}
    \maketitle
    
    
    \begin{abstract}
        Magnetic confinement reactors---in particular tokamaks---offer one of the most promising options for achieving practical nuclear fusion, with the potential to provide virtually limitless, clean energy. The theoretical and numerical modeling of tokamak plasmas is simultaneously an essential component of effective reactor design, and a great research barrier. Tokamak operational conditions exhibit comparatively low Knudsen numbers. Kinetic effects, including kinetic waves and instabilities, Landau damping, bump-on-tail instabilities and more, are therefore highly influential in tokamak plasma dynamics. Purely fluid models are inherently incapable of capturing these effects, whereas the high dimensionality in purely kinetic models render them practically intractable for most relevant purposes.

        We consider a $\delta\!f$ decomposition model, with a macroscopic fluid background and microscopic kinetic correction, both fully coupled to each other. A similar manner of discretization is proposed to that used in the recent \texttt{STRUPHY} code \cite{Holderied_Possanner_Wang_2021, Holderied_2022, Li_et_al_2023} with a finite-element model for the background and a pseudo-particle/PiC model for the correction.

        The fluid background satisfies the full, non-linear, resistive, compressible, Hall MHD equations. \cite{Laakmann_Hu_Farrell_2022} introduces finite-element(-in-space) implicit timesteppers for the incompressible analogue to this system with structure-preserving (SP) properties in the ideal case, alongside parameter-robust preconditioners. We show that these timesteppers can derive from a finite-element-in-time (FET) (and finite-element-in-space) interpretation. The benefits of this reformulation are discussed, including the derivation of timesteppers that are higher order in time, and the quantifiable dissipative SP properties in the non-ideal, resistive case.
        
        We discuss possible options for extending this FET approach to timesteppers for the compressible case.

        The kinetic corrections satisfy linearized Boltzmann equations. Using a Lénard--Bernstein collision operator, these take Fokker--Planck-like forms \cite{Fokker_1914, Planck_1917} wherein pseudo-particles in the numerical model obey the neoclassical transport equations, with particle-independent Brownian drift terms. This offers a rigorous methodology for incorporating collisions into the particle transport model, without coupling the equations of motions for each particle.
        
        Works by Chen, Chacón et al. \cite{Chen_Chacón_Barnes_2011, Chacón_Chen_Barnes_2013, Chen_Chacón_2014, Chen_Chacón_2015} have developed structure-preserving particle pushers for neoclassical transport in the Vlasov equations, derived from Crank--Nicolson integrators. We show these too can can derive from a FET interpretation, similarly offering potential extensions to higher-order-in-time particle pushers. The FET formulation is used also to consider how the stochastic drift terms can be incorporated into the pushers. Stochastic gyrokinetic expansions are also discussed.

        Different options for the numerical implementation of these schemes are considered.

        Due to the efficacy of FET in the development of SP timesteppers for both the fluid and kinetic component, we hope this approach will prove effective in the future for developing SP timesteppers for the full hybrid model. We hope this will give us the opportunity to incorporate previously inaccessible kinetic effects into the highly effective, modern, finite-element MHD models.
    \end{abstract}
    
    
    \newpage
    \tableofcontents
    
    
    \newpage
    \pagenumbering{arabic}
    %\linenumbers\renewcommand\thelinenumber{\color{black!50}\arabic{linenumber}}
            \input{0 - introduction/main.tex}
        \part{Research}
            \input{1 - low-noise PiC models/main.tex}
            \input{2 - kinetic component/main.tex}
            \input{3 - fluid component/main.tex}
            \input{4 - numerical implementation/main.tex}
        \part{Project Overview}
            \input{5 - research plan/main.tex}
            \input{6 - summary/main.tex}
    
    
    %\section{}
    \newpage
    \pagenumbering{gobble}
        \printbibliography


    \newpage
    \pagenumbering{roman}
    \appendix
        \part{Appendices}
            \input{8 - Hilbert complexes/main.tex}
            \input{9 - weak conservation proofs/main.tex}
\end{document}

            \documentclass[12pt, a4paper]{report}

\input{template/main.tex}

\title{\BA{Title in Progress...}}
\author{Boris Andrews}
\affil{Mathematical Institute, University of Oxford}
\date{\today}


\begin{document}
    \pagenumbering{gobble}
    \maketitle
    
    
    \begin{abstract}
        Magnetic confinement reactors---in particular tokamaks---offer one of the most promising options for achieving practical nuclear fusion, with the potential to provide virtually limitless, clean energy. The theoretical and numerical modeling of tokamak plasmas is simultaneously an essential component of effective reactor design, and a great research barrier. Tokamak operational conditions exhibit comparatively low Knudsen numbers. Kinetic effects, including kinetic waves and instabilities, Landau damping, bump-on-tail instabilities and more, are therefore highly influential in tokamak plasma dynamics. Purely fluid models are inherently incapable of capturing these effects, whereas the high dimensionality in purely kinetic models render them practically intractable for most relevant purposes.

        We consider a $\delta\!f$ decomposition model, with a macroscopic fluid background and microscopic kinetic correction, both fully coupled to each other. A similar manner of discretization is proposed to that used in the recent \texttt{STRUPHY} code \cite{Holderied_Possanner_Wang_2021, Holderied_2022, Li_et_al_2023} with a finite-element model for the background and a pseudo-particle/PiC model for the correction.

        The fluid background satisfies the full, non-linear, resistive, compressible, Hall MHD equations. \cite{Laakmann_Hu_Farrell_2022} introduces finite-element(-in-space) implicit timesteppers for the incompressible analogue to this system with structure-preserving (SP) properties in the ideal case, alongside parameter-robust preconditioners. We show that these timesteppers can derive from a finite-element-in-time (FET) (and finite-element-in-space) interpretation. The benefits of this reformulation are discussed, including the derivation of timesteppers that are higher order in time, and the quantifiable dissipative SP properties in the non-ideal, resistive case.
        
        We discuss possible options for extending this FET approach to timesteppers for the compressible case.

        The kinetic corrections satisfy linearized Boltzmann equations. Using a Lénard--Bernstein collision operator, these take Fokker--Planck-like forms \cite{Fokker_1914, Planck_1917} wherein pseudo-particles in the numerical model obey the neoclassical transport equations, with particle-independent Brownian drift terms. This offers a rigorous methodology for incorporating collisions into the particle transport model, without coupling the equations of motions for each particle.
        
        Works by Chen, Chacón et al. \cite{Chen_Chacón_Barnes_2011, Chacón_Chen_Barnes_2013, Chen_Chacón_2014, Chen_Chacón_2015} have developed structure-preserving particle pushers for neoclassical transport in the Vlasov equations, derived from Crank--Nicolson integrators. We show these too can can derive from a FET interpretation, similarly offering potential extensions to higher-order-in-time particle pushers. The FET formulation is used also to consider how the stochastic drift terms can be incorporated into the pushers. Stochastic gyrokinetic expansions are also discussed.

        Different options for the numerical implementation of these schemes are considered.

        Due to the efficacy of FET in the development of SP timesteppers for both the fluid and kinetic component, we hope this approach will prove effective in the future for developing SP timesteppers for the full hybrid model. We hope this will give us the opportunity to incorporate previously inaccessible kinetic effects into the highly effective, modern, finite-element MHD models.
    \end{abstract}
    
    
    \newpage
    \tableofcontents
    
    
    \newpage
    \pagenumbering{arabic}
    %\linenumbers\renewcommand\thelinenumber{\color{black!50}\arabic{linenumber}}
            \input{0 - introduction/main.tex}
        \part{Research}
            \input{1 - low-noise PiC models/main.tex}
            \input{2 - kinetic component/main.tex}
            \input{3 - fluid component/main.tex}
            \input{4 - numerical implementation/main.tex}
        \part{Project Overview}
            \input{5 - research plan/main.tex}
            \input{6 - summary/main.tex}
    
    
    %\section{}
    \newpage
    \pagenumbering{gobble}
        \printbibliography


    \newpage
    \pagenumbering{roman}
    \appendix
        \part{Appendices}
            \input{8 - Hilbert complexes/main.tex}
            \input{9 - weak conservation proofs/main.tex}
\end{document}

            \documentclass[12pt, a4paper]{report}

\input{template/main.tex}

\title{\BA{Title in Progress...}}
\author{Boris Andrews}
\affil{Mathematical Institute, University of Oxford}
\date{\today}


\begin{document}
    \pagenumbering{gobble}
    \maketitle
    
    
    \begin{abstract}
        Magnetic confinement reactors---in particular tokamaks---offer one of the most promising options for achieving practical nuclear fusion, with the potential to provide virtually limitless, clean energy. The theoretical and numerical modeling of tokamak plasmas is simultaneously an essential component of effective reactor design, and a great research barrier. Tokamak operational conditions exhibit comparatively low Knudsen numbers. Kinetic effects, including kinetic waves and instabilities, Landau damping, bump-on-tail instabilities and more, are therefore highly influential in tokamak plasma dynamics. Purely fluid models are inherently incapable of capturing these effects, whereas the high dimensionality in purely kinetic models render them practically intractable for most relevant purposes.

        We consider a $\delta\!f$ decomposition model, with a macroscopic fluid background and microscopic kinetic correction, both fully coupled to each other. A similar manner of discretization is proposed to that used in the recent \texttt{STRUPHY} code \cite{Holderied_Possanner_Wang_2021, Holderied_2022, Li_et_al_2023} with a finite-element model for the background and a pseudo-particle/PiC model for the correction.

        The fluid background satisfies the full, non-linear, resistive, compressible, Hall MHD equations. \cite{Laakmann_Hu_Farrell_2022} introduces finite-element(-in-space) implicit timesteppers for the incompressible analogue to this system with structure-preserving (SP) properties in the ideal case, alongside parameter-robust preconditioners. We show that these timesteppers can derive from a finite-element-in-time (FET) (and finite-element-in-space) interpretation. The benefits of this reformulation are discussed, including the derivation of timesteppers that are higher order in time, and the quantifiable dissipative SP properties in the non-ideal, resistive case.
        
        We discuss possible options for extending this FET approach to timesteppers for the compressible case.

        The kinetic corrections satisfy linearized Boltzmann equations. Using a Lénard--Bernstein collision operator, these take Fokker--Planck-like forms \cite{Fokker_1914, Planck_1917} wherein pseudo-particles in the numerical model obey the neoclassical transport equations, with particle-independent Brownian drift terms. This offers a rigorous methodology for incorporating collisions into the particle transport model, without coupling the equations of motions for each particle.
        
        Works by Chen, Chacón et al. \cite{Chen_Chacón_Barnes_2011, Chacón_Chen_Barnes_2013, Chen_Chacón_2014, Chen_Chacón_2015} have developed structure-preserving particle pushers for neoclassical transport in the Vlasov equations, derived from Crank--Nicolson integrators. We show these too can can derive from a FET interpretation, similarly offering potential extensions to higher-order-in-time particle pushers. The FET formulation is used also to consider how the stochastic drift terms can be incorporated into the pushers. Stochastic gyrokinetic expansions are also discussed.

        Different options for the numerical implementation of these schemes are considered.

        Due to the efficacy of FET in the development of SP timesteppers for both the fluid and kinetic component, we hope this approach will prove effective in the future for developing SP timesteppers for the full hybrid model. We hope this will give us the opportunity to incorporate previously inaccessible kinetic effects into the highly effective, modern, finite-element MHD models.
    \end{abstract}
    
    
    \newpage
    \tableofcontents
    
    
    \newpage
    \pagenumbering{arabic}
    %\linenumbers\renewcommand\thelinenumber{\color{black!50}\arabic{linenumber}}
            \input{0 - introduction/main.tex}
        \part{Research}
            \input{1 - low-noise PiC models/main.tex}
            \input{2 - kinetic component/main.tex}
            \input{3 - fluid component/main.tex}
            \input{4 - numerical implementation/main.tex}
        \part{Project Overview}
            \input{5 - research plan/main.tex}
            \input{6 - summary/main.tex}
    
    
    %\section{}
    \newpage
    \pagenumbering{gobble}
        \printbibliography


    \newpage
    \pagenumbering{roman}
    \appendix
        \part{Appendices}
            \input{8 - Hilbert complexes/main.tex}
            \input{9 - weak conservation proofs/main.tex}
\end{document}

            \documentclass[12pt, a4paper]{report}

\input{template/main.tex}

\title{\BA{Title in Progress...}}
\author{Boris Andrews}
\affil{Mathematical Institute, University of Oxford}
\date{\today}


\begin{document}
    \pagenumbering{gobble}
    \maketitle
    
    
    \begin{abstract}
        Magnetic confinement reactors---in particular tokamaks---offer one of the most promising options for achieving practical nuclear fusion, with the potential to provide virtually limitless, clean energy. The theoretical and numerical modeling of tokamak plasmas is simultaneously an essential component of effective reactor design, and a great research barrier. Tokamak operational conditions exhibit comparatively low Knudsen numbers. Kinetic effects, including kinetic waves and instabilities, Landau damping, bump-on-tail instabilities and more, are therefore highly influential in tokamak plasma dynamics. Purely fluid models are inherently incapable of capturing these effects, whereas the high dimensionality in purely kinetic models render them practically intractable for most relevant purposes.

        We consider a $\delta\!f$ decomposition model, with a macroscopic fluid background and microscopic kinetic correction, both fully coupled to each other. A similar manner of discretization is proposed to that used in the recent \texttt{STRUPHY} code \cite{Holderied_Possanner_Wang_2021, Holderied_2022, Li_et_al_2023} with a finite-element model for the background and a pseudo-particle/PiC model for the correction.

        The fluid background satisfies the full, non-linear, resistive, compressible, Hall MHD equations. \cite{Laakmann_Hu_Farrell_2022} introduces finite-element(-in-space) implicit timesteppers for the incompressible analogue to this system with structure-preserving (SP) properties in the ideal case, alongside parameter-robust preconditioners. We show that these timesteppers can derive from a finite-element-in-time (FET) (and finite-element-in-space) interpretation. The benefits of this reformulation are discussed, including the derivation of timesteppers that are higher order in time, and the quantifiable dissipative SP properties in the non-ideal, resistive case.
        
        We discuss possible options for extending this FET approach to timesteppers for the compressible case.

        The kinetic corrections satisfy linearized Boltzmann equations. Using a Lénard--Bernstein collision operator, these take Fokker--Planck-like forms \cite{Fokker_1914, Planck_1917} wherein pseudo-particles in the numerical model obey the neoclassical transport equations, with particle-independent Brownian drift terms. This offers a rigorous methodology for incorporating collisions into the particle transport model, without coupling the equations of motions for each particle.
        
        Works by Chen, Chacón et al. \cite{Chen_Chacón_Barnes_2011, Chacón_Chen_Barnes_2013, Chen_Chacón_2014, Chen_Chacón_2015} have developed structure-preserving particle pushers for neoclassical transport in the Vlasov equations, derived from Crank--Nicolson integrators. We show these too can can derive from a FET interpretation, similarly offering potential extensions to higher-order-in-time particle pushers. The FET formulation is used also to consider how the stochastic drift terms can be incorporated into the pushers. Stochastic gyrokinetic expansions are also discussed.

        Different options for the numerical implementation of these schemes are considered.

        Due to the efficacy of FET in the development of SP timesteppers for both the fluid and kinetic component, we hope this approach will prove effective in the future for developing SP timesteppers for the full hybrid model. We hope this will give us the opportunity to incorporate previously inaccessible kinetic effects into the highly effective, modern, finite-element MHD models.
    \end{abstract}
    
    
    \newpage
    \tableofcontents
    
    
    \newpage
    \pagenumbering{arabic}
    %\linenumbers\renewcommand\thelinenumber{\color{black!50}\arabic{linenumber}}
            \input{0 - introduction/main.tex}
        \part{Research}
            \input{1 - low-noise PiC models/main.tex}
            \input{2 - kinetic component/main.tex}
            \input{3 - fluid component/main.tex}
            \input{4 - numerical implementation/main.tex}
        \part{Project Overview}
            \input{5 - research plan/main.tex}
            \input{6 - summary/main.tex}
    
    
    %\section{}
    \newpage
    \pagenumbering{gobble}
        \printbibliography


    \newpage
    \pagenumbering{roman}
    \appendix
        \part{Appendices}
            \input{8 - Hilbert complexes/main.tex}
            \input{9 - weak conservation proofs/main.tex}
\end{document}

        \part{Project Overview}
            \documentclass[12pt, a4paper]{report}

\input{template/main.tex}

\title{\BA{Title in Progress...}}
\author{Boris Andrews}
\affil{Mathematical Institute, University of Oxford}
\date{\today}


\begin{document}
    \pagenumbering{gobble}
    \maketitle
    
    
    \begin{abstract}
        Magnetic confinement reactors---in particular tokamaks---offer one of the most promising options for achieving practical nuclear fusion, with the potential to provide virtually limitless, clean energy. The theoretical and numerical modeling of tokamak plasmas is simultaneously an essential component of effective reactor design, and a great research barrier. Tokamak operational conditions exhibit comparatively low Knudsen numbers. Kinetic effects, including kinetic waves and instabilities, Landau damping, bump-on-tail instabilities and more, are therefore highly influential in tokamak plasma dynamics. Purely fluid models are inherently incapable of capturing these effects, whereas the high dimensionality in purely kinetic models render them practically intractable for most relevant purposes.

        We consider a $\delta\!f$ decomposition model, with a macroscopic fluid background and microscopic kinetic correction, both fully coupled to each other. A similar manner of discretization is proposed to that used in the recent \texttt{STRUPHY} code \cite{Holderied_Possanner_Wang_2021, Holderied_2022, Li_et_al_2023} with a finite-element model for the background and a pseudo-particle/PiC model for the correction.

        The fluid background satisfies the full, non-linear, resistive, compressible, Hall MHD equations. \cite{Laakmann_Hu_Farrell_2022} introduces finite-element(-in-space) implicit timesteppers for the incompressible analogue to this system with structure-preserving (SP) properties in the ideal case, alongside parameter-robust preconditioners. We show that these timesteppers can derive from a finite-element-in-time (FET) (and finite-element-in-space) interpretation. The benefits of this reformulation are discussed, including the derivation of timesteppers that are higher order in time, and the quantifiable dissipative SP properties in the non-ideal, resistive case.
        
        We discuss possible options for extending this FET approach to timesteppers for the compressible case.

        The kinetic corrections satisfy linearized Boltzmann equations. Using a Lénard--Bernstein collision operator, these take Fokker--Planck-like forms \cite{Fokker_1914, Planck_1917} wherein pseudo-particles in the numerical model obey the neoclassical transport equations, with particle-independent Brownian drift terms. This offers a rigorous methodology for incorporating collisions into the particle transport model, without coupling the equations of motions for each particle.
        
        Works by Chen, Chacón et al. \cite{Chen_Chacón_Barnes_2011, Chacón_Chen_Barnes_2013, Chen_Chacón_2014, Chen_Chacón_2015} have developed structure-preserving particle pushers for neoclassical transport in the Vlasov equations, derived from Crank--Nicolson integrators. We show these too can can derive from a FET interpretation, similarly offering potential extensions to higher-order-in-time particle pushers. The FET formulation is used also to consider how the stochastic drift terms can be incorporated into the pushers. Stochastic gyrokinetic expansions are also discussed.

        Different options for the numerical implementation of these schemes are considered.

        Due to the efficacy of FET in the development of SP timesteppers for both the fluid and kinetic component, we hope this approach will prove effective in the future for developing SP timesteppers for the full hybrid model. We hope this will give us the opportunity to incorporate previously inaccessible kinetic effects into the highly effective, modern, finite-element MHD models.
    \end{abstract}
    
    
    \newpage
    \tableofcontents
    
    
    \newpage
    \pagenumbering{arabic}
    %\linenumbers\renewcommand\thelinenumber{\color{black!50}\arabic{linenumber}}
            \input{0 - introduction/main.tex}
        \part{Research}
            \input{1 - low-noise PiC models/main.tex}
            \input{2 - kinetic component/main.tex}
            \input{3 - fluid component/main.tex}
            \input{4 - numerical implementation/main.tex}
        \part{Project Overview}
            \input{5 - research plan/main.tex}
            \input{6 - summary/main.tex}
    
    
    %\section{}
    \newpage
    \pagenumbering{gobble}
        \printbibliography


    \newpage
    \pagenumbering{roman}
    \appendix
        \part{Appendices}
            \input{8 - Hilbert complexes/main.tex}
            \input{9 - weak conservation proofs/main.tex}
\end{document}

            \documentclass[12pt, a4paper]{report}

\input{template/main.tex}

\title{\BA{Title in Progress...}}
\author{Boris Andrews}
\affil{Mathematical Institute, University of Oxford}
\date{\today}


\begin{document}
    \pagenumbering{gobble}
    \maketitle
    
    
    \begin{abstract}
        Magnetic confinement reactors---in particular tokamaks---offer one of the most promising options for achieving practical nuclear fusion, with the potential to provide virtually limitless, clean energy. The theoretical and numerical modeling of tokamak plasmas is simultaneously an essential component of effective reactor design, and a great research barrier. Tokamak operational conditions exhibit comparatively low Knudsen numbers. Kinetic effects, including kinetic waves and instabilities, Landau damping, bump-on-tail instabilities and more, are therefore highly influential in tokamak plasma dynamics. Purely fluid models are inherently incapable of capturing these effects, whereas the high dimensionality in purely kinetic models render them practically intractable for most relevant purposes.

        We consider a $\delta\!f$ decomposition model, with a macroscopic fluid background and microscopic kinetic correction, both fully coupled to each other. A similar manner of discretization is proposed to that used in the recent \texttt{STRUPHY} code \cite{Holderied_Possanner_Wang_2021, Holderied_2022, Li_et_al_2023} with a finite-element model for the background and a pseudo-particle/PiC model for the correction.

        The fluid background satisfies the full, non-linear, resistive, compressible, Hall MHD equations. \cite{Laakmann_Hu_Farrell_2022} introduces finite-element(-in-space) implicit timesteppers for the incompressible analogue to this system with structure-preserving (SP) properties in the ideal case, alongside parameter-robust preconditioners. We show that these timesteppers can derive from a finite-element-in-time (FET) (and finite-element-in-space) interpretation. The benefits of this reformulation are discussed, including the derivation of timesteppers that are higher order in time, and the quantifiable dissipative SP properties in the non-ideal, resistive case.
        
        We discuss possible options for extending this FET approach to timesteppers for the compressible case.

        The kinetic corrections satisfy linearized Boltzmann equations. Using a Lénard--Bernstein collision operator, these take Fokker--Planck-like forms \cite{Fokker_1914, Planck_1917} wherein pseudo-particles in the numerical model obey the neoclassical transport equations, with particle-independent Brownian drift terms. This offers a rigorous methodology for incorporating collisions into the particle transport model, without coupling the equations of motions for each particle.
        
        Works by Chen, Chacón et al. \cite{Chen_Chacón_Barnes_2011, Chacón_Chen_Barnes_2013, Chen_Chacón_2014, Chen_Chacón_2015} have developed structure-preserving particle pushers for neoclassical transport in the Vlasov equations, derived from Crank--Nicolson integrators. We show these too can can derive from a FET interpretation, similarly offering potential extensions to higher-order-in-time particle pushers. The FET formulation is used also to consider how the stochastic drift terms can be incorporated into the pushers. Stochastic gyrokinetic expansions are also discussed.

        Different options for the numerical implementation of these schemes are considered.

        Due to the efficacy of FET in the development of SP timesteppers for both the fluid and kinetic component, we hope this approach will prove effective in the future for developing SP timesteppers for the full hybrid model. We hope this will give us the opportunity to incorporate previously inaccessible kinetic effects into the highly effective, modern, finite-element MHD models.
    \end{abstract}
    
    
    \newpage
    \tableofcontents
    
    
    \newpage
    \pagenumbering{arabic}
    %\linenumbers\renewcommand\thelinenumber{\color{black!50}\arabic{linenumber}}
            \input{0 - introduction/main.tex}
        \part{Research}
            \input{1 - low-noise PiC models/main.tex}
            \input{2 - kinetic component/main.tex}
            \input{3 - fluid component/main.tex}
            \input{4 - numerical implementation/main.tex}
        \part{Project Overview}
            \input{5 - research plan/main.tex}
            \input{6 - summary/main.tex}
    
    
    %\section{}
    \newpage
    \pagenumbering{gobble}
        \printbibliography


    \newpage
    \pagenumbering{roman}
    \appendix
        \part{Appendices}
            \input{8 - Hilbert complexes/main.tex}
            \input{9 - weak conservation proofs/main.tex}
\end{document}

    
    
    %\section{}
    \newpage
    \pagenumbering{gobble}
        \printbibliography


    \newpage
    \pagenumbering{roman}
    \appendix
        \part{Appendices}
            \documentclass[12pt, a4paper]{report}

\input{template/main.tex}

\title{\BA{Title in Progress...}}
\author{Boris Andrews}
\affil{Mathematical Institute, University of Oxford}
\date{\today}


\begin{document}
    \pagenumbering{gobble}
    \maketitle
    
    
    \begin{abstract}
        Magnetic confinement reactors---in particular tokamaks---offer one of the most promising options for achieving practical nuclear fusion, with the potential to provide virtually limitless, clean energy. The theoretical and numerical modeling of tokamak plasmas is simultaneously an essential component of effective reactor design, and a great research barrier. Tokamak operational conditions exhibit comparatively low Knudsen numbers. Kinetic effects, including kinetic waves and instabilities, Landau damping, bump-on-tail instabilities and more, are therefore highly influential in tokamak plasma dynamics. Purely fluid models are inherently incapable of capturing these effects, whereas the high dimensionality in purely kinetic models render them practically intractable for most relevant purposes.

        We consider a $\delta\!f$ decomposition model, with a macroscopic fluid background and microscopic kinetic correction, both fully coupled to each other. A similar manner of discretization is proposed to that used in the recent \texttt{STRUPHY} code \cite{Holderied_Possanner_Wang_2021, Holderied_2022, Li_et_al_2023} with a finite-element model for the background and a pseudo-particle/PiC model for the correction.

        The fluid background satisfies the full, non-linear, resistive, compressible, Hall MHD equations. \cite{Laakmann_Hu_Farrell_2022} introduces finite-element(-in-space) implicit timesteppers for the incompressible analogue to this system with structure-preserving (SP) properties in the ideal case, alongside parameter-robust preconditioners. We show that these timesteppers can derive from a finite-element-in-time (FET) (and finite-element-in-space) interpretation. The benefits of this reformulation are discussed, including the derivation of timesteppers that are higher order in time, and the quantifiable dissipative SP properties in the non-ideal, resistive case.
        
        We discuss possible options for extending this FET approach to timesteppers for the compressible case.

        The kinetic corrections satisfy linearized Boltzmann equations. Using a Lénard--Bernstein collision operator, these take Fokker--Planck-like forms \cite{Fokker_1914, Planck_1917} wherein pseudo-particles in the numerical model obey the neoclassical transport equations, with particle-independent Brownian drift terms. This offers a rigorous methodology for incorporating collisions into the particle transport model, without coupling the equations of motions for each particle.
        
        Works by Chen, Chacón et al. \cite{Chen_Chacón_Barnes_2011, Chacón_Chen_Barnes_2013, Chen_Chacón_2014, Chen_Chacón_2015} have developed structure-preserving particle pushers for neoclassical transport in the Vlasov equations, derived from Crank--Nicolson integrators. We show these too can can derive from a FET interpretation, similarly offering potential extensions to higher-order-in-time particle pushers. The FET formulation is used also to consider how the stochastic drift terms can be incorporated into the pushers. Stochastic gyrokinetic expansions are also discussed.

        Different options for the numerical implementation of these schemes are considered.

        Due to the efficacy of FET in the development of SP timesteppers for both the fluid and kinetic component, we hope this approach will prove effective in the future for developing SP timesteppers for the full hybrid model. We hope this will give us the opportunity to incorporate previously inaccessible kinetic effects into the highly effective, modern, finite-element MHD models.
    \end{abstract}
    
    
    \newpage
    \tableofcontents
    
    
    \newpage
    \pagenumbering{arabic}
    %\linenumbers\renewcommand\thelinenumber{\color{black!50}\arabic{linenumber}}
            \input{0 - introduction/main.tex}
        \part{Research}
            \input{1 - low-noise PiC models/main.tex}
            \input{2 - kinetic component/main.tex}
            \input{3 - fluid component/main.tex}
            \input{4 - numerical implementation/main.tex}
        \part{Project Overview}
            \input{5 - research plan/main.tex}
            \input{6 - summary/main.tex}
    
    
    %\section{}
    \newpage
    \pagenumbering{gobble}
        \printbibliography


    \newpage
    \pagenumbering{roman}
    \appendix
        \part{Appendices}
            \input{8 - Hilbert complexes/main.tex}
            \input{9 - weak conservation proofs/main.tex}
\end{document}

            \documentclass[12pt, a4paper]{report}

\input{template/main.tex}

\title{\BA{Title in Progress...}}
\author{Boris Andrews}
\affil{Mathematical Institute, University of Oxford}
\date{\today}


\begin{document}
    \pagenumbering{gobble}
    \maketitle
    
    
    \begin{abstract}
        Magnetic confinement reactors---in particular tokamaks---offer one of the most promising options for achieving practical nuclear fusion, with the potential to provide virtually limitless, clean energy. The theoretical and numerical modeling of tokamak plasmas is simultaneously an essential component of effective reactor design, and a great research barrier. Tokamak operational conditions exhibit comparatively low Knudsen numbers. Kinetic effects, including kinetic waves and instabilities, Landau damping, bump-on-tail instabilities and more, are therefore highly influential in tokamak plasma dynamics. Purely fluid models are inherently incapable of capturing these effects, whereas the high dimensionality in purely kinetic models render them practically intractable for most relevant purposes.

        We consider a $\delta\!f$ decomposition model, with a macroscopic fluid background and microscopic kinetic correction, both fully coupled to each other. A similar manner of discretization is proposed to that used in the recent \texttt{STRUPHY} code \cite{Holderied_Possanner_Wang_2021, Holderied_2022, Li_et_al_2023} with a finite-element model for the background and a pseudo-particle/PiC model for the correction.

        The fluid background satisfies the full, non-linear, resistive, compressible, Hall MHD equations. \cite{Laakmann_Hu_Farrell_2022} introduces finite-element(-in-space) implicit timesteppers for the incompressible analogue to this system with structure-preserving (SP) properties in the ideal case, alongside parameter-robust preconditioners. We show that these timesteppers can derive from a finite-element-in-time (FET) (and finite-element-in-space) interpretation. The benefits of this reformulation are discussed, including the derivation of timesteppers that are higher order in time, and the quantifiable dissipative SP properties in the non-ideal, resistive case.
        
        We discuss possible options for extending this FET approach to timesteppers for the compressible case.

        The kinetic corrections satisfy linearized Boltzmann equations. Using a Lénard--Bernstein collision operator, these take Fokker--Planck-like forms \cite{Fokker_1914, Planck_1917} wherein pseudo-particles in the numerical model obey the neoclassical transport equations, with particle-independent Brownian drift terms. This offers a rigorous methodology for incorporating collisions into the particle transport model, without coupling the equations of motions for each particle.
        
        Works by Chen, Chacón et al. \cite{Chen_Chacón_Barnes_2011, Chacón_Chen_Barnes_2013, Chen_Chacón_2014, Chen_Chacón_2015} have developed structure-preserving particle pushers for neoclassical transport in the Vlasov equations, derived from Crank--Nicolson integrators. We show these too can can derive from a FET interpretation, similarly offering potential extensions to higher-order-in-time particle pushers. The FET formulation is used also to consider how the stochastic drift terms can be incorporated into the pushers. Stochastic gyrokinetic expansions are also discussed.

        Different options for the numerical implementation of these schemes are considered.

        Due to the efficacy of FET in the development of SP timesteppers for both the fluid and kinetic component, we hope this approach will prove effective in the future for developing SP timesteppers for the full hybrid model. We hope this will give us the opportunity to incorporate previously inaccessible kinetic effects into the highly effective, modern, finite-element MHD models.
    \end{abstract}
    
    
    \newpage
    \tableofcontents
    
    
    \newpage
    \pagenumbering{arabic}
    %\linenumbers\renewcommand\thelinenumber{\color{black!50}\arabic{linenumber}}
            \input{0 - introduction/main.tex}
        \part{Research}
            \input{1 - low-noise PiC models/main.tex}
            \input{2 - kinetic component/main.tex}
            \input{3 - fluid component/main.tex}
            \input{4 - numerical implementation/main.tex}
        \part{Project Overview}
            \input{5 - research plan/main.tex}
            \input{6 - summary/main.tex}
    
    
    %\section{}
    \newpage
    \pagenumbering{gobble}
        \printbibliography


    \newpage
    \pagenumbering{roman}
    \appendix
        \part{Appendices}
            \input{8 - Hilbert complexes/main.tex}
            \input{9 - weak conservation proofs/main.tex}
\end{document}

\end{document}

        \part{Project Overview}
            \documentclass[12pt, a4paper]{report}

\documentclass[12pt, a4paper]{report}

\input{template/main.tex}

\title{\BA{Title in Progress...}}
\author{Boris Andrews}
\affil{Mathematical Institute, University of Oxford}
\date{\today}


\begin{document}
    \pagenumbering{gobble}
    \maketitle
    
    
    \begin{abstract}
        Magnetic confinement reactors---in particular tokamaks---offer one of the most promising options for achieving practical nuclear fusion, with the potential to provide virtually limitless, clean energy. The theoretical and numerical modeling of tokamak plasmas is simultaneously an essential component of effective reactor design, and a great research barrier. Tokamak operational conditions exhibit comparatively low Knudsen numbers. Kinetic effects, including kinetic waves and instabilities, Landau damping, bump-on-tail instabilities and more, are therefore highly influential in tokamak plasma dynamics. Purely fluid models are inherently incapable of capturing these effects, whereas the high dimensionality in purely kinetic models render them practically intractable for most relevant purposes.

        We consider a $\delta\!f$ decomposition model, with a macroscopic fluid background and microscopic kinetic correction, both fully coupled to each other. A similar manner of discretization is proposed to that used in the recent \texttt{STRUPHY} code \cite{Holderied_Possanner_Wang_2021, Holderied_2022, Li_et_al_2023} with a finite-element model for the background and a pseudo-particle/PiC model for the correction.

        The fluid background satisfies the full, non-linear, resistive, compressible, Hall MHD equations. \cite{Laakmann_Hu_Farrell_2022} introduces finite-element(-in-space) implicit timesteppers for the incompressible analogue to this system with structure-preserving (SP) properties in the ideal case, alongside parameter-robust preconditioners. We show that these timesteppers can derive from a finite-element-in-time (FET) (and finite-element-in-space) interpretation. The benefits of this reformulation are discussed, including the derivation of timesteppers that are higher order in time, and the quantifiable dissipative SP properties in the non-ideal, resistive case.
        
        We discuss possible options for extending this FET approach to timesteppers for the compressible case.

        The kinetic corrections satisfy linearized Boltzmann equations. Using a Lénard--Bernstein collision operator, these take Fokker--Planck-like forms \cite{Fokker_1914, Planck_1917} wherein pseudo-particles in the numerical model obey the neoclassical transport equations, with particle-independent Brownian drift terms. This offers a rigorous methodology for incorporating collisions into the particle transport model, without coupling the equations of motions for each particle.
        
        Works by Chen, Chacón et al. \cite{Chen_Chacón_Barnes_2011, Chacón_Chen_Barnes_2013, Chen_Chacón_2014, Chen_Chacón_2015} have developed structure-preserving particle pushers for neoclassical transport in the Vlasov equations, derived from Crank--Nicolson integrators. We show these too can can derive from a FET interpretation, similarly offering potential extensions to higher-order-in-time particle pushers. The FET formulation is used also to consider how the stochastic drift terms can be incorporated into the pushers. Stochastic gyrokinetic expansions are also discussed.

        Different options for the numerical implementation of these schemes are considered.

        Due to the efficacy of FET in the development of SP timesteppers for both the fluid and kinetic component, we hope this approach will prove effective in the future for developing SP timesteppers for the full hybrid model. We hope this will give us the opportunity to incorporate previously inaccessible kinetic effects into the highly effective, modern, finite-element MHD models.
    \end{abstract}
    
    
    \newpage
    \tableofcontents
    
    
    \newpage
    \pagenumbering{arabic}
    %\linenumbers\renewcommand\thelinenumber{\color{black!50}\arabic{linenumber}}
            \input{0 - introduction/main.tex}
        \part{Research}
            \input{1 - low-noise PiC models/main.tex}
            \input{2 - kinetic component/main.tex}
            \input{3 - fluid component/main.tex}
            \input{4 - numerical implementation/main.tex}
        \part{Project Overview}
            \input{5 - research plan/main.tex}
            \input{6 - summary/main.tex}
    
    
    %\section{}
    \newpage
    \pagenumbering{gobble}
        \printbibliography


    \newpage
    \pagenumbering{roman}
    \appendix
        \part{Appendices}
            \input{8 - Hilbert complexes/main.tex}
            \input{9 - weak conservation proofs/main.tex}
\end{document}


\title{\BA{Title in Progress...}}
\author{Boris Andrews}
\affil{Mathematical Institute, University of Oxford}
\date{\today}


\begin{document}
    \pagenumbering{gobble}
    \maketitle
    
    
    \begin{abstract}
        Magnetic confinement reactors---in particular tokamaks---offer one of the most promising options for achieving practical nuclear fusion, with the potential to provide virtually limitless, clean energy. The theoretical and numerical modeling of tokamak plasmas is simultaneously an essential component of effective reactor design, and a great research barrier. Tokamak operational conditions exhibit comparatively low Knudsen numbers. Kinetic effects, including kinetic waves and instabilities, Landau damping, bump-on-tail instabilities and more, are therefore highly influential in tokamak plasma dynamics. Purely fluid models are inherently incapable of capturing these effects, whereas the high dimensionality in purely kinetic models render them practically intractable for most relevant purposes.

        We consider a $\delta\!f$ decomposition model, with a macroscopic fluid background and microscopic kinetic correction, both fully coupled to each other. A similar manner of discretization is proposed to that used in the recent \texttt{STRUPHY} code \cite{Holderied_Possanner_Wang_2021, Holderied_2022, Li_et_al_2023} with a finite-element model for the background and a pseudo-particle/PiC model for the correction.

        The fluid background satisfies the full, non-linear, resistive, compressible, Hall MHD equations. \cite{Laakmann_Hu_Farrell_2022} introduces finite-element(-in-space) implicit timesteppers for the incompressible analogue to this system with structure-preserving (SP) properties in the ideal case, alongside parameter-robust preconditioners. We show that these timesteppers can derive from a finite-element-in-time (FET) (and finite-element-in-space) interpretation. The benefits of this reformulation are discussed, including the derivation of timesteppers that are higher order in time, and the quantifiable dissipative SP properties in the non-ideal, resistive case.
        
        We discuss possible options for extending this FET approach to timesteppers for the compressible case.

        The kinetic corrections satisfy linearized Boltzmann equations. Using a Lénard--Bernstein collision operator, these take Fokker--Planck-like forms \cite{Fokker_1914, Planck_1917} wherein pseudo-particles in the numerical model obey the neoclassical transport equations, with particle-independent Brownian drift terms. This offers a rigorous methodology for incorporating collisions into the particle transport model, without coupling the equations of motions for each particle.
        
        Works by Chen, Chacón et al. \cite{Chen_Chacón_Barnes_2011, Chacón_Chen_Barnes_2013, Chen_Chacón_2014, Chen_Chacón_2015} have developed structure-preserving particle pushers for neoclassical transport in the Vlasov equations, derived from Crank--Nicolson integrators. We show these too can can derive from a FET interpretation, similarly offering potential extensions to higher-order-in-time particle pushers. The FET formulation is used also to consider how the stochastic drift terms can be incorporated into the pushers. Stochastic gyrokinetic expansions are also discussed.

        Different options for the numerical implementation of these schemes are considered.

        Due to the efficacy of FET in the development of SP timesteppers for both the fluid and kinetic component, we hope this approach will prove effective in the future for developing SP timesteppers for the full hybrid model. We hope this will give us the opportunity to incorporate previously inaccessible kinetic effects into the highly effective, modern, finite-element MHD models.
    \end{abstract}
    
    
    \newpage
    \tableofcontents
    
    
    \newpage
    \pagenumbering{arabic}
    %\linenumbers\renewcommand\thelinenumber{\color{black!50}\arabic{linenumber}}
            \documentclass[12pt, a4paper]{report}

\input{template/main.tex}

\title{\BA{Title in Progress...}}
\author{Boris Andrews}
\affil{Mathematical Institute, University of Oxford}
\date{\today}


\begin{document}
    \pagenumbering{gobble}
    \maketitle
    
    
    \begin{abstract}
        Magnetic confinement reactors---in particular tokamaks---offer one of the most promising options for achieving practical nuclear fusion, with the potential to provide virtually limitless, clean energy. The theoretical and numerical modeling of tokamak plasmas is simultaneously an essential component of effective reactor design, and a great research barrier. Tokamak operational conditions exhibit comparatively low Knudsen numbers. Kinetic effects, including kinetic waves and instabilities, Landau damping, bump-on-tail instabilities and more, are therefore highly influential in tokamak plasma dynamics. Purely fluid models are inherently incapable of capturing these effects, whereas the high dimensionality in purely kinetic models render them practically intractable for most relevant purposes.

        We consider a $\delta\!f$ decomposition model, with a macroscopic fluid background and microscopic kinetic correction, both fully coupled to each other. A similar manner of discretization is proposed to that used in the recent \texttt{STRUPHY} code \cite{Holderied_Possanner_Wang_2021, Holderied_2022, Li_et_al_2023} with a finite-element model for the background and a pseudo-particle/PiC model for the correction.

        The fluid background satisfies the full, non-linear, resistive, compressible, Hall MHD equations. \cite{Laakmann_Hu_Farrell_2022} introduces finite-element(-in-space) implicit timesteppers for the incompressible analogue to this system with structure-preserving (SP) properties in the ideal case, alongside parameter-robust preconditioners. We show that these timesteppers can derive from a finite-element-in-time (FET) (and finite-element-in-space) interpretation. The benefits of this reformulation are discussed, including the derivation of timesteppers that are higher order in time, and the quantifiable dissipative SP properties in the non-ideal, resistive case.
        
        We discuss possible options for extending this FET approach to timesteppers for the compressible case.

        The kinetic corrections satisfy linearized Boltzmann equations. Using a Lénard--Bernstein collision operator, these take Fokker--Planck-like forms \cite{Fokker_1914, Planck_1917} wherein pseudo-particles in the numerical model obey the neoclassical transport equations, with particle-independent Brownian drift terms. This offers a rigorous methodology for incorporating collisions into the particle transport model, without coupling the equations of motions for each particle.
        
        Works by Chen, Chacón et al. \cite{Chen_Chacón_Barnes_2011, Chacón_Chen_Barnes_2013, Chen_Chacón_2014, Chen_Chacón_2015} have developed structure-preserving particle pushers for neoclassical transport in the Vlasov equations, derived from Crank--Nicolson integrators. We show these too can can derive from a FET interpretation, similarly offering potential extensions to higher-order-in-time particle pushers. The FET formulation is used also to consider how the stochastic drift terms can be incorporated into the pushers. Stochastic gyrokinetic expansions are also discussed.

        Different options for the numerical implementation of these schemes are considered.

        Due to the efficacy of FET in the development of SP timesteppers for both the fluid and kinetic component, we hope this approach will prove effective in the future for developing SP timesteppers for the full hybrid model. We hope this will give us the opportunity to incorporate previously inaccessible kinetic effects into the highly effective, modern, finite-element MHD models.
    \end{abstract}
    
    
    \newpage
    \tableofcontents
    
    
    \newpage
    \pagenumbering{arabic}
    %\linenumbers\renewcommand\thelinenumber{\color{black!50}\arabic{linenumber}}
            \input{0 - introduction/main.tex}
        \part{Research}
            \input{1 - low-noise PiC models/main.tex}
            \input{2 - kinetic component/main.tex}
            \input{3 - fluid component/main.tex}
            \input{4 - numerical implementation/main.tex}
        \part{Project Overview}
            \input{5 - research plan/main.tex}
            \input{6 - summary/main.tex}
    
    
    %\section{}
    \newpage
    \pagenumbering{gobble}
        \printbibliography


    \newpage
    \pagenumbering{roman}
    \appendix
        \part{Appendices}
            \input{8 - Hilbert complexes/main.tex}
            \input{9 - weak conservation proofs/main.tex}
\end{document}

        \part{Research}
            \documentclass[12pt, a4paper]{report}

\input{template/main.tex}

\title{\BA{Title in Progress...}}
\author{Boris Andrews}
\affil{Mathematical Institute, University of Oxford}
\date{\today}


\begin{document}
    \pagenumbering{gobble}
    \maketitle
    
    
    \begin{abstract}
        Magnetic confinement reactors---in particular tokamaks---offer one of the most promising options for achieving practical nuclear fusion, with the potential to provide virtually limitless, clean energy. The theoretical and numerical modeling of tokamak plasmas is simultaneously an essential component of effective reactor design, and a great research barrier. Tokamak operational conditions exhibit comparatively low Knudsen numbers. Kinetic effects, including kinetic waves and instabilities, Landau damping, bump-on-tail instabilities and more, are therefore highly influential in tokamak plasma dynamics. Purely fluid models are inherently incapable of capturing these effects, whereas the high dimensionality in purely kinetic models render them practically intractable for most relevant purposes.

        We consider a $\delta\!f$ decomposition model, with a macroscopic fluid background and microscopic kinetic correction, both fully coupled to each other. A similar manner of discretization is proposed to that used in the recent \texttt{STRUPHY} code \cite{Holderied_Possanner_Wang_2021, Holderied_2022, Li_et_al_2023} with a finite-element model for the background and a pseudo-particle/PiC model for the correction.

        The fluid background satisfies the full, non-linear, resistive, compressible, Hall MHD equations. \cite{Laakmann_Hu_Farrell_2022} introduces finite-element(-in-space) implicit timesteppers for the incompressible analogue to this system with structure-preserving (SP) properties in the ideal case, alongside parameter-robust preconditioners. We show that these timesteppers can derive from a finite-element-in-time (FET) (and finite-element-in-space) interpretation. The benefits of this reformulation are discussed, including the derivation of timesteppers that are higher order in time, and the quantifiable dissipative SP properties in the non-ideal, resistive case.
        
        We discuss possible options for extending this FET approach to timesteppers for the compressible case.

        The kinetic corrections satisfy linearized Boltzmann equations. Using a Lénard--Bernstein collision operator, these take Fokker--Planck-like forms \cite{Fokker_1914, Planck_1917} wherein pseudo-particles in the numerical model obey the neoclassical transport equations, with particle-independent Brownian drift terms. This offers a rigorous methodology for incorporating collisions into the particle transport model, without coupling the equations of motions for each particle.
        
        Works by Chen, Chacón et al. \cite{Chen_Chacón_Barnes_2011, Chacón_Chen_Barnes_2013, Chen_Chacón_2014, Chen_Chacón_2015} have developed structure-preserving particle pushers for neoclassical transport in the Vlasov equations, derived from Crank--Nicolson integrators. We show these too can can derive from a FET interpretation, similarly offering potential extensions to higher-order-in-time particle pushers. The FET formulation is used also to consider how the stochastic drift terms can be incorporated into the pushers. Stochastic gyrokinetic expansions are also discussed.

        Different options for the numerical implementation of these schemes are considered.

        Due to the efficacy of FET in the development of SP timesteppers for both the fluid and kinetic component, we hope this approach will prove effective in the future for developing SP timesteppers for the full hybrid model. We hope this will give us the opportunity to incorporate previously inaccessible kinetic effects into the highly effective, modern, finite-element MHD models.
    \end{abstract}
    
    
    \newpage
    \tableofcontents
    
    
    \newpage
    \pagenumbering{arabic}
    %\linenumbers\renewcommand\thelinenumber{\color{black!50}\arabic{linenumber}}
            \input{0 - introduction/main.tex}
        \part{Research}
            \input{1 - low-noise PiC models/main.tex}
            \input{2 - kinetic component/main.tex}
            \input{3 - fluid component/main.tex}
            \input{4 - numerical implementation/main.tex}
        \part{Project Overview}
            \input{5 - research plan/main.tex}
            \input{6 - summary/main.tex}
    
    
    %\section{}
    \newpage
    \pagenumbering{gobble}
        \printbibliography


    \newpage
    \pagenumbering{roman}
    \appendix
        \part{Appendices}
            \input{8 - Hilbert complexes/main.tex}
            \input{9 - weak conservation proofs/main.tex}
\end{document}

            \documentclass[12pt, a4paper]{report}

\input{template/main.tex}

\title{\BA{Title in Progress...}}
\author{Boris Andrews}
\affil{Mathematical Institute, University of Oxford}
\date{\today}


\begin{document}
    \pagenumbering{gobble}
    \maketitle
    
    
    \begin{abstract}
        Magnetic confinement reactors---in particular tokamaks---offer one of the most promising options for achieving practical nuclear fusion, with the potential to provide virtually limitless, clean energy. The theoretical and numerical modeling of tokamak plasmas is simultaneously an essential component of effective reactor design, and a great research barrier. Tokamak operational conditions exhibit comparatively low Knudsen numbers. Kinetic effects, including kinetic waves and instabilities, Landau damping, bump-on-tail instabilities and more, are therefore highly influential in tokamak plasma dynamics. Purely fluid models are inherently incapable of capturing these effects, whereas the high dimensionality in purely kinetic models render them practically intractable for most relevant purposes.

        We consider a $\delta\!f$ decomposition model, with a macroscopic fluid background and microscopic kinetic correction, both fully coupled to each other. A similar manner of discretization is proposed to that used in the recent \texttt{STRUPHY} code \cite{Holderied_Possanner_Wang_2021, Holderied_2022, Li_et_al_2023} with a finite-element model for the background and a pseudo-particle/PiC model for the correction.

        The fluid background satisfies the full, non-linear, resistive, compressible, Hall MHD equations. \cite{Laakmann_Hu_Farrell_2022} introduces finite-element(-in-space) implicit timesteppers for the incompressible analogue to this system with structure-preserving (SP) properties in the ideal case, alongside parameter-robust preconditioners. We show that these timesteppers can derive from a finite-element-in-time (FET) (and finite-element-in-space) interpretation. The benefits of this reformulation are discussed, including the derivation of timesteppers that are higher order in time, and the quantifiable dissipative SP properties in the non-ideal, resistive case.
        
        We discuss possible options for extending this FET approach to timesteppers for the compressible case.

        The kinetic corrections satisfy linearized Boltzmann equations. Using a Lénard--Bernstein collision operator, these take Fokker--Planck-like forms \cite{Fokker_1914, Planck_1917} wherein pseudo-particles in the numerical model obey the neoclassical transport equations, with particle-independent Brownian drift terms. This offers a rigorous methodology for incorporating collisions into the particle transport model, without coupling the equations of motions for each particle.
        
        Works by Chen, Chacón et al. \cite{Chen_Chacón_Barnes_2011, Chacón_Chen_Barnes_2013, Chen_Chacón_2014, Chen_Chacón_2015} have developed structure-preserving particle pushers for neoclassical transport in the Vlasov equations, derived from Crank--Nicolson integrators. We show these too can can derive from a FET interpretation, similarly offering potential extensions to higher-order-in-time particle pushers. The FET formulation is used also to consider how the stochastic drift terms can be incorporated into the pushers. Stochastic gyrokinetic expansions are also discussed.

        Different options for the numerical implementation of these schemes are considered.

        Due to the efficacy of FET in the development of SP timesteppers for both the fluid and kinetic component, we hope this approach will prove effective in the future for developing SP timesteppers for the full hybrid model. We hope this will give us the opportunity to incorporate previously inaccessible kinetic effects into the highly effective, modern, finite-element MHD models.
    \end{abstract}
    
    
    \newpage
    \tableofcontents
    
    
    \newpage
    \pagenumbering{arabic}
    %\linenumbers\renewcommand\thelinenumber{\color{black!50}\arabic{linenumber}}
            \input{0 - introduction/main.tex}
        \part{Research}
            \input{1 - low-noise PiC models/main.tex}
            \input{2 - kinetic component/main.tex}
            \input{3 - fluid component/main.tex}
            \input{4 - numerical implementation/main.tex}
        \part{Project Overview}
            \input{5 - research plan/main.tex}
            \input{6 - summary/main.tex}
    
    
    %\section{}
    \newpage
    \pagenumbering{gobble}
        \printbibliography


    \newpage
    \pagenumbering{roman}
    \appendix
        \part{Appendices}
            \input{8 - Hilbert complexes/main.tex}
            \input{9 - weak conservation proofs/main.tex}
\end{document}

            \documentclass[12pt, a4paper]{report}

\input{template/main.tex}

\title{\BA{Title in Progress...}}
\author{Boris Andrews}
\affil{Mathematical Institute, University of Oxford}
\date{\today}


\begin{document}
    \pagenumbering{gobble}
    \maketitle
    
    
    \begin{abstract}
        Magnetic confinement reactors---in particular tokamaks---offer one of the most promising options for achieving practical nuclear fusion, with the potential to provide virtually limitless, clean energy. The theoretical and numerical modeling of tokamak plasmas is simultaneously an essential component of effective reactor design, and a great research barrier. Tokamak operational conditions exhibit comparatively low Knudsen numbers. Kinetic effects, including kinetic waves and instabilities, Landau damping, bump-on-tail instabilities and more, are therefore highly influential in tokamak plasma dynamics. Purely fluid models are inherently incapable of capturing these effects, whereas the high dimensionality in purely kinetic models render them practically intractable for most relevant purposes.

        We consider a $\delta\!f$ decomposition model, with a macroscopic fluid background and microscopic kinetic correction, both fully coupled to each other. A similar manner of discretization is proposed to that used in the recent \texttt{STRUPHY} code \cite{Holderied_Possanner_Wang_2021, Holderied_2022, Li_et_al_2023} with a finite-element model for the background and a pseudo-particle/PiC model for the correction.

        The fluid background satisfies the full, non-linear, resistive, compressible, Hall MHD equations. \cite{Laakmann_Hu_Farrell_2022} introduces finite-element(-in-space) implicit timesteppers for the incompressible analogue to this system with structure-preserving (SP) properties in the ideal case, alongside parameter-robust preconditioners. We show that these timesteppers can derive from a finite-element-in-time (FET) (and finite-element-in-space) interpretation. The benefits of this reformulation are discussed, including the derivation of timesteppers that are higher order in time, and the quantifiable dissipative SP properties in the non-ideal, resistive case.
        
        We discuss possible options for extending this FET approach to timesteppers for the compressible case.

        The kinetic corrections satisfy linearized Boltzmann equations. Using a Lénard--Bernstein collision operator, these take Fokker--Planck-like forms \cite{Fokker_1914, Planck_1917} wherein pseudo-particles in the numerical model obey the neoclassical transport equations, with particle-independent Brownian drift terms. This offers a rigorous methodology for incorporating collisions into the particle transport model, without coupling the equations of motions for each particle.
        
        Works by Chen, Chacón et al. \cite{Chen_Chacón_Barnes_2011, Chacón_Chen_Barnes_2013, Chen_Chacón_2014, Chen_Chacón_2015} have developed structure-preserving particle pushers for neoclassical transport in the Vlasov equations, derived from Crank--Nicolson integrators. We show these too can can derive from a FET interpretation, similarly offering potential extensions to higher-order-in-time particle pushers. The FET formulation is used also to consider how the stochastic drift terms can be incorporated into the pushers. Stochastic gyrokinetic expansions are also discussed.

        Different options for the numerical implementation of these schemes are considered.

        Due to the efficacy of FET in the development of SP timesteppers for both the fluid and kinetic component, we hope this approach will prove effective in the future for developing SP timesteppers for the full hybrid model. We hope this will give us the opportunity to incorporate previously inaccessible kinetic effects into the highly effective, modern, finite-element MHD models.
    \end{abstract}
    
    
    \newpage
    \tableofcontents
    
    
    \newpage
    \pagenumbering{arabic}
    %\linenumbers\renewcommand\thelinenumber{\color{black!50}\arabic{linenumber}}
            \input{0 - introduction/main.tex}
        \part{Research}
            \input{1 - low-noise PiC models/main.tex}
            \input{2 - kinetic component/main.tex}
            \input{3 - fluid component/main.tex}
            \input{4 - numerical implementation/main.tex}
        \part{Project Overview}
            \input{5 - research plan/main.tex}
            \input{6 - summary/main.tex}
    
    
    %\section{}
    \newpage
    \pagenumbering{gobble}
        \printbibliography


    \newpage
    \pagenumbering{roman}
    \appendix
        \part{Appendices}
            \input{8 - Hilbert complexes/main.tex}
            \input{9 - weak conservation proofs/main.tex}
\end{document}

            \documentclass[12pt, a4paper]{report}

\input{template/main.tex}

\title{\BA{Title in Progress...}}
\author{Boris Andrews}
\affil{Mathematical Institute, University of Oxford}
\date{\today}


\begin{document}
    \pagenumbering{gobble}
    \maketitle
    
    
    \begin{abstract}
        Magnetic confinement reactors---in particular tokamaks---offer one of the most promising options for achieving practical nuclear fusion, with the potential to provide virtually limitless, clean energy. The theoretical and numerical modeling of tokamak plasmas is simultaneously an essential component of effective reactor design, and a great research barrier. Tokamak operational conditions exhibit comparatively low Knudsen numbers. Kinetic effects, including kinetic waves and instabilities, Landau damping, bump-on-tail instabilities and more, are therefore highly influential in tokamak plasma dynamics. Purely fluid models are inherently incapable of capturing these effects, whereas the high dimensionality in purely kinetic models render them practically intractable for most relevant purposes.

        We consider a $\delta\!f$ decomposition model, with a macroscopic fluid background and microscopic kinetic correction, both fully coupled to each other. A similar manner of discretization is proposed to that used in the recent \texttt{STRUPHY} code \cite{Holderied_Possanner_Wang_2021, Holderied_2022, Li_et_al_2023} with a finite-element model for the background and a pseudo-particle/PiC model for the correction.

        The fluid background satisfies the full, non-linear, resistive, compressible, Hall MHD equations. \cite{Laakmann_Hu_Farrell_2022} introduces finite-element(-in-space) implicit timesteppers for the incompressible analogue to this system with structure-preserving (SP) properties in the ideal case, alongside parameter-robust preconditioners. We show that these timesteppers can derive from a finite-element-in-time (FET) (and finite-element-in-space) interpretation. The benefits of this reformulation are discussed, including the derivation of timesteppers that are higher order in time, and the quantifiable dissipative SP properties in the non-ideal, resistive case.
        
        We discuss possible options for extending this FET approach to timesteppers for the compressible case.

        The kinetic corrections satisfy linearized Boltzmann equations. Using a Lénard--Bernstein collision operator, these take Fokker--Planck-like forms \cite{Fokker_1914, Planck_1917} wherein pseudo-particles in the numerical model obey the neoclassical transport equations, with particle-independent Brownian drift terms. This offers a rigorous methodology for incorporating collisions into the particle transport model, without coupling the equations of motions for each particle.
        
        Works by Chen, Chacón et al. \cite{Chen_Chacón_Barnes_2011, Chacón_Chen_Barnes_2013, Chen_Chacón_2014, Chen_Chacón_2015} have developed structure-preserving particle pushers for neoclassical transport in the Vlasov equations, derived from Crank--Nicolson integrators. We show these too can can derive from a FET interpretation, similarly offering potential extensions to higher-order-in-time particle pushers. The FET formulation is used also to consider how the stochastic drift terms can be incorporated into the pushers. Stochastic gyrokinetic expansions are also discussed.

        Different options for the numerical implementation of these schemes are considered.

        Due to the efficacy of FET in the development of SP timesteppers for both the fluid and kinetic component, we hope this approach will prove effective in the future for developing SP timesteppers for the full hybrid model. We hope this will give us the opportunity to incorporate previously inaccessible kinetic effects into the highly effective, modern, finite-element MHD models.
    \end{abstract}
    
    
    \newpage
    \tableofcontents
    
    
    \newpage
    \pagenumbering{arabic}
    %\linenumbers\renewcommand\thelinenumber{\color{black!50}\arabic{linenumber}}
            \input{0 - introduction/main.tex}
        \part{Research}
            \input{1 - low-noise PiC models/main.tex}
            \input{2 - kinetic component/main.tex}
            \input{3 - fluid component/main.tex}
            \input{4 - numerical implementation/main.tex}
        \part{Project Overview}
            \input{5 - research plan/main.tex}
            \input{6 - summary/main.tex}
    
    
    %\section{}
    \newpage
    \pagenumbering{gobble}
        \printbibliography


    \newpage
    \pagenumbering{roman}
    \appendix
        \part{Appendices}
            \input{8 - Hilbert complexes/main.tex}
            \input{9 - weak conservation proofs/main.tex}
\end{document}

        \part{Project Overview}
            \documentclass[12pt, a4paper]{report}

\input{template/main.tex}

\title{\BA{Title in Progress...}}
\author{Boris Andrews}
\affil{Mathematical Institute, University of Oxford}
\date{\today}


\begin{document}
    \pagenumbering{gobble}
    \maketitle
    
    
    \begin{abstract}
        Magnetic confinement reactors---in particular tokamaks---offer one of the most promising options for achieving practical nuclear fusion, with the potential to provide virtually limitless, clean energy. The theoretical and numerical modeling of tokamak plasmas is simultaneously an essential component of effective reactor design, and a great research barrier. Tokamak operational conditions exhibit comparatively low Knudsen numbers. Kinetic effects, including kinetic waves and instabilities, Landau damping, bump-on-tail instabilities and more, are therefore highly influential in tokamak plasma dynamics. Purely fluid models are inherently incapable of capturing these effects, whereas the high dimensionality in purely kinetic models render them practically intractable for most relevant purposes.

        We consider a $\delta\!f$ decomposition model, with a macroscopic fluid background and microscopic kinetic correction, both fully coupled to each other. A similar manner of discretization is proposed to that used in the recent \texttt{STRUPHY} code \cite{Holderied_Possanner_Wang_2021, Holderied_2022, Li_et_al_2023} with a finite-element model for the background and a pseudo-particle/PiC model for the correction.

        The fluid background satisfies the full, non-linear, resistive, compressible, Hall MHD equations. \cite{Laakmann_Hu_Farrell_2022} introduces finite-element(-in-space) implicit timesteppers for the incompressible analogue to this system with structure-preserving (SP) properties in the ideal case, alongside parameter-robust preconditioners. We show that these timesteppers can derive from a finite-element-in-time (FET) (and finite-element-in-space) interpretation. The benefits of this reformulation are discussed, including the derivation of timesteppers that are higher order in time, and the quantifiable dissipative SP properties in the non-ideal, resistive case.
        
        We discuss possible options for extending this FET approach to timesteppers for the compressible case.

        The kinetic corrections satisfy linearized Boltzmann equations. Using a Lénard--Bernstein collision operator, these take Fokker--Planck-like forms \cite{Fokker_1914, Planck_1917} wherein pseudo-particles in the numerical model obey the neoclassical transport equations, with particle-independent Brownian drift terms. This offers a rigorous methodology for incorporating collisions into the particle transport model, without coupling the equations of motions for each particle.
        
        Works by Chen, Chacón et al. \cite{Chen_Chacón_Barnes_2011, Chacón_Chen_Barnes_2013, Chen_Chacón_2014, Chen_Chacón_2015} have developed structure-preserving particle pushers for neoclassical transport in the Vlasov equations, derived from Crank--Nicolson integrators. We show these too can can derive from a FET interpretation, similarly offering potential extensions to higher-order-in-time particle pushers. The FET formulation is used also to consider how the stochastic drift terms can be incorporated into the pushers. Stochastic gyrokinetic expansions are also discussed.

        Different options for the numerical implementation of these schemes are considered.

        Due to the efficacy of FET in the development of SP timesteppers for both the fluid and kinetic component, we hope this approach will prove effective in the future for developing SP timesteppers for the full hybrid model. We hope this will give us the opportunity to incorporate previously inaccessible kinetic effects into the highly effective, modern, finite-element MHD models.
    \end{abstract}
    
    
    \newpage
    \tableofcontents
    
    
    \newpage
    \pagenumbering{arabic}
    %\linenumbers\renewcommand\thelinenumber{\color{black!50}\arabic{linenumber}}
            \input{0 - introduction/main.tex}
        \part{Research}
            \input{1 - low-noise PiC models/main.tex}
            \input{2 - kinetic component/main.tex}
            \input{3 - fluid component/main.tex}
            \input{4 - numerical implementation/main.tex}
        \part{Project Overview}
            \input{5 - research plan/main.tex}
            \input{6 - summary/main.tex}
    
    
    %\section{}
    \newpage
    \pagenumbering{gobble}
        \printbibliography


    \newpage
    \pagenumbering{roman}
    \appendix
        \part{Appendices}
            \input{8 - Hilbert complexes/main.tex}
            \input{9 - weak conservation proofs/main.tex}
\end{document}

            \documentclass[12pt, a4paper]{report}

\input{template/main.tex}

\title{\BA{Title in Progress...}}
\author{Boris Andrews}
\affil{Mathematical Institute, University of Oxford}
\date{\today}


\begin{document}
    \pagenumbering{gobble}
    \maketitle
    
    
    \begin{abstract}
        Magnetic confinement reactors---in particular tokamaks---offer one of the most promising options for achieving practical nuclear fusion, with the potential to provide virtually limitless, clean energy. The theoretical and numerical modeling of tokamak plasmas is simultaneously an essential component of effective reactor design, and a great research barrier. Tokamak operational conditions exhibit comparatively low Knudsen numbers. Kinetic effects, including kinetic waves and instabilities, Landau damping, bump-on-tail instabilities and more, are therefore highly influential in tokamak plasma dynamics. Purely fluid models are inherently incapable of capturing these effects, whereas the high dimensionality in purely kinetic models render them practically intractable for most relevant purposes.

        We consider a $\delta\!f$ decomposition model, with a macroscopic fluid background and microscopic kinetic correction, both fully coupled to each other. A similar manner of discretization is proposed to that used in the recent \texttt{STRUPHY} code \cite{Holderied_Possanner_Wang_2021, Holderied_2022, Li_et_al_2023} with a finite-element model for the background and a pseudo-particle/PiC model for the correction.

        The fluid background satisfies the full, non-linear, resistive, compressible, Hall MHD equations. \cite{Laakmann_Hu_Farrell_2022} introduces finite-element(-in-space) implicit timesteppers for the incompressible analogue to this system with structure-preserving (SP) properties in the ideal case, alongside parameter-robust preconditioners. We show that these timesteppers can derive from a finite-element-in-time (FET) (and finite-element-in-space) interpretation. The benefits of this reformulation are discussed, including the derivation of timesteppers that are higher order in time, and the quantifiable dissipative SP properties in the non-ideal, resistive case.
        
        We discuss possible options for extending this FET approach to timesteppers for the compressible case.

        The kinetic corrections satisfy linearized Boltzmann equations. Using a Lénard--Bernstein collision operator, these take Fokker--Planck-like forms \cite{Fokker_1914, Planck_1917} wherein pseudo-particles in the numerical model obey the neoclassical transport equations, with particle-independent Brownian drift terms. This offers a rigorous methodology for incorporating collisions into the particle transport model, without coupling the equations of motions for each particle.
        
        Works by Chen, Chacón et al. \cite{Chen_Chacón_Barnes_2011, Chacón_Chen_Barnes_2013, Chen_Chacón_2014, Chen_Chacón_2015} have developed structure-preserving particle pushers for neoclassical transport in the Vlasov equations, derived from Crank--Nicolson integrators. We show these too can can derive from a FET interpretation, similarly offering potential extensions to higher-order-in-time particle pushers. The FET formulation is used also to consider how the stochastic drift terms can be incorporated into the pushers. Stochastic gyrokinetic expansions are also discussed.

        Different options for the numerical implementation of these schemes are considered.

        Due to the efficacy of FET in the development of SP timesteppers for both the fluid and kinetic component, we hope this approach will prove effective in the future for developing SP timesteppers for the full hybrid model. We hope this will give us the opportunity to incorporate previously inaccessible kinetic effects into the highly effective, modern, finite-element MHD models.
    \end{abstract}
    
    
    \newpage
    \tableofcontents
    
    
    \newpage
    \pagenumbering{arabic}
    %\linenumbers\renewcommand\thelinenumber{\color{black!50}\arabic{linenumber}}
            \input{0 - introduction/main.tex}
        \part{Research}
            \input{1 - low-noise PiC models/main.tex}
            \input{2 - kinetic component/main.tex}
            \input{3 - fluid component/main.tex}
            \input{4 - numerical implementation/main.tex}
        \part{Project Overview}
            \input{5 - research plan/main.tex}
            \input{6 - summary/main.tex}
    
    
    %\section{}
    \newpage
    \pagenumbering{gobble}
        \printbibliography


    \newpage
    \pagenumbering{roman}
    \appendix
        \part{Appendices}
            \input{8 - Hilbert complexes/main.tex}
            \input{9 - weak conservation proofs/main.tex}
\end{document}

    
    
    %\section{}
    \newpage
    \pagenumbering{gobble}
        \printbibliography


    \newpage
    \pagenumbering{roman}
    \appendix
        \part{Appendices}
            \documentclass[12pt, a4paper]{report}

\input{template/main.tex}

\title{\BA{Title in Progress...}}
\author{Boris Andrews}
\affil{Mathematical Institute, University of Oxford}
\date{\today}


\begin{document}
    \pagenumbering{gobble}
    \maketitle
    
    
    \begin{abstract}
        Magnetic confinement reactors---in particular tokamaks---offer one of the most promising options for achieving practical nuclear fusion, with the potential to provide virtually limitless, clean energy. The theoretical and numerical modeling of tokamak plasmas is simultaneously an essential component of effective reactor design, and a great research barrier. Tokamak operational conditions exhibit comparatively low Knudsen numbers. Kinetic effects, including kinetic waves and instabilities, Landau damping, bump-on-tail instabilities and more, are therefore highly influential in tokamak plasma dynamics. Purely fluid models are inherently incapable of capturing these effects, whereas the high dimensionality in purely kinetic models render them practically intractable for most relevant purposes.

        We consider a $\delta\!f$ decomposition model, with a macroscopic fluid background and microscopic kinetic correction, both fully coupled to each other. A similar manner of discretization is proposed to that used in the recent \texttt{STRUPHY} code \cite{Holderied_Possanner_Wang_2021, Holderied_2022, Li_et_al_2023} with a finite-element model for the background and a pseudo-particle/PiC model for the correction.

        The fluid background satisfies the full, non-linear, resistive, compressible, Hall MHD equations. \cite{Laakmann_Hu_Farrell_2022} introduces finite-element(-in-space) implicit timesteppers for the incompressible analogue to this system with structure-preserving (SP) properties in the ideal case, alongside parameter-robust preconditioners. We show that these timesteppers can derive from a finite-element-in-time (FET) (and finite-element-in-space) interpretation. The benefits of this reformulation are discussed, including the derivation of timesteppers that are higher order in time, and the quantifiable dissipative SP properties in the non-ideal, resistive case.
        
        We discuss possible options for extending this FET approach to timesteppers for the compressible case.

        The kinetic corrections satisfy linearized Boltzmann equations. Using a Lénard--Bernstein collision operator, these take Fokker--Planck-like forms \cite{Fokker_1914, Planck_1917} wherein pseudo-particles in the numerical model obey the neoclassical transport equations, with particle-independent Brownian drift terms. This offers a rigorous methodology for incorporating collisions into the particle transport model, without coupling the equations of motions for each particle.
        
        Works by Chen, Chacón et al. \cite{Chen_Chacón_Barnes_2011, Chacón_Chen_Barnes_2013, Chen_Chacón_2014, Chen_Chacón_2015} have developed structure-preserving particle pushers for neoclassical transport in the Vlasov equations, derived from Crank--Nicolson integrators. We show these too can can derive from a FET interpretation, similarly offering potential extensions to higher-order-in-time particle pushers. The FET formulation is used also to consider how the stochastic drift terms can be incorporated into the pushers. Stochastic gyrokinetic expansions are also discussed.

        Different options for the numerical implementation of these schemes are considered.

        Due to the efficacy of FET in the development of SP timesteppers for both the fluid and kinetic component, we hope this approach will prove effective in the future for developing SP timesteppers for the full hybrid model. We hope this will give us the opportunity to incorporate previously inaccessible kinetic effects into the highly effective, modern, finite-element MHD models.
    \end{abstract}
    
    
    \newpage
    \tableofcontents
    
    
    \newpage
    \pagenumbering{arabic}
    %\linenumbers\renewcommand\thelinenumber{\color{black!50}\arabic{linenumber}}
            \input{0 - introduction/main.tex}
        \part{Research}
            \input{1 - low-noise PiC models/main.tex}
            \input{2 - kinetic component/main.tex}
            \input{3 - fluid component/main.tex}
            \input{4 - numerical implementation/main.tex}
        \part{Project Overview}
            \input{5 - research plan/main.tex}
            \input{6 - summary/main.tex}
    
    
    %\section{}
    \newpage
    \pagenumbering{gobble}
        \printbibliography


    \newpage
    \pagenumbering{roman}
    \appendix
        \part{Appendices}
            \input{8 - Hilbert complexes/main.tex}
            \input{9 - weak conservation proofs/main.tex}
\end{document}

            \documentclass[12pt, a4paper]{report}

\input{template/main.tex}

\title{\BA{Title in Progress...}}
\author{Boris Andrews}
\affil{Mathematical Institute, University of Oxford}
\date{\today}


\begin{document}
    \pagenumbering{gobble}
    \maketitle
    
    
    \begin{abstract}
        Magnetic confinement reactors---in particular tokamaks---offer one of the most promising options for achieving practical nuclear fusion, with the potential to provide virtually limitless, clean energy. The theoretical and numerical modeling of tokamak plasmas is simultaneously an essential component of effective reactor design, and a great research barrier. Tokamak operational conditions exhibit comparatively low Knudsen numbers. Kinetic effects, including kinetic waves and instabilities, Landau damping, bump-on-tail instabilities and more, are therefore highly influential in tokamak plasma dynamics. Purely fluid models are inherently incapable of capturing these effects, whereas the high dimensionality in purely kinetic models render them practically intractable for most relevant purposes.

        We consider a $\delta\!f$ decomposition model, with a macroscopic fluid background and microscopic kinetic correction, both fully coupled to each other. A similar manner of discretization is proposed to that used in the recent \texttt{STRUPHY} code \cite{Holderied_Possanner_Wang_2021, Holderied_2022, Li_et_al_2023} with a finite-element model for the background and a pseudo-particle/PiC model for the correction.

        The fluid background satisfies the full, non-linear, resistive, compressible, Hall MHD equations. \cite{Laakmann_Hu_Farrell_2022} introduces finite-element(-in-space) implicit timesteppers for the incompressible analogue to this system with structure-preserving (SP) properties in the ideal case, alongside parameter-robust preconditioners. We show that these timesteppers can derive from a finite-element-in-time (FET) (and finite-element-in-space) interpretation. The benefits of this reformulation are discussed, including the derivation of timesteppers that are higher order in time, and the quantifiable dissipative SP properties in the non-ideal, resistive case.
        
        We discuss possible options for extending this FET approach to timesteppers for the compressible case.

        The kinetic corrections satisfy linearized Boltzmann equations. Using a Lénard--Bernstein collision operator, these take Fokker--Planck-like forms \cite{Fokker_1914, Planck_1917} wherein pseudo-particles in the numerical model obey the neoclassical transport equations, with particle-independent Brownian drift terms. This offers a rigorous methodology for incorporating collisions into the particle transport model, without coupling the equations of motions for each particle.
        
        Works by Chen, Chacón et al. \cite{Chen_Chacón_Barnes_2011, Chacón_Chen_Barnes_2013, Chen_Chacón_2014, Chen_Chacón_2015} have developed structure-preserving particle pushers for neoclassical transport in the Vlasov equations, derived from Crank--Nicolson integrators. We show these too can can derive from a FET interpretation, similarly offering potential extensions to higher-order-in-time particle pushers. The FET formulation is used also to consider how the stochastic drift terms can be incorporated into the pushers. Stochastic gyrokinetic expansions are also discussed.

        Different options for the numerical implementation of these schemes are considered.

        Due to the efficacy of FET in the development of SP timesteppers for both the fluid and kinetic component, we hope this approach will prove effective in the future for developing SP timesteppers for the full hybrid model. We hope this will give us the opportunity to incorporate previously inaccessible kinetic effects into the highly effective, modern, finite-element MHD models.
    \end{abstract}
    
    
    \newpage
    \tableofcontents
    
    
    \newpage
    \pagenumbering{arabic}
    %\linenumbers\renewcommand\thelinenumber{\color{black!50}\arabic{linenumber}}
            \input{0 - introduction/main.tex}
        \part{Research}
            \input{1 - low-noise PiC models/main.tex}
            \input{2 - kinetic component/main.tex}
            \input{3 - fluid component/main.tex}
            \input{4 - numerical implementation/main.tex}
        \part{Project Overview}
            \input{5 - research plan/main.tex}
            \input{6 - summary/main.tex}
    
    
    %\section{}
    \newpage
    \pagenumbering{gobble}
        \printbibliography


    \newpage
    \pagenumbering{roman}
    \appendix
        \part{Appendices}
            \input{8 - Hilbert complexes/main.tex}
            \input{9 - weak conservation proofs/main.tex}
\end{document}

\end{document}

            \documentclass[12pt, a4paper]{report}

\documentclass[12pt, a4paper]{report}

\input{template/main.tex}

\title{\BA{Title in Progress...}}
\author{Boris Andrews}
\affil{Mathematical Institute, University of Oxford}
\date{\today}


\begin{document}
    \pagenumbering{gobble}
    \maketitle
    
    
    \begin{abstract}
        Magnetic confinement reactors---in particular tokamaks---offer one of the most promising options for achieving practical nuclear fusion, with the potential to provide virtually limitless, clean energy. The theoretical and numerical modeling of tokamak plasmas is simultaneously an essential component of effective reactor design, and a great research barrier. Tokamak operational conditions exhibit comparatively low Knudsen numbers. Kinetic effects, including kinetic waves and instabilities, Landau damping, bump-on-tail instabilities and more, are therefore highly influential in tokamak plasma dynamics. Purely fluid models are inherently incapable of capturing these effects, whereas the high dimensionality in purely kinetic models render them practically intractable for most relevant purposes.

        We consider a $\delta\!f$ decomposition model, with a macroscopic fluid background and microscopic kinetic correction, both fully coupled to each other. A similar manner of discretization is proposed to that used in the recent \texttt{STRUPHY} code \cite{Holderied_Possanner_Wang_2021, Holderied_2022, Li_et_al_2023} with a finite-element model for the background and a pseudo-particle/PiC model for the correction.

        The fluid background satisfies the full, non-linear, resistive, compressible, Hall MHD equations. \cite{Laakmann_Hu_Farrell_2022} introduces finite-element(-in-space) implicit timesteppers for the incompressible analogue to this system with structure-preserving (SP) properties in the ideal case, alongside parameter-robust preconditioners. We show that these timesteppers can derive from a finite-element-in-time (FET) (and finite-element-in-space) interpretation. The benefits of this reformulation are discussed, including the derivation of timesteppers that are higher order in time, and the quantifiable dissipative SP properties in the non-ideal, resistive case.
        
        We discuss possible options for extending this FET approach to timesteppers for the compressible case.

        The kinetic corrections satisfy linearized Boltzmann equations. Using a Lénard--Bernstein collision operator, these take Fokker--Planck-like forms \cite{Fokker_1914, Planck_1917} wherein pseudo-particles in the numerical model obey the neoclassical transport equations, with particle-independent Brownian drift terms. This offers a rigorous methodology for incorporating collisions into the particle transport model, without coupling the equations of motions for each particle.
        
        Works by Chen, Chacón et al. \cite{Chen_Chacón_Barnes_2011, Chacón_Chen_Barnes_2013, Chen_Chacón_2014, Chen_Chacón_2015} have developed structure-preserving particle pushers for neoclassical transport in the Vlasov equations, derived from Crank--Nicolson integrators. We show these too can can derive from a FET interpretation, similarly offering potential extensions to higher-order-in-time particle pushers. The FET formulation is used also to consider how the stochastic drift terms can be incorporated into the pushers. Stochastic gyrokinetic expansions are also discussed.

        Different options for the numerical implementation of these schemes are considered.

        Due to the efficacy of FET in the development of SP timesteppers for both the fluid and kinetic component, we hope this approach will prove effective in the future for developing SP timesteppers for the full hybrid model. We hope this will give us the opportunity to incorporate previously inaccessible kinetic effects into the highly effective, modern, finite-element MHD models.
    \end{abstract}
    
    
    \newpage
    \tableofcontents
    
    
    \newpage
    \pagenumbering{arabic}
    %\linenumbers\renewcommand\thelinenumber{\color{black!50}\arabic{linenumber}}
            \input{0 - introduction/main.tex}
        \part{Research}
            \input{1 - low-noise PiC models/main.tex}
            \input{2 - kinetic component/main.tex}
            \input{3 - fluid component/main.tex}
            \input{4 - numerical implementation/main.tex}
        \part{Project Overview}
            \input{5 - research plan/main.tex}
            \input{6 - summary/main.tex}
    
    
    %\section{}
    \newpage
    \pagenumbering{gobble}
        \printbibliography


    \newpage
    \pagenumbering{roman}
    \appendix
        \part{Appendices}
            \input{8 - Hilbert complexes/main.tex}
            \input{9 - weak conservation proofs/main.tex}
\end{document}


\title{\BA{Title in Progress...}}
\author{Boris Andrews}
\affil{Mathematical Institute, University of Oxford}
\date{\today}


\begin{document}
    \pagenumbering{gobble}
    \maketitle
    
    
    \begin{abstract}
        Magnetic confinement reactors---in particular tokamaks---offer one of the most promising options for achieving practical nuclear fusion, with the potential to provide virtually limitless, clean energy. The theoretical and numerical modeling of tokamak plasmas is simultaneously an essential component of effective reactor design, and a great research barrier. Tokamak operational conditions exhibit comparatively low Knudsen numbers. Kinetic effects, including kinetic waves and instabilities, Landau damping, bump-on-tail instabilities and more, are therefore highly influential in tokamak plasma dynamics. Purely fluid models are inherently incapable of capturing these effects, whereas the high dimensionality in purely kinetic models render them practically intractable for most relevant purposes.

        We consider a $\delta\!f$ decomposition model, with a macroscopic fluid background and microscopic kinetic correction, both fully coupled to each other. A similar manner of discretization is proposed to that used in the recent \texttt{STRUPHY} code \cite{Holderied_Possanner_Wang_2021, Holderied_2022, Li_et_al_2023} with a finite-element model for the background and a pseudo-particle/PiC model for the correction.

        The fluid background satisfies the full, non-linear, resistive, compressible, Hall MHD equations. \cite{Laakmann_Hu_Farrell_2022} introduces finite-element(-in-space) implicit timesteppers for the incompressible analogue to this system with structure-preserving (SP) properties in the ideal case, alongside parameter-robust preconditioners. We show that these timesteppers can derive from a finite-element-in-time (FET) (and finite-element-in-space) interpretation. The benefits of this reformulation are discussed, including the derivation of timesteppers that are higher order in time, and the quantifiable dissipative SP properties in the non-ideal, resistive case.
        
        We discuss possible options for extending this FET approach to timesteppers for the compressible case.

        The kinetic corrections satisfy linearized Boltzmann equations. Using a Lénard--Bernstein collision operator, these take Fokker--Planck-like forms \cite{Fokker_1914, Planck_1917} wherein pseudo-particles in the numerical model obey the neoclassical transport equations, with particle-independent Brownian drift terms. This offers a rigorous methodology for incorporating collisions into the particle transport model, without coupling the equations of motions for each particle.
        
        Works by Chen, Chacón et al. \cite{Chen_Chacón_Barnes_2011, Chacón_Chen_Barnes_2013, Chen_Chacón_2014, Chen_Chacón_2015} have developed structure-preserving particle pushers for neoclassical transport in the Vlasov equations, derived from Crank--Nicolson integrators. We show these too can can derive from a FET interpretation, similarly offering potential extensions to higher-order-in-time particle pushers. The FET formulation is used also to consider how the stochastic drift terms can be incorporated into the pushers. Stochastic gyrokinetic expansions are also discussed.

        Different options for the numerical implementation of these schemes are considered.

        Due to the efficacy of FET in the development of SP timesteppers for both the fluid and kinetic component, we hope this approach will prove effective in the future for developing SP timesteppers for the full hybrid model. We hope this will give us the opportunity to incorporate previously inaccessible kinetic effects into the highly effective, modern, finite-element MHD models.
    \end{abstract}
    
    
    \newpage
    \tableofcontents
    
    
    \newpage
    \pagenumbering{arabic}
    %\linenumbers\renewcommand\thelinenumber{\color{black!50}\arabic{linenumber}}
            \documentclass[12pt, a4paper]{report}

\input{template/main.tex}

\title{\BA{Title in Progress...}}
\author{Boris Andrews}
\affil{Mathematical Institute, University of Oxford}
\date{\today}


\begin{document}
    \pagenumbering{gobble}
    \maketitle
    
    
    \begin{abstract}
        Magnetic confinement reactors---in particular tokamaks---offer one of the most promising options for achieving practical nuclear fusion, with the potential to provide virtually limitless, clean energy. The theoretical and numerical modeling of tokamak plasmas is simultaneously an essential component of effective reactor design, and a great research barrier. Tokamak operational conditions exhibit comparatively low Knudsen numbers. Kinetic effects, including kinetic waves and instabilities, Landau damping, bump-on-tail instabilities and more, are therefore highly influential in tokamak plasma dynamics. Purely fluid models are inherently incapable of capturing these effects, whereas the high dimensionality in purely kinetic models render them practically intractable for most relevant purposes.

        We consider a $\delta\!f$ decomposition model, with a macroscopic fluid background and microscopic kinetic correction, both fully coupled to each other. A similar manner of discretization is proposed to that used in the recent \texttt{STRUPHY} code \cite{Holderied_Possanner_Wang_2021, Holderied_2022, Li_et_al_2023} with a finite-element model for the background and a pseudo-particle/PiC model for the correction.

        The fluid background satisfies the full, non-linear, resistive, compressible, Hall MHD equations. \cite{Laakmann_Hu_Farrell_2022} introduces finite-element(-in-space) implicit timesteppers for the incompressible analogue to this system with structure-preserving (SP) properties in the ideal case, alongside parameter-robust preconditioners. We show that these timesteppers can derive from a finite-element-in-time (FET) (and finite-element-in-space) interpretation. The benefits of this reformulation are discussed, including the derivation of timesteppers that are higher order in time, and the quantifiable dissipative SP properties in the non-ideal, resistive case.
        
        We discuss possible options for extending this FET approach to timesteppers for the compressible case.

        The kinetic corrections satisfy linearized Boltzmann equations. Using a Lénard--Bernstein collision operator, these take Fokker--Planck-like forms \cite{Fokker_1914, Planck_1917} wherein pseudo-particles in the numerical model obey the neoclassical transport equations, with particle-independent Brownian drift terms. This offers a rigorous methodology for incorporating collisions into the particle transport model, without coupling the equations of motions for each particle.
        
        Works by Chen, Chacón et al. \cite{Chen_Chacón_Barnes_2011, Chacón_Chen_Barnes_2013, Chen_Chacón_2014, Chen_Chacón_2015} have developed structure-preserving particle pushers for neoclassical transport in the Vlasov equations, derived from Crank--Nicolson integrators. We show these too can can derive from a FET interpretation, similarly offering potential extensions to higher-order-in-time particle pushers. The FET formulation is used also to consider how the stochastic drift terms can be incorporated into the pushers. Stochastic gyrokinetic expansions are also discussed.

        Different options for the numerical implementation of these schemes are considered.

        Due to the efficacy of FET in the development of SP timesteppers for both the fluid and kinetic component, we hope this approach will prove effective in the future for developing SP timesteppers for the full hybrid model. We hope this will give us the opportunity to incorporate previously inaccessible kinetic effects into the highly effective, modern, finite-element MHD models.
    \end{abstract}
    
    
    \newpage
    \tableofcontents
    
    
    \newpage
    \pagenumbering{arabic}
    %\linenumbers\renewcommand\thelinenumber{\color{black!50}\arabic{linenumber}}
            \input{0 - introduction/main.tex}
        \part{Research}
            \input{1 - low-noise PiC models/main.tex}
            \input{2 - kinetic component/main.tex}
            \input{3 - fluid component/main.tex}
            \input{4 - numerical implementation/main.tex}
        \part{Project Overview}
            \input{5 - research plan/main.tex}
            \input{6 - summary/main.tex}
    
    
    %\section{}
    \newpage
    \pagenumbering{gobble}
        \printbibliography


    \newpage
    \pagenumbering{roman}
    \appendix
        \part{Appendices}
            \input{8 - Hilbert complexes/main.tex}
            \input{9 - weak conservation proofs/main.tex}
\end{document}

        \part{Research}
            \documentclass[12pt, a4paper]{report}

\input{template/main.tex}

\title{\BA{Title in Progress...}}
\author{Boris Andrews}
\affil{Mathematical Institute, University of Oxford}
\date{\today}


\begin{document}
    \pagenumbering{gobble}
    \maketitle
    
    
    \begin{abstract}
        Magnetic confinement reactors---in particular tokamaks---offer one of the most promising options for achieving practical nuclear fusion, with the potential to provide virtually limitless, clean energy. The theoretical and numerical modeling of tokamak plasmas is simultaneously an essential component of effective reactor design, and a great research barrier. Tokamak operational conditions exhibit comparatively low Knudsen numbers. Kinetic effects, including kinetic waves and instabilities, Landau damping, bump-on-tail instabilities and more, are therefore highly influential in tokamak plasma dynamics. Purely fluid models are inherently incapable of capturing these effects, whereas the high dimensionality in purely kinetic models render them practically intractable for most relevant purposes.

        We consider a $\delta\!f$ decomposition model, with a macroscopic fluid background and microscopic kinetic correction, both fully coupled to each other. A similar manner of discretization is proposed to that used in the recent \texttt{STRUPHY} code \cite{Holderied_Possanner_Wang_2021, Holderied_2022, Li_et_al_2023} with a finite-element model for the background and a pseudo-particle/PiC model for the correction.

        The fluid background satisfies the full, non-linear, resistive, compressible, Hall MHD equations. \cite{Laakmann_Hu_Farrell_2022} introduces finite-element(-in-space) implicit timesteppers for the incompressible analogue to this system with structure-preserving (SP) properties in the ideal case, alongside parameter-robust preconditioners. We show that these timesteppers can derive from a finite-element-in-time (FET) (and finite-element-in-space) interpretation. The benefits of this reformulation are discussed, including the derivation of timesteppers that are higher order in time, and the quantifiable dissipative SP properties in the non-ideal, resistive case.
        
        We discuss possible options for extending this FET approach to timesteppers for the compressible case.

        The kinetic corrections satisfy linearized Boltzmann equations. Using a Lénard--Bernstein collision operator, these take Fokker--Planck-like forms \cite{Fokker_1914, Planck_1917} wherein pseudo-particles in the numerical model obey the neoclassical transport equations, with particle-independent Brownian drift terms. This offers a rigorous methodology for incorporating collisions into the particle transport model, without coupling the equations of motions for each particle.
        
        Works by Chen, Chacón et al. \cite{Chen_Chacón_Barnes_2011, Chacón_Chen_Barnes_2013, Chen_Chacón_2014, Chen_Chacón_2015} have developed structure-preserving particle pushers for neoclassical transport in the Vlasov equations, derived from Crank--Nicolson integrators. We show these too can can derive from a FET interpretation, similarly offering potential extensions to higher-order-in-time particle pushers. The FET formulation is used also to consider how the stochastic drift terms can be incorporated into the pushers. Stochastic gyrokinetic expansions are also discussed.

        Different options for the numerical implementation of these schemes are considered.

        Due to the efficacy of FET in the development of SP timesteppers for both the fluid and kinetic component, we hope this approach will prove effective in the future for developing SP timesteppers for the full hybrid model. We hope this will give us the opportunity to incorporate previously inaccessible kinetic effects into the highly effective, modern, finite-element MHD models.
    \end{abstract}
    
    
    \newpage
    \tableofcontents
    
    
    \newpage
    \pagenumbering{arabic}
    %\linenumbers\renewcommand\thelinenumber{\color{black!50}\arabic{linenumber}}
            \input{0 - introduction/main.tex}
        \part{Research}
            \input{1 - low-noise PiC models/main.tex}
            \input{2 - kinetic component/main.tex}
            \input{3 - fluid component/main.tex}
            \input{4 - numerical implementation/main.tex}
        \part{Project Overview}
            \input{5 - research plan/main.tex}
            \input{6 - summary/main.tex}
    
    
    %\section{}
    \newpage
    \pagenumbering{gobble}
        \printbibliography


    \newpage
    \pagenumbering{roman}
    \appendix
        \part{Appendices}
            \input{8 - Hilbert complexes/main.tex}
            \input{9 - weak conservation proofs/main.tex}
\end{document}

            \documentclass[12pt, a4paper]{report}

\input{template/main.tex}

\title{\BA{Title in Progress...}}
\author{Boris Andrews}
\affil{Mathematical Institute, University of Oxford}
\date{\today}


\begin{document}
    \pagenumbering{gobble}
    \maketitle
    
    
    \begin{abstract}
        Magnetic confinement reactors---in particular tokamaks---offer one of the most promising options for achieving practical nuclear fusion, with the potential to provide virtually limitless, clean energy. The theoretical and numerical modeling of tokamak plasmas is simultaneously an essential component of effective reactor design, and a great research barrier. Tokamak operational conditions exhibit comparatively low Knudsen numbers. Kinetic effects, including kinetic waves and instabilities, Landau damping, bump-on-tail instabilities and more, are therefore highly influential in tokamak plasma dynamics. Purely fluid models are inherently incapable of capturing these effects, whereas the high dimensionality in purely kinetic models render them practically intractable for most relevant purposes.

        We consider a $\delta\!f$ decomposition model, with a macroscopic fluid background and microscopic kinetic correction, both fully coupled to each other. A similar manner of discretization is proposed to that used in the recent \texttt{STRUPHY} code \cite{Holderied_Possanner_Wang_2021, Holderied_2022, Li_et_al_2023} with a finite-element model for the background and a pseudo-particle/PiC model for the correction.

        The fluid background satisfies the full, non-linear, resistive, compressible, Hall MHD equations. \cite{Laakmann_Hu_Farrell_2022} introduces finite-element(-in-space) implicit timesteppers for the incompressible analogue to this system with structure-preserving (SP) properties in the ideal case, alongside parameter-robust preconditioners. We show that these timesteppers can derive from a finite-element-in-time (FET) (and finite-element-in-space) interpretation. The benefits of this reformulation are discussed, including the derivation of timesteppers that are higher order in time, and the quantifiable dissipative SP properties in the non-ideal, resistive case.
        
        We discuss possible options for extending this FET approach to timesteppers for the compressible case.

        The kinetic corrections satisfy linearized Boltzmann equations. Using a Lénard--Bernstein collision operator, these take Fokker--Planck-like forms \cite{Fokker_1914, Planck_1917} wherein pseudo-particles in the numerical model obey the neoclassical transport equations, with particle-independent Brownian drift terms. This offers a rigorous methodology for incorporating collisions into the particle transport model, without coupling the equations of motions for each particle.
        
        Works by Chen, Chacón et al. \cite{Chen_Chacón_Barnes_2011, Chacón_Chen_Barnes_2013, Chen_Chacón_2014, Chen_Chacón_2015} have developed structure-preserving particle pushers for neoclassical transport in the Vlasov equations, derived from Crank--Nicolson integrators. We show these too can can derive from a FET interpretation, similarly offering potential extensions to higher-order-in-time particle pushers. The FET formulation is used also to consider how the stochastic drift terms can be incorporated into the pushers. Stochastic gyrokinetic expansions are also discussed.

        Different options for the numerical implementation of these schemes are considered.

        Due to the efficacy of FET in the development of SP timesteppers for both the fluid and kinetic component, we hope this approach will prove effective in the future for developing SP timesteppers for the full hybrid model. We hope this will give us the opportunity to incorporate previously inaccessible kinetic effects into the highly effective, modern, finite-element MHD models.
    \end{abstract}
    
    
    \newpage
    \tableofcontents
    
    
    \newpage
    \pagenumbering{arabic}
    %\linenumbers\renewcommand\thelinenumber{\color{black!50}\arabic{linenumber}}
            \input{0 - introduction/main.tex}
        \part{Research}
            \input{1 - low-noise PiC models/main.tex}
            \input{2 - kinetic component/main.tex}
            \input{3 - fluid component/main.tex}
            \input{4 - numerical implementation/main.tex}
        \part{Project Overview}
            \input{5 - research plan/main.tex}
            \input{6 - summary/main.tex}
    
    
    %\section{}
    \newpage
    \pagenumbering{gobble}
        \printbibliography


    \newpage
    \pagenumbering{roman}
    \appendix
        \part{Appendices}
            \input{8 - Hilbert complexes/main.tex}
            \input{9 - weak conservation proofs/main.tex}
\end{document}

            \documentclass[12pt, a4paper]{report}

\input{template/main.tex}

\title{\BA{Title in Progress...}}
\author{Boris Andrews}
\affil{Mathematical Institute, University of Oxford}
\date{\today}


\begin{document}
    \pagenumbering{gobble}
    \maketitle
    
    
    \begin{abstract}
        Magnetic confinement reactors---in particular tokamaks---offer one of the most promising options for achieving practical nuclear fusion, with the potential to provide virtually limitless, clean energy. The theoretical and numerical modeling of tokamak plasmas is simultaneously an essential component of effective reactor design, and a great research barrier. Tokamak operational conditions exhibit comparatively low Knudsen numbers. Kinetic effects, including kinetic waves and instabilities, Landau damping, bump-on-tail instabilities and more, are therefore highly influential in tokamak plasma dynamics. Purely fluid models are inherently incapable of capturing these effects, whereas the high dimensionality in purely kinetic models render them practically intractable for most relevant purposes.

        We consider a $\delta\!f$ decomposition model, with a macroscopic fluid background and microscopic kinetic correction, both fully coupled to each other. A similar manner of discretization is proposed to that used in the recent \texttt{STRUPHY} code \cite{Holderied_Possanner_Wang_2021, Holderied_2022, Li_et_al_2023} with a finite-element model for the background and a pseudo-particle/PiC model for the correction.

        The fluid background satisfies the full, non-linear, resistive, compressible, Hall MHD equations. \cite{Laakmann_Hu_Farrell_2022} introduces finite-element(-in-space) implicit timesteppers for the incompressible analogue to this system with structure-preserving (SP) properties in the ideal case, alongside parameter-robust preconditioners. We show that these timesteppers can derive from a finite-element-in-time (FET) (and finite-element-in-space) interpretation. The benefits of this reformulation are discussed, including the derivation of timesteppers that are higher order in time, and the quantifiable dissipative SP properties in the non-ideal, resistive case.
        
        We discuss possible options for extending this FET approach to timesteppers for the compressible case.

        The kinetic corrections satisfy linearized Boltzmann equations. Using a Lénard--Bernstein collision operator, these take Fokker--Planck-like forms \cite{Fokker_1914, Planck_1917} wherein pseudo-particles in the numerical model obey the neoclassical transport equations, with particle-independent Brownian drift terms. This offers a rigorous methodology for incorporating collisions into the particle transport model, without coupling the equations of motions for each particle.
        
        Works by Chen, Chacón et al. \cite{Chen_Chacón_Barnes_2011, Chacón_Chen_Barnes_2013, Chen_Chacón_2014, Chen_Chacón_2015} have developed structure-preserving particle pushers for neoclassical transport in the Vlasov equations, derived from Crank--Nicolson integrators. We show these too can can derive from a FET interpretation, similarly offering potential extensions to higher-order-in-time particle pushers. The FET formulation is used also to consider how the stochastic drift terms can be incorporated into the pushers. Stochastic gyrokinetic expansions are also discussed.

        Different options for the numerical implementation of these schemes are considered.

        Due to the efficacy of FET in the development of SP timesteppers for both the fluid and kinetic component, we hope this approach will prove effective in the future for developing SP timesteppers for the full hybrid model. We hope this will give us the opportunity to incorporate previously inaccessible kinetic effects into the highly effective, modern, finite-element MHD models.
    \end{abstract}
    
    
    \newpage
    \tableofcontents
    
    
    \newpage
    \pagenumbering{arabic}
    %\linenumbers\renewcommand\thelinenumber{\color{black!50}\arabic{linenumber}}
            \input{0 - introduction/main.tex}
        \part{Research}
            \input{1 - low-noise PiC models/main.tex}
            \input{2 - kinetic component/main.tex}
            \input{3 - fluid component/main.tex}
            \input{4 - numerical implementation/main.tex}
        \part{Project Overview}
            \input{5 - research plan/main.tex}
            \input{6 - summary/main.tex}
    
    
    %\section{}
    \newpage
    \pagenumbering{gobble}
        \printbibliography


    \newpage
    \pagenumbering{roman}
    \appendix
        \part{Appendices}
            \input{8 - Hilbert complexes/main.tex}
            \input{9 - weak conservation proofs/main.tex}
\end{document}

            \documentclass[12pt, a4paper]{report}

\input{template/main.tex}

\title{\BA{Title in Progress...}}
\author{Boris Andrews}
\affil{Mathematical Institute, University of Oxford}
\date{\today}


\begin{document}
    \pagenumbering{gobble}
    \maketitle
    
    
    \begin{abstract}
        Magnetic confinement reactors---in particular tokamaks---offer one of the most promising options for achieving practical nuclear fusion, with the potential to provide virtually limitless, clean energy. The theoretical and numerical modeling of tokamak plasmas is simultaneously an essential component of effective reactor design, and a great research barrier. Tokamak operational conditions exhibit comparatively low Knudsen numbers. Kinetic effects, including kinetic waves and instabilities, Landau damping, bump-on-tail instabilities and more, are therefore highly influential in tokamak plasma dynamics. Purely fluid models are inherently incapable of capturing these effects, whereas the high dimensionality in purely kinetic models render them practically intractable for most relevant purposes.

        We consider a $\delta\!f$ decomposition model, with a macroscopic fluid background and microscopic kinetic correction, both fully coupled to each other. A similar manner of discretization is proposed to that used in the recent \texttt{STRUPHY} code \cite{Holderied_Possanner_Wang_2021, Holderied_2022, Li_et_al_2023} with a finite-element model for the background and a pseudo-particle/PiC model for the correction.

        The fluid background satisfies the full, non-linear, resistive, compressible, Hall MHD equations. \cite{Laakmann_Hu_Farrell_2022} introduces finite-element(-in-space) implicit timesteppers for the incompressible analogue to this system with structure-preserving (SP) properties in the ideal case, alongside parameter-robust preconditioners. We show that these timesteppers can derive from a finite-element-in-time (FET) (and finite-element-in-space) interpretation. The benefits of this reformulation are discussed, including the derivation of timesteppers that are higher order in time, and the quantifiable dissipative SP properties in the non-ideal, resistive case.
        
        We discuss possible options for extending this FET approach to timesteppers for the compressible case.

        The kinetic corrections satisfy linearized Boltzmann equations. Using a Lénard--Bernstein collision operator, these take Fokker--Planck-like forms \cite{Fokker_1914, Planck_1917} wherein pseudo-particles in the numerical model obey the neoclassical transport equations, with particle-independent Brownian drift terms. This offers a rigorous methodology for incorporating collisions into the particle transport model, without coupling the equations of motions for each particle.
        
        Works by Chen, Chacón et al. \cite{Chen_Chacón_Barnes_2011, Chacón_Chen_Barnes_2013, Chen_Chacón_2014, Chen_Chacón_2015} have developed structure-preserving particle pushers for neoclassical transport in the Vlasov equations, derived from Crank--Nicolson integrators. We show these too can can derive from a FET interpretation, similarly offering potential extensions to higher-order-in-time particle pushers. The FET formulation is used also to consider how the stochastic drift terms can be incorporated into the pushers. Stochastic gyrokinetic expansions are also discussed.

        Different options for the numerical implementation of these schemes are considered.

        Due to the efficacy of FET in the development of SP timesteppers for both the fluid and kinetic component, we hope this approach will prove effective in the future for developing SP timesteppers for the full hybrid model. We hope this will give us the opportunity to incorporate previously inaccessible kinetic effects into the highly effective, modern, finite-element MHD models.
    \end{abstract}
    
    
    \newpage
    \tableofcontents
    
    
    \newpage
    \pagenumbering{arabic}
    %\linenumbers\renewcommand\thelinenumber{\color{black!50}\arabic{linenumber}}
            \input{0 - introduction/main.tex}
        \part{Research}
            \input{1 - low-noise PiC models/main.tex}
            \input{2 - kinetic component/main.tex}
            \input{3 - fluid component/main.tex}
            \input{4 - numerical implementation/main.tex}
        \part{Project Overview}
            \input{5 - research plan/main.tex}
            \input{6 - summary/main.tex}
    
    
    %\section{}
    \newpage
    \pagenumbering{gobble}
        \printbibliography


    \newpage
    \pagenumbering{roman}
    \appendix
        \part{Appendices}
            \input{8 - Hilbert complexes/main.tex}
            \input{9 - weak conservation proofs/main.tex}
\end{document}

        \part{Project Overview}
            \documentclass[12pt, a4paper]{report}

\input{template/main.tex}

\title{\BA{Title in Progress...}}
\author{Boris Andrews}
\affil{Mathematical Institute, University of Oxford}
\date{\today}


\begin{document}
    \pagenumbering{gobble}
    \maketitle
    
    
    \begin{abstract}
        Magnetic confinement reactors---in particular tokamaks---offer one of the most promising options for achieving practical nuclear fusion, with the potential to provide virtually limitless, clean energy. The theoretical and numerical modeling of tokamak plasmas is simultaneously an essential component of effective reactor design, and a great research barrier. Tokamak operational conditions exhibit comparatively low Knudsen numbers. Kinetic effects, including kinetic waves and instabilities, Landau damping, bump-on-tail instabilities and more, are therefore highly influential in tokamak plasma dynamics. Purely fluid models are inherently incapable of capturing these effects, whereas the high dimensionality in purely kinetic models render them practically intractable for most relevant purposes.

        We consider a $\delta\!f$ decomposition model, with a macroscopic fluid background and microscopic kinetic correction, both fully coupled to each other. A similar manner of discretization is proposed to that used in the recent \texttt{STRUPHY} code \cite{Holderied_Possanner_Wang_2021, Holderied_2022, Li_et_al_2023} with a finite-element model for the background and a pseudo-particle/PiC model for the correction.

        The fluid background satisfies the full, non-linear, resistive, compressible, Hall MHD equations. \cite{Laakmann_Hu_Farrell_2022} introduces finite-element(-in-space) implicit timesteppers for the incompressible analogue to this system with structure-preserving (SP) properties in the ideal case, alongside parameter-robust preconditioners. We show that these timesteppers can derive from a finite-element-in-time (FET) (and finite-element-in-space) interpretation. The benefits of this reformulation are discussed, including the derivation of timesteppers that are higher order in time, and the quantifiable dissipative SP properties in the non-ideal, resistive case.
        
        We discuss possible options for extending this FET approach to timesteppers for the compressible case.

        The kinetic corrections satisfy linearized Boltzmann equations. Using a Lénard--Bernstein collision operator, these take Fokker--Planck-like forms \cite{Fokker_1914, Planck_1917} wherein pseudo-particles in the numerical model obey the neoclassical transport equations, with particle-independent Brownian drift terms. This offers a rigorous methodology for incorporating collisions into the particle transport model, without coupling the equations of motions for each particle.
        
        Works by Chen, Chacón et al. \cite{Chen_Chacón_Barnes_2011, Chacón_Chen_Barnes_2013, Chen_Chacón_2014, Chen_Chacón_2015} have developed structure-preserving particle pushers for neoclassical transport in the Vlasov equations, derived from Crank--Nicolson integrators. We show these too can can derive from a FET interpretation, similarly offering potential extensions to higher-order-in-time particle pushers. The FET formulation is used also to consider how the stochastic drift terms can be incorporated into the pushers. Stochastic gyrokinetic expansions are also discussed.

        Different options for the numerical implementation of these schemes are considered.

        Due to the efficacy of FET in the development of SP timesteppers for both the fluid and kinetic component, we hope this approach will prove effective in the future for developing SP timesteppers for the full hybrid model. We hope this will give us the opportunity to incorporate previously inaccessible kinetic effects into the highly effective, modern, finite-element MHD models.
    \end{abstract}
    
    
    \newpage
    \tableofcontents
    
    
    \newpage
    \pagenumbering{arabic}
    %\linenumbers\renewcommand\thelinenumber{\color{black!50}\arabic{linenumber}}
            \input{0 - introduction/main.tex}
        \part{Research}
            \input{1 - low-noise PiC models/main.tex}
            \input{2 - kinetic component/main.tex}
            \input{3 - fluid component/main.tex}
            \input{4 - numerical implementation/main.tex}
        \part{Project Overview}
            \input{5 - research plan/main.tex}
            \input{6 - summary/main.tex}
    
    
    %\section{}
    \newpage
    \pagenumbering{gobble}
        \printbibliography


    \newpage
    \pagenumbering{roman}
    \appendix
        \part{Appendices}
            \input{8 - Hilbert complexes/main.tex}
            \input{9 - weak conservation proofs/main.tex}
\end{document}

            \documentclass[12pt, a4paper]{report}

\input{template/main.tex}

\title{\BA{Title in Progress...}}
\author{Boris Andrews}
\affil{Mathematical Institute, University of Oxford}
\date{\today}


\begin{document}
    \pagenumbering{gobble}
    \maketitle
    
    
    \begin{abstract}
        Magnetic confinement reactors---in particular tokamaks---offer one of the most promising options for achieving practical nuclear fusion, with the potential to provide virtually limitless, clean energy. The theoretical and numerical modeling of tokamak plasmas is simultaneously an essential component of effective reactor design, and a great research barrier. Tokamak operational conditions exhibit comparatively low Knudsen numbers. Kinetic effects, including kinetic waves and instabilities, Landau damping, bump-on-tail instabilities and more, are therefore highly influential in tokamak plasma dynamics. Purely fluid models are inherently incapable of capturing these effects, whereas the high dimensionality in purely kinetic models render them practically intractable for most relevant purposes.

        We consider a $\delta\!f$ decomposition model, with a macroscopic fluid background and microscopic kinetic correction, both fully coupled to each other. A similar manner of discretization is proposed to that used in the recent \texttt{STRUPHY} code \cite{Holderied_Possanner_Wang_2021, Holderied_2022, Li_et_al_2023} with a finite-element model for the background and a pseudo-particle/PiC model for the correction.

        The fluid background satisfies the full, non-linear, resistive, compressible, Hall MHD equations. \cite{Laakmann_Hu_Farrell_2022} introduces finite-element(-in-space) implicit timesteppers for the incompressible analogue to this system with structure-preserving (SP) properties in the ideal case, alongside parameter-robust preconditioners. We show that these timesteppers can derive from a finite-element-in-time (FET) (and finite-element-in-space) interpretation. The benefits of this reformulation are discussed, including the derivation of timesteppers that are higher order in time, and the quantifiable dissipative SP properties in the non-ideal, resistive case.
        
        We discuss possible options for extending this FET approach to timesteppers for the compressible case.

        The kinetic corrections satisfy linearized Boltzmann equations. Using a Lénard--Bernstein collision operator, these take Fokker--Planck-like forms \cite{Fokker_1914, Planck_1917} wherein pseudo-particles in the numerical model obey the neoclassical transport equations, with particle-independent Brownian drift terms. This offers a rigorous methodology for incorporating collisions into the particle transport model, without coupling the equations of motions for each particle.
        
        Works by Chen, Chacón et al. \cite{Chen_Chacón_Barnes_2011, Chacón_Chen_Barnes_2013, Chen_Chacón_2014, Chen_Chacón_2015} have developed structure-preserving particle pushers for neoclassical transport in the Vlasov equations, derived from Crank--Nicolson integrators. We show these too can can derive from a FET interpretation, similarly offering potential extensions to higher-order-in-time particle pushers. The FET formulation is used also to consider how the stochastic drift terms can be incorporated into the pushers. Stochastic gyrokinetic expansions are also discussed.

        Different options for the numerical implementation of these schemes are considered.

        Due to the efficacy of FET in the development of SP timesteppers for both the fluid and kinetic component, we hope this approach will prove effective in the future for developing SP timesteppers for the full hybrid model. We hope this will give us the opportunity to incorporate previously inaccessible kinetic effects into the highly effective, modern, finite-element MHD models.
    \end{abstract}
    
    
    \newpage
    \tableofcontents
    
    
    \newpage
    \pagenumbering{arabic}
    %\linenumbers\renewcommand\thelinenumber{\color{black!50}\arabic{linenumber}}
            \input{0 - introduction/main.tex}
        \part{Research}
            \input{1 - low-noise PiC models/main.tex}
            \input{2 - kinetic component/main.tex}
            \input{3 - fluid component/main.tex}
            \input{4 - numerical implementation/main.tex}
        \part{Project Overview}
            \input{5 - research plan/main.tex}
            \input{6 - summary/main.tex}
    
    
    %\section{}
    \newpage
    \pagenumbering{gobble}
        \printbibliography


    \newpage
    \pagenumbering{roman}
    \appendix
        \part{Appendices}
            \input{8 - Hilbert complexes/main.tex}
            \input{9 - weak conservation proofs/main.tex}
\end{document}

    
    
    %\section{}
    \newpage
    \pagenumbering{gobble}
        \printbibliography


    \newpage
    \pagenumbering{roman}
    \appendix
        \part{Appendices}
            \documentclass[12pt, a4paper]{report}

\input{template/main.tex}

\title{\BA{Title in Progress...}}
\author{Boris Andrews}
\affil{Mathematical Institute, University of Oxford}
\date{\today}


\begin{document}
    \pagenumbering{gobble}
    \maketitle
    
    
    \begin{abstract}
        Magnetic confinement reactors---in particular tokamaks---offer one of the most promising options for achieving practical nuclear fusion, with the potential to provide virtually limitless, clean energy. The theoretical and numerical modeling of tokamak plasmas is simultaneously an essential component of effective reactor design, and a great research barrier. Tokamak operational conditions exhibit comparatively low Knudsen numbers. Kinetic effects, including kinetic waves and instabilities, Landau damping, bump-on-tail instabilities and more, are therefore highly influential in tokamak plasma dynamics. Purely fluid models are inherently incapable of capturing these effects, whereas the high dimensionality in purely kinetic models render them practically intractable for most relevant purposes.

        We consider a $\delta\!f$ decomposition model, with a macroscopic fluid background and microscopic kinetic correction, both fully coupled to each other. A similar manner of discretization is proposed to that used in the recent \texttt{STRUPHY} code \cite{Holderied_Possanner_Wang_2021, Holderied_2022, Li_et_al_2023} with a finite-element model for the background and a pseudo-particle/PiC model for the correction.

        The fluid background satisfies the full, non-linear, resistive, compressible, Hall MHD equations. \cite{Laakmann_Hu_Farrell_2022} introduces finite-element(-in-space) implicit timesteppers for the incompressible analogue to this system with structure-preserving (SP) properties in the ideal case, alongside parameter-robust preconditioners. We show that these timesteppers can derive from a finite-element-in-time (FET) (and finite-element-in-space) interpretation. The benefits of this reformulation are discussed, including the derivation of timesteppers that are higher order in time, and the quantifiable dissipative SP properties in the non-ideal, resistive case.
        
        We discuss possible options for extending this FET approach to timesteppers for the compressible case.

        The kinetic corrections satisfy linearized Boltzmann equations. Using a Lénard--Bernstein collision operator, these take Fokker--Planck-like forms \cite{Fokker_1914, Planck_1917} wherein pseudo-particles in the numerical model obey the neoclassical transport equations, with particle-independent Brownian drift terms. This offers a rigorous methodology for incorporating collisions into the particle transport model, without coupling the equations of motions for each particle.
        
        Works by Chen, Chacón et al. \cite{Chen_Chacón_Barnes_2011, Chacón_Chen_Barnes_2013, Chen_Chacón_2014, Chen_Chacón_2015} have developed structure-preserving particle pushers for neoclassical transport in the Vlasov equations, derived from Crank--Nicolson integrators. We show these too can can derive from a FET interpretation, similarly offering potential extensions to higher-order-in-time particle pushers. The FET formulation is used also to consider how the stochastic drift terms can be incorporated into the pushers. Stochastic gyrokinetic expansions are also discussed.

        Different options for the numerical implementation of these schemes are considered.

        Due to the efficacy of FET in the development of SP timesteppers for both the fluid and kinetic component, we hope this approach will prove effective in the future for developing SP timesteppers for the full hybrid model. We hope this will give us the opportunity to incorporate previously inaccessible kinetic effects into the highly effective, modern, finite-element MHD models.
    \end{abstract}
    
    
    \newpage
    \tableofcontents
    
    
    \newpage
    \pagenumbering{arabic}
    %\linenumbers\renewcommand\thelinenumber{\color{black!50}\arabic{linenumber}}
            \input{0 - introduction/main.tex}
        \part{Research}
            \input{1 - low-noise PiC models/main.tex}
            \input{2 - kinetic component/main.tex}
            \input{3 - fluid component/main.tex}
            \input{4 - numerical implementation/main.tex}
        \part{Project Overview}
            \input{5 - research plan/main.tex}
            \input{6 - summary/main.tex}
    
    
    %\section{}
    \newpage
    \pagenumbering{gobble}
        \printbibliography


    \newpage
    \pagenumbering{roman}
    \appendix
        \part{Appendices}
            \input{8 - Hilbert complexes/main.tex}
            \input{9 - weak conservation proofs/main.tex}
\end{document}

            \documentclass[12pt, a4paper]{report}

\input{template/main.tex}

\title{\BA{Title in Progress...}}
\author{Boris Andrews}
\affil{Mathematical Institute, University of Oxford}
\date{\today}


\begin{document}
    \pagenumbering{gobble}
    \maketitle
    
    
    \begin{abstract}
        Magnetic confinement reactors---in particular tokamaks---offer one of the most promising options for achieving practical nuclear fusion, with the potential to provide virtually limitless, clean energy. The theoretical and numerical modeling of tokamak plasmas is simultaneously an essential component of effective reactor design, and a great research barrier. Tokamak operational conditions exhibit comparatively low Knudsen numbers. Kinetic effects, including kinetic waves and instabilities, Landau damping, bump-on-tail instabilities and more, are therefore highly influential in tokamak plasma dynamics. Purely fluid models are inherently incapable of capturing these effects, whereas the high dimensionality in purely kinetic models render them practically intractable for most relevant purposes.

        We consider a $\delta\!f$ decomposition model, with a macroscopic fluid background and microscopic kinetic correction, both fully coupled to each other. A similar manner of discretization is proposed to that used in the recent \texttt{STRUPHY} code \cite{Holderied_Possanner_Wang_2021, Holderied_2022, Li_et_al_2023} with a finite-element model for the background and a pseudo-particle/PiC model for the correction.

        The fluid background satisfies the full, non-linear, resistive, compressible, Hall MHD equations. \cite{Laakmann_Hu_Farrell_2022} introduces finite-element(-in-space) implicit timesteppers for the incompressible analogue to this system with structure-preserving (SP) properties in the ideal case, alongside parameter-robust preconditioners. We show that these timesteppers can derive from a finite-element-in-time (FET) (and finite-element-in-space) interpretation. The benefits of this reformulation are discussed, including the derivation of timesteppers that are higher order in time, and the quantifiable dissipative SP properties in the non-ideal, resistive case.
        
        We discuss possible options for extending this FET approach to timesteppers for the compressible case.

        The kinetic corrections satisfy linearized Boltzmann equations. Using a Lénard--Bernstein collision operator, these take Fokker--Planck-like forms \cite{Fokker_1914, Planck_1917} wherein pseudo-particles in the numerical model obey the neoclassical transport equations, with particle-independent Brownian drift terms. This offers a rigorous methodology for incorporating collisions into the particle transport model, without coupling the equations of motions for each particle.
        
        Works by Chen, Chacón et al. \cite{Chen_Chacón_Barnes_2011, Chacón_Chen_Barnes_2013, Chen_Chacón_2014, Chen_Chacón_2015} have developed structure-preserving particle pushers for neoclassical transport in the Vlasov equations, derived from Crank--Nicolson integrators. We show these too can can derive from a FET interpretation, similarly offering potential extensions to higher-order-in-time particle pushers. The FET formulation is used also to consider how the stochastic drift terms can be incorporated into the pushers. Stochastic gyrokinetic expansions are also discussed.

        Different options for the numerical implementation of these schemes are considered.

        Due to the efficacy of FET in the development of SP timesteppers for both the fluid and kinetic component, we hope this approach will prove effective in the future for developing SP timesteppers for the full hybrid model. We hope this will give us the opportunity to incorporate previously inaccessible kinetic effects into the highly effective, modern, finite-element MHD models.
    \end{abstract}
    
    
    \newpage
    \tableofcontents
    
    
    \newpage
    \pagenumbering{arabic}
    %\linenumbers\renewcommand\thelinenumber{\color{black!50}\arabic{linenumber}}
            \input{0 - introduction/main.tex}
        \part{Research}
            \input{1 - low-noise PiC models/main.tex}
            \input{2 - kinetic component/main.tex}
            \input{3 - fluid component/main.tex}
            \input{4 - numerical implementation/main.tex}
        \part{Project Overview}
            \input{5 - research plan/main.tex}
            \input{6 - summary/main.tex}
    
    
    %\section{}
    \newpage
    \pagenumbering{gobble}
        \printbibliography


    \newpage
    \pagenumbering{roman}
    \appendix
        \part{Appendices}
            \input{8 - Hilbert complexes/main.tex}
            \input{9 - weak conservation proofs/main.tex}
\end{document}

\end{document}

    
    
    %\section{}
    \newpage
    \pagenumbering{gobble}
        \printbibliography


    \newpage
    \pagenumbering{roman}
    \appendix
        \part{Appendices}
            \documentclass[12pt, a4paper]{report}

\documentclass[12pt, a4paper]{report}

\input{template/main.tex}

\title{\BA{Title in Progress...}}
\author{Boris Andrews}
\affil{Mathematical Institute, University of Oxford}
\date{\today}


\begin{document}
    \pagenumbering{gobble}
    \maketitle
    
    
    \begin{abstract}
        Magnetic confinement reactors---in particular tokamaks---offer one of the most promising options for achieving practical nuclear fusion, with the potential to provide virtually limitless, clean energy. The theoretical and numerical modeling of tokamak plasmas is simultaneously an essential component of effective reactor design, and a great research barrier. Tokamak operational conditions exhibit comparatively low Knudsen numbers. Kinetic effects, including kinetic waves and instabilities, Landau damping, bump-on-tail instabilities and more, are therefore highly influential in tokamak plasma dynamics. Purely fluid models are inherently incapable of capturing these effects, whereas the high dimensionality in purely kinetic models render them practically intractable for most relevant purposes.

        We consider a $\delta\!f$ decomposition model, with a macroscopic fluid background and microscopic kinetic correction, both fully coupled to each other. A similar manner of discretization is proposed to that used in the recent \texttt{STRUPHY} code \cite{Holderied_Possanner_Wang_2021, Holderied_2022, Li_et_al_2023} with a finite-element model for the background and a pseudo-particle/PiC model for the correction.

        The fluid background satisfies the full, non-linear, resistive, compressible, Hall MHD equations. \cite{Laakmann_Hu_Farrell_2022} introduces finite-element(-in-space) implicit timesteppers for the incompressible analogue to this system with structure-preserving (SP) properties in the ideal case, alongside parameter-robust preconditioners. We show that these timesteppers can derive from a finite-element-in-time (FET) (and finite-element-in-space) interpretation. The benefits of this reformulation are discussed, including the derivation of timesteppers that are higher order in time, and the quantifiable dissipative SP properties in the non-ideal, resistive case.
        
        We discuss possible options for extending this FET approach to timesteppers for the compressible case.

        The kinetic corrections satisfy linearized Boltzmann equations. Using a Lénard--Bernstein collision operator, these take Fokker--Planck-like forms \cite{Fokker_1914, Planck_1917} wherein pseudo-particles in the numerical model obey the neoclassical transport equations, with particle-independent Brownian drift terms. This offers a rigorous methodology for incorporating collisions into the particle transport model, without coupling the equations of motions for each particle.
        
        Works by Chen, Chacón et al. \cite{Chen_Chacón_Barnes_2011, Chacón_Chen_Barnes_2013, Chen_Chacón_2014, Chen_Chacón_2015} have developed structure-preserving particle pushers for neoclassical transport in the Vlasov equations, derived from Crank--Nicolson integrators. We show these too can can derive from a FET interpretation, similarly offering potential extensions to higher-order-in-time particle pushers. The FET formulation is used also to consider how the stochastic drift terms can be incorporated into the pushers. Stochastic gyrokinetic expansions are also discussed.

        Different options for the numerical implementation of these schemes are considered.

        Due to the efficacy of FET in the development of SP timesteppers for both the fluid and kinetic component, we hope this approach will prove effective in the future for developing SP timesteppers for the full hybrid model. We hope this will give us the opportunity to incorporate previously inaccessible kinetic effects into the highly effective, modern, finite-element MHD models.
    \end{abstract}
    
    
    \newpage
    \tableofcontents
    
    
    \newpage
    \pagenumbering{arabic}
    %\linenumbers\renewcommand\thelinenumber{\color{black!50}\arabic{linenumber}}
            \input{0 - introduction/main.tex}
        \part{Research}
            \input{1 - low-noise PiC models/main.tex}
            \input{2 - kinetic component/main.tex}
            \input{3 - fluid component/main.tex}
            \input{4 - numerical implementation/main.tex}
        \part{Project Overview}
            \input{5 - research plan/main.tex}
            \input{6 - summary/main.tex}
    
    
    %\section{}
    \newpage
    \pagenumbering{gobble}
        \printbibliography


    \newpage
    \pagenumbering{roman}
    \appendix
        \part{Appendices}
            \input{8 - Hilbert complexes/main.tex}
            \input{9 - weak conservation proofs/main.tex}
\end{document}


\title{\BA{Title in Progress...}}
\author{Boris Andrews}
\affil{Mathematical Institute, University of Oxford}
\date{\today}


\begin{document}
    \pagenumbering{gobble}
    \maketitle
    
    
    \begin{abstract}
        Magnetic confinement reactors---in particular tokamaks---offer one of the most promising options for achieving practical nuclear fusion, with the potential to provide virtually limitless, clean energy. The theoretical and numerical modeling of tokamak plasmas is simultaneously an essential component of effective reactor design, and a great research barrier. Tokamak operational conditions exhibit comparatively low Knudsen numbers. Kinetic effects, including kinetic waves and instabilities, Landau damping, bump-on-tail instabilities and more, are therefore highly influential in tokamak plasma dynamics. Purely fluid models are inherently incapable of capturing these effects, whereas the high dimensionality in purely kinetic models render them practically intractable for most relevant purposes.

        We consider a $\delta\!f$ decomposition model, with a macroscopic fluid background and microscopic kinetic correction, both fully coupled to each other. A similar manner of discretization is proposed to that used in the recent \texttt{STRUPHY} code \cite{Holderied_Possanner_Wang_2021, Holderied_2022, Li_et_al_2023} with a finite-element model for the background and a pseudo-particle/PiC model for the correction.

        The fluid background satisfies the full, non-linear, resistive, compressible, Hall MHD equations. \cite{Laakmann_Hu_Farrell_2022} introduces finite-element(-in-space) implicit timesteppers for the incompressible analogue to this system with structure-preserving (SP) properties in the ideal case, alongside parameter-robust preconditioners. We show that these timesteppers can derive from a finite-element-in-time (FET) (and finite-element-in-space) interpretation. The benefits of this reformulation are discussed, including the derivation of timesteppers that are higher order in time, and the quantifiable dissipative SP properties in the non-ideal, resistive case.
        
        We discuss possible options for extending this FET approach to timesteppers for the compressible case.

        The kinetic corrections satisfy linearized Boltzmann equations. Using a Lénard--Bernstein collision operator, these take Fokker--Planck-like forms \cite{Fokker_1914, Planck_1917} wherein pseudo-particles in the numerical model obey the neoclassical transport equations, with particle-independent Brownian drift terms. This offers a rigorous methodology for incorporating collisions into the particle transport model, without coupling the equations of motions for each particle.
        
        Works by Chen, Chacón et al. \cite{Chen_Chacón_Barnes_2011, Chacón_Chen_Barnes_2013, Chen_Chacón_2014, Chen_Chacón_2015} have developed structure-preserving particle pushers for neoclassical transport in the Vlasov equations, derived from Crank--Nicolson integrators. We show these too can can derive from a FET interpretation, similarly offering potential extensions to higher-order-in-time particle pushers. The FET formulation is used also to consider how the stochastic drift terms can be incorporated into the pushers. Stochastic gyrokinetic expansions are also discussed.

        Different options for the numerical implementation of these schemes are considered.

        Due to the efficacy of FET in the development of SP timesteppers for both the fluid and kinetic component, we hope this approach will prove effective in the future for developing SP timesteppers for the full hybrid model. We hope this will give us the opportunity to incorporate previously inaccessible kinetic effects into the highly effective, modern, finite-element MHD models.
    \end{abstract}
    
    
    \newpage
    \tableofcontents
    
    
    \newpage
    \pagenumbering{arabic}
    %\linenumbers\renewcommand\thelinenumber{\color{black!50}\arabic{linenumber}}
            \documentclass[12pt, a4paper]{report}

\input{template/main.tex}

\title{\BA{Title in Progress...}}
\author{Boris Andrews}
\affil{Mathematical Institute, University of Oxford}
\date{\today}


\begin{document}
    \pagenumbering{gobble}
    \maketitle
    
    
    \begin{abstract}
        Magnetic confinement reactors---in particular tokamaks---offer one of the most promising options for achieving practical nuclear fusion, with the potential to provide virtually limitless, clean energy. The theoretical and numerical modeling of tokamak plasmas is simultaneously an essential component of effective reactor design, and a great research barrier. Tokamak operational conditions exhibit comparatively low Knudsen numbers. Kinetic effects, including kinetic waves and instabilities, Landau damping, bump-on-tail instabilities and more, are therefore highly influential in tokamak plasma dynamics. Purely fluid models are inherently incapable of capturing these effects, whereas the high dimensionality in purely kinetic models render them practically intractable for most relevant purposes.

        We consider a $\delta\!f$ decomposition model, with a macroscopic fluid background and microscopic kinetic correction, both fully coupled to each other. A similar manner of discretization is proposed to that used in the recent \texttt{STRUPHY} code \cite{Holderied_Possanner_Wang_2021, Holderied_2022, Li_et_al_2023} with a finite-element model for the background and a pseudo-particle/PiC model for the correction.

        The fluid background satisfies the full, non-linear, resistive, compressible, Hall MHD equations. \cite{Laakmann_Hu_Farrell_2022} introduces finite-element(-in-space) implicit timesteppers for the incompressible analogue to this system with structure-preserving (SP) properties in the ideal case, alongside parameter-robust preconditioners. We show that these timesteppers can derive from a finite-element-in-time (FET) (and finite-element-in-space) interpretation. The benefits of this reformulation are discussed, including the derivation of timesteppers that are higher order in time, and the quantifiable dissipative SP properties in the non-ideal, resistive case.
        
        We discuss possible options for extending this FET approach to timesteppers for the compressible case.

        The kinetic corrections satisfy linearized Boltzmann equations. Using a Lénard--Bernstein collision operator, these take Fokker--Planck-like forms \cite{Fokker_1914, Planck_1917} wherein pseudo-particles in the numerical model obey the neoclassical transport equations, with particle-independent Brownian drift terms. This offers a rigorous methodology for incorporating collisions into the particle transport model, without coupling the equations of motions for each particle.
        
        Works by Chen, Chacón et al. \cite{Chen_Chacón_Barnes_2011, Chacón_Chen_Barnes_2013, Chen_Chacón_2014, Chen_Chacón_2015} have developed structure-preserving particle pushers for neoclassical transport in the Vlasov equations, derived from Crank--Nicolson integrators. We show these too can can derive from a FET interpretation, similarly offering potential extensions to higher-order-in-time particle pushers. The FET formulation is used also to consider how the stochastic drift terms can be incorporated into the pushers. Stochastic gyrokinetic expansions are also discussed.

        Different options for the numerical implementation of these schemes are considered.

        Due to the efficacy of FET in the development of SP timesteppers for both the fluid and kinetic component, we hope this approach will prove effective in the future for developing SP timesteppers for the full hybrid model. We hope this will give us the opportunity to incorporate previously inaccessible kinetic effects into the highly effective, modern, finite-element MHD models.
    \end{abstract}
    
    
    \newpage
    \tableofcontents
    
    
    \newpage
    \pagenumbering{arabic}
    %\linenumbers\renewcommand\thelinenumber{\color{black!50}\arabic{linenumber}}
            \input{0 - introduction/main.tex}
        \part{Research}
            \input{1 - low-noise PiC models/main.tex}
            \input{2 - kinetic component/main.tex}
            \input{3 - fluid component/main.tex}
            \input{4 - numerical implementation/main.tex}
        \part{Project Overview}
            \input{5 - research plan/main.tex}
            \input{6 - summary/main.tex}
    
    
    %\section{}
    \newpage
    \pagenumbering{gobble}
        \printbibliography


    \newpage
    \pagenumbering{roman}
    \appendix
        \part{Appendices}
            \input{8 - Hilbert complexes/main.tex}
            \input{9 - weak conservation proofs/main.tex}
\end{document}

        \part{Research}
            \documentclass[12pt, a4paper]{report}

\input{template/main.tex}

\title{\BA{Title in Progress...}}
\author{Boris Andrews}
\affil{Mathematical Institute, University of Oxford}
\date{\today}


\begin{document}
    \pagenumbering{gobble}
    \maketitle
    
    
    \begin{abstract}
        Magnetic confinement reactors---in particular tokamaks---offer one of the most promising options for achieving practical nuclear fusion, with the potential to provide virtually limitless, clean energy. The theoretical and numerical modeling of tokamak plasmas is simultaneously an essential component of effective reactor design, and a great research barrier. Tokamak operational conditions exhibit comparatively low Knudsen numbers. Kinetic effects, including kinetic waves and instabilities, Landau damping, bump-on-tail instabilities and more, are therefore highly influential in tokamak plasma dynamics. Purely fluid models are inherently incapable of capturing these effects, whereas the high dimensionality in purely kinetic models render them practically intractable for most relevant purposes.

        We consider a $\delta\!f$ decomposition model, with a macroscopic fluid background and microscopic kinetic correction, both fully coupled to each other. A similar manner of discretization is proposed to that used in the recent \texttt{STRUPHY} code \cite{Holderied_Possanner_Wang_2021, Holderied_2022, Li_et_al_2023} with a finite-element model for the background and a pseudo-particle/PiC model for the correction.

        The fluid background satisfies the full, non-linear, resistive, compressible, Hall MHD equations. \cite{Laakmann_Hu_Farrell_2022} introduces finite-element(-in-space) implicit timesteppers for the incompressible analogue to this system with structure-preserving (SP) properties in the ideal case, alongside parameter-robust preconditioners. We show that these timesteppers can derive from a finite-element-in-time (FET) (and finite-element-in-space) interpretation. The benefits of this reformulation are discussed, including the derivation of timesteppers that are higher order in time, and the quantifiable dissipative SP properties in the non-ideal, resistive case.
        
        We discuss possible options for extending this FET approach to timesteppers for the compressible case.

        The kinetic corrections satisfy linearized Boltzmann equations. Using a Lénard--Bernstein collision operator, these take Fokker--Planck-like forms \cite{Fokker_1914, Planck_1917} wherein pseudo-particles in the numerical model obey the neoclassical transport equations, with particle-independent Brownian drift terms. This offers a rigorous methodology for incorporating collisions into the particle transport model, without coupling the equations of motions for each particle.
        
        Works by Chen, Chacón et al. \cite{Chen_Chacón_Barnes_2011, Chacón_Chen_Barnes_2013, Chen_Chacón_2014, Chen_Chacón_2015} have developed structure-preserving particle pushers for neoclassical transport in the Vlasov equations, derived from Crank--Nicolson integrators. We show these too can can derive from a FET interpretation, similarly offering potential extensions to higher-order-in-time particle pushers. The FET formulation is used also to consider how the stochastic drift terms can be incorporated into the pushers. Stochastic gyrokinetic expansions are also discussed.

        Different options for the numerical implementation of these schemes are considered.

        Due to the efficacy of FET in the development of SP timesteppers for both the fluid and kinetic component, we hope this approach will prove effective in the future for developing SP timesteppers for the full hybrid model. We hope this will give us the opportunity to incorporate previously inaccessible kinetic effects into the highly effective, modern, finite-element MHD models.
    \end{abstract}
    
    
    \newpage
    \tableofcontents
    
    
    \newpage
    \pagenumbering{arabic}
    %\linenumbers\renewcommand\thelinenumber{\color{black!50}\arabic{linenumber}}
            \input{0 - introduction/main.tex}
        \part{Research}
            \input{1 - low-noise PiC models/main.tex}
            \input{2 - kinetic component/main.tex}
            \input{3 - fluid component/main.tex}
            \input{4 - numerical implementation/main.tex}
        \part{Project Overview}
            \input{5 - research plan/main.tex}
            \input{6 - summary/main.tex}
    
    
    %\section{}
    \newpage
    \pagenumbering{gobble}
        \printbibliography


    \newpage
    \pagenumbering{roman}
    \appendix
        \part{Appendices}
            \input{8 - Hilbert complexes/main.tex}
            \input{9 - weak conservation proofs/main.tex}
\end{document}

            \documentclass[12pt, a4paper]{report}

\input{template/main.tex}

\title{\BA{Title in Progress...}}
\author{Boris Andrews}
\affil{Mathematical Institute, University of Oxford}
\date{\today}


\begin{document}
    \pagenumbering{gobble}
    \maketitle
    
    
    \begin{abstract}
        Magnetic confinement reactors---in particular tokamaks---offer one of the most promising options for achieving practical nuclear fusion, with the potential to provide virtually limitless, clean energy. The theoretical and numerical modeling of tokamak plasmas is simultaneously an essential component of effective reactor design, and a great research barrier. Tokamak operational conditions exhibit comparatively low Knudsen numbers. Kinetic effects, including kinetic waves and instabilities, Landau damping, bump-on-tail instabilities and more, are therefore highly influential in tokamak plasma dynamics. Purely fluid models are inherently incapable of capturing these effects, whereas the high dimensionality in purely kinetic models render them practically intractable for most relevant purposes.

        We consider a $\delta\!f$ decomposition model, with a macroscopic fluid background and microscopic kinetic correction, both fully coupled to each other. A similar manner of discretization is proposed to that used in the recent \texttt{STRUPHY} code \cite{Holderied_Possanner_Wang_2021, Holderied_2022, Li_et_al_2023} with a finite-element model for the background and a pseudo-particle/PiC model for the correction.

        The fluid background satisfies the full, non-linear, resistive, compressible, Hall MHD equations. \cite{Laakmann_Hu_Farrell_2022} introduces finite-element(-in-space) implicit timesteppers for the incompressible analogue to this system with structure-preserving (SP) properties in the ideal case, alongside parameter-robust preconditioners. We show that these timesteppers can derive from a finite-element-in-time (FET) (and finite-element-in-space) interpretation. The benefits of this reformulation are discussed, including the derivation of timesteppers that are higher order in time, and the quantifiable dissipative SP properties in the non-ideal, resistive case.
        
        We discuss possible options for extending this FET approach to timesteppers for the compressible case.

        The kinetic corrections satisfy linearized Boltzmann equations. Using a Lénard--Bernstein collision operator, these take Fokker--Planck-like forms \cite{Fokker_1914, Planck_1917} wherein pseudo-particles in the numerical model obey the neoclassical transport equations, with particle-independent Brownian drift terms. This offers a rigorous methodology for incorporating collisions into the particle transport model, without coupling the equations of motions for each particle.
        
        Works by Chen, Chacón et al. \cite{Chen_Chacón_Barnes_2011, Chacón_Chen_Barnes_2013, Chen_Chacón_2014, Chen_Chacón_2015} have developed structure-preserving particle pushers for neoclassical transport in the Vlasov equations, derived from Crank--Nicolson integrators. We show these too can can derive from a FET interpretation, similarly offering potential extensions to higher-order-in-time particle pushers. The FET formulation is used also to consider how the stochastic drift terms can be incorporated into the pushers. Stochastic gyrokinetic expansions are also discussed.

        Different options for the numerical implementation of these schemes are considered.

        Due to the efficacy of FET in the development of SP timesteppers for both the fluid and kinetic component, we hope this approach will prove effective in the future for developing SP timesteppers for the full hybrid model. We hope this will give us the opportunity to incorporate previously inaccessible kinetic effects into the highly effective, modern, finite-element MHD models.
    \end{abstract}
    
    
    \newpage
    \tableofcontents
    
    
    \newpage
    \pagenumbering{arabic}
    %\linenumbers\renewcommand\thelinenumber{\color{black!50}\arabic{linenumber}}
            \input{0 - introduction/main.tex}
        \part{Research}
            \input{1 - low-noise PiC models/main.tex}
            \input{2 - kinetic component/main.tex}
            \input{3 - fluid component/main.tex}
            \input{4 - numerical implementation/main.tex}
        \part{Project Overview}
            \input{5 - research plan/main.tex}
            \input{6 - summary/main.tex}
    
    
    %\section{}
    \newpage
    \pagenumbering{gobble}
        \printbibliography


    \newpage
    \pagenumbering{roman}
    \appendix
        \part{Appendices}
            \input{8 - Hilbert complexes/main.tex}
            \input{9 - weak conservation proofs/main.tex}
\end{document}

            \documentclass[12pt, a4paper]{report}

\input{template/main.tex}

\title{\BA{Title in Progress...}}
\author{Boris Andrews}
\affil{Mathematical Institute, University of Oxford}
\date{\today}


\begin{document}
    \pagenumbering{gobble}
    \maketitle
    
    
    \begin{abstract}
        Magnetic confinement reactors---in particular tokamaks---offer one of the most promising options for achieving practical nuclear fusion, with the potential to provide virtually limitless, clean energy. The theoretical and numerical modeling of tokamak plasmas is simultaneously an essential component of effective reactor design, and a great research barrier. Tokamak operational conditions exhibit comparatively low Knudsen numbers. Kinetic effects, including kinetic waves and instabilities, Landau damping, bump-on-tail instabilities and more, are therefore highly influential in tokamak plasma dynamics. Purely fluid models are inherently incapable of capturing these effects, whereas the high dimensionality in purely kinetic models render them practically intractable for most relevant purposes.

        We consider a $\delta\!f$ decomposition model, with a macroscopic fluid background and microscopic kinetic correction, both fully coupled to each other. A similar manner of discretization is proposed to that used in the recent \texttt{STRUPHY} code \cite{Holderied_Possanner_Wang_2021, Holderied_2022, Li_et_al_2023} with a finite-element model for the background and a pseudo-particle/PiC model for the correction.

        The fluid background satisfies the full, non-linear, resistive, compressible, Hall MHD equations. \cite{Laakmann_Hu_Farrell_2022} introduces finite-element(-in-space) implicit timesteppers for the incompressible analogue to this system with structure-preserving (SP) properties in the ideal case, alongside parameter-robust preconditioners. We show that these timesteppers can derive from a finite-element-in-time (FET) (and finite-element-in-space) interpretation. The benefits of this reformulation are discussed, including the derivation of timesteppers that are higher order in time, and the quantifiable dissipative SP properties in the non-ideal, resistive case.
        
        We discuss possible options for extending this FET approach to timesteppers for the compressible case.

        The kinetic corrections satisfy linearized Boltzmann equations. Using a Lénard--Bernstein collision operator, these take Fokker--Planck-like forms \cite{Fokker_1914, Planck_1917} wherein pseudo-particles in the numerical model obey the neoclassical transport equations, with particle-independent Brownian drift terms. This offers a rigorous methodology for incorporating collisions into the particle transport model, without coupling the equations of motions for each particle.
        
        Works by Chen, Chacón et al. \cite{Chen_Chacón_Barnes_2011, Chacón_Chen_Barnes_2013, Chen_Chacón_2014, Chen_Chacón_2015} have developed structure-preserving particle pushers for neoclassical transport in the Vlasov equations, derived from Crank--Nicolson integrators. We show these too can can derive from a FET interpretation, similarly offering potential extensions to higher-order-in-time particle pushers. The FET formulation is used also to consider how the stochastic drift terms can be incorporated into the pushers. Stochastic gyrokinetic expansions are also discussed.

        Different options for the numerical implementation of these schemes are considered.

        Due to the efficacy of FET in the development of SP timesteppers for both the fluid and kinetic component, we hope this approach will prove effective in the future for developing SP timesteppers for the full hybrid model. We hope this will give us the opportunity to incorporate previously inaccessible kinetic effects into the highly effective, modern, finite-element MHD models.
    \end{abstract}
    
    
    \newpage
    \tableofcontents
    
    
    \newpage
    \pagenumbering{arabic}
    %\linenumbers\renewcommand\thelinenumber{\color{black!50}\arabic{linenumber}}
            \input{0 - introduction/main.tex}
        \part{Research}
            \input{1 - low-noise PiC models/main.tex}
            \input{2 - kinetic component/main.tex}
            \input{3 - fluid component/main.tex}
            \input{4 - numerical implementation/main.tex}
        \part{Project Overview}
            \input{5 - research plan/main.tex}
            \input{6 - summary/main.tex}
    
    
    %\section{}
    \newpage
    \pagenumbering{gobble}
        \printbibliography


    \newpage
    \pagenumbering{roman}
    \appendix
        \part{Appendices}
            \input{8 - Hilbert complexes/main.tex}
            \input{9 - weak conservation proofs/main.tex}
\end{document}

            \documentclass[12pt, a4paper]{report}

\input{template/main.tex}

\title{\BA{Title in Progress...}}
\author{Boris Andrews}
\affil{Mathematical Institute, University of Oxford}
\date{\today}


\begin{document}
    \pagenumbering{gobble}
    \maketitle
    
    
    \begin{abstract}
        Magnetic confinement reactors---in particular tokamaks---offer one of the most promising options for achieving practical nuclear fusion, with the potential to provide virtually limitless, clean energy. The theoretical and numerical modeling of tokamak plasmas is simultaneously an essential component of effective reactor design, and a great research barrier. Tokamak operational conditions exhibit comparatively low Knudsen numbers. Kinetic effects, including kinetic waves and instabilities, Landau damping, bump-on-tail instabilities and more, are therefore highly influential in tokamak plasma dynamics. Purely fluid models are inherently incapable of capturing these effects, whereas the high dimensionality in purely kinetic models render them practically intractable for most relevant purposes.

        We consider a $\delta\!f$ decomposition model, with a macroscopic fluid background and microscopic kinetic correction, both fully coupled to each other. A similar manner of discretization is proposed to that used in the recent \texttt{STRUPHY} code \cite{Holderied_Possanner_Wang_2021, Holderied_2022, Li_et_al_2023} with a finite-element model for the background and a pseudo-particle/PiC model for the correction.

        The fluid background satisfies the full, non-linear, resistive, compressible, Hall MHD equations. \cite{Laakmann_Hu_Farrell_2022} introduces finite-element(-in-space) implicit timesteppers for the incompressible analogue to this system with structure-preserving (SP) properties in the ideal case, alongside parameter-robust preconditioners. We show that these timesteppers can derive from a finite-element-in-time (FET) (and finite-element-in-space) interpretation. The benefits of this reformulation are discussed, including the derivation of timesteppers that are higher order in time, and the quantifiable dissipative SP properties in the non-ideal, resistive case.
        
        We discuss possible options for extending this FET approach to timesteppers for the compressible case.

        The kinetic corrections satisfy linearized Boltzmann equations. Using a Lénard--Bernstein collision operator, these take Fokker--Planck-like forms \cite{Fokker_1914, Planck_1917} wherein pseudo-particles in the numerical model obey the neoclassical transport equations, with particle-independent Brownian drift terms. This offers a rigorous methodology for incorporating collisions into the particle transport model, without coupling the equations of motions for each particle.
        
        Works by Chen, Chacón et al. \cite{Chen_Chacón_Barnes_2011, Chacón_Chen_Barnes_2013, Chen_Chacón_2014, Chen_Chacón_2015} have developed structure-preserving particle pushers for neoclassical transport in the Vlasov equations, derived from Crank--Nicolson integrators. We show these too can can derive from a FET interpretation, similarly offering potential extensions to higher-order-in-time particle pushers. The FET formulation is used also to consider how the stochastic drift terms can be incorporated into the pushers. Stochastic gyrokinetic expansions are also discussed.

        Different options for the numerical implementation of these schemes are considered.

        Due to the efficacy of FET in the development of SP timesteppers for both the fluid and kinetic component, we hope this approach will prove effective in the future for developing SP timesteppers for the full hybrid model. We hope this will give us the opportunity to incorporate previously inaccessible kinetic effects into the highly effective, modern, finite-element MHD models.
    \end{abstract}
    
    
    \newpage
    \tableofcontents
    
    
    \newpage
    \pagenumbering{arabic}
    %\linenumbers\renewcommand\thelinenumber{\color{black!50}\arabic{linenumber}}
            \input{0 - introduction/main.tex}
        \part{Research}
            \input{1 - low-noise PiC models/main.tex}
            \input{2 - kinetic component/main.tex}
            \input{3 - fluid component/main.tex}
            \input{4 - numerical implementation/main.tex}
        \part{Project Overview}
            \input{5 - research plan/main.tex}
            \input{6 - summary/main.tex}
    
    
    %\section{}
    \newpage
    \pagenumbering{gobble}
        \printbibliography


    \newpage
    \pagenumbering{roman}
    \appendix
        \part{Appendices}
            \input{8 - Hilbert complexes/main.tex}
            \input{9 - weak conservation proofs/main.tex}
\end{document}

        \part{Project Overview}
            \documentclass[12pt, a4paper]{report}

\input{template/main.tex}

\title{\BA{Title in Progress...}}
\author{Boris Andrews}
\affil{Mathematical Institute, University of Oxford}
\date{\today}


\begin{document}
    \pagenumbering{gobble}
    \maketitle
    
    
    \begin{abstract}
        Magnetic confinement reactors---in particular tokamaks---offer one of the most promising options for achieving practical nuclear fusion, with the potential to provide virtually limitless, clean energy. The theoretical and numerical modeling of tokamak plasmas is simultaneously an essential component of effective reactor design, and a great research barrier. Tokamak operational conditions exhibit comparatively low Knudsen numbers. Kinetic effects, including kinetic waves and instabilities, Landau damping, bump-on-tail instabilities and more, are therefore highly influential in tokamak plasma dynamics. Purely fluid models are inherently incapable of capturing these effects, whereas the high dimensionality in purely kinetic models render them practically intractable for most relevant purposes.

        We consider a $\delta\!f$ decomposition model, with a macroscopic fluid background and microscopic kinetic correction, both fully coupled to each other. A similar manner of discretization is proposed to that used in the recent \texttt{STRUPHY} code \cite{Holderied_Possanner_Wang_2021, Holderied_2022, Li_et_al_2023} with a finite-element model for the background and a pseudo-particle/PiC model for the correction.

        The fluid background satisfies the full, non-linear, resistive, compressible, Hall MHD equations. \cite{Laakmann_Hu_Farrell_2022} introduces finite-element(-in-space) implicit timesteppers for the incompressible analogue to this system with structure-preserving (SP) properties in the ideal case, alongside parameter-robust preconditioners. We show that these timesteppers can derive from a finite-element-in-time (FET) (and finite-element-in-space) interpretation. The benefits of this reformulation are discussed, including the derivation of timesteppers that are higher order in time, and the quantifiable dissipative SP properties in the non-ideal, resistive case.
        
        We discuss possible options for extending this FET approach to timesteppers for the compressible case.

        The kinetic corrections satisfy linearized Boltzmann equations. Using a Lénard--Bernstein collision operator, these take Fokker--Planck-like forms \cite{Fokker_1914, Planck_1917} wherein pseudo-particles in the numerical model obey the neoclassical transport equations, with particle-independent Brownian drift terms. This offers a rigorous methodology for incorporating collisions into the particle transport model, without coupling the equations of motions for each particle.
        
        Works by Chen, Chacón et al. \cite{Chen_Chacón_Barnes_2011, Chacón_Chen_Barnes_2013, Chen_Chacón_2014, Chen_Chacón_2015} have developed structure-preserving particle pushers for neoclassical transport in the Vlasov equations, derived from Crank--Nicolson integrators. We show these too can can derive from a FET interpretation, similarly offering potential extensions to higher-order-in-time particle pushers. The FET formulation is used also to consider how the stochastic drift terms can be incorporated into the pushers. Stochastic gyrokinetic expansions are also discussed.

        Different options for the numerical implementation of these schemes are considered.

        Due to the efficacy of FET in the development of SP timesteppers for both the fluid and kinetic component, we hope this approach will prove effective in the future for developing SP timesteppers for the full hybrid model. We hope this will give us the opportunity to incorporate previously inaccessible kinetic effects into the highly effective, modern, finite-element MHD models.
    \end{abstract}
    
    
    \newpage
    \tableofcontents
    
    
    \newpage
    \pagenumbering{arabic}
    %\linenumbers\renewcommand\thelinenumber{\color{black!50}\arabic{linenumber}}
            \input{0 - introduction/main.tex}
        \part{Research}
            \input{1 - low-noise PiC models/main.tex}
            \input{2 - kinetic component/main.tex}
            \input{3 - fluid component/main.tex}
            \input{4 - numerical implementation/main.tex}
        \part{Project Overview}
            \input{5 - research plan/main.tex}
            \input{6 - summary/main.tex}
    
    
    %\section{}
    \newpage
    \pagenumbering{gobble}
        \printbibliography


    \newpage
    \pagenumbering{roman}
    \appendix
        \part{Appendices}
            \input{8 - Hilbert complexes/main.tex}
            \input{9 - weak conservation proofs/main.tex}
\end{document}

            \documentclass[12pt, a4paper]{report}

\input{template/main.tex}

\title{\BA{Title in Progress...}}
\author{Boris Andrews}
\affil{Mathematical Institute, University of Oxford}
\date{\today}


\begin{document}
    \pagenumbering{gobble}
    \maketitle
    
    
    \begin{abstract}
        Magnetic confinement reactors---in particular tokamaks---offer one of the most promising options for achieving practical nuclear fusion, with the potential to provide virtually limitless, clean energy. The theoretical and numerical modeling of tokamak plasmas is simultaneously an essential component of effective reactor design, and a great research barrier. Tokamak operational conditions exhibit comparatively low Knudsen numbers. Kinetic effects, including kinetic waves and instabilities, Landau damping, bump-on-tail instabilities and more, are therefore highly influential in tokamak plasma dynamics. Purely fluid models are inherently incapable of capturing these effects, whereas the high dimensionality in purely kinetic models render them practically intractable for most relevant purposes.

        We consider a $\delta\!f$ decomposition model, with a macroscopic fluid background and microscopic kinetic correction, both fully coupled to each other. A similar manner of discretization is proposed to that used in the recent \texttt{STRUPHY} code \cite{Holderied_Possanner_Wang_2021, Holderied_2022, Li_et_al_2023} with a finite-element model for the background and a pseudo-particle/PiC model for the correction.

        The fluid background satisfies the full, non-linear, resistive, compressible, Hall MHD equations. \cite{Laakmann_Hu_Farrell_2022} introduces finite-element(-in-space) implicit timesteppers for the incompressible analogue to this system with structure-preserving (SP) properties in the ideal case, alongside parameter-robust preconditioners. We show that these timesteppers can derive from a finite-element-in-time (FET) (and finite-element-in-space) interpretation. The benefits of this reformulation are discussed, including the derivation of timesteppers that are higher order in time, and the quantifiable dissipative SP properties in the non-ideal, resistive case.
        
        We discuss possible options for extending this FET approach to timesteppers for the compressible case.

        The kinetic corrections satisfy linearized Boltzmann equations. Using a Lénard--Bernstein collision operator, these take Fokker--Planck-like forms \cite{Fokker_1914, Planck_1917} wherein pseudo-particles in the numerical model obey the neoclassical transport equations, with particle-independent Brownian drift terms. This offers a rigorous methodology for incorporating collisions into the particle transport model, without coupling the equations of motions for each particle.
        
        Works by Chen, Chacón et al. \cite{Chen_Chacón_Barnes_2011, Chacón_Chen_Barnes_2013, Chen_Chacón_2014, Chen_Chacón_2015} have developed structure-preserving particle pushers for neoclassical transport in the Vlasov equations, derived from Crank--Nicolson integrators. We show these too can can derive from a FET interpretation, similarly offering potential extensions to higher-order-in-time particle pushers. The FET formulation is used also to consider how the stochastic drift terms can be incorporated into the pushers. Stochastic gyrokinetic expansions are also discussed.

        Different options for the numerical implementation of these schemes are considered.

        Due to the efficacy of FET in the development of SP timesteppers for both the fluid and kinetic component, we hope this approach will prove effective in the future for developing SP timesteppers for the full hybrid model. We hope this will give us the opportunity to incorporate previously inaccessible kinetic effects into the highly effective, modern, finite-element MHD models.
    \end{abstract}
    
    
    \newpage
    \tableofcontents
    
    
    \newpage
    \pagenumbering{arabic}
    %\linenumbers\renewcommand\thelinenumber{\color{black!50}\arabic{linenumber}}
            \input{0 - introduction/main.tex}
        \part{Research}
            \input{1 - low-noise PiC models/main.tex}
            \input{2 - kinetic component/main.tex}
            \input{3 - fluid component/main.tex}
            \input{4 - numerical implementation/main.tex}
        \part{Project Overview}
            \input{5 - research plan/main.tex}
            \input{6 - summary/main.tex}
    
    
    %\section{}
    \newpage
    \pagenumbering{gobble}
        \printbibliography


    \newpage
    \pagenumbering{roman}
    \appendix
        \part{Appendices}
            \input{8 - Hilbert complexes/main.tex}
            \input{9 - weak conservation proofs/main.tex}
\end{document}

    
    
    %\section{}
    \newpage
    \pagenumbering{gobble}
        \printbibliography


    \newpage
    \pagenumbering{roman}
    \appendix
        \part{Appendices}
            \documentclass[12pt, a4paper]{report}

\input{template/main.tex}

\title{\BA{Title in Progress...}}
\author{Boris Andrews}
\affil{Mathematical Institute, University of Oxford}
\date{\today}


\begin{document}
    \pagenumbering{gobble}
    \maketitle
    
    
    \begin{abstract}
        Magnetic confinement reactors---in particular tokamaks---offer one of the most promising options for achieving practical nuclear fusion, with the potential to provide virtually limitless, clean energy. The theoretical and numerical modeling of tokamak plasmas is simultaneously an essential component of effective reactor design, and a great research barrier. Tokamak operational conditions exhibit comparatively low Knudsen numbers. Kinetic effects, including kinetic waves and instabilities, Landau damping, bump-on-tail instabilities and more, are therefore highly influential in tokamak plasma dynamics. Purely fluid models are inherently incapable of capturing these effects, whereas the high dimensionality in purely kinetic models render them practically intractable for most relevant purposes.

        We consider a $\delta\!f$ decomposition model, with a macroscopic fluid background and microscopic kinetic correction, both fully coupled to each other. A similar manner of discretization is proposed to that used in the recent \texttt{STRUPHY} code \cite{Holderied_Possanner_Wang_2021, Holderied_2022, Li_et_al_2023} with a finite-element model for the background and a pseudo-particle/PiC model for the correction.

        The fluid background satisfies the full, non-linear, resistive, compressible, Hall MHD equations. \cite{Laakmann_Hu_Farrell_2022} introduces finite-element(-in-space) implicit timesteppers for the incompressible analogue to this system with structure-preserving (SP) properties in the ideal case, alongside parameter-robust preconditioners. We show that these timesteppers can derive from a finite-element-in-time (FET) (and finite-element-in-space) interpretation. The benefits of this reformulation are discussed, including the derivation of timesteppers that are higher order in time, and the quantifiable dissipative SP properties in the non-ideal, resistive case.
        
        We discuss possible options for extending this FET approach to timesteppers for the compressible case.

        The kinetic corrections satisfy linearized Boltzmann equations. Using a Lénard--Bernstein collision operator, these take Fokker--Planck-like forms \cite{Fokker_1914, Planck_1917} wherein pseudo-particles in the numerical model obey the neoclassical transport equations, with particle-independent Brownian drift terms. This offers a rigorous methodology for incorporating collisions into the particle transport model, without coupling the equations of motions for each particle.
        
        Works by Chen, Chacón et al. \cite{Chen_Chacón_Barnes_2011, Chacón_Chen_Barnes_2013, Chen_Chacón_2014, Chen_Chacón_2015} have developed structure-preserving particle pushers for neoclassical transport in the Vlasov equations, derived from Crank--Nicolson integrators. We show these too can can derive from a FET interpretation, similarly offering potential extensions to higher-order-in-time particle pushers. The FET formulation is used also to consider how the stochastic drift terms can be incorporated into the pushers. Stochastic gyrokinetic expansions are also discussed.

        Different options for the numerical implementation of these schemes are considered.

        Due to the efficacy of FET in the development of SP timesteppers for both the fluid and kinetic component, we hope this approach will prove effective in the future for developing SP timesteppers for the full hybrid model. We hope this will give us the opportunity to incorporate previously inaccessible kinetic effects into the highly effective, modern, finite-element MHD models.
    \end{abstract}
    
    
    \newpage
    \tableofcontents
    
    
    \newpage
    \pagenumbering{arabic}
    %\linenumbers\renewcommand\thelinenumber{\color{black!50}\arabic{linenumber}}
            \input{0 - introduction/main.tex}
        \part{Research}
            \input{1 - low-noise PiC models/main.tex}
            \input{2 - kinetic component/main.tex}
            \input{3 - fluid component/main.tex}
            \input{4 - numerical implementation/main.tex}
        \part{Project Overview}
            \input{5 - research plan/main.tex}
            \input{6 - summary/main.tex}
    
    
    %\section{}
    \newpage
    \pagenumbering{gobble}
        \printbibliography


    \newpage
    \pagenumbering{roman}
    \appendix
        \part{Appendices}
            \input{8 - Hilbert complexes/main.tex}
            \input{9 - weak conservation proofs/main.tex}
\end{document}

            \documentclass[12pt, a4paper]{report}

\input{template/main.tex}

\title{\BA{Title in Progress...}}
\author{Boris Andrews}
\affil{Mathematical Institute, University of Oxford}
\date{\today}


\begin{document}
    \pagenumbering{gobble}
    \maketitle
    
    
    \begin{abstract}
        Magnetic confinement reactors---in particular tokamaks---offer one of the most promising options for achieving practical nuclear fusion, with the potential to provide virtually limitless, clean energy. The theoretical and numerical modeling of tokamak plasmas is simultaneously an essential component of effective reactor design, and a great research barrier. Tokamak operational conditions exhibit comparatively low Knudsen numbers. Kinetic effects, including kinetic waves and instabilities, Landau damping, bump-on-tail instabilities and more, are therefore highly influential in tokamak plasma dynamics. Purely fluid models are inherently incapable of capturing these effects, whereas the high dimensionality in purely kinetic models render them practically intractable for most relevant purposes.

        We consider a $\delta\!f$ decomposition model, with a macroscopic fluid background and microscopic kinetic correction, both fully coupled to each other. A similar manner of discretization is proposed to that used in the recent \texttt{STRUPHY} code \cite{Holderied_Possanner_Wang_2021, Holderied_2022, Li_et_al_2023} with a finite-element model for the background and a pseudo-particle/PiC model for the correction.

        The fluid background satisfies the full, non-linear, resistive, compressible, Hall MHD equations. \cite{Laakmann_Hu_Farrell_2022} introduces finite-element(-in-space) implicit timesteppers for the incompressible analogue to this system with structure-preserving (SP) properties in the ideal case, alongside parameter-robust preconditioners. We show that these timesteppers can derive from a finite-element-in-time (FET) (and finite-element-in-space) interpretation. The benefits of this reformulation are discussed, including the derivation of timesteppers that are higher order in time, and the quantifiable dissipative SP properties in the non-ideal, resistive case.
        
        We discuss possible options for extending this FET approach to timesteppers for the compressible case.

        The kinetic corrections satisfy linearized Boltzmann equations. Using a Lénard--Bernstein collision operator, these take Fokker--Planck-like forms \cite{Fokker_1914, Planck_1917} wherein pseudo-particles in the numerical model obey the neoclassical transport equations, with particle-independent Brownian drift terms. This offers a rigorous methodology for incorporating collisions into the particle transport model, without coupling the equations of motions for each particle.
        
        Works by Chen, Chacón et al. \cite{Chen_Chacón_Barnes_2011, Chacón_Chen_Barnes_2013, Chen_Chacón_2014, Chen_Chacón_2015} have developed structure-preserving particle pushers for neoclassical transport in the Vlasov equations, derived from Crank--Nicolson integrators. We show these too can can derive from a FET interpretation, similarly offering potential extensions to higher-order-in-time particle pushers. The FET formulation is used also to consider how the stochastic drift terms can be incorporated into the pushers. Stochastic gyrokinetic expansions are also discussed.

        Different options for the numerical implementation of these schemes are considered.

        Due to the efficacy of FET in the development of SP timesteppers for both the fluid and kinetic component, we hope this approach will prove effective in the future for developing SP timesteppers for the full hybrid model. We hope this will give us the opportunity to incorporate previously inaccessible kinetic effects into the highly effective, modern, finite-element MHD models.
    \end{abstract}
    
    
    \newpage
    \tableofcontents
    
    
    \newpage
    \pagenumbering{arabic}
    %\linenumbers\renewcommand\thelinenumber{\color{black!50}\arabic{linenumber}}
            \input{0 - introduction/main.tex}
        \part{Research}
            \input{1 - low-noise PiC models/main.tex}
            \input{2 - kinetic component/main.tex}
            \input{3 - fluid component/main.tex}
            \input{4 - numerical implementation/main.tex}
        \part{Project Overview}
            \input{5 - research plan/main.tex}
            \input{6 - summary/main.tex}
    
    
    %\section{}
    \newpage
    \pagenumbering{gobble}
        \printbibliography


    \newpage
    \pagenumbering{roman}
    \appendix
        \part{Appendices}
            \input{8 - Hilbert complexes/main.tex}
            \input{9 - weak conservation proofs/main.tex}
\end{document}

\end{document}

            \documentclass[12pt, a4paper]{report}

\documentclass[12pt, a4paper]{report}

\input{template/main.tex}

\title{\BA{Title in Progress...}}
\author{Boris Andrews}
\affil{Mathematical Institute, University of Oxford}
\date{\today}


\begin{document}
    \pagenumbering{gobble}
    \maketitle
    
    
    \begin{abstract}
        Magnetic confinement reactors---in particular tokamaks---offer one of the most promising options for achieving practical nuclear fusion, with the potential to provide virtually limitless, clean energy. The theoretical and numerical modeling of tokamak plasmas is simultaneously an essential component of effective reactor design, and a great research barrier. Tokamak operational conditions exhibit comparatively low Knudsen numbers. Kinetic effects, including kinetic waves and instabilities, Landau damping, bump-on-tail instabilities and more, are therefore highly influential in tokamak plasma dynamics. Purely fluid models are inherently incapable of capturing these effects, whereas the high dimensionality in purely kinetic models render them practically intractable for most relevant purposes.

        We consider a $\delta\!f$ decomposition model, with a macroscopic fluid background and microscopic kinetic correction, both fully coupled to each other. A similar manner of discretization is proposed to that used in the recent \texttt{STRUPHY} code \cite{Holderied_Possanner_Wang_2021, Holderied_2022, Li_et_al_2023} with a finite-element model for the background and a pseudo-particle/PiC model for the correction.

        The fluid background satisfies the full, non-linear, resistive, compressible, Hall MHD equations. \cite{Laakmann_Hu_Farrell_2022} introduces finite-element(-in-space) implicit timesteppers for the incompressible analogue to this system with structure-preserving (SP) properties in the ideal case, alongside parameter-robust preconditioners. We show that these timesteppers can derive from a finite-element-in-time (FET) (and finite-element-in-space) interpretation. The benefits of this reformulation are discussed, including the derivation of timesteppers that are higher order in time, and the quantifiable dissipative SP properties in the non-ideal, resistive case.
        
        We discuss possible options for extending this FET approach to timesteppers for the compressible case.

        The kinetic corrections satisfy linearized Boltzmann equations. Using a Lénard--Bernstein collision operator, these take Fokker--Planck-like forms \cite{Fokker_1914, Planck_1917} wherein pseudo-particles in the numerical model obey the neoclassical transport equations, with particle-independent Brownian drift terms. This offers a rigorous methodology for incorporating collisions into the particle transport model, without coupling the equations of motions for each particle.
        
        Works by Chen, Chacón et al. \cite{Chen_Chacón_Barnes_2011, Chacón_Chen_Barnes_2013, Chen_Chacón_2014, Chen_Chacón_2015} have developed structure-preserving particle pushers for neoclassical transport in the Vlasov equations, derived from Crank--Nicolson integrators. We show these too can can derive from a FET interpretation, similarly offering potential extensions to higher-order-in-time particle pushers. The FET formulation is used also to consider how the stochastic drift terms can be incorporated into the pushers. Stochastic gyrokinetic expansions are also discussed.

        Different options for the numerical implementation of these schemes are considered.

        Due to the efficacy of FET in the development of SP timesteppers for both the fluid and kinetic component, we hope this approach will prove effective in the future for developing SP timesteppers for the full hybrid model. We hope this will give us the opportunity to incorporate previously inaccessible kinetic effects into the highly effective, modern, finite-element MHD models.
    \end{abstract}
    
    
    \newpage
    \tableofcontents
    
    
    \newpage
    \pagenumbering{arabic}
    %\linenumbers\renewcommand\thelinenumber{\color{black!50}\arabic{linenumber}}
            \input{0 - introduction/main.tex}
        \part{Research}
            \input{1 - low-noise PiC models/main.tex}
            \input{2 - kinetic component/main.tex}
            \input{3 - fluid component/main.tex}
            \input{4 - numerical implementation/main.tex}
        \part{Project Overview}
            \input{5 - research plan/main.tex}
            \input{6 - summary/main.tex}
    
    
    %\section{}
    \newpage
    \pagenumbering{gobble}
        \printbibliography


    \newpage
    \pagenumbering{roman}
    \appendix
        \part{Appendices}
            \input{8 - Hilbert complexes/main.tex}
            \input{9 - weak conservation proofs/main.tex}
\end{document}


\title{\BA{Title in Progress...}}
\author{Boris Andrews}
\affil{Mathematical Institute, University of Oxford}
\date{\today}


\begin{document}
    \pagenumbering{gobble}
    \maketitle
    
    
    \begin{abstract}
        Magnetic confinement reactors---in particular tokamaks---offer one of the most promising options for achieving practical nuclear fusion, with the potential to provide virtually limitless, clean energy. The theoretical and numerical modeling of tokamak plasmas is simultaneously an essential component of effective reactor design, and a great research barrier. Tokamak operational conditions exhibit comparatively low Knudsen numbers. Kinetic effects, including kinetic waves and instabilities, Landau damping, bump-on-tail instabilities and more, are therefore highly influential in tokamak plasma dynamics. Purely fluid models are inherently incapable of capturing these effects, whereas the high dimensionality in purely kinetic models render them practically intractable for most relevant purposes.

        We consider a $\delta\!f$ decomposition model, with a macroscopic fluid background and microscopic kinetic correction, both fully coupled to each other. A similar manner of discretization is proposed to that used in the recent \texttt{STRUPHY} code \cite{Holderied_Possanner_Wang_2021, Holderied_2022, Li_et_al_2023} with a finite-element model for the background and a pseudo-particle/PiC model for the correction.

        The fluid background satisfies the full, non-linear, resistive, compressible, Hall MHD equations. \cite{Laakmann_Hu_Farrell_2022} introduces finite-element(-in-space) implicit timesteppers for the incompressible analogue to this system with structure-preserving (SP) properties in the ideal case, alongside parameter-robust preconditioners. We show that these timesteppers can derive from a finite-element-in-time (FET) (and finite-element-in-space) interpretation. The benefits of this reformulation are discussed, including the derivation of timesteppers that are higher order in time, and the quantifiable dissipative SP properties in the non-ideal, resistive case.
        
        We discuss possible options for extending this FET approach to timesteppers for the compressible case.

        The kinetic corrections satisfy linearized Boltzmann equations. Using a Lénard--Bernstein collision operator, these take Fokker--Planck-like forms \cite{Fokker_1914, Planck_1917} wherein pseudo-particles in the numerical model obey the neoclassical transport equations, with particle-independent Brownian drift terms. This offers a rigorous methodology for incorporating collisions into the particle transport model, without coupling the equations of motions for each particle.
        
        Works by Chen, Chacón et al. \cite{Chen_Chacón_Barnes_2011, Chacón_Chen_Barnes_2013, Chen_Chacón_2014, Chen_Chacón_2015} have developed structure-preserving particle pushers for neoclassical transport in the Vlasov equations, derived from Crank--Nicolson integrators. We show these too can can derive from a FET interpretation, similarly offering potential extensions to higher-order-in-time particle pushers. The FET formulation is used also to consider how the stochastic drift terms can be incorporated into the pushers. Stochastic gyrokinetic expansions are also discussed.

        Different options for the numerical implementation of these schemes are considered.

        Due to the efficacy of FET in the development of SP timesteppers for both the fluid and kinetic component, we hope this approach will prove effective in the future for developing SP timesteppers for the full hybrid model. We hope this will give us the opportunity to incorporate previously inaccessible kinetic effects into the highly effective, modern, finite-element MHD models.
    \end{abstract}
    
    
    \newpage
    \tableofcontents
    
    
    \newpage
    \pagenumbering{arabic}
    %\linenumbers\renewcommand\thelinenumber{\color{black!50}\arabic{linenumber}}
            \documentclass[12pt, a4paper]{report}

\input{template/main.tex}

\title{\BA{Title in Progress...}}
\author{Boris Andrews}
\affil{Mathematical Institute, University of Oxford}
\date{\today}


\begin{document}
    \pagenumbering{gobble}
    \maketitle
    
    
    \begin{abstract}
        Magnetic confinement reactors---in particular tokamaks---offer one of the most promising options for achieving practical nuclear fusion, with the potential to provide virtually limitless, clean energy. The theoretical and numerical modeling of tokamak plasmas is simultaneously an essential component of effective reactor design, and a great research barrier. Tokamak operational conditions exhibit comparatively low Knudsen numbers. Kinetic effects, including kinetic waves and instabilities, Landau damping, bump-on-tail instabilities and more, are therefore highly influential in tokamak plasma dynamics. Purely fluid models are inherently incapable of capturing these effects, whereas the high dimensionality in purely kinetic models render them practically intractable for most relevant purposes.

        We consider a $\delta\!f$ decomposition model, with a macroscopic fluid background and microscopic kinetic correction, both fully coupled to each other. A similar manner of discretization is proposed to that used in the recent \texttt{STRUPHY} code \cite{Holderied_Possanner_Wang_2021, Holderied_2022, Li_et_al_2023} with a finite-element model for the background and a pseudo-particle/PiC model for the correction.

        The fluid background satisfies the full, non-linear, resistive, compressible, Hall MHD equations. \cite{Laakmann_Hu_Farrell_2022} introduces finite-element(-in-space) implicit timesteppers for the incompressible analogue to this system with structure-preserving (SP) properties in the ideal case, alongside parameter-robust preconditioners. We show that these timesteppers can derive from a finite-element-in-time (FET) (and finite-element-in-space) interpretation. The benefits of this reformulation are discussed, including the derivation of timesteppers that are higher order in time, and the quantifiable dissipative SP properties in the non-ideal, resistive case.
        
        We discuss possible options for extending this FET approach to timesteppers for the compressible case.

        The kinetic corrections satisfy linearized Boltzmann equations. Using a Lénard--Bernstein collision operator, these take Fokker--Planck-like forms \cite{Fokker_1914, Planck_1917} wherein pseudo-particles in the numerical model obey the neoclassical transport equations, with particle-independent Brownian drift terms. This offers a rigorous methodology for incorporating collisions into the particle transport model, without coupling the equations of motions for each particle.
        
        Works by Chen, Chacón et al. \cite{Chen_Chacón_Barnes_2011, Chacón_Chen_Barnes_2013, Chen_Chacón_2014, Chen_Chacón_2015} have developed structure-preserving particle pushers for neoclassical transport in the Vlasov equations, derived from Crank--Nicolson integrators. We show these too can can derive from a FET interpretation, similarly offering potential extensions to higher-order-in-time particle pushers. The FET formulation is used also to consider how the stochastic drift terms can be incorporated into the pushers. Stochastic gyrokinetic expansions are also discussed.

        Different options for the numerical implementation of these schemes are considered.

        Due to the efficacy of FET in the development of SP timesteppers for both the fluid and kinetic component, we hope this approach will prove effective in the future for developing SP timesteppers for the full hybrid model. We hope this will give us the opportunity to incorporate previously inaccessible kinetic effects into the highly effective, modern, finite-element MHD models.
    \end{abstract}
    
    
    \newpage
    \tableofcontents
    
    
    \newpage
    \pagenumbering{arabic}
    %\linenumbers\renewcommand\thelinenumber{\color{black!50}\arabic{linenumber}}
            \input{0 - introduction/main.tex}
        \part{Research}
            \input{1 - low-noise PiC models/main.tex}
            \input{2 - kinetic component/main.tex}
            \input{3 - fluid component/main.tex}
            \input{4 - numerical implementation/main.tex}
        \part{Project Overview}
            \input{5 - research plan/main.tex}
            \input{6 - summary/main.tex}
    
    
    %\section{}
    \newpage
    \pagenumbering{gobble}
        \printbibliography


    \newpage
    \pagenumbering{roman}
    \appendix
        \part{Appendices}
            \input{8 - Hilbert complexes/main.tex}
            \input{9 - weak conservation proofs/main.tex}
\end{document}

        \part{Research}
            \documentclass[12pt, a4paper]{report}

\input{template/main.tex}

\title{\BA{Title in Progress...}}
\author{Boris Andrews}
\affil{Mathematical Institute, University of Oxford}
\date{\today}


\begin{document}
    \pagenumbering{gobble}
    \maketitle
    
    
    \begin{abstract}
        Magnetic confinement reactors---in particular tokamaks---offer one of the most promising options for achieving practical nuclear fusion, with the potential to provide virtually limitless, clean energy. The theoretical and numerical modeling of tokamak plasmas is simultaneously an essential component of effective reactor design, and a great research barrier. Tokamak operational conditions exhibit comparatively low Knudsen numbers. Kinetic effects, including kinetic waves and instabilities, Landau damping, bump-on-tail instabilities and more, are therefore highly influential in tokamak plasma dynamics. Purely fluid models are inherently incapable of capturing these effects, whereas the high dimensionality in purely kinetic models render them practically intractable for most relevant purposes.

        We consider a $\delta\!f$ decomposition model, with a macroscopic fluid background and microscopic kinetic correction, both fully coupled to each other. A similar manner of discretization is proposed to that used in the recent \texttt{STRUPHY} code \cite{Holderied_Possanner_Wang_2021, Holderied_2022, Li_et_al_2023} with a finite-element model for the background and a pseudo-particle/PiC model for the correction.

        The fluid background satisfies the full, non-linear, resistive, compressible, Hall MHD equations. \cite{Laakmann_Hu_Farrell_2022} introduces finite-element(-in-space) implicit timesteppers for the incompressible analogue to this system with structure-preserving (SP) properties in the ideal case, alongside parameter-robust preconditioners. We show that these timesteppers can derive from a finite-element-in-time (FET) (and finite-element-in-space) interpretation. The benefits of this reformulation are discussed, including the derivation of timesteppers that are higher order in time, and the quantifiable dissipative SP properties in the non-ideal, resistive case.
        
        We discuss possible options for extending this FET approach to timesteppers for the compressible case.

        The kinetic corrections satisfy linearized Boltzmann equations. Using a Lénard--Bernstein collision operator, these take Fokker--Planck-like forms \cite{Fokker_1914, Planck_1917} wherein pseudo-particles in the numerical model obey the neoclassical transport equations, with particle-independent Brownian drift terms. This offers a rigorous methodology for incorporating collisions into the particle transport model, without coupling the equations of motions for each particle.
        
        Works by Chen, Chacón et al. \cite{Chen_Chacón_Barnes_2011, Chacón_Chen_Barnes_2013, Chen_Chacón_2014, Chen_Chacón_2015} have developed structure-preserving particle pushers for neoclassical transport in the Vlasov equations, derived from Crank--Nicolson integrators. We show these too can can derive from a FET interpretation, similarly offering potential extensions to higher-order-in-time particle pushers. The FET formulation is used also to consider how the stochastic drift terms can be incorporated into the pushers. Stochastic gyrokinetic expansions are also discussed.

        Different options for the numerical implementation of these schemes are considered.

        Due to the efficacy of FET in the development of SP timesteppers for both the fluid and kinetic component, we hope this approach will prove effective in the future for developing SP timesteppers for the full hybrid model. We hope this will give us the opportunity to incorporate previously inaccessible kinetic effects into the highly effective, modern, finite-element MHD models.
    \end{abstract}
    
    
    \newpage
    \tableofcontents
    
    
    \newpage
    \pagenumbering{arabic}
    %\linenumbers\renewcommand\thelinenumber{\color{black!50}\arabic{linenumber}}
            \input{0 - introduction/main.tex}
        \part{Research}
            \input{1 - low-noise PiC models/main.tex}
            \input{2 - kinetic component/main.tex}
            \input{3 - fluid component/main.tex}
            \input{4 - numerical implementation/main.tex}
        \part{Project Overview}
            \input{5 - research plan/main.tex}
            \input{6 - summary/main.tex}
    
    
    %\section{}
    \newpage
    \pagenumbering{gobble}
        \printbibliography


    \newpage
    \pagenumbering{roman}
    \appendix
        \part{Appendices}
            \input{8 - Hilbert complexes/main.tex}
            \input{9 - weak conservation proofs/main.tex}
\end{document}

            \documentclass[12pt, a4paper]{report}

\input{template/main.tex}

\title{\BA{Title in Progress...}}
\author{Boris Andrews}
\affil{Mathematical Institute, University of Oxford}
\date{\today}


\begin{document}
    \pagenumbering{gobble}
    \maketitle
    
    
    \begin{abstract}
        Magnetic confinement reactors---in particular tokamaks---offer one of the most promising options for achieving practical nuclear fusion, with the potential to provide virtually limitless, clean energy. The theoretical and numerical modeling of tokamak plasmas is simultaneously an essential component of effective reactor design, and a great research barrier. Tokamak operational conditions exhibit comparatively low Knudsen numbers. Kinetic effects, including kinetic waves and instabilities, Landau damping, bump-on-tail instabilities and more, are therefore highly influential in tokamak plasma dynamics. Purely fluid models are inherently incapable of capturing these effects, whereas the high dimensionality in purely kinetic models render them practically intractable for most relevant purposes.

        We consider a $\delta\!f$ decomposition model, with a macroscopic fluid background and microscopic kinetic correction, both fully coupled to each other. A similar manner of discretization is proposed to that used in the recent \texttt{STRUPHY} code \cite{Holderied_Possanner_Wang_2021, Holderied_2022, Li_et_al_2023} with a finite-element model for the background and a pseudo-particle/PiC model for the correction.

        The fluid background satisfies the full, non-linear, resistive, compressible, Hall MHD equations. \cite{Laakmann_Hu_Farrell_2022} introduces finite-element(-in-space) implicit timesteppers for the incompressible analogue to this system with structure-preserving (SP) properties in the ideal case, alongside parameter-robust preconditioners. We show that these timesteppers can derive from a finite-element-in-time (FET) (and finite-element-in-space) interpretation. The benefits of this reformulation are discussed, including the derivation of timesteppers that are higher order in time, and the quantifiable dissipative SP properties in the non-ideal, resistive case.
        
        We discuss possible options for extending this FET approach to timesteppers for the compressible case.

        The kinetic corrections satisfy linearized Boltzmann equations. Using a Lénard--Bernstein collision operator, these take Fokker--Planck-like forms \cite{Fokker_1914, Planck_1917} wherein pseudo-particles in the numerical model obey the neoclassical transport equations, with particle-independent Brownian drift terms. This offers a rigorous methodology for incorporating collisions into the particle transport model, without coupling the equations of motions for each particle.
        
        Works by Chen, Chacón et al. \cite{Chen_Chacón_Barnes_2011, Chacón_Chen_Barnes_2013, Chen_Chacón_2014, Chen_Chacón_2015} have developed structure-preserving particle pushers for neoclassical transport in the Vlasov equations, derived from Crank--Nicolson integrators. We show these too can can derive from a FET interpretation, similarly offering potential extensions to higher-order-in-time particle pushers. The FET formulation is used also to consider how the stochastic drift terms can be incorporated into the pushers. Stochastic gyrokinetic expansions are also discussed.

        Different options for the numerical implementation of these schemes are considered.

        Due to the efficacy of FET in the development of SP timesteppers for both the fluid and kinetic component, we hope this approach will prove effective in the future for developing SP timesteppers for the full hybrid model. We hope this will give us the opportunity to incorporate previously inaccessible kinetic effects into the highly effective, modern, finite-element MHD models.
    \end{abstract}
    
    
    \newpage
    \tableofcontents
    
    
    \newpage
    \pagenumbering{arabic}
    %\linenumbers\renewcommand\thelinenumber{\color{black!50}\arabic{linenumber}}
            \input{0 - introduction/main.tex}
        \part{Research}
            \input{1 - low-noise PiC models/main.tex}
            \input{2 - kinetic component/main.tex}
            \input{3 - fluid component/main.tex}
            \input{4 - numerical implementation/main.tex}
        \part{Project Overview}
            \input{5 - research plan/main.tex}
            \input{6 - summary/main.tex}
    
    
    %\section{}
    \newpage
    \pagenumbering{gobble}
        \printbibliography


    \newpage
    \pagenumbering{roman}
    \appendix
        \part{Appendices}
            \input{8 - Hilbert complexes/main.tex}
            \input{9 - weak conservation proofs/main.tex}
\end{document}

            \documentclass[12pt, a4paper]{report}

\input{template/main.tex}

\title{\BA{Title in Progress...}}
\author{Boris Andrews}
\affil{Mathematical Institute, University of Oxford}
\date{\today}


\begin{document}
    \pagenumbering{gobble}
    \maketitle
    
    
    \begin{abstract}
        Magnetic confinement reactors---in particular tokamaks---offer one of the most promising options for achieving practical nuclear fusion, with the potential to provide virtually limitless, clean energy. The theoretical and numerical modeling of tokamak plasmas is simultaneously an essential component of effective reactor design, and a great research barrier. Tokamak operational conditions exhibit comparatively low Knudsen numbers. Kinetic effects, including kinetic waves and instabilities, Landau damping, bump-on-tail instabilities and more, are therefore highly influential in tokamak plasma dynamics. Purely fluid models are inherently incapable of capturing these effects, whereas the high dimensionality in purely kinetic models render them practically intractable for most relevant purposes.

        We consider a $\delta\!f$ decomposition model, with a macroscopic fluid background and microscopic kinetic correction, both fully coupled to each other. A similar manner of discretization is proposed to that used in the recent \texttt{STRUPHY} code \cite{Holderied_Possanner_Wang_2021, Holderied_2022, Li_et_al_2023} with a finite-element model for the background and a pseudo-particle/PiC model for the correction.

        The fluid background satisfies the full, non-linear, resistive, compressible, Hall MHD equations. \cite{Laakmann_Hu_Farrell_2022} introduces finite-element(-in-space) implicit timesteppers for the incompressible analogue to this system with structure-preserving (SP) properties in the ideal case, alongside parameter-robust preconditioners. We show that these timesteppers can derive from a finite-element-in-time (FET) (and finite-element-in-space) interpretation. The benefits of this reformulation are discussed, including the derivation of timesteppers that are higher order in time, and the quantifiable dissipative SP properties in the non-ideal, resistive case.
        
        We discuss possible options for extending this FET approach to timesteppers for the compressible case.

        The kinetic corrections satisfy linearized Boltzmann equations. Using a Lénard--Bernstein collision operator, these take Fokker--Planck-like forms \cite{Fokker_1914, Planck_1917} wherein pseudo-particles in the numerical model obey the neoclassical transport equations, with particle-independent Brownian drift terms. This offers a rigorous methodology for incorporating collisions into the particle transport model, without coupling the equations of motions for each particle.
        
        Works by Chen, Chacón et al. \cite{Chen_Chacón_Barnes_2011, Chacón_Chen_Barnes_2013, Chen_Chacón_2014, Chen_Chacón_2015} have developed structure-preserving particle pushers for neoclassical transport in the Vlasov equations, derived from Crank--Nicolson integrators. We show these too can can derive from a FET interpretation, similarly offering potential extensions to higher-order-in-time particle pushers. The FET formulation is used also to consider how the stochastic drift terms can be incorporated into the pushers. Stochastic gyrokinetic expansions are also discussed.

        Different options for the numerical implementation of these schemes are considered.

        Due to the efficacy of FET in the development of SP timesteppers for both the fluid and kinetic component, we hope this approach will prove effective in the future for developing SP timesteppers for the full hybrid model. We hope this will give us the opportunity to incorporate previously inaccessible kinetic effects into the highly effective, modern, finite-element MHD models.
    \end{abstract}
    
    
    \newpage
    \tableofcontents
    
    
    \newpage
    \pagenumbering{arabic}
    %\linenumbers\renewcommand\thelinenumber{\color{black!50}\arabic{linenumber}}
            \input{0 - introduction/main.tex}
        \part{Research}
            \input{1 - low-noise PiC models/main.tex}
            \input{2 - kinetic component/main.tex}
            \input{3 - fluid component/main.tex}
            \input{4 - numerical implementation/main.tex}
        \part{Project Overview}
            \input{5 - research plan/main.tex}
            \input{6 - summary/main.tex}
    
    
    %\section{}
    \newpage
    \pagenumbering{gobble}
        \printbibliography


    \newpage
    \pagenumbering{roman}
    \appendix
        \part{Appendices}
            \input{8 - Hilbert complexes/main.tex}
            \input{9 - weak conservation proofs/main.tex}
\end{document}

            \documentclass[12pt, a4paper]{report}

\input{template/main.tex}

\title{\BA{Title in Progress...}}
\author{Boris Andrews}
\affil{Mathematical Institute, University of Oxford}
\date{\today}


\begin{document}
    \pagenumbering{gobble}
    \maketitle
    
    
    \begin{abstract}
        Magnetic confinement reactors---in particular tokamaks---offer one of the most promising options for achieving practical nuclear fusion, with the potential to provide virtually limitless, clean energy. The theoretical and numerical modeling of tokamak plasmas is simultaneously an essential component of effective reactor design, and a great research barrier. Tokamak operational conditions exhibit comparatively low Knudsen numbers. Kinetic effects, including kinetic waves and instabilities, Landau damping, bump-on-tail instabilities and more, are therefore highly influential in tokamak plasma dynamics. Purely fluid models are inherently incapable of capturing these effects, whereas the high dimensionality in purely kinetic models render them practically intractable for most relevant purposes.

        We consider a $\delta\!f$ decomposition model, with a macroscopic fluid background and microscopic kinetic correction, both fully coupled to each other. A similar manner of discretization is proposed to that used in the recent \texttt{STRUPHY} code \cite{Holderied_Possanner_Wang_2021, Holderied_2022, Li_et_al_2023} with a finite-element model for the background and a pseudo-particle/PiC model for the correction.

        The fluid background satisfies the full, non-linear, resistive, compressible, Hall MHD equations. \cite{Laakmann_Hu_Farrell_2022} introduces finite-element(-in-space) implicit timesteppers for the incompressible analogue to this system with structure-preserving (SP) properties in the ideal case, alongside parameter-robust preconditioners. We show that these timesteppers can derive from a finite-element-in-time (FET) (and finite-element-in-space) interpretation. The benefits of this reformulation are discussed, including the derivation of timesteppers that are higher order in time, and the quantifiable dissipative SP properties in the non-ideal, resistive case.
        
        We discuss possible options for extending this FET approach to timesteppers for the compressible case.

        The kinetic corrections satisfy linearized Boltzmann equations. Using a Lénard--Bernstein collision operator, these take Fokker--Planck-like forms \cite{Fokker_1914, Planck_1917} wherein pseudo-particles in the numerical model obey the neoclassical transport equations, with particle-independent Brownian drift terms. This offers a rigorous methodology for incorporating collisions into the particle transport model, without coupling the equations of motions for each particle.
        
        Works by Chen, Chacón et al. \cite{Chen_Chacón_Barnes_2011, Chacón_Chen_Barnes_2013, Chen_Chacón_2014, Chen_Chacón_2015} have developed structure-preserving particle pushers for neoclassical transport in the Vlasov equations, derived from Crank--Nicolson integrators. We show these too can can derive from a FET interpretation, similarly offering potential extensions to higher-order-in-time particle pushers. The FET formulation is used also to consider how the stochastic drift terms can be incorporated into the pushers. Stochastic gyrokinetic expansions are also discussed.

        Different options for the numerical implementation of these schemes are considered.

        Due to the efficacy of FET in the development of SP timesteppers for both the fluid and kinetic component, we hope this approach will prove effective in the future for developing SP timesteppers for the full hybrid model. We hope this will give us the opportunity to incorporate previously inaccessible kinetic effects into the highly effective, modern, finite-element MHD models.
    \end{abstract}
    
    
    \newpage
    \tableofcontents
    
    
    \newpage
    \pagenumbering{arabic}
    %\linenumbers\renewcommand\thelinenumber{\color{black!50}\arabic{linenumber}}
            \input{0 - introduction/main.tex}
        \part{Research}
            \input{1 - low-noise PiC models/main.tex}
            \input{2 - kinetic component/main.tex}
            \input{3 - fluid component/main.tex}
            \input{4 - numerical implementation/main.tex}
        \part{Project Overview}
            \input{5 - research plan/main.tex}
            \input{6 - summary/main.tex}
    
    
    %\section{}
    \newpage
    \pagenumbering{gobble}
        \printbibliography


    \newpage
    \pagenumbering{roman}
    \appendix
        \part{Appendices}
            \input{8 - Hilbert complexes/main.tex}
            \input{9 - weak conservation proofs/main.tex}
\end{document}

        \part{Project Overview}
            \documentclass[12pt, a4paper]{report}

\input{template/main.tex}

\title{\BA{Title in Progress...}}
\author{Boris Andrews}
\affil{Mathematical Institute, University of Oxford}
\date{\today}


\begin{document}
    \pagenumbering{gobble}
    \maketitle
    
    
    \begin{abstract}
        Magnetic confinement reactors---in particular tokamaks---offer one of the most promising options for achieving practical nuclear fusion, with the potential to provide virtually limitless, clean energy. The theoretical and numerical modeling of tokamak plasmas is simultaneously an essential component of effective reactor design, and a great research barrier. Tokamak operational conditions exhibit comparatively low Knudsen numbers. Kinetic effects, including kinetic waves and instabilities, Landau damping, bump-on-tail instabilities and more, are therefore highly influential in tokamak plasma dynamics. Purely fluid models are inherently incapable of capturing these effects, whereas the high dimensionality in purely kinetic models render them practically intractable for most relevant purposes.

        We consider a $\delta\!f$ decomposition model, with a macroscopic fluid background and microscopic kinetic correction, both fully coupled to each other. A similar manner of discretization is proposed to that used in the recent \texttt{STRUPHY} code \cite{Holderied_Possanner_Wang_2021, Holderied_2022, Li_et_al_2023} with a finite-element model for the background and a pseudo-particle/PiC model for the correction.

        The fluid background satisfies the full, non-linear, resistive, compressible, Hall MHD equations. \cite{Laakmann_Hu_Farrell_2022} introduces finite-element(-in-space) implicit timesteppers for the incompressible analogue to this system with structure-preserving (SP) properties in the ideal case, alongside parameter-robust preconditioners. We show that these timesteppers can derive from a finite-element-in-time (FET) (and finite-element-in-space) interpretation. The benefits of this reformulation are discussed, including the derivation of timesteppers that are higher order in time, and the quantifiable dissipative SP properties in the non-ideal, resistive case.
        
        We discuss possible options for extending this FET approach to timesteppers for the compressible case.

        The kinetic corrections satisfy linearized Boltzmann equations. Using a Lénard--Bernstein collision operator, these take Fokker--Planck-like forms \cite{Fokker_1914, Planck_1917} wherein pseudo-particles in the numerical model obey the neoclassical transport equations, with particle-independent Brownian drift terms. This offers a rigorous methodology for incorporating collisions into the particle transport model, without coupling the equations of motions for each particle.
        
        Works by Chen, Chacón et al. \cite{Chen_Chacón_Barnes_2011, Chacón_Chen_Barnes_2013, Chen_Chacón_2014, Chen_Chacón_2015} have developed structure-preserving particle pushers for neoclassical transport in the Vlasov equations, derived from Crank--Nicolson integrators. We show these too can can derive from a FET interpretation, similarly offering potential extensions to higher-order-in-time particle pushers. The FET formulation is used also to consider how the stochastic drift terms can be incorporated into the pushers. Stochastic gyrokinetic expansions are also discussed.

        Different options for the numerical implementation of these schemes are considered.

        Due to the efficacy of FET in the development of SP timesteppers for both the fluid and kinetic component, we hope this approach will prove effective in the future for developing SP timesteppers for the full hybrid model. We hope this will give us the opportunity to incorporate previously inaccessible kinetic effects into the highly effective, modern, finite-element MHD models.
    \end{abstract}
    
    
    \newpage
    \tableofcontents
    
    
    \newpage
    \pagenumbering{arabic}
    %\linenumbers\renewcommand\thelinenumber{\color{black!50}\arabic{linenumber}}
            \input{0 - introduction/main.tex}
        \part{Research}
            \input{1 - low-noise PiC models/main.tex}
            \input{2 - kinetic component/main.tex}
            \input{3 - fluid component/main.tex}
            \input{4 - numerical implementation/main.tex}
        \part{Project Overview}
            \input{5 - research plan/main.tex}
            \input{6 - summary/main.tex}
    
    
    %\section{}
    \newpage
    \pagenumbering{gobble}
        \printbibliography


    \newpage
    \pagenumbering{roman}
    \appendix
        \part{Appendices}
            \input{8 - Hilbert complexes/main.tex}
            \input{9 - weak conservation proofs/main.tex}
\end{document}

            \documentclass[12pt, a4paper]{report}

\input{template/main.tex}

\title{\BA{Title in Progress...}}
\author{Boris Andrews}
\affil{Mathematical Institute, University of Oxford}
\date{\today}


\begin{document}
    \pagenumbering{gobble}
    \maketitle
    
    
    \begin{abstract}
        Magnetic confinement reactors---in particular tokamaks---offer one of the most promising options for achieving practical nuclear fusion, with the potential to provide virtually limitless, clean energy. The theoretical and numerical modeling of tokamak plasmas is simultaneously an essential component of effective reactor design, and a great research barrier. Tokamak operational conditions exhibit comparatively low Knudsen numbers. Kinetic effects, including kinetic waves and instabilities, Landau damping, bump-on-tail instabilities and more, are therefore highly influential in tokamak plasma dynamics. Purely fluid models are inherently incapable of capturing these effects, whereas the high dimensionality in purely kinetic models render them practically intractable for most relevant purposes.

        We consider a $\delta\!f$ decomposition model, with a macroscopic fluid background and microscopic kinetic correction, both fully coupled to each other. A similar manner of discretization is proposed to that used in the recent \texttt{STRUPHY} code \cite{Holderied_Possanner_Wang_2021, Holderied_2022, Li_et_al_2023} with a finite-element model for the background and a pseudo-particle/PiC model for the correction.

        The fluid background satisfies the full, non-linear, resistive, compressible, Hall MHD equations. \cite{Laakmann_Hu_Farrell_2022} introduces finite-element(-in-space) implicit timesteppers for the incompressible analogue to this system with structure-preserving (SP) properties in the ideal case, alongside parameter-robust preconditioners. We show that these timesteppers can derive from a finite-element-in-time (FET) (and finite-element-in-space) interpretation. The benefits of this reformulation are discussed, including the derivation of timesteppers that are higher order in time, and the quantifiable dissipative SP properties in the non-ideal, resistive case.
        
        We discuss possible options for extending this FET approach to timesteppers for the compressible case.

        The kinetic corrections satisfy linearized Boltzmann equations. Using a Lénard--Bernstein collision operator, these take Fokker--Planck-like forms \cite{Fokker_1914, Planck_1917} wherein pseudo-particles in the numerical model obey the neoclassical transport equations, with particle-independent Brownian drift terms. This offers a rigorous methodology for incorporating collisions into the particle transport model, without coupling the equations of motions for each particle.
        
        Works by Chen, Chacón et al. \cite{Chen_Chacón_Barnes_2011, Chacón_Chen_Barnes_2013, Chen_Chacón_2014, Chen_Chacón_2015} have developed structure-preserving particle pushers for neoclassical transport in the Vlasov equations, derived from Crank--Nicolson integrators. We show these too can can derive from a FET interpretation, similarly offering potential extensions to higher-order-in-time particle pushers. The FET formulation is used also to consider how the stochastic drift terms can be incorporated into the pushers. Stochastic gyrokinetic expansions are also discussed.

        Different options for the numerical implementation of these schemes are considered.

        Due to the efficacy of FET in the development of SP timesteppers for both the fluid and kinetic component, we hope this approach will prove effective in the future for developing SP timesteppers for the full hybrid model. We hope this will give us the opportunity to incorporate previously inaccessible kinetic effects into the highly effective, modern, finite-element MHD models.
    \end{abstract}
    
    
    \newpage
    \tableofcontents
    
    
    \newpage
    \pagenumbering{arabic}
    %\linenumbers\renewcommand\thelinenumber{\color{black!50}\arabic{linenumber}}
            \input{0 - introduction/main.tex}
        \part{Research}
            \input{1 - low-noise PiC models/main.tex}
            \input{2 - kinetic component/main.tex}
            \input{3 - fluid component/main.tex}
            \input{4 - numerical implementation/main.tex}
        \part{Project Overview}
            \input{5 - research plan/main.tex}
            \input{6 - summary/main.tex}
    
    
    %\section{}
    \newpage
    \pagenumbering{gobble}
        \printbibliography


    \newpage
    \pagenumbering{roman}
    \appendix
        \part{Appendices}
            \input{8 - Hilbert complexes/main.tex}
            \input{9 - weak conservation proofs/main.tex}
\end{document}

    
    
    %\section{}
    \newpage
    \pagenumbering{gobble}
        \printbibliography


    \newpage
    \pagenumbering{roman}
    \appendix
        \part{Appendices}
            \documentclass[12pt, a4paper]{report}

\input{template/main.tex}

\title{\BA{Title in Progress...}}
\author{Boris Andrews}
\affil{Mathematical Institute, University of Oxford}
\date{\today}


\begin{document}
    \pagenumbering{gobble}
    \maketitle
    
    
    \begin{abstract}
        Magnetic confinement reactors---in particular tokamaks---offer one of the most promising options for achieving practical nuclear fusion, with the potential to provide virtually limitless, clean energy. The theoretical and numerical modeling of tokamak plasmas is simultaneously an essential component of effective reactor design, and a great research barrier. Tokamak operational conditions exhibit comparatively low Knudsen numbers. Kinetic effects, including kinetic waves and instabilities, Landau damping, bump-on-tail instabilities and more, are therefore highly influential in tokamak plasma dynamics. Purely fluid models are inherently incapable of capturing these effects, whereas the high dimensionality in purely kinetic models render them practically intractable for most relevant purposes.

        We consider a $\delta\!f$ decomposition model, with a macroscopic fluid background and microscopic kinetic correction, both fully coupled to each other. A similar manner of discretization is proposed to that used in the recent \texttt{STRUPHY} code \cite{Holderied_Possanner_Wang_2021, Holderied_2022, Li_et_al_2023} with a finite-element model for the background and a pseudo-particle/PiC model for the correction.

        The fluid background satisfies the full, non-linear, resistive, compressible, Hall MHD equations. \cite{Laakmann_Hu_Farrell_2022} introduces finite-element(-in-space) implicit timesteppers for the incompressible analogue to this system with structure-preserving (SP) properties in the ideal case, alongside parameter-robust preconditioners. We show that these timesteppers can derive from a finite-element-in-time (FET) (and finite-element-in-space) interpretation. The benefits of this reformulation are discussed, including the derivation of timesteppers that are higher order in time, and the quantifiable dissipative SP properties in the non-ideal, resistive case.
        
        We discuss possible options for extending this FET approach to timesteppers for the compressible case.

        The kinetic corrections satisfy linearized Boltzmann equations. Using a Lénard--Bernstein collision operator, these take Fokker--Planck-like forms \cite{Fokker_1914, Planck_1917} wherein pseudo-particles in the numerical model obey the neoclassical transport equations, with particle-independent Brownian drift terms. This offers a rigorous methodology for incorporating collisions into the particle transport model, without coupling the equations of motions for each particle.
        
        Works by Chen, Chacón et al. \cite{Chen_Chacón_Barnes_2011, Chacón_Chen_Barnes_2013, Chen_Chacón_2014, Chen_Chacón_2015} have developed structure-preserving particle pushers for neoclassical transport in the Vlasov equations, derived from Crank--Nicolson integrators. We show these too can can derive from a FET interpretation, similarly offering potential extensions to higher-order-in-time particle pushers. The FET formulation is used also to consider how the stochastic drift terms can be incorporated into the pushers. Stochastic gyrokinetic expansions are also discussed.

        Different options for the numerical implementation of these schemes are considered.

        Due to the efficacy of FET in the development of SP timesteppers for both the fluid and kinetic component, we hope this approach will prove effective in the future for developing SP timesteppers for the full hybrid model. We hope this will give us the opportunity to incorporate previously inaccessible kinetic effects into the highly effective, modern, finite-element MHD models.
    \end{abstract}
    
    
    \newpage
    \tableofcontents
    
    
    \newpage
    \pagenumbering{arabic}
    %\linenumbers\renewcommand\thelinenumber{\color{black!50}\arabic{linenumber}}
            \input{0 - introduction/main.tex}
        \part{Research}
            \input{1 - low-noise PiC models/main.tex}
            \input{2 - kinetic component/main.tex}
            \input{3 - fluid component/main.tex}
            \input{4 - numerical implementation/main.tex}
        \part{Project Overview}
            \input{5 - research plan/main.tex}
            \input{6 - summary/main.tex}
    
    
    %\section{}
    \newpage
    \pagenumbering{gobble}
        \printbibliography


    \newpage
    \pagenumbering{roman}
    \appendix
        \part{Appendices}
            \input{8 - Hilbert complexes/main.tex}
            \input{9 - weak conservation proofs/main.tex}
\end{document}

            \documentclass[12pt, a4paper]{report}

\input{template/main.tex}

\title{\BA{Title in Progress...}}
\author{Boris Andrews}
\affil{Mathematical Institute, University of Oxford}
\date{\today}


\begin{document}
    \pagenumbering{gobble}
    \maketitle
    
    
    \begin{abstract}
        Magnetic confinement reactors---in particular tokamaks---offer one of the most promising options for achieving practical nuclear fusion, with the potential to provide virtually limitless, clean energy. The theoretical and numerical modeling of tokamak plasmas is simultaneously an essential component of effective reactor design, and a great research barrier. Tokamak operational conditions exhibit comparatively low Knudsen numbers. Kinetic effects, including kinetic waves and instabilities, Landau damping, bump-on-tail instabilities and more, are therefore highly influential in tokamak plasma dynamics. Purely fluid models are inherently incapable of capturing these effects, whereas the high dimensionality in purely kinetic models render them practically intractable for most relevant purposes.

        We consider a $\delta\!f$ decomposition model, with a macroscopic fluid background and microscopic kinetic correction, both fully coupled to each other. A similar manner of discretization is proposed to that used in the recent \texttt{STRUPHY} code \cite{Holderied_Possanner_Wang_2021, Holderied_2022, Li_et_al_2023} with a finite-element model for the background and a pseudo-particle/PiC model for the correction.

        The fluid background satisfies the full, non-linear, resistive, compressible, Hall MHD equations. \cite{Laakmann_Hu_Farrell_2022} introduces finite-element(-in-space) implicit timesteppers for the incompressible analogue to this system with structure-preserving (SP) properties in the ideal case, alongside parameter-robust preconditioners. We show that these timesteppers can derive from a finite-element-in-time (FET) (and finite-element-in-space) interpretation. The benefits of this reformulation are discussed, including the derivation of timesteppers that are higher order in time, and the quantifiable dissipative SP properties in the non-ideal, resistive case.
        
        We discuss possible options for extending this FET approach to timesteppers for the compressible case.

        The kinetic corrections satisfy linearized Boltzmann equations. Using a Lénard--Bernstein collision operator, these take Fokker--Planck-like forms \cite{Fokker_1914, Planck_1917} wherein pseudo-particles in the numerical model obey the neoclassical transport equations, with particle-independent Brownian drift terms. This offers a rigorous methodology for incorporating collisions into the particle transport model, without coupling the equations of motions for each particle.
        
        Works by Chen, Chacón et al. \cite{Chen_Chacón_Barnes_2011, Chacón_Chen_Barnes_2013, Chen_Chacón_2014, Chen_Chacón_2015} have developed structure-preserving particle pushers for neoclassical transport in the Vlasov equations, derived from Crank--Nicolson integrators. We show these too can can derive from a FET interpretation, similarly offering potential extensions to higher-order-in-time particle pushers. The FET formulation is used also to consider how the stochastic drift terms can be incorporated into the pushers. Stochastic gyrokinetic expansions are also discussed.

        Different options for the numerical implementation of these schemes are considered.

        Due to the efficacy of FET in the development of SP timesteppers for both the fluid and kinetic component, we hope this approach will prove effective in the future for developing SP timesteppers for the full hybrid model. We hope this will give us the opportunity to incorporate previously inaccessible kinetic effects into the highly effective, modern, finite-element MHD models.
    \end{abstract}
    
    
    \newpage
    \tableofcontents
    
    
    \newpage
    \pagenumbering{arabic}
    %\linenumbers\renewcommand\thelinenumber{\color{black!50}\arabic{linenumber}}
            \input{0 - introduction/main.tex}
        \part{Research}
            \input{1 - low-noise PiC models/main.tex}
            \input{2 - kinetic component/main.tex}
            \input{3 - fluid component/main.tex}
            \input{4 - numerical implementation/main.tex}
        \part{Project Overview}
            \input{5 - research plan/main.tex}
            \input{6 - summary/main.tex}
    
    
    %\section{}
    \newpage
    \pagenumbering{gobble}
        \printbibliography


    \newpage
    \pagenumbering{roman}
    \appendix
        \part{Appendices}
            \input{8 - Hilbert complexes/main.tex}
            \input{9 - weak conservation proofs/main.tex}
\end{document}

\end{document}

\end{document}

            \documentclass[12pt, a4paper]{report}

\documentclass[12pt, a4paper]{report}

\documentclass[12pt, a4paper]{report}

\input{template/main.tex}

\title{\BA{Title in Progress...}}
\author{Boris Andrews}
\affil{Mathematical Institute, University of Oxford}
\date{\today}


\begin{document}
    \pagenumbering{gobble}
    \maketitle
    
    
    \begin{abstract}
        Magnetic confinement reactors---in particular tokamaks---offer one of the most promising options for achieving practical nuclear fusion, with the potential to provide virtually limitless, clean energy. The theoretical and numerical modeling of tokamak plasmas is simultaneously an essential component of effective reactor design, and a great research barrier. Tokamak operational conditions exhibit comparatively low Knudsen numbers. Kinetic effects, including kinetic waves and instabilities, Landau damping, bump-on-tail instabilities and more, are therefore highly influential in tokamak plasma dynamics. Purely fluid models are inherently incapable of capturing these effects, whereas the high dimensionality in purely kinetic models render them practically intractable for most relevant purposes.

        We consider a $\delta\!f$ decomposition model, with a macroscopic fluid background and microscopic kinetic correction, both fully coupled to each other. A similar manner of discretization is proposed to that used in the recent \texttt{STRUPHY} code \cite{Holderied_Possanner_Wang_2021, Holderied_2022, Li_et_al_2023} with a finite-element model for the background and a pseudo-particle/PiC model for the correction.

        The fluid background satisfies the full, non-linear, resistive, compressible, Hall MHD equations. \cite{Laakmann_Hu_Farrell_2022} introduces finite-element(-in-space) implicit timesteppers for the incompressible analogue to this system with structure-preserving (SP) properties in the ideal case, alongside parameter-robust preconditioners. We show that these timesteppers can derive from a finite-element-in-time (FET) (and finite-element-in-space) interpretation. The benefits of this reformulation are discussed, including the derivation of timesteppers that are higher order in time, and the quantifiable dissipative SP properties in the non-ideal, resistive case.
        
        We discuss possible options for extending this FET approach to timesteppers for the compressible case.

        The kinetic corrections satisfy linearized Boltzmann equations. Using a Lénard--Bernstein collision operator, these take Fokker--Planck-like forms \cite{Fokker_1914, Planck_1917} wherein pseudo-particles in the numerical model obey the neoclassical transport equations, with particle-independent Brownian drift terms. This offers a rigorous methodology for incorporating collisions into the particle transport model, without coupling the equations of motions for each particle.
        
        Works by Chen, Chacón et al. \cite{Chen_Chacón_Barnes_2011, Chacón_Chen_Barnes_2013, Chen_Chacón_2014, Chen_Chacón_2015} have developed structure-preserving particle pushers for neoclassical transport in the Vlasov equations, derived from Crank--Nicolson integrators. We show these too can can derive from a FET interpretation, similarly offering potential extensions to higher-order-in-time particle pushers. The FET formulation is used also to consider how the stochastic drift terms can be incorporated into the pushers. Stochastic gyrokinetic expansions are also discussed.

        Different options for the numerical implementation of these schemes are considered.

        Due to the efficacy of FET in the development of SP timesteppers for both the fluid and kinetic component, we hope this approach will prove effective in the future for developing SP timesteppers for the full hybrid model. We hope this will give us the opportunity to incorporate previously inaccessible kinetic effects into the highly effective, modern, finite-element MHD models.
    \end{abstract}
    
    
    \newpage
    \tableofcontents
    
    
    \newpage
    \pagenumbering{arabic}
    %\linenumbers\renewcommand\thelinenumber{\color{black!50}\arabic{linenumber}}
            \input{0 - introduction/main.tex}
        \part{Research}
            \input{1 - low-noise PiC models/main.tex}
            \input{2 - kinetic component/main.tex}
            \input{3 - fluid component/main.tex}
            \input{4 - numerical implementation/main.tex}
        \part{Project Overview}
            \input{5 - research plan/main.tex}
            \input{6 - summary/main.tex}
    
    
    %\section{}
    \newpage
    \pagenumbering{gobble}
        \printbibliography


    \newpage
    \pagenumbering{roman}
    \appendix
        \part{Appendices}
            \input{8 - Hilbert complexes/main.tex}
            \input{9 - weak conservation proofs/main.tex}
\end{document}


\title{\BA{Title in Progress...}}
\author{Boris Andrews}
\affil{Mathematical Institute, University of Oxford}
\date{\today}


\begin{document}
    \pagenumbering{gobble}
    \maketitle
    
    
    \begin{abstract}
        Magnetic confinement reactors---in particular tokamaks---offer one of the most promising options for achieving practical nuclear fusion, with the potential to provide virtually limitless, clean energy. The theoretical and numerical modeling of tokamak plasmas is simultaneously an essential component of effective reactor design, and a great research barrier. Tokamak operational conditions exhibit comparatively low Knudsen numbers. Kinetic effects, including kinetic waves and instabilities, Landau damping, bump-on-tail instabilities and more, are therefore highly influential in tokamak plasma dynamics. Purely fluid models are inherently incapable of capturing these effects, whereas the high dimensionality in purely kinetic models render them practically intractable for most relevant purposes.

        We consider a $\delta\!f$ decomposition model, with a macroscopic fluid background and microscopic kinetic correction, both fully coupled to each other. A similar manner of discretization is proposed to that used in the recent \texttt{STRUPHY} code \cite{Holderied_Possanner_Wang_2021, Holderied_2022, Li_et_al_2023} with a finite-element model for the background and a pseudo-particle/PiC model for the correction.

        The fluid background satisfies the full, non-linear, resistive, compressible, Hall MHD equations. \cite{Laakmann_Hu_Farrell_2022} introduces finite-element(-in-space) implicit timesteppers for the incompressible analogue to this system with structure-preserving (SP) properties in the ideal case, alongside parameter-robust preconditioners. We show that these timesteppers can derive from a finite-element-in-time (FET) (and finite-element-in-space) interpretation. The benefits of this reformulation are discussed, including the derivation of timesteppers that are higher order in time, and the quantifiable dissipative SP properties in the non-ideal, resistive case.
        
        We discuss possible options for extending this FET approach to timesteppers for the compressible case.

        The kinetic corrections satisfy linearized Boltzmann equations. Using a Lénard--Bernstein collision operator, these take Fokker--Planck-like forms \cite{Fokker_1914, Planck_1917} wherein pseudo-particles in the numerical model obey the neoclassical transport equations, with particle-independent Brownian drift terms. This offers a rigorous methodology for incorporating collisions into the particle transport model, without coupling the equations of motions for each particle.
        
        Works by Chen, Chacón et al. \cite{Chen_Chacón_Barnes_2011, Chacón_Chen_Barnes_2013, Chen_Chacón_2014, Chen_Chacón_2015} have developed structure-preserving particle pushers for neoclassical transport in the Vlasov equations, derived from Crank--Nicolson integrators. We show these too can can derive from a FET interpretation, similarly offering potential extensions to higher-order-in-time particle pushers. The FET formulation is used also to consider how the stochastic drift terms can be incorporated into the pushers. Stochastic gyrokinetic expansions are also discussed.

        Different options for the numerical implementation of these schemes are considered.

        Due to the efficacy of FET in the development of SP timesteppers for both the fluid and kinetic component, we hope this approach will prove effective in the future for developing SP timesteppers for the full hybrid model. We hope this will give us the opportunity to incorporate previously inaccessible kinetic effects into the highly effective, modern, finite-element MHD models.
    \end{abstract}
    
    
    \newpage
    \tableofcontents
    
    
    \newpage
    \pagenumbering{arabic}
    %\linenumbers\renewcommand\thelinenumber{\color{black!50}\arabic{linenumber}}
            \documentclass[12pt, a4paper]{report}

\input{template/main.tex}

\title{\BA{Title in Progress...}}
\author{Boris Andrews}
\affil{Mathematical Institute, University of Oxford}
\date{\today}


\begin{document}
    \pagenumbering{gobble}
    \maketitle
    
    
    \begin{abstract}
        Magnetic confinement reactors---in particular tokamaks---offer one of the most promising options for achieving practical nuclear fusion, with the potential to provide virtually limitless, clean energy. The theoretical and numerical modeling of tokamak plasmas is simultaneously an essential component of effective reactor design, and a great research barrier. Tokamak operational conditions exhibit comparatively low Knudsen numbers. Kinetic effects, including kinetic waves and instabilities, Landau damping, bump-on-tail instabilities and more, are therefore highly influential in tokamak plasma dynamics. Purely fluid models are inherently incapable of capturing these effects, whereas the high dimensionality in purely kinetic models render them practically intractable for most relevant purposes.

        We consider a $\delta\!f$ decomposition model, with a macroscopic fluid background and microscopic kinetic correction, both fully coupled to each other. A similar manner of discretization is proposed to that used in the recent \texttt{STRUPHY} code \cite{Holderied_Possanner_Wang_2021, Holderied_2022, Li_et_al_2023} with a finite-element model for the background and a pseudo-particle/PiC model for the correction.

        The fluid background satisfies the full, non-linear, resistive, compressible, Hall MHD equations. \cite{Laakmann_Hu_Farrell_2022} introduces finite-element(-in-space) implicit timesteppers for the incompressible analogue to this system with structure-preserving (SP) properties in the ideal case, alongside parameter-robust preconditioners. We show that these timesteppers can derive from a finite-element-in-time (FET) (and finite-element-in-space) interpretation. The benefits of this reformulation are discussed, including the derivation of timesteppers that are higher order in time, and the quantifiable dissipative SP properties in the non-ideal, resistive case.
        
        We discuss possible options for extending this FET approach to timesteppers for the compressible case.

        The kinetic corrections satisfy linearized Boltzmann equations. Using a Lénard--Bernstein collision operator, these take Fokker--Planck-like forms \cite{Fokker_1914, Planck_1917} wherein pseudo-particles in the numerical model obey the neoclassical transport equations, with particle-independent Brownian drift terms. This offers a rigorous methodology for incorporating collisions into the particle transport model, without coupling the equations of motions for each particle.
        
        Works by Chen, Chacón et al. \cite{Chen_Chacón_Barnes_2011, Chacón_Chen_Barnes_2013, Chen_Chacón_2014, Chen_Chacón_2015} have developed structure-preserving particle pushers for neoclassical transport in the Vlasov equations, derived from Crank--Nicolson integrators. We show these too can can derive from a FET interpretation, similarly offering potential extensions to higher-order-in-time particle pushers. The FET formulation is used also to consider how the stochastic drift terms can be incorporated into the pushers. Stochastic gyrokinetic expansions are also discussed.

        Different options for the numerical implementation of these schemes are considered.

        Due to the efficacy of FET in the development of SP timesteppers for both the fluid and kinetic component, we hope this approach will prove effective in the future for developing SP timesteppers for the full hybrid model. We hope this will give us the opportunity to incorporate previously inaccessible kinetic effects into the highly effective, modern, finite-element MHD models.
    \end{abstract}
    
    
    \newpage
    \tableofcontents
    
    
    \newpage
    \pagenumbering{arabic}
    %\linenumbers\renewcommand\thelinenumber{\color{black!50}\arabic{linenumber}}
            \input{0 - introduction/main.tex}
        \part{Research}
            \input{1 - low-noise PiC models/main.tex}
            \input{2 - kinetic component/main.tex}
            \input{3 - fluid component/main.tex}
            \input{4 - numerical implementation/main.tex}
        \part{Project Overview}
            \input{5 - research plan/main.tex}
            \input{6 - summary/main.tex}
    
    
    %\section{}
    \newpage
    \pagenumbering{gobble}
        \printbibliography


    \newpage
    \pagenumbering{roman}
    \appendix
        \part{Appendices}
            \input{8 - Hilbert complexes/main.tex}
            \input{9 - weak conservation proofs/main.tex}
\end{document}

        \part{Research}
            \documentclass[12pt, a4paper]{report}

\input{template/main.tex}

\title{\BA{Title in Progress...}}
\author{Boris Andrews}
\affil{Mathematical Institute, University of Oxford}
\date{\today}


\begin{document}
    \pagenumbering{gobble}
    \maketitle
    
    
    \begin{abstract}
        Magnetic confinement reactors---in particular tokamaks---offer one of the most promising options for achieving practical nuclear fusion, with the potential to provide virtually limitless, clean energy. The theoretical and numerical modeling of tokamak plasmas is simultaneously an essential component of effective reactor design, and a great research barrier. Tokamak operational conditions exhibit comparatively low Knudsen numbers. Kinetic effects, including kinetic waves and instabilities, Landau damping, bump-on-tail instabilities and more, are therefore highly influential in tokamak plasma dynamics. Purely fluid models are inherently incapable of capturing these effects, whereas the high dimensionality in purely kinetic models render them practically intractable for most relevant purposes.

        We consider a $\delta\!f$ decomposition model, with a macroscopic fluid background and microscopic kinetic correction, both fully coupled to each other. A similar manner of discretization is proposed to that used in the recent \texttt{STRUPHY} code \cite{Holderied_Possanner_Wang_2021, Holderied_2022, Li_et_al_2023} with a finite-element model for the background and a pseudo-particle/PiC model for the correction.

        The fluid background satisfies the full, non-linear, resistive, compressible, Hall MHD equations. \cite{Laakmann_Hu_Farrell_2022} introduces finite-element(-in-space) implicit timesteppers for the incompressible analogue to this system with structure-preserving (SP) properties in the ideal case, alongside parameter-robust preconditioners. We show that these timesteppers can derive from a finite-element-in-time (FET) (and finite-element-in-space) interpretation. The benefits of this reformulation are discussed, including the derivation of timesteppers that are higher order in time, and the quantifiable dissipative SP properties in the non-ideal, resistive case.
        
        We discuss possible options for extending this FET approach to timesteppers for the compressible case.

        The kinetic corrections satisfy linearized Boltzmann equations. Using a Lénard--Bernstein collision operator, these take Fokker--Planck-like forms \cite{Fokker_1914, Planck_1917} wherein pseudo-particles in the numerical model obey the neoclassical transport equations, with particle-independent Brownian drift terms. This offers a rigorous methodology for incorporating collisions into the particle transport model, without coupling the equations of motions for each particle.
        
        Works by Chen, Chacón et al. \cite{Chen_Chacón_Barnes_2011, Chacón_Chen_Barnes_2013, Chen_Chacón_2014, Chen_Chacón_2015} have developed structure-preserving particle pushers for neoclassical transport in the Vlasov equations, derived from Crank--Nicolson integrators. We show these too can can derive from a FET interpretation, similarly offering potential extensions to higher-order-in-time particle pushers. The FET formulation is used also to consider how the stochastic drift terms can be incorporated into the pushers. Stochastic gyrokinetic expansions are also discussed.

        Different options for the numerical implementation of these schemes are considered.

        Due to the efficacy of FET in the development of SP timesteppers for both the fluid and kinetic component, we hope this approach will prove effective in the future for developing SP timesteppers for the full hybrid model. We hope this will give us the opportunity to incorporate previously inaccessible kinetic effects into the highly effective, modern, finite-element MHD models.
    \end{abstract}
    
    
    \newpage
    \tableofcontents
    
    
    \newpage
    \pagenumbering{arabic}
    %\linenumbers\renewcommand\thelinenumber{\color{black!50}\arabic{linenumber}}
            \input{0 - introduction/main.tex}
        \part{Research}
            \input{1 - low-noise PiC models/main.tex}
            \input{2 - kinetic component/main.tex}
            \input{3 - fluid component/main.tex}
            \input{4 - numerical implementation/main.tex}
        \part{Project Overview}
            \input{5 - research plan/main.tex}
            \input{6 - summary/main.tex}
    
    
    %\section{}
    \newpage
    \pagenumbering{gobble}
        \printbibliography


    \newpage
    \pagenumbering{roman}
    \appendix
        \part{Appendices}
            \input{8 - Hilbert complexes/main.tex}
            \input{9 - weak conservation proofs/main.tex}
\end{document}

            \documentclass[12pt, a4paper]{report}

\input{template/main.tex}

\title{\BA{Title in Progress...}}
\author{Boris Andrews}
\affil{Mathematical Institute, University of Oxford}
\date{\today}


\begin{document}
    \pagenumbering{gobble}
    \maketitle
    
    
    \begin{abstract}
        Magnetic confinement reactors---in particular tokamaks---offer one of the most promising options for achieving practical nuclear fusion, with the potential to provide virtually limitless, clean energy. The theoretical and numerical modeling of tokamak plasmas is simultaneously an essential component of effective reactor design, and a great research barrier. Tokamak operational conditions exhibit comparatively low Knudsen numbers. Kinetic effects, including kinetic waves and instabilities, Landau damping, bump-on-tail instabilities and more, are therefore highly influential in tokamak plasma dynamics. Purely fluid models are inherently incapable of capturing these effects, whereas the high dimensionality in purely kinetic models render them practically intractable for most relevant purposes.

        We consider a $\delta\!f$ decomposition model, with a macroscopic fluid background and microscopic kinetic correction, both fully coupled to each other. A similar manner of discretization is proposed to that used in the recent \texttt{STRUPHY} code \cite{Holderied_Possanner_Wang_2021, Holderied_2022, Li_et_al_2023} with a finite-element model for the background and a pseudo-particle/PiC model for the correction.

        The fluid background satisfies the full, non-linear, resistive, compressible, Hall MHD equations. \cite{Laakmann_Hu_Farrell_2022} introduces finite-element(-in-space) implicit timesteppers for the incompressible analogue to this system with structure-preserving (SP) properties in the ideal case, alongside parameter-robust preconditioners. We show that these timesteppers can derive from a finite-element-in-time (FET) (and finite-element-in-space) interpretation. The benefits of this reformulation are discussed, including the derivation of timesteppers that are higher order in time, and the quantifiable dissipative SP properties in the non-ideal, resistive case.
        
        We discuss possible options for extending this FET approach to timesteppers for the compressible case.

        The kinetic corrections satisfy linearized Boltzmann equations. Using a Lénard--Bernstein collision operator, these take Fokker--Planck-like forms \cite{Fokker_1914, Planck_1917} wherein pseudo-particles in the numerical model obey the neoclassical transport equations, with particle-independent Brownian drift terms. This offers a rigorous methodology for incorporating collisions into the particle transport model, without coupling the equations of motions for each particle.
        
        Works by Chen, Chacón et al. \cite{Chen_Chacón_Barnes_2011, Chacón_Chen_Barnes_2013, Chen_Chacón_2014, Chen_Chacón_2015} have developed structure-preserving particle pushers for neoclassical transport in the Vlasov equations, derived from Crank--Nicolson integrators. We show these too can can derive from a FET interpretation, similarly offering potential extensions to higher-order-in-time particle pushers. The FET formulation is used also to consider how the stochastic drift terms can be incorporated into the pushers. Stochastic gyrokinetic expansions are also discussed.

        Different options for the numerical implementation of these schemes are considered.

        Due to the efficacy of FET in the development of SP timesteppers for both the fluid and kinetic component, we hope this approach will prove effective in the future for developing SP timesteppers for the full hybrid model. We hope this will give us the opportunity to incorporate previously inaccessible kinetic effects into the highly effective, modern, finite-element MHD models.
    \end{abstract}
    
    
    \newpage
    \tableofcontents
    
    
    \newpage
    \pagenumbering{arabic}
    %\linenumbers\renewcommand\thelinenumber{\color{black!50}\arabic{linenumber}}
            \input{0 - introduction/main.tex}
        \part{Research}
            \input{1 - low-noise PiC models/main.tex}
            \input{2 - kinetic component/main.tex}
            \input{3 - fluid component/main.tex}
            \input{4 - numerical implementation/main.tex}
        \part{Project Overview}
            \input{5 - research plan/main.tex}
            \input{6 - summary/main.tex}
    
    
    %\section{}
    \newpage
    \pagenumbering{gobble}
        \printbibliography


    \newpage
    \pagenumbering{roman}
    \appendix
        \part{Appendices}
            \input{8 - Hilbert complexes/main.tex}
            \input{9 - weak conservation proofs/main.tex}
\end{document}

            \documentclass[12pt, a4paper]{report}

\input{template/main.tex}

\title{\BA{Title in Progress...}}
\author{Boris Andrews}
\affil{Mathematical Institute, University of Oxford}
\date{\today}


\begin{document}
    \pagenumbering{gobble}
    \maketitle
    
    
    \begin{abstract}
        Magnetic confinement reactors---in particular tokamaks---offer one of the most promising options for achieving practical nuclear fusion, with the potential to provide virtually limitless, clean energy. The theoretical and numerical modeling of tokamak plasmas is simultaneously an essential component of effective reactor design, and a great research barrier. Tokamak operational conditions exhibit comparatively low Knudsen numbers. Kinetic effects, including kinetic waves and instabilities, Landau damping, bump-on-tail instabilities and more, are therefore highly influential in tokamak plasma dynamics. Purely fluid models are inherently incapable of capturing these effects, whereas the high dimensionality in purely kinetic models render them practically intractable for most relevant purposes.

        We consider a $\delta\!f$ decomposition model, with a macroscopic fluid background and microscopic kinetic correction, both fully coupled to each other. A similar manner of discretization is proposed to that used in the recent \texttt{STRUPHY} code \cite{Holderied_Possanner_Wang_2021, Holderied_2022, Li_et_al_2023} with a finite-element model for the background and a pseudo-particle/PiC model for the correction.

        The fluid background satisfies the full, non-linear, resistive, compressible, Hall MHD equations. \cite{Laakmann_Hu_Farrell_2022} introduces finite-element(-in-space) implicit timesteppers for the incompressible analogue to this system with structure-preserving (SP) properties in the ideal case, alongside parameter-robust preconditioners. We show that these timesteppers can derive from a finite-element-in-time (FET) (and finite-element-in-space) interpretation. The benefits of this reformulation are discussed, including the derivation of timesteppers that are higher order in time, and the quantifiable dissipative SP properties in the non-ideal, resistive case.
        
        We discuss possible options for extending this FET approach to timesteppers for the compressible case.

        The kinetic corrections satisfy linearized Boltzmann equations. Using a Lénard--Bernstein collision operator, these take Fokker--Planck-like forms \cite{Fokker_1914, Planck_1917} wherein pseudo-particles in the numerical model obey the neoclassical transport equations, with particle-independent Brownian drift terms. This offers a rigorous methodology for incorporating collisions into the particle transport model, without coupling the equations of motions for each particle.
        
        Works by Chen, Chacón et al. \cite{Chen_Chacón_Barnes_2011, Chacón_Chen_Barnes_2013, Chen_Chacón_2014, Chen_Chacón_2015} have developed structure-preserving particle pushers for neoclassical transport in the Vlasov equations, derived from Crank--Nicolson integrators. We show these too can can derive from a FET interpretation, similarly offering potential extensions to higher-order-in-time particle pushers. The FET formulation is used also to consider how the stochastic drift terms can be incorporated into the pushers. Stochastic gyrokinetic expansions are also discussed.

        Different options for the numerical implementation of these schemes are considered.

        Due to the efficacy of FET in the development of SP timesteppers for both the fluid and kinetic component, we hope this approach will prove effective in the future for developing SP timesteppers for the full hybrid model. We hope this will give us the opportunity to incorporate previously inaccessible kinetic effects into the highly effective, modern, finite-element MHD models.
    \end{abstract}
    
    
    \newpage
    \tableofcontents
    
    
    \newpage
    \pagenumbering{arabic}
    %\linenumbers\renewcommand\thelinenumber{\color{black!50}\arabic{linenumber}}
            \input{0 - introduction/main.tex}
        \part{Research}
            \input{1 - low-noise PiC models/main.tex}
            \input{2 - kinetic component/main.tex}
            \input{3 - fluid component/main.tex}
            \input{4 - numerical implementation/main.tex}
        \part{Project Overview}
            \input{5 - research plan/main.tex}
            \input{6 - summary/main.tex}
    
    
    %\section{}
    \newpage
    \pagenumbering{gobble}
        \printbibliography


    \newpage
    \pagenumbering{roman}
    \appendix
        \part{Appendices}
            \input{8 - Hilbert complexes/main.tex}
            \input{9 - weak conservation proofs/main.tex}
\end{document}

            \documentclass[12pt, a4paper]{report}

\input{template/main.tex}

\title{\BA{Title in Progress...}}
\author{Boris Andrews}
\affil{Mathematical Institute, University of Oxford}
\date{\today}


\begin{document}
    \pagenumbering{gobble}
    \maketitle
    
    
    \begin{abstract}
        Magnetic confinement reactors---in particular tokamaks---offer one of the most promising options for achieving practical nuclear fusion, with the potential to provide virtually limitless, clean energy. The theoretical and numerical modeling of tokamak plasmas is simultaneously an essential component of effective reactor design, and a great research barrier. Tokamak operational conditions exhibit comparatively low Knudsen numbers. Kinetic effects, including kinetic waves and instabilities, Landau damping, bump-on-tail instabilities and more, are therefore highly influential in tokamak plasma dynamics. Purely fluid models are inherently incapable of capturing these effects, whereas the high dimensionality in purely kinetic models render them practically intractable for most relevant purposes.

        We consider a $\delta\!f$ decomposition model, with a macroscopic fluid background and microscopic kinetic correction, both fully coupled to each other. A similar manner of discretization is proposed to that used in the recent \texttt{STRUPHY} code \cite{Holderied_Possanner_Wang_2021, Holderied_2022, Li_et_al_2023} with a finite-element model for the background and a pseudo-particle/PiC model for the correction.

        The fluid background satisfies the full, non-linear, resistive, compressible, Hall MHD equations. \cite{Laakmann_Hu_Farrell_2022} introduces finite-element(-in-space) implicit timesteppers for the incompressible analogue to this system with structure-preserving (SP) properties in the ideal case, alongside parameter-robust preconditioners. We show that these timesteppers can derive from a finite-element-in-time (FET) (and finite-element-in-space) interpretation. The benefits of this reformulation are discussed, including the derivation of timesteppers that are higher order in time, and the quantifiable dissipative SP properties in the non-ideal, resistive case.
        
        We discuss possible options for extending this FET approach to timesteppers for the compressible case.

        The kinetic corrections satisfy linearized Boltzmann equations. Using a Lénard--Bernstein collision operator, these take Fokker--Planck-like forms \cite{Fokker_1914, Planck_1917} wherein pseudo-particles in the numerical model obey the neoclassical transport equations, with particle-independent Brownian drift terms. This offers a rigorous methodology for incorporating collisions into the particle transport model, without coupling the equations of motions for each particle.
        
        Works by Chen, Chacón et al. \cite{Chen_Chacón_Barnes_2011, Chacón_Chen_Barnes_2013, Chen_Chacón_2014, Chen_Chacón_2015} have developed structure-preserving particle pushers for neoclassical transport in the Vlasov equations, derived from Crank--Nicolson integrators. We show these too can can derive from a FET interpretation, similarly offering potential extensions to higher-order-in-time particle pushers. The FET formulation is used also to consider how the stochastic drift terms can be incorporated into the pushers. Stochastic gyrokinetic expansions are also discussed.

        Different options for the numerical implementation of these schemes are considered.

        Due to the efficacy of FET in the development of SP timesteppers for both the fluid and kinetic component, we hope this approach will prove effective in the future for developing SP timesteppers for the full hybrid model. We hope this will give us the opportunity to incorporate previously inaccessible kinetic effects into the highly effective, modern, finite-element MHD models.
    \end{abstract}
    
    
    \newpage
    \tableofcontents
    
    
    \newpage
    \pagenumbering{arabic}
    %\linenumbers\renewcommand\thelinenumber{\color{black!50}\arabic{linenumber}}
            \input{0 - introduction/main.tex}
        \part{Research}
            \input{1 - low-noise PiC models/main.tex}
            \input{2 - kinetic component/main.tex}
            \input{3 - fluid component/main.tex}
            \input{4 - numerical implementation/main.tex}
        \part{Project Overview}
            \input{5 - research plan/main.tex}
            \input{6 - summary/main.tex}
    
    
    %\section{}
    \newpage
    \pagenumbering{gobble}
        \printbibliography


    \newpage
    \pagenumbering{roman}
    \appendix
        \part{Appendices}
            \input{8 - Hilbert complexes/main.tex}
            \input{9 - weak conservation proofs/main.tex}
\end{document}

        \part{Project Overview}
            \documentclass[12pt, a4paper]{report}

\input{template/main.tex}

\title{\BA{Title in Progress...}}
\author{Boris Andrews}
\affil{Mathematical Institute, University of Oxford}
\date{\today}


\begin{document}
    \pagenumbering{gobble}
    \maketitle
    
    
    \begin{abstract}
        Magnetic confinement reactors---in particular tokamaks---offer one of the most promising options for achieving practical nuclear fusion, with the potential to provide virtually limitless, clean energy. The theoretical and numerical modeling of tokamak plasmas is simultaneously an essential component of effective reactor design, and a great research barrier. Tokamak operational conditions exhibit comparatively low Knudsen numbers. Kinetic effects, including kinetic waves and instabilities, Landau damping, bump-on-tail instabilities and more, are therefore highly influential in tokamak plasma dynamics. Purely fluid models are inherently incapable of capturing these effects, whereas the high dimensionality in purely kinetic models render them practically intractable for most relevant purposes.

        We consider a $\delta\!f$ decomposition model, with a macroscopic fluid background and microscopic kinetic correction, both fully coupled to each other. A similar manner of discretization is proposed to that used in the recent \texttt{STRUPHY} code \cite{Holderied_Possanner_Wang_2021, Holderied_2022, Li_et_al_2023} with a finite-element model for the background and a pseudo-particle/PiC model for the correction.

        The fluid background satisfies the full, non-linear, resistive, compressible, Hall MHD equations. \cite{Laakmann_Hu_Farrell_2022} introduces finite-element(-in-space) implicit timesteppers for the incompressible analogue to this system with structure-preserving (SP) properties in the ideal case, alongside parameter-robust preconditioners. We show that these timesteppers can derive from a finite-element-in-time (FET) (and finite-element-in-space) interpretation. The benefits of this reformulation are discussed, including the derivation of timesteppers that are higher order in time, and the quantifiable dissipative SP properties in the non-ideal, resistive case.
        
        We discuss possible options for extending this FET approach to timesteppers for the compressible case.

        The kinetic corrections satisfy linearized Boltzmann equations. Using a Lénard--Bernstein collision operator, these take Fokker--Planck-like forms \cite{Fokker_1914, Planck_1917} wherein pseudo-particles in the numerical model obey the neoclassical transport equations, with particle-independent Brownian drift terms. This offers a rigorous methodology for incorporating collisions into the particle transport model, without coupling the equations of motions for each particle.
        
        Works by Chen, Chacón et al. \cite{Chen_Chacón_Barnes_2011, Chacón_Chen_Barnes_2013, Chen_Chacón_2014, Chen_Chacón_2015} have developed structure-preserving particle pushers for neoclassical transport in the Vlasov equations, derived from Crank--Nicolson integrators. We show these too can can derive from a FET interpretation, similarly offering potential extensions to higher-order-in-time particle pushers. The FET formulation is used also to consider how the stochastic drift terms can be incorporated into the pushers. Stochastic gyrokinetic expansions are also discussed.

        Different options for the numerical implementation of these schemes are considered.

        Due to the efficacy of FET in the development of SP timesteppers for both the fluid and kinetic component, we hope this approach will prove effective in the future for developing SP timesteppers for the full hybrid model. We hope this will give us the opportunity to incorporate previously inaccessible kinetic effects into the highly effective, modern, finite-element MHD models.
    \end{abstract}
    
    
    \newpage
    \tableofcontents
    
    
    \newpage
    \pagenumbering{arabic}
    %\linenumbers\renewcommand\thelinenumber{\color{black!50}\arabic{linenumber}}
            \input{0 - introduction/main.tex}
        \part{Research}
            \input{1 - low-noise PiC models/main.tex}
            \input{2 - kinetic component/main.tex}
            \input{3 - fluid component/main.tex}
            \input{4 - numerical implementation/main.tex}
        \part{Project Overview}
            \input{5 - research plan/main.tex}
            \input{6 - summary/main.tex}
    
    
    %\section{}
    \newpage
    \pagenumbering{gobble}
        \printbibliography


    \newpage
    \pagenumbering{roman}
    \appendix
        \part{Appendices}
            \input{8 - Hilbert complexes/main.tex}
            \input{9 - weak conservation proofs/main.tex}
\end{document}

            \documentclass[12pt, a4paper]{report}

\input{template/main.tex}

\title{\BA{Title in Progress...}}
\author{Boris Andrews}
\affil{Mathematical Institute, University of Oxford}
\date{\today}


\begin{document}
    \pagenumbering{gobble}
    \maketitle
    
    
    \begin{abstract}
        Magnetic confinement reactors---in particular tokamaks---offer one of the most promising options for achieving practical nuclear fusion, with the potential to provide virtually limitless, clean energy. The theoretical and numerical modeling of tokamak plasmas is simultaneously an essential component of effective reactor design, and a great research barrier. Tokamak operational conditions exhibit comparatively low Knudsen numbers. Kinetic effects, including kinetic waves and instabilities, Landau damping, bump-on-tail instabilities and more, are therefore highly influential in tokamak plasma dynamics. Purely fluid models are inherently incapable of capturing these effects, whereas the high dimensionality in purely kinetic models render them practically intractable for most relevant purposes.

        We consider a $\delta\!f$ decomposition model, with a macroscopic fluid background and microscopic kinetic correction, both fully coupled to each other. A similar manner of discretization is proposed to that used in the recent \texttt{STRUPHY} code \cite{Holderied_Possanner_Wang_2021, Holderied_2022, Li_et_al_2023} with a finite-element model for the background and a pseudo-particle/PiC model for the correction.

        The fluid background satisfies the full, non-linear, resistive, compressible, Hall MHD equations. \cite{Laakmann_Hu_Farrell_2022} introduces finite-element(-in-space) implicit timesteppers for the incompressible analogue to this system with structure-preserving (SP) properties in the ideal case, alongside parameter-robust preconditioners. We show that these timesteppers can derive from a finite-element-in-time (FET) (and finite-element-in-space) interpretation. The benefits of this reformulation are discussed, including the derivation of timesteppers that are higher order in time, and the quantifiable dissipative SP properties in the non-ideal, resistive case.
        
        We discuss possible options for extending this FET approach to timesteppers for the compressible case.

        The kinetic corrections satisfy linearized Boltzmann equations. Using a Lénard--Bernstein collision operator, these take Fokker--Planck-like forms \cite{Fokker_1914, Planck_1917} wherein pseudo-particles in the numerical model obey the neoclassical transport equations, with particle-independent Brownian drift terms. This offers a rigorous methodology for incorporating collisions into the particle transport model, without coupling the equations of motions for each particle.
        
        Works by Chen, Chacón et al. \cite{Chen_Chacón_Barnes_2011, Chacón_Chen_Barnes_2013, Chen_Chacón_2014, Chen_Chacón_2015} have developed structure-preserving particle pushers for neoclassical transport in the Vlasov equations, derived from Crank--Nicolson integrators. We show these too can can derive from a FET interpretation, similarly offering potential extensions to higher-order-in-time particle pushers. The FET formulation is used also to consider how the stochastic drift terms can be incorporated into the pushers. Stochastic gyrokinetic expansions are also discussed.

        Different options for the numerical implementation of these schemes are considered.

        Due to the efficacy of FET in the development of SP timesteppers for both the fluid and kinetic component, we hope this approach will prove effective in the future for developing SP timesteppers for the full hybrid model. We hope this will give us the opportunity to incorporate previously inaccessible kinetic effects into the highly effective, modern, finite-element MHD models.
    \end{abstract}
    
    
    \newpage
    \tableofcontents
    
    
    \newpage
    \pagenumbering{arabic}
    %\linenumbers\renewcommand\thelinenumber{\color{black!50}\arabic{linenumber}}
            \input{0 - introduction/main.tex}
        \part{Research}
            \input{1 - low-noise PiC models/main.tex}
            \input{2 - kinetic component/main.tex}
            \input{3 - fluid component/main.tex}
            \input{4 - numerical implementation/main.tex}
        \part{Project Overview}
            \input{5 - research plan/main.tex}
            \input{6 - summary/main.tex}
    
    
    %\section{}
    \newpage
    \pagenumbering{gobble}
        \printbibliography


    \newpage
    \pagenumbering{roman}
    \appendix
        \part{Appendices}
            \input{8 - Hilbert complexes/main.tex}
            \input{9 - weak conservation proofs/main.tex}
\end{document}

    
    
    %\section{}
    \newpage
    \pagenumbering{gobble}
        \printbibliography


    \newpage
    \pagenumbering{roman}
    \appendix
        \part{Appendices}
            \documentclass[12pt, a4paper]{report}

\input{template/main.tex}

\title{\BA{Title in Progress...}}
\author{Boris Andrews}
\affil{Mathematical Institute, University of Oxford}
\date{\today}


\begin{document}
    \pagenumbering{gobble}
    \maketitle
    
    
    \begin{abstract}
        Magnetic confinement reactors---in particular tokamaks---offer one of the most promising options for achieving practical nuclear fusion, with the potential to provide virtually limitless, clean energy. The theoretical and numerical modeling of tokamak plasmas is simultaneously an essential component of effective reactor design, and a great research barrier. Tokamak operational conditions exhibit comparatively low Knudsen numbers. Kinetic effects, including kinetic waves and instabilities, Landau damping, bump-on-tail instabilities and more, are therefore highly influential in tokamak plasma dynamics. Purely fluid models are inherently incapable of capturing these effects, whereas the high dimensionality in purely kinetic models render them practically intractable for most relevant purposes.

        We consider a $\delta\!f$ decomposition model, with a macroscopic fluid background and microscopic kinetic correction, both fully coupled to each other. A similar manner of discretization is proposed to that used in the recent \texttt{STRUPHY} code \cite{Holderied_Possanner_Wang_2021, Holderied_2022, Li_et_al_2023} with a finite-element model for the background and a pseudo-particle/PiC model for the correction.

        The fluid background satisfies the full, non-linear, resistive, compressible, Hall MHD equations. \cite{Laakmann_Hu_Farrell_2022} introduces finite-element(-in-space) implicit timesteppers for the incompressible analogue to this system with structure-preserving (SP) properties in the ideal case, alongside parameter-robust preconditioners. We show that these timesteppers can derive from a finite-element-in-time (FET) (and finite-element-in-space) interpretation. The benefits of this reformulation are discussed, including the derivation of timesteppers that are higher order in time, and the quantifiable dissipative SP properties in the non-ideal, resistive case.
        
        We discuss possible options for extending this FET approach to timesteppers for the compressible case.

        The kinetic corrections satisfy linearized Boltzmann equations. Using a Lénard--Bernstein collision operator, these take Fokker--Planck-like forms \cite{Fokker_1914, Planck_1917} wherein pseudo-particles in the numerical model obey the neoclassical transport equations, with particle-independent Brownian drift terms. This offers a rigorous methodology for incorporating collisions into the particle transport model, without coupling the equations of motions for each particle.
        
        Works by Chen, Chacón et al. \cite{Chen_Chacón_Barnes_2011, Chacón_Chen_Barnes_2013, Chen_Chacón_2014, Chen_Chacón_2015} have developed structure-preserving particle pushers for neoclassical transport in the Vlasov equations, derived from Crank--Nicolson integrators. We show these too can can derive from a FET interpretation, similarly offering potential extensions to higher-order-in-time particle pushers. The FET formulation is used also to consider how the stochastic drift terms can be incorporated into the pushers. Stochastic gyrokinetic expansions are also discussed.

        Different options for the numerical implementation of these schemes are considered.

        Due to the efficacy of FET in the development of SP timesteppers for both the fluid and kinetic component, we hope this approach will prove effective in the future for developing SP timesteppers for the full hybrid model. We hope this will give us the opportunity to incorporate previously inaccessible kinetic effects into the highly effective, modern, finite-element MHD models.
    \end{abstract}
    
    
    \newpage
    \tableofcontents
    
    
    \newpage
    \pagenumbering{arabic}
    %\linenumbers\renewcommand\thelinenumber{\color{black!50}\arabic{linenumber}}
            \input{0 - introduction/main.tex}
        \part{Research}
            \input{1 - low-noise PiC models/main.tex}
            \input{2 - kinetic component/main.tex}
            \input{3 - fluid component/main.tex}
            \input{4 - numerical implementation/main.tex}
        \part{Project Overview}
            \input{5 - research plan/main.tex}
            \input{6 - summary/main.tex}
    
    
    %\section{}
    \newpage
    \pagenumbering{gobble}
        \printbibliography


    \newpage
    \pagenumbering{roman}
    \appendix
        \part{Appendices}
            \input{8 - Hilbert complexes/main.tex}
            \input{9 - weak conservation proofs/main.tex}
\end{document}

            \documentclass[12pt, a4paper]{report}

\input{template/main.tex}

\title{\BA{Title in Progress...}}
\author{Boris Andrews}
\affil{Mathematical Institute, University of Oxford}
\date{\today}


\begin{document}
    \pagenumbering{gobble}
    \maketitle
    
    
    \begin{abstract}
        Magnetic confinement reactors---in particular tokamaks---offer one of the most promising options for achieving practical nuclear fusion, with the potential to provide virtually limitless, clean energy. The theoretical and numerical modeling of tokamak plasmas is simultaneously an essential component of effective reactor design, and a great research barrier. Tokamak operational conditions exhibit comparatively low Knudsen numbers. Kinetic effects, including kinetic waves and instabilities, Landau damping, bump-on-tail instabilities and more, are therefore highly influential in tokamak plasma dynamics. Purely fluid models are inherently incapable of capturing these effects, whereas the high dimensionality in purely kinetic models render them practically intractable for most relevant purposes.

        We consider a $\delta\!f$ decomposition model, with a macroscopic fluid background and microscopic kinetic correction, both fully coupled to each other. A similar manner of discretization is proposed to that used in the recent \texttt{STRUPHY} code \cite{Holderied_Possanner_Wang_2021, Holderied_2022, Li_et_al_2023} with a finite-element model for the background and a pseudo-particle/PiC model for the correction.

        The fluid background satisfies the full, non-linear, resistive, compressible, Hall MHD equations. \cite{Laakmann_Hu_Farrell_2022} introduces finite-element(-in-space) implicit timesteppers for the incompressible analogue to this system with structure-preserving (SP) properties in the ideal case, alongside parameter-robust preconditioners. We show that these timesteppers can derive from a finite-element-in-time (FET) (and finite-element-in-space) interpretation. The benefits of this reformulation are discussed, including the derivation of timesteppers that are higher order in time, and the quantifiable dissipative SP properties in the non-ideal, resistive case.
        
        We discuss possible options for extending this FET approach to timesteppers for the compressible case.

        The kinetic corrections satisfy linearized Boltzmann equations. Using a Lénard--Bernstein collision operator, these take Fokker--Planck-like forms \cite{Fokker_1914, Planck_1917} wherein pseudo-particles in the numerical model obey the neoclassical transport equations, with particle-independent Brownian drift terms. This offers a rigorous methodology for incorporating collisions into the particle transport model, without coupling the equations of motions for each particle.
        
        Works by Chen, Chacón et al. \cite{Chen_Chacón_Barnes_2011, Chacón_Chen_Barnes_2013, Chen_Chacón_2014, Chen_Chacón_2015} have developed structure-preserving particle pushers for neoclassical transport in the Vlasov equations, derived from Crank--Nicolson integrators. We show these too can can derive from a FET interpretation, similarly offering potential extensions to higher-order-in-time particle pushers. The FET formulation is used also to consider how the stochastic drift terms can be incorporated into the pushers. Stochastic gyrokinetic expansions are also discussed.

        Different options for the numerical implementation of these schemes are considered.

        Due to the efficacy of FET in the development of SP timesteppers for both the fluid and kinetic component, we hope this approach will prove effective in the future for developing SP timesteppers for the full hybrid model. We hope this will give us the opportunity to incorporate previously inaccessible kinetic effects into the highly effective, modern, finite-element MHD models.
    \end{abstract}
    
    
    \newpage
    \tableofcontents
    
    
    \newpage
    \pagenumbering{arabic}
    %\linenumbers\renewcommand\thelinenumber{\color{black!50}\arabic{linenumber}}
            \input{0 - introduction/main.tex}
        \part{Research}
            \input{1 - low-noise PiC models/main.tex}
            \input{2 - kinetic component/main.tex}
            \input{3 - fluid component/main.tex}
            \input{4 - numerical implementation/main.tex}
        \part{Project Overview}
            \input{5 - research plan/main.tex}
            \input{6 - summary/main.tex}
    
    
    %\section{}
    \newpage
    \pagenumbering{gobble}
        \printbibliography


    \newpage
    \pagenumbering{roman}
    \appendix
        \part{Appendices}
            \input{8 - Hilbert complexes/main.tex}
            \input{9 - weak conservation proofs/main.tex}
\end{document}

\end{document}


\title{\BA{Title in Progress...}}
\author{Boris Andrews}
\affil{Mathematical Institute, University of Oxford}
\date{\today}


\begin{document}
    \pagenumbering{gobble}
    \maketitle
    
    
    \begin{abstract}
        Magnetic confinement reactors---in particular tokamaks---offer one of the most promising options for achieving practical nuclear fusion, with the potential to provide virtually limitless, clean energy. The theoretical and numerical modeling of tokamak plasmas is simultaneously an essential component of effective reactor design, and a great research barrier. Tokamak operational conditions exhibit comparatively low Knudsen numbers. Kinetic effects, including kinetic waves and instabilities, Landau damping, bump-on-tail instabilities and more, are therefore highly influential in tokamak plasma dynamics. Purely fluid models are inherently incapable of capturing these effects, whereas the high dimensionality in purely kinetic models render them practically intractable for most relevant purposes.

        We consider a $\delta\!f$ decomposition model, with a macroscopic fluid background and microscopic kinetic correction, both fully coupled to each other. A similar manner of discretization is proposed to that used in the recent \texttt{STRUPHY} code \cite{Holderied_Possanner_Wang_2021, Holderied_2022, Li_et_al_2023} with a finite-element model for the background and a pseudo-particle/PiC model for the correction.

        The fluid background satisfies the full, non-linear, resistive, compressible, Hall MHD equations. \cite{Laakmann_Hu_Farrell_2022} introduces finite-element(-in-space) implicit timesteppers for the incompressible analogue to this system with structure-preserving (SP) properties in the ideal case, alongside parameter-robust preconditioners. We show that these timesteppers can derive from a finite-element-in-time (FET) (and finite-element-in-space) interpretation. The benefits of this reformulation are discussed, including the derivation of timesteppers that are higher order in time, and the quantifiable dissipative SP properties in the non-ideal, resistive case.
        
        We discuss possible options for extending this FET approach to timesteppers for the compressible case.

        The kinetic corrections satisfy linearized Boltzmann equations. Using a Lénard--Bernstein collision operator, these take Fokker--Planck-like forms \cite{Fokker_1914, Planck_1917} wherein pseudo-particles in the numerical model obey the neoclassical transport equations, with particle-independent Brownian drift terms. This offers a rigorous methodology for incorporating collisions into the particle transport model, without coupling the equations of motions for each particle.
        
        Works by Chen, Chacón et al. \cite{Chen_Chacón_Barnes_2011, Chacón_Chen_Barnes_2013, Chen_Chacón_2014, Chen_Chacón_2015} have developed structure-preserving particle pushers for neoclassical transport in the Vlasov equations, derived from Crank--Nicolson integrators. We show these too can can derive from a FET interpretation, similarly offering potential extensions to higher-order-in-time particle pushers. The FET formulation is used also to consider how the stochastic drift terms can be incorporated into the pushers. Stochastic gyrokinetic expansions are also discussed.

        Different options for the numerical implementation of these schemes are considered.

        Due to the efficacy of FET in the development of SP timesteppers for both the fluid and kinetic component, we hope this approach will prove effective in the future for developing SP timesteppers for the full hybrid model. We hope this will give us the opportunity to incorporate previously inaccessible kinetic effects into the highly effective, modern, finite-element MHD models.
    \end{abstract}
    
    
    \newpage
    \tableofcontents
    
    
    \newpage
    \pagenumbering{arabic}
    %\linenumbers\renewcommand\thelinenumber{\color{black!50}\arabic{linenumber}}
            \documentclass[12pt, a4paper]{report}

\documentclass[12pt, a4paper]{report}

\input{template/main.tex}

\title{\BA{Title in Progress...}}
\author{Boris Andrews}
\affil{Mathematical Institute, University of Oxford}
\date{\today}


\begin{document}
    \pagenumbering{gobble}
    \maketitle
    
    
    \begin{abstract}
        Magnetic confinement reactors---in particular tokamaks---offer one of the most promising options for achieving practical nuclear fusion, with the potential to provide virtually limitless, clean energy. The theoretical and numerical modeling of tokamak plasmas is simultaneously an essential component of effective reactor design, and a great research barrier. Tokamak operational conditions exhibit comparatively low Knudsen numbers. Kinetic effects, including kinetic waves and instabilities, Landau damping, bump-on-tail instabilities and more, are therefore highly influential in tokamak plasma dynamics. Purely fluid models are inherently incapable of capturing these effects, whereas the high dimensionality in purely kinetic models render them practically intractable for most relevant purposes.

        We consider a $\delta\!f$ decomposition model, with a macroscopic fluid background and microscopic kinetic correction, both fully coupled to each other. A similar manner of discretization is proposed to that used in the recent \texttt{STRUPHY} code \cite{Holderied_Possanner_Wang_2021, Holderied_2022, Li_et_al_2023} with a finite-element model for the background and a pseudo-particle/PiC model for the correction.

        The fluid background satisfies the full, non-linear, resistive, compressible, Hall MHD equations. \cite{Laakmann_Hu_Farrell_2022} introduces finite-element(-in-space) implicit timesteppers for the incompressible analogue to this system with structure-preserving (SP) properties in the ideal case, alongside parameter-robust preconditioners. We show that these timesteppers can derive from a finite-element-in-time (FET) (and finite-element-in-space) interpretation. The benefits of this reformulation are discussed, including the derivation of timesteppers that are higher order in time, and the quantifiable dissipative SP properties in the non-ideal, resistive case.
        
        We discuss possible options for extending this FET approach to timesteppers for the compressible case.

        The kinetic corrections satisfy linearized Boltzmann equations. Using a Lénard--Bernstein collision operator, these take Fokker--Planck-like forms \cite{Fokker_1914, Planck_1917} wherein pseudo-particles in the numerical model obey the neoclassical transport equations, with particle-independent Brownian drift terms. This offers a rigorous methodology for incorporating collisions into the particle transport model, without coupling the equations of motions for each particle.
        
        Works by Chen, Chacón et al. \cite{Chen_Chacón_Barnes_2011, Chacón_Chen_Barnes_2013, Chen_Chacón_2014, Chen_Chacón_2015} have developed structure-preserving particle pushers for neoclassical transport in the Vlasov equations, derived from Crank--Nicolson integrators. We show these too can can derive from a FET interpretation, similarly offering potential extensions to higher-order-in-time particle pushers. The FET formulation is used also to consider how the stochastic drift terms can be incorporated into the pushers. Stochastic gyrokinetic expansions are also discussed.

        Different options for the numerical implementation of these schemes are considered.

        Due to the efficacy of FET in the development of SP timesteppers for both the fluid and kinetic component, we hope this approach will prove effective in the future for developing SP timesteppers for the full hybrid model. We hope this will give us the opportunity to incorporate previously inaccessible kinetic effects into the highly effective, modern, finite-element MHD models.
    \end{abstract}
    
    
    \newpage
    \tableofcontents
    
    
    \newpage
    \pagenumbering{arabic}
    %\linenumbers\renewcommand\thelinenumber{\color{black!50}\arabic{linenumber}}
            \input{0 - introduction/main.tex}
        \part{Research}
            \input{1 - low-noise PiC models/main.tex}
            \input{2 - kinetic component/main.tex}
            \input{3 - fluid component/main.tex}
            \input{4 - numerical implementation/main.tex}
        \part{Project Overview}
            \input{5 - research plan/main.tex}
            \input{6 - summary/main.tex}
    
    
    %\section{}
    \newpage
    \pagenumbering{gobble}
        \printbibliography


    \newpage
    \pagenumbering{roman}
    \appendix
        \part{Appendices}
            \input{8 - Hilbert complexes/main.tex}
            \input{9 - weak conservation proofs/main.tex}
\end{document}


\title{\BA{Title in Progress...}}
\author{Boris Andrews}
\affil{Mathematical Institute, University of Oxford}
\date{\today}


\begin{document}
    \pagenumbering{gobble}
    \maketitle
    
    
    \begin{abstract}
        Magnetic confinement reactors---in particular tokamaks---offer one of the most promising options for achieving practical nuclear fusion, with the potential to provide virtually limitless, clean energy. The theoretical and numerical modeling of tokamak plasmas is simultaneously an essential component of effective reactor design, and a great research barrier. Tokamak operational conditions exhibit comparatively low Knudsen numbers. Kinetic effects, including kinetic waves and instabilities, Landau damping, bump-on-tail instabilities and more, are therefore highly influential in tokamak plasma dynamics. Purely fluid models are inherently incapable of capturing these effects, whereas the high dimensionality in purely kinetic models render them practically intractable for most relevant purposes.

        We consider a $\delta\!f$ decomposition model, with a macroscopic fluid background and microscopic kinetic correction, both fully coupled to each other. A similar manner of discretization is proposed to that used in the recent \texttt{STRUPHY} code \cite{Holderied_Possanner_Wang_2021, Holderied_2022, Li_et_al_2023} with a finite-element model for the background and a pseudo-particle/PiC model for the correction.

        The fluid background satisfies the full, non-linear, resistive, compressible, Hall MHD equations. \cite{Laakmann_Hu_Farrell_2022} introduces finite-element(-in-space) implicit timesteppers for the incompressible analogue to this system with structure-preserving (SP) properties in the ideal case, alongside parameter-robust preconditioners. We show that these timesteppers can derive from a finite-element-in-time (FET) (and finite-element-in-space) interpretation. The benefits of this reformulation are discussed, including the derivation of timesteppers that are higher order in time, and the quantifiable dissipative SP properties in the non-ideal, resistive case.
        
        We discuss possible options for extending this FET approach to timesteppers for the compressible case.

        The kinetic corrections satisfy linearized Boltzmann equations. Using a Lénard--Bernstein collision operator, these take Fokker--Planck-like forms \cite{Fokker_1914, Planck_1917} wherein pseudo-particles in the numerical model obey the neoclassical transport equations, with particle-independent Brownian drift terms. This offers a rigorous methodology for incorporating collisions into the particle transport model, without coupling the equations of motions for each particle.
        
        Works by Chen, Chacón et al. \cite{Chen_Chacón_Barnes_2011, Chacón_Chen_Barnes_2013, Chen_Chacón_2014, Chen_Chacón_2015} have developed structure-preserving particle pushers for neoclassical transport in the Vlasov equations, derived from Crank--Nicolson integrators. We show these too can can derive from a FET interpretation, similarly offering potential extensions to higher-order-in-time particle pushers. The FET formulation is used also to consider how the stochastic drift terms can be incorporated into the pushers. Stochastic gyrokinetic expansions are also discussed.

        Different options for the numerical implementation of these schemes are considered.

        Due to the efficacy of FET in the development of SP timesteppers for both the fluid and kinetic component, we hope this approach will prove effective in the future for developing SP timesteppers for the full hybrid model. We hope this will give us the opportunity to incorporate previously inaccessible kinetic effects into the highly effective, modern, finite-element MHD models.
    \end{abstract}
    
    
    \newpage
    \tableofcontents
    
    
    \newpage
    \pagenumbering{arabic}
    %\linenumbers\renewcommand\thelinenumber{\color{black!50}\arabic{linenumber}}
            \documentclass[12pt, a4paper]{report}

\input{template/main.tex}

\title{\BA{Title in Progress...}}
\author{Boris Andrews}
\affil{Mathematical Institute, University of Oxford}
\date{\today}


\begin{document}
    \pagenumbering{gobble}
    \maketitle
    
    
    \begin{abstract}
        Magnetic confinement reactors---in particular tokamaks---offer one of the most promising options for achieving practical nuclear fusion, with the potential to provide virtually limitless, clean energy. The theoretical and numerical modeling of tokamak plasmas is simultaneously an essential component of effective reactor design, and a great research barrier. Tokamak operational conditions exhibit comparatively low Knudsen numbers. Kinetic effects, including kinetic waves and instabilities, Landau damping, bump-on-tail instabilities and more, are therefore highly influential in tokamak plasma dynamics. Purely fluid models are inherently incapable of capturing these effects, whereas the high dimensionality in purely kinetic models render them practically intractable for most relevant purposes.

        We consider a $\delta\!f$ decomposition model, with a macroscopic fluid background and microscopic kinetic correction, both fully coupled to each other. A similar manner of discretization is proposed to that used in the recent \texttt{STRUPHY} code \cite{Holderied_Possanner_Wang_2021, Holderied_2022, Li_et_al_2023} with a finite-element model for the background and a pseudo-particle/PiC model for the correction.

        The fluid background satisfies the full, non-linear, resistive, compressible, Hall MHD equations. \cite{Laakmann_Hu_Farrell_2022} introduces finite-element(-in-space) implicit timesteppers for the incompressible analogue to this system with structure-preserving (SP) properties in the ideal case, alongside parameter-robust preconditioners. We show that these timesteppers can derive from a finite-element-in-time (FET) (and finite-element-in-space) interpretation. The benefits of this reformulation are discussed, including the derivation of timesteppers that are higher order in time, and the quantifiable dissipative SP properties in the non-ideal, resistive case.
        
        We discuss possible options for extending this FET approach to timesteppers for the compressible case.

        The kinetic corrections satisfy linearized Boltzmann equations. Using a Lénard--Bernstein collision operator, these take Fokker--Planck-like forms \cite{Fokker_1914, Planck_1917} wherein pseudo-particles in the numerical model obey the neoclassical transport equations, with particle-independent Brownian drift terms. This offers a rigorous methodology for incorporating collisions into the particle transport model, without coupling the equations of motions for each particle.
        
        Works by Chen, Chacón et al. \cite{Chen_Chacón_Barnes_2011, Chacón_Chen_Barnes_2013, Chen_Chacón_2014, Chen_Chacón_2015} have developed structure-preserving particle pushers for neoclassical transport in the Vlasov equations, derived from Crank--Nicolson integrators. We show these too can can derive from a FET interpretation, similarly offering potential extensions to higher-order-in-time particle pushers. The FET formulation is used also to consider how the stochastic drift terms can be incorporated into the pushers. Stochastic gyrokinetic expansions are also discussed.

        Different options for the numerical implementation of these schemes are considered.

        Due to the efficacy of FET in the development of SP timesteppers for both the fluid and kinetic component, we hope this approach will prove effective in the future for developing SP timesteppers for the full hybrid model. We hope this will give us the opportunity to incorporate previously inaccessible kinetic effects into the highly effective, modern, finite-element MHD models.
    \end{abstract}
    
    
    \newpage
    \tableofcontents
    
    
    \newpage
    \pagenumbering{arabic}
    %\linenumbers\renewcommand\thelinenumber{\color{black!50}\arabic{linenumber}}
            \input{0 - introduction/main.tex}
        \part{Research}
            \input{1 - low-noise PiC models/main.tex}
            \input{2 - kinetic component/main.tex}
            \input{3 - fluid component/main.tex}
            \input{4 - numerical implementation/main.tex}
        \part{Project Overview}
            \input{5 - research plan/main.tex}
            \input{6 - summary/main.tex}
    
    
    %\section{}
    \newpage
    \pagenumbering{gobble}
        \printbibliography


    \newpage
    \pagenumbering{roman}
    \appendix
        \part{Appendices}
            \input{8 - Hilbert complexes/main.tex}
            \input{9 - weak conservation proofs/main.tex}
\end{document}

        \part{Research}
            \documentclass[12pt, a4paper]{report}

\input{template/main.tex}

\title{\BA{Title in Progress...}}
\author{Boris Andrews}
\affil{Mathematical Institute, University of Oxford}
\date{\today}


\begin{document}
    \pagenumbering{gobble}
    \maketitle
    
    
    \begin{abstract}
        Magnetic confinement reactors---in particular tokamaks---offer one of the most promising options for achieving practical nuclear fusion, with the potential to provide virtually limitless, clean energy. The theoretical and numerical modeling of tokamak plasmas is simultaneously an essential component of effective reactor design, and a great research barrier. Tokamak operational conditions exhibit comparatively low Knudsen numbers. Kinetic effects, including kinetic waves and instabilities, Landau damping, bump-on-tail instabilities and more, are therefore highly influential in tokamak plasma dynamics. Purely fluid models are inherently incapable of capturing these effects, whereas the high dimensionality in purely kinetic models render them practically intractable for most relevant purposes.

        We consider a $\delta\!f$ decomposition model, with a macroscopic fluid background and microscopic kinetic correction, both fully coupled to each other. A similar manner of discretization is proposed to that used in the recent \texttt{STRUPHY} code \cite{Holderied_Possanner_Wang_2021, Holderied_2022, Li_et_al_2023} with a finite-element model for the background and a pseudo-particle/PiC model for the correction.

        The fluid background satisfies the full, non-linear, resistive, compressible, Hall MHD equations. \cite{Laakmann_Hu_Farrell_2022} introduces finite-element(-in-space) implicit timesteppers for the incompressible analogue to this system with structure-preserving (SP) properties in the ideal case, alongside parameter-robust preconditioners. We show that these timesteppers can derive from a finite-element-in-time (FET) (and finite-element-in-space) interpretation. The benefits of this reformulation are discussed, including the derivation of timesteppers that are higher order in time, and the quantifiable dissipative SP properties in the non-ideal, resistive case.
        
        We discuss possible options for extending this FET approach to timesteppers for the compressible case.

        The kinetic corrections satisfy linearized Boltzmann equations. Using a Lénard--Bernstein collision operator, these take Fokker--Planck-like forms \cite{Fokker_1914, Planck_1917} wherein pseudo-particles in the numerical model obey the neoclassical transport equations, with particle-independent Brownian drift terms. This offers a rigorous methodology for incorporating collisions into the particle transport model, without coupling the equations of motions for each particle.
        
        Works by Chen, Chacón et al. \cite{Chen_Chacón_Barnes_2011, Chacón_Chen_Barnes_2013, Chen_Chacón_2014, Chen_Chacón_2015} have developed structure-preserving particle pushers for neoclassical transport in the Vlasov equations, derived from Crank--Nicolson integrators. We show these too can can derive from a FET interpretation, similarly offering potential extensions to higher-order-in-time particle pushers. The FET formulation is used also to consider how the stochastic drift terms can be incorporated into the pushers. Stochastic gyrokinetic expansions are also discussed.

        Different options for the numerical implementation of these schemes are considered.

        Due to the efficacy of FET in the development of SP timesteppers for both the fluid and kinetic component, we hope this approach will prove effective in the future for developing SP timesteppers for the full hybrid model. We hope this will give us the opportunity to incorporate previously inaccessible kinetic effects into the highly effective, modern, finite-element MHD models.
    \end{abstract}
    
    
    \newpage
    \tableofcontents
    
    
    \newpage
    \pagenumbering{arabic}
    %\linenumbers\renewcommand\thelinenumber{\color{black!50}\arabic{linenumber}}
            \input{0 - introduction/main.tex}
        \part{Research}
            \input{1 - low-noise PiC models/main.tex}
            \input{2 - kinetic component/main.tex}
            \input{3 - fluid component/main.tex}
            \input{4 - numerical implementation/main.tex}
        \part{Project Overview}
            \input{5 - research plan/main.tex}
            \input{6 - summary/main.tex}
    
    
    %\section{}
    \newpage
    \pagenumbering{gobble}
        \printbibliography


    \newpage
    \pagenumbering{roman}
    \appendix
        \part{Appendices}
            \input{8 - Hilbert complexes/main.tex}
            \input{9 - weak conservation proofs/main.tex}
\end{document}

            \documentclass[12pt, a4paper]{report}

\input{template/main.tex}

\title{\BA{Title in Progress...}}
\author{Boris Andrews}
\affil{Mathematical Institute, University of Oxford}
\date{\today}


\begin{document}
    \pagenumbering{gobble}
    \maketitle
    
    
    \begin{abstract}
        Magnetic confinement reactors---in particular tokamaks---offer one of the most promising options for achieving practical nuclear fusion, with the potential to provide virtually limitless, clean energy. The theoretical and numerical modeling of tokamak plasmas is simultaneously an essential component of effective reactor design, and a great research barrier. Tokamak operational conditions exhibit comparatively low Knudsen numbers. Kinetic effects, including kinetic waves and instabilities, Landau damping, bump-on-tail instabilities and more, are therefore highly influential in tokamak plasma dynamics. Purely fluid models are inherently incapable of capturing these effects, whereas the high dimensionality in purely kinetic models render them practically intractable for most relevant purposes.

        We consider a $\delta\!f$ decomposition model, with a macroscopic fluid background and microscopic kinetic correction, both fully coupled to each other. A similar manner of discretization is proposed to that used in the recent \texttt{STRUPHY} code \cite{Holderied_Possanner_Wang_2021, Holderied_2022, Li_et_al_2023} with a finite-element model for the background and a pseudo-particle/PiC model for the correction.

        The fluid background satisfies the full, non-linear, resistive, compressible, Hall MHD equations. \cite{Laakmann_Hu_Farrell_2022} introduces finite-element(-in-space) implicit timesteppers for the incompressible analogue to this system with structure-preserving (SP) properties in the ideal case, alongside parameter-robust preconditioners. We show that these timesteppers can derive from a finite-element-in-time (FET) (and finite-element-in-space) interpretation. The benefits of this reformulation are discussed, including the derivation of timesteppers that are higher order in time, and the quantifiable dissipative SP properties in the non-ideal, resistive case.
        
        We discuss possible options for extending this FET approach to timesteppers for the compressible case.

        The kinetic corrections satisfy linearized Boltzmann equations. Using a Lénard--Bernstein collision operator, these take Fokker--Planck-like forms \cite{Fokker_1914, Planck_1917} wherein pseudo-particles in the numerical model obey the neoclassical transport equations, with particle-independent Brownian drift terms. This offers a rigorous methodology for incorporating collisions into the particle transport model, without coupling the equations of motions for each particle.
        
        Works by Chen, Chacón et al. \cite{Chen_Chacón_Barnes_2011, Chacón_Chen_Barnes_2013, Chen_Chacón_2014, Chen_Chacón_2015} have developed structure-preserving particle pushers for neoclassical transport in the Vlasov equations, derived from Crank--Nicolson integrators. We show these too can can derive from a FET interpretation, similarly offering potential extensions to higher-order-in-time particle pushers. The FET formulation is used also to consider how the stochastic drift terms can be incorporated into the pushers. Stochastic gyrokinetic expansions are also discussed.

        Different options for the numerical implementation of these schemes are considered.

        Due to the efficacy of FET in the development of SP timesteppers for both the fluid and kinetic component, we hope this approach will prove effective in the future for developing SP timesteppers for the full hybrid model. We hope this will give us the opportunity to incorporate previously inaccessible kinetic effects into the highly effective, modern, finite-element MHD models.
    \end{abstract}
    
    
    \newpage
    \tableofcontents
    
    
    \newpage
    \pagenumbering{arabic}
    %\linenumbers\renewcommand\thelinenumber{\color{black!50}\arabic{linenumber}}
            \input{0 - introduction/main.tex}
        \part{Research}
            \input{1 - low-noise PiC models/main.tex}
            \input{2 - kinetic component/main.tex}
            \input{3 - fluid component/main.tex}
            \input{4 - numerical implementation/main.tex}
        \part{Project Overview}
            \input{5 - research plan/main.tex}
            \input{6 - summary/main.tex}
    
    
    %\section{}
    \newpage
    \pagenumbering{gobble}
        \printbibliography


    \newpage
    \pagenumbering{roman}
    \appendix
        \part{Appendices}
            \input{8 - Hilbert complexes/main.tex}
            \input{9 - weak conservation proofs/main.tex}
\end{document}

            \documentclass[12pt, a4paper]{report}

\input{template/main.tex}

\title{\BA{Title in Progress...}}
\author{Boris Andrews}
\affil{Mathematical Institute, University of Oxford}
\date{\today}


\begin{document}
    \pagenumbering{gobble}
    \maketitle
    
    
    \begin{abstract}
        Magnetic confinement reactors---in particular tokamaks---offer one of the most promising options for achieving practical nuclear fusion, with the potential to provide virtually limitless, clean energy. The theoretical and numerical modeling of tokamak plasmas is simultaneously an essential component of effective reactor design, and a great research barrier. Tokamak operational conditions exhibit comparatively low Knudsen numbers. Kinetic effects, including kinetic waves and instabilities, Landau damping, bump-on-tail instabilities and more, are therefore highly influential in tokamak plasma dynamics. Purely fluid models are inherently incapable of capturing these effects, whereas the high dimensionality in purely kinetic models render them practically intractable for most relevant purposes.

        We consider a $\delta\!f$ decomposition model, with a macroscopic fluid background and microscopic kinetic correction, both fully coupled to each other. A similar manner of discretization is proposed to that used in the recent \texttt{STRUPHY} code \cite{Holderied_Possanner_Wang_2021, Holderied_2022, Li_et_al_2023} with a finite-element model for the background and a pseudo-particle/PiC model for the correction.

        The fluid background satisfies the full, non-linear, resistive, compressible, Hall MHD equations. \cite{Laakmann_Hu_Farrell_2022} introduces finite-element(-in-space) implicit timesteppers for the incompressible analogue to this system with structure-preserving (SP) properties in the ideal case, alongside parameter-robust preconditioners. We show that these timesteppers can derive from a finite-element-in-time (FET) (and finite-element-in-space) interpretation. The benefits of this reformulation are discussed, including the derivation of timesteppers that are higher order in time, and the quantifiable dissipative SP properties in the non-ideal, resistive case.
        
        We discuss possible options for extending this FET approach to timesteppers for the compressible case.

        The kinetic corrections satisfy linearized Boltzmann equations. Using a Lénard--Bernstein collision operator, these take Fokker--Planck-like forms \cite{Fokker_1914, Planck_1917} wherein pseudo-particles in the numerical model obey the neoclassical transport equations, with particle-independent Brownian drift terms. This offers a rigorous methodology for incorporating collisions into the particle transport model, without coupling the equations of motions for each particle.
        
        Works by Chen, Chacón et al. \cite{Chen_Chacón_Barnes_2011, Chacón_Chen_Barnes_2013, Chen_Chacón_2014, Chen_Chacón_2015} have developed structure-preserving particle pushers for neoclassical transport in the Vlasov equations, derived from Crank--Nicolson integrators. We show these too can can derive from a FET interpretation, similarly offering potential extensions to higher-order-in-time particle pushers. The FET formulation is used also to consider how the stochastic drift terms can be incorporated into the pushers. Stochastic gyrokinetic expansions are also discussed.

        Different options for the numerical implementation of these schemes are considered.

        Due to the efficacy of FET in the development of SP timesteppers for both the fluid and kinetic component, we hope this approach will prove effective in the future for developing SP timesteppers for the full hybrid model. We hope this will give us the opportunity to incorporate previously inaccessible kinetic effects into the highly effective, modern, finite-element MHD models.
    \end{abstract}
    
    
    \newpage
    \tableofcontents
    
    
    \newpage
    \pagenumbering{arabic}
    %\linenumbers\renewcommand\thelinenumber{\color{black!50}\arabic{linenumber}}
            \input{0 - introduction/main.tex}
        \part{Research}
            \input{1 - low-noise PiC models/main.tex}
            \input{2 - kinetic component/main.tex}
            \input{3 - fluid component/main.tex}
            \input{4 - numerical implementation/main.tex}
        \part{Project Overview}
            \input{5 - research plan/main.tex}
            \input{6 - summary/main.tex}
    
    
    %\section{}
    \newpage
    \pagenumbering{gobble}
        \printbibliography


    \newpage
    \pagenumbering{roman}
    \appendix
        \part{Appendices}
            \input{8 - Hilbert complexes/main.tex}
            \input{9 - weak conservation proofs/main.tex}
\end{document}

            \documentclass[12pt, a4paper]{report}

\input{template/main.tex}

\title{\BA{Title in Progress...}}
\author{Boris Andrews}
\affil{Mathematical Institute, University of Oxford}
\date{\today}


\begin{document}
    \pagenumbering{gobble}
    \maketitle
    
    
    \begin{abstract}
        Magnetic confinement reactors---in particular tokamaks---offer one of the most promising options for achieving practical nuclear fusion, with the potential to provide virtually limitless, clean energy. The theoretical and numerical modeling of tokamak plasmas is simultaneously an essential component of effective reactor design, and a great research barrier. Tokamak operational conditions exhibit comparatively low Knudsen numbers. Kinetic effects, including kinetic waves and instabilities, Landau damping, bump-on-tail instabilities and more, are therefore highly influential in tokamak plasma dynamics. Purely fluid models are inherently incapable of capturing these effects, whereas the high dimensionality in purely kinetic models render them practically intractable for most relevant purposes.

        We consider a $\delta\!f$ decomposition model, with a macroscopic fluid background and microscopic kinetic correction, both fully coupled to each other. A similar manner of discretization is proposed to that used in the recent \texttt{STRUPHY} code \cite{Holderied_Possanner_Wang_2021, Holderied_2022, Li_et_al_2023} with a finite-element model for the background and a pseudo-particle/PiC model for the correction.

        The fluid background satisfies the full, non-linear, resistive, compressible, Hall MHD equations. \cite{Laakmann_Hu_Farrell_2022} introduces finite-element(-in-space) implicit timesteppers for the incompressible analogue to this system with structure-preserving (SP) properties in the ideal case, alongside parameter-robust preconditioners. We show that these timesteppers can derive from a finite-element-in-time (FET) (and finite-element-in-space) interpretation. The benefits of this reformulation are discussed, including the derivation of timesteppers that are higher order in time, and the quantifiable dissipative SP properties in the non-ideal, resistive case.
        
        We discuss possible options for extending this FET approach to timesteppers for the compressible case.

        The kinetic corrections satisfy linearized Boltzmann equations. Using a Lénard--Bernstein collision operator, these take Fokker--Planck-like forms \cite{Fokker_1914, Planck_1917} wherein pseudo-particles in the numerical model obey the neoclassical transport equations, with particle-independent Brownian drift terms. This offers a rigorous methodology for incorporating collisions into the particle transport model, without coupling the equations of motions for each particle.
        
        Works by Chen, Chacón et al. \cite{Chen_Chacón_Barnes_2011, Chacón_Chen_Barnes_2013, Chen_Chacón_2014, Chen_Chacón_2015} have developed structure-preserving particle pushers for neoclassical transport in the Vlasov equations, derived from Crank--Nicolson integrators. We show these too can can derive from a FET interpretation, similarly offering potential extensions to higher-order-in-time particle pushers. The FET formulation is used also to consider how the stochastic drift terms can be incorporated into the pushers. Stochastic gyrokinetic expansions are also discussed.

        Different options for the numerical implementation of these schemes are considered.

        Due to the efficacy of FET in the development of SP timesteppers for both the fluid and kinetic component, we hope this approach will prove effective in the future for developing SP timesteppers for the full hybrid model. We hope this will give us the opportunity to incorporate previously inaccessible kinetic effects into the highly effective, modern, finite-element MHD models.
    \end{abstract}
    
    
    \newpage
    \tableofcontents
    
    
    \newpage
    \pagenumbering{arabic}
    %\linenumbers\renewcommand\thelinenumber{\color{black!50}\arabic{linenumber}}
            \input{0 - introduction/main.tex}
        \part{Research}
            \input{1 - low-noise PiC models/main.tex}
            \input{2 - kinetic component/main.tex}
            \input{3 - fluid component/main.tex}
            \input{4 - numerical implementation/main.tex}
        \part{Project Overview}
            \input{5 - research plan/main.tex}
            \input{6 - summary/main.tex}
    
    
    %\section{}
    \newpage
    \pagenumbering{gobble}
        \printbibliography


    \newpage
    \pagenumbering{roman}
    \appendix
        \part{Appendices}
            \input{8 - Hilbert complexes/main.tex}
            \input{9 - weak conservation proofs/main.tex}
\end{document}

        \part{Project Overview}
            \documentclass[12pt, a4paper]{report}

\input{template/main.tex}

\title{\BA{Title in Progress...}}
\author{Boris Andrews}
\affil{Mathematical Institute, University of Oxford}
\date{\today}


\begin{document}
    \pagenumbering{gobble}
    \maketitle
    
    
    \begin{abstract}
        Magnetic confinement reactors---in particular tokamaks---offer one of the most promising options for achieving practical nuclear fusion, with the potential to provide virtually limitless, clean energy. The theoretical and numerical modeling of tokamak plasmas is simultaneously an essential component of effective reactor design, and a great research barrier. Tokamak operational conditions exhibit comparatively low Knudsen numbers. Kinetic effects, including kinetic waves and instabilities, Landau damping, bump-on-tail instabilities and more, are therefore highly influential in tokamak plasma dynamics. Purely fluid models are inherently incapable of capturing these effects, whereas the high dimensionality in purely kinetic models render them practically intractable for most relevant purposes.

        We consider a $\delta\!f$ decomposition model, with a macroscopic fluid background and microscopic kinetic correction, both fully coupled to each other. A similar manner of discretization is proposed to that used in the recent \texttt{STRUPHY} code \cite{Holderied_Possanner_Wang_2021, Holderied_2022, Li_et_al_2023} with a finite-element model for the background and a pseudo-particle/PiC model for the correction.

        The fluid background satisfies the full, non-linear, resistive, compressible, Hall MHD equations. \cite{Laakmann_Hu_Farrell_2022} introduces finite-element(-in-space) implicit timesteppers for the incompressible analogue to this system with structure-preserving (SP) properties in the ideal case, alongside parameter-robust preconditioners. We show that these timesteppers can derive from a finite-element-in-time (FET) (and finite-element-in-space) interpretation. The benefits of this reformulation are discussed, including the derivation of timesteppers that are higher order in time, and the quantifiable dissipative SP properties in the non-ideal, resistive case.
        
        We discuss possible options for extending this FET approach to timesteppers for the compressible case.

        The kinetic corrections satisfy linearized Boltzmann equations. Using a Lénard--Bernstein collision operator, these take Fokker--Planck-like forms \cite{Fokker_1914, Planck_1917} wherein pseudo-particles in the numerical model obey the neoclassical transport equations, with particle-independent Brownian drift terms. This offers a rigorous methodology for incorporating collisions into the particle transport model, without coupling the equations of motions for each particle.
        
        Works by Chen, Chacón et al. \cite{Chen_Chacón_Barnes_2011, Chacón_Chen_Barnes_2013, Chen_Chacón_2014, Chen_Chacón_2015} have developed structure-preserving particle pushers for neoclassical transport in the Vlasov equations, derived from Crank--Nicolson integrators. We show these too can can derive from a FET interpretation, similarly offering potential extensions to higher-order-in-time particle pushers. The FET formulation is used also to consider how the stochastic drift terms can be incorporated into the pushers. Stochastic gyrokinetic expansions are also discussed.

        Different options for the numerical implementation of these schemes are considered.

        Due to the efficacy of FET in the development of SP timesteppers for both the fluid and kinetic component, we hope this approach will prove effective in the future for developing SP timesteppers for the full hybrid model. We hope this will give us the opportunity to incorporate previously inaccessible kinetic effects into the highly effective, modern, finite-element MHD models.
    \end{abstract}
    
    
    \newpage
    \tableofcontents
    
    
    \newpage
    \pagenumbering{arabic}
    %\linenumbers\renewcommand\thelinenumber{\color{black!50}\arabic{linenumber}}
            \input{0 - introduction/main.tex}
        \part{Research}
            \input{1 - low-noise PiC models/main.tex}
            \input{2 - kinetic component/main.tex}
            \input{3 - fluid component/main.tex}
            \input{4 - numerical implementation/main.tex}
        \part{Project Overview}
            \input{5 - research plan/main.tex}
            \input{6 - summary/main.tex}
    
    
    %\section{}
    \newpage
    \pagenumbering{gobble}
        \printbibliography


    \newpage
    \pagenumbering{roman}
    \appendix
        \part{Appendices}
            \input{8 - Hilbert complexes/main.tex}
            \input{9 - weak conservation proofs/main.tex}
\end{document}

            \documentclass[12pt, a4paper]{report}

\input{template/main.tex}

\title{\BA{Title in Progress...}}
\author{Boris Andrews}
\affil{Mathematical Institute, University of Oxford}
\date{\today}


\begin{document}
    \pagenumbering{gobble}
    \maketitle
    
    
    \begin{abstract}
        Magnetic confinement reactors---in particular tokamaks---offer one of the most promising options for achieving practical nuclear fusion, with the potential to provide virtually limitless, clean energy. The theoretical and numerical modeling of tokamak plasmas is simultaneously an essential component of effective reactor design, and a great research barrier. Tokamak operational conditions exhibit comparatively low Knudsen numbers. Kinetic effects, including kinetic waves and instabilities, Landau damping, bump-on-tail instabilities and more, are therefore highly influential in tokamak plasma dynamics. Purely fluid models are inherently incapable of capturing these effects, whereas the high dimensionality in purely kinetic models render them practically intractable for most relevant purposes.

        We consider a $\delta\!f$ decomposition model, with a macroscopic fluid background and microscopic kinetic correction, both fully coupled to each other. A similar manner of discretization is proposed to that used in the recent \texttt{STRUPHY} code \cite{Holderied_Possanner_Wang_2021, Holderied_2022, Li_et_al_2023} with a finite-element model for the background and a pseudo-particle/PiC model for the correction.

        The fluid background satisfies the full, non-linear, resistive, compressible, Hall MHD equations. \cite{Laakmann_Hu_Farrell_2022} introduces finite-element(-in-space) implicit timesteppers for the incompressible analogue to this system with structure-preserving (SP) properties in the ideal case, alongside parameter-robust preconditioners. We show that these timesteppers can derive from a finite-element-in-time (FET) (and finite-element-in-space) interpretation. The benefits of this reformulation are discussed, including the derivation of timesteppers that are higher order in time, and the quantifiable dissipative SP properties in the non-ideal, resistive case.
        
        We discuss possible options for extending this FET approach to timesteppers for the compressible case.

        The kinetic corrections satisfy linearized Boltzmann equations. Using a Lénard--Bernstein collision operator, these take Fokker--Planck-like forms \cite{Fokker_1914, Planck_1917} wherein pseudo-particles in the numerical model obey the neoclassical transport equations, with particle-independent Brownian drift terms. This offers a rigorous methodology for incorporating collisions into the particle transport model, without coupling the equations of motions for each particle.
        
        Works by Chen, Chacón et al. \cite{Chen_Chacón_Barnes_2011, Chacón_Chen_Barnes_2013, Chen_Chacón_2014, Chen_Chacón_2015} have developed structure-preserving particle pushers for neoclassical transport in the Vlasov equations, derived from Crank--Nicolson integrators. We show these too can can derive from a FET interpretation, similarly offering potential extensions to higher-order-in-time particle pushers. The FET formulation is used also to consider how the stochastic drift terms can be incorporated into the pushers. Stochastic gyrokinetic expansions are also discussed.

        Different options for the numerical implementation of these schemes are considered.

        Due to the efficacy of FET in the development of SP timesteppers for both the fluid and kinetic component, we hope this approach will prove effective in the future for developing SP timesteppers for the full hybrid model. We hope this will give us the opportunity to incorporate previously inaccessible kinetic effects into the highly effective, modern, finite-element MHD models.
    \end{abstract}
    
    
    \newpage
    \tableofcontents
    
    
    \newpage
    \pagenumbering{arabic}
    %\linenumbers\renewcommand\thelinenumber{\color{black!50}\arabic{linenumber}}
            \input{0 - introduction/main.tex}
        \part{Research}
            \input{1 - low-noise PiC models/main.tex}
            \input{2 - kinetic component/main.tex}
            \input{3 - fluid component/main.tex}
            \input{4 - numerical implementation/main.tex}
        \part{Project Overview}
            \input{5 - research plan/main.tex}
            \input{6 - summary/main.tex}
    
    
    %\section{}
    \newpage
    \pagenumbering{gobble}
        \printbibliography


    \newpage
    \pagenumbering{roman}
    \appendix
        \part{Appendices}
            \input{8 - Hilbert complexes/main.tex}
            \input{9 - weak conservation proofs/main.tex}
\end{document}

    
    
    %\section{}
    \newpage
    \pagenumbering{gobble}
        \printbibliography


    \newpage
    \pagenumbering{roman}
    \appendix
        \part{Appendices}
            \documentclass[12pt, a4paper]{report}

\input{template/main.tex}

\title{\BA{Title in Progress...}}
\author{Boris Andrews}
\affil{Mathematical Institute, University of Oxford}
\date{\today}


\begin{document}
    \pagenumbering{gobble}
    \maketitle
    
    
    \begin{abstract}
        Magnetic confinement reactors---in particular tokamaks---offer one of the most promising options for achieving practical nuclear fusion, with the potential to provide virtually limitless, clean energy. The theoretical and numerical modeling of tokamak plasmas is simultaneously an essential component of effective reactor design, and a great research barrier. Tokamak operational conditions exhibit comparatively low Knudsen numbers. Kinetic effects, including kinetic waves and instabilities, Landau damping, bump-on-tail instabilities and more, are therefore highly influential in tokamak plasma dynamics. Purely fluid models are inherently incapable of capturing these effects, whereas the high dimensionality in purely kinetic models render them practically intractable for most relevant purposes.

        We consider a $\delta\!f$ decomposition model, with a macroscopic fluid background and microscopic kinetic correction, both fully coupled to each other. A similar manner of discretization is proposed to that used in the recent \texttt{STRUPHY} code \cite{Holderied_Possanner_Wang_2021, Holderied_2022, Li_et_al_2023} with a finite-element model for the background and a pseudo-particle/PiC model for the correction.

        The fluid background satisfies the full, non-linear, resistive, compressible, Hall MHD equations. \cite{Laakmann_Hu_Farrell_2022} introduces finite-element(-in-space) implicit timesteppers for the incompressible analogue to this system with structure-preserving (SP) properties in the ideal case, alongside parameter-robust preconditioners. We show that these timesteppers can derive from a finite-element-in-time (FET) (and finite-element-in-space) interpretation. The benefits of this reformulation are discussed, including the derivation of timesteppers that are higher order in time, and the quantifiable dissipative SP properties in the non-ideal, resistive case.
        
        We discuss possible options for extending this FET approach to timesteppers for the compressible case.

        The kinetic corrections satisfy linearized Boltzmann equations. Using a Lénard--Bernstein collision operator, these take Fokker--Planck-like forms \cite{Fokker_1914, Planck_1917} wherein pseudo-particles in the numerical model obey the neoclassical transport equations, with particle-independent Brownian drift terms. This offers a rigorous methodology for incorporating collisions into the particle transport model, without coupling the equations of motions for each particle.
        
        Works by Chen, Chacón et al. \cite{Chen_Chacón_Barnes_2011, Chacón_Chen_Barnes_2013, Chen_Chacón_2014, Chen_Chacón_2015} have developed structure-preserving particle pushers for neoclassical transport in the Vlasov equations, derived from Crank--Nicolson integrators. We show these too can can derive from a FET interpretation, similarly offering potential extensions to higher-order-in-time particle pushers. The FET formulation is used also to consider how the stochastic drift terms can be incorporated into the pushers. Stochastic gyrokinetic expansions are also discussed.

        Different options for the numerical implementation of these schemes are considered.

        Due to the efficacy of FET in the development of SP timesteppers for both the fluid and kinetic component, we hope this approach will prove effective in the future for developing SP timesteppers for the full hybrid model. We hope this will give us the opportunity to incorporate previously inaccessible kinetic effects into the highly effective, modern, finite-element MHD models.
    \end{abstract}
    
    
    \newpage
    \tableofcontents
    
    
    \newpage
    \pagenumbering{arabic}
    %\linenumbers\renewcommand\thelinenumber{\color{black!50}\arabic{linenumber}}
            \input{0 - introduction/main.tex}
        \part{Research}
            \input{1 - low-noise PiC models/main.tex}
            \input{2 - kinetic component/main.tex}
            \input{3 - fluid component/main.tex}
            \input{4 - numerical implementation/main.tex}
        \part{Project Overview}
            \input{5 - research plan/main.tex}
            \input{6 - summary/main.tex}
    
    
    %\section{}
    \newpage
    \pagenumbering{gobble}
        \printbibliography


    \newpage
    \pagenumbering{roman}
    \appendix
        \part{Appendices}
            \input{8 - Hilbert complexes/main.tex}
            \input{9 - weak conservation proofs/main.tex}
\end{document}

            \documentclass[12pt, a4paper]{report}

\input{template/main.tex}

\title{\BA{Title in Progress...}}
\author{Boris Andrews}
\affil{Mathematical Institute, University of Oxford}
\date{\today}


\begin{document}
    \pagenumbering{gobble}
    \maketitle
    
    
    \begin{abstract}
        Magnetic confinement reactors---in particular tokamaks---offer one of the most promising options for achieving practical nuclear fusion, with the potential to provide virtually limitless, clean energy. The theoretical and numerical modeling of tokamak plasmas is simultaneously an essential component of effective reactor design, and a great research barrier. Tokamak operational conditions exhibit comparatively low Knudsen numbers. Kinetic effects, including kinetic waves and instabilities, Landau damping, bump-on-tail instabilities and more, are therefore highly influential in tokamak plasma dynamics. Purely fluid models are inherently incapable of capturing these effects, whereas the high dimensionality in purely kinetic models render them practically intractable for most relevant purposes.

        We consider a $\delta\!f$ decomposition model, with a macroscopic fluid background and microscopic kinetic correction, both fully coupled to each other. A similar manner of discretization is proposed to that used in the recent \texttt{STRUPHY} code \cite{Holderied_Possanner_Wang_2021, Holderied_2022, Li_et_al_2023} with a finite-element model for the background and a pseudo-particle/PiC model for the correction.

        The fluid background satisfies the full, non-linear, resistive, compressible, Hall MHD equations. \cite{Laakmann_Hu_Farrell_2022} introduces finite-element(-in-space) implicit timesteppers for the incompressible analogue to this system with structure-preserving (SP) properties in the ideal case, alongside parameter-robust preconditioners. We show that these timesteppers can derive from a finite-element-in-time (FET) (and finite-element-in-space) interpretation. The benefits of this reformulation are discussed, including the derivation of timesteppers that are higher order in time, and the quantifiable dissipative SP properties in the non-ideal, resistive case.
        
        We discuss possible options for extending this FET approach to timesteppers for the compressible case.

        The kinetic corrections satisfy linearized Boltzmann equations. Using a Lénard--Bernstein collision operator, these take Fokker--Planck-like forms \cite{Fokker_1914, Planck_1917} wherein pseudo-particles in the numerical model obey the neoclassical transport equations, with particle-independent Brownian drift terms. This offers a rigorous methodology for incorporating collisions into the particle transport model, without coupling the equations of motions for each particle.
        
        Works by Chen, Chacón et al. \cite{Chen_Chacón_Barnes_2011, Chacón_Chen_Barnes_2013, Chen_Chacón_2014, Chen_Chacón_2015} have developed structure-preserving particle pushers for neoclassical transport in the Vlasov equations, derived from Crank--Nicolson integrators. We show these too can can derive from a FET interpretation, similarly offering potential extensions to higher-order-in-time particle pushers. The FET formulation is used also to consider how the stochastic drift terms can be incorporated into the pushers. Stochastic gyrokinetic expansions are also discussed.

        Different options for the numerical implementation of these schemes are considered.

        Due to the efficacy of FET in the development of SP timesteppers for both the fluid and kinetic component, we hope this approach will prove effective in the future for developing SP timesteppers for the full hybrid model. We hope this will give us the opportunity to incorporate previously inaccessible kinetic effects into the highly effective, modern, finite-element MHD models.
    \end{abstract}
    
    
    \newpage
    \tableofcontents
    
    
    \newpage
    \pagenumbering{arabic}
    %\linenumbers\renewcommand\thelinenumber{\color{black!50}\arabic{linenumber}}
            \input{0 - introduction/main.tex}
        \part{Research}
            \input{1 - low-noise PiC models/main.tex}
            \input{2 - kinetic component/main.tex}
            \input{3 - fluid component/main.tex}
            \input{4 - numerical implementation/main.tex}
        \part{Project Overview}
            \input{5 - research plan/main.tex}
            \input{6 - summary/main.tex}
    
    
    %\section{}
    \newpage
    \pagenumbering{gobble}
        \printbibliography


    \newpage
    \pagenumbering{roman}
    \appendix
        \part{Appendices}
            \input{8 - Hilbert complexes/main.tex}
            \input{9 - weak conservation proofs/main.tex}
\end{document}

\end{document}

        \part{Research}
            \documentclass[12pt, a4paper]{report}

\documentclass[12pt, a4paper]{report}

\input{template/main.tex}

\title{\BA{Title in Progress...}}
\author{Boris Andrews}
\affil{Mathematical Institute, University of Oxford}
\date{\today}


\begin{document}
    \pagenumbering{gobble}
    \maketitle
    
    
    \begin{abstract}
        Magnetic confinement reactors---in particular tokamaks---offer one of the most promising options for achieving practical nuclear fusion, with the potential to provide virtually limitless, clean energy. The theoretical and numerical modeling of tokamak plasmas is simultaneously an essential component of effective reactor design, and a great research barrier. Tokamak operational conditions exhibit comparatively low Knudsen numbers. Kinetic effects, including kinetic waves and instabilities, Landau damping, bump-on-tail instabilities and more, are therefore highly influential in tokamak plasma dynamics. Purely fluid models are inherently incapable of capturing these effects, whereas the high dimensionality in purely kinetic models render them practically intractable for most relevant purposes.

        We consider a $\delta\!f$ decomposition model, with a macroscopic fluid background and microscopic kinetic correction, both fully coupled to each other. A similar manner of discretization is proposed to that used in the recent \texttt{STRUPHY} code \cite{Holderied_Possanner_Wang_2021, Holderied_2022, Li_et_al_2023} with a finite-element model for the background and a pseudo-particle/PiC model for the correction.

        The fluid background satisfies the full, non-linear, resistive, compressible, Hall MHD equations. \cite{Laakmann_Hu_Farrell_2022} introduces finite-element(-in-space) implicit timesteppers for the incompressible analogue to this system with structure-preserving (SP) properties in the ideal case, alongside parameter-robust preconditioners. We show that these timesteppers can derive from a finite-element-in-time (FET) (and finite-element-in-space) interpretation. The benefits of this reformulation are discussed, including the derivation of timesteppers that are higher order in time, and the quantifiable dissipative SP properties in the non-ideal, resistive case.
        
        We discuss possible options for extending this FET approach to timesteppers for the compressible case.

        The kinetic corrections satisfy linearized Boltzmann equations. Using a Lénard--Bernstein collision operator, these take Fokker--Planck-like forms \cite{Fokker_1914, Planck_1917} wherein pseudo-particles in the numerical model obey the neoclassical transport equations, with particle-independent Brownian drift terms. This offers a rigorous methodology for incorporating collisions into the particle transport model, without coupling the equations of motions for each particle.
        
        Works by Chen, Chacón et al. \cite{Chen_Chacón_Barnes_2011, Chacón_Chen_Barnes_2013, Chen_Chacón_2014, Chen_Chacón_2015} have developed structure-preserving particle pushers for neoclassical transport in the Vlasov equations, derived from Crank--Nicolson integrators. We show these too can can derive from a FET interpretation, similarly offering potential extensions to higher-order-in-time particle pushers. The FET formulation is used also to consider how the stochastic drift terms can be incorporated into the pushers. Stochastic gyrokinetic expansions are also discussed.

        Different options for the numerical implementation of these schemes are considered.

        Due to the efficacy of FET in the development of SP timesteppers for both the fluid and kinetic component, we hope this approach will prove effective in the future for developing SP timesteppers for the full hybrid model. We hope this will give us the opportunity to incorporate previously inaccessible kinetic effects into the highly effective, modern, finite-element MHD models.
    \end{abstract}
    
    
    \newpage
    \tableofcontents
    
    
    \newpage
    \pagenumbering{arabic}
    %\linenumbers\renewcommand\thelinenumber{\color{black!50}\arabic{linenumber}}
            \input{0 - introduction/main.tex}
        \part{Research}
            \input{1 - low-noise PiC models/main.tex}
            \input{2 - kinetic component/main.tex}
            \input{3 - fluid component/main.tex}
            \input{4 - numerical implementation/main.tex}
        \part{Project Overview}
            \input{5 - research plan/main.tex}
            \input{6 - summary/main.tex}
    
    
    %\section{}
    \newpage
    \pagenumbering{gobble}
        \printbibliography


    \newpage
    \pagenumbering{roman}
    \appendix
        \part{Appendices}
            \input{8 - Hilbert complexes/main.tex}
            \input{9 - weak conservation proofs/main.tex}
\end{document}


\title{\BA{Title in Progress...}}
\author{Boris Andrews}
\affil{Mathematical Institute, University of Oxford}
\date{\today}


\begin{document}
    \pagenumbering{gobble}
    \maketitle
    
    
    \begin{abstract}
        Magnetic confinement reactors---in particular tokamaks---offer one of the most promising options for achieving practical nuclear fusion, with the potential to provide virtually limitless, clean energy. The theoretical and numerical modeling of tokamak plasmas is simultaneously an essential component of effective reactor design, and a great research barrier. Tokamak operational conditions exhibit comparatively low Knudsen numbers. Kinetic effects, including kinetic waves and instabilities, Landau damping, bump-on-tail instabilities and more, are therefore highly influential in tokamak plasma dynamics. Purely fluid models are inherently incapable of capturing these effects, whereas the high dimensionality in purely kinetic models render them practically intractable for most relevant purposes.

        We consider a $\delta\!f$ decomposition model, with a macroscopic fluid background and microscopic kinetic correction, both fully coupled to each other. A similar manner of discretization is proposed to that used in the recent \texttt{STRUPHY} code \cite{Holderied_Possanner_Wang_2021, Holderied_2022, Li_et_al_2023} with a finite-element model for the background and a pseudo-particle/PiC model for the correction.

        The fluid background satisfies the full, non-linear, resistive, compressible, Hall MHD equations. \cite{Laakmann_Hu_Farrell_2022} introduces finite-element(-in-space) implicit timesteppers for the incompressible analogue to this system with structure-preserving (SP) properties in the ideal case, alongside parameter-robust preconditioners. We show that these timesteppers can derive from a finite-element-in-time (FET) (and finite-element-in-space) interpretation. The benefits of this reformulation are discussed, including the derivation of timesteppers that are higher order in time, and the quantifiable dissipative SP properties in the non-ideal, resistive case.
        
        We discuss possible options for extending this FET approach to timesteppers for the compressible case.

        The kinetic corrections satisfy linearized Boltzmann equations. Using a Lénard--Bernstein collision operator, these take Fokker--Planck-like forms \cite{Fokker_1914, Planck_1917} wherein pseudo-particles in the numerical model obey the neoclassical transport equations, with particle-independent Brownian drift terms. This offers a rigorous methodology for incorporating collisions into the particle transport model, without coupling the equations of motions for each particle.
        
        Works by Chen, Chacón et al. \cite{Chen_Chacón_Barnes_2011, Chacón_Chen_Barnes_2013, Chen_Chacón_2014, Chen_Chacón_2015} have developed structure-preserving particle pushers for neoclassical transport in the Vlasov equations, derived from Crank--Nicolson integrators. We show these too can can derive from a FET interpretation, similarly offering potential extensions to higher-order-in-time particle pushers. The FET formulation is used also to consider how the stochastic drift terms can be incorporated into the pushers. Stochastic gyrokinetic expansions are also discussed.

        Different options for the numerical implementation of these schemes are considered.

        Due to the efficacy of FET in the development of SP timesteppers for both the fluid and kinetic component, we hope this approach will prove effective in the future for developing SP timesteppers for the full hybrid model. We hope this will give us the opportunity to incorporate previously inaccessible kinetic effects into the highly effective, modern, finite-element MHD models.
    \end{abstract}
    
    
    \newpage
    \tableofcontents
    
    
    \newpage
    \pagenumbering{arabic}
    %\linenumbers\renewcommand\thelinenumber{\color{black!50}\arabic{linenumber}}
            \documentclass[12pt, a4paper]{report}

\input{template/main.tex}

\title{\BA{Title in Progress...}}
\author{Boris Andrews}
\affil{Mathematical Institute, University of Oxford}
\date{\today}


\begin{document}
    \pagenumbering{gobble}
    \maketitle
    
    
    \begin{abstract}
        Magnetic confinement reactors---in particular tokamaks---offer one of the most promising options for achieving practical nuclear fusion, with the potential to provide virtually limitless, clean energy. The theoretical and numerical modeling of tokamak plasmas is simultaneously an essential component of effective reactor design, and a great research barrier. Tokamak operational conditions exhibit comparatively low Knudsen numbers. Kinetic effects, including kinetic waves and instabilities, Landau damping, bump-on-tail instabilities and more, are therefore highly influential in tokamak plasma dynamics. Purely fluid models are inherently incapable of capturing these effects, whereas the high dimensionality in purely kinetic models render them practically intractable for most relevant purposes.

        We consider a $\delta\!f$ decomposition model, with a macroscopic fluid background and microscopic kinetic correction, both fully coupled to each other. A similar manner of discretization is proposed to that used in the recent \texttt{STRUPHY} code \cite{Holderied_Possanner_Wang_2021, Holderied_2022, Li_et_al_2023} with a finite-element model for the background and a pseudo-particle/PiC model for the correction.

        The fluid background satisfies the full, non-linear, resistive, compressible, Hall MHD equations. \cite{Laakmann_Hu_Farrell_2022} introduces finite-element(-in-space) implicit timesteppers for the incompressible analogue to this system with structure-preserving (SP) properties in the ideal case, alongside parameter-robust preconditioners. We show that these timesteppers can derive from a finite-element-in-time (FET) (and finite-element-in-space) interpretation. The benefits of this reformulation are discussed, including the derivation of timesteppers that are higher order in time, and the quantifiable dissipative SP properties in the non-ideal, resistive case.
        
        We discuss possible options for extending this FET approach to timesteppers for the compressible case.

        The kinetic corrections satisfy linearized Boltzmann equations. Using a Lénard--Bernstein collision operator, these take Fokker--Planck-like forms \cite{Fokker_1914, Planck_1917} wherein pseudo-particles in the numerical model obey the neoclassical transport equations, with particle-independent Brownian drift terms. This offers a rigorous methodology for incorporating collisions into the particle transport model, without coupling the equations of motions for each particle.
        
        Works by Chen, Chacón et al. \cite{Chen_Chacón_Barnes_2011, Chacón_Chen_Barnes_2013, Chen_Chacón_2014, Chen_Chacón_2015} have developed structure-preserving particle pushers for neoclassical transport in the Vlasov equations, derived from Crank--Nicolson integrators. We show these too can can derive from a FET interpretation, similarly offering potential extensions to higher-order-in-time particle pushers. The FET formulation is used also to consider how the stochastic drift terms can be incorporated into the pushers. Stochastic gyrokinetic expansions are also discussed.

        Different options for the numerical implementation of these schemes are considered.

        Due to the efficacy of FET in the development of SP timesteppers for both the fluid and kinetic component, we hope this approach will prove effective in the future for developing SP timesteppers for the full hybrid model. We hope this will give us the opportunity to incorporate previously inaccessible kinetic effects into the highly effective, modern, finite-element MHD models.
    \end{abstract}
    
    
    \newpage
    \tableofcontents
    
    
    \newpage
    \pagenumbering{arabic}
    %\linenumbers\renewcommand\thelinenumber{\color{black!50}\arabic{linenumber}}
            \input{0 - introduction/main.tex}
        \part{Research}
            \input{1 - low-noise PiC models/main.tex}
            \input{2 - kinetic component/main.tex}
            \input{3 - fluid component/main.tex}
            \input{4 - numerical implementation/main.tex}
        \part{Project Overview}
            \input{5 - research plan/main.tex}
            \input{6 - summary/main.tex}
    
    
    %\section{}
    \newpage
    \pagenumbering{gobble}
        \printbibliography


    \newpage
    \pagenumbering{roman}
    \appendix
        \part{Appendices}
            \input{8 - Hilbert complexes/main.tex}
            \input{9 - weak conservation proofs/main.tex}
\end{document}

        \part{Research}
            \documentclass[12pt, a4paper]{report}

\input{template/main.tex}

\title{\BA{Title in Progress...}}
\author{Boris Andrews}
\affil{Mathematical Institute, University of Oxford}
\date{\today}


\begin{document}
    \pagenumbering{gobble}
    \maketitle
    
    
    \begin{abstract}
        Magnetic confinement reactors---in particular tokamaks---offer one of the most promising options for achieving practical nuclear fusion, with the potential to provide virtually limitless, clean energy. The theoretical and numerical modeling of tokamak plasmas is simultaneously an essential component of effective reactor design, and a great research barrier. Tokamak operational conditions exhibit comparatively low Knudsen numbers. Kinetic effects, including kinetic waves and instabilities, Landau damping, bump-on-tail instabilities and more, are therefore highly influential in tokamak plasma dynamics. Purely fluid models are inherently incapable of capturing these effects, whereas the high dimensionality in purely kinetic models render them practically intractable for most relevant purposes.

        We consider a $\delta\!f$ decomposition model, with a macroscopic fluid background and microscopic kinetic correction, both fully coupled to each other. A similar manner of discretization is proposed to that used in the recent \texttt{STRUPHY} code \cite{Holderied_Possanner_Wang_2021, Holderied_2022, Li_et_al_2023} with a finite-element model for the background and a pseudo-particle/PiC model for the correction.

        The fluid background satisfies the full, non-linear, resistive, compressible, Hall MHD equations. \cite{Laakmann_Hu_Farrell_2022} introduces finite-element(-in-space) implicit timesteppers for the incompressible analogue to this system with structure-preserving (SP) properties in the ideal case, alongside parameter-robust preconditioners. We show that these timesteppers can derive from a finite-element-in-time (FET) (and finite-element-in-space) interpretation. The benefits of this reformulation are discussed, including the derivation of timesteppers that are higher order in time, and the quantifiable dissipative SP properties in the non-ideal, resistive case.
        
        We discuss possible options for extending this FET approach to timesteppers for the compressible case.

        The kinetic corrections satisfy linearized Boltzmann equations. Using a Lénard--Bernstein collision operator, these take Fokker--Planck-like forms \cite{Fokker_1914, Planck_1917} wherein pseudo-particles in the numerical model obey the neoclassical transport equations, with particle-independent Brownian drift terms. This offers a rigorous methodology for incorporating collisions into the particle transport model, without coupling the equations of motions for each particle.
        
        Works by Chen, Chacón et al. \cite{Chen_Chacón_Barnes_2011, Chacón_Chen_Barnes_2013, Chen_Chacón_2014, Chen_Chacón_2015} have developed structure-preserving particle pushers for neoclassical transport in the Vlasov equations, derived from Crank--Nicolson integrators. We show these too can can derive from a FET interpretation, similarly offering potential extensions to higher-order-in-time particle pushers. The FET formulation is used also to consider how the stochastic drift terms can be incorporated into the pushers. Stochastic gyrokinetic expansions are also discussed.

        Different options for the numerical implementation of these schemes are considered.

        Due to the efficacy of FET in the development of SP timesteppers for both the fluid and kinetic component, we hope this approach will prove effective in the future for developing SP timesteppers for the full hybrid model. We hope this will give us the opportunity to incorporate previously inaccessible kinetic effects into the highly effective, modern, finite-element MHD models.
    \end{abstract}
    
    
    \newpage
    \tableofcontents
    
    
    \newpage
    \pagenumbering{arabic}
    %\linenumbers\renewcommand\thelinenumber{\color{black!50}\arabic{linenumber}}
            \input{0 - introduction/main.tex}
        \part{Research}
            \input{1 - low-noise PiC models/main.tex}
            \input{2 - kinetic component/main.tex}
            \input{3 - fluid component/main.tex}
            \input{4 - numerical implementation/main.tex}
        \part{Project Overview}
            \input{5 - research plan/main.tex}
            \input{6 - summary/main.tex}
    
    
    %\section{}
    \newpage
    \pagenumbering{gobble}
        \printbibliography


    \newpage
    \pagenumbering{roman}
    \appendix
        \part{Appendices}
            \input{8 - Hilbert complexes/main.tex}
            \input{9 - weak conservation proofs/main.tex}
\end{document}

            \documentclass[12pt, a4paper]{report}

\input{template/main.tex}

\title{\BA{Title in Progress...}}
\author{Boris Andrews}
\affil{Mathematical Institute, University of Oxford}
\date{\today}


\begin{document}
    \pagenumbering{gobble}
    \maketitle
    
    
    \begin{abstract}
        Magnetic confinement reactors---in particular tokamaks---offer one of the most promising options for achieving practical nuclear fusion, with the potential to provide virtually limitless, clean energy. The theoretical and numerical modeling of tokamak plasmas is simultaneously an essential component of effective reactor design, and a great research barrier. Tokamak operational conditions exhibit comparatively low Knudsen numbers. Kinetic effects, including kinetic waves and instabilities, Landau damping, bump-on-tail instabilities and more, are therefore highly influential in tokamak plasma dynamics. Purely fluid models are inherently incapable of capturing these effects, whereas the high dimensionality in purely kinetic models render them practically intractable for most relevant purposes.

        We consider a $\delta\!f$ decomposition model, with a macroscopic fluid background and microscopic kinetic correction, both fully coupled to each other. A similar manner of discretization is proposed to that used in the recent \texttt{STRUPHY} code \cite{Holderied_Possanner_Wang_2021, Holderied_2022, Li_et_al_2023} with a finite-element model for the background and a pseudo-particle/PiC model for the correction.

        The fluid background satisfies the full, non-linear, resistive, compressible, Hall MHD equations. \cite{Laakmann_Hu_Farrell_2022} introduces finite-element(-in-space) implicit timesteppers for the incompressible analogue to this system with structure-preserving (SP) properties in the ideal case, alongside parameter-robust preconditioners. We show that these timesteppers can derive from a finite-element-in-time (FET) (and finite-element-in-space) interpretation. The benefits of this reformulation are discussed, including the derivation of timesteppers that are higher order in time, and the quantifiable dissipative SP properties in the non-ideal, resistive case.
        
        We discuss possible options for extending this FET approach to timesteppers for the compressible case.

        The kinetic corrections satisfy linearized Boltzmann equations. Using a Lénard--Bernstein collision operator, these take Fokker--Planck-like forms \cite{Fokker_1914, Planck_1917} wherein pseudo-particles in the numerical model obey the neoclassical transport equations, with particle-independent Brownian drift terms. This offers a rigorous methodology for incorporating collisions into the particle transport model, without coupling the equations of motions for each particle.
        
        Works by Chen, Chacón et al. \cite{Chen_Chacón_Barnes_2011, Chacón_Chen_Barnes_2013, Chen_Chacón_2014, Chen_Chacón_2015} have developed structure-preserving particle pushers for neoclassical transport in the Vlasov equations, derived from Crank--Nicolson integrators. We show these too can can derive from a FET interpretation, similarly offering potential extensions to higher-order-in-time particle pushers. The FET formulation is used also to consider how the stochastic drift terms can be incorporated into the pushers. Stochastic gyrokinetic expansions are also discussed.

        Different options for the numerical implementation of these schemes are considered.

        Due to the efficacy of FET in the development of SP timesteppers for both the fluid and kinetic component, we hope this approach will prove effective in the future for developing SP timesteppers for the full hybrid model. We hope this will give us the opportunity to incorporate previously inaccessible kinetic effects into the highly effective, modern, finite-element MHD models.
    \end{abstract}
    
    
    \newpage
    \tableofcontents
    
    
    \newpage
    \pagenumbering{arabic}
    %\linenumbers\renewcommand\thelinenumber{\color{black!50}\arabic{linenumber}}
            \input{0 - introduction/main.tex}
        \part{Research}
            \input{1 - low-noise PiC models/main.tex}
            \input{2 - kinetic component/main.tex}
            \input{3 - fluid component/main.tex}
            \input{4 - numerical implementation/main.tex}
        \part{Project Overview}
            \input{5 - research plan/main.tex}
            \input{6 - summary/main.tex}
    
    
    %\section{}
    \newpage
    \pagenumbering{gobble}
        \printbibliography


    \newpage
    \pagenumbering{roman}
    \appendix
        \part{Appendices}
            \input{8 - Hilbert complexes/main.tex}
            \input{9 - weak conservation proofs/main.tex}
\end{document}

            \documentclass[12pt, a4paper]{report}

\input{template/main.tex}

\title{\BA{Title in Progress...}}
\author{Boris Andrews}
\affil{Mathematical Institute, University of Oxford}
\date{\today}


\begin{document}
    \pagenumbering{gobble}
    \maketitle
    
    
    \begin{abstract}
        Magnetic confinement reactors---in particular tokamaks---offer one of the most promising options for achieving practical nuclear fusion, with the potential to provide virtually limitless, clean energy. The theoretical and numerical modeling of tokamak plasmas is simultaneously an essential component of effective reactor design, and a great research barrier. Tokamak operational conditions exhibit comparatively low Knudsen numbers. Kinetic effects, including kinetic waves and instabilities, Landau damping, bump-on-tail instabilities and more, are therefore highly influential in tokamak plasma dynamics. Purely fluid models are inherently incapable of capturing these effects, whereas the high dimensionality in purely kinetic models render them practically intractable for most relevant purposes.

        We consider a $\delta\!f$ decomposition model, with a macroscopic fluid background and microscopic kinetic correction, both fully coupled to each other. A similar manner of discretization is proposed to that used in the recent \texttt{STRUPHY} code \cite{Holderied_Possanner_Wang_2021, Holderied_2022, Li_et_al_2023} with a finite-element model for the background and a pseudo-particle/PiC model for the correction.

        The fluid background satisfies the full, non-linear, resistive, compressible, Hall MHD equations. \cite{Laakmann_Hu_Farrell_2022} introduces finite-element(-in-space) implicit timesteppers for the incompressible analogue to this system with structure-preserving (SP) properties in the ideal case, alongside parameter-robust preconditioners. We show that these timesteppers can derive from a finite-element-in-time (FET) (and finite-element-in-space) interpretation. The benefits of this reformulation are discussed, including the derivation of timesteppers that are higher order in time, and the quantifiable dissipative SP properties in the non-ideal, resistive case.
        
        We discuss possible options for extending this FET approach to timesteppers for the compressible case.

        The kinetic corrections satisfy linearized Boltzmann equations. Using a Lénard--Bernstein collision operator, these take Fokker--Planck-like forms \cite{Fokker_1914, Planck_1917} wherein pseudo-particles in the numerical model obey the neoclassical transport equations, with particle-independent Brownian drift terms. This offers a rigorous methodology for incorporating collisions into the particle transport model, without coupling the equations of motions for each particle.
        
        Works by Chen, Chacón et al. \cite{Chen_Chacón_Barnes_2011, Chacón_Chen_Barnes_2013, Chen_Chacón_2014, Chen_Chacón_2015} have developed structure-preserving particle pushers for neoclassical transport in the Vlasov equations, derived from Crank--Nicolson integrators. We show these too can can derive from a FET interpretation, similarly offering potential extensions to higher-order-in-time particle pushers. The FET formulation is used also to consider how the stochastic drift terms can be incorporated into the pushers. Stochastic gyrokinetic expansions are also discussed.

        Different options for the numerical implementation of these schemes are considered.

        Due to the efficacy of FET in the development of SP timesteppers for both the fluid and kinetic component, we hope this approach will prove effective in the future for developing SP timesteppers for the full hybrid model. We hope this will give us the opportunity to incorporate previously inaccessible kinetic effects into the highly effective, modern, finite-element MHD models.
    \end{abstract}
    
    
    \newpage
    \tableofcontents
    
    
    \newpage
    \pagenumbering{arabic}
    %\linenumbers\renewcommand\thelinenumber{\color{black!50}\arabic{linenumber}}
            \input{0 - introduction/main.tex}
        \part{Research}
            \input{1 - low-noise PiC models/main.tex}
            \input{2 - kinetic component/main.tex}
            \input{3 - fluid component/main.tex}
            \input{4 - numerical implementation/main.tex}
        \part{Project Overview}
            \input{5 - research plan/main.tex}
            \input{6 - summary/main.tex}
    
    
    %\section{}
    \newpage
    \pagenumbering{gobble}
        \printbibliography


    \newpage
    \pagenumbering{roman}
    \appendix
        \part{Appendices}
            \input{8 - Hilbert complexes/main.tex}
            \input{9 - weak conservation proofs/main.tex}
\end{document}

            \documentclass[12pt, a4paper]{report}

\input{template/main.tex}

\title{\BA{Title in Progress...}}
\author{Boris Andrews}
\affil{Mathematical Institute, University of Oxford}
\date{\today}


\begin{document}
    \pagenumbering{gobble}
    \maketitle
    
    
    \begin{abstract}
        Magnetic confinement reactors---in particular tokamaks---offer one of the most promising options for achieving practical nuclear fusion, with the potential to provide virtually limitless, clean energy. The theoretical and numerical modeling of tokamak plasmas is simultaneously an essential component of effective reactor design, and a great research barrier. Tokamak operational conditions exhibit comparatively low Knudsen numbers. Kinetic effects, including kinetic waves and instabilities, Landau damping, bump-on-tail instabilities and more, are therefore highly influential in tokamak plasma dynamics. Purely fluid models are inherently incapable of capturing these effects, whereas the high dimensionality in purely kinetic models render them practically intractable for most relevant purposes.

        We consider a $\delta\!f$ decomposition model, with a macroscopic fluid background and microscopic kinetic correction, both fully coupled to each other. A similar manner of discretization is proposed to that used in the recent \texttt{STRUPHY} code \cite{Holderied_Possanner_Wang_2021, Holderied_2022, Li_et_al_2023} with a finite-element model for the background and a pseudo-particle/PiC model for the correction.

        The fluid background satisfies the full, non-linear, resistive, compressible, Hall MHD equations. \cite{Laakmann_Hu_Farrell_2022} introduces finite-element(-in-space) implicit timesteppers for the incompressible analogue to this system with structure-preserving (SP) properties in the ideal case, alongside parameter-robust preconditioners. We show that these timesteppers can derive from a finite-element-in-time (FET) (and finite-element-in-space) interpretation. The benefits of this reformulation are discussed, including the derivation of timesteppers that are higher order in time, and the quantifiable dissipative SP properties in the non-ideal, resistive case.
        
        We discuss possible options for extending this FET approach to timesteppers for the compressible case.

        The kinetic corrections satisfy linearized Boltzmann equations. Using a Lénard--Bernstein collision operator, these take Fokker--Planck-like forms \cite{Fokker_1914, Planck_1917} wherein pseudo-particles in the numerical model obey the neoclassical transport equations, with particle-independent Brownian drift terms. This offers a rigorous methodology for incorporating collisions into the particle transport model, without coupling the equations of motions for each particle.
        
        Works by Chen, Chacón et al. \cite{Chen_Chacón_Barnes_2011, Chacón_Chen_Barnes_2013, Chen_Chacón_2014, Chen_Chacón_2015} have developed structure-preserving particle pushers for neoclassical transport in the Vlasov equations, derived from Crank--Nicolson integrators. We show these too can can derive from a FET interpretation, similarly offering potential extensions to higher-order-in-time particle pushers. The FET formulation is used also to consider how the stochastic drift terms can be incorporated into the pushers. Stochastic gyrokinetic expansions are also discussed.

        Different options for the numerical implementation of these schemes are considered.

        Due to the efficacy of FET in the development of SP timesteppers for both the fluid and kinetic component, we hope this approach will prove effective in the future for developing SP timesteppers for the full hybrid model. We hope this will give us the opportunity to incorporate previously inaccessible kinetic effects into the highly effective, modern, finite-element MHD models.
    \end{abstract}
    
    
    \newpage
    \tableofcontents
    
    
    \newpage
    \pagenumbering{arabic}
    %\linenumbers\renewcommand\thelinenumber{\color{black!50}\arabic{linenumber}}
            \input{0 - introduction/main.tex}
        \part{Research}
            \input{1 - low-noise PiC models/main.tex}
            \input{2 - kinetic component/main.tex}
            \input{3 - fluid component/main.tex}
            \input{4 - numerical implementation/main.tex}
        \part{Project Overview}
            \input{5 - research plan/main.tex}
            \input{6 - summary/main.tex}
    
    
    %\section{}
    \newpage
    \pagenumbering{gobble}
        \printbibliography


    \newpage
    \pagenumbering{roman}
    \appendix
        \part{Appendices}
            \input{8 - Hilbert complexes/main.tex}
            \input{9 - weak conservation proofs/main.tex}
\end{document}

        \part{Project Overview}
            \documentclass[12pt, a4paper]{report}

\input{template/main.tex}

\title{\BA{Title in Progress...}}
\author{Boris Andrews}
\affil{Mathematical Institute, University of Oxford}
\date{\today}


\begin{document}
    \pagenumbering{gobble}
    \maketitle
    
    
    \begin{abstract}
        Magnetic confinement reactors---in particular tokamaks---offer one of the most promising options for achieving practical nuclear fusion, with the potential to provide virtually limitless, clean energy. The theoretical and numerical modeling of tokamak plasmas is simultaneously an essential component of effective reactor design, and a great research barrier. Tokamak operational conditions exhibit comparatively low Knudsen numbers. Kinetic effects, including kinetic waves and instabilities, Landau damping, bump-on-tail instabilities and more, are therefore highly influential in tokamak plasma dynamics. Purely fluid models are inherently incapable of capturing these effects, whereas the high dimensionality in purely kinetic models render them practically intractable for most relevant purposes.

        We consider a $\delta\!f$ decomposition model, with a macroscopic fluid background and microscopic kinetic correction, both fully coupled to each other. A similar manner of discretization is proposed to that used in the recent \texttt{STRUPHY} code \cite{Holderied_Possanner_Wang_2021, Holderied_2022, Li_et_al_2023} with a finite-element model for the background and a pseudo-particle/PiC model for the correction.

        The fluid background satisfies the full, non-linear, resistive, compressible, Hall MHD equations. \cite{Laakmann_Hu_Farrell_2022} introduces finite-element(-in-space) implicit timesteppers for the incompressible analogue to this system with structure-preserving (SP) properties in the ideal case, alongside parameter-robust preconditioners. We show that these timesteppers can derive from a finite-element-in-time (FET) (and finite-element-in-space) interpretation. The benefits of this reformulation are discussed, including the derivation of timesteppers that are higher order in time, and the quantifiable dissipative SP properties in the non-ideal, resistive case.
        
        We discuss possible options for extending this FET approach to timesteppers for the compressible case.

        The kinetic corrections satisfy linearized Boltzmann equations. Using a Lénard--Bernstein collision operator, these take Fokker--Planck-like forms \cite{Fokker_1914, Planck_1917} wherein pseudo-particles in the numerical model obey the neoclassical transport equations, with particle-independent Brownian drift terms. This offers a rigorous methodology for incorporating collisions into the particle transport model, without coupling the equations of motions for each particle.
        
        Works by Chen, Chacón et al. \cite{Chen_Chacón_Barnes_2011, Chacón_Chen_Barnes_2013, Chen_Chacón_2014, Chen_Chacón_2015} have developed structure-preserving particle pushers for neoclassical transport in the Vlasov equations, derived from Crank--Nicolson integrators. We show these too can can derive from a FET interpretation, similarly offering potential extensions to higher-order-in-time particle pushers. The FET formulation is used also to consider how the stochastic drift terms can be incorporated into the pushers. Stochastic gyrokinetic expansions are also discussed.

        Different options for the numerical implementation of these schemes are considered.

        Due to the efficacy of FET in the development of SP timesteppers for both the fluid and kinetic component, we hope this approach will prove effective in the future for developing SP timesteppers for the full hybrid model. We hope this will give us the opportunity to incorporate previously inaccessible kinetic effects into the highly effective, modern, finite-element MHD models.
    \end{abstract}
    
    
    \newpage
    \tableofcontents
    
    
    \newpage
    \pagenumbering{arabic}
    %\linenumbers\renewcommand\thelinenumber{\color{black!50}\arabic{linenumber}}
            \input{0 - introduction/main.tex}
        \part{Research}
            \input{1 - low-noise PiC models/main.tex}
            \input{2 - kinetic component/main.tex}
            \input{3 - fluid component/main.tex}
            \input{4 - numerical implementation/main.tex}
        \part{Project Overview}
            \input{5 - research plan/main.tex}
            \input{6 - summary/main.tex}
    
    
    %\section{}
    \newpage
    \pagenumbering{gobble}
        \printbibliography


    \newpage
    \pagenumbering{roman}
    \appendix
        \part{Appendices}
            \input{8 - Hilbert complexes/main.tex}
            \input{9 - weak conservation proofs/main.tex}
\end{document}

            \documentclass[12pt, a4paper]{report}

\input{template/main.tex}

\title{\BA{Title in Progress...}}
\author{Boris Andrews}
\affil{Mathematical Institute, University of Oxford}
\date{\today}


\begin{document}
    \pagenumbering{gobble}
    \maketitle
    
    
    \begin{abstract}
        Magnetic confinement reactors---in particular tokamaks---offer one of the most promising options for achieving practical nuclear fusion, with the potential to provide virtually limitless, clean energy. The theoretical and numerical modeling of tokamak plasmas is simultaneously an essential component of effective reactor design, and a great research barrier. Tokamak operational conditions exhibit comparatively low Knudsen numbers. Kinetic effects, including kinetic waves and instabilities, Landau damping, bump-on-tail instabilities and more, are therefore highly influential in tokamak plasma dynamics. Purely fluid models are inherently incapable of capturing these effects, whereas the high dimensionality in purely kinetic models render them practically intractable for most relevant purposes.

        We consider a $\delta\!f$ decomposition model, with a macroscopic fluid background and microscopic kinetic correction, both fully coupled to each other. A similar manner of discretization is proposed to that used in the recent \texttt{STRUPHY} code \cite{Holderied_Possanner_Wang_2021, Holderied_2022, Li_et_al_2023} with a finite-element model for the background and a pseudo-particle/PiC model for the correction.

        The fluid background satisfies the full, non-linear, resistive, compressible, Hall MHD equations. \cite{Laakmann_Hu_Farrell_2022} introduces finite-element(-in-space) implicit timesteppers for the incompressible analogue to this system with structure-preserving (SP) properties in the ideal case, alongside parameter-robust preconditioners. We show that these timesteppers can derive from a finite-element-in-time (FET) (and finite-element-in-space) interpretation. The benefits of this reformulation are discussed, including the derivation of timesteppers that are higher order in time, and the quantifiable dissipative SP properties in the non-ideal, resistive case.
        
        We discuss possible options for extending this FET approach to timesteppers for the compressible case.

        The kinetic corrections satisfy linearized Boltzmann equations. Using a Lénard--Bernstein collision operator, these take Fokker--Planck-like forms \cite{Fokker_1914, Planck_1917} wherein pseudo-particles in the numerical model obey the neoclassical transport equations, with particle-independent Brownian drift terms. This offers a rigorous methodology for incorporating collisions into the particle transport model, without coupling the equations of motions for each particle.
        
        Works by Chen, Chacón et al. \cite{Chen_Chacón_Barnes_2011, Chacón_Chen_Barnes_2013, Chen_Chacón_2014, Chen_Chacón_2015} have developed structure-preserving particle pushers for neoclassical transport in the Vlasov equations, derived from Crank--Nicolson integrators. We show these too can can derive from a FET interpretation, similarly offering potential extensions to higher-order-in-time particle pushers. The FET formulation is used also to consider how the stochastic drift terms can be incorporated into the pushers. Stochastic gyrokinetic expansions are also discussed.

        Different options for the numerical implementation of these schemes are considered.

        Due to the efficacy of FET in the development of SP timesteppers for both the fluid and kinetic component, we hope this approach will prove effective in the future for developing SP timesteppers for the full hybrid model. We hope this will give us the opportunity to incorporate previously inaccessible kinetic effects into the highly effective, modern, finite-element MHD models.
    \end{abstract}
    
    
    \newpage
    \tableofcontents
    
    
    \newpage
    \pagenumbering{arabic}
    %\linenumbers\renewcommand\thelinenumber{\color{black!50}\arabic{linenumber}}
            \input{0 - introduction/main.tex}
        \part{Research}
            \input{1 - low-noise PiC models/main.tex}
            \input{2 - kinetic component/main.tex}
            \input{3 - fluid component/main.tex}
            \input{4 - numerical implementation/main.tex}
        \part{Project Overview}
            \input{5 - research plan/main.tex}
            \input{6 - summary/main.tex}
    
    
    %\section{}
    \newpage
    \pagenumbering{gobble}
        \printbibliography


    \newpage
    \pagenumbering{roman}
    \appendix
        \part{Appendices}
            \input{8 - Hilbert complexes/main.tex}
            \input{9 - weak conservation proofs/main.tex}
\end{document}

    
    
    %\section{}
    \newpage
    \pagenumbering{gobble}
        \printbibliography


    \newpage
    \pagenumbering{roman}
    \appendix
        \part{Appendices}
            \documentclass[12pt, a4paper]{report}

\input{template/main.tex}

\title{\BA{Title in Progress...}}
\author{Boris Andrews}
\affil{Mathematical Institute, University of Oxford}
\date{\today}


\begin{document}
    \pagenumbering{gobble}
    \maketitle
    
    
    \begin{abstract}
        Magnetic confinement reactors---in particular tokamaks---offer one of the most promising options for achieving practical nuclear fusion, with the potential to provide virtually limitless, clean energy. The theoretical and numerical modeling of tokamak plasmas is simultaneously an essential component of effective reactor design, and a great research barrier. Tokamak operational conditions exhibit comparatively low Knudsen numbers. Kinetic effects, including kinetic waves and instabilities, Landau damping, bump-on-tail instabilities and more, are therefore highly influential in tokamak plasma dynamics. Purely fluid models are inherently incapable of capturing these effects, whereas the high dimensionality in purely kinetic models render them practically intractable for most relevant purposes.

        We consider a $\delta\!f$ decomposition model, with a macroscopic fluid background and microscopic kinetic correction, both fully coupled to each other. A similar manner of discretization is proposed to that used in the recent \texttt{STRUPHY} code \cite{Holderied_Possanner_Wang_2021, Holderied_2022, Li_et_al_2023} with a finite-element model for the background and a pseudo-particle/PiC model for the correction.

        The fluid background satisfies the full, non-linear, resistive, compressible, Hall MHD equations. \cite{Laakmann_Hu_Farrell_2022} introduces finite-element(-in-space) implicit timesteppers for the incompressible analogue to this system with structure-preserving (SP) properties in the ideal case, alongside parameter-robust preconditioners. We show that these timesteppers can derive from a finite-element-in-time (FET) (and finite-element-in-space) interpretation. The benefits of this reformulation are discussed, including the derivation of timesteppers that are higher order in time, and the quantifiable dissipative SP properties in the non-ideal, resistive case.
        
        We discuss possible options for extending this FET approach to timesteppers for the compressible case.

        The kinetic corrections satisfy linearized Boltzmann equations. Using a Lénard--Bernstein collision operator, these take Fokker--Planck-like forms \cite{Fokker_1914, Planck_1917} wherein pseudo-particles in the numerical model obey the neoclassical transport equations, with particle-independent Brownian drift terms. This offers a rigorous methodology for incorporating collisions into the particle transport model, without coupling the equations of motions for each particle.
        
        Works by Chen, Chacón et al. \cite{Chen_Chacón_Barnes_2011, Chacón_Chen_Barnes_2013, Chen_Chacón_2014, Chen_Chacón_2015} have developed structure-preserving particle pushers for neoclassical transport in the Vlasov equations, derived from Crank--Nicolson integrators. We show these too can can derive from a FET interpretation, similarly offering potential extensions to higher-order-in-time particle pushers. The FET formulation is used also to consider how the stochastic drift terms can be incorporated into the pushers. Stochastic gyrokinetic expansions are also discussed.

        Different options for the numerical implementation of these schemes are considered.

        Due to the efficacy of FET in the development of SP timesteppers for both the fluid and kinetic component, we hope this approach will prove effective in the future for developing SP timesteppers for the full hybrid model. We hope this will give us the opportunity to incorporate previously inaccessible kinetic effects into the highly effective, modern, finite-element MHD models.
    \end{abstract}
    
    
    \newpage
    \tableofcontents
    
    
    \newpage
    \pagenumbering{arabic}
    %\linenumbers\renewcommand\thelinenumber{\color{black!50}\arabic{linenumber}}
            \input{0 - introduction/main.tex}
        \part{Research}
            \input{1 - low-noise PiC models/main.tex}
            \input{2 - kinetic component/main.tex}
            \input{3 - fluid component/main.tex}
            \input{4 - numerical implementation/main.tex}
        \part{Project Overview}
            \input{5 - research plan/main.tex}
            \input{6 - summary/main.tex}
    
    
    %\section{}
    \newpage
    \pagenumbering{gobble}
        \printbibliography


    \newpage
    \pagenumbering{roman}
    \appendix
        \part{Appendices}
            \input{8 - Hilbert complexes/main.tex}
            \input{9 - weak conservation proofs/main.tex}
\end{document}

            \documentclass[12pt, a4paper]{report}

\input{template/main.tex}

\title{\BA{Title in Progress...}}
\author{Boris Andrews}
\affil{Mathematical Institute, University of Oxford}
\date{\today}


\begin{document}
    \pagenumbering{gobble}
    \maketitle
    
    
    \begin{abstract}
        Magnetic confinement reactors---in particular tokamaks---offer one of the most promising options for achieving practical nuclear fusion, with the potential to provide virtually limitless, clean energy. The theoretical and numerical modeling of tokamak plasmas is simultaneously an essential component of effective reactor design, and a great research barrier. Tokamak operational conditions exhibit comparatively low Knudsen numbers. Kinetic effects, including kinetic waves and instabilities, Landau damping, bump-on-tail instabilities and more, are therefore highly influential in tokamak plasma dynamics. Purely fluid models are inherently incapable of capturing these effects, whereas the high dimensionality in purely kinetic models render them practically intractable for most relevant purposes.

        We consider a $\delta\!f$ decomposition model, with a macroscopic fluid background and microscopic kinetic correction, both fully coupled to each other. A similar manner of discretization is proposed to that used in the recent \texttt{STRUPHY} code \cite{Holderied_Possanner_Wang_2021, Holderied_2022, Li_et_al_2023} with a finite-element model for the background and a pseudo-particle/PiC model for the correction.

        The fluid background satisfies the full, non-linear, resistive, compressible, Hall MHD equations. \cite{Laakmann_Hu_Farrell_2022} introduces finite-element(-in-space) implicit timesteppers for the incompressible analogue to this system with structure-preserving (SP) properties in the ideal case, alongside parameter-robust preconditioners. We show that these timesteppers can derive from a finite-element-in-time (FET) (and finite-element-in-space) interpretation. The benefits of this reformulation are discussed, including the derivation of timesteppers that are higher order in time, and the quantifiable dissipative SP properties in the non-ideal, resistive case.
        
        We discuss possible options for extending this FET approach to timesteppers for the compressible case.

        The kinetic corrections satisfy linearized Boltzmann equations. Using a Lénard--Bernstein collision operator, these take Fokker--Planck-like forms \cite{Fokker_1914, Planck_1917} wherein pseudo-particles in the numerical model obey the neoclassical transport equations, with particle-independent Brownian drift terms. This offers a rigorous methodology for incorporating collisions into the particle transport model, without coupling the equations of motions for each particle.
        
        Works by Chen, Chacón et al. \cite{Chen_Chacón_Barnes_2011, Chacón_Chen_Barnes_2013, Chen_Chacón_2014, Chen_Chacón_2015} have developed structure-preserving particle pushers for neoclassical transport in the Vlasov equations, derived from Crank--Nicolson integrators. We show these too can can derive from a FET interpretation, similarly offering potential extensions to higher-order-in-time particle pushers. The FET formulation is used also to consider how the stochastic drift terms can be incorporated into the pushers. Stochastic gyrokinetic expansions are also discussed.

        Different options for the numerical implementation of these schemes are considered.

        Due to the efficacy of FET in the development of SP timesteppers for both the fluid and kinetic component, we hope this approach will prove effective in the future for developing SP timesteppers for the full hybrid model. We hope this will give us the opportunity to incorporate previously inaccessible kinetic effects into the highly effective, modern, finite-element MHD models.
    \end{abstract}
    
    
    \newpage
    \tableofcontents
    
    
    \newpage
    \pagenumbering{arabic}
    %\linenumbers\renewcommand\thelinenumber{\color{black!50}\arabic{linenumber}}
            \input{0 - introduction/main.tex}
        \part{Research}
            \input{1 - low-noise PiC models/main.tex}
            \input{2 - kinetic component/main.tex}
            \input{3 - fluid component/main.tex}
            \input{4 - numerical implementation/main.tex}
        \part{Project Overview}
            \input{5 - research plan/main.tex}
            \input{6 - summary/main.tex}
    
    
    %\section{}
    \newpage
    \pagenumbering{gobble}
        \printbibliography


    \newpage
    \pagenumbering{roman}
    \appendix
        \part{Appendices}
            \input{8 - Hilbert complexes/main.tex}
            \input{9 - weak conservation proofs/main.tex}
\end{document}

\end{document}

            \documentclass[12pt, a4paper]{report}

\documentclass[12pt, a4paper]{report}

\input{template/main.tex}

\title{\BA{Title in Progress...}}
\author{Boris Andrews}
\affil{Mathematical Institute, University of Oxford}
\date{\today}


\begin{document}
    \pagenumbering{gobble}
    \maketitle
    
    
    \begin{abstract}
        Magnetic confinement reactors---in particular tokamaks---offer one of the most promising options for achieving practical nuclear fusion, with the potential to provide virtually limitless, clean energy. The theoretical and numerical modeling of tokamak plasmas is simultaneously an essential component of effective reactor design, and a great research barrier. Tokamak operational conditions exhibit comparatively low Knudsen numbers. Kinetic effects, including kinetic waves and instabilities, Landau damping, bump-on-tail instabilities and more, are therefore highly influential in tokamak plasma dynamics. Purely fluid models are inherently incapable of capturing these effects, whereas the high dimensionality in purely kinetic models render them practically intractable for most relevant purposes.

        We consider a $\delta\!f$ decomposition model, with a macroscopic fluid background and microscopic kinetic correction, both fully coupled to each other. A similar manner of discretization is proposed to that used in the recent \texttt{STRUPHY} code \cite{Holderied_Possanner_Wang_2021, Holderied_2022, Li_et_al_2023} with a finite-element model for the background and a pseudo-particle/PiC model for the correction.

        The fluid background satisfies the full, non-linear, resistive, compressible, Hall MHD equations. \cite{Laakmann_Hu_Farrell_2022} introduces finite-element(-in-space) implicit timesteppers for the incompressible analogue to this system with structure-preserving (SP) properties in the ideal case, alongside parameter-robust preconditioners. We show that these timesteppers can derive from a finite-element-in-time (FET) (and finite-element-in-space) interpretation. The benefits of this reformulation are discussed, including the derivation of timesteppers that are higher order in time, and the quantifiable dissipative SP properties in the non-ideal, resistive case.
        
        We discuss possible options for extending this FET approach to timesteppers for the compressible case.

        The kinetic corrections satisfy linearized Boltzmann equations. Using a Lénard--Bernstein collision operator, these take Fokker--Planck-like forms \cite{Fokker_1914, Planck_1917} wherein pseudo-particles in the numerical model obey the neoclassical transport equations, with particle-independent Brownian drift terms. This offers a rigorous methodology for incorporating collisions into the particle transport model, without coupling the equations of motions for each particle.
        
        Works by Chen, Chacón et al. \cite{Chen_Chacón_Barnes_2011, Chacón_Chen_Barnes_2013, Chen_Chacón_2014, Chen_Chacón_2015} have developed structure-preserving particle pushers for neoclassical transport in the Vlasov equations, derived from Crank--Nicolson integrators. We show these too can can derive from a FET interpretation, similarly offering potential extensions to higher-order-in-time particle pushers. The FET formulation is used also to consider how the stochastic drift terms can be incorporated into the pushers. Stochastic gyrokinetic expansions are also discussed.

        Different options for the numerical implementation of these schemes are considered.

        Due to the efficacy of FET in the development of SP timesteppers for both the fluid and kinetic component, we hope this approach will prove effective in the future for developing SP timesteppers for the full hybrid model. We hope this will give us the opportunity to incorporate previously inaccessible kinetic effects into the highly effective, modern, finite-element MHD models.
    \end{abstract}
    
    
    \newpage
    \tableofcontents
    
    
    \newpage
    \pagenumbering{arabic}
    %\linenumbers\renewcommand\thelinenumber{\color{black!50}\arabic{linenumber}}
            \input{0 - introduction/main.tex}
        \part{Research}
            \input{1 - low-noise PiC models/main.tex}
            \input{2 - kinetic component/main.tex}
            \input{3 - fluid component/main.tex}
            \input{4 - numerical implementation/main.tex}
        \part{Project Overview}
            \input{5 - research plan/main.tex}
            \input{6 - summary/main.tex}
    
    
    %\section{}
    \newpage
    \pagenumbering{gobble}
        \printbibliography


    \newpage
    \pagenumbering{roman}
    \appendix
        \part{Appendices}
            \input{8 - Hilbert complexes/main.tex}
            \input{9 - weak conservation proofs/main.tex}
\end{document}


\title{\BA{Title in Progress...}}
\author{Boris Andrews}
\affil{Mathematical Institute, University of Oxford}
\date{\today}


\begin{document}
    \pagenumbering{gobble}
    \maketitle
    
    
    \begin{abstract}
        Magnetic confinement reactors---in particular tokamaks---offer one of the most promising options for achieving practical nuclear fusion, with the potential to provide virtually limitless, clean energy. The theoretical and numerical modeling of tokamak plasmas is simultaneously an essential component of effective reactor design, and a great research barrier. Tokamak operational conditions exhibit comparatively low Knudsen numbers. Kinetic effects, including kinetic waves and instabilities, Landau damping, bump-on-tail instabilities and more, are therefore highly influential in tokamak plasma dynamics. Purely fluid models are inherently incapable of capturing these effects, whereas the high dimensionality in purely kinetic models render them practically intractable for most relevant purposes.

        We consider a $\delta\!f$ decomposition model, with a macroscopic fluid background and microscopic kinetic correction, both fully coupled to each other. A similar manner of discretization is proposed to that used in the recent \texttt{STRUPHY} code \cite{Holderied_Possanner_Wang_2021, Holderied_2022, Li_et_al_2023} with a finite-element model for the background and a pseudo-particle/PiC model for the correction.

        The fluid background satisfies the full, non-linear, resistive, compressible, Hall MHD equations. \cite{Laakmann_Hu_Farrell_2022} introduces finite-element(-in-space) implicit timesteppers for the incompressible analogue to this system with structure-preserving (SP) properties in the ideal case, alongside parameter-robust preconditioners. We show that these timesteppers can derive from a finite-element-in-time (FET) (and finite-element-in-space) interpretation. The benefits of this reformulation are discussed, including the derivation of timesteppers that are higher order in time, and the quantifiable dissipative SP properties in the non-ideal, resistive case.
        
        We discuss possible options for extending this FET approach to timesteppers for the compressible case.

        The kinetic corrections satisfy linearized Boltzmann equations. Using a Lénard--Bernstein collision operator, these take Fokker--Planck-like forms \cite{Fokker_1914, Planck_1917} wherein pseudo-particles in the numerical model obey the neoclassical transport equations, with particle-independent Brownian drift terms. This offers a rigorous methodology for incorporating collisions into the particle transport model, without coupling the equations of motions for each particle.
        
        Works by Chen, Chacón et al. \cite{Chen_Chacón_Barnes_2011, Chacón_Chen_Barnes_2013, Chen_Chacón_2014, Chen_Chacón_2015} have developed structure-preserving particle pushers for neoclassical transport in the Vlasov equations, derived from Crank--Nicolson integrators. We show these too can can derive from a FET interpretation, similarly offering potential extensions to higher-order-in-time particle pushers. The FET formulation is used also to consider how the stochastic drift terms can be incorporated into the pushers. Stochastic gyrokinetic expansions are also discussed.

        Different options for the numerical implementation of these schemes are considered.

        Due to the efficacy of FET in the development of SP timesteppers for both the fluid and kinetic component, we hope this approach will prove effective in the future for developing SP timesteppers for the full hybrid model. We hope this will give us the opportunity to incorporate previously inaccessible kinetic effects into the highly effective, modern, finite-element MHD models.
    \end{abstract}
    
    
    \newpage
    \tableofcontents
    
    
    \newpage
    \pagenumbering{arabic}
    %\linenumbers\renewcommand\thelinenumber{\color{black!50}\arabic{linenumber}}
            \documentclass[12pt, a4paper]{report}

\input{template/main.tex}

\title{\BA{Title in Progress...}}
\author{Boris Andrews}
\affil{Mathematical Institute, University of Oxford}
\date{\today}


\begin{document}
    \pagenumbering{gobble}
    \maketitle
    
    
    \begin{abstract}
        Magnetic confinement reactors---in particular tokamaks---offer one of the most promising options for achieving practical nuclear fusion, with the potential to provide virtually limitless, clean energy. The theoretical and numerical modeling of tokamak plasmas is simultaneously an essential component of effective reactor design, and a great research barrier. Tokamak operational conditions exhibit comparatively low Knudsen numbers. Kinetic effects, including kinetic waves and instabilities, Landau damping, bump-on-tail instabilities and more, are therefore highly influential in tokamak plasma dynamics. Purely fluid models are inherently incapable of capturing these effects, whereas the high dimensionality in purely kinetic models render them practically intractable for most relevant purposes.

        We consider a $\delta\!f$ decomposition model, with a macroscopic fluid background and microscopic kinetic correction, both fully coupled to each other. A similar manner of discretization is proposed to that used in the recent \texttt{STRUPHY} code \cite{Holderied_Possanner_Wang_2021, Holderied_2022, Li_et_al_2023} with a finite-element model for the background and a pseudo-particle/PiC model for the correction.

        The fluid background satisfies the full, non-linear, resistive, compressible, Hall MHD equations. \cite{Laakmann_Hu_Farrell_2022} introduces finite-element(-in-space) implicit timesteppers for the incompressible analogue to this system with structure-preserving (SP) properties in the ideal case, alongside parameter-robust preconditioners. We show that these timesteppers can derive from a finite-element-in-time (FET) (and finite-element-in-space) interpretation. The benefits of this reformulation are discussed, including the derivation of timesteppers that are higher order in time, and the quantifiable dissipative SP properties in the non-ideal, resistive case.
        
        We discuss possible options for extending this FET approach to timesteppers for the compressible case.

        The kinetic corrections satisfy linearized Boltzmann equations. Using a Lénard--Bernstein collision operator, these take Fokker--Planck-like forms \cite{Fokker_1914, Planck_1917} wherein pseudo-particles in the numerical model obey the neoclassical transport equations, with particle-independent Brownian drift terms. This offers a rigorous methodology for incorporating collisions into the particle transport model, without coupling the equations of motions for each particle.
        
        Works by Chen, Chacón et al. \cite{Chen_Chacón_Barnes_2011, Chacón_Chen_Barnes_2013, Chen_Chacón_2014, Chen_Chacón_2015} have developed structure-preserving particle pushers for neoclassical transport in the Vlasov equations, derived from Crank--Nicolson integrators. We show these too can can derive from a FET interpretation, similarly offering potential extensions to higher-order-in-time particle pushers. The FET formulation is used also to consider how the stochastic drift terms can be incorporated into the pushers. Stochastic gyrokinetic expansions are also discussed.

        Different options for the numerical implementation of these schemes are considered.

        Due to the efficacy of FET in the development of SP timesteppers for both the fluid and kinetic component, we hope this approach will prove effective in the future for developing SP timesteppers for the full hybrid model. We hope this will give us the opportunity to incorporate previously inaccessible kinetic effects into the highly effective, modern, finite-element MHD models.
    \end{abstract}
    
    
    \newpage
    \tableofcontents
    
    
    \newpage
    \pagenumbering{arabic}
    %\linenumbers\renewcommand\thelinenumber{\color{black!50}\arabic{linenumber}}
            \input{0 - introduction/main.tex}
        \part{Research}
            \input{1 - low-noise PiC models/main.tex}
            \input{2 - kinetic component/main.tex}
            \input{3 - fluid component/main.tex}
            \input{4 - numerical implementation/main.tex}
        \part{Project Overview}
            \input{5 - research plan/main.tex}
            \input{6 - summary/main.tex}
    
    
    %\section{}
    \newpage
    \pagenumbering{gobble}
        \printbibliography


    \newpage
    \pagenumbering{roman}
    \appendix
        \part{Appendices}
            \input{8 - Hilbert complexes/main.tex}
            \input{9 - weak conservation proofs/main.tex}
\end{document}

        \part{Research}
            \documentclass[12pt, a4paper]{report}

\input{template/main.tex}

\title{\BA{Title in Progress...}}
\author{Boris Andrews}
\affil{Mathematical Institute, University of Oxford}
\date{\today}


\begin{document}
    \pagenumbering{gobble}
    \maketitle
    
    
    \begin{abstract}
        Magnetic confinement reactors---in particular tokamaks---offer one of the most promising options for achieving practical nuclear fusion, with the potential to provide virtually limitless, clean energy. The theoretical and numerical modeling of tokamak plasmas is simultaneously an essential component of effective reactor design, and a great research barrier. Tokamak operational conditions exhibit comparatively low Knudsen numbers. Kinetic effects, including kinetic waves and instabilities, Landau damping, bump-on-tail instabilities and more, are therefore highly influential in tokamak plasma dynamics. Purely fluid models are inherently incapable of capturing these effects, whereas the high dimensionality in purely kinetic models render them practically intractable for most relevant purposes.

        We consider a $\delta\!f$ decomposition model, with a macroscopic fluid background and microscopic kinetic correction, both fully coupled to each other. A similar manner of discretization is proposed to that used in the recent \texttt{STRUPHY} code \cite{Holderied_Possanner_Wang_2021, Holderied_2022, Li_et_al_2023} with a finite-element model for the background and a pseudo-particle/PiC model for the correction.

        The fluid background satisfies the full, non-linear, resistive, compressible, Hall MHD equations. \cite{Laakmann_Hu_Farrell_2022} introduces finite-element(-in-space) implicit timesteppers for the incompressible analogue to this system with structure-preserving (SP) properties in the ideal case, alongside parameter-robust preconditioners. We show that these timesteppers can derive from a finite-element-in-time (FET) (and finite-element-in-space) interpretation. The benefits of this reformulation are discussed, including the derivation of timesteppers that are higher order in time, and the quantifiable dissipative SP properties in the non-ideal, resistive case.
        
        We discuss possible options for extending this FET approach to timesteppers for the compressible case.

        The kinetic corrections satisfy linearized Boltzmann equations. Using a Lénard--Bernstein collision operator, these take Fokker--Planck-like forms \cite{Fokker_1914, Planck_1917} wherein pseudo-particles in the numerical model obey the neoclassical transport equations, with particle-independent Brownian drift terms. This offers a rigorous methodology for incorporating collisions into the particle transport model, without coupling the equations of motions for each particle.
        
        Works by Chen, Chacón et al. \cite{Chen_Chacón_Barnes_2011, Chacón_Chen_Barnes_2013, Chen_Chacón_2014, Chen_Chacón_2015} have developed structure-preserving particle pushers for neoclassical transport in the Vlasov equations, derived from Crank--Nicolson integrators. We show these too can can derive from a FET interpretation, similarly offering potential extensions to higher-order-in-time particle pushers. The FET formulation is used also to consider how the stochastic drift terms can be incorporated into the pushers. Stochastic gyrokinetic expansions are also discussed.

        Different options for the numerical implementation of these schemes are considered.

        Due to the efficacy of FET in the development of SP timesteppers for both the fluid and kinetic component, we hope this approach will prove effective in the future for developing SP timesteppers for the full hybrid model. We hope this will give us the opportunity to incorporate previously inaccessible kinetic effects into the highly effective, modern, finite-element MHD models.
    \end{abstract}
    
    
    \newpage
    \tableofcontents
    
    
    \newpage
    \pagenumbering{arabic}
    %\linenumbers\renewcommand\thelinenumber{\color{black!50}\arabic{linenumber}}
            \input{0 - introduction/main.tex}
        \part{Research}
            \input{1 - low-noise PiC models/main.tex}
            \input{2 - kinetic component/main.tex}
            \input{3 - fluid component/main.tex}
            \input{4 - numerical implementation/main.tex}
        \part{Project Overview}
            \input{5 - research plan/main.tex}
            \input{6 - summary/main.tex}
    
    
    %\section{}
    \newpage
    \pagenumbering{gobble}
        \printbibliography


    \newpage
    \pagenumbering{roman}
    \appendix
        \part{Appendices}
            \input{8 - Hilbert complexes/main.tex}
            \input{9 - weak conservation proofs/main.tex}
\end{document}

            \documentclass[12pt, a4paper]{report}

\input{template/main.tex}

\title{\BA{Title in Progress...}}
\author{Boris Andrews}
\affil{Mathematical Institute, University of Oxford}
\date{\today}


\begin{document}
    \pagenumbering{gobble}
    \maketitle
    
    
    \begin{abstract}
        Magnetic confinement reactors---in particular tokamaks---offer one of the most promising options for achieving practical nuclear fusion, with the potential to provide virtually limitless, clean energy. The theoretical and numerical modeling of tokamak plasmas is simultaneously an essential component of effective reactor design, and a great research barrier. Tokamak operational conditions exhibit comparatively low Knudsen numbers. Kinetic effects, including kinetic waves and instabilities, Landau damping, bump-on-tail instabilities and more, are therefore highly influential in tokamak plasma dynamics. Purely fluid models are inherently incapable of capturing these effects, whereas the high dimensionality in purely kinetic models render them practically intractable for most relevant purposes.

        We consider a $\delta\!f$ decomposition model, with a macroscopic fluid background and microscopic kinetic correction, both fully coupled to each other. A similar manner of discretization is proposed to that used in the recent \texttt{STRUPHY} code \cite{Holderied_Possanner_Wang_2021, Holderied_2022, Li_et_al_2023} with a finite-element model for the background and a pseudo-particle/PiC model for the correction.

        The fluid background satisfies the full, non-linear, resistive, compressible, Hall MHD equations. \cite{Laakmann_Hu_Farrell_2022} introduces finite-element(-in-space) implicit timesteppers for the incompressible analogue to this system with structure-preserving (SP) properties in the ideal case, alongside parameter-robust preconditioners. We show that these timesteppers can derive from a finite-element-in-time (FET) (and finite-element-in-space) interpretation. The benefits of this reformulation are discussed, including the derivation of timesteppers that are higher order in time, and the quantifiable dissipative SP properties in the non-ideal, resistive case.
        
        We discuss possible options for extending this FET approach to timesteppers for the compressible case.

        The kinetic corrections satisfy linearized Boltzmann equations. Using a Lénard--Bernstein collision operator, these take Fokker--Planck-like forms \cite{Fokker_1914, Planck_1917} wherein pseudo-particles in the numerical model obey the neoclassical transport equations, with particle-independent Brownian drift terms. This offers a rigorous methodology for incorporating collisions into the particle transport model, without coupling the equations of motions for each particle.
        
        Works by Chen, Chacón et al. \cite{Chen_Chacón_Barnes_2011, Chacón_Chen_Barnes_2013, Chen_Chacón_2014, Chen_Chacón_2015} have developed structure-preserving particle pushers for neoclassical transport in the Vlasov equations, derived from Crank--Nicolson integrators. We show these too can can derive from a FET interpretation, similarly offering potential extensions to higher-order-in-time particle pushers. The FET formulation is used also to consider how the stochastic drift terms can be incorporated into the pushers. Stochastic gyrokinetic expansions are also discussed.

        Different options for the numerical implementation of these schemes are considered.

        Due to the efficacy of FET in the development of SP timesteppers for both the fluid and kinetic component, we hope this approach will prove effective in the future for developing SP timesteppers for the full hybrid model. We hope this will give us the opportunity to incorporate previously inaccessible kinetic effects into the highly effective, modern, finite-element MHD models.
    \end{abstract}
    
    
    \newpage
    \tableofcontents
    
    
    \newpage
    \pagenumbering{arabic}
    %\linenumbers\renewcommand\thelinenumber{\color{black!50}\arabic{linenumber}}
            \input{0 - introduction/main.tex}
        \part{Research}
            \input{1 - low-noise PiC models/main.tex}
            \input{2 - kinetic component/main.tex}
            \input{3 - fluid component/main.tex}
            \input{4 - numerical implementation/main.tex}
        \part{Project Overview}
            \input{5 - research plan/main.tex}
            \input{6 - summary/main.tex}
    
    
    %\section{}
    \newpage
    \pagenumbering{gobble}
        \printbibliography


    \newpage
    \pagenumbering{roman}
    \appendix
        \part{Appendices}
            \input{8 - Hilbert complexes/main.tex}
            \input{9 - weak conservation proofs/main.tex}
\end{document}

            \documentclass[12pt, a4paper]{report}

\input{template/main.tex}

\title{\BA{Title in Progress...}}
\author{Boris Andrews}
\affil{Mathematical Institute, University of Oxford}
\date{\today}


\begin{document}
    \pagenumbering{gobble}
    \maketitle
    
    
    \begin{abstract}
        Magnetic confinement reactors---in particular tokamaks---offer one of the most promising options for achieving practical nuclear fusion, with the potential to provide virtually limitless, clean energy. The theoretical and numerical modeling of tokamak plasmas is simultaneously an essential component of effective reactor design, and a great research barrier. Tokamak operational conditions exhibit comparatively low Knudsen numbers. Kinetic effects, including kinetic waves and instabilities, Landau damping, bump-on-tail instabilities and more, are therefore highly influential in tokamak plasma dynamics. Purely fluid models are inherently incapable of capturing these effects, whereas the high dimensionality in purely kinetic models render them practically intractable for most relevant purposes.

        We consider a $\delta\!f$ decomposition model, with a macroscopic fluid background and microscopic kinetic correction, both fully coupled to each other. A similar manner of discretization is proposed to that used in the recent \texttt{STRUPHY} code \cite{Holderied_Possanner_Wang_2021, Holderied_2022, Li_et_al_2023} with a finite-element model for the background and a pseudo-particle/PiC model for the correction.

        The fluid background satisfies the full, non-linear, resistive, compressible, Hall MHD equations. \cite{Laakmann_Hu_Farrell_2022} introduces finite-element(-in-space) implicit timesteppers for the incompressible analogue to this system with structure-preserving (SP) properties in the ideal case, alongside parameter-robust preconditioners. We show that these timesteppers can derive from a finite-element-in-time (FET) (and finite-element-in-space) interpretation. The benefits of this reformulation are discussed, including the derivation of timesteppers that are higher order in time, and the quantifiable dissipative SP properties in the non-ideal, resistive case.
        
        We discuss possible options for extending this FET approach to timesteppers for the compressible case.

        The kinetic corrections satisfy linearized Boltzmann equations. Using a Lénard--Bernstein collision operator, these take Fokker--Planck-like forms \cite{Fokker_1914, Planck_1917} wherein pseudo-particles in the numerical model obey the neoclassical transport equations, with particle-independent Brownian drift terms. This offers a rigorous methodology for incorporating collisions into the particle transport model, without coupling the equations of motions for each particle.
        
        Works by Chen, Chacón et al. \cite{Chen_Chacón_Barnes_2011, Chacón_Chen_Barnes_2013, Chen_Chacón_2014, Chen_Chacón_2015} have developed structure-preserving particle pushers for neoclassical transport in the Vlasov equations, derived from Crank--Nicolson integrators. We show these too can can derive from a FET interpretation, similarly offering potential extensions to higher-order-in-time particle pushers. The FET formulation is used also to consider how the stochastic drift terms can be incorporated into the pushers. Stochastic gyrokinetic expansions are also discussed.

        Different options for the numerical implementation of these schemes are considered.

        Due to the efficacy of FET in the development of SP timesteppers for both the fluid and kinetic component, we hope this approach will prove effective in the future for developing SP timesteppers for the full hybrid model. We hope this will give us the opportunity to incorporate previously inaccessible kinetic effects into the highly effective, modern, finite-element MHD models.
    \end{abstract}
    
    
    \newpage
    \tableofcontents
    
    
    \newpage
    \pagenumbering{arabic}
    %\linenumbers\renewcommand\thelinenumber{\color{black!50}\arabic{linenumber}}
            \input{0 - introduction/main.tex}
        \part{Research}
            \input{1 - low-noise PiC models/main.tex}
            \input{2 - kinetic component/main.tex}
            \input{3 - fluid component/main.tex}
            \input{4 - numerical implementation/main.tex}
        \part{Project Overview}
            \input{5 - research plan/main.tex}
            \input{6 - summary/main.tex}
    
    
    %\section{}
    \newpage
    \pagenumbering{gobble}
        \printbibliography


    \newpage
    \pagenumbering{roman}
    \appendix
        \part{Appendices}
            \input{8 - Hilbert complexes/main.tex}
            \input{9 - weak conservation proofs/main.tex}
\end{document}

            \documentclass[12pt, a4paper]{report}

\input{template/main.tex}

\title{\BA{Title in Progress...}}
\author{Boris Andrews}
\affil{Mathematical Institute, University of Oxford}
\date{\today}


\begin{document}
    \pagenumbering{gobble}
    \maketitle
    
    
    \begin{abstract}
        Magnetic confinement reactors---in particular tokamaks---offer one of the most promising options for achieving practical nuclear fusion, with the potential to provide virtually limitless, clean energy. The theoretical and numerical modeling of tokamak plasmas is simultaneously an essential component of effective reactor design, and a great research barrier. Tokamak operational conditions exhibit comparatively low Knudsen numbers. Kinetic effects, including kinetic waves and instabilities, Landau damping, bump-on-tail instabilities and more, are therefore highly influential in tokamak plasma dynamics. Purely fluid models are inherently incapable of capturing these effects, whereas the high dimensionality in purely kinetic models render them practically intractable for most relevant purposes.

        We consider a $\delta\!f$ decomposition model, with a macroscopic fluid background and microscopic kinetic correction, both fully coupled to each other. A similar manner of discretization is proposed to that used in the recent \texttt{STRUPHY} code \cite{Holderied_Possanner_Wang_2021, Holderied_2022, Li_et_al_2023} with a finite-element model for the background and a pseudo-particle/PiC model for the correction.

        The fluid background satisfies the full, non-linear, resistive, compressible, Hall MHD equations. \cite{Laakmann_Hu_Farrell_2022} introduces finite-element(-in-space) implicit timesteppers for the incompressible analogue to this system with structure-preserving (SP) properties in the ideal case, alongside parameter-robust preconditioners. We show that these timesteppers can derive from a finite-element-in-time (FET) (and finite-element-in-space) interpretation. The benefits of this reformulation are discussed, including the derivation of timesteppers that are higher order in time, and the quantifiable dissipative SP properties in the non-ideal, resistive case.
        
        We discuss possible options for extending this FET approach to timesteppers for the compressible case.

        The kinetic corrections satisfy linearized Boltzmann equations. Using a Lénard--Bernstein collision operator, these take Fokker--Planck-like forms \cite{Fokker_1914, Planck_1917} wherein pseudo-particles in the numerical model obey the neoclassical transport equations, with particle-independent Brownian drift terms. This offers a rigorous methodology for incorporating collisions into the particle transport model, without coupling the equations of motions for each particle.
        
        Works by Chen, Chacón et al. \cite{Chen_Chacón_Barnes_2011, Chacón_Chen_Barnes_2013, Chen_Chacón_2014, Chen_Chacón_2015} have developed structure-preserving particle pushers for neoclassical transport in the Vlasov equations, derived from Crank--Nicolson integrators. We show these too can can derive from a FET interpretation, similarly offering potential extensions to higher-order-in-time particle pushers. The FET formulation is used also to consider how the stochastic drift terms can be incorporated into the pushers. Stochastic gyrokinetic expansions are also discussed.

        Different options for the numerical implementation of these schemes are considered.

        Due to the efficacy of FET in the development of SP timesteppers for both the fluid and kinetic component, we hope this approach will prove effective in the future for developing SP timesteppers for the full hybrid model. We hope this will give us the opportunity to incorporate previously inaccessible kinetic effects into the highly effective, modern, finite-element MHD models.
    \end{abstract}
    
    
    \newpage
    \tableofcontents
    
    
    \newpage
    \pagenumbering{arabic}
    %\linenumbers\renewcommand\thelinenumber{\color{black!50}\arabic{linenumber}}
            \input{0 - introduction/main.tex}
        \part{Research}
            \input{1 - low-noise PiC models/main.tex}
            \input{2 - kinetic component/main.tex}
            \input{3 - fluid component/main.tex}
            \input{4 - numerical implementation/main.tex}
        \part{Project Overview}
            \input{5 - research plan/main.tex}
            \input{6 - summary/main.tex}
    
    
    %\section{}
    \newpage
    \pagenumbering{gobble}
        \printbibliography


    \newpage
    \pagenumbering{roman}
    \appendix
        \part{Appendices}
            \input{8 - Hilbert complexes/main.tex}
            \input{9 - weak conservation proofs/main.tex}
\end{document}

        \part{Project Overview}
            \documentclass[12pt, a4paper]{report}

\input{template/main.tex}

\title{\BA{Title in Progress...}}
\author{Boris Andrews}
\affil{Mathematical Institute, University of Oxford}
\date{\today}


\begin{document}
    \pagenumbering{gobble}
    \maketitle
    
    
    \begin{abstract}
        Magnetic confinement reactors---in particular tokamaks---offer one of the most promising options for achieving practical nuclear fusion, with the potential to provide virtually limitless, clean energy. The theoretical and numerical modeling of tokamak plasmas is simultaneously an essential component of effective reactor design, and a great research barrier. Tokamak operational conditions exhibit comparatively low Knudsen numbers. Kinetic effects, including kinetic waves and instabilities, Landau damping, bump-on-tail instabilities and more, are therefore highly influential in tokamak plasma dynamics. Purely fluid models are inherently incapable of capturing these effects, whereas the high dimensionality in purely kinetic models render them practically intractable for most relevant purposes.

        We consider a $\delta\!f$ decomposition model, with a macroscopic fluid background and microscopic kinetic correction, both fully coupled to each other. A similar manner of discretization is proposed to that used in the recent \texttt{STRUPHY} code \cite{Holderied_Possanner_Wang_2021, Holderied_2022, Li_et_al_2023} with a finite-element model for the background and a pseudo-particle/PiC model for the correction.

        The fluid background satisfies the full, non-linear, resistive, compressible, Hall MHD equations. \cite{Laakmann_Hu_Farrell_2022} introduces finite-element(-in-space) implicit timesteppers for the incompressible analogue to this system with structure-preserving (SP) properties in the ideal case, alongside parameter-robust preconditioners. We show that these timesteppers can derive from a finite-element-in-time (FET) (and finite-element-in-space) interpretation. The benefits of this reformulation are discussed, including the derivation of timesteppers that are higher order in time, and the quantifiable dissipative SP properties in the non-ideal, resistive case.
        
        We discuss possible options for extending this FET approach to timesteppers for the compressible case.

        The kinetic corrections satisfy linearized Boltzmann equations. Using a Lénard--Bernstein collision operator, these take Fokker--Planck-like forms \cite{Fokker_1914, Planck_1917} wherein pseudo-particles in the numerical model obey the neoclassical transport equations, with particle-independent Brownian drift terms. This offers a rigorous methodology for incorporating collisions into the particle transport model, without coupling the equations of motions for each particle.
        
        Works by Chen, Chacón et al. \cite{Chen_Chacón_Barnes_2011, Chacón_Chen_Barnes_2013, Chen_Chacón_2014, Chen_Chacón_2015} have developed structure-preserving particle pushers for neoclassical transport in the Vlasov equations, derived from Crank--Nicolson integrators. We show these too can can derive from a FET interpretation, similarly offering potential extensions to higher-order-in-time particle pushers. The FET formulation is used also to consider how the stochastic drift terms can be incorporated into the pushers. Stochastic gyrokinetic expansions are also discussed.

        Different options for the numerical implementation of these schemes are considered.

        Due to the efficacy of FET in the development of SP timesteppers for both the fluid and kinetic component, we hope this approach will prove effective in the future for developing SP timesteppers for the full hybrid model. We hope this will give us the opportunity to incorporate previously inaccessible kinetic effects into the highly effective, modern, finite-element MHD models.
    \end{abstract}
    
    
    \newpage
    \tableofcontents
    
    
    \newpage
    \pagenumbering{arabic}
    %\linenumbers\renewcommand\thelinenumber{\color{black!50}\arabic{linenumber}}
            \input{0 - introduction/main.tex}
        \part{Research}
            \input{1 - low-noise PiC models/main.tex}
            \input{2 - kinetic component/main.tex}
            \input{3 - fluid component/main.tex}
            \input{4 - numerical implementation/main.tex}
        \part{Project Overview}
            \input{5 - research plan/main.tex}
            \input{6 - summary/main.tex}
    
    
    %\section{}
    \newpage
    \pagenumbering{gobble}
        \printbibliography


    \newpage
    \pagenumbering{roman}
    \appendix
        \part{Appendices}
            \input{8 - Hilbert complexes/main.tex}
            \input{9 - weak conservation proofs/main.tex}
\end{document}

            \documentclass[12pt, a4paper]{report}

\input{template/main.tex}

\title{\BA{Title in Progress...}}
\author{Boris Andrews}
\affil{Mathematical Institute, University of Oxford}
\date{\today}


\begin{document}
    \pagenumbering{gobble}
    \maketitle
    
    
    \begin{abstract}
        Magnetic confinement reactors---in particular tokamaks---offer one of the most promising options for achieving practical nuclear fusion, with the potential to provide virtually limitless, clean energy. The theoretical and numerical modeling of tokamak plasmas is simultaneously an essential component of effective reactor design, and a great research barrier. Tokamak operational conditions exhibit comparatively low Knudsen numbers. Kinetic effects, including kinetic waves and instabilities, Landau damping, bump-on-tail instabilities and more, are therefore highly influential in tokamak plasma dynamics. Purely fluid models are inherently incapable of capturing these effects, whereas the high dimensionality in purely kinetic models render them practically intractable for most relevant purposes.

        We consider a $\delta\!f$ decomposition model, with a macroscopic fluid background and microscopic kinetic correction, both fully coupled to each other. A similar manner of discretization is proposed to that used in the recent \texttt{STRUPHY} code \cite{Holderied_Possanner_Wang_2021, Holderied_2022, Li_et_al_2023} with a finite-element model for the background and a pseudo-particle/PiC model for the correction.

        The fluid background satisfies the full, non-linear, resistive, compressible, Hall MHD equations. \cite{Laakmann_Hu_Farrell_2022} introduces finite-element(-in-space) implicit timesteppers for the incompressible analogue to this system with structure-preserving (SP) properties in the ideal case, alongside parameter-robust preconditioners. We show that these timesteppers can derive from a finite-element-in-time (FET) (and finite-element-in-space) interpretation. The benefits of this reformulation are discussed, including the derivation of timesteppers that are higher order in time, and the quantifiable dissipative SP properties in the non-ideal, resistive case.
        
        We discuss possible options for extending this FET approach to timesteppers for the compressible case.

        The kinetic corrections satisfy linearized Boltzmann equations. Using a Lénard--Bernstein collision operator, these take Fokker--Planck-like forms \cite{Fokker_1914, Planck_1917} wherein pseudo-particles in the numerical model obey the neoclassical transport equations, with particle-independent Brownian drift terms. This offers a rigorous methodology for incorporating collisions into the particle transport model, without coupling the equations of motions for each particle.
        
        Works by Chen, Chacón et al. \cite{Chen_Chacón_Barnes_2011, Chacón_Chen_Barnes_2013, Chen_Chacón_2014, Chen_Chacón_2015} have developed structure-preserving particle pushers for neoclassical transport in the Vlasov equations, derived from Crank--Nicolson integrators. We show these too can can derive from a FET interpretation, similarly offering potential extensions to higher-order-in-time particle pushers. The FET formulation is used also to consider how the stochastic drift terms can be incorporated into the pushers. Stochastic gyrokinetic expansions are also discussed.

        Different options for the numerical implementation of these schemes are considered.

        Due to the efficacy of FET in the development of SP timesteppers for both the fluid and kinetic component, we hope this approach will prove effective in the future for developing SP timesteppers for the full hybrid model. We hope this will give us the opportunity to incorporate previously inaccessible kinetic effects into the highly effective, modern, finite-element MHD models.
    \end{abstract}
    
    
    \newpage
    \tableofcontents
    
    
    \newpage
    \pagenumbering{arabic}
    %\linenumbers\renewcommand\thelinenumber{\color{black!50}\arabic{linenumber}}
            \input{0 - introduction/main.tex}
        \part{Research}
            \input{1 - low-noise PiC models/main.tex}
            \input{2 - kinetic component/main.tex}
            \input{3 - fluid component/main.tex}
            \input{4 - numerical implementation/main.tex}
        \part{Project Overview}
            \input{5 - research plan/main.tex}
            \input{6 - summary/main.tex}
    
    
    %\section{}
    \newpage
    \pagenumbering{gobble}
        \printbibliography


    \newpage
    \pagenumbering{roman}
    \appendix
        \part{Appendices}
            \input{8 - Hilbert complexes/main.tex}
            \input{9 - weak conservation proofs/main.tex}
\end{document}

    
    
    %\section{}
    \newpage
    \pagenumbering{gobble}
        \printbibliography


    \newpage
    \pagenumbering{roman}
    \appendix
        \part{Appendices}
            \documentclass[12pt, a4paper]{report}

\input{template/main.tex}

\title{\BA{Title in Progress...}}
\author{Boris Andrews}
\affil{Mathematical Institute, University of Oxford}
\date{\today}


\begin{document}
    \pagenumbering{gobble}
    \maketitle
    
    
    \begin{abstract}
        Magnetic confinement reactors---in particular tokamaks---offer one of the most promising options for achieving practical nuclear fusion, with the potential to provide virtually limitless, clean energy. The theoretical and numerical modeling of tokamak plasmas is simultaneously an essential component of effective reactor design, and a great research barrier. Tokamak operational conditions exhibit comparatively low Knudsen numbers. Kinetic effects, including kinetic waves and instabilities, Landau damping, bump-on-tail instabilities and more, are therefore highly influential in tokamak plasma dynamics. Purely fluid models are inherently incapable of capturing these effects, whereas the high dimensionality in purely kinetic models render them practically intractable for most relevant purposes.

        We consider a $\delta\!f$ decomposition model, with a macroscopic fluid background and microscopic kinetic correction, both fully coupled to each other. A similar manner of discretization is proposed to that used in the recent \texttt{STRUPHY} code \cite{Holderied_Possanner_Wang_2021, Holderied_2022, Li_et_al_2023} with a finite-element model for the background and a pseudo-particle/PiC model for the correction.

        The fluid background satisfies the full, non-linear, resistive, compressible, Hall MHD equations. \cite{Laakmann_Hu_Farrell_2022} introduces finite-element(-in-space) implicit timesteppers for the incompressible analogue to this system with structure-preserving (SP) properties in the ideal case, alongside parameter-robust preconditioners. We show that these timesteppers can derive from a finite-element-in-time (FET) (and finite-element-in-space) interpretation. The benefits of this reformulation are discussed, including the derivation of timesteppers that are higher order in time, and the quantifiable dissipative SP properties in the non-ideal, resistive case.
        
        We discuss possible options for extending this FET approach to timesteppers for the compressible case.

        The kinetic corrections satisfy linearized Boltzmann equations. Using a Lénard--Bernstein collision operator, these take Fokker--Planck-like forms \cite{Fokker_1914, Planck_1917} wherein pseudo-particles in the numerical model obey the neoclassical transport equations, with particle-independent Brownian drift terms. This offers a rigorous methodology for incorporating collisions into the particle transport model, without coupling the equations of motions for each particle.
        
        Works by Chen, Chacón et al. \cite{Chen_Chacón_Barnes_2011, Chacón_Chen_Barnes_2013, Chen_Chacón_2014, Chen_Chacón_2015} have developed structure-preserving particle pushers for neoclassical transport in the Vlasov equations, derived from Crank--Nicolson integrators. We show these too can can derive from a FET interpretation, similarly offering potential extensions to higher-order-in-time particle pushers. The FET formulation is used also to consider how the stochastic drift terms can be incorporated into the pushers. Stochastic gyrokinetic expansions are also discussed.

        Different options for the numerical implementation of these schemes are considered.

        Due to the efficacy of FET in the development of SP timesteppers for both the fluid and kinetic component, we hope this approach will prove effective in the future for developing SP timesteppers for the full hybrid model. We hope this will give us the opportunity to incorporate previously inaccessible kinetic effects into the highly effective, modern, finite-element MHD models.
    \end{abstract}
    
    
    \newpage
    \tableofcontents
    
    
    \newpage
    \pagenumbering{arabic}
    %\linenumbers\renewcommand\thelinenumber{\color{black!50}\arabic{linenumber}}
            \input{0 - introduction/main.tex}
        \part{Research}
            \input{1 - low-noise PiC models/main.tex}
            \input{2 - kinetic component/main.tex}
            \input{3 - fluid component/main.tex}
            \input{4 - numerical implementation/main.tex}
        \part{Project Overview}
            \input{5 - research plan/main.tex}
            \input{6 - summary/main.tex}
    
    
    %\section{}
    \newpage
    \pagenumbering{gobble}
        \printbibliography


    \newpage
    \pagenumbering{roman}
    \appendix
        \part{Appendices}
            \input{8 - Hilbert complexes/main.tex}
            \input{9 - weak conservation proofs/main.tex}
\end{document}

            \documentclass[12pt, a4paper]{report}

\input{template/main.tex}

\title{\BA{Title in Progress...}}
\author{Boris Andrews}
\affil{Mathematical Institute, University of Oxford}
\date{\today}


\begin{document}
    \pagenumbering{gobble}
    \maketitle
    
    
    \begin{abstract}
        Magnetic confinement reactors---in particular tokamaks---offer one of the most promising options for achieving practical nuclear fusion, with the potential to provide virtually limitless, clean energy. The theoretical and numerical modeling of tokamak plasmas is simultaneously an essential component of effective reactor design, and a great research barrier. Tokamak operational conditions exhibit comparatively low Knudsen numbers. Kinetic effects, including kinetic waves and instabilities, Landau damping, bump-on-tail instabilities and more, are therefore highly influential in tokamak plasma dynamics. Purely fluid models are inherently incapable of capturing these effects, whereas the high dimensionality in purely kinetic models render them practically intractable for most relevant purposes.

        We consider a $\delta\!f$ decomposition model, with a macroscopic fluid background and microscopic kinetic correction, both fully coupled to each other. A similar manner of discretization is proposed to that used in the recent \texttt{STRUPHY} code \cite{Holderied_Possanner_Wang_2021, Holderied_2022, Li_et_al_2023} with a finite-element model for the background and a pseudo-particle/PiC model for the correction.

        The fluid background satisfies the full, non-linear, resistive, compressible, Hall MHD equations. \cite{Laakmann_Hu_Farrell_2022} introduces finite-element(-in-space) implicit timesteppers for the incompressible analogue to this system with structure-preserving (SP) properties in the ideal case, alongside parameter-robust preconditioners. We show that these timesteppers can derive from a finite-element-in-time (FET) (and finite-element-in-space) interpretation. The benefits of this reformulation are discussed, including the derivation of timesteppers that are higher order in time, and the quantifiable dissipative SP properties in the non-ideal, resistive case.
        
        We discuss possible options for extending this FET approach to timesteppers for the compressible case.

        The kinetic corrections satisfy linearized Boltzmann equations. Using a Lénard--Bernstein collision operator, these take Fokker--Planck-like forms \cite{Fokker_1914, Planck_1917} wherein pseudo-particles in the numerical model obey the neoclassical transport equations, with particle-independent Brownian drift terms. This offers a rigorous methodology for incorporating collisions into the particle transport model, without coupling the equations of motions for each particle.
        
        Works by Chen, Chacón et al. \cite{Chen_Chacón_Barnes_2011, Chacón_Chen_Barnes_2013, Chen_Chacón_2014, Chen_Chacón_2015} have developed structure-preserving particle pushers for neoclassical transport in the Vlasov equations, derived from Crank--Nicolson integrators. We show these too can can derive from a FET interpretation, similarly offering potential extensions to higher-order-in-time particle pushers. The FET formulation is used also to consider how the stochastic drift terms can be incorporated into the pushers. Stochastic gyrokinetic expansions are also discussed.

        Different options for the numerical implementation of these schemes are considered.

        Due to the efficacy of FET in the development of SP timesteppers for both the fluid and kinetic component, we hope this approach will prove effective in the future for developing SP timesteppers for the full hybrid model. We hope this will give us the opportunity to incorporate previously inaccessible kinetic effects into the highly effective, modern, finite-element MHD models.
    \end{abstract}
    
    
    \newpage
    \tableofcontents
    
    
    \newpage
    \pagenumbering{arabic}
    %\linenumbers\renewcommand\thelinenumber{\color{black!50}\arabic{linenumber}}
            \input{0 - introduction/main.tex}
        \part{Research}
            \input{1 - low-noise PiC models/main.tex}
            \input{2 - kinetic component/main.tex}
            \input{3 - fluid component/main.tex}
            \input{4 - numerical implementation/main.tex}
        \part{Project Overview}
            \input{5 - research plan/main.tex}
            \input{6 - summary/main.tex}
    
    
    %\section{}
    \newpage
    \pagenumbering{gobble}
        \printbibliography


    \newpage
    \pagenumbering{roman}
    \appendix
        \part{Appendices}
            \input{8 - Hilbert complexes/main.tex}
            \input{9 - weak conservation proofs/main.tex}
\end{document}

\end{document}

            \documentclass[12pt, a4paper]{report}

\documentclass[12pt, a4paper]{report}

\input{template/main.tex}

\title{\BA{Title in Progress...}}
\author{Boris Andrews}
\affil{Mathematical Institute, University of Oxford}
\date{\today}


\begin{document}
    \pagenumbering{gobble}
    \maketitle
    
    
    \begin{abstract}
        Magnetic confinement reactors---in particular tokamaks---offer one of the most promising options for achieving practical nuclear fusion, with the potential to provide virtually limitless, clean energy. The theoretical and numerical modeling of tokamak plasmas is simultaneously an essential component of effective reactor design, and a great research barrier. Tokamak operational conditions exhibit comparatively low Knudsen numbers. Kinetic effects, including kinetic waves and instabilities, Landau damping, bump-on-tail instabilities and more, are therefore highly influential in tokamak plasma dynamics. Purely fluid models are inherently incapable of capturing these effects, whereas the high dimensionality in purely kinetic models render them practically intractable for most relevant purposes.

        We consider a $\delta\!f$ decomposition model, with a macroscopic fluid background and microscopic kinetic correction, both fully coupled to each other. A similar manner of discretization is proposed to that used in the recent \texttt{STRUPHY} code \cite{Holderied_Possanner_Wang_2021, Holderied_2022, Li_et_al_2023} with a finite-element model for the background and a pseudo-particle/PiC model for the correction.

        The fluid background satisfies the full, non-linear, resistive, compressible, Hall MHD equations. \cite{Laakmann_Hu_Farrell_2022} introduces finite-element(-in-space) implicit timesteppers for the incompressible analogue to this system with structure-preserving (SP) properties in the ideal case, alongside parameter-robust preconditioners. We show that these timesteppers can derive from a finite-element-in-time (FET) (and finite-element-in-space) interpretation. The benefits of this reformulation are discussed, including the derivation of timesteppers that are higher order in time, and the quantifiable dissipative SP properties in the non-ideal, resistive case.
        
        We discuss possible options for extending this FET approach to timesteppers for the compressible case.

        The kinetic corrections satisfy linearized Boltzmann equations. Using a Lénard--Bernstein collision operator, these take Fokker--Planck-like forms \cite{Fokker_1914, Planck_1917} wherein pseudo-particles in the numerical model obey the neoclassical transport equations, with particle-independent Brownian drift terms. This offers a rigorous methodology for incorporating collisions into the particle transport model, without coupling the equations of motions for each particle.
        
        Works by Chen, Chacón et al. \cite{Chen_Chacón_Barnes_2011, Chacón_Chen_Barnes_2013, Chen_Chacón_2014, Chen_Chacón_2015} have developed structure-preserving particle pushers for neoclassical transport in the Vlasov equations, derived from Crank--Nicolson integrators. We show these too can can derive from a FET interpretation, similarly offering potential extensions to higher-order-in-time particle pushers. The FET formulation is used also to consider how the stochastic drift terms can be incorporated into the pushers. Stochastic gyrokinetic expansions are also discussed.

        Different options for the numerical implementation of these schemes are considered.

        Due to the efficacy of FET in the development of SP timesteppers for both the fluid and kinetic component, we hope this approach will prove effective in the future for developing SP timesteppers for the full hybrid model. We hope this will give us the opportunity to incorporate previously inaccessible kinetic effects into the highly effective, modern, finite-element MHD models.
    \end{abstract}
    
    
    \newpage
    \tableofcontents
    
    
    \newpage
    \pagenumbering{arabic}
    %\linenumbers\renewcommand\thelinenumber{\color{black!50}\arabic{linenumber}}
            \input{0 - introduction/main.tex}
        \part{Research}
            \input{1 - low-noise PiC models/main.tex}
            \input{2 - kinetic component/main.tex}
            \input{3 - fluid component/main.tex}
            \input{4 - numerical implementation/main.tex}
        \part{Project Overview}
            \input{5 - research plan/main.tex}
            \input{6 - summary/main.tex}
    
    
    %\section{}
    \newpage
    \pagenumbering{gobble}
        \printbibliography


    \newpage
    \pagenumbering{roman}
    \appendix
        \part{Appendices}
            \input{8 - Hilbert complexes/main.tex}
            \input{9 - weak conservation proofs/main.tex}
\end{document}


\title{\BA{Title in Progress...}}
\author{Boris Andrews}
\affil{Mathematical Institute, University of Oxford}
\date{\today}


\begin{document}
    \pagenumbering{gobble}
    \maketitle
    
    
    \begin{abstract}
        Magnetic confinement reactors---in particular tokamaks---offer one of the most promising options for achieving practical nuclear fusion, with the potential to provide virtually limitless, clean energy. The theoretical and numerical modeling of tokamak plasmas is simultaneously an essential component of effective reactor design, and a great research barrier. Tokamak operational conditions exhibit comparatively low Knudsen numbers. Kinetic effects, including kinetic waves and instabilities, Landau damping, bump-on-tail instabilities and more, are therefore highly influential in tokamak plasma dynamics. Purely fluid models are inherently incapable of capturing these effects, whereas the high dimensionality in purely kinetic models render them practically intractable for most relevant purposes.

        We consider a $\delta\!f$ decomposition model, with a macroscopic fluid background and microscopic kinetic correction, both fully coupled to each other. A similar manner of discretization is proposed to that used in the recent \texttt{STRUPHY} code \cite{Holderied_Possanner_Wang_2021, Holderied_2022, Li_et_al_2023} with a finite-element model for the background and a pseudo-particle/PiC model for the correction.

        The fluid background satisfies the full, non-linear, resistive, compressible, Hall MHD equations. \cite{Laakmann_Hu_Farrell_2022} introduces finite-element(-in-space) implicit timesteppers for the incompressible analogue to this system with structure-preserving (SP) properties in the ideal case, alongside parameter-robust preconditioners. We show that these timesteppers can derive from a finite-element-in-time (FET) (and finite-element-in-space) interpretation. The benefits of this reformulation are discussed, including the derivation of timesteppers that are higher order in time, and the quantifiable dissipative SP properties in the non-ideal, resistive case.
        
        We discuss possible options for extending this FET approach to timesteppers for the compressible case.

        The kinetic corrections satisfy linearized Boltzmann equations. Using a Lénard--Bernstein collision operator, these take Fokker--Planck-like forms \cite{Fokker_1914, Planck_1917} wherein pseudo-particles in the numerical model obey the neoclassical transport equations, with particle-independent Brownian drift terms. This offers a rigorous methodology for incorporating collisions into the particle transport model, without coupling the equations of motions for each particle.
        
        Works by Chen, Chacón et al. \cite{Chen_Chacón_Barnes_2011, Chacón_Chen_Barnes_2013, Chen_Chacón_2014, Chen_Chacón_2015} have developed structure-preserving particle pushers for neoclassical transport in the Vlasov equations, derived from Crank--Nicolson integrators. We show these too can can derive from a FET interpretation, similarly offering potential extensions to higher-order-in-time particle pushers. The FET formulation is used also to consider how the stochastic drift terms can be incorporated into the pushers. Stochastic gyrokinetic expansions are also discussed.

        Different options for the numerical implementation of these schemes are considered.

        Due to the efficacy of FET in the development of SP timesteppers for both the fluid and kinetic component, we hope this approach will prove effective in the future for developing SP timesteppers for the full hybrid model. We hope this will give us the opportunity to incorporate previously inaccessible kinetic effects into the highly effective, modern, finite-element MHD models.
    \end{abstract}
    
    
    \newpage
    \tableofcontents
    
    
    \newpage
    \pagenumbering{arabic}
    %\linenumbers\renewcommand\thelinenumber{\color{black!50}\arabic{linenumber}}
            \documentclass[12pt, a4paper]{report}

\input{template/main.tex}

\title{\BA{Title in Progress...}}
\author{Boris Andrews}
\affil{Mathematical Institute, University of Oxford}
\date{\today}


\begin{document}
    \pagenumbering{gobble}
    \maketitle
    
    
    \begin{abstract}
        Magnetic confinement reactors---in particular tokamaks---offer one of the most promising options for achieving practical nuclear fusion, with the potential to provide virtually limitless, clean energy. The theoretical and numerical modeling of tokamak plasmas is simultaneously an essential component of effective reactor design, and a great research barrier. Tokamak operational conditions exhibit comparatively low Knudsen numbers. Kinetic effects, including kinetic waves and instabilities, Landau damping, bump-on-tail instabilities and more, are therefore highly influential in tokamak plasma dynamics. Purely fluid models are inherently incapable of capturing these effects, whereas the high dimensionality in purely kinetic models render them practically intractable for most relevant purposes.

        We consider a $\delta\!f$ decomposition model, with a macroscopic fluid background and microscopic kinetic correction, both fully coupled to each other. A similar manner of discretization is proposed to that used in the recent \texttt{STRUPHY} code \cite{Holderied_Possanner_Wang_2021, Holderied_2022, Li_et_al_2023} with a finite-element model for the background and a pseudo-particle/PiC model for the correction.

        The fluid background satisfies the full, non-linear, resistive, compressible, Hall MHD equations. \cite{Laakmann_Hu_Farrell_2022} introduces finite-element(-in-space) implicit timesteppers for the incompressible analogue to this system with structure-preserving (SP) properties in the ideal case, alongside parameter-robust preconditioners. We show that these timesteppers can derive from a finite-element-in-time (FET) (and finite-element-in-space) interpretation. The benefits of this reformulation are discussed, including the derivation of timesteppers that are higher order in time, and the quantifiable dissipative SP properties in the non-ideal, resistive case.
        
        We discuss possible options for extending this FET approach to timesteppers for the compressible case.

        The kinetic corrections satisfy linearized Boltzmann equations. Using a Lénard--Bernstein collision operator, these take Fokker--Planck-like forms \cite{Fokker_1914, Planck_1917} wherein pseudo-particles in the numerical model obey the neoclassical transport equations, with particle-independent Brownian drift terms. This offers a rigorous methodology for incorporating collisions into the particle transport model, without coupling the equations of motions for each particle.
        
        Works by Chen, Chacón et al. \cite{Chen_Chacón_Barnes_2011, Chacón_Chen_Barnes_2013, Chen_Chacón_2014, Chen_Chacón_2015} have developed structure-preserving particle pushers for neoclassical transport in the Vlasov equations, derived from Crank--Nicolson integrators. We show these too can can derive from a FET interpretation, similarly offering potential extensions to higher-order-in-time particle pushers. The FET formulation is used also to consider how the stochastic drift terms can be incorporated into the pushers. Stochastic gyrokinetic expansions are also discussed.

        Different options for the numerical implementation of these schemes are considered.

        Due to the efficacy of FET in the development of SP timesteppers for both the fluid and kinetic component, we hope this approach will prove effective in the future for developing SP timesteppers for the full hybrid model. We hope this will give us the opportunity to incorporate previously inaccessible kinetic effects into the highly effective, modern, finite-element MHD models.
    \end{abstract}
    
    
    \newpage
    \tableofcontents
    
    
    \newpage
    \pagenumbering{arabic}
    %\linenumbers\renewcommand\thelinenumber{\color{black!50}\arabic{linenumber}}
            \input{0 - introduction/main.tex}
        \part{Research}
            \input{1 - low-noise PiC models/main.tex}
            \input{2 - kinetic component/main.tex}
            \input{3 - fluid component/main.tex}
            \input{4 - numerical implementation/main.tex}
        \part{Project Overview}
            \input{5 - research plan/main.tex}
            \input{6 - summary/main.tex}
    
    
    %\section{}
    \newpage
    \pagenumbering{gobble}
        \printbibliography


    \newpage
    \pagenumbering{roman}
    \appendix
        \part{Appendices}
            \input{8 - Hilbert complexes/main.tex}
            \input{9 - weak conservation proofs/main.tex}
\end{document}

        \part{Research}
            \documentclass[12pt, a4paper]{report}

\input{template/main.tex}

\title{\BA{Title in Progress...}}
\author{Boris Andrews}
\affil{Mathematical Institute, University of Oxford}
\date{\today}


\begin{document}
    \pagenumbering{gobble}
    \maketitle
    
    
    \begin{abstract}
        Magnetic confinement reactors---in particular tokamaks---offer one of the most promising options for achieving practical nuclear fusion, with the potential to provide virtually limitless, clean energy. The theoretical and numerical modeling of tokamak plasmas is simultaneously an essential component of effective reactor design, and a great research barrier. Tokamak operational conditions exhibit comparatively low Knudsen numbers. Kinetic effects, including kinetic waves and instabilities, Landau damping, bump-on-tail instabilities and more, are therefore highly influential in tokamak plasma dynamics. Purely fluid models are inherently incapable of capturing these effects, whereas the high dimensionality in purely kinetic models render them practically intractable for most relevant purposes.

        We consider a $\delta\!f$ decomposition model, with a macroscopic fluid background and microscopic kinetic correction, both fully coupled to each other. A similar manner of discretization is proposed to that used in the recent \texttt{STRUPHY} code \cite{Holderied_Possanner_Wang_2021, Holderied_2022, Li_et_al_2023} with a finite-element model for the background and a pseudo-particle/PiC model for the correction.

        The fluid background satisfies the full, non-linear, resistive, compressible, Hall MHD equations. \cite{Laakmann_Hu_Farrell_2022} introduces finite-element(-in-space) implicit timesteppers for the incompressible analogue to this system with structure-preserving (SP) properties in the ideal case, alongside parameter-robust preconditioners. We show that these timesteppers can derive from a finite-element-in-time (FET) (and finite-element-in-space) interpretation. The benefits of this reformulation are discussed, including the derivation of timesteppers that are higher order in time, and the quantifiable dissipative SP properties in the non-ideal, resistive case.
        
        We discuss possible options for extending this FET approach to timesteppers for the compressible case.

        The kinetic corrections satisfy linearized Boltzmann equations. Using a Lénard--Bernstein collision operator, these take Fokker--Planck-like forms \cite{Fokker_1914, Planck_1917} wherein pseudo-particles in the numerical model obey the neoclassical transport equations, with particle-independent Brownian drift terms. This offers a rigorous methodology for incorporating collisions into the particle transport model, without coupling the equations of motions for each particle.
        
        Works by Chen, Chacón et al. \cite{Chen_Chacón_Barnes_2011, Chacón_Chen_Barnes_2013, Chen_Chacón_2014, Chen_Chacón_2015} have developed structure-preserving particle pushers for neoclassical transport in the Vlasov equations, derived from Crank--Nicolson integrators. We show these too can can derive from a FET interpretation, similarly offering potential extensions to higher-order-in-time particle pushers. The FET formulation is used also to consider how the stochastic drift terms can be incorporated into the pushers. Stochastic gyrokinetic expansions are also discussed.

        Different options for the numerical implementation of these schemes are considered.

        Due to the efficacy of FET in the development of SP timesteppers for both the fluid and kinetic component, we hope this approach will prove effective in the future for developing SP timesteppers for the full hybrid model. We hope this will give us the opportunity to incorporate previously inaccessible kinetic effects into the highly effective, modern, finite-element MHD models.
    \end{abstract}
    
    
    \newpage
    \tableofcontents
    
    
    \newpage
    \pagenumbering{arabic}
    %\linenumbers\renewcommand\thelinenumber{\color{black!50}\arabic{linenumber}}
            \input{0 - introduction/main.tex}
        \part{Research}
            \input{1 - low-noise PiC models/main.tex}
            \input{2 - kinetic component/main.tex}
            \input{3 - fluid component/main.tex}
            \input{4 - numerical implementation/main.tex}
        \part{Project Overview}
            \input{5 - research plan/main.tex}
            \input{6 - summary/main.tex}
    
    
    %\section{}
    \newpage
    \pagenumbering{gobble}
        \printbibliography


    \newpage
    \pagenumbering{roman}
    \appendix
        \part{Appendices}
            \input{8 - Hilbert complexes/main.tex}
            \input{9 - weak conservation proofs/main.tex}
\end{document}

            \documentclass[12pt, a4paper]{report}

\input{template/main.tex}

\title{\BA{Title in Progress...}}
\author{Boris Andrews}
\affil{Mathematical Institute, University of Oxford}
\date{\today}


\begin{document}
    \pagenumbering{gobble}
    \maketitle
    
    
    \begin{abstract}
        Magnetic confinement reactors---in particular tokamaks---offer one of the most promising options for achieving practical nuclear fusion, with the potential to provide virtually limitless, clean energy. The theoretical and numerical modeling of tokamak plasmas is simultaneously an essential component of effective reactor design, and a great research barrier. Tokamak operational conditions exhibit comparatively low Knudsen numbers. Kinetic effects, including kinetic waves and instabilities, Landau damping, bump-on-tail instabilities and more, are therefore highly influential in tokamak plasma dynamics. Purely fluid models are inherently incapable of capturing these effects, whereas the high dimensionality in purely kinetic models render them practically intractable for most relevant purposes.

        We consider a $\delta\!f$ decomposition model, with a macroscopic fluid background and microscopic kinetic correction, both fully coupled to each other. A similar manner of discretization is proposed to that used in the recent \texttt{STRUPHY} code \cite{Holderied_Possanner_Wang_2021, Holderied_2022, Li_et_al_2023} with a finite-element model for the background and a pseudo-particle/PiC model for the correction.

        The fluid background satisfies the full, non-linear, resistive, compressible, Hall MHD equations. \cite{Laakmann_Hu_Farrell_2022} introduces finite-element(-in-space) implicit timesteppers for the incompressible analogue to this system with structure-preserving (SP) properties in the ideal case, alongside parameter-robust preconditioners. We show that these timesteppers can derive from a finite-element-in-time (FET) (and finite-element-in-space) interpretation. The benefits of this reformulation are discussed, including the derivation of timesteppers that are higher order in time, and the quantifiable dissipative SP properties in the non-ideal, resistive case.
        
        We discuss possible options for extending this FET approach to timesteppers for the compressible case.

        The kinetic corrections satisfy linearized Boltzmann equations. Using a Lénard--Bernstein collision operator, these take Fokker--Planck-like forms \cite{Fokker_1914, Planck_1917} wherein pseudo-particles in the numerical model obey the neoclassical transport equations, with particle-independent Brownian drift terms. This offers a rigorous methodology for incorporating collisions into the particle transport model, without coupling the equations of motions for each particle.
        
        Works by Chen, Chacón et al. \cite{Chen_Chacón_Barnes_2011, Chacón_Chen_Barnes_2013, Chen_Chacón_2014, Chen_Chacón_2015} have developed structure-preserving particle pushers for neoclassical transport in the Vlasov equations, derived from Crank--Nicolson integrators. We show these too can can derive from a FET interpretation, similarly offering potential extensions to higher-order-in-time particle pushers. The FET formulation is used also to consider how the stochastic drift terms can be incorporated into the pushers. Stochastic gyrokinetic expansions are also discussed.

        Different options for the numerical implementation of these schemes are considered.

        Due to the efficacy of FET in the development of SP timesteppers for both the fluid and kinetic component, we hope this approach will prove effective in the future for developing SP timesteppers for the full hybrid model. We hope this will give us the opportunity to incorporate previously inaccessible kinetic effects into the highly effective, modern, finite-element MHD models.
    \end{abstract}
    
    
    \newpage
    \tableofcontents
    
    
    \newpage
    \pagenumbering{arabic}
    %\linenumbers\renewcommand\thelinenumber{\color{black!50}\arabic{linenumber}}
            \input{0 - introduction/main.tex}
        \part{Research}
            \input{1 - low-noise PiC models/main.tex}
            \input{2 - kinetic component/main.tex}
            \input{3 - fluid component/main.tex}
            \input{4 - numerical implementation/main.tex}
        \part{Project Overview}
            \input{5 - research plan/main.tex}
            \input{6 - summary/main.tex}
    
    
    %\section{}
    \newpage
    \pagenumbering{gobble}
        \printbibliography


    \newpage
    \pagenumbering{roman}
    \appendix
        \part{Appendices}
            \input{8 - Hilbert complexes/main.tex}
            \input{9 - weak conservation proofs/main.tex}
\end{document}

            \documentclass[12pt, a4paper]{report}

\input{template/main.tex}

\title{\BA{Title in Progress...}}
\author{Boris Andrews}
\affil{Mathematical Institute, University of Oxford}
\date{\today}


\begin{document}
    \pagenumbering{gobble}
    \maketitle
    
    
    \begin{abstract}
        Magnetic confinement reactors---in particular tokamaks---offer one of the most promising options for achieving practical nuclear fusion, with the potential to provide virtually limitless, clean energy. The theoretical and numerical modeling of tokamak plasmas is simultaneously an essential component of effective reactor design, and a great research barrier. Tokamak operational conditions exhibit comparatively low Knudsen numbers. Kinetic effects, including kinetic waves and instabilities, Landau damping, bump-on-tail instabilities and more, are therefore highly influential in tokamak plasma dynamics. Purely fluid models are inherently incapable of capturing these effects, whereas the high dimensionality in purely kinetic models render them practically intractable for most relevant purposes.

        We consider a $\delta\!f$ decomposition model, with a macroscopic fluid background and microscopic kinetic correction, both fully coupled to each other. A similar manner of discretization is proposed to that used in the recent \texttt{STRUPHY} code \cite{Holderied_Possanner_Wang_2021, Holderied_2022, Li_et_al_2023} with a finite-element model for the background and a pseudo-particle/PiC model for the correction.

        The fluid background satisfies the full, non-linear, resistive, compressible, Hall MHD equations. \cite{Laakmann_Hu_Farrell_2022} introduces finite-element(-in-space) implicit timesteppers for the incompressible analogue to this system with structure-preserving (SP) properties in the ideal case, alongside parameter-robust preconditioners. We show that these timesteppers can derive from a finite-element-in-time (FET) (and finite-element-in-space) interpretation. The benefits of this reformulation are discussed, including the derivation of timesteppers that are higher order in time, and the quantifiable dissipative SP properties in the non-ideal, resistive case.
        
        We discuss possible options for extending this FET approach to timesteppers for the compressible case.

        The kinetic corrections satisfy linearized Boltzmann equations. Using a Lénard--Bernstein collision operator, these take Fokker--Planck-like forms \cite{Fokker_1914, Planck_1917} wherein pseudo-particles in the numerical model obey the neoclassical transport equations, with particle-independent Brownian drift terms. This offers a rigorous methodology for incorporating collisions into the particle transport model, without coupling the equations of motions for each particle.
        
        Works by Chen, Chacón et al. \cite{Chen_Chacón_Barnes_2011, Chacón_Chen_Barnes_2013, Chen_Chacón_2014, Chen_Chacón_2015} have developed structure-preserving particle pushers for neoclassical transport in the Vlasov equations, derived from Crank--Nicolson integrators. We show these too can can derive from a FET interpretation, similarly offering potential extensions to higher-order-in-time particle pushers. The FET formulation is used also to consider how the stochastic drift terms can be incorporated into the pushers. Stochastic gyrokinetic expansions are also discussed.

        Different options for the numerical implementation of these schemes are considered.

        Due to the efficacy of FET in the development of SP timesteppers for both the fluid and kinetic component, we hope this approach will prove effective in the future for developing SP timesteppers for the full hybrid model. We hope this will give us the opportunity to incorporate previously inaccessible kinetic effects into the highly effective, modern, finite-element MHD models.
    \end{abstract}
    
    
    \newpage
    \tableofcontents
    
    
    \newpage
    \pagenumbering{arabic}
    %\linenumbers\renewcommand\thelinenumber{\color{black!50}\arabic{linenumber}}
            \input{0 - introduction/main.tex}
        \part{Research}
            \input{1 - low-noise PiC models/main.tex}
            \input{2 - kinetic component/main.tex}
            \input{3 - fluid component/main.tex}
            \input{4 - numerical implementation/main.tex}
        \part{Project Overview}
            \input{5 - research plan/main.tex}
            \input{6 - summary/main.tex}
    
    
    %\section{}
    \newpage
    \pagenumbering{gobble}
        \printbibliography


    \newpage
    \pagenumbering{roman}
    \appendix
        \part{Appendices}
            \input{8 - Hilbert complexes/main.tex}
            \input{9 - weak conservation proofs/main.tex}
\end{document}

            \documentclass[12pt, a4paper]{report}

\input{template/main.tex}

\title{\BA{Title in Progress...}}
\author{Boris Andrews}
\affil{Mathematical Institute, University of Oxford}
\date{\today}


\begin{document}
    \pagenumbering{gobble}
    \maketitle
    
    
    \begin{abstract}
        Magnetic confinement reactors---in particular tokamaks---offer one of the most promising options for achieving practical nuclear fusion, with the potential to provide virtually limitless, clean energy. The theoretical and numerical modeling of tokamak plasmas is simultaneously an essential component of effective reactor design, and a great research barrier. Tokamak operational conditions exhibit comparatively low Knudsen numbers. Kinetic effects, including kinetic waves and instabilities, Landau damping, bump-on-tail instabilities and more, are therefore highly influential in tokamak plasma dynamics. Purely fluid models are inherently incapable of capturing these effects, whereas the high dimensionality in purely kinetic models render them practically intractable for most relevant purposes.

        We consider a $\delta\!f$ decomposition model, with a macroscopic fluid background and microscopic kinetic correction, both fully coupled to each other. A similar manner of discretization is proposed to that used in the recent \texttt{STRUPHY} code \cite{Holderied_Possanner_Wang_2021, Holderied_2022, Li_et_al_2023} with a finite-element model for the background and a pseudo-particle/PiC model for the correction.

        The fluid background satisfies the full, non-linear, resistive, compressible, Hall MHD equations. \cite{Laakmann_Hu_Farrell_2022} introduces finite-element(-in-space) implicit timesteppers for the incompressible analogue to this system with structure-preserving (SP) properties in the ideal case, alongside parameter-robust preconditioners. We show that these timesteppers can derive from a finite-element-in-time (FET) (and finite-element-in-space) interpretation. The benefits of this reformulation are discussed, including the derivation of timesteppers that are higher order in time, and the quantifiable dissipative SP properties in the non-ideal, resistive case.
        
        We discuss possible options for extending this FET approach to timesteppers for the compressible case.

        The kinetic corrections satisfy linearized Boltzmann equations. Using a Lénard--Bernstein collision operator, these take Fokker--Planck-like forms \cite{Fokker_1914, Planck_1917} wherein pseudo-particles in the numerical model obey the neoclassical transport equations, with particle-independent Brownian drift terms. This offers a rigorous methodology for incorporating collisions into the particle transport model, without coupling the equations of motions for each particle.
        
        Works by Chen, Chacón et al. \cite{Chen_Chacón_Barnes_2011, Chacón_Chen_Barnes_2013, Chen_Chacón_2014, Chen_Chacón_2015} have developed structure-preserving particle pushers for neoclassical transport in the Vlasov equations, derived from Crank--Nicolson integrators. We show these too can can derive from a FET interpretation, similarly offering potential extensions to higher-order-in-time particle pushers. The FET formulation is used also to consider how the stochastic drift terms can be incorporated into the pushers. Stochastic gyrokinetic expansions are also discussed.

        Different options for the numerical implementation of these schemes are considered.

        Due to the efficacy of FET in the development of SP timesteppers for both the fluid and kinetic component, we hope this approach will prove effective in the future for developing SP timesteppers for the full hybrid model. We hope this will give us the opportunity to incorporate previously inaccessible kinetic effects into the highly effective, modern, finite-element MHD models.
    \end{abstract}
    
    
    \newpage
    \tableofcontents
    
    
    \newpage
    \pagenumbering{arabic}
    %\linenumbers\renewcommand\thelinenumber{\color{black!50}\arabic{linenumber}}
            \input{0 - introduction/main.tex}
        \part{Research}
            \input{1 - low-noise PiC models/main.tex}
            \input{2 - kinetic component/main.tex}
            \input{3 - fluid component/main.tex}
            \input{4 - numerical implementation/main.tex}
        \part{Project Overview}
            \input{5 - research plan/main.tex}
            \input{6 - summary/main.tex}
    
    
    %\section{}
    \newpage
    \pagenumbering{gobble}
        \printbibliography


    \newpage
    \pagenumbering{roman}
    \appendix
        \part{Appendices}
            \input{8 - Hilbert complexes/main.tex}
            \input{9 - weak conservation proofs/main.tex}
\end{document}

        \part{Project Overview}
            \documentclass[12pt, a4paper]{report}

\input{template/main.tex}

\title{\BA{Title in Progress...}}
\author{Boris Andrews}
\affil{Mathematical Institute, University of Oxford}
\date{\today}


\begin{document}
    \pagenumbering{gobble}
    \maketitle
    
    
    \begin{abstract}
        Magnetic confinement reactors---in particular tokamaks---offer one of the most promising options for achieving practical nuclear fusion, with the potential to provide virtually limitless, clean energy. The theoretical and numerical modeling of tokamak plasmas is simultaneously an essential component of effective reactor design, and a great research barrier. Tokamak operational conditions exhibit comparatively low Knudsen numbers. Kinetic effects, including kinetic waves and instabilities, Landau damping, bump-on-tail instabilities and more, are therefore highly influential in tokamak plasma dynamics. Purely fluid models are inherently incapable of capturing these effects, whereas the high dimensionality in purely kinetic models render them practically intractable for most relevant purposes.

        We consider a $\delta\!f$ decomposition model, with a macroscopic fluid background and microscopic kinetic correction, both fully coupled to each other. A similar manner of discretization is proposed to that used in the recent \texttt{STRUPHY} code \cite{Holderied_Possanner_Wang_2021, Holderied_2022, Li_et_al_2023} with a finite-element model for the background and a pseudo-particle/PiC model for the correction.

        The fluid background satisfies the full, non-linear, resistive, compressible, Hall MHD equations. \cite{Laakmann_Hu_Farrell_2022} introduces finite-element(-in-space) implicit timesteppers for the incompressible analogue to this system with structure-preserving (SP) properties in the ideal case, alongside parameter-robust preconditioners. We show that these timesteppers can derive from a finite-element-in-time (FET) (and finite-element-in-space) interpretation. The benefits of this reformulation are discussed, including the derivation of timesteppers that are higher order in time, and the quantifiable dissipative SP properties in the non-ideal, resistive case.
        
        We discuss possible options for extending this FET approach to timesteppers for the compressible case.

        The kinetic corrections satisfy linearized Boltzmann equations. Using a Lénard--Bernstein collision operator, these take Fokker--Planck-like forms \cite{Fokker_1914, Planck_1917} wherein pseudo-particles in the numerical model obey the neoclassical transport equations, with particle-independent Brownian drift terms. This offers a rigorous methodology for incorporating collisions into the particle transport model, without coupling the equations of motions for each particle.
        
        Works by Chen, Chacón et al. \cite{Chen_Chacón_Barnes_2011, Chacón_Chen_Barnes_2013, Chen_Chacón_2014, Chen_Chacón_2015} have developed structure-preserving particle pushers for neoclassical transport in the Vlasov equations, derived from Crank--Nicolson integrators. We show these too can can derive from a FET interpretation, similarly offering potential extensions to higher-order-in-time particle pushers. The FET formulation is used also to consider how the stochastic drift terms can be incorporated into the pushers. Stochastic gyrokinetic expansions are also discussed.

        Different options for the numerical implementation of these schemes are considered.

        Due to the efficacy of FET in the development of SP timesteppers for both the fluid and kinetic component, we hope this approach will prove effective in the future for developing SP timesteppers for the full hybrid model. We hope this will give us the opportunity to incorporate previously inaccessible kinetic effects into the highly effective, modern, finite-element MHD models.
    \end{abstract}
    
    
    \newpage
    \tableofcontents
    
    
    \newpage
    \pagenumbering{arabic}
    %\linenumbers\renewcommand\thelinenumber{\color{black!50}\arabic{linenumber}}
            \input{0 - introduction/main.tex}
        \part{Research}
            \input{1 - low-noise PiC models/main.tex}
            \input{2 - kinetic component/main.tex}
            \input{3 - fluid component/main.tex}
            \input{4 - numerical implementation/main.tex}
        \part{Project Overview}
            \input{5 - research plan/main.tex}
            \input{6 - summary/main.tex}
    
    
    %\section{}
    \newpage
    \pagenumbering{gobble}
        \printbibliography


    \newpage
    \pagenumbering{roman}
    \appendix
        \part{Appendices}
            \input{8 - Hilbert complexes/main.tex}
            \input{9 - weak conservation proofs/main.tex}
\end{document}

            \documentclass[12pt, a4paper]{report}

\input{template/main.tex}

\title{\BA{Title in Progress...}}
\author{Boris Andrews}
\affil{Mathematical Institute, University of Oxford}
\date{\today}


\begin{document}
    \pagenumbering{gobble}
    \maketitle
    
    
    \begin{abstract}
        Magnetic confinement reactors---in particular tokamaks---offer one of the most promising options for achieving practical nuclear fusion, with the potential to provide virtually limitless, clean energy. The theoretical and numerical modeling of tokamak plasmas is simultaneously an essential component of effective reactor design, and a great research barrier. Tokamak operational conditions exhibit comparatively low Knudsen numbers. Kinetic effects, including kinetic waves and instabilities, Landau damping, bump-on-tail instabilities and more, are therefore highly influential in tokamak plasma dynamics. Purely fluid models are inherently incapable of capturing these effects, whereas the high dimensionality in purely kinetic models render them practically intractable for most relevant purposes.

        We consider a $\delta\!f$ decomposition model, with a macroscopic fluid background and microscopic kinetic correction, both fully coupled to each other. A similar manner of discretization is proposed to that used in the recent \texttt{STRUPHY} code \cite{Holderied_Possanner_Wang_2021, Holderied_2022, Li_et_al_2023} with a finite-element model for the background and a pseudo-particle/PiC model for the correction.

        The fluid background satisfies the full, non-linear, resistive, compressible, Hall MHD equations. \cite{Laakmann_Hu_Farrell_2022} introduces finite-element(-in-space) implicit timesteppers for the incompressible analogue to this system with structure-preserving (SP) properties in the ideal case, alongside parameter-robust preconditioners. We show that these timesteppers can derive from a finite-element-in-time (FET) (and finite-element-in-space) interpretation. The benefits of this reformulation are discussed, including the derivation of timesteppers that are higher order in time, and the quantifiable dissipative SP properties in the non-ideal, resistive case.
        
        We discuss possible options for extending this FET approach to timesteppers for the compressible case.

        The kinetic corrections satisfy linearized Boltzmann equations. Using a Lénard--Bernstein collision operator, these take Fokker--Planck-like forms \cite{Fokker_1914, Planck_1917} wherein pseudo-particles in the numerical model obey the neoclassical transport equations, with particle-independent Brownian drift terms. This offers a rigorous methodology for incorporating collisions into the particle transport model, without coupling the equations of motions for each particle.
        
        Works by Chen, Chacón et al. \cite{Chen_Chacón_Barnes_2011, Chacón_Chen_Barnes_2013, Chen_Chacón_2014, Chen_Chacón_2015} have developed structure-preserving particle pushers for neoclassical transport in the Vlasov equations, derived from Crank--Nicolson integrators. We show these too can can derive from a FET interpretation, similarly offering potential extensions to higher-order-in-time particle pushers. The FET formulation is used also to consider how the stochastic drift terms can be incorporated into the pushers. Stochastic gyrokinetic expansions are also discussed.

        Different options for the numerical implementation of these schemes are considered.

        Due to the efficacy of FET in the development of SP timesteppers for both the fluid and kinetic component, we hope this approach will prove effective in the future for developing SP timesteppers for the full hybrid model. We hope this will give us the opportunity to incorporate previously inaccessible kinetic effects into the highly effective, modern, finite-element MHD models.
    \end{abstract}
    
    
    \newpage
    \tableofcontents
    
    
    \newpage
    \pagenumbering{arabic}
    %\linenumbers\renewcommand\thelinenumber{\color{black!50}\arabic{linenumber}}
            \input{0 - introduction/main.tex}
        \part{Research}
            \input{1 - low-noise PiC models/main.tex}
            \input{2 - kinetic component/main.tex}
            \input{3 - fluid component/main.tex}
            \input{4 - numerical implementation/main.tex}
        \part{Project Overview}
            \input{5 - research plan/main.tex}
            \input{6 - summary/main.tex}
    
    
    %\section{}
    \newpage
    \pagenumbering{gobble}
        \printbibliography


    \newpage
    \pagenumbering{roman}
    \appendix
        \part{Appendices}
            \input{8 - Hilbert complexes/main.tex}
            \input{9 - weak conservation proofs/main.tex}
\end{document}

    
    
    %\section{}
    \newpage
    \pagenumbering{gobble}
        \printbibliography


    \newpage
    \pagenumbering{roman}
    \appendix
        \part{Appendices}
            \documentclass[12pt, a4paper]{report}

\input{template/main.tex}

\title{\BA{Title in Progress...}}
\author{Boris Andrews}
\affil{Mathematical Institute, University of Oxford}
\date{\today}


\begin{document}
    \pagenumbering{gobble}
    \maketitle
    
    
    \begin{abstract}
        Magnetic confinement reactors---in particular tokamaks---offer one of the most promising options for achieving practical nuclear fusion, with the potential to provide virtually limitless, clean energy. The theoretical and numerical modeling of tokamak plasmas is simultaneously an essential component of effective reactor design, and a great research barrier. Tokamak operational conditions exhibit comparatively low Knudsen numbers. Kinetic effects, including kinetic waves and instabilities, Landau damping, bump-on-tail instabilities and more, are therefore highly influential in tokamak plasma dynamics. Purely fluid models are inherently incapable of capturing these effects, whereas the high dimensionality in purely kinetic models render them practically intractable for most relevant purposes.

        We consider a $\delta\!f$ decomposition model, with a macroscopic fluid background and microscopic kinetic correction, both fully coupled to each other. A similar manner of discretization is proposed to that used in the recent \texttt{STRUPHY} code \cite{Holderied_Possanner_Wang_2021, Holderied_2022, Li_et_al_2023} with a finite-element model for the background and a pseudo-particle/PiC model for the correction.

        The fluid background satisfies the full, non-linear, resistive, compressible, Hall MHD equations. \cite{Laakmann_Hu_Farrell_2022} introduces finite-element(-in-space) implicit timesteppers for the incompressible analogue to this system with structure-preserving (SP) properties in the ideal case, alongside parameter-robust preconditioners. We show that these timesteppers can derive from a finite-element-in-time (FET) (and finite-element-in-space) interpretation. The benefits of this reformulation are discussed, including the derivation of timesteppers that are higher order in time, and the quantifiable dissipative SP properties in the non-ideal, resistive case.
        
        We discuss possible options for extending this FET approach to timesteppers for the compressible case.

        The kinetic corrections satisfy linearized Boltzmann equations. Using a Lénard--Bernstein collision operator, these take Fokker--Planck-like forms \cite{Fokker_1914, Planck_1917} wherein pseudo-particles in the numerical model obey the neoclassical transport equations, with particle-independent Brownian drift terms. This offers a rigorous methodology for incorporating collisions into the particle transport model, without coupling the equations of motions for each particle.
        
        Works by Chen, Chacón et al. \cite{Chen_Chacón_Barnes_2011, Chacón_Chen_Barnes_2013, Chen_Chacón_2014, Chen_Chacón_2015} have developed structure-preserving particle pushers for neoclassical transport in the Vlasov equations, derived from Crank--Nicolson integrators. We show these too can can derive from a FET interpretation, similarly offering potential extensions to higher-order-in-time particle pushers. The FET formulation is used also to consider how the stochastic drift terms can be incorporated into the pushers. Stochastic gyrokinetic expansions are also discussed.

        Different options for the numerical implementation of these schemes are considered.

        Due to the efficacy of FET in the development of SP timesteppers for both the fluid and kinetic component, we hope this approach will prove effective in the future for developing SP timesteppers for the full hybrid model. We hope this will give us the opportunity to incorporate previously inaccessible kinetic effects into the highly effective, modern, finite-element MHD models.
    \end{abstract}
    
    
    \newpage
    \tableofcontents
    
    
    \newpage
    \pagenumbering{arabic}
    %\linenumbers\renewcommand\thelinenumber{\color{black!50}\arabic{linenumber}}
            \input{0 - introduction/main.tex}
        \part{Research}
            \input{1 - low-noise PiC models/main.tex}
            \input{2 - kinetic component/main.tex}
            \input{3 - fluid component/main.tex}
            \input{4 - numerical implementation/main.tex}
        \part{Project Overview}
            \input{5 - research plan/main.tex}
            \input{6 - summary/main.tex}
    
    
    %\section{}
    \newpage
    \pagenumbering{gobble}
        \printbibliography


    \newpage
    \pagenumbering{roman}
    \appendix
        \part{Appendices}
            \input{8 - Hilbert complexes/main.tex}
            \input{9 - weak conservation proofs/main.tex}
\end{document}

            \documentclass[12pt, a4paper]{report}

\input{template/main.tex}

\title{\BA{Title in Progress...}}
\author{Boris Andrews}
\affil{Mathematical Institute, University of Oxford}
\date{\today}


\begin{document}
    \pagenumbering{gobble}
    \maketitle
    
    
    \begin{abstract}
        Magnetic confinement reactors---in particular tokamaks---offer one of the most promising options for achieving practical nuclear fusion, with the potential to provide virtually limitless, clean energy. The theoretical and numerical modeling of tokamak plasmas is simultaneously an essential component of effective reactor design, and a great research barrier. Tokamak operational conditions exhibit comparatively low Knudsen numbers. Kinetic effects, including kinetic waves and instabilities, Landau damping, bump-on-tail instabilities and more, are therefore highly influential in tokamak plasma dynamics. Purely fluid models are inherently incapable of capturing these effects, whereas the high dimensionality in purely kinetic models render them practically intractable for most relevant purposes.

        We consider a $\delta\!f$ decomposition model, with a macroscopic fluid background and microscopic kinetic correction, both fully coupled to each other. A similar manner of discretization is proposed to that used in the recent \texttt{STRUPHY} code \cite{Holderied_Possanner_Wang_2021, Holderied_2022, Li_et_al_2023} with a finite-element model for the background and a pseudo-particle/PiC model for the correction.

        The fluid background satisfies the full, non-linear, resistive, compressible, Hall MHD equations. \cite{Laakmann_Hu_Farrell_2022} introduces finite-element(-in-space) implicit timesteppers for the incompressible analogue to this system with structure-preserving (SP) properties in the ideal case, alongside parameter-robust preconditioners. We show that these timesteppers can derive from a finite-element-in-time (FET) (and finite-element-in-space) interpretation. The benefits of this reformulation are discussed, including the derivation of timesteppers that are higher order in time, and the quantifiable dissipative SP properties in the non-ideal, resistive case.
        
        We discuss possible options for extending this FET approach to timesteppers for the compressible case.

        The kinetic corrections satisfy linearized Boltzmann equations. Using a Lénard--Bernstein collision operator, these take Fokker--Planck-like forms \cite{Fokker_1914, Planck_1917} wherein pseudo-particles in the numerical model obey the neoclassical transport equations, with particle-independent Brownian drift terms. This offers a rigorous methodology for incorporating collisions into the particle transport model, without coupling the equations of motions for each particle.
        
        Works by Chen, Chacón et al. \cite{Chen_Chacón_Barnes_2011, Chacón_Chen_Barnes_2013, Chen_Chacón_2014, Chen_Chacón_2015} have developed structure-preserving particle pushers for neoclassical transport in the Vlasov equations, derived from Crank--Nicolson integrators. We show these too can can derive from a FET interpretation, similarly offering potential extensions to higher-order-in-time particle pushers. The FET formulation is used also to consider how the stochastic drift terms can be incorporated into the pushers. Stochastic gyrokinetic expansions are also discussed.

        Different options for the numerical implementation of these schemes are considered.

        Due to the efficacy of FET in the development of SP timesteppers for both the fluid and kinetic component, we hope this approach will prove effective in the future for developing SP timesteppers for the full hybrid model. We hope this will give us the opportunity to incorporate previously inaccessible kinetic effects into the highly effective, modern, finite-element MHD models.
    \end{abstract}
    
    
    \newpage
    \tableofcontents
    
    
    \newpage
    \pagenumbering{arabic}
    %\linenumbers\renewcommand\thelinenumber{\color{black!50}\arabic{linenumber}}
            \input{0 - introduction/main.tex}
        \part{Research}
            \input{1 - low-noise PiC models/main.tex}
            \input{2 - kinetic component/main.tex}
            \input{3 - fluid component/main.tex}
            \input{4 - numerical implementation/main.tex}
        \part{Project Overview}
            \input{5 - research plan/main.tex}
            \input{6 - summary/main.tex}
    
    
    %\section{}
    \newpage
    \pagenumbering{gobble}
        \printbibliography


    \newpage
    \pagenumbering{roman}
    \appendix
        \part{Appendices}
            \input{8 - Hilbert complexes/main.tex}
            \input{9 - weak conservation proofs/main.tex}
\end{document}

\end{document}

            \documentclass[12pt, a4paper]{report}

\documentclass[12pt, a4paper]{report}

\input{template/main.tex}

\title{\BA{Title in Progress...}}
\author{Boris Andrews}
\affil{Mathematical Institute, University of Oxford}
\date{\today}


\begin{document}
    \pagenumbering{gobble}
    \maketitle
    
    
    \begin{abstract}
        Magnetic confinement reactors---in particular tokamaks---offer one of the most promising options for achieving practical nuclear fusion, with the potential to provide virtually limitless, clean energy. The theoretical and numerical modeling of tokamak plasmas is simultaneously an essential component of effective reactor design, and a great research barrier. Tokamak operational conditions exhibit comparatively low Knudsen numbers. Kinetic effects, including kinetic waves and instabilities, Landau damping, bump-on-tail instabilities and more, are therefore highly influential in tokamak plasma dynamics. Purely fluid models are inherently incapable of capturing these effects, whereas the high dimensionality in purely kinetic models render them practically intractable for most relevant purposes.

        We consider a $\delta\!f$ decomposition model, with a macroscopic fluid background and microscopic kinetic correction, both fully coupled to each other. A similar manner of discretization is proposed to that used in the recent \texttt{STRUPHY} code \cite{Holderied_Possanner_Wang_2021, Holderied_2022, Li_et_al_2023} with a finite-element model for the background and a pseudo-particle/PiC model for the correction.

        The fluid background satisfies the full, non-linear, resistive, compressible, Hall MHD equations. \cite{Laakmann_Hu_Farrell_2022} introduces finite-element(-in-space) implicit timesteppers for the incompressible analogue to this system with structure-preserving (SP) properties in the ideal case, alongside parameter-robust preconditioners. We show that these timesteppers can derive from a finite-element-in-time (FET) (and finite-element-in-space) interpretation. The benefits of this reformulation are discussed, including the derivation of timesteppers that are higher order in time, and the quantifiable dissipative SP properties in the non-ideal, resistive case.
        
        We discuss possible options for extending this FET approach to timesteppers for the compressible case.

        The kinetic corrections satisfy linearized Boltzmann equations. Using a Lénard--Bernstein collision operator, these take Fokker--Planck-like forms \cite{Fokker_1914, Planck_1917} wherein pseudo-particles in the numerical model obey the neoclassical transport equations, with particle-independent Brownian drift terms. This offers a rigorous methodology for incorporating collisions into the particle transport model, without coupling the equations of motions for each particle.
        
        Works by Chen, Chacón et al. \cite{Chen_Chacón_Barnes_2011, Chacón_Chen_Barnes_2013, Chen_Chacón_2014, Chen_Chacón_2015} have developed structure-preserving particle pushers for neoclassical transport in the Vlasov equations, derived from Crank--Nicolson integrators. We show these too can can derive from a FET interpretation, similarly offering potential extensions to higher-order-in-time particle pushers. The FET formulation is used also to consider how the stochastic drift terms can be incorporated into the pushers. Stochastic gyrokinetic expansions are also discussed.

        Different options for the numerical implementation of these schemes are considered.

        Due to the efficacy of FET in the development of SP timesteppers for both the fluid and kinetic component, we hope this approach will prove effective in the future for developing SP timesteppers for the full hybrid model. We hope this will give us the opportunity to incorporate previously inaccessible kinetic effects into the highly effective, modern, finite-element MHD models.
    \end{abstract}
    
    
    \newpage
    \tableofcontents
    
    
    \newpage
    \pagenumbering{arabic}
    %\linenumbers\renewcommand\thelinenumber{\color{black!50}\arabic{linenumber}}
            \input{0 - introduction/main.tex}
        \part{Research}
            \input{1 - low-noise PiC models/main.tex}
            \input{2 - kinetic component/main.tex}
            \input{3 - fluid component/main.tex}
            \input{4 - numerical implementation/main.tex}
        \part{Project Overview}
            \input{5 - research plan/main.tex}
            \input{6 - summary/main.tex}
    
    
    %\section{}
    \newpage
    \pagenumbering{gobble}
        \printbibliography


    \newpage
    \pagenumbering{roman}
    \appendix
        \part{Appendices}
            \input{8 - Hilbert complexes/main.tex}
            \input{9 - weak conservation proofs/main.tex}
\end{document}


\title{\BA{Title in Progress...}}
\author{Boris Andrews}
\affil{Mathematical Institute, University of Oxford}
\date{\today}


\begin{document}
    \pagenumbering{gobble}
    \maketitle
    
    
    \begin{abstract}
        Magnetic confinement reactors---in particular tokamaks---offer one of the most promising options for achieving practical nuclear fusion, with the potential to provide virtually limitless, clean energy. The theoretical and numerical modeling of tokamak plasmas is simultaneously an essential component of effective reactor design, and a great research barrier. Tokamak operational conditions exhibit comparatively low Knudsen numbers. Kinetic effects, including kinetic waves and instabilities, Landau damping, bump-on-tail instabilities and more, are therefore highly influential in tokamak plasma dynamics. Purely fluid models are inherently incapable of capturing these effects, whereas the high dimensionality in purely kinetic models render them practically intractable for most relevant purposes.

        We consider a $\delta\!f$ decomposition model, with a macroscopic fluid background and microscopic kinetic correction, both fully coupled to each other. A similar manner of discretization is proposed to that used in the recent \texttt{STRUPHY} code \cite{Holderied_Possanner_Wang_2021, Holderied_2022, Li_et_al_2023} with a finite-element model for the background and a pseudo-particle/PiC model for the correction.

        The fluid background satisfies the full, non-linear, resistive, compressible, Hall MHD equations. \cite{Laakmann_Hu_Farrell_2022} introduces finite-element(-in-space) implicit timesteppers for the incompressible analogue to this system with structure-preserving (SP) properties in the ideal case, alongside parameter-robust preconditioners. We show that these timesteppers can derive from a finite-element-in-time (FET) (and finite-element-in-space) interpretation. The benefits of this reformulation are discussed, including the derivation of timesteppers that are higher order in time, and the quantifiable dissipative SP properties in the non-ideal, resistive case.
        
        We discuss possible options for extending this FET approach to timesteppers for the compressible case.

        The kinetic corrections satisfy linearized Boltzmann equations. Using a Lénard--Bernstein collision operator, these take Fokker--Planck-like forms \cite{Fokker_1914, Planck_1917} wherein pseudo-particles in the numerical model obey the neoclassical transport equations, with particle-independent Brownian drift terms. This offers a rigorous methodology for incorporating collisions into the particle transport model, without coupling the equations of motions for each particle.
        
        Works by Chen, Chacón et al. \cite{Chen_Chacón_Barnes_2011, Chacón_Chen_Barnes_2013, Chen_Chacón_2014, Chen_Chacón_2015} have developed structure-preserving particle pushers for neoclassical transport in the Vlasov equations, derived from Crank--Nicolson integrators. We show these too can can derive from a FET interpretation, similarly offering potential extensions to higher-order-in-time particle pushers. The FET formulation is used also to consider how the stochastic drift terms can be incorporated into the pushers. Stochastic gyrokinetic expansions are also discussed.

        Different options for the numerical implementation of these schemes are considered.

        Due to the efficacy of FET in the development of SP timesteppers for both the fluid and kinetic component, we hope this approach will prove effective in the future for developing SP timesteppers for the full hybrid model. We hope this will give us the opportunity to incorporate previously inaccessible kinetic effects into the highly effective, modern, finite-element MHD models.
    \end{abstract}
    
    
    \newpage
    \tableofcontents
    
    
    \newpage
    \pagenumbering{arabic}
    %\linenumbers\renewcommand\thelinenumber{\color{black!50}\arabic{linenumber}}
            \documentclass[12pt, a4paper]{report}

\input{template/main.tex}

\title{\BA{Title in Progress...}}
\author{Boris Andrews}
\affil{Mathematical Institute, University of Oxford}
\date{\today}


\begin{document}
    \pagenumbering{gobble}
    \maketitle
    
    
    \begin{abstract}
        Magnetic confinement reactors---in particular tokamaks---offer one of the most promising options for achieving practical nuclear fusion, with the potential to provide virtually limitless, clean energy. The theoretical and numerical modeling of tokamak plasmas is simultaneously an essential component of effective reactor design, and a great research barrier. Tokamak operational conditions exhibit comparatively low Knudsen numbers. Kinetic effects, including kinetic waves and instabilities, Landau damping, bump-on-tail instabilities and more, are therefore highly influential in tokamak plasma dynamics. Purely fluid models are inherently incapable of capturing these effects, whereas the high dimensionality in purely kinetic models render them practically intractable for most relevant purposes.

        We consider a $\delta\!f$ decomposition model, with a macroscopic fluid background and microscopic kinetic correction, both fully coupled to each other. A similar manner of discretization is proposed to that used in the recent \texttt{STRUPHY} code \cite{Holderied_Possanner_Wang_2021, Holderied_2022, Li_et_al_2023} with a finite-element model for the background and a pseudo-particle/PiC model for the correction.

        The fluid background satisfies the full, non-linear, resistive, compressible, Hall MHD equations. \cite{Laakmann_Hu_Farrell_2022} introduces finite-element(-in-space) implicit timesteppers for the incompressible analogue to this system with structure-preserving (SP) properties in the ideal case, alongside parameter-robust preconditioners. We show that these timesteppers can derive from a finite-element-in-time (FET) (and finite-element-in-space) interpretation. The benefits of this reformulation are discussed, including the derivation of timesteppers that are higher order in time, and the quantifiable dissipative SP properties in the non-ideal, resistive case.
        
        We discuss possible options for extending this FET approach to timesteppers for the compressible case.

        The kinetic corrections satisfy linearized Boltzmann equations. Using a Lénard--Bernstein collision operator, these take Fokker--Planck-like forms \cite{Fokker_1914, Planck_1917} wherein pseudo-particles in the numerical model obey the neoclassical transport equations, with particle-independent Brownian drift terms. This offers a rigorous methodology for incorporating collisions into the particle transport model, without coupling the equations of motions for each particle.
        
        Works by Chen, Chacón et al. \cite{Chen_Chacón_Barnes_2011, Chacón_Chen_Barnes_2013, Chen_Chacón_2014, Chen_Chacón_2015} have developed structure-preserving particle pushers for neoclassical transport in the Vlasov equations, derived from Crank--Nicolson integrators. We show these too can can derive from a FET interpretation, similarly offering potential extensions to higher-order-in-time particle pushers. The FET formulation is used also to consider how the stochastic drift terms can be incorporated into the pushers. Stochastic gyrokinetic expansions are also discussed.

        Different options for the numerical implementation of these schemes are considered.

        Due to the efficacy of FET in the development of SP timesteppers for both the fluid and kinetic component, we hope this approach will prove effective in the future for developing SP timesteppers for the full hybrid model. We hope this will give us the opportunity to incorporate previously inaccessible kinetic effects into the highly effective, modern, finite-element MHD models.
    \end{abstract}
    
    
    \newpage
    \tableofcontents
    
    
    \newpage
    \pagenumbering{arabic}
    %\linenumbers\renewcommand\thelinenumber{\color{black!50}\arabic{linenumber}}
            \input{0 - introduction/main.tex}
        \part{Research}
            \input{1 - low-noise PiC models/main.tex}
            \input{2 - kinetic component/main.tex}
            \input{3 - fluid component/main.tex}
            \input{4 - numerical implementation/main.tex}
        \part{Project Overview}
            \input{5 - research plan/main.tex}
            \input{6 - summary/main.tex}
    
    
    %\section{}
    \newpage
    \pagenumbering{gobble}
        \printbibliography


    \newpage
    \pagenumbering{roman}
    \appendix
        \part{Appendices}
            \input{8 - Hilbert complexes/main.tex}
            \input{9 - weak conservation proofs/main.tex}
\end{document}

        \part{Research}
            \documentclass[12pt, a4paper]{report}

\input{template/main.tex}

\title{\BA{Title in Progress...}}
\author{Boris Andrews}
\affil{Mathematical Institute, University of Oxford}
\date{\today}


\begin{document}
    \pagenumbering{gobble}
    \maketitle
    
    
    \begin{abstract}
        Magnetic confinement reactors---in particular tokamaks---offer one of the most promising options for achieving practical nuclear fusion, with the potential to provide virtually limitless, clean energy. The theoretical and numerical modeling of tokamak plasmas is simultaneously an essential component of effective reactor design, and a great research barrier. Tokamak operational conditions exhibit comparatively low Knudsen numbers. Kinetic effects, including kinetic waves and instabilities, Landau damping, bump-on-tail instabilities and more, are therefore highly influential in tokamak plasma dynamics. Purely fluid models are inherently incapable of capturing these effects, whereas the high dimensionality in purely kinetic models render them practically intractable for most relevant purposes.

        We consider a $\delta\!f$ decomposition model, with a macroscopic fluid background and microscopic kinetic correction, both fully coupled to each other. A similar manner of discretization is proposed to that used in the recent \texttt{STRUPHY} code \cite{Holderied_Possanner_Wang_2021, Holderied_2022, Li_et_al_2023} with a finite-element model for the background and a pseudo-particle/PiC model for the correction.

        The fluid background satisfies the full, non-linear, resistive, compressible, Hall MHD equations. \cite{Laakmann_Hu_Farrell_2022} introduces finite-element(-in-space) implicit timesteppers for the incompressible analogue to this system with structure-preserving (SP) properties in the ideal case, alongside parameter-robust preconditioners. We show that these timesteppers can derive from a finite-element-in-time (FET) (and finite-element-in-space) interpretation. The benefits of this reformulation are discussed, including the derivation of timesteppers that are higher order in time, and the quantifiable dissipative SP properties in the non-ideal, resistive case.
        
        We discuss possible options for extending this FET approach to timesteppers for the compressible case.

        The kinetic corrections satisfy linearized Boltzmann equations. Using a Lénard--Bernstein collision operator, these take Fokker--Planck-like forms \cite{Fokker_1914, Planck_1917} wherein pseudo-particles in the numerical model obey the neoclassical transport equations, with particle-independent Brownian drift terms. This offers a rigorous methodology for incorporating collisions into the particle transport model, without coupling the equations of motions for each particle.
        
        Works by Chen, Chacón et al. \cite{Chen_Chacón_Barnes_2011, Chacón_Chen_Barnes_2013, Chen_Chacón_2014, Chen_Chacón_2015} have developed structure-preserving particle pushers for neoclassical transport in the Vlasov equations, derived from Crank--Nicolson integrators. We show these too can can derive from a FET interpretation, similarly offering potential extensions to higher-order-in-time particle pushers. The FET formulation is used also to consider how the stochastic drift terms can be incorporated into the pushers. Stochastic gyrokinetic expansions are also discussed.

        Different options for the numerical implementation of these schemes are considered.

        Due to the efficacy of FET in the development of SP timesteppers for both the fluid and kinetic component, we hope this approach will prove effective in the future for developing SP timesteppers for the full hybrid model. We hope this will give us the opportunity to incorporate previously inaccessible kinetic effects into the highly effective, modern, finite-element MHD models.
    \end{abstract}
    
    
    \newpage
    \tableofcontents
    
    
    \newpage
    \pagenumbering{arabic}
    %\linenumbers\renewcommand\thelinenumber{\color{black!50}\arabic{linenumber}}
            \input{0 - introduction/main.tex}
        \part{Research}
            \input{1 - low-noise PiC models/main.tex}
            \input{2 - kinetic component/main.tex}
            \input{3 - fluid component/main.tex}
            \input{4 - numerical implementation/main.tex}
        \part{Project Overview}
            \input{5 - research plan/main.tex}
            \input{6 - summary/main.tex}
    
    
    %\section{}
    \newpage
    \pagenumbering{gobble}
        \printbibliography


    \newpage
    \pagenumbering{roman}
    \appendix
        \part{Appendices}
            \input{8 - Hilbert complexes/main.tex}
            \input{9 - weak conservation proofs/main.tex}
\end{document}

            \documentclass[12pt, a4paper]{report}

\input{template/main.tex}

\title{\BA{Title in Progress...}}
\author{Boris Andrews}
\affil{Mathematical Institute, University of Oxford}
\date{\today}


\begin{document}
    \pagenumbering{gobble}
    \maketitle
    
    
    \begin{abstract}
        Magnetic confinement reactors---in particular tokamaks---offer one of the most promising options for achieving practical nuclear fusion, with the potential to provide virtually limitless, clean energy. The theoretical and numerical modeling of tokamak plasmas is simultaneously an essential component of effective reactor design, and a great research barrier. Tokamak operational conditions exhibit comparatively low Knudsen numbers. Kinetic effects, including kinetic waves and instabilities, Landau damping, bump-on-tail instabilities and more, are therefore highly influential in tokamak plasma dynamics. Purely fluid models are inherently incapable of capturing these effects, whereas the high dimensionality in purely kinetic models render them practically intractable for most relevant purposes.

        We consider a $\delta\!f$ decomposition model, with a macroscopic fluid background and microscopic kinetic correction, both fully coupled to each other. A similar manner of discretization is proposed to that used in the recent \texttt{STRUPHY} code \cite{Holderied_Possanner_Wang_2021, Holderied_2022, Li_et_al_2023} with a finite-element model for the background and a pseudo-particle/PiC model for the correction.

        The fluid background satisfies the full, non-linear, resistive, compressible, Hall MHD equations. \cite{Laakmann_Hu_Farrell_2022} introduces finite-element(-in-space) implicit timesteppers for the incompressible analogue to this system with structure-preserving (SP) properties in the ideal case, alongside parameter-robust preconditioners. We show that these timesteppers can derive from a finite-element-in-time (FET) (and finite-element-in-space) interpretation. The benefits of this reformulation are discussed, including the derivation of timesteppers that are higher order in time, and the quantifiable dissipative SP properties in the non-ideal, resistive case.
        
        We discuss possible options for extending this FET approach to timesteppers for the compressible case.

        The kinetic corrections satisfy linearized Boltzmann equations. Using a Lénard--Bernstein collision operator, these take Fokker--Planck-like forms \cite{Fokker_1914, Planck_1917} wherein pseudo-particles in the numerical model obey the neoclassical transport equations, with particle-independent Brownian drift terms. This offers a rigorous methodology for incorporating collisions into the particle transport model, without coupling the equations of motions for each particle.
        
        Works by Chen, Chacón et al. \cite{Chen_Chacón_Barnes_2011, Chacón_Chen_Barnes_2013, Chen_Chacón_2014, Chen_Chacón_2015} have developed structure-preserving particle pushers for neoclassical transport in the Vlasov equations, derived from Crank--Nicolson integrators. We show these too can can derive from a FET interpretation, similarly offering potential extensions to higher-order-in-time particle pushers. The FET formulation is used also to consider how the stochastic drift terms can be incorporated into the pushers. Stochastic gyrokinetic expansions are also discussed.

        Different options for the numerical implementation of these schemes are considered.

        Due to the efficacy of FET in the development of SP timesteppers for both the fluid and kinetic component, we hope this approach will prove effective in the future for developing SP timesteppers for the full hybrid model. We hope this will give us the opportunity to incorporate previously inaccessible kinetic effects into the highly effective, modern, finite-element MHD models.
    \end{abstract}
    
    
    \newpage
    \tableofcontents
    
    
    \newpage
    \pagenumbering{arabic}
    %\linenumbers\renewcommand\thelinenumber{\color{black!50}\arabic{linenumber}}
            \input{0 - introduction/main.tex}
        \part{Research}
            \input{1 - low-noise PiC models/main.tex}
            \input{2 - kinetic component/main.tex}
            \input{3 - fluid component/main.tex}
            \input{4 - numerical implementation/main.tex}
        \part{Project Overview}
            \input{5 - research plan/main.tex}
            \input{6 - summary/main.tex}
    
    
    %\section{}
    \newpage
    \pagenumbering{gobble}
        \printbibliography


    \newpage
    \pagenumbering{roman}
    \appendix
        \part{Appendices}
            \input{8 - Hilbert complexes/main.tex}
            \input{9 - weak conservation proofs/main.tex}
\end{document}

            \documentclass[12pt, a4paper]{report}

\input{template/main.tex}

\title{\BA{Title in Progress...}}
\author{Boris Andrews}
\affil{Mathematical Institute, University of Oxford}
\date{\today}


\begin{document}
    \pagenumbering{gobble}
    \maketitle
    
    
    \begin{abstract}
        Magnetic confinement reactors---in particular tokamaks---offer one of the most promising options for achieving practical nuclear fusion, with the potential to provide virtually limitless, clean energy. The theoretical and numerical modeling of tokamak plasmas is simultaneously an essential component of effective reactor design, and a great research barrier. Tokamak operational conditions exhibit comparatively low Knudsen numbers. Kinetic effects, including kinetic waves and instabilities, Landau damping, bump-on-tail instabilities and more, are therefore highly influential in tokamak plasma dynamics. Purely fluid models are inherently incapable of capturing these effects, whereas the high dimensionality in purely kinetic models render them practically intractable for most relevant purposes.

        We consider a $\delta\!f$ decomposition model, with a macroscopic fluid background and microscopic kinetic correction, both fully coupled to each other. A similar manner of discretization is proposed to that used in the recent \texttt{STRUPHY} code \cite{Holderied_Possanner_Wang_2021, Holderied_2022, Li_et_al_2023} with a finite-element model for the background and a pseudo-particle/PiC model for the correction.

        The fluid background satisfies the full, non-linear, resistive, compressible, Hall MHD equations. \cite{Laakmann_Hu_Farrell_2022} introduces finite-element(-in-space) implicit timesteppers for the incompressible analogue to this system with structure-preserving (SP) properties in the ideal case, alongside parameter-robust preconditioners. We show that these timesteppers can derive from a finite-element-in-time (FET) (and finite-element-in-space) interpretation. The benefits of this reformulation are discussed, including the derivation of timesteppers that are higher order in time, and the quantifiable dissipative SP properties in the non-ideal, resistive case.
        
        We discuss possible options for extending this FET approach to timesteppers for the compressible case.

        The kinetic corrections satisfy linearized Boltzmann equations. Using a Lénard--Bernstein collision operator, these take Fokker--Planck-like forms \cite{Fokker_1914, Planck_1917} wherein pseudo-particles in the numerical model obey the neoclassical transport equations, with particle-independent Brownian drift terms. This offers a rigorous methodology for incorporating collisions into the particle transport model, without coupling the equations of motions for each particle.
        
        Works by Chen, Chacón et al. \cite{Chen_Chacón_Barnes_2011, Chacón_Chen_Barnes_2013, Chen_Chacón_2014, Chen_Chacón_2015} have developed structure-preserving particle pushers for neoclassical transport in the Vlasov equations, derived from Crank--Nicolson integrators. We show these too can can derive from a FET interpretation, similarly offering potential extensions to higher-order-in-time particle pushers. The FET formulation is used also to consider how the stochastic drift terms can be incorporated into the pushers. Stochastic gyrokinetic expansions are also discussed.

        Different options for the numerical implementation of these schemes are considered.

        Due to the efficacy of FET in the development of SP timesteppers for both the fluid and kinetic component, we hope this approach will prove effective in the future for developing SP timesteppers for the full hybrid model. We hope this will give us the opportunity to incorporate previously inaccessible kinetic effects into the highly effective, modern, finite-element MHD models.
    \end{abstract}
    
    
    \newpage
    \tableofcontents
    
    
    \newpage
    \pagenumbering{arabic}
    %\linenumbers\renewcommand\thelinenumber{\color{black!50}\arabic{linenumber}}
            \input{0 - introduction/main.tex}
        \part{Research}
            \input{1 - low-noise PiC models/main.tex}
            \input{2 - kinetic component/main.tex}
            \input{3 - fluid component/main.tex}
            \input{4 - numerical implementation/main.tex}
        \part{Project Overview}
            \input{5 - research plan/main.tex}
            \input{6 - summary/main.tex}
    
    
    %\section{}
    \newpage
    \pagenumbering{gobble}
        \printbibliography


    \newpage
    \pagenumbering{roman}
    \appendix
        \part{Appendices}
            \input{8 - Hilbert complexes/main.tex}
            \input{9 - weak conservation proofs/main.tex}
\end{document}

            \documentclass[12pt, a4paper]{report}

\input{template/main.tex}

\title{\BA{Title in Progress...}}
\author{Boris Andrews}
\affil{Mathematical Institute, University of Oxford}
\date{\today}


\begin{document}
    \pagenumbering{gobble}
    \maketitle
    
    
    \begin{abstract}
        Magnetic confinement reactors---in particular tokamaks---offer one of the most promising options for achieving practical nuclear fusion, with the potential to provide virtually limitless, clean energy. The theoretical and numerical modeling of tokamak plasmas is simultaneously an essential component of effective reactor design, and a great research barrier. Tokamak operational conditions exhibit comparatively low Knudsen numbers. Kinetic effects, including kinetic waves and instabilities, Landau damping, bump-on-tail instabilities and more, are therefore highly influential in tokamak plasma dynamics. Purely fluid models are inherently incapable of capturing these effects, whereas the high dimensionality in purely kinetic models render them practically intractable for most relevant purposes.

        We consider a $\delta\!f$ decomposition model, with a macroscopic fluid background and microscopic kinetic correction, both fully coupled to each other. A similar manner of discretization is proposed to that used in the recent \texttt{STRUPHY} code \cite{Holderied_Possanner_Wang_2021, Holderied_2022, Li_et_al_2023} with a finite-element model for the background and a pseudo-particle/PiC model for the correction.

        The fluid background satisfies the full, non-linear, resistive, compressible, Hall MHD equations. \cite{Laakmann_Hu_Farrell_2022} introduces finite-element(-in-space) implicit timesteppers for the incompressible analogue to this system with structure-preserving (SP) properties in the ideal case, alongside parameter-robust preconditioners. We show that these timesteppers can derive from a finite-element-in-time (FET) (and finite-element-in-space) interpretation. The benefits of this reformulation are discussed, including the derivation of timesteppers that are higher order in time, and the quantifiable dissipative SP properties in the non-ideal, resistive case.
        
        We discuss possible options for extending this FET approach to timesteppers for the compressible case.

        The kinetic corrections satisfy linearized Boltzmann equations. Using a Lénard--Bernstein collision operator, these take Fokker--Planck-like forms \cite{Fokker_1914, Planck_1917} wherein pseudo-particles in the numerical model obey the neoclassical transport equations, with particle-independent Brownian drift terms. This offers a rigorous methodology for incorporating collisions into the particle transport model, without coupling the equations of motions for each particle.
        
        Works by Chen, Chacón et al. \cite{Chen_Chacón_Barnes_2011, Chacón_Chen_Barnes_2013, Chen_Chacón_2014, Chen_Chacón_2015} have developed structure-preserving particle pushers for neoclassical transport in the Vlasov equations, derived from Crank--Nicolson integrators. We show these too can can derive from a FET interpretation, similarly offering potential extensions to higher-order-in-time particle pushers. The FET formulation is used also to consider how the stochastic drift terms can be incorporated into the pushers. Stochastic gyrokinetic expansions are also discussed.

        Different options for the numerical implementation of these schemes are considered.

        Due to the efficacy of FET in the development of SP timesteppers for both the fluid and kinetic component, we hope this approach will prove effective in the future for developing SP timesteppers for the full hybrid model. We hope this will give us the opportunity to incorporate previously inaccessible kinetic effects into the highly effective, modern, finite-element MHD models.
    \end{abstract}
    
    
    \newpage
    \tableofcontents
    
    
    \newpage
    \pagenumbering{arabic}
    %\linenumbers\renewcommand\thelinenumber{\color{black!50}\arabic{linenumber}}
            \input{0 - introduction/main.tex}
        \part{Research}
            \input{1 - low-noise PiC models/main.tex}
            \input{2 - kinetic component/main.tex}
            \input{3 - fluid component/main.tex}
            \input{4 - numerical implementation/main.tex}
        \part{Project Overview}
            \input{5 - research plan/main.tex}
            \input{6 - summary/main.tex}
    
    
    %\section{}
    \newpage
    \pagenumbering{gobble}
        \printbibliography


    \newpage
    \pagenumbering{roman}
    \appendix
        \part{Appendices}
            \input{8 - Hilbert complexes/main.tex}
            \input{9 - weak conservation proofs/main.tex}
\end{document}

        \part{Project Overview}
            \documentclass[12pt, a4paper]{report}

\input{template/main.tex}

\title{\BA{Title in Progress...}}
\author{Boris Andrews}
\affil{Mathematical Institute, University of Oxford}
\date{\today}


\begin{document}
    \pagenumbering{gobble}
    \maketitle
    
    
    \begin{abstract}
        Magnetic confinement reactors---in particular tokamaks---offer one of the most promising options for achieving practical nuclear fusion, with the potential to provide virtually limitless, clean energy. The theoretical and numerical modeling of tokamak plasmas is simultaneously an essential component of effective reactor design, and a great research barrier. Tokamak operational conditions exhibit comparatively low Knudsen numbers. Kinetic effects, including kinetic waves and instabilities, Landau damping, bump-on-tail instabilities and more, are therefore highly influential in tokamak plasma dynamics. Purely fluid models are inherently incapable of capturing these effects, whereas the high dimensionality in purely kinetic models render them practically intractable for most relevant purposes.

        We consider a $\delta\!f$ decomposition model, with a macroscopic fluid background and microscopic kinetic correction, both fully coupled to each other. A similar manner of discretization is proposed to that used in the recent \texttt{STRUPHY} code \cite{Holderied_Possanner_Wang_2021, Holderied_2022, Li_et_al_2023} with a finite-element model for the background and a pseudo-particle/PiC model for the correction.

        The fluid background satisfies the full, non-linear, resistive, compressible, Hall MHD equations. \cite{Laakmann_Hu_Farrell_2022} introduces finite-element(-in-space) implicit timesteppers for the incompressible analogue to this system with structure-preserving (SP) properties in the ideal case, alongside parameter-robust preconditioners. We show that these timesteppers can derive from a finite-element-in-time (FET) (and finite-element-in-space) interpretation. The benefits of this reformulation are discussed, including the derivation of timesteppers that are higher order in time, and the quantifiable dissipative SP properties in the non-ideal, resistive case.
        
        We discuss possible options for extending this FET approach to timesteppers for the compressible case.

        The kinetic corrections satisfy linearized Boltzmann equations. Using a Lénard--Bernstein collision operator, these take Fokker--Planck-like forms \cite{Fokker_1914, Planck_1917} wherein pseudo-particles in the numerical model obey the neoclassical transport equations, with particle-independent Brownian drift terms. This offers a rigorous methodology for incorporating collisions into the particle transport model, without coupling the equations of motions for each particle.
        
        Works by Chen, Chacón et al. \cite{Chen_Chacón_Barnes_2011, Chacón_Chen_Barnes_2013, Chen_Chacón_2014, Chen_Chacón_2015} have developed structure-preserving particle pushers for neoclassical transport in the Vlasov equations, derived from Crank--Nicolson integrators. We show these too can can derive from a FET interpretation, similarly offering potential extensions to higher-order-in-time particle pushers. The FET formulation is used also to consider how the stochastic drift terms can be incorporated into the pushers. Stochastic gyrokinetic expansions are also discussed.

        Different options for the numerical implementation of these schemes are considered.

        Due to the efficacy of FET in the development of SP timesteppers for both the fluid and kinetic component, we hope this approach will prove effective in the future for developing SP timesteppers for the full hybrid model. We hope this will give us the opportunity to incorporate previously inaccessible kinetic effects into the highly effective, modern, finite-element MHD models.
    \end{abstract}
    
    
    \newpage
    \tableofcontents
    
    
    \newpage
    \pagenumbering{arabic}
    %\linenumbers\renewcommand\thelinenumber{\color{black!50}\arabic{linenumber}}
            \input{0 - introduction/main.tex}
        \part{Research}
            \input{1 - low-noise PiC models/main.tex}
            \input{2 - kinetic component/main.tex}
            \input{3 - fluid component/main.tex}
            \input{4 - numerical implementation/main.tex}
        \part{Project Overview}
            \input{5 - research plan/main.tex}
            \input{6 - summary/main.tex}
    
    
    %\section{}
    \newpage
    \pagenumbering{gobble}
        \printbibliography


    \newpage
    \pagenumbering{roman}
    \appendix
        \part{Appendices}
            \input{8 - Hilbert complexes/main.tex}
            \input{9 - weak conservation proofs/main.tex}
\end{document}

            \documentclass[12pt, a4paper]{report}

\input{template/main.tex}

\title{\BA{Title in Progress...}}
\author{Boris Andrews}
\affil{Mathematical Institute, University of Oxford}
\date{\today}


\begin{document}
    \pagenumbering{gobble}
    \maketitle
    
    
    \begin{abstract}
        Magnetic confinement reactors---in particular tokamaks---offer one of the most promising options for achieving practical nuclear fusion, with the potential to provide virtually limitless, clean energy. The theoretical and numerical modeling of tokamak plasmas is simultaneously an essential component of effective reactor design, and a great research barrier. Tokamak operational conditions exhibit comparatively low Knudsen numbers. Kinetic effects, including kinetic waves and instabilities, Landau damping, bump-on-tail instabilities and more, are therefore highly influential in tokamak plasma dynamics. Purely fluid models are inherently incapable of capturing these effects, whereas the high dimensionality in purely kinetic models render them practically intractable for most relevant purposes.

        We consider a $\delta\!f$ decomposition model, with a macroscopic fluid background and microscopic kinetic correction, both fully coupled to each other. A similar manner of discretization is proposed to that used in the recent \texttt{STRUPHY} code \cite{Holderied_Possanner_Wang_2021, Holderied_2022, Li_et_al_2023} with a finite-element model for the background and a pseudo-particle/PiC model for the correction.

        The fluid background satisfies the full, non-linear, resistive, compressible, Hall MHD equations. \cite{Laakmann_Hu_Farrell_2022} introduces finite-element(-in-space) implicit timesteppers for the incompressible analogue to this system with structure-preserving (SP) properties in the ideal case, alongside parameter-robust preconditioners. We show that these timesteppers can derive from a finite-element-in-time (FET) (and finite-element-in-space) interpretation. The benefits of this reformulation are discussed, including the derivation of timesteppers that are higher order in time, and the quantifiable dissipative SP properties in the non-ideal, resistive case.
        
        We discuss possible options for extending this FET approach to timesteppers for the compressible case.

        The kinetic corrections satisfy linearized Boltzmann equations. Using a Lénard--Bernstein collision operator, these take Fokker--Planck-like forms \cite{Fokker_1914, Planck_1917} wherein pseudo-particles in the numerical model obey the neoclassical transport equations, with particle-independent Brownian drift terms. This offers a rigorous methodology for incorporating collisions into the particle transport model, without coupling the equations of motions for each particle.
        
        Works by Chen, Chacón et al. \cite{Chen_Chacón_Barnes_2011, Chacón_Chen_Barnes_2013, Chen_Chacón_2014, Chen_Chacón_2015} have developed structure-preserving particle pushers for neoclassical transport in the Vlasov equations, derived from Crank--Nicolson integrators. We show these too can can derive from a FET interpretation, similarly offering potential extensions to higher-order-in-time particle pushers. The FET formulation is used also to consider how the stochastic drift terms can be incorporated into the pushers. Stochastic gyrokinetic expansions are also discussed.

        Different options for the numerical implementation of these schemes are considered.

        Due to the efficacy of FET in the development of SP timesteppers for both the fluid and kinetic component, we hope this approach will prove effective in the future for developing SP timesteppers for the full hybrid model. We hope this will give us the opportunity to incorporate previously inaccessible kinetic effects into the highly effective, modern, finite-element MHD models.
    \end{abstract}
    
    
    \newpage
    \tableofcontents
    
    
    \newpage
    \pagenumbering{arabic}
    %\linenumbers\renewcommand\thelinenumber{\color{black!50}\arabic{linenumber}}
            \input{0 - introduction/main.tex}
        \part{Research}
            \input{1 - low-noise PiC models/main.tex}
            \input{2 - kinetic component/main.tex}
            \input{3 - fluid component/main.tex}
            \input{4 - numerical implementation/main.tex}
        \part{Project Overview}
            \input{5 - research plan/main.tex}
            \input{6 - summary/main.tex}
    
    
    %\section{}
    \newpage
    \pagenumbering{gobble}
        \printbibliography


    \newpage
    \pagenumbering{roman}
    \appendix
        \part{Appendices}
            \input{8 - Hilbert complexes/main.tex}
            \input{9 - weak conservation proofs/main.tex}
\end{document}

    
    
    %\section{}
    \newpage
    \pagenumbering{gobble}
        \printbibliography


    \newpage
    \pagenumbering{roman}
    \appendix
        \part{Appendices}
            \documentclass[12pt, a4paper]{report}

\input{template/main.tex}

\title{\BA{Title in Progress...}}
\author{Boris Andrews}
\affil{Mathematical Institute, University of Oxford}
\date{\today}


\begin{document}
    \pagenumbering{gobble}
    \maketitle
    
    
    \begin{abstract}
        Magnetic confinement reactors---in particular tokamaks---offer one of the most promising options for achieving practical nuclear fusion, with the potential to provide virtually limitless, clean energy. The theoretical and numerical modeling of tokamak plasmas is simultaneously an essential component of effective reactor design, and a great research barrier. Tokamak operational conditions exhibit comparatively low Knudsen numbers. Kinetic effects, including kinetic waves and instabilities, Landau damping, bump-on-tail instabilities and more, are therefore highly influential in tokamak plasma dynamics. Purely fluid models are inherently incapable of capturing these effects, whereas the high dimensionality in purely kinetic models render them practically intractable for most relevant purposes.

        We consider a $\delta\!f$ decomposition model, with a macroscopic fluid background and microscopic kinetic correction, both fully coupled to each other. A similar manner of discretization is proposed to that used in the recent \texttt{STRUPHY} code \cite{Holderied_Possanner_Wang_2021, Holderied_2022, Li_et_al_2023} with a finite-element model for the background and a pseudo-particle/PiC model for the correction.

        The fluid background satisfies the full, non-linear, resistive, compressible, Hall MHD equations. \cite{Laakmann_Hu_Farrell_2022} introduces finite-element(-in-space) implicit timesteppers for the incompressible analogue to this system with structure-preserving (SP) properties in the ideal case, alongside parameter-robust preconditioners. We show that these timesteppers can derive from a finite-element-in-time (FET) (and finite-element-in-space) interpretation. The benefits of this reformulation are discussed, including the derivation of timesteppers that are higher order in time, and the quantifiable dissipative SP properties in the non-ideal, resistive case.
        
        We discuss possible options for extending this FET approach to timesteppers for the compressible case.

        The kinetic corrections satisfy linearized Boltzmann equations. Using a Lénard--Bernstein collision operator, these take Fokker--Planck-like forms \cite{Fokker_1914, Planck_1917} wherein pseudo-particles in the numerical model obey the neoclassical transport equations, with particle-independent Brownian drift terms. This offers a rigorous methodology for incorporating collisions into the particle transport model, without coupling the equations of motions for each particle.
        
        Works by Chen, Chacón et al. \cite{Chen_Chacón_Barnes_2011, Chacón_Chen_Barnes_2013, Chen_Chacón_2014, Chen_Chacón_2015} have developed structure-preserving particle pushers for neoclassical transport in the Vlasov equations, derived from Crank--Nicolson integrators. We show these too can can derive from a FET interpretation, similarly offering potential extensions to higher-order-in-time particle pushers. The FET formulation is used also to consider how the stochastic drift terms can be incorporated into the pushers. Stochastic gyrokinetic expansions are also discussed.

        Different options for the numerical implementation of these schemes are considered.

        Due to the efficacy of FET in the development of SP timesteppers for both the fluid and kinetic component, we hope this approach will prove effective in the future for developing SP timesteppers for the full hybrid model. We hope this will give us the opportunity to incorporate previously inaccessible kinetic effects into the highly effective, modern, finite-element MHD models.
    \end{abstract}
    
    
    \newpage
    \tableofcontents
    
    
    \newpage
    \pagenumbering{arabic}
    %\linenumbers\renewcommand\thelinenumber{\color{black!50}\arabic{linenumber}}
            \input{0 - introduction/main.tex}
        \part{Research}
            \input{1 - low-noise PiC models/main.tex}
            \input{2 - kinetic component/main.tex}
            \input{3 - fluid component/main.tex}
            \input{4 - numerical implementation/main.tex}
        \part{Project Overview}
            \input{5 - research plan/main.tex}
            \input{6 - summary/main.tex}
    
    
    %\section{}
    \newpage
    \pagenumbering{gobble}
        \printbibliography


    \newpage
    \pagenumbering{roman}
    \appendix
        \part{Appendices}
            \input{8 - Hilbert complexes/main.tex}
            \input{9 - weak conservation proofs/main.tex}
\end{document}

            \documentclass[12pt, a4paper]{report}

\input{template/main.tex}

\title{\BA{Title in Progress...}}
\author{Boris Andrews}
\affil{Mathematical Institute, University of Oxford}
\date{\today}


\begin{document}
    \pagenumbering{gobble}
    \maketitle
    
    
    \begin{abstract}
        Magnetic confinement reactors---in particular tokamaks---offer one of the most promising options for achieving practical nuclear fusion, with the potential to provide virtually limitless, clean energy. The theoretical and numerical modeling of tokamak plasmas is simultaneously an essential component of effective reactor design, and a great research barrier. Tokamak operational conditions exhibit comparatively low Knudsen numbers. Kinetic effects, including kinetic waves and instabilities, Landau damping, bump-on-tail instabilities and more, are therefore highly influential in tokamak plasma dynamics. Purely fluid models are inherently incapable of capturing these effects, whereas the high dimensionality in purely kinetic models render them practically intractable for most relevant purposes.

        We consider a $\delta\!f$ decomposition model, with a macroscopic fluid background and microscopic kinetic correction, both fully coupled to each other. A similar manner of discretization is proposed to that used in the recent \texttt{STRUPHY} code \cite{Holderied_Possanner_Wang_2021, Holderied_2022, Li_et_al_2023} with a finite-element model for the background and a pseudo-particle/PiC model for the correction.

        The fluid background satisfies the full, non-linear, resistive, compressible, Hall MHD equations. \cite{Laakmann_Hu_Farrell_2022} introduces finite-element(-in-space) implicit timesteppers for the incompressible analogue to this system with structure-preserving (SP) properties in the ideal case, alongside parameter-robust preconditioners. We show that these timesteppers can derive from a finite-element-in-time (FET) (and finite-element-in-space) interpretation. The benefits of this reformulation are discussed, including the derivation of timesteppers that are higher order in time, and the quantifiable dissipative SP properties in the non-ideal, resistive case.
        
        We discuss possible options for extending this FET approach to timesteppers for the compressible case.

        The kinetic corrections satisfy linearized Boltzmann equations. Using a Lénard--Bernstein collision operator, these take Fokker--Planck-like forms \cite{Fokker_1914, Planck_1917} wherein pseudo-particles in the numerical model obey the neoclassical transport equations, with particle-independent Brownian drift terms. This offers a rigorous methodology for incorporating collisions into the particle transport model, without coupling the equations of motions for each particle.
        
        Works by Chen, Chacón et al. \cite{Chen_Chacón_Barnes_2011, Chacón_Chen_Barnes_2013, Chen_Chacón_2014, Chen_Chacón_2015} have developed structure-preserving particle pushers for neoclassical transport in the Vlasov equations, derived from Crank--Nicolson integrators. We show these too can can derive from a FET interpretation, similarly offering potential extensions to higher-order-in-time particle pushers. The FET formulation is used also to consider how the stochastic drift terms can be incorporated into the pushers. Stochastic gyrokinetic expansions are also discussed.

        Different options for the numerical implementation of these schemes are considered.

        Due to the efficacy of FET in the development of SP timesteppers for both the fluid and kinetic component, we hope this approach will prove effective in the future for developing SP timesteppers for the full hybrid model. We hope this will give us the opportunity to incorporate previously inaccessible kinetic effects into the highly effective, modern, finite-element MHD models.
    \end{abstract}
    
    
    \newpage
    \tableofcontents
    
    
    \newpage
    \pagenumbering{arabic}
    %\linenumbers\renewcommand\thelinenumber{\color{black!50}\arabic{linenumber}}
            \input{0 - introduction/main.tex}
        \part{Research}
            \input{1 - low-noise PiC models/main.tex}
            \input{2 - kinetic component/main.tex}
            \input{3 - fluid component/main.tex}
            \input{4 - numerical implementation/main.tex}
        \part{Project Overview}
            \input{5 - research plan/main.tex}
            \input{6 - summary/main.tex}
    
    
    %\section{}
    \newpage
    \pagenumbering{gobble}
        \printbibliography


    \newpage
    \pagenumbering{roman}
    \appendix
        \part{Appendices}
            \input{8 - Hilbert complexes/main.tex}
            \input{9 - weak conservation proofs/main.tex}
\end{document}

\end{document}

        \part{Project Overview}
            \documentclass[12pt, a4paper]{report}

\documentclass[12pt, a4paper]{report}

\input{template/main.tex}

\title{\BA{Title in Progress...}}
\author{Boris Andrews}
\affil{Mathematical Institute, University of Oxford}
\date{\today}


\begin{document}
    \pagenumbering{gobble}
    \maketitle
    
    
    \begin{abstract}
        Magnetic confinement reactors---in particular tokamaks---offer one of the most promising options for achieving practical nuclear fusion, with the potential to provide virtually limitless, clean energy. The theoretical and numerical modeling of tokamak plasmas is simultaneously an essential component of effective reactor design, and a great research barrier. Tokamak operational conditions exhibit comparatively low Knudsen numbers. Kinetic effects, including kinetic waves and instabilities, Landau damping, bump-on-tail instabilities and more, are therefore highly influential in tokamak plasma dynamics. Purely fluid models are inherently incapable of capturing these effects, whereas the high dimensionality in purely kinetic models render them practically intractable for most relevant purposes.

        We consider a $\delta\!f$ decomposition model, with a macroscopic fluid background and microscopic kinetic correction, both fully coupled to each other. A similar manner of discretization is proposed to that used in the recent \texttt{STRUPHY} code \cite{Holderied_Possanner_Wang_2021, Holderied_2022, Li_et_al_2023} with a finite-element model for the background and a pseudo-particle/PiC model for the correction.

        The fluid background satisfies the full, non-linear, resistive, compressible, Hall MHD equations. \cite{Laakmann_Hu_Farrell_2022} introduces finite-element(-in-space) implicit timesteppers for the incompressible analogue to this system with structure-preserving (SP) properties in the ideal case, alongside parameter-robust preconditioners. We show that these timesteppers can derive from a finite-element-in-time (FET) (and finite-element-in-space) interpretation. The benefits of this reformulation are discussed, including the derivation of timesteppers that are higher order in time, and the quantifiable dissipative SP properties in the non-ideal, resistive case.
        
        We discuss possible options for extending this FET approach to timesteppers for the compressible case.

        The kinetic corrections satisfy linearized Boltzmann equations. Using a Lénard--Bernstein collision operator, these take Fokker--Planck-like forms \cite{Fokker_1914, Planck_1917} wherein pseudo-particles in the numerical model obey the neoclassical transport equations, with particle-independent Brownian drift terms. This offers a rigorous methodology for incorporating collisions into the particle transport model, without coupling the equations of motions for each particle.
        
        Works by Chen, Chacón et al. \cite{Chen_Chacón_Barnes_2011, Chacón_Chen_Barnes_2013, Chen_Chacón_2014, Chen_Chacón_2015} have developed structure-preserving particle pushers for neoclassical transport in the Vlasov equations, derived from Crank--Nicolson integrators. We show these too can can derive from a FET interpretation, similarly offering potential extensions to higher-order-in-time particle pushers. The FET formulation is used also to consider how the stochastic drift terms can be incorporated into the pushers. Stochastic gyrokinetic expansions are also discussed.

        Different options for the numerical implementation of these schemes are considered.

        Due to the efficacy of FET in the development of SP timesteppers for both the fluid and kinetic component, we hope this approach will prove effective in the future for developing SP timesteppers for the full hybrid model. We hope this will give us the opportunity to incorporate previously inaccessible kinetic effects into the highly effective, modern, finite-element MHD models.
    \end{abstract}
    
    
    \newpage
    \tableofcontents
    
    
    \newpage
    \pagenumbering{arabic}
    %\linenumbers\renewcommand\thelinenumber{\color{black!50}\arabic{linenumber}}
            \input{0 - introduction/main.tex}
        \part{Research}
            \input{1 - low-noise PiC models/main.tex}
            \input{2 - kinetic component/main.tex}
            \input{3 - fluid component/main.tex}
            \input{4 - numerical implementation/main.tex}
        \part{Project Overview}
            \input{5 - research plan/main.tex}
            \input{6 - summary/main.tex}
    
    
    %\section{}
    \newpage
    \pagenumbering{gobble}
        \printbibliography


    \newpage
    \pagenumbering{roman}
    \appendix
        \part{Appendices}
            \input{8 - Hilbert complexes/main.tex}
            \input{9 - weak conservation proofs/main.tex}
\end{document}


\title{\BA{Title in Progress...}}
\author{Boris Andrews}
\affil{Mathematical Institute, University of Oxford}
\date{\today}


\begin{document}
    \pagenumbering{gobble}
    \maketitle
    
    
    \begin{abstract}
        Magnetic confinement reactors---in particular tokamaks---offer one of the most promising options for achieving practical nuclear fusion, with the potential to provide virtually limitless, clean energy. The theoretical and numerical modeling of tokamak plasmas is simultaneously an essential component of effective reactor design, and a great research barrier. Tokamak operational conditions exhibit comparatively low Knudsen numbers. Kinetic effects, including kinetic waves and instabilities, Landau damping, bump-on-tail instabilities and more, are therefore highly influential in tokamak plasma dynamics. Purely fluid models are inherently incapable of capturing these effects, whereas the high dimensionality in purely kinetic models render them practically intractable for most relevant purposes.

        We consider a $\delta\!f$ decomposition model, with a macroscopic fluid background and microscopic kinetic correction, both fully coupled to each other. A similar manner of discretization is proposed to that used in the recent \texttt{STRUPHY} code \cite{Holderied_Possanner_Wang_2021, Holderied_2022, Li_et_al_2023} with a finite-element model for the background and a pseudo-particle/PiC model for the correction.

        The fluid background satisfies the full, non-linear, resistive, compressible, Hall MHD equations. \cite{Laakmann_Hu_Farrell_2022} introduces finite-element(-in-space) implicit timesteppers for the incompressible analogue to this system with structure-preserving (SP) properties in the ideal case, alongside parameter-robust preconditioners. We show that these timesteppers can derive from a finite-element-in-time (FET) (and finite-element-in-space) interpretation. The benefits of this reformulation are discussed, including the derivation of timesteppers that are higher order in time, and the quantifiable dissipative SP properties in the non-ideal, resistive case.
        
        We discuss possible options for extending this FET approach to timesteppers for the compressible case.

        The kinetic corrections satisfy linearized Boltzmann equations. Using a Lénard--Bernstein collision operator, these take Fokker--Planck-like forms \cite{Fokker_1914, Planck_1917} wherein pseudo-particles in the numerical model obey the neoclassical transport equations, with particle-independent Brownian drift terms. This offers a rigorous methodology for incorporating collisions into the particle transport model, without coupling the equations of motions for each particle.
        
        Works by Chen, Chacón et al. \cite{Chen_Chacón_Barnes_2011, Chacón_Chen_Barnes_2013, Chen_Chacón_2014, Chen_Chacón_2015} have developed structure-preserving particle pushers for neoclassical transport in the Vlasov equations, derived from Crank--Nicolson integrators. We show these too can can derive from a FET interpretation, similarly offering potential extensions to higher-order-in-time particle pushers. The FET formulation is used also to consider how the stochastic drift terms can be incorporated into the pushers. Stochastic gyrokinetic expansions are also discussed.

        Different options for the numerical implementation of these schemes are considered.

        Due to the efficacy of FET in the development of SP timesteppers for both the fluid and kinetic component, we hope this approach will prove effective in the future for developing SP timesteppers for the full hybrid model. We hope this will give us the opportunity to incorporate previously inaccessible kinetic effects into the highly effective, modern, finite-element MHD models.
    \end{abstract}
    
    
    \newpage
    \tableofcontents
    
    
    \newpage
    \pagenumbering{arabic}
    %\linenumbers\renewcommand\thelinenumber{\color{black!50}\arabic{linenumber}}
            \documentclass[12pt, a4paper]{report}

\input{template/main.tex}

\title{\BA{Title in Progress...}}
\author{Boris Andrews}
\affil{Mathematical Institute, University of Oxford}
\date{\today}


\begin{document}
    \pagenumbering{gobble}
    \maketitle
    
    
    \begin{abstract}
        Magnetic confinement reactors---in particular tokamaks---offer one of the most promising options for achieving practical nuclear fusion, with the potential to provide virtually limitless, clean energy. The theoretical and numerical modeling of tokamak plasmas is simultaneously an essential component of effective reactor design, and a great research barrier. Tokamak operational conditions exhibit comparatively low Knudsen numbers. Kinetic effects, including kinetic waves and instabilities, Landau damping, bump-on-tail instabilities and more, are therefore highly influential in tokamak plasma dynamics. Purely fluid models are inherently incapable of capturing these effects, whereas the high dimensionality in purely kinetic models render them practically intractable for most relevant purposes.

        We consider a $\delta\!f$ decomposition model, with a macroscopic fluid background and microscopic kinetic correction, both fully coupled to each other. A similar manner of discretization is proposed to that used in the recent \texttt{STRUPHY} code \cite{Holderied_Possanner_Wang_2021, Holderied_2022, Li_et_al_2023} with a finite-element model for the background and a pseudo-particle/PiC model for the correction.

        The fluid background satisfies the full, non-linear, resistive, compressible, Hall MHD equations. \cite{Laakmann_Hu_Farrell_2022} introduces finite-element(-in-space) implicit timesteppers for the incompressible analogue to this system with structure-preserving (SP) properties in the ideal case, alongside parameter-robust preconditioners. We show that these timesteppers can derive from a finite-element-in-time (FET) (and finite-element-in-space) interpretation. The benefits of this reformulation are discussed, including the derivation of timesteppers that are higher order in time, and the quantifiable dissipative SP properties in the non-ideal, resistive case.
        
        We discuss possible options for extending this FET approach to timesteppers for the compressible case.

        The kinetic corrections satisfy linearized Boltzmann equations. Using a Lénard--Bernstein collision operator, these take Fokker--Planck-like forms \cite{Fokker_1914, Planck_1917} wherein pseudo-particles in the numerical model obey the neoclassical transport equations, with particle-independent Brownian drift terms. This offers a rigorous methodology for incorporating collisions into the particle transport model, without coupling the equations of motions for each particle.
        
        Works by Chen, Chacón et al. \cite{Chen_Chacón_Barnes_2011, Chacón_Chen_Barnes_2013, Chen_Chacón_2014, Chen_Chacón_2015} have developed structure-preserving particle pushers for neoclassical transport in the Vlasov equations, derived from Crank--Nicolson integrators. We show these too can can derive from a FET interpretation, similarly offering potential extensions to higher-order-in-time particle pushers. The FET formulation is used also to consider how the stochastic drift terms can be incorporated into the pushers. Stochastic gyrokinetic expansions are also discussed.

        Different options for the numerical implementation of these schemes are considered.

        Due to the efficacy of FET in the development of SP timesteppers for both the fluid and kinetic component, we hope this approach will prove effective in the future for developing SP timesteppers for the full hybrid model. We hope this will give us the opportunity to incorporate previously inaccessible kinetic effects into the highly effective, modern, finite-element MHD models.
    \end{abstract}
    
    
    \newpage
    \tableofcontents
    
    
    \newpage
    \pagenumbering{arabic}
    %\linenumbers\renewcommand\thelinenumber{\color{black!50}\arabic{linenumber}}
            \input{0 - introduction/main.tex}
        \part{Research}
            \input{1 - low-noise PiC models/main.tex}
            \input{2 - kinetic component/main.tex}
            \input{3 - fluid component/main.tex}
            \input{4 - numerical implementation/main.tex}
        \part{Project Overview}
            \input{5 - research plan/main.tex}
            \input{6 - summary/main.tex}
    
    
    %\section{}
    \newpage
    \pagenumbering{gobble}
        \printbibliography


    \newpage
    \pagenumbering{roman}
    \appendix
        \part{Appendices}
            \input{8 - Hilbert complexes/main.tex}
            \input{9 - weak conservation proofs/main.tex}
\end{document}

        \part{Research}
            \documentclass[12pt, a4paper]{report}

\input{template/main.tex}

\title{\BA{Title in Progress...}}
\author{Boris Andrews}
\affil{Mathematical Institute, University of Oxford}
\date{\today}


\begin{document}
    \pagenumbering{gobble}
    \maketitle
    
    
    \begin{abstract}
        Magnetic confinement reactors---in particular tokamaks---offer one of the most promising options for achieving practical nuclear fusion, with the potential to provide virtually limitless, clean energy. The theoretical and numerical modeling of tokamak plasmas is simultaneously an essential component of effective reactor design, and a great research barrier. Tokamak operational conditions exhibit comparatively low Knudsen numbers. Kinetic effects, including kinetic waves and instabilities, Landau damping, bump-on-tail instabilities and more, are therefore highly influential in tokamak plasma dynamics. Purely fluid models are inherently incapable of capturing these effects, whereas the high dimensionality in purely kinetic models render them practically intractable for most relevant purposes.

        We consider a $\delta\!f$ decomposition model, with a macroscopic fluid background and microscopic kinetic correction, both fully coupled to each other. A similar manner of discretization is proposed to that used in the recent \texttt{STRUPHY} code \cite{Holderied_Possanner_Wang_2021, Holderied_2022, Li_et_al_2023} with a finite-element model for the background and a pseudo-particle/PiC model for the correction.

        The fluid background satisfies the full, non-linear, resistive, compressible, Hall MHD equations. \cite{Laakmann_Hu_Farrell_2022} introduces finite-element(-in-space) implicit timesteppers for the incompressible analogue to this system with structure-preserving (SP) properties in the ideal case, alongside parameter-robust preconditioners. We show that these timesteppers can derive from a finite-element-in-time (FET) (and finite-element-in-space) interpretation. The benefits of this reformulation are discussed, including the derivation of timesteppers that are higher order in time, and the quantifiable dissipative SP properties in the non-ideal, resistive case.
        
        We discuss possible options for extending this FET approach to timesteppers for the compressible case.

        The kinetic corrections satisfy linearized Boltzmann equations. Using a Lénard--Bernstein collision operator, these take Fokker--Planck-like forms \cite{Fokker_1914, Planck_1917} wherein pseudo-particles in the numerical model obey the neoclassical transport equations, with particle-independent Brownian drift terms. This offers a rigorous methodology for incorporating collisions into the particle transport model, without coupling the equations of motions for each particle.
        
        Works by Chen, Chacón et al. \cite{Chen_Chacón_Barnes_2011, Chacón_Chen_Barnes_2013, Chen_Chacón_2014, Chen_Chacón_2015} have developed structure-preserving particle pushers for neoclassical transport in the Vlasov equations, derived from Crank--Nicolson integrators. We show these too can can derive from a FET interpretation, similarly offering potential extensions to higher-order-in-time particle pushers. The FET formulation is used also to consider how the stochastic drift terms can be incorporated into the pushers. Stochastic gyrokinetic expansions are also discussed.

        Different options for the numerical implementation of these schemes are considered.

        Due to the efficacy of FET in the development of SP timesteppers for both the fluid and kinetic component, we hope this approach will prove effective in the future for developing SP timesteppers for the full hybrid model. We hope this will give us the opportunity to incorporate previously inaccessible kinetic effects into the highly effective, modern, finite-element MHD models.
    \end{abstract}
    
    
    \newpage
    \tableofcontents
    
    
    \newpage
    \pagenumbering{arabic}
    %\linenumbers\renewcommand\thelinenumber{\color{black!50}\arabic{linenumber}}
            \input{0 - introduction/main.tex}
        \part{Research}
            \input{1 - low-noise PiC models/main.tex}
            \input{2 - kinetic component/main.tex}
            \input{3 - fluid component/main.tex}
            \input{4 - numerical implementation/main.tex}
        \part{Project Overview}
            \input{5 - research plan/main.tex}
            \input{6 - summary/main.tex}
    
    
    %\section{}
    \newpage
    \pagenumbering{gobble}
        \printbibliography


    \newpage
    \pagenumbering{roman}
    \appendix
        \part{Appendices}
            \input{8 - Hilbert complexes/main.tex}
            \input{9 - weak conservation proofs/main.tex}
\end{document}

            \documentclass[12pt, a4paper]{report}

\input{template/main.tex}

\title{\BA{Title in Progress...}}
\author{Boris Andrews}
\affil{Mathematical Institute, University of Oxford}
\date{\today}


\begin{document}
    \pagenumbering{gobble}
    \maketitle
    
    
    \begin{abstract}
        Magnetic confinement reactors---in particular tokamaks---offer one of the most promising options for achieving practical nuclear fusion, with the potential to provide virtually limitless, clean energy. The theoretical and numerical modeling of tokamak plasmas is simultaneously an essential component of effective reactor design, and a great research barrier. Tokamak operational conditions exhibit comparatively low Knudsen numbers. Kinetic effects, including kinetic waves and instabilities, Landau damping, bump-on-tail instabilities and more, are therefore highly influential in tokamak plasma dynamics. Purely fluid models are inherently incapable of capturing these effects, whereas the high dimensionality in purely kinetic models render them practically intractable for most relevant purposes.

        We consider a $\delta\!f$ decomposition model, with a macroscopic fluid background and microscopic kinetic correction, both fully coupled to each other. A similar manner of discretization is proposed to that used in the recent \texttt{STRUPHY} code \cite{Holderied_Possanner_Wang_2021, Holderied_2022, Li_et_al_2023} with a finite-element model for the background and a pseudo-particle/PiC model for the correction.

        The fluid background satisfies the full, non-linear, resistive, compressible, Hall MHD equations. \cite{Laakmann_Hu_Farrell_2022} introduces finite-element(-in-space) implicit timesteppers for the incompressible analogue to this system with structure-preserving (SP) properties in the ideal case, alongside parameter-robust preconditioners. We show that these timesteppers can derive from a finite-element-in-time (FET) (and finite-element-in-space) interpretation. The benefits of this reformulation are discussed, including the derivation of timesteppers that are higher order in time, and the quantifiable dissipative SP properties in the non-ideal, resistive case.
        
        We discuss possible options for extending this FET approach to timesteppers for the compressible case.

        The kinetic corrections satisfy linearized Boltzmann equations. Using a Lénard--Bernstein collision operator, these take Fokker--Planck-like forms \cite{Fokker_1914, Planck_1917} wherein pseudo-particles in the numerical model obey the neoclassical transport equations, with particle-independent Brownian drift terms. This offers a rigorous methodology for incorporating collisions into the particle transport model, without coupling the equations of motions for each particle.
        
        Works by Chen, Chacón et al. \cite{Chen_Chacón_Barnes_2011, Chacón_Chen_Barnes_2013, Chen_Chacón_2014, Chen_Chacón_2015} have developed structure-preserving particle pushers for neoclassical transport in the Vlasov equations, derived from Crank--Nicolson integrators. We show these too can can derive from a FET interpretation, similarly offering potential extensions to higher-order-in-time particle pushers. The FET formulation is used also to consider how the stochastic drift terms can be incorporated into the pushers. Stochastic gyrokinetic expansions are also discussed.

        Different options for the numerical implementation of these schemes are considered.

        Due to the efficacy of FET in the development of SP timesteppers for both the fluid and kinetic component, we hope this approach will prove effective in the future for developing SP timesteppers for the full hybrid model. We hope this will give us the opportunity to incorporate previously inaccessible kinetic effects into the highly effective, modern, finite-element MHD models.
    \end{abstract}
    
    
    \newpage
    \tableofcontents
    
    
    \newpage
    \pagenumbering{arabic}
    %\linenumbers\renewcommand\thelinenumber{\color{black!50}\arabic{linenumber}}
            \input{0 - introduction/main.tex}
        \part{Research}
            \input{1 - low-noise PiC models/main.tex}
            \input{2 - kinetic component/main.tex}
            \input{3 - fluid component/main.tex}
            \input{4 - numerical implementation/main.tex}
        \part{Project Overview}
            \input{5 - research plan/main.tex}
            \input{6 - summary/main.tex}
    
    
    %\section{}
    \newpage
    \pagenumbering{gobble}
        \printbibliography


    \newpage
    \pagenumbering{roman}
    \appendix
        \part{Appendices}
            \input{8 - Hilbert complexes/main.tex}
            \input{9 - weak conservation proofs/main.tex}
\end{document}

            \documentclass[12pt, a4paper]{report}

\input{template/main.tex}

\title{\BA{Title in Progress...}}
\author{Boris Andrews}
\affil{Mathematical Institute, University of Oxford}
\date{\today}


\begin{document}
    \pagenumbering{gobble}
    \maketitle
    
    
    \begin{abstract}
        Magnetic confinement reactors---in particular tokamaks---offer one of the most promising options for achieving practical nuclear fusion, with the potential to provide virtually limitless, clean energy. The theoretical and numerical modeling of tokamak plasmas is simultaneously an essential component of effective reactor design, and a great research barrier. Tokamak operational conditions exhibit comparatively low Knudsen numbers. Kinetic effects, including kinetic waves and instabilities, Landau damping, bump-on-tail instabilities and more, are therefore highly influential in tokamak plasma dynamics. Purely fluid models are inherently incapable of capturing these effects, whereas the high dimensionality in purely kinetic models render them practically intractable for most relevant purposes.

        We consider a $\delta\!f$ decomposition model, with a macroscopic fluid background and microscopic kinetic correction, both fully coupled to each other. A similar manner of discretization is proposed to that used in the recent \texttt{STRUPHY} code \cite{Holderied_Possanner_Wang_2021, Holderied_2022, Li_et_al_2023} with a finite-element model for the background and a pseudo-particle/PiC model for the correction.

        The fluid background satisfies the full, non-linear, resistive, compressible, Hall MHD equations. \cite{Laakmann_Hu_Farrell_2022} introduces finite-element(-in-space) implicit timesteppers for the incompressible analogue to this system with structure-preserving (SP) properties in the ideal case, alongside parameter-robust preconditioners. We show that these timesteppers can derive from a finite-element-in-time (FET) (and finite-element-in-space) interpretation. The benefits of this reformulation are discussed, including the derivation of timesteppers that are higher order in time, and the quantifiable dissipative SP properties in the non-ideal, resistive case.
        
        We discuss possible options for extending this FET approach to timesteppers for the compressible case.

        The kinetic corrections satisfy linearized Boltzmann equations. Using a Lénard--Bernstein collision operator, these take Fokker--Planck-like forms \cite{Fokker_1914, Planck_1917} wherein pseudo-particles in the numerical model obey the neoclassical transport equations, with particle-independent Brownian drift terms. This offers a rigorous methodology for incorporating collisions into the particle transport model, without coupling the equations of motions for each particle.
        
        Works by Chen, Chacón et al. \cite{Chen_Chacón_Barnes_2011, Chacón_Chen_Barnes_2013, Chen_Chacón_2014, Chen_Chacón_2015} have developed structure-preserving particle pushers for neoclassical transport in the Vlasov equations, derived from Crank--Nicolson integrators. We show these too can can derive from a FET interpretation, similarly offering potential extensions to higher-order-in-time particle pushers. The FET formulation is used also to consider how the stochastic drift terms can be incorporated into the pushers. Stochastic gyrokinetic expansions are also discussed.

        Different options for the numerical implementation of these schemes are considered.

        Due to the efficacy of FET in the development of SP timesteppers for both the fluid and kinetic component, we hope this approach will prove effective in the future for developing SP timesteppers for the full hybrid model. We hope this will give us the opportunity to incorporate previously inaccessible kinetic effects into the highly effective, modern, finite-element MHD models.
    \end{abstract}
    
    
    \newpage
    \tableofcontents
    
    
    \newpage
    \pagenumbering{arabic}
    %\linenumbers\renewcommand\thelinenumber{\color{black!50}\arabic{linenumber}}
            \input{0 - introduction/main.tex}
        \part{Research}
            \input{1 - low-noise PiC models/main.tex}
            \input{2 - kinetic component/main.tex}
            \input{3 - fluid component/main.tex}
            \input{4 - numerical implementation/main.tex}
        \part{Project Overview}
            \input{5 - research plan/main.tex}
            \input{6 - summary/main.tex}
    
    
    %\section{}
    \newpage
    \pagenumbering{gobble}
        \printbibliography


    \newpage
    \pagenumbering{roman}
    \appendix
        \part{Appendices}
            \input{8 - Hilbert complexes/main.tex}
            \input{9 - weak conservation proofs/main.tex}
\end{document}

            \documentclass[12pt, a4paper]{report}

\input{template/main.tex}

\title{\BA{Title in Progress...}}
\author{Boris Andrews}
\affil{Mathematical Institute, University of Oxford}
\date{\today}


\begin{document}
    \pagenumbering{gobble}
    \maketitle
    
    
    \begin{abstract}
        Magnetic confinement reactors---in particular tokamaks---offer one of the most promising options for achieving practical nuclear fusion, with the potential to provide virtually limitless, clean energy. The theoretical and numerical modeling of tokamak plasmas is simultaneously an essential component of effective reactor design, and a great research barrier. Tokamak operational conditions exhibit comparatively low Knudsen numbers. Kinetic effects, including kinetic waves and instabilities, Landau damping, bump-on-tail instabilities and more, are therefore highly influential in tokamak plasma dynamics. Purely fluid models are inherently incapable of capturing these effects, whereas the high dimensionality in purely kinetic models render them practically intractable for most relevant purposes.

        We consider a $\delta\!f$ decomposition model, with a macroscopic fluid background and microscopic kinetic correction, both fully coupled to each other. A similar manner of discretization is proposed to that used in the recent \texttt{STRUPHY} code \cite{Holderied_Possanner_Wang_2021, Holderied_2022, Li_et_al_2023} with a finite-element model for the background and a pseudo-particle/PiC model for the correction.

        The fluid background satisfies the full, non-linear, resistive, compressible, Hall MHD equations. \cite{Laakmann_Hu_Farrell_2022} introduces finite-element(-in-space) implicit timesteppers for the incompressible analogue to this system with structure-preserving (SP) properties in the ideal case, alongside parameter-robust preconditioners. We show that these timesteppers can derive from a finite-element-in-time (FET) (and finite-element-in-space) interpretation. The benefits of this reformulation are discussed, including the derivation of timesteppers that are higher order in time, and the quantifiable dissipative SP properties in the non-ideal, resistive case.
        
        We discuss possible options for extending this FET approach to timesteppers for the compressible case.

        The kinetic corrections satisfy linearized Boltzmann equations. Using a Lénard--Bernstein collision operator, these take Fokker--Planck-like forms \cite{Fokker_1914, Planck_1917} wherein pseudo-particles in the numerical model obey the neoclassical transport equations, with particle-independent Brownian drift terms. This offers a rigorous methodology for incorporating collisions into the particle transport model, without coupling the equations of motions for each particle.
        
        Works by Chen, Chacón et al. \cite{Chen_Chacón_Barnes_2011, Chacón_Chen_Barnes_2013, Chen_Chacón_2014, Chen_Chacón_2015} have developed structure-preserving particle pushers for neoclassical transport in the Vlasov equations, derived from Crank--Nicolson integrators. We show these too can can derive from a FET interpretation, similarly offering potential extensions to higher-order-in-time particle pushers. The FET formulation is used also to consider how the stochastic drift terms can be incorporated into the pushers. Stochastic gyrokinetic expansions are also discussed.

        Different options for the numerical implementation of these schemes are considered.

        Due to the efficacy of FET in the development of SP timesteppers for both the fluid and kinetic component, we hope this approach will prove effective in the future for developing SP timesteppers for the full hybrid model. We hope this will give us the opportunity to incorporate previously inaccessible kinetic effects into the highly effective, modern, finite-element MHD models.
    \end{abstract}
    
    
    \newpage
    \tableofcontents
    
    
    \newpage
    \pagenumbering{arabic}
    %\linenumbers\renewcommand\thelinenumber{\color{black!50}\arabic{linenumber}}
            \input{0 - introduction/main.tex}
        \part{Research}
            \input{1 - low-noise PiC models/main.tex}
            \input{2 - kinetic component/main.tex}
            \input{3 - fluid component/main.tex}
            \input{4 - numerical implementation/main.tex}
        \part{Project Overview}
            \input{5 - research plan/main.tex}
            \input{6 - summary/main.tex}
    
    
    %\section{}
    \newpage
    \pagenumbering{gobble}
        \printbibliography


    \newpage
    \pagenumbering{roman}
    \appendix
        \part{Appendices}
            \input{8 - Hilbert complexes/main.tex}
            \input{9 - weak conservation proofs/main.tex}
\end{document}

        \part{Project Overview}
            \documentclass[12pt, a4paper]{report}

\input{template/main.tex}

\title{\BA{Title in Progress...}}
\author{Boris Andrews}
\affil{Mathematical Institute, University of Oxford}
\date{\today}


\begin{document}
    \pagenumbering{gobble}
    \maketitle
    
    
    \begin{abstract}
        Magnetic confinement reactors---in particular tokamaks---offer one of the most promising options for achieving practical nuclear fusion, with the potential to provide virtually limitless, clean energy. The theoretical and numerical modeling of tokamak plasmas is simultaneously an essential component of effective reactor design, and a great research barrier. Tokamak operational conditions exhibit comparatively low Knudsen numbers. Kinetic effects, including kinetic waves and instabilities, Landau damping, bump-on-tail instabilities and more, are therefore highly influential in tokamak plasma dynamics. Purely fluid models are inherently incapable of capturing these effects, whereas the high dimensionality in purely kinetic models render them practically intractable for most relevant purposes.

        We consider a $\delta\!f$ decomposition model, with a macroscopic fluid background and microscopic kinetic correction, both fully coupled to each other. A similar manner of discretization is proposed to that used in the recent \texttt{STRUPHY} code \cite{Holderied_Possanner_Wang_2021, Holderied_2022, Li_et_al_2023} with a finite-element model for the background and a pseudo-particle/PiC model for the correction.

        The fluid background satisfies the full, non-linear, resistive, compressible, Hall MHD equations. \cite{Laakmann_Hu_Farrell_2022} introduces finite-element(-in-space) implicit timesteppers for the incompressible analogue to this system with structure-preserving (SP) properties in the ideal case, alongside parameter-robust preconditioners. We show that these timesteppers can derive from a finite-element-in-time (FET) (and finite-element-in-space) interpretation. The benefits of this reformulation are discussed, including the derivation of timesteppers that are higher order in time, and the quantifiable dissipative SP properties in the non-ideal, resistive case.
        
        We discuss possible options for extending this FET approach to timesteppers for the compressible case.

        The kinetic corrections satisfy linearized Boltzmann equations. Using a Lénard--Bernstein collision operator, these take Fokker--Planck-like forms \cite{Fokker_1914, Planck_1917} wherein pseudo-particles in the numerical model obey the neoclassical transport equations, with particle-independent Brownian drift terms. This offers a rigorous methodology for incorporating collisions into the particle transport model, without coupling the equations of motions for each particle.
        
        Works by Chen, Chacón et al. \cite{Chen_Chacón_Barnes_2011, Chacón_Chen_Barnes_2013, Chen_Chacón_2014, Chen_Chacón_2015} have developed structure-preserving particle pushers for neoclassical transport in the Vlasov equations, derived from Crank--Nicolson integrators. We show these too can can derive from a FET interpretation, similarly offering potential extensions to higher-order-in-time particle pushers. The FET formulation is used also to consider how the stochastic drift terms can be incorporated into the pushers. Stochastic gyrokinetic expansions are also discussed.

        Different options for the numerical implementation of these schemes are considered.

        Due to the efficacy of FET in the development of SP timesteppers for both the fluid and kinetic component, we hope this approach will prove effective in the future for developing SP timesteppers for the full hybrid model. We hope this will give us the opportunity to incorporate previously inaccessible kinetic effects into the highly effective, modern, finite-element MHD models.
    \end{abstract}
    
    
    \newpage
    \tableofcontents
    
    
    \newpage
    \pagenumbering{arabic}
    %\linenumbers\renewcommand\thelinenumber{\color{black!50}\arabic{linenumber}}
            \input{0 - introduction/main.tex}
        \part{Research}
            \input{1 - low-noise PiC models/main.tex}
            \input{2 - kinetic component/main.tex}
            \input{3 - fluid component/main.tex}
            \input{4 - numerical implementation/main.tex}
        \part{Project Overview}
            \input{5 - research plan/main.tex}
            \input{6 - summary/main.tex}
    
    
    %\section{}
    \newpage
    \pagenumbering{gobble}
        \printbibliography


    \newpage
    \pagenumbering{roman}
    \appendix
        \part{Appendices}
            \input{8 - Hilbert complexes/main.tex}
            \input{9 - weak conservation proofs/main.tex}
\end{document}

            \documentclass[12pt, a4paper]{report}

\input{template/main.tex}

\title{\BA{Title in Progress...}}
\author{Boris Andrews}
\affil{Mathematical Institute, University of Oxford}
\date{\today}


\begin{document}
    \pagenumbering{gobble}
    \maketitle
    
    
    \begin{abstract}
        Magnetic confinement reactors---in particular tokamaks---offer one of the most promising options for achieving practical nuclear fusion, with the potential to provide virtually limitless, clean energy. The theoretical and numerical modeling of tokamak plasmas is simultaneously an essential component of effective reactor design, and a great research barrier. Tokamak operational conditions exhibit comparatively low Knudsen numbers. Kinetic effects, including kinetic waves and instabilities, Landau damping, bump-on-tail instabilities and more, are therefore highly influential in tokamak plasma dynamics. Purely fluid models are inherently incapable of capturing these effects, whereas the high dimensionality in purely kinetic models render them practically intractable for most relevant purposes.

        We consider a $\delta\!f$ decomposition model, with a macroscopic fluid background and microscopic kinetic correction, both fully coupled to each other. A similar manner of discretization is proposed to that used in the recent \texttt{STRUPHY} code \cite{Holderied_Possanner_Wang_2021, Holderied_2022, Li_et_al_2023} with a finite-element model for the background and a pseudo-particle/PiC model for the correction.

        The fluid background satisfies the full, non-linear, resistive, compressible, Hall MHD equations. \cite{Laakmann_Hu_Farrell_2022} introduces finite-element(-in-space) implicit timesteppers for the incompressible analogue to this system with structure-preserving (SP) properties in the ideal case, alongside parameter-robust preconditioners. We show that these timesteppers can derive from a finite-element-in-time (FET) (and finite-element-in-space) interpretation. The benefits of this reformulation are discussed, including the derivation of timesteppers that are higher order in time, and the quantifiable dissipative SP properties in the non-ideal, resistive case.
        
        We discuss possible options for extending this FET approach to timesteppers for the compressible case.

        The kinetic corrections satisfy linearized Boltzmann equations. Using a Lénard--Bernstein collision operator, these take Fokker--Planck-like forms \cite{Fokker_1914, Planck_1917} wherein pseudo-particles in the numerical model obey the neoclassical transport equations, with particle-independent Brownian drift terms. This offers a rigorous methodology for incorporating collisions into the particle transport model, without coupling the equations of motions for each particle.
        
        Works by Chen, Chacón et al. \cite{Chen_Chacón_Barnes_2011, Chacón_Chen_Barnes_2013, Chen_Chacón_2014, Chen_Chacón_2015} have developed structure-preserving particle pushers for neoclassical transport in the Vlasov equations, derived from Crank--Nicolson integrators. We show these too can can derive from a FET interpretation, similarly offering potential extensions to higher-order-in-time particle pushers. The FET formulation is used also to consider how the stochastic drift terms can be incorporated into the pushers. Stochastic gyrokinetic expansions are also discussed.

        Different options for the numerical implementation of these schemes are considered.

        Due to the efficacy of FET in the development of SP timesteppers for both the fluid and kinetic component, we hope this approach will prove effective in the future for developing SP timesteppers for the full hybrid model. We hope this will give us the opportunity to incorporate previously inaccessible kinetic effects into the highly effective, modern, finite-element MHD models.
    \end{abstract}
    
    
    \newpage
    \tableofcontents
    
    
    \newpage
    \pagenumbering{arabic}
    %\linenumbers\renewcommand\thelinenumber{\color{black!50}\arabic{linenumber}}
            \input{0 - introduction/main.tex}
        \part{Research}
            \input{1 - low-noise PiC models/main.tex}
            \input{2 - kinetic component/main.tex}
            \input{3 - fluid component/main.tex}
            \input{4 - numerical implementation/main.tex}
        \part{Project Overview}
            \input{5 - research plan/main.tex}
            \input{6 - summary/main.tex}
    
    
    %\section{}
    \newpage
    \pagenumbering{gobble}
        \printbibliography


    \newpage
    \pagenumbering{roman}
    \appendix
        \part{Appendices}
            \input{8 - Hilbert complexes/main.tex}
            \input{9 - weak conservation proofs/main.tex}
\end{document}

    
    
    %\section{}
    \newpage
    \pagenumbering{gobble}
        \printbibliography


    \newpage
    \pagenumbering{roman}
    \appendix
        \part{Appendices}
            \documentclass[12pt, a4paper]{report}

\input{template/main.tex}

\title{\BA{Title in Progress...}}
\author{Boris Andrews}
\affil{Mathematical Institute, University of Oxford}
\date{\today}


\begin{document}
    \pagenumbering{gobble}
    \maketitle
    
    
    \begin{abstract}
        Magnetic confinement reactors---in particular tokamaks---offer one of the most promising options for achieving practical nuclear fusion, with the potential to provide virtually limitless, clean energy. The theoretical and numerical modeling of tokamak plasmas is simultaneously an essential component of effective reactor design, and a great research barrier. Tokamak operational conditions exhibit comparatively low Knudsen numbers. Kinetic effects, including kinetic waves and instabilities, Landau damping, bump-on-tail instabilities and more, are therefore highly influential in tokamak plasma dynamics. Purely fluid models are inherently incapable of capturing these effects, whereas the high dimensionality in purely kinetic models render them practically intractable for most relevant purposes.

        We consider a $\delta\!f$ decomposition model, with a macroscopic fluid background and microscopic kinetic correction, both fully coupled to each other. A similar manner of discretization is proposed to that used in the recent \texttt{STRUPHY} code \cite{Holderied_Possanner_Wang_2021, Holderied_2022, Li_et_al_2023} with a finite-element model for the background and a pseudo-particle/PiC model for the correction.

        The fluid background satisfies the full, non-linear, resistive, compressible, Hall MHD equations. \cite{Laakmann_Hu_Farrell_2022} introduces finite-element(-in-space) implicit timesteppers for the incompressible analogue to this system with structure-preserving (SP) properties in the ideal case, alongside parameter-robust preconditioners. We show that these timesteppers can derive from a finite-element-in-time (FET) (and finite-element-in-space) interpretation. The benefits of this reformulation are discussed, including the derivation of timesteppers that are higher order in time, and the quantifiable dissipative SP properties in the non-ideal, resistive case.
        
        We discuss possible options for extending this FET approach to timesteppers for the compressible case.

        The kinetic corrections satisfy linearized Boltzmann equations. Using a Lénard--Bernstein collision operator, these take Fokker--Planck-like forms \cite{Fokker_1914, Planck_1917} wherein pseudo-particles in the numerical model obey the neoclassical transport equations, with particle-independent Brownian drift terms. This offers a rigorous methodology for incorporating collisions into the particle transport model, without coupling the equations of motions for each particle.
        
        Works by Chen, Chacón et al. \cite{Chen_Chacón_Barnes_2011, Chacón_Chen_Barnes_2013, Chen_Chacón_2014, Chen_Chacón_2015} have developed structure-preserving particle pushers for neoclassical transport in the Vlasov equations, derived from Crank--Nicolson integrators. We show these too can can derive from a FET interpretation, similarly offering potential extensions to higher-order-in-time particle pushers. The FET formulation is used also to consider how the stochastic drift terms can be incorporated into the pushers. Stochastic gyrokinetic expansions are also discussed.

        Different options for the numerical implementation of these schemes are considered.

        Due to the efficacy of FET in the development of SP timesteppers for both the fluid and kinetic component, we hope this approach will prove effective in the future for developing SP timesteppers for the full hybrid model. We hope this will give us the opportunity to incorporate previously inaccessible kinetic effects into the highly effective, modern, finite-element MHD models.
    \end{abstract}
    
    
    \newpage
    \tableofcontents
    
    
    \newpage
    \pagenumbering{arabic}
    %\linenumbers\renewcommand\thelinenumber{\color{black!50}\arabic{linenumber}}
            \input{0 - introduction/main.tex}
        \part{Research}
            \input{1 - low-noise PiC models/main.tex}
            \input{2 - kinetic component/main.tex}
            \input{3 - fluid component/main.tex}
            \input{4 - numerical implementation/main.tex}
        \part{Project Overview}
            \input{5 - research plan/main.tex}
            \input{6 - summary/main.tex}
    
    
    %\section{}
    \newpage
    \pagenumbering{gobble}
        \printbibliography


    \newpage
    \pagenumbering{roman}
    \appendix
        \part{Appendices}
            \input{8 - Hilbert complexes/main.tex}
            \input{9 - weak conservation proofs/main.tex}
\end{document}

            \documentclass[12pt, a4paper]{report}

\input{template/main.tex}

\title{\BA{Title in Progress...}}
\author{Boris Andrews}
\affil{Mathematical Institute, University of Oxford}
\date{\today}


\begin{document}
    \pagenumbering{gobble}
    \maketitle
    
    
    \begin{abstract}
        Magnetic confinement reactors---in particular tokamaks---offer one of the most promising options for achieving practical nuclear fusion, with the potential to provide virtually limitless, clean energy. The theoretical and numerical modeling of tokamak plasmas is simultaneously an essential component of effective reactor design, and a great research barrier. Tokamak operational conditions exhibit comparatively low Knudsen numbers. Kinetic effects, including kinetic waves and instabilities, Landau damping, bump-on-tail instabilities and more, are therefore highly influential in tokamak plasma dynamics. Purely fluid models are inherently incapable of capturing these effects, whereas the high dimensionality in purely kinetic models render them practically intractable for most relevant purposes.

        We consider a $\delta\!f$ decomposition model, with a macroscopic fluid background and microscopic kinetic correction, both fully coupled to each other. A similar manner of discretization is proposed to that used in the recent \texttt{STRUPHY} code \cite{Holderied_Possanner_Wang_2021, Holderied_2022, Li_et_al_2023} with a finite-element model for the background and a pseudo-particle/PiC model for the correction.

        The fluid background satisfies the full, non-linear, resistive, compressible, Hall MHD equations. \cite{Laakmann_Hu_Farrell_2022} introduces finite-element(-in-space) implicit timesteppers for the incompressible analogue to this system with structure-preserving (SP) properties in the ideal case, alongside parameter-robust preconditioners. We show that these timesteppers can derive from a finite-element-in-time (FET) (and finite-element-in-space) interpretation. The benefits of this reformulation are discussed, including the derivation of timesteppers that are higher order in time, and the quantifiable dissipative SP properties in the non-ideal, resistive case.
        
        We discuss possible options for extending this FET approach to timesteppers for the compressible case.

        The kinetic corrections satisfy linearized Boltzmann equations. Using a Lénard--Bernstein collision operator, these take Fokker--Planck-like forms \cite{Fokker_1914, Planck_1917} wherein pseudo-particles in the numerical model obey the neoclassical transport equations, with particle-independent Brownian drift terms. This offers a rigorous methodology for incorporating collisions into the particle transport model, without coupling the equations of motions for each particle.
        
        Works by Chen, Chacón et al. \cite{Chen_Chacón_Barnes_2011, Chacón_Chen_Barnes_2013, Chen_Chacón_2014, Chen_Chacón_2015} have developed structure-preserving particle pushers for neoclassical transport in the Vlasov equations, derived from Crank--Nicolson integrators. We show these too can can derive from a FET interpretation, similarly offering potential extensions to higher-order-in-time particle pushers. The FET formulation is used also to consider how the stochastic drift terms can be incorporated into the pushers. Stochastic gyrokinetic expansions are also discussed.

        Different options for the numerical implementation of these schemes are considered.

        Due to the efficacy of FET in the development of SP timesteppers for both the fluid and kinetic component, we hope this approach will prove effective in the future for developing SP timesteppers for the full hybrid model. We hope this will give us the opportunity to incorporate previously inaccessible kinetic effects into the highly effective, modern, finite-element MHD models.
    \end{abstract}
    
    
    \newpage
    \tableofcontents
    
    
    \newpage
    \pagenumbering{arabic}
    %\linenumbers\renewcommand\thelinenumber{\color{black!50}\arabic{linenumber}}
            \input{0 - introduction/main.tex}
        \part{Research}
            \input{1 - low-noise PiC models/main.tex}
            \input{2 - kinetic component/main.tex}
            \input{3 - fluid component/main.tex}
            \input{4 - numerical implementation/main.tex}
        \part{Project Overview}
            \input{5 - research plan/main.tex}
            \input{6 - summary/main.tex}
    
    
    %\section{}
    \newpage
    \pagenumbering{gobble}
        \printbibliography


    \newpage
    \pagenumbering{roman}
    \appendix
        \part{Appendices}
            \input{8 - Hilbert complexes/main.tex}
            \input{9 - weak conservation proofs/main.tex}
\end{document}

\end{document}

            \documentclass[12pt, a4paper]{report}

\documentclass[12pt, a4paper]{report}

\input{template/main.tex}

\title{\BA{Title in Progress...}}
\author{Boris Andrews}
\affil{Mathematical Institute, University of Oxford}
\date{\today}


\begin{document}
    \pagenumbering{gobble}
    \maketitle
    
    
    \begin{abstract}
        Magnetic confinement reactors---in particular tokamaks---offer one of the most promising options for achieving practical nuclear fusion, with the potential to provide virtually limitless, clean energy. The theoretical and numerical modeling of tokamak plasmas is simultaneously an essential component of effective reactor design, and a great research barrier. Tokamak operational conditions exhibit comparatively low Knudsen numbers. Kinetic effects, including kinetic waves and instabilities, Landau damping, bump-on-tail instabilities and more, are therefore highly influential in tokamak plasma dynamics. Purely fluid models are inherently incapable of capturing these effects, whereas the high dimensionality in purely kinetic models render them practically intractable for most relevant purposes.

        We consider a $\delta\!f$ decomposition model, with a macroscopic fluid background and microscopic kinetic correction, both fully coupled to each other. A similar manner of discretization is proposed to that used in the recent \texttt{STRUPHY} code \cite{Holderied_Possanner_Wang_2021, Holderied_2022, Li_et_al_2023} with a finite-element model for the background and a pseudo-particle/PiC model for the correction.

        The fluid background satisfies the full, non-linear, resistive, compressible, Hall MHD equations. \cite{Laakmann_Hu_Farrell_2022} introduces finite-element(-in-space) implicit timesteppers for the incompressible analogue to this system with structure-preserving (SP) properties in the ideal case, alongside parameter-robust preconditioners. We show that these timesteppers can derive from a finite-element-in-time (FET) (and finite-element-in-space) interpretation. The benefits of this reformulation are discussed, including the derivation of timesteppers that are higher order in time, and the quantifiable dissipative SP properties in the non-ideal, resistive case.
        
        We discuss possible options for extending this FET approach to timesteppers for the compressible case.

        The kinetic corrections satisfy linearized Boltzmann equations. Using a Lénard--Bernstein collision operator, these take Fokker--Planck-like forms \cite{Fokker_1914, Planck_1917} wherein pseudo-particles in the numerical model obey the neoclassical transport equations, with particle-independent Brownian drift terms. This offers a rigorous methodology for incorporating collisions into the particle transport model, without coupling the equations of motions for each particle.
        
        Works by Chen, Chacón et al. \cite{Chen_Chacón_Barnes_2011, Chacón_Chen_Barnes_2013, Chen_Chacón_2014, Chen_Chacón_2015} have developed structure-preserving particle pushers for neoclassical transport in the Vlasov equations, derived from Crank--Nicolson integrators. We show these too can can derive from a FET interpretation, similarly offering potential extensions to higher-order-in-time particle pushers. The FET formulation is used also to consider how the stochastic drift terms can be incorporated into the pushers. Stochastic gyrokinetic expansions are also discussed.

        Different options for the numerical implementation of these schemes are considered.

        Due to the efficacy of FET in the development of SP timesteppers for both the fluid and kinetic component, we hope this approach will prove effective in the future for developing SP timesteppers for the full hybrid model. We hope this will give us the opportunity to incorporate previously inaccessible kinetic effects into the highly effective, modern, finite-element MHD models.
    \end{abstract}
    
    
    \newpage
    \tableofcontents
    
    
    \newpage
    \pagenumbering{arabic}
    %\linenumbers\renewcommand\thelinenumber{\color{black!50}\arabic{linenumber}}
            \input{0 - introduction/main.tex}
        \part{Research}
            \input{1 - low-noise PiC models/main.tex}
            \input{2 - kinetic component/main.tex}
            \input{3 - fluid component/main.tex}
            \input{4 - numerical implementation/main.tex}
        \part{Project Overview}
            \input{5 - research plan/main.tex}
            \input{6 - summary/main.tex}
    
    
    %\section{}
    \newpage
    \pagenumbering{gobble}
        \printbibliography


    \newpage
    \pagenumbering{roman}
    \appendix
        \part{Appendices}
            \input{8 - Hilbert complexes/main.tex}
            \input{9 - weak conservation proofs/main.tex}
\end{document}


\title{\BA{Title in Progress...}}
\author{Boris Andrews}
\affil{Mathematical Institute, University of Oxford}
\date{\today}


\begin{document}
    \pagenumbering{gobble}
    \maketitle
    
    
    \begin{abstract}
        Magnetic confinement reactors---in particular tokamaks---offer one of the most promising options for achieving practical nuclear fusion, with the potential to provide virtually limitless, clean energy. The theoretical and numerical modeling of tokamak plasmas is simultaneously an essential component of effective reactor design, and a great research barrier. Tokamak operational conditions exhibit comparatively low Knudsen numbers. Kinetic effects, including kinetic waves and instabilities, Landau damping, bump-on-tail instabilities and more, are therefore highly influential in tokamak plasma dynamics. Purely fluid models are inherently incapable of capturing these effects, whereas the high dimensionality in purely kinetic models render them practically intractable for most relevant purposes.

        We consider a $\delta\!f$ decomposition model, with a macroscopic fluid background and microscopic kinetic correction, both fully coupled to each other. A similar manner of discretization is proposed to that used in the recent \texttt{STRUPHY} code \cite{Holderied_Possanner_Wang_2021, Holderied_2022, Li_et_al_2023} with a finite-element model for the background and a pseudo-particle/PiC model for the correction.

        The fluid background satisfies the full, non-linear, resistive, compressible, Hall MHD equations. \cite{Laakmann_Hu_Farrell_2022} introduces finite-element(-in-space) implicit timesteppers for the incompressible analogue to this system with structure-preserving (SP) properties in the ideal case, alongside parameter-robust preconditioners. We show that these timesteppers can derive from a finite-element-in-time (FET) (and finite-element-in-space) interpretation. The benefits of this reformulation are discussed, including the derivation of timesteppers that are higher order in time, and the quantifiable dissipative SP properties in the non-ideal, resistive case.
        
        We discuss possible options for extending this FET approach to timesteppers for the compressible case.

        The kinetic corrections satisfy linearized Boltzmann equations. Using a Lénard--Bernstein collision operator, these take Fokker--Planck-like forms \cite{Fokker_1914, Planck_1917} wherein pseudo-particles in the numerical model obey the neoclassical transport equations, with particle-independent Brownian drift terms. This offers a rigorous methodology for incorporating collisions into the particle transport model, without coupling the equations of motions for each particle.
        
        Works by Chen, Chacón et al. \cite{Chen_Chacón_Barnes_2011, Chacón_Chen_Barnes_2013, Chen_Chacón_2014, Chen_Chacón_2015} have developed structure-preserving particle pushers for neoclassical transport in the Vlasov equations, derived from Crank--Nicolson integrators. We show these too can can derive from a FET interpretation, similarly offering potential extensions to higher-order-in-time particle pushers. The FET formulation is used also to consider how the stochastic drift terms can be incorporated into the pushers. Stochastic gyrokinetic expansions are also discussed.

        Different options for the numerical implementation of these schemes are considered.

        Due to the efficacy of FET in the development of SP timesteppers for both the fluid and kinetic component, we hope this approach will prove effective in the future for developing SP timesteppers for the full hybrid model. We hope this will give us the opportunity to incorporate previously inaccessible kinetic effects into the highly effective, modern, finite-element MHD models.
    \end{abstract}
    
    
    \newpage
    \tableofcontents
    
    
    \newpage
    \pagenumbering{arabic}
    %\linenumbers\renewcommand\thelinenumber{\color{black!50}\arabic{linenumber}}
            \documentclass[12pt, a4paper]{report}

\input{template/main.tex}

\title{\BA{Title in Progress...}}
\author{Boris Andrews}
\affil{Mathematical Institute, University of Oxford}
\date{\today}


\begin{document}
    \pagenumbering{gobble}
    \maketitle
    
    
    \begin{abstract}
        Magnetic confinement reactors---in particular tokamaks---offer one of the most promising options for achieving practical nuclear fusion, with the potential to provide virtually limitless, clean energy. The theoretical and numerical modeling of tokamak plasmas is simultaneously an essential component of effective reactor design, and a great research barrier. Tokamak operational conditions exhibit comparatively low Knudsen numbers. Kinetic effects, including kinetic waves and instabilities, Landau damping, bump-on-tail instabilities and more, are therefore highly influential in tokamak plasma dynamics. Purely fluid models are inherently incapable of capturing these effects, whereas the high dimensionality in purely kinetic models render them practically intractable for most relevant purposes.

        We consider a $\delta\!f$ decomposition model, with a macroscopic fluid background and microscopic kinetic correction, both fully coupled to each other. A similar manner of discretization is proposed to that used in the recent \texttt{STRUPHY} code \cite{Holderied_Possanner_Wang_2021, Holderied_2022, Li_et_al_2023} with a finite-element model for the background and a pseudo-particle/PiC model for the correction.

        The fluid background satisfies the full, non-linear, resistive, compressible, Hall MHD equations. \cite{Laakmann_Hu_Farrell_2022} introduces finite-element(-in-space) implicit timesteppers for the incompressible analogue to this system with structure-preserving (SP) properties in the ideal case, alongside parameter-robust preconditioners. We show that these timesteppers can derive from a finite-element-in-time (FET) (and finite-element-in-space) interpretation. The benefits of this reformulation are discussed, including the derivation of timesteppers that are higher order in time, and the quantifiable dissipative SP properties in the non-ideal, resistive case.
        
        We discuss possible options for extending this FET approach to timesteppers for the compressible case.

        The kinetic corrections satisfy linearized Boltzmann equations. Using a Lénard--Bernstein collision operator, these take Fokker--Planck-like forms \cite{Fokker_1914, Planck_1917} wherein pseudo-particles in the numerical model obey the neoclassical transport equations, with particle-independent Brownian drift terms. This offers a rigorous methodology for incorporating collisions into the particle transport model, without coupling the equations of motions for each particle.
        
        Works by Chen, Chacón et al. \cite{Chen_Chacón_Barnes_2011, Chacón_Chen_Barnes_2013, Chen_Chacón_2014, Chen_Chacón_2015} have developed structure-preserving particle pushers for neoclassical transport in the Vlasov equations, derived from Crank--Nicolson integrators. We show these too can can derive from a FET interpretation, similarly offering potential extensions to higher-order-in-time particle pushers. The FET formulation is used also to consider how the stochastic drift terms can be incorporated into the pushers. Stochastic gyrokinetic expansions are also discussed.

        Different options for the numerical implementation of these schemes are considered.

        Due to the efficacy of FET in the development of SP timesteppers for both the fluid and kinetic component, we hope this approach will prove effective in the future for developing SP timesteppers for the full hybrid model. We hope this will give us the opportunity to incorporate previously inaccessible kinetic effects into the highly effective, modern, finite-element MHD models.
    \end{abstract}
    
    
    \newpage
    \tableofcontents
    
    
    \newpage
    \pagenumbering{arabic}
    %\linenumbers\renewcommand\thelinenumber{\color{black!50}\arabic{linenumber}}
            \input{0 - introduction/main.tex}
        \part{Research}
            \input{1 - low-noise PiC models/main.tex}
            \input{2 - kinetic component/main.tex}
            \input{3 - fluid component/main.tex}
            \input{4 - numerical implementation/main.tex}
        \part{Project Overview}
            \input{5 - research plan/main.tex}
            \input{6 - summary/main.tex}
    
    
    %\section{}
    \newpage
    \pagenumbering{gobble}
        \printbibliography


    \newpage
    \pagenumbering{roman}
    \appendix
        \part{Appendices}
            \input{8 - Hilbert complexes/main.tex}
            \input{9 - weak conservation proofs/main.tex}
\end{document}

        \part{Research}
            \documentclass[12pt, a4paper]{report}

\input{template/main.tex}

\title{\BA{Title in Progress...}}
\author{Boris Andrews}
\affil{Mathematical Institute, University of Oxford}
\date{\today}


\begin{document}
    \pagenumbering{gobble}
    \maketitle
    
    
    \begin{abstract}
        Magnetic confinement reactors---in particular tokamaks---offer one of the most promising options for achieving practical nuclear fusion, with the potential to provide virtually limitless, clean energy. The theoretical and numerical modeling of tokamak plasmas is simultaneously an essential component of effective reactor design, and a great research barrier. Tokamak operational conditions exhibit comparatively low Knudsen numbers. Kinetic effects, including kinetic waves and instabilities, Landau damping, bump-on-tail instabilities and more, are therefore highly influential in tokamak plasma dynamics. Purely fluid models are inherently incapable of capturing these effects, whereas the high dimensionality in purely kinetic models render them practically intractable for most relevant purposes.

        We consider a $\delta\!f$ decomposition model, with a macroscopic fluid background and microscopic kinetic correction, both fully coupled to each other. A similar manner of discretization is proposed to that used in the recent \texttt{STRUPHY} code \cite{Holderied_Possanner_Wang_2021, Holderied_2022, Li_et_al_2023} with a finite-element model for the background and a pseudo-particle/PiC model for the correction.

        The fluid background satisfies the full, non-linear, resistive, compressible, Hall MHD equations. \cite{Laakmann_Hu_Farrell_2022} introduces finite-element(-in-space) implicit timesteppers for the incompressible analogue to this system with structure-preserving (SP) properties in the ideal case, alongside parameter-robust preconditioners. We show that these timesteppers can derive from a finite-element-in-time (FET) (and finite-element-in-space) interpretation. The benefits of this reformulation are discussed, including the derivation of timesteppers that are higher order in time, and the quantifiable dissipative SP properties in the non-ideal, resistive case.
        
        We discuss possible options for extending this FET approach to timesteppers for the compressible case.

        The kinetic corrections satisfy linearized Boltzmann equations. Using a Lénard--Bernstein collision operator, these take Fokker--Planck-like forms \cite{Fokker_1914, Planck_1917} wherein pseudo-particles in the numerical model obey the neoclassical transport equations, with particle-independent Brownian drift terms. This offers a rigorous methodology for incorporating collisions into the particle transport model, without coupling the equations of motions for each particle.
        
        Works by Chen, Chacón et al. \cite{Chen_Chacón_Barnes_2011, Chacón_Chen_Barnes_2013, Chen_Chacón_2014, Chen_Chacón_2015} have developed structure-preserving particle pushers for neoclassical transport in the Vlasov equations, derived from Crank--Nicolson integrators. We show these too can can derive from a FET interpretation, similarly offering potential extensions to higher-order-in-time particle pushers. The FET formulation is used also to consider how the stochastic drift terms can be incorporated into the pushers. Stochastic gyrokinetic expansions are also discussed.

        Different options for the numerical implementation of these schemes are considered.

        Due to the efficacy of FET in the development of SP timesteppers for both the fluid and kinetic component, we hope this approach will prove effective in the future for developing SP timesteppers for the full hybrid model. We hope this will give us the opportunity to incorporate previously inaccessible kinetic effects into the highly effective, modern, finite-element MHD models.
    \end{abstract}
    
    
    \newpage
    \tableofcontents
    
    
    \newpage
    \pagenumbering{arabic}
    %\linenumbers\renewcommand\thelinenumber{\color{black!50}\arabic{linenumber}}
            \input{0 - introduction/main.tex}
        \part{Research}
            \input{1 - low-noise PiC models/main.tex}
            \input{2 - kinetic component/main.tex}
            \input{3 - fluid component/main.tex}
            \input{4 - numerical implementation/main.tex}
        \part{Project Overview}
            \input{5 - research plan/main.tex}
            \input{6 - summary/main.tex}
    
    
    %\section{}
    \newpage
    \pagenumbering{gobble}
        \printbibliography


    \newpage
    \pagenumbering{roman}
    \appendix
        \part{Appendices}
            \input{8 - Hilbert complexes/main.tex}
            \input{9 - weak conservation proofs/main.tex}
\end{document}

            \documentclass[12pt, a4paper]{report}

\input{template/main.tex}

\title{\BA{Title in Progress...}}
\author{Boris Andrews}
\affil{Mathematical Institute, University of Oxford}
\date{\today}


\begin{document}
    \pagenumbering{gobble}
    \maketitle
    
    
    \begin{abstract}
        Magnetic confinement reactors---in particular tokamaks---offer one of the most promising options for achieving practical nuclear fusion, with the potential to provide virtually limitless, clean energy. The theoretical and numerical modeling of tokamak plasmas is simultaneously an essential component of effective reactor design, and a great research barrier. Tokamak operational conditions exhibit comparatively low Knudsen numbers. Kinetic effects, including kinetic waves and instabilities, Landau damping, bump-on-tail instabilities and more, are therefore highly influential in tokamak plasma dynamics. Purely fluid models are inherently incapable of capturing these effects, whereas the high dimensionality in purely kinetic models render them practically intractable for most relevant purposes.

        We consider a $\delta\!f$ decomposition model, with a macroscopic fluid background and microscopic kinetic correction, both fully coupled to each other. A similar manner of discretization is proposed to that used in the recent \texttt{STRUPHY} code \cite{Holderied_Possanner_Wang_2021, Holderied_2022, Li_et_al_2023} with a finite-element model for the background and a pseudo-particle/PiC model for the correction.

        The fluid background satisfies the full, non-linear, resistive, compressible, Hall MHD equations. \cite{Laakmann_Hu_Farrell_2022} introduces finite-element(-in-space) implicit timesteppers for the incompressible analogue to this system with structure-preserving (SP) properties in the ideal case, alongside parameter-robust preconditioners. We show that these timesteppers can derive from a finite-element-in-time (FET) (and finite-element-in-space) interpretation. The benefits of this reformulation are discussed, including the derivation of timesteppers that are higher order in time, and the quantifiable dissipative SP properties in the non-ideal, resistive case.
        
        We discuss possible options for extending this FET approach to timesteppers for the compressible case.

        The kinetic corrections satisfy linearized Boltzmann equations. Using a Lénard--Bernstein collision operator, these take Fokker--Planck-like forms \cite{Fokker_1914, Planck_1917} wherein pseudo-particles in the numerical model obey the neoclassical transport equations, with particle-independent Brownian drift terms. This offers a rigorous methodology for incorporating collisions into the particle transport model, without coupling the equations of motions for each particle.
        
        Works by Chen, Chacón et al. \cite{Chen_Chacón_Barnes_2011, Chacón_Chen_Barnes_2013, Chen_Chacón_2014, Chen_Chacón_2015} have developed structure-preserving particle pushers for neoclassical transport in the Vlasov equations, derived from Crank--Nicolson integrators. We show these too can can derive from a FET interpretation, similarly offering potential extensions to higher-order-in-time particle pushers. The FET formulation is used also to consider how the stochastic drift terms can be incorporated into the pushers. Stochastic gyrokinetic expansions are also discussed.

        Different options for the numerical implementation of these schemes are considered.

        Due to the efficacy of FET in the development of SP timesteppers for both the fluid and kinetic component, we hope this approach will prove effective in the future for developing SP timesteppers for the full hybrid model. We hope this will give us the opportunity to incorporate previously inaccessible kinetic effects into the highly effective, modern, finite-element MHD models.
    \end{abstract}
    
    
    \newpage
    \tableofcontents
    
    
    \newpage
    \pagenumbering{arabic}
    %\linenumbers\renewcommand\thelinenumber{\color{black!50}\arabic{linenumber}}
            \input{0 - introduction/main.tex}
        \part{Research}
            \input{1 - low-noise PiC models/main.tex}
            \input{2 - kinetic component/main.tex}
            \input{3 - fluid component/main.tex}
            \input{4 - numerical implementation/main.tex}
        \part{Project Overview}
            \input{5 - research plan/main.tex}
            \input{6 - summary/main.tex}
    
    
    %\section{}
    \newpage
    \pagenumbering{gobble}
        \printbibliography


    \newpage
    \pagenumbering{roman}
    \appendix
        \part{Appendices}
            \input{8 - Hilbert complexes/main.tex}
            \input{9 - weak conservation proofs/main.tex}
\end{document}

            \documentclass[12pt, a4paper]{report}

\input{template/main.tex}

\title{\BA{Title in Progress...}}
\author{Boris Andrews}
\affil{Mathematical Institute, University of Oxford}
\date{\today}


\begin{document}
    \pagenumbering{gobble}
    \maketitle
    
    
    \begin{abstract}
        Magnetic confinement reactors---in particular tokamaks---offer one of the most promising options for achieving practical nuclear fusion, with the potential to provide virtually limitless, clean energy. The theoretical and numerical modeling of tokamak plasmas is simultaneously an essential component of effective reactor design, and a great research barrier. Tokamak operational conditions exhibit comparatively low Knudsen numbers. Kinetic effects, including kinetic waves and instabilities, Landau damping, bump-on-tail instabilities and more, are therefore highly influential in tokamak plasma dynamics. Purely fluid models are inherently incapable of capturing these effects, whereas the high dimensionality in purely kinetic models render them practically intractable for most relevant purposes.

        We consider a $\delta\!f$ decomposition model, with a macroscopic fluid background and microscopic kinetic correction, both fully coupled to each other. A similar manner of discretization is proposed to that used in the recent \texttt{STRUPHY} code \cite{Holderied_Possanner_Wang_2021, Holderied_2022, Li_et_al_2023} with a finite-element model for the background and a pseudo-particle/PiC model for the correction.

        The fluid background satisfies the full, non-linear, resistive, compressible, Hall MHD equations. \cite{Laakmann_Hu_Farrell_2022} introduces finite-element(-in-space) implicit timesteppers for the incompressible analogue to this system with structure-preserving (SP) properties in the ideal case, alongside parameter-robust preconditioners. We show that these timesteppers can derive from a finite-element-in-time (FET) (and finite-element-in-space) interpretation. The benefits of this reformulation are discussed, including the derivation of timesteppers that are higher order in time, and the quantifiable dissipative SP properties in the non-ideal, resistive case.
        
        We discuss possible options for extending this FET approach to timesteppers for the compressible case.

        The kinetic corrections satisfy linearized Boltzmann equations. Using a Lénard--Bernstein collision operator, these take Fokker--Planck-like forms \cite{Fokker_1914, Planck_1917} wherein pseudo-particles in the numerical model obey the neoclassical transport equations, with particle-independent Brownian drift terms. This offers a rigorous methodology for incorporating collisions into the particle transport model, without coupling the equations of motions for each particle.
        
        Works by Chen, Chacón et al. \cite{Chen_Chacón_Barnes_2011, Chacón_Chen_Barnes_2013, Chen_Chacón_2014, Chen_Chacón_2015} have developed structure-preserving particle pushers for neoclassical transport in the Vlasov equations, derived from Crank--Nicolson integrators. We show these too can can derive from a FET interpretation, similarly offering potential extensions to higher-order-in-time particle pushers. The FET formulation is used also to consider how the stochastic drift terms can be incorporated into the pushers. Stochastic gyrokinetic expansions are also discussed.

        Different options for the numerical implementation of these schemes are considered.

        Due to the efficacy of FET in the development of SP timesteppers for both the fluid and kinetic component, we hope this approach will prove effective in the future for developing SP timesteppers for the full hybrid model. We hope this will give us the opportunity to incorporate previously inaccessible kinetic effects into the highly effective, modern, finite-element MHD models.
    \end{abstract}
    
    
    \newpage
    \tableofcontents
    
    
    \newpage
    \pagenumbering{arabic}
    %\linenumbers\renewcommand\thelinenumber{\color{black!50}\arabic{linenumber}}
            \input{0 - introduction/main.tex}
        \part{Research}
            \input{1 - low-noise PiC models/main.tex}
            \input{2 - kinetic component/main.tex}
            \input{3 - fluid component/main.tex}
            \input{4 - numerical implementation/main.tex}
        \part{Project Overview}
            \input{5 - research plan/main.tex}
            \input{6 - summary/main.tex}
    
    
    %\section{}
    \newpage
    \pagenumbering{gobble}
        \printbibliography


    \newpage
    \pagenumbering{roman}
    \appendix
        \part{Appendices}
            \input{8 - Hilbert complexes/main.tex}
            \input{9 - weak conservation proofs/main.tex}
\end{document}

            \documentclass[12pt, a4paper]{report}

\input{template/main.tex}

\title{\BA{Title in Progress...}}
\author{Boris Andrews}
\affil{Mathematical Institute, University of Oxford}
\date{\today}


\begin{document}
    \pagenumbering{gobble}
    \maketitle
    
    
    \begin{abstract}
        Magnetic confinement reactors---in particular tokamaks---offer one of the most promising options for achieving practical nuclear fusion, with the potential to provide virtually limitless, clean energy. The theoretical and numerical modeling of tokamak plasmas is simultaneously an essential component of effective reactor design, and a great research barrier. Tokamak operational conditions exhibit comparatively low Knudsen numbers. Kinetic effects, including kinetic waves and instabilities, Landau damping, bump-on-tail instabilities and more, are therefore highly influential in tokamak plasma dynamics. Purely fluid models are inherently incapable of capturing these effects, whereas the high dimensionality in purely kinetic models render them practically intractable for most relevant purposes.

        We consider a $\delta\!f$ decomposition model, with a macroscopic fluid background and microscopic kinetic correction, both fully coupled to each other. A similar manner of discretization is proposed to that used in the recent \texttt{STRUPHY} code \cite{Holderied_Possanner_Wang_2021, Holderied_2022, Li_et_al_2023} with a finite-element model for the background and a pseudo-particle/PiC model for the correction.

        The fluid background satisfies the full, non-linear, resistive, compressible, Hall MHD equations. \cite{Laakmann_Hu_Farrell_2022} introduces finite-element(-in-space) implicit timesteppers for the incompressible analogue to this system with structure-preserving (SP) properties in the ideal case, alongside parameter-robust preconditioners. We show that these timesteppers can derive from a finite-element-in-time (FET) (and finite-element-in-space) interpretation. The benefits of this reformulation are discussed, including the derivation of timesteppers that are higher order in time, and the quantifiable dissipative SP properties in the non-ideal, resistive case.
        
        We discuss possible options for extending this FET approach to timesteppers for the compressible case.

        The kinetic corrections satisfy linearized Boltzmann equations. Using a Lénard--Bernstein collision operator, these take Fokker--Planck-like forms \cite{Fokker_1914, Planck_1917} wherein pseudo-particles in the numerical model obey the neoclassical transport equations, with particle-independent Brownian drift terms. This offers a rigorous methodology for incorporating collisions into the particle transport model, without coupling the equations of motions for each particle.
        
        Works by Chen, Chacón et al. \cite{Chen_Chacón_Barnes_2011, Chacón_Chen_Barnes_2013, Chen_Chacón_2014, Chen_Chacón_2015} have developed structure-preserving particle pushers for neoclassical transport in the Vlasov equations, derived from Crank--Nicolson integrators. We show these too can can derive from a FET interpretation, similarly offering potential extensions to higher-order-in-time particle pushers. The FET formulation is used also to consider how the stochastic drift terms can be incorporated into the pushers. Stochastic gyrokinetic expansions are also discussed.

        Different options for the numerical implementation of these schemes are considered.

        Due to the efficacy of FET in the development of SP timesteppers for both the fluid and kinetic component, we hope this approach will prove effective in the future for developing SP timesteppers for the full hybrid model. We hope this will give us the opportunity to incorporate previously inaccessible kinetic effects into the highly effective, modern, finite-element MHD models.
    \end{abstract}
    
    
    \newpage
    \tableofcontents
    
    
    \newpage
    \pagenumbering{arabic}
    %\linenumbers\renewcommand\thelinenumber{\color{black!50}\arabic{linenumber}}
            \input{0 - introduction/main.tex}
        \part{Research}
            \input{1 - low-noise PiC models/main.tex}
            \input{2 - kinetic component/main.tex}
            \input{3 - fluid component/main.tex}
            \input{4 - numerical implementation/main.tex}
        \part{Project Overview}
            \input{5 - research plan/main.tex}
            \input{6 - summary/main.tex}
    
    
    %\section{}
    \newpage
    \pagenumbering{gobble}
        \printbibliography


    \newpage
    \pagenumbering{roman}
    \appendix
        \part{Appendices}
            \input{8 - Hilbert complexes/main.tex}
            \input{9 - weak conservation proofs/main.tex}
\end{document}

        \part{Project Overview}
            \documentclass[12pt, a4paper]{report}

\input{template/main.tex}

\title{\BA{Title in Progress...}}
\author{Boris Andrews}
\affil{Mathematical Institute, University of Oxford}
\date{\today}


\begin{document}
    \pagenumbering{gobble}
    \maketitle
    
    
    \begin{abstract}
        Magnetic confinement reactors---in particular tokamaks---offer one of the most promising options for achieving practical nuclear fusion, with the potential to provide virtually limitless, clean energy. The theoretical and numerical modeling of tokamak plasmas is simultaneously an essential component of effective reactor design, and a great research barrier. Tokamak operational conditions exhibit comparatively low Knudsen numbers. Kinetic effects, including kinetic waves and instabilities, Landau damping, bump-on-tail instabilities and more, are therefore highly influential in tokamak plasma dynamics. Purely fluid models are inherently incapable of capturing these effects, whereas the high dimensionality in purely kinetic models render them practically intractable for most relevant purposes.

        We consider a $\delta\!f$ decomposition model, with a macroscopic fluid background and microscopic kinetic correction, both fully coupled to each other. A similar manner of discretization is proposed to that used in the recent \texttt{STRUPHY} code \cite{Holderied_Possanner_Wang_2021, Holderied_2022, Li_et_al_2023} with a finite-element model for the background and a pseudo-particle/PiC model for the correction.

        The fluid background satisfies the full, non-linear, resistive, compressible, Hall MHD equations. \cite{Laakmann_Hu_Farrell_2022} introduces finite-element(-in-space) implicit timesteppers for the incompressible analogue to this system with structure-preserving (SP) properties in the ideal case, alongside parameter-robust preconditioners. We show that these timesteppers can derive from a finite-element-in-time (FET) (and finite-element-in-space) interpretation. The benefits of this reformulation are discussed, including the derivation of timesteppers that are higher order in time, and the quantifiable dissipative SP properties in the non-ideal, resistive case.
        
        We discuss possible options for extending this FET approach to timesteppers for the compressible case.

        The kinetic corrections satisfy linearized Boltzmann equations. Using a Lénard--Bernstein collision operator, these take Fokker--Planck-like forms \cite{Fokker_1914, Planck_1917} wherein pseudo-particles in the numerical model obey the neoclassical transport equations, with particle-independent Brownian drift terms. This offers a rigorous methodology for incorporating collisions into the particle transport model, without coupling the equations of motions for each particle.
        
        Works by Chen, Chacón et al. \cite{Chen_Chacón_Barnes_2011, Chacón_Chen_Barnes_2013, Chen_Chacón_2014, Chen_Chacón_2015} have developed structure-preserving particle pushers for neoclassical transport in the Vlasov equations, derived from Crank--Nicolson integrators. We show these too can can derive from a FET interpretation, similarly offering potential extensions to higher-order-in-time particle pushers. The FET formulation is used also to consider how the stochastic drift terms can be incorporated into the pushers. Stochastic gyrokinetic expansions are also discussed.

        Different options for the numerical implementation of these schemes are considered.

        Due to the efficacy of FET in the development of SP timesteppers for both the fluid and kinetic component, we hope this approach will prove effective in the future for developing SP timesteppers for the full hybrid model. We hope this will give us the opportunity to incorporate previously inaccessible kinetic effects into the highly effective, modern, finite-element MHD models.
    \end{abstract}
    
    
    \newpage
    \tableofcontents
    
    
    \newpage
    \pagenumbering{arabic}
    %\linenumbers\renewcommand\thelinenumber{\color{black!50}\arabic{linenumber}}
            \input{0 - introduction/main.tex}
        \part{Research}
            \input{1 - low-noise PiC models/main.tex}
            \input{2 - kinetic component/main.tex}
            \input{3 - fluid component/main.tex}
            \input{4 - numerical implementation/main.tex}
        \part{Project Overview}
            \input{5 - research plan/main.tex}
            \input{6 - summary/main.tex}
    
    
    %\section{}
    \newpage
    \pagenumbering{gobble}
        \printbibliography


    \newpage
    \pagenumbering{roman}
    \appendix
        \part{Appendices}
            \input{8 - Hilbert complexes/main.tex}
            \input{9 - weak conservation proofs/main.tex}
\end{document}

            \documentclass[12pt, a4paper]{report}

\input{template/main.tex}

\title{\BA{Title in Progress...}}
\author{Boris Andrews}
\affil{Mathematical Institute, University of Oxford}
\date{\today}


\begin{document}
    \pagenumbering{gobble}
    \maketitle
    
    
    \begin{abstract}
        Magnetic confinement reactors---in particular tokamaks---offer one of the most promising options for achieving practical nuclear fusion, with the potential to provide virtually limitless, clean energy. The theoretical and numerical modeling of tokamak plasmas is simultaneously an essential component of effective reactor design, and a great research barrier. Tokamak operational conditions exhibit comparatively low Knudsen numbers. Kinetic effects, including kinetic waves and instabilities, Landau damping, bump-on-tail instabilities and more, are therefore highly influential in tokamak plasma dynamics. Purely fluid models are inherently incapable of capturing these effects, whereas the high dimensionality in purely kinetic models render them practically intractable for most relevant purposes.

        We consider a $\delta\!f$ decomposition model, with a macroscopic fluid background and microscopic kinetic correction, both fully coupled to each other. A similar manner of discretization is proposed to that used in the recent \texttt{STRUPHY} code \cite{Holderied_Possanner_Wang_2021, Holderied_2022, Li_et_al_2023} with a finite-element model for the background and a pseudo-particle/PiC model for the correction.

        The fluid background satisfies the full, non-linear, resistive, compressible, Hall MHD equations. \cite{Laakmann_Hu_Farrell_2022} introduces finite-element(-in-space) implicit timesteppers for the incompressible analogue to this system with structure-preserving (SP) properties in the ideal case, alongside parameter-robust preconditioners. We show that these timesteppers can derive from a finite-element-in-time (FET) (and finite-element-in-space) interpretation. The benefits of this reformulation are discussed, including the derivation of timesteppers that are higher order in time, and the quantifiable dissipative SP properties in the non-ideal, resistive case.
        
        We discuss possible options for extending this FET approach to timesteppers for the compressible case.

        The kinetic corrections satisfy linearized Boltzmann equations. Using a Lénard--Bernstein collision operator, these take Fokker--Planck-like forms \cite{Fokker_1914, Planck_1917} wherein pseudo-particles in the numerical model obey the neoclassical transport equations, with particle-independent Brownian drift terms. This offers a rigorous methodology for incorporating collisions into the particle transport model, without coupling the equations of motions for each particle.
        
        Works by Chen, Chacón et al. \cite{Chen_Chacón_Barnes_2011, Chacón_Chen_Barnes_2013, Chen_Chacón_2014, Chen_Chacón_2015} have developed structure-preserving particle pushers for neoclassical transport in the Vlasov equations, derived from Crank--Nicolson integrators. We show these too can can derive from a FET interpretation, similarly offering potential extensions to higher-order-in-time particle pushers. The FET formulation is used also to consider how the stochastic drift terms can be incorporated into the pushers. Stochastic gyrokinetic expansions are also discussed.

        Different options for the numerical implementation of these schemes are considered.

        Due to the efficacy of FET in the development of SP timesteppers for both the fluid and kinetic component, we hope this approach will prove effective in the future for developing SP timesteppers for the full hybrid model. We hope this will give us the opportunity to incorporate previously inaccessible kinetic effects into the highly effective, modern, finite-element MHD models.
    \end{abstract}
    
    
    \newpage
    \tableofcontents
    
    
    \newpage
    \pagenumbering{arabic}
    %\linenumbers\renewcommand\thelinenumber{\color{black!50}\arabic{linenumber}}
            \input{0 - introduction/main.tex}
        \part{Research}
            \input{1 - low-noise PiC models/main.tex}
            \input{2 - kinetic component/main.tex}
            \input{3 - fluid component/main.tex}
            \input{4 - numerical implementation/main.tex}
        \part{Project Overview}
            \input{5 - research plan/main.tex}
            \input{6 - summary/main.tex}
    
    
    %\section{}
    \newpage
    \pagenumbering{gobble}
        \printbibliography


    \newpage
    \pagenumbering{roman}
    \appendix
        \part{Appendices}
            \input{8 - Hilbert complexes/main.tex}
            \input{9 - weak conservation proofs/main.tex}
\end{document}

    
    
    %\section{}
    \newpage
    \pagenumbering{gobble}
        \printbibliography


    \newpage
    \pagenumbering{roman}
    \appendix
        \part{Appendices}
            \documentclass[12pt, a4paper]{report}

\input{template/main.tex}

\title{\BA{Title in Progress...}}
\author{Boris Andrews}
\affil{Mathematical Institute, University of Oxford}
\date{\today}


\begin{document}
    \pagenumbering{gobble}
    \maketitle
    
    
    \begin{abstract}
        Magnetic confinement reactors---in particular tokamaks---offer one of the most promising options for achieving practical nuclear fusion, with the potential to provide virtually limitless, clean energy. The theoretical and numerical modeling of tokamak plasmas is simultaneously an essential component of effective reactor design, and a great research barrier. Tokamak operational conditions exhibit comparatively low Knudsen numbers. Kinetic effects, including kinetic waves and instabilities, Landau damping, bump-on-tail instabilities and more, are therefore highly influential in tokamak plasma dynamics. Purely fluid models are inherently incapable of capturing these effects, whereas the high dimensionality in purely kinetic models render them practically intractable for most relevant purposes.

        We consider a $\delta\!f$ decomposition model, with a macroscopic fluid background and microscopic kinetic correction, both fully coupled to each other. A similar manner of discretization is proposed to that used in the recent \texttt{STRUPHY} code \cite{Holderied_Possanner_Wang_2021, Holderied_2022, Li_et_al_2023} with a finite-element model for the background and a pseudo-particle/PiC model for the correction.

        The fluid background satisfies the full, non-linear, resistive, compressible, Hall MHD equations. \cite{Laakmann_Hu_Farrell_2022} introduces finite-element(-in-space) implicit timesteppers for the incompressible analogue to this system with structure-preserving (SP) properties in the ideal case, alongside parameter-robust preconditioners. We show that these timesteppers can derive from a finite-element-in-time (FET) (and finite-element-in-space) interpretation. The benefits of this reformulation are discussed, including the derivation of timesteppers that are higher order in time, and the quantifiable dissipative SP properties in the non-ideal, resistive case.
        
        We discuss possible options for extending this FET approach to timesteppers for the compressible case.

        The kinetic corrections satisfy linearized Boltzmann equations. Using a Lénard--Bernstein collision operator, these take Fokker--Planck-like forms \cite{Fokker_1914, Planck_1917} wherein pseudo-particles in the numerical model obey the neoclassical transport equations, with particle-independent Brownian drift terms. This offers a rigorous methodology for incorporating collisions into the particle transport model, without coupling the equations of motions for each particle.
        
        Works by Chen, Chacón et al. \cite{Chen_Chacón_Barnes_2011, Chacón_Chen_Barnes_2013, Chen_Chacón_2014, Chen_Chacón_2015} have developed structure-preserving particle pushers for neoclassical transport in the Vlasov equations, derived from Crank--Nicolson integrators. We show these too can can derive from a FET interpretation, similarly offering potential extensions to higher-order-in-time particle pushers. The FET formulation is used also to consider how the stochastic drift terms can be incorporated into the pushers. Stochastic gyrokinetic expansions are also discussed.

        Different options for the numerical implementation of these schemes are considered.

        Due to the efficacy of FET in the development of SP timesteppers for both the fluid and kinetic component, we hope this approach will prove effective in the future for developing SP timesteppers for the full hybrid model. We hope this will give us the opportunity to incorporate previously inaccessible kinetic effects into the highly effective, modern, finite-element MHD models.
    \end{abstract}
    
    
    \newpage
    \tableofcontents
    
    
    \newpage
    \pagenumbering{arabic}
    %\linenumbers\renewcommand\thelinenumber{\color{black!50}\arabic{linenumber}}
            \input{0 - introduction/main.tex}
        \part{Research}
            \input{1 - low-noise PiC models/main.tex}
            \input{2 - kinetic component/main.tex}
            \input{3 - fluid component/main.tex}
            \input{4 - numerical implementation/main.tex}
        \part{Project Overview}
            \input{5 - research plan/main.tex}
            \input{6 - summary/main.tex}
    
    
    %\section{}
    \newpage
    \pagenumbering{gobble}
        \printbibliography


    \newpage
    \pagenumbering{roman}
    \appendix
        \part{Appendices}
            \input{8 - Hilbert complexes/main.tex}
            \input{9 - weak conservation proofs/main.tex}
\end{document}

            \documentclass[12pt, a4paper]{report}

\input{template/main.tex}

\title{\BA{Title in Progress...}}
\author{Boris Andrews}
\affil{Mathematical Institute, University of Oxford}
\date{\today}


\begin{document}
    \pagenumbering{gobble}
    \maketitle
    
    
    \begin{abstract}
        Magnetic confinement reactors---in particular tokamaks---offer one of the most promising options for achieving practical nuclear fusion, with the potential to provide virtually limitless, clean energy. The theoretical and numerical modeling of tokamak plasmas is simultaneously an essential component of effective reactor design, and a great research barrier. Tokamak operational conditions exhibit comparatively low Knudsen numbers. Kinetic effects, including kinetic waves and instabilities, Landau damping, bump-on-tail instabilities and more, are therefore highly influential in tokamak plasma dynamics. Purely fluid models are inherently incapable of capturing these effects, whereas the high dimensionality in purely kinetic models render them practically intractable for most relevant purposes.

        We consider a $\delta\!f$ decomposition model, with a macroscopic fluid background and microscopic kinetic correction, both fully coupled to each other. A similar manner of discretization is proposed to that used in the recent \texttt{STRUPHY} code \cite{Holderied_Possanner_Wang_2021, Holderied_2022, Li_et_al_2023} with a finite-element model for the background and a pseudo-particle/PiC model for the correction.

        The fluid background satisfies the full, non-linear, resistive, compressible, Hall MHD equations. \cite{Laakmann_Hu_Farrell_2022} introduces finite-element(-in-space) implicit timesteppers for the incompressible analogue to this system with structure-preserving (SP) properties in the ideal case, alongside parameter-robust preconditioners. We show that these timesteppers can derive from a finite-element-in-time (FET) (and finite-element-in-space) interpretation. The benefits of this reformulation are discussed, including the derivation of timesteppers that are higher order in time, and the quantifiable dissipative SP properties in the non-ideal, resistive case.
        
        We discuss possible options for extending this FET approach to timesteppers for the compressible case.

        The kinetic corrections satisfy linearized Boltzmann equations. Using a Lénard--Bernstein collision operator, these take Fokker--Planck-like forms \cite{Fokker_1914, Planck_1917} wherein pseudo-particles in the numerical model obey the neoclassical transport equations, with particle-independent Brownian drift terms. This offers a rigorous methodology for incorporating collisions into the particle transport model, without coupling the equations of motions for each particle.
        
        Works by Chen, Chacón et al. \cite{Chen_Chacón_Barnes_2011, Chacón_Chen_Barnes_2013, Chen_Chacón_2014, Chen_Chacón_2015} have developed structure-preserving particle pushers for neoclassical transport in the Vlasov equations, derived from Crank--Nicolson integrators. We show these too can can derive from a FET interpretation, similarly offering potential extensions to higher-order-in-time particle pushers. The FET formulation is used also to consider how the stochastic drift terms can be incorporated into the pushers. Stochastic gyrokinetic expansions are also discussed.

        Different options for the numerical implementation of these schemes are considered.

        Due to the efficacy of FET in the development of SP timesteppers for both the fluid and kinetic component, we hope this approach will prove effective in the future for developing SP timesteppers for the full hybrid model. We hope this will give us the opportunity to incorporate previously inaccessible kinetic effects into the highly effective, modern, finite-element MHD models.
    \end{abstract}
    
    
    \newpage
    \tableofcontents
    
    
    \newpage
    \pagenumbering{arabic}
    %\linenumbers\renewcommand\thelinenumber{\color{black!50}\arabic{linenumber}}
            \input{0 - introduction/main.tex}
        \part{Research}
            \input{1 - low-noise PiC models/main.tex}
            \input{2 - kinetic component/main.tex}
            \input{3 - fluid component/main.tex}
            \input{4 - numerical implementation/main.tex}
        \part{Project Overview}
            \input{5 - research plan/main.tex}
            \input{6 - summary/main.tex}
    
    
    %\section{}
    \newpage
    \pagenumbering{gobble}
        \printbibliography


    \newpage
    \pagenumbering{roman}
    \appendix
        \part{Appendices}
            \input{8 - Hilbert complexes/main.tex}
            \input{9 - weak conservation proofs/main.tex}
\end{document}

\end{document}

    
    
    %\section{}
    \newpage
    \pagenumbering{gobble}
        \printbibliography


    \newpage
    \pagenumbering{roman}
    \appendix
        \part{Appendices}
            \documentclass[12pt, a4paper]{report}

\documentclass[12pt, a4paper]{report}

\input{template/main.tex}

\title{\BA{Title in Progress...}}
\author{Boris Andrews}
\affil{Mathematical Institute, University of Oxford}
\date{\today}


\begin{document}
    \pagenumbering{gobble}
    \maketitle
    
    
    \begin{abstract}
        Magnetic confinement reactors---in particular tokamaks---offer one of the most promising options for achieving practical nuclear fusion, with the potential to provide virtually limitless, clean energy. The theoretical and numerical modeling of tokamak plasmas is simultaneously an essential component of effective reactor design, and a great research barrier. Tokamak operational conditions exhibit comparatively low Knudsen numbers. Kinetic effects, including kinetic waves and instabilities, Landau damping, bump-on-tail instabilities and more, are therefore highly influential in tokamak plasma dynamics. Purely fluid models are inherently incapable of capturing these effects, whereas the high dimensionality in purely kinetic models render them practically intractable for most relevant purposes.

        We consider a $\delta\!f$ decomposition model, with a macroscopic fluid background and microscopic kinetic correction, both fully coupled to each other. A similar manner of discretization is proposed to that used in the recent \texttt{STRUPHY} code \cite{Holderied_Possanner_Wang_2021, Holderied_2022, Li_et_al_2023} with a finite-element model for the background and a pseudo-particle/PiC model for the correction.

        The fluid background satisfies the full, non-linear, resistive, compressible, Hall MHD equations. \cite{Laakmann_Hu_Farrell_2022} introduces finite-element(-in-space) implicit timesteppers for the incompressible analogue to this system with structure-preserving (SP) properties in the ideal case, alongside parameter-robust preconditioners. We show that these timesteppers can derive from a finite-element-in-time (FET) (and finite-element-in-space) interpretation. The benefits of this reformulation are discussed, including the derivation of timesteppers that are higher order in time, and the quantifiable dissipative SP properties in the non-ideal, resistive case.
        
        We discuss possible options for extending this FET approach to timesteppers for the compressible case.

        The kinetic corrections satisfy linearized Boltzmann equations. Using a Lénard--Bernstein collision operator, these take Fokker--Planck-like forms \cite{Fokker_1914, Planck_1917} wherein pseudo-particles in the numerical model obey the neoclassical transport equations, with particle-independent Brownian drift terms. This offers a rigorous methodology for incorporating collisions into the particle transport model, without coupling the equations of motions for each particle.
        
        Works by Chen, Chacón et al. \cite{Chen_Chacón_Barnes_2011, Chacón_Chen_Barnes_2013, Chen_Chacón_2014, Chen_Chacón_2015} have developed structure-preserving particle pushers for neoclassical transport in the Vlasov equations, derived from Crank--Nicolson integrators. We show these too can can derive from a FET interpretation, similarly offering potential extensions to higher-order-in-time particle pushers. The FET formulation is used also to consider how the stochastic drift terms can be incorporated into the pushers. Stochastic gyrokinetic expansions are also discussed.

        Different options for the numerical implementation of these schemes are considered.

        Due to the efficacy of FET in the development of SP timesteppers for both the fluid and kinetic component, we hope this approach will prove effective in the future for developing SP timesteppers for the full hybrid model. We hope this will give us the opportunity to incorporate previously inaccessible kinetic effects into the highly effective, modern, finite-element MHD models.
    \end{abstract}
    
    
    \newpage
    \tableofcontents
    
    
    \newpage
    \pagenumbering{arabic}
    %\linenumbers\renewcommand\thelinenumber{\color{black!50}\arabic{linenumber}}
            \input{0 - introduction/main.tex}
        \part{Research}
            \input{1 - low-noise PiC models/main.tex}
            \input{2 - kinetic component/main.tex}
            \input{3 - fluid component/main.tex}
            \input{4 - numerical implementation/main.tex}
        \part{Project Overview}
            \input{5 - research plan/main.tex}
            \input{6 - summary/main.tex}
    
    
    %\section{}
    \newpage
    \pagenumbering{gobble}
        \printbibliography


    \newpage
    \pagenumbering{roman}
    \appendix
        \part{Appendices}
            \input{8 - Hilbert complexes/main.tex}
            \input{9 - weak conservation proofs/main.tex}
\end{document}


\title{\BA{Title in Progress...}}
\author{Boris Andrews}
\affil{Mathematical Institute, University of Oxford}
\date{\today}


\begin{document}
    \pagenumbering{gobble}
    \maketitle
    
    
    \begin{abstract}
        Magnetic confinement reactors---in particular tokamaks---offer one of the most promising options for achieving practical nuclear fusion, with the potential to provide virtually limitless, clean energy. The theoretical and numerical modeling of tokamak plasmas is simultaneously an essential component of effective reactor design, and a great research barrier. Tokamak operational conditions exhibit comparatively low Knudsen numbers. Kinetic effects, including kinetic waves and instabilities, Landau damping, bump-on-tail instabilities and more, are therefore highly influential in tokamak plasma dynamics. Purely fluid models are inherently incapable of capturing these effects, whereas the high dimensionality in purely kinetic models render them practically intractable for most relevant purposes.

        We consider a $\delta\!f$ decomposition model, with a macroscopic fluid background and microscopic kinetic correction, both fully coupled to each other. A similar manner of discretization is proposed to that used in the recent \texttt{STRUPHY} code \cite{Holderied_Possanner_Wang_2021, Holderied_2022, Li_et_al_2023} with a finite-element model for the background and a pseudo-particle/PiC model for the correction.

        The fluid background satisfies the full, non-linear, resistive, compressible, Hall MHD equations. \cite{Laakmann_Hu_Farrell_2022} introduces finite-element(-in-space) implicit timesteppers for the incompressible analogue to this system with structure-preserving (SP) properties in the ideal case, alongside parameter-robust preconditioners. We show that these timesteppers can derive from a finite-element-in-time (FET) (and finite-element-in-space) interpretation. The benefits of this reformulation are discussed, including the derivation of timesteppers that are higher order in time, and the quantifiable dissipative SP properties in the non-ideal, resistive case.
        
        We discuss possible options for extending this FET approach to timesteppers for the compressible case.

        The kinetic corrections satisfy linearized Boltzmann equations. Using a Lénard--Bernstein collision operator, these take Fokker--Planck-like forms \cite{Fokker_1914, Planck_1917} wherein pseudo-particles in the numerical model obey the neoclassical transport equations, with particle-independent Brownian drift terms. This offers a rigorous methodology for incorporating collisions into the particle transport model, without coupling the equations of motions for each particle.
        
        Works by Chen, Chacón et al. \cite{Chen_Chacón_Barnes_2011, Chacón_Chen_Barnes_2013, Chen_Chacón_2014, Chen_Chacón_2015} have developed structure-preserving particle pushers for neoclassical transport in the Vlasov equations, derived from Crank--Nicolson integrators. We show these too can can derive from a FET interpretation, similarly offering potential extensions to higher-order-in-time particle pushers. The FET formulation is used also to consider how the stochastic drift terms can be incorporated into the pushers. Stochastic gyrokinetic expansions are also discussed.

        Different options for the numerical implementation of these schemes are considered.

        Due to the efficacy of FET in the development of SP timesteppers for both the fluid and kinetic component, we hope this approach will prove effective in the future for developing SP timesteppers for the full hybrid model. We hope this will give us the opportunity to incorporate previously inaccessible kinetic effects into the highly effective, modern, finite-element MHD models.
    \end{abstract}
    
    
    \newpage
    \tableofcontents
    
    
    \newpage
    \pagenumbering{arabic}
    %\linenumbers\renewcommand\thelinenumber{\color{black!50}\arabic{linenumber}}
            \documentclass[12pt, a4paper]{report}

\input{template/main.tex}

\title{\BA{Title in Progress...}}
\author{Boris Andrews}
\affil{Mathematical Institute, University of Oxford}
\date{\today}


\begin{document}
    \pagenumbering{gobble}
    \maketitle
    
    
    \begin{abstract}
        Magnetic confinement reactors---in particular tokamaks---offer one of the most promising options for achieving practical nuclear fusion, with the potential to provide virtually limitless, clean energy. The theoretical and numerical modeling of tokamak plasmas is simultaneously an essential component of effective reactor design, and a great research barrier. Tokamak operational conditions exhibit comparatively low Knudsen numbers. Kinetic effects, including kinetic waves and instabilities, Landau damping, bump-on-tail instabilities and more, are therefore highly influential in tokamak plasma dynamics. Purely fluid models are inherently incapable of capturing these effects, whereas the high dimensionality in purely kinetic models render them practically intractable for most relevant purposes.

        We consider a $\delta\!f$ decomposition model, with a macroscopic fluid background and microscopic kinetic correction, both fully coupled to each other. A similar manner of discretization is proposed to that used in the recent \texttt{STRUPHY} code \cite{Holderied_Possanner_Wang_2021, Holderied_2022, Li_et_al_2023} with a finite-element model for the background and a pseudo-particle/PiC model for the correction.

        The fluid background satisfies the full, non-linear, resistive, compressible, Hall MHD equations. \cite{Laakmann_Hu_Farrell_2022} introduces finite-element(-in-space) implicit timesteppers for the incompressible analogue to this system with structure-preserving (SP) properties in the ideal case, alongside parameter-robust preconditioners. We show that these timesteppers can derive from a finite-element-in-time (FET) (and finite-element-in-space) interpretation. The benefits of this reformulation are discussed, including the derivation of timesteppers that are higher order in time, and the quantifiable dissipative SP properties in the non-ideal, resistive case.
        
        We discuss possible options for extending this FET approach to timesteppers for the compressible case.

        The kinetic corrections satisfy linearized Boltzmann equations. Using a Lénard--Bernstein collision operator, these take Fokker--Planck-like forms \cite{Fokker_1914, Planck_1917} wherein pseudo-particles in the numerical model obey the neoclassical transport equations, with particle-independent Brownian drift terms. This offers a rigorous methodology for incorporating collisions into the particle transport model, without coupling the equations of motions for each particle.
        
        Works by Chen, Chacón et al. \cite{Chen_Chacón_Barnes_2011, Chacón_Chen_Barnes_2013, Chen_Chacón_2014, Chen_Chacón_2015} have developed structure-preserving particle pushers for neoclassical transport in the Vlasov equations, derived from Crank--Nicolson integrators. We show these too can can derive from a FET interpretation, similarly offering potential extensions to higher-order-in-time particle pushers. The FET formulation is used also to consider how the stochastic drift terms can be incorporated into the pushers. Stochastic gyrokinetic expansions are also discussed.

        Different options for the numerical implementation of these schemes are considered.

        Due to the efficacy of FET in the development of SP timesteppers for both the fluid and kinetic component, we hope this approach will prove effective in the future for developing SP timesteppers for the full hybrid model. We hope this will give us the opportunity to incorporate previously inaccessible kinetic effects into the highly effective, modern, finite-element MHD models.
    \end{abstract}
    
    
    \newpage
    \tableofcontents
    
    
    \newpage
    \pagenumbering{arabic}
    %\linenumbers\renewcommand\thelinenumber{\color{black!50}\arabic{linenumber}}
            \input{0 - introduction/main.tex}
        \part{Research}
            \input{1 - low-noise PiC models/main.tex}
            \input{2 - kinetic component/main.tex}
            \input{3 - fluid component/main.tex}
            \input{4 - numerical implementation/main.tex}
        \part{Project Overview}
            \input{5 - research plan/main.tex}
            \input{6 - summary/main.tex}
    
    
    %\section{}
    \newpage
    \pagenumbering{gobble}
        \printbibliography


    \newpage
    \pagenumbering{roman}
    \appendix
        \part{Appendices}
            \input{8 - Hilbert complexes/main.tex}
            \input{9 - weak conservation proofs/main.tex}
\end{document}

        \part{Research}
            \documentclass[12pt, a4paper]{report}

\input{template/main.tex}

\title{\BA{Title in Progress...}}
\author{Boris Andrews}
\affil{Mathematical Institute, University of Oxford}
\date{\today}


\begin{document}
    \pagenumbering{gobble}
    \maketitle
    
    
    \begin{abstract}
        Magnetic confinement reactors---in particular tokamaks---offer one of the most promising options for achieving practical nuclear fusion, with the potential to provide virtually limitless, clean energy. The theoretical and numerical modeling of tokamak plasmas is simultaneously an essential component of effective reactor design, and a great research barrier. Tokamak operational conditions exhibit comparatively low Knudsen numbers. Kinetic effects, including kinetic waves and instabilities, Landau damping, bump-on-tail instabilities and more, are therefore highly influential in tokamak plasma dynamics. Purely fluid models are inherently incapable of capturing these effects, whereas the high dimensionality in purely kinetic models render them practically intractable for most relevant purposes.

        We consider a $\delta\!f$ decomposition model, with a macroscopic fluid background and microscopic kinetic correction, both fully coupled to each other. A similar manner of discretization is proposed to that used in the recent \texttt{STRUPHY} code \cite{Holderied_Possanner_Wang_2021, Holderied_2022, Li_et_al_2023} with a finite-element model for the background and a pseudo-particle/PiC model for the correction.

        The fluid background satisfies the full, non-linear, resistive, compressible, Hall MHD equations. \cite{Laakmann_Hu_Farrell_2022} introduces finite-element(-in-space) implicit timesteppers for the incompressible analogue to this system with structure-preserving (SP) properties in the ideal case, alongside parameter-robust preconditioners. We show that these timesteppers can derive from a finite-element-in-time (FET) (and finite-element-in-space) interpretation. The benefits of this reformulation are discussed, including the derivation of timesteppers that are higher order in time, and the quantifiable dissipative SP properties in the non-ideal, resistive case.
        
        We discuss possible options for extending this FET approach to timesteppers for the compressible case.

        The kinetic corrections satisfy linearized Boltzmann equations. Using a Lénard--Bernstein collision operator, these take Fokker--Planck-like forms \cite{Fokker_1914, Planck_1917} wherein pseudo-particles in the numerical model obey the neoclassical transport equations, with particle-independent Brownian drift terms. This offers a rigorous methodology for incorporating collisions into the particle transport model, without coupling the equations of motions for each particle.
        
        Works by Chen, Chacón et al. \cite{Chen_Chacón_Barnes_2011, Chacón_Chen_Barnes_2013, Chen_Chacón_2014, Chen_Chacón_2015} have developed structure-preserving particle pushers for neoclassical transport in the Vlasov equations, derived from Crank--Nicolson integrators. We show these too can can derive from a FET interpretation, similarly offering potential extensions to higher-order-in-time particle pushers. The FET formulation is used also to consider how the stochastic drift terms can be incorporated into the pushers. Stochastic gyrokinetic expansions are also discussed.

        Different options for the numerical implementation of these schemes are considered.

        Due to the efficacy of FET in the development of SP timesteppers for both the fluid and kinetic component, we hope this approach will prove effective in the future for developing SP timesteppers for the full hybrid model. We hope this will give us the opportunity to incorporate previously inaccessible kinetic effects into the highly effective, modern, finite-element MHD models.
    \end{abstract}
    
    
    \newpage
    \tableofcontents
    
    
    \newpage
    \pagenumbering{arabic}
    %\linenumbers\renewcommand\thelinenumber{\color{black!50}\arabic{linenumber}}
            \input{0 - introduction/main.tex}
        \part{Research}
            \input{1 - low-noise PiC models/main.tex}
            \input{2 - kinetic component/main.tex}
            \input{3 - fluid component/main.tex}
            \input{4 - numerical implementation/main.tex}
        \part{Project Overview}
            \input{5 - research plan/main.tex}
            \input{6 - summary/main.tex}
    
    
    %\section{}
    \newpage
    \pagenumbering{gobble}
        \printbibliography


    \newpage
    \pagenumbering{roman}
    \appendix
        \part{Appendices}
            \input{8 - Hilbert complexes/main.tex}
            \input{9 - weak conservation proofs/main.tex}
\end{document}

            \documentclass[12pt, a4paper]{report}

\input{template/main.tex}

\title{\BA{Title in Progress...}}
\author{Boris Andrews}
\affil{Mathematical Institute, University of Oxford}
\date{\today}


\begin{document}
    \pagenumbering{gobble}
    \maketitle
    
    
    \begin{abstract}
        Magnetic confinement reactors---in particular tokamaks---offer one of the most promising options for achieving practical nuclear fusion, with the potential to provide virtually limitless, clean energy. The theoretical and numerical modeling of tokamak plasmas is simultaneously an essential component of effective reactor design, and a great research barrier. Tokamak operational conditions exhibit comparatively low Knudsen numbers. Kinetic effects, including kinetic waves and instabilities, Landau damping, bump-on-tail instabilities and more, are therefore highly influential in tokamak plasma dynamics. Purely fluid models are inherently incapable of capturing these effects, whereas the high dimensionality in purely kinetic models render them practically intractable for most relevant purposes.

        We consider a $\delta\!f$ decomposition model, with a macroscopic fluid background and microscopic kinetic correction, both fully coupled to each other. A similar manner of discretization is proposed to that used in the recent \texttt{STRUPHY} code \cite{Holderied_Possanner_Wang_2021, Holderied_2022, Li_et_al_2023} with a finite-element model for the background and a pseudo-particle/PiC model for the correction.

        The fluid background satisfies the full, non-linear, resistive, compressible, Hall MHD equations. \cite{Laakmann_Hu_Farrell_2022} introduces finite-element(-in-space) implicit timesteppers for the incompressible analogue to this system with structure-preserving (SP) properties in the ideal case, alongside parameter-robust preconditioners. We show that these timesteppers can derive from a finite-element-in-time (FET) (and finite-element-in-space) interpretation. The benefits of this reformulation are discussed, including the derivation of timesteppers that are higher order in time, and the quantifiable dissipative SP properties in the non-ideal, resistive case.
        
        We discuss possible options for extending this FET approach to timesteppers for the compressible case.

        The kinetic corrections satisfy linearized Boltzmann equations. Using a Lénard--Bernstein collision operator, these take Fokker--Planck-like forms \cite{Fokker_1914, Planck_1917} wherein pseudo-particles in the numerical model obey the neoclassical transport equations, with particle-independent Brownian drift terms. This offers a rigorous methodology for incorporating collisions into the particle transport model, without coupling the equations of motions for each particle.
        
        Works by Chen, Chacón et al. \cite{Chen_Chacón_Barnes_2011, Chacón_Chen_Barnes_2013, Chen_Chacón_2014, Chen_Chacón_2015} have developed structure-preserving particle pushers for neoclassical transport in the Vlasov equations, derived from Crank--Nicolson integrators. We show these too can can derive from a FET interpretation, similarly offering potential extensions to higher-order-in-time particle pushers. The FET formulation is used also to consider how the stochastic drift terms can be incorporated into the pushers. Stochastic gyrokinetic expansions are also discussed.

        Different options for the numerical implementation of these schemes are considered.

        Due to the efficacy of FET in the development of SP timesteppers for both the fluid and kinetic component, we hope this approach will prove effective in the future for developing SP timesteppers for the full hybrid model. We hope this will give us the opportunity to incorporate previously inaccessible kinetic effects into the highly effective, modern, finite-element MHD models.
    \end{abstract}
    
    
    \newpage
    \tableofcontents
    
    
    \newpage
    \pagenumbering{arabic}
    %\linenumbers\renewcommand\thelinenumber{\color{black!50}\arabic{linenumber}}
            \input{0 - introduction/main.tex}
        \part{Research}
            \input{1 - low-noise PiC models/main.tex}
            \input{2 - kinetic component/main.tex}
            \input{3 - fluid component/main.tex}
            \input{4 - numerical implementation/main.tex}
        \part{Project Overview}
            \input{5 - research plan/main.tex}
            \input{6 - summary/main.tex}
    
    
    %\section{}
    \newpage
    \pagenumbering{gobble}
        \printbibliography


    \newpage
    \pagenumbering{roman}
    \appendix
        \part{Appendices}
            \input{8 - Hilbert complexes/main.tex}
            \input{9 - weak conservation proofs/main.tex}
\end{document}

            \documentclass[12pt, a4paper]{report}

\input{template/main.tex}

\title{\BA{Title in Progress...}}
\author{Boris Andrews}
\affil{Mathematical Institute, University of Oxford}
\date{\today}


\begin{document}
    \pagenumbering{gobble}
    \maketitle
    
    
    \begin{abstract}
        Magnetic confinement reactors---in particular tokamaks---offer one of the most promising options for achieving practical nuclear fusion, with the potential to provide virtually limitless, clean energy. The theoretical and numerical modeling of tokamak plasmas is simultaneously an essential component of effective reactor design, and a great research barrier. Tokamak operational conditions exhibit comparatively low Knudsen numbers. Kinetic effects, including kinetic waves and instabilities, Landau damping, bump-on-tail instabilities and more, are therefore highly influential in tokamak plasma dynamics. Purely fluid models are inherently incapable of capturing these effects, whereas the high dimensionality in purely kinetic models render them practically intractable for most relevant purposes.

        We consider a $\delta\!f$ decomposition model, with a macroscopic fluid background and microscopic kinetic correction, both fully coupled to each other. A similar manner of discretization is proposed to that used in the recent \texttt{STRUPHY} code \cite{Holderied_Possanner_Wang_2021, Holderied_2022, Li_et_al_2023} with a finite-element model for the background and a pseudo-particle/PiC model for the correction.

        The fluid background satisfies the full, non-linear, resistive, compressible, Hall MHD equations. \cite{Laakmann_Hu_Farrell_2022} introduces finite-element(-in-space) implicit timesteppers for the incompressible analogue to this system with structure-preserving (SP) properties in the ideal case, alongside parameter-robust preconditioners. We show that these timesteppers can derive from a finite-element-in-time (FET) (and finite-element-in-space) interpretation. The benefits of this reformulation are discussed, including the derivation of timesteppers that are higher order in time, and the quantifiable dissipative SP properties in the non-ideal, resistive case.
        
        We discuss possible options for extending this FET approach to timesteppers for the compressible case.

        The kinetic corrections satisfy linearized Boltzmann equations. Using a Lénard--Bernstein collision operator, these take Fokker--Planck-like forms \cite{Fokker_1914, Planck_1917} wherein pseudo-particles in the numerical model obey the neoclassical transport equations, with particle-independent Brownian drift terms. This offers a rigorous methodology for incorporating collisions into the particle transport model, without coupling the equations of motions for each particle.
        
        Works by Chen, Chacón et al. \cite{Chen_Chacón_Barnes_2011, Chacón_Chen_Barnes_2013, Chen_Chacón_2014, Chen_Chacón_2015} have developed structure-preserving particle pushers for neoclassical transport in the Vlasov equations, derived from Crank--Nicolson integrators. We show these too can can derive from a FET interpretation, similarly offering potential extensions to higher-order-in-time particle pushers. The FET formulation is used also to consider how the stochastic drift terms can be incorporated into the pushers. Stochastic gyrokinetic expansions are also discussed.

        Different options for the numerical implementation of these schemes are considered.

        Due to the efficacy of FET in the development of SP timesteppers for both the fluid and kinetic component, we hope this approach will prove effective in the future for developing SP timesteppers for the full hybrid model. We hope this will give us the opportunity to incorporate previously inaccessible kinetic effects into the highly effective, modern, finite-element MHD models.
    \end{abstract}
    
    
    \newpage
    \tableofcontents
    
    
    \newpage
    \pagenumbering{arabic}
    %\linenumbers\renewcommand\thelinenumber{\color{black!50}\arabic{linenumber}}
            \input{0 - introduction/main.tex}
        \part{Research}
            \input{1 - low-noise PiC models/main.tex}
            \input{2 - kinetic component/main.tex}
            \input{3 - fluid component/main.tex}
            \input{4 - numerical implementation/main.tex}
        \part{Project Overview}
            \input{5 - research plan/main.tex}
            \input{6 - summary/main.tex}
    
    
    %\section{}
    \newpage
    \pagenumbering{gobble}
        \printbibliography


    \newpage
    \pagenumbering{roman}
    \appendix
        \part{Appendices}
            \input{8 - Hilbert complexes/main.tex}
            \input{9 - weak conservation proofs/main.tex}
\end{document}

            \documentclass[12pt, a4paper]{report}

\input{template/main.tex}

\title{\BA{Title in Progress...}}
\author{Boris Andrews}
\affil{Mathematical Institute, University of Oxford}
\date{\today}


\begin{document}
    \pagenumbering{gobble}
    \maketitle
    
    
    \begin{abstract}
        Magnetic confinement reactors---in particular tokamaks---offer one of the most promising options for achieving practical nuclear fusion, with the potential to provide virtually limitless, clean energy. The theoretical and numerical modeling of tokamak plasmas is simultaneously an essential component of effective reactor design, and a great research barrier. Tokamak operational conditions exhibit comparatively low Knudsen numbers. Kinetic effects, including kinetic waves and instabilities, Landau damping, bump-on-tail instabilities and more, are therefore highly influential in tokamak plasma dynamics. Purely fluid models are inherently incapable of capturing these effects, whereas the high dimensionality in purely kinetic models render them practically intractable for most relevant purposes.

        We consider a $\delta\!f$ decomposition model, with a macroscopic fluid background and microscopic kinetic correction, both fully coupled to each other. A similar manner of discretization is proposed to that used in the recent \texttt{STRUPHY} code \cite{Holderied_Possanner_Wang_2021, Holderied_2022, Li_et_al_2023} with a finite-element model for the background and a pseudo-particle/PiC model for the correction.

        The fluid background satisfies the full, non-linear, resistive, compressible, Hall MHD equations. \cite{Laakmann_Hu_Farrell_2022} introduces finite-element(-in-space) implicit timesteppers for the incompressible analogue to this system with structure-preserving (SP) properties in the ideal case, alongside parameter-robust preconditioners. We show that these timesteppers can derive from a finite-element-in-time (FET) (and finite-element-in-space) interpretation. The benefits of this reformulation are discussed, including the derivation of timesteppers that are higher order in time, and the quantifiable dissipative SP properties in the non-ideal, resistive case.
        
        We discuss possible options for extending this FET approach to timesteppers for the compressible case.

        The kinetic corrections satisfy linearized Boltzmann equations. Using a Lénard--Bernstein collision operator, these take Fokker--Planck-like forms \cite{Fokker_1914, Planck_1917} wherein pseudo-particles in the numerical model obey the neoclassical transport equations, with particle-independent Brownian drift terms. This offers a rigorous methodology for incorporating collisions into the particle transport model, without coupling the equations of motions for each particle.
        
        Works by Chen, Chacón et al. \cite{Chen_Chacón_Barnes_2011, Chacón_Chen_Barnes_2013, Chen_Chacón_2014, Chen_Chacón_2015} have developed structure-preserving particle pushers for neoclassical transport in the Vlasov equations, derived from Crank--Nicolson integrators. We show these too can can derive from a FET interpretation, similarly offering potential extensions to higher-order-in-time particle pushers. The FET formulation is used also to consider how the stochastic drift terms can be incorporated into the pushers. Stochastic gyrokinetic expansions are also discussed.

        Different options for the numerical implementation of these schemes are considered.

        Due to the efficacy of FET in the development of SP timesteppers for both the fluid and kinetic component, we hope this approach will prove effective in the future for developing SP timesteppers for the full hybrid model. We hope this will give us the opportunity to incorporate previously inaccessible kinetic effects into the highly effective, modern, finite-element MHD models.
    \end{abstract}
    
    
    \newpage
    \tableofcontents
    
    
    \newpage
    \pagenumbering{arabic}
    %\linenumbers\renewcommand\thelinenumber{\color{black!50}\arabic{linenumber}}
            \input{0 - introduction/main.tex}
        \part{Research}
            \input{1 - low-noise PiC models/main.tex}
            \input{2 - kinetic component/main.tex}
            \input{3 - fluid component/main.tex}
            \input{4 - numerical implementation/main.tex}
        \part{Project Overview}
            \input{5 - research plan/main.tex}
            \input{6 - summary/main.tex}
    
    
    %\section{}
    \newpage
    \pagenumbering{gobble}
        \printbibliography


    \newpage
    \pagenumbering{roman}
    \appendix
        \part{Appendices}
            \input{8 - Hilbert complexes/main.tex}
            \input{9 - weak conservation proofs/main.tex}
\end{document}

        \part{Project Overview}
            \documentclass[12pt, a4paper]{report}

\input{template/main.tex}

\title{\BA{Title in Progress...}}
\author{Boris Andrews}
\affil{Mathematical Institute, University of Oxford}
\date{\today}


\begin{document}
    \pagenumbering{gobble}
    \maketitle
    
    
    \begin{abstract}
        Magnetic confinement reactors---in particular tokamaks---offer one of the most promising options for achieving practical nuclear fusion, with the potential to provide virtually limitless, clean energy. The theoretical and numerical modeling of tokamak plasmas is simultaneously an essential component of effective reactor design, and a great research barrier. Tokamak operational conditions exhibit comparatively low Knudsen numbers. Kinetic effects, including kinetic waves and instabilities, Landau damping, bump-on-tail instabilities and more, are therefore highly influential in tokamak plasma dynamics. Purely fluid models are inherently incapable of capturing these effects, whereas the high dimensionality in purely kinetic models render them practically intractable for most relevant purposes.

        We consider a $\delta\!f$ decomposition model, with a macroscopic fluid background and microscopic kinetic correction, both fully coupled to each other. A similar manner of discretization is proposed to that used in the recent \texttt{STRUPHY} code \cite{Holderied_Possanner_Wang_2021, Holderied_2022, Li_et_al_2023} with a finite-element model for the background and a pseudo-particle/PiC model for the correction.

        The fluid background satisfies the full, non-linear, resistive, compressible, Hall MHD equations. \cite{Laakmann_Hu_Farrell_2022} introduces finite-element(-in-space) implicit timesteppers for the incompressible analogue to this system with structure-preserving (SP) properties in the ideal case, alongside parameter-robust preconditioners. We show that these timesteppers can derive from a finite-element-in-time (FET) (and finite-element-in-space) interpretation. The benefits of this reformulation are discussed, including the derivation of timesteppers that are higher order in time, and the quantifiable dissipative SP properties in the non-ideal, resistive case.
        
        We discuss possible options for extending this FET approach to timesteppers for the compressible case.

        The kinetic corrections satisfy linearized Boltzmann equations. Using a Lénard--Bernstein collision operator, these take Fokker--Planck-like forms \cite{Fokker_1914, Planck_1917} wherein pseudo-particles in the numerical model obey the neoclassical transport equations, with particle-independent Brownian drift terms. This offers a rigorous methodology for incorporating collisions into the particle transport model, without coupling the equations of motions for each particle.
        
        Works by Chen, Chacón et al. \cite{Chen_Chacón_Barnes_2011, Chacón_Chen_Barnes_2013, Chen_Chacón_2014, Chen_Chacón_2015} have developed structure-preserving particle pushers for neoclassical transport in the Vlasov equations, derived from Crank--Nicolson integrators. We show these too can can derive from a FET interpretation, similarly offering potential extensions to higher-order-in-time particle pushers. The FET formulation is used also to consider how the stochastic drift terms can be incorporated into the pushers. Stochastic gyrokinetic expansions are also discussed.

        Different options for the numerical implementation of these schemes are considered.

        Due to the efficacy of FET in the development of SP timesteppers for both the fluid and kinetic component, we hope this approach will prove effective in the future for developing SP timesteppers for the full hybrid model. We hope this will give us the opportunity to incorporate previously inaccessible kinetic effects into the highly effective, modern, finite-element MHD models.
    \end{abstract}
    
    
    \newpage
    \tableofcontents
    
    
    \newpage
    \pagenumbering{arabic}
    %\linenumbers\renewcommand\thelinenumber{\color{black!50}\arabic{linenumber}}
            \input{0 - introduction/main.tex}
        \part{Research}
            \input{1 - low-noise PiC models/main.tex}
            \input{2 - kinetic component/main.tex}
            \input{3 - fluid component/main.tex}
            \input{4 - numerical implementation/main.tex}
        \part{Project Overview}
            \input{5 - research plan/main.tex}
            \input{6 - summary/main.tex}
    
    
    %\section{}
    \newpage
    \pagenumbering{gobble}
        \printbibliography


    \newpage
    \pagenumbering{roman}
    \appendix
        \part{Appendices}
            \input{8 - Hilbert complexes/main.tex}
            \input{9 - weak conservation proofs/main.tex}
\end{document}

            \documentclass[12pt, a4paper]{report}

\input{template/main.tex}

\title{\BA{Title in Progress...}}
\author{Boris Andrews}
\affil{Mathematical Institute, University of Oxford}
\date{\today}


\begin{document}
    \pagenumbering{gobble}
    \maketitle
    
    
    \begin{abstract}
        Magnetic confinement reactors---in particular tokamaks---offer one of the most promising options for achieving practical nuclear fusion, with the potential to provide virtually limitless, clean energy. The theoretical and numerical modeling of tokamak plasmas is simultaneously an essential component of effective reactor design, and a great research barrier. Tokamak operational conditions exhibit comparatively low Knudsen numbers. Kinetic effects, including kinetic waves and instabilities, Landau damping, bump-on-tail instabilities and more, are therefore highly influential in tokamak plasma dynamics. Purely fluid models are inherently incapable of capturing these effects, whereas the high dimensionality in purely kinetic models render them practically intractable for most relevant purposes.

        We consider a $\delta\!f$ decomposition model, with a macroscopic fluid background and microscopic kinetic correction, both fully coupled to each other. A similar manner of discretization is proposed to that used in the recent \texttt{STRUPHY} code \cite{Holderied_Possanner_Wang_2021, Holderied_2022, Li_et_al_2023} with a finite-element model for the background and a pseudo-particle/PiC model for the correction.

        The fluid background satisfies the full, non-linear, resistive, compressible, Hall MHD equations. \cite{Laakmann_Hu_Farrell_2022} introduces finite-element(-in-space) implicit timesteppers for the incompressible analogue to this system with structure-preserving (SP) properties in the ideal case, alongside parameter-robust preconditioners. We show that these timesteppers can derive from a finite-element-in-time (FET) (and finite-element-in-space) interpretation. The benefits of this reformulation are discussed, including the derivation of timesteppers that are higher order in time, and the quantifiable dissipative SP properties in the non-ideal, resistive case.
        
        We discuss possible options for extending this FET approach to timesteppers for the compressible case.

        The kinetic corrections satisfy linearized Boltzmann equations. Using a Lénard--Bernstein collision operator, these take Fokker--Planck-like forms \cite{Fokker_1914, Planck_1917} wherein pseudo-particles in the numerical model obey the neoclassical transport equations, with particle-independent Brownian drift terms. This offers a rigorous methodology for incorporating collisions into the particle transport model, without coupling the equations of motions for each particle.
        
        Works by Chen, Chacón et al. \cite{Chen_Chacón_Barnes_2011, Chacón_Chen_Barnes_2013, Chen_Chacón_2014, Chen_Chacón_2015} have developed structure-preserving particle pushers for neoclassical transport in the Vlasov equations, derived from Crank--Nicolson integrators. We show these too can can derive from a FET interpretation, similarly offering potential extensions to higher-order-in-time particle pushers. The FET formulation is used also to consider how the stochastic drift terms can be incorporated into the pushers. Stochastic gyrokinetic expansions are also discussed.

        Different options for the numerical implementation of these schemes are considered.

        Due to the efficacy of FET in the development of SP timesteppers for both the fluid and kinetic component, we hope this approach will prove effective in the future for developing SP timesteppers for the full hybrid model. We hope this will give us the opportunity to incorporate previously inaccessible kinetic effects into the highly effective, modern, finite-element MHD models.
    \end{abstract}
    
    
    \newpage
    \tableofcontents
    
    
    \newpage
    \pagenumbering{arabic}
    %\linenumbers\renewcommand\thelinenumber{\color{black!50}\arabic{linenumber}}
            \input{0 - introduction/main.tex}
        \part{Research}
            \input{1 - low-noise PiC models/main.tex}
            \input{2 - kinetic component/main.tex}
            \input{3 - fluid component/main.tex}
            \input{4 - numerical implementation/main.tex}
        \part{Project Overview}
            \input{5 - research plan/main.tex}
            \input{6 - summary/main.tex}
    
    
    %\section{}
    \newpage
    \pagenumbering{gobble}
        \printbibliography


    \newpage
    \pagenumbering{roman}
    \appendix
        \part{Appendices}
            \input{8 - Hilbert complexes/main.tex}
            \input{9 - weak conservation proofs/main.tex}
\end{document}

    
    
    %\section{}
    \newpage
    \pagenumbering{gobble}
        \printbibliography


    \newpage
    \pagenumbering{roman}
    \appendix
        \part{Appendices}
            \documentclass[12pt, a4paper]{report}

\input{template/main.tex}

\title{\BA{Title in Progress...}}
\author{Boris Andrews}
\affil{Mathematical Institute, University of Oxford}
\date{\today}


\begin{document}
    \pagenumbering{gobble}
    \maketitle
    
    
    \begin{abstract}
        Magnetic confinement reactors---in particular tokamaks---offer one of the most promising options for achieving practical nuclear fusion, with the potential to provide virtually limitless, clean energy. The theoretical and numerical modeling of tokamak plasmas is simultaneously an essential component of effective reactor design, and a great research barrier. Tokamak operational conditions exhibit comparatively low Knudsen numbers. Kinetic effects, including kinetic waves and instabilities, Landau damping, bump-on-tail instabilities and more, are therefore highly influential in tokamak plasma dynamics. Purely fluid models are inherently incapable of capturing these effects, whereas the high dimensionality in purely kinetic models render them practically intractable for most relevant purposes.

        We consider a $\delta\!f$ decomposition model, with a macroscopic fluid background and microscopic kinetic correction, both fully coupled to each other. A similar manner of discretization is proposed to that used in the recent \texttt{STRUPHY} code \cite{Holderied_Possanner_Wang_2021, Holderied_2022, Li_et_al_2023} with a finite-element model for the background and a pseudo-particle/PiC model for the correction.

        The fluid background satisfies the full, non-linear, resistive, compressible, Hall MHD equations. \cite{Laakmann_Hu_Farrell_2022} introduces finite-element(-in-space) implicit timesteppers for the incompressible analogue to this system with structure-preserving (SP) properties in the ideal case, alongside parameter-robust preconditioners. We show that these timesteppers can derive from a finite-element-in-time (FET) (and finite-element-in-space) interpretation. The benefits of this reformulation are discussed, including the derivation of timesteppers that are higher order in time, and the quantifiable dissipative SP properties in the non-ideal, resistive case.
        
        We discuss possible options for extending this FET approach to timesteppers for the compressible case.

        The kinetic corrections satisfy linearized Boltzmann equations. Using a Lénard--Bernstein collision operator, these take Fokker--Planck-like forms \cite{Fokker_1914, Planck_1917} wherein pseudo-particles in the numerical model obey the neoclassical transport equations, with particle-independent Brownian drift terms. This offers a rigorous methodology for incorporating collisions into the particle transport model, without coupling the equations of motions for each particle.
        
        Works by Chen, Chacón et al. \cite{Chen_Chacón_Barnes_2011, Chacón_Chen_Barnes_2013, Chen_Chacón_2014, Chen_Chacón_2015} have developed structure-preserving particle pushers for neoclassical transport in the Vlasov equations, derived from Crank--Nicolson integrators. We show these too can can derive from a FET interpretation, similarly offering potential extensions to higher-order-in-time particle pushers. The FET formulation is used also to consider how the stochastic drift terms can be incorporated into the pushers. Stochastic gyrokinetic expansions are also discussed.

        Different options for the numerical implementation of these schemes are considered.

        Due to the efficacy of FET in the development of SP timesteppers for both the fluid and kinetic component, we hope this approach will prove effective in the future for developing SP timesteppers for the full hybrid model. We hope this will give us the opportunity to incorporate previously inaccessible kinetic effects into the highly effective, modern, finite-element MHD models.
    \end{abstract}
    
    
    \newpage
    \tableofcontents
    
    
    \newpage
    \pagenumbering{arabic}
    %\linenumbers\renewcommand\thelinenumber{\color{black!50}\arabic{linenumber}}
            \input{0 - introduction/main.tex}
        \part{Research}
            \input{1 - low-noise PiC models/main.tex}
            \input{2 - kinetic component/main.tex}
            \input{3 - fluid component/main.tex}
            \input{4 - numerical implementation/main.tex}
        \part{Project Overview}
            \input{5 - research plan/main.tex}
            \input{6 - summary/main.tex}
    
    
    %\section{}
    \newpage
    \pagenumbering{gobble}
        \printbibliography


    \newpage
    \pagenumbering{roman}
    \appendix
        \part{Appendices}
            \input{8 - Hilbert complexes/main.tex}
            \input{9 - weak conservation proofs/main.tex}
\end{document}

            \documentclass[12pt, a4paper]{report}

\input{template/main.tex}

\title{\BA{Title in Progress...}}
\author{Boris Andrews}
\affil{Mathematical Institute, University of Oxford}
\date{\today}


\begin{document}
    \pagenumbering{gobble}
    \maketitle
    
    
    \begin{abstract}
        Magnetic confinement reactors---in particular tokamaks---offer one of the most promising options for achieving practical nuclear fusion, with the potential to provide virtually limitless, clean energy. The theoretical and numerical modeling of tokamak plasmas is simultaneously an essential component of effective reactor design, and a great research barrier. Tokamak operational conditions exhibit comparatively low Knudsen numbers. Kinetic effects, including kinetic waves and instabilities, Landau damping, bump-on-tail instabilities and more, are therefore highly influential in tokamak plasma dynamics. Purely fluid models are inherently incapable of capturing these effects, whereas the high dimensionality in purely kinetic models render them practically intractable for most relevant purposes.

        We consider a $\delta\!f$ decomposition model, with a macroscopic fluid background and microscopic kinetic correction, both fully coupled to each other. A similar manner of discretization is proposed to that used in the recent \texttt{STRUPHY} code \cite{Holderied_Possanner_Wang_2021, Holderied_2022, Li_et_al_2023} with a finite-element model for the background and a pseudo-particle/PiC model for the correction.

        The fluid background satisfies the full, non-linear, resistive, compressible, Hall MHD equations. \cite{Laakmann_Hu_Farrell_2022} introduces finite-element(-in-space) implicit timesteppers for the incompressible analogue to this system with structure-preserving (SP) properties in the ideal case, alongside parameter-robust preconditioners. We show that these timesteppers can derive from a finite-element-in-time (FET) (and finite-element-in-space) interpretation. The benefits of this reformulation are discussed, including the derivation of timesteppers that are higher order in time, and the quantifiable dissipative SP properties in the non-ideal, resistive case.
        
        We discuss possible options for extending this FET approach to timesteppers for the compressible case.

        The kinetic corrections satisfy linearized Boltzmann equations. Using a Lénard--Bernstein collision operator, these take Fokker--Planck-like forms \cite{Fokker_1914, Planck_1917} wherein pseudo-particles in the numerical model obey the neoclassical transport equations, with particle-independent Brownian drift terms. This offers a rigorous methodology for incorporating collisions into the particle transport model, without coupling the equations of motions for each particle.
        
        Works by Chen, Chacón et al. \cite{Chen_Chacón_Barnes_2011, Chacón_Chen_Barnes_2013, Chen_Chacón_2014, Chen_Chacón_2015} have developed structure-preserving particle pushers for neoclassical transport in the Vlasov equations, derived from Crank--Nicolson integrators. We show these too can can derive from a FET interpretation, similarly offering potential extensions to higher-order-in-time particle pushers. The FET formulation is used also to consider how the stochastic drift terms can be incorporated into the pushers. Stochastic gyrokinetic expansions are also discussed.

        Different options for the numerical implementation of these schemes are considered.

        Due to the efficacy of FET in the development of SP timesteppers for both the fluid and kinetic component, we hope this approach will prove effective in the future for developing SP timesteppers for the full hybrid model. We hope this will give us the opportunity to incorporate previously inaccessible kinetic effects into the highly effective, modern, finite-element MHD models.
    \end{abstract}
    
    
    \newpage
    \tableofcontents
    
    
    \newpage
    \pagenumbering{arabic}
    %\linenumbers\renewcommand\thelinenumber{\color{black!50}\arabic{linenumber}}
            \input{0 - introduction/main.tex}
        \part{Research}
            \input{1 - low-noise PiC models/main.tex}
            \input{2 - kinetic component/main.tex}
            \input{3 - fluid component/main.tex}
            \input{4 - numerical implementation/main.tex}
        \part{Project Overview}
            \input{5 - research plan/main.tex}
            \input{6 - summary/main.tex}
    
    
    %\section{}
    \newpage
    \pagenumbering{gobble}
        \printbibliography


    \newpage
    \pagenumbering{roman}
    \appendix
        \part{Appendices}
            \input{8 - Hilbert complexes/main.tex}
            \input{9 - weak conservation proofs/main.tex}
\end{document}

\end{document}

            \documentclass[12pt, a4paper]{report}

\documentclass[12pt, a4paper]{report}

\input{template/main.tex}

\title{\BA{Title in Progress...}}
\author{Boris Andrews}
\affil{Mathematical Institute, University of Oxford}
\date{\today}


\begin{document}
    \pagenumbering{gobble}
    \maketitle
    
    
    \begin{abstract}
        Magnetic confinement reactors---in particular tokamaks---offer one of the most promising options for achieving practical nuclear fusion, with the potential to provide virtually limitless, clean energy. The theoretical and numerical modeling of tokamak plasmas is simultaneously an essential component of effective reactor design, and a great research barrier. Tokamak operational conditions exhibit comparatively low Knudsen numbers. Kinetic effects, including kinetic waves and instabilities, Landau damping, bump-on-tail instabilities and more, are therefore highly influential in tokamak plasma dynamics. Purely fluid models are inherently incapable of capturing these effects, whereas the high dimensionality in purely kinetic models render them practically intractable for most relevant purposes.

        We consider a $\delta\!f$ decomposition model, with a macroscopic fluid background and microscopic kinetic correction, both fully coupled to each other. A similar manner of discretization is proposed to that used in the recent \texttt{STRUPHY} code \cite{Holderied_Possanner_Wang_2021, Holderied_2022, Li_et_al_2023} with a finite-element model for the background and a pseudo-particle/PiC model for the correction.

        The fluid background satisfies the full, non-linear, resistive, compressible, Hall MHD equations. \cite{Laakmann_Hu_Farrell_2022} introduces finite-element(-in-space) implicit timesteppers for the incompressible analogue to this system with structure-preserving (SP) properties in the ideal case, alongside parameter-robust preconditioners. We show that these timesteppers can derive from a finite-element-in-time (FET) (and finite-element-in-space) interpretation. The benefits of this reformulation are discussed, including the derivation of timesteppers that are higher order in time, and the quantifiable dissipative SP properties in the non-ideal, resistive case.
        
        We discuss possible options for extending this FET approach to timesteppers for the compressible case.

        The kinetic corrections satisfy linearized Boltzmann equations. Using a Lénard--Bernstein collision operator, these take Fokker--Planck-like forms \cite{Fokker_1914, Planck_1917} wherein pseudo-particles in the numerical model obey the neoclassical transport equations, with particle-independent Brownian drift terms. This offers a rigorous methodology for incorporating collisions into the particle transport model, without coupling the equations of motions for each particle.
        
        Works by Chen, Chacón et al. \cite{Chen_Chacón_Barnes_2011, Chacón_Chen_Barnes_2013, Chen_Chacón_2014, Chen_Chacón_2015} have developed structure-preserving particle pushers for neoclassical transport in the Vlasov equations, derived from Crank--Nicolson integrators. We show these too can can derive from a FET interpretation, similarly offering potential extensions to higher-order-in-time particle pushers. The FET formulation is used also to consider how the stochastic drift terms can be incorporated into the pushers. Stochastic gyrokinetic expansions are also discussed.

        Different options for the numerical implementation of these schemes are considered.

        Due to the efficacy of FET in the development of SP timesteppers for both the fluid and kinetic component, we hope this approach will prove effective in the future for developing SP timesteppers for the full hybrid model. We hope this will give us the opportunity to incorporate previously inaccessible kinetic effects into the highly effective, modern, finite-element MHD models.
    \end{abstract}
    
    
    \newpage
    \tableofcontents
    
    
    \newpage
    \pagenumbering{arabic}
    %\linenumbers\renewcommand\thelinenumber{\color{black!50}\arabic{linenumber}}
            \input{0 - introduction/main.tex}
        \part{Research}
            \input{1 - low-noise PiC models/main.tex}
            \input{2 - kinetic component/main.tex}
            \input{3 - fluid component/main.tex}
            \input{4 - numerical implementation/main.tex}
        \part{Project Overview}
            \input{5 - research plan/main.tex}
            \input{6 - summary/main.tex}
    
    
    %\section{}
    \newpage
    \pagenumbering{gobble}
        \printbibliography


    \newpage
    \pagenumbering{roman}
    \appendix
        \part{Appendices}
            \input{8 - Hilbert complexes/main.tex}
            \input{9 - weak conservation proofs/main.tex}
\end{document}


\title{\BA{Title in Progress...}}
\author{Boris Andrews}
\affil{Mathematical Institute, University of Oxford}
\date{\today}


\begin{document}
    \pagenumbering{gobble}
    \maketitle
    
    
    \begin{abstract}
        Magnetic confinement reactors---in particular tokamaks---offer one of the most promising options for achieving practical nuclear fusion, with the potential to provide virtually limitless, clean energy. The theoretical and numerical modeling of tokamak plasmas is simultaneously an essential component of effective reactor design, and a great research barrier. Tokamak operational conditions exhibit comparatively low Knudsen numbers. Kinetic effects, including kinetic waves and instabilities, Landau damping, bump-on-tail instabilities and more, are therefore highly influential in tokamak plasma dynamics. Purely fluid models are inherently incapable of capturing these effects, whereas the high dimensionality in purely kinetic models render them practically intractable for most relevant purposes.

        We consider a $\delta\!f$ decomposition model, with a macroscopic fluid background and microscopic kinetic correction, both fully coupled to each other. A similar manner of discretization is proposed to that used in the recent \texttt{STRUPHY} code \cite{Holderied_Possanner_Wang_2021, Holderied_2022, Li_et_al_2023} with a finite-element model for the background and a pseudo-particle/PiC model for the correction.

        The fluid background satisfies the full, non-linear, resistive, compressible, Hall MHD equations. \cite{Laakmann_Hu_Farrell_2022} introduces finite-element(-in-space) implicit timesteppers for the incompressible analogue to this system with structure-preserving (SP) properties in the ideal case, alongside parameter-robust preconditioners. We show that these timesteppers can derive from a finite-element-in-time (FET) (and finite-element-in-space) interpretation. The benefits of this reformulation are discussed, including the derivation of timesteppers that are higher order in time, and the quantifiable dissipative SP properties in the non-ideal, resistive case.
        
        We discuss possible options for extending this FET approach to timesteppers for the compressible case.

        The kinetic corrections satisfy linearized Boltzmann equations. Using a Lénard--Bernstein collision operator, these take Fokker--Planck-like forms \cite{Fokker_1914, Planck_1917} wherein pseudo-particles in the numerical model obey the neoclassical transport equations, with particle-independent Brownian drift terms. This offers a rigorous methodology for incorporating collisions into the particle transport model, without coupling the equations of motions for each particle.
        
        Works by Chen, Chacón et al. \cite{Chen_Chacón_Barnes_2011, Chacón_Chen_Barnes_2013, Chen_Chacón_2014, Chen_Chacón_2015} have developed structure-preserving particle pushers for neoclassical transport in the Vlasov equations, derived from Crank--Nicolson integrators. We show these too can can derive from a FET interpretation, similarly offering potential extensions to higher-order-in-time particle pushers. The FET formulation is used also to consider how the stochastic drift terms can be incorporated into the pushers. Stochastic gyrokinetic expansions are also discussed.

        Different options for the numerical implementation of these schemes are considered.

        Due to the efficacy of FET in the development of SP timesteppers for both the fluid and kinetic component, we hope this approach will prove effective in the future for developing SP timesteppers for the full hybrid model. We hope this will give us the opportunity to incorporate previously inaccessible kinetic effects into the highly effective, modern, finite-element MHD models.
    \end{abstract}
    
    
    \newpage
    \tableofcontents
    
    
    \newpage
    \pagenumbering{arabic}
    %\linenumbers\renewcommand\thelinenumber{\color{black!50}\arabic{linenumber}}
            \documentclass[12pt, a4paper]{report}

\input{template/main.tex}

\title{\BA{Title in Progress...}}
\author{Boris Andrews}
\affil{Mathematical Institute, University of Oxford}
\date{\today}


\begin{document}
    \pagenumbering{gobble}
    \maketitle
    
    
    \begin{abstract}
        Magnetic confinement reactors---in particular tokamaks---offer one of the most promising options for achieving practical nuclear fusion, with the potential to provide virtually limitless, clean energy. The theoretical and numerical modeling of tokamak plasmas is simultaneously an essential component of effective reactor design, and a great research barrier. Tokamak operational conditions exhibit comparatively low Knudsen numbers. Kinetic effects, including kinetic waves and instabilities, Landau damping, bump-on-tail instabilities and more, are therefore highly influential in tokamak plasma dynamics. Purely fluid models are inherently incapable of capturing these effects, whereas the high dimensionality in purely kinetic models render them practically intractable for most relevant purposes.

        We consider a $\delta\!f$ decomposition model, with a macroscopic fluid background and microscopic kinetic correction, both fully coupled to each other. A similar manner of discretization is proposed to that used in the recent \texttt{STRUPHY} code \cite{Holderied_Possanner_Wang_2021, Holderied_2022, Li_et_al_2023} with a finite-element model for the background and a pseudo-particle/PiC model for the correction.

        The fluid background satisfies the full, non-linear, resistive, compressible, Hall MHD equations. \cite{Laakmann_Hu_Farrell_2022} introduces finite-element(-in-space) implicit timesteppers for the incompressible analogue to this system with structure-preserving (SP) properties in the ideal case, alongside parameter-robust preconditioners. We show that these timesteppers can derive from a finite-element-in-time (FET) (and finite-element-in-space) interpretation. The benefits of this reformulation are discussed, including the derivation of timesteppers that are higher order in time, and the quantifiable dissipative SP properties in the non-ideal, resistive case.
        
        We discuss possible options for extending this FET approach to timesteppers for the compressible case.

        The kinetic corrections satisfy linearized Boltzmann equations. Using a Lénard--Bernstein collision operator, these take Fokker--Planck-like forms \cite{Fokker_1914, Planck_1917} wherein pseudo-particles in the numerical model obey the neoclassical transport equations, with particle-independent Brownian drift terms. This offers a rigorous methodology for incorporating collisions into the particle transport model, without coupling the equations of motions for each particle.
        
        Works by Chen, Chacón et al. \cite{Chen_Chacón_Barnes_2011, Chacón_Chen_Barnes_2013, Chen_Chacón_2014, Chen_Chacón_2015} have developed structure-preserving particle pushers for neoclassical transport in the Vlasov equations, derived from Crank--Nicolson integrators. We show these too can can derive from a FET interpretation, similarly offering potential extensions to higher-order-in-time particle pushers. The FET formulation is used also to consider how the stochastic drift terms can be incorporated into the pushers. Stochastic gyrokinetic expansions are also discussed.

        Different options for the numerical implementation of these schemes are considered.

        Due to the efficacy of FET in the development of SP timesteppers for both the fluid and kinetic component, we hope this approach will prove effective in the future for developing SP timesteppers for the full hybrid model. We hope this will give us the opportunity to incorporate previously inaccessible kinetic effects into the highly effective, modern, finite-element MHD models.
    \end{abstract}
    
    
    \newpage
    \tableofcontents
    
    
    \newpage
    \pagenumbering{arabic}
    %\linenumbers\renewcommand\thelinenumber{\color{black!50}\arabic{linenumber}}
            \input{0 - introduction/main.tex}
        \part{Research}
            \input{1 - low-noise PiC models/main.tex}
            \input{2 - kinetic component/main.tex}
            \input{3 - fluid component/main.tex}
            \input{4 - numerical implementation/main.tex}
        \part{Project Overview}
            \input{5 - research plan/main.tex}
            \input{6 - summary/main.tex}
    
    
    %\section{}
    \newpage
    \pagenumbering{gobble}
        \printbibliography


    \newpage
    \pagenumbering{roman}
    \appendix
        \part{Appendices}
            \input{8 - Hilbert complexes/main.tex}
            \input{9 - weak conservation proofs/main.tex}
\end{document}

        \part{Research}
            \documentclass[12pt, a4paper]{report}

\input{template/main.tex}

\title{\BA{Title in Progress...}}
\author{Boris Andrews}
\affil{Mathematical Institute, University of Oxford}
\date{\today}


\begin{document}
    \pagenumbering{gobble}
    \maketitle
    
    
    \begin{abstract}
        Magnetic confinement reactors---in particular tokamaks---offer one of the most promising options for achieving practical nuclear fusion, with the potential to provide virtually limitless, clean energy. The theoretical and numerical modeling of tokamak plasmas is simultaneously an essential component of effective reactor design, and a great research barrier. Tokamak operational conditions exhibit comparatively low Knudsen numbers. Kinetic effects, including kinetic waves and instabilities, Landau damping, bump-on-tail instabilities and more, are therefore highly influential in tokamak plasma dynamics. Purely fluid models are inherently incapable of capturing these effects, whereas the high dimensionality in purely kinetic models render them practically intractable for most relevant purposes.

        We consider a $\delta\!f$ decomposition model, with a macroscopic fluid background and microscopic kinetic correction, both fully coupled to each other. A similar manner of discretization is proposed to that used in the recent \texttt{STRUPHY} code \cite{Holderied_Possanner_Wang_2021, Holderied_2022, Li_et_al_2023} with a finite-element model for the background and a pseudo-particle/PiC model for the correction.

        The fluid background satisfies the full, non-linear, resistive, compressible, Hall MHD equations. \cite{Laakmann_Hu_Farrell_2022} introduces finite-element(-in-space) implicit timesteppers for the incompressible analogue to this system with structure-preserving (SP) properties in the ideal case, alongside parameter-robust preconditioners. We show that these timesteppers can derive from a finite-element-in-time (FET) (and finite-element-in-space) interpretation. The benefits of this reformulation are discussed, including the derivation of timesteppers that are higher order in time, and the quantifiable dissipative SP properties in the non-ideal, resistive case.
        
        We discuss possible options for extending this FET approach to timesteppers for the compressible case.

        The kinetic corrections satisfy linearized Boltzmann equations. Using a Lénard--Bernstein collision operator, these take Fokker--Planck-like forms \cite{Fokker_1914, Planck_1917} wherein pseudo-particles in the numerical model obey the neoclassical transport equations, with particle-independent Brownian drift terms. This offers a rigorous methodology for incorporating collisions into the particle transport model, without coupling the equations of motions for each particle.
        
        Works by Chen, Chacón et al. \cite{Chen_Chacón_Barnes_2011, Chacón_Chen_Barnes_2013, Chen_Chacón_2014, Chen_Chacón_2015} have developed structure-preserving particle pushers for neoclassical transport in the Vlasov equations, derived from Crank--Nicolson integrators. We show these too can can derive from a FET interpretation, similarly offering potential extensions to higher-order-in-time particle pushers. The FET formulation is used also to consider how the stochastic drift terms can be incorporated into the pushers. Stochastic gyrokinetic expansions are also discussed.

        Different options for the numerical implementation of these schemes are considered.

        Due to the efficacy of FET in the development of SP timesteppers for both the fluid and kinetic component, we hope this approach will prove effective in the future for developing SP timesteppers for the full hybrid model. We hope this will give us the opportunity to incorporate previously inaccessible kinetic effects into the highly effective, modern, finite-element MHD models.
    \end{abstract}
    
    
    \newpage
    \tableofcontents
    
    
    \newpage
    \pagenumbering{arabic}
    %\linenumbers\renewcommand\thelinenumber{\color{black!50}\arabic{linenumber}}
            \input{0 - introduction/main.tex}
        \part{Research}
            \input{1 - low-noise PiC models/main.tex}
            \input{2 - kinetic component/main.tex}
            \input{3 - fluid component/main.tex}
            \input{4 - numerical implementation/main.tex}
        \part{Project Overview}
            \input{5 - research plan/main.tex}
            \input{6 - summary/main.tex}
    
    
    %\section{}
    \newpage
    \pagenumbering{gobble}
        \printbibliography


    \newpage
    \pagenumbering{roman}
    \appendix
        \part{Appendices}
            \input{8 - Hilbert complexes/main.tex}
            \input{9 - weak conservation proofs/main.tex}
\end{document}

            \documentclass[12pt, a4paper]{report}

\input{template/main.tex}

\title{\BA{Title in Progress...}}
\author{Boris Andrews}
\affil{Mathematical Institute, University of Oxford}
\date{\today}


\begin{document}
    \pagenumbering{gobble}
    \maketitle
    
    
    \begin{abstract}
        Magnetic confinement reactors---in particular tokamaks---offer one of the most promising options for achieving practical nuclear fusion, with the potential to provide virtually limitless, clean energy. The theoretical and numerical modeling of tokamak plasmas is simultaneously an essential component of effective reactor design, and a great research barrier. Tokamak operational conditions exhibit comparatively low Knudsen numbers. Kinetic effects, including kinetic waves and instabilities, Landau damping, bump-on-tail instabilities and more, are therefore highly influential in tokamak plasma dynamics. Purely fluid models are inherently incapable of capturing these effects, whereas the high dimensionality in purely kinetic models render them practically intractable for most relevant purposes.

        We consider a $\delta\!f$ decomposition model, with a macroscopic fluid background and microscopic kinetic correction, both fully coupled to each other. A similar manner of discretization is proposed to that used in the recent \texttt{STRUPHY} code \cite{Holderied_Possanner_Wang_2021, Holderied_2022, Li_et_al_2023} with a finite-element model for the background and a pseudo-particle/PiC model for the correction.

        The fluid background satisfies the full, non-linear, resistive, compressible, Hall MHD equations. \cite{Laakmann_Hu_Farrell_2022} introduces finite-element(-in-space) implicit timesteppers for the incompressible analogue to this system with structure-preserving (SP) properties in the ideal case, alongside parameter-robust preconditioners. We show that these timesteppers can derive from a finite-element-in-time (FET) (and finite-element-in-space) interpretation. The benefits of this reformulation are discussed, including the derivation of timesteppers that are higher order in time, and the quantifiable dissipative SP properties in the non-ideal, resistive case.
        
        We discuss possible options for extending this FET approach to timesteppers for the compressible case.

        The kinetic corrections satisfy linearized Boltzmann equations. Using a Lénard--Bernstein collision operator, these take Fokker--Planck-like forms \cite{Fokker_1914, Planck_1917} wherein pseudo-particles in the numerical model obey the neoclassical transport equations, with particle-independent Brownian drift terms. This offers a rigorous methodology for incorporating collisions into the particle transport model, without coupling the equations of motions for each particle.
        
        Works by Chen, Chacón et al. \cite{Chen_Chacón_Barnes_2011, Chacón_Chen_Barnes_2013, Chen_Chacón_2014, Chen_Chacón_2015} have developed structure-preserving particle pushers for neoclassical transport in the Vlasov equations, derived from Crank--Nicolson integrators. We show these too can can derive from a FET interpretation, similarly offering potential extensions to higher-order-in-time particle pushers. The FET formulation is used also to consider how the stochastic drift terms can be incorporated into the pushers. Stochastic gyrokinetic expansions are also discussed.

        Different options for the numerical implementation of these schemes are considered.

        Due to the efficacy of FET in the development of SP timesteppers for both the fluid and kinetic component, we hope this approach will prove effective in the future for developing SP timesteppers for the full hybrid model. We hope this will give us the opportunity to incorporate previously inaccessible kinetic effects into the highly effective, modern, finite-element MHD models.
    \end{abstract}
    
    
    \newpage
    \tableofcontents
    
    
    \newpage
    \pagenumbering{arabic}
    %\linenumbers\renewcommand\thelinenumber{\color{black!50}\arabic{linenumber}}
            \input{0 - introduction/main.tex}
        \part{Research}
            \input{1 - low-noise PiC models/main.tex}
            \input{2 - kinetic component/main.tex}
            \input{3 - fluid component/main.tex}
            \input{4 - numerical implementation/main.tex}
        \part{Project Overview}
            \input{5 - research plan/main.tex}
            \input{6 - summary/main.tex}
    
    
    %\section{}
    \newpage
    \pagenumbering{gobble}
        \printbibliography


    \newpage
    \pagenumbering{roman}
    \appendix
        \part{Appendices}
            \input{8 - Hilbert complexes/main.tex}
            \input{9 - weak conservation proofs/main.tex}
\end{document}

            \documentclass[12pt, a4paper]{report}

\input{template/main.tex}

\title{\BA{Title in Progress...}}
\author{Boris Andrews}
\affil{Mathematical Institute, University of Oxford}
\date{\today}


\begin{document}
    \pagenumbering{gobble}
    \maketitle
    
    
    \begin{abstract}
        Magnetic confinement reactors---in particular tokamaks---offer one of the most promising options for achieving practical nuclear fusion, with the potential to provide virtually limitless, clean energy. The theoretical and numerical modeling of tokamak plasmas is simultaneously an essential component of effective reactor design, and a great research barrier. Tokamak operational conditions exhibit comparatively low Knudsen numbers. Kinetic effects, including kinetic waves and instabilities, Landau damping, bump-on-tail instabilities and more, are therefore highly influential in tokamak plasma dynamics. Purely fluid models are inherently incapable of capturing these effects, whereas the high dimensionality in purely kinetic models render them practically intractable for most relevant purposes.

        We consider a $\delta\!f$ decomposition model, with a macroscopic fluid background and microscopic kinetic correction, both fully coupled to each other. A similar manner of discretization is proposed to that used in the recent \texttt{STRUPHY} code \cite{Holderied_Possanner_Wang_2021, Holderied_2022, Li_et_al_2023} with a finite-element model for the background and a pseudo-particle/PiC model for the correction.

        The fluid background satisfies the full, non-linear, resistive, compressible, Hall MHD equations. \cite{Laakmann_Hu_Farrell_2022} introduces finite-element(-in-space) implicit timesteppers for the incompressible analogue to this system with structure-preserving (SP) properties in the ideal case, alongside parameter-robust preconditioners. We show that these timesteppers can derive from a finite-element-in-time (FET) (and finite-element-in-space) interpretation. The benefits of this reformulation are discussed, including the derivation of timesteppers that are higher order in time, and the quantifiable dissipative SP properties in the non-ideal, resistive case.
        
        We discuss possible options for extending this FET approach to timesteppers for the compressible case.

        The kinetic corrections satisfy linearized Boltzmann equations. Using a Lénard--Bernstein collision operator, these take Fokker--Planck-like forms \cite{Fokker_1914, Planck_1917} wherein pseudo-particles in the numerical model obey the neoclassical transport equations, with particle-independent Brownian drift terms. This offers a rigorous methodology for incorporating collisions into the particle transport model, without coupling the equations of motions for each particle.
        
        Works by Chen, Chacón et al. \cite{Chen_Chacón_Barnes_2011, Chacón_Chen_Barnes_2013, Chen_Chacón_2014, Chen_Chacón_2015} have developed structure-preserving particle pushers for neoclassical transport in the Vlasov equations, derived from Crank--Nicolson integrators. We show these too can can derive from a FET interpretation, similarly offering potential extensions to higher-order-in-time particle pushers. The FET formulation is used also to consider how the stochastic drift terms can be incorporated into the pushers. Stochastic gyrokinetic expansions are also discussed.

        Different options for the numerical implementation of these schemes are considered.

        Due to the efficacy of FET in the development of SP timesteppers for both the fluid and kinetic component, we hope this approach will prove effective in the future for developing SP timesteppers for the full hybrid model. We hope this will give us the opportunity to incorporate previously inaccessible kinetic effects into the highly effective, modern, finite-element MHD models.
    \end{abstract}
    
    
    \newpage
    \tableofcontents
    
    
    \newpage
    \pagenumbering{arabic}
    %\linenumbers\renewcommand\thelinenumber{\color{black!50}\arabic{linenumber}}
            \input{0 - introduction/main.tex}
        \part{Research}
            \input{1 - low-noise PiC models/main.tex}
            \input{2 - kinetic component/main.tex}
            \input{3 - fluid component/main.tex}
            \input{4 - numerical implementation/main.tex}
        \part{Project Overview}
            \input{5 - research plan/main.tex}
            \input{6 - summary/main.tex}
    
    
    %\section{}
    \newpage
    \pagenumbering{gobble}
        \printbibliography


    \newpage
    \pagenumbering{roman}
    \appendix
        \part{Appendices}
            \input{8 - Hilbert complexes/main.tex}
            \input{9 - weak conservation proofs/main.tex}
\end{document}

            \documentclass[12pt, a4paper]{report}

\input{template/main.tex}

\title{\BA{Title in Progress...}}
\author{Boris Andrews}
\affil{Mathematical Institute, University of Oxford}
\date{\today}


\begin{document}
    \pagenumbering{gobble}
    \maketitle
    
    
    \begin{abstract}
        Magnetic confinement reactors---in particular tokamaks---offer one of the most promising options for achieving practical nuclear fusion, with the potential to provide virtually limitless, clean energy. The theoretical and numerical modeling of tokamak plasmas is simultaneously an essential component of effective reactor design, and a great research barrier. Tokamak operational conditions exhibit comparatively low Knudsen numbers. Kinetic effects, including kinetic waves and instabilities, Landau damping, bump-on-tail instabilities and more, are therefore highly influential in tokamak plasma dynamics. Purely fluid models are inherently incapable of capturing these effects, whereas the high dimensionality in purely kinetic models render them practically intractable for most relevant purposes.

        We consider a $\delta\!f$ decomposition model, with a macroscopic fluid background and microscopic kinetic correction, both fully coupled to each other. A similar manner of discretization is proposed to that used in the recent \texttt{STRUPHY} code \cite{Holderied_Possanner_Wang_2021, Holderied_2022, Li_et_al_2023} with a finite-element model for the background and a pseudo-particle/PiC model for the correction.

        The fluid background satisfies the full, non-linear, resistive, compressible, Hall MHD equations. \cite{Laakmann_Hu_Farrell_2022} introduces finite-element(-in-space) implicit timesteppers for the incompressible analogue to this system with structure-preserving (SP) properties in the ideal case, alongside parameter-robust preconditioners. We show that these timesteppers can derive from a finite-element-in-time (FET) (and finite-element-in-space) interpretation. The benefits of this reformulation are discussed, including the derivation of timesteppers that are higher order in time, and the quantifiable dissipative SP properties in the non-ideal, resistive case.
        
        We discuss possible options for extending this FET approach to timesteppers for the compressible case.

        The kinetic corrections satisfy linearized Boltzmann equations. Using a Lénard--Bernstein collision operator, these take Fokker--Planck-like forms \cite{Fokker_1914, Planck_1917} wherein pseudo-particles in the numerical model obey the neoclassical transport equations, with particle-independent Brownian drift terms. This offers a rigorous methodology for incorporating collisions into the particle transport model, without coupling the equations of motions for each particle.
        
        Works by Chen, Chacón et al. \cite{Chen_Chacón_Barnes_2011, Chacón_Chen_Barnes_2013, Chen_Chacón_2014, Chen_Chacón_2015} have developed structure-preserving particle pushers for neoclassical transport in the Vlasov equations, derived from Crank--Nicolson integrators. We show these too can can derive from a FET interpretation, similarly offering potential extensions to higher-order-in-time particle pushers. The FET formulation is used also to consider how the stochastic drift terms can be incorporated into the pushers. Stochastic gyrokinetic expansions are also discussed.

        Different options for the numerical implementation of these schemes are considered.

        Due to the efficacy of FET in the development of SP timesteppers for both the fluid and kinetic component, we hope this approach will prove effective in the future for developing SP timesteppers for the full hybrid model. We hope this will give us the opportunity to incorporate previously inaccessible kinetic effects into the highly effective, modern, finite-element MHD models.
    \end{abstract}
    
    
    \newpage
    \tableofcontents
    
    
    \newpage
    \pagenumbering{arabic}
    %\linenumbers\renewcommand\thelinenumber{\color{black!50}\arabic{linenumber}}
            \input{0 - introduction/main.tex}
        \part{Research}
            \input{1 - low-noise PiC models/main.tex}
            \input{2 - kinetic component/main.tex}
            \input{3 - fluid component/main.tex}
            \input{4 - numerical implementation/main.tex}
        \part{Project Overview}
            \input{5 - research plan/main.tex}
            \input{6 - summary/main.tex}
    
    
    %\section{}
    \newpage
    \pagenumbering{gobble}
        \printbibliography


    \newpage
    \pagenumbering{roman}
    \appendix
        \part{Appendices}
            \input{8 - Hilbert complexes/main.tex}
            \input{9 - weak conservation proofs/main.tex}
\end{document}

        \part{Project Overview}
            \documentclass[12pt, a4paper]{report}

\input{template/main.tex}

\title{\BA{Title in Progress...}}
\author{Boris Andrews}
\affil{Mathematical Institute, University of Oxford}
\date{\today}


\begin{document}
    \pagenumbering{gobble}
    \maketitle
    
    
    \begin{abstract}
        Magnetic confinement reactors---in particular tokamaks---offer one of the most promising options for achieving practical nuclear fusion, with the potential to provide virtually limitless, clean energy. The theoretical and numerical modeling of tokamak plasmas is simultaneously an essential component of effective reactor design, and a great research barrier. Tokamak operational conditions exhibit comparatively low Knudsen numbers. Kinetic effects, including kinetic waves and instabilities, Landau damping, bump-on-tail instabilities and more, are therefore highly influential in tokamak plasma dynamics. Purely fluid models are inherently incapable of capturing these effects, whereas the high dimensionality in purely kinetic models render them practically intractable for most relevant purposes.

        We consider a $\delta\!f$ decomposition model, with a macroscopic fluid background and microscopic kinetic correction, both fully coupled to each other. A similar manner of discretization is proposed to that used in the recent \texttt{STRUPHY} code \cite{Holderied_Possanner_Wang_2021, Holderied_2022, Li_et_al_2023} with a finite-element model for the background and a pseudo-particle/PiC model for the correction.

        The fluid background satisfies the full, non-linear, resistive, compressible, Hall MHD equations. \cite{Laakmann_Hu_Farrell_2022} introduces finite-element(-in-space) implicit timesteppers for the incompressible analogue to this system with structure-preserving (SP) properties in the ideal case, alongside parameter-robust preconditioners. We show that these timesteppers can derive from a finite-element-in-time (FET) (and finite-element-in-space) interpretation. The benefits of this reformulation are discussed, including the derivation of timesteppers that are higher order in time, and the quantifiable dissipative SP properties in the non-ideal, resistive case.
        
        We discuss possible options for extending this FET approach to timesteppers for the compressible case.

        The kinetic corrections satisfy linearized Boltzmann equations. Using a Lénard--Bernstein collision operator, these take Fokker--Planck-like forms \cite{Fokker_1914, Planck_1917} wherein pseudo-particles in the numerical model obey the neoclassical transport equations, with particle-independent Brownian drift terms. This offers a rigorous methodology for incorporating collisions into the particle transport model, without coupling the equations of motions for each particle.
        
        Works by Chen, Chacón et al. \cite{Chen_Chacón_Barnes_2011, Chacón_Chen_Barnes_2013, Chen_Chacón_2014, Chen_Chacón_2015} have developed structure-preserving particle pushers for neoclassical transport in the Vlasov equations, derived from Crank--Nicolson integrators. We show these too can can derive from a FET interpretation, similarly offering potential extensions to higher-order-in-time particle pushers. The FET formulation is used also to consider how the stochastic drift terms can be incorporated into the pushers. Stochastic gyrokinetic expansions are also discussed.

        Different options for the numerical implementation of these schemes are considered.

        Due to the efficacy of FET in the development of SP timesteppers for both the fluid and kinetic component, we hope this approach will prove effective in the future for developing SP timesteppers for the full hybrid model. We hope this will give us the opportunity to incorporate previously inaccessible kinetic effects into the highly effective, modern, finite-element MHD models.
    \end{abstract}
    
    
    \newpage
    \tableofcontents
    
    
    \newpage
    \pagenumbering{arabic}
    %\linenumbers\renewcommand\thelinenumber{\color{black!50}\arabic{linenumber}}
            \input{0 - introduction/main.tex}
        \part{Research}
            \input{1 - low-noise PiC models/main.tex}
            \input{2 - kinetic component/main.tex}
            \input{3 - fluid component/main.tex}
            \input{4 - numerical implementation/main.tex}
        \part{Project Overview}
            \input{5 - research plan/main.tex}
            \input{6 - summary/main.tex}
    
    
    %\section{}
    \newpage
    \pagenumbering{gobble}
        \printbibliography


    \newpage
    \pagenumbering{roman}
    \appendix
        \part{Appendices}
            \input{8 - Hilbert complexes/main.tex}
            \input{9 - weak conservation proofs/main.tex}
\end{document}

            \documentclass[12pt, a4paper]{report}

\input{template/main.tex}

\title{\BA{Title in Progress...}}
\author{Boris Andrews}
\affil{Mathematical Institute, University of Oxford}
\date{\today}


\begin{document}
    \pagenumbering{gobble}
    \maketitle
    
    
    \begin{abstract}
        Magnetic confinement reactors---in particular tokamaks---offer one of the most promising options for achieving practical nuclear fusion, with the potential to provide virtually limitless, clean energy. The theoretical and numerical modeling of tokamak plasmas is simultaneously an essential component of effective reactor design, and a great research barrier. Tokamak operational conditions exhibit comparatively low Knudsen numbers. Kinetic effects, including kinetic waves and instabilities, Landau damping, bump-on-tail instabilities and more, are therefore highly influential in tokamak plasma dynamics. Purely fluid models are inherently incapable of capturing these effects, whereas the high dimensionality in purely kinetic models render them practically intractable for most relevant purposes.

        We consider a $\delta\!f$ decomposition model, with a macroscopic fluid background and microscopic kinetic correction, both fully coupled to each other. A similar manner of discretization is proposed to that used in the recent \texttt{STRUPHY} code \cite{Holderied_Possanner_Wang_2021, Holderied_2022, Li_et_al_2023} with a finite-element model for the background and a pseudo-particle/PiC model for the correction.

        The fluid background satisfies the full, non-linear, resistive, compressible, Hall MHD equations. \cite{Laakmann_Hu_Farrell_2022} introduces finite-element(-in-space) implicit timesteppers for the incompressible analogue to this system with structure-preserving (SP) properties in the ideal case, alongside parameter-robust preconditioners. We show that these timesteppers can derive from a finite-element-in-time (FET) (and finite-element-in-space) interpretation. The benefits of this reformulation are discussed, including the derivation of timesteppers that are higher order in time, and the quantifiable dissipative SP properties in the non-ideal, resistive case.
        
        We discuss possible options for extending this FET approach to timesteppers for the compressible case.

        The kinetic corrections satisfy linearized Boltzmann equations. Using a Lénard--Bernstein collision operator, these take Fokker--Planck-like forms \cite{Fokker_1914, Planck_1917} wherein pseudo-particles in the numerical model obey the neoclassical transport equations, with particle-independent Brownian drift terms. This offers a rigorous methodology for incorporating collisions into the particle transport model, without coupling the equations of motions for each particle.
        
        Works by Chen, Chacón et al. \cite{Chen_Chacón_Barnes_2011, Chacón_Chen_Barnes_2013, Chen_Chacón_2014, Chen_Chacón_2015} have developed structure-preserving particle pushers for neoclassical transport in the Vlasov equations, derived from Crank--Nicolson integrators. We show these too can can derive from a FET interpretation, similarly offering potential extensions to higher-order-in-time particle pushers. The FET formulation is used also to consider how the stochastic drift terms can be incorporated into the pushers. Stochastic gyrokinetic expansions are also discussed.

        Different options for the numerical implementation of these schemes are considered.

        Due to the efficacy of FET in the development of SP timesteppers for both the fluid and kinetic component, we hope this approach will prove effective in the future for developing SP timesteppers for the full hybrid model. We hope this will give us the opportunity to incorporate previously inaccessible kinetic effects into the highly effective, modern, finite-element MHD models.
    \end{abstract}
    
    
    \newpage
    \tableofcontents
    
    
    \newpage
    \pagenumbering{arabic}
    %\linenumbers\renewcommand\thelinenumber{\color{black!50}\arabic{linenumber}}
            \input{0 - introduction/main.tex}
        \part{Research}
            \input{1 - low-noise PiC models/main.tex}
            \input{2 - kinetic component/main.tex}
            \input{3 - fluid component/main.tex}
            \input{4 - numerical implementation/main.tex}
        \part{Project Overview}
            \input{5 - research plan/main.tex}
            \input{6 - summary/main.tex}
    
    
    %\section{}
    \newpage
    \pagenumbering{gobble}
        \printbibliography


    \newpage
    \pagenumbering{roman}
    \appendix
        \part{Appendices}
            \input{8 - Hilbert complexes/main.tex}
            \input{9 - weak conservation proofs/main.tex}
\end{document}

    
    
    %\section{}
    \newpage
    \pagenumbering{gobble}
        \printbibliography


    \newpage
    \pagenumbering{roman}
    \appendix
        \part{Appendices}
            \documentclass[12pt, a4paper]{report}

\input{template/main.tex}

\title{\BA{Title in Progress...}}
\author{Boris Andrews}
\affil{Mathematical Institute, University of Oxford}
\date{\today}


\begin{document}
    \pagenumbering{gobble}
    \maketitle
    
    
    \begin{abstract}
        Magnetic confinement reactors---in particular tokamaks---offer one of the most promising options for achieving practical nuclear fusion, with the potential to provide virtually limitless, clean energy. The theoretical and numerical modeling of tokamak plasmas is simultaneously an essential component of effective reactor design, and a great research barrier. Tokamak operational conditions exhibit comparatively low Knudsen numbers. Kinetic effects, including kinetic waves and instabilities, Landau damping, bump-on-tail instabilities and more, are therefore highly influential in tokamak plasma dynamics. Purely fluid models are inherently incapable of capturing these effects, whereas the high dimensionality in purely kinetic models render them practically intractable for most relevant purposes.

        We consider a $\delta\!f$ decomposition model, with a macroscopic fluid background and microscopic kinetic correction, both fully coupled to each other. A similar manner of discretization is proposed to that used in the recent \texttt{STRUPHY} code \cite{Holderied_Possanner_Wang_2021, Holderied_2022, Li_et_al_2023} with a finite-element model for the background and a pseudo-particle/PiC model for the correction.

        The fluid background satisfies the full, non-linear, resistive, compressible, Hall MHD equations. \cite{Laakmann_Hu_Farrell_2022} introduces finite-element(-in-space) implicit timesteppers for the incompressible analogue to this system with structure-preserving (SP) properties in the ideal case, alongside parameter-robust preconditioners. We show that these timesteppers can derive from a finite-element-in-time (FET) (and finite-element-in-space) interpretation. The benefits of this reformulation are discussed, including the derivation of timesteppers that are higher order in time, and the quantifiable dissipative SP properties in the non-ideal, resistive case.
        
        We discuss possible options for extending this FET approach to timesteppers for the compressible case.

        The kinetic corrections satisfy linearized Boltzmann equations. Using a Lénard--Bernstein collision operator, these take Fokker--Planck-like forms \cite{Fokker_1914, Planck_1917} wherein pseudo-particles in the numerical model obey the neoclassical transport equations, with particle-independent Brownian drift terms. This offers a rigorous methodology for incorporating collisions into the particle transport model, without coupling the equations of motions for each particle.
        
        Works by Chen, Chacón et al. \cite{Chen_Chacón_Barnes_2011, Chacón_Chen_Barnes_2013, Chen_Chacón_2014, Chen_Chacón_2015} have developed structure-preserving particle pushers for neoclassical transport in the Vlasov equations, derived from Crank--Nicolson integrators. We show these too can can derive from a FET interpretation, similarly offering potential extensions to higher-order-in-time particle pushers. The FET formulation is used also to consider how the stochastic drift terms can be incorporated into the pushers. Stochastic gyrokinetic expansions are also discussed.

        Different options for the numerical implementation of these schemes are considered.

        Due to the efficacy of FET in the development of SP timesteppers for both the fluid and kinetic component, we hope this approach will prove effective in the future for developing SP timesteppers for the full hybrid model. We hope this will give us the opportunity to incorporate previously inaccessible kinetic effects into the highly effective, modern, finite-element MHD models.
    \end{abstract}
    
    
    \newpage
    \tableofcontents
    
    
    \newpage
    \pagenumbering{arabic}
    %\linenumbers\renewcommand\thelinenumber{\color{black!50}\arabic{linenumber}}
            \input{0 - introduction/main.tex}
        \part{Research}
            \input{1 - low-noise PiC models/main.tex}
            \input{2 - kinetic component/main.tex}
            \input{3 - fluid component/main.tex}
            \input{4 - numerical implementation/main.tex}
        \part{Project Overview}
            \input{5 - research plan/main.tex}
            \input{6 - summary/main.tex}
    
    
    %\section{}
    \newpage
    \pagenumbering{gobble}
        \printbibliography


    \newpage
    \pagenumbering{roman}
    \appendix
        \part{Appendices}
            \input{8 - Hilbert complexes/main.tex}
            \input{9 - weak conservation proofs/main.tex}
\end{document}

            \documentclass[12pt, a4paper]{report}

\input{template/main.tex}

\title{\BA{Title in Progress...}}
\author{Boris Andrews}
\affil{Mathematical Institute, University of Oxford}
\date{\today}


\begin{document}
    \pagenumbering{gobble}
    \maketitle
    
    
    \begin{abstract}
        Magnetic confinement reactors---in particular tokamaks---offer one of the most promising options for achieving practical nuclear fusion, with the potential to provide virtually limitless, clean energy. The theoretical and numerical modeling of tokamak plasmas is simultaneously an essential component of effective reactor design, and a great research barrier. Tokamak operational conditions exhibit comparatively low Knudsen numbers. Kinetic effects, including kinetic waves and instabilities, Landau damping, bump-on-tail instabilities and more, are therefore highly influential in tokamak plasma dynamics. Purely fluid models are inherently incapable of capturing these effects, whereas the high dimensionality in purely kinetic models render them practically intractable for most relevant purposes.

        We consider a $\delta\!f$ decomposition model, with a macroscopic fluid background and microscopic kinetic correction, both fully coupled to each other. A similar manner of discretization is proposed to that used in the recent \texttt{STRUPHY} code \cite{Holderied_Possanner_Wang_2021, Holderied_2022, Li_et_al_2023} with a finite-element model for the background and a pseudo-particle/PiC model for the correction.

        The fluid background satisfies the full, non-linear, resistive, compressible, Hall MHD equations. \cite{Laakmann_Hu_Farrell_2022} introduces finite-element(-in-space) implicit timesteppers for the incompressible analogue to this system with structure-preserving (SP) properties in the ideal case, alongside parameter-robust preconditioners. We show that these timesteppers can derive from a finite-element-in-time (FET) (and finite-element-in-space) interpretation. The benefits of this reformulation are discussed, including the derivation of timesteppers that are higher order in time, and the quantifiable dissipative SP properties in the non-ideal, resistive case.
        
        We discuss possible options for extending this FET approach to timesteppers for the compressible case.

        The kinetic corrections satisfy linearized Boltzmann equations. Using a Lénard--Bernstein collision operator, these take Fokker--Planck-like forms \cite{Fokker_1914, Planck_1917} wherein pseudo-particles in the numerical model obey the neoclassical transport equations, with particle-independent Brownian drift terms. This offers a rigorous methodology for incorporating collisions into the particle transport model, without coupling the equations of motions for each particle.
        
        Works by Chen, Chacón et al. \cite{Chen_Chacón_Barnes_2011, Chacón_Chen_Barnes_2013, Chen_Chacón_2014, Chen_Chacón_2015} have developed structure-preserving particle pushers for neoclassical transport in the Vlasov equations, derived from Crank--Nicolson integrators. We show these too can can derive from a FET interpretation, similarly offering potential extensions to higher-order-in-time particle pushers. The FET formulation is used also to consider how the stochastic drift terms can be incorporated into the pushers. Stochastic gyrokinetic expansions are also discussed.

        Different options for the numerical implementation of these schemes are considered.

        Due to the efficacy of FET in the development of SP timesteppers for both the fluid and kinetic component, we hope this approach will prove effective in the future for developing SP timesteppers for the full hybrid model. We hope this will give us the opportunity to incorporate previously inaccessible kinetic effects into the highly effective, modern, finite-element MHD models.
    \end{abstract}
    
    
    \newpage
    \tableofcontents
    
    
    \newpage
    \pagenumbering{arabic}
    %\linenumbers\renewcommand\thelinenumber{\color{black!50}\arabic{linenumber}}
            \input{0 - introduction/main.tex}
        \part{Research}
            \input{1 - low-noise PiC models/main.tex}
            \input{2 - kinetic component/main.tex}
            \input{3 - fluid component/main.tex}
            \input{4 - numerical implementation/main.tex}
        \part{Project Overview}
            \input{5 - research plan/main.tex}
            \input{6 - summary/main.tex}
    
    
    %\section{}
    \newpage
    \pagenumbering{gobble}
        \printbibliography


    \newpage
    \pagenumbering{roman}
    \appendix
        \part{Appendices}
            \input{8 - Hilbert complexes/main.tex}
            \input{9 - weak conservation proofs/main.tex}
\end{document}

\end{document}

\end{document}

            \documentclass[12pt, a4paper]{report}

\documentclass[12pt, a4paper]{report}

\documentclass[12pt, a4paper]{report}

\input{template/main.tex}

\title{\BA{Title in Progress...}}
\author{Boris Andrews}
\affil{Mathematical Institute, University of Oxford}
\date{\today}


\begin{document}
    \pagenumbering{gobble}
    \maketitle
    
    
    \begin{abstract}
        Magnetic confinement reactors---in particular tokamaks---offer one of the most promising options for achieving practical nuclear fusion, with the potential to provide virtually limitless, clean energy. The theoretical and numerical modeling of tokamak plasmas is simultaneously an essential component of effective reactor design, and a great research barrier. Tokamak operational conditions exhibit comparatively low Knudsen numbers. Kinetic effects, including kinetic waves and instabilities, Landau damping, bump-on-tail instabilities and more, are therefore highly influential in tokamak plasma dynamics. Purely fluid models are inherently incapable of capturing these effects, whereas the high dimensionality in purely kinetic models render them practically intractable for most relevant purposes.

        We consider a $\delta\!f$ decomposition model, with a macroscopic fluid background and microscopic kinetic correction, both fully coupled to each other. A similar manner of discretization is proposed to that used in the recent \texttt{STRUPHY} code \cite{Holderied_Possanner_Wang_2021, Holderied_2022, Li_et_al_2023} with a finite-element model for the background and a pseudo-particle/PiC model for the correction.

        The fluid background satisfies the full, non-linear, resistive, compressible, Hall MHD equations. \cite{Laakmann_Hu_Farrell_2022} introduces finite-element(-in-space) implicit timesteppers for the incompressible analogue to this system with structure-preserving (SP) properties in the ideal case, alongside parameter-robust preconditioners. We show that these timesteppers can derive from a finite-element-in-time (FET) (and finite-element-in-space) interpretation. The benefits of this reformulation are discussed, including the derivation of timesteppers that are higher order in time, and the quantifiable dissipative SP properties in the non-ideal, resistive case.
        
        We discuss possible options for extending this FET approach to timesteppers for the compressible case.

        The kinetic corrections satisfy linearized Boltzmann equations. Using a Lénard--Bernstein collision operator, these take Fokker--Planck-like forms \cite{Fokker_1914, Planck_1917} wherein pseudo-particles in the numerical model obey the neoclassical transport equations, with particle-independent Brownian drift terms. This offers a rigorous methodology for incorporating collisions into the particle transport model, without coupling the equations of motions for each particle.
        
        Works by Chen, Chacón et al. \cite{Chen_Chacón_Barnes_2011, Chacón_Chen_Barnes_2013, Chen_Chacón_2014, Chen_Chacón_2015} have developed structure-preserving particle pushers for neoclassical transport in the Vlasov equations, derived from Crank--Nicolson integrators. We show these too can can derive from a FET interpretation, similarly offering potential extensions to higher-order-in-time particle pushers. The FET formulation is used also to consider how the stochastic drift terms can be incorporated into the pushers. Stochastic gyrokinetic expansions are also discussed.

        Different options for the numerical implementation of these schemes are considered.

        Due to the efficacy of FET in the development of SP timesteppers for both the fluid and kinetic component, we hope this approach will prove effective in the future for developing SP timesteppers for the full hybrid model. We hope this will give us the opportunity to incorporate previously inaccessible kinetic effects into the highly effective, modern, finite-element MHD models.
    \end{abstract}
    
    
    \newpage
    \tableofcontents
    
    
    \newpage
    \pagenumbering{arabic}
    %\linenumbers\renewcommand\thelinenumber{\color{black!50}\arabic{linenumber}}
            \input{0 - introduction/main.tex}
        \part{Research}
            \input{1 - low-noise PiC models/main.tex}
            \input{2 - kinetic component/main.tex}
            \input{3 - fluid component/main.tex}
            \input{4 - numerical implementation/main.tex}
        \part{Project Overview}
            \input{5 - research plan/main.tex}
            \input{6 - summary/main.tex}
    
    
    %\section{}
    \newpage
    \pagenumbering{gobble}
        \printbibliography


    \newpage
    \pagenumbering{roman}
    \appendix
        \part{Appendices}
            \input{8 - Hilbert complexes/main.tex}
            \input{9 - weak conservation proofs/main.tex}
\end{document}


\title{\BA{Title in Progress...}}
\author{Boris Andrews}
\affil{Mathematical Institute, University of Oxford}
\date{\today}


\begin{document}
    \pagenumbering{gobble}
    \maketitle
    
    
    \begin{abstract}
        Magnetic confinement reactors---in particular tokamaks---offer one of the most promising options for achieving practical nuclear fusion, with the potential to provide virtually limitless, clean energy. The theoretical and numerical modeling of tokamak plasmas is simultaneously an essential component of effective reactor design, and a great research barrier. Tokamak operational conditions exhibit comparatively low Knudsen numbers. Kinetic effects, including kinetic waves and instabilities, Landau damping, bump-on-tail instabilities and more, are therefore highly influential in tokamak plasma dynamics. Purely fluid models are inherently incapable of capturing these effects, whereas the high dimensionality in purely kinetic models render them practically intractable for most relevant purposes.

        We consider a $\delta\!f$ decomposition model, with a macroscopic fluid background and microscopic kinetic correction, both fully coupled to each other. A similar manner of discretization is proposed to that used in the recent \texttt{STRUPHY} code \cite{Holderied_Possanner_Wang_2021, Holderied_2022, Li_et_al_2023} with a finite-element model for the background and a pseudo-particle/PiC model for the correction.

        The fluid background satisfies the full, non-linear, resistive, compressible, Hall MHD equations. \cite{Laakmann_Hu_Farrell_2022} introduces finite-element(-in-space) implicit timesteppers for the incompressible analogue to this system with structure-preserving (SP) properties in the ideal case, alongside parameter-robust preconditioners. We show that these timesteppers can derive from a finite-element-in-time (FET) (and finite-element-in-space) interpretation. The benefits of this reformulation are discussed, including the derivation of timesteppers that are higher order in time, and the quantifiable dissipative SP properties in the non-ideal, resistive case.
        
        We discuss possible options for extending this FET approach to timesteppers for the compressible case.

        The kinetic corrections satisfy linearized Boltzmann equations. Using a Lénard--Bernstein collision operator, these take Fokker--Planck-like forms \cite{Fokker_1914, Planck_1917} wherein pseudo-particles in the numerical model obey the neoclassical transport equations, with particle-independent Brownian drift terms. This offers a rigorous methodology for incorporating collisions into the particle transport model, without coupling the equations of motions for each particle.
        
        Works by Chen, Chacón et al. \cite{Chen_Chacón_Barnes_2011, Chacón_Chen_Barnes_2013, Chen_Chacón_2014, Chen_Chacón_2015} have developed structure-preserving particle pushers for neoclassical transport in the Vlasov equations, derived from Crank--Nicolson integrators. We show these too can can derive from a FET interpretation, similarly offering potential extensions to higher-order-in-time particle pushers. The FET formulation is used also to consider how the stochastic drift terms can be incorporated into the pushers. Stochastic gyrokinetic expansions are also discussed.

        Different options for the numerical implementation of these schemes are considered.

        Due to the efficacy of FET in the development of SP timesteppers for both the fluid and kinetic component, we hope this approach will prove effective in the future for developing SP timesteppers for the full hybrid model. We hope this will give us the opportunity to incorporate previously inaccessible kinetic effects into the highly effective, modern, finite-element MHD models.
    \end{abstract}
    
    
    \newpage
    \tableofcontents
    
    
    \newpage
    \pagenumbering{arabic}
    %\linenumbers\renewcommand\thelinenumber{\color{black!50}\arabic{linenumber}}
            \documentclass[12pt, a4paper]{report}

\input{template/main.tex}

\title{\BA{Title in Progress...}}
\author{Boris Andrews}
\affil{Mathematical Institute, University of Oxford}
\date{\today}


\begin{document}
    \pagenumbering{gobble}
    \maketitle
    
    
    \begin{abstract}
        Magnetic confinement reactors---in particular tokamaks---offer one of the most promising options for achieving practical nuclear fusion, with the potential to provide virtually limitless, clean energy. The theoretical and numerical modeling of tokamak plasmas is simultaneously an essential component of effective reactor design, and a great research barrier. Tokamak operational conditions exhibit comparatively low Knudsen numbers. Kinetic effects, including kinetic waves and instabilities, Landau damping, bump-on-tail instabilities and more, are therefore highly influential in tokamak plasma dynamics. Purely fluid models are inherently incapable of capturing these effects, whereas the high dimensionality in purely kinetic models render them practically intractable for most relevant purposes.

        We consider a $\delta\!f$ decomposition model, with a macroscopic fluid background and microscopic kinetic correction, both fully coupled to each other. A similar manner of discretization is proposed to that used in the recent \texttt{STRUPHY} code \cite{Holderied_Possanner_Wang_2021, Holderied_2022, Li_et_al_2023} with a finite-element model for the background and a pseudo-particle/PiC model for the correction.

        The fluid background satisfies the full, non-linear, resistive, compressible, Hall MHD equations. \cite{Laakmann_Hu_Farrell_2022} introduces finite-element(-in-space) implicit timesteppers for the incompressible analogue to this system with structure-preserving (SP) properties in the ideal case, alongside parameter-robust preconditioners. We show that these timesteppers can derive from a finite-element-in-time (FET) (and finite-element-in-space) interpretation. The benefits of this reformulation are discussed, including the derivation of timesteppers that are higher order in time, and the quantifiable dissipative SP properties in the non-ideal, resistive case.
        
        We discuss possible options for extending this FET approach to timesteppers for the compressible case.

        The kinetic corrections satisfy linearized Boltzmann equations. Using a Lénard--Bernstein collision operator, these take Fokker--Planck-like forms \cite{Fokker_1914, Planck_1917} wherein pseudo-particles in the numerical model obey the neoclassical transport equations, with particle-independent Brownian drift terms. This offers a rigorous methodology for incorporating collisions into the particle transport model, without coupling the equations of motions for each particle.
        
        Works by Chen, Chacón et al. \cite{Chen_Chacón_Barnes_2011, Chacón_Chen_Barnes_2013, Chen_Chacón_2014, Chen_Chacón_2015} have developed structure-preserving particle pushers for neoclassical transport in the Vlasov equations, derived from Crank--Nicolson integrators. We show these too can can derive from a FET interpretation, similarly offering potential extensions to higher-order-in-time particle pushers. The FET formulation is used also to consider how the stochastic drift terms can be incorporated into the pushers. Stochastic gyrokinetic expansions are also discussed.

        Different options for the numerical implementation of these schemes are considered.

        Due to the efficacy of FET in the development of SP timesteppers for both the fluid and kinetic component, we hope this approach will prove effective in the future for developing SP timesteppers for the full hybrid model. We hope this will give us the opportunity to incorporate previously inaccessible kinetic effects into the highly effective, modern, finite-element MHD models.
    \end{abstract}
    
    
    \newpage
    \tableofcontents
    
    
    \newpage
    \pagenumbering{arabic}
    %\linenumbers\renewcommand\thelinenumber{\color{black!50}\arabic{linenumber}}
            \input{0 - introduction/main.tex}
        \part{Research}
            \input{1 - low-noise PiC models/main.tex}
            \input{2 - kinetic component/main.tex}
            \input{3 - fluid component/main.tex}
            \input{4 - numerical implementation/main.tex}
        \part{Project Overview}
            \input{5 - research plan/main.tex}
            \input{6 - summary/main.tex}
    
    
    %\section{}
    \newpage
    \pagenumbering{gobble}
        \printbibliography


    \newpage
    \pagenumbering{roman}
    \appendix
        \part{Appendices}
            \input{8 - Hilbert complexes/main.tex}
            \input{9 - weak conservation proofs/main.tex}
\end{document}

        \part{Research}
            \documentclass[12pt, a4paper]{report}

\input{template/main.tex}

\title{\BA{Title in Progress...}}
\author{Boris Andrews}
\affil{Mathematical Institute, University of Oxford}
\date{\today}


\begin{document}
    \pagenumbering{gobble}
    \maketitle
    
    
    \begin{abstract}
        Magnetic confinement reactors---in particular tokamaks---offer one of the most promising options for achieving practical nuclear fusion, with the potential to provide virtually limitless, clean energy. The theoretical and numerical modeling of tokamak plasmas is simultaneously an essential component of effective reactor design, and a great research barrier. Tokamak operational conditions exhibit comparatively low Knudsen numbers. Kinetic effects, including kinetic waves and instabilities, Landau damping, bump-on-tail instabilities and more, are therefore highly influential in tokamak plasma dynamics. Purely fluid models are inherently incapable of capturing these effects, whereas the high dimensionality in purely kinetic models render them practically intractable for most relevant purposes.

        We consider a $\delta\!f$ decomposition model, with a macroscopic fluid background and microscopic kinetic correction, both fully coupled to each other. A similar manner of discretization is proposed to that used in the recent \texttt{STRUPHY} code \cite{Holderied_Possanner_Wang_2021, Holderied_2022, Li_et_al_2023} with a finite-element model for the background and a pseudo-particle/PiC model for the correction.

        The fluid background satisfies the full, non-linear, resistive, compressible, Hall MHD equations. \cite{Laakmann_Hu_Farrell_2022} introduces finite-element(-in-space) implicit timesteppers for the incompressible analogue to this system with structure-preserving (SP) properties in the ideal case, alongside parameter-robust preconditioners. We show that these timesteppers can derive from a finite-element-in-time (FET) (and finite-element-in-space) interpretation. The benefits of this reformulation are discussed, including the derivation of timesteppers that are higher order in time, and the quantifiable dissipative SP properties in the non-ideal, resistive case.
        
        We discuss possible options for extending this FET approach to timesteppers for the compressible case.

        The kinetic corrections satisfy linearized Boltzmann equations. Using a Lénard--Bernstein collision operator, these take Fokker--Planck-like forms \cite{Fokker_1914, Planck_1917} wherein pseudo-particles in the numerical model obey the neoclassical transport equations, with particle-independent Brownian drift terms. This offers a rigorous methodology for incorporating collisions into the particle transport model, without coupling the equations of motions for each particle.
        
        Works by Chen, Chacón et al. \cite{Chen_Chacón_Barnes_2011, Chacón_Chen_Barnes_2013, Chen_Chacón_2014, Chen_Chacón_2015} have developed structure-preserving particle pushers for neoclassical transport in the Vlasov equations, derived from Crank--Nicolson integrators. We show these too can can derive from a FET interpretation, similarly offering potential extensions to higher-order-in-time particle pushers. The FET formulation is used also to consider how the stochastic drift terms can be incorporated into the pushers. Stochastic gyrokinetic expansions are also discussed.

        Different options for the numerical implementation of these schemes are considered.

        Due to the efficacy of FET in the development of SP timesteppers for both the fluid and kinetic component, we hope this approach will prove effective in the future for developing SP timesteppers for the full hybrid model. We hope this will give us the opportunity to incorporate previously inaccessible kinetic effects into the highly effective, modern, finite-element MHD models.
    \end{abstract}
    
    
    \newpage
    \tableofcontents
    
    
    \newpage
    \pagenumbering{arabic}
    %\linenumbers\renewcommand\thelinenumber{\color{black!50}\arabic{linenumber}}
            \input{0 - introduction/main.tex}
        \part{Research}
            \input{1 - low-noise PiC models/main.tex}
            \input{2 - kinetic component/main.tex}
            \input{3 - fluid component/main.tex}
            \input{4 - numerical implementation/main.tex}
        \part{Project Overview}
            \input{5 - research plan/main.tex}
            \input{6 - summary/main.tex}
    
    
    %\section{}
    \newpage
    \pagenumbering{gobble}
        \printbibliography


    \newpage
    \pagenumbering{roman}
    \appendix
        \part{Appendices}
            \input{8 - Hilbert complexes/main.tex}
            \input{9 - weak conservation proofs/main.tex}
\end{document}

            \documentclass[12pt, a4paper]{report}

\input{template/main.tex}

\title{\BA{Title in Progress...}}
\author{Boris Andrews}
\affil{Mathematical Institute, University of Oxford}
\date{\today}


\begin{document}
    \pagenumbering{gobble}
    \maketitle
    
    
    \begin{abstract}
        Magnetic confinement reactors---in particular tokamaks---offer one of the most promising options for achieving practical nuclear fusion, with the potential to provide virtually limitless, clean energy. The theoretical and numerical modeling of tokamak plasmas is simultaneously an essential component of effective reactor design, and a great research barrier. Tokamak operational conditions exhibit comparatively low Knudsen numbers. Kinetic effects, including kinetic waves and instabilities, Landau damping, bump-on-tail instabilities and more, are therefore highly influential in tokamak plasma dynamics. Purely fluid models are inherently incapable of capturing these effects, whereas the high dimensionality in purely kinetic models render them practically intractable for most relevant purposes.

        We consider a $\delta\!f$ decomposition model, with a macroscopic fluid background and microscopic kinetic correction, both fully coupled to each other. A similar manner of discretization is proposed to that used in the recent \texttt{STRUPHY} code \cite{Holderied_Possanner_Wang_2021, Holderied_2022, Li_et_al_2023} with a finite-element model for the background and a pseudo-particle/PiC model for the correction.

        The fluid background satisfies the full, non-linear, resistive, compressible, Hall MHD equations. \cite{Laakmann_Hu_Farrell_2022} introduces finite-element(-in-space) implicit timesteppers for the incompressible analogue to this system with structure-preserving (SP) properties in the ideal case, alongside parameter-robust preconditioners. We show that these timesteppers can derive from a finite-element-in-time (FET) (and finite-element-in-space) interpretation. The benefits of this reformulation are discussed, including the derivation of timesteppers that are higher order in time, and the quantifiable dissipative SP properties in the non-ideal, resistive case.
        
        We discuss possible options for extending this FET approach to timesteppers for the compressible case.

        The kinetic corrections satisfy linearized Boltzmann equations. Using a Lénard--Bernstein collision operator, these take Fokker--Planck-like forms \cite{Fokker_1914, Planck_1917} wherein pseudo-particles in the numerical model obey the neoclassical transport equations, with particle-independent Brownian drift terms. This offers a rigorous methodology for incorporating collisions into the particle transport model, without coupling the equations of motions for each particle.
        
        Works by Chen, Chacón et al. \cite{Chen_Chacón_Barnes_2011, Chacón_Chen_Barnes_2013, Chen_Chacón_2014, Chen_Chacón_2015} have developed structure-preserving particle pushers for neoclassical transport in the Vlasov equations, derived from Crank--Nicolson integrators. We show these too can can derive from a FET interpretation, similarly offering potential extensions to higher-order-in-time particle pushers. The FET formulation is used also to consider how the stochastic drift terms can be incorporated into the pushers. Stochastic gyrokinetic expansions are also discussed.

        Different options for the numerical implementation of these schemes are considered.

        Due to the efficacy of FET in the development of SP timesteppers for both the fluid and kinetic component, we hope this approach will prove effective in the future for developing SP timesteppers for the full hybrid model. We hope this will give us the opportunity to incorporate previously inaccessible kinetic effects into the highly effective, modern, finite-element MHD models.
    \end{abstract}
    
    
    \newpage
    \tableofcontents
    
    
    \newpage
    \pagenumbering{arabic}
    %\linenumbers\renewcommand\thelinenumber{\color{black!50}\arabic{linenumber}}
            \input{0 - introduction/main.tex}
        \part{Research}
            \input{1 - low-noise PiC models/main.tex}
            \input{2 - kinetic component/main.tex}
            \input{3 - fluid component/main.tex}
            \input{4 - numerical implementation/main.tex}
        \part{Project Overview}
            \input{5 - research plan/main.tex}
            \input{6 - summary/main.tex}
    
    
    %\section{}
    \newpage
    \pagenumbering{gobble}
        \printbibliography


    \newpage
    \pagenumbering{roman}
    \appendix
        \part{Appendices}
            \input{8 - Hilbert complexes/main.tex}
            \input{9 - weak conservation proofs/main.tex}
\end{document}

            \documentclass[12pt, a4paper]{report}

\input{template/main.tex}

\title{\BA{Title in Progress...}}
\author{Boris Andrews}
\affil{Mathematical Institute, University of Oxford}
\date{\today}


\begin{document}
    \pagenumbering{gobble}
    \maketitle
    
    
    \begin{abstract}
        Magnetic confinement reactors---in particular tokamaks---offer one of the most promising options for achieving practical nuclear fusion, with the potential to provide virtually limitless, clean energy. The theoretical and numerical modeling of tokamak plasmas is simultaneously an essential component of effective reactor design, and a great research barrier. Tokamak operational conditions exhibit comparatively low Knudsen numbers. Kinetic effects, including kinetic waves and instabilities, Landau damping, bump-on-tail instabilities and more, are therefore highly influential in tokamak plasma dynamics. Purely fluid models are inherently incapable of capturing these effects, whereas the high dimensionality in purely kinetic models render them practically intractable for most relevant purposes.

        We consider a $\delta\!f$ decomposition model, with a macroscopic fluid background and microscopic kinetic correction, both fully coupled to each other. A similar manner of discretization is proposed to that used in the recent \texttt{STRUPHY} code \cite{Holderied_Possanner_Wang_2021, Holderied_2022, Li_et_al_2023} with a finite-element model for the background and a pseudo-particle/PiC model for the correction.

        The fluid background satisfies the full, non-linear, resistive, compressible, Hall MHD equations. \cite{Laakmann_Hu_Farrell_2022} introduces finite-element(-in-space) implicit timesteppers for the incompressible analogue to this system with structure-preserving (SP) properties in the ideal case, alongside parameter-robust preconditioners. We show that these timesteppers can derive from a finite-element-in-time (FET) (and finite-element-in-space) interpretation. The benefits of this reformulation are discussed, including the derivation of timesteppers that are higher order in time, and the quantifiable dissipative SP properties in the non-ideal, resistive case.
        
        We discuss possible options for extending this FET approach to timesteppers for the compressible case.

        The kinetic corrections satisfy linearized Boltzmann equations. Using a Lénard--Bernstein collision operator, these take Fokker--Planck-like forms \cite{Fokker_1914, Planck_1917} wherein pseudo-particles in the numerical model obey the neoclassical transport equations, with particle-independent Brownian drift terms. This offers a rigorous methodology for incorporating collisions into the particle transport model, without coupling the equations of motions for each particle.
        
        Works by Chen, Chacón et al. \cite{Chen_Chacón_Barnes_2011, Chacón_Chen_Barnes_2013, Chen_Chacón_2014, Chen_Chacón_2015} have developed structure-preserving particle pushers for neoclassical transport in the Vlasov equations, derived from Crank--Nicolson integrators. We show these too can can derive from a FET interpretation, similarly offering potential extensions to higher-order-in-time particle pushers. The FET formulation is used also to consider how the stochastic drift terms can be incorporated into the pushers. Stochastic gyrokinetic expansions are also discussed.

        Different options for the numerical implementation of these schemes are considered.

        Due to the efficacy of FET in the development of SP timesteppers for both the fluid and kinetic component, we hope this approach will prove effective in the future for developing SP timesteppers for the full hybrid model. We hope this will give us the opportunity to incorporate previously inaccessible kinetic effects into the highly effective, modern, finite-element MHD models.
    \end{abstract}
    
    
    \newpage
    \tableofcontents
    
    
    \newpage
    \pagenumbering{arabic}
    %\linenumbers\renewcommand\thelinenumber{\color{black!50}\arabic{linenumber}}
            \input{0 - introduction/main.tex}
        \part{Research}
            \input{1 - low-noise PiC models/main.tex}
            \input{2 - kinetic component/main.tex}
            \input{3 - fluid component/main.tex}
            \input{4 - numerical implementation/main.tex}
        \part{Project Overview}
            \input{5 - research plan/main.tex}
            \input{6 - summary/main.tex}
    
    
    %\section{}
    \newpage
    \pagenumbering{gobble}
        \printbibliography


    \newpage
    \pagenumbering{roman}
    \appendix
        \part{Appendices}
            \input{8 - Hilbert complexes/main.tex}
            \input{9 - weak conservation proofs/main.tex}
\end{document}

            \documentclass[12pt, a4paper]{report}

\input{template/main.tex}

\title{\BA{Title in Progress...}}
\author{Boris Andrews}
\affil{Mathematical Institute, University of Oxford}
\date{\today}


\begin{document}
    \pagenumbering{gobble}
    \maketitle
    
    
    \begin{abstract}
        Magnetic confinement reactors---in particular tokamaks---offer one of the most promising options for achieving practical nuclear fusion, with the potential to provide virtually limitless, clean energy. The theoretical and numerical modeling of tokamak plasmas is simultaneously an essential component of effective reactor design, and a great research barrier. Tokamak operational conditions exhibit comparatively low Knudsen numbers. Kinetic effects, including kinetic waves and instabilities, Landau damping, bump-on-tail instabilities and more, are therefore highly influential in tokamak plasma dynamics. Purely fluid models are inherently incapable of capturing these effects, whereas the high dimensionality in purely kinetic models render them practically intractable for most relevant purposes.

        We consider a $\delta\!f$ decomposition model, with a macroscopic fluid background and microscopic kinetic correction, both fully coupled to each other. A similar manner of discretization is proposed to that used in the recent \texttt{STRUPHY} code \cite{Holderied_Possanner_Wang_2021, Holderied_2022, Li_et_al_2023} with a finite-element model for the background and a pseudo-particle/PiC model for the correction.

        The fluid background satisfies the full, non-linear, resistive, compressible, Hall MHD equations. \cite{Laakmann_Hu_Farrell_2022} introduces finite-element(-in-space) implicit timesteppers for the incompressible analogue to this system with structure-preserving (SP) properties in the ideal case, alongside parameter-robust preconditioners. We show that these timesteppers can derive from a finite-element-in-time (FET) (and finite-element-in-space) interpretation. The benefits of this reformulation are discussed, including the derivation of timesteppers that are higher order in time, and the quantifiable dissipative SP properties in the non-ideal, resistive case.
        
        We discuss possible options for extending this FET approach to timesteppers for the compressible case.

        The kinetic corrections satisfy linearized Boltzmann equations. Using a Lénard--Bernstein collision operator, these take Fokker--Planck-like forms \cite{Fokker_1914, Planck_1917} wherein pseudo-particles in the numerical model obey the neoclassical transport equations, with particle-independent Brownian drift terms. This offers a rigorous methodology for incorporating collisions into the particle transport model, without coupling the equations of motions for each particle.
        
        Works by Chen, Chacón et al. \cite{Chen_Chacón_Barnes_2011, Chacón_Chen_Barnes_2013, Chen_Chacón_2014, Chen_Chacón_2015} have developed structure-preserving particle pushers for neoclassical transport in the Vlasov equations, derived from Crank--Nicolson integrators. We show these too can can derive from a FET interpretation, similarly offering potential extensions to higher-order-in-time particle pushers. The FET formulation is used also to consider how the stochastic drift terms can be incorporated into the pushers. Stochastic gyrokinetic expansions are also discussed.

        Different options for the numerical implementation of these schemes are considered.

        Due to the efficacy of FET in the development of SP timesteppers for both the fluid and kinetic component, we hope this approach will prove effective in the future for developing SP timesteppers for the full hybrid model. We hope this will give us the opportunity to incorporate previously inaccessible kinetic effects into the highly effective, modern, finite-element MHD models.
    \end{abstract}
    
    
    \newpage
    \tableofcontents
    
    
    \newpage
    \pagenumbering{arabic}
    %\linenumbers\renewcommand\thelinenumber{\color{black!50}\arabic{linenumber}}
            \input{0 - introduction/main.tex}
        \part{Research}
            \input{1 - low-noise PiC models/main.tex}
            \input{2 - kinetic component/main.tex}
            \input{3 - fluid component/main.tex}
            \input{4 - numerical implementation/main.tex}
        \part{Project Overview}
            \input{5 - research plan/main.tex}
            \input{6 - summary/main.tex}
    
    
    %\section{}
    \newpage
    \pagenumbering{gobble}
        \printbibliography


    \newpage
    \pagenumbering{roman}
    \appendix
        \part{Appendices}
            \input{8 - Hilbert complexes/main.tex}
            \input{9 - weak conservation proofs/main.tex}
\end{document}

        \part{Project Overview}
            \documentclass[12pt, a4paper]{report}

\input{template/main.tex}

\title{\BA{Title in Progress...}}
\author{Boris Andrews}
\affil{Mathematical Institute, University of Oxford}
\date{\today}


\begin{document}
    \pagenumbering{gobble}
    \maketitle
    
    
    \begin{abstract}
        Magnetic confinement reactors---in particular tokamaks---offer one of the most promising options for achieving practical nuclear fusion, with the potential to provide virtually limitless, clean energy. The theoretical and numerical modeling of tokamak plasmas is simultaneously an essential component of effective reactor design, and a great research barrier. Tokamak operational conditions exhibit comparatively low Knudsen numbers. Kinetic effects, including kinetic waves and instabilities, Landau damping, bump-on-tail instabilities and more, are therefore highly influential in tokamak plasma dynamics. Purely fluid models are inherently incapable of capturing these effects, whereas the high dimensionality in purely kinetic models render them practically intractable for most relevant purposes.

        We consider a $\delta\!f$ decomposition model, with a macroscopic fluid background and microscopic kinetic correction, both fully coupled to each other. A similar manner of discretization is proposed to that used in the recent \texttt{STRUPHY} code \cite{Holderied_Possanner_Wang_2021, Holderied_2022, Li_et_al_2023} with a finite-element model for the background and a pseudo-particle/PiC model for the correction.

        The fluid background satisfies the full, non-linear, resistive, compressible, Hall MHD equations. \cite{Laakmann_Hu_Farrell_2022} introduces finite-element(-in-space) implicit timesteppers for the incompressible analogue to this system with structure-preserving (SP) properties in the ideal case, alongside parameter-robust preconditioners. We show that these timesteppers can derive from a finite-element-in-time (FET) (and finite-element-in-space) interpretation. The benefits of this reformulation are discussed, including the derivation of timesteppers that are higher order in time, and the quantifiable dissipative SP properties in the non-ideal, resistive case.
        
        We discuss possible options for extending this FET approach to timesteppers for the compressible case.

        The kinetic corrections satisfy linearized Boltzmann equations. Using a Lénard--Bernstein collision operator, these take Fokker--Planck-like forms \cite{Fokker_1914, Planck_1917} wherein pseudo-particles in the numerical model obey the neoclassical transport equations, with particle-independent Brownian drift terms. This offers a rigorous methodology for incorporating collisions into the particle transport model, without coupling the equations of motions for each particle.
        
        Works by Chen, Chacón et al. \cite{Chen_Chacón_Barnes_2011, Chacón_Chen_Barnes_2013, Chen_Chacón_2014, Chen_Chacón_2015} have developed structure-preserving particle pushers for neoclassical transport in the Vlasov equations, derived from Crank--Nicolson integrators. We show these too can can derive from a FET interpretation, similarly offering potential extensions to higher-order-in-time particle pushers. The FET formulation is used also to consider how the stochastic drift terms can be incorporated into the pushers. Stochastic gyrokinetic expansions are also discussed.

        Different options for the numerical implementation of these schemes are considered.

        Due to the efficacy of FET in the development of SP timesteppers for both the fluid and kinetic component, we hope this approach will prove effective in the future for developing SP timesteppers for the full hybrid model. We hope this will give us the opportunity to incorporate previously inaccessible kinetic effects into the highly effective, modern, finite-element MHD models.
    \end{abstract}
    
    
    \newpage
    \tableofcontents
    
    
    \newpage
    \pagenumbering{arabic}
    %\linenumbers\renewcommand\thelinenumber{\color{black!50}\arabic{linenumber}}
            \input{0 - introduction/main.tex}
        \part{Research}
            \input{1 - low-noise PiC models/main.tex}
            \input{2 - kinetic component/main.tex}
            \input{3 - fluid component/main.tex}
            \input{4 - numerical implementation/main.tex}
        \part{Project Overview}
            \input{5 - research plan/main.tex}
            \input{6 - summary/main.tex}
    
    
    %\section{}
    \newpage
    \pagenumbering{gobble}
        \printbibliography


    \newpage
    \pagenumbering{roman}
    \appendix
        \part{Appendices}
            \input{8 - Hilbert complexes/main.tex}
            \input{9 - weak conservation proofs/main.tex}
\end{document}

            \documentclass[12pt, a4paper]{report}

\input{template/main.tex}

\title{\BA{Title in Progress...}}
\author{Boris Andrews}
\affil{Mathematical Institute, University of Oxford}
\date{\today}


\begin{document}
    \pagenumbering{gobble}
    \maketitle
    
    
    \begin{abstract}
        Magnetic confinement reactors---in particular tokamaks---offer one of the most promising options for achieving practical nuclear fusion, with the potential to provide virtually limitless, clean energy. The theoretical and numerical modeling of tokamak plasmas is simultaneously an essential component of effective reactor design, and a great research barrier. Tokamak operational conditions exhibit comparatively low Knudsen numbers. Kinetic effects, including kinetic waves and instabilities, Landau damping, bump-on-tail instabilities and more, are therefore highly influential in tokamak plasma dynamics. Purely fluid models are inherently incapable of capturing these effects, whereas the high dimensionality in purely kinetic models render them practically intractable for most relevant purposes.

        We consider a $\delta\!f$ decomposition model, with a macroscopic fluid background and microscopic kinetic correction, both fully coupled to each other. A similar manner of discretization is proposed to that used in the recent \texttt{STRUPHY} code \cite{Holderied_Possanner_Wang_2021, Holderied_2022, Li_et_al_2023} with a finite-element model for the background and a pseudo-particle/PiC model for the correction.

        The fluid background satisfies the full, non-linear, resistive, compressible, Hall MHD equations. \cite{Laakmann_Hu_Farrell_2022} introduces finite-element(-in-space) implicit timesteppers for the incompressible analogue to this system with structure-preserving (SP) properties in the ideal case, alongside parameter-robust preconditioners. We show that these timesteppers can derive from a finite-element-in-time (FET) (and finite-element-in-space) interpretation. The benefits of this reformulation are discussed, including the derivation of timesteppers that are higher order in time, and the quantifiable dissipative SP properties in the non-ideal, resistive case.
        
        We discuss possible options for extending this FET approach to timesteppers for the compressible case.

        The kinetic corrections satisfy linearized Boltzmann equations. Using a Lénard--Bernstein collision operator, these take Fokker--Planck-like forms \cite{Fokker_1914, Planck_1917} wherein pseudo-particles in the numerical model obey the neoclassical transport equations, with particle-independent Brownian drift terms. This offers a rigorous methodology for incorporating collisions into the particle transport model, without coupling the equations of motions for each particle.
        
        Works by Chen, Chacón et al. \cite{Chen_Chacón_Barnes_2011, Chacón_Chen_Barnes_2013, Chen_Chacón_2014, Chen_Chacón_2015} have developed structure-preserving particle pushers for neoclassical transport in the Vlasov equations, derived from Crank--Nicolson integrators. We show these too can can derive from a FET interpretation, similarly offering potential extensions to higher-order-in-time particle pushers. The FET formulation is used also to consider how the stochastic drift terms can be incorporated into the pushers. Stochastic gyrokinetic expansions are also discussed.

        Different options for the numerical implementation of these schemes are considered.

        Due to the efficacy of FET in the development of SP timesteppers for both the fluid and kinetic component, we hope this approach will prove effective in the future for developing SP timesteppers for the full hybrid model. We hope this will give us the opportunity to incorporate previously inaccessible kinetic effects into the highly effective, modern, finite-element MHD models.
    \end{abstract}
    
    
    \newpage
    \tableofcontents
    
    
    \newpage
    \pagenumbering{arabic}
    %\linenumbers\renewcommand\thelinenumber{\color{black!50}\arabic{linenumber}}
            \input{0 - introduction/main.tex}
        \part{Research}
            \input{1 - low-noise PiC models/main.tex}
            \input{2 - kinetic component/main.tex}
            \input{3 - fluid component/main.tex}
            \input{4 - numerical implementation/main.tex}
        \part{Project Overview}
            \input{5 - research plan/main.tex}
            \input{6 - summary/main.tex}
    
    
    %\section{}
    \newpage
    \pagenumbering{gobble}
        \printbibliography


    \newpage
    \pagenumbering{roman}
    \appendix
        \part{Appendices}
            \input{8 - Hilbert complexes/main.tex}
            \input{9 - weak conservation proofs/main.tex}
\end{document}

    
    
    %\section{}
    \newpage
    \pagenumbering{gobble}
        \printbibliography


    \newpage
    \pagenumbering{roman}
    \appendix
        \part{Appendices}
            \documentclass[12pt, a4paper]{report}

\input{template/main.tex}

\title{\BA{Title in Progress...}}
\author{Boris Andrews}
\affil{Mathematical Institute, University of Oxford}
\date{\today}


\begin{document}
    \pagenumbering{gobble}
    \maketitle
    
    
    \begin{abstract}
        Magnetic confinement reactors---in particular tokamaks---offer one of the most promising options for achieving practical nuclear fusion, with the potential to provide virtually limitless, clean energy. The theoretical and numerical modeling of tokamak plasmas is simultaneously an essential component of effective reactor design, and a great research barrier. Tokamak operational conditions exhibit comparatively low Knudsen numbers. Kinetic effects, including kinetic waves and instabilities, Landau damping, bump-on-tail instabilities and more, are therefore highly influential in tokamak plasma dynamics. Purely fluid models are inherently incapable of capturing these effects, whereas the high dimensionality in purely kinetic models render them practically intractable for most relevant purposes.

        We consider a $\delta\!f$ decomposition model, with a macroscopic fluid background and microscopic kinetic correction, both fully coupled to each other. A similar manner of discretization is proposed to that used in the recent \texttt{STRUPHY} code \cite{Holderied_Possanner_Wang_2021, Holderied_2022, Li_et_al_2023} with a finite-element model for the background and a pseudo-particle/PiC model for the correction.

        The fluid background satisfies the full, non-linear, resistive, compressible, Hall MHD equations. \cite{Laakmann_Hu_Farrell_2022} introduces finite-element(-in-space) implicit timesteppers for the incompressible analogue to this system with structure-preserving (SP) properties in the ideal case, alongside parameter-robust preconditioners. We show that these timesteppers can derive from a finite-element-in-time (FET) (and finite-element-in-space) interpretation. The benefits of this reformulation are discussed, including the derivation of timesteppers that are higher order in time, and the quantifiable dissipative SP properties in the non-ideal, resistive case.
        
        We discuss possible options for extending this FET approach to timesteppers for the compressible case.

        The kinetic corrections satisfy linearized Boltzmann equations. Using a Lénard--Bernstein collision operator, these take Fokker--Planck-like forms \cite{Fokker_1914, Planck_1917} wherein pseudo-particles in the numerical model obey the neoclassical transport equations, with particle-independent Brownian drift terms. This offers a rigorous methodology for incorporating collisions into the particle transport model, without coupling the equations of motions for each particle.
        
        Works by Chen, Chacón et al. \cite{Chen_Chacón_Barnes_2011, Chacón_Chen_Barnes_2013, Chen_Chacón_2014, Chen_Chacón_2015} have developed structure-preserving particle pushers for neoclassical transport in the Vlasov equations, derived from Crank--Nicolson integrators. We show these too can can derive from a FET interpretation, similarly offering potential extensions to higher-order-in-time particle pushers. The FET formulation is used also to consider how the stochastic drift terms can be incorporated into the pushers. Stochastic gyrokinetic expansions are also discussed.

        Different options for the numerical implementation of these schemes are considered.

        Due to the efficacy of FET in the development of SP timesteppers for both the fluid and kinetic component, we hope this approach will prove effective in the future for developing SP timesteppers for the full hybrid model. We hope this will give us the opportunity to incorporate previously inaccessible kinetic effects into the highly effective, modern, finite-element MHD models.
    \end{abstract}
    
    
    \newpage
    \tableofcontents
    
    
    \newpage
    \pagenumbering{arabic}
    %\linenumbers\renewcommand\thelinenumber{\color{black!50}\arabic{linenumber}}
            \input{0 - introduction/main.tex}
        \part{Research}
            \input{1 - low-noise PiC models/main.tex}
            \input{2 - kinetic component/main.tex}
            \input{3 - fluid component/main.tex}
            \input{4 - numerical implementation/main.tex}
        \part{Project Overview}
            \input{5 - research plan/main.tex}
            \input{6 - summary/main.tex}
    
    
    %\section{}
    \newpage
    \pagenumbering{gobble}
        \printbibliography


    \newpage
    \pagenumbering{roman}
    \appendix
        \part{Appendices}
            \input{8 - Hilbert complexes/main.tex}
            \input{9 - weak conservation proofs/main.tex}
\end{document}

            \documentclass[12pt, a4paper]{report}

\input{template/main.tex}

\title{\BA{Title in Progress...}}
\author{Boris Andrews}
\affil{Mathematical Institute, University of Oxford}
\date{\today}


\begin{document}
    \pagenumbering{gobble}
    \maketitle
    
    
    \begin{abstract}
        Magnetic confinement reactors---in particular tokamaks---offer one of the most promising options for achieving practical nuclear fusion, with the potential to provide virtually limitless, clean energy. The theoretical and numerical modeling of tokamak plasmas is simultaneously an essential component of effective reactor design, and a great research barrier. Tokamak operational conditions exhibit comparatively low Knudsen numbers. Kinetic effects, including kinetic waves and instabilities, Landau damping, bump-on-tail instabilities and more, are therefore highly influential in tokamak plasma dynamics. Purely fluid models are inherently incapable of capturing these effects, whereas the high dimensionality in purely kinetic models render them practically intractable for most relevant purposes.

        We consider a $\delta\!f$ decomposition model, with a macroscopic fluid background and microscopic kinetic correction, both fully coupled to each other. A similar manner of discretization is proposed to that used in the recent \texttt{STRUPHY} code \cite{Holderied_Possanner_Wang_2021, Holderied_2022, Li_et_al_2023} with a finite-element model for the background and a pseudo-particle/PiC model for the correction.

        The fluid background satisfies the full, non-linear, resistive, compressible, Hall MHD equations. \cite{Laakmann_Hu_Farrell_2022} introduces finite-element(-in-space) implicit timesteppers for the incompressible analogue to this system with structure-preserving (SP) properties in the ideal case, alongside parameter-robust preconditioners. We show that these timesteppers can derive from a finite-element-in-time (FET) (and finite-element-in-space) interpretation. The benefits of this reformulation are discussed, including the derivation of timesteppers that are higher order in time, and the quantifiable dissipative SP properties in the non-ideal, resistive case.
        
        We discuss possible options for extending this FET approach to timesteppers for the compressible case.

        The kinetic corrections satisfy linearized Boltzmann equations. Using a Lénard--Bernstein collision operator, these take Fokker--Planck-like forms \cite{Fokker_1914, Planck_1917} wherein pseudo-particles in the numerical model obey the neoclassical transport equations, with particle-independent Brownian drift terms. This offers a rigorous methodology for incorporating collisions into the particle transport model, without coupling the equations of motions for each particle.
        
        Works by Chen, Chacón et al. \cite{Chen_Chacón_Barnes_2011, Chacón_Chen_Barnes_2013, Chen_Chacón_2014, Chen_Chacón_2015} have developed structure-preserving particle pushers for neoclassical transport in the Vlasov equations, derived from Crank--Nicolson integrators. We show these too can can derive from a FET interpretation, similarly offering potential extensions to higher-order-in-time particle pushers. The FET formulation is used also to consider how the stochastic drift terms can be incorporated into the pushers. Stochastic gyrokinetic expansions are also discussed.

        Different options for the numerical implementation of these schemes are considered.

        Due to the efficacy of FET in the development of SP timesteppers for both the fluid and kinetic component, we hope this approach will prove effective in the future for developing SP timesteppers for the full hybrid model. We hope this will give us the opportunity to incorporate previously inaccessible kinetic effects into the highly effective, modern, finite-element MHD models.
    \end{abstract}
    
    
    \newpage
    \tableofcontents
    
    
    \newpage
    \pagenumbering{arabic}
    %\linenumbers\renewcommand\thelinenumber{\color{black!50}\arabic{linenumber}}
            \input{0 - introduction/main.tex}
        \part{Research}
            \input{1 - low-noise PiC models/main.tex}
            \input{2 - kinetic component/main.tex}
            \input{3 - fluid component/main.tex}
            \input{4 - numerical implementation/main.tex}
        \part{Project Overview}
            \input{5 - research plan/main.tex}
            \input{6 - summary/main.tex}
    
    
    %\section{}
    \newpage
    \pagenumbering{gobble}
        \printbibliography


    \newpage
    \pagenumbering{roman}
    \appendix
        \part{Appendices}
            \input{8 - Hilbert complexes/main.tex}
            \input{9 - weak conservation proofs/main.tex}
\end{document}

\end{document}


\title{\BA{Title in Progress...}}
\author{Boris Andrews}
\affil{Mathematical Institute, University of Oxford}
\date{\today}


\begin{document}
    \pagenumbering{gobble}
    \maketitle
    
    
    \begin{abstract}
        Magnetic confinement reactors---in particular tokamaks---offer one of the most promising options for achieving practical nuclear fusion, with the potential to provide virtually limitless, clean energy. The theoretical and numerical modeling of tokamak plasmas is simultaneously an essential component of effective reactor design, and a great research barrier. Tokamak operational conditions exhibit comparatively low Knudsen numbers. Kinetic effects, including kinetic waves and instabilities, Landau damping, bump-on-tail instabilities and more, are therefore highly influential in tokamak plasma dynamics. Purely fluid models are inherently incapable of capturing these effects, whereas the high dimensionality in purely kinetic models render them practically intractable for most relevant purposes.

        We consider a $\delta\!f$ decomposition model, with a macroscopic fluid background and microscopic kinetic correction, both fully coupled to each other. A similar manner of discretization is proposed to that used in the recent \texttt{STRUPHY} code \cite{Holderied_Possanner_Wang_2021, Holderied_2022, Li_et_al_2023} with a finite-element model for the background and a pseudo-particle/PiC model for the correction.

        The fluid background satisfies the full, non-linear, resistive, compressible, Hall MHD equations. \cite{Laakmann_Hu_Farrell_2022} introduces finite-element(-in-space) implicit timesteppers for the incompressible analogue to this system with structure-preserving (SP) properties in the ideal case, alongside parameter-robust preconditioners. We show that these timesteppers can derive from a finite-element-in-time (FET) (and finite-element-in-space) interpretation. The benefits of this reformulation are discussed, including the derivation of timesteppers that are higher order in time, and the quantifiable dissipative SP properties in the non-ideal, resistive case.
        
        We discuss possible options for extending this FET approach to timesteppers for the compressible case.

        The kinetic corrections satisfy linearized Boltzmann equations. Using a Lénard--Bernstein collision operator, these take Fokker--Planck-like forms \cite{Fokker_1914, Planck_1917} wherein pseudo-particles in the numerical model obey the neoclassical transport equations, with particle-independent Brownian drift terms. This offers a rigorous methodology for incorporating collisions into the particle transport model, without coupling the equations of motions for each particle.
        
        Works by Chen, Chacón et al. \cite{Chen_Chacón_Barnes_2011, Chacón_Chen_Barnes_2013, Chen_Chacón_2014, Chen_Chacón_2015} have developed structure-preserving particle pushers for neoclassical transport in the Vlasov equations, derived from Crank--Nicolson integrators. We show these too can can derive from a FET interpretation, similarly offering potential extensions to higher-order-in-time particle pushers. The FET formulation is used also to consider how the stochastic drift terms can be incorporated into the pushers. Stochastic gyrokinetic expansions are also discussed.

        Different options for the numerical implementation of these schemes are considered.

        Due to the efficacy of FET in the development of SP timesteppers for both the fluid and kinetic component, we hope this approach will prove effective in the future for developing SP timesteppers for the full hybrid model. We hope this will give us the opportunity to incorporate previously inaccessible kinetic effects into the highly effective, modern, finite-element MHD models.
    \end{abstract}
    
    
    \newpage
    \tableofcontents
    
    
    \newpage
    \pagenumbering{arabic}
    %\linenumbers\renewcommand\thelinenumber{\color{black!50}\arabic{linenumber}}
            \documentclass[12pt, a4paper]{report}

\documentclass[12pt, a4paper]{report}

\input{template/main.tex}

\title{\BA{Title in Progress...}}
\author{Boris Andrews}
\affil{Mathematical Institute, University of Oxford}
\date{\today}


\begin{document}
    \pagenumbering{gobble}
    \maketitle
    
    
    \begin{abstract}
        Magnetic confinement reactors---in particular tokamaks---offer one of the most promising options for achieving practical nuclear fusion, with the potential to provide virtually limitless, clean energy. The theoretical and numerical modeling of tokamak plasmas is simultaneously an essential component of effective reactor design, and a great research barrier. Tokamak operational conditions exhibit comparatively low Knudsen numbers. Kinetic effects, including kinetic waves and instabilities, Landau damping, bump-on-tail instabilities and more, are therefore highly influential in tokamak plasma dynamics. Purely fluid models are inherently incapable of capturing these effects, whereas the high dimensionality in purely kinetic models render them practically intractable for most relevant purposes.

        We consider a $\delta\!f$ decomposition model, with a macroscopic fluid background and microscopic kinetic correction, both fully coupled to each other. A similar manner of discretization is proposed to that used in the recent \texttt{STRUPHY} code \cite{Holderied_Possanner_Wang_2021, Holderied_2022, Li_et_al_2023} with a finite-element model for the background and a pseudo-particle/PiC model for the correction.

        The fluid background satisfies the full, non-linear, resistive, compressible, Hall MHD equations. \cite{Laakmann_Hu_Farrell_2022} introduces finite-element(-in-space) implicit timesteppers for the incompressible analogue to this system with structure-preserving (SP) properties in the ideal case, alongside parameter-robust preconditioners. We show that these timesteppers can derive from a finite-element-in-time (FET) (and finite-element-in-space) interpretation. The benefits of this reformulation are discussed, including the derivation of timesteppers that are higher order in time, and the quantifiable dissipative SP properties in the non-ideal, resistive case.
        
        We discuss possible options for extending this FET approach to timesteppers for the compressible case.

        The kinetic corrections satisfy linearized Boltzmann equations. Using a Lénard--Bernstein collision operator, these take Fokker--Planck-like forms \cite{Fokker_1914, Planck_1917} wherein pseudo-particles in the numerical model obey the neoclassical transport equations, with particle-independent Brownian drift terms. This offers a rigorous methodology for incorporating collisions into the particle transport model, without coupling the equations of motions for each particle.
        
        Works by Chen, Chacón et al. \cite{Chen_Chacón_Barnes_2011, Chacón_Chen_Barnes_2013, Chen_Chacón_2014, Chen_Chacón_2015} have developed structure-preserving particle pushers for neoclassical transport in the Vlasov equations, derived from Crank--Nicolson integrators. We show these too can can derive from a FET interpretation, similarly offering potential extensions to higher-order-in-time particle pushers. The FET formulation is used also to consider how the stochastic drift terms can be incorporated into the pushers. Stochastic gyrokinetic expansions are also discussed.

        Different options for the numerical implementation of these schemes are considered.

        Due to the efficacy of FET in the development of SP timesteppers for both the fluid and kinetic component, we hope this approach will prove effective in the future for developing SP timesteppers for the full hybrid model. We hope this will give us the opportunity to incorporate previously inaccessible kinetic effects into the highly effective, modern, finite-element MHD models.
    \end{abstract}
    
    
    \newpage
    \tableofcontents
    
    
    \newpage
    \pagenumbering{arabic}
    %\linenumbers\renewcommand\thelinenumber{\color{black!50}\arabic{linenumber}}
            \input{0 - introduction/main.tex}
        \part{Research}
            \input{1 - low-noise PiC models/main.tex}
            \input{2 - kinetic component/main.tex}
            \input{3 - fluid component/main.tex}
            \input{4 - numerical implementation/main.tex}
        \part{Project Overview}
            \input{5 - research plan/main.tex}
            \input{6 - summary/main.tex}
    
    
    %\section{}
    \newpage
    \pagenumbering{gobble}
        \printbibliography


    \newpage
    \pagenumbering{roman}
    \appendix
        \part{Appendices}
            \input{8 - Hilbert complexes/main.tex}
            \input{9 - weak conservation proofs/main.tex}
\end{document}


\title{\BA{Title in Progress...}}
\author{Boris Andrews}
\affil{Mathematical Institute, University of Oxford}
\date{\today}


\begin{document}
    \pagenumbering{gobble}
    \maketitle
    
    
    \begin{abstract}
        Magnetic confinement reactors---in particular tokamaks---offer one of the most promising options for achieving practical nuclear fusion, with the potential to provide virtually limitless, clean energy. The theoretical and numerical modeling of tokamak plasmas is simultaneously an essential component of effective reactor design, and a great research barrier. Tokamak operational conditions exhibit comparatively low Knudsen numbers. Kinetic effects, including kinetic waves and instabilities, Landau damping, bump-on-tail instabilities and more, are therefore highly influential in tokamak plasma dynamics. Purely fluid models are inherently incapable of capturing these effects, whereas the high dimensionality in purely kinetic models render them practically intractable for most relevant purposes.

        We consider a $\delta\!f$ decomposition model, with a macroscopic fluid background and microscopic kinetic correction, both fully coupled to each other. A similar manner of discretization is proposed to that used in the recent \texttt{STRUPHY} code \cite{Holderied_Possanner_Wang_2021, Holderied_2022, Li_et_al_2023} with a finite-element model for the background and a pseudo-particle/PiC model for the correction.

        The fluid background satisfies the full, non-linear, resistive, compressible, Hall MHD equations. \cite{Laakmann_Hu_Farrell_2022} introduces finite-element(-in-space) implicit timesteppers for the incompressible analogue to this system with structure-preserving (SP) properties in the ideal case, alongside parameter-robust preconditioners. We show that these timesteppers can derive from a finite-element-in-time (FET) (and finite-element-in-space) interpretation. The benefits of this reformulation are discussed, including the derivation of timesteppers that are higher order in time, and the quantifiable dissipative SP properties in the non-ideal, resistive case.
        
        We discuss possible options for extending this FET approach to timesteppers for the compressible case.

        The kinetic corrections satisfy linearized Boltzmann equations. Using a Lénard--Bernstein collision operator, these take Fokker--Planck-like forms \cite{Fokker_1914, Planck_1917} wherein pseudo-particles in the numerical model obey the neoclassical transport equations, with particle-independent Brownian drift terms. This offers a rigorous methodology for incorporating collisions into the particle transport model, without coupling the equations of motions for each particle.
        
        Works by Chen, Chacón et al. \cite{Chen_Chacón_Barnes_2011, Chacón_Chen_Barnes_2013, Chen_Chacón_2014, Chen_Chacón_2015} have developed structure-preserving particle pushers for neoclassical transport in the Vlasov equations, derived from Crank--Nicolson integrators. We show these too can can derive from a FET interpretation, similarly offering potential extensions to higher-order-in-time particle pushers. The FET formulation is used also to consider how the stochastic drift terms can be incorporated into the pushers. Stochastic gyrokinetic expansions are also discussed.

        Different options for the numerical implementation of these schemes are considered.

        Due to the efficacy of FET in the development of SP timesteppers for both the fluid and kinetic component, we hope this approach will prove effective in the future for developing SP timesteppers for the full hybrid model. We hope this will give us the opportunity to incorporate previously inaccessible kinetic effects into the highly effective, modern, finite-element MHD models.
    \end{abstract}
    
    
    \newpage
    \tableofcontents
    
    
    \newpage
    \pagenumbering{arabic}
    %\linenumbers\renewcommand\thelinenumber{\color{black!50}\arabic{linenumber}}
            \documentclass[12pt, a4paper]{report}

\input{template/main.tex}

\title{\BA{Title in Progress...}}
\author{Boris Andrews}
\affil{Mathematical Institute, University of Oxford}
\date{\today}


\begin{document}
    \pagenumbering{gobble}
    \maketitle
    
    
    \begin{abstract}
        Magnetic confinement reactors---in particular tokamaks---offer one of the most promising options for achieving practical nuclear fusion, with the potential to provide virtually limitless, clean energy. The theoretical and numerical modeling of tokamak plasmas is simultaneously an essential component of effective reactor design, and a great research barrier. Tokamak operational conditions exhibit comparatively low Knudsen numbers. Kinetic effects, including kinetic waves and instabilities, Landau damping, bump-on-tail instabilities and more, are therefore highly influential in tokamak plasma dynamics. Purely fluid models are inherently incapable of capturing these effects, whereas the high dimensionality in purely kinetic models render them practically intractable for most relevant purposes.

        We consider a $\delta\!f$ decomposition model, with a macroscopic fluid background and microscopic kinetic correction, both fully coupled to each other. A similar manner of discretization is proposed to that used in the recent \texttt{STRUPHY} code \cite{Holderied_Possanner_Wang_2021, Holderied_2022, Li_et_al_2023} with a finite-element model for the background and a pseudo-particle/PiC model for the correction.

        The fluid background satisfies the full, non-linear, resistive, compressible, Hall MHD equations. \cite{Laakmann_Hu_Farrell_2022} introduces finite-element(-in-space) implicit timesteppers for the incompressible analogue to this system with structure-preserving (SP) properties in the ideal case, alongside parameter-robust preconditioners. We show that these timesteppers can derive from a finite-element-in-time (FET) (and finite-element-in-space) interpretation. The benefits of this reformulation are discussed, including the derivation of timesteppers that are higher order in time, and the quantifiable dissipative SP properties in the non-ideal, resistive case.
        
        We discuss possible options for extending this FET approach to timesteppers for the compressible case.

        The kinetic corrections satisfy linearized Boltzmann equations. Using a Lénard--Bernstein collision operator, these take Fokker--Planck-like forms \cite{Fokker_1914, Planck_1917} wherein pseudo-particles in the numerical model obey the neoclassical transport equations, with particle-independent Brownian drift terms. This offers a rigorous methodology for incorporating collisions into the particle transport model, without coupling the equations of motions for each particle.
        
        Works by Chen, Chacón et al. \cite{Chen_Chacón_Barnes_2011, Chacón_Chen_Barnes_2013, Chen_Chacón_2014, Chen_Chacón_2015} have developed structure-preserving particle pushers for neoclassical transport in the Vlasov equations, derived from Crank--Nicolson integrators. We show these too can can derive from a FET interpretation, similarly offering potential extensions to higher-order-in-time particle pushers. The FET formulation is used also to consider how the stochastic drift terms can be incorporated into the pushers. Stochastic gyrokinetic expansions are also discussed.

        Different options for the numerical implementation of these schemes are considered.

        Due to the efficacy of FET in the development of SP timesteppers for both the fluid and kinetic component, we hope this approach will prove effective in the future for developing SP timesteppers for the full hybrid model. We hope this will give us the opportunity to incorporate previously inaccessible kinetic effects into the highly effective, modern, finite-element MHD models.
    \end{abstract}
    
    
    \newpage
    \tableofcontents
    
    
    \newpage
    \pagenumbering{arabic}
    %\linenumbers\renewcommand\thelinenumber{\color{black!50}\arabic{linenumber}}
            \input{0 - introduction/main.tex}
        \part{Research}
            \input{1 - low-noise PiC models/main.tex}
            \input{2 - kinetic component/main.tex}
            \input{3 - fluid component/main.tex}
            \input{4 - numerical implementation/main.tex}
        \part{Project Overview}
            \input{5 - research plan/main.tex}
            \input{6 - summary/main.tex}
    
    
    %\section{}
    \newpage
    \pagenumbering{gobble}
        \printbibliography


    \newpage
    \pagenumbering{roman}
    \appendix
        \part{Appendices}
            \input{8 - Hilbert complexes/main.tex}
            \input{9 - weak conservation proofs/main.tex}
\end{document}

        \part{Research}
            \documentclass[12pt, a4paper]{report}

\input{template/main.tex}

\title{\BA{Title in Progress...}}
\author{Boris Andrews}
\affil{Mathematical Institute, University of Oxford}
\date{\today}


\begin{document}
    \pagenumbering{gobble}
    \maketitle
    
    
    \begin{abstract}
        Magnetic confinement reactors---in particular tokamaks---offer one of the most promising options for achieving practical nuclear fusion, with the potential to provide virtually limitless, clean energy. The theoretical and numerical modeling of tokamak plasmas is simultaneously an essential component of effective reactor design, and a great research barrier. Tokamak operational conditions exhibit comparatively low Knudsen numbers. Kinetic effects, including kinetic waves and instabilities, Landau damping, bump-on-tail instabilities and more, are therefore highly influential in tokamak plasma dynamics. Purely fluid models are inherently incapable of capturing these effects, whereas the high dimensionality in purely kinetic models render them practically intractable for most relevant purposes.

        We consider a $\delta\!f$ decomposition model, with a macroscopic fluid background and microscopic kinetic correction, both fully coupled to each other. A similar manner of discretization is proposed to that used in the recent \texttt{STRUPHY} code \cite{Holderied_Possanner_Wang_2021, Holderied_2022, Li_et_al_2023} with a finite-element model for the background and a pseudo-particle/PiC model for the correction.

        The fluid background satisfies the full, non-linear, resistive, compressible, Hall MHD equations. \cite{Laakmann_Hu_Farrell_2022} introduces finite-element(-in-space) implicit timesteppers for the incompressible analogue to this system with structure-preserving (SP) properties in the ideal case, alongside parameter-robust preconditioners. We show that these timesteppers can derive from a finite-element-in-time (FET) (and finite-element-in-space) interpretation. The benefits of this reformulation are discussed, including the derivation of timesteppers that are higher order in time, and the quantifiable dissipative SP properties in the non-ideal, resistive case.
        
        We discuss possible options for extending this FET approach to timesteppers for the compressible case.

        The kinetic corrections satisfy linearized Boltzmann equations. Using a Lénard--Bernstein collision operator, these take Fokker--Planck-like forms \cite{Fokker_1914, Planck_1917} wherein pseudo-particles in the numerical model obey the neoclassical transport equations, with particle-independent Brownian drift terms. This offers a rigorous methodology for incorporating collisions into the particle transport model, without coupling the equations of motions for each particle.
        
        Works by Chen, Chacón et al. \cite{Chen_Chacón_Barnes_2011, Chacón_Chen_Barnes_2013, Chen_Chacón_2014, Chen_Chacón_2015} have developed structure-preserving particle pushers for neoclassical transport in the Vlasov equations, derived from Crank--Nicolson integrators. We show these too can can derive from a FET interpretation, similarly offering potential extensions to higher-order-in-time particle pushers. The FET formulation is used also to consider how the stochastic drift terms can be incorporated into the pushers. Stochastic gyrokinetic expansions are also discussed.

        Different options for the numerical implementation of these schemes are considered.

        Due to the efficacy of FET in the development of SP timesteppers for both the fluid and kinetic component, we hope this approach will prove effective in the future for developing SP timesteppers for the full hybrid model. We hope this will give us the opportunity to incorporate previously inaccessible kinetic effects into the highly effective, modern, finite-element MHD models.
    \end{abstract}
    
    
    \newpage
    \tableofcontents
    
    
    \newpage
    \pagenumbering{arabic}
    %\linenumbers\renewcommand\thelinenumber{\color{black!50}\arabic{linenumber}}
            \input{0 - introduction/main.tex}
        \part{Research}
            \input{1 - low-noise PiC models/main.tex}
            \input{2 - kinetic component/main.tex}
            \input{3 - fluid component/main.tex}
            \input{4 - numerical implementation/main.tex}
        \part{Project Overview}
            \input{5 - research plan/main.tex}
            \input{6 - summary/main.tex}
    
    
    %\section{}
    \newpage
    \pagenumbering{gobble}
        \printbibliography


    \newpage
    \pagenumbering{roman}
    \appendix
        \part{Appendices}
            \input{8 - Hilbert complexes/main.tex}
            \input{9 - weak conservation proofs/main.tex}
\end{document}

            \documentclass[12pt, a4paper]{report}

\input{template/main.tex}

\title{\BA{Title in Progress...}}
\author{Boris Andrews}
\affil{Mathematical Institute, University of Oxford}
\date{\today}


\begin{document}
    \pagenumbering{gobble}
    \maketitle
    
    
    \begin{abstract}
        Magnetic confinement reactors---in particular tokamaks---offer one of the most promising options for achieving practical nuclear fusion, with the potential to provide virtually limitless, clean energy. The theoretical and numerical modeling of tokamak plasmas is simultaneously an essential component of effective reactor design, and a great research barrier. Tokamak operational conditions exhibit comparatively low Knudsen numbers. Kinetic effects, including kinetic waves and instabilities, Landau damping, bump-on-tail instabilities and more, are therefore highly influential in tokamak plasma dynamics. Purely fluid models are inherently incapable of capturing these effects, whereas the high dimensionality in purely kinetic models render them practically intractable for most relevant purposes.

        We consider a $\delta\!f$ decomposition model, with a macroscopic fluid background and microscopic kinetic correction, both fully coupled to each other. A similar manner of discretization is proposed to that used in the recent \texttt{STRUPHY} code \cite{Holderied_Possanner_Wang_2021, Holderied_2022, Li_et_al_2023} with a finite-element model for the background and a pseudo-particle/PiC model for the correction.

        The fluid background satisfies the full, non-linear, resistive, compressible, Hall MHD equations. \cite{Laakmann_Hu_Farrell_2022} introduces finite-element(-in-space) implicit timesteppers for the incompressible analogue to this system with structure-preserving (SP) properties in the ideal case, alongside parameter-robust preconditioners. We show that these timesteppers can derive from a finite-element-in-time (FET) (and finite-element-in-space) interpretation. The benefits of this reformulation are discussed, including the derivation of timesteppers that are higher order in time, and the quantifiable dissipative SP properties in the non-ideal, resistive case.
        
        We discuss possible options for extending this FET approach to timesteppers for the compressible case.

        The kinetic corrections satisfy linearized Boltzmann equations. Using a Lénard--Bernstein collision operator, these take Fokker--Planck-like forms \cite{Fokker_1914, Planck_1917} wherein pseudo-particles in the numerical model obey the neoclassical transport equations, with particle-independent Brownian drift terms. This offers a rigorous methodology for incorporating collisions into the particle transport model, without coupling the equations of motions for each particle.
        
        Works by Chen, Chacón et al. \cite{Chen_Chacón_Barnes_2011, Chacón_Chen_Barnes_2013, Chen_Chacón_2014, Chen_Chacón_2015} have developed structure-preserving particle pushers for neoclassical transport in the Vlasov equations, derived from Crank--Nicolson integrators. We show these too can can derive from a FET interpretation, similarly offering potential extensions to higher-order-in-time particle pushers. The FET formulation is used also to consider how the stochastic drift terms can be incorporated into the pushers. Stochastic gyrokinetic expansions are also discussed.

        Different options for the numerical implementation of these schemes are considered.

        Due to the efficacy of FET in the development of SP timesteppers for both the fluid and kinetic component, we hope this approach will prove effective in the future for developing SP timesteppers for the full hybrid model. We hope this will give us the opportunity to incorporate previously inaccessible kinetic effects into the highly effective, modern, finite-element MHD models.
    \end{abstract}
    
    
    \newpage
    \tableofcontents
    
    
    \newpage
    \pagenumbering{arabic}
    %\linenumbers\renewcommand\thelinenumber{\color{black!50}\arabic{linenumber}}
            \input{0 - introduction/main.tex}
        \part{Research}
            \input{1 - low-noise PiC models/main.tex}
            \input{2 - kinetic component/main.tex}
            \input{3 - fluid component/main.tex}
            \input{4 - numerical implementation/main.tex}
        \part{Project Overview}
            \input{5 - research plan/main.tex}
            \input{6 - summary/main.tex}
    
    
    %\section{}
    \newpage
    \pagenumbering{gobble}
        \printbibliography


    \newpage
    \pagenumbering{roman}
    \appendix
        \part{Appendices}
            \input{8 - Hilbert complexes/main.tex}
            \input{9 - weak conservation proofs/main.tex}
\end{document}

            \documentclass[12pt, a4paper]{report}

\input{template/main.tex}

\title{\BA{Title in Progress...}}
\author{Boris Andrews}
\affil{Mathematical Institute, University of Oxford}
\date{\today}


\begin{document}
    \pagenumbering{gobble}
    \maketitle
    
    
    \begin{abstract}
        Magnetic confinement reactors---in particular tokamaks---offer one of the most promising options for achieving practical nuclear fusion, with the potential to provide virtually limitless, clean energy. The theoretical and numerical modeling of tokamak plasmas is simultaneously an essential component of effective reactor design, and a great research barrier. Tokamak operational conditions exhibit comparatively low Knudsen numbers. Kinetic effects, including kinetic waves and instabilities, Landau damping, bump-on-tail instabilities and more, are therefore highly influential in tokamak plasma dynamics. Purely fluid models are inherently incapable of capturing these effects, whereas the high dimensionality in purely kinetic models render them practically intractable for most relevant purposes.

        We consider a $\delta\!f$ decomposition model, with a macroscopic fluid background and microscopic kinetic correction, both fully coupled to each other. A similar manner of discretization is proposed to that used in the recent \texttt{STRUPHY} code \cite{Holderied_Possanner_Wang_2021, Holderied_2022, Li_et_al_2023} with a finite-element model for the background and a pseudo-particle/PiC model for the correction.

        The fluid background satisfies the full, non-linear, resistive, compressible, Hall MHD equations. \cite{Laakmann_Hu_Farrell_2022} introduces finite-element(-in-space) implicit timesteppers for the incompressible analogue to this system with structure-preserving (SP) properties in the ideal case, alongside parameter-robust preconditioners. We show that these timesteppers can derive from a finite-element-in-time (FET) (and finite-element-in-space) interpretation. The benefits of this reformulation are discussed, including the derivation of timesteppers that are higher order in time, and the quantifiable dissipative SP properties in the non-ideal, resistive case.
        
        We discuss possible options for extending this FET approach to timesteppers for the compressible case.

        The kinetic corrections satisfy linearized Boltzmann equations. Using a Lénard--Bernstein collision operator, these take Fokker--Planck-like forms \cite{Fokker_1914, Planck_1917} wherein pseudo-particles in the numerical model obey the neoclassical transport equations, with particle-independent Brownian drift terms. This offers a rigorous methodology for incorporating collisions into the particle transport model, without coupling the equations of motions for each particle.
        
        Works by Chen, Chacón et al. \cite{Chen_Chacón_Barnes_2011, Chacón_Chen_Barnes_2013, Chen_Chacón_2014, Chen_Chacón_2015} have developed structure-preserving particle pushers for neoclassical transport in the Vlasov equations, derived from Crank--Nicolson integrators. We show these too can can derive from a FET interpretation, similarly offering potential extensions to higher-order-in-time particle pushers. The FET formulation is used also to consider how the stochastic drift terms can be incorporated into the pushers. Stochastic gyrokinetic expansions are also discussed.

        Different options for the numerical implementation of these schemes are considered.

        Due to the efficacy of FET in the development of SP timesteppers for both the fluid and kinetic component, we hope this approach will prove effective in the future for developing SP timesteppers for the full hybrid model. We hope this will give us the opportunity to incorporate previously inaccessible kinetic effects into the highly effective, modern, finite-element MHD models.
    \end{abstract}
    
    
    \newpage
    \tableofcontents
    
    
    \newpage
    \pagenumbering{arabic}
    %\linenumbers\renewcommand\thelinenumber{\color{black!50}\arabic{linenumber}}
            \input{0 - introduction/main.tex}
        \part{Research}
            \input{1 - low-noise PiC models/main.tex}
            \input{2 - kinetic component/main.tex}
            \input{3 - fluid component/main.tex}
            \input{4 - numerical implementation/main.tex}
        \part{Project Overview}
            \input{5 - research plan/main.tex}
            \input{6 - summary/main.tex}
    
    
    %\section{}
    \newpage
    \pagenumbering{gobble}
        \printbibliography


    \newpage
    \pagenumbering{roman}
    \appendix
        \part{Appendices}
            \input{8 - Hilbert complexes/main.tex}
            \input{9 - weak conservation proofs/main.tex}
\end{document}

            \documentclass[12pt, a4paper]{report}

\input{template/main.tex}

\title{\BA{Title in Progress...}}
\author{Boris Andrews}
\affil{Mathematical Institute, University of Oxford}
\date{\today}


\begin{document}
    \pagenumbering{gobble}
    \maketitle
    
    
    \begin{abstract}
        Magnetic confinement reactors---in particular tokamaks---offer one of the most promising options for achieving practical nuclear fusion, with the potential to provide virtually limitless, clean energy. The theoretical and numerical modeling of tokamak plasmas is simultaneously an essential component of effective reactor design, and a great research barrier. Tokamak operational conditions exhibit comparatively low Knudsen numbers. Kinetic effects, including kinetic waves and instabilities, Landau damping, bump-on-tail instabilities and more, are therefore highly influential in tokamak plasma dynamics. Purely fluid models are inherently incapable of capturing these effects, whereas the high dimensionality in purely kinetic models render them practically intractable for most relevant purposes.

        We consider a $\delta\!f$ decomposition model, with a macroscopic fluid background and microscopic kinetic correction, both fully coupled to each other. A similar manner of discretization is proposed to that used in the recent \texttt{STRUPHY} code \cite{Holderied_Possanner_Wang_2021, Holderied_2022, Li_et_al_2023} with a finite-element model for the background and a pseudo-particle/PiC model for the correction.

        The fluid background satisfies the full, non-linear, resistive, compressible, Hall MHD equations. \cite{Laakmann_Hu_Farrell_2022} introduces finite-element(-in-space) implicit timesteppers for the incompressible analogue to this system with structure-preserving (SP) properties in the ideal case, alongside parameter-robust preconditioners. We show that these timesteppers can derive from a finite-element-in-time (FET) (and finite-element-in-space) interpretation. The benefits of this reformulation are discussed, including the derivation of timesteppers that are higher order in time, and the quantifiable dissipative SP properties in the non-ideal, resistive case.
        
        We discuss possible options for extending this FET approach to timesteppers for the compressible case.

        The kinetic corrections satisfy linearized Boltzmann equations. Using a Lénard--Bernstein collision operator, these take Fokker--Planck-like forms \cite{Fokker_1914, Planck_1917} wherein pseudo-particles in the numerical model obey the neoclassical transport equations, with particle-independent Brownian drift terms. This offers a rigorous methodology for incorporating collisions into the particle transport model, without coupling the equations of motions for each particle.
        
        Works by Chen, Chacón et al. \cite{Chen_Chacón_Barnes_2011, Chacón_Chen_Barnes_2013, Chen_Chacón_2014, Chen_Chacón_2015} have developed structure-preserving particle pushers for neoclassical transport in the Vlasov equations, derived from Crank--Nicolson integrators. We show these too can can derive from a FET interpretation, similarly offering potential extensions to higher-order-in-time particle pushers. The FET formulation is used also to consider how the stochastic drift terms can be incorporated into the pushers. Stochastic gyrokinetic expansions are also discussed.

        Different options for the numerical implementation of these schemes are considered.

        Due to the efficacy of FET in the development of SP timesteppers for both the fluid and kinetic component, we hope this approach will prove effective in the future for developing SP timesteppers for the full hybrid model. We hope this will give us the opportunity to incorporate previously inaccessible kinetic effects into the highly effective, modern, finite-element MHD models.
    \end{abstract}
    
    
    \newpage
    \tableofcontents
    
    
    \newpage
    \pagenumbering{arabic}
    %\linenumbers\renewcommand\thelinenumber{\color{black!50}\arabic{linenumber}}
            \input{0 - introduction/main.tex}
        \part{Research}
            \input{1 - low-noise PiC models/main.tex}
            \input{2 - kinetic component/main.tex}
            \input{3 - fluid component/main.tex}
            \input{4 - numerical implementation/main.tex}
        \part{Project Overview}
            \input{5 - research plan/main.tex}
            \input{6 - summary/main.tex}
    
    
    %\section{}
    \newpage
    \pagenumbering{gobble}
        \printbibliography


    \newpage
    \pagenumbering{roman}
    \appendix
        \part{Appendices}
            \input{8 - Hilbert complexes/main.tex}
            \input{9 - weak conservation proofs/main.tex}
\end{document}

        \part{Project Overview}
            \documentclass[12pt, a4paper]{report}

\input{template/main.tex}

\title{\BA{Title in Progress...}}
\author{Boris Andrews}
\affil{Mathematical Institute, University of Oxford}
\date{\today}


\begin{document}
    \pagenumbering{gobble}
    \maketitle
    
    
    \begin{abstract}
        Magnetic confinement reactors---in particular tokamaks---offer one of the most promising options for achieving practical nuclear fusion, with the potential to provide virtually limitless, clean energy. The theoretical and numerical modeling of tokamak plasmas is simultaneously an essential component of effective reactor design, and a great research barrier. Tokamak operational conditions exhibit comparatively low Knudsen numbers. Kinetic effects, including kinetic waves and instabilities, Landau damping, bump-on-tail instabilities and more, are therefore highly influential in tokamak plasma dynamics. Purely fluid models are inherently incapable of capturing these effects, whereas the high dimensionality in purely kinetic models render them practically intractable for most relevant purposes.

        We consider a $\delta\!f$ decomposition model, with a macroscopic fluid background and microscopic kinetic correction, both fully coupled to each other. A similar manner of discretization is proposed to that used in the recent \texttt{STRUPHY} code \cite{Holderied_Possanner_Wang_2021, Holderied_2022, Li_et_al_2023} with a finite-element model for the background and a pseudo-particle/PiC model for the correction.

        The fluid background satisfies the full, non-linear, resistive, compressible, Hall MHD equations. \cite{Laakmann_Hu_Farrell_2022} introduces finite-element(-in-space) implicit timesteppers for the incompressible analogue to this system with structure-preserving (SP) properties in the ideal case, alongside parameter-robust preconditioners. We show that these timesteppers can derive from a finite-element-in-time (FET) (and finite-element-in-space) interpretation. The benefits of this reformulation are discussed, including the derivation of timesteppers that are higher order in time, and the quantifiable dissipative SP properties in the non-ideal, resistive case.
        
        We discuss possible options for extending this FET approach to timesteppers for the compressible case.

        The kinetic corrections satisfy linearized Boltzmann equations. Using a Lénard--Bernstein collision operator, these take Fokker--Planck-like forms \cite{Fokker_1914, Planck_1917} wherein pseudo-particles in the numerical model obey the neoclassical transport equations, with particle-independent Brownian drift terms. This offers a rigorous methodology for incorporating collisions into the particle transport model, without coupling the equations of motions for each particle.
        
        Works by Chen, Chacón et al. \cite{Chen_Chacón_Barnes_2011, Chacón_Chen_Barnes_2013, Chen_Chacón_2014, Chen_Chacón_2015} have developed structure-preserving particle pushers for neoclassical transport in the Vlasov equations, derived from Crank--Nicolson integrators. We show these too can can derive from a FET interpretation, similarly offering potential extensions to higher-order-in-time particle pushers. The FET formulation is used also to consider how the stochastic drift terms can be incorporated into the pushers. Stochastic gyrokinetic expansions are also discussed.

        Different options for the numerical implementation of these schemes are considered.

        Due to the efficacy of FET in the development of SP timesteppers for both the fluid and kinetic component, we hope this approach will prove effective in the future for developing SP timesteppers for the full hybrid model. We hope this will give us the opportunity to incorporate previously inaccessible kinetic effects into the highly effective, modern, finite-element MHD models.
    \end{abstract}
    
    
    \newpage
    \tableofcontents
    
    
    \newpage
    \pagenumbering{arabic}
    %\linenumbers\renewcommand\thelinenumber{\color{black!50}\arabic{linenumber}}
            \input{0 - introduction/main.tex}
        \part{Research}
            \input{1 - low-noise PiC models/main.tex}
            \input{2 - kinetic component/main.tex}
            \input{3 - fluid component/main.tex}
            \input{4 - numerical implementation/main.tex}
        \part{Project Overview}
            \input{5 - research plan/main.tex}
            \input{6 - summary/main.tex}
    
    
    %\section{}
    \newpage
    \pagenumbering{gobble}
        \printbibliography


    \newpage
    \pagenumbering{roman}
    \appendix
        \part{Appendices}
            \input{8 - Hilbert complexes/main.tex}
            \input{9 - weak conservation proofs/main.tex}
\end{document}

            \documentclass[12pt, a4paper]{report}

\input{template/main.tex}

\title{\BA{Title in Progress...}}
\author{Boris Andrews}
\affil{Mathematical Institute, University of Oxford}
\date{\today}


\begin{document}
    \pagenumbering{gobble}
    \maketitle
    
    
    \begin{abstract}
        Magnetic confinement reactors---in particular tokamaks---offer one of the most promising options for achieving practical nuclear fusion, with the potential to provide virtually limitless, clean energy. The theoretical and numerical modeling of tokamak plasmas is simultaneously an essential component of effective reactor design, and a great research barrier. Tokamak operational conditions exhibit comparatively low Knudsen numbers. Kinetic effects, including kinetic waves and instabilities, Landau damping, bump-on-tail instabilities and more, are therefore highly influential in tokamak plasma dynamics. Purely fluid models are inherently incapable of capturing these effects, whereas the high dimensionality in purely kinetic models render them practically intractable for most relevant purposes.

        We consider a $\delta\!f$ decomposition model, with a macroscopic fluid background and microscopic kinetic correction, both fully coupled to each other. A similar manner of discretization is proposed to that used in the recent \texttt{STRUPHY} code \cite{Holderied_Possanner_Wang_2021, Holderied_2022, Li_et_al_2023} with a finite-element model for the background and a pseudo-particle/PiC model for the correction.

        The fluid background satisfies the full, non-linear, resistive, compressible, Hall MHD equations. \cite{Laakmann_Hu_Farrell_2022} introduces finite-element(-in-space) implicit timesteppers for the incompressible analogue to this system with structure-preserving (SP) properties in the ideal case, alongside parameter-robust preconditioners. We show that these timesteppers can derive from a finite-element-in-time (FET) (and finite-element-in-space) interpretation. The benefits of this reformulation are discussed, including the derivation of timesteppers that are higher order in time, and the quantifiable dissipative SP properties in the non-ideal, resistive case.
        
        We discuss possible options for extending this FET approach to timesteppers for the compressible case.

        The kinetic corrections satisfy linearized Boltzmann equations. Using a Lénard--Bernstein collision operator, these take Fokker--Planck-like forms \cite{Fokker_1914, Planck_1917} wherein pseudo-particles in the numerical model obey the neoclassical transport equations, with particle-independent Brownian drift terms. This offers a rigorous methodology for incorporating collisions into the particle transport model, without coupling the equations of motions for each particle.
        
        Works by Chen, Chacón et al. \cite{Chen_Chacón_Barnes_2011, Chacón_Chen_Barnes_2013, Chen_Chacón_2014, Chen_Chacón_2015} have developed structure-preserving particle pushers for neoclassical transport in the Vlasov equations, derived from Crank--Nicolson integrators. We show these too can can derive from a FET interpretation, similarly offering potential extensions to higher-order-in-time particle pushers. The FET formulation is used also to consider how the stochastic drift terms can be incorporated into the pushers. Stochastic gyrokinetic expansions are also discussed.

        Different options for the numerical implementation of these schemes are considered.

        Due to the efficacy of FET in the development of SP timesteppers for both the fluid and kinetic component, we hope this approach will prove effective in the future for developing SP timesteppers for the full hybrid model. We hope this will give us the opportunity to incorporate previously inaccessible kinetic effects into the highly effective, modern, finite-element MHD models.
    \end{abstract}
    
    
    \newpage
    \tableofcontents
    
    
    \newpage
    \pagenumbering{arabic}
    %\linenumbers\renewcommand\thelinenumber{\color{black!50}\arabic{linenumber}}
            \input{0 - introduction/main.tex}
        \part{Research}
            \input{1 - low-noise PiC models/main.tex}
            \input{2 - kinetic component/main.tex}
            \input{3 - fluid component/main.tex}
            \input{4 - numerical implementation/main.tex}
        \part{Project Overview}
            \input{5 - research plan/main.tex}
            \input{6 - summary/main.tex}
    
    
    %\section{}
    \newpage
    \pagenumbering{gobble}
        \printbibliography


    \newpage
    \pagenumbering{roman}
    \appendix
        \part{Appendices}
            \input{8 - Hilbert complexes/main.tex}
            \input{9 - weak conservation proofs/main.tex}
\end{document}

    
    
    %\section{}
    \newpage
    \pagenumbering{gobble}
        \printbibliography


    \newpage
    \pagenumbering{roman}
    \appendix
        \part{Appendices}
            \documentclass[12pt, a4paper]{report}

\input{template/main.tex}

\title{\BA{Title in Progress...}}
\author{Boris Andrews}
\affil{Mathematical Institute, University of Oxford}
\date{\today}


\begin{document}
    \pagenumbering{gobble}
    \maketitle
    
    
    \begin{abstract}
        Magnetic confinement reactors---in particular tokamaks---offer one of the most promising options for achieving practical nuclear fusion, with the potential to provide virtually limitless, clean energy. The theoretical and numerical modeling of tokamak plasmas is simultaneously an essential component of effective reactor design, and a great research barrier. Tokamak operational conditions exhibit comparatively low Knudsen numbers. Kinetic effects, including kinetic waves and instabilities, Landau damping, bump-on-tail instabilities and more, are therefore highly influential in tokamak plasma dynamics. Purely fluid models are inherently incapable of capturing these effects, whereas the high dimensionality in purely kinetic models render them practically intractable for most relevant purposes.

        We consider a $\delta\!f$ decomposition model, with a macroscopic fluid background and microscopic kinetic correction, both fully coupled to each other. A similar manner of discretization is proposed to that used in the recent \texttt{STRUPHY} code \cite{Holderied_Possanner_Wang_2021, Holderied_2022, Li_et_al_2023} with a finite-element model for the background and a pseudo-particle/PiC model for the correction.

        The fluid background satisfies the full, non-linear, resistive, compressible, Hall MHD equations. \cite{Laakmann_Hu_Farrell_2022} introduces finite-element(-in-space) implicit timesteppers for the incompressible analogue to this system with structure-preserving (SP) properties in the ideal case, alongside parameter-robust preconditioners. We show that these timesteppers can derive from a finite-element-in-time (FET) (and finite-element-in-space) interpretation. The benefits of this reformulation are discussed, including the derivation of timesteppers that are higher order in time, and the quantifiable dissipative SP properties in the non-ideal, resistive case.
        
        We discuss possible options for extending this FET approach to timesteppers for the compressible case.

        The kinetic corrections satisfy linearized Boltzmann equations. Using a Lénard--Bernstein collision operator, these take Fokker--Planck-like forms \cite{Fokker_1914, Planck_1917} wherein pseudo-particles in the numerical model obey the neoclassical transport equations, with particle-independent Brownian drift terms. This offers a rigorous methodology for incorporating collisions into the particle transport model, without coupling the equations of motions for each particle.
        
        Works by Chen, Chacón et al. \cite{Chen_Chacón_Barnes_2011, Chacón_Chen_Barnes_2013, Chen_Chacón_2014, Chen_Chacón_2015} have developed structure-preserving particle pushers for neoclassical transport in the Vlasov equations, derived from Crank--Nicolson integrators. We show these too can can derive from a FET interpretation, similarly offering potential extensions to higher-order-in-time particle pushers. The FET formulation is used also to consider how the stochastic drift terms can be incorporated into the pushers. Stochastic gyrokinetic expansions are also discussed.

        Different options for the numerical implementation of these schemes are considered.

        Due to the efficacy of FET in the development of SP timesteppers for both the fluid and kinetic component, we hope this approach will prove effective in the future for developing SP timesteppers for the full hybrid model. We hope this will give us the opportunity to incorporate previously inaccessible kinetic effects into the highly effective, modern, finite-element MHD models.
    \end{abstract}
    
    
    \newpage
    \tableofcontents
    
    
    \newpage
    \pagenumbering{arabic}
    %\linenumbers\renewcommand\thelinenumber{\color{black!50}\arabic{linenumber}}
            \input{0 - introduction/main.tex}
        \part{Research}
            \input{1 - low-noise PiC models/main.tex}
            \input{2 - kinetic component/main.tex}
            \input{3 - fluid component/main.tex}
            \input{4 - numerical implementation/main.tex}
        \part{Project Overview}
            \input{5 - research plan/main.tex}
            \input{6 - summary/main.tex}
    
    
    %\section{}
    \newpage
    \pagenumbering{gobble}
        \printbibliography


    \newpage
    \pagenumbering{roman}
    \appendix
        \part{Appendices}
            \input{8 - Hilbert complexes/main.tex}
            \input{9 - weak conservation proofs/main.tex}
\end{document}

            \documentclass[12pt, a4paper]{report}

\input{template/main.tex}

\title{\BA{Title in Progress...}}
\author{Boris Andrews}
\affil{Mathematical Institute, University of Oxford}
\date{\today}


\begin{document}
    \pagenumbering{gobble}
    \maketitle
    
    
    \begin{abstract}
        Magnetic confinement reactors---in particular tokamaks---offer one of the most promising options for achieving practical nuclear fusion, with the potential to provide virtually limitless, clean energy. The theoretical and numerical modeling of tokamak plasmas is simultaneously an essential component of effective reactor design, and a great research barrier. Tokamak operational conditions exhibit comparatively low Knudsen numbers. Kinetic effects, including kinetic waves and instabilities, Landau damping, bump-on-tail instabilities and more, are therefore highly influential in tokamak plasma dynamics. Purely fluid models are inherently incapable of capturing these effects, whereas the high dimensionality in purely kinetic models render them practically intractable for most relevant purposes.

        We consider a $\delta\!f$ decomposition model, with a macroscopic fluid background and microscopic kinetic correction, both fully coupled to each other. A similar manner of discretization is proposed to that used in the recent \texttt{STRUPHY} code \cite{Holderied_Possanner_Wang_2021, Holderied_2022, Li_et_al_2023} with a finite-element model for the background and a pseudo-particle/PiC model for the correction.

        The fluid background satisfies the full, non-linear, resistive, compressible, Hall MHD equations. \cite{Laakmann_Hu_Farrell_2022} introduces finite-element(-in-space) implicit timesteppers for the incompressible analogue to this system with structure-preserving (SP) properties in the ideal case, alongside parameter-robust preconditioners. We show that these timesteppers can derive from a finite-element-in-time (FET) (and finite-element-in-space) interpretation. The benefits of this reformulation are discussed, including the derivation of timesteppers that are higher order in time, and the quantifiable dissipative SP properties in the non-ideal, resistive case.
        
        We discuss possible options for extending this FET approach to timesteppers for the compressible case.

        The kinetic corrections satisfy linearized Boltzmann equations. Using a Lénard--Bernstein collision operator, these take Fokker--Planck-like forms \cite{Fokker_1914, Planck_1917} wherein pseudo-particles in the numerical model obey the neoclassical transport equations, with particle-independent Brownian drift terms. This offers a rigorous methodology for incorporating collisions into the particle transport model, without coupling the equations of motions for each particle.
        
        Works by Chen, Chacón et al. \cite{Chen_Chacón_Barnes_2011, Chacón_Chen_Barnes_2013, Chen_Chacón_2014, Chen_Chacón_2015} have developed structure-preserving particle pushers for neoclassical transport in the Vlasov equations, derived from Crank--Nicolson integrators. We show these too can can derive from a FET interpretation, similarly offering potential extensions to higher-order-in-time particle pushers. The FET formulation is used also to consider how the stochastic drift terms can be incorporated into the pushers. Stochastic gyrokinetic expansions are also discussed.

        Different options for the numerical implementation of these schemes are considered.

        Due to the efficacy of FET in the development of SP timesteppers for both the fluid and kinetic component, we hope this approach will prove effective in the future for developing SP timesteppers for the full hybrid model. We hope this will give us the opportunity to incorporate previously inaccessible kinetic effects into the highly effective, modern, finite-element MHD models.
    \end{abstract}
    
    
    \newpage
    \tableofcontents
    
    
    \newpage
    \pagenumbering{arabic}
    %\linenumbers\renewcommand\thelinenumber{\color{black!50}\arabic{linenumber}}
            \input{0 - introduction/main.tex}
        \part{Research}
            \input{1 - low-noise PiC models/main.tex}
            \input{2 - kinetic component/main.tex}
            \input{3 - fluid component/main.tex}
            \input{4 - numerical implementation/main.tex}
        \part{Project Overview}
            \input{5 - research plan/main.tex}
            \input{6 - summary/main.tex}
    
    
    %\section{}
    \newpage
    \pagenumbering{gobble}
        \printbibliography


    \newpage
    \pagenumbering{roman}
    \appendix
        \part{Appendices}
            \input{8 - Hilbert complexes/main.tex}
            \input{9 - weak conservation proofs/main.tex}
\end{document}

\end{document}

        \part{Research}
            \documentclass[12pt, a4paper]{report}

\documentclass[12pt, a4paper]{report}

\input{template/main.tex}

\title{\BA{Title in Progress...}}
\author{Boris Andrews}
\affil{Mathematical Institute, University of Oxford}
\date{\today}


\begin{document}
    \pagenumbering{gobble}
    \maketitle
    
    
    \begin{abstract}
        Magnetic confinement reactors---in particular tokamaks---offer one of the most promising options for achieving practical nuclear fusion, with the potential to provide virtually limitless, clean energy. The theoretical and numerical modeling of tokamak plasmas is simultaneously an essential component of effective reactor design, and a great research barrier. Tokamak operational conditions exhibit comparatively low Knudsen numbers. Kinetic effects, including kinetic waves and instabilities, Landau damping, bump-on-tail instabilities and more, are therefore highly influential in tokamak plasma dynamics. Purely fluid models are inherently incapable of capturing these effects, whereas the high dimensionality in purely kinetic models render them practically intractable for most relevant purposes.

        We consider a $\delta\!f$ decomposition model, with a macroscopic fluid background and microscopic kinetic correction, both fully coupled to each other. A similar manner of discretization is proposed to that used in the recent \texttt{STRUPHY} code \cite{Holderied_Possanner_Wang_2021, Holderied_2022, Li_et_al_2023} with a finite-element model for the background and a pseudo-particle/PiC model for the correction.

        The fluid background satisfies the full, non-linear, resistive, compressible, Hall MHD equations. \cite{Laakmann_Hu_Farrell_2022} introduces finite-element(-in-space) implicit timesteppers for the incompressible analogue to this system with structure-preserving (SP) properties in the ideal case, alongside parameter-robust preconditioners. We show that these timesteppers can derive from a finite-element-in-time (FET) (and finite-element-in-space) interpretation. The benefits of this reformulation are discussed, including the derivation of timesteppers that are higher order in time, and the quantifiable dissipative SP properties in the non-ideal, resistive case.
        
        We discuss possible options for extending this FET approach to timesteppers for the compressible case.

        The kinetic corrections satisfy linearized Boltzmann equations. Using a Lénard--Bernstein collision operator, these take Fokker--Planck-like forms \cite{Fokker_1914, Planck_1917} wherein pseudo-particles in the numerical model obey the neoclassical transport equations, with particle-independent Brownian drift terms. This offers a rigorous methodology for incorporating collisions into the particle transport model, without coupling the equations of motions for each particle.
        
        Works by Chen, Chacón et al. \cite{Chen_Chacón_Barnes_2011, Chacón_Chen_Barnes_2013, Chen_Chacón_2014, Chen_Chacón_2015} have developed structure-preserving particle pushers for neoclassical transport in the Vlasov equations, derived from Crank--Nicolson integrators. We show these too can can derive from a FET interpretation, similarly offering potential extensions to higher-order-in-time particle pushers. The FET formulation is used also to consider how the stochastic drift terms can be incorporated into the pushers. Stochastic gyrokinetic expansions are also discussed.

        Different options for the numerical implementation of these schemes are considered.

        Due to the efficacy of FET in the development of SP timesteppers for both the fluid and kinetic component, we hope this approach will prove effective in the future for developing SP timesteppers for the full hybrid model. We hope this will give us the opportunity to incorporate previously inaccessible kinetic effects into the highly effective, modern, finite-element MHD models.
    \end{abstract}
    
    
    \newpage
    \tableofcontents
    
    
    \newpage
    \pagenumbering{arabic}
    %\linenumbers\renewcommand\thelinenumber{\color{black!50}\arabic{linenumber}}
            \input{0 - introduction/main.tex}
        \part{Research}
            \input{1 - low-noise PiC models/main.tex}
            \input{2 - kinetic component/main.tex}
            \input{3 - fluid component/main.tex}
            \input{4 - numerical implementation/main.tex}
        \part{Project Overview}
            \input{5 - research plan/main.tex}
            \input{6 - summary/main.tex}
    
    
    %\section{}
    \newpage
    \pagenumbering{gobble}
        \printbibliography


    \newpage
    \pagenumbering{roman}
    \appendix
        \part{Appendices}
            \input{8 - Hilbert complexes/main.tex}
            \input{9 - weak conservation proofs/main.tex}
\end{document}


\title{\BA{Title in Progress...}}
\author{Boris Andrews}
\affil{Mathematical Institute, University of Oxford}
\date{\today}


\begin{document}
    \pagenumbering{gobble}
    \maketitle
    
    
    \begin{abstract}
        Magnetic confinement reactors---in particular tokamaks---offer one of the most promising options for achieving practical nuclear fusion, with the potential to provide virtually limitless, clean energy. The theoretical and numerical modeling of tokamak plasmas is simultaneously an essential component of effective reactor design, and a great research barrier. Tokamak operational conditions exhibit comparatively low Knudsen numbers. Kinetic effects, including kinetic waves and instabilities, Landau damping, bump-on-tail instabilities and more, are therefore highly influential in tokamak plasma dynamics. Purely fluid models are inherently incapable of capturing these effects, whereas the high dimensionality in purely kinetic models render them practically intractable for most relevant purposes.

        We consider a $\delta\!f$ decomposition model, with a macroscopic fluid background and microscopic kinetic correction, both fully coupled to each other. A similar manner of discretization is proposed to that used in the recent \texttt{STRUPHY} code \cite{Holderied_Possanner_Wang_2021, Holderied_2022, Li_et_al_2023} with a finite-element model for the background and a pseudo-particle/PiC model for the correction.

        The fluid background satisfies the full, non-linear, resistive, compressible, Hall MHD equations. \cite{Laakmann_Hu_Farrell_2022} introduces finite-element(-in-space) implicit timesteppers for the incompressible analogue to this system with structure-preserving (SP) properties in the ideal case, alongside parameter-robust preconditioners. We show that these timesteppers can derive from a finite-element-in-time (FET) (and finite-element-in-space) interpretation. The benefits of this reformulation are discussed, including the derivation of timesteppers that are higher order in time, and the quantifiable dissipative SP properties in the non-ideal, resistive case.
        
        We discuss possible options for extending this FET approach to timesteppers for the compressible case.

        The kinetic corrections satisfy linearized Boltzmann equations. Using a Lénard--Bernstein collision operator, these take Fokker--Planck-like forms \cite{Fokker_1914, Planck_1917} wherein pseudo-particles in the numerical model obey the neoclassical transport equations, with particle-independent Brownian drift terms. This offers a rigorous methodology for incorporating collisions into the particle transport model, without coupling the equations of motions for each particle.
        
        Works by Chen, Chacón et al. \cite{Chen_Chacón_Barnes_2011, Chacón_Chen_Barnes_2013, Chen_Chacón_2014, Chen_Chacón_2015} have developed structure-preserving particle pushers for neoclassical transport in the Vlasov equations, derived from Crank--Nicolson integrators. We show these too can can derive from a FET interpretation, similarly offering potential extensions to higher-order-in-time particle pushers. The FET formulation is used also to consider how the stochastic drift terms can be incorporated into the pushers. Stochastic gyrokinetic expansions are also discussed.

        Different options for the numerical implementation of these schemes are considered.

        Due to the efficacy of FET in the development of SP timesteppers for both the fluid and kinetic component, we hope this approach will prove effective in the future for developing SP timesteppers for the full hybrid model. We hope this will give us the opportunity to incorporate previously inaccessible kinetic effects into the highly effective, modern, finite-element MHD models.
    \end{abstract}
    
    
    \newpage
    \tableofcontents
    
    
    \newpage
    \pagenumbering{arabic}
    %\linenumbers\renewcommand\thelinenumber{\color{black!50}\arabic{linenumber}}
            \documentclass[12pt, a4paper]{report}

\input{template/main.tex}

\title{\BA{Title in Progress...}}
\author{Boris Andrews}
\affil{Mathematical Institute, University of Oxford}
\date{\today}


\begin{document}
    \pagenumbering{gobble}
    \maketitle
    
    
    \begin{abstract}
        Magnetic confinement reactors---in particular tokamaks---offer one of the most promising options for achieving practical nuclear fusion, with the potential to provide virtually limitless, clean energy. The theoretical and numerical modeling of tokamak plasmas is simultaneously an essential component of effective reactor design, and a great research barrier. Tokamak operational conditions exhibit comparatively low Knudsen numbers. Kinetic effects, including kinetic waves and instabilities, Landau damping, bump-on-tail instabilities and more, are therefore highly influential in tokamak plasma dynamics. Purely fluid models are inherently incapable of capturing these effects, whereas the high dimensionality in purely kinetic models render them practically intractable for most relevant purposes.

        We consider a $\delta\!f$ decomposition model, with a macroscopic fluid background and microscopic kinetic correction, both fully coupled to each other. A similar manner of discretization is proposed to that used in the recent \texttt{STRUPHY} code \cite{Holderied_Possanner_Wang_2021, Holderied_2022, Li_et_al_2023} with a finite-element model for the background and a pseudo-particle/PiC model for the correction.

        The fluid background satisfies the full, non-linear, resistive, compressible, Hall MHD equations. \cite{Laakmann_Hu_Farrell_2022} introduces finite-element(-in-space) implicit timesteppers for the incompressible analogue to this system with structure-preserving (SP) properties in the ideal case, alongside parameter-robust preconditioners. We show that these timesteppers can derive from a finite-element-in-time (FET) (and finite-element-in-space) interpretation. The benefits of this reformulation are discussed, including the derivation of timesteppers that are higher order in time, and the quantifiable dissipative SP properties in the non-ideal, resistive case.
        
        We discuss possible options for extending this FET approach to timesteppers for the compressible case.

        The kinetic corrections satisfy linearized Boltzmann equations. Using a Lénard--Bernstein collision operator, these take Fokker--Planck-like forms \cite{Fokker_1914, Planck_1917} wherein pseudo-particles in the numerical model obey the neoclassical transport equations, with particle-independent Brownian drift terms. This offers a rigorous methodology for incorporating collisions into the particle transport model, without coupling the equations of motions for each particle.
        
        Works by Chen, Chacón et al. \cite{Chen_Chacón_Barnes_2011, Chacón_Chen_Barnes_2013, Chen_Chacón_2014, Chen_Chacón_2015} have developed structure-preserving particle pushers for neoclassical transport in the Vlasov equations, derived from Crank--Nicolson integrators. We show these too can can derive from a FET interpretation, similarly offering potential extensions to higher-order-in-time particle pushers. The FET formulation is used also to consider how the stochastic drift terms can be incorporated into the pushers. Stochastic gyrokinetic expansions are also discussed.

        Different options for the numerical implementation of these schemes are considered.

        Due to the efficacy of FET in the development of SP timesteppers for both the fluid and kinetic component, we hope this approach will prove effective in the future for developing SP timesteppers for the full hybrid model. We hope this will give us the opportunity to incorporate previously inaccessible kinetic effects into the highly effective, modern, finite-element MHD models.
    \end{abstract}
    
    
    \newpage
    \tableofcontents
    
    
    \newpage
    \pagenumbering{arabic}
    %\linenumbers\renewcommand\thelinenumber{\color{black!50}\arabic{linenumber}}
            \input{0 - introduction/main.tex}
        \part{Research}
            \input{1 - low-noise PiC models/main.tex}
            \input{2 - kinetic component/main.tex}
            \input{3 - fluid component/main.tex}
            \input{4 - numerical implementation/main.tex}
        \part{Project Overview}
            \input{5 - research plan/main.tex}
            \input{6 - summary/main.tex}
    
    
    %\section{}
    \newpage
    \pagenumbering{gobble}
        \printbibliography


    \newpage
    \pagenumbering{roman}
    \appendix
        \part{Appendices}
            \input{8 - Hilbert complexes/main.tex}
            \input{9 - weak conservation proofs/main.tex}
\end{document}

        \part{Research}
            \documentclass[12pt, a4paper]{report}

\input{template/main.tex}

\title{\BA{Title in Progress...}}
\author{Boris Andrews}
\affil{Mathematical Institute, University of Oxford}
\date{\today}


\begin{document}
    \pagenumbering{gobble}
    \maketitle
    
    
    \begin{abstract}
        Magnetic confinement reactors---in particular tokamaks---offer one of the most promising options for achieving practical nuclear fusion, with the potential to provide virtually limitless, clean energy. The theoretical and numerical modeling of tokamak plasmas is simultaneously an essential component of effective reactor design, and a great research barrier. Tokamak operational conditions exhibit comparatively low Knudsen numbers. Kinetic effects, including kinetic waves and instabilities, Landau damping, bump-on-tail instabilities and more, are therefore highly influential in tokamak plasma dynamics. Purely fluid models are inherently incapable of capturing these effects, whereas the high dimensionality in purely kinetic models render them practically intractable for most relevant purposes.

        We consider a $\delta\!f$ decomposition model, with a macroscopic fluid background and microscopic kinetic correction, both fully coupled to each other. A similar manner of discretization is proposed to that used in the recent \texttt{STRUPHY} code \cite{Holderied_Possanner_Wang_2021, Holderied_2022, Li_et_al_2023} with a finite-element model for the background and a pseudo-particle/PiC model for the correction.

        The fluid background satisfies the full, non-linear, resistive, compressible, Hall MHD equations. \cite{Laakmann_Hu_Farrell_2022} introduces finite-element(-in-space) implicit timesteppers for the incompressible analogue to this system with structure-preserving (SP) properties in the ideal case, alongside parameter-robust preconditioners. We show that these timesteppers can derive from a finite-element-in-time (FET) (and finite-element-in-space) interpretation. The benefits of this reformulation are discussed, including the derivation of timesteppers that are higher order in time, and the quantifiable dissipative SP properties in the non-ideal, resistive case.
        
        We discuss possible options for extending this FET approach to timesteppers for the compressible case.

        The kinetic corrections satisfy linearized Boltzmann equations. Using a Lénard--Bernstein collision operator, these take Fokker--Planck-like forms \cite{Fokker_1914, Planck_1917} wherein pseudo-particles in the numerical model obey the neoclassical transport equations, with particle-independent Brownian drift terms. This offers a rigorous methodology for incorporating collisions into the particle transport model, without coupling the equations of motions for each particle.
        
        Works by Chen, Chacón et al. \cite{Chen_Chacón_Barnes_2011, Chacón_Chen_Barnes_2013, Chen_Chacón_2014, Chen_Chacón_2015} have developed structure-preserving particle pushers for neoclassical transport in the Vlasov equations, derived from Crank--Nicolson integrators. We show these too can can derive from a FET interpretation, similarly offering potential extensions to higher-order-in-time particle pushers. The FET formulation is used also to consider how the stochastic drift terms can be incorporated into the pushers. Stochastic gyrokinetic expansions are also discussed.

        Different options for the numerical implementation of these schemes are considered.

        Due to the efficacy of FET in the development of SP timesteppers for both the fluid and kinetic component, we hope this approach will prove effective in the future for developing SP timesteppers for the full hybrid model. We hope this will give us the opportunity to incorporate previously inaccessible kinetic effects into the highly effective, modern, finite-element MHD models.
    \end{abstract}
    
    
    \newpage
    \tableofcontents
    
    
    \newpage
    \pagenumbering{arabic}
    %\linenumbers\renewcommand\thelinenumber{\color{black!50}\arabic{linenumber}}
            \input{0 - introduction/main.tex}
        \part{Research}
            \input{1 - low-noise PiC models/main.tex}
            \input{2 - kinetic component/main.tex}
            \input{3 - fluid component/main.tex}
            \input{4 - numerical implementation/main.tex}
        \part{Project Overview}
            \input{5 - research plan/main.tex}
            \input{6 - summary/main.tex}
    
    
    %\section{}
    \newpage
    \pagenumbering{gobble}
        \printbibliography


    \newpage
    \pagenumbering{roman}
    \appendix
        \part{Appendices}
            \input{8 - Hilbert complexes/main.tex}
            \input{9 - weak conservation proofs/main.tex}
\end{document}

            \documentclass[12pt, a4paper]{report}

\input{template/main.tex}

\title{\BA{Title in Progress...}}
\author{Boris Andrews}
\affil{Mathematical Institute, University of Oxford}
\date{\today}


\begin{document}
    \pagenumbering{gobble}
    \maketitle
    
    
    \begin{abstract}
        Magnetic confinement reactors---in particular tokamaks---offer one of the most promising options for achieving practical nuclear fusion, with the potential to provide virtually limitless, clean energy. The theoretical and numerical modeling of tokamak plasmas is simultaneously an essential component of effective reactor design, and a great research barrier. Tokamak operational conditions exhibit comparatively low Knudsen numbers. Kinetic effects, including kinetic waves and instabilities, Landau damping, bump-on-tail instabilities and more, are therefore highly influential in tokamak plasma dynamics. Purely fluid models are inherently incapable of capturing these effects, whereas the high dimensionality in purely kinetic models render them practically intractable for most relevant purposes.

        We consider a $\delta\!f$ decomposition model, with a macroscopic fluid background and microscopic kinetic correction, both fully coupled to each other. A similar manner of discretization is proposed to that used in the recent \texttt{STRUPHY} code \cite{Holderied_Possanner_Wang_2021, Holderied_2022, Li_et_al_2023} with a finite-element model for the background and a pseudo-particle/PiC model for the correction.

        The fluid background satisfies the full, non-linear, resistive, compressible, Hall MHD equations. \cite{Laakmann_Hu_Farrell_2022} introduces finite-element(-in-space) implicit timesteppers for the incompressible analogue to this system with structure-preserving (SP) properties in the ideal case, alongside parameter-robust preconditioners. We show that these timesteppers can derive from a finite-element-in-time (FET) (and finite-element-in-space) interpretation. The benefits of this reformulation are discussed, including the derivation of timesteppers that are higher order in time, and the quantifiable dissipative SP properties in the non-ideal, resistive case.
        
        We discuss possible options for extending this FET approach to timesteppers for the compressible case.

        The kinetic corrections satisfy linearized Boltzmann equations. Using a Lénard--Bernstein collision operator, these take Fokker--Planck-like forms \cite{Fokker_1914, Planck_1917} wherein pseudo-particles in the numerical model obey the neoclassical transport equations, with particle-independent Brownian drift terms. This offers a rigorous methodology for incorporating collisions into the particle transport model, without coupling the equations of motions for each particle.
        
        Works by Chen, Chacón et al. \cite{Chen_Chacón_Barnes_2011, Chacón_Chen_Barnes_2013, Chen_Chacón_2014, Chen_Chacón_2015} have developed structure-preserving particle pushers for neoclassical transport in the Vlasov equations, derived from Crank--Nicolson integrators. We show these too can can derive from a FET interpretation, similarly offering potential extensions to higher-order-in-time particle pushers. The FET formulation is used also to consider how the stochastic drift terms can be incorporated into the pushers. Stochastic gyrokinetic expansions are also discussed.

        Different options for the numerical implementation of these schemes are considered.

        Due to the efficacy of FET in the development of SP timesteppers for both the fluid and kinetic component, we hope this approach will prove effective in the future for developing SP timesteppers for the full hybrid model. We hope this will give us the opportunity to incorporate previously inaccessible kinetic effects into the highly effective, modern, finite-element MHD models.
    \end{abstract}
    
    
    \newpage
    \tableofcontents
    
    
    \newpage
    \pagenumbering{arabic}
    %\linenumbers\renewcommand\thelinenumber{\color{black!50}\arabic{linenumber}}
            \input{0 - introduction/main.tex}
        \part{Research}
            \input{1 - low-noise PiC models/main.tex}
            \input{2 - kinetic component/main.tex}
            \input{3 - fluid component/main.tex}
            \input{4 - numerical implementation/main.tex}
        \part{Project Overview}
            \input{5 - research plan/main.tex}
            \input{6 - summary/main.tex}
    
    
    %\section{}
    \newpage
    \pagenumbering{gobble}
        \printbibliography


    \newpage
    \pagenumbering{roman}
    \appendix
        \part{Appendices}
            \input{8 - Hilbert complexes/main.tex}
            \input{9 - weak conservation proofs/main.tex}
\end{document}

            \documentclass[12pt, a4paper]{report}

\input{template/main.tex}

\title{\BA{Title in Progress...}}
\author{Boris Andrews}
\affil{Mathematical Institute, University of Oxford}
\date{\today}


\begin{document}
    \pagenumbering{gobble}
    \maketitle
    
    
    \begin{abstract}
        Magnetic confinement reactors---in particular tokamaks---offer one of the most promising options for achieving practical nuclear fusion, with the potential to provide virtually limitless, clean energy. The theoretical and numerical modeling of tokamak plasmas is simultaneously an essential component of effective reactor design, and a great research barrier. Tokamak operational conditions exhibit comparatively low Knudsen numbers. Kinetic effects, including kinetic waves and instabilities, Landau damping, bump-on-tail instabilities and more, are therefore highly influential in tokamak plasma dynamics. Purely fluid models are inherently incapable of capturing these effects, whereas the high dimensionality in purely kinetic models render them practically intractable for most relevant purposes.

        We consider a $\delta\!f$ decomposition model, with a macroscopic fluid background and microscopic kinetic correction, both fully coupled to each other. A similar manner of discretization is proposed to that used in the recent \texttt{STRUPHY} code \cite{Holderied_Possanner_Wang_2021, Holderied_2022, Li_et_al_2023} with a finite-element model for the background and a pseudo-particle/PiC model for the correction.

        The fluid background satisfies the full, non-linear, resistive, compressible, Hall MHD equations. \cite{Laakmann_Hu_Farrell_2022} introduces finite-element(-in-space) implicit timesteppers for the incompressible analogue to this system with structure-preserving (SP) properties in the ideal case, alongside parameter-robust preconditioners. We show that these timesteppers can derive from a finite-element-in-time (FET) (and finite-element-in-space) interpretation. The benefits of this reformulation are discussed, including the derivation of timesteppers that are higher order in time, and the quantifiable dissipative SP properties in the non-ideal, resistive case.
        
        We discuss possible options for extending this FET approach to timesteppers for the compressible case.

        The kinetic corrections satisfy linearized Boltzmann equations. Using a Lénard--Bernstein collision operator, these take Fokker--Planck-like forms \cite{Fokker_1914, Planck_1917} wherein pseudo-particles in the numerical model obey the neoclassical transport equations, with particle-independent Brownian drift terms. This offers a rigorous methodology for incorporating collisions into the particle transport model, without coupling the equations of motions for each particle.
        
        Works by Chen, Chacón et al. \cite{Chen_Chacón_Barnes_2011, Chacón_Chen_Barnes_2013, Chen_Chacón_2014, Chen_Chacón_2015} have developed structure-preserving particle pushers for neoclassical transport in the Vlasov equations, derived from Crank--Nicolson integrators. We show these too can can derive from a FET interpretation, similarly offering potential extensions to higher-order-in-time particle pushers. The FET formulation is used also to consider how the stochastic drift terms can be incorporated into the pushers. Stochastic gyrokinetic expansions are also discussed.

        Different options for the numerical implementation of these schemes are considered.

        Due to the efficacy of FET in the development of SP timesteppers for both the fluid and kinetic component, we hope this approach will prove effective in the future for developing SP timesteppers for the full hybrid model. We hope this will give us the opportunity to incorporate previously inaccessible kinetic effects into the highly effective, modern, finite-element MHD models.
    \end{abstract}
    
    
    \newpage
    \tableofcontents
    
    
    \newpage
    \pagenumbering{arabic}
    %\linenumbers\renewcommand\thelinenumber{\color{black!50}\arabic{linenumber}}
            \input{0 - introduction/main.tex}
        \part{Research}
            \input{1 - low-noise PiC models/main.tex}
            \input{2 - kinetic component/main.tex}
            \input{3 - fluid component/main.tex}
            \input{4 - numerical implementation/main.tex}
        \part{Project Overview}
            \input{5 - research plan/main.tex}
            \input{6 - summary/main.tex}
    
    
    %\section{}
    \newpage
    \pagenumbering{gobble}
        \printbibliography


    \newpage
    \pagenumbering{roman}
    \appendix
        \part{Appendices}
            \input{8 - Hilbert complexes/main.tex}
            \input{9 - weak conservation proofs/main.tex}
\end{document}

            \documentclass[12pt, a4paper]{report}

\input{template/main.tex}

\title{\BA{Title in Progress...}}
\author{Boris Andrews}
\affil{Mathematical Institute, University of Oxford}
\date{\today}


\begin{document}
    \pagenumbering{gobble}
    \maketitle
    
    
    \begin{abstract}
        Magnetic confinement reactors---in particular tokamaks---offer one of the most promising options for achieving practical nuclear fusion, with the potential to provide virtually limitless, clean energy. The theoretical and numerical modeling of tokamak plasmas is simultaneously an essential component of effective reactor design, and a great research barrier. Tokamak operational conditions exhibit comparatively low Knudsen numbers. Kinetic effects, including kinetic waves and instabilities, Landau damping, bump-on-tail instabilities and more, are therefore highly influential in tokamak plasma dynamics. Purely fluid models are inherently incapable of capturing these effects, whereas the high dimensionality in purely kinetic models render them practically intractable for most relevant purposes.

        We consider a $\delta\!f$ decomposition model, with a macroscopic fluid background and microscopic kinetic correction, both fully coupled to each other. A similar manner of discretization is proposed to that used in the recent \texttt{STRUPHY} code \cite{Holderied_Possanner_Wang_2021, Holderied_2022, Li_et_al_2023} with a finite-element model for the background and a pseudo-particle/PiC model for the correction.

        The fluid background satisfies the full, non-linear, resistive, compressible, Hall MHD equations. \cite{Laakmann_Hu_Farrell_2022} introduces finite-element(-in-space) implicit timesteppers for the incompressible analogue to this system with structure-preserving (SP) properties in the ideal case, alongside parameter-robust preconditioners. We show that these timesteppers can derive from a finite-element-in-time (FET) (and finite-element-in-space) interpretation. The benefits of this reformulation are discussed, including the derivation of timesteppers that are higher order in time, and the quantifiable dissipative SP properties in the non-ideal, resistive case.
        
        We discuss possible options for extending this FET approach to timesteppers for the compressible case.

        The kinetic corrections satisfy linearized Boltzmann equations. Using a Lénard--Bernstein collision operator, these take Fokker--Planck-like forms \cite{Fokker_1914, Planck_1917} wherein pseudo-particles in the numerical model obey the neoclassical transport equations, with particle-independent Brownian drift terms. This offers a rigorous methodology for incorporating collisions into the particle transport model, without coupling the equations of motions for each particle.
        
        Works by Chen, Chacón et al. \cite{Chen_Chacón_Barnes_2011, Chacón_Chen_Barnes_2013, Chen_Chacón_2014, Chen_Chacón_2015} have developed structure-preserving particle pushers for neoclassical transport in the Vlasov equations, derived from Crank--Nicolson integrators. We show these too can can derive from a FET interpretation, similarly offering potential extensions to higher-order-in-time particle pushers. The FET formulation is used also to consider how the stochastic drift terms can be incorporated into the pushers. Stochastic gyrokinetic expansions are also discussed.

        Different options for the numerical implementation of these schemes are considered.

        Due to the efficacy of FET in the development of SP timesteppers for both the fluid and kinetic component, we hope this approach will prove effective in the future for developing SP timesteppers for the full hybrid model. We hope this will give us the opportunity to incorporate previously inaccessible kinetic effects into the highly effective, modern, finite-element MHD models.
    \end{abstract}
    
    
    \newpage
    \tableofcontents
    
    
    \newpage
    \pagenumbering{arabic}
    %\linenumbers\renewcommand\thelinenumber{\color{black!50}\arabic{linenumber}}
            \input{0 - introduction/main.tex}
        \part{Research}
            \input{1 - low-noise PiC models/main.tex}
            \input{2 - kinetic component/main.tex}
            \input{3 - fluid component/main.tex}
            \input{4 - numerical implementation/main.tex}
        \part{Project Overview}
            \input{5 - research plan/main.tex}
            \input{6 - summary/main.tex}
    
    
    %\section{}
    \newpage
    \pagenumbering{gobble}
        \printbibliography


    \newpage
    \pagenumbering{roman}
    \appendix
        \part{Appendices}
            \input{8 - Hilbert complexes/main.tex}
            \input{9 - weak conservation proofs/main.tex}
\end{document}

        \part{Project Overview}
            \documentclass[12pt, a4paper]{report}

\input{template/main.tex}

\title{\BA{Title in Progress...}}
\author{Boris Andrews}
\affil{Mathematical Institute, University of Oxford}
\date{\today}


\begin{document}
    \pagenumbering{gobble}
    \maketitle
    
    
    \begin{abstract}
        Magnetic confinement reactors---in particular tokamaks---offer one of the most promising options for achieving practical nuclear fusion, with the potential to provide virtually limitless, clean energy. The theoretical and numerical modeling of tokamak plasmas is simultaneously an essential component of effective reactor design, and a great research barrier. Tokamak operational conditions exhibit comparatively low Knudsen numbers. Kinetic effects, including kinetic waves and instabilities, Landau damping, bump-on-tail instabilities and more, are therefore highly influential in tokamak plasma dynamics. Purely fluid models are inherently incapable of capturing these effects, whereas the high dimensionality in purely kinetic models render them practically intractable for most relevant purposes.

        We consider a $\delta\!f$ decomposition model, with a macroscopic fluid background and microscopic kinetic correction, both fully coupled to each other. A similar manner of discretization is proposed to that used in the recent \texttt{STRUPHY} code \cite{Holderied_Possanner_Wang_2021, Holderied_2022, Li_et_al_2023} with a finite-element model for the background and a pseudo-particle/PiC model for the correction.

        The fluid background satisfies the full, non-linear, resistive, compressible, Hall MHD equations. \cite{Laakmann_Hu_Farrell_2022} introduces finite-element(-in-space) implicit timesteppers for the incompressible analogue to this system with structure-preserving (SP) properties in the ideal case, alongside parameter-robust preconditioners. We show that these timesteppers can derive from a finite-element-in-time (FET) (and finite-element-in-space) interpretation. The benefits of this reformulation are discussed, including the derivation of timesteppers that are higher order in time, and the quantifiable dissipative SP properties in the non-ideal, resistive case.
        
        We discuss possible options for extending this FET approach to timesteppers for the compressible case.

        The kinetic corrections satisfy linearized Boltzmann equations. Using a Lénard--Bernstein collision operator, these take Fokker--Planck-like forms \cite{Fokker_1914, Planck_1917} wherein pseudo-particles in the numerical model obey the neoclassical transport equations, with particle-independent Brownian drift terms. This offers a rigorous methodology for incorporating collisions into the particle transport model, without coupling the equations of motions for each particle.
        
        Works by Chen, Chacón et al. \cite{Chen_Chacón_Barnes_2011, Chacón_Chen_Barnes_2013, Chen_Chacón_2014, Chen_Chacón_2015} have developed structure-preserving particle pushers for neoclassical transport in the Vlasov equations, derived from Crank--Nicolson integrators. We show these too can can derive from a FET interpretation, similarly offering potential extensions to higher-order-in-time particle pushers. The FET formulation is used also to consider how the stochastic drift terms can be incorporated into the pushers. Stochastic gyrokinetic expansions are also discussed.

        Different options for the numerical implementation of these schemes are considered.

        Due to the efficacy of FET in the development of SP timesteppers for both the fluid and kinetic component, we hope this approach will prove effective in the future for developing SP timesteppers for the full hybrid model. We hope this will give us the opportunity to incorporate previously inaccessible kinetic effects into the highly effective, modern, finite-element MHD models.
    \end{abstract}
    
    
    \newpage
    \tableofcontents
    
    
    \newpage
    \pagenumbering{arabic}
    %\linenumbers\renewcommand\thelinenumber{\color{black!50}\arabic{linenumber}}
            \input{0 - introduction/main.tex}
        \part{Research}
            \input{1 - low-noise PiC models/main.tex}
            \input{2 - kinetic component/main.tex}
            \input{3 - fluid component/main.tex}
            \input{4 - numerical implementation/main.tex}
        \part{Project Overview}
            \input{5 - research plan/main.tex}
            \input{6 - summary/main.tex}
    
    
    %\section{}
    \newpage
    \pagenumbering{gobble}
        \printbibliography


    \newpage
    \pagenumbering{roman}
    \appendix
        \part{Appendices}
            \input{8 - Hilbert complexes/main.tex}
            \input{9 - weak conservation proofs/main.tex}
\end{document}

            \documentclass[12pt, a4paper]{report}

\input{template/main.tex}

\title{\BA{Title in Progress...}}
\author{Boris Andrews}
\affil{Mathematical Institute, University of Oxford}
\date{\today}


\begin{document}
    \pagenumbering{gobble}
    \maketitle
    
    
    \begin{abstract}
        Magnetic confinement reactors---in particular tokamaks---offer one of the most promising options for achieving practical nuclear fusion, with the potential to provide virtually limitless, clean energy. The theoretical and numerical modeling of tokamak plasmas is simultaneously an essential component of effective reactor design, and a great research barrier. Tokamak operational conditions exhibit comparatively low Knudsen numbers. Kinetic effects, including kinetic waves and instabilities, Landau damping, bump-on-tail instabilities and more, are therefore highly influential in tokamak plasma dynamics. Purely fluid models are inherently incapable of capturing these effects, whereas the high dimensionality in purely kinetic models render them practically intractable for most relevant purposes.

        We consider a $\delta\!f$ decomposition model, with a macroscopic fluid background and microscopic kinetic correction, both fully coupled to each other. A similar manner of discretization is proposed to that used in the recent \texttt{STRUPHY} code \cite{Holderied_Possanner_Wang_2021, Holderied_2022, Li_et_al_2023} with a finite-element model for the background and a pseudo-particle/PiC model for the correction.

        The fluid background satisfies the full, non-linear, resistive, compressible, Hall MHD equations. \cite{Laakmann_Hu_Farrell_2022} introduces finite-element(-in-space) implicit timesteppers for the incompressible analogue to this system with structure-preserving (SP) properties in the ideal case, alongside parameter-robust preconditioners. We show that these timesteppers can derive from a finite-element-in-time (FET) (and finite-element-in-space) interpretation. The benefits of this reformulation are discussed, including the derivation of timesteppers that are higher order in time, and the quantifiable dissipative SP properties in the non-ideal, resistive case.
        
        We discuss possible options for extending this FET approach to timesteppers for the compressible case.

        The kinetic corrections satisfy linearized Boltzmann equations. Using a Lénard--Bernstein collision operator, these take Fokker--Planck-like forms \cite{Fokker_1914, Planck_1917} wherein pseudo-particles in the numerical model obey the neoclassical transport equations, with particle-independent Brownian drift terms. This offers a rigorous methodology for incorporating collisions into the particle transport model, without coupling the equations of motions for each particle.
        
        Works by Chen, Chacón et al. \cite{Chen_Chacón_Barnes_2011, Chacón_Chen_Barnes_2013, Chen_Chacón_2014, Chen_Chacón_2015} have developed structure-preserving particle pushers for neoclassical transport in the Vlasov equations, derived from Crank--Nicolson integrators. We show these too can can derive from a FET interpretation, similarly offering potential extensions to higher-order-in-time particle pushers. The FET formulation is used also to consider how the stochastic drift terms can be incorporated into the pushers. Stochastic gyrokinetic expansions are also discussed.

        Different options for the numerical implementation of these schemes are considered.

        Due to the efficacy of FET in the development of SP timesteppers for both the fluid and kinetic component, we hope this approach will prove effective in the future for developing SP timesteppers for the full hybrid model. We hope this will give us the opportunity to incorporate previously inaccessible kinetic effects into the highly effective, modern, finite-element MHD models.
    \end{abstract}
    
    
    \newpage
    \tableofcontents
    
    
    \newpage
    \pagenumbering{arabic}
    %\linenumbers\renewcommand\thelinenumber{\color{black!50}\arabic{linenumber}}
            \input{0 - introduction/main.tex}
        \part{Research}
            \input{1 - low-noise PiC models/main.tex}
            \input{2 - kinetic component/main.tex}
            \input{3 - fluid component/main.tex}
            \input{4 - numerical implementation/main.tex}
        \part{Project Overview}
            \input{5 - research plan/main.tex}
            \input{6 - summary/main.tex}
    
    
    %\section{}
    \newpage
    \pagenumbering{gobble}
        \printbibliography


    \newpage
    \pagenumbering{roman}
    \appendix
        \part{Appendices}
            \input{8 - Hilbert complexes/main.tex}
            \input{9 - weak conservation proofs/main.tex}
\end{document}

    
    
    %\section{}
    \newpage
    \pagenumbering{gobble}
        \printbibliography


    \newpage
    \pagenumbering{roman}
    \appendix
        \part{Appendices}
            \documentclass[12pt, a4paper]{report}

\input{template/main.tex}

\title{\BA{Title in Progress...}}
\author{Boris Andrews}
\affil{Mathematical Institute, University of Oxford}
\date{\today}


\begin{document}
    \pagenumbering{gobble}
    \maketitle
    
    
    \begin{abstract}
        Magnetic confinement reactors---in particular tokamaks---offer one of the most promising options for achieving practical nuclear fusion, with the potential to provide virtually limitless, clean energy. The theoretical and numerical modeling of tokamak plasmas is simultaneously an essential component of effective reactor design, and a great research barrier. Tokamak operational conditions exhibit comparatively low Knudsen numbers. Kinetic effects, including kinetic waves and instabilities, Landau damping, bump-on-tail instabilities and more, are therefore highly influential in tokamak plasma dynamics. Purely fluid models are inherently incapable of capturing these effects, whereas the high dimensionality in purely kinetic models render them practically intractable for most relevant purposes.

        We consider a $\delta\!f$ decomposition model, with a macroscopic fluid background and microscopic kinetic correction, both fully coupled to each other. A similar manner of discretization is proposed to that used in the recent \texttt{STRUPHY} code \cite{Holderied_Possanner_Wang_2021, Holderied_2022, Li_et_al_2023} with a finite-element model for the background and a pseudo-particle/PiC model for the correction.

        The fluid background satisfies the full, non-linear, resistive, compressible, Hall MHD equations. \cite{Laakmann_Hu_Farrell_2022} introduces finite-element(-in-space) implicit timesteppers for the incompressible analogue to this system with structure-preserving (SP) properties in the ideal case, alongside parameter-robust preconditioners. We show that these timesteppers can derive from a finite-element-in-time (FET) (and finite-element-in-space) interpretation. The benefits of this reformulation are discussed, including the derivation of timesteppers that are higher order in time, and the quantifiable dissipative SP properties in the non-ideal, resistive case.
        
        We discuss possible options for extending this FET approach to timesteppers for the compressible case.

        The kinetic corrections satisfy linearized Boltzmann equations. Using a Lénard--Bernstein collision operator, these take Fokker--Planck-like forms \cite{Fokker_1914, Planck_1917} wherein pseudo-particles in the numerical model obey the neoclassical transport equations, with particle-independent Brownian drift terms. This offers a rigorous methodology for incorporating collisions into the particle transport model, without coupling the equations of motions for each particle.
        
        Works by Chen, Chacón et al. \cite{Chen_Chacón_Barnes_2011, Chacón_Chen_Barnes_2013, Chen_Chacón_2014, Chen_Chacón_2015} have developed structure-preserving particle pushers for neoclassical transport in the Vlasov equations, derived from Crank--Nicolson integrators. We show these too can can derive from a FET interpretation, similarly offering potential extensions to higher-order-in-time particle pushers. The FET formulation is used also to consider how the stochastic drift terms can be incorporated into the pushers. Stochastic gyrokinetic expansions are also discussed.

        Different options for the numerical implementation of these schemes are considered.

        Due to the efficacy of FET in the development of SP timesteppers for both the fluid and kinetic component, we hope this approach will prove effective in the future for developing SP timesteppers for the full hybrid model. We hope this will give us the opportunity to incorporate previously inaccessible kinetic effects into the highly effective, modern, finite-element MHD models.
    \end{abstract}
    
    
    \newpage
    \tableofcontents
    
    
    \newpage
    \pagenumbering{arabic}
    %\linenumbers\renewcommand\thelinenumber{\color{black!50}\arabic{linenumber}}
            \input{0 - introduction/main.tex}
        \part{Research}
            \input{1 - low-noise PiC models/main.tex}
            \input{2 - kinetic component/main.tex}
            \input{3 - fluid component/main.tex}
            \input{4 - numerical implementation/main.tex}
        \part{Project Overview}
            \input{5 - research plan/main.tex}
            \input{6 - summary/main.tex}
    
    
    %\section{}
    \newpage
    \pagenumbering{gobble}
        \printbibliography


    \newpage
    \pagenumbering{roman}
    \appendix
        \part{Appendices}
            \input{8 - Hilbert complexes/main.tex}
            \input{9 - weak conservation proofs/main.tex}
\end{document}

            \documentclass[12pt, a4paper]{report}

\input{template/main.tex}

\title{\BA{Title in Progress...}}
\author{Boris Andrews}
\affil{Mathematical Institute, University of Oxford}
\date{\today}


\begin{document}
    \pagenumbering{gobble}
    \maketitle
    
    
    \begin{abstract}
        Magnetic confinement reactors---in particular tokamaks---offer one of the most promising options for achieving practical nuclear fusion, with the potential to provide virtually limitless, clean energy. The theoretical and numerical modeling of tokamak plasmas is simultaneously an essential component of effective reactor design, and a great research barrier. Tokamak operational conditions exhibit comparatively low Knudsen numbers. Kinetic effects, including kinetic waves and instabilities, Landau damping, bump-on-tail instabilities and more, are therefore highly influential in tokamak plasma dynamics. Purely fluid models are inherently incapable of capturing these effects, whereas the high dimensionality in purely kinetic models render them practically intractable for most relevant purposes.

        We consider a $\delta\!f$ decomposition model, with a macroscopic fluid background and microscopic kinetic correction, both fully coupled to each other. A similar manner of discretization is proposed to that used in the recent \texttt{STRUPHY} code \cite{Holderied_Possanner_Wang_2021, Holderied_2022, Li_et_al_2023} with a finite-element model for the background and a pseudo-particle/PiC model for the correction.

        The fluid background satisfies the full, non-linear, resistive, compressible, Hall MHD equations. \cite{Laakmann_Hu_Farrell_2022} introduces finite-element(-in-space) implicit timesteppers for the incompressible analogue to this system with structure-preserving (SP) properties in the ideal case, alongside parameter-robust preconditioners. We show that these timesteppers can derive from a finite-element-in-time (FET) (and finite-element-in-space) interpretation. The benefits of this reformulation are discussed, including the derivation of timesteppers that are higher order in time, and the quantifiable dissipative SP properties in the non-ideal, resistive case.
        
        We discuss possible options for extending this FET approach to timesteppers for the compressible case.

        The kinetic corrections satisfy linearized Boltzmann equations. Using a Lénard--Bernstein collision operator, these take Fokker--Planck-like forms \cite{Fokker_1914, Planck_1917} wherein pseudo-particles in the numerical model obey the neoclassical transport equations, with particle-independent Brownian drift terms. This offers a rigorous methodology for incorporating collisions into the particle transport model, without coupling the equations of motions for each particle.
        
        Works by Chen, Chacón et al. \cite{Chen_Chacón_Barnes_2011, Chacón_Chen_Barnes_2013, Chen_Chacón_2014, Chen_Chacón_2015} have developed structure-preserving particle pushers for neoclassical transport in the Vlasov equations, derived from Crank--Nicolson integrators. We show these too can can derive from a FET interpretation, similarly offering potential extensions to higher-order-in-time particle pushers. The FET formulation is used also to consider how the stochastic drift terms can be incorporated into the pushers. Stochastic gyrokinetic expansions are also discussed.

        Different options for the numerical implementation of these schemes are considered.

        Due to the efficacy of FET in the development of SP timesteppers for both the fluid and kinetic component, we hope this approach will prove effective in the future for developing SP timesteppers for the full hybrid model. We hope this will give us the opportunity to incorporate previously inaccessible kinetic effects into the highly effective, modern, finite-element MHD models.
    \end{abstract}
    
    
    \newpage
    \tableofcontents
    
    
    \newpage
    \pagenumbering{arabic}
    %\linenumbers\renewcommand\thelinenumber{\color{black!50}\arabic{linenumber}}
            \input{0 - introduction/main.tex}
        \part{Research}
            \input{1 - low-noise PiC models/main.tex}
            \input{2 - kinetic component/main.tex}
            \input{3 - fluid component/main.tex}
            \input{4 - numerical implementation/main.tex}
        \part{Project Overview}
            \input{5 - research plan/main.tex}
            \input{6 - summary/main.tex}
    
    
    %\section{}
    \newpage
    \pagenumbering{gobble}
        \printbibliography


    \newpage
    \pagenumbering{roman}
    \appendix
        \part{Appendices}
            \input{8 - Hilbert complexes/main.tex}
            \input{9 - weak conservation proofs/main.tex}
\end{document}

\end{document}

            \documentclass[12pt, a4paper]{report}

\documentclass[12pt, a4paper]{report}

\input{template/main.tex}

\title{\BA{Title in Progress...}}
\author{Boris Andrews}
\affil{Mathematical Institute, University of Oxford}
\date{\today}


\begin{document}
    \pagenumbering{gobble}
    \maketitle
    
    
    \begin{abstract}
        Magnetic confinement reactors---in particular tokamaks---offer one of the most promising options for achieving practical nuclear fusion, with the potential to provide virtually limitless, clean energy. The theoretical and numerical modeling of tokamak plasmas is simultaneously an essential component of effective reactor design, and a great research barrier. Tokamak operational conditions exhibit comparatively low Knudsen numbers. Kinetic effects, including kinetic waves and instabilities, Landau damping, bump-on-tail instabilities and more, are therefore highly influential in tokamak plasma dynamics. Purely fluid models are inherently incapable of capturing these effects, whereas the high dimensionality in purely kinetic models render them practically intractable for most relevant purposes.

        We consider a $\delta\!f$ decomposition model, with a macroscopic fluid background and microscopic kinetic correction, both fully coupled to each other. A similar manner of discretization is proposed to that used in the recent \texttt{STRUPHY} code \cite{Holderied_Possanner_Wang_2021, Holderied_2022, Li_et_al_2023} with a finite-element model for the background and a pseudo-particle/PiC model for the correction.

        The fluid background satisfies the full, non-linear, resistive, compressible, Hall MHD equations. \cite{Laakmann_Hu_Farrell_2022} introduces finite-element(-in-space) implicit timesteppers for the incompressible analogue to this system with structure-preserving (SP) properties in the ideal case, alongside parameter-robust preconditioners. We show that these timesteppers can derive from a finite-element-in-time (FET) (and finite-element-in-space) interpretation. The benefits of this reformulation are discussed, including the derivation of timesteppers that are higher order in time, and the quantifiable dissipative SP properties in the non-ideal, resistive case.
        
        We discuss possible options for extending this FET approach to timesteppers for the compressible case.

        The kinetic corrections satisfy linearized Boltzmann equations. Using a Lénard--Bernstein collision operator, these take Fokker--Planck-like forms \cite{Fokker_1914, Planck_1917} wherein pseudo-particles in the numerical model obey the neoclassical transport equations, with particle-independent Brownian drift terms. This offers a rigorous methodology for incorporating collisions into the particle transport model, without coupling the equations of motions for each particle.
        
        Works by Chen, Chacón et al. \cite{Chen_Chacón_Barnes_2011, Chacón_Chen_Barnes_2013, Chen_Chacón_2014, Chen_Chacón_2015} have developed structure-preserving particle pushers for neoclassical transport in the Vlasov equations, derived from Crank--Nicolson integrators. We show these too can can derive from a FET interpretation, similarly offering potential extensions to higher-order-in-time particle pushers. The FET formulation is used also to consider how the stochastic drift terms can be incorporated into the pushers. Stochastic gyrokinetic expansions are also discussed.

        Different options for the numerical implementation of these schemes are considered.

        Due to the efficacy of FET in the development of SP timesteppers for both the fluid and kinetic component, we hope this approach will prove effective in the future for developing SP timesteppers for the full hybrid model. We hope this will give us the opportunity to incorporate previously inaccessible kinetic effects into the highly effective, modern, finite-element MHD models.
    \end{abstract}
    
    
    \newpage
    \tableofcontents
    
    
    \newpage
    \pagenumbering{arabic}
    %\linenumbers\renewcommand\thelinenumber{\color{black!50}\arabic{linenumber}}
            \input{0 - introduction/main.tex}
        \part{Research}
            \input{1 - low-noise PiC models/main.tex}
            \input{2 - kinetic component/main.tex}
            \input{3 - fluid component/main.tex}
            \input{4 - numerical implementation/main.tex}
        \part{Project Overview}
            \input{5 - research plan/main.tex}
            \input{6 - summary/main.tex}
    
    
    %\section{}
    \newpage
    \pagenumbering{gobble}
        \printbibliography


    \newpage
    \pagenumbering{roman}
    \appendix
        \part{Appendices}
            \input{8 - Hilbert complexes/main.tex}
            \input{9 - weak conservation proofs/main.tex}
\end{document}


\title{\BA{Title in Progress...}}
\author{Boris Andrews}
\affil{Mathematical Institute, University of Oxford}
\date{\today}


\begin{document}
    \pagenumbering{gobble}
    \maketitle
    
    
    \begin{abstract}
        Magnetic confinement reactors---in particular tokamaks---offer one of the most promising options for achieving practical nuclear fusion, with the potential to provide virtually limitless, clean energy. The theoretical and numerical modeling of tokamak plasmas is simultaneously an essential component of effective reactor design, and a great research barrier. Tokamak operational conditions exhibit comparatively low Knudsen numbers. Kinetic effects, including kinetic waves and instabilities, Landau damping, bump-on-tail instabilities and more, are therefore highly influential in tokamak plasma dynamics. Purely fluid models are inherently incapable of capturing these effects, whereas the high dimensionality in purely kinetic models render them practically intractable for most relevant purposes.

        We consider a $\delta\!f$ decomposition model, with a macroscopic fluid background and microscopic kinetic correction, both fully coupled to each other. A similar manner of discretization is proposed to that used in the recent \texttt{STRUPHY} code \cite{Holderied_Possanner_Wang_2021, Holderied_2022, Li_et_al_2023} with a finite-element model for the background and a pseudo-particle/PiC model for the correction.

        The fluid background satisfies the full, non-linear, resistive, compressible, Hall MHD equations. \cite{Laakmann_Hu_Farrell_2022} introduces finite-element(-in-space) implicit timesteppers for the incompressible analogue to this system with structure-preserving (SP) properties in the ideal case, alongside parameter-robust preconditioners. We show that these timesteppers can derive from a finite-element-in-time (FET) (and finite-element-in-space) interpretation. The benefits of this reformulation are discussed, including the derivation of timesteppers that are higher order in time, and the quantifiable dissipative SP properties in the non-ideal, resistive case.
        
        We discuss possible options for extending this FET approach to timesteppers for the compressible case.

        The kinetic corrections satisfy linearized Boltzmann equations. Using a Lénard--Bernstein collision operator, these take Fokker--Planck-like forms \cite{Fokker_1914, Planck_1917} wherein pseudo-particles in the numerical model obey the neoclassical transport equations, with particle-independent Brownian drift terms. This offers a rigorous methodology for incorporating collisions into the particle transport model, without coupling the equations of motions for each particle.
        
        Works by Chen, Chacón et al. \cite{Chen_Chacón_Barnes_2011, Chacón_Chen_Barnes_2013, Chen_Chacón_2014, Chen_Chacón_2015} have developed structure-preserving particle pushers for neoclassical transport in the Vlasov equations, derived from Crank--Nicolson integrators. We show these too can can derive from a FET interpretation, similarly offering potential extensions to higher-order-in-time particle pushers. The FET formulation is used also to consider how the stochastic drift terms can be incorporated into the pushers. Stochastic gyrokinetic expansions are also discussed.

        Different options for the numerical implementation of these schemes are considered.

        Due to the efficacy of FET in the development of SP timesteppers for both the fluid and kinetic component, we hope this approach will prove effective in the future for developing SP timesteppers for the full hybrid model. We hope this will give us the opportunity to incorporate previously inaccessible kinetic effects into the highly effective, modern, finite-element MHD models.
    \end{abstract}
    
    
    \newpage
    \tableofcontents
    
    
    \newpage
    \pagenumbering{arabic}
    %\linenumbers\renewcommand\thelinenumber{\color{black!50}\arabic{linenumber}}
            \documentclass[12pt, a4paper]{report}

\input{template/main.tex}

\title{\BA{Title in Progress...}}
\author{Boris Andrews}
\affil{Mathematical Institute, University of Oxford}
\date{\today}


\begin{document}
    \pagenumbering{gobble}
    \maketitle
    
    
    \begin{abstract}
        Magnetic confinement reactors---in particular tokamaks---offer one of the most promising options for achieving practical nuclear fusion, with the potential to provide virtually limitless, clean energy. The theoretical and numerical modeling of tokamak plasmas is simultaneously an essential component of effective reactor design, and a great research barrier. Tokamak operational conditions exhibit comparatively low Knudsen numbers. Kinetic effects, including kinetic waves and instabilities, Landau damping, bump-on-tail instabilities and more, are therefore highly influential in tokamak plasma dynamics. Purely fluid models are inherently incapable of capturing these effects, whereas the high dimensionality in purely kinetic models render them practically intractable for most relevant purposes.

        We consider a $\delta\!f$ decomposition model, with a macroscopic fluid background and microscopic kinetic correction, both fully coupled to each other. A similar manner of discretization is proposed to that used in the recent \texttt{STRUPHY} code \cite{Holderied_Possanner_Wang_2021, Holderied_2022, Li_et_al_2023} with a finite-element model for the background and a pseudo-particle/PiC model for the correction.

        The fluid background satisfies the full, non-linear, resistive, compressible, Hall MHD equations. \cite{Laakmann_Hu_Farrell_2022} introduces finite-element(-in-space) implicit timesteppers for the incompressible analogue to this system with structure-preserving (SP) properties in the ideal case, alongside parameter-robust preconditioners. We show that these timesteppers can derive from a finite-element-in-time (FET) (and finite-element-in-space) interpretation. The benefits of this reformulation are discussed, including the derivation of timesteppers that are higher order in time, and the quantifiable dissipative SP properties in the non-ideal, resistive case.
        
        We discuss possible options for extending this FET approach to timesteppers for the compressible case.

        The kinetic corrections satisfy linearized Boltzmann equations. Using a Lénard--Bernstein collision operator, these take Fokker--Planck-like forms \cite{Fokker_1914, Planck_1917} wherein pseudo-particles in the numerical model obey the neoclassical transport equations, with particle-independent Brownian drift terms. This offers a rigorous methodology for incorporating collisions into the particle transport model, without coupling the equations of motions for each particle.
        
        Works by Chen, Chacón et al. \cite{Chen_Chacón_Barnes_2011, Chacón_Chen_Barnes_2013, Chen_Chacón_2014, Chen_Chacón_2015} have developed structure-preserving particle pushers for neoclassical transport in the Vlasov equations, derived from Crank--Nicolson integrators. We show these too can can derive from a FET interpretation, similarly offering potential extensions to higher-order-in-time particle pushers. The FET formulation is used also to consider how the stochastic drift terms can be incorporated into the pushers. Stochastic gyrokinetic expansions are also discussed.

        Different options for the numerical implementation of these schemes are considered.

        Due to the efficacy of FET in the development of SP timesteppers for both the fluid and kinetic component, we hope this approach will prove effective in the future for developing SP timesteppers for the full hybrid model. We hope this will give us the opportunity to incorporate previously inaccessible kinetic effects into the highly effective, modern, finite-element MHD models.
    \end{abstract}
    
    
    \newpage
    \tableofcontents
    
    
    \newpage
    \pagenumbering{arabic}
    %\linenumbers\renewcommand\thelinenumber{\color{black!50}\arabic{linenumber}}
            \input{0 - introduction/main.tex}
        \part{Research}
            \input{1 - low-noise PiC models/main.tex}
            \input{2 - kinetic component/main.tex}
            \input{3 - fluid component/main.tex}
            \input{4 - numerical implementation/main.tex}
        \part{Project Overview}
            \input{5 - research plan/main.tex}
            \input{6 - summary/main.tex}
    
    
    %\section{}
    \newpage
    \pagenumbering{gobble}
        \printbibliography


    \newpage
    \pagenumbering{roman}
    \appendix
        \part{Appendices}
            \input{8 - Hilbert complexes/main.tex}
            \input{9 - weak conservation proofs/main.tex}
\end{document}

        \part{Research}
            \documentclass[12pt, a4paper]{report}

\input{template/main.tex}

\title{\BA{Title in Progress...}}
\author{Boris Andrews}
\affil{Mathematical Institute, University of Oxford}
\date{\today}


\begin{document}
    \pagenumbering{gobble}
    \maketitle
    
    
    \begin{abstract}
        Magnetic confinement reactors---in particular tokamaks---offer one of the most promising options for achieving practical nuclear fusion, with the potential to provide virtually limitless, clean energy. The theoretical and numerical modeling of tokamak plasmas is simultaneously an essential component of effective reactor design, and a great research barrier. Tokamak operational conditions exhibit comparatively low Knudsen numbers. Kinetic effects, including kinetic waves and instabilities, Landau damping, bump-on-tail instabilities and more, are therefore highly influential in tokamak plasma dynamics. Purely fluid models are inherently incapable of capturing these effects, whereas the high dimensionality in purely kinetic models render them practically intractable for most relevant purposes.

        We consider a $\delta\!f$ decomposition model, with a macroscopic fluid background and microscopic kinetic correction, both fully coupled to each other. A similar manner of discretization is proposed to that used in the recent \texttt{STRUPHY} code \cite{Holderied_Possanner_Wang_2021, Holderied_2022, Li_et_al_2023} with a finite-element model for the background and a pseudo-particle/PiC model for the correction.

        The fluid background satisfies the full, non-linear, resistive, compressible, Hall MHD equations. \cite{Laakmann_Hu_Farrell_2022} introduces finite-element(-in-space) implicit timesteppers for the incompressible analogue to this system with structure-preserving (SP) properties in the ideal case, alongside parameter-robust preconditioners. We show that these timesteppers can derive from a finite-element-in-time (FET) (and finite-element-in-space) interpretation. The benefits of this reformulation are discussed, including the derivation of timesteppers that are higher order in time, and the quantifiable dissipative SP properties in the non-ideal, resistive case.
        
        We discuss possible options for extending this FET approach to timesteppers for the compressible case.

        The kinetic corrections satisfy linearized Boltzmann equations. Using a Lénard--Bernstein collision operator, these take Fokker--Planck-like forms \cite{Fokker_1914, Planck_1917} wherein pseudo-particles in the numerical model obey the neoclassical transport equations, with particle-independent Brownian drift terms. This offers a rigorous methodology for incorporating collisions into the particle transport model, without coupling the equations of motions for each particle.
        
        Works by Chen, Chacón et al. \cite{Chen_Chacón_Barnes_2011, Chacón_Chen_Barnes_2013, Chen_Chacón_2014, Chen_Chacón_2015} have developed structure-preserving particle pushers for neoclassical transport in the Vlasov equations, derived from Crank--Nicolson integrators. We show these too can can derive from a FET interpretation, similarly offering potential extensions to higher-order-in-time particle pushers. The FET formulation is used also to consider how the stochastic drift terms can be incorporated into the pushers. Stochastic gyrokinetic expansions are also discussed.

        Different options for the numerical implementation of these schemes are considered.

        Due to the efficacy of FET in the development of SP timesteppers for both the fluid and kinetic component, we hope this approach will prove effective in the future for developing SP timesteppers for the full hybrid model. We hope this will give us the opportunity to incorporate previously inaccessible kinetic effects into the highly effective, modern, finite-element MHD models.
    \end{abstract}
    
    
    \newpage
    \tableofcontents
    
    
    \newpage
    \pagenumbering{arabic}
    %\linenumbers\renewcommand\thelinenumber{\color{black!50}\arabic{linenumber}}
            \input{0 - introduction/main.tex}
        \part{Research}
            \input{1 - low-noise PiC models/main.tex}
            \input{2 - kinetic component/main.tex}
            \input{3 - fluid component/main.tex}
            \input{4 - numerical implementation/main.tex}
        \part{Project Overview}
            \input{5 - research plan/main.tex}
            \input{6 - summary/main.tex}
    
    
    %\section{}
    \newpage
    \pagenumbering{gobble}
        \printbibliography


    \newpage
    \pagenumbering{roman}
    \appendix
        \part{Appendices}
            \input{8 - Hilbert complexes/main.tex}
            \input{9 - weak conservation proofs/main.tex}
\end{document}

            \documentclass[12pt, a4paper]{report}

\input{template/main.tex}

\title{\BA{Title in Progress...}}
\author{Boris Andrews}
\affil{Mathematical Institute, University of Oxford}
\date{\today}


\begin{document}
    \pagenumbering{gobble}
    \maketitle
    
    
    \begin{abstract}
        Magnetic confinement reactors---in particular tokamaks---offer one of the most promising options for achieving practical nuclear fusion, with the potential to provide virtually limitless, clean energy. The theoretical and numerical modeling of tokamak plasmas is simultaneously an essential component of effective reactor design, and a great research barrier. Tokamak operational conditions exhibit comparatively low Knudsen numbers. Kinetic effects, including kinetic waves and instabilities, Landau damping, bump-on-tail instabilities and more, are therefore highly influential in tokamak plasma dynamics. Purely fluid models are inherently incapable of capturing these effects, whereas the high dimensionality in purely kinetic models render them practically intractable for most relevant purposes.

        We consider a $\delta\!f$ decomposition model, with a macroscopic fluid background and microscopic kinetic correction, both fully coupled to each other. A similar manner of discretization is proposed to that used in the recent \texttt{STRUPHY} code \cite{Holderied_Possanner_Wang_2021, Holderied_2022, Li_et_al_2023} with a finite-element model for the background and a pseudo-particle/PiC model for the correction.

        The fluid background satisfies the full, non-linear, resistive, compressible, Hall MHD equations. \cite{Laakmann_Hu_Farrell_2022} introduces finite-element(-in-space) implicit timesteppers for the incompressible analogue to this system with structure-preserving (SP) properties in the ideal case, alongside parameter-robust preconditioners. We show that these timesteppers can derive from a finite-element-in-time (FET) (and finite-element-in-space) interpretation. The benefits of this reformulation are discussed, including the derivation of timesteppers that are higher order in time, and the quantifiable dissipative SP properties in the non-ideal, resistive case.
        
        We discuss possible options for extending this FET approach to timesteppers for the compressible case.

        The kinetic corrections satisfy linearized Boltzmann equations. Using a Lénard--Bernstein collision operator, these take Fokker--Planck-like forms \cite{Fokker_1914, Planck_1917} wherein pseudo-particles in the numerical model obey the neoclassical transport equations, with particle-independent Brownian drift terms. This offers a rigorous methodology for incorporating collisions into the particle transport model, without coupling the equations of motions for each particle.
        
        Works by Chen, Chacón et al. \cite{Chen_Chacón_Barnes_2011, Chacón_Chen_Barnes_2013, Chen_Chacón_2014, Chen_Chacón_2015} have developed structure-preserving particle pushers for neoclassical transport in the Vlasov equations, derived from Crank--Nicolson integrators. We show these too can can derive from a FET interpretation, similarly offering potential extensions to higher-order-in-time particle pushers. The FET formulation is used also to consider how the stochastic drift terms can be incorporated into the pushers. Stochastic gyrokinetic expansions are also discussed.

        Different options for the numerical implementation of these schemes are considered.

        Due to the efficacy of FET in the development of SP timesteppers for both the fluid and kinetic component, we hope this approach will prove effective in the future for developing SP timesteppers for the full hybrid model. We hope this will give us the opportunity to incorporate previously inaccessible kinetic effects into the highly effective, modern, finite-element MHD models.
    \end{abstract}
    
    
    \newpage
    \tableofcontents
    
    
    \newpage
    \pagenumbering{arabic}
    %\linenumbers\renewcommand\thelinenumber{\color{black!50}\arabic{linenumber}}
            \input{0 - introduction/main.tex}
        \part{Research}
            \input{1 - low-noise PiC models/main.tex}
            \input{2 - kinetic component/main.tex}
            \input{3 - fluid component/main.tex}
            \input{4 - numerical implementation/main.tex}
        \part{Project Overview}
            \input{5 - research plan/main.tex}
            \input{6 - summary/main.tex}
    
    
    %\section{}
    \newpage
    \pagenumbering{gobble}
        \printbibliography


    \newpage
    \pagenumbering{roman}
    \appendix
        \part{Appendices}
            \input{8 - Hilbert complexes/main.tex}
            \input{9 - weak conservation proofs/main.tex}
\end{document}

            \documentclass[12pt, a4paper]{report}

\input{template/main.tex}

\title{\BA{Title in Progress...}}
\author{Boris Andrews}
\affil{Mathematical Institute, University of Oxford}
\date{\today}


\begin{document}
    \pagenumbering{gobble}
    \maketitle
    
    
    \begin{abstract}
        Magnetic confinement reactors---in particular tokamaks---offer one of the most promising options for achieving practical nuclear fusion, with the potential to provide virtually limitless, clean energy. The theoretical and numerical modeling of tokamak plasmas is simultaneously an essential component of effective reactor design, and a great research barrier. Tokamak operational conditions exhibit comparatively low Knudsen numbers. Kinetic effects, including kinetic waves and instabilities, Landau damping, bump-on-tail instabilities and more, are therefore highly influential in tokamak plasma dynamics. Purely fluid models are inherently incapable of capturing these effects, whereas the high dimensionality in purely kinetic models render them practically intractable for most relevant purposes.

        We consider a $\delta\!f$ decomposition model, with a macroscopic fluid background and microscopic kinetic correction, both fully coupled to each other. A similar manner of discretization is proposed to that used in the recent \texttt{STRUPHY} code \cite{Holderied_Possanner_Wang_2021, Holderied_2022, Li_et_al_2023} with a finite-element model for the background and a pseudo-particle/PiC model for the correction.

        The fluid background satisfies the full, non-linear, resistive, compressible, Hall MHD equations. \cite{Laakmann_Hu_Farrell_2022} introduces finite-element(-in-space) implicit timesteppers for the incompressible analogue to this system with structure-preserving (SP) properties in the ideal case, alongside parameter-robust preconditioners. We show that these timesteppers can derive from a finite-element-in-time (FET) (and finite-element-in-space) interpretation. The benefits of this reformulation are discussed, including the derivation of timesteppers that are higher order in time, and the quantifiable dissipative SP properties in the non-ideal, resistive case.
        
        We discuss possible options for extending this FET approach to timesteppers for the compressible case.

        The kinetic corrections satisfy linearized Boltzmann equations. Using a Lénard--Bernstein collision operator, these take Fokker--Planck-like forms \cite{Fokker_1914, Planck_1917} wherein pseudo-particles in the numerical model obey the neoclassical transport equations, with particle-independent Brownian drift terms. This offers a rigorous methodology for incorporating collisions into the particle transport model, without coupling the equations of motions for each particle.
        
        Works by Chen, Chacón et al. \cite{Chen_Chacón_Barnes_2011, Chacón_Chen_Barnes_2013, Chen_Chacón_2014, Chen_Chacón_2015} have developed structure-preserving particle pushers for neoclassical transport in the Vlasov equations, derived from Crank--Nicolson integrators. We show these too can can derive from a FET interpretation, similarly offering potential extensions to higher-order-in-time particle pushers. The FET formulation is used also to consider how the stochastic drift terms can be incorporated into the pushers. Stochastic gyrokinetic expansions are also discussed.

        Different options for the numerical implementation of these schemes are considered.

        Due to the efficacy of FET in the development of SP timesteppers for both the fluid and kinetic component, we hope this approach will prove effective in the future for developing SP timesteppers for the full hybrid model. We hope this will give us the opportunity to incorporate previously inaccessible kinetic effects into the highly effective, modern, finite-element MHD models.
    \end{abstract}
    
    
    \newpage
    \tableofcontents
    
    
    \newpage
    \pagenumbering{arabic}
    %\linenumbers\renewcommand\thelinenumber{\color{black!50}\arabic{linenumber}}
            \input{0 - introduction/main.tex}
        \part{Research}
            \input{1 - low-noise PiC models/main.tex}
            \input{2 - kinetic component/main.tex}
            \input{3 - fluid component/main.tex}
            \input{4 - numerical implementation/main.tex}
        \part{Project Overview}
            \input{5 - research plan/main.tex}
            \input{6 - summary/main.tex}
    
    
    %\section{}
    \newpage
    \pagenumbering{gobble}
        \printbibliography


    \newpage
    \pagenumbering{roman}
    \appendix
        \part{Appendices}
            \input{8 - Hilbert complexes/main.tex}
            \input{9 - weak conservation proofs/main.tex}
\end{document}

            \documentclass[12pt, a4paper]{report}

\input{template/main.tex}

\title{\BA{Title in Progress...}}
\author{Boris Andrews}
\affil{Mathematical Institute, University of Oxford}
\date{\today}


\begin{document}
    \pagenumbering{gobble}
    \maketitle
    
    
    \begin{abstract}
        Magnetic confinement reactors---in particular tokamaks---offer one of the most promising options for achieving practical nuclear fusion, with the potential to provide virtually limitless, clean energy. The theoretical and numerical modeling of tokamak plasmas is simultaneously an essential component of effective reactor design, and a great research barrier. Tokamak operational conditions exhibit comparatively low Knudsen numbers. Kinetic effects, including kinetic waves and instabilities, Landau damping, bump-on-tail instabilities and more, are therefore highly influential in tokamak plasma dynamics. Purely fluid models are inherently incapable of capturing these effects, whereas the high dimensionality in purely kinetic models render them practically intractable for most relevant purposes.

        We consider a $\delta\!f$ decomposition model, with a macroscopic fluid background and microscopic kinetic correction, both fully coupled to each other. A similar manner of discretization is proposed to that used in the recent \texttt{STRUPHY} code \cite{Holderied_Possanner_Wang_2021, Holderied_2022, Li_et_al_2023} with a finite-element model for the background and a pseudo-particle/PiC model for the correction.

        The fluid background satisfies the full, non-linear, resistive, compressible, Hall MHD equations. \cite{Laakmann_Hu_Farrell_2022} introduces finite-element(-in-space) implicit timesteppers for the incompressible analogue to this system with structure-preserving (SP) properties in the ideal case, alongside parameter-robust preconditioners. We show that these timesteppers can derive from a finite-element-in-time (FET) (and finite-element-in-space) interpretation. The benefits of this reformulation are discussed, including the derivation of timesteppers that are higher order in time, and the quantifiable dissipative SP properties in the non-ideal, resistive case.
        
        We discuss possible options for extending this FET approach to timesteppers for the compressible case.

        The kinetic corrections satisfy linearized Boltzmann equations. Using a Lénard--Bernstein collision operator, these take Fokker--Planck-like forms \cite{Fokker_1914, Planck_1917} wherein pseudo-particles in the numerical model obey the neoclassical transport equations, with particle-independent Brownian drift terms. This offers a rigorous methodology for incorporating collisions into the particle transport model, without coupling the equations of motions for each particle.
        
        Works by Chen, Chacón et al. \cite{Chen_Chacón_Barnes_2011, Chacón_Chen_Barnes_2013, Chen_Chacón_2014, Chen_Chacón_2015} have developed structure-preserving particle pushers for neoclassical transport in the Vlasov equations, derived from Crank--Nicolson integrators. We show these too can can derive from a FET interpretation, similarly offering potential extensions to higher-order-in-time particle pushers. The FET formulation is used also to consider how the stochastic drift terms can be incorporated into the pushers. Stochastic gyrokinetic expansions are also discussed.

        Different options for the numerical implementation of these schemes are considered.

        Due to the efficacy of FET in the development of SP timesteppers for both the fluid and kinetic component, we hope this approach will prove effective in the future for developing SP timesteppers for the full hybrid model. We hope this will give us the opportunity to incorporate previously inaccessible kinetic effects into the highly effective, modern, finite-element MHD models.
    \end{abstract}
    
    
    \newpage
    \tableofcontents
    
    
    \newpage
    \pagenumbering{arabic}
    %\linenumbers\renewcommand\thelinenumber{\color{black!50}\arabic{linenumber}}
            \input{0 - introduction/main.tex}
        \part{Research}
            \input{1 - low-noise PiC models/main.tex}
            \input{2 - kinetic component/main.tex}
            \input{3 - fluid component/main.tex}
            \input{4 - numerical implementation/main.tex}
        \part{Project Overview}
            \input{5 - research plan/main.tex}
            \input{6 - summary/main.tex}
    
    
    %\section{}
    \newpage
    \pagenumbering{gobble}
        \printbibliography


    \newpage
    \pagenumbering{roman}
    \appendix
        \part{Appendices}
            \input{8 - Hilbert complexes/main.tex}
            \input{9 - weak conservation proofs/main.tex}
\end{document}

        \part{Project Overview}
            \documentclass[12pt, a4paper]{report}

\input{template/main.tex}

\title{\BA{Title in Progress...}}
\author{Boris Andrews}
\affil{Mathematical Institute, University of Oxford}
\date{\today}


\begin{document}
    \pagenumbering{gobble}
    \maketitle
    
    
    \begin{abstract}
        Magnetic confinement reactors---in particular tokamaks---offer one of the most promising options for achieving practical nuclear fusion, with the potential to provide virtually limitless, clean energy. The theoretical and numerical modeling of tokamak plasmas is simultaneously an essential component of effective reactor design, and a great research barrier. Tokamak operational conditions exhibit comparatively low Knudsen numbers. Kinetic effects, including kinetic waves and instabilities, Landau damping, bump-on-tail instabilities and more, are therefore highly influential in tokamak plasma dynamics. Purely fluid models are inherently incapable of capturing these effects, whereas the high dimensionality in purely kinetic models render them practically intractable for most relevant purposes.

        We consider a $\delta\!f$ decomposition model, with a macroscopic fluid background and microscopic kinetic correction, both fully coupled to each other. A similar manner of discretization is proposed to that used in the recent \texttt{STRUPHY} code \cite{Holderied_Possanner_Wang_2021, Holderied_2022, Li_et_al_2023} with a finite-element model for the background and a pseudo-particle/PiC model for the correction.

        The fluid background satisfies the full, non-linear, resistive, compressible, Hall MHD equations. \cite{Laakmann_Hu_Farrell_2022} introduces finite-element(-in-space) implicit timesteppers for the incompressible analogue to this system with structure-preserving (SP) properties in the ideal case, alongside parameter-robust preconditioners. We show that these timesteppers can derive from a finite-element-in-time (FET) (and finite-element-in-space) interpretation. The benefits of this reformulation are discussed, including the derivation of timesteppers that are higher order in time, and the quantifiable dissipative SP properties in the non-ideal, resistive case.
        
        We discuss possible options for extending this FET approach to timesteppers for the compressible case.

        The kinetic corrections satisfy linearized Boltzmann equations. Using a Lénard--Bernstein collision operator, these take Fokker--Planck-like forms \cite{Fokker_1914, Planck_1917} wherein pseudo-particles in the numerical model obey the neoclassical transport equations, with particle-independent Brownian drift terms. This offers a rigorous methodology for incorporating collisions into the particle transport model, without coupling the equations of motions for each particle.
        
        Works by Chen, Chacón et al. \cite{Chen_Chacón_Barnes_2011, Chacón_Chen_Barnes_2013, Chen_Chacón_2014, Chen_Chacón_2015} have developed structure-preserving particle pushers for neoclassical transport in the Vlasov equations, derived from Crank--Nicolson integrators. We show these too can can derive from a FET interpretation, similarly offering potential extensions to higher-order-in-time particle pushers. The FET formulation is used also to consider how the stochastic drift terms can be incorporated into the pushers. Stochastic gyrokinetic expansions are also discussed.

        Different options for the numerical implementation of these schemes are considered.

        Due to the efficacy of FET in the development of SP timesteppers for both the fluid and kinetic component, we hope this approach will prove effective in the future for developing SP timesteppers for the full hybrid model. We hope this will give us the opportunity to incorporate previously inaccessible kinetic effects into the highly effective, modern, finite-element MHD models.
    \end{abstract}
    
    
    \newpage
    \tableofcontents
    
    
    \newpage
    \pagenumbering{arabic}
    %\linenumbers\renewcommand\thelinenumber{\color{black!50}\arabic{linenumber}}
            \input{0 - introduction/main.tex}
        \part{Research}
            \input{1 - low-noise PiC models/main.tex}
            \input{2 - kinetic component/main.tex}
            \input{3 - fluid component/main.tex}
            \input{4 - numerical implementation/main.tex}
        \part{Project Overview}
            \input{5 - research plan/main.tex}
            \input{6 - summary/main.tex}
    
    
    %\section{}
    \newpage
    \pagenumbering{gobble}
        \printbibliography


    \newpage
    \pagenumbering{roman}
    \appendix
        \part{Appendices}
            \input{8 - Hilbert complexes/main.tex}
            \input{9 - weak conservation proofs/main.tex}
\end{document}

            \documentclass[12pt, a4paper]{report}

\input{template/main.tex}

\title{\BA{Title in Progress...}}
\author{Boris Andrews}
\affil{Mathematical Institute, University of Oxford}
\date{\today}


\begin{document}
    \pagenumbering{gobble}
    \maketitle
    
    
    \begin{abstract}
        Magnetic confinement reactors---in particular tokamaks---offer one of the most promising options for achieving practical nuclear fusion, with the potential to provide virtually limitless, clean energy. The theoretical and numerical modeling of tokamak plasmas is simultaneously an essential component of effective reactor design, and a great research barrier. Tokamak operational conditions exhibit comparatively low Knudsen numbers. Kinetic effects, including kinetic waves and instabilities, Landau damping, bump-on-tail instabilities and more, are therefore highly influential in tokamak plasma dynamics. Purely fluid models are inherently incapable of capturing these effects, whereas the high dimensionality in purely kinetic models render them practically intractable for most relevant purposes.

        We consider a $\delta\!f$ decomposition model, with a macroscopic fluid background and microscopic kinetic correction, both fully coupled to each other. A similar manner of discretization is proposed to that used in the recent \texttt{STRUPHY} code \cite{Holderied_Possanner_Wang_2021, Holderied_2022, Li_et_al_2023} with a finite-element model for the background and a pseudo-particle/PiC model for the correction.

        The fluid background satisfies the full, non-linear, resistive, compressible, Hall MHD equations. \cite{Laakmann_Hu_Farrell_2022} introduces finite-element(-in-space) implicit timesteppers for the incompressible analogue to this system with structure-preserving (SP) properties in the ideal case, alongside parameter-robust preconditioners. We show that these timesteppers can derive from a finite-element-in-time (FET) (and finite-element-in-space) interpretation. The benefits of this reformulation are discussed, including the derivation of timesteppers that are higher order in time, and the quantifiable dissipative SP properties in the non-ideal, resistive case.
        
        We discuss possible options for extending this FET approach to timesteppers for the compressible case.

        The kinetic corrections satisfy linearized Boltzmann equations. Using a Lénard--Bernstein collision operator, these take Fokker--Planck-like forms \cite{Fokker_1914, Planck_1917} wherein pseudo-particles in the numerical model obey the neoclassical transport equations, with particle-independent Brownian drift terms. This offers a rigorous methodology for incorporating collisions into the particle transport model, without coupling the equations of motions for each particle.
        
        Works by Chen, Chacón et al. \cite{Chen_Chacón_Barnes_2011, Chacón_Chen_Barnes_2013, Chen_Chacón_2014, Chen_Chacón_2015} have developed structure-preserving particle pushers for neoclassical transport in the Vlasov equations, derived from Crank--Nicolson integrators. We show these too can can derive from a FET interpretation, similarly offering potential extensions to higher-order-in-time particle pushers. The FET formulation is used also to consider how the stochastic drift terms can be incorporated into the pushers. Stochastic gyrokinetic expansions are also discussed.

        Different options for the numerical implementation of these schemes are considered.

        Due to the efficacy of FET in the development of SP timesteppers for both the fluid and kinetic component, we hope this approach will prove effective in the future for developing SP timesteppers for the full hybrid model. We hope this will give us the opportunity to incorporate previously inaccessible kinetic effects into the highly effective, modern, finite-element MHD models.
    \end{abstract}
    
    
    \newpage
    \tableofcontents
    
    
    \newpage
    \pagenumbering{arabic}
    %\linenumbers\renewcommand\thelinenumber{\color{black!50}\arabic{linenumber}}
            \input{0 - introduction/main.tex}
        \part{Research}
            \input{1 - low-noise PiC models/main.tex}
            \input{2 - kinetic component/main.tex}
            \input{3 - fluid component/main.tex}
            \input{4 - numerical implementation/main.tex}
        \part{Project Overview}
            \input{5 - research plan/main.tex}
            \input{6 - summary/main.tex}
    
    
    %\section{}
    \newpage
    \pagenumbering{gobble}
        \printbibliography


    \newpage
    \pagenumbering{roman}
    \appendix
        \part{Appendices}
            \input{8 - Hilbert complexes/main.tex}
            \input{9 - weak conservation proofs/main.tex}
\end{document}

    
    
    %\section{}
    \newpage
    \pagenumbering{gobble}
        \printbibliography


    \newpage
    \pagenumbering{roman}
    \appendix
        \part{Appendices}
            \documentclass[12pt, a4paper]{report}

\input{template/main.tex}

\title{\BA{Title in Progress...}}
\author{Boris Andrews}
\affil{Mathematical Institute, University of Oxford}
\date{\today}


\begin{document}
    \pagenumbering{gobble}
    \maketitle
    
    
    \begin{abstract}
        Magnetic confinement reactors---in particular tokamaks---offer one of the most promising options for achieving practical nuclear fusion, with the potential to provide virtually limitless, clean energy. The theoretical and numerical modeling of tokamak plasmas is simultaneously an essential component of effective reactor design, and a great research barrier. Tokamak operational conditions exhibit comparatively low Knudsen numbers. Kinetic effects, including kinetic waves and instabilities, Landau damping, bump-on-tail instabilities and more, are therefore highly influential in tokamak plasma dynamics. Purely fluid models are inherently incapable of capturing these effects, whereas the high dimensionality in purely kinetic models render them practically intractable for most relevant purposes.

        We consider a $\delta\!f$ decomposition model, with a macroscopic fluid background and microscopic kinetic correction, both fully coupled to each other. A similar manner of discretization is proposed to that used in the recent \texttt{STRUPHY} code \cite{Holderied_Possanner_Wang_2021, Holderied_2022, Li_et_al_2023} with a finite-element model for the background and a pseudo-particle/PiC model for the correction.

        The fluid background satisfies the full, non-linear, resistive, compressible, Hall MHD equations. \cite{Laakmann_Hu_Farrell_2022} introduces finite-element(-in-space) implicit timesteppers for the incompressible analogue to this system with structure-preserving (SP) properties in the ideal case, alongside parameter-robust preconditioners. We show that these timesteppers can derive from a finite-element-in-time (FET) (and finite-element-in-space) interpretation. The benefits of this reformulation are discussed, including the derivation of timesteppers that are higher order in time, and the quantifiable dissipative SP properties in the non-ideal, resistive case.
        
        We discuss possible options for extending this FET approach to timesteppers for the compressible case.

        The kinetic corrections satisfy linearized Boltzmann equations. Using a Lénard--Bernstein collision operator, these take Fokker--Planck-like forms \cite{Fokker_1914, Planck_1917} wherein pseudo-particles in the numerical model obey the neoclassical transport equations, with particle-independent Brownian drift terms. This offers a rigorous methodology for incorporating collisions into the particle transport model, without coupling the equations of motions for each particle.
        
        Works by Chen, Chacón et al. \cite{Chen_Chacón_Barnes_2011, Chacón_Chen_Barnes_2013, Chen_Chacón_2014, Chen_Chacón_2015} have developed structure-preserving particle pushers for neoclassical transport in the Vlasov equations, derived from Crank--Nicolson integrators. We show these too can can derive from a FET interpretation, similarly offering potential extensions to higher-order-in-time particle pushers. The FET formulation is used also to consider how the stochastic drift terms can be incorporated into the pushers. Stochastic gyrokinetic expansions are also discussed.

        Different options for the numerical implementation of these schemes are considered.

        Due to the efficacy of FET in the development of SP timesteppers for both the fluid and kinetic component, we hope this approach will prove effective in the future for developing SP timesteppers for the full hybrid model. We hope this will give us the opportunity to incorporate previously inaccessible kinetic effects into the highly effective, modern, finite-element MHD models.
    \end{abstract}
    
    
    \newpage
    \tableofcontents
    
    
    \newpage
    \pagenumbering{arabic}
    %\linenumbers\renewcommand\thelinenumber{\color{black!50}\arabic{linenumber}}
            \input{0 - introduction/main.tex}
        \part{Research}
            \input{1 - low-noise PiC models/main.tex}
            \input{2 - kinetic component/main.tex}
            \input{3 - fluid component/main.tex}
            \input{4 - numerical implementation/main.tex}
        \part{Project Overview}
            \input{5 - research plan/main.tex}
            \input{6 - summary/main.tex}
    
    
    %\section{}
    \newpage
    \pagenumbering{gobble}
        \printbibliography


    \newpage
    \pagenumbering{roman}
    \appendix
        \part{Appendices}
            \input{8 - Hilbert complexes/main.tex}
            \input{9 - weak conservation proofs/main.tex}
\end{document}

            \documentclass[12pt, a4paper]{report}

\input{template/main.tex}

\title{\BA{Title in Progress...}}
\author{Boris Andrews}
\affil{Mathematical Institute, University of Oxford}
\date{\today}


\begin{document}
    \pagenumbering{gobble}
    \maketitle
    
    
    \begin{abstract}
        Magnetic confinement reactors---in particular tokamaks---offer one of the most promising options for achieving practical nuclear fusion, with the potential to provide virtually limitless, clean energy. The theoretical and numerical modeling of tokamak plasmas is simultaneously an essential component of effective reactor design, and a great research barrier. Tokamak operational conditions exhibit comparatively low Knudsen numbers. Kinetic effects, including kinetic waves and instabilities, Landau damping, bump-on-tail instabilities and more, are therefore highly influential in tokamak plasma dynamics. Purely fluid models are inherently incapable of capturing these effects, whereas the high dimensionality in purely kinetic models render them practically intractable for most relevant purposes.

        We consider a $\delta\!f$ decomposition model, with a macroscopic fluid background and microscopic kinetic correction, both fully coupled to each other. A similar manner of discretization is proposed to that used in the recent \texttt{STRUPHY} code \cite{Holderied_Possanner_Wang_2021, Holderied_2022, Li_et_al_2023} with a finite-element model for the background and a pseudo-particle/PiC model for the correction.

        The fluid background satisfies the full, non-linear, resistive, compressible, Hall MHD equations. \cite{Laakmann_Hu_Farrell_2022} introduces finite-element(-in-space) implicit timesteppers for the incompressible analogue to this system with structure-preserving (SP) properties in the ideal case, alongside parameter-robust preconditioners. We show that these timesteppers can derive from a finite-element-in-time (FET) (and finite-element-in-space) interpretation. The benefits of this reformulation are discussed, including the derivation of timesteppers that are higher order in time, and the quantifiable dissipative SP properties in the non-ideal, resistive case.
        
        We discuss possible options for extending this FET approach to timesteppers for the compressible case.

        The kinetic corrections satisfy linearized Boltzmann equations. Using a Lénard--Bernstein collision operator, these take Fokker--Planck-like forms \cite{Fokker_1914, Planck_1917} wherein pseudo-particles in the numerical model obey the neoclassical transport equations, with particle-independent Brownian drift terms. This offers a rigorous methodology for incorporating collisions into the particle transport model, without coupling the equations of motions for each particle.
        
        Works by Chen, Chacón et al. \cite{Chen_Chacón_Barnes_2011, Chacón_Chen_Barnes_2013, Chen_Chacón_2014, Chen_Chacón_2015} have developed structure-preserving particle pushers for neoclassical transport in the Vlasov equations, derived from Crank--Nicolson integrators. We show these too can can derive from a FET interpretation, similarly offering potential extensions to higher-order-in-time particle pushers. The FET formulation is used also to consider how the stochastic drift terms can be incorporated into the pushers. Stochastic gyrokinetic expansions are also discussed.

        Different options for the numerical implementation of these schemes are considered.

        Due to the efficacy of FET in the development of SP timesteppers for both the fluid and kinetic component, we hope this approach will prove effective in the future for developing SP timesteppers for the full hybrid model. We hope this will give us the opportunity to incorporate previously inaccessible kinetic effects into the highly effective, modern, finite-element MHD models.
    \end{abstract}
    
    
    \newpage
    \tableofcontents
    
    
    \newpage
    \pagenumbering{arabic}
    %\linenumbers\renewcommand\thelinenumber{\color{black!50}\arabic{linenumber}}
            \input{0 - introduction/main.tex}
        \part{Research}
            \input{1 - low-noise PiC models/main.tex}
            \input{2 - kinetic component/main.tex}
            \input{3 - fluid component/main.tex}
            \input{4 - numerical implementation/main.tex}
        \part{Project Overview}
            \input{5 - research plan/main.tex}
            \input{6 - summary/main.tex}
    
    
    %\section{}
    \newpage
    \pagenumbering{gobble}
        \printbibliography


    \newpage
    \pagenumbering{roman}
    \appendix
        \part{Appendices}
            \input{8 - Hilbert complexes/main.tex}
            \input{9 - weak conservation proofs/main.tex}
\end{document}

\end{document}

            \documentclass[12pt, a4paper]{report}

\documentclass[12pt, a4paper]{report}

\input{template/main.tex}

\title{\BA{Title in Progress...}}
\author{Boris Andrews}
\affil{Mathematical Institute, University of Oxford}
\date{\today}


\begin{document}
    \pagenumbering{gobble}
    \maketitle
    
    
    \begin{abstract}
        Magnetic confinement reactors---in particular tokamaks---offer one of the most promising options for achieving practical nuclear fusion, with the potential to provide virtually limitless, clean energy. The theoretical and numerical modeling of tokamak plasmas is simultaneously an essential component of effective reactor design, and a great research barrier. Tokamak operational conditions exhibit comparatively low Knudsen numbers. Kinetic effects, including kinetic waves and instabilities, Landau damping, bump-on-tail instabilities and more, are therefore highly influential in tokamak plasma dynamics. Purely fluid models are inherently incapable of capturing these effects, whereas the high dimensionality in purely kinetic models render them practically intractable for most relevant purposes.

        We consider a $\delta\!f$ decomposition model, with a macroscopic fluid background and microscopic kinetic correction, both fully coupled to each other. A similar manner of discretization is proposed to that used in the recent \texttt{STRUPHY} code \cite{Holderied_Possanner_Wang_2021, Holderied_2022, Li_et_al_2023} with a finite-element model for the background and a pseudo-particle/PiC model for the correction.

        The fluid background satisfies the full, non-linear, resistive, compressible, Hall MHD equations. \cite{Laakmann_Hu_Farrell_2022} introduces finite-element(-in-space) implicit timesteppers for the incompressible analogue to this system with structure-preserving (SP) properties in the ideal case, alongside parameter-robust preconditioners. We show that these timesteppers can derive from a finite-element-in-time (FET) (and finite-element-in-space) interpretation. The benefits of this reformulation are discussed, including the derivation of timesteppers that are higher order in time, and the quantifiable dissipative SP properties in the non-ideal, resistive case.
        
        We discuss possible options for extending this FET approach to timesteppers for the compressible case.

        The kinetic corrections satisfy linearized Boltzmann equations. Using a Lénard--Bernstein collision operator, these take Fokker--Planck-like forms \cite{Fokker_1914, Planck_1917} wherein pseudo-particles in the numerical model obey the neoclassical transport equations, with particle-independent Brownian drift terms. This offers a rigorous methodology for incorporating collisions into the particle transport model, without coupling the equations of motions for each particle.
        
        Works by Chen, Chacón et al. \cite{Chen_Chacón_Barnes_2011, Chacón_Chen_Barnes_2013, Chen_Chacón_2014, Chen_Chacón_2015} have developed structure-preserving particle pushers for neoclassical transport in the Vlasov equations, derived from Crank--Nicolson integrators. We show these too can can derive from a FET interpretation, similarly offering potential extensions to higher-order-in-time particle pushers. The FET formulation is used also to consider how the stochastic drift terms can be incorporated into the pushers. Stochastic gyrokinetic expansions are also discussed.

        Different options for the numerical implementation of these schemes are considered.

        Due to the efficacy of FET in the development of SP timesteppers for both the fluid and kinetic component, we hope this approach will prove effective in the future for developing SP timesteppers for the full hybrid model. We hope this will give us the opportunity to incorporate previously inaccessible kinetic effects into the highly effective, modern, finite-element MHD models.
    \end{abstract}
    
    
    \newpage
    \tableofcontents
    
    
    \newpage
    \pagenumbering{arabic}
    %\linenumbers\renewcommand\thelinenumber{\color{black!50}\arabic{linenumber}}
            \input{0 - introduction/main.tex}
        \part{Research}
            \input{1 - low-noise PiC models/main.tex}
            \input{2 - kinetic component/main.tex}
            \input{3 - fluid component/main.tex}
            \input{4 - numerical implementation/main.tex}
        \part{Project Overview}
            \input{5 - research plan/main.tex}
            \input{6 - summary/main.tex}
    
    
    %\section{}
    \newpage
    \pagenumbering{gobble}
        \printbibliography


    \newpage
    \pagenumbering{roman}
    \appendix
        \part{Appendices}
            \input{8 - Hilbert complexes/main.tex}
            \input{9 - weak conservation proofs/main.tex}
\end{document}


\title{\BA{Title in Progress...}}
\author{Boris Andrews}
\affil{Mathematical Institute, University of Oxford}
\date{\today}


\begin{document}
    \pagenumbering{gobble}
    \maketitle
    
    
    \begin{abstract}
        Magnetic confinement reactors---in particular tokamaks---offer one of the most promising options for achieving practical nuclear fusion, with the potential to provide virtually limitless, clean energy. The theoretical and numerical modeling of tokamak plasmas is simultaneously an essential component of effective reactor design, and a great research barrier. Tokamak operational conditions exhibit comparatively low Knudsen numbers. Kinetic effects, including kinetic waves and instabilities, Landau damping, bump-on-tail instabilities and more, are therefore highly influential in tokamak plasma dynamics. Purely fluid models are inherently incapable of capturing these effects, whereas the high dimensionality in purely kinetic models render them practically intractable for most relevant purposes.

        We consider a $\delta\!f$ decomposition model, with a macroscopic fluid background and microscopic kinetic correction, both fully coupled to each other. A similar manner of discretization is proposed to that used in the recent \texttt{STRUPHY} code \cite{Holderied_Possanner_Wang_2021, Holderied_2022, Li_et_al_2023} with a finite-element model for the background and a pseudo-particle/PiC model for the correction.

        The fluid background satisfies the full, non-linear, resistive, compressible, Hall MHD equations. \cite{Laakmann_Hu_Farrell_2022} introduces finite-element(-in-space) implicit timesteppers for the incompressible analogue to this system with structure-preserving (SP) properties in the ideal case, alongside parameter-robust preconditioners. We show that these timesteppers can derive from a finite-element-in-time (FET) (and finite-element-in-space) interpretation. The benefits of this reformulation are discussed, including the derivation of timesteppers that are higher order in time, and the quantifiable dissipative SP properties in the non-ideal, resistive case.
        
        We discuss possible options for extending this FET approach to timesteppers for the compressible case.

        The kinetic corrections satisfy linearized Boltzmann equations. Using a Lénard--Bernstein collision operator, these take Fokker--Planck-like forms \cite{Fokker_1914, Planck_1917} wherein pseudo-particles in the numerical model obey the neoclassical transport equations, with particle-independent Brownian drift terms. This offers a rigorous methodology for incorporating collisions into the particle transport model, without coupling the equations of motions for each particle.
        
        Works by Chen, Chacón et al. \cite{Chen_Chacón_Barnes_2011, Chacón_Chen_Barnes_2013, Chen_Chacón_2014, Chen_Chacón_2015} have developed structure-preserving particle pushers for neoclassical transport in the Vlasov equations, derived from Crank--Nicolson integrators. We show these too can can derive from a FET interpretation, similarly offering potential extensions to higher-order-in-time particle pushers. The FET formulation is used also to consider how the stochastic drift terms can be incorporated into the pushers. Stochastic gyrokinetic expansions are also discussed.

        Different options for the numerical implementation of these schemes are considered.

        Due to the efficacy of FET in the development of SP timesteppers for both the fluid and kinetic component, we hope this approach will prove effective in the future for developing SP timesteppers for the full hybrid model. We hope this will give us the opportunity to incorporate previously inaccessible kinetic effects into the highly effective, modern, finite-element MHD models.
    \end{abstract}
    
    
    \newpage
    \tableofcontents
    
    
    \newpage
    \pagenumbering{arabic}
    %\linenumbers\renewcommand\thelinenumber{\color{black!50}\arabic{linenumber}}
            \documentclass[12pt, a4paper]{report}

\input{template/main.tex}

\title{\BA{Title in Progress...}}
\author{Boris Andrews}
\affil{Mathematical Institute, University of Oxford}
\date{\today}


\begin{document}
    \pagenumbering{gobble}
    \maketitle
    
    
    \begin{abstract}
        Magnetic confinement reactors---in particular tokamaks---offer one of the most promising options for achieving practical nuclear fusion, with the potential to provide virtually limitless, clean energy. The theoretical and numerical modeling of tokamak plasmas is simultaneously an essential component of effective reactor design, and a great research barrier. Tokamak operational conditions exhibit comparatively low Knudsen numbers. Kinetic effects, including kinetic waves and instabilities, Landau damping, bump-on-tail instabilities and more, are therefore highly influential in tokamak plasma dynamics. Purely fluid models are inherently incapable of capturing these effects, whereas the high dimensionality in purely kinetic models render them practically intractable for most relevant purposes.

        We consider a $\delta\!f$ decomposition model, with a macroscopic fluid background and microscopic kinetic correction, both fully coupled to each other. A similar manner of discretization is proposed to that used in the recent \texttt{STRUPHY} code \cite{Holderied_Possanner_Wang_2021, Holderied_2022, Li_et_al_2023} with a finite-element model for the background and a pseudo-particle/PiC model for the correction.

        The fluid background satisfies the full, non-linear, resistive, compressible, Hall MHD equations. \cite{Laakmann_Hu_Farrell_2022} introduces finite-element(-in-space) implicit timesteppers for the incompressible analogue to this system with structure-preserving (SP) properties in the ideal case, alongside parameter-robust preconditioners. We show that these timesteppers can derive from a finite-element-in-time (FET) (and finite-element-in-space) interpretation. The benefits of this reformulation are discussed, including the derivation of timesteppers that are higher order in time, and the quantifiable dissipative SP properties in the non-ideal, resistive case.
        
        We discuss possible options for extending this FET approach to timesteppers for the compressible case.

        The kinetic corrections satisfy linearized Boltzmann equations. Using a Lénard--Bernstein collision operator, these take Fokker--Planck-like forms \cite{Fokker_1914, Planck_1917} wherein pseudo-particles in the numerical model obey the neoclassical transport equations, with particle-independent Brownian drift terms. This offers a rigorous methodology for incorporating collisions into the particle transport model, without coupling the equations of motions for each particle.
        
        Works by Chen, Chacón et al. \cite{Chen_Chacón_Barnes_2011, Chacón_Chen_Barnes_2013, Chen_Chacón_2014, Chen_Chacón_2015} have developed structure-preserving particle pushers for neoclassical transport in the Vlasov equations, derived from Crank--Nicolson integrators. We show these too can can derive from a FET interpretation, similarly offering potential extensions to higher-order-in-time particle pushers. The FET formulation is used also to consider how the stochastic drift terms can be incorporated into the pushers. Stochastic gyrokinetic expansions are also discussed.

        Different options for the numerical implementation of these schemes are considered.

        Due to the efficacy of FET in the development of SP timesteppers for both the fluid and kinetic component, we hope this approach will prove effective in the future for developing SP timesteppers for the full hybrid model. We hope this will give us the opportunity to incorporate previously inaccessible kinetic effects into the highly effective, modern, finite-element MHD models.
    \end{abstract}
    
    
    \newpage
    \tableofcontents
    
    
    \newpage
    \pagenumbering{arabic}
    %\linenumbers\renewcommand\thelinenumber{\color{black!50}\arabic{linenumber}}
            \input{0 - introduction/main.tex}
        \part{Research}
            \input{1 - low-noise PiC models/main.tex}
            \input{2 - kinetic component/main.tex}
            \input{3 - fluid component/main.tex}
            \input{4 - numerical implementation/main.tex}
        \part{Project Overview}
            \input{5 - research plan/main.tex}
            \input{6 - summary/main.tex}
    
    
    %\section{}
    \newpage
    \pagenumbering{gobble}
        \printbibliography


    \newpage
    \pagenumbering{roman}
    \appendix
        \part{Appendices}
            \input{8 - Hilbert complexes/main.tex}
            \input{9 - weak conservation proofs/main.tex}
\end{document}

        \part{Research}
            \documentclass[12pt, a4paper]{report}

\input{template/main.tex}

\title{\BA{Title in Progress...}}
\author{Boris Andrews}
\affil{Mathematical Institute, University of Oxford}
\date{\today}


\begin{document}
    \pagenumbering{gobble}
    \maketitle
    
    
    \begin{abstract}
        Magnetic confinement reactors---in particular tokamaks---offer one of the most promising options for achieving practical nuclear fusion, with the potential to provide virtually limitless, clean energy. The theoretical and numerical modeling of tokamak plasmas is simultaneously an essential component of effective reactor design, and a great research barrier. Tokamak operational conditions exhibit comparatively low Knudsen numbers. Kinetic effects, including kinetic waves and instabilities, Landau damping, bump-on-tail instabilities and more, are therefore highly influential in tokamak plasma dynamics. Purely fluid models are inherently incapable of capturing these effects, whereas the high dimensionality in purely kinetic models render them practically intractable for most relevant purposes.

        We consider a $\delta\!f$ decomposition model, with a macroscopic fluid background and microscopic kinetic correction, both fully coupled to each other. A similar manner of discretization is proposed to that used in the recent \texttt{STRUPHY} code \cite{Holderied_Possanner_Wang_2021, Holderied_2022, Li_et_al_2023} with a finite-element model for the background and a pseudo-particle/PiC model for the correction.

        The fluid background satisfies the full, non-linear, resistive, compressible, Hall MHD equations. \cite{Laakmann_Hu_Farrell_2022} introduces finite-element(-in-space) implicit timesteppers for the incompressible analogue to this system with structure-preserving (SP) properties in the ideal case, alongside parameter-robust preconditioners. We show that these timesteppers can derive from a finite-element-in-time (FET) (and finite-element-in-space) interpretation. The benefits of this reformulation are discussed, including the derivation of timesteppers that are higher order in time, and the quantifiable dissipative SP properties in the non-ideal, resistive case.
        
        We discuss possible options for extending this FET approach to timesteppers for the compressible case.

        The kinetic corrections satisfy linearized Boltzmann equations. Using a Lénard--Bernstein collision operator, these take Fokker--Planck-like forms \cite{Fokker_1914, Planck_1917} wherein pseudo-particles in the numerical model obey the neoclassical transport equations, with particle-independent Brownian drift terms. This offers a rigorous methodology for incorporating collisions into the particle transport model, without coupling the equations of motions for each particle.
        
        Works by Chen, Chacón et al. \cite{Chen_Chacón_Barnes_2011, Chacón_Chen_Barnes_2013, Chen_Chacón_2014, Chen_Chacón_2015} have developed structure-preserving particle pushers for neoclassical transport in the Vlasov equations, derived from Crank--Nicolson integrators. We show these too can can derive from a FET interpretation, similarly offering potential extensions to higher-order-in-time particle pushers. The FET formulation is used also to consider how the stochastic drift terms can be incorporated into the pushers. Stochastic gyrokinetic expansions are also discussed.

        Different options for the numerical implementation of these schemes are considered.

        Due to the efficacy of FET in the development of SP timesteppers for both the fluid and kinetic component, we hope this approach will prove effective in the future for developing SP timesteppers for the full hybrid model. We hope this will give us the opportunity to incorporate previously inaccessible kinetic effects into the highly effective, modern, finite-element MHD models.
    \end{abstract}
    
    
    \newpage
    \tableofcontents
    
    
    \newpage
    \pagenumbering{arabic}
    %\linenumbers\renewcommand\thelinenumber{\color{black!50}\arabic{linenumber}}
            \input{0 - introduction/main.tex}
        \part{Research}
            \input{1 - low-noise PiC models/main.tex}
            \input{2 - kinetic component/main.tex}
            \input{3 - fluid component/main.tex}
            \input{4 - numerical implementation/main.tex}
        \part{Project Overview}
            \input{5 - research plan/main.tex}
            \input{6 - summary/main.tex}
    
    
    %\section{}
    \newpage
    \pagenumbering{gobble}
        \printbibliography


    \newpage
    \pagenumbering{roman}
    \appendix
        \part{Appendices}
            \input{8 - Hilbert complexes/main.tex}
            \input{9 - weak conservation proofs/main.tex}
\end{document}

            \documentclass[12pt, a4paper]{report}

\input{template/main.tex}

\title{\BA{Title in Progress...}}
\author{Boris Andrews}
\affil{Mathematical Institute, University of Oxford}
\date{\today}


\begin{document}
    \pagenumbering{gobble}
    \maketitle
    
    
    \begin{abstract}
        Magnetic confinement reactors---in particular tokamaks---offer one of the most promising options for achieving practical nuclear fusion, with the potential to provide virtually limitless, clean energy. The theoretical and numerical modeling of tokamak plasmas is simultaneously an essential component of effective reactor design, and a great research barrier. Tokamak operational conditions exhibit comparatively low Knudsen numbers. Kinetic effects, including kinetic waves and instabilities, Landau damping, bump-on-tail instabilities and more, are therefore highly influential in tokamak plasma dynamics. Purely fluid models are inherently incapable of capturing these effects, whereas the high dimensionality in purely kinetic models render them practically intractable for most relevant purposes.

        We consider a $\delta\!f$ decomposition model, with a macroscopic fluid background and microscopic kinetic correction, both fully coupled to each other. A similar manner of discretization is proposed to that used in the recent \texttt{STRUPHY} code \cite{Holderied_Possanner_Wang_2021, Holderied_2022, Li_et_al_2023} with a finite-element model for the background and a pseudo-particle/PiC model for the correction.

        The fluid background satisfies the full, non-linear, resistive, compressible, Hall MHD equations. \cite{Laakmann_Hu_Farrell_2022} introduces finite-element(-in-space) implicit timesteppers for the incompressible analogue to this system with structure-preserving (SP) properties in the ideal case, alongside parameter-robust preconditioners. We show that these timesteppers can derive from a finite-element-in-time (FET) (and finite-element-in-space) interpretation. The benefits of this reformulation are discussed, including the derivation of timesteppers that are higher order in time, and the quantifiable dissipative SP properties in the non-ideal, resistive case.
        
        We discuss possible options for extending this FET approach to timesteppers for the compressible case.

        The kinetic corrections satisfy linearized Boltzmann equations. Using a Lénard--Bernstein collision operator, these take Fokker--Planck-like forms \cite{Fokker_1914, Planck_1917} wherein pseudo-particles in the numerical model obey the neoclassical transport equations, with particle-independent Brownian drift terms. This offers a rigorous methodology for incorporating collisions into the particle transport model, without coupling the equations of motions for each particle.
        
        Works by Chen, Chacón et al. \cite{Chen_Chacón_Barnes_2011, Chacón_Chen_Barnes_2013, Chen_Chacón_2014, Chen_Chacón_2015} have developed structure-preserving particle pushers for neoclassical transport in the Vlasov equations, derived from Crank--Nicolson integrators. We show these too can can derive from a FET interpretation, similarly offering potential extensions to higher-order-in-time particle pushers. The FET formulation is used also to consider how the stochastic drift terms can be incorporated into the pushers. Stochastic gyrokinetic expansions are also discussed.

        Different options for the numerical implementation of these schemes are considered.

        Due to the efficacy of FET in the development of SP timesteppers for both the fluid and kinetic component, we hope this approach will prove effective in the future for developing SP timesteppers for the full hybrid model. We hope this will give us the opportunity to incorporate previously inaccessible kinetic effects into the highly effective, modern, finite-element MHD models.
    \end{abstract}
    
    
    \newpage
    \tableofcontents
    
    
    \newpage
    \pagenumbering{arabic}
    %\linenumbers\renewcommand\thelinenumber{\color{black!50}\arabic{linenumber}}
            \input{0 - introduction/main.tex}
        \part{Research}
            \input{1 - low-noise PiC models/main.tex}
            \input{2 - kinetic component/main.tex}
            \input{3 - fluid component/main.tex}
            \input{4 - numerical implementation/main.tex}
        \part{Project Overview}
            \input{5 - research plan/main.tex}
            \input{6 - summary/main.tex}
    
    
    %\section{}
    \newpage
    \pagenumbering{gobble}
        \printbibliography


    \newpage
    \pagenumbering{roman}
    \appendix
        \part{Appendices}
            \input{8 - Hilbert complexes/main.tex}
            \input{9 - weak conservation proofs/main.tex}
\end{document}

            \documentclass[12pt, a4paper]{report}

\input{template/main.tex}

\title{\BA{Title in Progress...}}
\author{Boris Andrews}
\affil{Mathematical Institute, University of Oxford}
\date{\today}


\begin{document}
    \pagenumbering{gobble}
    \maketitle
    
    
    \begin{abstract}
        Magnetic confinement reactors---in particular tokamaks---offer one of the most promising options for achieving practical nuclear fusion, with the potential to provide virtually limitless, clean energy. The theoretical and numerical modeling of tokamak plasmas is simultaneously an essential component of effective reactor design, and a great research barrier. Tokamak operational conditions exhibit comparatively low Knudsen numbers. Kinetic effects, including kinetic waves and instabilities, Landau damping, bump-on-tail instabilities and more, are therefore highly influential in tokamak plasma dynamics. Purely fluid models are inherently incapable of capturing these effects, whereas the high dimensionality in purely kinetic models render them practically intractable for most relevant purposes.

        We consider a $\delta\!f$ decomposition model, with a macroscopic fluid background and microscopic kinetic correction, both fully coupled to each other. A similar manner of discretization is proposed to that used in the recent \texttt{STRUPHY} code \cite{Holderied_Possanner_Wang_2021, Holderied_2022, Li_et_al_2023} with a finite-element model for the background and a pseudo-particle/PiC model for the correction.

        The fluid background satisfies the full, non-linear, resistive, compressible, Hall MHD equations. \cite{Laakmann_Hu_Farrell_2022} introduces finite-element(-in-space) implicit timesteppers for the incompressible analogue to this system with structure-preserving (SP) properties in the ideal case, alongside parameter-robust preconditioners. We show that these timesteppers can derive from a finite-element-in-time (FET) (and finite-element-in-space) interpretation. The benefits of this reformulation are discussed, including the derivation of timesteppers that are higher order in time, and the quantifiable dissipative SP properties in the non-ideal, resistive case.
        
        We discuss possible options for extending this FET approach to timesteppers for the compressible case.

        The kinetic corrections satisfy linearized Boltzmann equations. Using a Lénard--Bernstein collision operator, these take Fokker--Planck-like forms \cite{Fokker_1914, Planck_1917} wherein pseudo-particles in the numerical model obey the neoclassical transport equations, with particle-independent Brownian drift terms. This offers a rigorous methodology for incorporating collisions into the particle transport model, without coupling the equations of motions for each particle.
        
        Works by Chen, Chacón et al. \cite{Chen_Chacón_Barnes_2011, Chacón_Chen_Barnes_2013, Chen_Chacón_2014, Chen_Chacón_2015} have developed structure-preserving particle pushers for neoclassical transport in the Vlasov equations, derived from Crank--Nicolson integrators. We show these too can can derive from a FET interpretation, similarly offering potential extensions to higher-order-in-time particle pushers. The FET formulation is used also to consider how the stochastic drift terms can be incorporated into the pushers. Stochastic gyrokinetic expansions are also discussed.

        Different options for the numerical implementation of these schemes are considered.

        Due to the efficacy of FET in the development of SP timesteppers for both the fluid and kinetic component, we hope this approach will prove effective in the future for developing SP timesteppers for the full hybrid model. We hope this will give us the opportunity to incorporate previously inaccessible kinetic effects into the highly effective, modern, finite-element MHD models.
    \end{abstract}
    
    
    \newpage
    \tableofcontents
    
    
    \newpage
    \pagenumbering{arabic}
    %\linenumbers\renewcommand\thelinenumber{\color{black!50}\arabic{linenumber}}
            \input{0 - introduction/main.tex}
        \part{Research}
            \input{1 - low-noise PiC models/main.tex}
            \input{2 - kinetic component/main.tex}
            \input{3 - fluid component/main.tex}
            \input{4 - numerical implementation/main.tex}
        \part{Project Overview}
            \input{5 - research plan/main.tex}
            \input{6 - summary/main.tex}
    
    
    %\section{}
    \newpage
    \pagenumbering{gobble}
        \printbibliography


    \newpage
    \pagenumbering{roman}
    \appendix
        \part{Appendices}
            \input{8 - Hilbert complexes/main.tex}
            \input{9 - weak conservation proofs/main.tex}
\end{document}

            \documentclass[12pt, a4paper]{report}

\input{template/main.tex}

\title{\BA{Title in Progress...}}
\author{Boris Andrews}
\affil{Mathematical Institute, University of Oxford}
\date{\today}


\begin{document}
    \pagenumbering{gobble}
    \maketitle
    
    
    \begin{abstract}
        Magnetic confinement reactors---in particular tokamaks---offer one of the most promising options for achieving practical nuclear fusion, with the potential to provide virtually limitless, clean energy. The theoretical and numerical modeling of tokamak plasmas is simultaneously an essential component of effective reactor design, and a great research barrier. Tokamak operational conditions exhibit comparatively low Knudsen numbers. Kinetic effects, including kinetic waves and instabilities, Landau damping, bump-on-tail instabilities and more, are therefore highly influential in tokamak plasma dynamics. Purely fluid models are inherently incapable of capturing these effects, whereas the high dimensionality in purely kinetic models render them practically intractable for most relevant purposes.

        We consider a $\delta\!f$ decomposition model, with a macroscopic fluid background and microscopic kinetic correction, both fully coupled to each other. A similar manner of discretization is proposed to that used in the recent \texttt{STRUPHY} code \cite{Holderied_Possanner_Wang_2021, Holderied_2022, Li_et_al_2023} with a finite-element model for the background and a pseudo-particle/PiC model for the correction.

        The fluid background satisfies the full, non-linear, resistive, compressible, Hall MHD equations. \cite{Laakmann_Hu_Farrell_2022} introduces finite-element(-in-space) implicit timesteppers for the incompressible analogue to this system with structure-preserving (SP) properties in the ideal case, alongside parameter-robust preconditioners. We show that these timesteppers can derive from a finite-element-in-time (FET) (and finite-element-in-space) interpretation. The benefits of this reformulation are discussed, including the derivation of timesteppers that are higher order in time, and the quantifiable dissipative SP properties in the non-ideal, resistive case.
        
        We discuss possible options for extending this FET approach to timesteppers for the compressible case.

        The kinetic corrections satisfy linearized Boltzmann equations. Using a Lénard--Bernstein collision operator, these take Fokker--Planck-like forms \cite{Fokker_1914, Planck_1917} wherein pseudo-particles in the numerical model obey the neoclassical transport equations, with particle-independent Brownian drift terms. This offers a rigorous methodology for incorporating collisions into the particle transport model, without coupling the equations of motions for each particle.
        
        Works by Chen, Chacón et al. \cite{Chen_Chacón_Barnes_2011, Chacón_Chen_Barnes_2013, Chen_Chacón_2014, Chen_Chacón_2015} have developed structure-preserving particle pushers for neoclassical transport in the Vlasov equations, derived from Crank--Nicolson integrators. We show these too can can derive from a FET interpretation, similarly offering potential extensions to higher-order-in-time particle pushers. The FET formulation is used also to consider how the stochastic drift terms can be incorporated into the pushers. Stochastic gyrokinetic expansions are also discussed.

        Different options for the numerical implementation of these schemes are considered.

        Due to the efficacy of FET in the development of SP timesteppers for both the fluid and kinetic component, we hope this approach will prove effective in the future for developing SP timesteppers for the full hybrid model. We hope this will give us the opportunity to incorporate previously inaccessible kinetic effects into the highly effective, modern, finite-element MHD models.
    \end{abstract}
    
    
    \newpage
    \tableofcontents
    
    
    \newpage
    \pagenumbering{arabic}
    %\linenumbers\renewcommand\thelinenumber{\color{black!50}\arabic{linenumber}}
            \input{0 - introduction/main.tex}
        \part{Research}
            \input{1 - low-noise PiC models/main.tex}
            \input{2 - kinetic component/main.tex}
            \input{3 - fluid component/main.tex}
            \input{4 - numerical implementation/main.tex}
        \part{Project Overview}
            \input{5 - research plan/main.tex}
            \input{6 - summary/main.tex}
    
    
    %\section{}
    \newpage
    \pagenumbering{gobble}
        \printbibliography


    \newpage
    \pagenumbering{roman}
    \appendix
        \part{Appendices}
            \input{8 - Hilbert complexes/main.tex}
            \input{9 - weak conservation proofs/main.tex}
\end{document}

        \part{Project Overview}
            \documentclass[12pt, a4paper]{report}

\input{template/main.tex}

\title{\BA{Title in Progress...}}
\author{Boris Andrews}
\affil{Mathematical Institute, University of Oxford}
\date{\today}


\begin{document}
    \pagenumbering{gobble}
    \maketitle
    
    
    \begin{abstract}
        Magnetic confinement reactors---in particular tokamaks---offer one of the most promising options for achieving practical nuclear fusion, with the potential to provide virtually limitless, clean energy. The theoretical and numerical modeling of tokamak plasmas is simultaneously an essential component of effective reactor design, and a great research barrier. Tokamak operational conditions exhibit comparatively low Knudsen numbers. Kinetic effects, including kinetic waves and instabilities, Landau damping, bump-on-tail instabilities and more, are therefore highly influential in tokamak plasma dynamics. Purely fluid models are inherently incapable of capturing these effects, whereas the high dimensionality in purely kinetic models render them practically intractable for most relevant purposes.

        We consider a $\delta\!f$ decomposition model, with a macroscopic fluid background and microscopic kinetic correction, both fully coupled to each other. A similar manner of discretization is proposed to that used in the recent \texttt{STRUPHY} code \cite{Holderied_Possanner_Wang_2021, Holderied_2022, Li_et_al_2023} with a finite-element model for the background and a pseudo-particle/PiC model for the correction.

        The fluid background satisfies the full, non-linear, resistive, compressible, Hall MHD equations. \cite{Laakmann_Hu_Farrell_2022} introduces finite-element(-in-space) implicit timesteppers for the incompressible analogue to this system with structure-preserving (SP) properties in the ideal case, alongside parameter-robust preconditioners. We show that these timesteppers can derive from a finite-element-in-time (FET) (and finite-element-in-space) interpretation. The benefits of this reformulation are discussed, including the derivation of timesteppers that are higher order in time, and the quantifiable dissipative SP properties in the non-ideal, resistive case.
        
        We discuss possible options for extending this FET approach to timesteppers for the compressible case.

        The kinetic corrections satisfy linearized Boltzmann equations. Using a Lénard--Bernstein collision operator, these take Fokker--Planck-like forms \cite{Fokker_1914, Planck_1917} wherein pseudo-particles in the numerical model obey the neoclassical transport equations, with particle-independent Brownian drift terms. This offers a rigorous methodology for incorporating collisions into the particle transport model, without coupling the equations of motions for each particle.
        
        Works by Chen, Chacón et al. \cite{Chen_Chacón_Barnes_2011, Chacón_Chen_Barnes_2013, Chen_Chacón_2014, Chen_Chacón_2015} have developed structure-preserving particle pushers for neoclassical transport in the Vlasov equations, derived from Crank--Nicolson integrators. We show these too can can derive from a FET interpretation, similarly offering potential extensions to higher-order-in-time particle pushers. The FET formulation is used also to consider how the stochastic drift terms can be incorporated into the pushers. Stochastic gyrokinetic expansions are also discussed.

        Different options for the numerical implementation of these schemes are considered.

        Due to the efficacy of FET in the development of SP timesteppers for both the fluid and kinetic component, we hope this approach will prove effective in the future for developing SP timesteppers for the full hybrid model. We hope this will give us the opportunity to incorporate previously inaccessible kinetic effects into the highly effective, modern, finite-element MHD models.
    \end{abstract}
    
    
    \newpage
    \tableofcontents
    
    
    \newpage
    \pagenumbering{arabic}
    %\linenumbers\renewcommand\thelinenumber{\color{black!50}\arabic{linenumber}}
            \input{0 - introduction/main.tex}
        \part{Research}
            \input{1 - low-noise PiC models/main.tex}
            \input{2 - kinetic component/main.tex}
            \input{3 - fluid component/main.tex}
            \input{4 - numerical implementation/main.tex}
        \part{Project Overview}
            \input{5 - research plan/main.tex}
            \input{6 - summary/main.tex}
    
    
    %\section{}
    \newpage
    \pagenumbering{gobble}
        \printbibliography


    \newpage
    \pagenumbering{roman}
    \appendix
        \part{Appendices}
            \input{8 - Hilbert complexes/main.tex}
            \input{9 - weak conservation proofs/main.tex}
\end{document}

            \documentclass[12pt, a4paper]{report}

\input{template/main.tex}

\title{\BA{Title in Progress...}}
\author{Boris Andrews}
\affil{Mathematical Institute, University of Oxford}
\date{\today}


\begin{document}
    \pagenumbering{gobble}
    \maketitle
    
    
    \begin{abstract}
        Magnetic confinement reactors---in particular tokamaks---offer one of the most promising options for achieving practical nuclear fusion, with the potential to provide virtually limitless, clean energy. The theoretical and numerical modeling of tokamak plasmas is simultaneously an essential component of effective reactor design, and a great research barrier. Tokamak operational conditions exhibit comparatively low Knudsen numbers. Kinetic effects, including kinetic waves and instabilities, Landau damping, bump-on-tail instabilities and more, are therefore highly influential in tokamak plasma dynamics. Purely fluid models are inherently incapable of capturing these effects, whereas the high dimensionality in purely kinetic models render them practically intractable for most relevant purposes.

        We consider a $\delta\!f$ decomposition model, with a macroscopic fluid background and microscopic kinetic correction, both fully coupled to each other. A similar manner of discretization is proposed to that used in the recent \texttt{STRUPHY} code \cite{Holderied_Possanner_Wang_2021, Holderied_2022, Li_et_al_2023} with a finite-element model for the background and a pseudo-particle/PiC model for the correction.

        The fluid background satisfies the full, non-linear, resistive, compressible, Hall MHD equations. \cite{Laakmann_Hu_Farrell_2022} introduces finite-element(-in-space) implicit timesteppers for the incompressible analogue to this system with structure-preserving (SP) properties in the ideal case, alongside parameter-robust preconditioners. We show that these timesteppers can derive from a finite-element-in-time (FET) (and finite-element-in-space) interpretation. The benefits of this reformulation are discussed, including the derivation of timesteppers that are higher order in time, and the quantifiable dissipative SP properties in the non-ideal, resistive case.
        
        We discuss possible options for extending this FET approach to timesteppers for the compressible case.

        The kinetic corrections satisfy linearized Boltzmann equations. Using a Lénard--Bernstein collision operator, these take Fokker--Planck-like forms \cite{Fokker_1914, Planck_1917} wherein pseudo-particles in the numerical model obey the neoclassical transport equations, with particle-independent Brownian drift terms. This offers a rigorous methodology for incorporating collisions into the particle transport model, without coupling the equations of motions for each particle.
        
        Works by Chen, Chacón et al. \cite{Chen_Chacón_Barnes_2011, Chacón_Chen_Barnes_2013, Chen_Chacón_2014, Chen_Chacón_2015} have developed structure-preserving particle pushers for neoclassical transport in the Vlasov equations, derived from Crank--Nicolson integrators. We show these too can can derive from a FET interpretation, similarly offering potential extensions to higher-order-in-time particle pushers. The FET formulation is used also to consider how the stochastic drift terms can be incorporated into the pushers. Stochastic gyrokinetic expansions are also discussed.

        Different options for the numerical implementation of these schemes are considered.

        Due to the efficacy of FET in the development of SP timesteppers for both the fluid and kinetic component, we hope this approach will prove effective in the future for developing SP timesteppers for the full hybrid model. We hope this will give us the opportunity to incorporate previously inaccessible kinetic effects into the highly effective, modern, finite-element MHD models.
    \end{abstract}
    
    
    \newpage
    \tableofcontents
    
    
    \newpage
    \pagenumbering{arabic}
    %\linenumbers\renewcommand\thelinenumber{\color{black!50}\arabic{linenumber}}
            \input{0 - introduction/main.tex}
        \part{Research}
            \input{1 - low-noise PiC models/main.tex}
            \input{2 - kinetic component/main.tex}
            \input{3 - fluid component/main.tex}
            \input{4 - numerical implementation/main.tex}
        \part{Project Overview}
            \input{5 - research plan/main.tex}
            \input{6 - summary/main.tex}
    
    
    %\section{}
    \newpage
    \pagenumbering{gobble}
        \printbibliography


    \newpage
    \pagenumbering{roman}
    \appendix
        \part{Appendices}
            \input{8 - Hilbert complexes/main.tex}
            \input{9 - weak conservation proofs/main.tex}
\end{document}

    
    
    %\section{}
    \newpage
    \pagenumbering{gobble}
        \printbibliography


    \newpage
    \pagenumbering{roman}
    \appendix
        \part{Appendices}
            \documentclass[12pt, a4paper]{report}

\input{template/main.tex}

\title{\BA{Title in Progress...}}
\author{Boris Andrews}
\affil{Mathematical Institute, University of Oxford}
\date{\today}


\begin{document}
    \pagenumbering{gobble}
    \maketitle
    
    
    \begin{abstract}
        Magnetic confinement reactors---in particular tokamaks---offer one of the most promising options for achieving practical nuclear fusion, with the potential to provide virtually limitless, clean energy. The theoretical and numerical modeling of tokamak plasmas is simultaneously an essential component of effective reactor design, and a great research barrier. Tokamak operational conditions exhibit comparatively low Knudsen numbers. Kinetic effects, including kinetic waves and instabilities, Landau damping, bump-on-tail instabilities and more, are therefore highly influential in tokamak plasma dynamics. Purely fluid models are inherently incapable of capturing these effects, whereas the high dimensionality in purely kinetic models render them practically intractable for most relevant purposes.

        We consider a $\delta\!f$ decomposition model, with a macroscopic fluid background and microscopic kinetic correction, both fully coupled to each other. A similar manner of discretization is proposed to that used in the recent \texttt{STRUPHY} code \cite{Holderied_Possanner_Wang_2021, Holderied_2022, Li_et_al_2023} with a finite-element model for the background and a pseudo-particle/PiC model for the correction.

        The fluid background satisfies the full, non-linear, resistive, compressible, Hall MHD equations. \cite{Laakmann_Hu_Farrell_2022} introduces finite-element(-in-space) implicit timesteppers for the incompressible analogue to this system with structure-preserving (SP) properties in the ideal case, alongside parameter-robust preconditioners. We show that these timesteppers can derive from a finite-element-in-time (FET) (and finite-element-in-space) interpretation. The benefits of this reformulation are discussed, including the derivation of timesteppers that are higher order in time, and the quantifiable dissipative SP properties in the non-ideal, resistive case.
        
        We discuss possible options for extending this FET approach to timesteppers for the compressible case.

        The kinetic corrections satisfy linearized Boltzmann equations. Using a Lénard--Bernstein collision operator, these take Fokker--Planck-like forms \cite{Fokker_1914, Planck_1917} wherein pseudo-particles in the numerical model obey the neoclassical transport equations, with particle-independent Brownian drift terms. This offers a rigorous methodology for incorporating collisions into the particle transport model, without coupling the equations of motions for each particle.
        
        Works by Chen, Chacón et al. \cite{Chen_Chacón_Barnes_2011, Chacón_Chen_Barnes_2013, Chen_Chacón_2014, Chen_Chacón_2015} have developed structure-preserving particle pushers for neoclassical transport in the Vlasov equations, derived from Crank--Nicolson integrators. We show these too can can derive from a FET interpretation, similarly offering potential extensions to higher-order-in-time particle pushers. The FET formulation is used also to consider how the stochastic drift terms can be incorporated into the pushers. Stochastic gyrokinetic expansions are also discussed.

        Different options for the numerical implementation of these schemes are considered.

        Due to the efficacy of FET in the development of SP timesteppers for both the fluid and kinetic component, we hope this approach will prove effective in the future for developing SP timesteppers for the full hybrid model. We hope this will give us the opportunity to incorporate previously inaccessible kinetic effects into the highly effective, modern, finite-element MHD models.
    \end{abstract}
    
    
    \newpage
    \tableofcontents
    
    
    \newpage
    \pagenumbering{arabic}
    %\linenumbers\renewcommand\thelinenumber{\color{black!50}\arabic{linenumber}}
            \input{0 - introduction/main.tex}
        \part{Research}
            \input{1 - low-noise PiC models/main.tex}
            \input{2 - kinetic component/main.tex}
            \input{3 - fluid component/main.tex}
            \input{4 - numerical implementation/main.tex}
        \part{Project Overview}
            \input{5 - research plan/main.tex}
            \input{6 - summary/main.tex}
    
    
    %\section{}
    \newpage
    \pagenumbering{gobble}
        \printbibliography


    \newpage
    \pagenumbering{roman}
    \appendix
        \part{Appendices}
            \input{8 - Hilbert complexes/main.tex}
            \input{9 - weak conservation proofs/main.tex}
\end{document}

            \documentclass[12pt, a4paper]{report}

\input{template/main.tex}

\title{\BA{Title in Progress...}}
\author{Boris Andrews}
\affil{Mathematical Institute, University of Oxford}
\date{\today}


\begin{document}
    \pagenumbering{gobble}
    \maketitle
    
    
    \begin{abstract}
        Magnetic confinement reactors---in particular tokamaks---offer one of the most promising options for achieving practical nuclear fusion, with the potential to provide virtually limitless, clean energy. The theoretical and numerical modeling of tokamak plasmas is simultaneously an essential component of effective reactor design, and a great research barrier. Tokamak operational conditions exhibit comparatively low Knudsen numbers. Kinetic effects, including kinetic waves and instabilities, Landau damping, bump-on-tail instabilities and more, are therefore highly influential in tokamak plasma dynamics. Purely fluid models are inherently incapable of capturing these effects, whereas the high dimensionality in purely kinetic models render them practically intractable for most relevant purposes.

        We consider a $\delta\!f$ decomposition model, with a macroscopic fluid background and microscopic kinetic correction, both fully coupled to each other. A similar manner of discretization is proposed to that used in the recent \texttt{STRUPHY} code \cite{Holderied_Possanner_Wang_2021, Holderied_2022, Li_et_al_2023} with a finite-element model for the background and a pseudo-particle/PiC model for the correction.

        The fluid background satisfies the full, non-linear, resistive, compressible, Hall MHD equations. \cite{Laakmann_Hu_Farrell_2022} introduces finite-element(-in-space) implicit timesteppers for the incompressible analogue to this system with structure-preserving (SP) properties in the ideal case, alongside parameter-robust preconditioners. We show that these timesteppers can derive from a finite-element-in-time (FET) (and finite-element-in-space) interpretation. The benefits of this reformulation are discussed, including the derivation of timesteppers that are higher order in time, and the quantifiable dissipative SP properties in the non-ideal, resistive case.
        
        We discuss possible options for extending this FET approach to timesteppers for the compressible case.

        The kinetic corrections satisfy linearized Boltzmann equations. Using a Lénard--Bernstein collision operator, these take Fokker--Planck-like forms \cite{Fokker_1914, Planck_1917} wherein pseudo-particles in the numerical model obey the neoclassical transport equations, with particle-independent Brownian drift terms. This offers a rigorous methodology for incorporating collisions into the particle transport model, without coupling the equations of motions for each particle.
        
        Works by Chen, Chacón et al. \cite{Chen_Chacón_Barnes_2011, Chacón_Chen_Barnes_2013, Chen_Chacón_2014, Chen_Chacón_2015} have developed structure-preserving particle pushers for neoclassical transport in the Vlasov equations, derived from Crank--Nicolson integrators. We show these too can can derive from a FET interpretation, similarly offering potential extensions to higher-order-in-time particle pushers. The FET formulation is used also to consider how the stochastic drift terms can be incorporated into the pushers. Stochastic gyrokinetic expansions are also discussed.

        Different options for the numerical implementation of these schemes are considered.

        Due to the efficacy of FET in the development of SP timesteppers for both the fluid and kinetic component, we hope this approach will prove effective in the future for developing SP timesteppers for the full hybrid model. We hope this will give us the opportunity to incorporate previously inaccessible kinetic effects into the highly effective, modern, finite-element MHD models.
    \end{abstract}
    
    
    \newpage
    \tableofcontents
    
    
    \newpage
    \pagenumbering{arabic}
    %\linenumbers\renewcommand\thelinenumber{\color{black!50}\arabic{linenumber}}
            \input{0 - introduction/main.tex}
        \part{Research}
            \input{1 - low-noise PiC models/main.tex}
            \input{2 - kinetic component/main.tex}
            \input{3 - fluid component/main.tex}
            \input{4 - numerical implementation/main.tex}
        \part{Project Overview}
            \input{5 - research plan/main.tex}
            \input{6 - summary/main.tex}
    
    
    %\section{}
    \newpage
    \pagenumbering{gobble}
        \printbibliography


    \newpage
    \pagenumbering{roman}
    \appendix
        \part{Appendices}
            \input{8 - Hilbert complexes/main.tex}
            \input{9 - weak conservation proofs/main.tex}
\end{document}

\end{document}

            \documentclass[12pt, a4paper]{report}

\documentclass[12pt, a4paper]{report}

\input{template/main.tex}

\title{\BA{Title in Progress...}}
\author{Boris Andrews}
\affil{Mathematical Institute, University of Oxford}
\date{\today}


\begin{document}
    \pagenumbering{gobble}
    \maketitle
    
    
    \begin{abstract}
        Magnetic confinement reactors---in particular tokamaks---offer one of the most promising options for achieving practical nuclear fusion, with the potential to provide virtually limitless, clean energy. The theoretical and numerical modeling of tokamak plasmas is simultaneously an essential component of effective reactor design, and a great research barrier. Tokamak operational conditions exhibit comparatively low Knudsen numbers. Kinetic effects, including kinetic waves and instabilities, Landau damping, bump-on-tail instabilities and more, are therefore highly influential in tokamak plasma dynamics. Purely fluid models are inherently incapable of capturing these effects, whereas the high dimensionality in purely kinetic models render them practically intractable for most relevant purposes.

        We consider a $\delta\!f$ decomposition model, with a macroscopic fluid background and microscopic kinetic correction, both fully coupled to each other. A similar manner of discretization is proposed to that used in the recent \texttt{STRUPHY} code \cite{Holderied_Possanner_Wang_2021, Holderied_2022, Li_et_al_2023} with a finite-element model for the background and a pseudo-particle/PiC model for the correction.

        The fluid background satisfies the full, non-linear, resistive, compressible, Hall MHD equations. \cite{Laakmann_Hu_Farrell_2022} introduces finite-element(-in-space) implicit timesteppers for the incompressible analogue to this system with structure-preserving (SP) properties in the ideal case, alongside parameter-robust preconditioners. We show that these timesteppers can derive from a finite-element-in-time (FET) (and finite-element-in-space) interpretation. The benefits of this reformulation are discussed, including the derivation of timesteppers that are higher order in time, and the quantifiable dissipative SP properties in the non-ideal, resistive case.
        
        We discuss possible options for extending this FET approach to timesteppers for the compressible case.

        The kinetic corrections satisfy linearized Boltzmann equations. Using a Lénard--Bernstein collision operator, these take Fokker--Planck-like forms \cite{Fokker_1914, Planck_1917} wherein pseudo-particles in the numerical model obey the neoclassical transport equations, with particle-independent Brownian drift terms. This offers a rigorous methodology for incorporating collisions into the particle transport model, without coupling the equations of motions for each particle.
        
        Works by Chen, Chacón et al. \cite{Chen_Chacón_Barnes_2011, Chacón_Chen_Barnes_2013, Chen_Chacón_2014, Chen_Chacón_2015} have developed structure-preserving particle pushers for neoclassical transport in the Vlasov equations, derived from Crank--Nicolson integrators. We show these too can can derive from a FET interpretation, similarly offering potential extensions to higher-order-in-time particle pushers. The FET formulation is used also to consider how the stochastic drift terms can be incorporated into the pushers. Stochastic gyrokinetic expansions are also discussed.

        Different options for the numerical implementation of these schemes are considered.

        Due to the efficacy of FET in the development of SP timesteppers for both the fluid and kinetic component, we hope this approach will prove effective in the future for developing SP timesteppers for the full hybrid model. We hope this will give us the opportunity to incorporate previously inaccessible kinetic effects into the highly effective, modern, finite-element MHD models.
    \end{abstract}
    
    
    \newpage
    \tableofcontents
    
    
    \newpage
    \pagenumbering{arabic}
    %\linenumbers\renewcommand\thelinenumber{\color{black!50}\arabic{linenumber}}
            \input{0 - introduction/main.tex}
        \part{Research}
            \input{1 - low-noise PiC models/main.tex}
            \input{2 - kinetic component/main.tex}
            \input{3 - fluid component/main.tex}
            \input{4 - numerical implementation/main.tex}
        \part{Project Overview}
            \input{5 - research plan/main.tex}
            \input{6 - summary/main.tex}
    
    
    %\section{}
    \newpage
    \pagenumbering{gobble}
        \printbibliography


    \newpage
    \pagenumbering{roman}
    \appendix
        \part{Appendices}
            \input{8 - Hilbert complexes/main.tex}
            \input{9 - weak conservation proofs/main.tex}
\end{document}


\title{\BA{Title in Progress...}}
\author{Boris Andrews}
\affil{Mathematical Institute, University of Oxford}
\date{\today}


\begin{document}
    \pagenumbering{gobble}
    \maketitle
    
    
    \begin{abstract}
        Magnetic confinement reactors---in particular tokamaks---offer one of the most promising options for achieving practical nuclear fusion, with the potential to provide virtually limitless, clean energy. The theoretical and numerical modeling of tokamak plasmas is simultaneously an essential component of effective reactor design, and a great research barrier. Tokamak operational conditions exhibit comparatively low Knudsen numbers. Kinetic effects, including kinetic waves and instabilities, Landau damping, bump-on-tail instabilities and more, are therefore highly influential in tokamak plasma dynamics. Purely fluid models are inherently incapable of capturing these effects, whereas the high dimensionality in purely kinetic models render them practically intractable for most relevant purposes.

        We consider a $\delta\!f$ decomposition model, with a macroscopic fluid background and microscopic kinetic correction, both fully coupled to each other. A similar manner of discretization is proposed to that used in the recent \texttt{STRUPHY} code \cite{Holderied_Possanner_Wang_2021, Holderied_2022, Li_et_al_2023} with a finite-element model for the background and a pseudo-particle/PiC model for the correction.

        The fluid background satisfies the full, non-linear, resistive, compressible, Hall MHD equations. \cite{Laakmann_Hu_Farrell_2022} introduces finite-element(-in-space) implicit timesteppers for the incompressible analogue to this system with structure-preserving (SP) properties in the ideal case, alongside parameter-robust preconditioners. We show that these timesteppers can derive from a finite-element-in-time (FET) (and finite-element-in-space) interpretation. The benefits of this reformulation are discussed, including the derivation of timesteppers that are higher order in time, and the quantifiable dissipative SP properties in the non-ideal, resistive case.
        
        We discuss possible options for extending this FET approach to timesteppers for the compressible case.

        The kinetic corrections satisfy linearized Boltzmann equations. Using a Lénard--Bernstein collision operator, these take Fokker--Planck-like forms \cite{Fokker_1914, Planck_1917} wherein pseudo-particles in the numerical model obey the neoclassical transport equations, with particle-independent Brownian drift terms. This offers a rigorous methodology for incorporating collisions into the particle transport model, without coupling the equations of motions for each particle.
        
        Works by Chen, Chacón et al. \cite{Chen_Chacón_Barnes_2011, Chacón_Chen_Barnes_2013, Chen_Chacón_2014, Chen_Chacón_2015} have developed structure-preserving particle pushers for neoclassical transport in the Vlasov equations, derived from Crank--Nicolson integrators. We show these too can can derive from a FET interpretation, similarly offering potential extensions to higher-order-in-time particle pushers. The FET formulation is used also to consider how the stochastic drift terms can be incorporated into the pushers. Stochastic gyrokinetic expansions are also discussed.

        Different options for the numerical implementation of these schemes are considered.

        Due to the efficacy of FET in the development of SP timesteppers for both the fluid and kinetic component, we hope this approach will prove effective in the future for developing SP timesteppers for the full hybrid model. We hope this will give us the opportunity to incorporate previously inaccessible kinetic effects into the highly effective, modern, finite-element MHD models.
    \end{abstract}
    
    
    \newpage
    \tableofcontents
    
    
    \newpage
    \pagenumbering{arabic}
    %\linenumbers\renewcommand\thelinenumber{\color{black!50}\arabic{linenumber}}
            \documentclass[12pt, a4paper]{report}

\input{template/main.tex}

\title{\BA{Title in Progress...}}
\author{Boris Andrews}
\affil{Mathematical Institute, University of Oxford}
\date{\today}


\begin{document}
    \pagenumbering{gobble}
    \maketitle
    
    
    \begin{abstract}
        Magnetic confinement reactors---in particular tokamaks---offer one of the most promising options for achieving practical nuclear fusion, with the potential to provide virtually limitless, clean energy. The theoretical and numerical modeling of tokamak plasmas is simultaneously an essential component of effective reactor design, and a great research barrier. Tokamak operational conditions exhibit comparatively low Knudsen numbers. Kinetic effects, including kinetic waves and instabilities, Landau damping, bump-on-tail instabilities and more, are therefore highly influential in tokamak plasma dynamics. Purely fluid models are inherently incapable of capturing these effects, whereas the high dimensionality in purely kinetic models render them practically intractable for most relevant purposes.

        We consider a $\delta\!f$ decomposition model, with a macroscopic fluid background and microscopic kinetic correction, both fully coupled to each other. A similar manner of discretization is proposed to that used in the recent \texttt{STRUPHY} code \cite{Holderied_Possanner_Wang_2021, Holderied_2022, Li_et_al_2023} with a finite-element model for the background and a pseudo-particle/PiC model for the correction.

        The fluid background satisfies the full, non-linear, resistive, compressible, Hall MHD equations. \cite{Laakmann_Hu_Farrell_2022} introduces finite-element(-in-space) implicit timesteppers for the incompressible analogue to this system with structure-preserving (SP) properties in the ideal case, alongside parameter-robust preconditioners. We show that these timesteppers can derive from a finite-element-in-time (FET) (and finite-element-in-space) interpretation. The benefits of this reformulation are discussed, including the derivation of timesteppers that are higher order in time, and the quantifiable dissipative SP properties in the non-ideal, resistive case.
        
        We discuss possible options for extending this FET approach to timesteppers for the compressible case.

        The kinetic corrections satisfy linearized Boltzmann equations. Using a Lénard--Bernstein collision operator, these take Fokker--Planck-like forms \cite{Fokker_1914, Planck_1917} wherein pseudo-particles in the numerical model obey the neoclassical transport equations, with particle-independent Brownian drift terms. This offers a rigorous methodology for incorporating collisions into the particle transport model, without coupling the equations of motions for each particle.
        
        Works by Chen, Chacón et al. \cite{Chen_Chacón_Barnes_2011, Chacón_Chen_Barnes_2013, Chen_Chacón_2014, Chen_Chacón_2015} have developed structure-preserving particle pushers for neoclassical transport in the Vlasov equations, derived from Crank--Nicolson integrators. We show these too can can derive from a FET interpretation, similarly offering potential extensions to higher-order-in-time particle pushers. The FET formulation is used also to consider how the stochastic drift terms can be incorporated into the pushers. Stochastic gyrokinetic expansions are also discussed.

        Different options for the numerical implementation of these schemes are considered.

        Due to the efficacy of FET in the development of SP timesteppers for both the fluid and kinetic component, we hope this approach will prove effective in the future for developing SP timesteppers for the full hybrid model. We hope this will give us the opportunity to incorporate previously inaccessible kinetic effects into the highly effective, modern, finite-element MHD models.
    \end{abstract}
    
    
    \newpage
    \tableofcontents
    
    
    \newpage
    \pagenumbering{arabic}
    %\linenumbers\renewcommand\thelinenumber{\color{black!50}\arabic{linenumber}}
            \input{0 - introduction/main.tex}
        \part{Research}
            \input{1 - low-noise PiC models/main.tex}
            \input{2 - kinetic component/main.tex}
            \input{3 - fluid component/main.tex}
            \input{4 - numerical implementation/main.tex}
        \part{Project Overview}
            \input{5 - research plan/main.tex}
            \input{6 - summary/main.tex}
    
    
    %\section{}
    \newpage
    \pagenumbering{gobble}
        \printbibliography


    \newpage
    \pagenumbering{roman}
    \appendix
        \part{Appendices}
            \input{8 - Hilbert complexes/main.tex}
            \input{9 - weak conservation proofs/main.tex}
\end{document}

        \part{Research}
            \documentclass[12pt, a4paper]{report}

\input{template/main.tex}

\title{\BA{Title in Progress...}}
\author{Boris Andrews}
\affil{Mathematical Institute, University of Oxford}
\date{\today}


\begin{document}
    \pagenumbering{gobble}
    \maketitle
    
    
    \begin{abstract}
        Magnetic confinement reactors---in particular tokamaks---offer one of the most promising options for achieving practical nuclear fusion, with the potential to provide virtually limitless, clean energy. The theoretical and numerical modeling of tokamak plasmas is simultaneously an essential component of effective reactor design, and a great research barrier. Tokamak operational conditions exhibit comparatively low Knudsen numbers. Kinetic effects, including kinetic waves and instabilities, Landau damping, bump-on-tail instabilities and more, are therefore highly influential in tokamak plasma dynamics. Purely fluid models are inherently incapable of capturing these effects, whereas the high dimensionality in purely kinetic models render them practically intractable for most relevant purposes.

        We consider a $\delta\!f$ decomposition model, with a macroscopic fluid background and microscopic kinetic correction, both fully coupled to each other. A similar manner of discretization is proposed to that used in the recent \texttt{STRUPHY} code \cite{Holderied_Possanner_Wang_2021, Holderied_2022, Li_et_al_2023} with a finite-element model for the background and a pseudo-particle/PiC model for the correction.

        The fluid background satisfies the full, non-linear, resistive, compressible, Hall MHD equations. \cite{Laakmann_Hu_Farrell_2022} introduces finite-element(-in-space) implicit timesteppers for the incompressible analogue to this system with structure-preserving (SP) properties in the ideal case, alongside parameter-robust preconditioners. We show that these timesteppers can derive from a finite-element-in-time (FET) (and finite-element-in-space) interpretation. The benefits of this reformulation are discussed, including the derivation of timesteppers that are higher order in time, and the quantifiable dissipative SP properties in the non-ideal, resistive case.
        
        We discuss possible options for extending this FET approach to timesteppers for the compressible case.

        The kinetic corrections satisfy linearized Boltzmann equations. Using a Lénard--Bernstein collision operator, these take Fokker--Planck-like forms \cite{Fokker_1914, Planck_1917} wherein pseudo-particles in the numerical model obey the neoclassical transport equations, with particle-independent Brownian drift terms. This offers a rigorous methodology for incorporating collisions into the particle transport model, without coupling the equations of motions for each particle.
        
        Works by Chen, Chacón et al. \cite{Chen_Chacón_Barnes_2011, Chacón_Chen_Barnes_2013, Chen_Chacón_2014, Chen_Chacón_2015} have developed structure-preserving particle pushers for neoclassical transport in the Vlasov equations, derived from Crank--Nicolson integrators. We show these too can can derive from a FET interpretation, similarly offering potential extensions to higher-order-in-time particle pushers. The FET formulation is used also to consider how the stochastic drift terms can be incorporated into the pushers. Stochastic gyrokinetic expansions are also discussed.

        Different options for the numerical implementation of these schemes are considered.

        Due to the efficacy of FET in the development of SP timesteppers for both the fluid and kinetic component, we hope this approach will prove effective in the future for developing SP timesteppers for the full hybrid model. We hope this will give us the opportunity to incorporate previously inaccessible kinetic effects into the highly effective, modern, finite-element MHD models.
    \end{abstract}
    
    
    \newpage
    \tableofcontents
    
    
    \newpage
    \pagenumbering{arabic}
    %\linenumbers\renewcommand\thelinenumber{\color{black!50}\arabic{linenumber}}
            \input{0 - introduction/main.tex}
        \part{Research}
            \input{1 - low-noise PiC models/main.tex}
            \input{2 - kinetic component/main.tex}
            \input{3 - fluid component/main.tex}
            \input{4 - numerical implementation/main.tex}
        \part{Project Overview}
            \input{5 - research plan/main.tex}
            \input{6 - summary/main.tex}
    
    
    %\section{}
    \newpage
    \pagenumbering{gobble}
        \printbibliography


    \newpage
    \pagenumbering{roman}
    \appendix
        \part{Appendices}
            \input{8 - Hilbert complexes/main.tex}
            \input{9 - weak conservation proofs/main.tex}
\end{document}

            \documentclass[12pt, a4paper]{report}

\input{template/main.tex}

\title{\BA{Title in Progress...}}
\author{Boris Andrews}
\affil{Mathematical Institute, University of Oxford}
\date{\today}


\begin{document}
    \pagenumbering{gobble}
    \maketitle
    
    
    \begin{abstract}
        Magnetic confinement reactors---in particular tokamaks---offer one of the most promising options for achieving practical nuclear fusion, with the potential to provide virtually limitless, clean energy. The theoretical and numerical modeling of tokamak plasmas is simultaneously an essential component of effective reactor design, and a great research barrier. Tokamak operational conditions exhibit comparatively low Knudsen numbers. Kinetic effects, including kinetic waves and instabilities, Landau damping, bump-on-tail instabilities and more, are therefore highly influential in tokamak plasma dynamics. Purely fluid models are inherently incapable of capturing these effects, whereas the high dimensionality in purely kinetic models render them practically intractable for most relevant purposes.

        We consider a $\delta\!f$ decomposition model, with a macroscopic fluid background and microscopic kinetic correction, both fully coupled to each other. A similar manner of discretization is proposed to that used in the recent \texttt{STRUPHY} code \cite{Holderied_Possanner_Wang_2021, Holderied_2022, Li_et_al_2023} with a finite-element model for the background and a pseudo-particle/PiC model for the correction.

        The fluid background satisfies the full, non-linear, resistive, compressible, Hall MHD equations. \cite{Laakmann_Hu_Farrell_2022} introduces finite-element(-in-space) implicit timesteppers for the incompressible analogue to this system with structure-preserving (SP) properties in the ideal case, alongside parameter-robust preconditioners. We show that these timesteppers can derive from a finite-element-in-time (FET) (and finite-element-in-space) interpretation. The benefits of this reformulation are discussed, including the derivation of timesteppers that are higher order in time, and the quantifiable dissipative SP properties in the non-ideal, resistive case.
        
        We discuss possible options for extending this FET approach to timesteppers for the compressible case.

        The kinetic corrections satisfy linearized Boltzmann equations. Using a Lénard--Bernstein collision operator, these take Fokker--Planck-like forms \cite{Fokker_1914, Planck_1917} wherein pseudo-particles in the numerical model obey the neoclassical transport equations, with particle-independent Brownian drift terms. This offers a rigorous methodology for incorporating collisions into the particle transport model, without coupling the equations of motions for each particle.
        
        Works by Chen, Chacón et al. \cite{Chen_Chacón_Barnes_2011, Chacón_Chen_Barnes_2013, Chen_Chacón_2014, Chen_Chacón_2015} have developed structure-preserving particle pushers for neoclassical transport in the Vlasov equations, derived from Crank--Nicolson integrators. We show these too can can derive from a FET interpretation, similarly offering potential extensions to higher-order-in-time particle pushers. The FET formulation is used also to consider how the stochastic drift terms can be incorporated into the pushers. Stochastic gyrokinetic expansions are also discussed.

        Different options for the numerical implementation of these schemes are considered.

        Due to the efficacy of FET in the development of SP timesteppers for both the fluid and kinetic component, we hope this approach will prove effective in the future for developing SP timesteppers for the full hybrid model. We hope this will give us the opportunity to incorporate previously inaccessible kinetic effects into the highly effective, modern, finite-element MHD models.
    \end{abstract}
    
    
    \newpage
    \tableofcontents
    
    
    \newpage
    \pagenumbering{arabic}
    %\linenumbers\renewcommand\thelinenumber{\color{black!50}\arabic{linenumber}}
            \input{0 - introduction/main.tex}
        \part{Research}
            \input{1 - low-noise PiC models/main.tex}
            \input{2 - kinetic component/main.tex}
            \input{3 - fluid component/main.tex}
            \input{4 - numerical implementation/main.tex}
        \part{Project Overview}
            \input{5 - research plan/main.tex}
            \input{6 - summary/main.tex}
    
    
    %\section{}
    \newpage
    \pagenumbering{gobble}
        \printbibliography


    \newpage
    \pagenumbering{roman}
    \appendix
        \part{Appendices}
            \input{8 - Hilbert complexes/main.tex}
            \input{9 - weak conservation proofs/main.tex}
\end{document}

            \documentclass[12pt, a4paper]{report}

\input{template/main.tex}

\title{\BA{Title in Progress...}}
\author{Boris Andrews}
\affil{Mathematical Institute, University of Oxford}
\date{\today}


\begin{document}
    \pagenumbering{gobble}
    \maketitle
    
    
    \begin{abstract}
        Magnetic confinement reactors---in particular tokamaks---offer one of the most promising options for achieving practical nuclear fusion, with the potential to provide virtually limitless, clean energy. The theoretical and numerical modeling of tokamak plasmas is simultaneously an essential component of effective reactor design, and a great research barrier. Tokamak operational conditions exhibit comparatively low Knudsen numbers. Kinetic effects, including kinetic waves and instabilities, Landau damping, bump-on-tail instabilities and more, are therefore highly influential in tokamak plasma dynamics. Purely fluid models are inherently incapable of capturing these effects, whereas the high dimensionality in purely kinetic models render them practically intractable for most relevant purposes.

        We consider a $\delta\!f$ decomposition model, with a macroscopic fluid background and microscopic kinetic correction, both fully coupled to each other. A similar manner of discretization is proposed to that used in the recent \texttt{STRUPHY} code \cite{Holderied_Possanner_Wang_2021, Holderied_2022, Li_et_al_2023} with a finite-element model for the background and a pseudo-particle/PiC model for the correction.

        The fluid background satisfies the full, non-linear, resistive, compressible, Hall MHD equations. \cite{Laakmann_Hu_Farrell_2022} introduces finite-element(-in-space) implicit timesteppers for the incompressible analogue to this system with structure-preserving (SP) properties in the ideal case, alongside parameter-robust preconditioners. We show that these timesteppers can derive from a finite-element-in-time (FET) (and finite-element-in-space) interpretation. The benefits of this reformulation are discussed, including the derivation of timesteppers that are higher order in time, and the quantifiable dissipative SP properties in the non-ideal, resistive case.
        
        We discuss possible options for extending this FET approach to timesteppers for the compressible case.

        The kinetic corrections satisfy linearized Boltzmann equations. Using a Lénard--Bernstein collision operator, these take Fokker--Planck-like forms \cite{Fokker_1914, Planck_1917} wherein pseudo-particles in the numerical model obey the neoclassical transport equations, with particle-independent Brownian drift terms. This offers a rigorous methodology for incorporating collisions into the particle transport model, without coupling the equations of motions for each particle.
        
        Works by Chen, Chacón et al. \cite{Chen_Chacón_Barnes_2011, Chacón_Chen_Barnes_2013, Chen_Chacón_2014, Chen_Chacón_2015} have developed structure-preserving particle pushers for neoclassical transport in the Vlasov equations, derived from Crank--Nicolson integrators. We show these too can can derive from a FET interpretation, similarly offering potential extensions to higher-order-in-time particle pushers. The FET formulation is used also to consider how the stochastic drift terms can be incorporated into the pushers. Stochastic gyrokinetic expansions are also discussed.

        Different options for the numerical implementation of these schemes are considered.

        Due to the efficacy of FET in the development of SP timesteppers for both the fluid and kinetic component, we hope this approach will prove effective in the future for developing SP timesteppers for the full hybrid model. We hope this will give us the opportunity to incorporate previously inaccessible kinetic effects into the highly effective, modern, finite-element MHD models.
    \end{abstract}
    
    
    \newpage
    \tableofcontents
    
    
    \newpage
    \pagenumbering{arabic}
    %\linenumbers\renewcommand\thelinenumber{\color{black!50}\arabic{linenumber}}
            \input{0 - introduction/main.tex}
        \part{Research}
            \input{1 - low-noise PiC models/main.tex}
            \input{2 - kinetic component/main.tex}
            \input{3 - fluid component/main.tex}
            \input{4 - numerical implementation/main.tex}
        \part{Project Overview}
            \input{5 - research plan/main.tex}
            \input{6 - summary/main.tex}
    
    
    %\section{}
    \newpage
    \pagenumbering{gobble}
        \printbibliography


    \newpage
    \pagenumbering{roman}
    \appendix
        \part{Appendices}
            \input{8 - Hilbert complexes/main.tex}
            \input{9 - weak conservation proofs/main.tex}
\end{document}

            \documentclass[12pt, a4paper]{report}

\input{template/main.tex}

\title{\BA{Title in Progress...}}
\author{Boris Andrews}
\affil{Mathematical Institute, University of Oxford}
\date{\today}


\begin{document}
    \pagenumbering{gobble}
    \maketitle
    
    
    \begin{abstract}
        Magnetic confinement reactors---in particular tokamaks---offer one of the most promising options for achieving practical nuclear fusion, with the potential to provide virtually limitless, clean energy. The theoretical and numerical modeling of tokamak plasmas is simultaneously an essential component of effective reactor design, and a great research barrier. Tokamak operational conditions exhibit comparatively low Knudsen numbers. Kinetic effects, including kinetic waves and instabilities, Landau damping, bump-on-tail instabilities and more, are therefore highly influential in tokamak plasma dynamics. Purely fluid models are inherently incapable of capturing these effects, whereas the high dimensionality in purely kinetic models render them practically intractable for most relevant purposes.

        We consider a $\delta\!f$ decomposition model, with a macroscopic fluid background and microscopic kinetic correction, both fully coupled to each other. A similar manner of discretization is proposed to that used in the recent \texttt{STRUPHY} code \cite{Holderied_Possanner_Wang_2021, Holderied_2022, Li_et_al_2023} with a finite-element model for the background and a pseudo-particle/PiC model for the correction.

        The fluid background satisfies the full, non-linear, resistive, compressible, Hall MHD equations. \cite{Laakmann_Hu_Farrell_2022} introduces finite-element(-in-space) implicit timesteppers for the incompressible analogue to this system with structure-preserving (SP) properties in the ideal case, alongside parameter-robust preconditioners. We show that these timesteppers can derive from a finite-element-in-time (FET) (and finite-element-in-space) interpretation. The benefits of this reformulation are discussed, including the derivation of timesteppers that are higher order in time, and the quantifiable dissipative SP properties in the non-ideal, resistive case.
        
        We discuss possible options for extending this FET approach to timesteppers for the compressible case.

        The kinetic corrections satisfy linearized Boltzmann equations. Using a Lénard--Bernstein collision operator, these take Fokker--Planck-like forms \cite{Fokker_1914, Planck_1917} wherein pseudo-particles in the numerical model obey the neoclassical transport equations, with particle-independent Brownian drift terms. This offers a rigorous methodology for incorporating collisions into the particle transport model, without coupling the equations of motions for each particle.
        
        Works by Chen, Chacón et al. \cite{Chen_Chacón_Barnes_2011, Chacón_Chen_Barnes_2013, Chen_Chacón_2014, Chen_Chacón_2015} have developed structure-preserving particle pushers for neoclassical transport in the Vlasov equations, derived from Crank--Nicolson integrators. We show these too can can derive from a FET interpretation, similarly offering potential extensions to higher-order-in-time particle pushers. The FET formulation is used also to consider how the stochastic drift terms can be incorporated into the pushers. Stochastic gyrokinetic expansions are also discussed.

        Different options for the numerical implementation of these schemes are considered.

        Due to the efficacy of FET in the development of SP timesteppers for both the fluid and kinetic component, we hope this approach will prove effective in the future for developing SP timesteppers for the full hybrid model. We hope this will give us the opportunity to incorporate previously inaccessible kinetic effects into the highly effective, modern, finite-element MHD models.
    \end{abstract}
    
    
    \newpage
    \tableofcontents
    
    
    \newpage
    \pagenumbering{arabic}
    %\linenumbers\renewcommand\thelinenumber{\color{black!50}\arabic{linenumber}}
            \input{0 - introduction/main.tex}
        \part{Research}
            \input{1 - low-noise PiC models/main.tex}
            \input{2 - kinetic component/main.tex}
            \input{3 - fluid component/main.tex}
            \input{4 - numerical implementation/main.tex}
        \part{Project Overview}
            \input{5 - research plan/main.tex}
            \input{6 - summary/main.tex}
    
    
    %\section{}
    \newpage
    \pagenumbering{gobble}
        \printbibliography


    \newpage
    \pagenumbering{roman}
    \appendix
        \part{Appendices}
            \input{8 - Hilbert complexes/main.tex}
            \input{9 - weak conservation proofs/main.tex}
\end{document}

        \part{Project Overview}
            \documentclass[12pt, a4paper]{report}

\input{template/main.tex}

\title{\BA{Title in Progress...}}
\author{Boris Andrews}
\affil{Mathematical Institute, University of Oxford}
\date{\today}


\begin{document}
    \pagenumbering{gobble}
    \maketitle
    
    
    \begin{abstract}
        Magnetic confinement reactors---in particular tokamaks---offer one of the most promising options for achieving practical nuclear fusion, with the potential to provide virtually limitless, clean energy. The theoretical and numerical modeling of tokamak plasmas is simultaneously an essential component of effective reactor design, and a great research barrier. Tokamak operational conditions exhibit comparatively low Knudsen numbers. Kinetic effects, including kinetic waves and instabilities, Landau damping, bump-on-tail instabilities and more, are therefore highly influential in tokamak plasma dynamics. Purely fluid models are inherently incapable of capturing these effects, whereas the high dimensionality in purely kinetic models render them practically intractable for most relevant purposes.

        We consider a $\delta\!f$ decomposition model, with a macroscopic fluid background and microscopic kinetic correction, both fully coupled to each other. A similar manner of discretization is proposed to that used in the recent \texttt{STRUPHY} code \cite{Holderied_Possanner_Wang_2021, Holderied_2022, Li_et_al_2023} with a finite-element model for the background and a pseudo-particle/PiC model for the correction.

        The fluid background satisfies the full, non-linear, resistive, compressible, Hall MHD equations. \cite{Laakmann_Hu_Farrell_2022} introduces finite-element(-in-space) implicit timesteppers for the incompressible analogue to this system with structure-preserving (SP) properties in the ideal case, alongside parameter-robust preconditioners. We show that these timesteppers can derive from a finite-element-in-time (FET) (and finite-element-in-space) interpretation. The benefits of this reformulation are discussed, including the derivation of timesteppers that are higher order in time, and the quantifiable dissipative SP properties in the non-ideal, resistive case.
        
        We discuss possible options for extending this FET approach to timesteppers for the compressible case.

        The kinetic corrections satisfy linearized Boltzmann equations. Using a Lénard--Bernstein collision operator, these take Fokker--Planck-like forms \cite{Fokker_1914, Planck_1917} wherein pseudo-particles in the numerical model obey the neoclassical transport equations, with particle-independent Brownian drift terms. This offers a rigorous methodology for incorporating collisions into the particle transport model, without coupling the equations of motions for each particle.
        
        Works by Chen, Chacón et al. \cite{Chen_Chacón_Barnes_2011, Chacón_Chen_Barnes_2013, Chen_Chacón_2014, Chen_Chacón_2015} have developed structure-preserving particle pushers for neoclassical transport in the Vlasov equations, derived from Crank--Nicolson integrators. We show these too can can derive from a FET interpretation, similarly offering potential extensions to higher-order-in-time particle pushers. The FET formulation is used also to consider how the stochastic drift terms can be incorporated into the pushers. Stochastic gyrokinetic expansions are also discussed.

        Different options for the numerical implementation of these schemes are considered.

        Due to the efficacy of FET in the development of SP timesteppers for both the fluid and kinetic component, we hope this approach will prove effective in the future for developing SP timesteppers for the full hybrid model. We hope this will give us the opportunity to incorporate previously inaccessible kinetic effects into the highly effective, modern, finite-element MHD models.
    \end{abstract}
    
    
    \newpage
    \tableofcontents
    
    
    \newpage
    \pagenumbering{arabic}
    %\linenumbers\renewcommand\thelinenumber{\color{black!50}\arabic{linenumber}}
            \input{0 - introduction/main.tex}
        \part{Research}
            \input{1 - low-noise PiC models/main.tex}
            \input{2 - kinetic component/main.tex}
            \input{3 - fluid component/main.tex}
            \input{4 - numerical implementation/main.tex}
        \part{Project Overview}
            \input{5 - research plan/main.tex}
            \input{6 - summary/main.tex}
    
    
    %\section{}
    \newpage
    \pagenumbering{gobble}
        \printbibliography


    \newpage
    \pagenumbering{roman}
    \appendix
        \part{Appendices}
            \input{8 - Hilbert complexes/main.tex}
            \input{9 - weak conservation proofs/main.tex}
\end{document}

            \documentclass[12pt, a4paper]{report}

\input{template/main.tex}

\title{\BA{Title in Progress...}}
\author{Boris Andrews}
\affil{Mathematical Institute, University of Oxford}
\date{\today}


\begin{document}
    \pagenumbering{gobble}
    \maketitle
    
    
    \begin{abstract}
        Magnetic confinement reactors---in particular tokamaks---offer one of the most promising options for achieving practical nuclear fusion, with the potential to provide virtually limitless, clean energy. The theoretical and numerical modeling of tokamak plasmas is simultaneously an essential component of effective reactor design, and a great research barrier. Tokamak operational conditions exhibit comparatively low Knudsen numbers. Kinetic effects, including kinetic waves and instabilities, Landau damping, bump-on-tail instabilities and more, are therefore highly influential in tokamak plasma dynamics. Purely fluid models are inherently incapable of capturing these effects, whereas the high dimensionality in purely kinetic models render them practically intractable for most relevant purposes.

        We consider a $\delta\!f$ decomposition model, with a macroscopic fluid background and microscopic kinetic correction, both fully coupled to each other. A similar manner of discretization is proposed to that used in the recent \texttt{STRUPHY} code \cite{Holderied_Possanner_Wang_2021, Holderied_2022, Li_et_al_2023} with a finite-element model for the background and a pseudo-particle/PiC model for the correction.

        The fluid background satisfies the full, non-linear, resistive, compressible, Hall MHD equations. \cite{Laakmann_Hu_Farrell_2022} introduces finite-element(-in-space) implicit timesteppers for the incompressible analogue to this system with structure-preserving (SP) properties in the ideal case, alongside parameter-robust preconditioners. We show that these timesteppers can derive from a finite-element-in-time (FET) (and finite-element-in-space) interpretation. The benefits of this reformulation are discussed, including the derivation of timesteppers that are higher order in time, and the quantifiable dissipative SP properties in the non-ideal, resistive case.
        
        We discuss possible options for extending this FET approach to timesteppers for the compressible case.

        The kinetic corrections satisfy linearized Boltzmann equations. Using a Lénard--Bernstein collision operator, these take Fokker--Planck-like forms \cite{Fokker_1914, Planck_1917} wherein pseudo-particles in the numerical model obey the neoclassical transport equations, with particle-independent Brownian drift terms. This offers a rigorous methodology for incorporating collisions into the particle transport model, without coupling the equations of motions for each particle.
        
        Works by Chen, Chacón et al. \cite{Chen_Chacón_Barnes_2011, Chacón_Chen_Barnes_2013, Chen_Chacón_2014, Chen_Chacón_2015} have developed structure-preserving particle pushers for neoclassical transport in the Vlasov equations, derived from Crank--Nicolson integrators. We show these too can can derive from a FET interpretation, similarly offering potential extensions to higher-order-in-time particle pushers. The FET formulation is used also to consider how the stochastic drift terms can be incorporated into the pushers. Stochastic gyrokinetic expansions are also discussed.

        Different options for the numerical implementation of these schemes are considered.

        Due to the efficacy of FET in the development of SP timesteppers for both the fluid and kinetic component, we hope this approach will prove effective in the future for developing SP timesteppers for the full hybrid model. We hope this will give us the opportunity to incorporate previously inaccessible kinetic effects into the highly effective, modern, finite-element MHD models.
    \end{abstract}
    
    
    \newpage
    \tableofcontents
    
    
    \newpage
    \pagenumbering{arabic}
    %\linenumbers\renewcommand\thelinenumber{\color{black!50}\arabic{linenumber}}
            \input{0 - introduction/main.tex}
        \part{Research}
            \input{1 - low-noise PiC models/main.tex}
            \input{2 - kinetic component/main.tex}
            \input{3 - fluid component/main.tex}
            \input{4 - numerical implementation/main.tex}
        \part{Project Overview}
            \input{5 - research plan/main.tex}
            \input{6 - summary/main.tex}
    
    
    %\section{}
    \newpage
    \pagenumbering{gobble}
        \printbibliography


    \newpage
    \pagenumbering{roman}
    \appendix
        \part{Appendices}
            \input{8 - Hilbert complexes/main.tex}
            \input{9 - weak conservation proofs/main.tex}
\end{document}

    
    
    %\section{}
    \newpage
    \pagenumbering{gobble}
        \printbibliography


    \newpage
    \pagenumbering{roman}
    \appendix
        \part{Appendices}
            \documentclass[12pt, a4paper]{report}

\input{template/main.tex}

\title{\BA{Title in Progress...}}
\author{Boris Andrews}
\affil{Mathematical Institute, University of Oxford}
\date{\today}


\begin{document}
    \pagenumbering{gobble}
    \maketitle
    
    
    \begin{abstract}
        Magnetic confinement reactors---in particular tokamaks---offer one of the most promising options for achieving practical nuclear fusion, with the potential to provide virtually limitless, clean energy. The theoretical and numerical modeling of tokamak plasmas is simultaneously an essential component of effective reactor design, and a great research barrier. Tokamak operational conditions exhibit comparatively low Knudsen numbers. Kinetic effects, including kinetic waves and instabilities, Landau damping, bump-on-tail instabilities and more, are therefore highly influential in tokamak plasma dynamics. Purely fluid models are inherently incapable of capturing these effects, whereas the high dimensionality in purely kinetic models render them practically intractable for most relevant purposes.

        We consider a $\delta\!f$ decomposition model, with a macroscopic fluid background and microscopic kinetic correction, both fully coupled to each other. A similar manner of discretization is proposed to that used in the recent \texttt{STRUPHY} code \cite{Holderied_Possanner_Wang_2021, Holderied_2022, Li_et_al_2023} with a finite-element model for the background and a pseudo-particle/PiC model for the correction.

        The fluid background satisfies the full, non-linear, resistive, compressible, Hall MHD equations. \cite{Laakmann_Hu_Farrell_2022} introduces finite-element(-in-space) implicit timesteppers for the incompressible analogue to this system with structure-preserving (SP) properties in the ideal case, alongside parameter-robust preconditioners. We show that these timesteppers can derive from a finite-element-in-time (FET) (and finite-element-in-space) interpretation. The benefits of this reformulation are discussed, including the derivation of timesteppers that are higher order in time, and the quantifiable dissipative SP properties in the non-ideal, resistive case.
        
        We discuss possible options for extending this FET approach to timesteppers for the compressible case.

        The kinetic corrections satisfy linearized Boltzmann equations. Using a Lénard--Bernstein collision operator, these take Fokker--Planck-like forms \cite{Fokker_1914, Planck_1917} wherein pseudo-particles in the numerical model obey the neoclassical transport equations, with particle-independent Brownian drift terms. This offers a rigorous methodology for incorporating collisions into the particle transport model, without coupling the equations of motions for each particle.
        
        Works by Chen, Chacón et al. \cite{Chen_Chacón_Barnes_2011, Chacón_Chen_Barnes_2013, Chen_Chacón_2014, Chen_Chacón_2015} have developed structure-preserving particle pushers for neoclassical transport in the Vlasov equations, derived from Crank--Nicolson integrators. We show these too can can derive from a FET interpretation, similarly offering potential extensions to higher-order-in-time particle pushers. The FET formulation is used also to consider how the stochastic drift terms can be incorporated into the pushers. Stochastic gyrokinetic expansions are also discussed.

        Different options for the numerical implementation of these schemes are considered.

        Due to the efficacy of FET in the development of SP timesteppers for both the fluid and kinetic component, we hope this approach will prove effective in the future for developing SP timesteppers for the full hybrid model. We hope this will give us the opportunity to incorporate previously inaccessible kinetic effects into the highly effective, modern, finite-element MHD models.
    \end{abstract}
    
    
    \newpage
    \tableofcontents
    
    
    \newpage
    \pagenumbering{arabic}
    %\linenumbers\renewcommand\thelinenumber{\color{black!50}\arabic{linenumber}}
            \input{0 - introduction/main.tex}
        \part{Research}
            \input{1 - low-noise PiC models/main.tex}
            \input{2 - kinetic component/main.tex}
            \input{3 - fluid component/main.tex}
            \input{4 - numerical implementation/main.tex}
        \part{Project Overview}
            \input{5 - research plan/main.tex}
            \input{6 - summary/main.tex}
    
    
    %\section{}
    \newpage
    \pagenumbering{gobble}
        \printbibliography


    \newpage
    \pagenumbering{roman}
    \appendix
        \part{Appendices}
            \input{8 - Hilbert complexes/main.tex}
            \input{9 - weak conservation proofs/main.tex}
\end{document}

            \documentclass[12pt, a4paper]{report}

\input{template/main.tex}

\title{\BA{Title in Progress...}}
\author{Boris Andrews}
\affil{Mathematical Institute, University of Oxford}
\date{\today}


\begin{document}
    \pagenumbering{gobble}
    \maketitle
    
    
    \begin{abstract}
        Magnetic confinement reactors---in particular tokamaks---offer one of the most promising options for achieving practical nuclear fusion, with the potential to provide virtually limitless, clean energy. The theoretical and numerical modeling of tokamak plasmas is simultaneously an essential component of effective reactor design, and a great research barrier. Tokamak operational conditions exhibit comparatively low Knudsen numbers. Kinetic effects, including kinetic waves and instabilities, Landau damping, bump-on-tail instabilities and more, are therefore highly influential in tokamak plasma dynamics. Purely fluid models are inherently incapable of capturing these effects, whereas the high dimensionality in purely kinetic models render them practically intractable for most relevant purposes.

        We consider a $\delta\!f$ decomposition model, with a macroscopic fluid background and microscopic kinetic correction, both fully coupled to each other. A similar manner of discretization is proposed to that used in the recent \texttt{STRUPHY} code \cite{Holderied_Possanner_Wang_2021, Holderied_2022, Li_et_al_2023} with a finite-element model for the background and a pseudo-particle/PiC model for the correction.

        The fluid background satisfies the full, non-linear, resistive, compressible, Hall MHD equations. \cite{Laakmann_Hu_Farrell_2022} introduces finite-element(-in-space) implicit timesteppers for the incompressible analogue to this system with structure-preserving (SP) properties in the ideal case, alongside parameter-robust preconditioners. We show that these timesteppers can derive from a finite-element-in-time (FET) (and finite-element-in-space) interpretation. The benefits of this reformulation are discussed, including the derivation of timesteppers that are higher order in time, and the quantifiable dissipative SP properties in the non-ideal, resistive case.
        
        We discuss possible options for extending this FET approach to timesteppers for the compressible case.

        The kinetic corrections satisfy linearized Boltzmann equations. Using a Lénard--Bernstein collision operator, these take Fokker--Planck-like forms \cite{Fokker_1914, Planck_1917} wherein pseudo-particles in the numerical model obey the neoclassical transport equations, with particle-independent Brownian drift terms. This offers a rigorous methodology for incorporating collisions into the particle transport model, without coupling the equations of motions for each particle.
        
        Works by Chen, Chacón et al. \cite{Chen_Chacón_Barnes_2011, Chacón_Chen_Barnes_2013, Chen_Chacón_2014, Chen_Chacón_2015} have developed structure-preserving particle pushers for neoclassical transport in the Vlasov equations, derived from Crank--Nicolson integrators. We show these too can can derive from a FET interpretation, similarly offering potential extensions to higher-order-in-time particle pushers. The FET formulation is used also to consider how the stochastic drift terms can be incorporated into the pushers. Stochastic gyrokinetic expansions are also discussed.

        Different options for the numerical implementation of these schemes are considered.

        Due to the efficacy of FET in the development of SP timesteppers for both the fluid and kinetic component, we hope this approach will prove effective in the future for developing SP timesteppers for the full hybrid model. We hope this will give us the opportunity to incorporate previously inaccessible kinetic effects into the highly effective, modern, finite-element MHD models.
    \end{abstract}
    
    
    \newpage
    \tableofcontents
    
    
    \newpage
    \pagenumbering{arabic}
    %\linenumbers\renewcommand\thelinenumber{\color{black!50}\arabic{linenumber}}
            \input{0 - introduction/main.tex}
        \part{Research}
            \input{1 - low-noise PiC models/main.tex}
            \input{2 - kinetic component/main.tex}
            \input{3 - fluid component/main.tex}
            \input{4 - numerical implementation/main.tex}
        \part{Project Overview}
            \input{5 - research plan/main.tex}
            \input{6 - summary/main.tex}
    
    
    %\section{}
    \newpage
    \pagenumbering{gobble}
        \printbibliography


    \newpage
    \pagenumbering{roman}
    \appendix
        \part{Appendices}
            \input{8 - Hilbert complexes/main.tex}
            \input{9 - weak conservation proofs/main.tex}
\end{document}

\end{document}

        \part{Project Overview}
            \documentclass[12pt, a4paper]{report}

\documentclass[12pt, a4paper]{report}

\input{template/main.tex}

\title{\BA{Title in Progress...}}
\author{Boris Andrews}
\affil{Mathematical Institute, University of Oxford}
\date{\today}


\begin{document}
    \pagenumbering{gobble}
    \maketitle
    
    
    \begin{abstract}
        Magnetic confinement reactors---in particular tokamaks---offer one of the most promising options for achieving practical nuclear fusion, with the potential to provide virtually limitless, clean energy. The theoretical and numerical modeling of tokamak plasmas is simultaneously an essential component of effective reactor design, and a great research barrier. Tokamak operational conditions exhibit comparatively low Knudsen numbers. Kinetic effects, including kinetic waves and instabilities, Landau damping, bump-on-tail instabilities and more, are therefore highly influential in tokamak plasma dynamics. Purely fluid models are inherently incapable of capturing these effects, whereas the high dimensionality in purely kinetic models render them practically intractable for most relevant purposes.

        We consider a $\delta\!f$ decomposition model, with a macroscopic fluid background and microscopic kinetic correction, both fully coupled to each other. A similar manner of discretization is proposed to that used in the recent \texttt{STRUPHY} code \cite{Holderied_Possanner_Wang_2021, Holderied_2022, Li_et_al_2023} with a finite-element model for the background and a pseudo-particle/PiC model for the correction.

        The fluid background satisfies the full, non-linear, resistive, compressible, Hall MHD equations. \cite{Laakmann_Hu_Farrell_2022} introduces finite-element(-in-space) implicit timesteppers for the incompressible analogue to this system with structure-preserving (SP) properties in the ideal case, alongside parameter-robust preconditioners. We show that these timesteppers can derive from a finite-element-in-time (FET) (and finite-element-in-space) interpretation. The benefits of this reformulation are discussed, including the derivation of timesteppers that are higher order in time, and the quantifiable dissipative SP properties in the non-ideal, resistive case.
        
        We discuss possible options for extending this FET approach to timesteppers for the compressible case.

        The kinetic corrections satisfy linearized Boltzmann equations. Using a Lénard--Bernstein collision operator, these take Fokker--Planck-like forms \cite{Fokker_1914, Planck_1917} wherein pseudo-particles in the numerical model obey the neoclassical transport equations, with particle-independent Brownian drift terms. This offers a rigorous methodology for incorporating collisions into the particle transport model, without coupling the equations of motions for each particle.
        
        Works by Chen, Chacón et al. \cite{Chen_Chacón_Barnes_2011, Chacón_Chen_Barnes_2013, Chen_Chacón_2014, Chen_Chacón_2015} have developed structure-preserving particle pushers for neoclassical transport in the Vlasov equations, derived from Crank--Nicolson integrators. We show these too can can derive from a FET interpretation, similarly offering potential extensions to higher-order-in-time particle pushers. The FET formulation is used also to consider how the stochastic drift terms can be incorporated into the pushers. Stochastic gyrokinetic expansions are also discussed.

        Different options for the numerical implementation of these schemes are considered.

        Due to the efficacy of FET in the development of SP timesteppers for both the fluid and kinetic component, we hope this approach will prove effective in the future for developing SP timesteppers for the full hybrid model. We hope this will give us the opportunity to incorporate previously inaccessible kinetic effects into the highly effective, modern, finite-element MHD models.
    \end{abstract}
    
    
    \newpage
    \tableofcontents
    
    
    \newpage
    \pagenumbering{arabic}
    %\linenumbers\renewcommand\thelinenumber{\color{black!50}\arabic{linenumber}}
            \input{0 - introduction/main.tex}
        \part{Research}
            \input{1 - low-noise PiC models/main.tex}
            \input{2 - kinetic component/main.tex}
            \input{3 - fluid component/main.tex}
            \input{4 - numerical implementation/main.tex}
        \part{Project Overview}
            \input{5 - research plan/main.tex}
            \input{6 - summary/main.tex}
    
    
    %\section{}
    \newpage
    \pagenumbering{gobble}
        \printbibliography


    \newpage
    \pagenumbering{roman}
    \appendix
        \part{Appendices}
            \input{8 - Hilbert complexes/main.tex}
            \input{9 - weak conservation proofs/main.tex}
\end{document}


\title{\BA{Title in Progress...}}
\author{Boris Andrews}
\affil{Mathematical Institute, University of Oxford}
\date{\today}


\begin{document}
    \pagenumbering{gobble}
    \maketitle
    
    
    \begin{abstract}
        Magnetic confinement reactors---in particular tokamaks---offer one of the most promising options for achieving practical nuclear fusion, with the potential to provide virtually limitless, clean energy. The theoretical and numerical modeling of tokamak plasmas is simultaneously an essential component of effective reactor design, and a great research barrier. Tokamak operational conditions exhibit comparatively low Knudsen numbers. Kinetic effects, including kinetic waves and instabilities, Landau damping, bump-on-tail instabilities and more, are therefore highly influential in tokamak plasma dynamics. Purely fluid models are inherently incapable of capturing these effects, whereas the high dimensionality in purely kinetic models render them practically intractable for most relevant purposes.

        We consider a $\delta\!f$ decomposition model, with a macroscopic fluid background and microscopic kinetic correction, both fully coupled to each other. A similar manner of discretization is proposed to that used in the recent \texttt{STRUPHY} code \cite{Holderied_Possanner_Wang_2021, Holderied_2022, Li_et_al_2023} with a finite-element model for the background and a pseudo-particle/PiC model for the correction.

        The fluid background satisfies the full, non-linear, resistive, compressible, Hall MHD equations. \cite{Laakmann_Hu_Farrell_2022} introduces finite-element(-in-space) implicit timesteppers for the incompressible analogue to this system with structure-preserving (SP) properties in the ideal case, alongside parameter-robust preconditioners. We show that these timesteppers can derive from a finite-element-in-time (FET) (and finite-element-in-space) interpretation. The benefits of this reformulation are discussed, including the derivation of timesteppers that are higher order in time, and the quantifiable dissipative SP properties in the non-ideal, resistive case.
        
        We discuss possible options for extending this FET approach to timesteppers for the compressible case.

        The kinetic corrections satisfy linearized Boltzmann equations. Using a Lénard--Bernstein collision operator, these take Fokker--Planck-like forms \cite{Fokker_1914, Planck_1917} wherein pseudo-particles in the numerical model obey the neoclassical transport equations, with particle-independent Brownian drift terms. This offers a rigorous methodology for incorporating collisions into the particle transport model, without coupling the equations of motions for each particle.
        
        Works by Chen, Chacón et al. \cite{Chen_Chacón_Barnes_2011, Chacón_Chen_Barnes_2013, Chen_Chacón_2014, Chen_Chacón_2015} have developed structure-preserving particle pushers for neoclassical transport in the Vlasov equations, derived from Crank--Nicolson integrators. We show these too can can derive from a FET interpretation, similarly offering potential extensions to higher-order-in-time particle pushers. The FET formulation is used also to consider how the stochastic drift terms can be incorporated into the pushers. Stochastic gyrokinetic expansions are also discussed.

        Different options for the numerical implementation of these schemes are considered.

        Due to the efficacy of FET in the development of SP timesteppers for both the fluid and kinetic component, we hope this approach will prove effective in the future for developing SP timesteppers for the full hybrid model. We hope this will give us the opportunity to incorporate previously inaccessible kinetic effects into the highly effective, modern, finite-element MHD models.
    \end{abstract}
    
    
    \newpage
    \tableofcontents
    
    
    \newpage
    \pagenumbering{arabic}
    %\linenumbers\renewcommand\thelinenumber{\color{black!50}\arabic{linenumber}}
            \documentclass[12pt, a4paper]{report}

\input{template/main.tex}

\title{\BA{Title in Progress...}}
\author{Boris Andrews}
\affil{Mathematical Institute, University of Oxford}
\date{\today}


\begin{document}
    \pagenumbering{gobble}
    \maketitle
    
    
    \begin{abstract}
        Magnetic confinement reactors---in particular tokamaks---offer one of the most promising options for achieving practical nuclear fusion, with the potential to provide virtually limitless, clean energy. The theoretical and numerical modeling of tokamak plasmas is simultaneously an essential component of effective reactor design, and a great research barrier. Tokamak operational conditions exhibit comparatively low Knudsen numbers. Kinetic effects, including kinetic waves and instabilities, Landau damping, bump-on-tail instabilities and more, are therefore highly influential in tokamak plasma dynamics. Purely fluid models are inherently incapable of capturing these effects, whereas the high dimensionality in purely kinetic models render them practically intractable for most relevant purposes.

        We consider a $\delta\!f$ decomposition model, with a macroscopic fluid background and microscopic kinetic correction, both fully coupled to each other. A similar manner of discretization is proposed to that used in the recent \texttt{STRUPHY} code \cite{Holderied_Possanner_Wang_2021, Holderied_2022, Li_et_al_2023} with a finite-element model for the background and a pseudo-particle/PiC model for the correction.

        The fluid background satisfies the full, non-linear, resistive, compressible, Hall MHD equations. \cite{Laakmann_Hu_Farrell_2022} introduces finite-element(-in-space) implicit timesteppers for the incompressible analogue to this system with structure-preserving (SP) properties in the ideal case, alongside parameter-robust preconditioners. We show that these timesteppers can derive from a finite-element-in-time (FET) (and finite-element-in-space) interpretation. The benefits of this reformulation are discussed, including the derivation of timesteppers that are higher order in time, and the quantifiable dissipative SP properties in the non-ideal, resistive case.
        
        We discuss possible options for extending this FET approach to timesteppers for the compressible case.

        The kinetic corrections satisfy linearized Boltzmann equations. Using a Lénard--Bernstein collision operator, these take Fokker--Planck-like forms \cite{Fokker_1914, Planck_1917} wherein pseudo-particles in the numerical model obey the neoclassical transport equations, with particle-independent Brownian drift terms. This offers a rigorous methodology for incorporating collisions into the particle transport model, without coupling the equations of motions for each particle.
        
        Works by Chen, Chacón et al. \cite{Chen_Chacón_Barnes_2011, Chacón_Chen_Barnes_2013, Chen_Chacón_2014, Chen_Chacón_2015} have developed structure-preserving particle pushers for neoclassical transport in the Vlasov equations, derived from Crank--Nicolson integrators. We show these too can can derive from a FET interpretation, similarly offering potential extensions to higher-order-in-time particle pushers. The FET formulation is used also to consider how the stochastic drift terms can be incorporated into the pushers. Stochastic gyrokinetic expansions are also discussed.

        Different options for the numerical implementation of these schemes are considered.

        Due to the efficacy of FET in the development of SP timesteppers for both the fluid and kinetic component, we hope this approach will prove effective in the future for developing SP timesteppers for the full hybrid model. We hope this will give us the opportunity to incorporate previously inaccessible kinetic effects into the highly effective, modern, finite-element MHD models.
    \end{abstract}
    
    
    \newpage
    \tableofcontents
    
    
    \newpage
    \pagenumbering{arabic}
    %\linenumbers\renewcommand\thelinenumber{\color{black!50}\arabic{linenumber}}
            \input{0 - introduction/main.tex}
        \part{Research}
            \input{1 - low-noise PiC models/main.tex}
            \input{2 - kinetic component/main.tex}
            \input{3 - fluid component/main.tex}
            \input{4 - numerical implementation/main.tex}
        \part{Project Overview}
            \input{5 - research plan/main.tex}
            \input{6 - summary/main.tex}
    
    
    %\section{}
    \newpage
    \pagenumbering{gobble}
        \printbibliography


    \newpage
    \pagenumbering{roman}
    \appendix
        \part{Appendices}
            \input{8 - Hilbert complexes/main.tex}
            \input{9 - weak conservation proofs/main.tex}
\end{document}

        \part{Research}
            \documentclass[12pt, a4paper]{report}

\input{template/main.tex}

\title{\BA{Title in Progress...}}
\author{Boris Andrews}
\affil{Mathematical Institute, University of Oxford}
\date{\today}


\begin{document}
    \pagenumbering{gobble}
    \maketitle
    
    
    \begin{abstract}
        Magnetic confinement reactors---in particular tokamaks---offer one of the most promising options for achieving practical nuclear fusion, with the potential to provide virtually limitless, clean energy. The theoretical and numerical modeling of tokamak plasmas is simultaneously an essential component of effective reactor design, and a great research barrier. Tokamak operational conditions exhibit comparatively low Knudsen numbers. Kinetic effects, including kinetic waves and instabilities, Landau damping, bump-on-tail instabilities and more, are therefore highly influential in tokamak plasma dynamics. Purely fluid models are inherently incapable of capturing these effects, whereas the high dimensionality in purely kinetic models render them practically intractable for most relevant purposes.

        We consider a $\delta\!f$ decomposition model, with a macroscopic fluid background and microscopic kinetic correction, both fully coupled to each other. A similar manner of discretization is proposed to that used in the recent \texttt{STRUPHY} code \cite{Holderied_Possanner_Wang_2021, Holderied_2022, Li_et_al_2023} with a finite-element model for the background and a pseudo-particle/PiC model for the correction.

        The fluid background satisfies the full, non-linear, resistive, compressible, Hall MHD equations. \cite{Laakmann_Hu_Farrell_2022} introduces finite-element(-in-space) implicit timesteppers for the incompressible analogue to this system with structure-preserving (SP) properties in the ideal case, alongside parameter-robust preconditioners. We show that these timesteppers can derive from a finite-element-in-time (FET) (and finite-element-in-space) interpretation. The benefits of this reformulation are discussed, including the derivation of timesteppers that are higher order in time, and the quantifiable dissipative SP properties in the non-ideal, resistive case.
        
        We discuss possible options for extending this FET approach to timesteppers for the compressible case.

        The kinetic corrections satisfy linearized Boltzmann equations. Using a Lénard--Bernstein collision operator, these take Fokker--Planck-like forms \cite{Fokker_1914, Planck_1917} wherein pseudo-particles in the numerical model obey the neoclassical transport equations, with particle-independent Brownian drift terms. This offers a rigorous methodology for incorporating collisions into the particle transport model, without coupling the equations of motions for each particle.
        
        Works by Chen, Chacón et al. \cite{Chen_Chacón_Barnes_2011, Chacón_Chen_Barnes_2013, Chen_Chacón_2014, Chen_Chacón_2015} have developed structure-preserving particle pushers for neoclassical transport in the Vlasov equations, derived from Crank--Nicolson integrators. We show these too can can derive from a FET interpretation, similarly offering potential extensions to higher-order-in-time particle pushers. The FET formulation is used also to consider how the stochastic drift terms can be incorporated into the pushers. Stochastic gyrokinetic expansions are also discussed.

        Different options for the numerical implementation of these schemes are considered.

        Due to the efficacy of FET in the development of SP timesteppers for both the fluid and kinetic component, we hope this approach will prove effective in the future for developing SP timesteppers for the full hybrid model. We hope this will give us the opportunity to incorporate previously inaccessible kinetic effects into the highly effective, modern, finite-element MHD models.
    \end{abstract}
    
    
    \newpage
    \tableofcontents
    
    
    \newpage
    \pagenumbering{arabic}
    %\linenumbers\renewcommand\thelinenumber{\color{black!50}\arabic{linenumber}}
            \input{0 - introduction/main.tex}
        \part{Research}
            \input{1 - low-noise PiC models/main.tex}
            \input{2 - kinetic component/main.tex}
            \input{3 - fluid component/main.tex}
            \input{4 - numerical implementation/main.tex}
        \part{Project Overview}
            \input{5 - research plan/main.tex}
            \input{6 - summary/main.tex}
    
    
    %\section{}
    \newpage
    \pagenumbering{gobble}
        \printbibliography


    \newpage
    \pagenumbering{roman}
    \appendix
        \part{Appendices}
            \input{8 - Hilbert complexes/main.tex}
            \input{9 - weak conservation proofs/main.tex}
\end{document}

            \documentclass[12pt, a4paper]{report}

\input{template/main.tex}

\title{\BA{Title in Progress...}}
\author{Boris Andrews}
\affil{Mathematical Institute, University of Oxford}
\date{\today}


\begin{document}
    \pagenumbering{gobble}
    \maketitle
    
    
    \begin{abstract}
        Magnetic confinement reactors---in particular tokamaks---offer one of the most promising options for achieving practical nuclear fusion, with the potential to provide virtually limitless, clean energy. The theoretical and numerical modeling of tokamak plasmas is simultaneously an essential component of effective reactor design, and a great research barrier. Tokamak operational conditions exhibit comparatively low Knudsen numbers. Kinetic effects, including kinetic waves and instabilities, Landau damping, bump-on-tail instabilities and more, are therefore highly influential in tokamak plasma dynamics. Purely fluid models are inherently incapable of capturing these effects, whereas the high dimensionality in purely kinetic models render them practically intractable for most relevant purposes.

        We consider a $\delta\!f$ decomposition model, with a macroscopic fluid background and microscopic kinetic correction, both fully coupled to each other. A similar manner of discretization is proposed to that used in the recent \texttt{STRUPHY} code \cite{Holderied_Possanner_Wang_2021, Holderied_2022, Li_et_al_2023} with a finite-element model for the background and a pseudo-particle/PiC model for the correction.

        The fluid background satisfies the full, non-linear, resistive, compressible, Hall MHD equations. \cite{Laakmann_Hu_Farrell_2022} introduces finite-element(-in-space) implicit timesteppers for the incompressible analogue to this system with structure-preserving (SP) properties in the ideal case, alongside parameter-robust preconditioners. We show that these timesteppers can derive from a finite-element-in-time (FET) (and finite-element-in-space) interpretation. The benefits of this reformulation are discussed, including the derivation of timesteppers that are higher order in time, and the quantifiable dissipative SP properties in the non-ideal, resistive case.
        
        We discuss possible options for extending this FET approach to timesteppers for the compressible case.

        The kinetic corrections satisfy linearized Boltzmann equations. Using a Lénard--Bernstein collision operator, these take Fokker--Planck-like forms \cite{Fokker_1914, Planck_1917} wherein pseudo-particles in the numerical model obey the neoclassical transport equations, with particle-independent Brownian drift terms. This offers a rigorous methodology for incorporating collisions into the particle transport model, without coupling the equations of motions for each particle.
        
        Works by Chen, Chacón et al. \cite{Chen_Chacón_Barnes_2011, Chacón_Chen_Barnes_2013, Chen_Chacón_2014, Chen_Chacón_2015} have developed structure-preserving particle pushers for neoclassical transport in the Vlasov equations, derived from Crank--Nicolson integrators. We show these too can can derive from a FET interpretation, similarly offering potential extensions to higher-order-in-time particle pushers. The FET formulation is used also to consider how the stochastic drift terms can be incorporated into the pushers. Stochastic gyrokinetic expansions are also discussed.

        Different options for the numerical implementation of these schemes are considered.

        Due to the efficacy of FET in the development of SP timesteppers for both the fluid and kinetic component, we hope this approach will prove effective in the future for developing SP timesteppers for the full hybrid model. We hope this will give us the opportunity to incorporate previously inaccessible kinetic effects into the highly effective, modern, finite-element MHD models.
    \end{abstract}
    
    
    \newpage
    \tableofcontents
    
    
    \newpage
    \pagenumbering{arabic}
    %\linenumbers\renewcommand\thelinenumber{\color{black!50}\arabic{linenumber}}
            \input{0 - introduction/main.tex}
        \part{Research}
            \input{1 - low-noise PiC models/main.tex}
            \input{2 - kinetic component/main.tex}
            \input{3 - fluid component/main.tex}
            \input{4 - numerical implementation/main.tex}
        \part{Project Overview}
            \input{5 - research plan/main.tex}
            \input{6 - summary/main.tex}
    
    
    %\section{}
    \newpage
    \pagenumbering{gobble}
        \printbibliography


    \newpage
    \pagenumbering{roman}
    \appendix
        \part{Appendices}
            \input{8 - Hilbert complexes/main.tex}
            \input{9 - weak conservation proofs/main.tex}
\end{document}

            \documentclass[12pt, a4paper]{report}

\input{template/main.tex}

\title{\BA{Title in Progress...}}
\author{Boris Andrews}
\affil{Mathematical Institute, University of Oxford}
\date{\today}


\begin{document}
    \pagenumbering{gobble}
    \maketitle
    
    
    \begin{abstract}
        Magnetic confinement reactors---in particular tokamaks---offer one of the most promising options for achieving practical nuclear fusion, with the potential to provide virtually limitless, clean energy. The theoretical and numerical modeling of tokamak plasmas is simultaneously an essential component of effective reactor design, and a great research barrier. Tokamak operational conditions exhibit comparatively low Knudsen numbers. Kinetic effects, including kinetic waves and instabilities, Landau damping, bump-on-tail instabilities and more, are therefore highly influential in tokamak plasma dynamics. Purely fluid models are inherently incapable of capturing these effects, whereas the high dimensionality in purely kinetic models render them practically intractable for most relevant purposes.

        We consider a $\delta\!f$ decomposition model, with a macroscopic fluid background and microscopic kinetic correction, both fully coupled to each other. A similar manner of discretization is proposed to that used in the recent \texttt{STRUPHY} code \cite{Holderied_Possanner_Wang_2021, Holderied_2022, Li_et_al_2023} with a finite-element model for the background and a pseudo-particle/PiC model for the correction.

        The fluid background satisfies the full, non-linear, resistive, compressible, Hall MHD equations. \cite{Laakmann_Hu_Farrell_2022} introduces finite-element(-in-space) implicit timesteppers for the incompressible analogue to this system with structure-preserving (SP) properties in the ideal case, alongside parameter-robust preconditioners. We show that these timesteppers can derive from a finite-element-in-time (FET) (and finite-element-in-space) interpretation. The benefits of this reformulation are discussed, including the derivation of timesteppers that are higher order in time, and the quantifiable dissipative SP properties in the non-ideal, resistive case.
        
        We discuss possible options for extending this FET approach to timesteppers for the compressible case.

        The kinetic corrections satisfy linearized Boltzmann equations. Using a Lénard--Bernstein collision operator, these take Fokker--Planck-like forms \cite{Fokker_1914, Planck_1917} wherein pseudo-particles in the numerical model obey the neoclassical transport equations, with particle-independent Brownian drift terms. This offers a rigorous methodology for incorporating collisions into the particle transport model, without coupling the equations of motions for each particle.
        
        Works by Chen, Chacón et al. \cite{Chen_Chacón_Barnes_2011, Chacón_Chen_Barnes_2013, Chen_Chacón_2014, Chen_Chacón_2015} have developed structure-preserving particle pushers for neoclassical transport in the Vlasov equations, derived from Crank--Nicolson integrators. We show these too can can derive from a FET interpretation, similarly offering potential extensions to higher-order-in-time particle pushers. The FET formulation is used also to consider how the stochastic drift terms can be incorporated into the pushers. Stochastic gyrokinetic expansions are also discussed.

        Different options for the numerical implementation of these schemes are considered.

        Due to the efficacy of FET in the development of SP timesteppers for both the fluid and kinetic component, we hope this approach will prove effective in the future for developing SP timesteppers for the full hybrid model. We hope this will give us the opportunity to incorporate previously inaccessible kinetic effects into the highly effective, modern, finite-element MHD models.
    \end{abstract}
    
    
    \newpage
    \tableofcontents
    
    
    \newpage
    \pagenumbering{arabic}
    %\linenumbers\renewcommand\thelinenumber{\color{black!50}\arabic{linenumber}}
            \input{0 - introduction/main.tex}
        \part{Research}
            \input{1 - low-noise PiC models/main.tex}
            \input{2 - kinetic component/main.tex}
            \input{3 - fluid component/main.tex}
            \input{4 - numerical implementation/main.tex}
        \part{Project Overview}
            \input{5 - research plan/main.tex}
            \input{6 - summary/main.tex}
    
    
    %\section{}
    \newpage
    \pagenumbering{gobble}
        \printbibliography


    \newpage
    \pagenumbering{roman}
    \appendix
        \part{Appendices}
            \input{8 - Hilbert complexes/main.tex}
            \input{9 - weak conservation proofs/main.tex}
\end{document}

            \documentclass[12pt, a4paper]{report}

\input{template/main.tex}

\title{\BA{Title in Progress...}}
\author{Boris Andrews}
\affil{Mathematical Institute, University of Oxford}
\date{\today}


\begin{document}
    \pagenumbering{gobble}
    \maketitle
    
    
    \begin{abstract}
        Magnetic confinement reactors---in particular tokamaks---offer one of the most promising options for achieving practical nuclear fusion, with the potential to provide virtually limitless, clean energy. The theoretical and numerical modeling of tokamak plasmas is simultaneously an essential component of effective reactor design, and a great research barrier. Tokamak operational conditions exhibit comparatively low Knudsen numbers. Kinetic effects, including kinetic waves and instabilities, Landau damping, bump-on-tail instabilities and more, are therefore highly influential in tokamak plasma dynamics. Purely fluid models are inherently incapable of capturing these effects, whereas the high dimensionality in purely kinetic models render them practically intractable for most relevant purposes.

        We consider a $\delta\!f$ decomposition model, with a macroscopic fluid background and microscopic kinetic correction, both fully coupled to each other. A similar manner of discretization is proposed to that used in the recent \texttt{STRUPHY} code \cite{Holderied_Possanner_Wang_2021, Holderied_2022, Li_et_al_2023} with a finite-element model for the background and a pseudo-particle/PiC model for the correction.

        The fluid background satisfies the full, non-linear, resistive, compressible, Hall MHD equations. \cite{Laakmann_Hu_Farrell_2022} introduces finite-element(-in-space) implicit timesteppers for the incompressible analogue to this system with structure-preserving (SP) properties in the ideal case, alongside parameter-robust preconditioners. We show that these timesteppers can derive from a finite-element-in-time (FET) (and finite-element-in-space) interpretation. The benefits of this reformulation are discussed, including the derivation of timesteppers that are higher order in time, and the quantifiable dissipative SP properties in the non-ideal, resistive case.
        
        We discuss possible options for extending this FET approach to timesteppers for the compressible case.

        The kinetic corrections satisfy linearized Boltzmann equations. Using a Lénard--Bernstein collision operator, these take Fokker--Planck-like forms \cite{Fokker_1914, Planck_1917} wherein pseudo-particles in the numerical model obey the neoclassical transport equations, with particle-independent Brownian drift terms. This offers a rigorous methodology for incorporating collisions into the particle transport model, without coupling the equations of motions for each particle.
        
        Works by Chen, Chacón et al. \cite{Chen_Chacón_Barnes_2011, Chacón_Chen_Barnes_2013, Chen_Chacón_2014, Chen_Chacón_2015} have developed structure-preserving particle pushers for neoclassical transport in the Vlasov equations, derived from Crank--Nicolson integrators. We show these too can can derive from a FET interpretation, similarly offering potential extensions to higher-order-in-time particle pushers. The FET formulation is used also to consider how the stochastic drift terms can be incorporated into the pushers. Stochastic gyrokinetic expansions are also discussed.

        Different options for the numerical implementation of these schemes are considered.

        Due to the efficacy of FET in the development of SP timesteppers for both the fluid and kinetic component, we hope this approach will prove effective in the future for developing SP timesteppers for the full hybrid model. We hope this will give us the opportunity to incorporate previously inaccessible kinetic effects into the highly effective, modern, finite-element MHD models.
    \end{abstract}
    
    
    \newpage
    \tableofcontents
    
    
    \newpage
    \pagenumbering{arabic}
    %\linenumbers\renewcommand\thelinenumber{\color{black!50}\arabic{linenumber}}
            \input{0 - introduction/main.tex}
        \part{Research}
            \input{1 - low-noise PiC models/main.tex}
            \input{2 - kinetic component/main.tex}
            \input{3 - fluid component/main.tex}
            \input{4 - numerical implementation/main.tex}
        \part{Project Overview}
            \input{5 - research plan/main.tex}
            \input{6 - summary/main.tex}
    
    
    %\section{}
    \newpage
    \pagenumbering{gobble}
        \printbibliography


    \newpage
    \pagenumbering{roman}
    \appendix
        \part{Appendices}
            \input{8 - Hilbert complexes/main.tex}
            \input{9 - weak conservation proofs/main.tex}
\end{document}

        \part{Project Overview}
            \documentclass[12pt, a4paper]{report}

\input{template/main.tex}

\title{\BA{Title in Progress...}}
\author{Boris Andrews}
\affil{Mathematical Institute, University of Oxford}
\date{\today}


\begin{document}
    \pagenumbering{gobble}
    \maketitle
    
    
    \begin{abstract}
        Magnetic confinement reactors---in particular tokamaks---offer one of the most promising options for achieving practical nuclear fusion, with the potential to provide virtually limitless, clean energy. The theoretical and numerical modeling of tokamak plasmas is simultaneously an essential component of effective reactor design, and a great research barrier. Tokamak operational conditions exhibit comparatively low Knudsen numbers. Kinetic effects, including kinetic waves and instabilities, Landau damping, bump-on-tail instabilities and more, are therefore highly influential in tokamak plasma dynamics. Purely fluid models are inherently incapable of capturing these effects, whereas the high dimensionality in purely kinetic models render them practically intractable for most relevant purposes.

        We consider a $\delta\!f$ decomposition model, with a macroscopic fluid background and microscopic kinetic correction, both fully coupled to each other. A similar manner of discretization is proposed to that used in the recent \texttt{STRUPHY} code \cite{Holderied_Possanner_Wang_2021, Holderied_2022, Li_et_al_2023} with a finite-element model for the background and a pseudo-particle/PiC model for the correction.

        The fluid background satisfies the full, non-linear, resistive, compressible, Hall MHD equations. \cite{Laakmann_Hu_Farrell_2022} introduces finite-element(-in-space) implicit timesteppers for the incompressible analogue to this system with structure-preserving (SP) properties in the ideal case, alongside parameter-robust preconditioners. We show that these timesteppers can derive from a finite-element-in-time (FET) (and finite-element-in-space) interpretation. The benefits of this reformulation are discussed, including the derivation of timesteppers that are higher order in time, and the quantifiable dissipative SP properties in the non-ideal, resistive case.
        
        We discuss possible options for extending this FET approach to timesteppers for the compressible case.

        The kinetic corrections satisfy linearized Boltzmann equations. Using a Lénard--Bernstein collision operator, these take Fokker--Planck-like forms \cite{Fokker_1914, Planck_1917} wherein pseudo-particles in the numerical model obey the neoclassical transport equations, with particle-independent Brownian drift terms. This offers a rigorous methodology for incorporating collisions into the particle transport model, without coupling the equations of motions for each particle.
        
        Works by Chen, Chacón et al. \cite{Chen_Chacón_Barnes_2011, Chacón_Chen_Barnes_2013, Chen_Chacón_2014, Chen_Chacón_2015} have developed structure-preserving particle pushers for neoclassical transport in the Vlasov equations, derived from Crank--Nicolson integrators. We show these too can can derive from a FET interpretation, similarly offering potential extensions to higher-order-in-time particle pushers. The FET formulation is used also to consider how the stochastic drift terms can be incorporated into the pushers. Stochastic gyrokinetic expansions are also discussed.

        Different options for the numerical implementation of these schemes are considered.

        Due to the efficacy of FET in the development of SP timesteppers for both the fluid and kinetic component, we hope this approach will prove effective in the future for developing SP timesteppers for the full hybrid model. We hope this will give us the opportunity to incorporate previously inaccessible kinetic effects into the highly effective, modern, finite-element MHD models.
    \end{abstract}
    
    
    \newpage
    \tableofcontents
    
    
    \newpage
    \pagenumbering{arabic}
    %\linenumbers\renewcommand\thelinenumber{\color{black!50}\arabic{linenumber}}
            \input{0 - introduction/main.tex}
        \part{Research}
            \input{1 - low-noise PiC models/main.tex}
            \input{2 - kinetic component/main.tex}
            \input{3 - fluid component/main.tex}
            \input{4 - numerical implementation/main.tex}
        \part{Project Overview}
            \input{5 - research plan/main.tex}
            \input{6 - summary/main.tex}
    
    
    %\section{}
    \newpage
    \pagenumbering{gobble}
        \printbibliography


    \newpage
    \pagenumbering{roman}
    \appendix
        \part{Appendices}
            \input{8 - Hilbert complexes/main.tex}
            \input{9 - weak conservation proofs/main.tex}
\end{document}

            \documentclass[12pt, a4paper]{report}

\input{template/main.tex}

\title{\BA{Title in Progress...}}
\author{Boris Andrews}
\affil{Mathematical Institute, University of Oxford}
\date{\today}


\begin{document}
    \pagenumbering{gobble}
    \maketitle
    
    
    \begin{abstract}
        Magnetic confinement reactors---in particular tokamaks---offer one of the most promising options for achieving practical nuclear fusion, with the potential to provide virtually limitless, clean energy. The theoretical and numerical modeling of tokamak plasmas is simultaneously an essential component of effective reactor design, and a great research barrier. Tokamak operational conditions exhibit comparatively low Knudsen numbers. Kinetic effects, including kinetic waves and instabilities, Landau damping, bump-on-tail instabilities and more, are therefore highly influential in tokamak plasma dynamics. Purely fluid models are inherently incapable of capturing these effects, whereas the high dimensionality in purely kinetic models render them practically intractable for most relevant purposes.

        We consider a $\delta\!f$ decomposition model, with a macroscopic fluid background and microscopic kinetic correction, both fully coupled to each other. A similar manner of discretization is proposed to that used in the recent \texttt{STRUPHY} code \cite{Holderied_Possanner_Wang_2021, Holderied_2022, Li_et_al_2023} with a finite-element model for the background and a pseudo-particle/PiC model for the correction.

        The fluid background satisfies the full, non-linear, resistive, compressible, Hall MHD equations. \cite{Laakmann_Hu_Farrell_2022} introduces finite-element(-in-space) implicit timesteppers for the incompressible analogue to this system with structure-preserving (SP) properties in the ideal case, alongside parameter-robust preconditioners. We show that these timesteppers can derive from a finite-element-in-time (FET) (and finite-element-in-space) interpretation. The benefits of this reformulation are discussed, including the derivation of timesteppers that are higher order in time, and the quantifiable dissipative SP properties in the non-ideal, resistive case.
        
        We discuss possible options for extending this FET approach to timesteppers for the compressible case.

        The kinetic corrections satisfy linearized Boltzmann equations. Using a Lénard--Bernstein collision operator, these take Fokker--Planck-like forms \cite{Fokker_1914, Planck_1917} wherein pseudo-particles in the numerical model obey the neoclassical transport equations, with particle-independent Brownian drift terms. This offers a rigorous methodology for incorporating collisions into the particle transport model, without coupling the equations of motions for each particle.
        
        Works by Chen, Chacón et al. \cite{Chen_Chacón_Barnes_2011, Chacón_Chen_Barnes_2013, Chen_Chacón_2014, Chen_Chacón_2015} have developed structure-preserving particle pushers for neoclassical transport in the Vlasov equations, derived from Crank--Nicolson integrators. We show these too can can derive from a FET interpretation, similarly offering potential extensions to higher-order-in-time particle pushers. The FET formulation is used also to consider how the stochastic drift terms can be incorporated into the pushers. Stochastic gyrokinetic expansions are also discussed.

        Different options for the numerical implementation of these schemes are considered.

        Due to the efficacy of FET in the development of SP timesteppers for both the fluid and kinetic component, we hope this approach will prove effective in the future for developing SP timesteppers for the full hybrid model. We hope this will give us the opportunity to incorporate previously inaccessible kinetic effects into the highly effective, modern, finite-element MHD models.
    \end{abstract}
    
    
    \newpage
    \tableofcontents
    
    
    \newpage
    \pagenumbering{arabic}
    %\linenumbers\renewcommand\thelinenumber{\color{black!50}\arabic{linenumber}}
            \input{0 - introduction/main.tex}
        \part{Research}
            \input{1 - low-noise PiC models/main.tex}
            \input{2 - kinetic component/main.tex}
            \input{3 - fluid component/main.tex}
            \input{4 - numerical implementation/main.tex}
        \part{Project Overview}
            \input{5 - research plan/main.tex}
            \input{6 - summary/main.tex}
    
    
    %\section{}
    \newpage
    \pagenumbering{gobble}
        \printbibliography


    \newpage
    \pagenumbering{roman}
    \appendix
        \part{Appendices}
            \input{8 - Hilbert complexes/main.tex}
            \input{9 - weak conservation proofs/main.tex}
\end{document}

    
    
    %\section{}
    \newpage
    \pagenumbering{gobble}
        \printbibliography


    \newpage
    \pagenumbering{roman}
    \appendix
        \part{Appendices}
            \documentclass[12pt, a4paper]{report}

\input{template/main.tex}

\title{\BA{Title in Progress...}}
\author{Boris Andrews}
\affil{Mathematical Institute, University of Oxford}
\date{\today}


\begin{document}
    \pagenumbering{gobble}
    \maketitle
    
    
    \begin{abstract}
        Magnetic confinement reactors---in particular tokamaks---offer one of the most promising options for achieving practical nuclear fusion, with the potential to provide virtually limitless, clean energy. The theoretical and numerical modeling of tokamak plasmas is simultaneously an essential component of effective reactor design, and a great research barrier. Tokamak operational conditions exhibit comparatively low Knudsen numbers. Kinetic effects, including kinetic waves and instabilities, Landau damping, bump-on-tail instabilities and more, are therefore highly influential in tokamak plasma dynamics. Purely fluid models are inherently incapable of capturing these effects, whereas the high dimensionality in purely kinetic models render them practically intractable for most relevant purposes.

        We consider a $\delta\!f$ decomposition model, with a macroscopic fluid background and microscopic kinetic correction, both fully coupled to each other. A similar manner of discretization is proposed to that used in the recent \texttt{STRUPHY} code \cite{Holderied_Possanner_Wang_2021, Holderied_2022, Li_et_al_2023} with a finite-element model for the background and a pseudo-particle/PiC model for the correction.

        The fluid background satisfies the full, non-linear, resistive, compressible, Hall MHD equations. \cite{Laakmann_Hu_Farrell_2022} introduces finite-element(-in-space) implicit timesteppers for the incompressible analogue to this system with structure-preserving (SP) properties in the ideal case, alongside parameter-robust preconditioners. We show that these timesteppers can derive from a finite-element-in-time (FET) (and finite-element-in-space) interpretation. The benefits of this reformulation are discussed, including the derivation of timesteppers that are higher order in time, and the quantifiable dissipative SP properties in the non-ideal, resistive case.
        
        We discuss possible options for extending this FET approach to timesteppers for the compressible case.

        The kinetic corrections satisfy linearized Boltzmann equations. Using a Lénard--Bernstein collision operator, these take Fokker--Planck-like forms \cite{Fokker_1914, Planck_1917} wherein pseudo-particles in the numerical model obey the neoclassical transport equations, with particle-independent Brownian drift terms. This offers a rigorous methodology for incorporating collisions into the particle transport model, without coupling the equations of motions for each particle.
        
        Works by Chen, Chacón et al. \cite{Chen_Chacón_Barnes_2011, Chacón_Chen_Barnes_2013, Chen_Chacón_2014, Chen_Chacón_2015} have developed structure-preserving particle pushers for neoclassical transport in the Vlasov equations, derived from Crank--Nicolson integrators. We show these too can can derive from a FET interpretation, similarly offering potential extensions to higher-order-in-time particle pushers. The FET formulation is used also to consider how the stochastic drift terms can be incorporated into the pushers. Stochastic gyrokinetic expansions are also discussed.

        Different options for the numerical implementation of these schemes are considered.

        Due to the efficacy of FET in the development of SP timesteppers for both the fluid and kinetic component, we hope this approach will prove effective in the future for developing SP timesteppers for the full hybrid model. We hope this will give us the opportunity to incorporate previously inaccessible kinetic effects into the highly effective, modern, finite-element MHD models.
    \end{abstract}
    
    
    \newpage
    \tableofcontents
    
    
    \newpage
    \pagenumbering{arabic}
    %\linenumbers\renewcommand\thelinenumber{\color{black!50}\arabic{linenumber}}
            \input{0 - introduction/main.tex}
        \part{Research}
            \input{1 - low-noise PiC models/main.tex}
            \input{2 - kinetic component/main.tex}
            \input{3 - fluid component/main.tex}
            \input{4 - numerical implementation/main.tex}
        \part{Project Overview}
            \input{5 - research plan/main.tex}
            \input{6 - summary/main.tex}
    
    
    %\section{}
    \newpage
    \pagenumbering{gobble}
        \printbibliography


    \newpage
    \pagenumbering{roman}
    \appendix
        \part{Appendices}
            \input{8 - Hilbert complexes/main.tex}
            \input{9 - weak conservation proofs/main.tex}
\end{document}

            \documentclass[12pt, a4paper]{report}

\input{template/main.tex}

\title{\BA{Title in Progress...}}
\author{Boris Andrews}
\affil{Mathematical Institute, University of Oxford}
\date{\today}


\begin{document}
    \pagenumbering{gobble}
    \maketitle
    
    
    \begin{abstract}
        Magnetic confinement reactors---in particular tokamaks---offer one of the most promising options for achieving practical nuclear fusion, with the potential to provide virtually limitless, clean energy. The theoretical and numerical modeling of tokamak plasmas is simultaneously an essential component of effective reactor design, and a great research barrier. Tokamak operational conditions exhibit comparatively low Knudsen numbers. Kinetic effects, including kinetic waves and instabilities, Landau damping, bump-on-tail instabilities and more, are therefore highly influential in tokamak plasma dynamics. Purely fluid models are inherently incapable of capturing these effects, whereas the high dimensionality in purely kinetic models render them practically intractable for most relevant purposes.

        We consider a $\delta\!f$ decomposition model, with a macroscopic fluid background and microscopic kinetic correction, both fully coupled to each other. A similar manner of discretization is proposed to that used in the recent \texttt{STRUPHY} code \cite{Holderied_Possanner_Wang_2021, Holderied_2022, Li_et_al_2023} with a finite-element model for the background and a pseudo-particle/PiC model for the correction.

        The fluid background satisfies the full, non-linear, resistive, compressible, Hall MHD equations. \cite{Laakmann_Hu_Farrell_2022} introduces finite-element(-in-space) implicit timesteppers for the incompressible analogue to this system with structure-preserving (SP) properties in the ideal case, alongside parameter-robust preconditioners. We show that these timesteppers can derive from a finite-element-in-time (FET) (and finite-element-in-space) interpretation. The benefits of this reformulation are discussed, including the derivation of timesteppers that are higher order in time, and the quantifiable dissipative SP properties in the non-ideal, resistive case.
        
        We discuss possible options for extending this FET approach to timesteppers for the compressible case.

        The kinetic corrections satisfy linearized Boltzmann equations. Using a Lénard--Bernstein collision operator, these take Fokker--Planck-like forms \cite{Fokker_1914, Planck_1917} wherein pseudo-particles in the numerical model obey the neoclassical transport equations, with particle-independent Brownian drift terms. This offers a rigorous methodology for incorporating collisions into the particle transport model, without coupling the equations of motions for each particle.
        
        Works by Chen, Chacón et al. \cite{Chen_Chacón_Barnes_2011, Chacón_Chen_Barnes_2013, Chen_Chacón_2014, Chen_Chacón_2015} have developed structure-preserving particle pushers for neoclassical transport in the Vlasov equations, derived from Crank--Nicolson integrators. We show these too can can derive from a FET interpretation, similarly offering potential extensions to higher-order-in-time particle pushers. The FET formulation is used also to consider how the stochastic drift terms can be incorporated into the pushers. Stochastic gyrokinetic expansions are also discussed.

        Different options for the numerical implementation of these schemes are considered.

        Due to the efficacy of FET in the development of SP timesteppers for both the fluid and kinetic component, we hope this approach will prove effective in the future for developing SP timesteppers for the full hybrid model. We hope this will give us the opportunity to incorporate previously inaccessible kinetic effects into the highly effective, modern, finite-element MHD models.
    \end{abstract}
    
    
    \newpage
    \tableofcontents
    
    
    \newpage
    \pagenumbering{arabic}
    %\linenumbers\renewcommand\thelinenumber{\color{black!50}\arabic{linenumber}}
            \input{0 - introduction/main.tex}
        \part{Research}
            \input{1 - low-noise PiC models/main.tex}
            \input{2 - kinetic component/main.tex}
            \input{3 - fluid component/main.tex}
            \input{4 - numerical implementation/main.tex}
        \part{Project Overview}
            \input{5 - research plan/main.tex}
            \input{6 - summary/main.tex}
    
    
    %\section{}
    \newpage
    \pagenumbering{gobble}
        \printbibliography


    \newpage
    \pagenumbering{roman}
    \appendix
        \part{Appendices}
            \input{8 - Hilbert complexes/main.tex}
            \input{9 - weak conservation proofs/main.tex}
\end{document}

\end{document}

            \documentclass[12pt, a4paper]{report}

\documentclass[12pt, a4paper]{report}

\input{template/main.tex}

\title{\BA{Title in Progress...}}
\author{Boris Andrews}
\affil{Mathematical Institute, University of Oxford}
\date{\today}


\begin{document}
    \pagenumbering{gobble}
    \maketitle
    
    
    \begin{abstract}
        Magnetic confinement reactors---in particular tokamaks---offer one of the most promising options for achieving practical nuclear fusion, with the potential to provide virtually limitless, clean energy. The theoretical and numerical modeling of tokamak plasmas is simultaneously an essential component of effective reactor design, and a great research barrier. Tokamak operational conditions exhibit comparatively low Knudsen numbers. Kinetic effects, including kinetic waves and instabilities, Landau damping, bump-on-tail instabilities and more, are therefore highly influential in tokamak plasma dynamics. Purely fluid models are inherently incapable of capturing these effects, whereas the high dimensionality in purely kinetic models render them practically intractable for most relevant purposes.

        We consider a $\delta\!f$ decomposition model, with a macroscopic fluid background and microscopic kinetic correction, both fully coupled to each other. A similar manner of discretization is proposed to that used in the recent \texttt{STRUPHY} code \cite{Holderied_Possanner_Wang_2021, Holderied_2022, Li_et_al_2023} with a finite-element model for the background and a pseudo-particle/PiC model for the correction.

        The fluid background satisfies the full, non-linear, resistive, compressible, Hall MHD equations. \cite{Laakmann_Hu_Farrell_2022} introduces finite-element(-in-space) implicit timesteppers for the incompressible analogue to this system with structure-preserving (SP) properties in the ideal case, alongside parameter-robust preconditioners. We show that these timesteppers can derive from a finite-element-in-time (FET) (and finite-element-in-space) interpretation. The benefits of this reformulation are discussed, including the derivation of timesteppers that are higher order in time, and the quantifiable dissipative SP properties in the non-ideal, resistive case.
        
        We discuss possible options for extending this FET approach to timesteppers for the compressible case.

        The kinetic corrections satisfy linearized Boltzmann equations. Using a Lénard--Bernstein collision operator, these take Fokker--Planck-like forms \cite{Fokker_1914, Planck_1917} wherein pseudo-particles in the numerical model obey the neoclassical transport equations, with particle-independent Brownian drift terms. This offers a rigorous methodology for incorporating collisions into the particle transport model, without coupling the equations of motions for each particle.
        
        Works by Chen, Chacón et al. \cite{Chen_Chacón_Barnes_2011, Chacón_Chen_Barnes_2013, Chen_Chacón_2014, Chen_Chacón_2015} have developed structure-preserving particle pushers for neoclassical transport in the Vlasov equations, derived from Crank--Nicolson integrators. We show these too can can derive from a FET interpretation, similarly offering potential extensions to higher-order-in-time particle pushers. The FET formulation is used also to consider how the stochastic drift terms can be incorporated into the pushers. Stochastic gyrokinetic expansions are also discussed.

        Different options for the numerical implementation of these schemes are considered.

        Due to the efficacy of FET in the development of SP timesteppers for both the fluid and kinetic component, we hope this approach will prove effective in the future for developing SP timesteppers for the full hybrid model. We hope this will give us the opportunity to incorporate previously inaccessible kinetic effects into the highly effective, modern, finite-element MHD models.
    \end{abstract}
    
    
    \newpage
    \tableofcontents
    
    
    \newpage
    \pagenumbering{arabic}
    %\linenumbers\renewcommand\thelinenumber{\color{black!50}\arabic{linenumber}}
            \input{0 - introduction/main.tex}
        \part{Research}
            \input{1 - low-noise PiC models/main.tex}
            \input{2 - kinetic component/main.tex}
            \input{3 - fluid component/main.tex}
            \input{4 - numerical implementation/main.tex}
        \part{Project Overview}
            \input{5 - research plan/main.tex}
            \input{6 - summary/main.tex}
    
    
    %\section{}
    \newpage
    \pagenumbering{gobble}
        \printbibliography


    \newpage
    \pagenumbering{roman}
    \appendix
        \part{Appendices}
            \input{8 - Hilbert complexes/main.tex}
            \input{9 - weak conservation proofs/main.tex}
\end{document}


\title{\BA{Title in Progress...}}
\author{Boris Andrews}
\affil{Mathematical Institute, University of Oxford}
\date{\today}


\begin{document}
    \pagenumbering{gobble}
    \maketitle
    
    
    \begin{abstract}
        Magnetic confinement reactors---in particular tokamaks---offer one of the most promising options for achieving practical nuclear fusion, with the potential to provide virtually limitless, clean energy. The theoretical and numerical modeling of tokamak plasmas is simultaneously an essential component of effective reactor design, and a great research barrier. Tokamak operational conditions exhibit comparatively low Knudsen numbers. Kinetic effects, including kinetic waves and instabilities, Landau damping, bump-on-tail instabilities and more, are therefore highly influential in tokamak plasma dynamics. Purely fluid models are inherently incapable of capturing these effects, whereas the high dimensionality in purely kinetic models render them practically intractable for most relevant purposes.

        We consider a $\delta\!f$ decomposition model, with a macroscopic fluid background and microscopic kinetic correction, both fully coupled to each other. A similar manner of discretization is proposed to that used in the recent \texttt{STRUPHY} code \cite{Holderied_Possanner_Wang_2021, Holderied_2022, Li_et_al_2023} with a finite-element model for the background and a pseudo-particle/PiC model for the correction.

        The fluid background satisfies the full, non-linear, resistive, compressible, Hall MHD equations. \cite{Laakmann_Hu_Farrell_2022} introduces finite-element(-in-space) implicit timesteppers for the incompressible analogue to this system with structure-preserving (SP) properties in the ideal case, alongside parameter-robust preconditioners. We show that these timesteppers can derive from a finite-element-in-time (FET) (and finite-element-in-space) interpretation. The benefits of this reformulation are discussed, including the derivation of timesteppers that are higher order in time, and the quantifiable dissipative SP properties in the non-ideal, resistive case.
        
        We discuss possible options for extending this FET approach to timesteppers for the compressible case.

        The kinetic corrections satisfy linearized Boltzmann equations. Using a Lénard--Bernstein collision operator, these take Fokker--Planck-like forms \cite{Fokker_1914, Planck_1917} wherein pseudo-particles in the numerical model obey the neoclassical transport equations, with particle-independent Brownian drift terms. This offers a rigorous methodology for incorporating collisions into the particle transport model, without coupling the equations of motions for each particle.
        
        Works by Chen, Chacón et al. \cite{Chen_Chacón_Barnes_2011, Chacón_Chen_Barnes_2013, Chen_Chacón_2014, Chen_Chacón_2015} have developed structure-preserving particle pushers for neoclassical transport in the Vlasov equations, derived from Crank--Nicolson integrators. We show these too can can derive from a FET interpretation, similarly offering potential extensions to higher-order-in-time particle pushers. The FET formulation is used also to consider how the stochastic drift terms can be incorporated into the pushers. Stochastic gyrokinetic expansions are also discussed.

        Different options for the numerical implementation of these schemes are considered.

        Due to the efficacy of FET in the development of SP timesteppers for both the fluid and kinetic component, we hope this approach will prove effective in the future for developing SP timesteppers for the full hybrid model. We hope this will give us the opportunity to incorporate previously inaccessible kinetic effects into the highly effective, modern, finite-element MHD models.
    \end{abstract}
    
    
    \newpage
    \tableofcontents
    
    
    \newpage
    \pagenumbering{arabic}
    %\linenumbers\renewcommand\thelinenumber{\color{black!50}\arabic{linenumber}}
            \documentclass[12pt, a4paper]{report}

\input{template/main.tex}

\title{\BA{Title in Progress...}}
\author{Boris Andrews}
\affil{Mathematical Institute, University of Oxford}
\date{\today}


\begin{document}
    \pagenumbering{gobble}
    \maketitle
    
    
    \begin{abstract}
        Magnetic confinement reactors---in particular tokamaks---offer one of the most promising options for achieving practical nuclear fusion, with the potential to provide virtually limitless, clean energy. The theoretical and numerical modeling of tokamak plasmas is simultaneously an essential component of effective reactor design, and a great research barrier. Tokamak operational conditions exhibit comparatively low Knudsen numbers. Kinetic effects, including kinetic waves and instabilities, Landau damping, bump-on-tail instabilities and more, are therefore highly influential in tokamak plasma dynamics. Purely fluid models are inherently incapable of capturing these effects, whereas the high dimensionality in purely kinetic models render them practically intractable for most relevant purposes.

        We consider a $\delta\!f$ decomposition model, with a macroscopic fluid background and microscopic kinetic correction, both fully coupled to each other. A similar manner of discretization is proposed to that used in the recent \texttt{STRUPHY} code \cite{Holderied_Possanner_Wang_2021, Holderied_2022, Li_et_al_2023} with a finite-element model for the background and a pseudo-particle/PiC model for the correction.

        The fluid background satisfies the full, non-linear, resistive, compressible, Hall MHD equations. \cite{Laakmann_Hu_Farrell_2022} introduces finite-element(-in-space) implicit timesteppers for the incompressible analogue to this system with structure-preserving (SP) properties in the ideal case, alongside parameter-robust preconditioners. We show that these timesteppers can derive from a finite-element-in-time (FET) (and finite-element-in-space) interpretation. The benefits of this reformulation are discussed, including the derivation of timesteppers that are higher order in time, and the quantifiable dissipative SP properties in the non-ideal, resistive case.
        
        We discuss possible options for extending this FET approach to timesteppers for the compressible case.

        The kinetic corrections satisfy linearized Boltzmann equations. Using a Lénard--Bernstein collision operator, these take Fokker--Planck-like forms \cite{Fokker_1914, Planck_1917} wherein pseudo-particles in the numerical model obey the neoclassical transport equations, with particle-independent Brownian drift terms. This offers a rigorous methodology for incorporating collisions into the particle transport model, without coupling the equations of motions for each particle.
        
        Works by Chen, Chacón et al. \cite{Chen_Chacón_Barnes_2011, Chacón_Chen_Barnes_2013, Chen_Chacón_2014, Chen_Chacón_2015} have developed structure-preserving particle pushers for neoclassical transport in the Vlasov equations, derived from Crank--Nicolson integrators. We show these too can can derive from a FET interpretation, similarly offering potential extensions to higher-order-in-time particle pushers. The FET formulation is used also to consider how the stochastic drift terms can be incorporated into the pushers. Stochastic gyrokinetic expansions are also discussed.

        Different options for the numerical implementation of these schemes are considered.

        Due to the efficacy of FET in the development of SP timesteppers for both the fluid and kinetic component, we hope this approach will prove effective in the future for developing SP timesteppers for the full hybrid model. We hope this will give us the opportunity to incorporate previously inaccessible kinetic effects into the highly effective, modern, finite-element MHD models.
    \end{abstract}
    
    
    \newpage
    \tableofcontents
    
    
    \newpage
    \pagenumbering{arabic}
    %\linenumbers\renewcommand\thelinenumber{\color{black!50}\arabic{linenumber}}
            \input{0 - introduction/main.tex}
        \part{Research}
            \input{1 - low-noise PiC models/main.tex}
            \input{2 - kinetic component/main.tex}
            \input{3 - fluid component/main.tex}
            \input{4 - numerical implementation/main.tex}
        \part{Project Overview}
            \input{5 - research plan/main.tex}
            \input{6 - summary/main.tex}
    
    
    %\section{}
    \newpage
    \pagenumbering{gobble}
        \printbibliography


    \newpage
    \pagenumbering{roman}
    \appendix
        \part{Appendices}
            \input{8 - Hilbert complexes/main.tex}
            \input{9 - weak conservation proofs/main.tex}
\end{document}

        \part{Research}
            \documentclass[12pt, a4paper]{report}

\input{template/main.tex}

\title{\BA{Title in Progress...}}
\author{Boris Andrews}
\affil{Mathematical Institute, University of Oxford}
\date{\today}


\begin{document}
    \pagenumbering{gobble}
    \maketitle
    
    
    \begin{abstract}
        Magnetic confinement reactors---in particular tokamaks---offer one of the most promising options for achieving practical nuclear fusion, with the potential to provide virtually limitless, clean energy. The theoretical and numerical modeling of tokamak plasmas is simultaneously an essential component of effective reactor design, and a great research barrier. Tokamak operational conditions exhibit comparatively low Knudsen numbers. Kinetic effects, including kinetic waves and instabilities, Landau damping, bump-on-tail instabilities and more, are therefore highly influential in tokamak plasma dynamics. Purely fluid models are inherently incapable of capturing these effects, whereas the high dimensionality in purely kinetic models render them practically intractable for most relevant purposes.

        We consider a $\delta\!f$ decomposition model, with a macroscopic fluid background and microscopic kinetic correction, both fully coupled to each other. A similar manner of discretization is proposed to that used in the recent \texttt{STRUPHY} code \cite{Holderied_Possanner_Wang_2021, Holderied_2022, Li_et_al_2023} with a finite-element model for the background and a pseudo-particle/PiC model for the correction.

        The fluid background satisfies the full, non-linear, resistive, compressible, Hall MHD equations. \cite{Laakmann_Hu_Farrell_2022} introduces finite-element(-in-space) implicit timesteppers for the incompressible analogue to this system with structure-preserving (SP) properties in the ideal case, alongside parameter-robust preconditioners. We show that these timesteppers can derive from a finite-element-in-time (FET) (and finite-element-in-space) interpretation. The benefits of this reformulation are discussed, including the derivation of timesteppers that are higher order in time, and the quantifiable dissipative SP properties in the non-ideal, resistive case.
        
        We discuss possible options for extending this FET approach to timesteppers for the compressible case.

        The kinetic corrections satisfy linearized Boltzmann equations. Using a Lénard--Bernstein collision operator, these take Fokker--Planck-like forms \cite{Fokker_1914, Planck_1917} wherein pseudo-particles in the numerical model obey the neoclassical transport equations, with particle-independent Brownian drift terms. This offers a rigorous methodology for incorporating collisions into the particle transport model, without coupling the equations of motions for each particle.
        
        Works by Chen, Chacón et al. \cite{Chen_Chacón_Barnes_2011, Chacón_Chen_Barnes_2013, Chen_Chacón_2014, Chen_Chacón_2015} have developed structure-preserving particle pushers for neoclassical transport in the Vlasov equations, derived from Crank--Nicolson integrators. We show these too can can derive from a FET interpretation, similarly offering potential extensions to higher-order-in-time particle pushers. The FET formulation is used also to consider how the stochastic drift terms can be incorporated into the pushers. Stochastic gyrokinetic expansions are also discussed.

        Different options for the numerical implementation of these schemes are considered.

        Due to the efficacy of FET in the development of SP timesteppers for both the fluid and kinetic component, we hope this approach will prove effective in the future for developing SP timesteppers for the full hybrid model. We hope this will give us the opportunity to incorporate previously inaccessible kinetic effects into the highly effective, modern, finite-element MHD models.
    \end{abstract}
    
    
    \newpage
    \tableofcontents
    
    
    \newpage
    \pagenumbering{arabic}
    %\linenumbers\renewcommand\thelinenumber{\color{black!50}\arabic{linenumber}}
            \input{0 - introduction/main.tex}
        \part{Research}
            \input{1 - low-noise PiC models/main.tex}
            \input{2 - kinetic component/main.tex}
            \input{3 - fluid component/main.tex}
            \input{4 - numerical implementation/main.tex}
        \part{Project Overview}
            \input{5 - research plan/main.tex}
            \input{6 - summary/main.tex}
    
    
    %\section{}
    \newpage
    \pagenumbering{gobble}
        \printbibliography


    \newpage
    \pagenumbering{roman}
    \appendix
        \part{Appendices}
            \input{8 - Hilbert complexes/main.tex}
            \input{9 - weak conservation proofs/main.tex}
\end{document}

            \documentclass[12pt, a4paper]{report}

\input{template/main.tex}

\title{\BA{Title in Progress...}}
\author{Boris Andrews}
\affil{Mathematical Institute, University of Oxford}
\date{\today}


\begin{document}
    \pagenumbering{gobble}
    \maketitle
    
    
    \begin{abstract}
        Magnetic confinement reactors---in particular tokamaks---offer one of the most promising options for achieving practical nuclear fusion, with the potential to provide virtually limitless, clean energy. The theoretical and numerical modeling of tokamak plasmas is simultaneously an essential component of effective reactor design, and a great research barrier. Tokamak operational conditions exhibit comparatively low Knudsen numbers. Kinetic effects, including kinetic waves and instabilities, Landau damping, bump-on-tail instabilities and more, are therefore highly influential in tokamak plasma dynamics. Purely fluid models are inherently incapable of capturing these effects, whereas the high dimensionality in purely kinetic models render them practically intractable for most relevant purposes.

        We consider a $\delta\!f$ decomposition model, with a macroscopic fluid background and microscopic kinetic correction, both fully coupled to each other. A similar manner of discretization is proposed to that used in the recent \texttt{STRUPHY} code \cite{Holderied_Possanner_Wang_2021, Holderied_2022, Li_et_al_2023} with a finite-element model for the background and a pseudo-particle/PiC model for the correction.

        The fluid background satisfies the full, non-linear, resistive, compressible, Hall MHD equations. \cite{Laakmann_Hu_Farrell_2022} introduces finite-element(-in-space) implicit timesteppers for the incompressible analogue to this system with structure-preserving (SP) properties in the ideal case, alongside parameter-robust preconditioners. We show that these timesteppers can derive from a finite-element-in-time (FET) (and finite-element-in-space) interpretation. The benefits of this reformulation are discussed, including the derivation of timesteppers that are higher order in time, and the quantifiable dissipative SP properties in the non-ideal, resistive case.
        
        We discuss possible options for extending this FET approach to timesteppers for the compressible case.

        The kinetic corrections satisfy linearized Boltzmann equations. Using a Lénard--Bernstein collision operator, these take Fokker--Planck-like forms \cite{Fokker_1914, Planck_1917} wherein pseudo-particles in the numerical model obey the neoclassical transport equations, with particle-independent Brownian drift terms. This offers a rigorous methodology for incorporating collisions into the particle transport model, without coupling the equations of motions for each particle.
        
        Works by Chen, Chacón et al. \cite{Chen_Chacón_Barnes_2011, Chacón_Chen_Barnes_2013, Chen_Chacón_2014, Chen_Chacón_2015} have developed structure-preserving particle pushers for neoclassical transport in the Vlasov equations, derived from Crank--Nicolson integrators. We show these too can can derive from a FET interpretation, similarly offering potential extensions to higher-order-in-time particle pushers. The FET formulation is used also to consider how the stochastic drift terms can be incorporated into the pushers. Stochastic gyrokinetic expansions are also discussed.

        Different options for the numerical implementation of these schemes are considered.

        Due to the efficacy of FET in the development of SP timesteppers for both the fluid and kinetic component, we hope this approach will prove effective in the future for developing SP timesteppers for the full hybrid model. We hope this will give us the opportunity to incorporate previously inaccessible kinetic effects into the highly effective, modern, finite-element MHD models.
    \end{abstract}
    
    
    \newpage
    \tableofcontents
    
    
    \newpage
    \pagenumbering{arabic}
    %\linenumbers\renewcommand\thelinenumber{\color{black!50}\arabic{linenumber}}
            \input{0 - introduction/main.tex}
        \part{Research}
            \input{1 - low-noise PiC models/main.tex}
            \input{2 - kinetic component/main.tex}
            \input{3 - fluid component/main.tex}
            \input{4 - numerical implementation/main.tex}
        \part{Project Overview}
            \input{5 - research plan/main.tex}
            \input{6 - summary/main.tex}
    
    
    %\section{}
    \newpage
    \pagenumbering{gobble}
        \printbibliography


    \newpage
    \pagenumbering{roman}
    \appendix
        \part{Appendices}
            \input{8 - Hilbert complexes/main.tex}
            \input{9 - weak conservation proofs/main.tex}
\end{document}

            \documentclass[12pt, a4paper]{report}

\input{template/main.tex}

\title{\BA{Title in Progress...}}
\author{Boris Andrews}
\affil{Mathematical Institute, University of Oxford}
\date{\today}


\begin{document}
    \pagenumbering{gobble}
    \maketitle
    
    
    \begin{abstract}
        Magnetic confinement reactors---in particular tokamaks---offer one of the most promising options for achieving practical nuclear fusion, with the potential to provide virtually limitless, clean energy. The theoretical and numerical modeling of tokamak plasmas is simultaneously an essential component of effective reactor design, and a great research barrier. Tokamak operational conditions exhibit comparatively low Knudsen numbers. Kinetic effects, including kinetic waves and instabilities, Landau damping, bump-on-tail instabilities and more, are therefore highly influential in tokamak plasma dynamics. Purely fluid models are inherently incapable of capturing these effects, whereas the high dimensionality in purely kinetic models render them practically intractable for most relevant purposes.

        We consider a $\delta\!f$ decomposition model, with a macroscopic fluid background and microscopic kinetic correction, both fully coupled to each other. A similar manner of discretization is proposed to that used in the recent \texttt{STRUPHY} code \cite{Holderied_Possanner_Wang_2021, Holderied_2022, Li_et_al_2023} with a finite-element model for the background and a pseudo-particle/PiC model for the correction.

        The fluid background satisfies the full, non-linear, resistive, compressible, Hall MHD equations. \cite{Laakmann_Hu_Farrell_2022} introduces finite-element(-in-space) implicit timesteppers for the incompressible analogue to this system with structure-preserving (SP) properties in the ideal case, alongside parameter-robust preconditioners. We show that these timesteppers can derive from a finite-element-in-time (FET) (and finite-element-in-space) interpretation. The benefits of this reformulation are discussed, including the derivation of timesteppers that are higher order in time, and the quantifiable dissipative SP properties in the non-ideal, resistive case.
        
        We discuss possible options for extending this FET approach to timesteppers for the compressible case.

        The kinetic corrections satisfy linearized Boltzmann equations. Using a Lénard--Bernstein collision operator, these take Fokker--Planck-like forms \cite{Fokker_1914, Planck_1917} wherein pseudo-particles in the numerical model obey the neoclassical transport equations, with particle-independent Brownian drift terms. This offers a rigorous methodology for incorporating collisions into the particle transport model, without coupling the equations of motions for each particle.
        
        Works by Chen, Chacón et al. \cite{Chen_Chacón_Barnes_2011, Chacón_Chen_Barnes_2013, Chen_Chacón_2014, Chen_Chacón_2015} have developed structure-preserving particle pushers for neoclassical transport in the Vlasov equations, derived from Crank--Nicolson integrators. We show these too can can derive from a FET interpretation, similarly offering potential extensions to higher-order-in-time particle pushers. The FET formulation is used also to consider how the stochastic drift terms can be incorporated into the pushers. Stochastic gyrokinetic expansions are also discussed.

        Different options for the numerical implementation of these schemes are considered.

        Due to the efficacy of FET in the development of SP timesteppers for both the fluid and kinetic component, we hope this approach will prove effective in the future for developing SP timesteppers for the full hybrid model. We hope this will give us the opportunity to incorporate previously inaccessible kinetic effects into the highly effective, modern, finite-element MHD models.
    \end{abstract}
    
    
    \newpage
    \tableofcontents
    
    
    \newpage
    \pagenumbering{arabic}
    %\linenumbers\renewcommand\thelinenumber{\color{black!50}\arabic{linenumber}}
            \input{0 - introduction/main.tex}
        \part{Research}
            \input{1 - low-noise PiC models/main.tex}
            \input{2 - kinetic component/main.tex}
            \input{3 - fluid component/main.tex}
            \input{4 - numerical implementation/main.tex}
        \part{Project Overview}
            \input{5 - research plan/main.tex}
            \input{6 - summary/main.tex}
    
    
    %\section{}
    \newpage
    \pagenumbering{gobble}
        \printbibliography


    \newpage
    \pagenumbering{roman}
    \appendix
        \part{Appendices}
            \input{8 - Hilbert complexes/main.tex}
            \input{9 - weak conservation proofs/main.tex}
\end{document}

            \documentclass[12pt, a4paper]{report}

\input{template/main.tex}

\title{\BA{Title in Progress...}}
\author{Boris Andrews}
\affil{Mathematical Institute, University of Oxford}
\date{\today}


\begin{document}
    \pagenumbering{gobble}
    \maketitle
    
    
    \begin{abstract}
        Magnetic confinement reactors---in particular tokamaks---offer one of the most promising options for achieving practical nuclear fusion, with the potential to provide virtually limitless, clean energy. The theoretical and numerical modeling of tokamak plasmas is simultaneously an essential component of effective reactor design, and a great research barrier. Tokamak operational conditions exhibit comparatively low Knudsen numbers. Kinetic effects, including kinetic waves and instabilities, Landau damping, bump-on-tail instabilities and more, are therefore highly influential in tokamak plasma dynamics. Purely fluid models are inherently incapable of capturing these effects, whereas the high dimensionality in purely kinetic models render them practically intractable for most relevant purposes.

        We consider a $\delta\!f$ decomposition model, with a macroscopic fluid background and microscopic kinetic correction, both fully coupled to each other. A similar manner of discretization is proposed to that used in the recent \texttt{STRUPHY} code \cite{Holderied_Possanner_Wang_2021, Holderied_2022, Li_et_al_2023} with a finite-element model for the background and a pseudo-particle/PiC model for the correction.

        The fluid background satisfies the full, non-linear, resistive, compressible, Hall MHD equations. \cite{Laakmann_Hu_Farrell_2022} introduces finite-element(-in-space) implicit timesteppers for the incompressible analogue to this system with structure-preserving (SP) properties in the ideal case, alongside parameter-robust preconditioners. We show that these timesteppers can derive from a finite-element-in-time (FET) (and finite-element-in-space) interpretation. The benefits of this reformulation are discussed, including the derivation of timesteppers that are higher order in time, and the quantifiable dissipative SP properties in the non-ideal, resistive case.
        
        We discuss possible options for extending this FET approach to timesteppers for the compressible case.

        The kinetic corrections satisfy linearized Boltzmann equations. Using a Lénard--Bernstein collision operator, these take Fokker--Planck-like forms \cite{Fokker_1914, Planck_1917} wherein pseudo-particles in the numerical model obey the neoclassical transport equations, with particle-independent Brownian drift terms. This offers a rigorous methodology for incorporating collisions into the particle transport model, without coupling the equations of motions for each particle.
        
        Works by Chen, Chacón et al. \cite{Chen_Chacón_Barnes_2011, Chacón_Chen_Barnes_2013, Chen_Chacón_2014, Chen_Chacón_2015} have developed structure-preserving particle pushers for neoclassical transport in the Vlasov equations, derived from Crank--Nicolson integrators. We show these too can can derive from a FET interpretation, similarly offering potential extensions to higher-order-in-time particle pushers. The FET formulation is used also to consider how the stochastic drift terms can be incorporated into the pushers. Stochastic gyrokinetic expansions are also discussed.

        Different options for the numerical implementation of these schemes are considered.

        Due to the efficacy of FET in the development of SP timesteppers for both the fluid and kinetic component, we hope this approach will prove effective in the future for developing SP timesteppers for the full hybrid model. We hope this will give us the opportunity to incorporate previously inaccessible kinetic effects into the highly effective, modern, finite-element MHD models.
    \end{abstract}
    
    
    \newpage
    \tableofcontents
    
    
    \newpage
    \pagenumbering{arabic}
    %\linenumbers\renewcommand\thelinenumber{\color{black!50}\arabic{linenumber}}
            \input{0 - introduction/main.tex}
        \part{Research}
            \input{1 - low-noise PiC models/main.tex}
            \input{2 - kinetic component/main.tex}
            \input{3 - fluid component/main.tex}
            \input{4 - numerical implementation/main.tex}
        \part{Project Overview}
            \input{5 - research plan/main.tex}
            \input{6 - summary/main.tex}
    
    
    %\section{}
    \newpage
    \pagenumbering{gobble}
        \printbibliography


    \newpage
    \pagenumbering{roman}
    \appendix
        \part{Appendices}
            \input{8 - Hilbert complexes/main.tex}
            \input{9 - weak conservation proofs/main.tex}
\end{document}

        \part{Project Overview}
            \documentclass[12pt, a4paper]{report}

\input{template/main.tex}

\title{\BA{Title in Progress...}}
\author{Boris Andrews}
\affil{Mathematical Institute, University of Oxford}
\date{\today}


\begin{document}
    \pagenumbering{gobble}
    \maketitle
    
    
    \begin{abstract}
        Magnetic confinement reactors---in particular tokamaks---offer one of the most promising options for achieving practical nuclear fusion, with the potential to provide virtually limitless, clean energy. The theoretical and numerical modeling of tokamak plasmas is simultaneously an essential component of effective reactor design, and a great research barrier. Tokamak operational conditions exhibit comparatively low Knudsen numbers. Kinetic effects, including kinetic waves and instabilities, Landau damping, bump-on-tail instabilities and more, are therefore highly influential in tokamak plasma dynamics. Purely fluid models are inherently incapable of capturing these effects, whereas the high dimensionality in purely kinetic models render them practically intractable for most relevant purposes.

        We consider a $\delta\!f$ decomposition model, with a macroscopic fluid background and microscopic kinetic correction, both fully coupled to each other. A similar manner of discretization is proposed to that used in the recent \texttt{STRUPHY} code \cite{Holderied_Possanner_Wang_2021, Holderied_2022, Li_et_al_2023} with a finite-element model for the background and a pseudo-particle/PiC model for the correction.

        The fluid background satisfies the full, non-linear, resistive, compressible, Hall MHD equations. \cite{Laakmann_Hu_Farrell_2022} introduces finite-element(-in-space) implicit timesteppers for the incompressible analogue to this system with structure-preserving (SP) properties in the ideal case, alongside parameter-robust preconditioners. We show that these timesteppers can derive from a finite-element-in-time (FET) (and finite-element-in-space) interpretation. The benefits of this reformulation are discussed, including the derivation of timesteppers that are higher order in time, and the quantifiable dissipative SP properties in the non-ideal, resistive case.
        
        We discuss possible options for extending this FET approach to timesteppers for the compressible case.

        The kinetic corrections satisfy linearized Boltzmann equations. Using a Lénard--Bernstein collision operator, these take Fokker--Planck-like forms \cite{Fokker_1914, Planck_1917} wherein pseudo-particles in the numerical model obey the neoclassical transport equations, with particle-independent Brownian drift terms. This offers a rigorous methodology for incorporating collisions into the particle transport model, without coupling the equations of motions for each particle.
        
        Works by Chen, Chacón et al. \cite{Chen_Chacón_Barnes_2011, Chacón_Chen_Barnes_2013, Chen_Chacón_2014, Chen_Chacón_2015} have developed structure-preserving particle pushers for neoclassical transport in the Vlasov equations, derived from Crank--Nicolson integrators. We show these too can can derive from a FET interpretation, similarly offering potential extensions to higher-order-in-time particle pushers. The FET formulation is used also to consider how the stochastic drift terms can be incorporated into the pushers. Stochastic gyrokinetic expansions are also discussed.

        Different options for the numerical implementation of these schemes are considered.

        Due to the efficacy of FET in the development of SP timesteppers for both the fluid and kinetic component, we hope this approach will prove effective in the future for developing SP timesteppers for the full hybrid model. We hope this will give us the opportunity to incorporate previously inaccessible kinetic effects into the highly effective, modern, finite-element MHD models.
    \end{abstract}
    
    
    \newpage
    \tableofcontents
    
    
    \newpage
    \pagenumbering{arabic}
    %\linenumbers\renewcommand\thelinenumber{\color{black!50}\arabic{linenumber}}
            \input{0 - introduction/main.tex}
        \part{Research}
            \input{1 - low-noise PiC models/main.tex}
            \input{2 - kinetic component/main.tex}
            \input{3 - fluid component/main.tex}
            \input{4 - numerical implementation/main.tex}
        \part{Project Overview}
            \input{5 - research plan/main.tex}
            \input{6 - summary/main.tex}
    
    
    %\section{}
    \newpage
    \pagenumbering{gobble}
        \printbibliography


    \newpage
    \pagenumbering{roman}
    \appendix
        \part{Appendices}
            \input{8 - Hilbert complexes/main.tex}
            \input{9 - weak conservation proofs/main.tex}
\end{document}

            \documentclass[12pt, a4paper]{report}

\input{template/main.tex}

\title{\BA{Title in Progress...}}
\author{Boris Andrews}
\affil{Mathematical Institute, University of Oxford}
\date{\today}


\begin{document}
    \pagenumbering{gobble}
    \maketitle
    
    
    \begin{abstract}
        Magnetic confinement reactors---in particular tokamaks---offer one of the most promising options for achieving practical nuclear fusion, with the potential to provide virtually limitless, clean energy. The theoretical and numerical modeling of tokamak plasmas is simultaneously an essential component of effective reactor design, and a great research barrier. Tokamak operational conditions exhibit comparatively low Knudsen numbers. Kinetic effects, including kinetic waves and instabilities, Landau damping, bump-on-tail instabilities and more, are therefore highly influential in tokamak plasma dynamics. Purely fluid models are inherently incapable of capturing these effects, whereas the high dimensionality in purely kinetic models render them practically intractable for most relevant purposes.

        We consider a $\delta\!f$ decomposition model, with a macroscopic fluid background and microscopic kinetic correction, both fully coupled to each other. A similar manner of discretization is proposed to that used in the recent \texttt{STRUPHY} code \cite{Holderied_Possanner_Wang_2021, Holderied_2022, Li_et_al_2023} with a finite-element model for the background and a pseudo-particle/PiC model for the correction.

        The fluid background satisfies the full, non-linear, resistive, compressible, Hall MHD equations. \cite{Laakmann_Hu_Farrell_2022} introduces finite-element(-in-space) implicit timesteppers for the incompressible analogue to this system with structure-preserving (SP) properties in the ideal case, alongside parameter-robust preconditioners. We show that these timesteppers can derive from a finite-element-in-time (FET) (and finite-element-in-space) interpretation. The benefits of this reformulation are discussed, including the derivation of timesteppers that are higher order in time, and the quantifiable dissipative SP properties in the non-ideal, resistive case.
        
        We discuss possible options for extending this FET approach to timesteppers for the compressible case.

        The kinetic corrections satisfy linearized Boltzmann equations. Using a Lénard--Bernstein collision operator, these take Fokker--Planck-like forms \cite{Fokker_1914, Planck_1917} wherein pseudo-particles in the numerical model obey the neoclassical transport equations, with particle-independent Brownian drift terms. This offers a rigorous methodology for incorporating collisions into the particle transport model, without coupling the equations of motions for each particle.
        
        Works by Chen, Chacón et al. \cite{Chen_Chacón_Barnes_2011, Chacón_Chen_Barnes_2013, Chen_Chacón_2014, Chen_Chacón_2015} have developed structure-preserving particle pushers for neoclassical transport in the Vlasov equations, derived from Crank--Nicolson integrators. We show these too can can derive from a FET interpretation, similarly offering potential extensions to higher-order-in-time particle pushers. The FET formulation is used also to consider how the stochastic drift terms can be incorporated into the pushers. Stochastic gyrokinetic expansions are also discussed.

        Different options for the numerical implementation of these schemes are considered.

        Due to the efficacy of FET in the development of SP timesteppers for both the fluid and kinetic component, we hope this approach will prove effective in the future for developing SP timesteppers for the full hybrid model. We hope this will give us the opportunity to incorporate previously inaccessible kinetic effects into the highly effective, modern, finite-element MHD models.
    \end{abstract}
    
    
    \newpage
    \tableofcontents
    
    
    \newpage
    \pagenumbering{arabic}
    %\linenumbers\renewcommand\thelinenumber{\color{black!50}\arabic{linenumber}}
            \input{0 - introduction/main.tex}
        \part{Research}
            \input{1 - low-noise PiC models/main.tex}
            \input{2 - kinetic component/main.tex}
            \input{3 - fluid component/main.tex}
            \input{4 - numerical implementation/main.tex}
        \part{Project Overview}
            \input{5 - research plan/main.tex}
            \input{6 - summary/main.tex}
    
    
    %\section{}
    \newpage
    \pagenumbering{gobble}
        \printbibliography


    \newpage
    \pagenumbering{roman}
    \appendix
        \part{Appendices}
            \input{8 - Hilbert complexes/main.tex}
            \input{9 - weak conservation proofs/main.tex}
\end{document}

    
    
    %\section{}
    \newpage
    \pagenumbering{gobble}
        \printbibliography


    \newpage
    \pagenumbering{roman}
    \appendix
        \part{Appendices}
            \documentclass[12pt, a4paper]{report}

\input{template/main.tex}

\title{\BA{Title in Progress...}}
\author{Boris Andrews}
\affil{Mathematical Institute, University of Oxford}
\date{\today}


\begin{document}
    \pagenumbering{gobble}
    \maketitle
    
    
    \begin{abstract}
        Magnetic confinement reactors---in particular tokamaks---offer one of the most promising options for achieving practical nuclear fusion, with the potential to provide virtually limitless, clean energy. The theoretical and numerical modeling of tokamak plasmas is simultaneously an essential component of effective reactor design, and a great research barrier. Tokamak operational conditions exhibit comparatively low Knudsen numbers. Kinetic effects, including kinetic waves and instabilities, Landau damping, bump-on-tail instabilities and more, are therefore highly influential in tokamak plasma dynamics. Purely fluid models are inherently incapable of capturing these effects, whereas the high dimensionality in purely kinetic models render them practically intractable for most relevant purposes.

        We consider a $\delta\!f$ decomposition model, with a macroscopic fluid background and microscopic kinetic correction, both fully coupled to each other. A similar manner of discretization is proposed to that used in the recent \texttt{STRUPHY} code \cite{Holderied_Possanner_Wang_2021, Holderied_2022, Li_et_al_2023} with a finite-element model for the background and a pseudo-particle/PiC model for the correction.

        The fluid background satisfies the full, non-linear, resistive, compressible, Hall MHD equations. \cite{Laakmann_Hu_Farrell_2022} introduces finite-element(-in-space) implicit timesteppers for the incompressible analogue to this system with structure-preserving (SP) properties in the ideal case, alongside parameter-robust preconditioners. We show that these timesteppers can derive from a finite-element-in-time (FET) (and finite-element-in-space) interpretation. The benefits of this reformulation are discussed, including the derivation of timesteppers that are higher order in time, and the quantifiable dissipative SP properties in the non-ideal, resistive case.
        
        We discuss possible options for extending this FET approach to timesteppers for the compressible case.

        The kinetic corrections satisfy linearized Boltzmann equations. Using a Lénard--Bernstein collision operator, these take Fokker--Planck-like forms \cite{Fokker_1914, Planck_1917} wherein pseudo-particles in the numerical model obey the neoclassical transport equations, with particle-independent Brownian drift terms. This offers a rigorous methodology for incorporating collisions into the particle transport model, without coupling the equations of motions for each particle.
        
        Works by Chen, Chacón et al. \cite{Chen_Chacón_Barnes_2011, Chacón_Chen_Barnes_2013, Chen_Chacón_2014, Chen_Chacón_2015} have developed structure-preserving particle pushers for neoclassical transport in the Vlasov equations, derived from Crank--Nicolson integrators. We show these too can can derive from a FET interpretation, similarly offering potential extensions to higher-order-in-time particle pushers. The FET formulation is used also to consider how the stochastic drift terms can be incorporated into the pushers. Stochastic gyrokinetic expansions are also discussed.

        Different options for the numerical implementation of these schemes are considered.

        Due to the efficacy of FET in the development of SP timesteppers for both the fluid and kinetic component, we hope this approach will prove effective in the future for developing SP timesteppers for the full hybrid model. We hope this will give us the opportunity to incorporate previously inaccessible kinetic effects into the highly effective, modern, finite-element MHD models.
    \end{abstract}
    
    
    \newpage
    \tableofcontents
    
    
    \newpage
    \pagenumbering{arabic}
    %\linenumbers\renewcommand\thelinenumber{\color{black!50}\arabic{linenumber}}
            \input{0 - introduction/main.tex}
        \part{Research}
            \input{1 - low-noise PiC models/main.tex}
            \input{2 - kinetic component/main.tex}
            \input{3 - fluid component/main.tex}
            \input{4 - numerical implementation/main.tex}
        \part{Project Overview}
            \input{5 - research plan/main.tex}
            \input{6 - summary/main.tex}
    
    
    %\section{}
    \newpage
    \pagenumbering{gobble}
        \printbibliography


    \newpage
    \pagenumbering{roman}
    \appendix
        \part{Appendices}
            \input{8 - Hilbert complexes/main.tex}
            \input{9 - weak conservation proofs/main.tex}
\end{document}

            \documentclass[12pt, a4paper]{report}

\input{template/main.tex}

\title{\BA{Title in Progress...}}
\author{Boris Andrews}
\affil{Mathematical Institute, University of Oxford}
\date{\today}


\begin{document}
    \pagenumbering{gobble}
    \maketitle
    
    
    \begin{abstract}
        Magnetic confinement reactors---in particular tokamaks---offer one of the most promising options for achieving practical nuclear fusion, with the potential to provide virtually limitless, clean energy. The theoretical and numerical modeling of tokamak plasmas is simultaneously an essential component of effective reactor design, and a great research barrier. Tokamak operational conditions exhibit comparatively low Knudsen numbers. Kinetic effects, including kinetic waves and instabilities, Landau damping, bump-on-tail instabilities and more, are therefore highly influential in tokamak plasma dynamics. Purely fluid models are inherently incapable of capturing these effects, whereas the high dimensionality in purely kinetic models render them practically intractable for most relevant purposes.

        We consider a $\delta\!f$ decomposition model, with a macroscopic fluid background and microscopic kinetic correction, both fully coupled to each other. A similar manner of discretization is proposed to that used in the recent \texttt{STRUPHY} code \cite{Holderied_Possanner_Wang_2021, Holderied_2022, Li_et_al_2023} with a finite-element model for the background and a pseudo-particle/PiC model for the correction.

        The fluid background satisfies the full, non-linear, resistive, compressible, Hall MHD equations. \cite{Laakmann_Hu_Farrell_2022} introduces finite-element(-in-space) implicit timesteppers for the incompressible analogue to this system with structure-preserving (SP) properties in the ideal case, alongside parameter-robust preconditioners. We show that these timesteppers can derive from a finite-element-in-time (FET) (and finite-element-in-space) interpretation. The benefits of this reformulation are discussed, including the derivation of timesteppers that are higher order in time, and the quantifiable dissipative SP properties in the non-ideal, resistive case.
        
        We discuss possible options for extending this FET approach to timesteppers for the compressible case.

        The kinetic corrections satisfy linearized Boltzmann equations. Using a Lénard--Bernstein collision operator, these take Fokker--Planck-like forms \cite{Fokker_1914, Planck_1917} wherein pseudo-particles in the numerical model obey the neoclassical transport equations, with particle-independent Brownian drift terms. This offers a rigorous methodology for incorporating collisions into the particle transport model, without coupling the equations of motions for each particle.
        
        Works by Chen, Chacón et al. \cite{Chen_Chacón_Barnes_2011, Chacón_Chen_Barnes_2013, Chen_Chacón_2014, Chen_Chacón_2015} have developed structure-preserving particle pushers for neoclassical transport in the Vlasov equations, derived from Crank--Nicolson integrators. We show these too can can derive from a FET interpretation, similarly offering potential extensions to higher-order-in-time particle pushers. The FET formulation is used also to consider how the stochastic drift terms can be incorporated into the pushers. Stochastic gyrokinetic expansions are also discussed.

        Different options for the numerical implementation of these schemes are considered.

        Due to the efficacy of FET in the development of SP timesteppers for both the fluid and kinetic component, we hope this approach will prove effective in the future for developing SP timesteppers for the full hybrid model. We hope this will give us the opportunity to incorporate previously inaccessible kinetic effects into the highly effective, modern, finite-element MHD models.
    \end{abstract}
    
    
    \newpage
    \tableofcontents
    
    
    \newpage
    \pagenumbering{arabic}
    %\linenumbers\renewcommand\thelinenumber{\color{black!50}\arabic{linenumber}}
            \input{0 - introduction/main.tex}
        \part{Research}
            \input{1 - low-noise PiC models/main.tex}
            \input{2 - kinetic component/main.tex}
            \input{3 - fluid component/main.tex}
            \input{4 - numerical implementation/main.tex}
        \part{Project Overview}
            \input{5 - research plan/main.tex}
            \input{6 - summary/main.tex}
    
    
    %\section{}
    \newpage
    \pagenumbering{gobble}
        \printbibliography


    \newpage
    \pagenumbering{roman}
    \appendix
        \part{Appendices}
            \input{8 - Hilbert complexes/main.tex}
            \input{9 - weak conservation proofs/main.tex}
\end{document}

\end{document}

    
    
    %\section{}
    \newpage
    \pagenumbering{gobble}
        \printbibliography


    \newpage
    \pagenumbering{roman}
    \appendix
        \part{Appendices}
            \documentclass[12pt, a4paper]{report}

\documentclass[12pt, a4paper]{report}

\input{template/main.tex}

\title{\BA{Title in Progress...}}
\author{Boris Andrews}
\affil{Mathematical Institute, University of Oxford}
\date{\today}


\begin{document}
    \pagenumbering{gobble}
    \maketitle
    
    
    \begin{abstract}
        Magnetic confinement reactors---in particular tokamaks---offer one of the most promising options for achieving practical nuclear fusion, with the potential to provide virtually limitless, clean energy. The theoretical and numerical modeling of tokamak plasmas is simultaneously an essential component of effective reactor design, and a great research barrier. Tokamak operational conditions exhibit comparatively low Knudsen numbers. Kinetic effects, including kinetic waves and instabilities, Landau damping, bump-on-tail instabilities and more, are therefore highly influential in tokamak plasma dynamics. Purely fluid models are inherently incapable of capturing these effects, whereas the high dimensionality in purely kinetic models render them practically intractable for most relevant purposes.

        We consider a $\delta\!f$ decomposition model, with a macroscopic fluid background and microscopic kinetic correction, both fully coupled to each other. A similar manner of discretization is proposed to that used in the recent \texttt{STRUPHY} code \cite{Holderied_Possanner_Wang_2021, Holderied_2022, Li_et_al_2023} with a finite-element model for the background and a pseudo-particle/PiC model for the correction.

        The fluid background satisfies the full, non-linear, resistive, compressible, Hall MHD equations. \cite{Laakmann_Hu_Farrell_2022} introduces finite-element(-in-space) implicit timesteppers for the incompressible analogue to this system with structure-preserving (SP) properties in the ideal case, alongside parameter-robust preconditioners. We show that these timesteppers can derive from a finite-element-in-time (FET) (and finite-element-in-space) interpretation. The benefits of this reformulation are discussed, including the derivation of timesteppers that are higher order in time, and the quantifiable dissipative SP properties in the non-ideal, resistive case.
        
        We discuss possible options for extending this FET approach to timesteppers for the compressible case.

        The kinetic corrections satisfy linearized Boltzmann equations. Using a Lénard--Bernstein collision operator, these take Fokker--Planck-like forms \cite{Fokker_1914, Planck_1917} wherein pseudo-particles in the numerical model obey the neoclassical transport equations, with particle-independent Brownian drift terms. This offers a rigorous methodology for incorporating collisions into the particle transport model, without coupling the equations of motions for each particle.
        
        Works by Chen, Chacón et al. \cite{Chen_Chacón_Barnes_2011, Chacón_Chen_Barnes_2013, Chen_Chacón_2014, Chen_Chacón_2015} have developed structure-preserving particle pushers for neoclassical transport in the Vlasov equations, derived from Crank--Nicolson integrators. We show these too can can derive from a FET interpretation, similarly offering potential extensions to higher-order-in-time particle pushers. The FET formulation is used also to consider how the stochastic drift terms can be incorporated into the pushers. Stochastic gyrokinetic expansions are also discussed.

        Different options for the numerical implementation of these schemes are considered.

        Due to the efficacy of FET in the development of SP timesteppers for both the fluid and kinetic component, we hope this approach will prove effective in the future for developing SP timesteppers for the full hybrid model. We hope this will give us the opportunity to incorporate previously inaccessible kinetic effects into the highly effective, modern, finite-element MHD models.
    \end{abstract}
    
    
    \newpage
    \tableofcontents
    
    
    \newpage
    \pagenumbering{arabic}
    %\linenumbers\renewcommand\thelinenumber{\color{black!50}\arabic{linenumber}}
            \input{0 - introduction/main.tex}
        \part{Research}
            \input{1 - low-noise PiC models/main.tex}
            \input{2 - kinetic component/main.tex}
            \input{3 - fluid component/main.tex}
            \input{4 - numerical implementation/main.tex}
        \part{Project Overview}
            \input{5 - research plan/main.tex}
            \input{6 - summary/main.tex}
    
    
    %\section{}
    \newpage
    \pagenumbering{gobble}
        \printbibliography


    \newpage
    \pagenumbering{roman}
    \appendix
        \part{Appendices}
            \input{8 - Hilbert complexes/main.tex}
            \input{9 - weak conservation proofs/main.tex}
\end{document}


\title{\BA{Title in Progress...}}
\author{Boris Andrews}
\affil{Mathematical Institute, University of Oxford}
\date{\today}


\begin{document}
    \pagenumbering{gobble}
    \maketitle
    
    
    \begin{abstract}
        Magnetic confinement reactors---in particular tokamaks---offer one of the most promising options for achieving practical nuclear fusion, with the potential to provide virtually limitless, clean energy. The theoretical and numerical modeling of tokamak plasmas is simultaneously an essential component of effective reactor design, and a great research barrier. Tokamak operational conditions exhibit comparatively low Knudsen numbers. Kinetic effects, including kinetic waves and instabilities, Landau damping, bump-on-tail instabilities and more, are therefore highly influential in tokamak plasma dynamics. Purely fluid models are inherently incapable of capturing these effects, whereas the high dimensionality in purely kinetic models render them practically intractable for most relevant purposes.

        We consider a $\delta\!f$ decomposition model, with a macroscopic fluid background and microscopic kinetic correction, both fully coupled to each other. A similar manner of discretization is proposed to that used in the recent \texttt{STRUPHY} code \cite{Holderied_Possanner_Wang_2021, Holderied_2022, Li_et_al_2023} with a finite-element model for the background and a pseudo-particle/PiC model for the correction.

        The fluid background satisfies the full, non-linear, resistive, compressible, Hall MHD equations. \cite{Laakmann_Hu_Farrell_2022} introduces finite-element(-in-space) implicit timesteppers for the incompressible analogue to this system with structure-preserving (SP) properties in the ideal case, alongside parameter-robust preconditioners. We show that these timesteppers can derive from a finite-element-in-time (FET) (and finite-element-in-space) interpretation. The benefits of this reformulation are discussed, including the derivation of timesteppers that are higher order in time, and the quantifiable dissipative SP properties in the non-ideal, resistive case.
        
        We discuss possible options for extending this FET approach to timesteppers for the compressible case.

        The kinetic corrections satisfy linearized Boltzmann equations. Using a Lénard--Bernstein collision operator, these take Fokker--Planck-like forms \cite{Fokker_1914, Planck_1917} wherein pseudo-particles in the numerical model obey the neoclassical transport equations, with particle-independent Brownian drift terms. This offers a rigorous methodology for incorporating collisions into the particle transport model, without coupling the equations of motions for each particle.
        
        Works by Chen, Chacón et al. \cite{Chen_Chacón_Barnes_2011, Chacón_Chen_Barnes_2013, Chen_Chacón_2014, Chen_Chacón_2015} have developed structure-preserving particle pushers for neoclassical transport in the Vlasov equations, derived from Crank--Nicolson integrators. We show these too can can derive from a FET interpretation, similarly offering potential extensions to higher-order-in-time particle pushers. The FET formulation is used also to consider how the stochastic drift terms can be incorporated into the pushers. Stochastic gyrokinetic expansions are also discussed.

        Different options for the numerical implementation of these schemes are considered.

        Due to the efficacy of FET in the development of SP timesteppers for both the fluid and kinetic component, we hope this approach will prove effective in the future for developing SP timesteppers for the full hybrid model. We hope this will give us the opportunity to incorporate previously inaccessible kinetic effects into the highly effective, modern, finite-element MHD models.
    \end{abstract}
    
    
    \newpage
    \tableofcontents
    
    
    \newpage
    \pagenumbering{arabic}
    %\linenumbers\renewcommand\thelinenumber{\color{black!50}\arabic{linenumber}}
            \documentclass[12pt, a4paper]{report}

\input{template/main.tex}

\title{\BA{Title in Progress...}}
\author{Boris Andrews}
\affil{Mathematical Institute, University of Oxford}
\date{\today}


\begin{document}
    \pagenumbering{gobble}
    \maketitle
    
    
    \begin{abstract}
        Magnetic confinement reactors---in particular tokamaks---offer one of the most promising options for achieving practical nuclear fusion, with the potential to provide virtually limitless, clean energy. The theoretical and numerical modeling of tokamak plasmas is simultaneously an essential component of effective reactor design, and a great research barrier. Tokamak operational conditions exhibit comparatively low Knudsen numbers. Kinetic effects, including kinetic waves and instabilities, Landau damping, bump-on-tail instabilities and more, are therefore highly influential in tokamak plasma dynamics. Purely fluid models are inherently incapable of capturing these effects, whereas the high dimensionality in purely kinetic models render them practically intractable for most relevant purposes.

        We consider a $\delta\!f$ decomposition model, with a macroscopic fluid background and microscopic kinetic correction, both fully coupled to each other. A similar manner of discretization is proposed to that used in the recent \texttt{STRUPHY} code \cite{Holderied_Possanner_Wang_2021, Holderied_2022, Li_et_al_2023} with a finite-element model for the background and a pseudo-particle/PiC model for the correction.

        The fluid background satisfies the full, non-linear, resistive, compressible, Hall MHD equations. \cite{Laakmann_Hu_Farrell_2022} introduces finite-element(-in-space) implicit timesteppers for the incompressible analogue to this system with structure-preserving (SP) properties in the ideal case, alongside parameter-robust preconditioners. We show that these timesteppers can derive from a finite-element-in-time (FET) (and finite-element-in-space) interpretation. The benefits of this reformulation are discussed, including the derivation of timesteppers that are higher order in time, and the quantifiable dissipative SP properties in the non-ideal, resistive case.
        
        We discuss possible options for extending this FET approach to timesteppers for the compressible case.

        The kinetic corrections satisfy linearized Boltzmann equations. Using a Lénard--Bernstein collision operator, these take Fokker--Planck-like forms \cite{Fokker_1914, Planck_1917} wherein pseudo-particles in the numerical model obey the neoclassical transport equations, with particle-independent Brownian drift terms. This offers a rigorous methodology for incorporating collisions into the particle transport model, without coupling the equations of motions for each particle.
        
        Works by Chen, Chacón et al. \cite{Chen_Chacón_Barnes_2011, Chacón_Chen_Barnes_2013, Chen_Chacón_2014, Chen_Chacón_2015} have developed structure-preserving particle pushers for neoclassical transport in the Vlasov equations, derived from Crank--Nicolson integrators. We show these too can can derive from a FET interpretation, similarly offering potential extensions to higher-order-in-time particle pushers. The FET formulation is used also to consider how the stochastic drift terms can be incorporated into the pushers. Stochastic gyrokinetic expansions are also discussed.

        Different options for the numerical implementation of these schemes are considered.

        Due to the efficacy of FET in the development of SP timesteppers for both the fluid and kinetic component, we hope this approach will prove effective in the future for developing SP timesteppers for the full hybrid model. We hope this will give us the opportunity to incorporate previously inaccessible kinetic effects into the highly effective, modern, finite-element MHD models.
    \end{abstract}
    
    
    \newpage
    \tableofcontents
    
    
    \newpage
    \pagenumbering{arabic}
    %\linenumbers\renewcommand\thelinenumber{\color{black!50}\arabic{linenumber}}
            \input{0 - introduction/main.tex}
        \part{Research}
            \input{1 - low-noise PiC models/main.tex}
            \input{2 - kinetic component/main.tex}
            \input{3 - fluid component/main.tex}
            \input{4 - numerical implementation/main.tex}
        \part{Project Overview}
            \input{5 - research plan/main.tex}
            \input{6 - summary/main.tex}
    
    
    %\section{}
    \newpage
    \pagenumbering{gobble}
        \printbibliography


    \newpage
    \pagenumbering{roman}
    \appendix
        \part{Appendices}
            \input{8 - Hilbert complexes/main.tex}
            \input{9 - weak conservation proofs/main.tex}
\end{document}

        \part{Research}
            \documentclass[12pt, a4paper]{report}

\input{template/main.tex}

\title{\BA{Title in Progress...}}
\author{Boris Andrews}
\affil{Mathematical Institute, University of Oxford}
\date{\today}


\begin{document}
    \pagenumbering{gobble}
    \maketitle
    
    
    \begin{abstract}
        Magnetic confinement reactors---in particular tokamaks---offer one of the most promising options for achieving practical nuclear fusion, with the potential to provide virtually limitless, clean energy. The theoretical and numerical modeling of tokamak plasmas is simultaneously an essential component of effective reactor design, and a great research barrier. Tokamak operational conditions exhibit comparatively low Knudsen numbers. Kinetic effects, including kinetic waves and instabilities, Landau damping, bump-on-tail instabilities and more, are therefore highly influential in tokamak plasma dynamics. Purely fluid models are inherently incapable of capturing these effects, whereas the high dimensionality in purely kinetic models render them practically intractable for most relevant purposes.

        We consider a $\delta\!f$ decomposition model, with a macroscopic fluid background and microscopic kinetic correction, both fully coupled to each other. A similar manner of discretization is proposed to that used in the recent \texttt{STRUPHY} code \cite{Holderied_Possanner_Wang_2021, Holderied_2022, Li_et_al_2023} with a finite-element model for the background and a pseudo-particle/PiC model for the correction.

        The fluid background satisfies the full, non-linear, resistive, compressible, Hall MHD equations. \cite{Laakmann_Hu_Farrell_2022} introduces finite-element(-in-space) implicit timesteppers for the incompressible analogue to this system with structure-preserving (SP) properties in the ideal case, alongside parameter-robust preconditioners. We show that these timesteppers can derive from a finite-element-in-time (FET) (and finite-element-in-space) interpretation. The benefits of this reformulation are discussed, including the derivation of timesteppers that are higher order in time, and the quantifiable dissipative SP properties in the non-ideal, resistive case.
        
        We discuss possible options for extending this FET approach to timesteppers for the compressible case.

        The kinetic corrections satisfy linearized Boltzmann equations. Using a Lénard--Bernstein collision operator, these take Fokker--Planck-like forms \cite{Fokker_1914, Planck_1917} wherein pseudo-particles in the numerical model obey the neoclassical transport equations, with particle-independent Brownian drift terms. This offers a rigorous methodology for incorporating collisions into the particle transport model, without coupling the equations of motions for each particle.
        
        Works by Chen, Chacón et al. \cite{Chen_Chacón_Barnes_2011, Chacón_Chen_Barnes_2013, Chen_Chacón_2014, Chen_Chacón_2015} have developed structure-preserving particle pushers for neoclassical transport in the Vlasov equations, derived from Crank--Nicolson integrators. We show these too can can derive from a FET interpretation, similarly offering potential extensions to higher-order-in-time particle pushers. The FET formulation is used also to consider how the stochastic drift terms can be incorporated into the pushers. Stochastic gyrokinetic expansions are also discussed.

        Different options for the numerical implementation of these schemes are considered.

        Due to the efficacy of FET in the development of SP timesteppers for both the fluid and kinetic component, we hope this approach will prove effective in the future for developing SP timesteppers for the full hybrid model. We hope this will give us the opportunity to incorporate previously inaccessible kinetic effects into the highly effective, modern, finite-element MHD models.
    \end{abstract}
    
    
    \newpage
    \tableofcontents
    
    
    \newpage
    \pagenumbering{arabic}
    %\linenumbers\renewcommand\thelinenumber{\color{black!50}\arabic{linenumber}}
            \input{0 - introduction/main.tex}
        \part{Research}
            \input{1 - low-noise PiC models/main.tex}
            \input{2 - kinetic component/main.tex}
            \input{3 - fluid component/main.tex}
            \input{4 - numerical implementation/main.tex}
        \part{Project Overview}
            \input{5 - research plan/main.tex}
            \input{6 - summary/main.tex}
    
    
    %\section{}
    \newpage
    \pagenumbering{gobble}
        \printbibliography


    \newpage
    \pagenumbering{roman}
    \appendix
        \part{Appendices}
            \input{8 - Hilbert complexes/main.tex}
            \input{9 - weak conservation proofs/main.tex}
\end{document}

            \documentclass[12pt, a4paper]{report}

\input{template/main.tex}

\title{\BA{Title in Progress...}}
\author{Boris Andrews}
\affil{Mathematical Institute, University of Oxford}
\date{\today}


\begin{document}
    \pagenumbering{gobble}
    \maketitle
    
    
    \begin{abstract}
        Magnetic confinement reactors---in particular tokamaks---offer one of the most promising options for achieving practical nuclear fusion, with the potential to provide virtually limitless, clean energy. The theoretical and numerical modeling of tokamak plasmas is simultaneously an essential component of effective reactor design, and a great research barrier. Tokamak operational conditions exhibit comparatively low Knudsen numbers. Kinetic effects, including kinetic waves and instabilities, Landau damping, bump-on-tail instabilities and more, are therefore highly influential in tokamak plasma dynamics. Purely fluid models are inherently incapable of capturing these effects, whereas the high dimensionality in purely kinetic models render them practically intractable for most relevant purposes.

        We consider a $\delta\!f$ decomposition model, with a macroscopic fluid background and microscopic kinetic correction, both fully coupled to each other. A similar manner of discretization is proposed to that used in the recent \texttt{STRUPHY} code \cite{Holderied_Possanner_Wang_2021, Holderied_2022, Li_et_al_2023} with a finite-element model for the background and a pseudo-particle/PiC model for the correction.

        The fluid background satisfies the full, non-linear, resistive, compressible, Hall MHD equations. \cite{Laakmann_Hu_Farrell_2022} introduces finite-element(-in-space) implicit timesteppers for the incompressible analogue to this system with structure-preserving (SP) properties in the ideal case, alongside parameter-robust preconditioners. We show that these timesteppers can derive from a finite-element-in-time (FET) (and finite-element-in-space) interpretation. The benefits of this reformulation are discussed, including the derivation of timesteppers that are higher order in time, and the quantifiable dissipative SP properties in the non-ideal, resistive case.
        
        We discuss possible options for extending this FET approach to timesteppers for the compressible case.

        The kinetic corrections satisfy linearized Boltzmann equations. Using a Lénard--Bernstein collision operator, these take Fokker--Planck-like forms \cite{Fokker_1914, Planck_1917} wherein pseudo-particles in the numerical model obey the neoclassical transport equations, with particle-independent Brownian drift terms. This offers a rigorous methodology for incorporating collisions into the particle transport model, without coupling the equations of motions for each particle.
        
        Works by Chen, Chacón et al. \cite{Chen_Chacón_Barnes_2011, Chacón_Chen_Barnes_2013, Chen_Chacón_2014, Chen_Chacón_2015} have developed structure-preserving particle pushers for neoclassical transport in the Vlasov equations, derived from Crank--Nicolson integrators. We show these too can can derive from a FET interpretation, similarly offering potential extensions to higher-order-in-time particle pushers. The FET formulation is used also to consider how the stochastic drift terms can be incorporated into the pushers. Stochastic gyrokinetic expansions are also discussed.

        Different options for the numerical implementation of these schemes are considered.

        Due to the efficacy of FET in the development of SP timesteppers for both the fluid and kinetic component, we hope this approach will prove effective in the future for developing SP timesteppers for the full hybrid model. We hope this will give us the opportunity to incorporate previously inaccessible kinetic effects into the highly effective, modern, finite-element MHD models.
    \end{abstract}
    
    
    \newpage
    \tableofcontents
    
    
    \newpage
    \pagenumbering{arabic}
    %\linenumbers\renewcommand\thelinenumber{\color{black!50}\arabic{linenumber}}
            \input{0 - introduction/main.tex}
        \part{Research}
            \input{1 - low-noise PiC models/main.tex}
            \input{2 - kinetic component/main.tex}
            \input{3 - fluid component/main.tex}
            \input{4 - numerical implementation/main.tex}
        \part{Project Overview}
            \input{5 - research plan/main.tex}
            \input{6 - summary/main.tex}
    
    
    %\section{}
    \newpage
    \pagenumbering{gobble}
        \printbibliography


    \newpage
    \pagenumbering{roman}
    \appendix
        \part{Appendices}
            \input{8 - Hilbert complexes/main.tex}
            \input{9 - weak conservation proofs/main.tex}
\end{document}

            \documentclass[12pt, a4paper]{report}

\input{template/main.tex}

\title{\BA{Title in Progress...}}
\author{Boris Andrews}
\affil{Mathematical Institute, University of Oxford}
\date{\today}


\begin{document}
    \pagenumbering{gobble}
    \maketitle
    
    
    \begin{abstract}
        Magnetic confinement reactors---in particular tokamaks---offer one of the most promising options for achieving practical nuclear fusion, with the potential to provide virtually limitless, clean energy. The theoretical and numerical modeling of tokamak plasmas is simultaneously an essential component of effective reactor design, and a great research barrier. Tokamak operational conditions exhibit comparatively low Knudsen numbers. Kinetic effects, including kinetic waves and instabilities, Landau damping, bump-on-tail instabilities and more, are therefore highly influential in tokamak plasma dynamics. Purely fluid models are inherently incapable of capturing these effects, whereas the high dimensionality in purely kinetic models render them practically intractable for most relevant purposes.

        We consider a $\delta\!f$ decomposition model, with a macroscopic fluid background and microscopic kinetic correction, both fully coupled to each other. A similar manner of discretization is proposed to that used in the recent \texttt{STRUPHY} code \cite{Holderied_Possanner_Wang_2021, Holderied_2022, Li_et_al_2023} with a finite-element model for the background and a pseudo-particle/PiC model for the correction.

        The fluid background satisfies the full, non-linear, resistive, compressible, Hall MHD equations. \cite{Laakmann_Hu_Farrell_2022} introduces finite-element(-in-space) implicit timesteppers for the incompressible analogue to this system with structure-preserving (SP) properties in the ideal case, alongside parameter-robust preconditioners. We show that these timesteppers can derive from a finite-element-in-time (FET) (and finite-element-in-space) interpretation. The benefits of this reformulation are discussed, including the derivation of timesteppers that are higher order in time, and the quantifiable dissipative SP properties in the non-ideal, resistive case.
        
        We discuss possible options for extending this FET approach to timesteppers for the compressible case.

        The kinetic corrections satisfy linearized Boltzmann equations. Using a Lénard--Bernstein collision operator, these take Fokker--Planck-like forms \cite{Fokker_1914, Planck_1917} wherein pseudo-particles in the numerical model obey the neoclassical transport equations, with particle-independent Brownian drift terms. This offers a rigorous methodology for incorporating collisions into the particle transport model, without coupling the equations of motions for each particle.
        
        Works by Chen, Chacón et al. \cite{Chen_Chacón_Barnes_2011, Chacón_Chen_Barnes_2013, Chen_Chacón_2014, Chen_Chacón_2015} have developed structure-preserving particle pushers for neoclassical transport in the Vlasov equations, derived from Crank--Nicolson integrators. We show these too can can derive from a FET interpretation, similarly offering potential extensions to higher-order-in-time particle pushers. The FET formulation is used also to consider how the stochastic drift terms can be incorporated into the pushers. Stochastic gyrokinetic expansions are also discussed.

        Different options for the numerical implementation of these schemes are considered.

        Due to the efficacy of FET in the development of SP timesteppers for both the fluid and kinetic component, we hope this approach will prove effective in the future for developing SP timesteppers for the full hybrid model. We hope this will give us the opportunity to incorporate previously inaccessible kinetic effects into the highly effective, modern, finite-element MHD models.
    \end{abstract}
    
    
    \newpage
    \tableofcontents
    
    
    \newpage
    \pagenumbering{arabic}
    %\linenumbers\renewcommand\thelinenumber{\color{black!50}\arabic{linenumber}}
            \input{0 - introduction/main.tex}
        \part{Research}
            \input{1 - low-noise PiC models/main.tex}
            \input{2 - kinetic component/main.tex}
            \input{3 - fluid component/main.tex}
            \input{4 - numerical implementation/main.tex}
        \part{Project Overview}
            \input{5 - research plan/main.tex}
            \input{6 - summary/main.tex}
    
    
    %\section{}
    \newpage
    \pagenumbering{gobble}
        \printbibliography


    \newpage
    \pagenumbering{roman}
    \appendix
        \part{Appendices}
            \input{8 - Hilbert complexes/main.tex}
            \input{9 - weak conservation proofs/main.tex}
\end{document}

            \documentclass[12pt, a4paper]{report}

\input{template/main.tex}

\title{\BA{Title in Progress...}}
\author{Boris Andrews}
\affil{Mathematical Institute, University of Oxford}
\date{\today}


\begin{document}
    \pagenumbering{gobble}
    \maketitle
    
    
    \begin{abstract}
        Magnetic confinement reactors---in particular tokamaks---offer one of the most promising options for achieving practical nuclear fusion, with the potential to provide virtually limitless, clean energy. The theoretical and numerical modeling of tokamak plasmas is simultaneously an essential component of effective reactor design, and a great research barrier. Tokamak operational conditions exhibit comparatively low Knudsen numbers. Kinetic effects, including kinetic waves and instabilities, Landau damping, bump-on-tail instabilities and more, are therefore highly influential in tokamak plasma dynamics. Purely fluid models are inherently incapable of capturing these effects, whereas the high dimensionality in purely kinetic models render them practically intractable for most relevant purposes.

        We consider a $\delta\!f$ decomposition model, with a macroscopic fluid background and microscopic kinetic correction, both fully coupled to each other. A similar manner of discretization is proposed to that used in the recent \texttt{STRUPHY} code \cite{Holderied_Possanner_Wang_2021, Holderied_2022, Li_et_al_2023} with a finite-element model for the background and a pseudo-particle/PiC model for the correction.

        The fluid background satisfies the full, non-linear, resistive, compressible, Hall MHD equations. \cite{Laakmann_Hu_Farrell_2022} introduces finite-element(-in-space) implicit timesteppers for the incompressible analogue to this system with structure-preserving (SP) properties in the ideal case, alongside parameter-robust preconditioners. We show that these timesteppers can derive from a finite-element-in-time (FET) (and finite-element-in-space) interpretation. The benefits of this reformulation are discussed, including the derivation of timesteppers that are higher order in time, and the quantifiable dissipative SP properties in the non-ideal, resistive case.
        
        We discuss possible options for extending this FET approach to timesteppers for the compressible case.

        The kinetic corrections satisfy linearized Boltzmann equations. Using a Lénard--Bernstein collision operator, these take Fokker--Planck-like forms \cite{Fokker_1914, Planck_1917} wherein pseudo-particles in the numerical model obey the neoclassical transport equations, with particle-independent Brownian drift terms. This offers a rigorous methodology for incorporating collisions into the particle transport model, without coupling the equations of motions for each particle.
        
        Works by Chen, Chacón et al. \cite{Chen_Chacón_Barnes_2011, Chacón_Chen_Barnes_2013, Chen_Chacón_2014, Chen_Chacón_2015} have developed structure-preserving particle pushers for neoclassical transport in the Vlasov equations, derived from Crank--Nicolson integrators. We show these too can can derive from a FET interpretation, similarly offering potential extensions to higher-order-in-time particle pushers. The FET formulation is used also to consider how the stochastic drift terms can be incorporated into the pushers. Stochastic gyrokinetic expansions are also discussed.

        Different options for the numerical implementation of these schemes are considered.

        Due to the efficacy of FET in the development of SP timesteppers for both the fluid and kinetic component, we hope this approach will prove effective in the future for developing SP timesteppers for the full hybrid model. We hope this will give us the opportunity to incorporate previously inaccessible kinetic effects into the highly effective, modern, finite-element MHD models.
    \end{abstract}
    
    
    \newpage
    \tableofcontents
    
    
    \newpage
    \pagenumbering{arabic}
    %\linenumbers\renewcommand\thelinenumber{\color{black!50}\arabic{linenumber}}
            \input{0 - introduction/main.tex}
        \part{Research}
            \input{1 - low-noise PiC models/main.tex}
            \input{2 - kinetic component/main.tex}
            \input{3 - fluid component/main.tex}
            \input{4 - numerical implementation/main.tex}
        \part{Project Overview}
            \input{5 - research plan/main.tex}
            \input{6 - summary/main.tex}
    
    
    %\section{}
    \newpage
    \pagenumbering{gobble}
        \printbibliography


    \newpage
    \pagenumbering{roman}
    \appendix
        \part{Appendices}
            \input{8 - Hilbert complexes/main.tex}
            \input{9 - weak conservation proofs/main.tex}
\end{document}

        \part{Project Overview}
            \documentclass[12pt, a4paper]{report}

\input{template/main.tex}

\title{\BA{Title in Progress...}}
\author{Boris Andrews}
\affil{Mathematical Institute, University of Oxford}
\date{\today}


\begin{document}
    \pagenumbering{gobble}
    \maketitle
    
    
    \begin{abstract}
        Magnetic confinement reactors---in particular tokamaks---offer one of the most promising options for achieving practical nuclear fusion, with the potential to provide virtually limitless, clean energy. The theoretical and numerical modeling of tokamak plasmas is simultaneously an essential component of effective reactor design, and a great research barrier. Tokamak operational conditions exhibit comparatively low Knudsen numbers. Kinetic effects, including kinetic waves and instabilities, Landau damping, bump-on-tail instabilities and more, are therefore highly influential in tokamak plasma dynamics. Purely fluid models are inherently incapable of capturing these effects, whereas the high dimensionality in purely kinetic models render them practically intractable for most relevant purposes.

        We consider a $\delta\!f$ decomposition model, with a macroscopic fluid background and microscopic kinetic correction, both fully coupled to each other. A similar manner of discretization is proposed to that used in the recent \texttt{STRUPHY} code \cite{Holderied_Possanner_Wang_2021, Holderied_2022, Li_et_al_2023} with a finite-element model for the background and a pseudo-particle/PiC model for the correction.

        The fluid background satisfies the full, non-linear, resistive, compressible, Hall MHD equations. \cite{Laakmann_Hu_Farrell_2022} introduces finite-element(-in-space) implicit timesteppers for the incompressible analogue to this system with structure-preserving (SP) properties in the ideal case, alongside parameter-robust preconditioners. We show that these timesteppers can derive from a finite-element-in-time (FET) (and finite-element-in-space) interpretation. The benefits of this reformulation are discussed, including the derivation of timesteppers that are higher order in time, and the quantifiable dissipative SP properties in the non-ideal, resistive case.
        
        We discuss possible options for extending this FET approach to timesteppers for the compressible case.

        The kinetic corrections satisfy linearized Boltzmann equations. Using a Lénard--Bernstein collision operator, these take Fokker--Planck-like forms \cite{Fokker_1914, Planck_1917} wherein pseudo-particles in the numerical model obey the neoclassical transport equations, with particle-independent Brownian drift terms. This offers a rigorous methodology for incorporating collisions into the particle transport model, without coupling the equations of motions for each particle.
        
        Works by Chen, Chacón et al. \cite{Chen_Chacón_Barnes_2011, Chacón_Chen_Barnes_2013, Chen_Chacón_2014, Chen_Chacón_2015} have developed structure-preserving particle pushers for neoclassical transport in the Vlasov equations, derived from Crank--Nicolson integrators. We show these too can can derive from a FET interpretation, similarly offering potential extensions to higher-order-in-time particle pushers. The FET formulation is used also to consider how the stochastic drift terms can be incorporated into the pushers. Stochastic gyrokinetic expansions are also discussed.

        Different options for the numerical implementation of these schemes are considered.

        Due to the efficacy of FET in the development of SP timesteppers for both the fluid and kinetic component, we hope this approach will prove effective in the future for developing SP timesteppers for the full hybrid model. We hope this will give us the opportunity to incorporate previously inaccessible kinetic effects into the highly effective, modern, finite-element MHD models.
    \end{abstract}
    
    
    \newpage
    \tableofcontents
    
    
    \newpage
    \pagenumbering{arabic}
    %\linenumbers\renewcommand\thelinenumber{\color{black!50}\arabic{linenumber}}
            \input{0 - introduction/main.tex}
        \part{Research}
            \input{1 - low-noise PiC models/main.tex}
            \input{2 - kinetic component/main.tex}
            \input{3 - fluid component/main.tex}
            \input{4 - numerical implementation/main.tex}
        \part{Project Overview}
            \input{5 - research plan/main.tex}
            \input{6 - summary/main.tex}
    
    
    %\section{}
    \newpage
    \pagenumbering{gobble}
        \printbibliography


    \newpage
    \pagenumbering{roman}
    \appendix
        \part{Appendices}
            \input{8 - Hilbert complexes/main.tex}
            \input{9 - weak conservation proofs/main.tex}
\end{document}

            \documentclass[12pt, a4paper]{report}

\input{template/main.tex}

\title{\BA{Title in Progress...}}
\author{Boris Andrews}
\affil{Mathematical Institute, University of Oxford}
\date{\today}


\begin{document}
    \pagenumbering{gobble}
    \maketitle
    
    
    \begin{abstract}
        Magnetic confinement reactors---in particular tokamaks---offer one of the most promising options for achieving practical nuclear fusion, with the potential to provide virtually limitless, clean energy. The theoretical and numerical modeling of tokamak plasmas is simultaneously an essential component of effective reactor design, and a great research barrier. Tokamak operational conditions exhibit comparatively low Knudsen numbers. Kinetic effects, including kinetic waves and instabilities, Landau damping, bump-on-tail instabilities and more, are therefore highly influential in tokamak plasma dynamics. Purely fluid models are inherently incapable of capturing these effects, whereas the high dimensionality in purely kinetic models render them practically intractable for most relevant purposes.

        We consider a $\delta\!f$ decomposition model, with a macroscopic fluid background and microscopic kinetic correction, both fully coupled to each other. A similar manner of discretization is proposed to that used in the recent \texttt{STRUPHY} code \cite{Holderied_Possanner_Wang_2021, Holderied_2022, Li_et_al_2023} with a finite-element model for the background and a pseudo-particle/PiC model for the correction.

        The fluid background satisfies the full, non-linear, resistive, compressible, Hall MHD equations. \cite{Laakmann_Hu_Farrell_2022} introduces finite-element(-in-space) implicit timesteppers for the incompressible analogue to this system with structure-preserving (SP) properties in the ideal case, alongside parameter-robust preconditioners. We show that these timesteppers can derive from a finite-element-in-time (FET) (and finite-element-in-space) interpretation. The benefits of this reformulation are discussed, including the derivation of timesteppers that are higher order in time, and the quantifiable dissipative SP properties in the non-ideal, resistive case.
        
        We discuss possible options for extending this FET approach to timesteppers for the compressible case.

        The kinetic corrections satisfy linearized Boltzmann equations. Using a Lénard--Bernstein collision operator, these take Fokker--Planck-like forms \cite{Fokker_1914, Planck_1917} wherein pseudo-particles in the numerical model obey the neoclassical transport equations, with particle-independent Brownian drift terms. This offers a rigorous methodology for incorporating collisions into the particle transport model, without coupling the equations of motions for each particle.
        
        Works by Chen, Chacón et al. \cite{Chen_Chacón_Barnes_2011, Chacón_Chen_Barnes_2013, Chen_Chacón_2014, Chen_Chacón_2015} have developed structure-preserving particle pushers for neoclassical transport in the Vlasov equations, derived from Crank--Nicolson integrators. We show these too can can derive from a FET interpretation, similarly offering potential extensions to higher-order-in-time particle pushers. The FET formulation is used also to consider how the stochastic drift terms can be incorporated into the pushers. Stochastic gyrokinetic expansions are also discussed.

        Different options for the numerical implementation of these schemes are considered.

        Due to the efficacy of FET in the development of SP timesteppers for both the fluid and kinetic component, we hope this approach will prove effective in the future for developing SP timesteppers for the full hybrid model. We hope this will give us the opportunity to incorporate previously inaccessible kinetic effects into the highly effective, modern, finite-element MHD models.
    \end{abstract}
    
    
    \newpage
    \tableofcontents
    
    
    \newpage
    \pagenumbering{arabic}
    %\linenumbers\renewcommand\thelinenumber{\color{black!50}\arabic{linenumber}}
            \input{0 - introduction/main.tex}
        \part{Research}
            \input{1 - low-noise PiC models/main.tex}
            \input{2 - kinetic component/main.tex}
            \input{3 - fluid component/main.tex}
            \input{4 - numerical implementation/main.tex}
        \part{Project Overview}
            \input{5 - research plan/main.tex}
            \input{6 - summary/main.tex}
    
    
    %\section{}
    \newpage
    \pagenumbering{gobble}
        \printbibliography


    \newpage
    \pagenumbering{roman}
    \appendix
        \part{Appendices}
            \input{8 - Hilbert complexes/main.tex}
            \input{9 - weak conservation proofs/main.tex}
\end{document}

    
    
    %\section{}
    \newpage
    \pagenumbering{gobble}
        \printbibliography


    \newpage
    \pagenumbering{roman}
    \appendix
        \part{Appendices}
            \documentclass[12pt, a4paper]{report}

\input{template/main.tex}

\title{\BA{Title in Progress...}}
\author{Boris Andrews}
\affil{Mathematical Institute, University of Oxford}
\date{\today}


\begin{document}
    \pagenumbering{gobble}
    \maketitle
    
    
    \begin{abstract}
        Magnetic confinement reactors---in particular tokamaks---offer one of the most promising options for achieving practical nuclear fusion, with the potential to provide virtually limitless, clean energy. The theoretical and numerical modeling of tokamak plasmas is simultaneously an essential component of effective reactor design, and a great research barrier. Tokamak operational conditions exhibit comparatively low Knudsen numbers. Kinetic effects, including kinetic waves and instabilities, Landau damping, bump-on-tail instabilities and more, are therefore highly influential in tokamak plasma dynamics. Purely fluid models are inherently incapable of capturing these effects, whereas the high dimensionality in purely kinetic models render them practically intractable for most relevant purposes.

        We consider a $\delta\!f$ decomposition model, with a macroscopic fluid background and microscopic kinetic correction, both fully coupled to each other. A similar manner of discretization is proposed to that used in the recent \texttt{STRUPHY} code \cite{Holderied_Possanner_Wang_2021, Holderied_2022, Li_et_al_2023} with a finite-element model for the background and a pseudo-particle/PiC model for the correction.

        The fluid background satisfies the full, non-linear, resistive, compressible, Hall MHD equations. \cite{Laakmann_Hu_Farrell_2022} introduces finite-element(-in-space) implicit timesteppers for the incompressible analogue to this system with structure-preserving (SP) properties in the ideal case, alongside parameter-robust preconditioners. We show that these timesteppers can derive from a finite-element-in-time (FET) (and finite-element-in-space) interpretation. The benefits of this reformulation are discussed, including the derivation of timesteppers that are higher order in time, and the quantifiable dissipative SP properties in the non-ideal, resistive case.
        
        We discuss possible options for extending this FET approach to timesteppers for the compressible case.

        The kinetic corrections satisfy linearized Boltzmann equations. Using a Lénard--Bernstein collision operator, these take Fokker--Planck-like forms \cite{Fokker_1914, Planck_1917} wherein pseudo-particles in the numerical model obey the neoclassical transport equations, with particle-independent Brownian drift terms. This offers a rigorous methodology for incorporating collisions into the particle transport model, without coupling the equations of motions for each particle.
        
        Works by Chen, Chacón et al. \cite{Chen_Chacón_Barnes_2011, Chacón_Chen_Barnes_2013, Chen_Chacón_2014, Chen_Chacón_2015} have developed structure-preserving particle pushers for neoclassical transport in the Vlasov equations, derived from Crank--Nicolson integrators. We show these too can can derive from a FET interpretation, similarly offering potential extensions to higher-order-in-time particle pushers. The FET formulation is used also to consider how the stochastic drift terms can be incorporated into the pushers. Stochastic gyrokinetic expansions are also discussed.

        Different options for the numerical implementation of these schemes are considered.

        Due to the efficacy of FET in the development of SP timesteppers for both the fluid and kinetic component, we hope this approach will prove effective in the future for developing SP timesteppers for the full hybrid model. We hope this will give us the opportunity to incorporate previously inaccessible kinetic effects into the highly effective, modern, finite-element MHD models.
    \end{abstract}
    
    
    \newpage
    \tableofcontents
    
    
    \newpage
    \pagenumbering{arabic}
    %\linenumbers\renewcommand\thelinenumber{\color{black!50}\arabic{linenumber}}
            \input{0 - introduction/main.tex}
        \part{Research}
            \input{1 - low-noise PiC models/main.tex}
            \input{2 - kinetic component/main.tex}
            \input{3 - fluid component/main.tex}
            \input{4 - numerical implementation/main.tex}
        \part{Project Overview}
            \input{5 - research plan/main.tex}
            \input{6 - summary/main.tex}
    
    
    %\section{}
    \newpage
    \pagenumbering{gobble}
        \printbibliography


    \newpage
    \pagenumbering{roman}
    \appendix
        \part{Appendices}
            \input{8 - Hilbert complexes/main.tex}
            \input{9 - weak conservation proofs/main.tex}
\end{document}

            \documentclass[12pt, a4paper]{report}

\input{template/main.tex}

\title{\BA{Title in Progress...}}
\author{Boris Andrews}
\affil{Mathematical Institute, University of Oxford}
\date{\today}


\begin{document}
    \pagenumbering{gobble}
    \maketitle
    
    
    \begin{abstract}
        Magnetic confinement reactors---in particular tokamaks---offer one of the most promising options for achieving practical nuclear fusion, with the potential to provide virtually limitless, clean energy. The theoretical and numerical modeling of tokamak plasmas is simultaneously an essential component of effective reactor design, and a great research barrier. Tokamak operational conditions exhibit comparatively low Knudsen numbers. Kinetic effects, including kinetic waves and instabilities, Landau damping, bump-on-tail instabilities and more, are therefore highly influential in tokamak plasma dynamics. Purely fluid models are inherently incapable of capturing these effects, whereas the high dimensionality in purely kinetic models render them practically intractable for most relevant purposes.

        We consider a $\delta\!f$ decomposition model, with a macroscopic fluid background and microscopic kinetic correction, both fully coupled to each other. A similar manner of discretization is proposed to that used in the recent \texttt{STRUPHY} code \cite{Holderied_Possanner_Wang_2021, Holderied_2022, Li_et_al_2023} with a finite-element model for the background and a pseudo-particle/PiC model for the correction.

        The fluid background satisfies the full, non-linear, resistive, compressible, Hall MHD equations. \cite{Laakmann_Hu_Farrell_2022} introduces finite-element(-in-space) implicit timesteppers for the incompressible analogue to this system with structure-preserving (SP) properties in the ideal case, alongside parameter-robust preconditioners. We show that these timesteppers can derive from a finite-element-in-time (FET) (and finite-element-in-space) interpretation. The benefits of this reformulation are discussed, including the derivation of timesteppers that are higher order in time, and the quantifiable dissipative SP properties in the non-ideal, resistive case.
        
        We discuss possible options for extending this FET approach to timesteppers for the compressible case.

        The kinetic corrections satisfy linearized Boltzmann equations. Using a Lénard--Bernstein collision operator, these take Fokker--Planck-like forms \cite{Fokker_1914, Planck_1917} wherein pseudo-particles in the numerical model obey the neoclassical transport equations, with particle-independent Brownian drift terms. This offers a rigorous methodology for incorporating collisions into the particle transport model, without coupling the equations of motions for each particle.
        
        Works by Chen, Chacón et al. \cite{Chen_Chacón_Barnes_2011, Chacón_Chen_Barnes_2013, Chen_Chacón_2014, Chen_Chacón_2015} have developed structure-preserving particle pushers for neoclassical transport in the Vlasov equations, derived from Crank--Nicolson integrators. We show these too can can derive from a FET interpretation, similarly offering potential extensions to higher-order-in-time particle pushers. The FET formulation is used also to consider how the stochastic drift terms can be incorporated into the pushers. Stochastic gyrokinetic expansions are also discussed.

        Different options for the numerical implementation of these schemes are considered.

        Due to the efficacy of FET in the development of SP timesteppers for both the fluid and kinetic component, we hope this approach will prove effective in the future for developing SP timesteppers for the full hybrid model. We hope this will give us the opportunity to incorporate previously inaccessible kinetic effects into the highly effective, modern, finite-element MHD models.
    \end{abstract}
    
    
    \newpage
    \tableofcontents
    
    
    \newpage
    \pagenumbering{arabic}
    %\linenumbers\renewcommand\thelinenumber{\color{black!50}\arabic{linenumber}}
            \input{0 - introduction/main.tex}
        \part{Research}
            \input{1 - low-noise PiC models/main.tex}
            \input{2 - kinetic component/main.tex}
            \input{3 - fluid component/main.tex}
            \input{4 - numerical implementation/main.tex}
        \part{Project Overview}
            \input{5 - research plan/main.tex}
            \input{6 - summary/main.tex}
    
    
    %\section{}
    \newpage
    \pagenumbering{gobble}
        \printbibliography


    \newpage
    \pagenumbering{roman}
    \appendix
        \part{Appendices}
            \input{8 - Hilbert complexes/main.tex}
            \input{9 - weak conservation proofs/main.tex}
\end{document}

\end{document}

            \documentclass[12pt, a4paper]{report}

\documentclass[12pt, a4paper]{report}

\input{template/main.tex}

\title{\BA{Title in Progress...}}
\author{Boris Andrews}
\affil{Mathematical Institute, University of Oxford}
\date{\today}


\begin{document}
    \pagenumbering{gobble}
    \maketitle
    
    
    \begin{abstract}
        Magnetic confinement reactors---in particular tokamaks---offer one of the most promising options for achieving practical nuclear fusion, with the potential to provide virtually limitless, clean energy. The theoretical and numerical modeling of tokamak plasmas is simultaneously an essential component of effective reactor design, and a great research barrier. Tokamak operational conditions exhibit comparatively low Knudsen numbers. Kinetic effects, including kinetic waves and instabilities, Landau damping, bump-on-tail instabilities and more, are therefore highly influential in tokamak plasma dynamics. Purely fluid models are inherently incapable of capturing these effects, whereas the high dimensionality in purely kinetic models render them practically intractable for most relevant purposes.

        We consider a $\delta\!f$ decomposition model, with a macroscopic fluid background and microscopic kinetic correction, both fully coupled to each other. A similar manner of discretization is proposed to that used in the recent \texttt{STRUPHY} code \cite{Holderied_Possanner_Wang_2021, Holderied_2022, Li_et_al_2023} with a finite-element model for the background and a pseudo-particle/PiC model for the correction.

        The fluid background satisfies the full, non-linear, resistive, compressible, Hall MHD equations. \cite{Laakmann_Hu_Farrell_2022} introduces finite-element(-in-space) implicit timesteppers for the incompressible analogue to this system with structure-preserving (SP) properties in the ideal case, alongside parameter-robust preconditioners. We show that these timesteppers can derive from a finite-element-in-time (FET) (and finite-element-in-space) interpretation. The benefits of this reformulation are discussed, including the derivation of timesteppers that are higher order in time, and the quantifiable dissipative SP properties in the non-ideal, resistive case.
        
        We discuss possible options for extending this FET approach to timesteppers for the compressible case.

        The kinetic corrections satisfy linearized Boltzmann equations. Using a Lénard--Bernstein collision operator, these take Fokker--Planck-like forms \cite{Fokker_1914, Planck_1917} wherein pseudo-particles in the numerical model obey the neoclassical transport equations, with particle-independent Brownian drift terms. This offers a rigorous methodology for incorporating collisions into the particle transport model, without coupling the equations of motions for each particle.
        
        Works by Chen, Chacón et al. \cite{Chen_Chacón_Barnes_2011, Chacón_Chen_Barnes_2013, Chen_Chacón_2014, Chen_Chacón_2015} have developed structure-preserving particle pushers for neoclassical transport in the Vlasov equations, derived from Crank--Nicolson integrators. We show these too can can derive from a FET interpretation, similarly offering potential extensions to higher-order-in-time particle pushers. The FET formulation is used also to consider how the stochastic drift terms can be incorporated into the pushers. Stochastic gyrokinetic expansions are also discussed.

        Different options for the numerical implementation of these schemes are considered.

        Due to the efficacy of FET in the development of SP timesteppers for both the fluid and kinetic component, we hope this approach will prove effective in the future for developing SP timesteppers for the full hybrid model. We hope this will give us the opportunity to incorporate previously inaccessible kinetic effects into the highly effective, modern, finite-element MHD models.
    \end{abstract}
    
    
    \newpage
    \tableofcontents
    
    
    \newpage
    \pagenumbering{arabic}
    %\linenumbers\renewcommand\thelinenumber{\color{black!50}\arabic{linenumber}}
            \input{0 - introduction/main.tex}
        \part{Research}
            \input{1 - low-noise PiC models/main.tex}
            \input{2 - kinetic component/main.tex}
            \input{3 - fluid component/main.tex}
            \input{4 - numerical implementation/main.tex}
        \part{Project Overview}
            \input{5 - research plan/main.tex}
            \input{6 - summary/main.tex}
    
    
    %\section{}
    \newpage
    \pagenumbering{gobble}
        \printbibliography


    \newpage
    \pagenumbering{roman}
    \appendix
        \part{Appendices}
            \input{8 - Hilbert complexes/main.tex}
            \input{9 - weak conservation proofs/main.tex}
\end{document}


\title{\BA{Title in Progress...}}
\author{Boris Andrews}
\affil{Mathematical Institute, University of Oxford}
\date{\today}


\begin{document}
    \pagenumbering{gobble}
    \maketitle
    
    
    \begin{abstract}
        Magnetic confinement reactors---in particular tokamaks---offer one of the most promising options for achieving practical nuclear fusion, with the potential to provide virtually limitless, clean energy. The theoretical and numerical modeling of tokamak plasmas is simultaneously an essential component of effective reactor design, and a great research barrier. Tokamak operational conditions exhibit comparatively low Knudsen numbers. Kinetic effects, including kinetic waves and instabilities, Landau damping, bump-on-tail instabilities and more, are therefore highly influential in tokamak plasma dynamics. Purely fluid models are inherently incapable of capturing these effects, whereas the high dimensionality in purely kinetic models render them practically intractable for most relevant purposes.

        We consider a $\delta\!f$ decomposition model, with a macroscopic fluid background and microscopic kinetic correction, both fully coupled to each other. A similar manner of discretization is proposed to that used in the recent \texttt{STRUPHY} code \cite{Holderied_Possanner_Wang_2021, Holderied_2022, Li_et_al_2023} with a finite-element model for the background and a pseudo-particle/PiC model for the correction.

        The fluid background satisfies the full, non-linear, resistive, compressible, Hall MHD equations. \cite{Laakmann_Hu_Farrell_2022} introduces finite-element(-in-space) implicit timesteppers for the incompressible analogue to this system with structure-preserving (SP) properties in the ideal case, alongside parameter-robust preconditioners. We show that these timesteppers can derive from a finite-element-in-time (FET) (and finite-element-in-space) interpretation. The benefits of this reformulation are discussed, including the derivation of timesteppers that are higher order in time, and the quantifiable dissipative SP properties in the non-ideal, resistive case.
        
        We discuss possible options for extending this FET approach to timesteppers for the compressible case.

        The kinetic corrections satisfy linearized Boltzmann equations. Using a Lénard--Bernstein collision operator, these take Fokker--Planck-like forms \cite{Fokker_1914, Planck_1917} wherein pseudo-particles in the numerical model obey the neoclassical transport equations, with particle-independent Brownian drift terms. This offers a rigorous methodology for incorporating collisions into the particle transport model, without coupling the equations of motions for each particle.
        
        Works by Chen, Chacón et al. \cite{Chen_Chacón_Barnes_2011, Chacón_Chen_Barnes_2013, Chen_Chacón_2014, Chen_Chacón_2015} have developed structure-preserving particle pushers for neoclassical transport in the Vlasov equations, derived from Crank--Nicolson integrators. We show these too can can derive from a FET interpretation, similarly offering potential extensions to higher-order-in-time particle pushers. The FET formulation is used also to consider how the stochastic drift terms can be incorporated into the pushers. Stochastic gyrokinetic expansions are also discussed.

        Different options for the numerical implementation of these schemes are considered.

        Due to the efficacy of FET in the development of SP timesteppers for both the fluid and kinetic component, we hope this approach will prove effective in the future for developing SP timesteppers for the full hybrid model. We hope this will give us the opportunity to incorporate previously inaccessible kinetic effects into the highly effective, modern, finite-element MHD models.
    \end{abstract}
    
    
    \newpage
    \tableofcontents
    
    
    \newpage
    \pagenumbering{arabic}
    %\linenumbers\renewcommand\thelinenumber{\color{black!50}\arabic{linenumber}}
            \documentclass[12pt, a4paper]{report}

\input{template/main.tex}

\title{\BA{Title in Progress...}}
\author{Boris Andrews}
\affil{Mathematical Institute, University of Oxford}
\date{\today}


\begin{document}
    \pagenumbering{gobble}
    \maketitle
    
    
    \begin{abstract}
        Magnetic confinement reactors---in particular tokamaks---offer one of the most promising options for achieving practical nuclear fusion, with the potential to provide virtually limitless, clean energy. The theoretical and numerical modeling of tokamak plasmas is simultaneously an essential component of effective reactor design, and a great research barrier. Tokamak operational conditions exhibit comparatively low Knudsen numbers. Kinetic effects, including kinetic waves and instabilities, Landau damping, bump-on-tail instabilities and more, are therefore highly influential in tokamak plasma dynamics. Purely fluid models are inherently incapable of capturing these effects, whereas the high dimensionality in purely kinetic models render them practically intractable for most relevant purposes.

        We consider a $\delta\!f$ decomposition model, with a macroscopic fluid background and microscopic kinetic correction, both fully coupled to each other. A similar manner of discretization is proposed to that used in the recent \texttt{STRUPHY} code \cite{Holderied_Possanner_Wang_2021, Holderied_2022, Li_et_al_2023} with a finite-element model for the background and a pseudo-particle/PiC model for the correction.

        The fluid background satisfies the full, non-linear, resistive, compressible, Hall MHD equations. \cite{Laakmann_Hu_Farrell_2022} introduces finite-element(-in-space) implicit timesteppers for the incompressible analogue to this system with structure-preserving (SP) properties in the ideal case, alongside parameter-robust preconditioners. We show that these timesteppers can derive from a finite-element-in-time (FET) (and finite-element-in-space) interpretation. The benefits of this reformulation are discussed, including the derivation of timesteppers that are higher order in time, and the quantifiable dissipative SP properties in the non-ideal, resistive case.
        
        We discuss possible options for extending this FET approach to timesteppers for the compressible case.

        The kinetic corrections satisfy linearized Boltzmann equations. Using a Lénard--Bernstein collision operator, these take Fokker--Planck-like forms \cite{Fokker_1914, Planck_1917} wherein pseudo-particles in the numerical model obey the neoclassical transport equations, with particle-independent Brownian drift terms. This offers a rigorous methodology for incorporating collisions into the particle transport model, without coupling the equations of motions for each particle.
        
        Works by Chen, Chacón et al. \cite{Chen_Chacón_Barnes_2011, Chacón_Chen_Barnes_2013, Chen_Chacón_2014, Chen_Chacón_2015} have developed structure-preserving particle pushers for neoclassical transport in the Vlasov equations, derived from Crank--Nicolson integrators. We show these too can can derive from a FET interpretation, similarly offering potential extensions to higher-order-in-time particle pushers. The FET formulation is used also to consider how the stochastic drift terms can be incorporated into the pushers. Stochastic gyrokinetic expansions are also discussed.

        Different options for the numerical implementation of these schemes are considered.

        Due to the efficacy of FET in the development of SP timesteppers for both the fluid and kinetic component, we hope this approach will prove effective in the future for developing SP timesteppers for the full hybrid model. We hope this will give us the opportunity to incorporate previously inaccessible kinetic effects into the highly effective, modern, finite-element MHD models.
    \end{abstract}
    
    
    \newpage
    \tableofcontents
    
    
    \newpage
    \pagenumbering{arabic}
    %\linenumbers\renewcommand\thelinenumber{\color{black!50}\arabic{linenumber}}
            \input{0 - introduction/main.tex}
        \part{Research}
            \input{1 - low-noise PiC models/main.tex}
            \input{2 - kinetic component/main.tex}
            \input{3 - fluid component/main.tex}
            \input{4 - numerical implementation/main.tex}
        \part{Project Overview}
            \input{5 - research plan/main.tex}
            \input{6 - summary/main.tex}
    
    
    %\section{}
    \newpage
    \pagenumbering{gobble}
        \printbibliography


    \newpage
    \pagenumbering{roman}
    \appendix
        \part{Appendices}
            \input{8 - Hilbert complexes/main.tex}
            \input{9 - weak conservation proofs/main.tex}
\end{document}

        \part{Research}
            \documentclass[12pt, a4paper]{report}

\input{template/main.tex}

\title{\BA{Title in Progress...}}
\author{Boris Andrews}
\affil{Mathematical Institute, University of Oxford}
\date{\today}


\begin{document}
    \pagenumbering{gobble}
    \maketitle
    
    
    \begin{abstract}
        Magnetic confinement reactors---in particular tokamaks---offer one of the most promising options for achieving practical nuclear fusion, with the potential to provide virtually limitless, clean energy. The theoretical and numerical modeling of tokamak plasmas is simultaneously an essential component of effective reactor design, and a great research barrier. Tokamak operational conditions exhibit comparatively low Knudsen numbers. Kinetic effects, including kinetic waves and instabilities, Landau damping, bump-on-tail instabilities and more, are therefore highly influential in tokamak plasma dynamics. Purely fluid models are inherently incapable of capturing these effects, whereas the high dimensionality in purely kinetic models render them practically intractable for most relevant purposes.

        We consider a $\delta\!f$ decomposition model, with a macroscopic fluid background and microscopic kinetic correction, both fully coupled to each other. A similar manner of discretization is proposed to that used in the recent \texttt{STRUPHY} code \cite{Holderied_Possanner_Wang_2021, Holderied_2022, Li_et_al_2023} with a finite-element model for the background and a pseudo-particle/PiC model for the correction.

        The fluid background satisfies the full, non-linear, resistive, compressible, Hall MHD equations. \cite{Laakmann_Hu_Farrell_2022} introduces finite-element(-in-space) implicit timesteppers for the incompressible analogue to this system with structure-preserving (SP) properties in the ideal case, alongside parameter-robust preconditioners. We show that these timesteppers can derive from a finite-element-in-time (FET) (and finite-element-in-space) interpretation. The benefits of this reformulation are discussed, including the derivation of timesteppers that are higher order in time, and the quantifiable dissipative SP properties in the non-ideal, resistive case.
        
        We discuss possible options for extending this FET approach to timesteppers for the compressible case.

        The kinetic corrections satisfy linearized Boltzmann equations. Using a Lénard--Bernstein collision operator, these take Fokker--Planck-like forms \cite{Fokker_1914, Planck_1917} wherein pseudo-particles in the numerical model obey the neoclassical transport equations, with particle-independent Brownian drift terms. This offers a rigorous methodology for incorporating collisions into the particle transport model, without coupling the equations of motions for each particle.
        
        Works by Chen, Chacón et al. \cite{Chen_Chacón_Barnes_2011, Chacón_Chen_Barnes_2013, Chen_Chacón_2014, Chen_Chacón_2015} have developed structure-preserving particle pushers for neoclassical transport in the Vlasov equations, derived from Crank--Nicolson integrators. We show these too can can derive from a FET interpretation, similarly offering potential extensions to higher-order-in-time particle pushers. The FET formulation is used also to consider how the stochastic drift terms can be incorporated into the pushers. Stochastic gyrokinetic expansions are also discussed.

        Different options for the numerical implementation of these schemes are considered.

        Due to the efficacy of FET in the development of SP timesteppers for both the fluid and kinetic component, we hope this approach will prove effective in the future for developing SP timesteppers for the full hybrid model. We hope this will give us the opportunity to incorporate previously inaccessible kinetic effects into the highly effective, modern, finite-element MHD models.
    \end{abstract}
    
    
    \newpage
    \tableofcontents
    
    
    \newpage
    \pagenumbering{arabic}
    %\linenumbers\renewcommand\thelinenumber{\color{black!50}\arabic{linenumber}}
            \input{0 - introduction/main.tex}
        \part{Research}
            \input{1 - low-noise PiC models/main.tex}
            \input{2 - kinetic component/main.tex}
            \input{3 - fluid component/main.tex}
            \input{4 - numerical implementation/main.tex}
        \part{Project Overview}
            \input{5 - research plan/main.tex}
            \input{6 - summary/main.tex}
    
    
    %\section{}
    \newpage
    \pagenumbering{gobble}
        \printbibliography


    \newpage
    \pagenumbering{roman}
    \appendix
        \part{Appendices}
            \input{8 - Hilbert complexes/main.tex}
            \input{9 - weak conservation proofs/main.tex}
\end{document}

            \documentclass[12pt, a4paper]{report}

\input{template/main.tex}

\title{\BA{Title in Progress...}}
\author{Boris Andrews}
\affil{Mathematical Institute, University of Oxford}
\date{\today}


\begin{document}
    \pagenumbering{gobble}
    \maketitle
    
    
    \begin{abstract}
        Magnetic confinement reactors---in particular tokamaks---offer one of the most promising options for achieving practical nuclear fusion, with the potential to provide virtually limitless, clean energy. The theoretical and numerical modeling of tokamak plasmas is simultaneously an essential component of effective reactor design, and a great research barrier. Tokamak operational conditions exhibit comparatively low Knudsen numbers. Kinetic effects, including kinetic waves and instabilities, Landau damping, bump-on-tail instabilities and more, are therefore highly influential in tokamak plasma dynamics. Purely fluid models are inherently incapable of capturing these effects, whereas the high dimensionality in purely kinetic models render them practically intractable for most relevant purposes.

        We consider a $\delta\!f$ decomposition model, with a macroscopic fluid background and microscopic kinetic correction, both fully coupled to each other. A similar manner of discretization is proposed to that used in the recent \texttt{STRUPHY} code \cite{Holderied_Possanner_Wang_2021, Holderied_2022, Li_et_al_2023} with a finite-element model for the background and a pseudo-particle/PiC model for the correction.

        The fluid background satisfies the full, non-linear, resistive, compressible, Hall MHD equations. \cite{Laakmann_Hu_Farrell_2022} introduces finite-element(-in-space) implicit timesteppers for the incompressible analogue to this system with structure-preserving (SP) properties in the ideal case, alongside parameter-robust preconditioners. We show that these timesteppers can derive from a finite-element-in-time (FET) (and finite-element-in-space) interpretation. The benefits of this reformulation are discussed, including the derivation of timesteppers that are higher order in time, and the quantifiable dissipative SP properties in the non-ideal, resistive case.
        
        We discuss possible options for extending this FET approach to timesteppers for the compressible case.

        The kinetic corrections satisfy linearized Boltzmann equations. Using a Lénard--Bernstein collision operator, these take Fokker--Planck-like forms \cite{Fokker_1914, Planck_1917} wherein pseudo-particles in the numerical model obey the neoclassical transport equations, with particle-independent Brownian drift terms. This offers a rigorous methodology for incorporating collisions into the particle transport model, without coupling the equations of motions for each particle.
        
        Works by Chen, Chacón et al. \cite{Chen_Chacón_Barnes_2011, Chacón_Chen_Barnes_2013, Chen_Chacón_2014, Chen_Chacón_2015} have developed structure-preserving particle pushers for neoclassical transport in the Vlasov equations, derived from Crank--Nicolson integrators. We show these too can can derive from a FET interpretation, similarly offering potential extensions to higher-order-in-time particle pushers. The FET formulation is used also to consider how the stochastic drift terms can be incorporated into the pushers. Stochastic gyrokinetic expansions are also discussed.

        Different options for the numerical implementation of these schemes are considered.

        Due to the efficacy of FET in the development of SP timesteppers for both the fluid and kinetic component, we hope this approach will prove effective in the future for developing SP timesteppers for the full hybrid model. We hope this will give us the opportunity to incorporate previously inaccessible kinetic effects into the highly effective, modern, finite-element MHD models.
    \end{abstract}
    
    
    \newpage
    \tableofcontents
    
    
    \newpage
    \pagenumbering{arabic}
    %\linenumbers\renewcommand\thelinenumber{\color{black!50}\arabic{linenumber}}
            \input{0 - introduction/main.tex}
        \part{Research}
            \input{1 - low-noise PiC models/main.tex}
            \input{2 - kinetic component/main.tex}
            \input{3 - fluid component/main.tex}
            \input{4 - numerical implementation/main.tex}
        \part{Project Overview}
            \input{5 - research plan/main.tex}
            \input{6 - summary/main.tex}
    
    
    %\section{}
    \newpage
    \pagenumbering{gobble}
        \printbibliography


    \newpage
    \pagenumbering{roman}
    \appendix
        \part{Appendices}
            \input{8 - Hilbert complexes/main.tex}
            \input{9 - weak conservation proofs/main.tex}
\end{document}

            \documentclass[12pt, a4paper]{report}

\input{template/main.tex}

\title{\BA{Title in Progress...}}
\author{Boris Andrews}
\affil{Mathematical Institute, University of Oxford}
\date{\today}


\begin{document}
    \pagenumbering{gobble}
    \maketitle
    
    
    \begin{abstract}
        Magnetic confinement reactors---in particular tokamaks---offer one of the most promising options for achieving practical nuclear fusion, with the potential to provide virtually limitless, clean energy. The theoretical and numerical modeling of tokamak plasmas is simultaneously an essential component of effective reactor design, and a great research barrier. Tokamak operational conditions exhibit comparatively low Knudsen numbers. Kinetic effects, including kinetic waves and instabilities, Landau damping, bump-on-tail instabilities and more, are therefore highly influential in tokamak plasma dynamics. Purely fluid models are inherently incapable of capturing these effects, whereas the high dimensionality in purely kinetic models render them practically intractable for most relevant purposes.

        We consider a $\delta\!f$ decomposition model, with a macroscopic fluid background and microscopic kinetic correction, both fully coupled to each other. A similar manner of discretization is proposed to that used in the recent \texttt{STRUPHY} code \cite{Holderied_Possanner_Wang_2021, Holderied_2022, Li_et_al_2023} with a finite-element model for the background and a pseudo-particle/PiC model for the correction.

        The fluid background satisfies the full, non-linear, resistive, compressible, Hall MHD equations. \cite{Laakmann_Hu_Farrell_2022} introduces finite-element(-in-space) implicit timesteppers for the incompressible analogue to this system with structure-preserving (SP) properties in the ideal case, alongside parameter-robust preconditioners. We show that these timesteppers can derive from a finite-element-in-time (FET) (and finite-element-in-space) interpretation. The benefits of this reformulation are discussed, including the derivation of timesteppers that are higher order in time, and the quantifiable dissipative SP properties in the non-ideal, resistive case.
        
        We discuss possible options for extending this FET approach to timesteppers for the compressible case.

        The kinetic corrections satisfy linearized Boltzmann equations. Using a Lénard--Bernstein collision operator, these take Fokker--Planck-like forms \cite{Fokker_1914, Planck_1917} wherein pseudo-particles in the numerical model obey the neoclassical transport equations, with particle-independent Brownian drift terms. This offers a rigorous methodology for incorporating collisions into the particle transport model, without coupling the equations of motions for each particle.
        
        Works by Chen, Chacón et al. \cite{Chen_Chacón_Barnes_2011, Chacón_Chen_Barnes_2013, Chen_Chacón_2014, Chen_Chacón_2015} have developed structure-preserving particle pushers for neoclassical transport in the Vlasov equations, derived from Crank--Nicolson integrators. We show these too can can derive from a FET interpretation, similarly offering potential extensions to higher-order-in-time particle pushers. The FET formulation is used also to consider how the stochastic drift terms can be incorporated into the pushers. Stochastic gyrokinetic expansions are also discussed.

        Different options for the numerical implementation of these schemes are considered.

        Due to the efficacy of FET in the development of SP timesteppers for both the fluid and kinetic component, we hope this approach will prove effective in the future for developing SP timesteppers for the full hybrid model. We hope this will give us the opportunity to incorporate previously inaccessible kinetic effects into the highly effective, modern, finite-element MHD models.
    \end{abstract}
    
    
    \newpage
    \tableofcontents
    
    
    \newpage
    \pagenumbering{arabic}
    %\linenumbers\renewcommand\thelinenumber{\color{black!50}\arabic{linenumber}}
            \input{0 - introduction/main.tex}
        \part{Research}
            \input{1 - low-noise PiC models/main.tex}
            \input{2 - kinetic component/main.tex}
            \input{3 - fluid component/main.tex}
            \input{4 - numerical implementation/main.tex}
        \part{Project Overview}
            \input{5 - research plan/main.tex}
            \input{6 - summary/main.tex}
    
    
    %\section{}
    \newpage
    \pagenumbering{gobble}
        \printbibliography


    \newpage
    \pagenumbering{roman}
    \appendix
        \part{Appendices}
            \input{8 - Hilbert complexes/main.tex}
            \input{9 - weak conservation proofs/main.tex}
\end{document}

            \documentclass[12pt, a4paper]{report}

\input{template/main.tex}

\title{\BA{Title in Progress...}}
\author{Boris Andrews}
\affil{Mathematical Institute, University of Oxford}
\date{\today}


\begin{document}
    \pagenumbering{gobble}
    \maketitle
    
    
    \begin{abstract}
        Magnetic confinement reactors---in particular tokamaks---offer one of the most promising options for achieving practical nuclear fusion, with the potential to provide virtually limitless, clean energy. The theoretical and numerical modeling of tokamak plasmas is simultaneously an essential component of effective reactor design, and a great research barrier. Tokamak operational conditions exhibit comparatively low Knudsen numbers. Kinetic effects, including kinetic waves and instabilities, Landau damping, bump-on-tail instabilities and more, are therefore highly influential in tokamak plasma dynamics. Purely fluid models are inherently incapable of capturing these effects, whereas the high dimensionality in purely kinetic models render them practically intractable for most relevant purposes.

        We consider a $\delta\!f$ decomposition model, with a macroscopic fluid background and microscopic kinetic correction, both fully coupled to each other. A similar manner of discretization is proposed to that used in the recent \texttt{STRUPHY} code \cite{Holderied_Possanner_Wang_2021, Holderied_2022, Li_et_al_2023} with a finite-element model for the background and a pseudo-particle/PiC model for the correction.

        The fluid background satisfies the full, non-linear, resistive, compressible, Hall MHD equations. \cite{Laakmann_Hu_Farrell_2022} introduces finite-element(-in-space) implicit timesteppers for the incompressible analogue to this system with structure-preserving (SP) properties in the ideal case, alongside parameter-robust preconditioners. We show that these timesteppers can derive from a finite-element-in-time (FET) (and finite-element-in-space) interpretation. The benefits of this reformulation are discussed, including the derivation of timesteppers that are higher order in time, and the quantifiable dissipative SP properties in the non-ideal, resistive case.
        
        We discuss possible options for extending this FET approach to timesteppers for the compressible case.

        The kinetic corrections satisfy linearized Boltzmann equations. Using a Lénard--Bernstein collision operator, these take Fokker--Planck-like forms \cite{Fokker_1914, Planck_1917} wherein pseudo-particles in the numerical model obey the neoclassical transport equations, with particle-independent Brownian drift terms. This offers a rigorous methodology for incorporating collisions into the particle transport model, without coupling the equations of motions for each particle.
        
        Works by Chen, Chacón et al. \cite{Chen_Chacón_Barnes_2011, Chacón_Chen_Barnes_2013, Chen_Chacón_2014, Chen_Chacón_2015} have developed structure-preserving particle pushers for neoclassical transport in the Vlasov equations, derived from Crank--Nicolson integrators. We show these too can can derive from a FET interpretation, similarly offering potential extensions to higher-order-in-time particle pushers. The FET formulation is used also to consider how the stochastic drift terms can be incorporated into the pushers. Stochastic gyrokinetic expansions are also discussed.

        Different options for the numerical implementation of these schemes are considered.

        Due to the efficacy of FET in the development of SP timesteppers for both the fluid and kinetic component, we hope this approach will prove effective in the future for developing SP timesteppers for the full hybrid model. We hope this will give us the opportunity to incorporate previously inaccessible kinetic effects into the highly effective, modern, finite-element MHD models.
    \end{abstract}
    
    
    \newpage
    \tableofcontents
    
    
    \newpage
    \pagenumbering{arabic}
    %\linenumbers\renewcommand\thelinenumber{\color{black!50}\arabic{linenumber}}
            \input{0 - introduction/main.tex}
        \part{Research}
            \input{1 - low-noise PiC models/main.tex}
            \input{2 - kinetic component/main.tex}
            \input{3 - fluid component/main.tex}
            \input{4 - numerical implementation/main.tex}
        \part{Project Overview}
            \input{5 - research plan/main.tex}
            \input{6 - summary/main.tex}
    
    
    %\section{}
    \newpage
    \pagenumbering{gobble}
        \printbibliography


    \newpage
    \pagenumbering{roman}
    \appendix
        \part{Appendices}
            \input{8 - Hilbert complexes/main.tex}
            \input{9 - weak conservation proofs/main.tex}
\end{document}

        \part{Project Overview}
            \documentclass[12pt, a4paper]{report}

\input{template/main.tex}

\title{\BA{Title in Progress...}}
\author{Boris Andrews}
\affil{Mathematical Institute, University of Oxford}
\date{\today}


\begin{document}
    \pagenumbering{gobble}
    \maketitle
    
    
    \begin{abstract}
        Magnetic confinement reactors---in particular tokamaks---offer one of the most promising options for achieving practical nuclear fusion, with the potential to provide virtually limitless, clean energy. The theoretical and numerical modeling of tokamak plasmas is simultaneously an essential component of effective reactor design, and a great research barrier. Tokamak operational conditions exhibit comparatively low Knudsen numbers. Kinetic effects, including kinetic waves and instabilities, Landau damping, bump-on-tail instabilities and more, are therefore highly influential in tokamak plasma dynamics. Purely fluid models are inherently incapable of capturing these effects, whereas the high dimensionality in purely kinetic models render them practically intractable for most relevant purposes.

        We consider a $\delta\!f$ decomposition model, with a macroscopic fluid background and microscopic kinetic correction, both fully coupled to each other. A similar manner of discretization is proposed to that used in the recent \texttt{STRUPHY} code \cite{Holderied_Possanner_Wang_2021, Holderied_2022, Li_et_al_2023} with a finite-element model for the background and a pseudo-particle/PiC model for the correction.

        The fluid background satisfies the full, non-linear, resistive, compressible, Hall MHD equations. \cite{Laakmann_Hu_Farrell_2022} introduces finite-element(-in-space) implicit timesteppers for the incompressible analogue to this system with structure-preserving (SP) properties in the ideal case, alongside parameter-robust preconditioners. We show that these timesteppers can derive from a finite-element-in-time (FET) (and finite-element-in-space) interpretation. The benefits of this reformulation are discussed, including the derivation of timesteppers that are higher order in time, and the quantifiable dissipative SP properties in the non-ideal, resistive case.
        
        We discuss possible options for extending this FET approach to timesteppers for the compressible case.

        The kinetic corrections satisfy linearized Boltzmann equations. Using a Lénard--Bernstein collision operator, these take Fokker--Planck-like forms \cite{Fokker_1914, Planck_1917} wherein pseudo-particles in the numerical model obey the neoclassical transport equations, with particle-independent Brownian drift terms. This offers a rigorous methodology for incorporating collisions into the particle transport model, without coupling the equations of motions for each particle.
        
        Works by Chen, Chacón et al. \cite{Chen_Chacón_Barnes_2011, Chacón_Chen_Barnes_2013, Chen_Chacón_2014, Chen_Chacón_2015} have developed structure-preserving particle pushers for neoclassical transport in the Vlasov equations, derived from Crank--Nicolson integrators. We show these too can can derive from a FET interpretation, similarly offering potential extensions to higher-order-in-time particle pushers. The FET formulation is used also to consider how the stochastic drift terms can be incorporated into the pushers. Stochastic gyrokinetic expansions are also discussed.

        Different options for the numerical implementation of these schemes are considered.

        Due to the efficacy of FET in the development of SP timesteppers for both the fluid and kinetic component, we hope this approach will prove effective in the future for developing SP timesteppers for the full hybrid model. We hope this will give us the opportunity to incorporate previously inaccessible kinetic effects into the highly effective, modern, finite-element MHD models.
    \end{abstract}
    
    
    \newpage
    \tableofcontents
    
    
    \newpage
    \pagenumbering{arabic}
    %\linenumbers\renewcommand\thelinenumber{\color{black!50}\arabic{linenumber}}
            \input{0 - introduction/main.tex}
        \part{Research}
            \input{1 - low-noise PiC models/main.tex}
            \input{2 - kinetic component/main.tex}
            \input{3 - fluid component/main.tex}
            \input{4 - numerical implementation/main.tex}
        \part{Project Overview}
            \input{5 - research plan/main.tex}
            \input{6 - summary/main.tex}
    
    
    %\section{}
    \newpage
    \pagenumbering{gobble}
        \printbibliography


    \newpage
    \pagenumbering{roman}
    \appendix
        \part{Appendices}
            \input{8 - Hilbert complexes/main.tex}
            \input{9 - weak conservation proofs/main.tex}
\end{document}

            \documentclass[12pt, a4paper]{report}

\input{template/main.tex}

\title{\BA{Title in Progress...}}
\author{Boris Andrews}
\affil{Mathematical Institute, University of Oxford}
\date{\today}


\begin{document}
    \pagenumbering{gobble}
    \maketitle
    
    
    \begin{abstract}
        Magnetic confinement reactors---in particular tokamaks---offer one of the most promising options for achieving practical nuclear fusion, with the potential to provide virtually limitless, clean energy. The theoretical and numerical modeling of tokamak plasmas is simultaneously an essential component of effective reactor design, and a great research barrier. Tokamak operational conditions exhibit comparatively low Knudsen numbers. Kinetic effects, including kinetic waves and instabilities, Landau damping, bump-on-tail instabilities and more, are therefore highly influential in tokamak plasma dynamics. Purely fluid models are inherently incapable of capturing these effects, whereas the high dimensionality in purely kinetic models render them practically intractable for most relevant purposes.

        We consider a $\delta\!f$ decomposition model, with a macroscopic fluid background and microscopic kinetic correction, both fully coupled to each other. A similar manner of discretization is proposed to that used in the recent \texttt{STRUPHY} code \cite{Holderied_Possanner_Wang_2021, Holderied_2022, Li_et_al_2023} with a finite-element model for the background and a pseudo-particle/PiC model for the correction.

        The fluid background satisfies the full, non-linear, resistive, compressible, Hall MHD equations. \cite{Laakmann_Hu_Farrell_2022} introduces finite-element(-in-space) implicit timesteppers for the incompressible analogue to this system with structure-preserving (SP) properties in the ideal case, alongside parameter-robust preconditioners. We show that these timesteppers can derive from a finite-element-in-time (FET) (and finite-element-in-space) interpretation. The benefits of this reformulation are discussed, including the derivation of timesteppers that are higher order in time, and the quantifiable dissipative SP properties in the non-ideal, resistive case.
        
        We discuss possible options for extending this FET approach to timesteppers for the compressible case.

        The kinetic corrections satisfy linearized Boltzmann equations. Using a Lénard--Bernstein collision operator, these take Fokker--Planck-like forms \cite{Fokker_1914, Planck_1917} wherein pseudo-particles in the numerical model obey the neoclassical transport equations, with particle-independent Brownian drift terms. This offers a rigorous methodology for incorporating collisions into the particle transport model, without coupling the equations of motions for each particle.
        
        Works by Chen, Chacón et al. \cite{Chen_Chacón_Barnes_2011, Chacón_Chen_Barnes_2013, Chen_Chacón_2014, Chen_Chacón_2015} have developed structure-preserving particle pushers for neoclassical transport in the Vlasov equations, derived from Crank--Nicolson integrators. We show these too can can derive from a FET interpretation, similarly offering potential extensions to higher-order-in-time particle pushers. The FET formulation is used also to consider how the stochastic drift terms can be incorporated into the pushers. Stochastic gyrokinetic expansions are also discussed.

        Different options for the numerical implementation of these schemes are considered.

        Due to the efficacy of FET in the development of SP timesteppers for both the fluid and kinetic component, we hope this approach will prove effective in the future for developing SP timesteppers for the full hybrid model. We hope this will give us the opportunity to incorporate previously inaccessible kinetic effects into the highly effective, modern, finite-element MHD models.
    \end{abstract}
    
    
    \newpage
    \tableofcontents
    
    
    \newpage
    \pagenumbering{arabic}
    %\linenumbers\renewcommand\thelinenumber{\color{black!50}\arabic{linenumber}}
            \input{0 - introduction/main.tex}
        \part{Research}
            \input{1 - low-noise PiC models/main.tex}
            \input{2 - kinetic component/main.tex}
            \input{3 - fluid component/main.tex}
            \input{4 - numerical implementation/main.tex}
        \part{Project Overview}
            \input{5 - research plan/main.tex}
            \input{6 - summary/main.tex}
    
    
    %\section{}
    \newpage
    \pagenumbering{gobble}
        \printbibliography


    \newpage
    \pagenumbering{roman}
    \appendix
        \part{Appendices}
            \input{8 - Hilbert complexes/main.tex}
            \input{9 - weak conservation proofs/main.tex}
\end{document}

    
    
    %\section{}
    \newpage
    \pagenumbering{gobble}
        \printbibliography


    \newpage
    \pagenumbering{roman}
    \appendix
        \part{Appendices}
            \documentclass[12pt, a4paper]{report}

\input{template/main.tex}

\title{\BA{Title in Progress...}}
\author{Boris Andrews}
\affil{Mathematical Institute, University of Oxford}
\date{\today}


\begin{document}
    \pagenumbering{gobble}
    \maketitle
    
    
    \begin{abstract}
        Magnetic confinement reactors---in particular tokamaks---offer one of the most promising options for achieving practical nuclear fusion, with the potential to provide virtually limitless, clean energy. The theoretical and numerical modeling of tokamak plasmas is simultaneously an essential component of effective reactor design, and a great research barrier. Tokamak operational conditions exhibit comparatively low Knudsen numbers. Kinetic effects, including kinetic waves and instabilities, Landau damping, bump-on-tail instabilities and more, are therefore highly influential in tokamak plasma dynamics. Purely fluid models are inherently incapable of capturing these effects, whereas the high dimensionality in purely kinetic models render them practically intractable for most relevant purposes.

        We consider a $\delta\!f$ decomposition model, with a macroscopic fluid background and microscopic kinetic correction, both fully coupled to each other. A similar manner of discretization is proposed to that used in the recent \texttt{STRUPHY} code \cite{Holderied_Possanner_Wang_2021, Holderied_2022, Li_et_al_2023} with a finite-element model for the background and a pseudo-particle/PiC model for the correction.

        The fluid background satisfies the full, non-linear, resistive, compressible, Hall MHD equations. \cite{Laakmann_Hu_Farrell_2022} introduces finite-element(-in-space) implicit timesteppers for the incompressible analogue to this system with structure-preserving (SP) properties in the ideal case, alongside parameter-robust preconditioners. We show that these timesteppers can derive from a finite-element-in-time (FET) (and finite-element-in-space) interpretation. The benefits of this reformulation are discussed, including the derivation of timesteppers that are higher order in time, and the quantifiable dissipative SP properties in the non-ideal, resistive case.
        
        We discuss possible options for extending this FET approach to timesteppers for the compressible case.

        The kinetic corrections satisfy linearized Boltzmann equations. Using a Lénard--Bernstein collision operator, these take Fokker--Planck-like forms \cite{Fokker_1914, Planck_1917} wherein pseudo-particles in the numerical model obey the neoclassical transport equations, with particle-independent Brownian drift terms. This offers a rigorous methodology for incorporating collisions into the particle transport model, without coupling the equations of motions for each particle.
        
        Works by Chen, Chacón et al. \cite{Chen_Chacón_Barnes_2011, Chacón_Chen_Barnes_2013, Chen_Chacón_2014, Chen_Chacón_2015} have developed structure-preserving particle pushers for neoclassical transport in the Vlasov equations, derived from Crank--Nicolson integrators. We show these too can can derive from a FET interpretation, similarly offering potential extensions to higher-order-in-time particle pushers. The FET formulation is used also to consider how the stochastic drift terms can be incorporated into the pushers. Stochastic gyrokinetic expansions are also discussed.

        Different options for the numerical implementation of these schemes are considered.

        Due to the efficacy of FET in the development of SP timesteppers for both the fluid and kinetic component, we hope this approach will prove effective in the future for developing SP timesteppers for the full hybrid model. We hope this will give us the opportunity to incorporate previously inaccessible kinetic effects into the highly effective, modern, finite-element MHD models.
    \end{abstract}
    
    
    \newpage
    \tableofcontents
    
    
    \newpage
    \pagenumbering{arabic}
    %\linenumbers\renewcommand\thelinenumber{\color{black!50}\arabic{linenumber}}
            \input{0 - introduction/main.tex}
        \part{Research}
            \input{1 - low-noise PiC models/main.tex}
            \input{2 - kinetic component/main.tex}
            \input{3 - fluid component/main.tex}
            \input{4 - numerical implementation/main.tex}
        \part{Project Overview}
            \input{5 - research plan/main.tex}
            \input{6 - summary/main.tex}
    
    
    %\section{}
    \newpage
    \pagenumbering{gobble}
        \printbibliography


    \newpage
    \pagenumbering{roman}
    \appendix
        \part{Appendices}
            \input{8 - Hilbert complexes/main.tex}
            \input{9 - weak conservation proofs/main.tex}
\end{document}

            \documentclass[12pt, a4paper]{report}

\input{template/main.tex}

\title{\BA{Title in Progress...}}
\author{Boris Andrews}
\affil{Mathematical Institute, University of Oxford}
\date{\today}


\begin{document}
    \pagenumbering{gobble}
    \maketitle
    
    
    \begin{abstract}
        Magnetic confinement reactors---in particular tokamaks---offer one of the most promising options for achieving practical nuclear fusion, with the potential to provide virtually limitless, clean energy. The theoretical and numerical modeling of tokamak plasmas is simultaneously an essential component of effective reactor design, and a great research barrier. Tokamak operational conditions exhibit comparatively low Knudsen numbers. Kinetic effects, including kinetic waves and instabilities, Landau damping, bump-on-tail instabilities and more, are therefore highly influential in tokamak plasma dynamics. Purely fluid models are inherently incapable of capturing these effects, whereas the high dimensionality in purely kinetic models render them practically intractable for most relevant purposes.

        We consider a $\delta\!f$ decomposition model, with a macroscopic fluid background and microscopic kinetic correction, both fully coupled to each other. A similar manner of discretization is proposed to that used in the recent \texttt{STRUPHY} code \cite{Holderied_Possanner_Wang_2021, Holderied_2022, Li_et_al_2023} with a finite-element model for the background and a pseudo-particle/PiC model for the correction.

        The fluid background satisfies the full, non-linear, resistive, compressible, Hall MHD equations. \cite{Laakmann_Hu_Farrell_2022} introduces finite-element(-in-space) implicit timesteppers for the incompressible analogue to this system with structure-preserving (SP) properties in the ideal case, alongside parameter-robust preconditioners. We show that these timesteppers can derive from a finite-element-in-time (FET) (and finite-element-in-space) interpretation. The benefits of this reformulation are discussed, including the derivation of timesteppers that are higher order in time, and the quantifiable dissipative SP properties in the non-ideal, resistive case.
        
        We discuss possible options for extending this FET approach to timesteppers for the compressible case.

        The kinetic corrections satisfy linearized Boltzmann equations. Using a Lénard--Bernstein collision operator, these take Fokker--Planck-like forms \cite{Fokker_1914, Planck_1917} wherein pseudo-particles in the numerical model obey the neoclassical transport equations, with particle-independent Brownian drift terms. This offers a rigorous methodology for incorporating collisions into the particle transport model, without coupling the equations of motions for each particle.
        
        Works by Chen, Chacón et al. \cite{Chen_Chacón_Barnes_2011, Chacón_Chen_Barnes_2013, Chen_Chacón_2014, Chen_Chacón_2015} have developed structure-preserving particle pushers for neoclassical transport in the Vlasov equations, derived from Crank--Nicolson integrators. We show these too can can derive from a FET interpretation, similarly offering potential extensions to higher-order-in-time particle pushers. The FET formulation is used also to consider how the stochastic drift terms can be incorporated into the pushers. Stochastic gyrokinetic expansions are also discussed.

        Different options for the numerical implementation of these schemes are considered.

        Due to the efficacy of FET in the development of SP timesteppers for both the fluid and kinetic component, we hope this approach will prove effective in the future for developing SP timesteppers for the full hybrid model. We hope this will give us the opportunity to incorporate previously inaccessible kinetic effects into the highly effective, modern, finite-element MHD models.
    \end{abstract}
    
    
    \newpage
    \tableofcontents
    
    
    \newpage
    \pagenumbering{arabic}
    %\linenumbers\renewcommand\thelinenumber{\color{black!50}\arabic{linenumber}}
            \input{0 - introduction/main.tex}
        \part{Research}
            \input{1 - low-noise PiC models/main.tex}
            \input{2 - kinetic component/main.tex}
            \input{3 - fluid component/main.tex}
            \input{4 - numerical implementation/main.tex}
        \part{Project Overview}
            \input{5 - research plan/main.tex}
            \input{6 - summary/main.tex}
    
    
    %\section{}
    \newpage
    \pagenumbering{gobble}
        \printbibliography


    \newpage
    \pagenumbering{roman}
    \appendix
        \part{Appendices}
            \input{8 - Hilbert complexes/main.tex}
            \input{9 - weak conservation proofs/main.tex}
\end{document}

\end{document}

\end{document}

        \part{Project Overview}
            \documentclass[12pt, a4paper]{report}

\documentclass[12pt, a4paper]{report}

\documentclass[12pt, a4paper]{report}

\input{template/main.tex}

\title{\BA{Title in Progress...}}
\author{Boris Andrews}
\affil{Mathematical Institute, University of Oxford}
\date{\today}


\begin{document}
    \pagenumbering{gobble}
    \maketitle
    
    
    \begin{abstract}
        Magnetic confinement reactors---in particular tokamaks---offer one of the most promising options for achieving practical nuclear fusion, with the potential to provide virtually limitless, clean energy. The theoretical and numerical modeling of tokamak plasmas is simultaneously an essential component of effective reactor design, and a great research barrier. Tokamak operational conditions exhibit comparatively low Knudsen numbers. Kinetic effects, including kinetic waves and instabilities, Landau damping, bump-on-tail instabilities and more, are therefore highly influential in tokamak plasma dynamics. Purely fluid models are inherently incapable of capturing these effects, whereas the high dimensionality in purely kinetic models render them practically intractable for most relevant purposes.

        We consider a $\delta\!f$ decomposition model, with a macroscopic fluid background and microscopic kinetic correction, both fully coupled to each other. A similar manner of discretization is proposed to that used in the recent \texttt{STRUPHY} code \cite{Holderied_Possanner_Wang_2021, Holderied_2022, Li_et_al_2023} with a finite-element model for the background and a pseudo-particle/PiC model for the correction.

        The fluid background satisfies the full, non-linear, resistive, compressible, Hall MHD equations. \cite{Laakmann_Hu_Farrell_2022} introduces finite-element(-in-space) implicit timesteppers for the incompressible analogue to this system with structure-preserving (SP) properties in the ideal case, alongside parameter-robust preconditioners. We show that these timesteppers can derive from a finite-element-in-time (FET) (and finite-element-in-space) interpretation. The benefits of this reformulation are discussed, including the derivation of timesteppers that are higher order in time, and the quantifiable dissipative SP properties in the non-ideal, resistive case.
        
        We discuss possible options for extending this FET approach to timesteppers for the compressible case.

        The kinetic corrections satisfy linearized Boltzmann equations. Using a Lénard--Bernstein collision operator, these take Fokker--Planck-like forms \cite{Fokker_1914, Planck_1917} wherein pseudo-particles in the numerical model obey the neoclassical transport equations, with particle-independent Brownian drift terms. This offers a rigorous methodology for incorporating collisions into the particle transport model, without coupling the equations of motions for each particle.
        
        Works by Chen, Chacón et al. \cite{Chen_Chacón_Barnes_2011, Chacón_Chen_Barnes_2013, Chen_Chacón_2014, Chen_Chacón_2015} have developed structure-preserving particle pushers for neoclassical transport in the Vlasov equations, derived from Crank--Nicolson integrators. We show these too can can derive from a FET interpretation, similarly offering potential extensions to higher-order-in-time particle pushers. The FET formulation is used also to consider how the stochastic drift terms can be incorporated into the pushers. Stochastic gyrokinetic expansions are also discussed.

        Different options for the numerical implementation of these schemes are considered.

        Due to the efficacy of FET in the development of SP timesteppers for both the fluid and kinetic component, we hope this approach will prove effective in the future for developing SP timesteppers for the full hybrid model. We hope this will give us the opportunity to incorporate previously inaccessible kinetic effects into the highly effective, modern, finite-element MHD models.
    \end{abstract}
    
    
    \newpage
    \tableofcontents
    
    
    \newpage
    \pagenumbering{arabic}
    %\linenumbers\renewcommand\thelinenumber{\color{black!50}\arabic{linenumber}}
            \input{0 - introduction/main.tex}
        \part{Research}
            \input{1 - low-noise PiC models/main.tex}
            \input{2 - kinetic component/main.tex}
            \input{3 - fluid component/main.tex}
            \input{4 - numerical implementation/main.tex}
        \part{Project Overview}
            \input{5 - research plan/main.tex}
            \input{6 - summary/main.tex}
    
    
    %\section{}
    \newpage
    \pagenumbering{gobble}
        \printbibliography


    \newpage
    \pagenumbering{roman}
    \appendix
        \part{Appendices}
            \input{8 - Hilbert complexes/main.tex}
            \input{9 - weak conservation proofs/main.tex}
\end{document}


\title{\BA{Title in Progress...}}
\author{Boris Andrews}
\affil{Mathematical Institute, University of Oxford}
\date{\today}


\begin{document}
    \pagenumbering{gobble}
    \maketitle
    
    
    \begin{abstract}
        Magnetic confinement reactors---in particular tokamaks---offer one of the most promising options for achieving practical nuclear fusion, with the potential to provide virtually limitless, clean energy. The theoretical and numerical modeling of tokamak plasmas is simultaneously an essential component of effective reactor design, and a great research barrier. Tokamak operational conditions exhibit comparatively low Knudsen numbers. Kinetic effects, including kinetic waves and instabilities, Landau damping, bump-on-tail instabilities and more, are therefore highly influential in tokamak plasma dynamics. Purely fluid models are inherently incapable of capturing these effects, whereas the high dimensionality in purely kinetic models render them practically intractable for most relevant purposes.

        We consider a $\delta\!f$ decomposition model, with a macroscopic fluid background and microscopic kinetic correction, both fully coupled to each other. A similar manner of discretization is proposed to that used in the recent \texttt{STRUPHY} code \cite{Holderied_Possanner_Wang_2021, Holderied_2022, Li_et_al_2023} with a finite-element model for the background and a pseudo-particle/PiC model for the correction.

        The fluid background satisfies the full, non-linear, resistive, compressible, Hall MHD equations. \cite{Laakmann_Hu_Farrell_2022} introduces finite-element(-in-space) implicit timesteppers for the incompressible analogue to this system with structure-preserving (SP) properties in the ideal case, alongside parameter-robust preconditioners. We show that these timesteppers can derive from a finite-element-in-time (FET) (and finite-element-in-space) interpretation. The benefits of this reformulation are discussed, including the derivation of timesteppers that are higher order in time, and the quantifiable dissipative SP properties in the non-ideal, resistive case.
        
        We discuss possible options for extending this FET approach to timesteppers for the compressible case.

        The kinetic corrections satisfy linearized Boltzmann equations. Using a Lénard--Bernstein collision operator, these take Fokker--Planck-like forms \cite{Fokker_1914, Planck_1917} wherein pseudo-particles in the numerical model obey the neoclassical transport equations, with particle-independent Brownian drift terms. This offers a rigorous methodology for incorporating collisions into the particle transport model, without coupling the equations of motions for each particle.
        
        Works by Chen, Chacón et al. \cite{Chen_Chacón_Barnes_2011, Chacón_Chen_Barnes_2013, Chen_Chacón_2014, Chen_Chacón_2015} have developed structure-preserving particle pushers for neoclassical transport in the Vlasov equations, derived from Crank--Nicolson integrators. We show these too can can derive from a FET interpretation, similarly offering potential extensions to higher-order-in-time particle pushers. The FET formulation is used also to consider how the stochastic drift terms can be incorporated into the pushers. Stochastic gyrokinetic expansions are also discussed.

        Different options for the numerical implementation of these schemes are considered.

        Due to the efficacy of FET in the development of SP timesteppers for both the fluid and kinetic component, we hope this approach will prove effective in the future for developing SP timesteppers for the full hybrid model. We hope this will give us the opportunity to incorporate previously inaccessible kinetic effects into the highly effective, modern, finite-element MHD models.
    \end{abstract}
    
    
    \newpage
    \tableofcontents
    
    
    \newpage
    \pagenumbering{arabic}
    %\linenumbers\renewcommand\thelinenumber{\color{black!50}\arabic{linenumber}}
            \documentclass[12pt, a4paper]{report}

\input{template/main.tex}

\title{\BA{Title in Progress...}}
\author{Boris Andrews}
\affil{Mathematical Institute, University of Oxford}
\date{\today}


\begin{document}
    \pagenumbering{gobble}
    \maketitle
    
    
    \begin{abstract}
        Magnetic confinement reactors---in particular tokamaks---offer one of the most promising options for achieving practical nuclear fusion, with the potential to provide virtually limitless, clean energy. The theoretical and numerical modeling of tokamak plasmas is simultaneously an essential component of effective reactor design, and a great research barrier. Tokamak operational conditions exhibit comparatively low Knudsen numbers. Kinetic effects, including kinetic waves and instabilities, Landau damping, bump-on-tail instabilities and more, are therefore highly influential in tokamak plasma dynamics. Purely fluid models are inherently incapable of capturing these effects, whereas the high dimensionality in purely kinetic models render them practically intractable for most relevant purposes.

        We consider a $\delta\!f$ decomposition model, with a macroscopic fluid background and microscopic kinetic correction, both fully coupled to each other. A similar manner of discretization is proposed to that used in the recent \texttt{STRUPHY} code \cite{Holderied_Possanner_Wang_2021, Holderied_2022, Li_et_al_2023} with a finite-element model for the background and a pseudo-particle/PiC model for the correction.

        The fluid background satisfies the full, non-linear, resistive, compressible, Hall MHD equations. \cite{Laakmann_Hu_Farrell_2022} introduces finite-element(-in-space) implicit timesteppers for the incompressible analogue to this system with structure-preserving (SP) properties in the ideal case, alongside parameter-robust preconditioners. We show that these timesteppers can derive from a finite-element-in-time (FET) (and finite-element-in-space) interpretation. The benefits of this reformulation are discussed, including the derivation of timesteppers that are higher order in time, and the quantifiable dissipative SP properties in the non-ideal, resistive case.
        
        We discuss possible options for extending this FET approach to timesteppers for the compressible case.

        The kinetic corrections satisfy linearized Boltzmann equations. Using a Lénard--Bernstein collision operator, these take Fokker--Planck-like forms \cite{Fokker_1914, Planck_1917} wherein pseudo-particles in the numerical model obey the neoclassical transport equations, with particle-independent Brownian drift terms. This offers a rigorous methodology for incorporating collisions into the particle transport model, without coupling the equations of motions for each particle.
        
        Works by Chen, Chacón et al. \cite{Chen_Chacón_Barnes_2011, Chacón_Chen_Barnes_2013, Chen_Chacón_2014, Chen_Chacón_2015} have developed structure-preserving particle pushers for neoclassical transport in the Vlasov equations, derived from Crank--Nicolson integrators. We show these too can can derive from a FET interpretation, similarly offering potential extensions to higher-order-in-time particle pushers. The FET formulation is used also to consider how the stochastic drift terms can be incorporated into the pushers. Stochastic gyrokinetic expansions are also discussed.

        Different options for the numerical implementation of these schemes are considered.

        Due to the efficacy of FET in the development of SP timesteppers for both the fluid and kinetic component, we hope this approach will prove effective in the future for developing SP timesteppers for the full hybrid model. We hope this will give us the opportunity to incorporate previously inaccessible kinetic effects into the highly effective, modern, finite-element MHD models.
    \end{abstract}
    
    
    \newpage
    \tableofcontents
    
    
    \newpage
    \pagenumbering{arabic}
    %\linenumbers\renewcommand\thelinenumber{\color{black!50}\arabic{linenumber}}
            \input{0 - introduction/main.tex}
        \part{Research}
            \input{1 - low-noise PiC models/main.tex}
            \input{2 - kinetic component/main.tex}
            \input{3 - fluid component/main.tex}
            \input{4 - numerical implementation/main.tex}
        \part{Project Overview}
            \input{5 - research plan/main.tex}
            \input{6 - summary/main.tex}
    
    
    %\section{}
    \newpage
    \pagenumbering{gobble}
        \printbibliography


    \newpage
    \pagenumbering{roman}
    \appendix
        \part{Appendices}
            \input{8 - Hilbert complexes/main.tex}
            \input{9 - weak conservation proofs/main.tex}
\end{document}

        \part{Research}
            \documentclass[12pt, a4paper]{report}

\input{template/main.tex}

\title{\BA{Title in Progress...}}
\author{Boris Andrews}
\affil{Mathematical Institute, University of Oxford}
\date{\today}


\begin{document}
    \pagenumbering{gobble}
    \maketitle
    
    
    \begin{abstract}
        Magnetic confinement reactors---in particular tokamaks---offer one of the most promising options for achieving practical nuclear fusion, with the potential to provide virtually limitless, clean energy. The theoretical and numerical modeling of tokamak plasmas is simultaneously an essential component of effective reactor design, and a great research barrier. Tokamak operational conditions exhibit comparatively low Knudsen numbers. Kinetic effects, including kinetic waves and instabilities, Landau damping, bump-on-tail instabilities and more, are therefore highly influential in tokamak plasma dynamics. Purely fluid models are inherently incapable of capturing these effects, whereas the high dimensionality in purely kinetic models render them practically intractable for most relevant purposes.

        We consider a $\delta\!f$ decomposition model, with a macroscopic fluid background and microscopic kinetic correction, both fully coupled to each other. A similar manner of discretization is proposed to that used in the recent \texttt{STRUPHY} code \cite{Holderied_Possanner_Wang_2021, Holderied_2022, Li_et_al_2023} with a finite-element model for the background and a pseudo-particle/PiC model for the correction.

        The fluid background satisfies the full, non-linear, resistive, compressible, Hall MHD equations. \cite{Laakmann_Hu_Farrell_2022} introduces finite-element(-in-space) implicit timesteppers for the incompressible analogue to this system with structure-preserving (SP) properties in the ideal case, alongside parameter-robust preconditioners. We show that these timesteppers can derive from a finite-element-in-time (FET) (and finite-element-in-space) interpretation. The benefits of this reformulation are discussed, including the derivation of timesteppers that are higher order in time, and the quantifiable dissipative SP properties in the non-ideal, resistive case.
        
        We discuss possible options for extending this FET approach to timesteppers for the compressible case.

        The kinetic corrections satisfy linearized Boltzmann equations. Using a Lénard--Bernstein collision operator, these take Fokker--Planck-like forms \cite{Fokker_1914, Planck_1917} wherein pseudo-particles in the numerical model obey the neoclassical transport equations, with particle-independent Brownian drift terms. This offers a rigorous methodology for incorporating collisions into the particle transport model, without coupling the equations of motions for each particle.
        
        Works by Chen, Chacón et al. \cite{Chen_Chacón_Barnes_2011, Chacón_Chen_Barnes_2013, Chen_Chacón_2014, Chen_Chacón_2015} have developed structure-preserving particle pushers for neoclassical transport in the Vlasov equations, derived from Crank--Nicolson integrators. We show these too can can derive from a FET interpretation, similarly offering potential extensions to higher-order-in-time particle pushers. The FET formulation is used also to consider how the stochastic drift terms can be incorporated into the pushers. Stochastic gyrokinetic expansions are also discussed.

        Different options for the numerical implementation of these schemes are considered.

        Due to the efficacy of FET in the development of SP timesteppers for both the fluid and kinetic component, we hope this approach will prove effective in the future for developing SP timesteppers for the full hybrid model. We hope this will give us the opportunity to incorporate previously inaccessible kinetic effects into the highly effective, modern, finite-element MHD models.
    \end{abstract}
    
    
    \newpage
    \tableofcontents
    
    
    \newpage
    \pagenumbering{arabic}
    %\linenumbers\renewcommand\thelinenumber{\color{black!50}\arabic{linenumber}}
            \input{0 - introduction/main.tex}
        \part{Research}
            \input{1 - low-noise PiC models/main.tex}
            \input{2 - kinetic component/main.tex}
            \input{3 - fluid component/main.tex}
            \input{4 - numerical implementation/main.tex}
        \part{Project Overview}
            \input{5 - research plan/main.tex}
            \input{6 - summary/main.tex}
    
    
    %\section{}
    \newpage
    \pagenumbering{gobble}
        \printbibliography


    \newpage
    \pagenumbering{roman}
    \appendix
        \part{Appendices}
            \input{8 - Hilbert complexes/main.tex}
            \input{9 - weak conservation proofs/main.tex}
\end{document}

            \documentclass[12pt, a4paper]{report}

\input{template/main.tex}

\title{\BA{Title in Progress...}}
\author{Boris Andrews}
\affil{Mathematical Institute, University of Oxford}
\date{\today}


\begin{document}
    \pagenumbering{gobble}
    \maketitle
    
    
    \begin{abstract}
        Magnetic confinement reactors---in particular tokamaks---offer one of the most promising options for achieving practical nuclear fusion, with the potential to provide virtually limitless, clean energy. The theoretical and numerical modeling of tokamak plasmas is simultaneously an essential component of effective reactor design, and a great research barrier. Tokamak operational conditions exhibit comparatively low Knudsen numbers. Kinetic effects, including kinetic waves and instabilities, Landau damping, bump-on-tail instabilities and more, are therefore highly influential in tokamak plasma dynamics. Purely fluid models are inherently incapable of capturing these effects, whereas the high dimensionality in purely kinetic models render them practically intractable for most relevant purposes.

        We consider a $\delta\!f$ decomposition model, with a macroscopic fluid background and microscopic kinetic correction, both fully coupled to each other. A similar manner of discretization is proposed to that used in the recent \texttt{STRUPHY} code \cite{Holderied_Possanner_Wang_2021, Holderied_2022, Li_et_al_2023} with a finite-element model for the background and a pseudo-particle/PiC model for the correction.

        The fluid background satisfies the full, non-linear, resistive, compressible, Hall MHD equations. \cite{Laakmann_Hu_Farrell_2022} introduces finite-element(-in-space) implicit timesteppers for the incompressible analogue to this system with structure-preserving (SP) properties in the ideal case, alongside parameter-robust preconditioners. We show that these timesteppers can derive from a finite-element-in-time (FET) (and finite-element-in-space) interpretation. The benefits of this reformulation are discussed, including the derivation of timesteppers that are higher order in time, and the quantifiable dissipative SP properties in the non-ideal, resistive case.
        
        We discuss possible options for extending this FET approach to timesteppers for the compressible case.

        The kinetic corrections satisfy linearized Boltzmann equations. Using a Lénard--Bernstein collision operator, these take Fokker--Planck-like forms \cite{Fokker_1914, Planck_1917} wherein pseudo-particles in the numerical model obey the neoclassical transport equations, with particle-independent Brownian drift terms. This offers a rigorous methodology for incorporating collisions into the particle transport model, without coupling the equations of motions for each particle.
        
        Works by Chen, Chacón et al. \cite{Chen_Chacón_Barnes_2011, Chacón_Chen_Barnes_2013, Chen_Chacón_2014, Chen_Chacón_2015} have developed structure-preserving particle pushers for neoclassical transport in the Vlasov equations, derived from Crank--Nicolson integrators. We show these too can can derive from a FET interpretation, similarly offering potential extensions to higher-order-in-time particle pushers. The FET formulation is used also to consider how the stochastic drift terms can be incorporated into the pushers. Stochastic gyrokinetic expansions are also discussed.

        Different options for the numerical implementation of these schemes are considered.

        Due to the efficacy of FET in the development of SP timesteppers for both the fluid and kinetic component, we hope this approach will prove effective in the future for developing SP timesteppers for the full hybrid model. We hope this will give us the opportunity to incorporate previously inaccessible kinetic effects into the highly effective, modern, finite-element MHD models.
    \end{abstract}
    
    
    \newpage
    \tableofcontents
    
    
    \newpage
    \pagenumbering{arabic}
    %\linenumbers\renewcommand\thelinenumber{\color{black!50}\arabic{linenumber}}
            \input{0 - introduction/main.tex}
        \part{Research}
            \input{1 - low-noise PiC models/main.tex}
            \input{2 - kinetic component/main.tex}
            \input{3 - fluid component/main.tex}
            \input{4 - numerical implementation/main.tex}
        \part{Project Overview}
            \input{5 - research plan/main.tex}
            \input{6 - summary/main.tex}
    
    
    %\section{}
    \newpage
    \pagenumbering{gobble}
        \printbibliography


    \newpage
    \pagenumbering{roman}
    \appendix
        \part{Appendices}
            \input{8 - Hilbert complexes/main.tex}
            \input{9 - weak conservation proofs/main.tex}
\end{document}

            \documentclass[12pt, a4paper]{report}

\input{template/main.tex}

\title{\BA{Title in Progress...}}
\author{Boris Andrews}
\affil{Mathematical Institute, University of Oxford}
\date{\today}


\begin{document}
    \pagenumbering{gobble}
    \maketitle
    
    
    \begin{abstract}
        Magnetic confinement reactors---in particular tokamaks---offer one of the most promising options for achieving practical nuclear fusion, with the potential to provide virtually limitless, clean energy. The theoretical and numerical modeling of tokamak plasmas is simultaneously an essential component of effective reactor design, and a great research barrier. Tokamak operational conditions exhibit comparatively low Knudsen numbers. Kinetic effects, including kinetic waves and instabilities, Landau damping, bump-on-tail instabilities and more, are therefore highly influential in tokamak plasma dynamics. Purely fluid models are inherently incapable of capturing these effects, whereas the high dimensionality in purely kinetic models render them practically intractable for most relevant purposes.

        We consider a $\delta\!f$ decomposition model, with a macroscopic fluid background and microscopic kinetic correction, both fully coupled to each other. A similar manner of discretization is proposed to that used in the recent \texttt{STRUPHY} code \cite{Holderied_Possanner_Wang_2021, Holderied_2022, Li_et_al_2023} with a finite-element model for the background and a pseudo-particle/PiC model for the correction.

        The fluid background satisfies the full, non-linear, resistive, compressible, Hall MHD equations. \cite{Laakmann_Hu_Farrell_2022} introduces finite-element(-in-space) implicit timesteppers for the incompressible analogue to this system with structure-preserving (SP) properties in the ideal case, alongside parameter-robust preconditioners. We show that these timesteppers can derive from a finite-element-in-time (FET) (and finite-element-in-space) interpretation. The benefits of this reformulation are discussed, including the derivation of timesteppers that are higher order in time, and the quantifiable dissipative SP properties in the non-ideal, resistive case.
        
        We discuss possible options for extending this FET approach to timesteppers for the compressible case.

        The kinetic corrections satisfy linearized Boltzmann equations. Using a Lénard--Bernstein collision operator, these take Fokker--Planck-like forms \cite{Fokker_1914, Planck_1917} wherein pseudo-particles in the numerical model obey the neoclassical transport equations, with particle-independent Brownian drift terms. This offers a rigorous methodology for incorporating collisions into the particle transport model, without coupling the equations of motions for each particle.
        
        Works by Chen, Chacón et al. \cite{Chen_Chacón_Barnes_2011, Chacón_Chen_Barnes_2013, Chen_Chacón_2014, Chen_Chacón_2015} have developed structure-preserving particle pushers for neoclassical transport in the Vlasov equations, derived from Crank--Nicolson integrators. We show these too can can derive from a FET interpretation, similarly offering potential extensions to higher-order-in-time particle pushers. The FET formulation is used also to consider how the stochastic drift terms can be incorporated into the pushers. Stochastic gyrokinetic expansions are also discussed.

        Different options for the numerical implementation of these schemes are considered.

        Due to the efficacy of FET in the development of SP timesteppers for both the fluid and kinetic component, we hope this approach will prove effective in the future for developing SP timesteppers for the full hybrid model. We hope this will give us the opportunity to incorporate previously inaccessible kinetic effects into the highly effective, modern, finite-element MHD models.
    \end{abstract}
    
    
    \newpage
    \tableofcontents
    
    
    \newpage
    \pagenumbering{arabic}
    %\linenumbers\renewcommand\thelinenumber{\color{black!50}\arabic{linenumber}}
            \input{0 - introduction/main.tex}
        \part{Research}
            \input{1 - low-noise PiC models/main.tex}
            \input{2 - kinetic component/main.tex}
            \input{3 - fluid component/main.tex}
            \input{4 - numerical implementation/main.tex}
        \part{Project Overview}
            \input{5 - research plan/main.tex}
            \input{6 - summary/main.tex}
    
    
    %\section{}
    \newpage
    \pagenumbering{gobble}
        \printbibliography


    \newpage
    \pagenumbering{roman}
    \appendix
        \part{Appendices}
            \input{8 - Hilbert complexes/main.tex}
            \input{9 - weak conservation proofs/main.tex}
\end{document}

            \documentclass[12pt, a4paper]{report}

\input{template/main.tex}

\title{\BA{Title in Progress...}}
\author{Boris Andrews}
\affil{Mathematical Institute, University of Oxford}
\date{\today}


\begin{document}
    \pagenumbering{gobble}
    \maketitle
    
    
    \begin{abstract}
        Magnetic confinement reactors---in particular tokamaks---offer one of the most promising options for achieving practical nuclear fusion, with the potential to provide virtually limitless, clean energy. The theoretical and numerical modeling of tokamak plasmas is simultaneously an essential component of effective reactor design, and a great research barrier. Tokamak operational conditions exhibit comparatively low Knudsen numbers. Kinetic effects, including kinetic waves and instabilities, Landau damping, bump-on-tail instabilities and more, are therefore highly influential in tokamak plasma dynamics. Purely fluid models are inherently incapable of capturing these effects, whereas the high dimensionality in purely kinetic models render them practically intractable for most relevant purposes.

        We consider a $\delta\!f$ decomposition model, with a macroscopic fluid background and microscopic kinetic correction, both fully coupled to each other. A similar manner of discretization is proposed to that used in the recent \texttt{STRUPHY} code \cite{Holderied_Possanner_Wang_2021, Holderied_2022, Li_et_al_2023} with a finite-element model for the background and a pseudo-particle/PiC model for the correction.

        The fluid background satisfies the full, non-linear, resistive, compressible, Hall MHD equations. \cite{Laakmann_Hu_Farrell_2022} introduces finite-element(-in-space) implicit timesteppers for the incompressible analogue to this system with structure-preserving (SP) properties in the ideal case, alongside parameter-robust preconditioners. We show that these timesteppers can derive from a finite-element-in-time (FET) (and finite-element-in-space) interpretation. The benefits of this reformulation are discussed, including the derivation of timesteppers that are higher order in time, and the quantifiable dissipative SP properties in the non-ideal, resistive case.
        
        We discuss possible options for extending this FET approach to timesteppers for the compressible case.

        The kinetic corrections satisfy linearized Boltzmann equations. Using a Lénard--Bernstein collision operator, these take Fokker--Planck-like forms \cite{Fokker_1914, Planck_1917} wherein pseudo-particles in the numerical model obey the neoclassical transport equations, with particle-independent Brownian drift terms. This offers a rigorous methodology for incorporating collisions into the particle transport model, without coupling the equations of motions for each particle.
        
        Works by Chen, Chacón et al. \cite{Chen_Chacón_Barnes_2011, Chacón_Chen_Barnes_2013, Chen_Chacón_2014, Chen_Chacón_2015} have developed structure-preserving particle pushers for neoclassical transport in the Vlasov equations, derived from Crank--Nicolson integrators. We show these too can can derive from a FET interpretation, similarly offering potential extensions to higher-order-in-time particle pushers. The FET formulation is used also to consider how the stochastic drift terms can be incorporated into the pushers. Stochastic gyrokinetic expansions are also discussed.

        Different options for the numerical implementation of these schemes are considered.

        Due to the efficacy of FET in the development of SP timesteppers for both the fluid and kinetic component, we hope this approach will prove effective in the future for developing SP timesteppers for the full hybrid model. We hope this will give us the opportunity to incorporate previously inaccessible kinetic effects into the highly effective, modern, finite-element MHD models.
    \end{abstract}
    
    
    \newpage
    \tableofcontents
    
    
    \newpage
    \pagenumbering{arabic}
    %\linenumbers\renewcommand\thelinenumber{\color{black!50}\arabic{linenumber}}
            \input{0 - introduction/main.tex}
        \part{Research}
            \input{1 - low-noise PiC models/main.tex}
            \input{2 - kinetic component/main.tex}
            \input{3 - fluid component/main.tex}
            \input{4 - numerical implementation/main.tex}
        \part{Project Overview}
            \input{5 - research plan/main.tex}
            \input{6 - summary/main.tex}
    
    
    %\section{}
    \newpage
    \pagenumbering{gobble}
        \printbibliography


    \newpage
    \pagenumbering{roman}
    \appendix
        \part{Appendices}
            \input{8 - Hilbert complexes/main.tex}
            \input{9 - weak conservation proofs/main.tex}
\end{document}

        \part{Project Overview}
            \documentclass[12pt, a4paper]{report}

\input{template/main.tex}

\title{\BA{Title in Progress...}}
\author{Boris Andrews}
\affil{Mathematical Institute, University of Oxford}
\date{\today}


\begin{document}
    \pagenumbering{gobble}
    \maketitle
    
    
    \begin{abstract}
        Magnetic confinement reactors---in particular tokamaks---offer one of the most promising options for achieving practical nuclear fusion, with the potential to provide virtually limitless, clean energy. The theoretical and numerical modeling of tokamak plasmas is simultaneously an essential component of effective reactor design, and a great research barrier. Tokamak operational conditions exhibit comparatively low Knudsen numbers. Kinetic effects, including kinetic waves and instabilities, Landau damping, bump-on-tail instabilities and more, are therefore highly influential in tokamak plasma dynamics. Purely fluid models are inherently incapable of capturing these effects, whereas the high dimensionality in purely kinetic models render them practically intractable for most relevant purposes.

        We consider a $\delta\!f$ decomposition model, with a macroscopic fluid background and microscopic kinetic correction, both fully coupled to each other. A similar manner of discretization is proposed to that used in the recent \texttt{STRUPHY} code \cite{Holderied_Possanner_Wang_2021, Holderied_2022, Li_et_al_2023} with a finite-element model for the background and a pseudo-particle/PiC model for the correction.

        The fluid background satisfies the full, non-linear, resistive, compressible, Hall MHD equations. \cite{Laakmann_Hu_Farrell_2022} introduces finite-element(-in-space) implicit timesteppers for the incompressible analogue to this system with structure-preserving (SP) properties in the ideal case, alongside parameter-robust preconditioners. We show that these timesteppers can derive from a finite-element-in-time (FET) (and finite-element-in-space) interpretation. The benefits of this reformulation are discussed, including the derivation of timesteppers that are higher order in time, and the quantifiable dissipative SP properties in the non-ideal, resistive case.
        
        We discuss possible options for extending this FET approach to timesteppers for the compressible case.

        The kinetic corrections satisfy linearized Boltzmann equations. Using a Lénard--Bernstein collision operator, these take Fokker--Planck-like forms \cite{Fokker_1914, Planck_1917} wherein pseudo-particles in the numerical model obey the neoclassical transport equations, with particle-independent Brownian drift terms. This offers a rigorous methodology for incorporating collisions into the particle transport model, without coupling the equations of motions for each particle.
        
        Works by Chen, Chacón et al. \cite{Chen_Chacón_Barnes_2011, Chacón_Chen_Barnes_2013, Chen_Chacón_2014, Chen_Chacón_2015} have developed structure-preserving particle pushers for neoclassical transport in the Vlasov equations, derived from Crank--Nicolson integrators. We show these too can can derive from a FET interpretation, similarly offering potential extensions to higher-order-in-time particle pushers. The FET formulation is used also to consider how the stochastic drift terms can be incorporated into the pushers. Stochastic gyrokinetic expansions are also discussed.

        Different options for the numerical implementation of these schemes are considered.

        Due to the efficacy of FET in the development of SP timesteppers for both the fluid and kinetic component, we hope this approach will prove effective in the future for developing SP timesteppers for the full hybrid model. We hope this will give us the opportunity to incorporate previously inaccessible kinetic effects into the highly effective, modern, finite-element MHD models.
    \end{abstract}
    
    
    \newpage
    \tableofcontents
    
    
    \newpage
    \pagenumbering{arabic}
    %\linenumbers\renewcommand\thelinenumber{\color{black!50}\arabic{linenumber}}
            \input{0 - introduction/main.tex}
        \part{Research}
            \input{1 - low-noise PiC models/main.tex}
            \input{2 - kinetic component/main.tex}
            \input{3 - fluid component/main.tex}
            \input{4 - numerical implementation/main.tex}
        \part{Project Overview}
            \input{5 - research plan/main.tex}
            \input{6 - summary/main.tex}
    
    
    %\section{}
    \newpage
    \pagenumbering{gobble}
        \printbibliography


    \newpage
    \pagenumbering{roman}
    \appendix
        \part{Appendices}
            \input{8 - Hilbert complexes/main.tex}
            \input{9 - weak conservation proofs/main.tex}
\end{document}

            \documentclass[12pt, a4paper]{report}

\input{template/main.tex}

\title{\BA{Title in Progress...}}
\author{Boris Andrews}
\affil{Mathematical Institute, University of Oxford}
\date{\today}


\begin{document}
    \pagenumbering{gobble}
    \maketitle
    
    
    \begin{abstract}
        Magnetic confinement reactors---in particular tokamaks---offer one of the most promising options for achieving practical nuclear fusion, with the potential to provide virtually limitless, clean energy. The theoretical and numerical modeling of tokamak plasmas is simultaneously an essential component of effective reactor design, and a great research barrier. Tokamak operational conditions exhibit comparatively low Knudsen numbers. Kinetic effects, including kinetic waves and instabilities, Landau damping, bump-on-tail instabilities and more, are therefore highly influential in tokamak plasma dynamics. Purely fluid models are inherently incapable of capturing these effects, whereas the high dimensionality in purely kinetic models render them practically intractable for most relevant purposes.

        We consider a $\delta\!f$ decomposition model, with a macroscopic fluid background and microscopic kinetic correction, both fully coupled to each other. A similar manner of discretization is proposed to that used in the recent \texttt{STRUPHY} code \cite{Holderied_Possanner_Wang_2021, Holderied_2022, Li_et_al_2023} with a finite-element model for the background and a pseudo-particle/PiC model for the correction.

        The fluid background satisfies the full, non-linear, resistive, compressible, Hall MHD equations. \cite{Laakmann_Hu_Farrell_2022} introduces finite-element(-in-space) implicit timesteppers for the incompressible analogue to this system with structure-preserving (SP) properties in the ideal case, alongside parameter-robust preconditioners. We show that these timesteppers can derive from a finite-element-in-time (FET) (and finite-element-in-space) interpretation. The benefits of this reformulation are discussed, including the derivation of timesteppers that are higher order in time, and the quantifiable dissipative SP properties in the non-ideal, resistive case.
        
        We discuss possible options for extending this FET approach to timesteppers for the compressible case.

        The kinetic corrections satisfy linearized Boltzmann equations. Using a Lénard--Bernstein collision operator, these take Fokker--Planck-like forms \cite{Fokker_1914, Planck_1917} wherein pseudo-particles in the numerical model obey the neoclassical transport equations, with particle-independent Brownian drift terms. This offers a rigorous methodology for incorporating collisions into the particle transport model, without coupling the equations of motions for each particle.
        
        Works by Chen, Chacón et al. \cite{Chen_Chacón_Barnes_2011, Chacón_Chen_Barnes_2013, Chen_Chacón_2014, Chen_Chacón_2015} have developed structure-preserving particle pushers for neoclassical transport in the Vlasov equations, derived from Crank--Nicolson integrators. We show these too can can derive from a FET interpretation, similarly offering potential extensions to higher-order-in-time particle pushers. The FET formulation is used also to consider how the stochastic drift terms can be incorporated into the pushers. Stochastic gyrokinetic expansions are also discussed.

        Different options for the numerical implementation of these schemes are considered.

        Due to the efficacy of FET in the development of SP timesteppers for both the fluid and kinetic component, we hope this approach will prove effective in the future for developing SP timesteppers for the full hybrid model. We hope this will give us the opportunity to incorporate previously inaccessible kinetic effects into the highly effective, modern, finite-element MHD models.
    \end{abstract}
    
    
    \newpage
    \tableofcontents
    
    
    \newpage
    \pagenumbering{arabic}
    %\linenumbers\renewcommand\thelinenumber{\color{black!50}\arabic{linenumber}}
            \input{0 - introduction/main.tex}
        \part{Research}
            \input{1 - low-noise PiC models/main.tex}
            \input{2 - kinetic component/main.tex}
            \input{3 - fluid component/main.tex}
            \input{4 - numerical implementation/main.tex}
        \part{Project Overview}
            \input{5 - research plan/main.tex}
            \input{6 - summary/main.tex}
    
    
    %\section{}
    \newpage
    \pagenumbering{gobble}
        \printbibliography


    \newpage
    \pagenumbering{roman}
    \appendix
        \part{Appendices}
            \input{8 - Hilbert complexes/main.tex}
            \input{9 - weak conservation proofs/main.tex}
\end{document}

    
    
    %\section{}
    \newpage
    \pagenumbering{gobble}
        \printbibliography


    \newpage
    \pagenumbering{roman}
    \appendix
        \part{Appendices}
            \documentclass[12pt, a4paper]{report}

\input{template/main.tex}

\title{\BA{Title in Progress...}}
\author{Boris Andrews}
\affil{Mathematical Institute, University of Oxford}
\date{\today}


\begin{document}
    \pagenumbering{gobble}
    \maketitle
    
    
    \begin{abstract}
        Magnetic confinement reactors---in particular tokamaks---offer one of the most promising options for achieving practical nuclear fusion, with the potential to provide virtually limitless, clean energy. The theoretical and numerical modeling of tokamak plasmas is simultaneously an essential component of effective reactor design, and a great research barrier. Tokamak operational conditions exhibit comparatively low Knudsen numbers. Kinetic effects, including kinetic waves and instabilities, Landau damping, bump-on-tail instabilities and more, are therefore highly influential in tokamak plasma dynamics. Purely fluid models are inherently incapable of capturing these effects, whereas the high dimensionality in purely kinetic models render them practically intractable for most relevant purposes.

        We consider a $\delta\!f$ decomposition model, with a macroscopic fluid background and microscopic kinetic correction, both fully coupled to each other. A similar manner of discretization is proposed to that used in the recent \texttt{STRUPHY} code \cite{Holderied_Possanner_Wang_2021, Holderied_2022, Li_et_al_2023} with a finite-element model for the background and a pseudo-particle/PiC model for the correction.

        The fluid background satisfies the full, non-linear, resistive, compressible, Hall MHD equations. \cite{Laakmann_Hu_Farrell_2022} introduces finite-element(-in-space) implicit timesteppers for the incompressible analogue to this system with structure-preserving (SP) properties in the ideal case, alongside parameter-robust preconditioners. We show that these timesteppers can derive from a finite-element-in-time (FET) (and finite-element-in-space) interpretation. The benefits of this reformulation are discussed, including the derivation of timesteppers that are higher order in time, and the quantifiable dissipative SP properties in the non-ideal, resistive case.
        
        We discuss possible options for extending this FET approach to timesteppers for the compressible case.

        The kinetic corrections satisfy linearized Boltzmann equations. Using a Lénard--Bernstein collision operator, these take Fokker--Planck-like forms \cite{Fokker_1914, Planck_1917} wherein pseudo-particles in the numerical model obey the neoclassical transport equations, with particle-independent Brownian drift terms. This offers a rigorous methodology for incorporating collisions into the particle transport model, without coupling the equations of motions for each particle.
        
        Works by Chen, Chacón et al. \cite{Chen_Chacón_Barnes_2011, Chacón_Chen_Barnes_2013, Chen_Chacón_2014, Chen_Chacón_2015} have developed structure-preserving particle pushers for neoclassical transport in the Vlasov equations, derived from Crank--Nicolson integrators. We show these too can can derive from a FET interpretation, similarly offering potential extensions to higher-order-in-time particle pushers. The FET formulation is used also to consider how the stochastic drift terms can be incorporated into the pushers. Stochastic gyrokinetic expansions are also discussed.

        Different options for the numerical implementation of these schemes are considered.

        Due to the efficacy of FET in the development of SP timesteppers for both the fluid and kinetic component, we hope this approach will prove effective in the future for developing SP timesteppers for the full hybrid model. We hope this will give us the opportunity to incorporate previously inaccessible kinetic effects into the highly effective, modern, finite-element MHD models.
    \end{abstract}
    
    
    \newpage
    \tableofcontents
    
    
    \newpage
    \pagenumbering{arabic}
    %\linenumbers\renewcommand\thelinenumber{\color{black!50}\arabic{linenumber}}
            \input{0 - introduction/main.tex}
        \part{Research}
            \input{1 - low-noise PiC models/main.tex}
            \input{2 - kinetic component/main.tex}
            \input{3 - fluid component/main.tex}
            \input{4 - numerical implementation/main.tex}
        \part{Project Overview}
            \input{5 - research plan/main.tex}
            \input{6 - summary/main.tex}
    
    
    %\section{}
    \newpage
    \pagenumbering{gobble}
        \printbibliography


    \newpage
    \pagenumbering{roman}
    \appendix
        \part{Appendices}
            \input{8 - Hilbert complexes/main.tex}
            \input{9 - weak conservation proofs/main.tex}
\end{document}

            \documentclass[12pt, a4paper]{report}

\input{template/main.tex}

\title{\BA{Title in Progress...}}
\author{Boris Andrews}
\affil{Mathematical Institute, University of Oxford}
\date{\today}


\begin{document}
    \pagenumbering{gobble}
    \maketitle
    
    
    \begin{abstract}
        Magnetic confinement reactors---in particular tokamaks---offer one of the most promising options for achieving practical nuclear fusion, with the potential to provide virtually limitless, clean energy. The theoretical and numerical modeling of tokamak plasmas is simultaneously an essential component of effective reactor design, and a great research barrier. Tokamak operational conditions exhibit comparatively low Knudsen numbers. Kinetic effects, including kinetic waves and instabilities, Landau damping, bump-on-tail instabilities and more, are therefore highly influential in tokamak plasma dynamics. Purely fluid models are inherently incapable of capturing these effects, whereas the high dimensionality in purely kinetic models render them practically intractable for most relevant purposes.

        We consider a $\delta\!f$ decomposition model, with a macroscopic fluid background and microscopic kinetic correction, both fully coupled to each other. A similar manner of discretization is proposed to that used in the recent \texttt{STRUPHY} code \cite{Holderied_Possanner_Wang_2021, Holderied_2022, Li_et_al_2023} with a finite-element model for the background and a pseudo-particle/PiC model for the correction.

        The fluid background satisfies the full, non-linear, resistive, compressible, Hall MHD equations. \cite{Laakmann_Hu_Farrell_2022} introduces finite-element(-in-space) implicit timesteppers for the incompressible analogue to this system with structure-preserving (SP) properties in the ideal case, alongside parameter-robust preconditioners. We show that these timesteppers can derive from a finite-element-in-time (FET) (and finite-element-in-space) interpretation. The benefits of this reformulation are discussed, including the derivation of timesteppers that are higher order in time, and the quantifiable dissipative SP properties in the non-ideal, resistive case.
        
        We discuss possible options for extending this FET approach to timesteppers for the compressible case.

        The kinetic corrections satisfy linearized Boltzmann equations. Using a Lénard--Bernstein collision operator, these take Fokker--Planck-like forms \cite{Fokker_1914, Planck_1917} wherein pseudo-particles in the numerical model obey the neoclassical transport equations, with particle-independent Brownian drift terms. This offers a rigorous methodology for incorporating collisions into the particle transport model, without coupling the equations of motions for each particle.
        
        Works by Chen, Chacón et al. \cite{Chen_Chacón_Barnes_2011, Chacón_Chen_Barnes_2013, Chen_Chacón_2014, Chen_Chacón_2015} have developed structure-preserving particle pushers for neoclassical transport in the Vlasov equations, derived from Crank--Nicolson integrators. We show these too can can derive from a FET interpretation, similarly offering potential extensions to higher-order-in-time particle pushers. The FET formulation is used also to consider how the stochastic drift terms can be incorporated into the pushers. Stochastic gyrokinetic expansions are also discussed.

        Different options for the numerical implementation of these schemes are considered.

        Due to the efficacy of FET in the development of SP timesteppers for both the fluid and kinetic component, we hope this approach will prove effective in the future for developing SP timesteppers for the full hybrid model. We hope this will give us the opportunity to incorporate previously inaccessible kinetic effects into the highly effective, modern, finite-element MHD models.
    \end{abstract}
    
    
    \newpage
    \tableofcontents
    
    
    \newpage
    \pagenumbering{arabic}
    %\linenumbers\renewcommand\thelinenumber{\color{black!50}\arabic{linenumber}}
            \input{0 - introduction/main.tex}
        \part{Research}
            \input{1 - low-noise PiC models/main.tex}
            \input{2 - kinetic component/main.tex}
            \input{3 - fluid component/main.tex}
            \input{4 - numerical implementation/main.tex}
        \part{Project Overview}
            \input{5 - research plan/main.tex}
            \input{6 - summary/main.tex}
    
    
    %\section{}
    \newpage
    \pagenumbering{gobble}
        \printbibliography


    \newpage
    \pagenumbering{roman}
    \appendix
        \part{Appendices}
            \input{8 - Hilbert complexes/main.tex}
            \input{9 - weak conservation proofs/main.tex}
\end{document}

\end{document}


\title{\BA{Title in Progress...}}
\author{Boris Andrews}
\affil{Mathematical Institute, University of Oxford}
\date{\today}


\begin{document}
    \pagenumbering{gobble}
    \maketitle
    
    
    \begin{abstract}
        Magnetic confinement reactors---in particular tokamaks---offer one of the most promising options for achieving practical nuclear fusion, with the potential to provide virtually limitless, clean energy. The theoretical and numerical modeling of tokamak plasmas is simultaneously an essential component of effective reactor design, and a great research barrier. Tokamak operational conditions exhibit comparatively low Knudsen numbers. Kinetic effects, including kinetic waves and instabilities, Landau damping, bump-on-tail instabilities and more, are therefore highly influential in tokamak plasma dynamics. Purely fluid models are inherently incapable of capturing these effects, whereas the high dimensionality in purely kinetic models render them practically intractable for most relevant purposes.

        We consider a $\delta\!f$ decomposition model, with a macroscopic fluid background and microscopic kinetic correction, both fully coupled to each other. A similar manner of discretization is proposed to that used in the recent \texttt{STRUPHY} code \cite{Holderied_Possanner_Wang_2021, Holderied_2022, Li_et_al_2023} with a finite-element model for the background and a pseudo-particle/PiC model for the correction.

        The fluid background satisfies the full, non-linear, resistive, compressible, Hall MHD equations. \cite{Laakmann_Hu_Farrell_2022} introduces finite-element(-in-space) implicit timesteppers for the incompressible analogue to this system with structure-preserving (SP) properties in the ideal case, alongside parameter-robust preconditioners. We show that these timesteppers can derive from a finite-element-in-time (FET) (and finite-element-in-space) interpretation. The benefits of this reformulation are discussed, including the derivation of timesteppers that are higher order in time, and the quantifiable dissipative SP properties in the non-ideal, resistive case.
        
        We discuss possible options for extending this FET approach to timesteppers for the compressible case.

        The kinetic corrections satisfy linearized Boltzmann equations. Using a Lénard--Bernstein collision operator, these take Fokker--Planck-like forms \cite{Fokker_1914, Planck_1917} wherein pseudo-particles in the numerical model obey the neoclassical transport equations, with particle-independent Brownian drift terms. This offers a rigorous methodology for incorporating collisions into the particle transport model, without coupling the equations of motions for each particle.
        
        Works by Chen, Chacón et al. \cite{Chen_Chacón_Barnes_2011, Chacón_Chen_Barnes_2013, Chen_Chacón_2014, Chen_Chacón_2015} have developed structure-preserving particle pushers for neoclassical transport in the Vlasov equations, derived from Crank--Nicolson integrators. We show these too can can derive from a FET interpretation, similarly offering potential extensions to higher-order-in-time particle pushers. The FET formulation is used also to consider how the stochastic drift terms can be incorporated into the pushers. Stochastic gyrokinetic expansions are also discussed.

        Different options for the numerical implementation of these schemes are considered.

        Due to the efficacy of FET in the development of SP timesteppers for both the fluid and kinetic component, we hope this approach will prove effective in the future for developing SP timesteppers for the full hybrid model. We hope this will give us the opportunity to incorporate previously inaccessible kinetic effects into the highly effective, modern, finite-element MHD models.
    \end{abstract}
    
    
    \newpage
    \tableofcontents
    
    
    \newpage
    \pagenumbering{arabic}
    %\linenumbers\renewcommand\thelinenumber{\color{black!50}\arabic{linenumber}}
            \documentclass[12pt, a4paper]{report}

\documentclass[12pt, a4paper]{report}

\input{template/main.tex}

\title{\BA{Title in Progress...}}
\author{Boris Andrews}
\affil{Mathematical Institute, University of Oxford}
\date{\today}


\begin{document}
    \pagenumbering{gobble}
    \maketitle
    
    
    \begin{abstract}
        Magnetic confinement reactors---in particular tokamaks---offer one of the most promising options for achieving practical nuclear fusion, with the potential to provide virtually limitless, clean energy. The theoretical and numerical modeling of tokamak plasmas is simultaneously an essential component of effective reactor design, and a great research barrier. Tokamak operational conditions exhibit comparatively low Knudsen numbers. Kinetic effects, including kinetic waves and instabilities, Landau damping, bump-on-tail instabilities and more, are therefore highly influential in tokamak plasma dynamics. Purely fluid models are inherently incapable of capturing these effects, whereas the high dimensionality in purely kinetic models render them practically intractable for most relevant purposes.

        We consider a $\delta\!f$ decomposition model, with a macroscopic fluid background and microscopic kinetic correction, both fully coupled to each other. A similar manner of discretization is proposed to that used in the recent \texttt{STRUPHY} code \cite{Holderied_Possanner_Wang_2021, Holderied_2022, Li_et_al_2023} with a finite-element model for the background and a pseudo-particle/PiC model for the correction.

        The fluid background satisfies the full, non-linear, resistive, compressible, Hall MHD equations. \cite{Laakmann_Hu_Farrell_2022} introduces finite-element(-in-space) implicit timesteppers for the incompressible analogue to this system with structure-preserving (SP) properties in the ideal case, alongside parameter-robust preconditioners. We show that these timesteppers can derive from a finite-element-in-time (FET) (and finite-element-in-space) interpretation. The benefits of this reformulation are discussed, including the derivation of timesteppers that are higher order in time, and the quantifiable dissipative SP properties in the non-ideal, resistive case.
        
        We discuss possible options for extending this FET approach to timesteppers for the compressible case.

        The kinetic corrections satisfy linearized Boltzmann equations. Using a Lénard--Bernstein collision operator, these take Fokker--Planck-like forms \cite{Fokker_1914, Planck_1917} wherein pseudo-particles in the numerical model obey the neoclassical transport equations, with particle-independent Brownian drift terms. This offers a rigorous methodology for incorporating collisions into the particle transport model, without coupling the equations of motions for each particle.
        
        Works by Chen, Chacón et al. \cite{Chen_Chacón_Barnes_2011, Chacón_Chen_Barnes_2013, Chen_Chacón_2014, Chen_Chacón_2015} have developed structure-preserving particle pushers for neoclassical transport in the Vlasov equations, derived from Crank--Nicolson integrators. We show these too can can derive from a FET interpretation, similarly offering potential extensions to higher-order-in-time particle pushers. The FET formulation is used also to consider how the stochastic drift terms can be incorporated into the pushers. Stochastic gyrokinetic expansions are also discussed.

        Different options for the numerical implementation of these schemes are considered.

        Due to the efficacy of FET in the development of SP timesteppers for both the fluid and kinetic component, we hope this approach will prove effective in the future for developing SP timesteppers for the full hybrid model. We hope this will give us the opportunity to incorporate previously inaccessible kinetic effects into the highly effective, modern, finite-element MHD models.
    \end{abstract}
    
    
    \newpage
    \tableofcontents
    
    
    \newpage
    \pagenumbering{arabic}
    %\linenumbers\renewcommand\thelinenumber{\color{black!50}\arabic{linenumber}}
            \input{0 - introduction/main.tex}
        \part{Research}
            \input{1 - low-noise PiC models/main.tex}
            \input{2 - kinetic component/main.tex}
            \input{3 - fluid component/main.tex}
            \input{4 - numerical implementation/main.tex}
        \part{Project Overview}
            \input{5 - research plan/main.tex}
            \input{6 - summary/main.tex}
    
    
    %\section{}
    \newpage
    \pagenumbering{gobble}
        \printbibliography


    \newpage
    \pagenumbering{roman}
    \appendix
        \part{Appendices}
            \input{8 - Hilbert complexes/main.tex}
            \input{9 - weak conservation proofs/main.tex}
\end{document}


\title{\BA{Title in Progress...}}
\author{Boris Andrews}
\affil{Mathematical Institute, University of Oxford}
\date{\today}


\begin{document}
    \pagenumbering{gobble}
    \maketitle
    
    
    \begin{abstract}
        Magnetic confinement reactors---in particular tokamaks---offer one of the most promising options for achieving practical nuclear fusion, with the potential to provide virtually limitless, clean energy. The theoretical and numerical modeling of tokamak plasmas is simultaneously an essential component of effective reactor design, and a great research barrier. Tokamak operational conditions exhibit comparatively low Knudsen numbers. Kinetic effects, including kinetic waves and instabilities, Landau damping, bump-on-tail instabilities and more, are therefore highly influential in tokamak plasma dynamics. Purely fluid models are inherently incapable of capturing these effects, whereas the high dimensionality in purely kinetic models render them practically intractable for most relevant purposes.

        We consider a $\delta\!f$ decomposition model, with a macroscopic fluid background and microscopic kinetic correction, both fully coupled to each other. A similar manner of discretization is proposed to that used in the recent \texttt{STRUPHY} code \cite{Holderied_Possanner_Wang_2021, Holderied_2022, Li_et_al_2023} with a finite-element model for the background and a pseudo-particle/PiC model for the correction.

        The fluid background satisfies the full, non-linear, resistive, compressible, Hall MHD equations. \cite{Laakmann_Hu_Farrell_2022} introduces finite-element(-in-space) implicit timesteppers for the incompressible analogue to this system with structure-preserving (SP) properties in the ideal case, alongside parameter-robust preconditioners. We show that these timesteppers can derive from a finite-element-in-time (FET) (and finite-element-in-space) interpretation. The benefits of this reformulation are discussed, including the derivation of timesteppers that are higher order in time, and the quantifiable dissipative SP properties in the non-ideal, resistive case.
        
        We discuss possible options for extending this FET approach to timesteppers for the compressible case.

        The kinetic corrections satisfy linearized Boltzmann equations. Using a Lénard--Bernstein collision operator, these take Fokker--Planck-like forms \cite{Fokker_1914, Planck_1917} wherein pseudo-particles in the numerical model obey the neoclassical transport equations, with particle-independent Brownian drift terms. This offers a rigorous methodology for incorporating collisions into the particle transport model, without coupling the equations of motions for each particle.
        
        Works by Chen, Chacón et al. \cite{Chen_Chacón_Barnes_2011, Chacón_Chen_Barnes_2013, Chen_Chacón_2014, Chen_Chacón_2015} have developed structure-preserving particle pushers for neoclassical transport in the Vlasov equations, derived from Crank--Nicolson integrators. We show these too can can derive from a FET interpretation, similarly offering potential extensions to higher-order-in-time particle pushers. The FET formulation is used also to consider how the stochastic drift terms can be incorporated into the pushers. Stochastic gyrokinetic expansions are also discussed.

        Different options for the numerical implementation of these schemes are considered.

        Due to the efficacy of FET in the development of SP timesteppers for both the fluid and kinetic component, we hope this approach will prove effective in the future for developing SP timesteppers for the full hybrid model. We hope this will give us the opportunity to incorporate previously inaccessible kinetic effects into the highly effective, modern, finite-element MHD models.
    \end{abstract}
    
    
    \newpage
    \tableofcontents
    
    
    \newpage
    \pagenumbering{arabic}
    %\linenumbers\renewcommand\thelinenumber{\color{black!50}\arabic{linenumber}}
            \documentclass[12pt, a4paper]{report}

\input{template/main.tex}

\title{\BA{Title in Progress...}}
\author{Boris Andrews}
\affil{Mathematical Institute, University of Oxford}
\date{\today}


\begin{document}
    \pagenumbering{gobble}
    \maketitle
    
    
    \begin{abstract}
        Magnetic confinement reactors---in particular tokamaks---offer one of the most promising options for achieving practical nuclear fusion, with the potential to provide virtually limitless, clean energy. The theoretical and numerical modeling of tokamak plasmas is simultaneously an essential component of effective reactor design, and a great research barrier. Tokamak operational conditions exhibit comparatively low Knudsen numbers. Kinetic effects, including kinetic waves and instabilities, Landau damping, bump-on-tail instabilities and more, are therefore highly influential in tokamak plasma dynamics. Purely fluid models are inherently incapable of capturing these effects, whereas the high dimensionality in purely kinetic models render them practically intractable for most relevant purposes.

        We consider a $\delta\!f$ decomposition model, with a macroscopic fluid background and microscopic kinetic correction, both fully coupled to each other. A similar manner of discretization is proposed to that used in the recent \texttt{STRUPHY} code \cite{Holderied_Possanner_Wang_2021, Holderied_2022, Li_et_al_2023} with a finite-element model for the background and a pseudo-particle/PiC model for the correction.

        The fluid background satisfies the full, non-linear, resistive, compressible, Hall MHD equations. \cite{Laakmann_Hu_Farrell_2022} introduces finite-element(-in-space) implicit timesteppers for the incompressible analogue to this system with structure-preserving (SP) properties in the ideal case, alongside parameter-robust preconditioners. We show that these timesteppers can derive from a finite-element-in-time (FET) (and finite-element-in-space) interpretation. The benefits of this reformulation are discussed, including the derivation of timesteppers that are higher order in time, and the quantifiable dissipative SP properties in the non-ideal, resistive case.
        
        We discuss possible options for extending this FET approach to timesteppers for the compressible case.

        The kinetic corrections satisfy linearized Boltzmann equations. Using a Lénard--Bernstein collision operator, these take Fokker--Planck-like forms \cite{Fokker_1914, Planck_1917} wherein pseudo-particles in the numerical model obey the neoclassical transport equations, with particle-independent Brownian drift terms. This offers a rigorous methodology for incorporating collisions into the particle transport model, without coupling the equations of motions for each particle.
        
        Works by Chen, Chacón et al. \cite{Chen_Chacón_Barnes_2011, Chacón_Chen_Barnes_2013, Chen_Chacón_2014, Chen_Chacón_2015} have developed structure-preserving particle pushers for neoclassical transport in the Vlasov equations, derived from Crank--Nicolson integrators. We show these too can can derive from a FET interpretation, similarly offering potential extensions to higher-order-in-time particle pushers. The FET formulation is used also to consider how the stochastic drift terms can be incorporated into the pushers. Stochastic gyrokinetic expansions are also discussed.

        Different options for the numerical implementation of these schemes are considered.

        Due to the efficacy of FET in the development of SP timesteppers for both the fluid and kinetic component, we hope this approach will prove effective in the future for developing SP timesteppers for the full hybrid model. We hope this will give us the opportunity to incorporate previously inaccessible kinetic effects into the highly effective, modern, finite-element MHD models.
    \end{abstract}
    
    
    \newpage
    \tableofcontents
    
    
    \newpage
    \pagenumbering{arabic}
    %\linenumbers\renewcommand\thelinenumber{\color{black!50}\arabic{linenumber}}
            \input{0 - introduction/main.tex}
        \part{Research}
            \input{1 - low-noise PiC models/main.tex}
            \input{2 - kinetic component/main.tex}
            \input{3 - fluid component/main.tex}
            \input{4 - numerical implementation/main.tex}
        \part{Project Overview}
            \input{5 - research plan/main.tex}
            \input{6 - summary/main.tex}
    
    
    %\section{}
    \newpage
    \pagenumbering{gobble}
        \printbibliography


    \newpage
    \pagenumbering{roman}
    \appendix
        \part{Appendices}
            \input{8 - Hilbert complexes/main.tex}
            \input{9 - weak conservation proofs/main.tex}
\end{document}

        \part{Research}
            \documentclass[12pt, a4paper]{report}

\input{template/main.tex}

\title{\BA{Title in Progress...}}
\author{Boris Andrews}
\affil{Mathematical Institute, University of Oxford}
\date{\today}


\begin{document}
    \pagenumbering{gobble}
    \maketitle
    
    
    \begin{abstract}
        Magnetic confinement reactors---in particular tokamaks---offer one of the most promising options for achieving practical nuclear fusion, with the potential to provide virtually limitless, clean energy. The theoretical and numerical modeling of tokamak plasmas is simultaneously an essential component of effective reactor design, and a great research barrier. Tokamak operational conditions exhibit comparatively low Knudsen numbers. Kinetic effects, including kinetic waves and instabilities, Landau damping, bump-on-tail instabilities and more, are therefore highly influential in tokamak plasma dynamics. Purely fluid models are inherently incapable of capturing these effects, whereas the high dimensionality in purely kinetic models render them practically intractable for most relevant purposes.

        We consider a $\delta\!f$ decomposition model, with a macroscopic fluid background and microscopic kinetic correction, both fully coupled to each other. A similar manner of discretization is proposed to that used in the recent \texttt{STRUPHY} code \cite{Holderied_Possanner_Wang_2021, Holderied_2022, Li_et_al_2023} with a finite-element model for the background and a pseudo-particle/PiC model for the correction.

        The fluid background satisfies the full, non-linear, resistive, compressible, Hall MHD equations. \cite{Laakmann_Hu_Farrell_2022} introduces finite-element(-in-space) implicit timesteppers for the incompressible analogue to this system with structure-preserving (SP) properties in the ideal case, alongside parameter-robust preconditioners. We show that these timesteppers can derive from a finite-element-in-time (FET) (and finite-element-in-space) interpretation. The benefits of this reformulation are discussed, including the derivation of timesteppers that are higher order in time, and the quantifiable dissipative SP properties in the non-ideal, resistive case.
        
        We discuss possible options for extending this FET approach to timesteppers for the compressible case.

        The kinetic corrections satisfy linearized Boltzmann equations. Using a Lénard--Bernstein collision operator, these take Fokker--Planck-like forms \cite{Fokker_1914, Planck_1917} wherein pseudo-particles in the numerical model obey the neoclassical transport equations, with particle-independent Brownian drift terms. This offers a rigorous methodology for incorporating collisions into the particle transport model, without coupling the equations of motions for each particle.
        
        Works by Chen, Chacón et al. \cite{Chen_Chacón_Barnes_2011, Chacón_Chen_Barnes_2013, Chen_Chacón_2014, Chen_Chacón_2015} have developed structure-preserving particle pushers for neoclassical transport in the Vlasov equations, derived from Crank--Nicolson integrators. We show these too can can derive from a FET interpretation, similarly offering potential extensions to higher-order-in-time particle pushers. The FET formulation is used also to consider how the stochastic drift terms can be incorporated into the pushers. Stochastic gyrokinetic expansions are also discussed.

        Different options for the numerical implementation of these schemes are considered.

        Due to the efficacy of FET in the development of SP timesteppers for both the fluid and kinetic component, we hope this approach will prove effective in the future for developing SP timesteppers for the full hybrid model. We hope this will give us the opportunity to incorporate previously inaccessible kinetic effects into the highly effective, modern, finite-element MHD models.
    \end{abstract}
    
    
    \newpage
    \tableofcontents
    
    
    \newpage
    \pagenumbering{arabic}
    %\linenumbers\renewcommand\thelinenumber{\color{black!50}\arabic{linenumber}}
            \input{0 - introduction/main.tex}
        \part{Research}
            \input{1 - low-noise PiC models/main.tex}
            \input{2 - kinetic component/main.tex}
            \input{3 - fluid component/main.tex}
            \input{4 - numerical implementation/main.tex}
        \part{Project Overview}
            \input{5 - research plan/main.tex}
            \input{6 - summary/main.tex}
    
    
    %\section{}
    \newpage
    \pagenumbering{gobble}
        \printbibliography


    \newpage
    \pagenumbering{roman}
    \appendix
        \part{Appendices}
            \input{8 - Hilbert complexes/main.tex}
            \input{9 - weak conservation proofs/main.tex}
\end{document}

            \documentclass[12pt, a4paper]{report}

\input{template/main.tex}

\title{\BA{Title in Progress...}}
\author{Boris Andrews}
\affil{Mathematical Institute, University of Oxford}
\date{\today}


\begin{document}
    \pagenumbering{gobble}
    \maketitle
    
    
    \begin{abstract}
        Magnetic confinement reactors---in particular tokamaks---offer one of the most promising options for achieving practical nuclear fusion, with the potential to provide virtually limitless, clean energy. The theoretical and numerical modeling of tokamak plasmas is simultaneously an essential component of effective reactor design, and a great research barrier. Tokamak operational conditions exhibit comparatively low Knudsen numbers. Kinetic effects, including kinetic waves and instabilities, Landau damping, bump-on-tail instabilities and more, are therefore highly influential in tokamak plasma dynamics. Purely fluid models are inherently incapable of capturing these effects, whereas the high dimensionality in purely kinetic models render them practically intractable for most relevant purposes.

        We consider a $\delta\!f$ decomposition model, with a macroscopic fluid background and microscopic kinetic correction, both fully coupled to each other. A similar manner of discretization is proposed to that used in the recent \texttt{STRUPHY} code \cite{Holderied_Possanner_Wang_2021, Holderied_2022, Li_et_al_2023} with a finite-element model for the background and a pseudo-particle/PiC model for the correction.

        The fluid background satisfies the full, non-linear, resistive, compressible, Hall MHD equations. \cite{Laakmann_Hu_Farrell_2022} introduces finite-element(-in-space) implicit timesteppers for the incompressible analogue to this system with structure-preserving (SP) properties in the ideal case, alongside parameter-robust preconditioners. We show that these timesteppers can derive from a finite-element-in-time (FET) (and finite-element-in-space) interpretation. The benefits of this reformulation are discussed, including the derivation of timesteppers that are higher order in time, and the quantifiable dissipative SP properties in the non-ideal, resistive case.
        
        We discuss possible options for extending this FET approach to timesteppers for the compressible case.

        The kinetic corrections satisfy linearized Boltzmann equations. Using a Lénard--Bernstein collision operator, these take Fokker--Planck-like forms \cite{Fokker_1914, Planck_1917} wherein pseudo-particles in the numerical model obey the neoclassical transport equations, with particle-independent Brownian drift terms. This offers a rigorous methodology for incorporating collisions into the particle transport model, without coupling the equations of motions for each particle.
        
        Works by Chen, Chacón et al. \cite{Chen_Chacón_Barnes_2011, Chacón_Chen_Barnes_2013, Chen_Chacón_2014, Chen_Chacón_2015} have developed structure-preserving particle pushers for neoclassical transport in the Vlasov equations, derived from Crank--Nicolson integrators. We show these too can can derive from a FET interpretation, similarly offering potential extensions to higher-order-in-time particle pushers. The FET formulation is used also to consider how the stochastic drift terms can be incorporated into the pushers. Stochastic gyrokinetic expansions are also discussed.

        Different options for the numerical implementation of these schemes are considered.

        Due to the efficacy of FET in the development of SP timesteppers for both the fluid and kinetic component, we hope this approach will prove effective in the future for developing SP timesteppers for the full hybrid model. We hope this will give us the opportunity to incorporate previously inaccessible kinetic effects into the highly effective, modern, finite-element MHD models.
    \end{abstract}
    
    
    \newpage
    \tableofcontents
    
    
    \newpage
    \pagenumbering{arabic}
    %\linenumbers\renewcommand\thelinenumber{\color{black!50}\arabic{linenumber}}
            \input{0 - introduction/main.tex}
        \part{Research}
            \input{1 - low-noise PiC models/main.tex}
            \input{2 - kinetic component/main.tex}
            \input{3 - fluid component/main.tex}
            \input{4 - numerical implementation/main.tex}
        \part{Project Overview}
            \input{5 - research plan/main.tex}
            \input{6 - summary/main.tex}
    
    
    %\section{}
    \newpage
    \pagenumbering{gobble}
        \printbibliography


    \newpage
    \pagenumbering{roman}
    \appendix
        \part{Appendices}
            \input{8 - Hilbert complexes/main.tex}
            \input{9 - weak conservation proofs/main.tex}
\end{document}

            \documentclass[12pt, a4paper]{report}

\input{template/main.tex}

\title{\BA{Title in Progress...}}
\author{Boris Andrews}
\affil{Mathematical Institute, University of Oxford}
\date{\today}


\begin{document}
    \pagenumbering{gobble}
    \maketitle
    
    
    \begin{abstract}
        Magnetic confinement reactors---in particular tokamaks---offer one of the most promising options for achieving practical nuclear fusion, with the potential to provide virtually limitless, clean energy. The theoretical and numerical modeling of tokamak plasmas is simultaneously an essential component of effective reactor design, and a great research barrier. Tokamak operational conditions exhibit comparatively low Knudsen numbers. Kinetic effects, including kinetic waves and instabilities, Landau damping, bump-on-tail instabilities and more, are therefore highly influential in tokamak plasma dynamics. Purely fluid models are inherently incapable of capturing these effects, whereas the high dimensionality in purely kinetic models render them practically intractable for most relevant purposes.

        We consider a $\delta\!f$ decomposition model, with a macroscopic fluid background and microscopic kinetic correction, both fully coupled to each other. A similar manner of discretization is proposed to that used in the recent \texttt{STRUPHY} code \cite{Holderied_Possanner_Wang_2021, Holderied_2022, Li_et_al_2023} with a finite-element model for the background and a pseudo-particle/PiC model for the correction.

        The fluid background satisfies the full, non-linear, resistive, compressible, Hall MHD equations. \cite{Laakmann_Hu_Farrell_2022} introduces finite-element(-in-space) implicit timesteppers for the incompressible analogue to this system with structure-preserving (SP) properties in the ideal case, alongside parameter-robust preconditioners. We show that these timesteppers can derive from a finite-element-in-time (FET) (and finite-element-in-space) interpretation. The benefits of this reformulation are discussed, including the derivation of timesteppers that are higher order in time, and the quantifiable dissipative SP properties in the non-ideal, resistive case.
        
        We discuss possible options for extending this FET approach to timesteppers for the compressible case.

        The kinetic corrections satisfy linearized Boltzmann equations. Using a Lénard--Bernstein collision operator, these take Fokker--Planck-like forms \cite{Fokker_1914, Planck_1917} wherein pseudo-particles in the numerical model obey the neoclassical transport equations, with particle-independent Brownian drift terms. This offers a rigorous methodology for incorporating collisions into the particle transport model, without coupling the equations of motions for each particle.
        
        Works by Chen, Chacón et al. \cite{Chen_Chacón_Barnes_2011, Chacón_Chen_Barnes_2013, Chen_Chacón_2014, Chen_Chacón_2015} have developed structure-preserving particle pushers for neoclassical transport in the Vlasov equations, derived from Crank--Nicolson integrators. We show these too can can derive from a FET interpretation, similarly offering potential extensions to higher-order-in-time particle pushers. The FET formulation is used also to consider how the stochastic drift terms can be incorporated into the pushers. Stochastic gyrokinetic expansions are also discussed.

        Different options for the numerical implementation of these schemes are considered.

        Due to the efficacy of FET in the development of SP timesteppers for both the fluid and kinetic component, we hope this approach will prove effective in the future for developing SP timesteppers for the full hybrid model. We hope this will give us the opportunity to incorporate previously inaccessible kinetic effects into the highly effective, modern, finite-element MHD models.
    \end{abstract}
    
    
    \newpage
    \tableofcontents
    
    
    \newpage
    \pagenumbering{arabic}
    %\linenumbers\renewcommand\thelinenumber{\color{black!50}\arabic{linenumber}}
            \input{0 - introduction/main.tex}
        \part{Research}
            \input{1 - low-noise PiC models/main.tex}
            \input{2 - kinetic component/main.tex}
            \input{3 - fluid component/main.tex}
            \input{4 - numerical implementation/main.tex}
        \part{Project Overview}
            \input{5 - research plan/main.tex}
            \input{6 - summary/main.tex}
    
    
    %\section{}
    \newpage
    \pagenumbering{gobble}
        \printbibliography


    \newpage
    \pagenumbering{roman}
    \appendix
        \part{Appendices}
            \input{8 - Hilbert complexes/main.tex}
            \input{9 - weak conservation proofs/main.tex}
\end{document}

            \documentclass[12pt, a4paper]{report}

\input{template/main.tex}

\title{\BA{Title in Progress...}}
\author{Boris Andrews}
\affil{Mathematical Institute, University of Oxford}
\date{\today}


\begin{document}
    \pagenumbering{gobble}
    \maketitle
    
    
    \begin{abstract}
        Magnetic confinement reactors---in particular tokamaks---offer one of the most promising options for achieving practical nuclear fusion, with the potential to provide virtually limitless, clean energy. The theoretical and numerical modeling of tokamak plasmas is simultaneously an essential component of effective reactor design, and a great research barrier. Tokamak operational conditions exhibit comparatively low Knudsen numbers. Kinetic effects, including kinetic waves and instabilities, Landau damping, bump-on-tail instabilities and more, are therefore highly influential in tokamak plasma dynamics. Purely fluid models are inherently incapable of capturing these effects, whereas the high dimensionality in purely kinetic models render them practically intractable for most relevant purposes.

        We consider a $\delta\!f$ decomposition model, with a macroscopic fluid background and microscopic kinetic correction, both fully coupled to each other. A similar manner of discretization is proposed to that used in the recent \texttt{STRUPHY} code \cite{Holderied_Possanner_Wang_2021, Holderied_2022, Li_et_al_2023} with a finite-element model for the background and a pseudo-particle/PiC model for the correction.

        The fluid background satisfies the full, non-linear, resistive, compressible, Hall MHD equations. \cite{Laakmann_Hu_Farrell_2022} introduces finite-element(-in-space) implicit timesteppers for the incompressible analogue to this system with structure-preserving (SP) properties in the ideal case, alongside parameter-robust preconditioners. We show that these timesteppers can derive from a finite-element-in-time (FET) (and finite-element-in-space) interpretation. The benefits of this reformulation are discussed, including the derivation of timesteppers that are higher order in time, and the quantifiable dissipative SP properties in the non-ideal, resistive case.
        
        We discuss possible options for extending this FET approach to timesteppers for the compressible case.

        The kinetic corrections satisfy linearized Boltzmann equations. Using a Lénard--Bernstein collision operator, these take Fokker--Planck-like forms \cite{Fokker_1914, Planck_1917} wherein pseudo-particles in the numerical model obey the neoclassical transport equations, with particle-independent Brownian drift terms. This offers a rigorous methodology for incorporating collisions into the particle transport model, without coupling the equations of motions for each particle.
        
        Works by Chen, Chacón et al. \cite{Chen_Chacón_Barnes_2011, Chacón_Chen_Barnes_2013, Chen_Chacón_2014, Chen_Chacón_2015} have developed structure-preserving particle pushers for neoclassical transport in the Vlasov equations, derived from Crank--Nicolson integrators. We show these too can can derive from a FET interpretation, similarly offering potential extensions to higher-order-in-time particle pushers. The FET formulation is used also to consider how the stochastic drift terms can be incorporated into the pushers. Stochastic gyrokinetic expansions are also discussed.

        Different options for the numerical implementation of these schemes are considered.

        Due to the efficacy of FET in the development of SP timesteppers for both the fluid and kinetic component, we hope this approach will prove effective in the future for developing SP timesteppers for the full hybrid model. We hope this will give us the opportunity to incorporate previously inaccessible kinetic effects into the highly effective, modern, finite-element MHD models.
    \end{abstract}
    
    
    \newpage
    \tableofcontents
    
    
    \newpage
    \pagenumbering{arabic}
    %\linenumbers\renewcommand\thelinenumber{\color{black!50}\arabic{linenumber}}
            \input{0 - introduction/main.tex}
        \part{Research}
            \input{1 - low-noise PiC models/main.tex}
            \input{2 - kinetic component/main.tex}
            \input{3 - fluid component/main.tex}
            \input{4 - numerical implementation/main.tex}
        \part{Project Overview}
            \input{5 - research plan/main.tex}
            \input{6 - summary/main.tex}
    
    
    %\section{}
    \newpage
    \pagenumbering{gobble}
        \printbibliography


    \newpage
    \pagenumbering{roman}
    \appendix
        \part{Appendices}
            \input{8 - Hilbert complexes/main.tex}
            \input{9 - weak conservation proofs/main.tex}
\end{document}

        \part{Project Overview}
            \documentclass[12pt, a4paper]{report}

\input{template/main.tex}

\title{\BA{Title in Progress...}}
\author{Boris Andrews}
\affil{Mathematical Institute, University of Oxford}
\date{\today}


\begin{document}
    \pagenumbering{gobble}
    \maketitle
    
    
    \begin{abstract}
        Magnetic confinement reactors---in particular tokamaks---offer one of the most promising options for achieving practical nuclear fusion, with the potential to provide virtually limitless, clean energy. The theoretical and numerical modeling of tokamak plasmas is simultaneously an essential component of effective reactor design, and a great research barrier. Tokamak operational conditions exhibit comparatively low Knudsen numbers. Kinetic effects, including kinetic waves and instabilities, Landau damping, bump-on-tail instabilities and more, are therefore highly influential in tokamak plasma dynamics. Purely fluid models are inherently incapable of capturing these effects, whereas the high dimensionality in purely kinetic models render them practically intractable for most relevant purposes.

        We consider a $\delta\!f$ decomposition model, with a macroscopic fluid background and microscopic kinetic correction, both fully coupled to each other. A similar manner of discretization is proposed to that used in the recent \texttt{STRUPHY} code \cite{Holderied_Possanner_Wang_2021, Holderied_2022, Li_et_al_2023} with a finite-element model for the background and a pseudo-particle/PiC model for the correction.

        The fluid background satisfies the full, non-linear, resistive, compressible, Hall MHD equations. \cite{Laakmann_Hu_Farrell_2022} introduces finite-element(-in-space) implicit timesteppers for the incompressible analogue to this system with structure-preserving (SP) properties in the ideal case, alongside parameter-robust preconditioners. We show that these timesteppers can derive from a finite-element-in-time (FET) (and finite-element-in-space) interpretation. The benefits of this reformulation are discussed, including the derivation of timesteppers that are higher order in time, and the quantifiable dissipative SP properties in the non-ideal, resistive case.
        
        We discuss possible options for extending this FET approach to timesteppers for the compressible case.

        The kinetic corrections satisfy linearized Boltzmann equations. Using a Lénard--Bernstein collision operator, these take Fokker--Planck-like forms \cite{Fokker_1914, Planck_1917} wherein pseudo-particles in the numerical model obey the neoclassical transport equations, with particle-independent Brownian drift terms. This offers a rigorous methodology for incorporating collisions into the particle transport model, without coupling the equations of motions for each particle.
        
        Works by Chen, Chacón et al. \cite{Chen_Chacón_Barnes_2011, Chacón_Chen_Barnes_2013, Chen_Chacón_2014, Chen_Chacón_2015} have developed structure-preserving particle pushers for neoclassical transport in the Vlasov equations, derived from Crank--Nicolson integrators. We show these too can can derive from a FET interpretation, similarly offering potential extensions to higher-order-in-time particle pushers. The FET formulation is used also to consider how the stochastic drift terms can be incorporated into the pushers. Stochastic gyrokinetic expansions are also discussed.

        Different options for the numerical implementation of these schemes are considered.

        Due to the efficacy of FET in the development of SP timesteppers for both the fluid and kinetic component, we hope this approach will prove effective in the future for developing SP timesteppers for the full hybrid model. We hope this will give us the opportunity to incorporate previously inaccessible kinetic effects into the highly effective, modern, finite-element MHD models.
    \end{abstract}
    
    
    \newpage
    \tableofcontents
    
    
    \newpage
    \pagenumbering{arabic}
    %\linenumbers\renewcommand\thelinenumber{\color{black!50}\arabic{linenumber}}
            \input{0 - introduction/main.tex}
        \part{Research}
            \input{1 - low-noise PiC models/main.tex}
            \input{2 - kinetic component/main.tex}
            \input{3 - fluid component/main.tex}
            \input{4 - numerical implementation/main.tex}
        \part{Project Overview}
            \input{5 - research plan/main.tex}
            \input{6 - summary/main.tex}
    
    
    %\section{}
    \newpage
    \pagenumbering{gobble}
        \printbibliography


    \newpage
    \pagenumbering{roman}
    \appendix
        \part{Appendices}
            \input{8 - Hilbert complexes/main.tex}
            \input{9 - weak conservation proofs/main.tex}
\end{document}

            \documentclass[12pt, a4paper]{report}

\input{template/main.tex}

\title{\BA{Title in Progress...}}
\author{Boris Andrews}
\affil{Mathematical Institute, University of Oxford}
\date{\today}


\begin{document}
    \pagenumbering{gobble}
    \maketitle
    
    
    \begin{abstract}
        Magnetic confinement reactors---in particular tokamaks---offer one of the most promising options for achieving practical nuclear fusion, with the potential to provide virtually limitless, clean energy. The theoretical and numerical modeling of tokamak plasmas is simultaneously an essential component of effective reactor design, and a great research barrier. Tokamak operational conditions exhibit comparatively low Knudsen numbers. Kinetic effects, including kinetic waves and instabilities, Landau damping, bump-on-tail instabilities and more, are therefore highly influential in tokamak plasma dynamics. Purely fluid models are inherently incapable of capturing these effects, whereas the high dimensionality in purely kinetic models render them practically intractable for most relevant purposes.

        We consider a $\delta\!f$ decomposition model, with a macroscopic fluid background and microscopic kinetic correction, both fully coupled to each other. A similar manner of discretization is proposed to that used in the recent \texttt{STRUPHY} code \cite{Holderied_Possanner_Wang_2021, Holderied_2022, Li_et_al_2023} with a finite-element model for the background and a pseudo-particle/PiC model for the correction.

        The fluid background satisfies the full, non-linear, resistive, compressible, Hall MHD equations. \cite{Laakmann_Hu_Farrell_2022} introduces finite-element(-in-space) implicit timesteppers for the incompressible analogue to this system with structure-preserving (SP) properties in the ideal case, alongside parameter-robust preconditioners. We show that these timesteppers can derive from a finite-element-in-time (FET) (and finite-element-in-space) interpretation. The benefits of this reformulation are discussed, including the derivation of timesteppers that are higher order in time, and the quantifiable dissipative SP properties in the non-ideal, resistive case.
        
        We discuss possible options for extending this FET approach to timesteppers for the compressible case.

        The kinetic corrections satisfy linearized Boltzmann equations. Using a Lénard--Bernstein collision operator, these take Fokker--Planck-like forms \cite{Fokker_1914, Planck_1917} wherein pseudo-particles in the numerical model obey the neoclassical transport equations, with particle-independent Brownian drift terms. This offers a rigorous methodology for incorporating collisions into the particle transport model, without coupling the equations of motions for each particle.
        
        Works by Chen, Chacón et al. \cite{Chen_Chacón_Barnes_2011, Chacón_Chen_Barnes_2013, Chen_Chacón_2014, Chen_Chacón_2015} have developed structure-preserving particle pushers for neoclassical transport in the Vlasov equations, derived from Crank--Nicolson integrators. We show these too can can derive from a FET interpretation, similarly offering potential extensions to higher-order-in-time particle pushers. The FET formulation is used also to consider how the stochastic drift terms can be incorporated into the pushers. Stochastic gyrokinetic expansions are also discussed.

        Different options for the numerical implementation of these schemes are considered.

        Due to the efficacy of FET in the development of SP timesteppers for both the fluid and kinetic component, we hope this approach will prove effective in the future for developing SP timesteppers for the full hybrid model. We hope this will give us the opportunity to incorporate previously inaccessible kinetic effects into the highly effective, modern, finite-element MHD models.
    \end{abstract}
    
    
    \newpage
    \tableofcontents
    
    
    \newpage
    \pagenumbering{arabic}
    %\linenumbers\renewcommand\thelinenumber{\color{black!50}\arabic{linenumber}}
            \input{0 - introduction/main.tex}
        \part{Research}
            \input{1 - low-noise PiC models/main.tex}
            \input{2 - kinetic component/main.tex}
            \input{3 - fluid component/main.tex}
            \input{4 - numerical implementation/main.tex}
        \part{Project Overview}
            \input{5 - research plan/main.tex}
            \input{6 - summary/main.tex}
    
    
    %\section{}
    \newpage
    \pagenumbering{gobble}
        \printbibliography


    \newpage
    \pagenumbering{roman}
    \appendix
        \part{Appendices}
            \input{8 - Hilbert complexes/main.tex}
            \input{9 - weak conservation proofs/main.tex}
\end{document}

    
    
    %\section{}
    \newpage
    \pagenumbering{gobble}
        \printbibliography


    \newpage
    \pagenumbering{roman}
    \appendix
        \part{Appendices}
            \documentclass[12pt, a4paper]{report}

\input{template/main.tex}

\title{\BA{Title in Progress...}}
\author{Boris Andrews}
\affil{Mathematical Institute, University of Oxford}
\date{\today}


\begin{document}
    \pagenumbering{gobble}
    \maketitle
    
    
    \begin{abstract}
        Magnetic confinement reactors---in particular tokamaks---offer one of the most promising options for achieving practical nuclear fusion, with the potential to provide virtually limitless, clean energy. The theoretical and numerical modeling of tokamak plasmas is simultaneously an essential component of effective reactor design, and a great research barrier. Tokamak operational conditions exhibit comparatively low Knudsen numbers. Kinetic effects, including kinetic waves and instabilities, Landau damping, bump-on-tail instabilities and more, are therefore highly influential in tokamak plasma dynamics. Purely fluid models are inherently incapable of capturing these effects, whereas the high dimensionality in purely kinetic models render them practically intractable for most relevant purposes.

        We consider a $\delta\!f$ decomposition model, with a macroscopic fluid background and microscopic kinetic correction, both fully coupled to each other. A similar manner of discretization is proposed to that used in the recent \texttt{STRUPHY} code \cite{Holderied_Possanner_Wang_2021, Holderied_2022, Li_et_al_2023} with a finite-element model for the background and a pseudo-particle/PiC model for the correction.

        The fluid background satisfies the full, non-linear, resistive, compressible, Hall MHD equations. \cite{Laakmann_Hu_Farrell_2022} introduces finite-element(-in-space) implicit timesteppers for the incompressible analogue to this system with structure-preserving (SP) properties in the ideal case, alongside parameter-robust preconditioners. We show that these timesteppers can derive from a finite-element-in-time (FET) (and finite-element-in-space) interpretation. The benefits of this reformulation are discussed, including the derivation of timesteppers that are higher order in time, and the quantifiable dissipative SP properties in the non-ideal, resistive case.
        
        We discuss possible options for extending this FET approach to timesteppers for the compressible case.

        The kinetic corrections satisfy linearized Boltzmann equations. Using a Lénard--Bernstein collision operator, these take Fokker--Planck-like forms \cite{Fokker_1914, Planck_1917} wherein pseudo-particles in the numerical model obey the neoclassical transport equations, with particle-independent Brownian drift terms. This offers a rigorous methodology for incorporating collisions into the particle transport model, without coupling the equations of motions for each particle.
        
        Works by Chen, Chacón et al. \cite{Chen_Chacón_Barnes_2011, Chacón_Chen_Barnes_2013, Chen_Chacón_2014, Chen_Chacón_2015} have developed structure-preserving particle pushers for neoclassical transport in the Vlasov equations, derived from Crank--Nicolson integrators. We show these too can can derive from a FET interpretation, similarly offering potential extensions to higher-order-in-time particle pushers. The FET formulation is used also to consider how the stochastic drift terms can be incorporated into the pushers. Stochastic gyrokinetic expansions are also discussed.

        Different options for the numerical implementation of these schemes are considered.

        Due to the efficacy of FET in the development of SP timesteppers for both the fluid and kinetic component, we hope this approach will prove effective in the future for developing SP timesteppers for the full hybrid model. We hope this will give us the opportunity to incorporate previously inaccessible kinetic effects into the highly effective, modern, finite-element MHD models.
    \end{abstract}
    
    
    \newpage
    \tableofcontents
    
    
    \newpage
    \pagenumbering{arabic}
    %\linenumbers\renewcommand\thelinenumber{\color{black!50}\arabic{linenumber}}
            \input{0 - introduction/main.tex}
        \part{Research}
            \input{1 - low-noise PiC models/main.tex}
            \input{2 - kinetic component/main.tex}
            \input{3 - fluid component/main.tex}
            \input{4 - numerical implementation/main.tex}
        \part{Project Overview}
            \input{5 - research plan/main.tex}
            \input{6 - summary/main.tex}
    
    
    %\section{}
    \newpage
    \pagenumbering{gobble}
        \printbibliography


    \newpage
    \pagenumbering{roman}
    \appendix
        \part{Appendices}
            \input{8 - Hilbert complexes/main.tex}
            \input{9 - weak conservation proofs/main.tex}
\end{document}

            \documentclass[12pt, a4paper]{report}

\input{template/main.tex}

\title{\BA{Title in Progress...}}
\author{Boris Andrews}
\affil{Mathematical Institute, University of Oxford}
\date{\today}


\begin{document}
    \pagenumbering{gobble}
    \maketitle
    
    
    \begin{abstract}
        Magnetic confinement reactors---in particular tokamaks---offer one of the most promising options for achieving practical nuclear fusion, with the potential to provide virtually limitless, clean energy. The theoretical and numerical modeling of tokamak plasmas is simultaneously an essential component of effective reactor design, and a great research barrier. Tokamak operational conditions exhibit comparatively low Knudsen numbers. Kinetic effects, including kinetic waves and instabilities, Landau damping, bump-on-tail instabilities and more, are therefore highly influential in tokamak plasma dynamics. Purely fluid models are inherently incapable of capturing these effects, whereas the high dimensionality in purely kinetic models render them practically intractable for most relevant purposes.

        We consider a $\delta\!f$ decomposition model, with a macroscopic fluid background and microscopic kinetic correction, both fully coupled to each other. A similar manner of discretization is proposed to that used in the recent \texttt{STRUPHY} code \cite{Holderied_Possanner_Wang_2021, Holderied_2022, Li_et_al_2023} with a finite-element model for the background and a pseudo-particle/PiC model for the correction.

        The fluid background satisfies the full, non-linear, resistive, compressible, Hall MHD equations. \cite{Laakmann_Hu_Farrell_2022} introduces finite-element(-in-space) implicit timesteppers for the incompressible analogue to this system with structure-preserving (SP) properties in the ideal case, alongside parameter-robust preconditioners. We show that these timesteppers can derive from a finite-element-in-time (FET) (and finite-element-in-space) interpretation. The benefits of this reformulation are discussed, including the derivation of timesteppers that are higher order in time, and the quantifiable dissipative SP properties in the non-ideal, resistive case.
        
        We discuss possible options for extending this FET approach to timesteppers for the compressible case.

        The kinetic corrections satisfy linearized Boltzmann equations. Using a Lénard--Bernstein collision operator, these take Fokker--Planck-like forms \cite{Fokker_1914, Planck_1917} wherein pseudo-particles in the numerical model obey the neoclassical transport equations, with particle-independent Brownian drift terms. This offers a rigorous methodology for incorporating collisions into the particle transport model, without coupling the equations of motions for each particle.
        
        Works by Chen, Chacón et al. \cite{Chen_Chacón_Barnes_2011, Chacón_Chen_Barnes_2013, Chen_Chacón_2014, Chen_Chacón_2015} have developed structure-preserving particle pushers for neoclassical transport in the Vlasov equations, derived from Crank--Nicolson integrators. We show these too can can derive from a FET interpretation, similarly offering potential extensions to higher-order-in-time particle pushers. The FET formulation is used also to consider how the stochastic drift terms can be incorporated into the pushers. Stochastic gyrokinetic expansions are also discussed.

        Different options for the numerical implementation of these schemes are considered.

        Due to the efficacy of FET in the development of SP timesteppers for both the fluid and kinetic component, we hope this approach will prove effective in the future for developing SP timesteppers for the full hybrid model. We hope this will give us the opportunity to incorporate previously inaccessible kinetic effects into the highly effective, modern, finite-element MHD models.
    \end{abstract}
    
    
    \newpage
    \tableofcontents
    
    
    \newpage
    \pagenumbering{arabic}
    %\linenumbers\renewcommand\thelinenumber{\color{black!50}\arabic{linenumber}}
            \input{0 - introduction/main.tex}
        \part{Research}
            \input{1 - low-noise PiC models/main.tex}
            \input{2 - kinetic component/main.tex}
            \input{3 - fluid component/main.tex}
            \input{4 - numerical implementation/main.tex}
        \part{Project Overview}
            \input{5 - research plan/main.tex}
            \input{6 - summary/main.tex}
    
    
    %\section{}
    \newpage
    \pagenumbering{gobble}
        \printbibliography


    \newpage
    \pagenumbering{roman}
    \appendix
        \part{Appendices}
            \input{8 - Hilbert complexes/main.tex}
            \input{9 - weak conservation proofs/main.tex}
\end{document}

\end{document}

        \part{Research}
            \documentclass[12pt, a4paper]{report}

\documentclass[12pt, a4paper]{report}

\input{template/main.tex}

\title{\BA{Title in Progress...}}
\author{Boris Andrews}
\affil{Mathematical Institute, University of Oxford}
\date{\today}


\begin{document}
    \pagenumbering{gobble}
    \maketitle
    
    
    \begin{abstract}
        Magnetic confinement reactors---in particular tokamaks---offer one of the most promising options for achieving practical nuclear fusion, with the potential to provide virtually limitless, clean energy. The theoretical and numerical modeling of tokamak plasmas is simultaneously an essential component of effective reactor design, and a great research barrier. Tokamak operational conditions exhibit comparatively low Knudsen numbers. Kinetic effects, including kinetic waves and instabilities, Landau damping, bump-on-tail instabilities and more, are therefore highly influential in tokamak plasma dynamics. Purely fluid models are inherently incapable of capturing these effects, whereas the high dimensionality in purely kinetic models render them practically intractable for most relevant purposes.

        We consider a $\delta\!f$ decomposition model, with a macroscopic fluid background and microscopic kinetic correction, both fully coupled to each other. A similar manner of discretization is proposed to that used in the recent \texttt{STRUPHY} code \cite{Holderied_Possanner_Wang_2021, Holderied_2022, Li_et_al_2023} with a finite-element model for the background and a pseudo-particle/PiC model for the correction.

        The fluid background satisfies the full, non-linear, resistive, compressible, Hall MHD equations. \cite{Laakmann_Hu_Farrell_2022} introduces finite-element(-in-space) implicit timesteppers for the incompressible analogue to this system with structure-preserving (SP) properties in the ideal case, alongside parameter-robust preconditioners. We show that these timesteppers can derive from a finite-element-in-time (FET) (and finite-element-in-space) interpretation. The benefits of this reformulation are discussed, including the derivation of timesteppers that are higher order in time, and the quantifiable dissipative SP properties in the non-ideal, resistive case.
        
        We discuss possible options for extending this FET approach to timesteppers for the compressible case.

        The kinetic corrections satisfy linearized Boltzmann equations. Using a Lénard--Bernstein collision operator, these take Fokker--Planck-like forms \cite{Fokker_1914, Planck_1917} wherein pseudo-particles in the numerical model obey the neoclassical transport equations, with particle-independent Brownian drift terms. This offers a rigorous methodology for incorporating collisions into the particle transport model, without coupling the equations of motions for each particle.
        
        Works by Chen, Chacón et al. \cite{Chen_Chacón_Barnes_2011, Chacón_Chen_Barnes_2013, Chen_Chacón_2014, Chen_Chacón_2015} have developed structure-preserving particle pushers for neoclassical transport in the Vlasov equations, derived from Crank--Nicolson integrators. We show these too can can derive from a FET interpretation, similarly offering potential extensions to higher-order-in-time particle pushers. The FET formulation is used also to consider how the stochastic drift terms can be incorporated into the pushers. Stochastic gyrokinetic expansions are also discussed.

        Different options for the numerical implementation of these schemes are considered.

        Due to the efficacy of FET in the development of SP timesteppers for both the fluid and kinetic component, we hope this approach will prove effective in the future for developing SP timesteppers for the full hybrid model. We hope this will give us the opportunity to incorporate previously inaccessible kinetic effects into the highly effective, modern, finite-element MHD models.
    \end{abstract}
    
    
    \newpage
    \tableofcontents
    
    
    \newpage
    \pagenumbering{arabic}
    %\linenumbers\renewcommand\thelinenumber{\color{black!50}\arabic{linenumber}}
            \input{0 - introduction/main.tex}
        \part{Research}
            \input{1 - low-noise PiC models/main.tex}
            \input{2 - kinetic component/main.tex}
            \input{3 - fluid component/main.tex}
            \input{4 - numerical implementation/main.tex}
        \part{Project Overview}
            \input{5 - research plan/main.tex}
            \input{6 - summary/main.tex}
    
    
    %\section{}
    \newpage
    \pagenumbering{gobble}
        \printbibliography


    \newpage
    \pagenumbering{roman}
    \appendix
        \part{Appendices}
            \input{8 - Hilbert complexes/main.tex}
            \input{9 - weak conservation proofs/main.tex}
\end{document}


\title{\BA{Title in Progress...}}
\author{Boris Andrews}
\affil{Mathematical Institute, University of Oxford}
\date{\today}


\begin{document}
    \pagenumbering{gobble}
    \maketitle
    
    
    \begin{abstract}
        Magnetic confinement reactors---in particular tokamaks---offer one of the most promising options for achieving practical nuclear fusion, with the potential to provide virtually limitless, clean energy. The theoretical and numerical modeling of tokamak plasmas is simultaneously an essential component of effective reactor design, and a great research barrier. Tokamak operational conditions exhibit comparatively low Knudsen numbers. Kinetic effects, including kinetic waves and instabilities, Landau damping, bump-on-tail instabilities and more, are therefore highly influential in tokamak plasma dynamics. Purely fluid models are inherently incapable of capturing these effects, whereas the high dimensionality in purely kinetic models render them practically intractable for most relevant purposes.

        We consider a $\delta\!f$ decomposition model, with a macroscopic fluid background and microscopic kinetic correction, both fully coupled to each other. A similar manner of discretization is proposed to that used in the recent \texttt{STRUPHY} code \cite{Holderied_Possanner_Wang_2021, Holderied_2022, Li_et_al_2023} with a finite-element model for the background and a pseudo-particle/PiC model for the correction.

        The fluid background satisfies the full, non-linear, resistive, compressible, Hall MHD equations. \cite{Laakmann_Hu_Farrell_2022} introduces finite-element(-in-space) implicit timesteppers for the incompressible analogue to this system with structure-preserving (SP) properties in the ideal case, alongside parameter-robust preconditioners. We show that these timesteppers can derive from a finite-element-in-time (FET) (and finite-element-in-space) interpretation. The benefits of this reformulation are discussed, including the derivation of timesteppers that are higher order in time, and the quantifiable dissipative SP properties in the non-ideal, resistive case.
        
        We discuss possible options for extending this FET approach to timesteppers for the compressible case.

        The kinetic corrections satisfy linearized Boltzmann equations. Using a Lénard--Bernstein collision operator, these take Fokker--Planck-like forms \cite{Fokker_1914, Planck_1917} wherein pseudo-particles in the numerical model obey the neoclassical transport equations, with particle-independent Brownian drift terms. This offers a rigorous methodology for incorporating collisions into the particle transport model, without coupling the equations of motions for each particle.
        
        Works by Chen, Chacón et al. \cite{Chen_Chacón_Barnes_2011, Chacón_Chen_Barnes_2013, Chen_Chacón_2014, Chen_Chacón_2015} have developed structure-preserving particle pushers for neoclassical transport in the Vlasov equations, derived from Crank--Nicolson integrators. We show these too can can derive from a FET interpretation, similarly offering potential extensions to higher-order-in-time particle pushers. The FET formulation is used also to consider how the stochastic drift terms can be incorporated into the pushers. Stochastic gyrokinetic expansions are also discussed.

        Different options for the numerical implementation of these schemes are considered.

        Due to the efficacy of FET in the development of SP timesteppers for both the fluid and kinetic component, we hope this approach will prove effective in the future for developing SP timesteppers for the full hybrid model. We hope this will give us the opportunity to incorporate previously inaccessible kinetic effects into the highly effective, modern, finite-element MHD models.
    \end{abstract}
    
    
    \newpage
    \tableofcontents
    
    
    \newpage
    \pagenumbering{arabic}
    %\linenumbers\renewcommand\thelinenumber{\color{black!50}\arabic{linenumber}}
            \documentclass[12pt, a4paper]{report}

\input{template/main.tex}

\title{\BA{Title in Progress...}}
\author{Boris Andrews}
\affil{Mathematical Institute, University of Oxford}
\date{\today}


\begin{document}
    \pagenumbering{gobble}
    \maketitle
    
    
    \begin{abstract}
        Magnetic confinement reactors---in particular tokamaks---offer one of the most promising options for achieving practical nuclear fusion, with the potential to provide virtually limitless, clean energy. The theoretical and numerical modeling of tokamak plasmas is simultaneously an essential component of effective reactor design, and a great research barrier. Tokamak operational conditions exhibit comparatively low Knudsen numbers. Kinetic effects, including kinetic waves and instabilities, Landau damping, bump-on-tail instabilities and more, are therefore highly influential in tokamak plasma dynamics. Purely fluid models are inherently incapable of capturing these effects, whereas the high dimensionality in purely kinetic models render them practically intractable for most relevant purposes.

        We consider a $\delta\!f$ decomposition model, with a macroscopic fluid background and microscopic kinetic correction, both fully coupled to each other. A similar manner of discretization is proposed to that used in the recent \texttt{STRUPHY} code \cite{Holderied_Possanner_Wang_2021, Holderied_2022, Li_et_al_2023} with a finite-element model for the background and a pseudo-particle/PiC model for the correction.

        The fluid background satisfies the full, non-linear, resistive, compressible, Hall MHD equations. \cite{Laakmann_Hu_Farrell_2022} introduces finite-element(-in-space) implicit timesteppers for the incompressible analogue to this system with structure-preserving (SP) properties in the ideal case, alongside parameter-robust preconditioners. We show that these timesteppers can derive from a finite-element-in-time (FET) (and finite-element-in-space) interpretation. The benefits of this reformulation are discussed, including the derivation of timesteppers that are higher order in time, and the quantifiable dissipative SP properties in the non-ideal, resistive case.
        
        We discuss possible options for extending this FET approach to timesteppers for the compressible case.

        The kinetic corrections satisfy linearized Boltzmann equations. Using a Lénard--Bernstein collision operator, these take Fokker--Planck-like forms \cite{Fokker_1914, Planck_1917} wherein pseudo-particles in the numerical model obey the neoclassical transport equations, with particle-independent Brownian drift terms. This offers a rigorous methodology for incorporating collisions into the particle transport model, without coupling the equations of motions for each particle.
        
        Works by Chen, Chacón et al. \cite{Chen_Chacón_Barnes_2011, Chacón_Chen_Barnes_2013, Chen_Chacón_2014, Chen_Chacón_2015} have developed structure-preserving particle pushers for neoclassical transport in the Vlasov equations, derived from Crank--Nicolson integrators. We show these too can can derive from a FET interpretation, similarly offering potential extensions to higher-order-in-time particle pushers. The FET formulation is used also to consider how the stochastic drift terms can be incorporated into the pushers. Stochastic gyrokinetic expansions are also discussed.

        Different options for the numerical implementation of these schemes are considered.

        Due to the efficacy of FET in the development of SP timesteppers for both the fluid and kinetic component, we hope this approach will prove effective in the future for developing SP timesteppers for the full hybrid model. We hope this will give us the opportunity to incorporate previously inaccessible kinetic effects into the highly effective, modern, finite-element MHD models.
    \end{abstract}
    
    
    \newpage
    \tableofcontents
    
    
    \newpage
    \pagenumbering{arabic}
    %\linenumbers\renewcommand\thelinenumber{\color{black!50}\arabic{linenumber}}
            \input{0 - introduction/main.tex}
        \part{Research}
            \input{1 - low-noise PiC models/main.tex}
            \input{2 - kinetic component/main.tex}
            \input{3 - fluid component/main.tex}
            \input{4 - numerical implementation/main.tex}
        \part{Project Overview}
            \input{5 - research plan/main.tex}
            \input{6 - summary/main.tex}
    
    
    %\section{}
    \newpage
    \pagenumbering{gobble}
        \printbibliography


    \newpage
    \pagenumbering{roman}
    \appendix
        \part{Appendices}
            \input{8 - Hilbert complexes/main.tex}
            \input{9 - weak conservation proofs/main.tex}
\end{document}

        \part{Research}
            \documentclass[12pt, a4paper]{report}

\input{template/main.tex}

\title{\BA{Title in Progress...}}
\author{Boris Andrews}
\affil{Mathematical Institute, University of Oxford}
\date{\today}


\begin{document}
    \pagenumbering{gobble}
    \maketitle
    
    
    \begin{abstract}
        Magnetic confinement reactors---in particular tokamaks---offer one of the most promising options for achieving practical nuclear fusion, with the potential to provide virtually limitless, clean energy. The theoretical and numerical modeling of tokamak plasmas is simultaneously an essential component of effective reactor design, and a great research barrier. Tokamak operational conditions exhibit comparatively low Knudsen numbers. Kinetic effects, including kinetic waves and instabilities, Landau damping, bump-on-tail instabilities and more, are therefore highly influential in tokamak plasma dynamics. Purely fluid models are inherently incapable of capturing these effects, whereas the high dimensionality in purely kinetic models render them practically intractable for most relevant purposes.

        We consider a $\delta\!f$ decomposition model, with a macroscopic fluid background and microscopic kinetic correction, both fully coupled to each other. A similar manner of discretization is proposed to that used in the recent \texttt{STRUPHY} code \cite{Holderied_Possanner_Wang_2021, Holderied_2022, Li_et_al_2023} with a finite-element model for the background and a pseudo-particle/PiC model for the correction.

        The fluid background satisfies the full, non-linear, resistive, compressible, Hall MHD equations. \cite{Laakmann_Hu_Farrell_2022} introduces finite-element(-in-space) implicit timesteppers for the incompressible analogue to this system with structure-preserving (SP) properties in the ideal case, alongside parameter-robust preconditioners. We show that these timesteppers can derive from a finite-element-in-time (FET) (and finite-element-in-space) interpretation. The benefits of this reformulation are discussed, including the derivation of timesteppers that are higher order in time, and the quantifiable dissipative SP properties in the non-ideal, resistive case.
        
        We discuss possible options for extending this FET approach to timesteppers for the compressible case.

        The kinetic corrections satisfy linearized Boltzmann equations. Using a Lénard--Bernstein collision operator, these take Fokker--Planck-like forms \cite{Fokker_1914, Planck_1917} wherein pseudo-particles in the numerical model obey the neoclassical transport equations, with particle-independent Brownian drift terms. This offers a rigorous methodology for incorporating collisions into the particle transport model, without coupling the equations of motions for each particle.
        
        Works by Chen, Chacón et al. \cite{Chen_Chacón_Barnes_2011, Chacón_Chen_Barnes_2013, Chen_Chacón_2014, Chen_Chacón_2015} have developed structure-preserving particle pushers for neoclassical transport in the Vlasov equations, derived from Crank--Nicolson integrators. We show these too can can derive from a FET interpretation, similarly offering potential extensions to higher-order-in-time particle pushers. The FET formulation is used also to consider how the stochastic drift terms can be incorporated into the pushers. Stochastic gyrokinetic expansions are also discussed.

        Different options for the numerical implementation of these schemes are considered.

        Due to the efficacy of FET in the development of SP timesteppers for both the fluid and kinetic component, we hope this approach will prove effective in the future for developing SP timesteppers for the full hybrid model. We hope this will give us the opportunity to incorporate previously inaccessible kinetic effects into the highly effective, modern, finite-element MHD models.
    \end{abstract}
    
    
    \newpage
    \tableofcontents
    
    
    \newpage
    \pagenumbering{arabic}
    %\linenumbers\renewcommand\thelinenumber{\color{black!50}\arabic{linenumber}}
            \input{0 - introduction/main.tex}
        \part{Research}
            \input{1 - low-noise PiC models/main.tex}
            \input{2 - kinetic component/main.tex}
            \input{3 - fluid component/main.tex}
            \input{4 - numerical implementation/main.tex}
        \part{Project Overview}
            \input{5 - research plan/main.tex}
            \input{6 - summary/main.tex}
    
    
    %\section{}
    \newpage
    \pagenumbering{gobble}
        \printbibliography


    \newpage
    \pagenumbering{roman}
    \appendix
        \part{Appendices}
            \input{8 - Hilbert complexes/main.tex}
            \input{9 - weak conservation proofs/main.tex}
\end{document}

            \documentclass[12pt, a4paper]{report}

\input{template/main.tex}

\title{\BA{Title in Progress...}}
\author{Boris Andrews}
\affil{Mathematical Institute, University of Oxford}
\date{\today}


\begin{document}
    \pagenumbering{gobble}
    \maketitle
    
    
    \begin{abstract}
        Magnetic confinement reactors---in particular tokamaks---offer one of the most promising options for achieving practical nuclear fusion, with the potential to provide virtually limitless, clean energy. The theoretical and numerical modeling of tokamak plasmas is simultaneously an essential component of effective reactor design, and a great research barrier. Tokamak operational conditions exhibit comparatively low Knudsen numbers. Kinetic effects, including kinetic waves and instabilities, Landau damping, bump-on-tail instabilities and more, are therefore highly influential in tokamak plasma dynamics. Purely fluid models are inherently incapable of capturing these effects, whereas the high dimensionality in purely kinetic models render them practically intractable for most relevant purposes.

        We consider a $\delta\!f$ decomposition model, with a macroscopic fluid background and microscopic kinetic correction, both fully coupled to each other. A similar manner of discretization is proposed to that used in the recent \texttt{STRUPHY} code \cite{Holderied_Possanner_Wang_2021, Holderied_2022, Li_et_al_2023} with a finite-element model for the background and a pseudo-particle/PiC model for the correction.

        The fluid background satisfies the full, non-linear, resistive, compressible, Hall MHD equations. \cite{Laakmann_Hu_Farrell_2022} introduces finite-element(-in-space) implicit timesteppers for the incompressible analogue to this system with structure-preserving (SP) properties in the ideal case, alongside parameter-robust preconditioners. We show that these timesteppers can derive from a finite-element-in-time (FET) (and finite-element-in-space) interpretation. The benefits of this reformulation are discussed, including the derivation of timesteppers that are higher order in time, and the quantifiable dissipative SP properties in the non-ideal, resistive case.
        
        We discuss possible options for extending this FET approach to timesteppers for the compressible case.

        The kinetic corrections satisfy linearized Boltzmann equations. Using a Lénard--Bernstein collision operator, these take Fokker--Planck-like forms \cite{Fokker_1914, Planck_1917} wherein pseudo-particles in the numerical model obey the neoclassical transport equations, with particle-independent Brownian drift terms. This offers a rigorous methodology for incorporating collisions into the particle transport model, without coupling the equations of motions for each particle.
        
        Works by Chen, Chacón et al. \cite{Chen_Chacón_Barnes_2011, Chacón_Chen_Barnes_2013, Chen_Chacón_2014, Chen_Chacón_2015} have developed structure-preserving particle pushers for neoclassical transport in the Vlasov equations, derived from Crank--Nicolson integrators. We show these too can can derive from a FET interpretation, similarly offering potential extensions to higher-order-in-time particle pushers. The FET formulation is used also to consider how the stochastic drift terms can be incorporated into the pushers. Stochastic gyrokinetic expansions are also discussed.

        Different options for the numerical implementation of these schemes are considered.

        Due to the efficacy of FET in the development of SP timesteppers for both the fluid and kinetic component, we hope this approach will prove effective in the future for developing SP timesteppers for the full hybrid model. We hope this will give us the opportunity to incorporate previously inaccessible kinetic effects into the highly effective, modern, finite-element MHD models.
    \end{abstract}
    
    
    \newpage
    \tableofcontents
    
    
    \newpage
    \pagenumbering{arabic}
    %\linenumbers\renewcommand\thelinenumber{\color{black!50}\arabic{linenumber}}
            \input{0 - introduction/main.tex}
        \part{Research}
            \input{1 - low-noise PiC models/main.tex}
            \input{2 - kinetic component/main.tex}
            \input{3 - fluid component/main.tex}
            \input{4 - numerical implementation/main.tex}
        \part{Project Overview}
            \input{5 - research plan/main.tex}
            \input{6 - summary/main.tex}
    
    
    %\section{}
    \newpage
    \pagenumbering{gobble}
        \printbibliography


    \newpage
    \pagenumbering{roman}
    \appendix
        \part{Appendices}
            \input{8 - Hilbert complexes/main.tex}
            \input{9 - weak conservation proofs/main.tex}
\end{document}

            \documentclass[12pt, a4paper]{report}

\input{template/main.tex}

\title{\BA{Title in Progress...}}
\author{Boris Andrews}
\affil{Mathematical Institute, University of Oxford}
\date{\today}


\begin{document}
    \pagenumbering{gobble}
    \maketitle
    
    
    \begin{abstract}
        Magnetic confinement reactors---in particular tokamaks---offer one of the most promising options for achieving practical nuclear fusion, with the potential to provide virtually limitless, clean energy. The theoretical and numerical modeling of tokamak plasmas is simultaneously an essential component of effective reactor design, and a great research barrier. Tokamak operational conditions exhibit comparatively low Knudsen numbers. Kinetic effects, including kinetic waves and instabilities, Landau damping, bump-on-tail instabilities and more, are therefore highly influential in tokamak plasma dynamics. Purely fluid models are inherently incapable of capturing these effects, whereas the high dimensionality in purely kinetic models render them practically intractable for most relevant purposes.

        We consider a $\delta\!f$ decomposition model, with a macroscopic fluid background and microscopic kinetic correction, both fully coupled to each other. A similar manner of discretization is proposed to that used in the recent \texttt{STRUPHY} code \cite{Holderied_Possanner_Wang_2021, Holderied_2022, Li_et_al_2023} with a finite-element model for the background and a pseudo-particle/PiC model for the correction.

        The fluid background satisfies the full, non-linear, resistive, compressible, Hall MHD equations. \cite{Laakmann_Hu_Farrell_2022} introduces finite-element(-in-space) implicit timesteppers for the incompressible analogue to this system with structure-preserving (SP) properties in the ideal case, alongside parameter-robust preconditioners. We show that these timesteppers can derive from a finite-element-in-time (FET) (and finite-element-in-space) interpretation. The benefits of this reformulation are discussed, including the derivation of timesteppers that are higher order in time, and the quantifiable dissipative SP properties in the non-ideal, resistive case.
        
        We discuss possible options for extending this FET approach to timesteppers for the compressible case.

        The kinetic corrections satisfy linearized Boltzmann equations. Using a Lénard--Bernstein collision operator, these take Fokker--Planck-like forms \cite{Fokker_1914, Planck_1917} wherein pseudo-particles in the numerical model obey the neoclassical transport equations, with particle-independent Brownian drift terms. This offers a rigorous methodology for incorporating collisions into the particle transport model, without coupling the equations of motions for each particle.
        
        Works by Chen, Chacón et al. \cite{Chen_Chacón_Barnes_2011, Chacón_Chen_Barnes_2013, Chen_Chacón_2014, Chen_Chacón_2015} have developed structure-preserving particle pushers for neoclassical transport in the Vlasov equations, derived from Crank--Nicolson integrators. We show these too can can derive from a FET interpretation, similarly offering potential extensions to higher-order-in-time particle pushers. The FET formulation is used also to consider how the stochastic drift terms can be incorporated into the pushers. Stochastic gyrokinetic expansions are also discussed.

        Different options for the numerical implementation of these schemes are considered.

        Due to the efficacy of FET in the development of SP timesteppers for both the fluid and kinetic component, we hope this approach will prove effective in the future for developing SP timesteppers for the full hybrid model. We hope this will give us the opportunity to incorporate previously inaccessible kinetic effects into the highly effective, modern, finite-element MHD models.
    \end{abstract}
    
    
    \newpage
    \tableofcontents
    
    
    \newpage
    \pagenumbering{arabic}
    %\linenumbers\renewcommand\thelinenumber{\color{black!50}\arabic{linenumber}}
            \input{0 - introduction/main.tex}
        \part{Research}
            \input{1 - low-noise PiC models/main.tex}
            \input{2 - kinetic component/main.tex}
            \input{3 - fluid component/main.tex}
            \input{4 - numerical implementation/main.tex}
        \part{Project Overview}
            \input{5 - research plan/main.tex}
            \input{6 - summary/main.tex}
    
    
    %\section{}
    \newpage
    \pagenumbering{gobble}
        \printbibliography


    \newpage
    \pagenumbering{roman}
    \appendix
        \part{Appendices}
            \input{8 - Hilbert complexes/main.tex}
            \input{9 - weak conservation proofs/main.tex}
\end{document}

            \documentclass[12pt, a4paper]{report}

\input{template/main.tex}

\title{\BA{Title in Progress...}}
\author{Boris Andrews}
\affil{Mathematical Institute, University of Oxford}
\date{\today}


\begin{document}
    \pagenumbering{gobble}
    \maketitle
    
    
    \begin{abstract}
        Magnetic confinement reactors---in particular tokamaks---offer one of the most promising options for achieving practical nuclear fusion, with the potential to provide virtually limitless, clean energy. The theoretical and numerical modeling of tokamak plasmas is simultaneously an essential component of effective reactor design, and a great research barrier. Tokamak operational conditions exhibit comparatively low Knudsen numbers. Kinetic effects, including kinetic waves and instabilities, Landau damping, bump-on-tail instabilities and more, are therefore highly influential in tokamak plasma dynamics. Purely fluid models are inherently incapable of capturing these effects, whereas the high dimensionality in purely kinetic models render them practically intractable for most relevant purposes.

        We consider a $\delta\!f$ decomposition model, with a macroscopic fluid background and microscopic kinetic correction, both fully coupled to each other. A similar manner of discretization is proposed to that used in the recent \texttt{STRUPHY} code \cite{Holderied_Possanner_Wang_2021, Holderied_2022, Li_et_al_2023} with a finite-element model for the background and a pseudo-particle/PiC model for the correction.

        The fluid background satisfies the full, non-linear, resistive, compressible, Hall MHD equations. \cite{Laakmann_Hu_Farrell_2022} introduces finite-element(-in-space) implicit timesteppers for the incompressible analogue to this system with structure-preserving (SP) properties in the ideal case, alongside parameter-robust preconditioners. We show that these timesteppers can derive from a finite-element-in-time (FET) (and finite-element-in-space) interpretation. The benefits of this reformulation are discussed, including the derivation of timesteppers that are higher order in time, and the quantifiable dissipative SP properties in the non-ideal, resistive case.
        
        We discuss possible options for extending this FET approach to timesteppers for the compressible case.

        The kinetic corrections satisfy linearized Boltzmann equations. Using a Lénard--Bernstein collision operator, these take Fokker--Planck-like forms \cite{Fokker_1914, Planck_1917} wherein pseudo-particles in the numerical model obey the neoclassical transport equations, with particle-independent Brownian drift terms. This offers a rigorous methodology for incorporating collisions into the particle transport model, without coupling the equations of motions for each particle.
        
        Works by Chen, Chacón et al. \cite{Chen_Chacón_Barnes_2011, Chacón_Chen_Barnes_2013, Chen_Chacón_2014, Chen_Chacón_2015} have developed structure-preserving particle pushers for neoclassical transport in the Vlasov equations, derived from Crank--Nicolson integrators. We show these too can can derive from a FET interpretation, similarly offering potential extensions to higher-order-in-time particle pushers. The FET formulation is used also to consider how the stochastic drift terms can be incorporated into the pushers. Stochastic gyrokinetic expansions are also discussed.

        Different options for the numerical implementation of these schemes are considered.

        Due to the efficacy of FET in the development of SP timesteppers for both the fluid and kinetic component, we hope this approach will prove effective in the future for developing SP timesteppers for the full hybrid model. We hope this will give us the opportunity to incorporate previously inaccessible kinetic effects into the highly effective, modern, finite-element MHD models.
    \end{abstract}
    
    
    \newpage
    \tableofcontents
    
    
    \newpage
    \pagenumbering{arabic}
    %\linenumbers\renewcommand\thelinenumber{\color{black!50}\arabic{linenumber}}
            \input{0 - introduction/main.tex}
        \part{Research}
            \input{1 - low-noise PiC models/main.tex}
            \input{2 - kinetic component/main.tex}
            \input{3 - fluid component/main.tex}
            \input{4 - numerical implementation/main.tex}
        \part{Project Overview}
            \input{5 - research plan/main.tex}
            \input{6 - summary/main.tex}
    
    
    %\section{}
    \newpage
    \pagenumbering{gobble}
        \printbibliography


    \newpage
    \pagenumbering{roman}
    \appendix
        \part{Appendices}
            \input{8 - Hilbert complexes/main.tex}
            \input{9 - weak conservation proofs/main.tex}
\end{document}

        \part{Project Overview}
            \documentclass[12pt, a4paper]{report}

\input{template/main.tex}

\title{\BA{Title in Progress...}}
\author{Boris Andrews}
\affil{Mathematical Institute, University of Oxford}
\date{\today}


\begin{document}
    \pagenumbering{gobble}
    \maketitle
    
    
    \begin{abstract}
        Magnetic confinement reactors---in particular tokamaks---offer one of the most promising options for achieving practical nuclear fusion, with the potential to provide virtually limitless, clean energy. The theoretical and numerical modeling of tokamak plasmas is simultaneously an essential component of effective reactor design, and a great research barrier. Tokamak operational conditions exhibit comparatively low Knudsen numbers. Kinetic effects, including kinetic waves and instabilities, Landau damping, bump-on-tail instabilities and more, are therefore highly influential in tokamak plasma dynamics. Purely fluid models are inherently incapable of capturing these effects, whereas the high dimensionality in purely kinetic models render them practically intractable for most relevant purposes.

        We consider a $\delta\!f$ decomposition model, with a macroscopic fluid background and microscopic kinetic correction, both fully coupled to each other. A similar manner of discretization is proposed to that used in the recent \texttt{STRUPHY} code \cite{Holderied_Possanner_Wang_2021, Holderied_2022, Li_et_al_2023} with a finite-element model for the background and a pseudo-particle/PiC model for the correction.

        The fluid background satisfies the full, non-linear, resistive, compressible, Hall MHD equations. \cite{Laakmann_Hu_Farrell_2022} introduces finite-element(-in-space) implicit timesteppers for the incompressible analogue to this system with structure-preserving (SP) properties in the ideal case, alongside parameter-robust preconditioners. We show that these timesteppers can derive from a finite-element-in-time (FET) (and finite-element-in-space) interpretation. The benefits of this reformulation are discussed, including the derivation of timesteppers that are higher order in time, and the quantifiable dissipative SP properties in the non-ideal, resistive case.
        
        We discuss possible options for extending this FET approach to timesteppers for the compressible case.

        The kinetic corrections satisfy linearized Boltzmann equations. Using a Lénard--Bernstein collision operator, these take Fokker--Planck-like forms \cite{Fokker_1914, Planck_1917} wherein pseudo-particles in the numerical model obey the neoclassical transport equations, with particle-independent Brownian drift terms. This offers a rigorous methodology for incorporating collisions into the particle transport model, without coupling the equations of motions for each particle.
        
        Works by Chen, Chacón et al. \cite{Chen_Chacón_Barnes_2011, Chacón_Chen_Barnes_2013, Chen_Chacón_2014, Chen_Chacón_2015} have developed structure-preserving particle pushers for neoclassical transport in the Vlasov equations, derived from Crank--Nicolson integrators. We show these too can can derive from a FET interpretation, similarly offering potential extensions to higher-order-in-time particle pushers. The FET formulation is used also to consider how the stochastic drift terms can be incorporated into the pushers. Stochastic gyrokinetic expansions are also discussed.

        Different options for the numerical implementation of these schemes are considered.

        Due to the efficacy of FET in the development of SP timesteppers for both the fluid and kinetic component, we hope this approach will prove effective in the future for developing SP timesteppers for the full hybrid model. We hope this will give us the opportunity to incorporate previously inaccessible kinetic effects into the highly effective, modern, finite-element MHD models.
    \end{abstract}
    
    
    \newpage
    \tableofcontents
    
    
    \newpage
    \pagenumbering{arabic}
    %\linenumbers\renewcommand\thelinenumber{\color{black!50}\arabic{linenumber}}
            \input{0 - introduction/main.tex}
        \part{Research}
            \input{1 - low-noise PiC models/main.tex}
            \input{2 - kinetic component/main.tex}
            \input{3 - fluid component/main.tex}
            \input{4 - numerical implementation/main.tex}
        \part{Project Overview}
            \input{5 - research plan/main.tex}
            \input{6 - summary/main.tex}
    
    
    %\section{}
    \newpage
    \pagenumbering{gobble}
        \printbibliography


    \newpage
    \pagenumbering{roman}
    \appendix
        \part{Appendices}
            \input{8 - Hilbert complexes/main.tex}
            \input{9 - weak conservation proofs/main.tex}
\end{document}

            \documentclass[12pt, a4paper]{report}

\input{template/main.tex}

\title{\BA{Title in Progress...}}
\author{Boris Andrews}
\affil{Mathematical Institute, University of Oxford}
\date{\today}


\begin{document}
    \pagenumbering{gobble}
    \maketitle
    
    
    \begin{abstract}
        Magnetic confinement reactors---in particular tokamaks---offer one of the most promising options for achieving practical nuclear fusion, with the potential to provide virtually limitless, clean energy. The theoretical and numerical modeling of tokamak plasmas is simultaneously an essential component of effective reactor design, and a great research barrier. Tokamak operational conditions exhibit comparatively low Knudsen numbers. Kinetic effects, including kinetic waves and instabilities, Landau damping, bump-on-tail instabilities and more, are therefore highly influential in tokamak plasma dynamics. Purely fluid models are inherently incapable of capturing these effects, whereas the high dimensionality in purely kinetic models render them practically intractable for most relevant purposes.

        We consider a $\delta\!f$ decomposition model, with a macroscopic fluid background and microscopic kinetic correction, both fully coupled to each other. A similar manner of discretization is proposed to that used in the recent \texttt{STRUPHY} code \cite{Holderied_Possanner_Wang_2021, Holderied_2022, Li_et_al_2023} with a finite-element model for the background and a pseudo-particle/PiC model for the correction.

        The fluid background satisfies the full, non-linear, resistive, compressible, Hall MHD equations. \cite{Laakmann_Hu_Farrell_2022} introduces finite-element(-in-space) implicit timesteppers for the incompressible analogue to this system with structure-preserving (SP) properties in the ideal case, alongside parameter-robust preconditioners. We show that these timesteppers can derive from a finite-element-in-time (FET) (and finite-element-in-space) interpretation. The benefits of this reformulation are discussed, including the derivation of timesteppers that are higher order in time, and the quantifiable dissipative SP properties in the non-ideal, resistive case.
        
        We discuss possible options for extending this FET approach to timesteppers for the compressible case.

        The kinetic corrections satisfy linearized Boltzmann equations. Using a Lénard--Bernstein collision operator, these take Fokker--Planck-like forms \cite{Fokker_1914, Planck_1917} wherein pseudo-particles in the numerical model obey the neoclassical transport equations, with particle-independent Brownian drift terms. This offers a rigorous methodology for incorporating collisions into the particle transport model, without coupling the equations of motions for each particle.
        
        Works by Chen, Chacón et al. \cite{Chen_Chacón_Barnes_2011, Chacón_Chen_Barnes_2013, Chen_Chacón_2014, Chen_Chacón_2015} have developed structure-preserving particle pushers for neoclassical transport in the Vlasov equations, derived from Crank--Nicolson integrators. We show these too can can derive from a FET interpretation, similarly offering potential extensions to higher-order-in-time particle pushers. The FET formulation is used also to consider how the stochastic drift terms can be incorporated into the pushers. Stochastic gyrokinetic expansions are also discussed.

        Different options for the numerical implementation of these schemes are considered.

        Due to the efficacy of FET in the development of SP timesteppers for both the fluid and kinetic component, we hope this approach will prove effective in the future for developing SP timesteppers for the full hybrid model. We hope this will give us the opportunity to incorporate previously inaccessible kinetic effects into the highly effective, modern, finite-element MHD models.
    \end{abstract}
    
    
    \newpage
    \tableofcontents
    
    
    \newpage
    \pagenumbering{arabic}
    %\linenumbers\renewcommand\thelinenumber{\color{black!50}\arabic{linenumber}}
            \input{0 - introduction/main.tex}
        \part{Research}
            \input{1 - low-noise PiC models/main.tex}
            \input{2 - kinetic component/main.tex}
            \input{3 - fluid component/main.tex}
            \input{4 - numerical implementation/main.tex}
        \part{Project Overview}
            \input{5 - research plan/main.tex}
            \input{6 - summary/main.tex}
    
    
    %\section{}
    \newpage
    \pagenumbering{gobble}
        \printbibliography


    \newpage
    \pagenumbering{roman}
    \appendix
        \part{Appendices}
            \input{8 - Hilbert complexes/main.tex}
            \input{9 - weak conservation proofs/main.tex}
\end{document}

    
    
    %\section{}
    \newpage
    \pagenumbering{gobble}
        \printbibliography


    \newpage
    \pagenumbering{roman}
    \appendix
        \part{Appendices}
            \documentclass[12pt, a4paper]{report}

\input{template/main.tex}

\title{\BA{Title in Progress...}}
\author{Boris Andrews}
\affil{Mathematical Institute, University of Oxford}
\date{\today}


\begin{document}
    \pagenumbering{gobble}
    \maketitle
    
    
    \begin{abstract}
        Magnetic confinement reactors---in particular tokamaks---offer one of the most promising options for achieving practical nuclear fusion, with the potential to provide virtually limitless, clean energy. The theoretical and numerical modeling of tokamak plasmas is simultaneously an essential component of effective reactor design, and a great research barrier. Tokamak operational conditions exhibit comparatively low Knudsen numbers. Kinetic effects, including kinetic waves and instabilities, Landau damping, bump-on-tail instabilities and more, are therefore highly influential in tokamak plasma dynamics. Purely fluid models are inherently incapable of capturing these effects, whereas the high dimensionality in purely kinetic models render them practically intractable for most relevant purposes.

        We consider a $\delta\!f$ decomposition model, with a macroscopic fluid background and microscopic kinetic correction, both fully coupled to each other. A similar manner of discretization is proposed to that used in the recent \texttt{STRUPHY} code \cite{Holderied_Possanner_Wang_2021, Holderied_2022, Li_et_al_2023} with a finite-element model for the background and a pseudo-particle/PiC model for the correction.

        The fluid background satisfies the full, non-linear, resistive, compressible, Hall MHD equations. \cite{Laakmann_Hu_Farrell_2022} introduces finite-element(-in-space) implicit timesteppers for the incompressible analogue to this system with structure-preserving (SP) properties in the ideal case, alongside parameter-robust preconditioners. We show that these timesteppers can derive from a finite-element-in-time (FET) (and finite-element-in-space) interpretation. The benefits of this reformulation are discussed, including the derivation of timesteppers that are higher order in time, and the quantifiable dissipative SP properties in the non-ideal, resistive case.
        
        We discuss possible options for extending this FET approach to timesteppers for the compressible case.

        The kinetic corrections satisfy linearized Boltzmann equations. Using a Lénard--Bernstein collision operator, these take Fokker--Planck-like forms \cite{Fokker_1914, Planck_1917} wherein pseudo-particles in the numerical model obey the neoclassical transport equations, with particle-independent Brownian drift terms. This offers a rigorous methodology for incorporating collisions into the particle transport model, without coupling the equations of motions for each particle.
        
        Works by Chen, Chacón et al. \cite{Chen_Chacón_Barnes_2011, Chacón_Chen_Barnes_2013, Chen_Chacón_2014, Chen_Chacón_2015} have developed structure-preserving particle pushers for neoclassical transport in the Vlasov equations, derived from Crank--Nicolson integrators. We show these too can can derive from a FET interpretation, similarly offering potential extensions to higher-order-in-time particle pushers. The FET formulation is used also to consider how the stochastic drift terms can be incorporated into the pushers. Stochastic gyrokinetic expansions are also discussed.

        Different options for the numerical implementation of these schemes are considered.

        Due to the efficacy of FET in the development of SP timesteppers for both the fluid and kinetic component, we hope this approach will prove effective in the future for developing SP timesteppers for the full hybrid model. We hope this will give us the opportunity to incorporate previously inaccessible kinetic effects into the highly effective, modern, finite-element MHD models.
    \end{abstract}
    
    
    \newpage
    \tableofcontents
    
    
    \newpage
    \pagenumbering{arabic}
    %\linenumbers\renewcommand\thelinenumber{\color{black!50}\arabic{linenumber}}
            \input{0 - introduction/main.tex}
        \part{Research}
            \input{1 - low-noise PiC models/main.tex}
            \input{2 - kinetic component/main.tex}
            \input{3 - fluid component/main.tex}
            \input{4 - numerical implementation/main.tex}
        \part{Project Overview}
            \input{5 - research plan/main.tex}
            \input{6 - summary/main.tex}
    
    
    %\section{}
    \newpage
    \pagenumbering{gobble}
        \printbibliography


    \newpage
    \pagenumbering{roman}
    \appendix
        \part{Appendices}
            \input{8 - Hilbert complexes/main.tex}
            \input{9 - weak conservation proofs/main.tex}
\end{document}

            \documentclass[12pt, a4paper]{report}

\input{template/main.tex}

\title{\BA{Title in Progress...}}
\author{Boris Andrews}
\affil{Mathematical Institute, University of Oxford}
\date{\today}


\begin{document}
    \pagenumbering{gobble}
    \maketitle
    
    
    \begin{abstract}
        Magnetic confinement reactors---in particular tokamaks---offer one of the most promising options for achieving practical nuclear fusion, with the potential to provide virtually limitless, clean energy. The theoretical and numerical modeling of tokamak plasmas is simultaneously an essential component of effective reactor design, and a great research barrier. Tokamak operational conditions exhibit comparatively low Knudsen numbers. Kinetic effects, including kinetic waves and instabilities, Landau damping, bump-on-tail instabilities and more, are therefore highly influential in tokamak plasma dynamics. Purely fluid models are inherently incapable of capturing these effects, whereas the high dimensionality in purely kinetic models render them practically intractable for most relevant purposes.

        We consider a $\delta\!f$ decomposition model, with a macroscopic fluid background and microscopic kinetic correction, both fully coupled to each other. A similar manner of discretization is proposed to that used in the recent \texttt{STRUPHY} code \cite{Holderied_Possanner_Wang_2021, Holderied_2022, Li_et_al_2023} with a finite-element model for the background and a pseudo-particle/PiC model for the correction.

        The fluid background satisfies the full, non-linear, resistive, compressible, Hall MHD equations. \cite{Laakmann_Hu_Farrell_2022} introduces finite-element(-in-space) implicit timesteppers for the incompressible analogue to this system with structure-preserving (SP) properties in the ideal case, alongside parameter-robust preconditioners. We show that these timesteppers can derive from a finite-element-in-time (FET) (and finite-element-in-space) interpretation. The benefits of this reformulation are discussed, including the derivation of timesteppers that are higher order in time, and the quantifiable dissipative SP properties in the non-ideal, resistive case.
        
        We discuss possible options for extending this FET approach to timesteppers for the compressible case.

        The kinetic corrections satisfy linearized Boltzmann equations. Using a Lénard--Bernstein collision operator, these take Fokker--Planck-like forms \cite{Fokker_1914, Planck_1917} wherein pseudo-particles in the numerical model obey the neoclassical transport equations, with particle-independent Brownian drift terms. This offers a rigorous methodology for incorporating collisions into the particle transport model, without coupling the equations of motions for each particle.
        
        Works by Chen, Chacón et al. \cite{Chen_Chacón_Barnes_2011, Chacón_Chen_Barnes_2013, Chen_Chacón_2014, Chen_Chacón_2015} have developed structure-preserving particle pushers for neoclassical transport in the Vlasov equations, derived from Crank--Nicolson integrators. We show these too can can derive from a FET interpretation, similarly offering potential extensions to higher-order-in-time particle pushers. The FET formulation is used also to consider how the stochastic drift terms can be incorporated into the pushers. Stochastic gyrokinetic expansions are also discussed.

        Different options for the numerical implementation of these schemes are considered.

        Due to the efficacy of FET in the development of SP timesteppers for both the fluid and kinetic component, we hope this approach will prove effective in the future for developing SP timesteppers for the full hybrid model. We hope this will give us the opportunity to incorporate previously inaccessible kinetic effects into the highly effective, modern, finite-element MHD models.
    \end{abstract}
    
    
    \newpage
    \tableofcontents
    
    
    \newpage
    \pagenumbering{arabic}
    %\linenumbers\renewcommand\thelinenumber{\color{black!50}\arabic{linenumber}}
            \input{0 - introduction/main.tex}
        \part{Research}
            \input{1 - low-noise PiC models/main.tex}
            \input{2 - kinetic component/main.tex}
            \input{3 - fluid component/main.tex}
            \input{4 - numerical implementation/main.tex}
        \part{Project Overview}
            \input{5 - research plan/main.tex}
            \input{6 - summary/main.tex}
    
    
    %\section{}
    \newpage
    \pagenumbering{gobble}
        \printbibliography


    \newpage
    \pagenumbering{roman}
    \appendix
        \part{Appendices}
            \input{8 - Hilbert complexes/main.tex}
            \input{9 - weak conservation proofs/main.tex}
\end{document}

\end{document}

            \documentclass[12pt, a4paper]{report}

\documentclass[12pt, a4paper]{report}

\input{template/main.tex}

\title{\BA{Title in Progress...}}
\author{Boris Andrews}
\affil{Mathematical Institute, University of Oxford}
\date{\today}


\begin{document}
    \pagenumbering{gobble}
    \maketitle
    
    
    \begin{abstract}
        Magnetic confinement reactors---in particular tokamaks---offer one of the most promising options for achieving practical nuclear fusion, with the potential to provide virtually limitless, clean energy. The theoretical and numerical modeling of tokamak plasmas is simultaneously an essential component of effective reactor design, and a great research barrier. Tokamak operational conditions exhibit comparatively low Knudsen numbers. Kinetic effects, including kinetic waves and instabilities, Landau damping, bump-on-tail instabilities and more, are therefore highly influential in tokamak plasma dynamics. Purely fluid models are inherently incapable of capturing these effects, whereas the high dimensionality in purely kinetic models render them practically intractable for most relevant purposes.

        We consider a $\delta\!f$ decomposition model, with a macroscopic fluid background and microscopic kinetic correction, both fully coupled to each other. A similar manner of discretization is proposed to that used in the recent \texttt{STRUPHY} code \cite{Holderied_Possanner_Wang_2021, Holderied_2022, Li_et_al_2023} with a finite-element model for the background and a pseudo-particle/PiC model for the correction.

        The fluid background satisfies the full, non-linear, resistive, compressible, Hall MHD equations. \cite{Laakmann_Hu_Farrell_2022} introduces finite-element(-in-space) implicit timesteppers for the incompressible analogue to this system with structure-preserving (SP) properties in the ideal case, alongside parameter-robust preconditioners. We show that these timesteppers can derive from a finite-element-in-time (FET) (and finite-element-in-space) interpretation. The benefits of this reformulation are discussed, including the derivation of timesteppers that are higher order in time, and the quantifiable dissipative SP properties in the non-ideal, resistive case.
        
        We discuss possible options for extending this FET approach to timesteppers for the compressible case.

        The kinetic corrections satisfy linearized Boltzmann equations. Using a Lénard--Bernstein collision operator, these take Fokker--Planck-like forms \cite{Fokker_1914, Planck_1917} wherein pseudo-particles in the numerical model obey the neoclassical transport equations, with particle-independent Brownian drift terms. This offers a rigorous methodology for incorporating collisions into the particle transport model, without coupling the equations of motions for each particle.
        
        Works by Chen, Chacón et al. \cite{Chen_Chacón_Barnes_2011, Chacón_Chen_Barnes_2013, Chen_Chacón_2014, Chen_Chacón_2015} have developed structure-preserving particle pushers for neoclassical transport in the Vlasov equations, derived from Crank--Nicolson integrators. We show these too can can derive from a FET interpretation, similarly offering potential extensions to higher-order-in-time particle pushers. The FET formulation is used also to consider how the stochastic drift terms can be incorporated into the pushers. Stochastic gyrokinetic expansions are also discussed.

        Different options for the numerical implementation of these schemes are considered.

        Due to the efficacy of FET in the development of SP timesteppers for both the fluid and kinetic component, we hope this approach will prove effective in the future for developing SP timesteppers for the full hybrid model. We hope this will give us the opportunity to incorporate previously inaccessible kinetic effects into the highly effective, modern, finite-element MHD models.
    \end{abstract}
    
    
    \newpage
    \tableofcontents
    
    
    \newpage
    \pagenumbering{arabic}
    %\linenumbers\renewcommand\thelinenumber{\color{black!50}\arabic{linenumber}}
            \input{0 - introduction/main.tex}
        \part{Research}
            \input{1 - low-noise PiC models/main.tex}
            \input{2 - kinetic component/main.tex}
            \input{3 - fluid component/main.tex}
            \input{4 - numerical implementation/main.tex}
        \part{Project Overview}
            \input{5 - research plan/main.tex}
            \input{6 - summary/main.tex}
    
    
    %\section{}
    \newpage
    \pagenumbering{gobble}
        \printbibliography


    \newpage
    \pagenumbering{roman}
    \appendix
        \part{Appendices}
            \input{8 - Hilbert complexes/main.tex}
            \input{9 - weak conservation proofs/main.tex}
\end{document}


\title{\BA{Title in Progress...}}
\author{Boris Andrews}
\affil{Mathematical Institute, University of Oxford}
\date{\today}


\begin{document}
    \pagenumbering{gobble}
    \maketitle
    
    
    \begin{abstract}
        Magnetic confinement reactors---in particular tokamaks---offer one of the most promising options for achieving practical nuclear fusion, with the potential to provide virtually limitless, clean energy. The theoretical and numerical modeling of tokamak plasmas is simultaneously an essential component of effective reactor design, and a great research barrier. Tokamak operational conditions exhibit comparatively low Knudsen numbers. Kinetic effects, including kinetic waves and instabilities, Landau damping, bump-on-tail instabilities and more, are therefore highly influential in tokamak plasma dynamics. Purely fluid models are inherently incapable of capturing these effects, whereas the high dimensionality in purely kinetic models render them practically intractable for most relevant purposes.

        We consider a $\delta\!f$ decomposition model, with a macroscopic fluid background and microscopic kinetic correction, both fully coupled to each other. A similar manner of discretization is proposed to that used in the recent \texttt{STRUPHY} code \cite{Holderied_Possanner_Wang_2021, Holderied_2022, Li_et_al_2023} with a finite-element model for the background and a pseudo-particle/PiC model for the correction.

        The fluid background satisfies the full, non-linear, resistive, compressible, Hall MHD equations. \cite{Laakmann_Hu_Farrell_2022} introduces finite-element(-in-space) implicit timesteppers for the incompressible analogue to this system with structure-preserving (SP) properties in the ideal case, alongside parameter-robust preconditioners. We show that these timesteppers can derive from a finite-element-in-time (FET) (and finite-element-in-space) interpretation. The benefits of this reformulation are discussed, including the derivation of timesteppers that are higher order in time, and the quantifiable dissipative SP properties in the non-ideal, resistive case.
        
        We discuss possible options for extending this FET approach to timesteppers for the compressible case.

        The kinetic corrections satisfy linearized Boltzmann equations. Using a Lénard--Bernstein collision operator, these take Fokker--Planck-like forms \cite{Fokker_1914, Planck_1917} wherein pseudo-particles in the numerical model obey the neoclassical transport equations, with particle-independent Brownian drift terms. This offers a rigorous methodology for incorporating collisions into the particle transport model, without coupling the equations of motions for each particle.
        
        Works by Chen, Chacón et al. \cite{Chen_Chacón_Barnes_2011, Chacón_Chen_Barnes_2013, Chen_Chacón_2014, Chen_Chacón_2015} have developed structure-preserving particle pushers for neoclassical transport in the Vlasov equations, derived from Crank--Nicolson integrators. We show these too can can derive from a FET interpretation, similarly offering potential extensions to higher-order-in-time particle pushers. The FET formulation is used also to consider how the stochastic drift terms can be incorporated into the pushers. Stochastic gyrokinetic expansions are also discussed.

        Different options for the numerical implementation of these schemes are considered.

        Due to the efficacy of FET in the development of SP timesteppers for both the fluid and kinetic component, we hope this approach will prove effective in the future for developing SP timesteppers for the full hybrid model. We hope this will give us the opportunity to incorporate previously inaccessible kinetic effects into the highly effective, modern, finite-element MHD models.
    \end{abstract}
    
    
    \newpage
    \tableofcontents
    
    
    \newpage
    \pagenumbering{arabic}
    %\linenumbers\renewcommand\thelinenumber{\color{black!50}\arabic{linenumber}}
            \documentclass[12pt, a4paper]{report}

\input{template/main.tex}

\title{\BA{Title in Progress...}}
\author{Boris Andrews}
\affil{Mathematical Institute, University of Oxford}
\date{\today}


\begin{document}
    \pagenumbering{gobble}
    \maketitle
    
    
    \begin{abstract}
        Magnetic confinement reactors---in particular tokamaks---offer one of the most promising options for achieving practical nuclear fusion, with the potential to provide virtually limitless, clean energy. The theoretical and numerical modeling of tokamak plasmas is simultaneously an essential component of effective reactor design, and a great research barrier. Tokamak operational conditions exhibit comparatively low Knudsen numbers. Kinetic effects, including kinetic waves and instabilities, Landau damping, bump-on-tail instabilities and more, are therefore highly influential in tokamak plasma dynamics. Purely fluid models are inherently incapable of capturing these effects, whereas the high dimensionality in purely kinetic models render them practically intractable for most relevant purposes.

        We consider a $\delta\!f$ decomposition model, with a macroscopic fluid background and microscopic kinetic correction, both fully coupled to each other. A similar manner of discretization is proposed to that used in the recent \texttt{STRUPHY} code \cite{Holderied_Possanner_Wang_2021, Holderied_2022, Li_et_al_2023} with a finite-element model for the background and a pseudo-particle/PiC model for the correction.

        The fluid background satisfies the full, non-linear, resistive, compressible, Hall MHD equations. \cite{Laakmann_Hu_Farrell_2022} introduces finite-element(-in-space) implicit timesteppers for the incompressible analogue to this system with structure-preserving (SP) properties in the ideal case, alongside parameter-robust preconditioners. We show that these timesteppers can derive from a finite-element-in-time (FET) (and finite-element-in-space) interpretation. The benefits of this reformulation are discussed, including the derivation of timesteppers that are higher order in time, and the quantifiable dissipative SP properties in the non-ideal, resistive case.
        
        We discuss possible options for extending this FET approach to timesteppers for the compressible case.

        The kinetic corrections satisfy linearized Boltzmann equations. Using a Lénard--Bernstein collision operator, these take Fokker--Planck-like forms \cite{Fokker_1914, Planck_1917} wherein pseudo-particles in the numerical model obey the neoclassical transport equations, with particle-independent Brownian drift terms. This offers a rigorous methodology for incorporating collisions into the particle transport model, without coupling the equations of motions for each particle.
        
        Works by Chen, Chacón et al. \cite{Chen_Chacón_Barnes_2011, Chacón_Chen_Barnes_2013, Chen_Chacón_2014, Chen_Chacón_2015} have developed structure-preserving particle pushers for neoclassical transport in the Vlasov equations, derived from Crank--Nicolson integrators. We show these too can can derive from a FET interpretation, similarly offering potential extensions to higher-order-in-time particle pushers. The FET formulation is used also to consider how the stochastic drift terms can be incorporated into the pushers. Stochastic gyrokinetic expansions are also discussed.

        Different options for the numerical implementation of these schemes are considered.

        Due to the efficacy of FET in the development of SP timesteppers for both the fluid and kinetic component, we hope this approach will prove effective in the future for developing SP timesteppers for the full hybrid model. We hope this will give us the opportunity to incorporate previously inaccessible kinetic effects into the highly effective, modern, finite-element MHD models.
    \end{abstract}
    
    
    \newpage
    \tableofcontents
    
    
    \newpage
    \pagenumbering{arabic}
    %\linenumbers\renewcommand\thelinenumber{\color{black!50}\arabic{linenumber}}
            \input{0 - introduction/main.tex}
        \part{Research}
            \input{1 - low-noise PiC models/main.tex}
            \input{2 - kinetic component/main.tex}
            \input{3 - fluid component/main.tex}
            \input{4 - numerical implementation/main.tex}
        \part{Project Overview}
            \input{5 - research plan/main.tex}
            \input{6 - summary/main.tex}
    
    
    %\section{}
    \newpage
    \pagenumbering{gobble}
        \printbibliography


    \newpage
    \pagenumbering{roman}
    \appendix
        \part{Appendices}
            \input{8 - Hilbert complexes/main.tex}
            \input{9 - weak conservation proofs/main.tex}
\end{document}

        \part{Research}
            \documentclass[12pt, a4paper]{report}

\input{template/main.tex}

\title{\BA{Title in Progress...}}
\author{Boris Andrews}
\affil{Mathematical Institute, University of Oxford}
\date{\today}


\begin{document}
    \pagenumbering{gobble}
    \maketitle
    
    
    \begin{abstract}
        Magnetic confinement reactors---in particular tokamaks---offer one of the most promising options for achieving practical nuclear fusion, with the potential to provide virtually limitless, clean energy. The theoretical and numerical modeling of tokamak plasmas is simultaneously an essential component of effective reactor design, and a great research barrier. Tokamak operational conditions exhibit comparatively low Knudsen numbers. Kinetic effects, including kinetic waves and instabilities, Landau damping, bump-on-tail instabilities and more, are therefore highly influential in tokamak plasma dynamics. Purely fluid models are inherently incapable of capturing these effects, whereas the high dimensionality in purely kinetic models render them practically intractable for most relevant purposes.

        We consider a $\delta\!f$ decomposition model, with a macroscopic fluid background and microscopic kinetic correction, both fully coupled to each other. A similar manner of discretization is proposed to that used in the recent \texttt{STRUPHY} code \cite{Holderied_Possanner_Wang_2021, Holderied_2022, Li_et_al_2023} with a finite-element model for the background and a pseudo-particle/PiC model for the correction.

        The fluid background satisfies the full, non-linear, resistive, compressible, Hall MHD equations. \cite{Laakmann_Hu_Farrell_2022} introduces finite-element(-in-space) implicit timesteppers for the incompressible analogue to this system with structure-preserving (SP) properties in the ideal case, alongside parameter-robust preconditioners. We show that these timesteppers can derive from a finite-element-in-time (FET) (and finite-element-in-space) interpretation. The benefits of this reformulation are discussed, including the derivation of timesteppers that are higher order in time, and the quantifiable dissipative SP properties in the non-ideal, resistive case.
        
        We discuss possible options for extending this FET approach to timesteppers for the compressible case.

        The kinetic corrections satisfy linearized Boltzmann equations. Using a Lénard--Bernstein collision operator, these take Fokker--Planck-like forms \cite{Fokker_1914, Planck_1917} wherein pseudo-particles in the numerical model obey the neoclassical transport equations, with particle-independent Brownian drift terms. This offers a rigorous methodology for incorporating collisions into the particle transport model, without coupling the equations of motions for each particle.
        
        Works by Chen, Chacón et al. \cite{Chen_Chacón_Barnes_2011, Chacón_Chen_Barnes_2013, Chen_Chacón_2014, Chen_Chacón_2015} have developed structure-preserving particle pushers for neoclassical transport in the Vlasov equations, derived from Crank--Nicolson integrators. We show these too can can derive from a FET interpretation, similarly offering potential extensions to higher-order-in-time particle pushers. The FET formulation is used also to consider how the stochastic drift terms can be incorporated into the pushers. Stochastic gyrokinetic expansions are also discussed.

        Different options for the numerical implementation of these schemes are considered.

        Due to the efficacy of FET in the development of SP timesteppers for both the fluid and kinetic component, we hope this approach will prove effective in the future for developing SP timesteppers for the full hybrid model. We hope this will give us the opportunity to incorporate previously inaccessible kinetic effects into the highly effective, modern, finite-element MHD models.
    \end{abstract}
    
    
    \newpage
    \tableofcontents
    
    
    \newpage
    \pagenumbering{arabic}
    %\linenumbers\renewcommand\thelinenumber{\color{black!50}\arabic{linenumber}}
            \input{0 - introduction/main.tex}
        \part{Research}
            \input{1 - low-noise PiC models/main.tex}
            \input{2 - kinetic component/main.tex}
            \input{3 - fluid component/main.tex}
            \input{4 - numerical implementation/main.tex}
        \part{Project Overview}
            \input{5 - research plan/main.tex}
            \input{6 - summary/main.tex}
    
    
    %\section{}
    \newpage
    \pagenumbering{gobble}
        \printbibliography


    \newpage
    \pagenumbering{roman}
    \appendix
        \part{Appendices}
            \input{8 - Hilbert complexes/main.tex}
            \input{9 - weak conservation proofs/main.tex}
\end{document}

            \documentclass[12pt, a4paper]{report}

\input{template/main.tex}

\title{\BA{Title in Progress...}}
\author{Boris Andrews}
\affil{Mathematical Institute, University of Oxford}
\date{\today}


\begin{document}
    \pagenumbering{gobble}
    \maketitle
    
    
    \begin{abstract}
        Magnetic confinement reactors---in particular tokamaks---offer one of the most promising options for achieving practical nuclear fusion, with the potential to provide virtually limitless, clean energy. The theoretical and numerical modeling of tokamak plasmas is simultaneously an essential component of effective reactor design, and a great research barrier. Tokamak operational conditions exhibit comparatively low Knudsen numbers. Kinetic effects, including kinetic waves and instabilities, Landau damping, bump-on-tail instabilities and more, are therefore highly influential in tokamak plasma dynamics. Purely fluid models are inherently incapable of capturing these effects, whereas the high dimensionality in purely kinetic models render them practically intractable for most relevant purposes.

        We consider a $\delta\!f$ decomposition model, with a macroscopic fluid background and microscopic kinetic correction, both fully coupled to each other. A similar manner of discretization is proposed to that used in the recent \texttt{STRUPHY} code \cite{Holderied_Possanner_Wang_2021, Holderied_2022, Li_et_al_2023} with a finite-element model for the background and a pseudo-particle/PiC model for the correction.

        The fluid background satisfies the full, non-linear, resistive, compressible, Hall MHD equations. \cite{Laakmann_Hu_Farrell_2022} introduces finite-element(-in-space) implicit timesteppers for the incompressible analogue to this system with structure-preserving (SP) properties in the ideal case, alongside parameter-robust preconditioners. We show that these timesteppers can derive from a finite-element-in-time (FET) (and finite-element-in-space) interpretation. The benefits of this reformulation are discussed, including the derivation of timesteppers that are higher order in time, and the quantifiable dissipative SP properties in the non-ideal, resistive case.
        
        We discuss possible options for extending this FET approach to timesteppers for the compressible case.

        The kinetic corrections satisfy linearized Boltzmann equations. Using a Lénard--Bernstein collision operator, these take Fokker--Planck-like forms \cite{Fokker_1914, Planck_1917} wherein pseudo-particles in the numerical model obey the neoclassical transport equations, with particle-independent Brownian drift terms. This offers a rigorous methodology for incorporating collisions into the particle transport model, without coupling the equations of motions for each particle.
        
        Works by Chen, Chacón et al. \cite{Chen_Chacón_Barnes_2011, Chacón_Chen_Barnes_2013, Chen_Chacón_2014, Chen_Chacón_2015} have developed structure-preserving particle pushers for neoclassical transport in the Vlasov equations, derived from Crank--Nicolson integrators. We show these too can can derive from a FET interpretation, similarly offering potential extensions to higher-order-in-time particle pushers. The FET formulation is used also to consider how the stochastic drift terms can be incorporated into the pushers. Stochastic gyrokinetic expansions are also discussed.

        Different options for the numerical implementation of these schemes are considered.

        Due to the efficacy of FET in the development of SP timesteppers for both the fluid and kinetic component, we hope this approach will prove effective in the future for developing SP timesteppers for the full hybrid model. We hope this will give us the opportunity to incorporate previously inaccessible kinetic effects into the highly effective, modern, finite-element MHD models.
    \end{abstract}
    
    
    \newpage
    \tableofcontents
    
    
    \newpage
    \pagenumbering{arabic}
    %\linenumbers\renewcommand\thelinenumber{\color{black!50}\arabic{linenumber}}
            \input{0 - introduction/main.tex}
        \part{Research}
            \input{1 - low-noise PiC models/main.tex}
            \input{2 - kinetic component/main.tex}
            \input{3 - fluid component/main.tex}
            \input{4 - numerical implementation/main.tex}
        \part{Project Overview}
            \input{5 - research plan/main.tex}
            \input{6 - summary/main.tex}
    
    
    %\section{}
    \newpage
    \pagenumbering{gobble}
        \printbibliography


    \newpage
    \pagenumbering{roman}
    \appendix
        \part{Appendices}
            \input{8 - Hilbert complexes/main.tex}
            \input{9 - weak conservation proofs/main.tex}
\end{document}

            \documentclass[12pt, a4paper]{report}

\input{template/main.tex}

\title{\BA{Title in Progress...}}
\author{Boris Andrews}
\affil{Mathematical Institute, University of Oxford}
\date{\today}


\begin{document}
    \pagenumbering{gobble}
    \maketitle
    
    
    \begin{abstract}
        Magnetic confinement reactors---in particular tokamaks---offer one of the most promising options for achieving practical nuclear fusion, with the potential to provide virtually limitless, clean energy. The theoretical and numerical modeling of tokamak plasmas is simultaneously an essential component of effective reactor design, and a great research barrier. Tokamak operational conditions exhibit comparatively low Knudsen numbers. Kinetic effects, including kinetic waves and instabilities, Landau damping, bump-on-tail instabilities and more, are therefore highly influential in tokamak plasma dynamics. Purely fluid models are inherently incapable of capturing these effects, whereas the high dimensionality in purely kinetic models render them practically intractable for most relevant purposes.

        We consider a $\delta\!f$ decomposition model, with a macroscopic fluid background and microscopic kinetic correction, both fully coupled to each other. A similar manner of discretization is proposed to that used in the recent \texttt{STRUPHY} code \cite{Holderied_Possanner_Wang_2021, Holderied_2022, Li_et_al_2023} with a finite-element model for the background and a pseudo-particle/PiC model for the correction.

        The fluid background satisfies the full, non-linear, resistive, compressible, Hall MHD equations. \cite{Laakmann_Hu_Farrell_2022} introduces finite-element(-in-space) implicit timesteppers for the incompressible analogue to this system with structure-preserving (SP) properties in the ideal case, alongside parameter-robust preconditioners. We show that these timesteppers can derive from a finite-element-in-time (FET) (and finite-element-in-space) interpretation. The benefits of this reformulation are discussed, including the derivation of timesteppers that are higher order in time, and the quantifiable dissipative SP properties in the non-ideal, resistive case.
        
        We discuss possible options for extending this FET approach to timesteppers for the compressible case.

        The kinetic corrections satisfy linearized Boltzmann equations. Using a Lénard--Bernstein collision operator, these take Fokker--Planck-like forms \cite{Fokker_1914, Planck_1917} wherein pseudo-particles in the numerical model obey the neoclassical transport equations, with particle-independent Brownian drift terms. This offers a rigorous methodology for incorporating collisions into the particle transport model, without coupling the equations of motions for each particle.
        
        Works by Chen, Chacón et al. \cite{Chen_Chacón_Barnes_2011, Chacón_Chen_Barnes_2013, Chen_Chacón_2014, Chen_Chacón_2015} have developed structure-preserving particle pushers for neoclassical transport in the Vlasov equations, derived from Crank--Nicolson integrators. We show these too can can derive from a FET interpretation, similarly offering potential extensions to higher-order-in-time particle pushers. The FET formulation is used also to consider how the stochastic drift terms can be incorporated into the pushers. Stochastic gyrokinetic expansions are also discussed.

        Different options for the numerical implementation of these schemes are considered.

        Due to the efficacy of FET in the development of SP timesteppers for both the fluid and kinetic component, we hope this approach will prove effective in the future for developing SP timesteppers for the full hybrid model. We hope this will give us the opportunity to incorporate previously inaccessible kinetic effects into the highly effective, modern, finite-element MHD models.
    \end{abstract}
    
    
    \newpage
    \tableofcontents
    
    
    \newpage
    \pagenumbering{arabic}
    %\linenumbers\renewcommand\thelinenumber{\color{black!50}\arabic{linenumber}}
            \input{0 - introduction/main.tex}
        \part{Research}
            \input{1 - low-noise PiC models/main.tex}
            \input{2 - kinetic component/main.tex}
            \input{3 - fluid component/main.tex}
            \input{4 - numerical implementation/main.tex}
        \part{Project Overview}
            \input{5 - research plan/main.tex}
            \input{6 - summary/main.tex}
    
    
    %\section{}
    \newpage
    \pagenumbering{gobble}
        \printbibliography


    \newpage
    \pagenumbering{roman}
    \appendix
        \part{Appendices}
            \input{8 - Hilbert complexes/main.tex}
            \input{9 - weak conservation proofs/main.tex}
\end{document}

            \documentclass[12pt, a4paper]{report}

\input{template/main.tex}

\title{\BA{Title in Progress...}}
\author{Boris Andrews}
\affil{Mathematical Institute, University of Oxford}
\date{\today}


\begin{document}
    \pagenumbering{gobble}
    \maketitle
    
    
    \begin{abstract}
        Magnetic confinement reactors---in particular tokamaks---offer one of the most promising options for achieving practical nuclear fusion, with the potential to provide virtually limitless, clean energy. The theoretical and numerical modeling of tokamak plasmas is simultaneously an essential component of effective reactor design, and a great research barrier. Tokamak operational conditions exhibit comparatively low Knudsen numbers. Kinetic effects, including kinetic waves and instabilities, Landau damping, bump-on-tail instabilities and more, are therefore highly influential in tokamak plasma dynamics. Purely fluid models are inherently incapable of capturing these effects, whereas the high dimensionality in purely kinetic models render them practically intractable for most relevant purposes.

        We consider a $\delta\!f$ decomposition model, with a macroscopic fluid background and microscopic kinetic correction, both fully coupled to each other. A similar manner of discretization is proposed to that used in the recent \texttt{STRUPHY} code \cite{Holderied_Possanner_Wang_2021, Holderied_2022, Li_et_al_2023} with a finite-element model for the background and a pseudo-particle/PiC model for the correction.

        The fluid background satisfies the full, non-linear, resistive, compressible, Hall MHD equations. \cite{Laakmann_Hu_Farrell_2022} introduces finite-element(-in-space) implicit timesteppers for the incompressible analogue to this system with structure-preserving (SP) properties in the ideal case, alongside parameter-robust preconditioners. We show that these timesteppers can derive from a finite-element-in-time (FET) (and finite-element-in-space) interpretation. The benefits of this reformulation are discussed, including the derivation of timesteppers that are higher order in time, and the quantifiable dissipative SP properties in the non-ideal, resistive case.
        
        We discuss possible options for extending this FET approach to timesteppers for the compressible case.

        The kinetic corrections satisfy linearized Boltzmann equations. Using a Lénard--Bernstein collision operator, these take Fokker--Planck-like forms \cite{Fokker_1914, Planck_1917} wherein pseudo-particles in the numerical model obey the neoclassical transport equations, with particle-independent Brownian drift terms. This offers a rigorous methodology for incorporating collisions into the particle transport model, without coupling the equations of motions for each particle.
        
        Works by Chen, Chacón et al. \cite{Chen_Chacón_Barnes_2011, Chacón_Chen_Barnes_2013, Chen_Chacón_2014, Chen_Chacón_2015} have developed structure-preserving particle pushers for neoclassical transport in the Vlasov equations, derived from Crank--Nicolson integrators. We show these too can can derive from a FET interpretation, similarly offering potential extensions to higher-order-in-time particle pushers. The FET formulation is used also to consider how the stochastic drift terms can be incorporated into the pushers. Stochastic gyrokinetic expansions are also discussed.

        Different options for the numerical implementation of these schemes are considered.

        Due to the efficacy of FET in the development of SP timesteppers for both the fluid and kinetic component, we hope this approach will prove effective in the future for developing SP timesteppers for the full hybrid model. We hope this will give us the opportunity to incorporate previously inaccessible kinetic effects into the highly effective, modern, finite-element MHD models.
    \end{abstract}
    
    
    \newpage
    \tableofcontents
    
    
    \newpage
    \pagenumbering{arabic}
    %\linenumbers\renewcommand\thelinenumber{\color{black!50}\arabic{linenumber}}
            \input{0 - introduction/main.tex}
        \part{Research}
            \input{1 - low-noise PiC models/main.tex}
            \input{2 - kinetic component/main.tex}
            \input{3 - fluid component/main.tex}
            \input{4 - numerical implementation/main.tex}
        \part{Project Overview}
            \input{5 - research plan/main.tex}
            \input{6 - summary/main.tex}
    
    
    %\section{}
    \newpage
    \pagenumbering{gobble}
        \printbibliography


    \newpage
    \pagenumbering{roman}
    \appendix
        \part{Appendices}
            \input{8 - Hilbert complexes/main.tex}
            \input{9 - weak conservation proofs/main.tex}
\end{document}

        \part{Project Overview}
            \documentclass[12pt, a4paper]{report}

\input{template/main.tex}

\title{\BA{Title in Progress...}}
\author{Boris Andrews}
\affil{Mathematical Institute, University of Oxford}
\date{\today}


\begin{document}
    \pagenumbering{gobble}
    \maketitle
    
    
    \begin{abstract}
        Magnetic confinement reactors---in particular tokamaks---offer one of the most promising options for achieving practical nuclear fusion, with the potential to provide virtually limitless, clean energy. The theoretical and numerical modeling of tokamak plasmas is simultaneously an essential component of effective reactor design, and a great research barrier. Tokamak operational conditions exhibit comparatively low Knudsen numbers. Kinetic effects, including kinetic waves and instabilities, Landau damping, bump-on-tail instabilities and more, are therefore highly influential in tokamak plasma dynamics. Purely fluid models are inherently incapable of capturing these effects, whereas the high dimensionality in purely kinetic models render them practically intractable for most relevant purposes.

        We consider a $\delta\!f$ decomposition model, with a macroscopic fluid background and microscopic kinetic correction, both fully coupled to each other. A similar manner of discretization is proposed to that used in the recent \texttt{STRUPHY} code \cite{Holderied_Possanner_Wang_2021, Holderied_2022, Li_et_al_2023} with a finite-element model for the background and a pseudo-particle/PiC model for the correction.

        The fluid background satisfies the full, non-linear, resistive, compressible, Hall MHD equations. \cite{Laakmann_Hu_Farrell_2022} introduces finite-element(-in-space) implicit timesteppers for the incompressible analogue to this system with structure-preserving (SP) properties in the ideal case, alongside parameter-robust preconditioners. We show that these timesteppers can derive from a finite-element-in-time (FET) (and finite-element-in-space) interpretation. The benefits of this reformulation are discussed, including the derivation of timesteppers that are higher order in time, and the quantifiable dissipative SP properties in the non-ideal, resistive case.
        
        We discuss possible options for extending this FET approach to timesteppers for the compressible case.

        The kinetic corrections satisfy linearized Boltzmann equations. Using a Lénard--Bernstein collision operator, these take Fokker--Planck-like forms \cite{Fokker_1914, Planck_1917} wherein pseudo-particles in the numerical model obey the neoclassical transport equations, with particle-independent Brownian drift terms. This offers a rigorous methodology for incorporating collisions into the particle transport model, without coupling the equations of motions for each particle.
        
        Works by Chen, Chacón et al. \cite{Chen_Chacón_Barnes_2011, Chacón_Chen_Barnes_2013, Chen_Chacón_2014, Chen_Chacón_2015} have developed structure-preserving particle pushers for neoclassical transport in the Vlasov equations, derived from Crank--Nicolson integrators. We show these too can can derive from a FET interpretation, similarly offering potential extensions to higher-order-in-time particle pushers. The FET formulation is used also to consider how the stochastic drift terms can be incorporated into the pushers. Stochastic gyrokinetic expansions are also discussed.

        Different options for the numerical implementation of these schemes are considered.

        Due to the efficacy of FET in the development of SP timesteppers for both the fluid and kinetic component, we hope this approach will prove effective in the future for developing SP timesteppers for the full hybrid model. We hope this will give us the opportunity to incorporate previously inaccessible kinetic effects into the highly effective, modern, finite-element MHD models.
    \end{abstract}
    
    
    \newpage
    \tableofcontents
    
    
    \newpage
    \pagenumbering{arabic}
    %\linenumbers\renewcommand\thelinenumber{\color{black!50}\arabic{linenumber}}
            \input{0 - introduction/main.tex}
        \part{Research}
            \input{1 - low-noise PiC models/main.tex}
            \input{2 - kinetic component/main.tex}
            \input{3 - fluid component/main.tex}
            \input{4 - numerical implementation/main.tex}
        \part{Project Overview}
            \input{5 - research plan/main.tex}
            \input{6 - summary/main.tex}
    
    
    %\section{}
    \newpage
    \pagenumbering{gobble}
        \printbibliography


    \newpage
    \pagenumbering{roman}
    \appendix
        \part{Appendices}
            \input{8 - Hilbert complexes/main.tex}
            \input{9 - weak conservation proofs/main.tex}
\end{document}

            \documentclass[12pt, a4paper]{report}

\input{template/main.tex}

\title{\BA{Title in Progress...}}
\author{Boris Andrews}
\affil{Mathematical Institute, University of Oxford}
\date{\today}


\begin{document}
    \pagenumbering{gobble}
    \maketitle
    
    
    \begin{abstract}
        Magnetic confinement reactors---in particular tokamaks---offer one of the most promising options for achieving practical nuclear fusion, with the potential to provide virtually limitless, clean energy. The theoretical and numerical modeling of tokamak plasmas is simultaneously an essential component of effective reactor design, and a great research barrier. Tokamak operational conditions exhibit comparatively low Knudsen numbers. Kinetic effects, including kinetic waves and instabilities, Landau damping, bump-on-tail instabilities and more, are therefore highly influential in tokamak plasma dynamics. Purely fluid models are inherently incapable of capturing these effects, whereas the high dimensionality in purely kinetic models render them practically intractable for most relevant purposes.

        We consider a $\delta\!f$ decomposition model, with a macroscopic fluid background and microscopic kinetic correction, both fully coupled to each other. A similar manner of discretization is proposed to that used in the recent \texttt{STRUPHY} code \cite{Holderied_Possanner_Wang_2021, Holderied_2022, Li_et_al_2023} with a finite-element model for the background and a pseudo-particle/PiC model for the correction.

        The fluid background satisfies the full, non-linear, resistive, compressible, Hall MHD equations. \cite{Laakmann_Hu_Farrell_2022} introduces finite-element(-in-space) implicit timesteppers for the incompressible analogue to this system with structure-preserving (SP) properties in the ideal case, alongside parameter-robust preconditioners. We show that these timesteppers can derive from a finite-element-in-time (FET) (and finite-element-in-space) interpretation. The benefits of this reformulation are discussed, including the derivation of timesteppers that are higher order in time, and the quantifiable dissipative SP properties in the non-ideal, resistive case.
        
        We discuss possible options for extending this FET approach to timesteppers for the compressible case.

        The kinetic corrections satisfy linearized Boltzmann equations. Using a Lénard--Bernstein collision operator, these take Fokker--Planck-like forms \cite{Fokker_1914, Planck_1917} wherein pseudo-particles in the numerical model obey the neoclassical transport equations, with particle-independent Brownian drift terms. This offers a rigorous methodology for incorporating collisions into the particle transport model, without coupling the equations of motions for each particle.
        
        Works by Chen, Chacón et al. \cite{Chen_Chacón_Barnes_2011, Chacón_Chen_Barnes_2013, Chen_Chacón_2014, Chen_Chacón_2015} have developed structure-preserving particle pushers for neoclassical transport in the Vlasov equations, derived from Crank--Nicolson integrators. We show these too can can derive from a FET interpretation, similarly offering potential extensions to higher-order-in-time particle pushers. The FET formulation is used also to consider how the stochastic drift terms can be incorporated into the pushers. Stochastic gyrokinetic expansions are also discussed.

        Different options for the numerical implementation of these schemes are considered.

        Due to the efficacy of FET in the development of SP timesteppers for both the fluid and kinetic component, we hope this approach will prove effective in the future for developing SP timesteppers for the full hybrid model. We hope this will give us the opportunity to incorporate previously inaccessible kinetic effects into the highly effective, modern, finite-element MHD models.
    \end{abstract}
    
    
    \newpage
    \tableofcontents
    
    
    \newpage
    \pagenumbering{arabic}
    %\linenumbers\renewcommand\thelinenumber{\color{black!50}\arabic{linenumber}}
            \input{0 - introduction/main.tex}
        \part{Research}
            \input{1 - low-noise PiC models/main.tex}
            \input{2 - kinetic component/main.tex}
            \input{3 - fluid component/main.tex}
            \input{4 - numerical implementation/main.tex}
        \part{Project Overview}
            \input{5 - research plan/main.tex}
            \input{6 - summary/main.tex}
    
    
    %\section{}
    \newpage
    \pagenumbering{gobble}
        \printbibliography


    \newpage
    \pagenumbering{roman}
    \appendix
        \part{Appendices}
            \input{8 - Hilbert complexes/main.tex}
            \input{9 - weak conservation proofs/main.tex}
\end{document}

    
    
    %\section{}
    \newpage
    \pagenumbering{gobble}
        \printbibliography


    \newpage
    \pagenumbering{roman}
    \appendix
        \part{Appendices}
            \documentclass[12pt, a4paper]{report}

\input{template/main.tex}

\title{\BA{Title in Progress...}}
\author{Boris Andrews}
\affil{Mathematical Institute, University of Oxford}
\date{\today}


\begin{document}
    \pagenumbering{gobble}
    \maketitle
    
    
    \begin{abstract}
        Magnetic confinement reactors---in particular tokamaks---offer one of the most promising options for achieving practical nuclear fusion, with the potential to provide virtually limitless, clean energy. The theoretical and numerical modeling of tokamak plasmas is simultaneously an essential component of effective reactor design, and a great research barrier. Tokamak operational conditions exhibit comparatively low Knudsen numbers. Kinetic effects, including kinetic waves and instabilities, Landau damping, bump-on-tail instabilities and more, are therefore highly influential in tokamak plasma dynamics. Purely fluid models are inherently incapable of capturing these effects, whereas the high dimensionality in purely kinetic models render them practically intractable for most relevant purposes.

        We consider a $\delta\!f$ decomposition model, with a macroscopic fluid background and microscopic kinetic correction, both fully coupled to each other. A similar manner of discretization is proposed to that used in the recent \texttt{STRUPHY} code \cite{Holderied_Possanner_Wang_2021, Holderied_2022, Li_et_al_2023} with a finite-element model for the background and a pseudo-particle/PiC model for the correction.

        The fluid background satisfies the full, non-linear, resistive, compressible, Hall MHD equations. \cite{Laakmann_Hu_Farrell_2022} introduces finite-element(-in-space) implicit timesteppers for the incompressible analogue to this system with structure-preserving (SP) properties in the ideal case, alongside parameter-robust preconditioners. We show that these timesteppers can derive from a finite-element-in-time (FET) (and finite-element-in-space) interpretation. The benefits of this reformulation are discussed, including the derivation of timesteppers that are higher order in time, and the quantifiable dissipative SP properties in the non-ideal, resistive case.
        
        We discuss possible options for extending this FET approach to timesteppers for the compressible case.

        The kinetic corrections satisfy linearized Boltzmann equations. Using a Lénard--Bernstein collision operator, these take Fokker--Planck-like forms \cite{Fokker_1914, Planck_1917} wherein pseudo-particles in the numerical model obey the neoclassical transport equations, with particle-independent Brownian drift terms. This offers a rigorous methodology for incorporating collisions into the particle transport model, without coupling the equations of motions for each particle.
        
        Works by Chen, Chacón et al. \cite{Chen_Chacón_Barnes_2011, Chacón_Chen_Barnes_2013, Chen_Chacón_2014, Chen_Chacón_2015} have developed structure-preserving particle pushers for neoclassical transport in the Vlasov equations, derived from Crank--Nicolson integrators. We show these too can can derive from a FET interpretation, similarly offering potential extensions to higher-order-in-time particle pushers. The FET formulation is used also to consider how the stochastic drift terms can be incorporated into the pushers. Stochastic gyrokinetic expansions are also discussed.

        Different options for the numerical implementation of these schemes are considered.

        Due to the efficacy of FET in the development of SP timesteppers for both the fluid and kinetic component, we hope this approach will prove effective in the future for developing SP timesteppers for the full hybrid model. We hope this will give us the opportunity to incorporate previously inaccessible kinetic effects into the highly effective, modern, finite-element MHD models.
    \end{abstract}
    
    
    \newpage
    \tableofcontents
    
    
    \newpage
    \pagenumbering{arabic}
    %\linenumbers\renewcommand\thelinenumber{\color{black!50}\arabic{linenumber}}
            \input{0 - introduction/main.tex}
        \part{Research}
            \input{1 - low-noise PiC models/main.tex}
            \input{2 - kinetic component/main.tex}
            \input{3 - fluid component/main.tex}
            \input{4 - numerical implementation/main.tex}
        \part{Project Overview}
            \input{5 - research plan/main.tex}
            \input{6 - summary/main.tex}
    
    
    %\section{}
    \newpage
    \pagenumbering{gobble}
        \printbibliography


    \newpage
    \pagenumbering{roman}
    \appendix
        \part{Appendices}
            \input{8 - Hilbert complexes/main.tex}
            \input{9 - weak conservation proofs/main.tex}
\end{document}

            \documentclass[12pt, a4paper]{report}

\input{template/main.tex}

\title{\BA{Title in Progress...}}
\author{Boris Andrews}
\affil{Mathematical Institute, University of Oxford}
\date{\today}


\begin{document}
    \pagenumbering{gobble}
    \maketitle
    
    
    \begin{abstract}
        Magnetic confinement reactors---in particular tokamaks---offer one of the most promising options for achieving practical nuclear fusion, with the potential to provide virtually limitless, clean energy. The theoretical and numerical modeling of tokamak plasmas is simultaneously an essential component of effective reactor design, and a great research barrier. Tokamak operational conditions exhibit comparatively low Knudsen numbers. Kinetic effects, including kinetic waves and instabilities, Landau damping, bump-on-tail instabilities and more, are therefore highly influential in tokamak plasma dynamics. Purely fluid models are inherently incapable of capturing these effects, whereas the high dimensionality in purely kinetic models render them practically intractable for most relevant purposes.

        We consider a $\delta\!f$ decomposition model, with a macroscopic fluid background and microscopic kinetic correction, both fully coupled to each other. A similar manner of discretization is proposed to that used in the recent \texttt{STRUPHY} code \cite{Holderied_Possanner_Wang_2021, Holderied_2022, Li_et_al_2023} with a finite-element model for the background and a pseudo-particle/PiC model for the correction.

        The fluid background satisfies the full, non-linear, resistive, compressible, Hall MHD equations. \cite{Laakmann_Hu_Farrell_2022} introduces finite-element(-in-space) implicit timesteppers for the incompressible analogue to this system with structure-preserving (SP) properties in the ideal case, alongside parameter-robust preconditioners. We show that these timesteppers can derive from a finite-element-in-time (FET) (and finite-element-in-space) interpretation. The benefits of this reformulation are discussed, including the derivation of timesteppers that are higher order in time, and the quantifiable dissipative SP properties in the non-ideal, resistive case.
        
        We discuss possible options for extending this FET approach to timesteppers for the compressible case.

        The kinetic corrections satisfy linearized Boltzmann equations. Using a Lénard--Bernstein collision operator, these take Fokker--Planck-like forms \cite{Fokker_1914, Planck_1917} wherein pseudo-particles in the numerical model obey the neoclassical transport equations, with particle-independent Brownian drift terms. This offers a rigorous methodology for incorporating collisions into the particle transport model, without coupling the equations of motions for each particle.
        
        Works by Chen, Chacón et al. \cite{Chen_Chacón_Barnes_2011, Chacón_Chen_Barnes_2013, Chen_Chacón_2014, Chen_Chacón_2015} have developed structure-preserving particle pushers for neoclassical transport in the Vlasov equations, derived from Crank--Nicolson integrators. We show these too can can derive from a FET interpretation, similarly offering potential extensions to higher-order-in-time particle pushers. The FET formulation is used also to consider how the stochastic drift terms can be incorporated into the pushers. Stochastic gyrokinetic expansions are also discussed.

        Different options for the numerical implementation of these schemes are considered.

        Due to the efficacy of FET in the development of SP timesteppers for both the fluid and kinetic component, we hope this approach will prove effective in the future for developing SP timesteppers for the full hybrid model. We hope this will give us the opportunity to incorporate previously inaccessible kinetic effects into the highly effective, modern, finite-element MHD models.
    \end{abstract}
    
    
    \newpage
    \tableofcontents
    
    
    \newpage
    \pagenumbering{arabic}
    %\linenumbers\renewcommand\thelinenumber{\color{black!50}\arabic{linenumber}}
            \input{0 - introduction/main.tex}
        \part{Research}
            \input{1 - low-noise PiC models/main.tex}
            \input{2 - kinetic component/main.tex}
            \input{3 - fluid component/main.tex}
            \input{4 - numerical implementation/main.tex}
        \part{Project Overview}
            \input{5 - research plan/main.tex}
            \input{6 - summary/main.tex}
    
    
    %\section{}
    \newpage
    \pagenumbering{gobble}
        \printbibliography


    \newpage
    \pagenumbering{roman}
    \appendix
        \part{Appendices}
            \input{8 - Hilbert complexes/main.tex}
            \input{9 - weak conservation proofs/main.tex}
\end{document}

\end{document}

            \documentclass[12pt, a4paper]{report}

\documentclass[12pt, a4paper]{report}

\input{template/main.tex}

\title{\BA{Title in Progress...}}
\author{Boris Andrews}
\affil{Mathematical Institute, University of Oxford}
\date{\today}


\begin{document}
    \pagenumbering{gobble}
    \maketitle
    
    
    \begin{abstract}
        Magnetic confinement reactors---in particular tokamaks---offer one of the most promising options for achieving practical nuclear fusion, with the potential to provide virtually limitless, clean energy. The theoretical and numerical modeling of tokamak plasmas is simultaneously an essential component of effective reactor design, and a great research barrier. Tokamak operational conditions exhibit comparatively low Knudsen numbers. Kinetic effects, including kinetic waves and instabilities, Landau damping, bump-on-tail instabilities and more, are therefore highly influential in tokamak plasma dynamics. Purely fluid models are inherently incapable of capturing these effects, whereas the high dimensionality in purely kinetic models render them practically intractable for most relevant purposes.

        We consider a $\delta\!f$ decomposition model, with a macroscopic fluid background and microscopic kinetic correction, both fully coupled to each other. A similar manner of discretization is proposed to that used in the recent \texttt{STRUPHY} code \cite{Holderied_Possanner_Wang_2021, Holderied_2022, Li_et_al_2023} with a finite-element model for the background and a pseudo-particle/PiC model for the correction.

        The fluid background satisfies the full, non-linear, resistive, compressible, Hall MHD equations. \cite{Laakmann_Hu_Farrell_2022} introduces finite-element(-in-space) implicit timesteppers for the incompressible analogue to this system with structure-preserving (SP) properties in the ideal case, alongside parameter-robust preconditioners. We show that these timesteppers can derive from a finite-element-in-time (FET) (and finite-element-in-space) interpretation. The benefits of this reformulation are discussed, including the derivation of timesteppers that are higher order in time, and the quantifiable dissipative SP properties in the non-ideal, resistive case.
        
        We discuss possible options for extending this FET approach to timesteppers for the compressible case.

        The kinetic corrections satisfy linearized Boltzmann equations. Using a Lénard--Bernstein collision operator, these take Fokker--Planck-like forms \cite{Fokker_1914, Planck_1917} wherein pseudo-particles in the numerical model obey the neoclassical transport equations, with particle-independent Brownian drift terms. This offers a rigorous methodology for incorporating collisions into the particle transport model, without coupling the equations of motions for each particle.
        
        Works by Chen, Chacón et al. \cite{Chen_Chacón_Barnes_2011, Chacón_Chen_Barnes_2013, Chen_Chacón_2014, Chen_Chacón_2015} have developed structure-preserving particle pushers for neoclassical transport in the Vlasov equations, derived from Crank--Nicolson integrators. We show these too can can derive from a FET interpretation, similarly offering potential extensions to higher-order-in-time particle pushers. The FET formulation is used also to consider how the stochastic drift terms can be incorporated into the pushers. Stochastic gyrokinetic expansions are also discussed.

        Different options for the numerical implementation of these schemes are considered.

        Due to the efficacy of FET in the development of SP timesteppers for both the fluid and kinetic component, we hope this approach will prove effective in the future for developing SP timesteppers for the full hybrid model. We hope this will give us the opportunity to incorporate previously inaccessible kinetic effects into the highly effective, modern, finite-element MHD models.
    \end{abstract}
    
    
    \newpage
    \tableofcontents
    
    
    \newpage
    \pagenumbering{arabic}
    %\linenumbers\renewcommand\thelinenumber{\color{black!50}\arabic{linenumber}}
            \input{0 - introduction/main.tex}
        \part{Research}
            \input{1 - low-noise PiC models/main.tex}
            \input{2 - kinetic component/main.tex}
            \input{3 - fluid component/main.tex}
            \input{4 - numerical implementation/main.tex}
        \part{Project Overview}
            \input{5 - research plan/main.tex}
            \input{6 - summary/main.tex}
    
    
    %\section{}
    \newpage
    \pagenumbering{gobble}
        \printbibliography


    \newpage
    \pagenumbering{roman}
    \appendix
        \part{Appendices}
            \input{8 - Hilbert complexes/main.tex}
            \input{9 - weak conservation proofs/main.tex}
\end{document}


\title{\BA{Title in Progress...}}
\author{Boris Andrews}
\affil{Mathematical Institute, University of Oxford}
\date{\today}


\begin{document}
    \pagenumbering{gobble}
    \maketitle
    
    
    \begin{abstract}
        Magnetic confinement reactors---in particular tokamaks---offer one of the most promising options for achieving practical nuclear fusion, with the potential to provide virtually limitless, clean energy. The theoretical and numerical modeling of tokamak plasmas is simultaneously an essential component of effective reactor design, and a great research barrier. Tokamak operational conditions exhibit comparatively low Knudsen numbers. Kinetic effects, including kinetic waves and instabilities, Landau damping, bump-on-tail instabilities and more, are therefore highly influential in tokamak plasma dynamics. Purely fluid models are inherently incapable of capturing these effects, whereas the high dimensionality in purely kinetic models render them practically intractable for most relevant purposes.

        We consider a $\delta\!f$ decomposition model, with a macroscopic fluid background and microscopic kinetic correction, both fully coupled to each other. A similar manner of discretization is proposed to that used in the recent \texttt{STRUPHY} code \cite{Holderied_Possanner_Wang_2021, Holderied_2022, Li_et_al_2023} with a finite-element model for the background and a pseudo-particle/PiC model for the correction.

        The fluid background satisfies the full, non-linear, resistive, compressible, Hall MHD equations. \cite{Laakmann_Hu_Farrell_2022} introduces finite-element(-in-space) implicit timesteppers for the incompressible analogue to this system with structure-preserving (SP) properties in the ideal case, alongside parameter-robust preconditioners. We show that these timesteppers can derive from a finite-element-in-time (FET) (and finite-element-in-space) interpretation. The benefits of this reformulation are discussed, including the derivation of timesteppers that are higher order in time, and the quantifiable dissipative SP properties in the non-ideal, resistive case.
        
        We discuss possible options for extending this FET approach to timesteppers for the compressible case.

        The kinetic corrections satisfy linearized Boltzmann equations. Using a Lénard--Bernstein collision operator, these take Fokker--Planck-like forms \cite{Fokker_1914, Planck_1917} wherein pseudo-particles in the numerical model obey the neoclassical transport equations, with particle-independent Brownian drift terms. This offers a rigorous methodology for incorporating collisions into the particle transport model, without coupling the equations of motions for each particle.
        
        Works by Chen, Chacón et al. \cite{Chen_Chacón_Barnes_2011, Chacón_Chen_Barnes_2013, Chen_Chacón_2014, Chen_Chacón_2015} have developed structure-preserving particle pushers for neoclassical transport in the Vlasov equations, derived from Crank--Nicolson integrators. We show these too can can derive from a FET interpretation, similarly offering potential extensions to higher-order-in-time particle pushers. The FET formulation is used also to consider how the stochastic drift terms can be incorporated into the pushers. Stochastic gyrokinetic expansions are also discussed.

        Different options for the numerical implementation of these schemes are considered.

        Due to the efficacy of FET in the development of SP timesteppers for both the fluid and kinetic component, we hope this approach will prove effective in the future for developing SP timesteppers for the full hybrid model. We hope this will give us the opportunity to incorporate previously inaccessible kinetic effects into the highly effective, modern, finite-element MHD models.
    \end{abstract}
    
    
    \newpage
    \tableofcontents
    
    
    \newpage
    \pagenumbering{arabic}
    %\linenumbers\renewcommand\thelinenumber{\color{black!50}\arabic{linenumber}}
            \documentclass[12pt, a4paper]{report}

\input{template/main.tex}

\title{\BA{Title in Progress...}}
\author{Boris Andrews}
\affil{Mathematical Institute, University of Oxford}
\date{\today}


\begin{document}
    \pagenumbering{gobble}
    \maketitle
    
    
    \begin{abstract}
        Magnetic confinement reactors---in particular tokamaks---offer one of the most promising options for achieving practical nuclear fusion, with the potential to provide virtually limitless, clean energy. The theoretical and numerical modeling of tokamak plasmas is simultaneously an essential component of effective reactor design, and a great research barrier. Tokamak operational conditions exhibit comparatively low Knudsen numbers. Kinetic effects, including kinetic waves and instabilities, Landau damping, bump-on-tail instabilities and more, are therefore highly influential in tokamak plasma dynamics. Purely fluid models are inherently incapable of capturing these effects, whereas the high dimensionality in purely kinetic models render them practically intractable for most relevant purposes.

        We consider a $\delta\!f$ decomposition model, with a macroscopic fluid background and microscopic kinetic correction, both fully coupled to each other. A similar manner of discretization is proposed to that used in the recent \texttt{STRUPHY} code \cite{Holderied_Possanner_Wang_2021, Holderied_2022, Li_et_al_2023} with a finite-element model for the background and a pseudo-particle/PiC model for the correction.

        The fluid background satisfies the full, non-linear, resistive, compressible, Hall MHD equations. \cite{Laakmann_Hu_Farrell_2022} introduces finite-element(-in-space) implicit timesteppers for the incompressible analogue to this system with structure-preserving (SP) properties in the ideal case, alongside parameter-robust preconditioners. We show that these timesteppers can derive from a finite-element-in-time (FET) (and finite-element-in-space) interpretation. The benefits of this reformulation are discussed, including the derivation of timesteppers that are higher order in time, and the quantifiable dissipative SP properties in the non-ideal, resistive case.
        
        We discuss possible options for extending this FET approach to timesteppers for the compressible case.

        The kinetic corrections satisfy linearized Boltzmann equations. Using a Lénard--Bernstein collision operator, these take Fokker--Planck-like forms \cite{Fokker_1914, Planck_1917} wherein pseudo-particles in the numerical model obey the neoclassical transport equations, with particle-independent Brownian drift terms. This offers a rigorous methodology for incorporating collisions into the particle transport model, without coupling the equations of motions for each particle.
        
        Works by Chen, Chacón et al. \cite{Chen_Chacón_Barnes_2011, Chacón_Chen_Barnes_2013, Chen_Chacón_2014, Chen_Chacón_2015} have developed structure-preserving particle pushers for neoclassical transport in the Vlasov equations, derived from Crank--Nicolson integrators. We show these too can can derive from a FET interpretation, similarly offering potential extensions to higher-order-in-time particle pushers. The FET formulation is used also to consider how the stochastic drift terms can be incorporated into the pushers. Stochastic gyrokinetic expansions are also discussed.

        Different options for the numerical implementation of these schemes are considered.

        Due to the efficacy of FET in the development of SP timesteppers for both the fluid and kinetic component, we hope this approach will prove effective in the future for developing SP timesteppers for the full hybrid model. We hope this will give us the opportunity to incorporate previously inaccessible kinetic effects into the highly effective, modern, finite-element MHD models.
    \end{abstract}
    
    
    \newpage
    \tableofcontents
    
    
    \newpage
    \pagenumbering{arabic}
    %\linenumbers\renewcommand\thelinenumber{\color{black!50}\arabic{linenumber}}
            \input{0 - introduction/main.tex}
        \part{Research}
            \input{1 - low-noise PiC models/main.tex}
            \input{2 - kinetic component/main.tex}
            \input{3 - fluid component/main.tex}
            \input{4 - numerical implementation/main.tex}
        \part{Project Overview}
            \input{5 - research plan/main.tex}
            \input{6 - summary/main.tex}
    
    
    %\section{}
    \newpage
    \pagenumbering{gobble}
        \printbibliography


    \newpage
    \pagenumbering{roman}
    \appendix
        \part{Appendices}
            \input{8 - Hilbert complexes/main.tex}
            \input{9 - weak conservation proofs/main.tex}
\end{document}

        \part{Research}
            \documentclass[12pt, a4paper]{report}

\input{template/main.tex}

\title{\BA{Title in Progress...}}
\author{Boris Andrews}
\affil{Mathematical Institute, University of Oxford}
\date{\today}


\begin{document}
    \pagenumbering{gobble}
    \maketitle
    
    
    \begin{abstract}
        Magnetic confinement reactors---in particular tokamaks---offer one of the most promising options for achieving practical nuclear fusion, with the potential to provide virtually limitless, clean energy. The theoretical and numerical modeling of tokamak plasmas is simultaneously an essential component of effective reactor design, and a great research barrier. Tokamak operational conditions exhibit comparatively low Knudsen numbers. Kinetic effects, including kinetic waves and instabilities, Landau damping, bump-on-tail instabilities and more, are therefore highly influential in tokamak plasma dynamics. Purely fluid models are inherently incapable of capturing these effects, whereas the high dimensionality in purely kinetic models render them practically intractable for most relevant purposes.

        We consider a $\delta\!f$ decomposition model, with a macroscopic fluid background and microscopic kinetic correction, both fully coupled to each other. A similar manner of discretization is proposed to that used in the recent \texttt{STRUPHY} code \cite{Holderied_Possanner_Wang_2021, Holderied_2022, Li_et_al_2023} with a finite-element model for the background and a pseudo-particle/PiC model for the correction.

        The fluid background satisfies the full, non-linear, resistive, compressible, Hall MHD equations. \cite{Laakmann_Hu_Farrell_2022} introduces finite-element(-in-space) implicit timesteppers for the incompressible analogue to this system with structure-preserving (SP) properties in the ideal case, alongside parameter-robust preconditioners. We show that these timesteppers can derive from a finite-element-in-time (FET) (and finite-element-in-space) interpretation. The benefits of this reformulation are discussed, including the derivation of timesteppers that are higher order in time, and the quantifiable dissipative SP properties in the non-ideal, resistive case.
        
        We discuss possible options for extending this FET approach to timesteppers for the compressible case.

        The kinetic corrections satisfy linearized Boltzmann equations. Using a Lénard--Bernstein collision operator, these take Fokker--Planck-like forms \cite{Fokker_1914, Planck_1917} wherein pseudo-particles in the numerical model obey the neoclassical transport equations, with particle-independent Brownian drift terms. This offers a rigorous methodology for incorporating collisions into the particle transport model, without coupling the equations of motions for each particle.
        
        Works by Chen, Chacón et al. \cite{Chen_Chacón_Barnes_2011, Chacón_Chen_Barnes_2013, Chen_Chacón_2014, Chen_Chacón_2015} have developed structure-preserving particle pushers for neoclassical transport in the Vlasov equations, derived from Crank--Nicolson integrators. We show these too can can derive from a FET interpretation, similarly offering potential extensions to higher-order-in-time particle pushers. The FET formulation is used also to consider how the stochastic drift terms can be incorporated into the pushers. Stochastic gyrokinetic expansions are also discussed.

        Different options for the numerical implementation of these schemes are considered.

        Due to the efficacy of FET in the development of SP timesteppers for both the fluid and kinetic component, we hope this approach will prove effective in the future for developing SP timesteppers for the full hybrid model. We hope this will give us the opportunity to incorporate previously inaccessible kinetic effects into the highly effective, modern, finite-element MHD models.
    \end{abstract}
    
    
    \newpage
    \tableofcontents
    
    
    \newpage
    \pagenumbering{arabic}
    %\linenumbers\renewcommand\thelinenumber{\color{black!50}\arabic{linenumber}}
            \input{0 - introduction/main.tex}
        \part{Research}
            \input{1 - low-noise PiC models/main.tex}
            \input{2 - kinetic component/main.tex}
            \input{3 - fluid component/main.tex}
            \input{4 - numerical implementation/main.tex}
        \part{Project Overview}
            \input{5 - research plan/main.tex}
            \input{6 - summary/main.tex}
    
    
    %\section{}
    \newpage
    \pagenumbering{gobble}
        \printbibliography


    \newpage
    \pagenumbering{roman}
    \appendix
        \part{Appendices}
            \input{8 - Hilbert complexes/main.tex}
            \input{9 - weak conservation proofs/main.tex}
\end{document}

            \documentclass[12pt, a4paper]{report}

\input{template/main.tex}

\title{\BA{Title in Progress...}}
\author{Boris Andrews}
\affil{Mathematical Institute, University of Oxford}
\date{\today}


\begin{document}
    \pagenumbering{gobble}
    \maketitle
    
    
    \begin{abstract}
        Magnetic confinement reactors---in particular tokamaks---offer one of the most promising options for achieving practical nuclear fusion, with the potential to provide virtually limitless, clean energy. The theoretical and numerical modeling of tokamak plasmas is simultaneously an essential component of effective reactor design, and a great research barrier. Tokamak operational conditions exhibit comparatively low Knudsen numbers. Kinetic effects, including kinetic waves and instabilities, Landau damping, bump-on-tail instabilities and more, are therefore highly influential in tokamak plasma dynamics. Purely fluid models are inherently incapable of capturing these effects, whereas the high dimensionality in purely kinetic models render them practically intractable for most relevant purposes.

        We consider a $\delta\!f$ decomposition model, with a macroscopic fluid background and microscopic kinetic correction, both fully coupled to each other. A similar manner of discretization is proposed to that used in the recent \texttt{STRUPHY} code \cite{Holderied_Possanner_Wang_2021, Holderied_2022, Li_et_al_2023} with a finite-element model for the background and a pseudo-particle/PiC model for the correction.

        The fluid background satisfies the full, non-linear, resistive, compressible, Hall MHD equations. \cite{Laakmann_Hu_Farrell_2022} introduces finite-element(-in-space) implicit timesteppers for the incompressible analogue to this system with structure-preserving (SP) properties in the ideal case, alongside parameter-robust preconditioners. We show that these timesteppers can derive from a finite-element-in-time (FET) (and finite-element-in-space) interpretation. The benefits of this reformulation are discussed, including the derivation of timesteppers that are higher order in time, and the quantifiable dissipative SP properties in the non-ideal, resistive case.
        
        We discuss possible options for extending this FET approach to timesteppers for the compressible case.

        The kinetic corrections satisfy linearized Boltzmann equations. Using a Lénard--Bernstein collision operator, these take Fokker--Planck-like forms \cite{Fokker_1914, Planck_1917} wherein pseudo-particles in the numerical model obey the neoclassical transport equations, with particle-independent Brownian drift terms. This offers a rigorous methodology for incorporating collisions into the particle transport model, without coupling the equations of motions for each particle.
        
        Works by Chen, Chacón et al. \cite{Chen_Chacón_Barnes_2011, Chacón_Chen_Barnes_2013, Chen_Chacón_2014, Chen_Chacón_2015} have developed structure-preserving particle pushers for neoclassical transport in the Vlasov equations, derived from Crank--Nicolson integrators. We show these too can can derive from a FET interpretation, similarly offering potential extensions to higher-order-in-time particle pushers. The FET formulation is used also to consider how the stochastic drift terms can be incorporated into the pushers. Stochastic gyrokinetic expansions are also discussed.

        Different options for the numerical implementation of these schemes are considered.

        Due to the efficacy of FET in the development of SP timesteppers for both the fluid and kinetic component, we hope this approach will prove effective in the future for developing SP timesteppers for the full hybrid model. We hope this will give us the opportunity to incorporate previously inaccessible kinetic effects into the highly effective, modern, finite-element MHD models.
    \end{abstract}
    
    
    \newpage
    \tableofcontents
    
    
    \newpage
    \pagenumbering{arabic}
    %\linenumbers\renewcommand\thelinenumber{\color{black!50}\arabic{linenumber}}
            \input{0 - introduction/main.tex}
        \part{Research}
            \input{1 - low-noise PiC models/main.tex}
            \input{2 - kinetic component/main.tex}
            \input{3 - fluid component/main.tex}
            \input{4 - numerical implementation/main.tex}
        \part{Project Overview}
            \input{5 - research plan/main.tex}
            \input{6 - summary/main.tex}
    
    
    %\section{}
    \newpage
    \pagenumbering{gobble}
        \printbibliography


    \newpage
    \pagenumbering{roman}
    \appendix
        \part{Appendices}
            \input{8 - Hilbert complexes/main.tex}
            \input{9 - weak conservation proofs/main.tex}
\end{document}

            \documentclass[12pt, a4paper]{report}

\input{template/main.tex}

\title{\BA{Title in Progress...}}
\author{Boris Andrews}
\affil{Mathematical Institute, University of Oxford}
\date{\today}


\begin{document}
    \pagenumbering{gobble}
    \maketitle
    
    
    \begin{abstract}
        Magnetic confinement reactors---in particular tokamaks---offer one of the most promising options for achieving practical nuclear fusion, with the potential to provide virtually limitless, clean energy. The theoretical and numerical modeling of tokamak plasmas is simultaneously an essential component of effective reactor design, and a great research barrier. Tokamak operational conditions exhibit comparatively low Knudsen numbers. Kinetic effects, including kinetic waves and instabilities, Landau damping, bump-on-tail instabilities and more, are therefore highly influential in tokamak plasma dynamics. Purely fluid models are inherently incapable of capturing these effects, whereas the high dimensionality in purely kinetic models render them practically intractable for most relevant purposes.

        We consider a $\delta\!f$ decomposition model, with a macroscopic fluid background and microscopic kinetic correction, both fully coupled to each other. A similar manner of discretization is proposed to that used in the recent \texttt{STRUPHY} code \cite{Holderied_Possanner_Wang_2021, Holderied_2022, Li_et_al_2023} with a finite-element model for the background and a pseudo-particle/PiC model for the correction.

        The fluid background satisfies the full, non-linear, resistive, compressible, Hall MHD equations. \cite{Laakmann_Hu_Farrell_2022} introduces finite-element(-in-space) implicit timesteppers for the incompressible analogue to this system with structure-preserving (SP) properties in the ideal case, alongside parameter-robust preconditioners. We show that these timesteppers can derive from a finite-element-in-time (FET) (and finite-element-in-space) interpretation. The benefits of this reformulation are discussed, including the derivation of timesteppers that are higher order in time, and the quantifiable dissipative SP properties in the non-ideal, resistive case.
        
        We discuss possible options for extending this FET approach to timesteppers for the compressible case.

        The kinetic corrections satisfy linearized Boltzmann equations. Using a Lénard--Bernstein collision operator, these take Fokker--Planck-like forms \cite{Fokker_1914, Planck_1917} wherein pseudo-particles in the numerical model obey the neoclassical transport equations, with particle-independent Brownian drift terms. This offers a rigorous methodology for incorporating collisions into the particle transport model, without coupling the equations of motions for each particle.
        
        Works by Chen, Chacón et al. \cite{Chen_Chacón_Barnes_2011, Chacón_Chen_Barnes_2013, Chen_Chacón_2014, Chen_Chacón_2015} have developed structure-preserving particle pushers for neoclassical transport in the Vlasov equations, derived from Crank--Nicolson integrators. We show these too can can derive from a FET interpretation, similarly offering potential extensions to higher-order-in-time particle pushers. The FET formulation is used also to consider how the stochastic drift terms can be incorporated into the pushers. Stochastic gyrokinetic expansions are also discussed.

        Different options for the numerical implementation of these schemes are considered.

        Due to the efficacy of FET in the development of SP timesteppers for both the fluid and kinetic component, we hope this approach will prove effective in the future for developing SP timesteppers for the full hybrid model. We hope this will give us the opportunity to incorporate previously inaccessible kinetic effects into the highly effective, modern, finite-element MHD models.
    \end{abstract}
    
    
    \newpage
    \tableofcontents
    
    
    \newpage
    \pagenumbering{arabic}
    %\linenumbers\renewcommand\thelinenumber{\color{black!50}\arabic{linenumber}}
            \input{0 - introduction/main.tex}
        \part{Research}
            \input{1 - low-noise PiC models/main.tex}
            \input{2 - kinetic component/main.tex}
            \input{3 - fluid component/main.tex}
            \input{4 - numerical implementation/main.tex}
        \part{Project Overview}
            \input{5 - research plan/main.tex}
            \input{6 - summary/main.tex}
    
    
    %\section{}
    \newpage
    \pagenumbering{gobble}
        \printbibliography


    \newpage
    \pagenumbering{roman}
    \appendix
        \part{Appendices}
            \input{8 - Hilbert complexes/main.tex}
            \input{9 - weak conservation proofs/main.tex}
\end{document}

            \documentclass[12pt, a4paper]{report}

\input{template/main.tex}

\title{\BA{Title in Progress...}}
\author{Boris Andrews}
\affil{Mathematical Institute, University of Oxford}
\date{\today}


\begin{document}
    \pagenumbering{gobble}
    \maketitle
    
    
    \begin{abstract}
        Magnetic confinement reactors---in particular tokamaks---offer one of the most promising options for achieving practical nuclear fusion, with the potential to provide virtually limitless, clean energy. The theoretical and numerical modeling of tokamak plasmas is simultaneously an essential component of effective reactor design, and a great research barrier. Tokamak operational conditions exhibit comparatively low Knudsen numbers. Kinetic effects, including kinetic waves and instabilities, Landau damping, bump-on-tail instabilities and more, are therefore highly influential in tokamak plasma dynamics. Purely fluid models are inherently incapable of capturing these effects, whereas the high dimensionality in purely kinetic models render them practically intractable for most relevant purposes.

        We consider a $\delta\!f$ decomposition model, with a macroscopic fluid background and microscopic kinetic correction, both fully coupled to each other. A similar manner of discretization is proposed to that used in the recent \texttt{STRUPHY} code \cite{Holderied_Possanner_Wang_2021, Holderied_2022, Li_et_al_2023} with a finite-element model for the background and a pseudo-particle/PiC model for the correction.

        The fluid background satisfies the full, non-linear, resistive, compressible, Hall MHD equations. \cite{Laakmann_Hu_Farrell_2022} introduces finite-element(-in-space) implicit timesteppers for the incompressible analogue to this system with structure-preserving (SP) properties in the ideal case, alongside parameter-robust preconditioners. We show that these timesteppers can derive from a finite-element-in-time (FET) (and finite-element-in-space) interpretation. The benefits of this reformulation are discussed, including the derivation of timesteppers that are higher order in time, and the quantifiable dissipative SP properties in the non-ideal, resistive case.
        
        We discuss possible options for extending this FET approach to timesteppers for the compressible case.

        The kinetic corrections satisfy linearized Boltzmann equations. Using a Lénard--Bernstein collision operator, these take Fokker--Planck-like forms \cite{Fokker_1914, Planck_1917} wherein pseudo-particles in the numerical model obey the neoclassical transport equations, with particle-independent Brownian drift terms. This offers a rigorous methodology for incorporating collisions into the particle transport model, without coupling the equations of motions for each particle.
        
        Works by Chen, Chacón et al. \cite{Chen_Chacón_Barnes_2011, Chacón_Chen_Barnes_2013, Chen_Chacón_2014, Chen_Chacón_2015} have developed structure-preserving particle pushers for neoclassical transport in the Vlasov equations, derived from Crank--Nicolson integrators. We show these too can can derive from a FET interpretation, similarly offering potential extensions to higher-order-in-time particle pushers. The FET formulation is used also to consider how the stochastic drift terms can be incorporated into the pushers. Stochastic gyrokinetic expansions are also discussed.

        Different options for the numerical implementation of these schemes are considered.

        Due to the efficacy of FET in the development of SP timesteppers for both the fluid and kinetic component, we hope this approach will prove effective in the future for developing SP timesteppers for the full hybrid model. We hope this will give us the opportunity to incorporate previously inaccessible kinetic effects into the highly effective, modern, finite-element MHD models.
    \end{abstract}
    
    
    \newpage
    \tableofcontents
    
    
    \newpage
    \pagenumbering{arabic}
    %\linenumbers\renewcommand\thelinenumber{\color{black!50}\arabic{linenumber}}
            \input{0 - introduction/main.tex}
        \part{Research}
            \input{1 - low-noise PiC models/main.tex}
            \input{2 - kinetic component/main.tex}
            \input{3 - fluid component/main.tex}
            \input{4 - numerical implementation/main.tex}
        \part{Project Overview}
            \input{5 - research plan/main.tex}
            \input{6 - summary/main.tex}
    
    
    %\section{}
    \newpage
    \pagenumbering{gobble}
        \printbibliography


    \newpage
    \pagenumbering{roman}
    \appendix
        \part{Appendices}
            \input{8 - Hilbert complexes/main.tex}
            \input{9 - weak conservation proofs/main.tex}
\end{document}

        \part{Project Overview}
            \documentclass[12pt, a4paper]{report}

\input{template/main.tex}

\title{\BA{Title in Progress...}}
\author{Boris Andrews}
\affil{Mathematical Institute, University of Oxford}
\date{\today}


\begin{document}
    \pagenumbering{gobble}
    \maketitle
    
    
    \begin{abstract}
        Magnetic confinement reactors---in particular tokamaks---offer one of the most promising options for achieving practical nuclear fusion, with the potential to provide virtually limitless, clean energy. The theoretical and numerical modeling of tokamak plasmas is simultaneously an essential component of effective reactor design, and a great research barrier. Tokamak operational conditions exhibit comparatively low Knudsen numbers. Kinetic effects, including kinetic waves and instabilities, Landau damping, bump-on-tail instabilities and more, are therefore highly influential in tokamak plasma dynamics. Purely fluid models are inherently incapable of capturing these effects, whereas the high dimensionality in purely kinetic models render them practically intractable for most relevant purposes.

        We consider a $\delta\!f$ decomposition model, with a macroscopic fluid background and microscopic kinetic correction, both fully coupled to each other. A similar manner of discretization is proposed to that used in the recent \texttt{STRUPHY} code \cite{Holderied_Possanner_Wang_2021, Holderied_2022, Li_et_al_2023} with a finite-element model for the background and a pseudo-particle/PiC model for the correction.

        The fluid background satisfies the full, non-linear, resistive, compressible, Hall MHD equations. \cite{Laakmann_Hu_Farrell_2022} introduces finite-element(-in-space) implicit timesteppers for the incompressible analogue to this system with structure-preserving (SP) properties in the ideal case, alongside parameter-robust preconditioners. We show that these timesteppers can derive from a finite-element-in-time (FET) (and finite-element-in-space) interpretation. The benefits of this reformulation are discussed, including the derivation of timesteppers that are higher order in time, and the quantifiable dissipative SP properties in the non-ideal, resistive case.
        
        We discuss possible options for extending this FET approach to timesteppers for the compressible case.

        The kinetic corrections satisfy linearized Boltzmann equations. Using a Lénard--Bernstein collision operator, these take Fokker--Planck-like forms \cite{Fokker_1914, Planck_1917} wherein pseudo-particles in the numerical model obey the neoclassical transport equations, with particle-independent Brownian drift terms. This offers a rigorous methodology for incorporating collisions into the particle transport model, without coupling the equations of motions for each particle.
        
        Works by Chen, Chacón et al. \cite{Chen_Chacón_Barnes_2011, Chacón_Chen_Barnes_2013, Chen_Chacón_2014, Chen_Chacón_2015} have developed structure-preserving particle pushers for neoclassical transport in the Vlasov equations, derived from Crank--Nicolson integrators. We show these too can can derive from a FET interpretation, similarly offering potential extensions to higher-order-in-time particle pushers. The FET formulation is used also to consider how the stochastic drift terms can be incorporated into the pushers. Stochastic gyrokinetic expansions are also discussed.

        Different options for the numerical implementation of these schemes are considered.

        Due to the efficacy of FET in the development of SP timesteppers for both the fluid and kinetic component, we hope this approach will prove effective in the future for developing SP timesteppers for the full hybrid model. We hope this will give us the opportunity to incorporate previously inaccessible kinetic effects into the highly effective, modern, finite-element MHD models.
    \end{abstract}
    
    
    \newpage
    \tableofcontents
    
    
    \newpage
    \pagenumbering{arabic}
    %\linenumbers\renewcommand\thelinenumber{\color{black!50}\arabic{linenumber}}
            \input{0 - introduction/main.tex}
        \part{Research}
            \input{1 - low-noise PiC models/main.tex}
            \input{2 - kinetic component/main.tex}
            \input{3 - fluid component/main.tex}
            \input{4 - numerical implementation/main.tex}
        \part{Project Overview}
            \input{5 - research plan/main.tex}
            \input{6 - summary/main.tex}
    
    
    %\section{}
    \newpage
    \pagenumbering{gobble}
        \printbibliography


    \newpage
    \pagenumbering{roman}
    \appendix
        \part{Appendices}
            \input{8 - Hilbert complexes/main.tex}
            \input{9 - weak conservation proofs/main.tex}
\end{document}

            \documentclass[12pt, a4paper]{report}

\input{template/main.tex}

\title{\BA{Title in Progress...}}
\author{Boris Andrews}
\affil{Mathematical Institute, University of Oxford}
\date{\today}


\begin{document}
    \pagenumbering{gobble}
    \maketitle
    
    
    \begin{abstract}
        Magnetic confinement reactors---in particular tokamaks---offer one of the most promising options for achieving practical nuclear fusion, with the potential to provide virtually limitless, clean energy. The theoretical and numerical modeling of tokamak plasmas is simultaneously an essential component of effective reactor design, and a great research barrier. Tokamak operational conditions exhibit comparatively low Knudsen numbers. Kinetic effects, including kinetic waves and instabilities, Landau damping, bump-on-tail instabilities and more, are therefore highly influential in tokamak plasma dynamics. Purely fluid models are inherently incapable of capturing these effects, whereas the high dimensionality in purely kinetic models render them practically intractable for most relevant purposes.

        We consider a $\delta\!f$ decomposition model, with a macroscopic fluid background and microscopic kinetic correction, both fully coupled to each other. A similar manner of discretization is proposed to that used in the recent \texttt{STRUPHY} code \cite{Holderied_Possanner_Wang_2021, Holderied_2022, Li_et_al_2023} with a finite-element model for the background and a pseudo-particle/PiC model for the correction.

        The fluid background satisfies the full, non-linear, resistive, compressible, Hall MHD equations. \cite{Laakmann_Hu_Farrell_2022} introduces finite-element(-in-space) implicit timesteppers for the incompressible analogue to this system with structure-preserving (SP) properties in the ideal case, alongside parameter-robust preconditioners. We show that these timesteppers can derive from a finite-element-in-time (FET) (and finite-element-in-space) interpretation. The benefits of this reformulation are discussed, including the derivation of timesteppers that are higher order in time, and the quantifiable dissipative SP properties in the non-ideal, resistive case.
        
        We discuss possible options for extending this FET approach to timesteppers for the compressible case.

        The kinetic corrections satisfy linearized Boltzmann equations. Using a Lénard--Bernstein collision operator, these take Fokker--Planck-like forms \cite{Fokker_1914, Planck_1917} wherein pseudo-particles in the numerical model obey the neoclassical transport equations, with particle-independent Brownian drift terms. This offers a rigorous methodology for incorporating collisions into the particle transport model, without coupling the equations of motions for each particle.
        
        Works by Chen, Chacón et al. \cite{Chen_Chacón_Barnes_2011, Chacón_Chen_Barnes_2013, Chen_Chacón_2014, Chen_Chacón_2015} have developed structure-preserving particle pushers for neoclassical transport in the Vlasov equations, derived from Crank--Nicolson integrators. We show these too can can derive from a FET interpretation, similarly offering potential extensions to higher-order-in-time particle pushers. The FET formulation is used also to consider how the stochastic drift terms can be incorporated into the pushers. Stochastic gyrokinetic expansions are also discussed.

        Different options for the numerical implementation of these schemes are considered.

        Due to the efficacy of FET in the development of SP timesteppers for both the fluid and kinetic component, we hope this approach will prove effective in the future for developing SP timesteppers for the full hybrid model. We hope this will give us the opportunity to incorporate previously inaccessible kinetic effects into the highly effective, modern, finite-element MHD models.
    \end{abstract}
    
    
    \newpage
    \tableofcontents
    
    
    \newpage
    \pagenumbering{arabic}
    %\linenumbers\renewcommand\thelinenumber{\color{black!50}\arabic{linenumber}}
            \input{0 - introduction/main.tex}
        \part{Research}
            \input{1 - low-noise PiC models/main.tex}
            \input{2 - kinetic component/main.tex}
            \input{3 - fluid component/main.tex}
            \input{4 - numerical implementation/main.tex}
        \part{Project Overview}
            \input{5 - research plan/main.tex}
            \input{6 - summary/main.tex}
    
    
    %\section{}
    \newpage
    \pagenumbering{gobble}
        \printbibliography


    \newpage
    \pagenumbering{roman}
    \appendix
        \part{Appendices}
            \input{8 - Hilbert complexes/main.tex}
            \input{9 - weak conservation proofs/main.tex}
\end{document}

    
    
    %\section{}
    \newpage
    \pagenumbering{gobble}
        \printbibliography


    \newpage
    \pagenumbering{roman}
    \appendix
        \part{Appendices}
            \documentclass[12pt, a4paper]{report}

\input{template/main.tex}

\title{\BA{Title in Progress...}}
\author{Boris Andrews}
\affil{Mathematical Institute, University of Oxford}
\date{\today}


\begin{document}
    \pagenumbering{gobble}
    \maketitle
    
    
    \begin{abstract}
        Magnetic confinement reactors---in particular tokamaks---offer one of the most promising options for achieving practical nuclear fusion, with the potential to provide virtually limitless, clean energy. The theoretical and numerical modeling of tokamak plasmas is simultaneously an essential component of effective reactor design, and a great research barrier. Tokamak operational conditions exhibit comparatively low Knudsen numbers. Kinetic effects, including kinetic waves and instabilities, Landau damping, bump-on-tail instabilities and more, are therefore highly influential in tokamak plasma dynamics. Purely fluid models are inherently incapable of capturing these effects, whereas the high dimensionality in purely kinetic models render them practically intractable for most relevant purposes.

        We consider a $\delta\!f$ decomposition model, with a macroscopic fluid background and microscopic kinetic correction, both fully coupled to each other. A similar manner of discretization is proposed to that used in the recent \texttt{STRUPHY} code \cite{Holderied_Possanner_Wang_2021, Holderied_2022, Li_et_al_2023} with a finite-element model for the background and a pseudo-particle/PiC model for the correction.

        The fluid background satisfies the full, non-linear, resistive, compressible, Hall MHD equations. \cite{Laakmann_Hu_Farrell_2022} introduces finite-element(-in-space) implicit timesteppers for the incompressible analogue to this system with structure-preserving (SP) properties in the ideal case, alongside parameter-robust preconditioners. We show that these timesteppers can derive from a finite-element-in-time (FET) (and finite-element-in-space) interpretation. The benefits of this reformulation are discussed, including the derivation of timesteppers that are higher order in time, and the quantifiable dissipative SP properties in the non-ideal, resistive case.
        
        We discuss possible options for extending this FET approach to timesteppers for the compressible case.

        The kinetic corrections satisfy linearized Boltzmann equations. Using a Lénard--Bernstein collision operator, these take Fokker--Planck-like forms \cite{Fokker_1914, Planck_1917} wherein pseudo-particles in the numerical model obey the neoclassical transport equations, with particle-independent Brownian drift terms. This offers a rigorous methodology for incorporating collisions into the particle transport model, without coupling the equations of motions for each particle.
        
        Works by Chen, Chacón et al. \cite{Chen_Chacón_Barnes_2011, Chacón_Chen_Barnes_2013, Chen_Chacón_2014, Chen_Chacón_2015} have developed structure-preserving particle pushers for neoclassical transport in the Vlasov equations, derived from Crank--Nicolson integrators. We show these too can can derive from a FET interpretation, similarly offering potential extensions to higher-order-in-time particle pushers. The FET formulation is used also to consider how the stochastic drift terms can be incorporated into the pushers. Stochastic gyrokinetic expansions are also discussed.

        Different options for the numerical implementation of these schemes are considered.

        Due to the efficacy of FET in the development of SP timesteppers for both the fluid and kinetic component, we hope this approach will prove effective in the future for developing SP timesteppers for the full hybrid model. We hope this will give us the opportunity to incorporate previously inaccessible kinetic effects into the highly effective, modern, finite-element MHD models.
    \end{abstract}
    
    
    \newpage
    \tableofcontents
    
    
    \newpage
    \pagenumbering{arabic}
    %\linenumbers\renewcommand\thelinenumber{\color{black!50}\arabic{linenumber}}
            \input{0 - introduction/main.tex}
        \part{Research}
            \input{1 - low-noise PiC models/main.tex}
            \input{2 - kinetic component/main.tex}
            \input{3 - fluid component/main.tex}
            \input{4 - numerical implementation/main.tex}
        \part{Project Overview}
            \input{5 - research plan/main.tex}
            \input{6 - summary/main.tex}
    
    
    %\section{}
    \newpage
    \pagenumbering{gobble}
        \printbibliography


    \newpage
    \pagenumbering{roman}
    \appendix
        \part{Appendices}
            \input{8 - Hilbert complexes/main.tex}
            \input{9 - weak conservation proofs/main.tex}
\end{document}

            \documentclass[12pt, a4paper]{report}

\input{template/main.tex}

\title{\BA{Title in Progress...}}
\author{Boris Andrews}
\affil{Mathematical Institute, University of Oxford}
\date{\today}


\begin{document}
    \pagenumbering{gobble}
    \maketitle
    
    
    \begin{abstract}
        Magnetic confinement reactors---in particular tokamaks---offer one of the most promising options for achieving practical nuclear fusion, with the potential to provide virtually limitless, clean energy. The theoretical and numerical modeling of tokamak plasmas is simultaneously an essential component of effective reactor design, and a great research barrier. Tokamak operational conditions exhibit comparatively low Knudsen numbers. Kinetic effects, including kinetic waves and instabilities, Landau damping, bump-on-tail instabilities and more, are therefore highly influential in tokamak plasma dynamics. Purely fluid models are inherently incapable of capturing these effects, whereas the high dimensionality in purely kinetic models render them practically intractable for most relevant purposes.

        We consider a $\delta\!f$ decomposition model, with a macroscopic fluid background and microscopic kinetic correction, both fully coupled to each other. A similar manner of discretization is proposed to that used in the recent \texttt{STRUPHY} code \cite{Holderied_Possanner_Wang_2021, Holderied_2022, Li_et_al_2023} with a finite-element model for the background and a pseudo-particle/PiC model for the correction.

        The fluid background satisfies the full, non-linear, resistive, compressible, Hall MHD equations. \cite{Laakmann_Hu_Farrell_2022} introduces finite-element(-in-space) implicit timesteppers for the incompressible analogue to this system with structure-preserving (SP) properties in the ideal case, alongside parameter-robust preconditioners. We show that these timesteppers can derive from a finite-element-in-time (FET) (and finite-element-in-space) interpretation. The benefits of this reformulation are discussed, including the derivation of timesteppers that are higher order in time, and the quantifiable dissipative SP properties in the non-ideal, resistive case.
        
        We discuss possible options for extending this FET approach to timesteppers for the compressible case.

        The kinetic corrections satisfy linearized Boltzmann equations. Using a Lénard--Bernstein collision operator, these take Fokker--Planck-like forms \cite{Fokker_1914, Planck_1917} wherein pseudo-particles in the numerical model obey the neoclassical transport equations, with particle-independent Brownian drift terms. This offers a rigorous methodology for incorporating collisions into the particle transport model, without coupling the equations of motions for each particle.
        
        Works by Chen, Chacón et al. \cite{Chen_Chacón_Barnes_2011, Chacón_Chen_Barnes_2013, Chen_Chacón_2014, Chen_Chacón_2015} have developed structure-preserving particle pushers for neoclassical transport in the Vlasov equations, derived from Crank--Nicolson integrators. We show these too can can derive from a FET interpretation, similarly offering potential extensions to higher-order-in-time particle pushers. The FET formulation is used also to consider how the stochastic drift terms can be incorporated into the pushers. Stochastic gyrokinetic expansions are also discussed.

        Different options for the numerical implementation of these schemes are considered.

        Due to the efficacy of FET in the development of SP timesteppers for both the fluid and kinetic component, we hope this approach will prove effective in the future for developing SP timesteppers for the full hybrid model. We hope this will give us the opportunity to incorporate previously inaccessible kinetic effects into the highly effective, modern, finite-element MHD models.
    \end{abstract}
    
    
    \newpage
    \tableofcontents
    
    
    \newpage
    \pagenumbering{arabic}
    %\linenumbers\renewcommand\thelinenumber{\color{black!50}\arabic{linenumber}}
            \input{0 - introduction/main.tex}
        \part{Research}
            \input{1 - low-noise PiC models/main.tex}
            \input{2 - kinetic component/main.tex}
            \input{3 - fluid component/main.tex}
            \input{4 - numerical implementation/main.tex}
        \part{Project Overview}
            \input{5 - research plan/main.tex}
            \input{6 - summary/main.tex}
    
    
    %\section{}
    \newpage
    \pagenumbering{gobble}
        \printbibliography


    \newpage
    \pagenumbering{roman}
    \appendix
        \part{Appendices}
            \input{8 - Hilbert complexes/main.tex}
            \input{9 - weak conservation proofs/main.tex}
\end{document}

\end{document}

            \documentclass[12pt, a4paper]{report}

\documentclass[12pt, a4paper]{report}

\input{template/main.tex}

\title{\BA{Title in Progress...}}
\author{Boris Andrews}
\affil{Mathematical Institute, University of Oxford}
\date{\today}


\begin{document}
    \pagenumbering{gobble}
    \maketitle
    
    
    \begin{abstract}
        Magnetic confinement reactors---in particular tokamaks---offer one of the most promising options for achieving practical nuclear fusion, with the potential to provide virtually limitless, clean energy. The theoretical and numerical modeling of tokamak plasmas is simultaneously an essential component of effective reactor design, and a great research barrier. Tokamak operational conditions exhibit comparatively low Knudsen numbers. Kinetic effects, including kinetic waves and instabilities, Landau damping, bump-on-tail instabilities and more, are therefore highly influential in tokamak plasma dynamics. Purely fluid models are inherently incapable of capturing these effects, whereas the high dimensionality in purely kinetic models render them practically intractable for most relevant purposes.

        We consider a $\delta\!f$ decomposition model, with a macroscopic fluid background and microscopic kinetic correction, both fully coupled to each other. A similar manner of discretization is proposed to that used in the recent \texttt{STRUPHY} code \cite{Holderied_Possanner_Wang_2021, Holderied_2022, Li_et_al_2023} with a finite-element model for the background and a pseudo-particle/PiC model for the correction.

        The fluid background satisfies the full, non-linear, resistive, compressible, Hall MHD equations. \cite{Laakmann_Hu_Farrell_2022} introduces finite-element(-in-space) implicit timesteppers for the incompressible analogue to this system with structure-preserving (SP) properties in the ideal case, alongside parameter-robust preconditioners. We show that these timesteppers can derive from a finite-element-in-time (FET) (and finite-element-in-space) interpretation. The benefits of this reformulation are discussed, including the derivation of timesteppers that are higher order in time, and the quantifiable dissipative SP properties in the non-ideal, resistive case.
        
        We discuss possible options for extending this FET approach to timesteppers for the compressible case.

        The kinetic corrections satisfy linearized Boltzmann equations. Using a Lénard--Bernstein collision operator, these take Fokker--Planck-like forms \cite{Fokker_1914, Planck_1917} wherein pseudo-particles in the numerical model obey the neoclassical transport equations, with particle-independent Brownian drift terms. This offers a rigorous methodology for incorporating collisions into the particle transport model, without coupling the equations of motions for each particle.
        
        Works by Chen, Chacón et al. \cite{Chen_Chacón_Barnes_2011, Chacón_Chen_Barnes_2013, Chen_Chacón_2014, Chen_Chacón_2015} have developed structure-preserving particle pushers for neoclassical transport in the Vlasov equations, derived from Crank--Nicolson integrators. We show these too can can derive from a FET interpretation, similarly offering potential extensions to higher-order-in-time particle pushers. The FET formulation is used also to consider how the stochastic drift terms can be incorporated into the pushers. Stochastic gyrokinetic expansions are also discussed.

        Different options for the numerical implementation of these schemes are considered.

        Due to the efficacy of FET in the development of SP timesteppers for both the fluid and kinetic component, we hope this approach will prove effective in the future for developing SP timesteppers for the full hybrid model. We hope this will give us the opportunity to incorporate previously inaccessible kinetic effects into the highly effective, modern, finite-element MHD models.
    \end{abstract}
    
    
    \newpage
    \tableofcontents
    
    
    \newpage
    \pagenumbering{arabic}
    %\linenumbers\renewcommand\thelinenumber{\color{black!50}\arabic{linenumber}}
            \input{0 - introduction/main.tex}
        \part{Research}
            \input{1 - low-noise PiC models/main.tex}
            \input{2 - kinetic component/main.tex}
            \input{3 - fluid component/main.tex}
            \input{4 - numerical implementation/main.tex}
        \part{Project Overview}
            \input{5 - research plan/main.tex}
            \input{6 - summary/main.tex}
    
    
    %\section{}
    \newpage
    \pagenumbering{gobble}
        \printbibliography


    \newpage
    \pagenumbering{roman}
    \appendix
        \part{Appendices}
            \input{8 - Hilbert complexes/main.tex}
            \input{9 - weak conservation proofs/main.tex}
\end{document}


\title{\BA{Title in Progress...}}
\author{Boris Andrews}
\affil{Mathematical Institute, University of Oxford}
\date{\today}


\begin{document}
    \pagenumbering{gobble}
    \maketitle
    
    
    \begin{abstract}
        Magnetic confinement reactors---in particular tokamaks---offer one of the most promising options for achieving practical nuclear fusion, with the potential to provide virtually limitless, clean energy. The theoretical and numerical modeling of tokamak plasmas is simultaneously an essential component of effective reactor design, and a great research barrier. Tokamak operational conditions exhibit comparatively low Knudsen numbers. Kinetic effects, including kinetic waves and instabilities, Landau damping, bump-on-tail instabilities and more, are therefore highly influential in tokamak plasma dynamics. Purely fluid models are inherently incapable of capturing these effects, whereas the high dimensionality in purely kinetic models render them practically intractable for most relevant purposes.

        We consider a $\delta\!f$ decomposition model, with a macroscopic fluid background and microscopic kinetic correction, both fully coupled to each other. A similar manner of discretization is proposed to that used in the recent \texttt{STRUPHY} code \cite{Holderied_Possanner_Wang_2021, Holderied_2022, Li_et_al_2023} with a finite-element model for the background and a pseudo-particle/PiC model for the correction.

        The fluid background satisfies the full, non-linear, resistive, compressible, Hall MHD equations. \cite{Laakmann_Hu_Farrell_2022} introduces finite-element(-in-space) implicit timesteppers for the incompressible analogue to this system with structure-preserving (SP) properties in the ideal case, alongside parameter-robust preconditioners. We show that these timesteppers can derive from a finite-element-in-time (FET) (and finite-element-in-space) interpretation. The benefits of this reformulation are discussed, including the derivation of timesteppers that are higher order in time, and the quantifiable dissipative SP properties in the non-ideal, resistive case.
        
        We discuss possible options for extending this FET approach to timesteppers for the compressible case.

        The kinetic corrections satisfy linearized Boltzmann equations. Using a Lénard--Bernstein collision operator, these take Fokker--Planck-like forms \cite{Fokker_1914, Planck_1917} wherein pseudo-particles in the numerical model obey the neoclassical transport equations, with particle-independent Brownian drift terms. This offers a rigorous methodology for incorporating collisions into the particle transport model, without coupling the equations of motions for each particle.
        
        Works by Chen, Chacón et al. \cite{Chen_Chacón_Barnes_2011, Chacón_Chen_Barnes_2013, Chen_Chacón_2014, Chen_Chacón_2015} have developed structure-preserving particle pushers for neoclassical transport in the Vlasov equations, derived from Crank--Nicolson integrators. We show these too can can derive from a FET interpretation, similarly offering potential extensions to higher-order-in-time particle pushers. The FET formulation is used also to consider how the stochastic drift terms can be incorporated into the pushers. Stochastic gyrokinetic expansions are also discussed.

        Different options for the numerical implementation of these schemes are considered.

        Due to the efficacy of FET in the development of SP timesteppers for both the fluid and kinetic component, we hope this approach will prove effective in the future for developing SP timesteppers for the full hybrid model. We hope this will give us the opportunity to incorporate previously inaccessible kinetic effects into the highly effective, modern, finite-element MHD models.
    \end{abstract}
    
    
    \newpage
    \tableofcontents
    
    
    \newpage
    \pagenumbering{arabic}
    %\linenumbers\renewcommand\thelinenumber{\color{black!50}\arabic{linenumber}}
            \documentclass[12pt, a4paper]{report}

\input{template/main.tex}

\title{\BA{Title in Progress...}}
\author{Boris Andrews}
\affil{Mathematical Institute, University of Oxford}
\date{\today}


\begin{document}
    \pagenumbering{gobble}
    \maketitle
    
    
    \begin{abstract}
        Magnetic confinement reactors---in particular tokamaks---offer one of the most promising options for achieving practical nuclear fusion, with the potential to provide virtually limitless, clean energy. The theoretical and numerical modeling of tokamak plasmas is simultaneously an essential component of effective reactor design, and a great research barrier. Tokamak operational conditions exhibit comparatively low Knudsen numbers. Kinetic effects, including kinetic waves and instabilities, Landau damping, bump-on-tail instabilities and more, are therefore highly influential in tokamak plasma dynamics. Purely fluid models are inherently incapable of capturing these effects, whereas the high dimensionality in purely kinetic models render them practically intractable for most relevant purposes.

        We consider a $\delta\!f$ decomposition model, with a macroscopic fluid background and microscopic kinetic correction, both fully coupled to each other. A similar manner of discretization is proposed to that used in the recent \texttt{STRUPHY} code \cite{Holderied_Possanner_Wang_2021, Holderied_2022, Li_et_al_2023} with a finite-element model for the background and a pseudo-particle/PiC model for the correction.

        The fluid background satisfies the full, non-linear, resistive, compressible, Hall MHD equations. \cite{Laakmann_Hu_Farrell_2022} introduces finite-element(-in-space) implicit timesteppers for the incompressible analogue to this system with structure-preserving (SP) properties in the ideal case, alongside parameter-robust preconditioners. We show that these timesteppers can derive from a finite-element-in-time (FET) (and finite-element-in-space) interpretation. The benefits of this reformulation are discussed, including the derivation of timesteppers that are higher order in time, and the quantifiable dissipative SP properties in the non-ideal, resistive case.
        
        We discuss possible options for extending this FET approach to timesteppers for the compressible case.

        The kinetic corrections satisfy linearized Boltzmann equations. Using a Lénard--Bernstein collision operator, these take Fokker--Planck-like forms \cite{Fokker_1914, Planck_1917} wherein pseudo-particles in the numerical model obey the neoclassical transport equations, with particle-independent Brownian drift terms. This offers a rigorous methodology for incorporating collisions into the particle transport model, without coupling the equations of motions for each particle.
        
        Works by Chen, Chacón et al. \cite{Chen_Chacón_Barnes_2011, Chacón_Chen_Barnes_2013, Chen_Chacón_2014, Chen_Chacón_2015} have developed structure-preserving particle pushers for neoclassical transport in the Vlasov equations, derived from Crank--Nicolson integrators. We show these too can can derive from a FET interpretation, similarly offering potential extensions to higher-order-in-time particle pushers. The FET formulation is used also to consider how the stochastic drift terms can be incorporated into the pushers. Stochastic gyrokinetic expansions are also discussed.

        Different options for the numerical implementation of these schemes are considered.

        Due to the efficacy of FET in the development of SP timesteppers for both the fluid and kinetic component, we hope this approach will prove effective in the future for developing SP timesteppers for the full hybrid model. We hope this will give us the opportunity to incorporate previously inaccessible kinetic effects into the highly effective, modern, finite-element MHD models.
    \end{abstract}
    
    
    \newpage
    \tableofcontents
    
    
    \newpage
    \pagenumbering{arabic}
    %\linenumbers\renewcommand\thelinenumber{\color{black!50}\arabic{linenumber}}
            \input{0 - introduction/main.tex}
        \part{Research}
            \input{1 - low-noise PiC models/main.tex}
            \input{2 - kinetic component/main.tex}
            \input{3 - fluid component/main.tex}
            \input{4 - numerical implementation/main.tex}
        \part{Project Overview}
            \input{5 - research plan/main.tex}
            \input{6 - summary/main.tex}
    
    
    %\section{}
    \newpage
    \pagenumbering{gobble}
        \printbibliography


    \newpage
    \pagenumbering{roman}
    \appendix
        \part{Appendices}
            \input{8 - Hilbert complexes/main.tex}
            \input{9 - weak conservation proofs/main.tex}
\end{document}

        \part{Research}
            \documentclass[12pt, a4paper]{report}

\input{template/main.tex}

\title{\BA{Title in Progress...}}
\author{Boris Andrews}
\affil{Mathematical Institute, University of Oxford}
\date{\today}


\begin{document}
    \pagenumbering{gobble}
    \maketitle
    
    
    \begin{abstract}
        Magnetic confinement reactors---in particular tokamaks---offer one of the most promising options for achieving practical nuclear fusion, with the potential to provide virtually limitless, clean energy. The theoretical and numerical modeling of tokamak plasmas is simultaneously an essential component of effective reactor design, and a great research barrier. Tokamak operational conditions exhibit comparatively low Knudsen numbers. Kinetic effects, including kinetic waves and instabilities, Landau damping, bump-on-tail instabilities and more, are therefore highly influential in tokamak plasma dynamics. Purely fluid models are inherently incapable of capturing these effects, whereas the high dimensionality in purely kinetic models render them practically intractable for most relevant purposes.

        We consider a $\delta\!f$ decomposition model, with a macroscopic fluid background and microscopic kinetic correction, both fully coupled to each other. A similar manner of discretization is proposed to that used in the recent \texttt{STRUPHY} code \cite{Holderied_Possanner_Wang_2021, Holderied_2022, Li_et_al_2023} with a finite-element model for the background and a pseudo-particle/PiC model for the correction.

        The fluid background satisfies the full, non-linear, resistive, compressible, Hall MHD equations. \cite{Laakmann_Hu_Farrell_2022} introduces finite-element(-in-space) implicit timesteppers for the incompressible analogue to this system with structure-preserving (SP) properties in the ideal case, alongside parameter-robust preconditioners. We show that these timesteppers can derive from a finite-element-in-time (FET) (and finite-element-in-space) interpretation. The benefits of this reformulation are discussed, including the derivation of timesteppers that are higher order in time, and the quantifiable dissipative SP properties in the non-ideal, resistive case.
        
        We discuss possible options for extending this FET approach to timesteppers for the compressible case.

        The kinetic corrections satisfy linearized Boltzmann equations. Using a Lénard--Bernstein collision operator, these take Fokker--Planck-like forms \cite{Fokker_1914, Planck_1917} wherein pseudo-particles in the numerical model obey the neoclassical transport equations, with particle-independent Brownian drift terms. This offers a rigorous methodology for incorporating collisions into the particle transport model, without coupling the equations of motions for each particle.
        
        Works by Chen, Chacón et al. \cite{Chen_Chacón_Barnes_2011, Chacón_Chen_Barnes_2013, Chen_Chacón_2014, Chen_Chacón_2015} have developed structure-preserving particle pushers for neoclassical transport in the Vlasov equations, derived from Crank--Nicolson integrators. We show these too can can derive from a FET interpretation, similarly offering potential extensions to higher-order-in-time particle pushers. The FET formulation is used also to consider how the stochastic drift terms can be incorporated into the pushers. Stochastic gyrokinetic expansions are also discussed.

        Different options for the numerical implementation of these schemes are considered.

        Due to the efficacy of FET in the development of SP timesteppers for both the fluid and kinetic component, we hope this approach will prove effective in the future for developing SP timesteppers for the full hybrid model. We hope this will give us the opportunity to incorporate previously inaccessible kinetic effects into the highly effective, modern, finite-element MHD models.
    \end{abstract}
    
    
    \newpage
    \tableofcontents
    
    
    \newpage
    \pagenumbering{arabic}
    %\linenumbers\renewcommand\thelinenumber{\color{black!50}\arabic{linenumber}}
            \input{0 - introduction/main.tex}
        \part{Research}
            \input{1 - low-noise PiC models/main.tex}
            \input{2 - kinetic component/main.tex}
            \input{3 - fluid component/main.tex}
            \input{4 - numerical implementation/main.tex}
        \part{Project Overview}
            \input{5 - research plan/main.tex}
            \input{6 - summary/main.tex}
    
    
    %\section{}
    \newpage
    \pagenumbering{gobble}
        \printbibliography


    \newpage
    \pagenumbering{roman}
    \appendix
        \part{Appendices}
            \input{8 - Hilbert complexes/main.tex}
            \input{9 - weak conservation proofs/main.tex}
\end{document}

            \documentclass[12pt, a4paper]{report}

\input{template/main.tex}

\title{\BA{Title in Progress...}}
\author{Boris Andrews}
\affil{Mathematical Institute, University of Oxford}
\date{\today}


\begin{document}
    \pagenumbering{gobble}
    \maketitle
    
    
    \begin{abstract}
        Magnetic confinement reactors---in particular tokamaks---offer one of the most promising options for achieving practical nuclear fusion, with the potential to provide virtually limitless, clean energy. The theoretical and numerical modeling of tokamak plasmas is simultaneously an essential component of effective reactor design, and a great research barrier. Tokamak operational conditions exhibit comparatively low Knudsen numbers. Kinetic effects, including kinetic waves and instabilities, Landau damping, bump-on-tail instabilities and more, are therefore highly influential in tokamak plasma dynamics. Purely fluid models are inherently incapable of capturing these effects, whereas the high dimensionality in purely kinetic models render them practically intractable for most relevant purposes.

        We consider a $\delta\!f$ decomposition model, with a macroscopic fluid background and microscopic kinetic correction, both fully coupled to each other. A similar manner of discretization is proposed to that used in the recent \texttt{STRUPHY} code \cite{Holderied_Possanner_Wang_2021, Holderied_2022, Li_et_al_2023} with a finite-element model for the background and a pseudo-particle/PiC model for the correction.

        The fluid background satisfies the full, non-linear, resistive, compressible, Hall MHD equations. \cite{Laakmann_Hu_Farrell_2022} introduces finite-element(-in-space) implicit timesteppers for the incompressible analogue to this system with structure-preserving (SP) properties in the ideal case, alongside parameter-robust preconditioners. We show that these timesteppers can derive from a finite-element-in-time (FET) (and finite-element-in-space) interpretation. The benefits of this reformulation are discussed, including the derivation of timesteppers that are higher order in time, and the quantifiable dissipative SP properties in the non-ideal, resistive case.
        
        We discuss possible options for extending this FET approach to timesteppers for the compressible case.

        The kinetic corrections satisfy linearized Boltzmann equations. Using a Lénard--Bernstein collision operator, these take Fokker--Planck-like forms \cite{Fokker_1914, Planck_1917} wherein pseudo-particles in the numerical model obey the neoclassical transport equations, with particle-independent Brownian drift terms. This offers a rigorous methodology for incorporating collisions into the particle transport model, without coupling the equations of motions for each particle.
        
        Works by Chen, Chacón et al. \cite{Chen_Chacón_Barnes_2011, Chacón_Chen_Barnes_2013, Chen_Chacón_2014, Chen_Chacón_2015} have developed structure-preserving particle pushers for neoclassical transport in the Vlasov equations, derived from Crank--Nicolson integrators. We show these too can can derive from a FET interpretation, similarly offering potential extensions to higher-order-in-time particle pushers. The FET formulation is used also to consider how the stochastic drift terms can be incorporated into the pushers. Stochastic gyrokinetic expansions are also discussed.

        Different options for the numerical implementation of these schemes are considered.

        Due to the efficacy of FET in the development of SP timesteppers for both the fluid and kinetic component, we hope this approach will prove effective in the future for developing SP timesteppers for the full hybrid model. We hope this will give us the opportunity to incorporate previously inaccessible kinetic effects into the highly effective, modern, finite-element MHD models.
    \end{abstract}
    
    
    \newpage
    \tableofcontents
    
    
    \newpage
    \pagenumbering{arabic}
    %\linenumbers\renewcommand\thelinenumber{\color{black!50}\arabic{linenumber}}
            \input{0 - introduction/main.tex}
        \part{Research}
            \input{1 - low-noise PiC models/main.tex}
            \input{2 - kinetic component/main.tex}
            \input{3 - fluid component/main.tex}
            \input{4 - numerical implementation/main.tex}
        \part{Project Overview}
            \input{5 - research plan/main.tex}
            \input{6 - summary/main.tex}
    
    
    %\section{}
    \newpage
    \pagenumbering{gobble}
        \printbibliography


    \newpage
    \pagenumbering{roman}
    \appendix
        \part{Appendices}
            \input{8 - Hilbert complexes/main.tex}
            \input{9 - weak conservation proofs/main.tex}
\end{document}

            \documentclass[12pt, a4paper]{report}

\input{template/main.tex}

\title{\BA{Title in Progress...}}
\author{Boris Andrews}
\affil{Mathematical Institute, University of Oxford}
\date{\today}


\begin{document}
    \pagenumbering{gobble}
    \maketitle
    
    
    \begin{abstract}
        Magnetic confinement reactors---in particular tokamaks---offer one of the most promising options for achieving practical nuclear fusion, with the potential to provide virtually limitless, clean energy. The theoretical and numerical modeling of tokamak plasmas is simultaneously an essential component of effective reactor design, and a great research barrier. Tokamak operational conditions exhibit comparatively low Knudsen numbers. Kinetic effects, including kinetic waves and instabilities, Landau damping, bump-on-tail instabilities and more, are therefore highly influential in tokamak plasma dynamics. Purely fluid models are inherently incapable of capturing these effects, whereas the high dimensionality in purely kinetic models render them practically intractable for most relevant purposes.

        We consider a $\delta\!f$ decomposition model, with a macroscopic fluid background and microscopic kinetic correction, both fully coupled to each other. A similar manner of discretization is proposed to that used in the recent \texttt{STRUPHY} code \cite{Holderied_Possanner_Wang_2021, Holderied_2022, Li_et_al_2023} with a finite-element model for the background and a pseudo-particle/PiC model for the correction.

        The fluid background satisfies the full, non-linear, resistive, compressible, Hall MHD equations. \cite{Laakmann_Hu_Farrell_2022} introduces finite-element(-in-space) implicit timesteppers for the incompressible analogue to this system with structure-preserving (SP) properties in the ideal case, alongside parameter-robust preconditioners. We show that these timesteppers can derive from a finite-element-in-time (FET) (and finite-element-in-space) interpretation. The benefits of this reformulation are discussed, including the derivation of timesteppers that are higher order in time, and the quantifiable dissipative SP properties in the non-ideal, resistive case.
        
        We discuss possible options for extending this FET approach to timesteppers for the compressible case.

        The kinetic corrections satisfy linearized Boltzmann equations. Using a Lénard--Bernstein collision operator, these take Fokker--Planck-like forms \cite{Fokker_1914, Planck_1917} wherein pseudo-particles in the numerical model obey the neoclassical transport equations, with particle-independent Brownian drift terms. This offers a rigorous methodology for incorporating collisions into the particle transport model, without coupling the equations of motions for each particle.
        
        Works by Chen, Chacón et al. \cite{Chen_Chacón_Barnes_2011, Chacón_Chen_Barnes_2013, Chen_Chacón_2014, Chen_Chacón_2015} have developed structure-preserving particle pushers for neoclassical transport in the Vlasov equations, derived from Crank--Nicolson integrators. We show these too can can derive from a FET interpretation, similarly offering potential extensions to higher-order-in-time particle pushers. The FET formulation is used also to consider how the stochastic drift terms can be incorporated into the pushers. Stochastic gyrokinetic expansions are also discussed.

        Different options for the numerical implementation of these schemes are considered.

        Due to the efficacy of FET in the development of SP timesteppers for both the fluid and kinetic component, we hope this approach will prove effective in the future for developing SP timesteppers for the full hybrid model. We hope this will give us the opportunity to incorporate previously inaccessible kinetic effects into the highly effective, modern, finite-element MHD models.
    \end{abstract}
    
    
    \newpage
    \tableofcontents
    
    
    \newpage
    \pagenumbering{arabic}
    %\linenumbers\renewcommand\thelinenumber{\color{black!50}\arabic{linenumber}}
            \input{0 - introduction/main.tex}
        \part{Research}
            \input{1 - low-noise PiC models/main.tex}
            \input{2 - kinetic component/main.tex}
            \input{3 - fluid component/main.tex}
            \input{4 - numerical implementation/main.tex}
        \part{Project Overview}
            \input{5 - research plan/main.tex}
            \input{6 - summary/main.tex}
    
    
    %\section{}
    \newpage
    \pagenumbering{gobble}
        \printbibliography


    \newpage
    \pagenumbering{roman}
    \appendix
        \part{Appendices}
            \input{8 - Hilbert complexes/main.tex}
            \input{9 - weak conservation proofs/main.tex}
\end{document}

            \documentclass[12pt, a4paper]{report}

\input{template/main.tex}

\title{\BA{Title in Progress...}}
\author{Boris Andrews}
\affil{Mathematical Institute, University of Oxford}
\date{\today}


\begin{document}
    \pagenumbering{gobble}
    \maketitle
    
    
    \begin{abstract}
        Magnetic confinement reactors---in particular tokamaks---offer one of the most promising options for achieving practical nuclear fusion, with the potential to provide virtually limitless, clean energy. The theoretical and numerical modeling of tokamak plasmas is simultaneously an essential component of effective reactor design, and a great research barrier. Tokamak operational conditions exhibit comparatively low Knudsen numbers. Kinetic effects, including kinetic waves and instabilities, Landau damping, bump-on-tail instabilities and more, are therefore highly influential in tokamak plasma dynamics. Purely fluid models are inherently incapable of capturing these effects, whereas the high dimensionality in purely kinetic models render them practically intractable for most relevant purposes.

        We consider a $\delta\!f$ decomposition model, with a macroscopic fluid background and microscopic kinetic correction, both fully coupled to each other. A similar manner of discretization is proposed to that used in the recent \texttt{STRUPHY} code \cite{Holderied_Possanner_Wang_2021, Holderied_2022, Li_et_al_2023} with a finite-element model for the background and a pseudo-particle/PiC model for the correction.

        The fluid background satisfies the full, non-linear, resistive, compressible, Hall MHD equations. \cite{Laakmann_Hu_Farrell_2022} introduces finite-element(-in-space) implicit timesteppers for the incompressible analogue to this system with structure-preserving (SP) properties in the ideal case, alongside parameter-robust preconditioners. We show that these timesteppers can derive from a finite-element-in-time (FET) (and finite-element-in-space) interpretation. The benefits of this reformulation are discussed, including the derivation of timesteppers that are higher order in time, and the quantifiable dissipative SP properties in the non-ideal, resistive case.
        
        We discuss possible options for extending this FET approach to timesteppers for the compressible case.

        The kinetic corrections satisfy linearized Boltzmann equations. Using a Lénard--Bernstein collision operator, these take Fokker--Planck-like forms \cite{Fokker_1914, Planck_1917} wherein pseudo-particles in the numerical model obey the neoclassical transport equations, with particle-independent Brownian drift terms. This offers a rigorous methodology for incorporating collisions into the particle transport model, without coupling the equations of motions for each particle.
        
        Works by Chen, Chacón et al. \cite{Chen_Chacón_Barnes_2011, Chacón_Chen_Barnes_2013, Chen_Chacón_2014, Chen_Chacón_2015} have developed structure-preserving particle pushers for neoclassical transport in the Vlasov equations, derived from Crank--Nicolson integrators. We show these too can can derive from a FET interpretation, similarly offering potential extensions to higher-order-in-time particle pushers. The FET formulation is used also to consider how the stochastic drift terms can be incorporated into the pushers. Stochastic gyrokinetic expansions are also discussed.

        Different options for the numerical implementation of these schemes are considered.

        Due to the efficacy of FET in the development of SP timesteppers for both the fluid and kinetic component, we hope this approach will prove effective in the future for developing SP timesteppers for the full hybrid model. We hope this will give us the opportunity to incorporate previously inaccessible kinetic effects into the highly effective, modern, finite-element MHD models.
    \end{abstract}
    
    
    \newpage
    \tableofcontents
    
    
    \newpage
    \pagenumbering{arabic}
    %\linenumbers\renewcommand\thelinenumber{\color{black!50}\arabic{linenumber}}
            \input{0 - introduction/main.tex}
        \part{Research}
            \input{1 - low-noise PiC models/main.tex}
            \input{2 - kinetic component/main.tex}
            \input{3 - fluid component/main.tex}
            \input{4 - numerical implementation/main.tex}
        \part{Project Overview}
            \input{5 - research plan/main.tex}
            \input{6 - summary/main.tex}
    
    
    %\section{}
    \newpage
    \pagenumbering{gobble}
        \printbibliography


    \newpage
    \pagenumbering{roman}
    \appendix
        \part{Appendices}
            \input{8 - Hilbert complexes/main.tex}
            \input{9 - weak conservation proofs/main.tex}
\end{document}

        \part{Project Overview}
            \documentclass[12pt, a4paper]{report}

\input{template/main.tex}

\title{\BA{Title in Progress...}}
\author{Boris Andrews}
\affil{Mathematical Institute, University of Oxford}
\date{\today}


\begin{document}
    \pagenumbering{gobble}
    \maketitle
    
    
    \begin{abstract}
        Magnetic confinement reactors---in particular tokamaks---offer one of the most promising options for achieving practical nuclear fusion, with the potential to provide virtually limitless, clean energy. The theoretical and numerical modeling of tokamak plasmas is simultaneously an essential component of effective reactor design, and a great research barrier. Tokamak operational conditions exhibit comparatively low Knudsen numbers. Kinetic effects, including kinetic waves and instabilities, Landau damping, bump-on-tail instabilities and more, are therefore highly influential in tokamak plasma dynamics. Purely fluid models are inherently incapable of capturing these effects, whereas the high dimensionality in purely kinetic models render them practically intractable for most relevant purposes.

        We consider a $\delta\!f$ decomposition model, with a macroscopic fluid background and microscopic kinetic correction, both fully coupled to each other. A similar manner of discretization is proposed to that used in the recent \texttt{STRUPHY} code \cite{Holderied_Possanner_Wang_2021, Holderied_2022, Li_et_al_2023} with a finite-element model for the background and a pseudo-particle/PiC model for the correction.

        The fluid background satisfies the full, non-linear, resistive, compressible, Hall MHD equations. \cite{Laakmann_Hu_Farrell_2022} introduces finite-element(-in-space) implicit timesteppers for the incompressible analogue to this system with structure-preserving (SP) properties in the ideal case, alongside parameter-robust preconditioners. We show that these timesteppers can derive from a finite-element-in-time (FET) (and finite-element-in-space) interpretation. The benefits of this reformulation are discussed, including the derivation of timesteppers that are higher order in time, and the quantifiable dissipative SP properties in the non-ideal, resistive case.
        
        We discuss possible options for extending this FET approach to timesteppers for the compressible case.

        The kinetic corrections satisfy linearized Boltzmann equations. Using a Lénard--Bernstein collision operator, these take Fokker--Planck-like forms \cite{Fokker_1914, Planck_1917} wherein pseudo-particles in the numerical model obey the neoclassical transport equations, with particle-independent Brownian drift terms. This offers a rigorous methodology for incorporating collisions into the particle transport model, without coupling the equations of motions for each particle.
        
        Works by Chen, Chacón et al. \cite{Chen_Chacón_Barnes_2011, Chacón_Chen_Barnes_2013, Chen_Chacón_2014, Chen_Chacón_2015} have developed structure-preserving particle pushers for neoclassical transport in the Vlasov equations, derived from Crank--Nicolson integrators. We show these too can can derive from a FET interpretation, similarly offering potential extensions to higher-order-in-time particle pushers. The FET formulation is used also to consider how the stochastic drift terms can be incorporated into the pushers. Stochastic gyrokinetic expansions are also discussed.

        Different options for the numerical implementation of these schemes are considered.

        Due to the efficacy of FET in the development of SP timesteppers for both the fluid and kinetic component, we hope this approach will prove effective in the future for developing SP timesteppers for the full hybrid model. We hope this will give us the opportunity to incorporate previously inaccessible kinetic effects into the highly effective, modern, finite-element MHD models.
    \end{abstract}
    
    
    \newpage
    \tableofcontents
    
    
    \newpage
    \pagenumbering{arabic}
    %\linenumbers\renewcommand\thelinenumber{\color{black!50}\arabic{linenumber}}
            \input{0 - introduction/main.tex}
        \part{Research}
            \input{1 - low-noise PiC models/main.tex}
            \input{2 - kinetic component/main.tex}
            \input{3 - fluid component/main.tex}
            \input{4 - numerical implementation/main.tex}
        \part{Project Overview}
            \input{5 - research plan/main.tex}
            \input{6 - summary/main.tex}
    
    
    %\section{}
    \newpage
    \pagenumbering{gobble}
        \printbibliography


    \newpage
    \pagenumbering{roman}
    \appendix
        \part{Appendices}
            \input{8 - Hilbert complexes/main.tex}
            \input{9 - weak conservation proofs/main.tex}
\end{document}

            \documentclass[12pt, a4paper]{report}

\input{template/main.tex}

\title{\BA{Title in Progress...}}
\author{Boris Andrews}
\affil{Mathematical Institute, University of Oxford}
\date{\today}


\begin{document}
    \pagenumbering{gobble}
    \maketitle
    
    
    \begin{abstract}
        Magnetic confinement reactors---in particular tokamaks---offer one of the most promising options for achieving practical nuclear fusion, with the potential to provide virtually limitless, clean energy. The theoretical and numerical modeling of tokamak plasmas is simultaneously an essential component of effective reactor design, and a great research barrier. Tokamak operational conditions exhibit comparatively low Knudsen numbers. Kinetic effects, including kinetic waves and instabilities, Landau damping, bump-on-tail instabilities and more, are therefore highly influential in tokamak plasma dynamics. Purely fluid models are inherently incapable of capturing these effects, whereas the high dimensionality in purely kinetic models render them practically intractable for most relevant purposes.

        We consider a $\delta\!f$ decomposition model, with a macroscopic fluid background and microscopic kinetic correction, both fully coupled to each other. A similar manner of discretization is proposed to that used in the recent \texttt{STRUPHY} code \cite{Holderied_Possanner_Wang_2021, Holderied_2022, Li_et_al_2023} with a finite-element model for the background and a pseudo-particle/PiC model for the correction.

        The fluid background satisfies the full, non-linear, resistive, compressible, Hall MHD equations. \cite{Laakmann_Hu_Farrell_2022} introduces finite-element(-in-space) implicit timesteppers for the incompressible analogue to this system with structure-preserving (SP) properties in the ideal case, alongside parameter-robust preconditioners. We show that these timesteppers can derive from a finite-element-in-time (FET) (and finite-element-in-space) interpretation. The benefits of this reformulation are discussed, including the derivation of timesteppers that are higher order in time, and the quantifiable dissipative SP properties in the non-ideal, resistive case.
        
        We discuss possible options for extending this FET approach to timesteppers for the compressible case.

        The kinetic corrections satisfy linearized Boltzmann equations. Using a Lénard--Bernstein collision operator, these take Fokker--Planck-like forms \cite{Fokker_1914, Planck_1917} wherein pseudo-particles in the numerical model obey the neoclassical transport equations, with particle-independent Brownian drift terms. This offers a rigorous methodology for incorporating collisions into the particle transport model, without coupling the equations of motions for each particle.
        
        Works by Chen, Chacón et al. \cite{Chen_Chacón_Barnes_2011, Chacón_Chen_Barnes_2013, Chen_Chacón_2014, Chen_Chacón_2015} have developed structure-preserving particle pushers for neoclassical transport in the Vlasov equations, derived from Crank--Nicolson integrators. We show these too can can derive from a FET interpretation, similarly offering potential extensions to higher-order-in-time particle pushers. The FET formulation is used also to consider how the stochastic drift terms can be incorporated into the pushers. Stochastic gyrokinetic expansions are also discussed.

        Different options for the numerical implementation of these schemes are considered.

        Due to the efficacy of FET in the development of SP timesteppers for both the fluid and kinetic component, we hope this approach will prove effective in the future for developing SP timesteppers for the full hybrid model. We hope this will give us the opportunity to incorporate previously inaccessible kinetic effects into the highly effective, modern, finite-element MHD models.
    \end{abstract}
    
    
    \newpage
    \tableofcontents
    
    
    \newpage
    \pagenumbering{arabic}
    %\linenumbers\renewcommand\thelinenumber{\color{black!50}\arabic{linenumber}}
            \input{0 - introduction/main.tex}
        \part{Research}
            \input{1 - low-noise PiC models/main.tex}
            \input{2 - kinetic component/main.tex}
            \input{3 - fluid component/main.tex}
            \input{4 - numerical implementation/main.tex}
        \part{Project Overview}
            \input{5 - research plan/main.tex}
            \input{6 - summary/main.tex}
    
    
    %\section{}
    \newpage
    \pagenumbering{gobble}
        \printbibliography


    \newpage
    \pagenumbering{roman}
    \appendix
        \part{Appendices}
            \input{8 - Hilbert complexes/main.tex}
            \input{9 - weak conservation proofs/main.tex}
\end{document}

    
    
    %\section{}
    \newpage
    \pagenumbering{gobble}
        \printbibliography


    \newpage
    \pagenumbering{roman}
    \appendix
        \part{Appendices}
            \documentclass[12pt, a4paper]{report}

\input{template/main.tex}

\title{\BA{Title in Progress...}}
\author{Boris Andrews}
\affil{Mathematical Institute, University of Oxford}
\date{\today}


\begin{document}
    \pagenumbering{gobble}
    \maketitle
    
    
    \begin{abstract}
        Magnetic confinement reactors---in particular tokamaks---offer one of the most promising options for achieving practical nuclear fusion, with the potential to provide virtually limitless, clean energy. The theoretical and numerical modeling of tokamak plasmas is simultaneously an essential component of effective reactor design, and a great research barrier. Tokamak operational conditions exhibit comparatively low Knudsen numbers. Kinetic effects, including kinetic waves and instabilities, Landau damping, bump-on-tail instabilities and more, are therefore highly influential in tokamak plasma dynamics. Purely fluid models are inherently incapable of capturing these effects, whereas the high dimensionality in purely kinetic models render them practically intractable for most relevant purposes.

        We consider a $\delta\!f$ decomposition model, with a macroscopic fluid background and microscopic kinetic correction, both fully coupled to each other. A similar manner of discretization is proposed to that used in the recent \texttt{STRUPHY} code \cite{Holderied_Possanner_Wang_2021, Holderied_2022, Li_et_al_2023} with a finite-element model for the background and a pseudo-particle/PiC model for the correction.

        The fluid background satisfies the full, non-linear, resistive, compressible, Hall MHD equations. \cite{Laakmann_Hu_Farrell_2022} introduces finite-element(-in-space) implicit timesteppers for the incompressible analogue to this system with structure-preserving (SP) properties in the ideal case, alongside parameter-robust preconditioners. We show that these timesteppers can derive from a finite-element-in-time (FET) (and finite-element-in-space) interpretation. The benefits of this reformulation are discussed, including the derivation of timesteppers that are higher order in time, and the quantifiable dissipative SP properties in the non-ideal, resistive case.
        
        We discuss possible options for extending this FET approach to timesteppers for the compressible case.

        The kinetic corrections satisfy linearized Boltzmann equations. Using a Lénard--Bernstein collision operator, these take Fokker--Planck-like forms \cite{Fokker_1914, Planck_1917} wherein pseudo-particles in the numerical model obey the neoclassical transport equations, with particle-independent Brownian drift terms. This offers a rigorous methodology for incorporating collisions into the particle transport model, without coupling the equations of motions for each particle.
        
        Works by Chen, Chacón et al. \cite{Chen_Chacón_Barnes_2011, Chacón_Chen_Barnes_2013, Chen_Chacón_2014, Chen_Chacón_2015} have developed structure-preserving particle pushers for neoclassical transport in the Vlasov equations, derived from Crank--Nicolson integrators. We show these too can can derive from a FET interpretation, similarly offering potential extensions to higher-order-in-time particle pushers. The FET formulation is used also to consider how the stochastic drift terms can be incorporated into the pushers. Stochastic gyrokinetic expansions are also discussed.

        Different options for the numerical implementation of these schemes are considered.

        Due to the efficacy of FET in the development of SP timesteppers for both the fluid and kinetic component, we hope this approach will prove effective in the future for developing SP timesteppers for the full hybrid model. We hope this will give us the opportunity to incorporate previously inaccessible kinetic effects into the highly effective, modern, finite-element MHD models.
    \end{abstract}
    
    
    \newpage
    \tableofcontents
    
    
    \newpage
    \pagenumbering{arabic}
    %\linenumbers\renewcommand\thelinenumber{\color{black!50}\arabic{linenumber}}
            \input{0 - introduction/main.tex}
        \part{Research}
            \input{1 - low-noise PiC models/main.tex}
            \input{2 - kinetic component/main.tex}
            \input{3 - fluid component/main.tex}
            \input{4 - numerical implementation/main.tex}
        \part{Project Overview}
            \input{5 - research plan/main.tex}
            \input{6 - summary/main.tex}
    
    
    %\section{}
    \newpage
    \pagenumbering{gobble}
        \printbibliography


    \newpage
    \pagenumbering{roman}
    \appendix
        \part{Appendices}
            \input{8 - Hilbert complexes/main.tex}
            \input{9 - weak conservation proofs/main.tex}
\end{document}

            \documentclass[12pt, a4paper]{report}

\input{template/main.tex}

\title{\BA{Title in Progress...}}
\author{Boris Andrews}
\affil{Mathematical Institute, University of Oxford}
\date{\today}


\begin{document}
    \pagenumbering{gobble}
    \maketitle
    
    
    \begin{abstract}
        Magnetic confinement reactors---in particular tokamaks---offer one of the most promising options for achieving practical nuclear fusion, with the potential to provide virtually limitless, clean energy. The theoretical and numerical modeling of tokamak plasmas is simultaneously an essential component of effective reactor design, and a great research barrier. Tokamak operational conditions exhibit comparatively low Knudsen numbers. Kinetic effects, including kinetic waves and instabilities, Landau damping, bump-on-tail instabilities and more, are therefore highly influential in tokamak plasma dynamics. Purely fluid models are inherently incapable of capturing these effects, whereas the high dimensionality in purely kinetic models render them practically intractable for most relevant purposes.

        We consider a $\delta\!f$ decomposition model, with a macroscopic fluid background and microscopic kinetic correction, both fully coupled to each other. A similar manner of discretization is proposed to that used in the recent \texttt{STRUPHY} code \cite{Holderied_Possanner_Wang_2021, Holderied_2022, Li_et_al_2023} with a finite-element model for the background and a pseudo-particle/PiC model for the correction.

        The fluid background satisfies the full, non-linear, resistive, compressible, Hall MHD equations. \cite{Laakmann_Hu_Farrell_2022} introduces finite-element(-in-space) implicit timesteppers for the incompressible analogue to this system with structure-preserving (SP) properties in the ideal case, alongside parameter-robust preconditioners. We show that these timesteppers can derive from a finite-element-in-time (FET) (and finite-element-in-space) interpretation. The benefits of this reformulation are discussed, including the derivation of timesteppers that are higher order in time, and the quantifiable dissipative SP properties in the non-ideal, resistive case.
        
        We discuss possible options for extending this FET approach to timesteppers for the compressible case.

        The kinetic corrections satisfy linearized Boltzmann equations. Using a Lénard--Bernstein collision operator, these take Fokker--Planck-like forms \cite{Fokker_1914, Planck_1917} wherein pseudo-particles in the numerical model obey the neoclassical transport equations, with particle-independent Brownian drift terms. This offers a rigorous methodology for incorporating collisions into the particle transport model, without coupling the equations of motions for each particle.
        
        Works by Chen, Chacón et al. \cite{Chen_Chacón_Barnes_2011, Chacón_Chen_Barnes_2013, Chen_Chacón_2014, Chen_Chacón_2015} have developed structure-preserving particle pushers for neoclassical transport in the Vlasov equations, derived from Crank--Nicolson integrators. We show these too can can derive from a FET interpretation, similarly offering potential extensions to higher-order-in-time particle pushers. The FET formulation is used also to consider how the stochastic drift terms can be incorporated into the pushers. Stochastic gyrokinetic expansions are also discussed.

        Different options for the numerical implementation of these schemes are considered.

        Due to the efficacy of FET in the development of SP timesteppers for both the fluid and kinetic component, we hope this approach will prove effective in the future for developing SP timesteppers for the full hybrid model. We hope this will give us the opportunity to incorporate previously inaccessible kinetic effects into the highly effective, modern, finite-element MHD models.
    \end{abstract}
    
    
    \newpage
    \tableofcontents
    
    
    \newpage
    \pagenumbering{arabic}
    %\linenumbers\renewcommand\thelinenumber{\color{black!50}\arabic{linenumber}}
            \input{0 - introduction/main.tex}
        \part{Research}
            \input{1 - low-noise PiC models/main.tex}
            \input{2 - kinetic component/main.tex}
            \input{3 - fluid component/main.tex}
            \input{4 - numerical implementation/main.tex}
        \part{Project Overview}
            \input{5 - research plan/main.tex}
            \input{6 - summary/main.tex}
    
    
    %\section{}
    \newpage
    \pagenumbering{gobble}
        \printbibliography


    \newpage
    \pagenumbering{roman}
    \appendix
        \part{Appendices}
            \input{8 - Hilbert complexes/main.tex}
            \input{9 - weak conservation proofs/main.tex}
\end{document}

\end{document}

        \part{Project Overview}
            \documentclass[12pt, a4paper]{report}

\documentclass[12pt, a4paper]{report}

\input{template/main.tex}

\title{\BA{Title in Progress...}}
\author{Boris Andrews}
\affil{Mathematical Institute, University of Oxford}
\date{\today}


\begin{document}
    \pagenumbering{gobble}
    \maketitle
    
    
    \begin{abstract}
        Magnetic confinement reactors---in particular tokamaks---offer one of the most promising options for achieving practical nuclear fusion, with the potential to provide virtually limitless, clean energy. The theoretical and numerical modeling of tokamak plasmas is simultaneously an essential component of effective reactor design, and a great research barrier. Tokamak operational conditions exhibit comparatively low Knudsen numbers. Kinetic effects, including kinetic waves and instabilities, Landau damping, bump-on-tail instabilities and more, are therefore highly influential in tokamak plasma dynamics. Purely fluid models are inherently incapable of capturing these effects, whereas the high dimensionality in purely kinetic models render them practically intractable for most relevant purposes.

        We consider a $\delta\!f$ decomposition model, with a macroscopic fluid background and microscopic kinetic correction, both fully coupled to each other. A similar manner of discretization is proposed to that used in the recent \texttt{STRUPHY} code \cite{Holderied_Possanner_Wang_2021, Holderied_2022, Li_et_al_2023} with a finite-element model for the background and a pseudo-particle/PiC model for the correction.

        The fluid background satisfies the full, non-linear, resistive, compressible, Hall MHD equations. \cite{Laakmann_Hu_Farrell_2022} introduces finite-element(-in-space) implicit timesteppers for the incompressible analogue to this system with structure-preserving (SP) properties in the ideal case, alongside parameter-robust preconditioners. We show that these timesteppers can derive from a finite-element-in-time (FET) (and finite-element-in-space) interpretation. The benefits of this reformulation are discussed, including the derivation of timesteppers that are higher order in time, and the quantifiable dissipative SP properties in the non-ideal, resistive case.
        
        We discuss possible options for extending this FET approach to timesteppers for the compressible case.

        The kinetic corrections satisfy linearized Boltzmann equations. Using a Lénard--Bernstein collision operator, these take Fokker--Planck-like forms \cite{Fokker_1914, Planck_1917} wherein pseudo-particles in the numerical model obey the neoclassical transport equations, with particle-independent Brownian drift terms. This offers a rigorous methodology for incorporating collisions into the particle transport model, without coupling the equations of motions for each particle.
        
        Works by Chen, Chacón et al. \cite{Chen_Chacón_Barnes_2011, Chacón_Chen_Barnes_2013, Chen_Chacón_2014, Chen_Chacón_2015} have developed structure-preserving particle pushers for neoclassical transport in the Vlasov equations, derived from Crank--Nicolson integrators. We show these too can can derive from a FET interpretation, similarly offering potential extensions to higher-order-in-time particle pushers. The FET formulation is used also to consider how the stochastic drift terms can be incorporated into the pushers. Stochastic gyrokinetic expansions are also discussed.

        Different options for the numerical implementation of these schemes are considered.

        Due to the efficacy of FET in the development of SP timesteppers for both the fluid and kinetic component, we hope this approach will prove effective in the future for developing SP timesteppers for the full hybrid model. We hope this will give us the opportunity to incorporate previously inaccessible kinetic effects into the highly effective, modern, finite-element MHD models.
    \end{abstract}
    
    
    \newpage
    \tableofcontents
    
    
    \newpage
    \pagenumbering{arabic}
    %\linenumbers\renewcommand\thelinenumber{\color{black!50}\arabic{linenumber}}
            \input{0 - introduction/main.tex}
        \part{Research}
            \input{1 - low-noise PiC models/main.tex}
            \input{2 - kinetic component/main.tex}
            \input{3 - fluid component/main.tex}
            \input{4 - numerical implementation/main.tex}
        \part{Project Overview}
            \input{5 - research plan/main.tex}
            \input{6 - summary/main.tex}
    
    
    %\section{}
    \newpage
    \pagenumbering{gobble}
        \printbibliography


    \newpage
    \pagenumbering{roman}
    \appendix
        \part{Appendices}
            \input{8 - Hilbert complexes/main.tex}
            \input{9 - weak conservation proofs/main.tex}
\end{document}


\title{\BA{Title in Progress...}}
\author{Boris Andrews}
\affil{Mathematical Institute, University of Oxford}
\date{\today}


\begin{document}
    \pagenumbering{gobble}
    \maketitle
    
    
    \begin{abstract}
        Magnetic confinement reactors---in particular tokamaks---offer one of the most promising options for achieving practical nuclear fusion, with the potential to provide virtually limitless, clean energy. The theoretical and numerical modeling of tokamak plasmas is simultaneously an essential component of effective reactor design, and a great research barrier. Tokamak operational conditions exhibit comparatively low Knudsen numbers. Kinetic effects, including kinetic waves and instabilities, Landau damping, bump-on-tail instabilities and more, are therefore highly influential in tokamak plasma dynamics. Purely fluid models are inherently incapable of capturing these effects, whereas the high dimensionality in purely kinetic models render them practically intractable for most relevant purposes.

        We consider a $\delta\!f$ decomposition model, with a macroscopic fluid background and microscopic kinetic correction, both fully coupled to each other. A similar manner of discretization is proposed to that used in the recent \texttt{STRUPHY} code \cite{Holderied_Possanner_Wang_2021, Holderied_2022, Li_et_al_2023} with a finite-element model for the background and a pseudo-particle/PiC model for the correction.

        The fluid background satisfies the full, non-linear, resistive, compressible, Hall MHD equations. \cite{Laakmann_Hu_Farrell_2022} introduces finite-element(-in-space) implicit timesteppers for the incompressible analogue to this system with structure-preserving (SP) properties in the ideal case, alongside parameter-robust preconditioners. We show that these timesteppers can derive from a finite-element-in-time (FET) (and finite-element-in-space) interpretation. The benefits of this reformulation are discussed, including the derivation of timesteppers that are higher order in time, and the quantifiable dissipative SP properties in the non-ideal, resistive case.
        
        We discuss possible options for extending this FET approach to timesteppers for the compressible case.

        The kinetic corrections satisfy linearized Boltzmann equations. Using a Lénard--Bernstein collision operator, these take Fokker--Planck-like forms \cite{Fokker_1914, Planck_1917} wherein pseudo-particles in the numerical model obey the neoclassical transport equations, with particle-independent Brownian drift terms. This offers a rigorous methodology for incorporating collisions into the particle transport model, without coupling the equations of motions for each particle.
        
        Works by Chen, Chacón et al. \cite{Chen_Chacón_Barnes_2011, Chacón_Chen_Barnes_2013, Chen_Chacón_2014, Chen_Chacón_2015} have developed structure-preserving particle pushers for neoclassical transport in the Vlasov equations, derived from Crank--Nicolson integrators. We show these too can can derive from a FET interpretation, similarly offering potential extensions to higher-order-in-time particle pushers. The FET formulation is used also to consider how the stochastic drift terms can be incorporated into the pushers. Stochastic gyrokinetic expansions are also discussed.

        Different options for the numerical implementation of these schemes are considered.

        Due to the efficacy of FET in the development of SP timesteppers for both the fluid and kinetic component, we hope this approach will prove effective in the future for developing SP timesteppers for the full hybrid model. We hope this will give us the opportunity to incorporate previously inaccessible kinetic effects into the highly effective, modern, finite-element MHD models.
    \end{abstract}
    
    
    \newpage
    \tableofcontents
    
    
    \newpage
    \pagenumbering{arabic}
    %\linenumbers\renewcommand\thelinenumber{\color{black!50}\arabic{linenumber}}
            \documentclass[12pt, a4paper]{report}

\input{template/main.tex}

\title{\BA{Title in Progress...}}
\author{Boris Andrews}
\affil{Mathematical Institute, University of Oxford}
\date{\today}


\begin{document}
    \pagenumbering{gobble}
    \maketitle
    
    
    \begin{abstract}
        Magnetic confinement reactors---in particular tokamaks---offer one of the most promising options for achieving practical nuclear fusion, with the potential to provide virtually limitless, clean energy. The theoretical and numerical modeling of tokamak plasmas is simultaneously an essential component of effective reactor design, and a great research barrier. Tokamak operational conditions exhibit comparatively low Knudsen numbers. Kinetic effects, including kinetic waves and instabilities, Landau damping, bump-on-tail instabilities and more, are therefore highly influential in tokamak plasma dynamics. Purely fluid models are inherently incapable of capturing these effects, whereas the high dimensionality in purely kinetic models render them practically intractable for most relevant purposes.

        We consider a $\delta\!f$ decomposition model, with a macroscopic fluid background and microscopic kinetic correction, both fully coupled to each other. A similar manner of discretization is proposed to that used in the recent \texttt{STRUPHY} code \cite{Holderied_Possanner_Wang_2021, Holderied_2022, Li_et_al_2023} with a finite-element model for the background and a pseudo-particle/PiC model for the correction.

        The fluid background satisfies the full, non-linear, resistive, compressible, Hall MHD equations. \cite{Laakmann_Hu_Farrell_2022} introduces finite-element(-in-space) implicit timesteppers for the incompressible analogue to this system with structure-preserving (SP) properties in the ideal case, alongside parameter-robust preconditioners. We show that these timesteppers can derive from a finite-element-in-time (FET) (and finite-element-in-space) interpretation. The benefits of this reformulation are discussed, including the derivation of timesteppers that are higher order in time, and the quantifiable dissipative SP properties in the non-ideal, resistive case.
        
        We discuss possible options for extending this FET approach to timesteppers for the compressible case.

        The kinetic corrections satisfy linearized Boltzmann equations. Using a Lénard--Bernstein collision operator, these take Fokker--Planck-like forms \cite{Fokker_1914, Planck_1917} wherein pseudo-particles in the numerical model obey the neoclassical transport equations, with particle-independent Brownian drift terms. This offers a rigorous methodology for incorporating collisions into the particle transport model, without coupling the equations of motions for each particle.
        
        Works by Chen, Chacón et al. \cite{Chen_Chacón_Barnes_2011, Chacón_Chen_Barnes_2013, Chen_Chacón_2014, Chen_Chacón_2015} have developed structure-preserving particle pushers for neoclassical transport in the Vlasov equations, derived from Crank--Nicolson integrators. We show these too can can derive from a FET interpretation, similarly offering potential extensions to higher-order-in-time particle pushers. The FET formulation is used also to consider how the stochastic drift terms can be incorporated into the pushers. Stochastic gyrokinetic expansions are also discussed.

        Different options for the numerical implementation of these schemes are considered.

        Due to the efficacy of FET in the development of SP timesteppers for both the fluid and kinetic component, we hope this approach will prove effective in the future for developing SP timesteppers for the full hybrid model. We hope this will give us the opportunity to incorporate previously inaccessible kinetic effects into the highly effective, modern, finite-element MHD models.
    \end{abstract}
    
    
    \newpage
    \tableofcontents
    
    
    \newpage
    \pagenumbering{arabic}
    %\linenumbers\renewcommand\thelinenumber{\color{black!50}\arabic{linenumber}}
            \input{0 - introduction/main.tex}
        \part{Research}
            \input{1 - low-noise PiC models/main.tex}
            \input{2 - kinetic component/main.tex}
            \input{3 - fluid component/main.tex}
            \input{4 - numerical implementation/main.tex}
        \part{Project Overview}
            \input{5 - research plan/main.tex}
            \input{6 - summary/main.tex}
    
    
    %\section{}
    \newpage
    \pagenumbering{gobble}
        \printbibliography


    \newpage
    \pagenumbering{roman}
    \appendix
        \part{Appendices}
            \input{8 - Hilbert complexes/main.tex}
            \input{9 - weak conservation proofs/main.tex}
\end{document}

        \part{Research}
            \documentclass[12pt, a4paper]{report}

\input{template/main.tex}

\title{\BA{Title in Progress...}}
\author{Boris Andrews}
\affil{Mathematical Institute, University of Oxford}
\date{\today}


\begin{document}
    \pagenumbering{gobble}
    \maketitle
    
    
    \begin{abstract}
        Magnetic confinement reactors---in particular tokamaks---offer one of the most promising options for achieving practical nuclear fusion, with the potential to provide virtually limitless, clean energy. The theoretical and numerical modeling of tokamak plasmas is simultaneously an essential component of effective reactor design, and a great research barrier. Tokamak operational conditions exhibit comparatively low Knudsen numbers. Kinetic effects, including kinetic waves and instabilities, Landau damping, bump-on-tail instabilities and more, are therefore highly influential in tokamak plasma dynamics. Purely fluid models are inherently incapable of capturing these effects, whereas the high dimensionality in purely kinetic models render them practically intractable for most relevant purposes.

        We consider a $\delta\!f$ decomposition model, with a macroscopic fluid background and microscopic kinetic correction, both fully coupled to each other. A similar manner of discretization is proposed to that used in the recent \texttt{STRUPHY} code \cite{Holderied_Possanner_Wang_2021, Holderied_2022, Li_et_al_2023} with a finite-element model for the background and a pseudo-particle/PiC model for the correction.

        The fluid background satisfies the full, non-linear, resistive, compressible, Hall MHD equations. \cite{Laakmann_Hu_Farrell_2022} introduces finite-element(-in-space) implicit timesteppers for the incompressible analogue to this system with structure-preserving (SP) properties in the ideal case, alongside parameter-robust preconditioners. We show that these timesteppers can derive from a finite-element-in-time (FET) (and finite-element-in-space) interpretation. The benefits of this reformulation are discussed, including the derivation of timesteppers that are higher order in time, and the quantifiable dissipative SP properties in the non-ideal, resistive case.
        
        We discuss possible options for extending this FET approach to timesteppers for the compressible case.

        The kinetic corrections satisfy linearized Boltzmann equations. Using a Lénard--Bernstein collision operator, these take Fokker--Planck-like forms \cite{Fokker_1914, Planck_1917} wherein pseudo-particles in the numerical model obey the neoclassical transport equations, with particle-independent Brownian drift terms. This offers a rigorous methodology for incorporating collisions into the particle transport model, without coupling the equations of motions for each particle.
        
        Works by Chen, Chacón et al. \cite{Chen_Chacón_Barnes_2011, Chacón_Chen_Barnes_2013, Chen_Chacón_2014, Chen_Chacón_2015} have developed structure-preserving particle pushers for neoclassical transport in the Vlasov equations, derived from Crank--Nicolson integrators. We show these too can can derive from a FET interpretation, similarly offering potential extensions to higher-order-in-time particle pushers. The FET formulation is used also to consider how the stochastic drift terms can be incorporated into the pushers. Stochastic gyrokinetic expansions are also discussed.

        Different options for the numerical implementation of these schemes are considered.

        Due to the efficacy of FET in the development of SP timesteppers for both the fluid and kinetic component, we hope this approach will prove effective in the future for developing SP timesteppers for the full hybrid model. We hope this will give us the opportunity to incorporate previously inaccessible kinetic effects into the highly effective, modern, finite-element MHD models.
    \end{abstract}
    
    
    \newpage
    \tableofcontents
    
    
    \newpage
    \pagenumbering{arabic}
    %\linenumbers\renewcommand\thelinenumber{\color{black!50}\arabic{linenumber}}
            \input{0 - introduction/main.tex}
        \part{Research}
            \input{1 - low-noise PiC models/main.tex}
            \input{2 - kinetic component/main.tex}
            \input{3 - fluid component/main.tex}
            \input{4 - numerical implementation/main.tex}
        \part{Project Overview}
            \input{5 - research plan/main.tex}
            \input{6 - summary/main.tex}
    
    
    %\section{}
    \newpage
    \pagenumbering{gobble}
        \printbibliography


    \newpage
    \pagenumbering{roman}
    \appendix
        \part{Appendices}
            \input{8 - Hilbert complexes/main.tex}
            \input{9 - weak conservation proofs/main.tex}
\end{document}

            \documentclass[12pt, a4paper]{report}

\input{template/main.tex}

\title{\BA{Title in Progress...}}
\author{Boris Andrews}
\affil{Mathematical Institute, University of Oxford}
\date{\today}


\begin{document}
    \pagenumbering{gobble}
    \maketitle
    
    
    \begin{abstract}
        Magnetic confinement reactors---in particular tokamaks---offer one of the most promising options for achieving practical nuclear fusion, with the potential to provide virtually limitless, clean energy. The theoretical and numerical modeling of tokamak plasmas is simultaneously an essential component of effective reactor design, and a great research barrier. Tokamak operational conditions exhibit comparatively low Knudsen numbers. Kinetic effects, including kinetic waves and instabilities, Landau damping, bump-on-tail instabilities and more, are therefore highly influential in tokamak plasma dynamics. Purely fluid models are inherently incapable of capturing these effects, whereas the high dimensionality in purely kinetic models render them practically intractable for most relevant purposes.

        We consider a $\delta\!f$ decomposition model, with a macroscopic fluid background and microscopic kinetic correction, both fully coupled to each other. A similar manner of discretization is proposed to that used in the recent \texttt{STRUPHY} code \cite{Holderied_Possanner_Wang_2021, Holderied_2022, Li_et_al_2023} with a finite-element model for the background and a pseudo-particle/PiC model for the correction.

        The fluid background satisfies the full, non-linear, resistive, compressible, Hall MHD equations. \cite{Laakmann_Hu_Farrell_2022} introduces finite-element(-in-space) implicit timesteppers for the incompressible analogue to this system with structure-preserving (SP) properties in the ideal case, alongside parameter-robust preconditioners. We show that these timesteppers can derive from a finite-element-in-time (FET) (and finite-element-in-space) interpretation. The benefits of this reformulation are discussed, including the derivation of timesteppers that are higher order in time, and the quantifiable dissipative SP properties in the non-ideal, resistive case.
        
        We discuss possible options for extending this FET approach to timesteppers for the compressible case.

        The kinetic corrections satisfy linearized Boltzmann equations. Using a Lénard--Bernstein collision operator, these take Fokker--Planck-like forms \cite{Fokker_1914, Planck_1917} wherein pseudo-particles in the numerical model obey the neoclassical transport equations, with particle-independent Brownian drift terms. This offers a rigorous methodology for incorporating collisions into the particle transport model, without coupling the equations of motions for each particle.
        
        Works by Chen, Chacón et al. \cite{Chen_Chacón_Barnes_2011, Chacón_Chen_Barnes_2013, Chen_Chacón_2014, Chen_Chacón_2015} have developed structure-preserving particle pushers for neoclassical transport in the Vlasov equations, derived from Crank--Nicolson integrators. We show these too can can derive from a FET interpretation, similarly offering potential extensions to higher-order-in-time particle pushers. The FET formulation is used also to consider how the stochastic drift terms can be incorporated into the pushers. Stochastic gyrokinetic expansions are also discussed.

        Different options for the numerical implementation of these schemes are considered.

        Due to the efficacy of FET in the development of SP timesteppers for both the fluid and kinetic component, we hope this approach will prove effective in the future for developing SP timesteppers for the full hybrid model. We hope this will give us the opportunity to incorporate previously inaccessible kinetic effects into the highly effective, modern, finite-element MHD models.
    \end{abstract}
    
    
    \newpage
    \tableofcontents
    
    
    \newpage
    \pagenumbering{arabic}
    %\linenumbers\renewcommand\thelinenumber{\color{black!50}\arabic{linenumber}}
            \input{0 - introduction/main.tex}
        \part{Research}
            \input{1 - low-noise PiC models/main.tex}
            \input{2 - kinetic component/main.tex}
            \input{3 - fluid component/main.tex}
            \input{4 - numerical implementation/main.tex}
        \part{Project Overview}
            \input{5 - research plan/main.tex}
            \input{6 - summary/main.tex}
    
    
    %\section{}
    \newpage
    \pagenumbering{gobble}
        \printbibliography


    \newpage
    \pagenumbering{roman}
    \appendix
        \part{Appendices}
            \input{8 - Hilbert complexes/main.tex}
            \input{9 - weak conservation proofs/main.tex}
\end{document}

            \documentclass[12pt, a4paper]{report}

\input{template/main.tex}

\title{\BA{Title in Progress...}}
\author{Boris Andrews}
\affil{Mathematical Institute, University of Oxford}
\date{\today}


\begin{document}
    \pagenumbering{gobble}
    \maketitle
    
    
    \begin{abstract}
        Magnetic confinement reactors---in particular tokamaks---offer one of the most promising options for achieving practical nuclear fusion, with the potential to provide virtually limitless, clean energy. The theoretical and numerical modeling of tokamak plasmas is simultaneously an essential component of effective reactor design, and a great research barrier. Tokamak operational conditions exhibit comparatively low Knudsen numbers. Kinetic effects, including kinetic waves and instabilities, Landau damping, bump-on-tail instabilities and more, are therefore highly influential in tokamak plasma dynamics. Purely fluid models are inherently incapable of capturing these effects, whereas the high dimensionality in purely kinetic models render them practically intractable for most relevant purposes.

        We consider a $\delta\!f$ decomposition model, with a macroscopic fluid background and microscopic kinetic correction, both fully coupled to each other. A similar manner of discretization is proposed to that used in the recent \texttt{STRUPHY} code \cite{Holderied_Possanner_Wang_2021, Holderied_2022, Li_et_al_2023} with a finite-element model for the background and a pseudo-particle/PiC model for the correction.

        The fluid background satisfies the full, non-linear, resistive, compressible, Hall MHD equations. \cite{Laakmann_Hu_Farrell_2022} introduces finite-element(-in-space) implicit timesteppers for the incompressible analogue to this system with structure-preserving (SP) properties in the ideal case, alongside parameter-robust preconditioners. We show that these timesteppers can derive from a finite-element-in-time (FET) (and finite-element-in-space) interpretation. The benefits of this reformulation are discussed, including the derivation of timesteppers that are higher order in time, and the quantifiable dissipative SP properties in the non-ideal, resistive case.
        
        We discuss possible options for extending this FET approach to timesteppers for the compressible case.

        The kinetic corrections satisfy linearized Boltzmann equations. Using a Lénard--Bernstein collision operator, these take Fokker--Planck-like forms \cite{Fokker_1914, Planck_1917} wherein pseudo-particles in the numerical model obey the neoclassical transport equations, with particle-independent Brownian drift terms. This offers a rigorous methodology for incorporating collisions into the particle transport model, without coupling the equations of motions for each particle.
        
        Works by Chen, Chacón et al. \cite{Chen_Chacón_Barnes_2011, Chacón_Chen_Barnes_2013, Chen_Chacón_2014, Chen_Chacón_2015} have developed structure-preserving particle pushers for neoclassical transport in the Vlasov equations, derived from Crank--Nicolson integrators. We show these too can can derive from a FET interpretation, similarly offering potential extensions to higher-order-in-time particle pushers. The FET formulation is used also to consider how the stochastic drift terms can be incorporated into the pushers. Stochastic gyrokinetic expansions are also discussed.

        Different options for the numerical implementation of these schemes are considered.

        Due to the efficacy of FET in the development of SP timesteppers for both the fluid and kinetic component, we hope this approach will prove effective in the future for developing SP timesteppers for the full hybrid model. We hope this will give us the opportunity to incorporate previously inaccessible kinetic effects into the highly effective, modern, finite-element MHD models.
    \end{abstract}
    
    
    \newpage
    \tableofcontents
    
    
    \newpage
    \pagenumbering{arabic}
    %\linenumbers\renewcommand\thelinenumber{\color{black!50}\arabic{linenumber}}
            \input{0 - introduction/main.tex}
        \part{Research}
            \input{1 - low-noise PiC models/main.tex}
            \input{2 - kinetic component/main.tex}
            \input{3 - fluid component/main.tex}
            \input{4 - numerical implementation/main.tex}
        \part{Project Overview}
            \input{5 - research plan/main.tex}
            \input{6 - summary/main.tex}
    
    
    %\section{}
    \newpage
    \pagenumbering{gobble}
        \printbibliography


    \newpage
    \pagenumbering{roman}
    \appendix
        \part{Appendices}
            \input{8 - Hilbert complexes/main.tex}
            \input{9 - weak conservation proofs/main.tex}
\end{document}

            \documentclass[12pt, a4paper]{report}

\input{template/main.tex}

\title{\BA{Title in Progress...}}
\author{Boris Andrews}
\affil{Mathematical Institute, University of Oxford}
\date{\today}


\begin{document}
    \pagenumbering{gobble}
    \maketitle
    
    
    \begin{abstract}
        Magnetic confinement reactors---in particular tokamaks---offer one of the most promising options for achieving practical nuclear fusion, with the potential to provide virtually limitless, clean energy. The theoretical and numerical modeling of tokamak plasmas is simultaneously an essential component of effective reactor design, and a great research barrier. Tokamak operational conditions exhibit comparatively low Knudsen numbers. Kinetic effects, including kinetic waves and instabilities, Landau damping, bump-on-tail instabilities and more, are therefore highly influential in tokamak plasma dynamics. Purely fluid models are inherently incapable of capturing these effects, whereas the high dimensionality in purely kinetic models render them practically intractable for most relevant purposes.

        We consider a $\delta\!f$ decomposition model, with a macroscopic fluid background and microscopic kinetic correction, both fully coupled to each other. A similar manner of discretization is proposed to that used in the recent \texttt{STRUPHY} code \cite{Holderied_Possanner_Wang_2021, Holderied_2022, Li_et_al_2023} with a finite-element model for the background and a pseudo-particle/PiC model for the correction.

        The fluid background satisfies the full, non-linear, resistive, compressible, Hall MHD equations. \cite{Laakmann_Hu_Farrell_2022} introduces finite-element(-in-space) implicit timesteppers for the incompressible analogue to this system with structure-preserving (SP) properties in the ideal case, alongside parameter-robust preconditioners. We show that these timesteppers can derive from a finite-element-in-time (FET) (and finite-element-in-space) interpretation. The benefits of this reformulation are discussed, including the derivation of timesteppers that are higher order in time, and the quantifiable dissipative SP properties in the non-ideal, resistive case.
        
        We discuss possible options for extending this FET approach to timesteppers for the compressible case.

        The kinetic corrections satisfy linearized Boltzmann equations. Using a Lénard--Bernstein collision operator, these take Fokker--Planck-like forms \cite{Fokker_1914, Planck_1917} wherein pseudo-particles in the numerical model obey the neoclassical transport equations, with particle-independent Brownian drift terms. This offers a rigorous methodology for incorporating collisions into the particle transport model, without coupling the equations of motions for each particle.
        
        Works by Chen, Chacón et al. \cite{Chen_Chacón_Barnes_2011, Chacón_Chen_Barnes_2013, Chen_Chacón_2014, Chen_Chacón_2015} have developed structure-preserving particle pushers for neoclassical transport in the Vlasov equations, derived from Crank--Nicolson integrators. We show these too can can derive from a FET interpretation, similarly offering potential extensions to higher-order-in-time particle pushers. The FET formulation is used also to consider how the stochastic drift terms can be incorporated into the pushers. Stochastic gyrokinetic expansions are also discussed.

        Different options for the numerical implementation of these schemes are considered.

        Due to the efficacy of FET in the development of SP timesteppers for both the fluid and kinetic component, we hope this approach will prove effective in the future for developing SP timesteppers for the full hybrid model. We hope this will give us the opportunity to incorporate previously inaccessible kinetic effects into the highly effective, modern, finite-element MHD models.
    \end{abstract}
    
    
    \newpage
    \tableofcontents
    
    
    \newpage
    \pagenumbering{arabic}
    %\linenumbers\renewcommand\thelinenumber{\color{black!50}\arabic{linenumber}}
            \input{0 - introduction/main.tex}
        \part{Research}
            \input{1 - low-noise PiC models/main.tex}
            \input{2 - kinetic component/main.tex}
            \input{3 - fluid component/main.tex}
            \input{4 - numerical implementation/main.tex}
        \part{Project Overview}
            \input{5 - research plan/main.tex}
            \input{6 - summary/main.tex}
    
    
    %\section{}
    \newpage
    \pagenumbering{gobble}
        \printbibliography


    \newpage
    \pagenumbering{roman}
    \appendix
        \part{Appendices}
            \input{8 - Hilbert complexes/main.tex}
            \input{9 - weak conservation proofs/main.tex}
\end{document}

        \part{Project Overview}
            \documentclass[12pt, a4paper]{report}

\input{template/main.tex}

\title{\BA{Title in Progress...}}
\author{Boris Andrews}
\affil{Mathematical Institute, University of Oxford}
\date{\today}


\begin{document}
    \pagenumbering{gobble}
    \maketitle
    
    
    \begin{abstract}
        Magnetic confinement reactors---in particular tokamaks---offer one of the most promising options for achieving practical nuclear fusion, with the potential to provide virtually limitless, clean energy. The theoretical and numerical modeling of tokamak plasmas is simultaneously an essential component of effective reactor design, and a great research barrier. Tokamak operational conditions exhibit comparatively low Knudsen numbers. Kinetic effects, including kinetic waves and instabilities, Landau damping, bump-on-tail instabilities and more, are therefore highly influential in tokamak plasma dynamics. Purely fluid models are inherently incapable of capturing these effects, whereas the high dimensionality in purely kinetic models render them practically intractable for most relevant purposes.

        We consider a $\delta\!f$ decomposition model, with a macroscopic fluid background and microscopic kinetic correction, both fully coupled to each other. A similar manner of discretization is proposed to that used in the recent \texttt{STRUPHY} code \cite{Holderied_Possanner_Wang_2021, Holderied_2022, Li_et_al_2023} with a finite-element model for the background and a pseudo-particle/PiC model for the correction.

        The fluid background satisfies the full, non-linear, resistive, compressible, Hall MHD equations. \cite{Laakmann_Hu_Farrell_2022} introduces finite-element(-in-space) implicit timesteppers for the incompressible analogue to this system with structure-preserving (SP) properties in the ideal case, alongside parameter-robust preconditioners. We show that these timesteppers can derive from a finite-element-in-time (FET) (and finite-element-in-space) interpretation. The benefits of this reformulation are discussed, including the derivation of timesteppers that are higher order in time, and the quantifiable dissipative SP properties in the non-ideal, resistive case.
        
        We discuss possible options for extending this FET approach to timesteppers for the compressible case.

        The kinetic corrections satisfy linearized Boltzmann equations. Using a Lénard--Bernstein collision operator, these take Fokker--Planck-like forms \cite{Fokker_1914, Planck_1917} wherein pseudo-particles in the numerical model obey the neoclassical transport equations, with particle-independent Brownian drift terms. This offers a rigorous methodology for incorporating collisions into the particle transport model, without coupling the equations of motions for each particle.
        
        Works by Chen, Chacón et al. \cite{Chen_Chacón_Barnes_2011, Chacón_Chen_Barnes_2013, Chen_Chacón_2014, Chen_Chacón_2015} have developed structure-preserving particle pushers for neoclassical transport in the Vlasov equations, derived from Crank--Nicolson integrators. We show these too can can derive from a FET interpretation, similarly offering potential extensions to higher-order-in-time particle pushers. The FET formulation is used also to consider how the stochastic drift terms can be incorporated into the pushers. Stochastic gyrokinetic expansions are also discussed.

        Different options for the numerical implementation of these schemes are considered.

        Due to the efficacy of FET in the development of SP timesteppers for both the fluid and kinetic component, we hope this approach will prove effective in the future for developing SP timesteppers for the full hybrid model. We hope this will give us the opportunity to incorporate previously inaccessible kinetic effects into the highly effective, modern, finite-element MHD models.
    \end{abstract}
    
    
    \newpage
    \tableofcontents
    
    
    \newpage
    \pagenumbering{arabic}
    %\linenumbers\renewcommand\thelinenumber{\color{black!50}\arabic{linenumber}}
            \input{0 - introduction/main.tex}
        \part{Research}
            \input{1 - low-noise PiC models/main.tex}
            \input{2 - kinetic component/main.tex}
            \input{3 - fluid component/main.tex}
            \input{4 - numerical implementation/main.tex}
        \part{Project Overview}
            \input{5 - research plan/main.tex}
            \input{6 - summary/main.tex}
    
    
    %\section{}
    \newpage
    \pagenumbering{gobble}
        \printbibliography


    \newpage
    \pagenumbering{roman}
    \appendix
        \part{Appendices}
            \input{8 - Hilbert complexes/main.tex}
            \input{9 - weak conservation proofs/main.tex}
\end{document}

            \documentclass[12pt, a4paper]{report}

\input{template/main.tex}

\title{\BA{Title in Progress...}}
\author{Boris Andrews}
\affil{Mathematical Institute, University of Oxford}
\date{\today}


\begin{document}
    \pagenumbering{gobble}
    \maketitle
    
    
    \begin{abstract}
        Magnetic confinement reactors---in particular tokamaks---offer one of the most promising options for achieving practical nuclear fusion, with the potential to provide virtually limitless, clean energy. The theoretical and numerical modeling of tokamak plasmas is simultaneously an essential component of effective reactor design, and a great research barrier. Tokamak operational conditions exhibit comparatively low Knudsen numbers. Kinetic effects, including kinetic waves and instabilities, Landau damping, bump-on-tail instabilities and more, are therefore highly influential in tokamak plasma dynamics. Purely fluid models are inherently incapable of capturing these effects, whereas the high dimensionality in purely kinetic models render them practically intractable for most relevant purposes.

        We consider a $\delta\!f$ decomposition model, with a macroscopic fluid background and microscopic kinetic correction, both fully coupled to each other. A similar manner of discretization is proposed to that used in the recent \texttt{STRUPHY} code \cite{Holderied_Possanner_Wang_2021, Holderied_2022, Li_et_al_2023} with a finite-element model for the background and a pseudo-particle/PiC model for the correction.

        The fluid background satisfies the full, non-linear, resistive, compressible, Hall MHD equations. \cite{Laakmann_Hu_Farrell_2022} introduces finite-element(-in-space) implicit timesteppers for the incompressible analogue to this system with structure-preserving (SP) properties in the ideal case, alongside parameter-robust preconditioners. We show that these timesteppers can derive from a finite-element-in-time (FET) (and finite-element-in-space) interpretation. The benefits of this reformulation are discussed, including the derivation of timesteppers that are higher order in time, and the quantifiable dissipative SP properties in the non-ideal, resistive case.
        
        We discuss possible options for extending this FET approach to timesteppers for the compressible case.

        The kinetic corrections satisfy linearized Boltzmann equations. Using a Lénard--Bernstein collision operator, these take Fokker--Planck-like forms \cite{Fokker_1914, Planck_1917} wherein pseudo-particles in the numerical model obey the neoclassical transport equations, with particle-independent Brownian drift terms. This offers a rigorous methodology for incorporating collisions into the particle transport model, without coupling the equations of motions for each particle.
        
        Works by Chen, Chacón et al. \cite{Chen_Chacón_Barnes_2011, Chacón_Chen_Barnes_2013, Chen_Chacón_2014, Chen_Chacón_2015} have developed structure-preserving particle pushers for neoclassical transport in the Vlasov equations, derived from Crank--Nicolson integrators. We show these too can can derive from a FET interpretation, similarly offering potential extensions to higher-order-in-time particle pushers. The FET formulation is used also to consider how the stochastic drift terms can be incorporated into the pushers. Stochastic gyrokinetic expansions are also discussed.

        Different options for the numerical implementation of these schemes are considered.

        Due to the efficacy of FET in the development of SP timesteppers for both the fluid and kinetic component, we hope this approach will prove effective in the future for developing SP timesteppers for the full hybrid model. We hope this will give us the opportunity to incorporate previously inaccessible kinetic effects into the highly effective, modern, finite-element MHD models.
    \end{abstract}
    
    
    \newpage
    \tableofcontents
    
    
    \newpage
    \pagenumbering{arabic}
    %\linenumbers\renewcommand\thelinenumber{\color{black!50}\arabic{linenumber}}
            \input{0 - introduction/main.tex}
        \part{Research}
            \input{1 - low-noise PiC models/main.tex}
            \input{2 - kinetic component/main.tex}
            \input{3 - fluid component/main.tex}
            \input{4 - numerical implementation/main.tex}
        \part{Project Overview}
            \input{5 - research plan/main.tex}
            \input{6 - summary/main.tex}
    
    
    %\section{}
    \newpage
    \pagenumbering{gobble}
        \printbibliography


    \newpage
    \pagenumbering{roman}
    \appendix
        \part{Appendices}
            \input{8 - Hilbert complexes/main.tex}
            \input{9 - weak conservation proofs/main.tex}
\end{document}

    
    
    %\section{}
    \newpage
    \pagenumbering{gobble}
        \printbibliography


    \newpage
    \pagenumbering{roman}
    \appendix
        \part{Appendices}
            \documentclass[12pt, a4paper]{report}

\input{template/main.tex}

\title{\BA{Title in Progress...}}
\author{Boris Andrews}
\affil{Mathematical Institute, University of Oxford}
\date{\today}


\begin{document}
    \pagenumbering{gobble}
    \maketitle
    
    
    \begin{abstract}
        Magnetic confinement reactors---in particular tokamaks---offer one of the most promising options for achieving practical nuclear fusion, with the potential to provide virtually limitless, clean energy. The theoretical and numerical modeling of tokamak plasmas is simultaneously an essential component of effective reactor design, and a great research barrier. Tokamak operational conditions exhibit comparatively low Knudsen numbers. Kinetic effects, including kinetic waves and instabilities, Landau damping, bump-on-tail instabilities and more, are therefore highly influential in tokamak plasma dynamics. Purely fluid models are inherently incapable of capturing these effects, whereas the high dimensionality in purely kinetic models render them practically intractable for most relevant purposes.

        We consider a $\delta\!f$ decomposition model, with a macroscopic fluid background and microscopic kinetic correction, both fully coupled to each other. A similar manner of discretization is proposed to that used in the recent \texttt{STRUPHY} code \cite{Holderied_Possanner_Wang_2021, Holderied_2022, Li_et_al_2023} with a finite-element model for the background and a pseudo-particle/PiC model for the correction.

        The fluid background satisfies the full, non-linear, resistive, compressible, Hall MHD equations. \cite{Laakmann_Hu_Farrell_2022} introduces finite-element(-in-space) implicit timesteppers for the incompressible analogue to this system with structure-preserving (SP) properties in the ideal case, alongside parameter-robust preconditioners. We show that these timesteppers can derive from a finite-element-in-time (FET) (and finite-element-in-space) interpretation. The benefits of this reformulation are discussed, including the derivation of timesteppers that are higher order in time, and the quantifiable dissipative SP properties in the non-ideal, resistive case.
        
        We discuss possible options for extending this FET approach to timesteppers for the compressible case.

        The kinetic corrections satisfy linearized Boltzmann equations. Using a Lénard--Bernstein collision operator, these take Fokker--Planck-like forms \cite{Fokker_1914, Planck_1917} wherein pseudo-particles in the numerical model obey the neoclassical transport equations, with particle-independent Brownian drift terms. This offers a rigorous methodology for incorporating collisions into the particle transport model, without coupling the equations of motions for each particle.
        
        Works by Chen, Chacón et al. \cite{Chen_Chacón_Barnes_2011, Chacón_Chen_Barnes_2013, Chen_Chacón_2014, Chen_Chacón_2015} have developed structure-preserving particle pushers for neoclassical transport in the Vlasov equations, derived from Crank--Nicolson integrators. We show these too can can derive from a FET interpretation, similarly offering potential extensions to higher-order-in-time particle pushers. The FET formulation is used also to consider how the stochastic drift terms can be incorporated into the pushers. Stochastic gyrokinetic expansions are also discussed.

        Different options for the numerical implementation of these schemes are considered.

        Due to the efficacy of FET in the development of SP timesteppers for both the fluid and kinetic component, we hope this approach will prove effective in the future for developing SP timesteppers for the full hybrid model. We hope this will give us the opportunity to incorporate previously inaccessible kinetic effects into the highly effective, modern, finite-element MHD models.
    \end{abstract}
    
    
    \newpage
    \tableofcontents
    
    
    \newpage
    \pagenumbering{arabic}
    %\linenumbers\renewcommand\thelinenumber{\color{black!50}\arabic{linenumber}}
            \input{0 - introduction/main.tex}
        \part{Research}
            \input{1 - low-noise PiC models/main.tex}
            \input{2 - kinetic component/main.tex}
            \input{3 - fluid component/main.tex}
            \input{4 - numerical implementation/main.tex}
        \part{Project Overview}
            \input{5 - research plan/main.tex}
            \input{6 - summary/main.tex}
    
    
    %\section{}
    \newpage
    \pagenumbering{gobble}
        \printbibliography


    \newpage
    \pagenumbering{roman}
    \appendix
        \part{Appendices}
            \input{8 - Hilbert complexes/main.tex}
            \input{9 - weak conservation proofs/main.tex}
\end{document}

            \documentclass[12pt, a4paper]{report}

\input{template/main.tex}

\title{\BA{Title in Progress...}}
\author{Boris Andrews}
\affil{Mathematical Institute, University of Oxford}
\date{\today}


\begin{document}
    \pagenumbering{gobble}
    \maketitle
    
    
    \begin{abstract}
        Magnetic confinement reactors---in particular tokamaks---offer one of the most promising options for achieving practical nuclear fusion, with the potential to provide virtually limitless, clean energy. The theoretical and numerical modeling of tokamak plasmas is simultaneously an essential component of effective reactor design, and a great research barrier. Tokamak operational conditions exhibit comparatively low Knudsen numbers. Kinetic effects, including kinetic waves and instabilities, Landau damping, bump-on-tail instabilities and more, are therefore highly influential in tokamak plasma dynamics. Purely fluid models are inherently incapable of capturing these effects, whereas the high dimensionality in purely kinetic models render them practically intractable for most relevant purposes.

        We consider a $\delta\!f$ decomposition model, with a macroscopic fluid background and microscopic kinetic correction, both fully coupled to each other. A similar manner of discretization is proposed to that used in the recent \texttt{STRUPHY} code \cite{Holderied_Possanner_Wang_2021, Holderied_2022, Li_et_al_2023} with a finite-element model for the background and a pseudo-particle/PiC model for the correction.

        The fluid background satisfies the full, non-linear, resistive, compressible, Hall MHD equations. \cite{Laakmann_Hu_Farrell_2022} introduces finite-element(-in-space) implicit timesteppers for the incompressible analogue to this system with structure-preserving (SP) properties in the ideal case, alongside parameter-robust preconditioners. We show that these timesteppers can derive from a finite-element-in-time (FET) (and finite-element-in-space) interpretation. The benefits of this reformulation are discussed, including the derivation of timesteppers that are higher order in time, and the quantifiable dissipative SP properties in the non-ideal, resistive case.
        
        We discuss possible options for extending this FET approach to timesteppers for the compressible case.

        The kinetic corrections satisfy linearized Boltzmann equations. Using a Lénard--Bernstein collision operator, these take Fokker--Planck-like forms \cite{Fokker_1914, Planck_1917} wherein pseudo-particles in the numerical model obey the neoclassical transport equations, with particle-independent Brownian drift terms. This offers a rigorous methodology for incorporating collisions into the particle transport model, without coupling the equations of motions for each particle.
        
        Works by Chen, Chacón et al. \cite{Chen_Chacón_Barnes_2011, Chacón_Chen_Barnes_2013, Chen_Chacón_2014, Chen_Chacón_2015} have developed structure-preserving particle pushers for neoclassical transport in the Vlasov equations, derived from Crank--Nicolson integrators. We show these too can can derive from a FET interpretation, similarly offering potential extensions to higher-order-in-time particle pushers. The FET formulation is used also to consider how the stochastic drift terms can be incorporated into the pushers. Stochastic gyrokinetic expansions are also discussed.

        Different options for the numerical implementation of these schemes are considered.

        Due to the efficacy of FET in the development of SP timesteppers for both the fluid and kinetic component, we hope this approach will prove effective in the future for developing SP timesteppers for the full hybrid model. We hope this will give us the opportunity to incorporate previously inaccessible kinetic effects into the highly effective, modern, finite-element MHD models.
    \end{abstract}
    
    
    \newpage
    \tableofcontents
    
    
    \newpage
    \pagenumbering{arabic}
    %\linenumbers\renewcommand\thelinenumber{\color{black!50}\arabic{linenumber}}
            \input{0 - introduction/main.tex}
        \part{Research}
            \input{1 - low-noise PiC models/main.tex}
            \input{2 - kinetic component/main.tex}
            \input{3 - fluid component/main.tex}
            \input{4 - numerical implementation/main.tex}
        \part{Project Overview}
            \input{5 - research plan/main.tex}
            \input{6 - summary/main.tex}
    
    
    %\section{}
    \newpage
    \pagenumbering{gobble}
        \printbibliography


    \newpage
    \pagenumbering{roman}
    \appendix
        \part{Appendices}
            \input{8 - Hilbert complexes/main.tex}
            \input{9 - weak conservation proofs/main.tex}
\end{document}

\end{document}

            \documentclass[12pt, a4paper]{report}

\documentclass[12pt, a4paper]{report}

\input{template/main.tex}

\title{\BA{Title in Progress...}}
\author{Boris Andrews}
\affil{Mathematical Institute, University of Oxford}
\date{\today}


\begin{document}
    \pagenumbering{gobble}
    \maketitle
    
    
    \begin{abstract}
        Magnetic confinement reactors---in particular tokamaks---offer one of the most promising options for achieving practical nuclear fusion, with the potential to provide virtually limitless, clean energy. The theoretical and numerical modeling of tokamak plasmas is simultaneously an essential component of effective reactor design, and a great research barrier. Tokamak operational conditions exhibit comparatively low Knudsen numbers. Kinetic effects, including kinetic waves and instabilities, Landau damping, bump-on-tail instabilities and more, are therefore highly influential in tokamak plasma dynamics. Purely fluid models are inherently incapable of capturing these effects, whereas the high dimensionality in purely kinetic models render them practically intractable for most relevant purposes.

        We consider a $\delta\!f$ decomposition model, with a macroscopic fluid background and microscopic kinetic correction, both fully coupled to each other. A similar manner of discretization is proposed to that used in the recent \texttt{STRUPHY} code \cite{Holderied_Possanner_Wang_2021, Holderied_2022, Li_et_al_2023} with a finite-element model for the background and a pseudo-particle/PiC model for the correction.

        The fluid background satisfies the full, non-linear, resistive, compressible, Hall MHD equations. \cite{Laakmann_Hu_Farrell_2022} introduces finite-element(-in-space) implicit timesteppers for the incompressible analogue to this system with structure-preserving (SP) properties in the ideal case, alongside parameter-robust preconditioners. We show that these timesteppers can derive from a finite-element-in-time (FET) (and finite-element-in-space) interpretation. The benefits of this reformulation are discussed, including the derivation of timesteppers that are higher order in time, and the quantifiable dissipative SP properties in the non-ideal, resistive case.
        
        We discuss possible options for extending this FET approach to timesteppers for the compressible case.

        The kinetic corrections satisfy linearized Boltzmann equations. Using a Lénard--Bernstein collision operator, these take Fokker--Planck-like forms \cite{Fokker_1914, Planck_1917} wherein pseudo-particles in the numerical model obey the neoclassical transport equations, with particle-independent Brownian drift terms. This offers a rigorous methodology for incorporating collisions into the particle transport model, without coupling the equations of motions for each particle.
        
        Works by Chen, Chacón et al. \cite{Chen_Chacón_Barnes_2011, Chacón_Chen_Barnes_2013, Chen_Chacón_2014, Chen_Chacón_2015} have developed structure-preserving particle pushers for neoclassical transport in the Vlasov equations, derived from Crank--Nicolson integrators. We show these too can can derive from a FET interpretation, similarly offering potential extensions to higher-order-in-time particle pushers. The FET formulation is used also to consider how the stochastic drift terms can be incorporated into the pushers. Stochastic gyrokinetic expansions are also discussed.

        Different options for the numerical implementation of these schemes are considered.

        Due to the efficacy of FET in the development of SP timesteppers for both the fluid and kinetic component, we hope this approach will prove effective in the future for developing SP timesteppers for the full hybrid model. We hope this will give us the opportunity to incorporate previously inaccessible kinetic effects into the highly effective, modern, finite-element MHD models.
    \end{abstract}
    
    
    \newpage
    \tableofcontents
    
    
    \newpage
    \pagenumbering{arabic}
    %\linenumbers\renewcommand\thelinenumber{\color{black!50}\arabic{linenumber}}
            \input{0 - introduction/main.tex}
        \part{Research}
            \input{1 - low-noise PiC models/main.tex}
            \input{2 - kinetic component/main.tex}
            \input{3 - fluid component/main.tex}
            \input{4 - numerical implementation/main.tex}
        \part{Project Overview}
            \input{5 - research plan/main.tex}
            \input{6 - summary/main.tex}
    
    
    %\section{}
    \newpage
    \pagenumbering{gobble}
        \printbibliography


    \newpage
    \pagenumbering{roman}
    \appendix
        \part{Appendices}
            \input{8 - Hilbert complexes/main.tex}
            \input{9 - weak conservation proofs/main.tex}
\end{document}


\title{\BA{Title in Progress...}}
\author{Boris Andrews}
\affil{Mathematical Institute, University of Oxford}
\date{\today}


\begin{document}
    \pagenumbering{gobble}
    \maketitle
    
    
    \begin{abstract}
        Magnetic confinement reactors---in particular tokamaks---offer one of the most promising options for achieving practical nuclear fusion, with the potential to provide virtually limitless, clean energy. The theoretical and numerical modeling of tokamak plasmas is simultaneously an essential component of effective reactor design, and a great research barrier. Tokamak operational conditions exhibit comparatively low Knudsen numbers. Kinetic effects, including kinetic waves and instabilities, Landau damping, bump-on-tail instabilities and more, are therefore highly influential in tokamak plasma dynamics. Purely fluid models are inherently incapable of capturing these effects, whereas the high dimensionality in purely kinetic models render them practically intractable for most relevant purposes.

        We consider a $\delta\!f$ decomposition model, with a macroscopic fluid background and microscopic kinetic correction, both fully coupled to each other. A similar manner of discretization is proposed to that used in the recent \texttt{STRUPHY} code \cite{Holderied_Possanner_Wang_2021, Holderied_2022, Li_et_al_2023} with a finite-element model for the background and a pseudo-particle/PiC model for the correction.

        The fluid background satisfies the full, non-linear, resistive, compressible, Hall MHD equations. \cite{Laakmann_Hu_Farrell_2022} introduces finite-element(-in-space) implicit timesteppers for the incompressible analogue to this system with structure-preserving (SP) properties in the ideal case, alongside parameter-robust preconditioners. We show that these timesteppers can derive from a finite-element-in-time (FET) (and finite-element-in-space) interpretation. The benefits of this reformulation are discussed, including the derivation of timesteppers that are higher order in time, and the quantifiable dissipative SP properties in the non-ideal, resistive case.
        
        We discuss possible options for extending this FET approach to timesteppers for the compressible case.

        The kinetic corrections satisfy linearized Boltzmann equations. Using a Lénard--Bernstein collision operator, these take Fokker--Planck-like forms \cite{Fokker_1914, Planck_1917} wherein pseudo-particles in the numerical model obey the neoclassical transport equations, with particle-independent Brownian drift terms. This offers a rigorous methodology for incorporating collisions into the particle transport model, without coupling the equations of motions for each particle.
        
        Works by Chen, Chacón et al. \cite{Chen_Chacón_Barnes_2011, Chacón_Chen_Barnes_2013, Chen_Chacón_2014, Chen_Chacón_2015} have developed structure-preserving particle pushers for neoclassical transport in the Vlasov equations, derived from Crank--Nicolson integrators. We show these too can can derive from a FET interpretation, similarly offering potential extensions to higher-order-in-time particle pushers. The FET formulation is used also to consider how the stochastic drift terms can be incorporated into the pushers. Stochastic gyrokinetic expansions are also discussed.

        Different options for the numerical implementation of these schemes are considered.

        Due to the efficacy of FET in the development of SP timesteppers for both the fluid and kinetic component, we hope this approach will prove effective in the future for developing SP timesteppers for the full hybrid model. We hope this will give us the opportunity to incorporate previously inaccessible kinetic effects into the highly effective, modern, finite-element MHD models.
    \end{abstract}
    
    
    \newpage
    \tableofcontents
    
    
    \newpage
    \pagenumbering{arabic}
    %\linenumbers\renewcommand\thelinenumber{\color{black!50}\arabic{linenumber}}
            \documentclass[12pt, a4paper]{report}

\input{template/main.tex}

\title{\BA{Title in Progress...}}
\author{Boris Andrews}
\affil{Mathematical Institute, University of Oxford}
\date{\today}


\begin{document}
    \pagenumbering{gobble}
    \maketitle
    
    
    \begin{abstract}
        Magnetic confinement reactors---in particular tokamaks---offer one of the most promising options for achieving practical nuclear fusion, with the potential to provide virtually limitless, clean energy. The theoretical and numerical modeling of tokamak plasmas is simultaneously an essential component of effective reactor design, and a great research barrier. Tokamak operational conditions exhibit comparatively low Knudsen numbers. Kinetic effects, including kinetic waves and instabilities, Landau damping, bump-on-tail instabilities and more, are therefore highly influential in tokamak plasma dynamics. Purely fluid models are inherently incapable of capturing these effects, whereas the high dimensionality in purely kinetic models render them practically intractable for most relevant purposes.

        We consider a $\delta\!f$ decomposition model, with a macroscopic fluid background and microscopic kinetic correction, both fully coupled to each other. A similar manner of discretization is proposed to that used in the recent \texttt{STRUPHY} code \cite{Holderied_Possanner_Wang_2021, Holderied_2022, Li_et_al_2023} with a finite-element model for the background and a pseudo-particle/PiC model for the correction.

        The fluid background satisfies the full, non-linear, resistive, compressible, Hall MHD equations. \cite{Laakmann_Hu_Farrell_2022} introduces finite-element(-in-space) implicit timesteppers for the incompressible analogue to this system with structure-preserving (SP) properties in the ideal case, alongside parameter-robust preconditioners. We show that these timesteppers can derive from a finite-element-in-time (FET) (and finite-element-in-space) interpretation. The benefits of this reformulation are discussed, including the derivation of timesteppers that are higher order in time, and the quantifiable dissipative SP properties in the non-ideal, resistive case.
        
        We discuss possible options for extending this FET approach to timesteppers for the compressible case.

        The kinetic corrections satisfy linearized Boltzmann equations. Using a Lénard--Bernstein collision operator, these take Fokker--Planck-like forms \cite{Fokker_1914, Planck_1917} wherein pseudo-particles in the numerical model obey the neoclassical transport equations, with particle-independent Brownian drift terms. This offers a rigorous methodology for incorporating collisions into the particle transport model, without coupling the equations of motions for each particle.
        
        Works by Chen, Chacón et al. \cite{Chen_Chacón_Barnes_2011, Chacón_Chen_Barnes_2013, Chen_Chacón_2014, Chen_Chacón_2015} have developed structure-preserving particle pushers for neoclassical transport in the Vlasov equations, derived from Crank--Nicolson integrators. We show these too can can derive from a FET interpretation, similarly offering potential extensions to higher-order-in-time particle pushers. The FET formulation is used also to consider how the stochastic drift terms can be incorporated into the pushers. Stochastic gyrokinetic expansions are also discussed.

        Different options for the numerical implementation of these schemes are considered.

        Due to the efficacy of FET in the development of SP timesteppers for both the fluid and kinetic component, we hope this approach will prove effective in the future for developing SP timesteppers for the full hybrid model. We hope this will give us the opportunity to incorporate previously inaccessible kinetic effects into the highly effective, modern, finite-element MHD models.
    \end{abstract}
    
    
    \newpage
    \tableofcontents
    
    
    \newpage
    \pagenumbering{arabic}
    %\linenumbers\renewcommand\thelinenumber{\color{black!50}\arabic{linenumber}}
            \input{0 - introduction/main.tex}
        \part{Research}
            \input{1 - low-noise PiC models/main.tex}
            \input{2 - kinetic component/main.tex}
            \input{3 - fluid component/main.tex}
            \input{4 - numerical implementation/main.tex}
        \part{Project Overview}
            \input{5 - research plan/main.tex}
            \input{6 - summary/main.tex}
    
    
    %\section{}
    \newpage
    \pagenumbering{gobble}
        \printbibliography


    \newpage
    \pagenumbering{roman}
    \appendix
        \part{Appendices}
            \input{8 - Hilbert complexes/main.tex}
            \input{9 - weak conservation proofs/main.tex}
\end{document}

        \part{Research}
            \documentclass[12pt, a4paper]{report}

\input{template/main.tex}

\title{\BA{Title in Progress...}}
\author{Boris Andrews}
\affil{Mathematical Institute, University of Oxford}
\date{\today}


\begin{document}
    \pagenumbering{gobble}
    \maketitle
    
    
    \begin{abstract}
        Magnetic confinement reactors---in particular tokamaks---offer one of the most promising options for achieving practical nuclear fusion, with the potential to provide virtually limitless, clean energy. The theoretical and numerical modeling of tokamak plasmas is simultaneously an essential component of effective reactor design, and a great research barrier. Tokamak operational conditions exhibit comparatively low Knudsen numbers. Kinetic effects, including kinetic waves and instabilities, Landau damping, bump-on-tail instabilities and more, are therefore highly influential in tokamak plasma dynamics. Purely fluid models are inherently incapable of capturing these effects, whereas the high dimensionality in purely kinetic models render them practically intractable for most relevant purposes.

        We consider a $\delta\!f$ decomposition model, with a macroscopic fluid background and microscopic kinetic correction, both fully coupled to each other. A similar manner of discretization is proposed to that used in the recent \texttt{STRUPHY} code \cite{Holderied_Possanner_Wang_2021, Holderied_2022, Li_et_al_2023} with a finite-element model for the background and a pseudo-particle/PiC model for the correction.

        The fluid background satisfies the full, non-linear, resistive, compressible, Hall MHD equations. \cite{Laakmann_Hu_Farrell_2022} introduces finite-element(-in-space) implicit timesteppers for the incompressible analogue to this system with structure-preserving (SP) properties in the ideal case, alongside parameter-robust preconditioners. We show that these timesteppers can derive from a finite-element-in-time (FET) (and finite-element-in-space) interpretation. The benefits of this reformulation are discussed, including the derivation of timesteppers that are higher order in time, and the quantifiable dissipative SP properties in the non-ideal, resistive case.
        
        We discuss possible options for extending this FET approach to timesteppers for the compressible case.

        The kinetic corrections satisfy linearized Boltzmann equations. Using a Lénard--Bernstein collision operator, these take Fokker--Planck-like forms \cite{Fokker_1914, Planck_1917} wherein pseudo-particles in the numerical model obey the neoclassical transport equations, with particle-independent Brownian drift terms. This offers a rigorous methodology for incorporating collisions into the particle transport model, without coupling the equations of motions for each particle.
        
        Works by Chen, Chacón et al. \cite{Chen_Chacón_Barnes_2011, Chacón_Chen_Barnes_2013, Chen_Chacón_2014, Chen_Chacón_2015} have developed structure-preserving particle pushers for neoclassical transport in the Vlasov equations, derived from Crank--Nicolson integrators. We show these too can can derive from a FET interpretation, similarly offering potential extensions to higher-order-in-time particle pushers. The FET formulation is used also to consider how the stochastic drift terms can be incorporated into the pushers. Stochastic gyrokinetic expansions are also discussed.

        Different options for the numerical implementation of these schemes are considered.

        Due to the efficacy of FET in the development of SP timesteppers for both the fluid and kinetic component, we hope this approach will prove effective in the future for developing SP timesteppers for the full hybrid model. We hope this will give us the opportunity to incorporate previously inaccessible kinetic effects into the highly effective, modern, finite-element MHD models.
    \end{abstract}
    
    
    \newpage
    \tableofcontents
    
    
    \newpage
    \pagenumbering{arabic}
    %\linenumbers\renewcommand\thelinenumber{\color{black!50}\arabic{linenumber}}
            \input{0 - introduction/main.tex}
        \part{Research}
            \input{1 - low-noise PiC models/main.tex}
            \input{2 - kinetic component/main.tex}
            \input{3 - fluid component/main.tex}
            \input{4 - numerical implementation/main.tex}
        \part{Project Overview}
            \input{5 - research plan/main.tex}
            \input{6 - summary/main.tex}
    
    
    %\section{}
    \newpage
    \pagenumbering{gobble}
        \printbibliography


    \newpage
    \pagenumbering{roman}
    \appendix
        \part{Appendices}
            \input{8 - Hilbert complexes/main.tex}
            \input{9 - weak conservation proofs/main.tex}
\end{document}

            \documentclass[12pt, a4paper]{report}

\input{template/main.tex}

\title{\BA{Title in Progress...}}
\author{Boris Andrews}
\affil{Mathematical Institute, University of Oxford}
\date{\today}


\begin{document}
    \pagenumbering{gobble}
    \maketitle
    
    
    \begin{abstract}
        Magnetic confinement reactors---in particular tokamaks---offer one of the most promising options for achieving practical nuclear fusion, with the potential to provide virtually limitless, clean energy. The theoretical and numerical modeling of tokamak plasmas is simultaneously an essential component of effective reactor design, and a great research barrier. Tokamak operational conditions exhibit comparatively low Knudsen numbers. Kinetic effects, including kinetic waves and instabilities, Landau damping, bump-on-tail instabilities and more, are therefore highly influential in tokamak plasma dynamics. Purely fluid models are inherently incapable of capturing these effects, whereas the high dimensionality in purely kinetic models render them practically intractable for most relevant purposes.

        We consider a $\delta\!f$ decomposition model, with a macroscopic fluid background and microscopic kinetic correction, both fully coupled to each other. A similar manner of discretization is proposed to that used in the recent \texttt{STRUPHY} code \cite{Holderied_Possanner_Wang_2021, Holderied_2022, Li_et_al_2023} with a finite-element model for the background and a pseudo-particle/PiC model for the correction.

        The fluid background satisfies the full, non-linear, resistive, compressible, Hall MHD equations. \cite{Laakmann_Hu_Farrell_2022} introduces finite-element(-in-space) implicit timesteppers for the incompressible analogue to this system with structure-preserving (SP) properties in the ideal case, alongside parameter-robust preconditioners. We show that these timesteppers can derive from a finite-element-in-time (FET) (and finite-element-in-space) interpretation. The benefits of this reformulation are discussed, including the derivation of timesteppers that are higher order in time, and the quantifiable dissipative SP properties in the non-ideal, resistive case.
        
        We discuss possible options for extending this FET approach to timesteppers for the compressible case.

        The kinetic corrections satisfy linearized Boltzmann equations. Using a Lénard--Bernstein collision operator, these take Fokker--Planck-like forms \cite{Fokker_1914, Planck_1917} wherein pseudo-particles in the numerical model obey the neoclassical transport equations, with particle-independent Brownian drift terms. This offers a rigorous methodology for incorporating collisions into the particle transport model, without coupling the equations of motions for each particle.
        
        Works by Chen, Chacón et al. \cite{Chen_Chacón_Barnes_2011, Chacón_Chen_Barnes_2013, Chen_Chacón_2014, Chen_Chacón_2015} have developed structure-preserving particle pushers for neoclassical transport in the Vlasov equations, derived from Crank--Nicolson integrators. We show these too can can derive from a FET interpretation, similarly offering potential extensions to higher-order-in-time particle pushers. The FET formulation is used also to consider how the stochastic drift terms can be incorporated into the pushers. Stochastic gyrokinetic expansions are also discussed.

        Different options for the numerical implementation of these schemes are considered.

        Due to the efficacy of FET in the development of SP timesteppers for both the fluid and kinetic component, we hope this approach will prove effective in the future for developing SP timesteppers for the full hybrid model. We hope this will give us the opportunity to incorporate previously inaccessible kinetic effects into the highly effective, modern, finite-element MHD models.
    \end{abstract}
    
    
    \newpage
    \tableofcontents
    
    
    \newpage
    \pagenumbering{arabic}
    %\linenumbers\renewcommand\thelinenumber{\color{black!50}\arabic{linenumber}}
            \input{0 - introduction/main.tex}
        \part{Research}
            \input{1 - low-noise PiC models/main.tex}
            \input{2 - kinetic component/main.tex}
            \input{3 - fluid component/main.tex}
            \input{4 - numerical implementation/main.tex}
        \part{Project Overview}
            \input{5 - research plan/main.tex}
            \input{6 - summary/main.tex}
    
    
    %\section{}
    \newpage
    \pagenumbering{gobble}
        \printbibliography


    \newpage
    \pagenumbering{roman}
    \appendix
        \part{Appendices}
            \input{8 - Hilbert complexes/main.tex}
            \input{9 - weak conservation proofs/main.tex}
\end{document}

            \documentclass[12pt, a4paper]{report}

\input{template/main.tex}

\title{\BA{Title in Progress...}}
\author{Boris Andrews}
\affil{Mathematical Institute, University of Oxford}
\date{\today}


\begin{document}
    \pagenumbering{gobble}
    \maketitle
    
    
    \begin{abstract}
        Magnetic confinement reactors---in particular tokamaks---offer one of the most promising options for achieving practical nuclear fusion, with the potential to provide virtually limitless, clean energy. The theoretical and numerical modeling of tokamak plasmas is simultaneously an essential component of effective reactor design, and a great research barrier. Tokamak operational conditions exhibit comparatively low Knudsen numbers. Kinetic effects, including kinetic waves and instabilities, Landau damping, bump-on-tail instabilities and more, are therefore highly influential in tokamak plasma dynamics. Purely fluid models are inherently incapable of capturing these effects, whereas the high dimensionality in purely kinetic models render them practically intractable for most relevant purposes.

        We consider a $\delta\!f$ decomposition model, with a macroscopic fluid background and microscopic kinetic correction, both fully coupled to each other. A similar manner of discretization is proposed to that used in the recent \texttt{STRUPHY} code \cite{Holderied_Possanner_Wang_2021, Holderied_2022, Li_et_al_2023} with a finite-element model for the background and a pseudo-particle/PiC model for the correction.

        The fluid background satisfies the full, non-linear, resistive, compressible, Hall MHD equations. \cite{Laakmann_Hu_Farrell_2022} introduces finite-element(-in-space) implicit timesteppers for the incompressible analogue to this system with structure-preserving (SP) properties in the ideal case, alongside parameter-robust preconditioners. We show that these timesteppers can derive from a finite-element-in-time (FET) (and finite-element-in-space) interpretation. The benefits of this reformulation are discussed, including the derivation of timesteppers that are higher order in time, and the quantifiable dissipative SP properties in the non-ideal, resistive case.
        
        We discuss possible options for extending this FET approach to timesteppers for the compressible case.

        The kinetic corrections satisfy linearized Boltzmann equations. Using a Lénard--Bernstein collision operator, these take Fokker--Planck-like forms \cite{Fokker_1914, Planck_1917} wherein pseudo-particles in the numerical model obey the neoclassical transport equations, with particle-independent Brownian drift terms. This offers a rigorous methodology for incorporating collisions into the particle transport model, without coupling the equations of motions for each particle.
        
        Works by Chen, Chacón et al. \cite{Chen_Chacón_Barnes_2011, Chacón_Chen_Barnes_2013, Chen_Chacón_2014, Chen_Chacón_2015} have developed structure-preserving particle pushers for neoclassical transport in the Vlasov equations, derived from Crank--Nicolson integrators. We show these too can can derive from a FET interpretation, similarly offering potential extensions to higher-order-in-time particle pushers. The FET formulation is used also to consider how the stochastic drift terms can be incorporated into the pushers. Stochastic gyrokinetic expansions are also discussed.

        Different options for the numerical implementation of these schemes are considered.

        Due to the efficacy of FET in the development of SP timesteppers for both the fluid and kinetic component, we hope this approach will prove effective in the future for developing SP timesteppers for the full hybrid model. We hope this will give us the opportunity to incorporate previously inaccessible kinetic effects into the highly effective, modern, finite-element MHD models.
    \end{abstract}
    
    
    \newpage
    \tableofcontents
    
    
    \newpage
    \pagenumbering{arabic}
    %\linenumbers\renewcommand\thelinenumber{\color{black!50}\arabic{linenumber}}
            \input{0 - introduction/main.tex}
        \part{Research}
            \input{1 - low-noise PiC models/main.tex}
            \input{2 - kinetic component/main.tex}
            \input{3 - fluid component/main.tex}
            \input{4 - numerical implementation/main.tex}
        \part{Project Overview}
            \input{5 - research plan/main.tex}
            \input{6 - summary/main.tex}
    
    
    %\section{}
    \newpage
    \pagenumbering{gobble}
        \printbibliography


    \newpage
    \pagenumbering{roman}
    \appendix
        \part{Appendices}
            \input{8 - Hilbert complexes/main.tex}
            \input{9 - weak conservation proofs/main.tex}
\end{document}

            \documentclass[12pt, a4paper]{report}

\input{template/main.tex}

\title{\BA{Title in Progress...}}
\author{Boris Andrews}
\affil{Mathematical Institute, University of Oxford}
\date{\today}


\begin{document}
    \pagenumbering{gobble}
    \maketitle
    
    
    \begin{abstract}
        Magnetic confinement reactors---in particular tokamaks---offer one of the most promising options for achieving practical nuclear fusion, with the potential to provide virtually limitless, clean energy. The theoretical and numerical modeling of tokamak plasmas is simultaneously an essential component of effective reactor design, and a great research barrier. Tokamak operational conditions exhibit comparatively low Knudsen numbers. Kinetic effects, including kinetic waves and instabilities, Landau damping, bump-on-tail instabilities and more, are therefore highly influential in tokamak plasma dynamics. Purely fluid models are inherently incapable of capturing these effects, whereas the high dimensionality in purely kinetic models render them practically intractable for most relevant purposes.

        We consider a $\delta\!f$ decomposition model, with a macroscopic fluid background and microscopic kinetic correction, both fully coupled to each other. A similar manner of discretization is proposed to that used in the recent \texttt{STRUPHY} code \cite{Holderied_Possanner_Wang_2021, Holderied_2022, Li_et_al_2023} with a finite-element model for the background and a pseudo-particle/PiC model for the correction.

        The fluid background satisfies the full, non-linear, resistive, compressible, Hall MHD equations. \cite{Laakmann_Hu_Farrell_2022} introduces finite-element(-in-space) implicit timesteppers for the incompressible analogue to this system with structure-preserving (SP) properties in the ideal case, alongside parameter-robust preconditioners. We show that these timesteppers can derive from a finite-element-in-time (FET) (and finite-element-in-space) interpretation. The benefits of this reformulation are discussed, including the derivation of timesteppers that are higher order in time, and the quantifiable dissipative SP properties in the non-ideal, resistive case.
        
        We discuss possible options for extending this FET approach to timesteppers for the compressible case.

        The kinetic corrections satisfy linearized Boltzmann equations. Using a Lénard--Bernstein collision operator, these take Fokker--Planck-like forms \cite{Fokker_1914, Planck_1917} wherein pseudo-particles in the numerical model obey the neoclassical transport equations, with particle-independent Brownian drift terms. This offers a rigorous methodology for incorporating collisions into the particle transport model, without coupling the equations of motions for each particle.
        
        Works by Chen, Chacón et al. \cite{Chen_Chacón_Barnes_2011, Chacón_Chen_Barnes_2013, Chen_Chacón_2014, Chen_Chacón_2015} have developed structure-preserving particle pushers for neoclassical transport in the Vlasov equations, derived from Crank--Nicolson integrators. We show these too can can derive from a FET interpretation, similarly offering potential extensions to higher-order-in-time particle pushers. The FET formulation is used also to consider how the stochastic drift terms can be incorporated into the pushers. Stochastic gyrokinetic expansions are also discussed.

        Different options for the numerical implementation of these schemes are considered.

        Due to the efficacy of FET in the development of SP timesteppers for both the fluid and kinetic component, we hope this approach will prove effective in the future for developing SP timesteppers for the full hybrid model. We hope this will give us the opportunity to incorporate previously inaccessible kinetic effects into the highly effective, modern, finite-element MHD models.
    \end{abstract}
    
    
    \newpage
    \tableofcontents
    
    
    \newpage
    \pagenumbering{arabic}
    %\linenumbers\renewcommand\thelinenumber{\color{black!50}\arabic{linenumber}}
            \input{0 - introduction/main.tex}
        \part{Research}
            \input{1 - low-noise PiC models/main.tex}
            \input{2 - kinetic component/main.tex}
            \input{3 - fluid component/main.tex}
            \input{4 - numerical implementation/main.tex}
        \part{Project Overview}
            \input{5 - research plan/main.tex}
            \input{6 - summary/main.tex}
    
    
    %\section{}
    \newpage
    \pagenumbering{gobble}
        \printbibliography


    \newpage
    \pagenumbering{roman}
    \appendix
        \part{Appendices}
            \input{8 - Hilbert complexes/main.tex}
            \input{9 - weak conservation proofs/main.tex}
\end{document}

        \part{Project Overview}
            \documentclass[12pt, a4paper]{report}

\input{template/main.tex}

\title{\BA{Title in Progress...}}
\author{Boris Andrews}
\affil{Mathematical Institute, University of Oxford}
\date{\today}


\begin{document}
    \pagenumbering{gobble}
    \maketitle
    
    
    \begin{abstract}
        Magnetic confinement reactors---in particular tokamaks---offer one of the most promising options for achieving practical nuclear fusion, with the potential to provide virtually limitless, clean energy. The theoretical and numerical modeling of tokamak plasmas is simultaneously an essential component of effective reactor design, and a great research barrier. Tokamak operational conditions exhibit comparatively low Knudsen numbers. Kinetic effects, including kinetic waves and instabilities, Landau damping, bump-on-tail instabilities and more, are therefore highly influential in tokamak plasma dynamics. Purely fluid models are inherently incapable of capturing these effects, whereas the high dimensionality in purely kinetic models render them practically intractable for most relevant purposes.

        We consider a $\delta\!f$ decomposition model, with a macroscopic fluid background and microscopic kinetic correction, both fully coupled to each other. A similar manner of discretization is proposed to that used in the recent \texttt{STRUPHY} code \cite{Holderied_Possanner_Wang_2021, Holderied_2022, Li_et_al_2023} with a finite-element model for the background and a pseudo-particle/PiC model for the correction.

        The fluid background satisfies the full, non-linear, resistive, compressible, Hall MHD equations. \cite{Laakmann_Hu_Farrell_2022} introduces finite-element(-in-space) implicit timesteppers for the incompressible analogue to this system with structure-preserving (SP) properties in the ideal case, alongside parameter-robust preconditioners. We show that these timesteppers can derive from a finite-element-in-time (FET) (and finite-element-in-space) interpretation. The benefits of this reformulation are discussed, including the derivation of timesteppers that are higher order in time, and the quantifiable dissipative SP properties in the non-ideal, resistive case.
        
        We discuss possible options for extending this FET approach to timesteppers for the compressible case.

        The kinetic corrections satisfy linearized Boltzmann equations. Using a Lénard--Bernstein collision operator, these take Fokker--Planck-like forms \cite{Fokker_1914, Planck_1917} wherein pseudo-particles in the numerical model obey the neoclassical transport equations, with particle-independent Brownian drift terms. This offers a rigorous methodology for incorporating collisions into the particle transport model, without coupling the equations of motions for each particle.
        
        Works by Chen, Chacón et al. \cite{Chen_Chacón_Barnes_2011, Chacón_Chen_Barnes_2013, Chen_Chacón_2014, Chen_Chacón_2015} have developed structure-preserving particle pushers for neoclassical transport in the Vlasov equations, derived from Crank--Nicolson integrators. We show these too can can derive from a FET interpretation, similarly offering potential extensions to higher-order-in-time particle pushers. The FET formulation is used also to consider how the stochastic drift terms can be incorporated into the pushers. Stochastic gyrokinetic expansions are also discussed.

        Different options for the numerical implementation of these schemes are considered.

        Due to the efficacy of FET in the development of SP timesteppers for both the fluid and kinetic component, we hope this approach will prove effective in the future for developing SP timesteppers for the full hybrid model. We hope this will give us the opportunity to incorporate previously inaccessible kinetic effects into the highly effective, modern, finite-element MHD models.
    \end{abstract}
    
    
    \newpage
    \tableofcontents
    
    
    \newpage
    \pagenumbering{arabic}
    %\linenumbers\renewcommand\thelinenumber{\color{black!50}\arabic{linenumber}}
            \input{0 - introduction/main.tex}
        \part{Research}
            \input{1 - low-noise PiC models/main.tex}
            \input{2 - kinetic component/main.tex}
            \input{3 - fluid component/main.tex}
            \input{4 - numerical implementation/main.tex}
        \part{Project Overview}
            \input{5 - research plan/main.tex}
            \input{6 - summary/main.tex}
    
    
    %\section{}
    \newpage
    \pagenumbering{gobble}
        \printbibliography


    \newpage
    \pagenumbering{roman}
    \appendix
        \part{Appendices}
            \input{8 - Hilbert complexes/main.tex}
            \input{9 - weak conservation proofs/main.tex}
\end{document}

            \documentclass[12pt, a4paper]{report}

\input{template/main.tex}

\title{\BA{Title in Progress...}}
\author{Boris Andrews}
\affil{Mathematical Institute, University of Oxford}
\date{\today}


\begin{document}
    \pagenumbering{gobble}
    \maketitle
    
    
    \begin{abstract}
        Magnetic confinement reactors---in particular tokamaks---offer one of the most promising options for achieving practical nuclear fusion, with the potential to provide virtually limitless, clean energy. The theoretical and numerical modeling of tokamak plasmas is simultaneously an essential component of effective reactor design, and a great research barrier. Tokamak operational conditions exhibit comparatively low Knudsen numbers. Kinetic effects, including kinetic waves and instabilities, Landau damping, bump-on-tail instabilities and more, are therefore highly influential in tokamak plasma dynamics. Purely fluid models are inherently incapable of capturing these effects, whereas the high dimensionality in purely kinetic models render them practically intractable for most relevant purposes.

        We consider a $\delta\!f$ decomposition model, with a macroscopic fluid background and microscopic kinetic correction, both fully coupled to each other. A similar manner of discretization is proposed to that used in the recent \texttt{STRUPHY} code \cite{Holderied_Possanner_Wang_2021, Holderied_2022, Li_et_al_2023} with a finite-element model for the background and a pseudo-particle/PiC model for the correction.

        The fluid background satisfies the full, non-linear, resistive, compressible, Hall MHD equations. \cite{Laakmann_Hu_Farrell_2022} introduces finite-element(-in-space) implicit timesteppers for the incompressible analogue to this system with structure-preserving (SP) properties in the ideal case, alongside parameter-robust preconditioners. We show that these timesteppers can derive from a finite-element-in-time (FET) (and finite-element-in-space) interpretation. The benefits of this reformulation are discussed, including the derivation of timesteppers that are higher order in time, and the quantifiable dissipative SP properties in the non-ideal, resistive case.
        
        We discuss possible options for extending this FET approach to timesteppers for the compressible case.

        The kinetic corrections satisfy linearized Boltzmann equations. Using a Lénard--Bernstein collision operator, these take Fokker--Planck-like forms \cite{Fokker_1914, Planck_1917} wherein pseudo-particles in the numerical model obey the neoclassical transport equations, with particle-independent Brownian drift terms. This offers a rigorous methodology for incorporating collisions into the particle transport model, without coupling the equations of motions for each particle.
        
        Works by Chen, Chacón et al. \cite{Chen_Chacón_Barnes_2011, Chacón_Chen_Barnes_2013, Chen_Chacón_2014, Chen_Chacón_2015} have developed structure-preserving particle pushers for neoclassical transport in the Vlasov equations, derived from Crank--Nicolson integrators. We show these too can can derive from a FET interpretation, similarly offering potential extensions to higher-order-in-time particle pushers. The FET formulation is used also to consider how the stochastic drift terms can be incorporated into the pushers. Stochastic gyrokinetic expansions are also discussed.

        Different options for the numerical implementation of these schemes are considered.

        Due to the efficacy of FET in the development of SP timesteppers for both the fluid and kinetic component, we hope this approach will prove effective in the future for developing SP timesteppers for the full hybrid model. We hope this will give us the opportunity to incorporate previously inaccessible kinetic effects into the highly effective, modern, finite-element MHD models.
    \end{abstract}
    
    
    \newpage
    \tableofcontents
    
    
    \newpage
    \pagenumbering{arabic}
    %\linenumbers\renewcommand\thelinenumber{\color{black!50}\arabic{linenumber}}
            \input{0 - introduction/main.tex}
        \part{Research}
            \input{1 - low-noise PiC models/main.tex}
            \input{2 - kinetic component/main.tex}
            \input{3 - fluid component/main.tex}
            \input{4 - numerical implementation/main.tex}
        \part{Project Overview}
            \input{5 - research plan/main.tex}
            \input{6 - summary/main.tex}
    
    
    %\section{}
    \newpage
    \pagenumbering{gobble}
        \printbibliography


    \newpage
    \pagenumbering{roman}
    \appendix
        \part{Appendices}
            \input{8 - Hilbert complexes/main.tex}
            \input{9 - weak conservation proofs/main.tex}
\end{document}

    
    
    %\section{}
    \newpage
    \pagenumbering{gobble}
        \printbibliography


    \newpage
    \pagenumbering{roman}
    \appendix
        \part{Appendices}
            \documentclass[12pt, a4paper]{report}

\input{template/main.tex}

\title{\BA{Title in Progress...}}
\author{Boris Andrews}
\affil{Mathematical Institute, University of Oxford}
\date{\today}


\begin{document}
    \pagenumbering{gobble}
    \maketitle
    
    
    \begin{abstract}
        Magnetic confinement reactors---in particular tokamaks---offer one of the most promising options for achieving practical nuclear fusion, with the potential to provide virtually limitless, clean energy. The theoretical and numerical modeling of tokamak plasmas is simultaneously an essential component of effective reactor design, and a great research barrier. Tokamak operational conditions exhibit comparatively low Knudsen numbers. Kinetic effects, including kinetic waves and instabilities, Landau damping, bump-on-tail instabilities and more, are therefore highly influential in tokamak plasma dynamics. Purely fluid models are inherently incapable of capturing these effects, whereas the high dimensionality in purely kinetic models render them practically intractable for most relevant purposes.

        We consider a $\delta\!f$ decomposition model, with a macroscopic fluid background and microscopic kinetic correction, both fully coupled to each other. A similar manner of discretization is proposed to that used in the recent \texttt{STRUPHY} code \cite{Holderied_Possanner_Wang_2021, Holderied_2022, Li_et_al_2023} with a finite-element model for the background and a pseudo-particle/PiC model for the correction.

        The fluid background satisfies the full, non-linear, resistive, compressible, Hall MHD equations. \cite{Laakmann_Hu_Farrell_2022} introduces finite-element(-in-space) implicit timesteppers for the incompressible analogue to this system with structure-preserving (SP) properties in the ideal case, alongside parameter-robust preconditioners. We show that these timesteppers can derive from a finite-element-in-time (FET) (and finite-element-in-space) interpretation. The benefits of this reformulation are discussed, including the derivation of timesteppers that are higher order in time, and the quantifiable dissipative SP properties in the non-ideal, resistive case.
        
        We discuss possible options for extending this FET approach to timesteppers for the compressible case.

        The kinetic corrections satisfy linearized Boltzmann equations. Using a Lénard--Bernstein collision operator, these take Fokker--Planck-like forms \cite{Fokker_1914, Planck_1917} wherein pseudo-particles in the numerical model obey the neoclassical transport equations, with particle-independent Brownian drift terms. This offers a rigorous methodology for incorporating collisions into the particle transport model, without coupling the equations of motions for each particle.
        
        Works by Chen, Chacón et al. \cite{Chen_Chacón_Barnes_2011, Chacón_Chen_Barnes_2013, Chen_Chacón_2014, Chen_Chacón_2015} have developed structure-preserving particle pushers for neoclassical transport in the Vlasov equations, derived from Crank--Nicolson integrators. We show these too can can derive from a FET interpretation, similarly offering potential extensions to higher-order-in-time particle pushers. The FET formulation is used also to consider how the stochastic drift terms can be incorporated into the pushers. Stochastic gyrokinetic expansions are also discussed.

        Different options for the numerical implementation of these schemes are considered.

        Due to the efficacy of FET in the development of SP timesteppers for both the fluid and kinetic component, we hope this approach will prove effective in the future for developing SP timesteppers for the full hybrid model. We hope this will give us the opportunity to incorporate previously inaccessible kinetic effects into the highly effective, modern, finite-element MHD models.
    \end{abstract}
    
    
    \newpage
    \tableofcontents
    
    
    \newpage
    \pagenumbering{arabic}
    %\linenumbers\renewcommand\thelinenumber{\color{black!50}\arabic{linenumber}}
            \input{0 - introduction/main.tex}
        \part{Research}
            \input{1 - low-noise PiC models/main.tex}
            \input{2 - kinetic component/main.tex}
            \input{3 - fluid component/main.tex}
            \input{4 - numerical implementation/main.tex}
        \part{Project Overview}
            \input{5 - research plan/main.tex}
            \input{6 - summary/main.tex}
    
    
    %\section{}
    \newpage
    \pagenumbering{gobble}
        \printbibliography


    \newpage
    \pagenumbering{roman}
    \appendix
        \part{Appendices}
            \input{8 - Hilbert complexes/main.tex}
            \input{9 - weak conservation proofs/main.tex}
\end{document}

            \documentclass[12pt, a4paper]{report}

\input{template/main.tex}

\title{\BA{Title in Progress...}}
\author{Boris Andrews}
\affil{Mathematical Institute, University of Oxford}
\date{\today}


\begin{document}
    \pagenumbering{gobble}
    \maketitle
    
    
    \begin{abstract}
        Magnetic confinement reactors---in particular tokamaks---offer one of the most promising options for achieving practical nuclear fusion, with the potential to provide virtually limitless, clean energy. The theoretical and numerical modeling of tokamak plasmas is simultaneously an essential component of effective reactor design, and a great research barrier. Tokamak operational conditions exhibit comparatively low Knudsen numbers. Kinetic effects, including kinetic waves and instabilities, Landau damping, bump-on-tail instabilities and more, are therefore highly influential in tokamak plasma dynamics. Purely fluid models are inherently incapable of capturing these effects, whereas the high dimensionality in purely kinetic models render them practically intractable for most relevant purposes.

        We consider a $\delta\!f$ decomposition model, with a macroscopic fluid background and microscopic kinetic correction, both fully coupled to each other. A similar manner of discretization is proposed to that used in the recent \texttt{STRUPHY} code \cite{Holderied_Possanner_Wang_2021, Holderied_2022, Li_et_al_2023} with a finite-element model for the background and a pseudo-particle/PiC model for the correction.

        The fluid background satisfies the full, non-linear, resistive, compressible, Hall MHD equations. \cite{Laakmann_Hu_Farrell_2022} introduces finite-element(-in-space) implicit timesteppers for the incompressible analogue to this system with structure-preserving (SP) properties in the ideal case, alongside parameter-robust preconditioners. We show that these timesteppers can derive from a finite-element-in-time (FET) (and finite-element-in-space) interpretation. The benefits of this reformulation are discussed, including the derivation of timesteppers that are higher order in time, and the quantifiable dissipative SP properties in the non-ideal, resistive case.
        
        We discuss possible options for extending this FET approach to timesteppers for the compressible case.

        The kinetic corrections satisfy linearized Boltzmann equations. Using a Lénard--Bernstein collision operator, these take Fokker--Planck-like forms \cite{Fokker_1914, Planck_1917} wherein pseudo-particles in the numerical model obey the neoclassical transport equations, with particle-independent Brownian drift terms. This offers a rigorous methodology for incorporating collisions into the particle transport model, without coupling the equations of motions for each particle.
        
        Works by Chen, Chacón et al. \cite{Chen_Chacón_Barnes_2011, Chacón_Chen_Barnes_2013, Chen_Chacón_2014, Chen_Chacón_2015} have developed structure-preserving particle pushers for neoclassical transport in the Vlasov equations, derived from Crank--Nicolson integrators. We show these too can can derive from a FET interpretation, similarly offering potential extensions to higher-order-in-time particle pushers. The FET formulation is used also to consider how the stochastic drift terms can be incorporated into the pushers. Stochastic gyrokinetic expansions are also discussed.

        Different options for the numerical implementation of these schemes are considered.

        Due to the efficacy of FET in the development of SP timesteppers for both the fluid and kinetic component, we hope this approach will prove effective in the future for developing SP timesteppers for the full hybrid model. We hope this will give us the opportunity to incorporate previously inaccessible kinetic effects into the highly effective, modern, finite-element MHD models.
    \end{abstract}
    
    
    \newpage
    \tableofcontents
    
    
    \newpage
    \pagenumbering{arabic}
    %\linenumbers\renewcommand\thelinenumber{\color{black!50}\arabic{linenumber}}
            \input{0 - introduction/main.tex}
        \part{Research}
            \input{1 - low-noise PiC models/main.tex}
            \input{2 - kinetic component/main.tex}
            \input{3 - fluid component/main.tex}
            \input{4 - numerical implementation/main.tex}
        \part{Project Overview}
            \input{5 - research plan/main.tex}
            \input{6 - summary/main.tex}
    
    
    %\section{}
    \newpage
    \pagenumbering{gobble}
        \printbibliography


    \newpage
    \pagenumbering{roman}
    \appendix
        \part{Appendices}
            \input{8 - Hilbert complexes/main.tex}
            \input{9 - weak conservation proofs/main.tex}
\end{document}

\end{document}

    
    
    %\section{}
    \newpage
    \pagenumbering{gobble}
        \printbibliography


    \newpage
    \pagenumbering{roman}
    \appendix
        \part{Appendices}
            \documentclass[12pt, a4paper]{report}

\documentclass[12pt, a4paper]{report}

\input{template/main.tex}

\title{\BA{Title in Progress...}}
\author{Boris Andrews}
\affil{Mathematical Institute, University of Oxford}
\date{\today}


\begin{document}
    \pagenumbering{gobble}
    \maketitle
    
    
    \begin{abstract}
        Magnetic confinement reactors---in particular tokamaks---offer one of the most promising options for achieving practical nuclear fusion, with the potential to provide virtually limitless, clean energy. The theoretical and numerical modeling of tokamak plasmas is simultaneously an essential component of effective reactor design, and a great research barrier. Tokamak operational conditions exhibit comparatively low Knudsen numbers. Kinetic effects, including kinetic waves and instabilities, Landau damping, bump-on-tail instabilities and more, are therefore highly influential in tokamak plasma dynamics. Purely fluid models are inherently incapable of capturing these effects, whereas the high dimensionality in purely kinetic models render them practically intractable for most relevant purposes.

        We consider a $\delta\!f$ decomposition model, with a macroscopic fluid background and microscopic kinetic correction, both fully coupled to each other. A similar manner of discretization is proposed to that used in the recent \texttt{STRUPHY} code \cite{Holderied_Possanner_Wang_2021, Holderied_2022, Li_et_al_2023} with a finite-element model for the background and a pseudo-particle/PiC model for the correction.

        The fluid background satisfies the full, non-linear, resistive, compressible, Hall MHD equations. \cite{Laakmann_Hu_Farrell_2022} introduces finite-element(-in-space) implicit timesteppers for the incompressible analogue to this system with structure-preserving (SP) properties in the ideal case, alongside parameter-robust preconditioners. We show that these timesteppers can derive from a finite-element-in-time (FET) (and finite-element-in-space) interpretation. The benefits of this reformulation are discussed, including the derivation of timesteppers that are higher order in time, and the quantifiable dissipative SP properties in the non-ideal, resistive case.
        
        We discuss possible options for extending this FET approach to timesteppers for the compressible case.

        The kinetic corrections satisfy linearized Boltzmann equations. Using a Lénard--Bernstein collision operator, these take Fokker--Planck-like forms \cite{Fokker_1914, Planck_1917} wherein pseudo-particles in the numerical model obey the neoclassical transport equations, with particle-independent Brownian drift terms. This offers a rigorous methodology for incorporating collisions into the particle transport model, without coupling the equations of motions for each particle.
        
        Works by Chen, Chacón et al. \cite{Chen_Chacón_Barnes_2011, Chacón_Chen_Barnes_2013, Chen_Chacón_2014, Chen_Chacón_2015} have developed structure-preserving particle pushers for neoclassical transport in the Vlasov equations, derived from Crank--Nicolson integrators. We show these too can can derive from a FET interpretation, similarly offering potential extensions to higher-order-in-time particle pushers. The FET formulation is used also to consider how the stochastic drift terms can be incorporated into the pushers. Stochastic gyrokinetic expansions are also discussed.

        Different options for the numerical implementation of these schemes are considered.

        Due to the efficacy of FET in the development of SP timesteppers for both the fluid and kinetic component, we hope this approach will prove effective in the future for developing SP timesteppers for the full hybrid model. We hope this will give us the opportunity to incorporate previously inaccessible kinetic effects into the highly effective, modern, finite-element MHD models.
    \end{abstract}
    
    
    \newpage
    \tableofcontents
    
    
    \newpage
    \pagenumbering{arabic}
    %\linenumbers\renewcommand\thelinenumber{\color{black!50}\arabic{linenumber}}
            \input{0 - introduction/main.tex}
        \part{Research}
            \input{1 - low-noise PiC models/main.tex}
            \input{2 - kinetic component/main.tex}
            \input{3 - fluid component/main.tex}
            \input{4 - numerical implementation/main.tex}
        \part{Project Overview}
            \input{5 - research plan/main.tex}
            \input{6 - summary/main.tex}
    
    
    %\section{}
    \newpage
    \pagenumbering{gobble}
        \printbibliography


    \newpage
    \pagenumbering{roman}
    \appendix
        \part{Appendices}
            \input{8 - Hilbert complexes/main.tex}
            \input{9 - weak conservation proofs/main.tex}
\end{document}


\title{\BA{Title in Progress...}}
\author{Boris Andrews}
\affil{Mathematical Institute, University of Oxford}
\date{\today}


\begin{document}
    \pagenumbering{gobble}
    \maketitle
    
    
    \begin{abstract}
        Magnetic confinement reactors---in particular tokamaks---offer one of the most promising options for achieving practical nuclear fusion, with the potential to provide virtually limitless, clean energy. The theoretical and numerical modeling of tokamak plasmas is simultaneously an essential component of effective reactor design, and a great research barrier. Tokamak operational conditions exhibit comparatively low Knudsen numbers. Kinetic effects, including kinetic waves and instabilities, Landau damping, bump-on-tail instabilities and more, are therefore highly influential in tokamak plasma dynamics. Purely fluid models are inherently incapable of capturing these effects, whereas the high dimensionality in purely kinetic models render them practically intractable for most relevant purposes.

        We consider a $\delta\!f$ decomposition model, with a macroscopic fluid background and microscopic kinetic correction, both fully coupled to each other. A similar manner of discretization is proposed to that used in the recent \texttt{STRUPHY} code \cite{Holderied_Possanner_Wang_2021, Holderied_2022, Li_et_al_2023} with a finite-element model for the background and a pseudo-particle/PiC model for the correction.

        The fluid background satisfies the full, non-linear, resistive, compressible, Hall MHD equations. \cite{Laakmann_Hu_Farrell_2022} introduces finite-element(-in-space) implicit timesteppers for the incompressible analogue to this system with structure-preserving (SP) properties in the ideal case, alongside parameter-robust preconditioners. We show that these timesteppers can derive from a finite-element-in-time (FET) (and finite-element-in-space) interpretation. The benefits of this reformulation are discussed, including the derivation of timesteppers that are higher order in time, and the quantifiable dissipative SP properties in the non-ideal, resistive case.
        
        We discuss possible options for extending this FET approach to timesteppers for the compressible case.

        The kinetic corrections satisfy linearized Boltzmann equations. Using a Lénard--Bernstein collision operator, these take Fokker--Planck-like forms \cite{Fokker_1914, Planck_1917} wherein pseudo-particles in the numerical model obey the neoclassical transport equations, with particle-independent Brownian drift terms. This offers a rigorous methodology for incorporating collisions into the particle transport model, without coupling the equations of motions for each particle.
        
        Works by Chen, Chacón et al. \cite{Chen_Chacón_Barnes_2011, Chacón_Chen_Barnes_2013, Chen_Chacón_2014, Chen_Chacón_2015} have developed structure-preserving particle pushers for neoclassical transport in the Vlasov equations, derived from Crank--Nicolson integrators. We show these too can can derive from a FET interpretation, similarly offering potential extensions to higher-order-in-time particle pushers. The FET formulation is used also to consider how the stochastic drift terms can be incorporated into the pushers. Stochastic gyrokinetic expansions are also discussed.

        Different options for the numerical implementation of these schemes are considered.

        Due to the efficacy of FET in the development of SP timesteppers for both the fluid and kinetic component, we hope this approach will prove effective in the future for developing SP timesteppers for the full hybrid model. We hope this will give us the opportunity to incorporate previously inaccessible kinetic effects into the highly effective, modern, finite-element MHD models.
    \end{abstract}
    
    
    \newpage
    \tableofcontents
    
    
    \newpage
    \pagenumbering{arabic}
    %\linenumbers\renewcommand\thelinenumber{\color{black!50}\arabic{linenumber}}
            \documentclass[12pt, a4paper]{report}

\input{template/main.tex}

\title{\BA{Title in Progress...}}
\author{Boris Andrews}
\affil{Mathematical Institute, University of Oxford}
\date{\today}


\begin{document}
    \pagenumbering{gobble}
    \maketitle
    
    
    \begin{abstract}
        Magnetic confinement reactors---in particular tokamaks---offer one of the most promising options for achieving practical nuclear fusion, with the potential to provide virtually limitless, clean energy. The theoretical and numerical modeling of tokamak plasmas is simultaneously an essential component of effective reactor design, and a great research barrier. Tokamak operational conditions exhibit comparatively low Knudsen numbers. Kinetic effects, including kinetic waves and instabilities, Landau damping, bump-on-tail instabilities and more, are therefore highly influential in tokamak plasma dynamics. Purely fluid models are inherently incapable of capturing these effects, whereas the high dimensionality in purely kinetic models render them practically intractable for most relevant purposes.

        We consider a $\delta\!f$ decomposition model, with a macroscopic fluid background and microscopic kinetic correction, both fully coupled to each other. A similar manner of discretization is proposed to that used in the recent \texttt{STRUPHY} code \cite{Holderied_Possanner_Wang_2021, Holderied_2022, Li_et_al_2023} with a finite-element model for the background and a pseudo-particle/PiC model for the correction.

        The fluid background satisfies the full, non-linear, resistive, compressible, Hall MHD equations. \cite{Laakmann_Hu_Farrell_2022} introduces finite-element(-in-space) implicit timesteppers for the incompressible analogue to this system with structure-preserving (SP) properties in the ideal case, alongside parameter-robust preconditioners. We show that these timesteppers can derive from a finite-element-in-time (FET) (and finite-element-in-space) interpretation. The benefits of this reformulation are discussed, including the derivation of timesteppers that are higher order in time, and the quantifiable dissipative SP properties in the non-ideal, resistive case.
        
        We discuss possible options for extending this FET approach to timesteppers for the compressible case.

        The kinetic corrections satisfy linearized Boltzmann equations. Using a Lénard--Bernstein collision operator, these take Fokker--Planck-like forms \cite{Fokker_1914, Planck_1917} wherein pseudo-particles in the numerical model obey the neoclassical transport equations, with particle-independent Brownian drift terms. This offers a rigorous methodology for incorporating collisions into the particle transport model, without coupling the equations of motions for each particle.
        
        Works by Chen, Chacón et al. \cite{Chen_Chacón_Barnes_2011, Chacón_Chen_Barnes_2013, Chen_Chacón_2014, Chen_Chacón_2015} have developed structure-preserving particle pushers for neoclassical transport in the Vlasov equations, derived from Crank--Nicolson integrators. We show these too can can derive from a FET interpretation, similarly offering potential extensions to higher-order-in-time particle pushers. The FET formulation is used also to consider how the stochastic drift terms can be incorporated into the pushers. Stochastic gyrokinetic expansions are also discussed.

        Different options for the numerical implementation of these schemes are considered.

        Due to the efficacy of FET in the development of SP timesteppers for both the fluid and kinetic component, we hope this approach will prove effective in the future for developing SP timesteppers for the full hybrid model. We hope this will give us the opportunity to incorporate previously inaccessible kinetic effects into the highly effective, modern, finite-element MHD models.
    \end{abstract}
    
    
    \newpage
    \tableofcontents
    
    
    \newpage
    \pagenumbering{arabic}
    %\linenumbers\renewcommand\thelinenumber{\color{black!50}\arabic{linenumber}}
            \input{0 - introduction/main.tex}
        \part{Research}
            \input{1 - low-noise PiC models/main.tex}
            \input{2 - kinetic component/main.tex}
            \input{3 - fluid component/main.tex}
            \input{4 - numerical implementation/main.tex}
        \part{Project Overview}
            \input{5 - research plan/main.tex}
            \input{6 - summary/main.tex}
    
    
    %\section{}
    \newpage
    \pagenumbering{gobble}
        \printbibliography


    \newpage
    \pagenumbering{roman}
    \appendix
        \part{Appendices}
            \input{8 - Hilbert complexes/main.tex}
            \input{9 - weak conservation proofs/main.tex}
\end{document}

        \part{Research}
            \documentclass[12pt, a4paper]{report}

\input{template/main.tex}

\title{\BA{Title in Progress...}}
\author{Boris Andrews}
\affil{Mathematical Institute, University of Oxford}
\date{\today}


\begin{document}
    \pagenumbering{gobble}
    \maketitle
    
    
    \begin{abstract}
        Magnetic confinement reactors---in particular tokamaks---offer one of the most promising options for achieving practical nuclear fusion, with the potential to provide virtually limitless, clean energy. The theoretical and numerical modeling of tokamak plasmas is simultaneously an essential component of effective reactor design, and a great research barrier. Tokamak operational conditions exhibit comparatively low Knudsen numbers. Kinetic effects, including kinetic waves and instabilities, Landau damping, bump-on-tail instabilities and more, are therefore highly influential in tokamak plasma dynamics. Purely fluid models are inherently incapable of capturing these effects, whereas the high dimensionality in purely kinetic models render them practically intractable for most relevant purposes.

        We consider a $\delta\!f$ decomposition model, with a macroscopic fluid background and microscopic kinetic correction, both fully coupled to each other. A similar manner of discretization is proposed to that used in the recent \texttt{STRUPHY} code \cite{Holderied_Possanner_Wang_2021, Holderied_2022, Li_et_al_2023} with a finite-element model for the background and a pseudo-particle/PiC model for the correction.

        The fluid background satisfies the full, non-linear, resistive, compressible, Hall MHD equations. \cite{Laakmann_Hu_Farrell_2022} introduces finite-element(-in-space) implicit timesteppers for the incompressible analogue to this system with structure-preserving (SP) properties in the ideal case, alongside parameter-robust preconditioners. We show that these timesteppers can derive from a finite-element-in-time (FET) (and finite-element-in-space) interpretation. The benefits of this reformulation are discussed, including the derivation of timesteppers that are higher order in time, and the quantifiable dissipative SP properties in the non-ideal, resistive case.
        
        We discuss possible options for extending this FET approach to timesteppers for the compressible case.

        The kinetic corrections satisfy linearized Boltzmann equations. Using a Lénard--Bernstein collision operator, these take Fokker--Planck-like forms \cite{Fokker_1914, Planck_1917} wherein pseudo-particles in the numerical model obey the neoclassical transport equations, with particle-independent Brownian drift terms. This offers a rigorous methodology for incorporating collisions into the particle transport model, without coupling the equations of motions for each particle.
        
        Works by Chen, Chacón et al. \cite{Chen_Chacón_Barnes_2011, Chacón_Chen_Barnes_2013, Chen_Chacón_2014, Chen_Chacón_2015} have developed structure-preserving particle pushers for neoclassical transport in the Vlasov equations, derived from Crank--Nicolson integrators. We show these too can can derive from a FET interpretation, similarly offering potential extensions to higher-order-in-time particle pushers. The FET formulation is used also to consider how the stochastic drift terms can be incorporated into the pushers. Stochastic gyrokinetic expansions are also discussed.

        Different options for the numerical implementation of these schemes are considered.

        Due to the efficacy of FET in the development of SP timesteppers for both the fluid and kinetic component, we hope this approach will prove effective in the future for developing SP timesteppers for the full hybrid model. We hope this will give us the opportunity to incorporate previously inaccessible kinetic effects into the highly effective, modern, finite-element MHD models.
    \end{abstract}
    
    
    \newpage
    \tableofcontents
    
    
    \newpage
    \pagenumbering{arabic}
    %\linenumbers\renewcommand\thelinenumber{\color{black!50}\arabic{linenumber}}
            \input{0 - introduction/main.tex}
        \part{Research}
            \input{1 - low-noise PiC models/main.tex}
            \input{2 - kinetic component/main.tex}
            \input{3 - fluid component/main.tex}
            \input{4 - numerical implementation/main.tex}
        \part{Project Overview}
            \input{5 - research plan/main.tex}
            \input{6 - summary/main.tex}
    
    
    %\section{}
    \newpage
    \pagenumbering{gobble}
        \printbibliography


    \newpage
    \pagenumbering{roman}
    \appendix
        \part{Appendices}
            \input{8 - Hilbert complexes/main.tex}
            \input{9 - weak conservation proofs/main.tex}
\end{document}

            \documentclass[12pt, a4paper]{report}

\input{template/main.tex}

\title{\BA{Title in Progress...}}
\author{Boris Andrews}
\affil{Mathematical Institute, University of Oxford}
\date{\today}


\begin{document}
    \pagenumbering{gobble}
    \maketitle
    
    
    \begin{abstract}
        Magnetic confinement reactors---in particular tokamaks---offer one of the most promising options for achieving practical nuclear fusion, with the potential to provide virtually limitless, clean energy. The theoretical and numerical modeling of tokamak plasmas is simultaneously an essential component of effective reactor design, and a great research barrier. Tokamak operational conditions exhibit comparatively low Knudsen numbers. Kinetic effects, including kinetic waves and instabilities, Landau damping, bump-on-tail instabilities and more, are therefore highly influential in tokamak plasma dynamics. Purely fluid models are inherently incapable of capturing these effects, whereas the high dimensionality in purely kinetic models render them practically intractable for most relevant purposes.

        We consider a $\delta\!f$ decomposition model, with a macroscopic fluid background and microscopic kinetic correction, both fully coupled to each other. A similar manner of discretization is proposed to that used in the recent \texttt{STRUPHY} code \cite{Holderied_Possanner_Wang_2021, Holderied_2022, Li_et_al_2023} with a finite-element model for the background and a pseudo-particle/PiC model for the correction.

        The fluid background satisfies the full, non-linear, resistive, compressible, Hall MHD equations. \cite{Laakmann_Hu_Farrell_2022} introduces finite-element(-in-space) implicit timesteppers for the incompressible analogue to this system with structure-preserving (SP) properties in the ideal case, alongside parameter-robust preconditioners. We show that these timesteppers can derive from a finite-element-in-time (FET) (and finite-element-in-space) interpretation. The benefits of this reformulation are discussed, including the derivation of timesteppers that are higher order in time, and the quantifiable dissipative SP properties in the non-ideal, resistive case.
        
        We discuss possible options for extending this FET approach to timesteppers for the compressible case.

        The kinetic corrections satisfy linearized Boltzmann equations. Using a Lénard--Bernstein collision operator, these take Fokker--Planck-like forms \cite{Fokker_1914, Planck_1917} wherein pseudo-particles in the numerical model obey the neoclassical transport equations, with particle-independent Brownian drift terms. This offers a rigorous methodology for incorporating collisions into the particle transport model, without coupling the equations of motions for each particle.
        
        Works by Chen, Chacón et al. \cite{Chen_Chacón_Barnes_2011, Chacón_Chen_Barnes_2013, Chen_Chacón_2014, Chen_Chacón_2015} have developed structure-preserving particle pushers for neoclassical transport in the Vlasov equations, derived from Crank--Nicolson integrators. We show these too can can derive from a FET interpretation, similarly offering potential extensions to higher-order-in-time particle pushers. The FET formulation is used also to consider how the stochastic drift terms can be incorporated into the pushers. Stochastic gyrokinetic expansions are also discussed.

        Different options for the numerical implementation of these schemes are considered.

        Due to the efficacy of FET in the development of SP timesteppers for both the fluid and kinetic component, we hope this approach will prove effective in the future for developing SP timesteppers for the full hybrid model. We hope this will give us the opportunity to incorporate previously inaccessible kinetic effects into the highly effective, modern, finite-element MHD models.
    \end{abstract}
    
    
    \newpage
    \tableofcontents
    
    
    \newpage
    \pagenumbering{arabic}
    %\linenumbers\renewcommand\thelinenumber{\color{black!50}\arabic{linenumber}}
            \input{0 - introduction/main.tex}
        \part{Research}
            \input{1 - low-noise PiC models/main.tex}
            \input{2 - kinetic component/main.tex}
            \input{3 - fluid component/main.tex}
            \input{4 - numerical implementation/main.tex}
        \part{Project Overview}
            \input{5 - research plan/main.tex}
            \input{6 - summary/main.tex}
    
    
    %\section{}
    \newpage
    \pagenumbering{gobble}
        \printbibliography


    \newpage
    \pagenumbering{roman}
    \appendix
        \part{Appendices}
            \input{8 - Hilbert complexes/main.tex}
            \input{9 - weak conservation proofs/main.tex}
\end{document}

            \documentclass[12pt, a4paper]{report}

\input{template/main.tex}

\title{\BA{Title in Progress...}}
\author{Boris Andrews}
\affil{Mathematical Institute, University of Oxford}
\date{\today}


\begin{document}
    \pagenumbering{gobble}
    \maketitle
    
    
    \begin{abstract}
        Magnetic confinement reactors---in particular tokamaks---offer one of the most promising options for achieving practical nuclear fusion, with the potential to provide virtually limitless, clean energy. The theoretical and numerical modeling of tokamak plasmas is simultaneously an essential component of effective reactor design, and a great research barrier. Tokamak operational conditions exhibit comparatively low Knudsen numbers. Kinetic effects, including kinetic waves and instabilities, Landau damping, bump-on-tail instabilities and more, are therefore highly influential in tokamak plasma dynamics. Purely fluid models are inherently incapable of capturing these effects, whereas the high dimensionality in purely kinetic models render them practically intractable for most relevant purposes.

        We consider a $\delta\!f$ decomposition model, with a macroscopic fluid background and microscopic kinetic correction, both fully coupled to each other. A similar manner of discretization is proposed to that used in the recent \texttt{STRUPHY} code \cite{Holderied_Possanner_Wang_2021, Holderied_2022, Li_et_al_2023} with a finite-element model for the background and a pseudo-particle/PiC model for the correction.

        The fluid background satisfies the full, non-linear, resistive, compressible, Hall MHD equations. \cite{Laakmann_Hu_Farrell_2022} introduces finite-element(-in-space) implicit timesteppers for the incompressible analogue to this system with structure-preserving (SP) properties in the ideal case, alongside parameter-robust preconditioners. We show that these timesteppers can derive from a finite-element-in-time (FET) (and finite-element-in-space) interpretation. The benefits of this reformulation are discussed, including the derivation of timesteppers that are higher order in time, and the quantifiable dissipative SP properties in the non-ideal, resistive case.
        
        We discuss possible options for extending this FET approach to timesteppers for the compressible case.

        The kinetic corrections satisfy linearized Boltzmann equations. Using a Lénard--Bernstein collision operator, these take Fokker--Planck-like forms \cite{Fokker_1914, Planck_1917} wherein pseudo-particles in the numerical model obey the neoclassical transport equations, with particle-independent Brownian drift terms. This offers a rigorous methodology for incorporating collisions into the particle transport model, without coupling the equations of motions for each particle.
        
        Works by Chen, Chacón et al. \cite{Chen_Chacón_Barnes_2011, Chacón_Chen_Barnes_2013, Chen_Chacón_2014, Chen_Chacón_2015} have developed structure-preserving particle pushers for neoclassical transport in the Vlasov equations, derived from Crank--Nicolson integrators. We show these too can can derive from a FET interpretation, similarly offering potential extensions to higher-order-in-time particle pushers. The FET formulation is used also to consider how the stochastic drift terms can be incorporated into the pushers. Stochastic gyrokinetic expansions are also discussed.

        Different options for the numerical implementation of these schemes are considered.

        Due to the efficacy of FET in the development of SP timesteppers for both the fluid and kinetic component, we hope this approach will prove effective in the future for developing SP timesteppers for the full hybrid model. We hope this will give us the opportunity to incorporate previously inaccessible kinetic effects into the highly effective, modern, finite-element MHD models.
    \end{abstract}
    
    
    \newpage
    \tableofcontents
    
    
    \newpage
    \pagenumbering{arabic}
    %\linenumbers\renewcommand\thelinenumber{\color{black!50}\arabic{linenumber}}
            \input{0 - introduction/main.tex}
        \part{Research}
            \input{1 - low-noise PiC models/main.tex}
            \input{2 - kinetic component/main.tex}
            \input{3 - fluid component/main.tex}
            \input{4 - numerical implementation/main.tex}
        \part{Project Overview}
            \input{5 - research plan/main.tex}
            \input{6 - summary/main.tex}
    
    
    %\section{}
    \newpage
    \pagenumbering{gobble}
        \printbibliography


    \newpage
    \pagenumbering{roman}
    \appendix
        \part{Appendices}
            \input{8 - Hilbert complexes/main.tex}
            \input{9 - weak conservation proofs/main.tex}
\end{document}

            \documentclass[12pt, a4paper]{report}

\input{template/main.tex}

\title{\BA{Title in Progress...}}
\author{Boris Andrews}
\affil{Mathematical Institute, University of Oxford}
\date{\today}


\begin{document}
    \pagenumbering{gobble}
    \maketitle
    
    
    \begin{abstract}
        Magnetic confinement reactors---in particular tokamaks---offer one of the most promising options for achieving practical nuclear fusion, with the potential to provide virtually limitless, clean energy. The theoretical and numerical modeling of tokamak plasmas is simultaneously an essential component of effective reactor design, and a great research barrier. Tokamak operational conditions exhibit comparatively low Knudsen numbers. Kinetic effects, including kinetic waves and instabilities, Landau damping, bump-on-tail instabilities and more, are therefore highly influential in tokamak plasma dynamics. Purely fluid models are inherently incapable of capturing these effects, whereas the high dimensionality in purely kinetic models render them practically intractable for most relevant purposes.

        We consider a $\delta\!f$ decomposition model, with a macroscopic fluid background and microscopic kinetic correction, both fully coupled to each other. A similar manner of discretization is proposed to that used in the recent \texttt{STRUPHY} code \cite{Holderied_Possanner_Wang_2021, Holderied_2022, Li_et_al_2023} with a finite-element model for the background and a pseudo-particle/PiC model for the correction.

        The fluid background satisfies the full, non-linear, resistive, compressible, Hall MHD equations. \cite{Laakmann_Hu_Farrell_2022} introduces finite-element(-in-space) implicit timesteppers for the incompressible analogue to this system with structure-preserving (SP) properties in the ideal case, alongside parameter-robust preconditioners. We show that these timesteppers can derive from a finite-element-in-time (FET) (and finite-element-in-space) interpretation. The benefits of this reformulation are discussed, including the derivation of timesteppers that are higher order in time, and the quantifiable dissipative SP properties in the non-ideal, resistive case.
        
        We discuss possible options for extending this FET approach to timesteppers for the compressible case.

        The kinetic corrections satisfy linearized Boltzmann equations. Using a Lénard--Bernstein collision operator, these take Fokker--Planck-like forms \cite{Fokker_1914, Planck_1917} wherein pseudo-particles in the numerical model obey the neoclassical transport equations, with particle-independent Brownian drift terms. This offers a rigorous methodology for incorporating collisions into the particle transport model, without coupling the equations of motions for each particle.
        
        Works by Chen, Chacón et al. \cite{Chen_Chacón_Barnes_2011, Chacón_Chen_Barnes_2013, Chen_Chacón_2014, Chen_Chacón_2015} have developed structure-preserving particle pushers for neoclassical transport in the Vlasov equations, derived from Crank--Nicolson integrators. We show these too can can derive from a FET interpretation, similarly offering potential extensions to higher-order-in-time particle pushers. The FET formulation is used also to consider how the stochastic drift terms can be incorporated into the pushers. Stochastic gyrokinetic expansions are also discussed.

        Different options for the numerical implementation of these schemes are considered.

        Due to the efficacy of FET in the development of SP timesteppers for both the fluid and kinetic component, we hope this approach will prove effective in the future for developing SP timesteppers for the full hybrid model. We hope this will give us the opportunity to incorporate previously inaccessible kinetic effects into the highly effective, modern, finite-element MHD models.
    \end{abstract}
    
    
    \newpage
    \tableofcontents
    
    
    \newpage
    \pagenumbering{arabic}
    %\linenumbers\renewcommand\thelinenumber{\color{black!50}\arabic{linenumber}}
            \input{0 - introduction/main.tex}
        \part{Research}
            \input{1 - low-noise PiC models/main.tex}
            \input{2 - kinetic component/main.tex}
            \input{3 - fluid component/main.tex}
            \input{4 - numerical implementation/main.tex}
        \part{Project Overview}
            \input{5 - research plan/main.tex}
            \input{6 - summary/main.tex}
    
    
    %\section{}
    \newpage
    \pagenumbering{gobble}
        \printbibliography


    \newpage
    \pagenumbering{roman}
    \appendix
        \part{Appendices}
            \input{8 - Hilbert complexes/main.tex}
            \input{9 - weak conservation proofs/main.tex}
\end{document}

        \part{Project Overview}
            \documentclass[12pt, a4paper]{report}

\input{template/main.tex}

\title{\BA{Title in Progress...}}
\author{Boris Andrews}
\affil{Mathematical Institute, University of Oxford}
\date{\today}


\begin{document}
    \pagenumbering{gobble}
    \maketitle
    
    
    \begin{abstract}
        Magnetic confinement reactors---in particular tokamaks---offer one of the most promising options for achieving practical nuclear fusion, with the potential to provide virtually limitless, clean energy. The theoretical and numerical modeling of tokamak plasmas is simultaneously an essential component of effective reactor design, and a great research barrier. Tokamak operational conditions exhibit comparatively low Knudsen numbers. Kinetic effects, including kinetic waves and instabilities, Landau damping, bump-on-tail instabilities and more, are therefore highly influential in tokamak plasma dynamics. Purely fluid models are inherently incapable of capturing these effects, whereas the high dimensionality in purely kinetic models render them practically intractable for most relevant purposes.

        We consider a $\delta\!f$ decomposition model, with a macroscopic fluid background and microscopic kinetic correction, both fully coupled to each other. A similar manner of discretization is proposed to that used in the recent \texttt{STRUPHY} code \cite{Holderied_Possanner_Wang_2021, Holderied_2022, Li_et_al_2023} with a finite-element model for the background and a pseudo-particle/PiC model for the correction.

        The fluid background satisfies the full, non-linear, resistive, compressible, Hall MHD equations. \cite{Laakmann_Hu_Farrell_2022} introduces finite-element(-in-space) implicit timesteppers for the incompressible analogue to this system with structure-preserving (SP) properties in the ideal case, alongside parameter-robust preconditioners. We show that these timesteppers can derive from a finite-element-in-time (FET) (and finite-element-in-space) interpretation. The benefits of this reformulation are discussed, including the derivation of timesteppers that are higher order in time, and the quantifiable dissipative SP properties in the non-ideal, resistive case.
        
        We discuss possible options for extending this FET approach to timesteppers for the compressible case.

        The kinetic corrections satisfy linearized Boltzmann equations. Using a Lénard--Bernstein collision operator, these take Fokker--Planck-like forms \cite{Fokker_1914, Planck_1917} wherein pseudo-particles in the numerical model obey the neoclassical transport equations, with particle-independent Brownian drift terms. This offers a rigorous methodology for incorporating collisions into the particle transport model, without coupling the equations of motions for each particle.
        
        Works by Chen, Chacón et al. \cite{Chen_Chacón_Barnes_2011, Chacón_Chen_Barnes_2013, Chen_Chacón_2014, Chen_Chacón_2015} have developed structure-preserving particle pushers for neoclassical transport in the Vlasov equations, derived from Crank--Nicolson integrators. We show these too can can derive from a FET interpretation, similarly offering potential extensions to higher-order-in-time particle pushers. The FET formulation is used also to consider how the stochastic drift terms can be incorporated into the pushers. Stochastic gyrokinetic expansions are also discussed.

        Different options for the numerical implementation of these schemes are considered.

        Due to the efficacy of FET in the development of SP timesteppers for both the fluid and kinetic component, we hope this approach will prove effective in the future for developing SP timesteppers for the full hybrid model. We hope this will give us the opportunity to incorporate previously inaccessible kinetic effects into the highly effective, modern, finite-element MHD models.
    \end{abstract}
    
    
    \newpage
    \tableofcontents
    
    
    \newpage
    \pagenumbering{arabic}
    %\linenumbers\renewcommand\thelinenumber{\color{black!50}\arabic{linenumber}}
            \input{0 - introduction/main.tex}
        \part{Research}
            \input{1 - low-noise PiC models/main.tex}
            \input{2 - kinetic component/main.tex}
            \input{3 - fluid component/main.tex}
            \input{4 - numerical implementation/main.tex}
        \part{Project Overview}
            \input{5 - research plan/main.tex}
            \input{6 - summary/main.tex}
    
    
    %\section{}
    \newpage
    \pagenumbering{gobble}
        \printbibliography


    \newpage
    \pagenumbering{roman}
    \appendix
        \part{Appendices}
            \input{8 - Hilbert complexes/main.tex}
            \input{9 - weak conservation proofs/main.tex}
\end{document}

            \documentclass[12pt, a4paper]{report}

\input{template/main.tex}

\title{\BA{Title in Progress...}}
\author{Boris Andrews}
\affil{Mathematical Institute, University of Oxford}
\date{\today}


\begin{document}
    \pagenumbering{gobble}
    \maketitle
    
    
    \begin{abstract}
        Magnetic confinement reactors---in particular tokamaks---offer one of the most promising options for achieving practical nuclear fusion, with the potential to provide virtually limitless, clean energy. The theoretical and numerical modeling of tokamak plasmas is simultaneously an essential component of effective reactor design, and a great research barrier. Tokamak operational conditions exhibit comparatively low Knudsen numbers. Kinetic effects, including kinetic waves and instabilities, Landau damping, bump-on-tail instabilities and more, are therefore highly influential in tokamak plasma dynamics. Purely fluid models are inherently incapable of capturing these effects, whereas the high dimensionality in purely kinetic models render them practically intractable for most relevant purposes.

        We consider a $\delta\!f$ decomposition model, with a macroscopic fluid background and microscopic kinetic correction, both fully coupled to each other. A similar manner of discretization is proposed to that used in the recent \texttt{STRUPHY} code \cite{Holderied_Possanner_Wang_2021, Holderied_2022, Li_et_al_2023} with a finite-element model for the background and a pseudo-particle/PiC model for the correction.

        The fluid background satisfies the full, non-linear, resistive, compressible, Hall MHD equations. \cite{Laakmann_Hu_Farrell_2022} introduces finite-element(-in-space) implicit timesteppers for the incompressible analogue to this system with structure-preserving (SP) properties in the ideal case, alongside parameter-robust preconditioners. We show that these timesteppers can derive from a finite-element-in-time (FET) (and finite-element-in-space) interpretation. The benefits of this reformulation are discussed, including the derivation of timesteppers that are higher order in time, and the quantifiable dissipative SP properties in the non-ideal, resistive case.
        
        We discuss possible options for extending this FET approach to timesteppers for the compressible case.

        The kinetic corrections satisfy linearized Boltzmann equations. Using a Lénard--Bernstein collision operator, these take Fokker--Planck-like forms \cite{Fokker_1914, Planck_1917} wherein pseudo-particles in the numerical model obey the neoclassical transport equations, with particle-independent Brownian drift terms. This offers a rigorous methodology for incorporating collisions into the particle transport model, without coupling the equations of motions for each particle.
        
        Works by Chen, Chacón et al. \cite{Chen_Chacón_Barnes_2011, Chacón_Chen_Barnes_2013, Chen_Chacón_2014, Chen_Chacón_2015} have developed structure-preserving particle pushers for neoclassical transport in the Vlasov equations, derived from Crank--Nicolson integrators. We show these too can can derive from a FET interpretation, similarly offering potential extensions to higher-order-in-time particle pushers. The FET formulation is used also to consider how the stochastic drift terms can be incorporated into the pushers. Stochastic gyrokinetic expansions are also discussed.

        Different options for the numerical implementation of these schemes are considered.

        Due to the efficacy of FET in the development of SP timesteppers for both the fluid and kinetic component, we hope this approach will prove effective in the future for developing SP timesteppers for the full hybrid model. We hope this will give us the opportunity to incorporate previously inaccessible kinetic effects into the highly effective, modern, finite-element MHD models.
    \end{abstract}
    
    
    \newpage
    \tableofcontents
    
    
    \newpage
    \pagenumbering{arabic}
    %\linenumbers\renewcommand\thelinenumber{\color{black!50}\arabic{linenumber}}
            \input{0 - introduction/main.tex}
        \part{Research}
            \input{1 - low-noise PiC models/main.tex}
            \input{2 - kinetic component/main.tex}
            \input{3 - fluid component/main.tex}
            \input{4 - numerical implementation/main.tex}
        \part{Project Overview}
            \input{5 - research plan/main.tex}
            \input{6 - summary/main.tex}
    
    
    %\section{}
    \newpage
    \pagenumbering{gobble}
        \printbibliography


    \newpage
    \pagenumbering{roman}
    \appendix
        \part{Appendices}
            \input{8 - Hilbert complexes/main.tex}
            \input{9 - weak conservation proofs/main.tex}
\end{document}

    
    
    %\section{}
    \newpage
    \pagenumbering{gobble}
        \printbibliography


    \newpage
    \pagenumbering{roman}
    \appendix
        \part{Appendices}
            \documentclass[12pt, a4paper]{report}

\input{template/main.tex}

\title{\BA{Title in Progress...}}
\author{Boris Andrews}
\affil{Mathematical Institute, University of Oxford}
\date{\today}


\begin{document}
    \pagenumbering{gobble}
    \maketitle
    
    
    \begin{abstract}
        Magnetic confinement reactors---in particular tokamaks---offer one of the most promising options for achieving practical nuclear fusion, with the potential to provide virtually limitless, clean energy. The theoretical and numerical modeling of tokamak plasmas is simultaneously an essential component of effective reactor design, and a great research barrier. Tokamak operational conditions exhibit comparatively low Knudsen numbers. Kinetic effects, including kinetic waves and instabilities, Landau damping, bump-on-tail instabilities and more, are therefore highly influential in tokamak plasma dynamics. Purely fluid models are inherently incapable of capturing these effects, whereas the high dimensionality in purely kinetic models render them practically intractable for most relevant purposes.

        We consider a $\delta\!f$ decomposition model, with a macroscopic fluid background and microscopic kinetic correction, both fully coupled to each other. A similar manner of discretization is proposed to that used in the recent \texttt{STRUPHY} code \cite{Holderied_Possanner_Wang_2021, Holderied_2022, Li_et_al_2023} with a finite-element model for the background and a pseudo-particle/PiC model for the correction.

        The fluid background satisfies the full, non-linear, resistive, compressible, Hall MHD equations. \cite{Laakmann_Hu_Farrell_2022} introduces finite-element(-in-space) implicit timesteppers for the incompressible analogue to this system with structure-preserving (SP) properties in the ideal case, alongside parameter-robust preconditioners. We show that these timesteppers can derive from a finite-element-in-time (FET) (and finite-element-in-space) interpretation. The benefits of this reformulation are discussed, including the derivation of timesteppers that are higher order in time, and the quantifiable dissipative SP properties in the non-ideal, resistive case.
        
        We discuss possible options for extending this FET approach to timesteppers for the compressible case.

        The kinetic corrections satisfy linearized Boltzmann equations. Using a Lénard--Bernstein collision operator, these take Fokker--Planck-like forms \cite{Fokker_1914, Planck_1917} wherein pseudo-particles in the numerical model obey the neoclassical transport equations, with particle-independent Brownian drift terms. This offers a rigorous methodology for incorporating collisions into the particle transport model, without coupling the equations of motions for each particle.
        
        Works by Chen, Chacón et al. \cite{Chen_Chacón_Barnes_2011, Chacón_Chen_Barnes_2013, Chen_Chacón_2014, Chen_Chacón_2015} have developed structure-preserving particle pushers for neoclassical transport in the Vlasov equations, derived from Crank--Nicolson integrators. We show these too can can derive from a FET interpretation, similarly offering potential extensions to higher-order-in-time particle pushers. The FET formulation is used also to consider how the stochastic drift terms can be incorporated into the pushers. Stochastic gyrokinetic expansions are also discussed.

        Different options for the numerical implementation of these schemes are considered.

        Due to the efficacy of FET in the development of SP timesteppers for both the fluid and kinetic component, we hope this approach will prove effective in the future for developing SP timesteppers for the full hybrid model. We hope this will give us the opportunity to incorporate previously inaccessible kinetic effects into the highly effective, modern, finite-element MHD models.
    \end{abstract}
    
    
    \newpage
    \tableofcontents
    
    
    \newpage
    \pagenumbering{arabic}
    %\linenumbers\renewcommand\thelinenumber{\color{black!50}\arabic{linenumber}}
            \input{0 - introduction/main.tex}
        \part{Research}
            \input{1 - low-noise PiC models/main.tex}
            \input{2 - kinetic component/main.tex}
            \input{3 - fluid component/main.tex}
            \input{4 - numerical implementation/main.tex}
        \part{Project Overview}
            \input{5 - research plan/main.tex}
            \input{6 - summary/main.tex}
    
    
    %\section{}
    \newpage
    \pagenumbering{gobble}
        \printbibliography


    \newpage
    \pagenumbering{roman}
    \appendix
        \part{Appendices}
            \input{8 - Hilbert complexes/main.tex}
            \input{9 - weak conservation proofs/main.tex}
\end{document}

            \documentclass[12pt, a4paper]{report}

\input{template/main.tex}

\title{\BA{Title in Progress...}}
\author{Boris Andrews}
\affil{Mathematical Institute, University of Oxford}
\date{\today}


\begin{document}
    \pagenumbering{gobble}
    \maketitle
    
    
    \begin{abstract}
        Magnetic confinement reactors---in particular tokamaks---offer one of the most promising options for achieving practical nuclear fusion, with the potential to provide virtually limitless, clean energy. The theoretical and numerical modeling of tokamak plasmas is simultaneously an essential component of effective reactor design, and a great research barrier. Tokamak operational conditions exhibit comparatively low Knudsen numbers. Kinetic effects, including kinetic waves and instabilities, Landau damping, bump-on-tail instabilities and more, are therefore highly influential in tokamak plasma dynamics. Purely fluid models are inherently incapable of capturing these effects, whereas the high dimensionality in purely kinetic models render them practically intractable for most relevant purposes.

        We consider a $\delta\!f$ decomposition model, with a macroscopic fluid background and microscopic kinetic correction, both fully coupled to each other. A similar manner of discretization is proposed to that used in the recent \texttt{STRUPHY} code \cite{Holderied_Possanner_Wang_2021, Holderied_2022, Li_et_al_2023} with a finite-element model for the background and a pseudo-particle/PiC model for the correction.

        The fluid background satisfies the full, non-linear, resistive, compressible, Hall MHD equations. \cite{Laakmann_Hu_Farrell_2022} introduces finite-element(-in-space) implicit timesteppers for the incompressible analogue to this system with structure-preserving (SP) properties in the ideal case, alongside parameter-robust preconditioners. We show that these timesteppers can derive from a finite-element-in-time (FET) (and finite-element-in-space) interpretation. The benefits of this reformulation are discussed, including the derivation of timesteppers that are higher order in time, and the quantifiable dissipative SP properties in the non-ideal, resistive case.
        
        We discuss possible options for extending this FET approach to timesteppers for the compressible case.

        The kinetic corrections satisfy linearized Boltzmann equations. Using a Lénard--Bernstein collision operator, these take Fokker--Planck-like forms \cite{Fokker_1914, Planck_1917} wherein pseudo-particles in the numerical model obey the neoclassical transport equations, with particle-independent Brownian drift terms. This offers a rigorous methodology for incorporating collisions into the particle transport model, without coupling the equations of motions for each particle.
        
        Works by Chen, Chacón et al. \cite{Chen_Chacón_Barnes_2011, Chacón_Chen_Barnes_2013, Chen_Chacón_2014, Chen_Chacón_2015} have developed structure-preserving particle pushers for neoclassical transport in the Vlasov equations, derived from Crank--Nicolson integrators. We show these too can can derive from a FET interpretation, similarly offering potential extensions to higher-order-in-time particle pushers. The FET formulation is used also to consider how the stochastic drift terms can be incorporated into the pushers. Stochastic gyrokinetic expansions are also discussed.

        Different options for the numerical implementation of these schemes are considered.

        Due to the efficacy of FET in the development of SP timesteppers for both the fluid and kinetic component, we hope this approach will prove effective in the future for developing SP timesteppers for the full hybrid model. We hope this will give us the opportunity to incorporate previously inaccessible kinetic effects into the highly effective, modern, finite-element MHD models.
    \end{abstract}
    
    
    \newpage
    \tableofcontents
    
    
    \newpage
    \pagenumbering{arabic}
    %\linenumbers\renewcommand\thelinenumber{\color{black!50}\arabic{linenumber}}
            \input{0 - introduction/main.tex}
        \part{Research}
            \input{1 - low-noise PiC models/main.tex}
            \input{2 - kinetic component/main.tex}
            \input{3 - fluid component/main.tex}
            \input{4 - numerical implementation/main.tex}
        \part{Project Overview}
            \input{5 - research plan/main.tex}
            \input{6 - summary/main.tex}
    
    
    %\section{}
    \newpage
    \pagenumbering{gobble}
        \printbibliography


    \newpage
    \pagenumbering{roman}
    \appendix
        \part{Appendices}
            \input{8 - Hilbert complexes/main.tex}
            \input{9 - weak conservation proofs/main.tex}
\end{document}

\end{document}

            \documentclass[12pt, a4paper]{report}

\documentclass[12pt, a4paper]{report}

\input{template/main.tex}

\title{\BA{Title in Progress...}}
\author{Boris Andrews}
\affil{Mathematical Institute, University of Oxford}
\date{\today}


\begin{document}
    \pagenumbering{gobble}
    \maketitle
    
    
    \begin{abstract}
        Magnetic confinement reactors---in particular tokamaks---offer one of the most promising options for achieving practical nuclear fusion, with the potential to provide virtually limitless, clean energy. The theoretical and numerical modeling of tokamak plasmas is simultaneously an essential component of effective reactor design, and a great research barrier. Tokamak operational conditions exhibit comparatively low Knudsen numbers. Kinetic effects, including kinetic waves and instabilities, Landau damping, bump-on-tail instabilities and more, are therefore highly influential in tokamak plasma dynamics. Purely fluid models are inherently incapable of capturing these effects, whereas the high dimensionality in purely kinetic models render them practically intractable for most relevant purposes.

        We consider a $\delta\!f$ decomposition model, with a macroscopic fluid background and microscopic kinetic correction, both fully coupled to each other. A similar manner of discretization is proposed to that used in the recent \texttt{STRUPHY} code \cite{Holderied_Possanner_Wang_2021, Holderied_2022, Li_et_al_2023} with a finite-element model for the background and a pseudo-particle/PiC model for the correction.

        The fluid background satisfies the full, non-linear, resistive, compressible, Hall MHD equations. \cite{Laakmann_Hu_Farrell_2022} introduces finite-element(-in-space) implicit timesteppers for the incompressible analogue to this system with structure-preserving (SP) properties in the ideal case, alongside parameter-robust preconditioners. We show that these timesteppers can derive from a finite-element-in-time (FET) (and finite-element-in-space) interpretation. The benefits of this reformulation are discussed, including the derivation of timesteppers that are higher order in time, and the quantifiable dissipative SP properties in the non-ideal, resistive case.
        
        We discuss possible options for extending this FET approach to timesteppers for the compressible case.

        The kinetic corrections satisfy linearized Boltzmann equations. Using a Lénard--Bernstein collision operator, these take Fokker--Planck-like forms \cite{Fokker_1914, Planck_1917} wherein pseudo-particles in the numerical model obey the neoclassical transport equations, with particle-independent Brownian drift terms. This offers a rigorous methodology for incorporating collisions into the particle transport model, without coupling the equations of motions for each particle.
        
        Works by Chen, Chacón et al. \cite{Chen_Chacón_Barnes_2011, Chacón_Chen_Barnes_2013, Chen_Chacón_2014, Chen_Chacón_2015} have developed structure-preserving particle pushers for neoclassical transport in the Vlasov equations, derived from Crank--Nicolson integrators. We show these too can can derive from a FET interpretation, similarly offering potential extensions to higher-order-in-time particle pushers. The FET formulation is used also to consider how the stochastic drift terms can be incorporated into the pushers. Stochastic gyrokinetic expansions are also discussed.

        Different options for the numerical implementation of these schemes are considered.

        Due to the efficacy of FET in the development of SP timesteppers for both the fluid and kinetic component, we hope this approach will prove effective in the future for developing SP timesteppers for the full hybrid model. We hope this will give us the opportunity to incorporate previously inaccessible kinetic effects into the highly effective, modern, finite-element MHD models.
    \end{abstract}
    
    
    \newpage
    \tableofcontents
    
    
    \newpage
    \pagenumbering{arabic}
    %\linenumbers\renewcommand\thelinenumber{\color{black!50}\arabic{linenumber}}
            \input{0 - introduction/main.tex}
        \part{Research}
            \input{1 - low-noise PiC models/main.tex}
            \input{2 - kinetic component/main.tex}
            \input{3 - fluid component/main.tex}
            \input{4 - numerical implementation/main.tex}
        \part{Project Overview}
            \input{5 - research plan/main.tex}
            \input{6 - summary/main.tex}
    
    
    %\section{}
    \newpage
    \pagenumbering{gobble}
        \printbibliography


    \newpage
    \pagenumbering{roman}
    \appendix
        \part{Appendices}
            \input{8 - Hilbert complexes/main.tex}
            \input{9 - weak conservation proofs/main.tex}
\end{document}


\title{\BA{Title in Progress...}}
\author{Boris Andrews}
\affil{Mathematical Institute, University of Oxford}
\date{\today}


\begin{document}
    \pagenumbering{gobble}
    \maketitle
    
    
    \begin{abstract}
        Magnetic confinement reactors---in particular tokamaks---offer one of the most promising options for achieving practical nuclear fusion, with the potential to provide virtually limitless, clean energy. The theoretical and numerical modeling of tokamak plasmas is simultaneously an essential component of effective reactor design, and a great research barrier. Tokamak operational conditions exhibit comparatively low Knudsen numbers. Kinetic effects, including kinetic waves and instabilities, Landau damping, bump-on-tail instabilities and more, are therefore highly influential in tokamak plasma dynamics. Purely fluid models are inherently incapable of capturing these effects, whereas the high dimensionality in purely kinetic models render them practically intractable for most relevant purposes.

        We consider a $\delta\!f$ decomposition model, with a macroscopic fluid background and microscopic kinetic correction, both fully coupled to each other. A similar manner of discretization is proposed to that used in the recent \texttt{STRUPHY} code \cite{Holderied_Possanner_Wang_2021, Holderied_2022, Li_et_al_2023} with a finite-element model for the background and a pseudo-particle/PiC model for the correction.

        The fluid background satisfies the full, non-linear, resistive, compressible, Hall MHD equations. \cite{Laakmann_Hu_Farrell_2022} introduces finite-element(-in-space) implicit timesteppers for the incompressible analogue to this system with structure-preserving (SP) properties in the ideal case, alongside parameter-robust preconditioners. We show that these timesteppers can derive from a finite-element-in-time (FET) (and finite-element-in-space) interpretation. The benefits of this reformulation are discussed, including the derivation of timesteppers that are higher order in time, and the quantifiable dissipative SP properties in the non-ideal, resistive case.
        
        We discuss possible options for extending this FET approach to timesteppers for the compressible case.

        The kinetic corrections satisfy linearized Boltzmann equations. Using a Lénard--Bernstein collision operator, these take Fokker--Planck-like forms \cite{Fokker_1914, Planck_1917} wherein pseudo-particles in the numerical model obey the neoclassical transport equations, with particle-independent Brownian drift terms. This offers a rigorous methodology for incorporating collisions into the particle transport model, without coupling the equations of motions for each particle.
        
        Works by Chen, Chacón et al. \cite{Chen_Chacón_Barnes_2011, Chacón_Chen_Barnes_2013, Chen_Chacón_2014, Chen_Chacón_2015} have developed structure-preserving particle pushers for neoclassical transport in the Vlasov equations, derived from Crank--Nicolson integrators. We show these too can can derive from a FET interpretation, similarly offering potential extensions to higher-order-in-time particle pushers. The FET formulation is used also to consider how the stochastic drift terms can be incorporated into the pushers. Stochastic gyrokinetic expansions are also discussed.

        Different options for the numerical implementation of these schemes are considered.

        Due to the efficacy of FET in the development of SP timesteppers for both the fluid and kinetic component, we hope this approach will prove effective in the future for developing SP timesteppers for the full hybrid model. We hope this will give us the opportunity to incorporate previously inaccessible kinetic effects into the highly effective, modern, finite-element MHD models.
    \end{abstract}
    
    
    \newpage
    \tableofcontents
    
    
    \newpage
    \pagenumbering{arabic}
    %\linenumbers\renewcommand\thelinenumber{\color{black!50}\arabic{linenumber}}
            \documentclass[12pt, a4paper]{report}

\input{template/main.tex}

\title{\BA{Title in Progress...}}
\author{Boris Andrews}
\affil{Mathematical Institute, University of Oxford}
\date{\today}


\begin{document}
    \pagenumbering{gobble}
    \maketitle
    
    
    \begin{abstract}
        Magnetic confinement reactors---in particular tokamaks---offer one of the most promising options for achieving practical nuclear fusion, with the potential to provide virtually limitless, clean energy. The theoretical and numerical modeling of tokamak plasmas is simultaneously an essential component of effective reactor design, and a great research barrier. Tokamak operational conditions exhibit comparatively low Knudsen numbers. Kinetic effects, including kinetic waves and instabilities, Landau damping, bump-on-tail instabilities and more, are therefore highly influential in tokamak plasma dynamics. Purely fluid models are inherently incapable of capturing these effects, whereas the high dimensionality in purely kinetic models render them practically intractable for most relevant purposes.

        We consider a $\delta\!f$ decomposition model, with a macroscopic fluid background and microscopic kinetic correction, both fully coupled to each other. A similar manner of discretization is proposed to that used in the recent \texttt{STRUPHY} code \cite{Holderied_Possanner_Wang_2021, Holderied_2022, Li_et_al_2023} with a finite-element model for the background and a pseudo-particle/PiC model for the correction.

        The fluid background satisfies the full, non-linear, resistive, compressible, Hall MHD equations. \cite{Laakmann_Hu_Farrell_2022} introduces finite-element(-in-space) implicit timesteppers for the incompressible analogue to this system with structure-preserving (SP) properties in the ideal case, alongside parameter-robust preconditioners. We show that these timesteppers can derive from a finite-element-in-time (FET) (and finite-element-in-space) interpretation. The benefits of this reformulation are discussed, including the derivation of timesteppers that are higher order in time, and the quantifiable dissipative SP properties in the non-ideal, resistive case.
        
        We discuss possible options for extending this FET approach to timesteppers for the compressible case.

        The kinetic corrections satisfy linearized Boltzmann equations. Using a Lénard--Bernstein collision operator, these take Fokker--Planck-like forms \cite{Fokker_1914, Planck_1917} wherein pseudo-particles in the numerical model obey the neoclassical transport equations, with particle-independent Brownian drift terms. This offers a rigorous methodology for incorporating collisions into the particle transport model, without coupling the equations of motions for each particle.
        
        Works by Chen, Chacón et al. \cite{Chen_Chacón_Barnes_2011, Chacón_Chen_Barnes_2013, Chen_Chacón_2014, Chen_Chacón_2015} have developed structure-preserving particle pushers for neoclassical transport in the Vlasov equations, derived from Crank--Nicolson integrators. We show these too can can derive from a FET interpretation, similarly offering potential extensions to higher-order-in-time particle pushers. The FET formulation is used also to consider how the stochastic drift terms can be incorporated into the pushers. Stochastic gyrokinetic expansions are also discussed.

        Different options for the numerical implementation of these schemes are considered.

        Due to the efficacy of FET in the development of SP timesteppers for both the fluid and kinetic component, we hope this approach will prove effective in the future for developing SP timesteppers for the full hybrid model. We hope this will give us the opportunity to incorporate previously inaccessible kinetic effects into the highly effective, modern, finite-element MHD models.
    \end{abstract}
    
    
    \newpage
    \tableofcontents
    
    
    \newpage
    \pagenumbering{arabic}
    %\linenumbers\renewcommand\thelinenumber{\color{black!50}\arabic{linenumber}}
            \input{0 - introduction/main.tex}
        \part{Research}
            \input{1 - low-noise PiC models/main.tex}
            \input{2 - kinetic component/main.tex}
            \input{3 - fluid component/main.tex}
            \input{4 - numerical implementation/main.tex}
        \part{Project Overview}
            \input{5 - research plan/main.tex}
            \input{6 - summary/main.tex}
    
    
    %\section{}
    \newpage
    \pagenumbering{gobble}
        \printbibliography


    \newpage
    \pagenumbering{roman}
    \appendix
        \part{Appendices}
            \input{8 - Hilbert complexes/main.tex}
            \input{9 - weak conservation proofs/main.tex}
\end{document}

        \part{Research}
            \documentclass[12pt, a4paper]{report}

\input{template/main.tex}

\title{\BA{Title in Progress...}}
\author{Boris Andrews}
\affil{Mathematical Institute, University of Oxford}
\date{\today}


\begin{document}
    \pagenumbering{gobble}
    \maketitle
    
    
    \begin{abstract}
        Magnetic confinement reactors---in particular tokamaks---offer one of the most promising options for achieving practical nuclear fusion, with the potential to provide virtually limitless, clean energy. The theoretical and numerical modeling of tokamak plasmas is simultaneously an essential component of effective reactor design, and a great research barrier. Tokamak operational conditions exhibit comparatively low Knudsen numbers. Kinetic effects, including kinetic waves and instabilities, Landau damping, bump-on-tail instabilities and more, are therefore highly influential in tokamak plasma dynamics. Purely fluid models are inherently incapable of capturing these effects, whereas the high dimensionality in purely kinetic models render them practically intractable for most relevant purposes.

        We consider a $\delta\!f$ decomposition model, with a macroscopic fluid background and microscopic kinetic correction, both fully coupled to each other. A similar manner of discretization is proposed to that used in the recent \texttt{STRUPHY} code \cite{Holderied_Possanner_Wang_2021, Holderied_2022, Li_et_al_2023} with a finite-element model for the background and a pseudo-particle/PiC model for the correction.

        The fluid background satisfies the full, non-linear, resistive, compressible, Hall MHD equations. \cite{Laakmann_Hu_Farrell_2022} introduces finite-element(-in-space) implicit timesteppers for the incompressible analogue to this system with structure-preserving (SP) properties in the ideal case, alongside parameter-robust preconditioners. We show that these timesteppers can derive from a finite-element-in-time (FET) (and finite-element-in-space) interpretation. The benefits of this reformulation are discussed, including the derivation of timesteppers that are higher order in time, and the quantifiable dissipative SP properties in the non-ideal, resistive case.
        
        We discuss possible options for extending this FET approach to timesteppers for the compressible case.

        The kinetic corrections satisfy linearized Boltzmann equations. Using a Lénard--Bernstein collision operator, these take Fokker--Planck-like forms \cite{Fokker_1914, Planck_1917} wherein pseudo-particles in the numerical model obey the neoclassical transport equations, with particle-independent Brownian drift terms. This offers a rigorous methodology for incorporating collisions into the particle transport model, without coupling the equations of motions for each particle.
        
        Works by Chen, Chacón et al. \cite{Chen_Chacón_Barnes_2011, Chacón_Chen_Barnes_2013, Chen_Chacón_2014, Chen_Chacón_2015} have developed structure-preserving particle pushers for neoclassical transport in the Vlasov equations, derived from Crank--Nicolson integrators. We show these too can can derive from a FET interpretation, similarly offering potential extensions to higher-order-in-time particle pushers. The FET formulation is used also to consider how the stochastic drift terms can be incorporated into the pushers. Stochastic gyrokinetic expansions are also discussed.

        Different options for the numerical implementation of these schemes are considered.

        Due to the efficacy of FET in the development of SP timesteppers for both the fluid and kinetic component, we hope this approach will prove effective in the future for developing SP timesteppers for the full hybrid model. We hope this will give us the opportunity to incorporate previously inaccessible kinetic effects into the highly effective, modern, finite-element MHD models.
    \end{abstract}
    
    
    \newpage
    \tableofcontents
    
    
    \newpage
    \pagenumbering{arabic}
    %\linenumbers\renewcommand\thelinenumber{\color{black!50}\arabic{linenumber}}
            \input{0 - introduction/main.tex}
        \part{Research}
            \input{1 - low-noise PiC models/main.tex}
            \input{2 - kinetic component/main.tex}
            \input{3 - fluid component/main.tex}
            \input{4 - numerical implementation/main.tex}
        \part{Project Overview}
            \input{5 - research plan/main.tex}
            \input{6 - summary/main.tex}
    
    
    %\section{}
    \newpage
    \pagenumbering{gobble}
        \printbibliography


    \newpage
    \pagenumbering{roman}
    \appendix
        \part{Appendices}
            \input{8 - Hilbert complexes/main.tex}
            \input{9 - weak conservation proofs/main.tex}
\end{document}

            \documentclass[12pt, a4paper]{report}

\input{template/main.tex}

\title{\BA{Title in Progress...}}
\author{Boris Andrews}
\affil{Mathematical Institute, University of Oxford}
\date{\today}


\begin{document}
    \pagenumbering{gobble}
    \maketitle
    
    
    \begin{abstract}
        Magnetic confinement reactors---in particular tokamaks---offer one of the most promising options for achieving practical nuclear fusion, with the potential to provide virtually limitless, clean energy. The theoretical and numerical modeling of tokamak plasmas is simultaneously an essential component of effective reactor design, and a great research barrier. Tokamak operational conditions exhibit comparatively low Knudsen numbers. Kinetic effects, including kinetic waves and instabilities, Landau damping, bump-on-tail instabilities and more, are therefore highly influential in tokamak plasma dynamics. Purely fluid models are inherently incapable of capturing these effects, whereas the high dimensionality in purely kinetic models render them practically intractable for most relevant purposes.

        We consider a $\delta\!f$ decomposition model, with a macroscopic fluid background and microscopic kinetic correction, both fully coupled to each other. A similar manner of discretization is proposed to that used in the recent \texttt{STRUPHY} code \cite{Holderied_Possanner_Wang_2021, Holderied_2022, Li_et_al_2023} with a finite-element model for the background and a pseudo-particle/PiC model for the correction.

        The fluid background satisfies the full, non-linear, resistive, compressible, Hall MHD equations. \cite{Laakmann_Hu_Farrell_2022} introduces finite-element(-in-space) implicit timesteppers for the incompressible analogue to this system with structure-preserving (SP) properties in the ideal case, alongside parameter-robust preconditioners. We show that these timesteppers can derive from a finite-element-in-time (FET) (and finite-element-in-space) interpretation. The benefits of this reformulation are discussed, including the derivation of timesteppers that are higher order in time, and the quantifiable dissipative SP properties in the non-ideal, resistive case.
        
        We discuss possible options for extending this FET approach to timesteppers for the compressible case.

        The kinetic corrections satisfy linearized Boltzmann equations. Using a Lénard--Bernstein collision operator, these take Fokker--Planck-like forms \cite{Fokker_1914, Planck_1917} wherein pseudo-particles in the numerical model obey the neoclassical transport equations, with particle-independent Brownian drift terms. This offers a rigorous methodology for incorporating collisions into the particle transport model, without coupling the equations of motions for each particle.
        
        Works by Chen, Chacón et al. \cite{Chen_Chacón_Barnes_2011, Chacón_Chen_Barnes_2013, Chen_Chacón_2014, Chen_Chacón_2015} have developed structure-preserving particle pushers for neoclassical transport in the Vlasov equations, derived from Crank--Nicolson integrators. We show these too can can derive from a FET interpretation, similarly offering potential extensions to higher-order-in-time particle pushers. The FET formulation is used also to consider how the stochastic drift terms can be incorporated into the pushers. Stochastic gyrokinetic expansions are also discussed.

        Different options for the numerical implementation of these schemes are considered.

        Due to the efficacy of FET in the development of SP timesteppers for both the fluid and kinetic component, we hope this approach will prove effective in the future for developing SP timesteppers for the full hybrid model. We hope this will give us the opportunity to incorporate previously inaccessible kinetic effects into the highly effective, modern, finite-element MHD models.
    \end{abstract}
    
    
    \newpage
    \tableofcontents
    
    
    \newpage
    \pagenumbering{arabic}
    %\linenumbers\renewcommand\thelinenumber{\color{black!50}\arabic{linenumber}}
            \input{0 - introduction/main.tex}
        \part{Research}
            \input{1 - low-noise PiC models/main.tex}
            \input{2 - kinetic component/main.tex}
            \input{3 - fluid component/main.tex}
            \input{4 - numerical implementation/main.tex}
        \part{Project Overview}
            \input{5 - research plan/main.tex}
            \input{6 - summary/main.tex}
    
    
    %\section{}
    \newpage
    \pagenumbering{gobble}
        \printbibliography


    \newpage
    \pagenumbering{roman}
    \appendix
        \part{Appendices}
            \input{8 - Hilbert complexes/main.tex}
            \input{9 - weak conservation proofs/main.tex}
\end{document}

            \documentclass[12pt, a4paper]{report}

\input{template/main.tex}

\title{\BA{Title in Progress...}}
\author{Boris Andrews}
\affil{Mathematical Institute, University of Oxford}
\date{\today}


\begin{document}
    \pagenumbering{gobble}
    \maketitle
    
    
    \begin{abstract}
        Magnetic confinement reactors---in particular tokamaks---offer one of the most promising options for achieving practical nuclear fusion, with the potential to provide virtually limitless, clean energy. The theoretical and numerical modeling of tokamak plasmas is simultaneously an essential component of effective reactor design, and a great research barrier. Tokamak operational conditions exhibit comparatively low Knudsen numbers. Kinetic effects, including kinetic waves and instabilities, Landau damping, bump-on-tail instabilities and more, are therefore highly influential in tokamak plasma dynamics. Purely fluid models are inherently incapable of capturing these effects, whereas the high dimensionality in purely kinetic models render them practically intractable for most relevant purposes.

        We consider a $\delta\!f$ decomposition model, with a macroscopic fluid background and microscopic kinetic correction, both fully coupled to each other. A similar manner of discretization is proposed to that used in the recent \texttt{STRUPHY} code \cite{Holderied_Possanner_Wang_2021, Holderied_2022, Li_et_al_2023} with a finite-element model for the background and a pseudo-particle/PiC model for the correction.

        The fluid background satisfies the full, non-linear, resistive, compressible, Hall MHD equations. \cite{Laakmann_Hu_Farrell_2022} introduces finite-element(-in-space) implicit timesteppers for the incompressible analogue to this system with structure-preserving (SP) properties in the ideal case, alongside parameter-robust preconditioners. We show that these timesteppers can derive from a finite-element-in-time (FET) (and finite-element-in-space) interpretation. The benefits of this reformulation are discussed, including the derivation of timesteppers that are higher order in time, and the quantifiable dissipative SP properties in the non-ideal, resistive case.
        
        We discuss possible options for extending this FET approach to timesteppers for the compressible case.

        The kinetic corrections satisfy linearized Boltzmann equations. Using a Lénard--Bernstein collision operator, these take Fokker--Planck-like forms \cite{Fokker_1914, Planck_1917} wherein pseudo-particles in the numerical model obey the neoclassical transport equations, with particle-independent Brownian drift terms. This offers a rigorous methodology for incorporating collisions into the particle transport model, without coupling the equations of motions for each particle.
        
        Works by Chen, Chacón et al. \cite{Chen_Chacón_Barnes_2011, Chacón_Chen_Barnes_2013, Chen_Chacón_2014, Chen_Chacón_2015} have developed structure-preserving particle pushers for neoclassical transport in the Vlasov equations, derived from Crank--Nicolson integrators. We show these too can can derive from a FET interpretation, similarly offering potential extensions to higher-order-in-time particle pushers. The FET formulation is used also to consider how the stochastic drift terms can be incorporated into the pushers. Stochastic gyrokinetic expansions are also discussed.

        Different options for the numerical implementation of these schemes are considered.

        Due to the efficacy of FET in the development of SP timesteppers for both the fluid and kinetic component, we hope this approach will prove effective in the future for developing SP timesteppers for the full hybrid model. We hope this will give us the opportunity to incorporate previously inaccessible kinetic effects into the highly effective, modern, finite-element MHD models.
    \end{abstract}
    
    
    \newpage
    \tableofcontents
    
    
    \newpage
    \pagenumbering{arabic}
    %\linenumbers\renewcommand\thelinenumber{\color{black!50}\arabic{linenumber}}
            \input{0 - introduction/main.tex}
        \part{Research}
            \input{1 - low-noise PiC models/main.tex}
            \input{2 - kinetic component/main.tex}
            \input{3 - fluid component/main.tex}
            \input{4 - numerical implementation/main.tex}
        \part{Project Overview}
            \input{5 - research plan/main.tex}
            \input{6 - summary/main.tex}
    
    
    %\section{}
    \newpage
    \pagenumbering{gobble}
        \printbibliography


    \newpage
    \pagenumbering{roman}
    \appendix
        \part{Appendices}
            \input{8 - Hilbert complexes/main.tex}
            \input{9 - weak conservation proofs/main.tex}
\end{document}

            \documentclass[12pt, a4paper]{report}

\input{template/main.tex}

\title{\BA{Title in Progress...}}
\author{Boris Andrews}
\affil{Mathematical Institute, University of Oxford}
\date{\today}


\begin{document}
    \pagenumbering{gobble}
    \maketitle
    
    
    \begin{abstract}
        Magnetic confinement reactors---in particular tokamaks---offer one of the most promising options for achieving practical nuclear fusion, with the potential to provide virtually limitless, clean energy. The theoretical and numerical modeling of tokamak plasmas is simultaneously an essential component of effective reactor design, and a great research barrier. Tokamak operational conditions exhibit comparatively low Knudsen numbers. Kinetic effects, including kinetic waves and instabilities, Landau damping, bump-on-tail instabilities and more, are therefore highly influential in tokamak plasma dynamics. Purely fluid models are inherently incapable of capturing these effects, whereas the high dimensionality in purely kinetic models render them practically intractable for most relevant purposes.

        We consider a $\delta\!f$ decomposition model, with a macroscopic fluid background and microscopic kinetic correction, both fully coupled to each other. A similar manner of discretization is proposed to that used in the recent \texttt{STRUPHY} code \cite{Holderied_Possanner_Wang_2021, Holderied_2022, Li_et_al_2023} with a finite-element model for the background and a pseudo-particle/PiC model for the correction.

        The fluid background satisfies the full, non-linear, resistive, compressible, Hall MHD equations. \cite{Laakmann_Hu_Farrell_2022} introduces finite-element(-in-space) implicit timesteppers for the incompressible analogue to this system with structure-preserving (SP) properties in the ideal case, alongside parameter-robust preconditioners. We show that these timesteppers can derive from a finite-element-in-time (FET) (and finite-element-in-space) interpretation. The benefits of this reformulation are discussed, including the derivation of timesteppers that are higher order in time, and the quantifiable dissipative SP properties in the non-ideal, resistive case.
        
        We discuss possible options for extending this FET approach to timesteppers for the compressible case.

        The kinetic corrections satisfy linearized Boltzmann equations. Using a Lénard--Bernstein collision operator, these take Fokker--Planck-like forms \cite{Fokker_1914, Planck_1917} wherein pseudo-particles in the numerical model obey the neoclassical transport equations, with particle-independent Brownian drift terms. This offers a rigorous methodology for incorporating collisions into the particle transport model, without coupling the equations of motions for each particle.
        
        Works by Chen, Chacón et al. \cite{Chen_Chacón_Barnes_2011, Chacón_Chen_Barnes_2013, Chen_Chacón_2014, Chen_Chacón_2015} have developed structure-preserving particle pushers for neoclassical transport in the Vlasov equations, derived from Crank--Nicolson integrators. We show these too can can derive from a FET interpretation, similarly offering potential extensions to higher-order-in-time particle pushers. The FET formulation is used also to consider how the stochastic drift terms can be incorporated into the pushers. Stochastic gyrokinetic expansions are also discussed.

        Different options for the numerical implementation of these schemes are considered.

        Due to the efficacy of FET in the development of SP timesteppers for both the fluid and kinetic component, we hope this approach will prove effective in the future for developing SP timesteppers for the full hybrid model. We hope this will give us the opportunity to incorporate previously inaccessible kinetic effects into the highly effective, modern, finite-element MHD models.
    \end{abstract}
    
    
    \newpage
    \tableofcontents
    
    
    \newpage
    \pagenumbering{arabic}
    %\linenumbers\renewcommand\thelinenumber{\color{black!50}\arabic{linenumber}}
            \input{0 - introduction/main.tex}
        \part{Research}
            \input{1 - low-noise PiC models/main.tex}
            \input{2 - kinetic component/main.tex}
            \input{3 - fluid component/main.tex}
            \input{4 - numerical implementation/main.tex}
        \part{Project Overview}
            \input{5 - research plan/main.tex}
            \input{6 - summary/main.tex}
    
    
    %\section{}
    \newpage
    \pagenumbering{gobble}
        \printbibliography


    \newpage
    \pagenumbering{roman}
    \appendix
        \part{Appendices}
            \input{8 - Hilbert complexes/main.tex}
            \input{9 - weak conservation proofs/main.tex}
\end{document}

        \part{Project Overview}
            \documentclass[12pt, a4paper]{report}

\input{template/main.tex}

\title{\BA{Title in Progress...}}
\author{Boris Andrews}
\affil{Mathematical Institute, University of Oxford}
\date{\today}


\begin{document}
    \pagenumbering{gobble}
    \maketitle
    
    
    \begin{abstract}
        Magnetic confinement reactors---in particular tokamaks---offer one of the most promising options for achieving practical nuclear fusion, with the potential to provide virtually limitless, clean energy. The theoretical and numerical modeling of tokamak plasmas is simultaneously an essential component of effective reactor design, and a great research barrier. Tokamak operational conditions exhibit comparatively low Knudsen numbers. Kinetic effects, including kinetic waves and instabilities, Landau damping, bump-on-tail instabilities and more, are therefore highly influential in tokamak plasma dynamics. Purely fluid models are inherently incapable of capturing these effects, whereas the high dimensionality in purely kinetic models render them practically intractable for most relevant purposes.

        We consider a $\delta\!f$ decomposition model, with a macroscopic fluid background and microscopic kinetic correction, both fully coupled to each other. A similar manner of discretization is proposed to that used in the recent \texttt{STRUPHY} code \cite{Holderied_Possanner_Wang_2021, Holderied_2022, Li_et_al_2023} with a finite-element model for the background and a pseudo-particle/PiC model for the correction.

        The fluid background satisfies the full, non-linear, resistive, compressible, Hall MHD equations. \cite{Laakmann_Hu_Farrell_2022} introduces finite-element(-in-space) implicit timesteppers for the incompressible analogue to this system with structure-preserving (SP) properties in the ideal case, alongside parameter-robust preconditioners. We show that these timesteppers can derive from a finite-element-in-time (FET) (and finite-element-in-space) interpretation. The benefits of this reformulation are discussed, including the derivation of timesteppers that are higher order in time, and the quantifiable dissipative SP properties in the non-ideal, resistive case.
        
        We discuss possible options for extending this FET approach to timesteppers for the compressible case.

        The kinetic corrections satisfy linearized Boltzmann equations. Using a Lénard--Bernstein collision operator, these take Fokker--Planck-like forms \cite{Fokker_1914, Planck_1917} wherein pseudo-particles in the numerical model obey the neoclassical transport equations, with particle-independent Brownian drift terms. This offers a rigorous methodology for incorporating collisions into the particle transport model, without coupling the equations of motions for each particle.
        
        Works by Chen, Chacón et al. \cite{Chen_Chacón_Barnes_2011, Chacón_Chen_Barnes_2013, Chen_Chacón_2014, Chen_Chacón_2015} have developed structure-preserving particle pushers for neoclassical transport in the Vlasov equations, derived from Crank--Nicolson integrators. We show these too can can derive from a FET interpretation, similarly offering potential extensions to higher-order-in-time particle pushers. The FET formulation is used also to consider how the stochastic drift terms can be incorporated into the pushers. Stochastic gyrokinetic expansions are also discussed.

        Different options for the numerical implementation of these schemes are considered.

        Due to the efficacy of FET in the development of SP timesteppers for both the fluid and kinetic component, we hope this approach will prove effective in the future for developing SP timesteppers for the full hybrid model. We hope this will give us the opportunity to incorporate previously inaccessible kinetic effects into the highly effective, modern, finite-element MHD models.
    \end{abstract}
    
    
    \newpage
    \tableofcontents
    
    
    \newpage
    \pagenumbering{arabic}
    %\linenumbers\renewcommand\thelinenumber{\color{black!50}\arabic{linenumber}}
            \input{0 - introduction/main.tex}
        \part{Research}
            \input{1 - low-noise PiC models/main.tex}
            \input{2 - kinetic component/main.tex}
            \input{3 - fluid component/main.tex}
            \input{4 - numerical implementation/main.tex}
        \part{Project Overview}
            \input{5 - research plan/main.tex}
            \input{6 - summary/main.tex}
    
    
    %\section{}
    \newpage
    \pagenumbering{gobble}
        \printbibliography


    \newpage
    \pagenumbering{roman}
    \appendix
        \part{Appendices}
            \input{8 - Hilbert complexes/main.tex}
            \input{9 - weak conservation proofs/main.tex}
\end{document}

            \documentclass[12pt, a4paper]{report}

\input{template/main.tex}

\title{\BA{Title in Progress...}}
\author{Boris Andrews}
\affil{Mathematical Institute, University of Oxford}
\date{\today}


\begin{document}
    \pagenumbering{gobble}
    \maketitle
    
    
    \begin{abstract}
        Magnetic confinement reactors---in particular tokamaks---offer one of the most promising options for achieving practical nuclear fusion, with the potential to provide virtually limitless, clean energy. The theoretical and numerical modeling of tokamak plasmas is simultaneously an essential component of effective reactor design, and a great research barrier. Tokamak operational conditions exhibit comparatively low Knudsen numbers. Kinetic effects, including kinetic waves and instabilities, Landau damping, bump-on-tail instabilities and more, are therefore highly influential in tokamak plasma dynamics. Purely fluid models are inherently incapable of capturing these effects, whereas the high dimensionality in purely kinetic models render them practically intractable for most relevant purposes.

        We consider a $\delta\!f$ decomposition model, with a macroscopic fluid background and microscopic kinetic correction, both fully coupled to each other. A similar manner of discretization is proposed to that used in the recent \texttt{STRUPHY} code \cite{Holderied_Possanner_Wang_2021, Holderied_2022, Li_et_al_2023} with a finite-element model for the background and a pseudo-particle/PiC model for the correction.

        The fluid background satisfies the full, non-linear, resistive, compressible, Hall MHD equations. \cite{Laakmann_Hu_Farrell_2022} introduces finite-element(-in-space) implicit timesteppers for the incompressible analogue to this system with structure-preserving (SP) properties in the ideal case, alongside parameter-robust preconditioners. We show that these timesteppers can derive from a finite-element-in-time (FET) (and finite-element-in-space) interpretation. The benefits of this reformulation are discussed, including the derivation of timesteppers that are higher order in time, and the quantifiable dissipative SP properties in the non-ideal, resistive case.
        
        We discuss possible options for extending this FET approach to timesteppers for the compressible case.

        The kinetic corrections satisfy linearized Boltzmann equations. Using a Lénard--Bernstein collision operator, these take Fokker--Planck-like forms \cite{Fokker_1914, Planck_1917} wherein pseudo-particles in the numerical model obey the neoclassical transport equations, with particle-independent Brownian drift terms. This offers a rigorous methodology for incorporating collisions into the particle transport model, without coupling the equations of motions for each particle.
        
        Works by Chen, Chacón et al. \cite{Chen_Chacón_Barnes_2011, Chacón_Chen_Barnes_2013, Chen_Chacón_2014, Chen_Chacón_2015} have developed structure-preserving particle pushers for neoclassical transport in the Vlasov equations, derived from Crank--Nicolson integrators. We show these too can can derive from a FET interpretation, similarly offering potential extensions to higher-order-in-time particle pushers. The FET formulation is used also to consider how the stochastic drift terms can be incorporated into the pushers. Stochastic gyrokinetic expansions are also discussed.

        Different options for the numerical implementation of these schemes are considered.

        Due to the efficacy of FET in the development of SP timesteppers for both the fluid and kinetic component, we hope this approach will prove effective in the future for developing SP timesteppers for the full hybrid model. We hope this will give us the opportunity to incorporate previously inaccessible kinetic effects into the highly effective, modern, finite-element MHD models.
    \end{abstract}
    
    
    \newpage
    \tableofcontents
    
    
    \newpage
    \pagenumbering{arabic}
    %\linenumbers\renewcommand\thelinenumber{\color{black!50}\arabic{linenumber}}
            \input{0 - introduction/main.tex}
        \part{Research}
            \input{1 - low-noise PiC models/main.tex}
            \input{2 - kinetic component/main.tex}
            \input{3 - fluid component/main.tex}
            \input{4 - numerical implementation/main.tex}
        \part{Project Overview}
            \input{5 - research plan/main.tex}
            \input{6 - summary/main.tex}
    
    
    %\section{}
    \newpage
    \pagenumbering{gobble}
        \printbibliography


    \newpage
    \pagenumbering{roman}
    \appendix
        \part{Appendices}
            \input{8 - Hilbert complexes/main.tex}
            \input{9 - weak conservation proofs/main.tex}
\end{document}

    
    
    %\section{}
    \newpage
    \pagenumbering{gobble}
        \printbibliography


    \newpage
    \pagenumbering{roman}
    \appendix
        \part{Appendices}
            \documentclass[12pt, a4paper]{report}

\input{template/main.tex}

\title{\BA{Title in Progress...}}
\author{Boris Andrews}
\affil{Mathematical Institute, University of Oxford}
\date{\today}


\begin{document}
    \pagenumbering{gobble}
    \maketitle
    
    
    \begin{abstract}
        Magnetic confinement reactors---in particular tokamaks---offer one of the most promising options for achieving practical nuclear fusion, with the potential to provide virtually limitless, clean energy. The theoretical and numerical modeling of tokamak plasmas is simultaneously an essential component of effective reactor design, and a great research barrier. Tokamak operational conditions exhibit comparatively low Knudsen numbers. Kinetic effects, including kinetic waves and instabilities, Landau damping, bump-on-tail instabilities and more, are therefore highly influential in tokamak plasma dynamics. Purely fluid models are inherently incapable of capturing these effects, whereas the high dimensionality in purely kinetic models render them practically intractable for most relevant purposes.

        We consider a $\delta\!f$ decomposition model, with a macroscopic fluid background and microscopic kinetic correction, both fully coupled to each other. A similar manner of discretization is proposed to that used in the recent \texttt{STRUPHY} code \cite{Holderied_Possanner_Wang_2021, Holderied_2022, Li_et_al_2023} with a finite-element model for the background and a pseudo-particle/PiC model for the correction.

        The fluid background satisfies the full, non-linear, resistive, compressible, Hall MHD equations. \cite{Laakmann_Hu_Farrell_2022} introduces finite-element(-in-space) implicit timesteppers for the incompressible analogue to this system with structure-preserving (SP) properties in the ideal case, alongside parameter-robust preconditioners. We show that these timesteppers can derive from a finite-element-in-time (FET) (and finite-element-in-space) interpretation. The benefits of this reformulation are discussed, including the derivation of timesteppers that are higher order in time, and the quantifiable dissipative SP properties in the non-ideal, resistive case.
        
        We discuss possible options for extending this FET approach to timesteppers for the compressible case.

        The kinetic corrections satisfy linearized Boltzmann equations. Using a Lénard--Bernstein collision operator, these take Fokker--Planck-like forms \cite{Fokker_1914, Planck_1917} wherein pseudo-particles in the numerical model obey the neoclassical transport equations, with particle-independent Brownian drift terms. This offers a rigorous methodology for incorporating collisions into the particle transport model, without coupling the equations of motions for each particle.
        
        Works by Chen, Chacón et al. \cite{Chen_Chacón_Barnes_2011, Chacón_Chen_Barnes_2013, Chen_Chacón_2014, Chen_Chacón_2015} have developed structure-preserving particle pushers for neoclassical transport in the Vlasov equations, derived from Crank--Nicolson integrators. We show these too can can derive from a FET interpretation, similarly offering potential extensions to higher-order-in-time particle pushers. The FET formulation is used also to consider how the stochastic drift terms can be incorporated into the pushers. Stochastic gyrokinetic expansions are also discussed.

        Different options for the numerical implementation of these schemes are considered.

        Due to the efficacy of FET in the development of SP timesteppers for both the fluid and kinetic component, we hope this approach will prove effective in the future for developing SP timesteppers for the full hybrid model. We hope this will give us the opportunity to incorporate previously inaccessible kinetic effects into the highly effective, modern, finite-element MHD models.
    \end{abstract}
    
    
    \newpage
    \tableofcontents
    
    
    \newpage
    \pagenumbering{arabic}
    %\linenumbers\renewcommand\thelinenumber{\color{black!50}\arabic{linenumber}}
            \input{0 - introduction/main.tex}
        \part{Research}
            \input{1 - low-noise PiC models/main.tex}
            \input{2 - kinetic component/main.tex}
            \input{3 - fluid component/main.tex}
            \input{4 - numerical implementation/main.tex}
        \part{Project Overview}
            \input{5 - research plan/main.tex}
            \input{6 - summary/main.tex}
    
    
    %\section{}
    \newpage
    \pagenumbering{gobble}
        \printbibliography


    \newpage
    \pagenumbering{roman}
    \appendix
        \part{Appendices}
            \input{8 - Hilbert complexes/main.tex}
            \input{9 - weak conservation proofs/main.tex}
\end{document}

            \documentclass[12pt, a4paper]{report}

\input{template/main.tex}

\title{\BA{Title in Progress...}}
\author{Boris Andrews}
\affil{Mathematical Institute, University of Oxford}
\date{\today}


\begin{document}
    \pagenumbering{gobble}
    \maketitle
    
    
    \begin{abstract}
        Magnetic confinement reactors---in particular tokamaks---offer one of the most promising options for achieving practical nuclear fusion, with the potential to provide virtually limitless, clean energy. The theoretical and numerical modeling of tokamak plasmas is simultaneously an essential component of effective reactor design, and a great research barrier. Tokamak operational conditions exhibit comparatively low Knudsen numbers. Kinetic effects, including kinetic waves and instabilities, Landau damping, bump-on-tail instabilities and more, are therefore highly influential in tokamak plasma dynamics. Purely fluid models are inherently incapable of capturing these effects, whereas the high dimensionality in purely kinetic models render them practically intractable for most relevant purposes.

        We consider a $\delta\!f$ decomposition model, with a macroscopic fluid background and microscopic kinetic correction, both fully coupled to each other. A similar manner of discretization is proposed to that used in the recent \texttt{STRUPHY} code \cite{Holderied_Possanner_Wang_2021, Holderied_2022, Li_et_al_2023} with a finite-element model for the background and a pseudo-particle/PiC model for the correction.

        The fluid background satisfies the full, non-linear, resistive, compressible, Hall MHD equations. \cite{Laakmann_Hu_Farrell_2022} introduces finite-element(-in-space) implicit timesteppers for the incompressible analogue to this system with structure-preserving (SP) properties in the ideal case, alongside parameter-robust preconditioners. We show that these timesteppers can derive from a finite-element-in-time (FET) (and finite-element-in-space) interpretation. The benefits of this reformulation are discussed, including the derivation of timesteppers that are higher order in time, and the quantifiable dissipative SP properties in the non-ideal, resistive case.
        
        We discuss possible options for extending this FET approach to timesteppers for the compressible case.

        The kinetic corrections satisfy linearized Boltzmann equations. Using a Lénard--Bernstein collision operator, these take Fokker--Planck-like forms \cite{Fokker_1914, Planck_1917} wherein pseudo-particles in the numerical model obey the neoclassical transport equations, with particle-independent Brownian drift terms. This offers a rigorous methodology for incorporating collisions into the particle transport model, without coupling the equations of motions for each particle.
        
        Works by Chen, Chacón et al. \cite{Chen_Chacón_Barnes_2011, Chacón_Chen_Barnes_2013, Chen_Chacón_2014, Chen_Chacón_2015} have developed structure-preserving particle pushers for neoclassical transport in the Vlasov equations, derived from Crank--Nicolson integrators. We show these too can can derive from a FET interpretation, similarly offering potential extensions to higher-order-in-time particle pushers. The FET formulation is used also to consider how the stochastic drift terms can be incorporated into the pushers. Stochastic gyrokinetic expansions are also discussed.

        Different options for the numerical implementation of these schemes are considered.

        Due to the efficacy of FET in the development of SP timesteppers for both the fluid and kinetic component, we hope this approach will prove effective in the future for developing SP timesteppers for the full hybrid model. We hope this will give us the opportunity to incorporate previously inaccessible kinetic effects into the highly effective, modern, finite-element MHD models.
    \end{abstract}
    
    
    \newpage
    \tableofcontents
    
    
    \newpage
    \pagenumbering{arabic}
    %\linenumbers\renewcommand\thelinenumber{\color{black!50}\arabic{linenumber}}
            \input{0 - introduction/main.tex}
        \part{Research}
            \input{1 - low-noise PiC models/main.tex}
            \input{2 - kinetic component/main.tex}
            \input{3 - fluid component/main.tex}
            \input{4 - numerical implementation/main.tex}
        \part{Project Overview}
            \input{5 - research plan/main.tex}
            \input{6 - summary/main.tex}
    
    
    %\section{}
    \newpage
    \pagenumbering{gobble}
        \printbibliography


    \newpage
    \pagenumbering{roman}
    \appendix
        \part{Appendices}
            \input{8 - Hilbert complexes/main.tex}
            \input{9 - weak conservation proofs/main.tex}
\end{document}

\end{document}

\end{document}

            \documentclass[12pt, a4paper]{report}

\documentclass[12pt, a4paper]{report}

\documentclass[12pt, a4paper]{report}

\input{template/main.tex}

\title{\BA{Title in Progress...}}
\author{Boris Andrews}
\affil{Mathematical Institute, University of Oxford}
\date{\today}


\begin{document}
    \pagenumbering{gobble}
    \maketitle
    
    
    \begin{abstract}
        Magnetic confinement reactors---in particular tokamaks---offer one of the most promising options for achieving practical nuclear fusion, with the potential to provide virtually limitless, clean energy. The theoretical and numerical modeling of tokamak plasmas is simultaneously an essential component of effective reactor design, and a great research barrier. Tokamak operational conditions exhibit comparatively low Knudsen numbers. Kinetic effects, including kinetic waves and instabilities, Landau damping, bump-on-tail instabilities and more, are therefore highly influential in tokamak plasma dynamics. Purely fluid models are inherently incapable of capturing these effects, whereas the high dimensionality in purely kinetic models render them practically intractable for most relevant purposes.

        We consider a $\delta\!f$ decomposition model, with a macroscopic fluid background and microscopic kinetic correction, both fully coupled to each other. A similar manner of discretization is proposed to that used in the recent \texttt{STRUPHY} code \cite{Holderied_Possanner_Wang_2021, Holderied_2022, Li_et_al_2023} with a finite-element model for the background and a pseudo-particle/PiC model for the correction.

        The fluid background satisfies the full, non-linear, resistive, compressible, Hall MHD equations. \cite{Laakmann_Hu_Farrell_2022} introduces finite-element(-in-space) implicit timesteppers for the incompressible analogue to this system with structure-preserving (SP) properties in the ideal case, alongside parameter-robust preconditioners. We show that these timesteppers can derive from a finite-element-in-time (FET) (and finite-element-in-space) interpretation. The benefits of this reformulation are discussed, including the derivation of timesteppers that are higher order in time, and the quantifiable dissipative SP properties in the non-ideal, resistive case.
        
        We discuss possible options for extending this FET approach to timesteppers for the compressible case.

        The kinetic corrections satisfy linearized Boltzmann equations. Using a Lénard--Bernstein collision operator, these take Fokker--Planck-like forms \cite{Fokker_1914, Planck_1917} wherein pseudo-particles in the numerical model obey the neoclassical transport equations, with particle-independent Brownian drift terms. This offers a rigorous methodology for incorporating collisions into the particle transport model, without coupling the equations of motions for each particle.
        
        Works by Chen, Chacón et al. \cite{Chen_Chacón_Barnes_2011, Chacón_Chen_Barnes_2013, Chen_Chacón_2014, Chen_Chacón_2015} have developed structure-preserving particle pushers for neoclassical transport in the Vlasov equations, derived from Crank--Nicolson integrators. We show these too can can derive from a FET interpretation, similarly offering potential extensions to higher-order-in-time particle pushers. The FET formulation is used also to consider how the stochastic drift terms can be incorporated into the pushers. Stochastic gyrokinetic expansions are also discussed.

        Different options for the numerical implementation of these schemes are considered.

        Due to the efficacy of FET in the development of SP timesteppers for both the fluid and kinetic component, we hope this approach will prove effective in the future for developing SP timesteppers for the full hybrid model. We hope this will give us the opportunity to incorporate previously inaccessible kinetic effects into the highly effective, modern, finite-element MHD models.
    \end{abstract}
    
    
    \newpage
    \tableofcontents
    
    
    \newpage
    \pagenumbering{arabic}
    %\linenumbers\renewcommand\thelinenumber{\color{black!50}\arabic{linenumber}}
            \input{0 - introduction/main.tex}
        \part{Research}
            \input{1 - low-noise PiC models/main.tex}
            \input{2 - kinetic component/main.tex}
            \input{3 - fluid component/main.tex}
            \input{4 - numerical implementation/main.tex}
        \part{Project Overview}
            \input{5 - research plan/main.tex}
            \input{6 - summary/main.tex}
    
    
    %\section{}
    \newpage
    \pagenumbering{gobble}
        \printbibliography


    \newpage
    \pagenumbering{roman}
    \appendix
        \part{Appendices}
            \input{8 - Hilbert complexes/main.tex}
            \input{9 - weak conservation proofs/main.tex}
\end{document}


\title{\BA{Title in Progress...}}
\author{Boris Andrews}
\affil{Mathematical Institute, University of Oxford}
\date{\today}


\begin{document}
    \pagenumbering{gobble}
    \maketitle
    
    
    \begin{abstract}
        Magnetic confinement reactors---in particular tokamaks---offer one of the most promising options for achieving practical nuclear fusion, with the potential to provide virtually limitless, clean energy. The theoretical and numerical modeling of tokamak plasmas is simultaneously an essential component of effective reactor design, and a great research barrier. Tokamak operational conditions exhibit comparatively low Knudsen numbers. Kinetic effects, including kinetic waves and instabilities, Landau damping, bump-on-tail instabilities and more, are therefore highly influential in tokamak plasma dynamics. Purely fluid models are inherently incapable of capturing these effects, whereas the high dimensionality in purely kinetic models render them practically intractable for most relevant purposes.

        We consider a $\delta\!f$ decomposition model, with a macroscopic fluid background and microscopic kinetic correction, both fully coupled to each other. A similar manner of discretization is proposed to that used in the recent \texttt{STRUPHY} code \cite{Holderied_Possanner_Wang_2021, Holderied_2022, Li_et_al_2023} with a finite-element model for the background and a pseudo-particle/PiC model for the correction.

        The fluid background satisfies the full, non-linear, resistive, compressible, Hall MHD equations. \cite{Laakmann_Hu_Farrell_2022} introduces finite-element(-in-space) implicit timesteppers for the incompressible analogue to this system with structure-preserving (SP) properties in the ideal case, alongside parameter-robust preconditioners. We show that these timesteppers can derive from a finite-element-in-time (FET) (and finite-element-in-space) interpretation. The benefits of this reformulation are discussed, including the derivation of timesteppers that are higher order in time, and the quantifiable dissipative SP properties in the non-ideal, resistive case.
        
        We discuss possible options for extending this FET approach to timesteppers for the compressible case.

        The kinetic corrections satisfy linearized Boltzmann equations. Using a Lénard--Bernstein collision operator, these take Fokker--Planck-like forms \cite{Fokker_1914, Planck_1917} wherein pseudo-particles in the numerical model obey the neoclassical transport equations, with particle-independent Brownian drift terms. This offers a rigorous methodology for incorporating collisions into the particle transport model, without coupling the equations of motions for each particle.
        
        Works by Chen, Chacón et al. \cite{Chen_Chacón_Barnes_2011, Chacón_Chen_Barnes_2013, Chen_Chacón_2014, Chen_Chacón_2015} have developed structure-preserving particle pushers for neoclassical transport in the Vlasov equations, derived from Crank--Nicolson integrators. We show these too can can derive from a FET interpretation, similarly offering potential extensions to higher-order-in-time particle pushers. The FET formulation is used also to consider how the stochastic drift terms can be incorporated into the pushers. Stochastic gyrokinetic expansions are also discussed.

        Different options for the numerical implementation of these schemes are considered.

        Due to the efficacy of FET in the development of SP timesteppers for both the fluid and kinetic component, we hope this approach will prove effective in the future for developing SP timesteppers for the full hybrid model. We hope this will give us the opportunity to incorporate previously inaccessible kinetic effects into the highly effective, modern, finite-element MHD models.
    \end{abstract}
    
    
    \newpage
    \tableofcontents
    
    
    \newpage
    \pagenumbering{arabic}
    %\linenumbers\renewcommand\thelinenumber{\color{black!50}\arabic{linenumber}}
            \documentclass[12pt, a4paper]{report}

\input{template/main.tex}

\title{\BA{Title in Progress...}}
\author{Boris Andrews}
\affil{Mathematical Institute, University of Oxford}
\date{\today}


\begin{document}
    \pagenumbering{gobble}
    \maketitle
    
    
    \begin{abstract}
        Magnetic confinement reactors---in particular tokamaks---offer one of the most promising options for achieving practical nuclear fusion, with the potential to provide virtually limitless, clean energy. The theoretical and numerical modeling of tokamak plasmas is simultaneously an essential component of effective reactor design, and a great research barrier. Tokamak operational conditions exhibit comparatively low Knudsen numbers. Kinetic effects, including kinetic waves and instabilities, Landau damping, bump-on-tail instabilities and more, are therefore highly influential in tokamak plasma dynamics. Purely fluid models are inherently incapable of capturing these effects, whereas the high dimensionality in purely kinetic models render them practically intractable for most relevant purposes.

        We consider a $\delta\!f$ decomposition model, with a macroscopic fluid background and microscopic kinetic correction, both fully coupled to each other. A similar manner of discretization is proposed to that used in the recent \texttt{STRUPHY} code \cite{Holderied_Possanner_Wang_2021, Holderied_2022, Li_et_al_2023} with a finite-element model for the background and a pseudo-particle/PiC model for the correction.

        The fluid background satisfies the full, non-linear, resistive, compressible, Hall MHD equations. \cite{Laakmann_Hu_Farrell_2022} introduces finite-element(-in-space) implicit timesteppers for the incompressible analogue to this system with structure-preserving (SP) properties in the ideal case, alongside parameter-robust preconditioners. We show that these timesteppers can derive from a finite-element-in-time (FET) (and finite-element-in-space) interpretation. The benefits of this reformulation are discussed, including the derivation of timesteppers that are higher order in time, and the quantifiable dissipative SP properties in the non-ideal, resistive case.
        
        We discuss possible options for extending this FET approach to timesteppers for the compressible case.

        The kinetic corrections satisfy linearized Boltzmann equations. Using a Lénard--Bernstein collision operator, these take Fokker--Planck-like forms \cite{Fokker_1914, Planck_1917} wherein pseudo-particles in the numerical model obey the neoclassical transport equations, with particle-independent Brownian drift terms. This offers a rigorous methodology for incorporating collisions into the particle transport model, without coupling the equations of motions for each particle.
        
        Works by Chen, Chacón et al. \cite{Chen_Chacón_Barnes_2011, Chacón_Chen_Barnes_2013, Chen_Chacón_2014, Chen_Chacón_2015} have developed structure-preserving particle pushers for neoclassical transport in the Vlasov equations, derived from Crank--Nicolson integrators. We show these too can can derive from a FET interpretation, similarly offering potential extensions to higher-order-in-time particle pushers. The FET formulation is used also to consider how the stochastic drift terms can be incorporated into the pushers. Stochastic gyrokinetic expansions are also discussed.

        Different options for the numerical implementation of these schemes are considered.

        Due to the efficacy of FET in the development of SP timesteppers for both the fluid and kinetic component, we hope this approach will prove effective in the future for developing SP timesteppers for the full hybrid model. We hope this will give us the opportunity to incorporate previously inaccessible kinetic effects into the highly effective, modern, finite-element MHD models.
    \end{abstract}
    
    
    \newpage
    \tableofcontents
    
    
    \newpage
    \pagenumbering{arabic}
    %\linenumbers\renewcommand\thelinenumber{\color{black!50}\arabic{linenumber}}
            \input{0 - introduction/main.tex}
        \part{Research}
            \input{1 - low-noise PiC models/main.tex}
            \input{2 - kinetic component/main.tex}
            \input{3 - fluid component/main.tex}
            \input{4 - numerical implementation/main.tex}
        \part{Project Overview}
            \input{5 - research plan/main.tex}
            \input{6 - summary/main.tex}
    
    
    %\section{}
    \newpage
    \pagenumbering{gobble}
        \printbibliography


    \newpage
    \pagenumbering{roman}
    \appendix
        \part{Appendices}
            \input{8 - Hilbert complexes/main.tex}
            \input{9 - weak conservation proofs/main.tex}
\end{document}

        \part{Research}
            \documentclass[12pt, a4paper]{report}

\input{template/main.tex}

\title{\BA{Title in Progress...}}
\author{Boris Andrews}
\affil{Mathematical Institute, University of Oxford}
\date{\today}


\begin{document}
    \pagenumbering{gobble}
    \maketitle
    
    
    \begin{abstract}
        Magnetic confinement reactors---in particular tokamaks---offer one of the most promising options for achieving practical nuclear fusion, with the potential to provide virtually limitless, clean energy. The theoretical and numerical modeling of tokamak plasmas is simultaneously an essential component of effective reactor design, and a great research barrier. Tokamak operational conditions exhibit comparatively low Knudsen numbers. Kinetic effects, including kinetic waves and instabilities, Landau damping, bump-on-tail instabilities and more, are therefore highly influential in tokamak plasma dynamics. Purely fluid models are inherently incapable of capturing these effects, whereas the high dimensionality in purely kinetic models render them practically intractable for most relevant purposes.

        We consider a $\delta\!f$ decomposition model, with a macroscopic fluid background and microscopic kinetic correction, both fully coupled to each other. A similar manner of discretization is proposed to that used in the recent \texttt{STRUPHY} code \cite{Holderied_Possanner_Wang_2021, Holderied_2022, Li_et_al_2023} with a finite-element model for the background and a pseudo-particle/PiC model for the correction.

        The fluid background satisfies the full, non-linear, resistive, compressible, Hall MHD equations. \cite{Laakmann_Hu_Farrell_2022} introduces finite-element(-in-space) implicit timesteppers for the incompressible analogue to this system with structure-preserving (SP) properties in the ideal case, alongside parameter-robust preconditioners. We show that these timesteppers can derive from a finite-element-in-time (FET) (and finite-element-in-space) interpretation. The benefits of this reformulation are discussed, including the derivation of timesteppers that are higher order in time, and the quantifiable dissipative SP properties in the non-ideal, resistive case.
        
        We discuss possible options for extending this FET approach to timesteppers for the compressible case.

        The kinetic corrections satisfy linearized Boltzmann equations. Using a Lénard--Bernstein collision operator, these take Fokker--Planck-like forms \cite{Fokker_1914, Planck_1917} wherein pseudo-particles in the numerical model obey the neoclassical transport equations, with particle-independent Brownian drift terms. This offers a rigorous methodology for incorporating collisions into the particle transport model, without coupling the equations of motions for each particle.
        
        Works by Chen, Chacón et al. \cite{Chen_Chacón_Barnes_2011, Chacón_Chen_Barnes_2013, Chen_Chacón_2014, Chen_Chacón_2015} have developed structure-preserving particle pushers for neoclassical transport in the Vlasov equations, derived from Crank--Nicolson integrators. We show these too can can derive from a FET interpretation, similarly offering potential extensions to higher-order-in-time particle pushers. The FET formulation is used also to consider how the stochastic drift terms can be incorporated into the pushers. Stochastic gyrokinetic expansions are also discussed.

        Different options for the numerical implementation of these schemes are considered.

        Due to the efficacy of FET in the development of SP timesteppers for both the fluid and kinetic component, we hope this approach will prove effective in the future for developing SP timesteppers for the full hybrid model. We hope this will give us the opportunity to incorporate previously inaccessible kinetic effects into the highly effective, modern, finite-element MHD models.
    \end{abstract}
    
    
    \newpage
    \tableofcontents
    
    
    \newpage
    \pagenumbering{arabic}
    %\linenumbers\renewcommand\thelinenumber{\color{black!50}\arabic{linenumber}}
            \input{0 - introduction/main.tex}
        \part{Research}
            \input{1 - low-noise PiC models/main.tex}
            \input{2 - kinetic component/main.tex}
            \input{3 - fluid component/main.tex}
            \input{4 - numerical implementation/main.tex}
        \part{Project Overview}
            \input{5 - research plan/main.tex}
            \input{6 - summary/main.tex}
    
    
    %\section{}
    \newpage
    \pagenumbering{gobble}
        \printbibliography


    \newpage
    \pagenumbering{roman}
    \appendix
        \part{Appendices}
            \input{8 - Hilbert complexes/main.tex}
            \input{9 - weak conservation proofs/main.tex}
\end{document}

            \documentclass[12pt, a4paper]{report}

\input{template/main.tex}

\title{\BA{Title in Progress...}}
\author{Boris Andrews}
\affil{Mathematical Institute, University of Oxford}
\date{\today}


\begin{document}
    \pagenumbering{gobble}
    \maketitle
    
    
    \begin{abstract}
        Magnetic confinement reactors---in particular tokamaks---offer one of the most promising options for achieving practical nuclear fusion, with the potential to provide virtually limitless, clean energy. The theoretical and numerical modeling of tokamak plasmas is simultaneously an essential component of effective reactor design, and a great research barrier. Tokamak operational conditions exhibit comparatively low Knudsen numbers. Kinetic effects, including kinetic waves and instabilities, Landau damping, bump-on-tail instabilities and more, are therefore highly influential in tokamak plasma dynamics. Purely fluid models are inherently incapable of capturing these effects, whereas the high dimensionality in purely kinetic models render them practically intractable for most relevant purposes.

        We consider a $\delta\!f$ decomposition model, with a macroscopic fluid background and microscopic kinetic correction, both fully coupled to each other. A similar manner of discretization is proposed to that used in the recent \texttt{STRUPHY} code \cite{Holderied_Possanner_Wang_2021, Holderied_2022, Li_et_al_2023} with a finite-element model for the background and a pseudo-particle/PiC model for the correction.

        The fluid background satisfies the full, non-linear, resistive, compressible, Hall MHD equations. \cite{Laakmann_Hu_Farrell_2022} introduces finite-element(-in-space) implicit timesteppers for the incompressible analogue to this system with structure-preserving (SP) properties in the ideal case, alongside parameter-robust preconditioners. We show that these timesteppers can derive from a finite-element-in-time (FET) (and finite-element-in-space) interpretation. The benefits of this reformulation are discussed, including the derivation of timesteppers that are higher order in time, and the quantifiable dissipative SP properties in the non-ideal, resistive case.
        
        We discuss possible options for extending this FET approach to timesteppers for the compressible case.

        The kinetic corrections satisfy linearized Boltzmann equations. Using a Lénard--Bernstein collision operator, these take Fokker--Planck-like forms \cite{Fokker_1914, Planck_1917} wherein pseudo-particles in the numerical model obey the neoclassical transport equations, with particle-independent Brownian drift terms. This offers a rigorous methodology for incorporating collisions into the particle transport model, without coupling the equations of motions for each particle.
        
        Works by Chen, Chacón et al. \cite{Chen_Chacón_Barnes_2011, Chacón_Chen_Barnes_2013, Chen_Chacón_2014, Chen_Chacón_2015} have developed structure-preserving particle pushers for neoclassical transport in the Vlasov equations, derived from Crank--Nicolson integrators. We show these too can can derive from a FET interpretation, similarly offering potential extensions to higher-order-in-time particle pushers. The FET formulation is used also to consider how the stochastic drift terms can be incorporated into the pushers. Stochastic gyrokinetic expansions are also discussed.

        Different options for the numerical implementation of these schemes are considered.

        Due to the efficacy of FET in the development of SP timesteppers for both the fluid and kinetic component, we hope this approach will prove effective in the future for developing SP timesteppers for the full hybrid model. We hope this will give us the opportunity to incorporate previously inaccessible kinetic effects into the highly effective, modern, finite-element MHD models.
    \end{abstract}
    
    
    \newpage
    \tableofcontents
    
    
    \newpage
    \pagenumbering{arabic}
    %\linenumbers\renewcommand\thelinenumber{\color{black!50}\arabic{linenumber}}
            \input{0 - introduction/main.tex}
        \part{Research}
            \input{1 - low-noise PiC models/main.tex}
            \input{2 - kinetic component/main.tex}
            \input{3 - fluid component/main.tex}
            \input{4 - numerical implementation/main.tex}
        \part{Project Overview}
            \input{5 - research plan/main.tex}
            \input{6 - summary/main.tex}
    
    
    %\section{}
    \newpage
    \pagenumbering{gobble}
        \printbibliography


    \newpage
    \pagenumbering{roman}
    \appendix
        \part{Appendices}
            \input{8 - Hilbert complexes/main.tex}
            \input{9 - weak conservation proofs/main.tex}
\end{document}

            \documentclass[12pt, a4paper]{report}

\input{template/main.tex}

\title{\BA{Title in Progress...}}
\author{Boris Andrews}
\affil{Mathematical Institute, University of Oxford}
\date{\today}


\begin{document}
    \pagenumbering{gobble}
    \maketitle
    
    
    \begin{abstract}
        Magnetic confinement reactors---in particular tokamaks---offer one of the most promising options for achieving practical nuclear fusion, with the potential to provide virtually limitless, clean energy. The theoretical and numerical modeling of tokamak plasmas is simultaneously an essential component of effective reactor design, and a great research barrier. Tokamak operational conditions exhibit comparatively low Knudsen numbers. Kinetic effects, including kinetic waves and instabilities, Landau damping, bump-on-tail instabilities and more, are therefore highly influential in tokamak plasma dynamics. Purely fluid models are inherently incapable of capturing these effects, whereas the high dimensionality in purely kinetic models render them practically intractable for most relevant purposes.

        We consider a $\delta\!f$ decomposition model, with a macroscopic fluid background and microscopic kinetic correction, both fully coupled to each other. A similar manner of discretization is proposed to that used in the recent \texttt{STRUPHY} code \cite{Holderied_Possanner_Wang_2021, Holderied_2022, Li_et_al_2023} with a finite-element model for the background and a pseudo-particle/PiC model for the correction.

        The fluid background satisfies the full, non-linear, resistive, compressible, Hall MHD equations. \cite{Laakmann_Hu_Farrell_2022} introduces finite-element(-in-space) implicit timesteppers for the incompressible analogue to this system with structure-preserving (SP) properties in the ideal case, alongside parameter-robust preconditioners. We show that these timesteppers can derive from a finite-element-in-time (FET) (and finite-element-in-space) interpretation. The benefits of this reformulation are discussed, including the derivation of timesteppers that are higher order in time, and the quantifiable dissipative SP properties in the non-ideal, resistive case.
        
        We discuss possible options for extending this FET approach to timesteppers for the compressible case.

        The kinetic corrections satisfy linearized Boltzmann equations. Using a Lénard--Bernstein collision operator, these take Fokker--Planck-like forms \cite{Fokker_1914, Planck_1917} wherein pseudo-particles in the numerical model obey the neoclassical transport equations, with particle-independent Brownian drift terms. This offers a rigorous methodology for incorporating collisions into the particle transport model, without coupling the equations of motions for each particle.
        
        Works by Chen, Chacón et al. \cite{Chen_Chacón_Barnes_2011, Chacón_Chen_Barnes_2013, Chen_Chacón_2014, Chen_Chacón_2015} have developed structure-preserving particle pushers for neoclassical transport in the Vlasov equations, derived from Crank--Nicolson integrators. We show these too can can derive from a FET interpretation, similarly offering potential extensions to higher-order-in-time particle pushers. The FET formulation is used also to consider how the stochastic drift terms can be incorporated into the pushers. Stochastic gyrokinetic expansions are also discussed.

        Different options for the numerical implementation of these schemes are considered.

        Due to the efficacy of FET in the development of SP timesteppers for both the fluid and kinetic component, we hope this approach will prove effective in the future for developing SP timesteppers for the full hybrid model. We hope this will give us the opportunity to incorporate previously inaccessible kinetic effects into the highly effective, modern, finite-element MHD models.
    \end{abstract}
    
    
    \newpage
    \tableofcontents
    
    
    \newpage
    \pagenumbering{arabic}
    %\linenumbers\renewcommand\thelinenumber{\color{black!50}\arabic{linenumber}}
            \input{0 - introduction/main.tex}
        \part{Research}
            \input{1 - low-noise PiC models/main.tex}
            \input{2 - kinetic component/main.tex}
            \input{3 - fluid component/main.tex}
            \input{4 - numerical implementation/main.tex}
        \part{Project Overview}
            \input{5 - research plan/main.tex}
            \input{6 - summary/main.tex}
    
    
    %\section{}
    \newpage
    \pagenumbering{gobble}
        \printbibliography


    \newpage
    \pagenumbering{roman}
    \appendix
        \part{Appendices}
            \input{8 - Hilbert complexes/main.tex}
            \input{9 - weak conservation proofs/main.tex}
\end{document}

            \documentclass[12pt, a4paper]{report}

\input{template/main.tex}

\title{\BA{Title in Progress...}}
\author{Boris Andrews}
\affil{Mathematical Institute, University of Oxford}
\date{\today}


\begin{document}
    \pagenumbering{gobble}
    \maketitle
    
    
    \begin{abstract}
        Magnetic confinement reactors---in particular tokamaks---offer one of the most promising options for achieving practical nuclear fusion, with the potential to provide virtually limitless, clean energy. The theoretical and numerical modeling of tokamak plasmas is simultaneously an essential component of effective reactor design, and a great research barrier. Tokamak operational conditions exhibit comparatively low Knudsen numbers. Kinetic effects, including kinetic waves and instabilities, Landau damping, bump-on-tail instabilities and more, are therefore highly influential in tokamak plasma dynamics. Purely fluid models are inherently incapable of capturing these effects, whereas the high dimensionality in purely kinetic models render them practically intractable for most relevant purposes.

        We consider a $\delta\!f$ decomposition model, with a macroscopic fluid background and microscopic kinetic correction, both fully coupled to each other. A similar manner of discretization is proposed to that used in the recent \texttt{STRUPHY} code \cite{Holderied_Possanner_Wang_2021, Holderied_2022, Li_et_al_2023} with a finite-element model for the background and a pseudo-particle/PiC model for the correction.

        The fluid background satisfies the full, non-linear, resistive, compressible, Hall MHD equations. \cite{Laakmann_Hu_Farrell_2022} introduces finite-element(-in-space) implicit timesteppers for the incompressible analogue to this system with structure-preserving (SP) properties in the ideal case, alongside parameter-robust preconditioners. We show that these timesteppers can derive from a finite-element-in-time (FET) (and finite-element-in-space) interpretation. The benefits of this reformulation are discussed, including the derivation of timesteppers that are higher order in time, and the quantifiable dissipative SP properties in the non-ideal, resistive case.
        
        We discuss possible options for extending this FET approach to timesteppers for the compressible case.

        The kinetic corrections satisfy linearized Boltzmann equations. Using a Lénard--Bernstein collision operator, these take Fokker--Planck-like forms \cite{Fokker_1914, Planck_1917} wherein pseudo-particles in the numerical model obey the neoclassical transport equations, with particle-independent Brownian drift terms. This offers a rigorous methodology for incorporating collisions into the particle transport model, without coupling the equations of motions for each particle.
        
        Works by Chen, Chacón et al. \cite{Chen_Chacón_Barnes_2011, Chacón_Chen_Barnes_2013, Chen_Chacón_2014, Chen_Chacón_2015} have developed structure-preserving particle pushers for neoclassical transport in the Vlasov equations, derived from Crank--Nicolson integrators. We show these too can can derive from a FET interpretation, similarly offering potential extensions to higher-order-in-time particle pushers. The FET formulation is used also to consider how the stochastic drift terms can be incorporated into the pushers. Stochastic gyrokinetic expansions are also discussed.

        Different options for the numerical implementation of these schemes are considered.

        Due to the efficacy of FET in the development of SP timesteppers for both the fluid and kinetic component, we hope this approach will prove effective in the future for developing SP timesteppers for the full hybrid model. We hope this will give us the opportunity to incorporate previously inaccessible kinetic effects into the highly effective, modern, finite-element MHD models.
    \end{abstract}
    
    
    \newpage
    \tableofcontents
    
    
    \newpage
    \pagenumbering{arabic}
    %\linenumbers\renewcommand\thelinenumber{\color{black!50}\arabic{linenumber}}
            \input{0 - introduction/main.tex}
        \part{Research}
            \input{1 - low-noise PiC models/main.tex}
            \input{2 - kinetic component/main.tex}
            \input{3 - fluid component/main.tex}
            \input{4 - numerical implementation/main.tex}
        \part{Project Overview}
            \input{5 - research plan/main.tex}
            \input{6 - summary/main.tex}
    
    
    %\section{}
    \newpage
    \pagenumbering{gobble}
        \printbibliography


    \newpage
    \pagenumbering{roman}
    \appendix
        \part{Appendices}
            \input{8 - Hilbert complexes/main.tex}
            \input{9 - weak conservation proofs/main.tex}
\end{document}

        \part{Project Overview}
            \documentclass[12pt, a4paper]{report}

\input{template/main.tex}

\title{\BA{Title in Progress...}}
\author{Boris Andrews}
\affil{Mathematical Institute, University of Oxford}
\date{\today}


\begin{document}
    \pagenumbering{gobble}
    \maketitle
    
    
    \begin{abstract}
        Magnetic confinement reactors---in particular tokamaks---offer one of the most promising options for achieving practical nuclear fusion, with the potential to provide virtually limitless, clean energy. The theoretical and numerical modeling of tokamak plasmas is simultaneously an essential component of effective reactor design, and a great research barrier. Tokamak operational conditions exhibit comparatively low Knudsen numbers. Kinetic effects, including kinetic waves and instabilities, Landau damping, bump-on-tail instabilities and more, are therefore highly influential in tokamak plasma dynamics. Purely fluid models are inherently incapable of capturing these effects, whereas the high dimensionality in purely kinetic models render them practically intractable for most relevant purposes.

        We consider a $\delta\!f$ decomposition model, with a macroscopic fluid background and microscopic kinetic correction, both fully coupled to each other. A similar manner of discretization is proposed to that used in the recent \texttt{STRUPHY} code \cite{Holderied_Possanner_Wang_2021, Holderied_2022, Li_et_al_2023} with a finite-element model for the background and a pseudo-particle/PiC model for the correction.

        The fluid background satisfies the full, non-linear, resistive, compressible, Hall MHD equations. \cite{Laakmann_Hu_Farrell_2022} introduces finite-element(-in-space) implicit timesteppers for the incompressible analogue to this system with structure-preserving (SP) properties in the ideal case, alongside parameter-robust preconditioners. We show that these timesteppers can derive from a finite-element-in-time (FET) (and finite-element-in-space) interpretation. The benefits of this reformulation are discussed, including the derivation of timesteppers that are higher order in time, and the quantifiable dissipative SP properties in the non-ideal, resistive case.
        
        We discuss possible options for extending this FET approach to timesteppers for the compressible case.

        The kinetic corrections satisfy linearized Boltzmann equations. Using a Lénard--Bernstein collision operator, these take Fokker--Planck-like forms \cite{Fokker_1914, Planck_1917} wherein pseudo-particles in the numerical model obey the neoclassical transport equations, with particle-independent Brownian drift terms. This offers a rigorous methodology for incorporating collisions into the particle transport model, without coupling the equations of motions for each particle.
        
        Works by Chen, Chacón et al. \cite{Chen_Chacón_Barnes_2011, Chacón_Chen_Barnes_2013, Chen_Chacón_2014, Chen_Chacón_2015} have developed structure-preserving particle pushers for neoclassical transport in the Vlasov equations, derived from Crank--Nicolson integrators. We show these too can can derive from a FET interpretation, similarly offering potential extensions to higher-order-in-time particle pushers. The FET formulation is used also to consider how the stochastic drift terms can be incorporated into the pushers. Stochastic gyrokinetic expansions are also discussed.

        Different options for the numerical implementation of these schemes are considered.

        Due to the efficacy of FET in the development of SP timesteppers for both the fluid and kinetic component, we hope this approach will prove effective in the future for developing SP timesteppers for the full hybrid model. We hope this will give us the opportunity to incorporate previously inaccessible kinetic effects into the highly effective, modern, finite-element MHD models.
    \end{abstract}
    
    
    \newpage
    \tableofcontents
    
    
    \newpage
    \pagenumbering{arabic}
    %\linenumbers\renewcommand\thelinenumber{\color{black!50}\arabic{linenumber}}
            \input{0 - introduction/main.tex}
        \part{Research}
            \input{1 - low-noise PiC models/main.tex}
            \input{2 - kinetic component/main.tex}
            \input{3 - fluid component/main.tex}
            \input{4 - numerical implementation/main.tex}
        \part{Project Overview}
            \input{5 - research plan/main.tex}
            \input{6 - summary/main.tex}
    
    
    %\section{}
    \newpage
    \pagenumbering{gobble}
        \printbibliography


    \newpage
    \pagenumbering{roman}
    \appendix
        \part{Appendices}
            \input{8 - Hilbert complexes/main.tex}
            \input{9 - weak conservation proofs/main.tex}
\end{document}

            \documentclass[12pt, a4paper]{report}

\input{template/main.tex}

\title{\BA{Title in Progress...}}
\author{Boris Andrews}
\affil{Mathematical Institute, University of Oxford}
\date{\today}


\begin{document}
    \pagenumbering{gobble}
    \maketitle
    
    
    \begin{abstract}
        Magnetic confinement reactors---in particular tokamaks---offer one of the most promising options for achieving practical nuclear fusion, with the potential to provide virtually limitless, clean energy. The theoretical and numerical modeling of tokamak plasmas is simultaneously an essential component of effective reactor design, and a great research barrier. Tokamak operational conditions exhibit comparatively low Knudsen numbers. Kinetic effects, including kinetic waves and instabilities, Landau damping, bump-on-tail instabilities and more, are therefore highly influential in tokamak plasma dynamics. Purely fluid models are inherently incapable of capturing these effects, whereas the high dimensionality in purely kinetic models render them practically intractable for most relevant purposes.

        We consider a $\delta\!f$ decomposition model, with a macroscopic fluid background and microscopic kinetic correction, both fully coupled to each other. A similar manner of discretization is proposed to that used in the recent \texttt{STRUPHY} code \cite{Holderied_Possanner_Wang_2021, Holderied_2022, Li_et_al_2023} with a finite-element model for the background and a pseudo-particle/PiC model for the correction.

        The fluid background satisfies the full, non-linear, resistive, compressible, Hall MHD equations. \cite{Laakmann_Hu_Farrell_2022} introduces finite-element(-in-space) implicit timesteppers for the incompressible analogue to this system with structure-preserving (SP) properties in the ideal case, alongside parameter-robust preconditioners. We show that these timesteppers can derive from a finite-element-in-time (FET) (and finite-element-in-space) interpretation. The benefits of this reformulation are discussed, including the derivation of timesteppers that are higher order in time, and the quantifiable dissipative SP properties in the non-ideal, resistive case.
        
        We discuss possible options for extending this FET approach to timesteppers for the compressible case.

        The kinetic corrections satisfy linearized Boltzmann equations. Using a Lénard--Bernstein collision operator, these take Fokker--Planck-like forms \cite{Fokker_1914, Planck_1917} wherein pseudo-particles in the numerical model obey the neoclassical transport equations, with particle-independent Brownian drift terms. This offers a rigorous methodology for incorporating collisions into the particle transport model, without coupling the equations of motions for each particle.
        
        Works by Chen, Chacón et al. \cite{Chen_Chacón_Barnes_2011, Chacón_Chen_Barnes_2013, Chen_Chacón_2014, Chen_Chacón_2015} have developed structure-preserving particle pushers for neoclassical transport in the Vlasov equations, derived from Crank--Nicolson integrators. We show these too can can derive from a FET interpretation, similarly offering potential extensions to higher-order-in-time particle pushers. The FET formulation is used also to consider how the stochastic drift terms can be incorporated into the pushers. Stochastic gyrokinetic expansions are also discussed.

        Different options for the numerical implementation of these schemes are considered.

        Due to the efficacy of FET in the development of SP timesteppers for both the fluid and kinetic component, we hope this approach will prove effective in the future for developing SP timesteppers for the full hybrid model. We hope this will give us the opportunity to incorporate previously inaccessible kinetic effects into the highly effective, modern, finite-element MHD models.
    \end{abstract}
    
    
    \newpage
    \tableofcontents
    
    
    \newpage
    \pagenumbering{arabic}
    %\linenumbers\renewcommand\thelinenumber{\color{black!50}\arabic{linenumber}}
            \input{0 - introduction/main.tex}
        \part{Research}
            \input{1 - low-noise PiC models/main.tex}
            \input{2 - kinetic component/main.tex}
            \input{3 - fluid component/main.tex}
            \input{4 - numerical implementation/main.tex}
        \part{Project Overview}
            \input{5 - research plan/main.tex}
            \input{6 - summary/main.tex}
    
    
    %\section{}
    \newpage
    \pagenumbering{gobble}
        \printbibliography


    \newpage
    \pagenumbering{roman}
    \appendix
        \part{Appendices}
            \input{8 - Hilbert complexes/main.tex}
            \input{9 - weak conservation proofs/main.tex}
\end{document}

    
    
    %\section{}
    \newpage
    \pagenumbering{gobble}
        \printbibliography


    \newpage
    \pagenumbering{roman}
    \appendix
        \part{Appendices}
            \documentclass[12pt, a4paper]{report}

\input{template/main.tex}

\title{\BA{Title in Progress...}}
\author{Boris Andrews}
\affil{Mathematical Institute, University of Oxford}
\date{\today}


\begin{document}
    \pagenumbering{gobble}
    \maketitle
    
    
    \begin{abstract}
        Magnetic confinement reactors---in particular tokamaks---offer one of the most promising options for achieving practical nuclear fusion, with the potential to provide virtually limitless, clean energy. The theoretical and numerical modeling of tokamak plasmas is simultaneously an essential component of effective reactor design, and a great research barrier. Tokamak operational conditions exhibit comparatively low Knudsen numbers. Kinetic effects, including kinetic waves and instabilities, Landau damping, bump-on-tail instabilities and more, are therefore highly influential in tokamak plasma dynamics. Purely fluid models are inherently incapable of capturing these effects, whereas the high dimensionality in purely kinetic models render them practically intractable for most relevant purposes.

        We consider a $\delta\!f$ decomposition model, with a macroscopic fluid background and microscopic kinetic correction, both fully coupled to each other. A similar manner of discretization is proposed to that used in the recent \texttt{STRUPHY} code \cite{Holderied_Possanner_Wang_2021, Holderied_2022, Li_et_al_2023} with a finite-element model for the background and a pseudo-particle/PiC model for the correction.

        The fluid background satisfies the full, non-linear, resistive, compressible, Hall MHD equations. \cite{Laakmann_Hu_Farrell_2022} introduces finite-element(-in-space) implicit timesteppers for the incompressible analogue to this system with structure-preserving (SP) properties in the ideal case, alongside parameter-robust preconditioners. We show that these timesteppers can derive from a finite-element-in-time (FET) (and finite-element-in-space) interpretation. The benefits of this reformulation are discussed, including the derivation of timesteppers that are higher order in time, and the quantifiable dissipative SP properties in the non-ideal, resistive case.
        
        We discuss possible options for extending this FET approach to timesteppers for the compressible case.

        The kinetic corrections satisfy linearized Boltzmann equations. Using a Lénard--Bernstein collision operator, these take Fokker--Planck-like forms \cite{Fokker_1914, Planck_1917} wherein pseudo-particles in the numerical model obey the neoclassical transport equations, with particle-independent Brownian drift terms. This offers a rigorous methodology for incorporating collisions into the particle transport model, without coupling the equations of motions for each particle.
        
        Works by Chen, Chacón et al. \cite{Chen_Chacón_Barnes_2011, Chacón_Chen_Barnes_2013, Chen_Chacón_2014, Chen_Chacón_2015} have developed structure-preserving particle pushers for neoclassical transport in the Vlasov equations, derived from Crank--Nicolson integrators. We show these too can can derive from a FET interpretation, similarly offering potential extensions to higher-order-in-time particle pushers. The FET formulation is used also to consider how the stochastic drift terms can be incorporated into the pushers. Stochastic gyrokinetic expansions are also discussed.

        Different options for the numerical implementation of these schemes are considered.

        Due to the efficacy of FET in the development of SP timesteppers for both the fluid and kinetic component, we hope this approach will prove effective in the future for developing SP timesteppers for the full hybrid model. We hope this will give us the opportunity to incorporate previously inaccessible kinetic effects into the highly effective, modern, finite-element MHD models.
    \end{abstract}
    
    
    \newpage
    \tableofcontents
    
    
    \newpage
    \pagenumbering{arabic}
    %\linenumbers\renewcommand\thelinenumber{\color{black!50}\arabic{linenumber}}
            \input{0 - introduction/main.tex}
        \part{Research}
            \input{1 - low-noise PiC models/main.tex}
            \input{2 - kinetic component/main.tex}
            \input{3 - fluid component/main.tex}
            \input{4 - numerical implementation/main.tex}
        \part{Project Overview}
            \input{5 - research plan/main.tex}
            \input{6 - summary/main.tex}
    
    
    %\section{}
    \newpage
    \pagenumbering{gobble}
        \printbibliography


    \newpage
    \pagenumbering{roman}
    \appendix
        \part{Appendices}
            \input{8 - Hilbert complexes/main.tex}
            \input{9 - weak conservation proofs/main.tex}
\end{document}

            \documentclass[12pt, a4paper]{report}

\input{template/main.tex}

\title{\BA{Title in Progress...}}
\author{Boris Andrews}
\affil{Mathematical Institute, University of Oxford}
\date{\today}


\begin{document}
    \pagenumbering{gobble}
    \maketitle
    
    
    \begin{abstract}
        Magnetic confinement reactors---in particular tokamaks---offer one of the most promising options for achieving practical nuclear fusion, with the potential to provide virtually limitless, clean energy. The theoretical and numerical modeling of tokamak plasmas is simultaneously an essential component of effective reactor design, and a great research barrier. Tokamak operational conditions exhibit comparatively low Knudsen numbers. Kinetic effects, including kinetic waves and instabilities, Landau damping, bump-on-tail instabilities and more, are therefore highly influential in tokamak plasma dynamics. Purely fluid models are inherently incapable of capturing these effects, whereas the high dimensionality in purely kinetic models render them practically intractable for most relevant purposes.

        We consider a $\delta\!f$ decomposition model, with a macroscopic fluid background and microscopic kinetic correction, both fully coupled to each other. A similar manner of discretization is proposed to that used in the recent \texttt{STRUPHY} code \cite{Holderied_Possanner_Wang_2021, Holderied_2022, Li_et_al_2023} with a finite-element model for the background and a pseudo-particle/PiC model for the correction.

        The fluid background satisfies the full, non-linear, resistive, compressible, Hall MHD equations. \cite{Laakmann_Hu_Farrell_2022} introduces finite-element(-in-space) implicit timesteppers for the incompressible analogue to this system with structure-preserving (SP) properties in the ideal case, alongside parameter-robust preconditioners. We show that these timesteppers can derive from a finite-element-in-time (FET) (and finite-element-in-space) interpretation. The benefits of this reformulation are discussed, including the derivation of timesteppers that are higher order in time, and the quantifiable dissipative SP properties in the non-ideal, resistive case.
        
        We discuss possible options for extending this FET approach to timesteppers for the compressible case.

        The kinetic corrections satisfy linearized Boltzmann equations. Using a Lénard--Bernstein collision operator, these take Fokker--Planck-like forms \cite{Fokker_1914, Planck_1917} wherein pseudo-particles in the numerical model obey the neoclassical transport equations, with particle-independent Brownian drift terms. This offers a rigorous methodology for incorporating collisions into the particle transport model, without coupling the equations of motions for each particle.
        
        Works by Chen, Chacón et al. \cite{Chen_Chacón_Barnes_2011, Chacón_Chen_Barnes_2013, Chen_Chacón_2014, Chen_Chacón_2015} have developed structure-preserving particle pushers for neoclassical transport in the Vlasov equations, derived from Crank--Nicolson integrators. We show these too can can derive from a FET interpretation, similarly offering potential extensions to higher-order-in-time particle pushers. The FET formulation is used also to consider how the stochastic drift terms can be incorporated into the pushers. Stochastic gyrokinetic expansions are also discussed.

        Different options for the numerical implementation of these schemes are considered.

        Due to the efficacy of FET in the development of SP timesteppers for both the fluid and kinetic component, we hope this approach will prove effective in the future for developing SP timesteppers for the full hybrid model. We hope this will give us the opportunity to incorporate previously inaccessible kinetic effects into the highly effective, modern, finite-element MHD models.
    \end{abstract}
    
    
    \newpage
    \tableofcontents
    
    
    \newpage
    \pagenumbering{arabic}
    %\linenumbers\renewcommand\thelinenumber{\color{black!50}\arabic{linenumber}}
            \input{0 - introduction/main.tex}
        \part{Research}
            \input{1 - low-noise PiC models/main.tex}
            \input{2 - kinetic component/main.tex}
            \input{3 - fluid component/main.tex}
            \input{4 - numerical implementation/main.tex}
        \part{Project Overview}
            \input{5 - research plan/main.tex}
            \input{6 - summary/main.tex}
    
    
    %\section{}
    \newpage
    \pagenumbering{gobble}
        \printbibliography


    \newpage
    \pagenumbering{roman}
    \appendix
        \part{Appendices}
            \input{8 - Hilbert complexes/main.tex}
            \input{9 - weak conservation proofs/main.tex}
\end{document}

\end{document}


\title{\BA{Title in Progress...}}
\author{Boris Andrews}
\affil{Mathematical Institute, University of Oxford}
\date{\today}


\begin{document}
    \pagenumbering{gobble}
    \maketitle
    
    
    \begin{abstract}
        Magnetic confinement reactors---in particular tokamaks---offer one of the most promising options for achieving practical nuclear fusion, with the potential to provide virtually limitless, clean energy. The theoretical and numerical modeling of tokamak plasmas is simultaneously an essential component of effective reactor design, and a great research barrier. Tokamak operational conditions exhibit comparatively low Knudsen numbers. Kinetic effects, including kinetic waves and instabilities, Landau damping, bump-on-tail instabilities and more, are therefore highly influential in tokamak plasma dynamics. Purely fluid models are inherently incapable of capturing these effects, whereas the high dimensionality in purely kinetic models render them practically intractable for most relevant purposes.

        We consider a $\delta\!f$ decomposition model, with a macroscopic fluid background and microscopic kinetic correction, both fully coupled to each other. A similar manner of discretization is proposed to that used in the recent \texttt{STRUPHY} code \cite{Holderied_Possanner_Wang_2021, Holderied_2022, Li_et_al_2023} with a finite-element model for the background and a pseudo-particle/PiC model for the correction.

        The fluid background satisfies the full, non-linear, resistive, compressible, Hall MHD equations. \cite{Laakmann_Hu_Farrell_2022} introduces finite-element(-in-space) implicit timesteppers for the incompressible analogue to this system with structure-preserving (SP) properties in the ideal case, alongside parameter-robust preconditioners. We show that these timesteppers can derive from a finite-element-in-time (FET) (and finite-element-in-space) interpretation. The benefits of this reformulation are discussed, including the derivation of timesteppers that are higher order in time, and the quantifiable dissipative SP properties in the non-ideal, resistive case.
        
        We discuss possible options for extending this FET approach to timesteppers for the compressible case.

        The kinetic corrections satisfy linearized Boltzmann equations. Using a Lénard--Bernstein collision operator, these take Fokker--Planck-like forms \cite{Fokker_1914, Planck_1917} wherein pseudo-particles in the numerical model obey the neoclassical transport equations, with particle-independent Brownian drift terms. This offers a rigorous methodology for incorporating collisions into the particle transport model, without coupling the equations of motions for each particle.
        
        Works by Chen, Chacón et al. \cite{Chen_Chacón_Barnes_2011, Chacón_Chen_Barnes_2013, Chen_Chacón_2014, Chen_Chacón_2015} have developed structure-preserving particle pushers for neoclassical transport in the Vlasov equations, derived from Crank--Nicolson integrators. We show these too can can derive from a FET interpretation, similarly offering potential extensions to higher-order-in-time particle pushers. The FET formulation is used also to consider how the stochastic drift terms can be incorporated into the pushers. Stochastic gyrokinetic expansions are also discussed.

        Different options for the numerical implementation of these schemes are considered.

        Due to the efficacy of FET in the development of SP timesteppers for both the fluid and kinetic component, we hope this approach will prove effective in the future for developing SP timesteppers for the full hybrid model. We hope this will give us the opportunity to incorporate previously inaccessible kinetic effects into the highly effective, modern, finite-element MHD models.
    \end{abstract}
    
    
    \newpage
    \tableofcontents
    
    
    \newpage
    \pagenumbering{arabic}
    %\linenumbers\renewcommand\thelinenumber{\color{black!50}\arabic{linenumber}}
            \documentclass[12pt, a4paper]{report}

\documentclass[12pt, a4paper]{report}

\input{template/main.tex}

\title{\BA{Title in Progress...}}
\author{Boris Andrews}
\affil{Mathematical Institute, University of Oxford}
\date{\today}


\begin{document}
    \pagenumbering{gobble}
    \maketitle
    
    
    \begin{abstract}
        Magnetic confinement reactors---in particular tokamaks---offer one of the most promising options for achieving practical nuclear fusion, with the potential to provide virtually limitless, clean energy. The theoretical and numerical modeling of tokamak plasmas is simultaneously an essential component of effective reactor design, and a great research barrier. Tokamak operational conditions exhibit comparatively low Knudsen numbers. Kinetic effects, including kinetic waves and instabilities, Landau damping, bump-on-tail instabilities and more, are therefore highly influential in tokamak plasma dynamics. Purely fluid models are inherently incapable of capturing these effects, whereas the high dimensionality in purely kinetic models render them practically intractable for most relevant purposes.

        We consider a $\delta\!f$ decomposition model, with a macroscopic fluid background and microscopic kinetic correction, both fully coupled to each other. A similar manner of discretization is proposed to that used in the recent \texttt{STRUPHY} code \cite{Holderied_Possanner_Wang_2021, Holderied_2022, Li_et_al_2023} with a finite-element model for the background and a pseudo-particle/PiC model for the correction.

        The fluid background satisfies the full, non-linear, resistive, compressible, Hall MHD equations. \cite{Laakmann_Hu_Farrell_2022} introduces finite-element(-in-space) implicit timesteppers for the incompressible analogue to this system with structure-preserving (SP) properties in the ideal case, alongside parameter-robust preconditioners. We show that these timesteppers can derive from a finite-element-in-time (FET) (and finite-element-in-space) interpretation. The benefits of this reformulation are discussed, including the derivation of timesteppers that are higher order in time, and the quantifiable dissipative SP properties in the non-ideal, resistive case.
        
        We discuss possible options for extending this FET approach to timesteppers for the compressible case.

        The kinetic corrections satisfy linearized Boltzmann equations. Using a Lénard--Bernstein collision operator, these take Fokker--Planck-like forms \cite{Fokker_1914, Planck_1917} wherein pseudo-particles in the numerical model obey the neoclassical transport equations, with particle-independent Brownian drift terms. This offers a rigorous methodology for incorporating collisions into the particle transport model, without coupling the equations of motions for each particle.
        
        Works by Chen, Chacón et al. \cite{Chen_Chacón_Barnes_2011, Chacón_Chen_Barnes_2013, Chen_Chacón_2014, Chen_Chacón_2015} have developed structure-preserving particle pushers for neoclassical transport in the Vlasov equations, derived from Crank--Nicolson integrators. We show these too can can derive from a FET interpretation, similarly offering potential extensions to higher-order-in-time particle pushers. The FET formulation is used also to consider how the stochastic drift terms can be incorporated into the pushers. Stochastic gyrokinetic expansions are also discussed.

        Different options for the numerical implementation of these schemes are considered.

        Due to the efficacy of FET in the development of SP timesteppers for both the fluid and kinetic component, we hope this approach will prove effective in the future for developing SP timesteppers for the full hybrid model. We hope this will give us the opportunity to incorporate previously inaccessible kinetic effects into the highly effective, modern, finite-element MHD models.
    \end{abstract}
    
    
    \newpage
    \tableofcontents
    
    
    \newpage
    \pagenumbering{arabic}
    %\linenumbers\renewcommand\thelinenumber{\color{black!50}\arabic{linenumber}}
            \input{0 - introduction/main.tex}
        \part{Research}
            \input{1 - low-noise PiC models/main.tex}
            \input{2 - kinetic component/main.tex}
            \input{3 - fluid component/main.tex}
            \input{4 - numerical implementation/main.tex}
        \part{Project Overview}
            \input{5 - research plan/main.tex}
            \input{6 - summary/main.tex}
    
    
    %\section{}
    \newpage
    \pagenumbering{gobble}
        \printbibliography


    \newpage
    \pagenumbering{roman}
    \appendix
        \part{Appendices}
            \input{8 - Hilbert complexes/main.tex}
            \input{9 - weak conservation proofs/main.tex}
\end{document}


\title{\BA{Title in Progress...}}
\author{Boris Andrews}
\affil{Mathematical Institute, University of Oxford}
\date{\today}


\begin{document}
    \pagenumbering{gobble}
    \maketitle
    
    
    \begin{abstract}
        Magnetic confinement reactors---in particular tokamaks---offer one of the most promising options for achieving practical nuclear fusion, with the potential to provide virtually limitless, clean energy. The theoretical and numerical modeling of tokamak plasmas is simultaneously an essential component of effective reactor design, and a great research barrier. Tokamak operational conditions exhibit comparatively low Knudsen numbers. Kinetic effects, including kinetic waves and instabilities, Landau damping, bump-on-tail instabilities and more, are therefore highly influential in tokamak plasma dynamics. Purely fluid models are inherently incapable of capturing these effects, whereas the high dimensionality in purely kinetic models render them practically intractable for most relevant purposes.

        We consider a $\delta\!f$ decomposition model, with a macroscopic fluid background and microscopic kinetic correction, both fully coupled to each other. A similar manner of discretization is proposed to that used in the recent \texttt{STRUPHY} code \cite{Holderied_Possanner_Wang_2021, Holderied_2022, Li_et_al_2023} with a finite-element model for the background and a pseudo-particle/PiC model for the correction.

        The fluid background satisfies the full, non-linear, resistive, compressible, Hall MHD equations. \cite{Laakmann_Hu_Farrell_2022} introduces finite-element(-in-space) implicit timesteppers for the incompressible analogue to this system with structure-preserving (SP) properties in the ideal case, alongside parameter-robust preconditioners. We show that these timesteppers can derive from a finite-element-in-time (FET) (and finite-element-in-space) interpretation. The benefits of this reformulation are discussed, including the derivation of timesteppers that are higher order in time, and the quantifiable dissipative SP properties in the non-ideal, resistive case.
        
        We discuss possible options for extending this FET approach to timesteppers for the compressible case.

        The kinetic corrections satisfy linearized Boltzmann equations. Using a Lénard--Bernstein collision operator, these take Fokker--Planck-like forms \cite{Fokker_1914, Planck_1917} wherein pseudo-particles in the numerical model obey the neoclassical transport equations, with particle-independent Brownian drift terms. This offers a rigorous methodology for incorporating collisions into the particle transport model, without coupling the equations of motions for each particle.
        
        Works by Chen, Chacón et al. \cite{Chen_Chacón_Barnes_2011, Chacón_Chen_Barnes_2013, Chen_Chacón_2014, Chen_Chacón_2015} have developed structure-preserving particle pushers for neoclassical transport in the Vlasov equations, derived from Crank--Nicolson integrators. We show these too can can derive from a FET interpretation, similarly offering potential extensions to higher-order-in-time particle pushers. The FET formulation is used also to consider how the stochastic drift terms can be incorporated into the pushers. Stochastic gyrokinetic expansions are also discussed.

        Different options for the numerical implementation of these schemes are considered.

        Due to the efficacy of FET in the development of SP timesteppers for both the fluid and kinetic component, we hope this approach will prove effective in the future for developing SP timesteppers for the full hybrid model. We hope this will give us the opportunity to incorporate previously inaccessible kinetic effects into the highly effective, modern, finite-element MHD models.
    \end{abstract}
    
    
    \newpage
    \tableofcontents
    
    
    \newpage
    \pagenumbering{arabic}
    %\linenumbers\renewcommand\thelinenumber{\color{black!50}\arabic{linenumber}}
            \documentclass[12pt, a4paper]{report}

\input{template/main.tex}

\title{\BA{Title in Progress...}}
\author{Boris Andrews}
\affil{Mathematical Institute, University of Oxford}
\date{\today}


\begin{document}
    \pagenumbering{gobble}
    \maketitle
    
    
    \begin{abstract}
        Magnetic confinement reactors---in particular tokamaks---offer one of the most promising options for achieving practical nuclear fusion, with the potential to provide virtually limitless, clean energy. The theoretical and numerical modeling of tokamak plasmas is simultaneously an essential component of effective reactor design, and a great research barrier. Tokamak operational conditions exhibit comparatively low Knudsen numbers. Kinetic effects, including kinetic waves and instabilities, Landau damping, bump-on-tail instabilities and more, are therefore highly influential in tokamak plasma dynamics. Purely fluid models are inherently incapable of capturing these effects, whereas the high dimensionality in purely kinetic models render them practically intractable for most relevant purposes.

        We consider a $\delta\!f$ decomposition model, with a macroscopic fluid background and microscopic kinetic correction, both fully coupled to each other. A similar manner of discretization is proposed to that used in the recent \texttt{STRUPHY} code \cite{Holderied_Possanner_Wang_2021, Holderied_2022, Li_et_al_2023} with a finite-element model for the background and a pseudo-particle/PiC model for the correction.

        The fluid background satisfies the full, non-linear, resistive, compressible, Hall MHD equations. \cite{Laakmann_Hu_Farrell_2022} introduces finite-element(-in-space) implicit timesteppers for the incompressible analogue to this system with structure-preserving (SP) properties in the ideal case, alongside parameter-robust preconditioners. We show that these timesteppers can derive from a finite-element-in-time (FET) (and finite-element-in-space) interpretation. The benefits of this reformulation are discussed, including the derivation of timesteppers that are higher order in time, and the quantifiable dissipative SP properties in the non-ideal, resistive case.
        
        We discuss possible options for extending this FET approach to timesteppers for the compressible case.

        The kinetic corrections satisfy linearized Boltzmann equations. Using a Lénard--Bernstein collision operator, these take Fokker--Planck-like forms \cite{Fokker_1914, Planck_1917} wherein pseudo-particles in the numerical model obey the neoclassical transport equations, with particle-independent Brownian drift terms. This offers a rigorous methodology for incorporating collisions into the particle transport model, without coupling the equations of motions for each particle.
        
        Works by Chen, Chacón et al. \cite{Chen_Chacón_Barnes_2011, Chacón_Chen_Barnes_2013, Chen_Chacón_2014, Chen_Chacón_2015} have developed structure-preserving particle pushers for neoclassical transport in the Vlasov equations, derived from Crank--Nicolson integrators. We show these too can can derive from a FET interpretation, similarly offering potential extensions to higher-order-in-time particle pushers. The FET formulation is used also to consider how the stochastic drift terms can be incorporated into the pushers. Stochastic gyrokinetic expansions are also discussed.

        Different options for the numerical implementation of these schemes are considered.

        Due to the efficacy of FET in the development of SP timesteppers for both the fluid and kinetic component, we hope this approach will prove effective in the future for developing SP timesteppers for the full hybrid model. We hope this will give us the opportunity to incorporate previously inaccessible kinetic effects into the highly effective, modern, finite-element MHD models.
    \end{abstract}
    
    
    \newpage
    \tableofcontents
    
    
    \newpage
    \pagenumbering{arabic}
    %\linenumbers\renewcommand\thelinenumber{\color{black!50}\arabic{linenumber}}
            \input{0 - introduction/main.tex}
        \part{Research}
            \input{1 - low-noise PiC models/main.tex}
            \input{2 - kinetic component/main.tex}
            \input{3 - fluid component/main.tex}
            \input{4 - numerical implementation/main.tex}
        \part{Project Overview}
            \input{5 - research plan/main.tex}
            \input{6 - summary/main.tex}
    
    
    %\section{}
    \newpage
    \pagenumbering{gobble}
        \printbibliography


    \newpage
    \pagenumbering{roman}
    \appendix
        \part{Appendices}
            \input{8 - Hilbert complexes/main.tex}
            \input{9 - weak conservation proofs/main.tex}
\end{document}

        \part{Research}
            \documentclass[12pt, a4paper]{report}

\input{template/main.tex}

\title{\BA{Title in Progress...}}
\author{Boris Andrews}
\affil{Mathematical Institute, University of Oxford}
\date{\today}


\begin{document}
    \pagenumbering{gobble}
    \maketitle
    
    
    \begin{abstract}
        Magnetic confinement reactors---in particular tokamaks---offer one of the most promising options for achieving practical nuclear fusion, with the potential to provide virtually limitless, clean energy. The theoretical and numerical modeling of tokamak plasmas is simultaneously an essential component of effective reactor design, and a great research barrier. Tokamak operational conditions exhibit comparatively low Knudsen numbers. Kinetic effects, including kinetic waves and instabilities, Landau damping, bump-on-tail instabilities and more, are therefore highly influential in tokamak plasma dynamics. Purely fluid models are inherently incapable of capturing these effects, whereas the high dimensionality in purely kinetic models render them practically intractable for most relevant purposes.

        We consider a $\delta\!f$ decomposition model, with a macroscopic fluid background and microscopic kinetic correction, both fully coupled to each other. A similar manner of discretization is proposed to that used in the recent \texttt{STRUPHY} code \cite{Holderied_Possanner_Wang_2021, Holderied_2022, Li_et_al_2023} with a finite-element model for the background and a pseudo-particle/PiC model for the correction.

        The fluid background satisfies the full, non-linear, resistive, compressible, Hall MHD equations. \cite{Laakmann_Hu_Farrell_2022} introduces finite-element(-in-space) implicit timesteppers for the incompressible analogue to this system with structure-preserving (SP) properties in the ideal case, alongside parameter-robust preconditioners. We show that these timesteppers can derive from a finite-element-in-time (FET) (and finite-element-in-space) interpretation. The benefits of this reformulation are discussed, including the derivation of timesteppers that are higher order in time, and the quantifiable dissipative SP properties in the non-ideal, resistive case.
        
        We discuss possible options for extending this FET approach to timesteppers for the compressible case.

        The kinetic corrections satisfy linearized Boltzmann equations. Using a Lénard--Bernstein collision operator, these take Fokker--Planck-like forms \cite{Fokker_1914, Planck_1917} wherein pseudo-particles in the numerical model obey the neoclassical transport equations, with particle-independent Brownian drift terms. This offers a rigorous methodology for incorporating collisions into the particle transport model, without coupling the equations of motions for each particle.
        
        Works by Chen, Chacón et al. \cite{Chen_Chacón_Barnes_2011, Chacón_Chen_Barnes_2013, Chen_Chacón_2014, Chen_Chacón_2015} have developed structure-preserving particle pushers for neoclassical transport in the Vlasov equations, derived from Crank--Nicolson integrators. We show these too can can derive from a FET interpretation, similarly offering potential extensions to higher-order-in-time particle pushers. The FET formulation is used also to consider how the stochastic drift terms can be incorporated into the pushers. Stochastic gyrokinetic expansions are also discussed.

        Different options for the numerical implementation of these schemes are considered.

        Due to the efficacy of FET in the development of SP timesteppers for both the fluid and kinetic component, we hope this approach will prove effective in the future for developing SP timesteppers for the full hybrid model. We hope this will give us the opportunity to incorporate previously inaccessible kinetic effects into the highly effective, modern, finite-element MHD models.
    \end{abstract}
    
    
    \newpage
    \tableofcontents
    
    
    \newpage
    \pagenumbering{arabic}
    %\linenumbers\renewcommand\thelinenumber{\color{black!50}\arabic{linenumber}}
            \input{0 - introduction/main.tex}
        \part{Research}
            \input{1 - low-noise PiC models/main.tex}
            \input{2 - kinetic component/main.tex}
            \input{3 - fluid component/main.tex}
            \input{4 - numerical implementation/main.tex}
        \part{Project Overview}
            \input{5 - research plan/main.tex}
            \input{6 - summary/main.tex}
    
    
    %\section{}
    \newpage
    \pagenumbering{gobble}
        \printbibliography


    \newpage
    \pagenumbering{roman}
    \appendix
        \part{Appendices}
            \input{8 - Hilbert complexes/main.tex}
            \input{9 - weak conservation proofs/main.tex}
\end{document}

            \documentclass[12pt, a4paper]{report}

\input{template/main.tex}

\title{\BA{Title in Progress...}}
\author{Boris Andrews}
\affil{Mathematical Institute, University of Oxford}
\date{\today}


\begin{document}
    \pagenumbering{gobble}
    \maketitle
    
    
    \begin{abstract}
        Magnetic confinement reactors---in particular tokamaks---offer one of the most promising options for achieving practical nuclear fusion, with the potential to provide virtually limitless, clean energy. The theoretical and numerical modeling of tokamak plasmas is simultaneously an essential component of effective reactor design, and a great research barrier. Tokamak operational conditions exhibit comparatively low Knudsen numbers. Kinetic effects, including kinetic waves and instabilities, Landau damping, bump-on-tail instabilities and more, are therefore highly influential in tokamak plasma dynamics. Purely fluid models are inherently incapable of capturing these effects, whereas the high dimensionality in purely kinetic models render them practically intractable for most relevant purposes.

        We consider a $\delta\!f$ decomposition model, with a macroscopic fluid background and microscopic kinetic correction, both fully coupled to each other. A similar manner of discretization is proposed to that used in the recent \texttt{STRUPHY} code \cite{Holderied_Possanner_Wang_2021, Holderied_2022, Li_et_al_2023} with a finite-element model for the background and a pseudo-particle/PiC model for the correction.

        The fluid background satisfies the full, non-linear, resistive, compressible, Hall MHD equations. \cite{Laakmann_Hu_Farrell_2022} introduces finite-element(-in-space) implicit timesteppers for the incompressible analogue to this system with structure-preserving (SP) properties in the ideal case, alongside parameter-robust preconditioners. We show that these timesteppers can derive from a finite-element-in-time (FET) (and finite-element-in-space) interpretation. The benefits of this reformulation are discussed, including the derivation of timesteppers that are higher order in time, and the quantifiable dissipative SP properties in the non-ideal, resistive case.
        
        We discuss possible options for extending this FET approach to timesteppers for the compressible case.

        The kinetic corrections satisfy linearized Boltzmann equations. Using a Lénard--Bernstein collision operator, these take Fokker--Planck-like forms \cite{Fokker_1914, Planck_1917} wherein pseudo-particles in the numerical model obey the neoclassical transport equations, with particle-independent Brownian drift terms. This offers a rigorous methodology for incorporating collisions into the particle transport model, without coupling the equations of motions for each particle.
        
        Works by Chen, Chacón et al. \cite{Chen_Chacón_Barnes_2011, Chacón_Chen_Barnes_2013, Chen_Chacón_2014, Chen_Chacón_2015} have developed structure-preserving particle pushers for neoclassical transport in the Vlasov equations, derived from Crank--Nicolson integrators. We show these too can can derive from a FET interpretation, similarly offering potential extensions to higher-order-in-time particle pushers. The FET formulation is used also to consider how the stochastic drift terms can be incorporated into the pushers. Stochastic gyrokinetic expansions are also discussed.

        Different options for the numerical implementation of these schemes are considered.

        Due to the efficacy of FET in the development of SP timesteppers for both the fluid and kinetic component, we hope this approach will prove effective in the future for developing SP timesteppers for the full hybrid model. We hope this will give us the opportunity to incorporate previously inaccessible kinetic effects into the highly effective, modern, finite-element MHD models.
    \end{abstract}
    
    
    \newpage
    \tableofcontents
    
    
    \newpage
    \pagenumbering{arabic}
    %\linenumbers\renewcommand\thelinenumber{\color{black!50}\arabic{linenumber}}
            \input{0 - introduction/main.tex}
        \part{Research}
            \input{1 - low-noise PiC models/main.tex}
            \input{2 - kinetic component/main.tex}
            \input{3 - fluid component/main.tex}
            \input{4 - numerical implementation/main.tex}
        \part{Project Overview}
            \input{5 - research plan/main.tex}
            \input{6 - summary/main.tex}
    
    
    %\section{}
    \newpage
    \pagenumbering{gobble}
        \printbibliography


    \newpage
    \pagenumbering{roman}
    \appendix
        \part{Appendices}
            \input{8 - Hilbert complexes/main.tex}
            \input{9 - weak conservation proofs/main.tex}
\end{document}

            \documentclass[12pt, a4paper]{report}

\input{template/main.tex}

\title{\BA{Title in Progress...}}
\author{Boris Andrews}
\affil{Mathematical Institute, University of Oxford}
\date{\today}


\begin{document}
    \pagenumbering{gobble}
    \maketitle
    
    
    \begin{abstract}
        Magnetic confinement reactors---in particular tokamaks---offer one of the most promising options for achieving practical nuclear fusion, with the potential to provide virtually limitless, clean energy. The theoretical and numerical modeling of tokamak plasmas is simultaneously an essential component of effective reactor design, and a great research barrier. Tokamak operational conditions exhibit comparatively low Knudsen numbers. Kinetic effects, including kinetic waves and instabilities, Landau damping, bump-on-tail instabilities and more, are therefore highly influential in tokamak plasma dynamics. Purely fluid models are inherently incapable of capturing these effects, whereas the high dimensionality in purely kinetic models render them practically intractable for most relevant purposes.

        We consider a $\delta\!f$ decomposition model, with a macroscopic fluid background and microscopic kinetic correction, both fully coupled to each other. A similar manner of discretization is proposed to that used in the recent \texttt{STRUPHY} code \cite{Holderied_Possanner_Wang_2021, Holderied_2022, Li_et_al_2023} with a finite-element model for the background and a pseudo-particle/PiC model for the correction.

        The fluid background satisfies the full, non-linear, resistive, compressible, Hall MHD equations. \cite{Laakmann_Hu_Farrell_2022} introduces finite-element(-in-space) implicit timesteppers for the incompressible analogue to this system with structure-preserving (SP) properties in the ideal case, alongside parameter-robust preconditioners. We show that these timesteppers can derive from a finite-element-in-time (FET) (and finite-element-in-space) interpretation. The benefits of this reformulation are discussed, including the derivation of timesteppers that are higher order in time, and the quantifiable dissipative SP properties in the non-ideal, resistive case.
        
        We discuss possible options for extending this FET approach to timesteppers for the compressible case.

        The kinetic corrections satisfy linearized Boltzmann equations. Using a Lénard--Bernstein collision operator, these take Fokker--Planck-like forms \cite{Fokker_1914, Planck_1917} wherein pseudo-particles in the numerical model obey the neoclassical transport equations, with particle-independent Brownian drift terms. This offers a rigorous methodology for incorporating collisions into the particle transport model, without coupling the equations of motions for each particle.
        
        Works by Chen, Chacón et al. \cite{Chen_Chacón_Barnes_2011, Chacón_Chen_Barnes_2013, Chen_Chacón_2014, Chen_Chacón_2015} have developed structure-preserving particle pushers for neoclassical transport in the Vlasov equations, derived from Crank--Nicolson integrators. We show these too can can derive from a FET interpretation, similarly offering potential extensions to higher-order-in-time particle pushers. The FET formulation is used also to consider how the stochastic drift terms can be incorporated into the pushers. Stochastic gyrokinetic expansions are also discussed.

        Different options for the numerical implementation of these schemes are considered.

        Due to the efficacy of FET in the development of SP timesteppers for both the fluid and kinetic component, we hope this approach will prove effective in the future for developing SP timesteppers for the full hybrid model. We hope this will give us the opportunity to incorporate previously inaccessible kinetic effects into the highly effective, modern, finite-element MHD models.
    \end{abstract}
    
    
    \newpage
    \tableofcontents
    
    
    \newpage
    \pagenumbering{arabic}
    %\linenumbers\renewcommand\thelinenumber{\color{black!50}\arabic{linenumber}}
            \input{0 - introduction/main.tex}
        \part{Research}
            \input{1 - low-noise PiC models/main.tex}
            \input{2 - kinetic component/main.tex}
            \input{3 - fluid component/main.tex}
            \input{4 - numerical implementation/main.tex}
        \part{Project Overview}
            \input{5 - research plan/main.tex}
            \input{6 - summary/main.tex}
    
    
    %\section{}
    \newpage
    \pagenumbering{gobble}
        \printbibliography


    \newpage
    \pagenumbering{roman}
    \appendix
        \part{Appendices}
            \input{8 - Hilbert complexes/main.tex}
            \input{9 - weak conservation proofs/main.tex}
\end{document}

            \documentclass[12pt, a4paper]{report}

\input{template/main.tex}

\title{\BA{Title in Progress...}}
\author{Boris Andrews}
\affil{Mathematical Institute, University of Oxford}
\date{\today}


\begin{document}
    \pagenumbering{gobble}
    \maketitle
    
    
    \begin{abstract}
        Magnetic confinement reactors---in particular tokamaks---offer one of the most promising options for achieving practical nuclear fusion, with the potential to provide virtually limitless, clean energy. The theoretical and numerical modeling of tokamak plasmas is simultaneously an essential component of effective reactor design, and a great research barrier. Tokamak operational conditions exhibit comparatively low Knudsen numbers. Kinetic effects, including kinetic waves and instabilities, Landau damping, bump-on-tail instabilities and more, are therefore highly influential in tokamak plasma dynamics. Purely fluid models are inherently incapable of capturing these effects, whereas the high dimensionality in purely kinetic models render them practically intractable for most relevant purposes.

        We consider a $\delta\!f$ decomposition model, with a macroscopic fluid background and microscopic kinetic correction, both fully coupled to each other. A similar manner of discretization is proposed to that used in the recent \texttt{STRUPHY} code \cite{Holderied_Possanner_Wang_2021, Holderied_2022, Li_et_al_2023} with a finite-element model for the background and a pseudo-particle/PiC model for the correction.

        The fluid background satisfies the full, non-linear, resistive, compressible, Hall MHD equations. \cite{Laakmann_Hu_Farrell_2022} introduces finite-element(-in-space) implicit timesteppers for the incompressible analogue to this system with structure-preserving (SP) properties in the ideal case, alongside parameter-robust preconditioners. We show that these timesteppers can derive from a finite-element-in-time (FET) (and finite-element-in-space) interpretation. The benefits of this reformulation are discussed, including the derivation of timesteppers that are higher order in time, and the quantifiable dissipative SP properties in the non-ideal, resistive case.
        
        We discuss possible options for extending this FET approach to timesteppers for the compressible case.

        The kinetic corrections satisfy linearized Boltzmann equations. Using a Lénard--Bernstein collision operator, these take Fokker--Planck-like forms \cite{Fokker_1914, Planck_1917} wherein pseudo-particles in the numerical model obey the neoclassical transport equations, with particle-independent Brownian drift terms. This offers a rigorous methodology for incorporating collisions into the particle transport model, without coupling the equations of motions for each particle.
        
        Works by Chen, Chacón et al. \cite{Chen_Chacón_Barnes_2011, Chacón_Chen_Barnes_2013, Chen_Chacón_2014, Chen_Chacón_2015} have developed structure-preserving particle pushers for neoclassical transport in the Vlasov equations, derived from Crank--Nicolson integrators. We show these too can can derive from a FET interpretation, similarly offering potential extensions to higher-order-in-time particle pushers. The FET formulation is used also to consider how the stochastic drift terms can be incorporated into the pushers. Stochastic gyrokinetic expansions are also discussed.

        Different options for the numerical implementation of these schemes are considered.

        Due to the efficacy of FET in the development of SP timesteppers for both the fluid and kinetic component, we hope this approach will prove effective in the future for developing SP timesteppers for the full hybrid model. We hope this will give us the opportunity to incorporate previously inaccessible kinetic effects into the highly effective, modern, finite-element MHD models.
    \end{abstract}
    
    
    \newpage
    \tableofcontents
    
    
    \newpage
    \pagenumbering{arabic}
    %\linenumbers\renewcommand\thelinenumber{\color{black!50}\arabic{linenumber}}
            \input{0 - introduction/main.tex}
        \part{Research}
            \input{1 - low-noise PiC models/main.tex}
            \input{2 - kinetic component/main.tex}
            \input{3 - fluid component/main.tex}
            \input{4 - numerical implementation/main.tex}
        \part{Project Overview}
            \input{5 - research plan/main.tex}
            \input{6 - summary/main.tex}
    
    
    %\section{}
    \newpage
    \pagenumbering{gobble}
        \printbibliography


    \newpage
    \pagenumbering{roman}
    \appendix
        \part{Appendices}
            \input{8 - Hilbert complexes/main.tex}
            \input{9 - weak conservation proofs/main.tex}
\end{document}

        \part{Project Overview}
            \documentclass[12pt, a4paper]{report}

\input{template/main.tex}

\title{\BA{Title in Progress...}}
\author{Boris Andrews}
\affil{Mathematical Institute, University of Oxford}
\date{\today}


\begin{document}
    \pagenumbering{gobble}
    \maketitle
    
    
    \begin{abstract}
        Magnetic confinement reactors---in particular tokamaks---offer one of the most promising options for achieving practical nuclear fusion, with the potential to provide virtually limitless, clean energy. The theoretical and numerical modeling of tokamak plasmas is simultaneously an essential component of effective reactor design, and a great research barrier. Tokamak operational conditions exhibit comparatively low Knudsen numbers. Kinetic effects, including kinetic waves and instabilities, Landau damping, bump-on-tail instabilities and more, are therefore highly influential in tokamak plasma dynamics. Purely fluid models are inherently incapable of capturing these effects, whereas the high dimensionality in purely kinetic models render them practically intractable for most relevant purposes.

        We consider a $\delta\!f$ decomposition model, with a macroscopic fluid background and microscopic kinetic correction, both fully coupled to each other. A similar manner of discretization is proposed to that used in the recent \texttt{STRUPHY} code \cite{Holderied_Possanner_Wang_2021, Holderied_2022, Li_et_al_2023} with a finite-element model for the background and a pseudo-particle/PiC model for the correction.

        The fluid background satisfies the full, non-linear, resistive, compressible, Hall MHD equations. \cite{Laakmann_Hu_Farrell_2022} introduces finite-element(-in-space) implicit timesteppers for the incompressible analogue to this system with structure-preserving (SP) properties in the ideal case, alongside parameter-robust preconditioners. We show that these timesteppers can derive from a finite-element-in-time (FET) (and finite-element-in-space) interpretation. The benefits of this reformulation are discussed, including the derivation of timesteppers that are higher order in time, and the quantifiable dissipative SP properties in the non-ideal, resistive case.
        
        We discuss possible options for extending this FET approach to timesteppers for the compressible case.

        The kinetic corrections satisfy linearized Boltzmann equations. Using a Lénard--Bernstein collision operator, these take Fokker--Planck-like forms \cite{Fokker_1914, Planck_1917} wherein pseudo-particles in the numerical model obey the neoclassical transport equations, with particle-independent Brownian drift terms. This offers a rigorous methodology for incorporating collisions into the particle transport model, without coupling the equations of motions for each particle.
        
        Works by Chen, Chacón et al. \cite{Chen_Chacón_Barnes_2011, Chacón_Chen_Barnes_2013, Chen_Chacón_2014, Chen_Chacón_2015} have developed structure-preserving particle pushers for neoclassical transport in the Vlasov equations, derived from Crank--Nicolson integrators. We show these too can can derive from a FET interpretation, similarly offering potential extensions to higher-order-in-time particle pushers. The FET formulation is used also to consider how the stochastic drift terms can be incorporated into the pushers. Stochastic gyrokinetic expansions are also discussed.

        Different options for the numerical implementation of these schemes are considered.

        Due to the efficacy of FET in the development of SP timesteppers for both the fluid and kinetic component, we hope this approach will prove effective in the future for developing SP timesteppers for the full hybrid model. We hope this will give us the opportunity to incorporate previously inaccessible kinetic effects into the highly effective, modern, finite-element MHD models.
    \end{abstract}
    
    
    \newpage
    \tableofcontents
    
    
    \newpage
    \pagenumbering{arabic}
    %\linenumbers\renewcommand\thelinenumber{\color{black!50}\arabic{linenumber}}
            \input{0 - introduction/main.tex}
        \part{Research}
            \input{1 - low-noise PiC models/main.tex}
            \input{2 - kinetic component/main.tex}
            \input{3 - fluid component/main.tex}
            \input{4 - numerical implementation/main.tex}
        \part{Project Overview}
            \input{5 - research plan/main.tex}
            \input{6 - summary/main.tex}
    
    
    %\section{}
    \newpage
    \pagenumbering{gobble}
        \printbibliography


    \newpage
    \pagenumbering{roman}
    \appendix
        \part{Appendices}
            \input{8 - Hilbert complexes/main.tex}
            \input{9 - weak conservation proofs/main.tex}
\end{document}

            \documentclass[12pt, a4paper]{report}

\input{template/main.tex}

\title{\BA{Title in Progress...}}
\author{Boris Andrews}
\affil{Mathematical Institute, University of Oxford}
\date{\today}


\begin{document}
    \pagenumbering{gobble}
    \maketitle
    
    
    \begin{abstract}
        Magnetic confinement reactors---in particular tokamaks---offer one of the most promising options for achieving practical nuclear fusion, with the potential to provide virtually limitless, clean energy. The theoretical and numerical modeling of tokamak plasmas is simultaneously an essential component of effective reactor design, and a great research barrier. Tokamak operational conditions exhibit comparatively low Knudsen numbers. Kinetic effects, including kinetic waves and instabilities, Landau damping, bump-on-tail instabilities and more, are therefore highly influential in tokamak plasma dynamics. Purely fluid models are inherently incapable of capturing these effects, whereas the high dimensionality in purely kinetic models render them practically intractable for most relevant purposes.

        We consider a $\delta\!f$ decomposition model, with a macroscopic fluid background and microscopic kinetic correction, both fully coupled to each other. A similar manner of discretization is proposed to that used in the recent \texttt{STRUPHY} code \cite{Holderied_Possanner_Wang_2021, Holderied_2022, Li_et_al_2023} with a finite-element model for the background and a pseudo-particle/PiC model for the correction.

        The fluid background satisfies the full, non-linear, resistive, compressible, Hall MHD equations. \cite{Laakmann_Hu_Farrell_2022} introduces finite-element(-in-space) implicit timesteppers for the incompressible analogue to this system with structure-preserving (SP) properties in the ideal case, alongside parameter-robust preconditioners. We show that these timesteppers can derive from a finite-element-in-time (FET) (and finite-element-in-space) interpretation. The benefits of this reformulation are discussed, including the derivation of timesteppers that are higher order in time, and the quantifiable dissipative SP properties in the non-ideal, resistive case.
        
        We discuss possible options for extending this FET approach to timesteppers for the compressible case.

        The kinetic corrections satisfy linearized Boltzmann equations. Using a Lénard--Bernstein collision operator, these take Fokker--Planck-like forms \cite{Fokker_1914, Planck_1917} wherein pseudo-particles in the numerical model obey the neoclassical transport equations, with particle-independent Brownian drift terms. This offers a rigorous methodology for incorporating collisions into the particle transport model, without coupling the equations of motions for each particle.
        
        Works by Chen, Chacón et al. \cite{Chen_Chacón_Barnes_2011, Chacón_Chen_Barnes_2013, Chen_Chacón_2014, Chen_Chacón_2015} have developed structure-preserving particle pushers for neoclassical transport in the Vlasov equations, derived from Crank--Nicolson integrators. We show these too can can derive from a FET interpretation, similarly offering potential extensions to higher-order-in-time particle pushers. The FET formulation is used also to consider how the stochastic drift terms can be incorporated into the pushers. Stochastic gyrokinetic expansions are also discussed.

        Different options for the numerical implementation of these schemes are considered.

        Due to the efficacy of FET in the development of SP timesteppers for both the fluid and kinetic component, we hope this approach will prove effective in the future for developing SP timesteppers for the full hybrid model. We hope this will give us the opportunity to incorporate previously inaccessible kinetic effects into the highly effective, modern, finite-element MHD models.
    \end{abstract}
    
    
    \newpage
    \tableofcontents
    
    
    \newpage
    \pagenumbering{arabic}
    %\linenumbers\renewcommand\thelinenumber{\color{black!50}\arabic{linenumber}}
            \input{0 - introduction/main.tex}
        \part{Research}
            \input{1 - low-noise PiC models/main.tex}
            \input{2 - kinetic component/main.tex}
            \input{3 - fluid component/main.tex}
            \input{4 - numerical implementation/main.tex}
        \part{Project Overview}
            \input{5 - research plan/main.tex}
            \input{6 - summary/main.tex}
    
    
    %\section{}
    \newpage
    \pagenumbering{gobble}
        \printbibliography


    \newpage
    \pagenumbering{roman}
    \appendix
        \part{Appendices}
            \input{8 - Hilbert complexes/main.tex}
            \input{9 - weak conservation proofs/main.tex}
\end{document}

    
    
    %\section{}
    \newpage
    \pagenumbering{gobble}
        \printbibliography


    \newpage
    \pagenumbering{roman}
    \appendix
        \part{Appendices}
            \documentclass[12pt, a4paper]{report}

\input{template/main.tex}

\title{\BA{Title in Progress...}}
\author{Boris Andrews}
\affil{Mathematical Institute, University of Oxford}
\date{\today}


\begin{document}
    \pagenumbering{gobble}
    \maketitle
    
    
    \begin{abstract}
        Magnetic confinement reactors---in particular tokamaks---offer one of the most promising options for achieving practical nuclear fusion, with the potential to provide virtually limitless, clean energy. The theoretical and numerical modeling of tokamak plasmas is simultaneously an essential component of effective reactor design, and a great research barrier. Tokamak operational conditions exhibit comparatively low Knudsen numbers. Kinetic effects, including kinetic waves and instabilities, Landau damping, bump-on-tail instabilities and more, are therefore highly influential in tokamak plasma dynamics. Purely fluid models are inherently incapable of capturing these effects, whereas the high dimensionality in purely kinetic models render them practically intractable for most relevant purposes.

        We consider a $\delta\!f$ decomposition model, with a macroscopic fluid background and microscopic kinetic correction, both fully coupled to each other. A similar manner of discretization is proposed to that used in the recent \texttt{STRUPHY} code \cite{Holderied_Possanner_Wang_2021, Holderied_2022, Li_et_al_2023} with a finite-element model for the background and a pseudo-particle/PiC model for the correction.

        The fluid background satisfies the full, non-linear, resistive, compressible, Hall MHD equations. \cite{Laakmann_Hu_Farrell_2022} introduces finite-element(-in-space) implicit timesteppers for the incompressible analogue to this system with structure-preserving (SP) properties in the ideal case, alongside parameter-robust preconditioners. We show that these timesteppers can derive from a finite-element-in-time (FET) (and finite-element-in-space) interpretation. The benefits of this reformulation are discussed, including the derivation of timesteppers that are higher order in time, and the quantifiable dissipative SP properties in the non-ideal, resistive case.
        
        We discuss possible options for extending this FET approach to timesteppers for the compressible case.

        The kinetic corrections satisfy linearized Boltzmann equations. Using a Lénard--Bernstein collision operator, these take Fokker--Planck-like forms \cite{Fokker_1914, Planck_1917} wherein pseudo-particles in the numerical model obey the neoclassical transport equations, with particle-independent Brownian drift terms. This offers a rigorous methodology for incorporating collisions into the particle transport model, without coupling the equations of motions for each particle.
        
        Works by Chen, Chacón et al. \cite{Chen_Chacón_Barnes_2011, Chacón_Chen_Barnes_2013, Chen_Chacón_2014, Chen_Chacón_2015} have developed structure-preserving particle pushers for neoclassical transport in the Vlasov equations, derived from Crank--Nicolson integrators. We show these too can can derive from a FET interpretation, similarly offering potential extensions to higher-order-in-time particle pushers. The FET formulation is used also to consider how the stochastic drift terms can be incorporated into the pushers. Stochastic gyrokinetic expansions are also discussed.

        Different options for the numerical implementation of these schemes are considered.

        Due to the efficacy of FET in the development of SP timesteppers for both the fluid and kinetic component, we hope this approach will prove effective in the future for developing SP timesteppers for the full hybrid model. We hope this will give us the opportunity to incorporate previously inaccessible kinetic effects into the highly effective, modern, finite-element MHD models.
    \end{abstract}
    
    
    \newpage
    \tableofcontents
    
    
    \newpage
    \pagenumbering{arabic}
    %\linenumbers\renewcommand\thelinenumber{\color{black!50}\arabic{linenumber}}
            \input{0 - introduction/main.tex}
        \part{Research}
            \input{1 - low-noise PiC models/main.tex}
            \input{2 - kinetic component/main.tex}
            \input{3 - fluid component/main.tex}
            \input{4 - numerical implementation/main.tex}
        \part{Project Overview}
            \input{5 - research plan/main.tex}
            \input{6 - summary/main.tex}
    
    
    %\section{}
    \newpage
    \pagenumbering{gobble}
        \printbibliography


    \newpage
    \pagenumbering{roman}
    \appendix
        \part{Appendices}
            \input{8 - Hilbert complexes/main.tex}
            \input{9 - weak conservation proofs/main.tex}
\end{document}

            \documentclass[12pt, a4paper]{report}

\input{template/main.tex}

\title{\BA{Title in Progress...}}
\author{Boris Andrews}
\affil{Mathematical Institute, University of Oxford}
\date{\today}


\begin{document}
    \pagenumbering{gobble}
    \maketitle
    
    
    \begin{abstract}
        Magnetic confinement reactors---in particular tokamaks---offer one of the most promising options for achieving practical nuclear fusion, with the potential to provide virtually limitless, clean energy. The theoretical and numerical modeling of tokamak plasmas is simultaneously an essential component of effective reactor design, and a great research barrier. Tokamak operational conditions exhibit comparatively low Knudsen numbers. Kinetic effects, including kinetic waves and instabilities, Landau damping, bump-on-tail instabilities and more, are therefore highly influential in tokamak plasma dynamics. Purely fluid models are inherently incapable of capturing these effects, whereas the high dimensionality in purely kinetic models render them practically intractable for most relevant purposes.

        We consider a $\delta\!f$ decomposition model, with a macroscopic fluid background and microscopic kinetic correction, both fully coupled to each other. A similar manner of discretization is proposed to that used in the recent \texttt{STRUPHY} code \cite{Holderied_Possanner_Wang_2021, Holderied_2022, Li_et_al_2023} with a finite-element model for the background and a pseudo-particle/PiC model for the correction.

        The fluid background satisfies the full, non-linear, resistive, compressible, Hall MHD equations. \cite{Laakmann_Hu_Farrell_2022} introduces finite-element(-in-space) implicit timesteppers for the incompressible analogue to this system with structure-preserving (SP) properties in the ideal case, alongside parameter-robust preconditioners. We show that these timesteppers can derive from a finite-element-in-time (FET) (and finite-element-in-space) interpretation. The benefits of this reformulation are discussed, including the derivation of timesteppers that are higher order in time, and the quantifiable dissipative SP properties in the non-ideal, resistive case.
        
        We discuss possible options for extending this FET approach to timesteppers for the compressible case.

        The kinetic corrections satisfy linearized Boltzmann equations. Using a Lénard--Bernstein collision operator, these take Fokker--Planck-like forms \cite{Fokker_1914, Planck_1917} wherein pseudo-particles in the numerical model obey the neoclassical transport equations, with particle-independent Brownian drift terms. This offers a rigorous methodology for incorporating collisions into the particle transport model, without coupling the equations of motions for each particle.
        
        Works by Chen, Chacón et al. \cite{Chen_Chacón_Barnes_2011, Chacón_Chen_Barnes_2013, Chen_Chacón_2014, Chen_Chacón_2015} have developed structure-preserving particle pushers for neoclassical transport in the Vlasov equations, derived from Crank--Nicolson integrators. We show these too can can derive from a FET interpretation, similarly offering potential extensions to higher-order-in-time particle pushers. The FET formulation is used also to consider how the stochastic drift terms can be incorporated into the pushers. Stochastic gyrokinetic expansions are also discussed.

        Different options for the numerical implementation of these schemes are considered.

        Due to the efficacy of FET in the development of SP timesteppers for both the fluid and kinetic component, we hope this approach will prove effective in the future for developing SP timesteppers for the full hybrid model. We hope this will give us the opportunity to incorporate previously inaccessible kinetic effects into the highly effective, modern, finite-element MHD models.
    \end{abstract}
    
    
    \newpage
    \tableofcontents
    
    
    \newpage
    \pagenumbering{arabic}
    %\linenumbers\renewcommand\thelinenumber{\color{black!50}\arabic{linenumber}}
            \input{0 - introduction/main.tex}
        \part{Research}
            \input{1 - low-noise PiC models/main.tex}
            \input{2 - kinetic component/main.tex}
            \input{3 - fluid component/main.tex}
            \input{4 - numerical implementation/main.tex}
        \part{Project Overview}
            \input{5 - research plan/main.tex}
            \input{6 - summary/main.tex}
    
    
    %\section{}
    \newpage
    \pagenumbering{gobble}
        \printbibliography


    \newpage
    \pagenumbering{roman}
    \appendix
        \part{Appendices}
            \input{8 - Hilbert complexes/main.tex}
            \input{9 - weak conservation proofs/main.tex}
\end{document}

\end{document}

        \part{Research}
            \documentclass[12pt, a4paper]{report}

\documentclass[12pt, a4paper]{report}

\input{template/main.tex}

\title{\BA{Title in Progress...}}
\author{Boris Andrews}
\affil{Mathematical Institute, University of Oxford}
\date{\today}


\begin{document}
    \pagenumbering{gobble}
    \maketitle
    
    
    \begin{abstract}
        Magnetic confinement reactors---in particular tokamaks---offer one of the most promising options for achieving practical nuclear fusion, with the potential to provide virtually limitless, clean energy. The theoretical and numerical modeling of tokamak plasmas is simultaneously an essential component of effective reactor design, and a great research barrier. Tokamak operational conditions exhibit comparatively low Knudsen numbers. Kinetic effects, including kinetic waves and instabilities, Landau damping, bump-on-tail instabilities and more, are therefore highly influential in tokamak plasma dynamics. Purely fluid models are inherently incapable of capturing these effects, whereas the high dimensionality in purely kinetic models render them practically intractable for most relevant purposes.

        We consider a $\delta\!f$ decomposition model, with a macroscopic fluid background and microscopic kinetic correction, both fully coupled to each other. A similar manner of discretization is proposed to that used in the recent \texttt{STRUPHY} code \cite{Holderied_Possanner_Wang_2021, Holderied_2022, Li_et_al_2023} with a finite-element model for the background and a pseudo-particle/PiC model for the correction.

        The fluid background satisfies the full, non-linear, resistive, compressible, Hall MHD equations. \cite{Laakmann_Hu_Farrell_2022} introduces finite-element(-in-space) implicit timesteppers for the incompressible analogue to this system with structure-preserving (SP) properties in the ideal case, alongside parameter-robust preconditioners. We show that these timesteppers can derive from a finite-element-in-time (FET) (and finite-element-in-space) interpretation. The benefits of this reformulation are discussed, including the derivation of timesteppers that are higher order in time, and the quantifiable dissipative SP properties in the non-ideal, resistive case.
        
        We discuss possible options for extending this FET approach to timesteppers for the compressible case.

        The kinetic corrections satisfy linearized Boltzmann equations. Using a Lénard--Bernstein collision operator, these take Fokker--Planck-like forms \cite{Fokker_1914, Planck_1917} wherein pseudo-particles in the numerical model obey the neoclassical transport equations, with particle-independent Brownian drift terms. This offers a rigorous methodology for incorporating collisions into the particle transport model, without coupling the equations of motions for each particle.
        
        Works by Chen, Chacón et al. \cite{Chen_Chacón_Barnes_2011, Chacón_Chen_Barnes_2013, Chen_Chacón_2014, Chen_Chacón_2015} have developed structure-preserving particle pushers for neoclassical transport in the Vlasov equations, derived from Crank--Nicolson integrators. We show these too can can derive from a FET interpretation, similarly offering potential extensions to higher-order-in-time particle pushers. The FET formulation is used also to consider how the stochastic drift terms can be incorporated into the pushers. Stochastic gyrokinetic expansions are also discussed.

        Different options for the numerical implementation of these schemes are considered.

        Due to the efficacy of FET in the development of SP timesteppers for both the fluid and kinetic component, we hope this approach will prove effective in the future for developing SP timesteppers for the full hybrid model. We hope this will give us the opportunity to incorporate previously inaccessible kinetic effects into the highly effective, modern, finite-element MHD models.
    \end{abstract}
    
    
    \newpage
    \tableofcontents
    
    
    \newpage
    \pagenumbering{arabic}
    %\linenumbers\renewcommand\thelinenumber{\color{black!50}\arabic{linenumber}}
            \input{0 - introduction/main.tex}
        \part{Research}
            \input{1 - low-noise PiC models/main.tex}
            \input{2 - kinetic component/main.tex}
            \input{3 - fluid component/main.tex}
            \input{4 - numerical implementation/main.tex}
        \part{Project Overview}
            \input{5 - research plan/main.tex}
            \input{6 - summary/main.tex}
    
    
    %\section{}
    \newpage
    \pagenumbering{gobble}
        \printbibliography


    \newpage
    \pagenumbering{roman}
    \appendix
        \part{Appendices}
            \input{8 - Hilbert complexes/main.tex}
            \input{9 - weak conservation proofs/main.tex}
\end{document}


\title{\BA{Title in Progress...}}
\author{Boris Andrews}
\affil{Mathematical Institute, University of Oxford}
\date{\today}


\begin{document}
    \pagenumbering{gobble}
    \maketitle
    
    
    \begin{abstract}
        Magnetic confinement reactors---in particular tokamaks---offer one of the most promising options for achieving practical nuclear fusion, with the potential to provide virtually limitless, clean energy. The theoretical and numerical modeling of tokamak plasmas is simultaneously an essential component of effective reactor design, and a great research barrier. Tokamak operational conditions exhibit comparatively low Knudsen numbers. Kinetic effects, including kinetic waves and instabilities, Landau damping, bump-on-tail instabilities and more, are therefore highly influential in tokamak plasma dynamics. Purely fluid models are inherently incapable of capturing these effects, whereas the high dimensionality in purely kinetic models render them practically intractable for most relevant purposes.

        We consider a $\delta\!f$ decomposition model, with a macroscopic fluid background and microscopic kinetic correction, both fully coupled to each other. A similar manner of discretization is proposed to that used in the recent \texttt{STRUPHY} code \cite{Holderied_Possanner_Wang_2021, Holderied_2022, Li_et_al_2023} with a finite-element model for the background and a pseudo-particle/PiC model for the correction.

        The fluid background satisfies the full, non-linear, resistive, compressible, Hall MHD equations. \cite{Laakmann_Hu_Farrell_2022} introduces finite-element(-in-space) implicit timesteppers for the incompressible analogue to this system with structure-preserving (SP) properties in the ideal case, alongside parameter-robust preconditioners. We show that these timesteppers can derive from a finite-element-in-time (FET) (and finite-element-in-space) interpretation. The benefits of this reformulation are discussed, including the derivation of timesteppers that are higher order in time, and the quantifiable dissipative SP properties in the non-ideal, resistive case.
        
        We discuss possible options for extending this FET approach to timesteppers for the compressible case.

        The kinetic corrections satisfy linearized Boltzmann equations. Using a Lénard--Bernstein collision operator, these take Fokker--Planck-like forms \cite{Fokker_1914, Planck_1917} wherein pseudo-particles in the numerical model obey the neoclassical transport equations, with particle-independent Brownian drift terms. This offers a rigorous methodology for incorporating collisions into the particle transport model, without coupling the equations of motions for each particle.
        
        Works by Chen, Chacón et al. \cite{Chen_Chacón_Barnes_2011, Chacón_Chen_Barnes_2013, Chen_Chacón_2014, Chen_Chacón_2015} have developed structure-preserving particle pushers for neoclassical transport in the Vlasov equations, derived from Crank--Nicolson integrators. We show these too can can derive from a FET interpretation, similarly offering potential extensions to higher-order-in-time particle pushers. The FET formulation is used also to consider how the stochastic drift terms can be incorporated into the pushers. Stochastic gyrokinetic expansions are also discussed.

        Different options for the numerical implementation of these schemes are considered.

        Due to the efficacy of FET in the development of SP timesteppers for both the fluid and kinetic component, we hope this approach will prove effective in the future for developing SP timesteppers for the full hybrid model. We hope this will give us the opportunity to incorporate previously inaccessible kinetic effects into the highly effective, modern, finite-element MHD models.
    \end{abstract}
    
    
    \newpage
    \tableofcontents
    
    
    \newpage
    \pagenumbering{arabic}
    %\linenumbers\renewcommand\thelinenumber{\color{black!50}\arabic{linenumber}}
            \documentclass[12pt, a4paper]{report}

\input{template/main.tex}

\title{\BA{Title in Progress...}}
\author{Boris Andrews}
\affil{Mathematical Institute, University of Oxford}
\date{\today}


\begin{document}
    \pagenumbering{gobble}
    \maketitle
    
    
    \begin{abstract}
        Magnetic confinement reactors---in particular tokamaks---offer one of the most promising options for achieving practical nuclear fusion, with the potential to provide virtually limitless, clean energy. The theoretical and numerical modeling of tokamak plasmas is simultaneously an essential component of effective reactor design, and a great research barrier. Tokamak operational conditions exhibit comparatively low Knudsen numbers. Kinetic effects, including kinetic waves and instabilities, Landau damping, bump-on-tail instabilities and more, are therefore highly influential in tokamak plasma dynamics. Purely fluid models are inherently incapable of capturing these effects, whereas the high dimensionality in purely kinetic models render them practically intractable for most relevant purposes.

        We consider a $\delta\!f$ decomposition model, with a macroscopic fluid background and microscopic kinetic correction, both fully coupled to each other. A similar manner of discretization is proposed to that used in the recent \texttt{STRUPHY} code \cite{Holderied_Possanner_Wang_2021, Holderied_2022, Li_et_al_2023} with a finite-element model for the background and a pseudo-particle/PiC model for the correction.

        The fluid background satisfies the full, non-linear, resistive, compressible, Hall MHD equations. \cite{Laakmann_Hu_Farrell_2022} introduces finite-element(-in-space) implicit timesteppers for the incompressible analogue to this system with structure-preserving (SP) properties in the ideal case, alongside parameter-robust preconditioners. We show that these timesteppers can derive from a finite-element-in-time (FET) (and finite-element-in-space) interpretation. The benefits of this reformulation are discussed, including the derivation of timesteppers that are higher order in time, and the quantifiable dissipative SP properties in the non-ideal, resistive case.
        
        We discuss possible options for extending this FET approach to timesteppers for the compressible case.

        The kinetic corrections satisfy linearized Boltzmann equations. Using a Lénard--Bernstein collision operator, these take Fokker--Planck-like forms \cite{Fokker_1914, Planck_1917} wherein pseudo-particles in the numerical model obey the neoclassical transport equations, with particle-independent Brownian drift terms. This offers a rigorous methodology for incorporating collisions into the particle transport model, without coupling the equations of motions for each particle.
        
        Works by Chen, Chacón et al. \cite{Chen_Chacón_Barnes_2011, Chacón_Chen_Barnes_2013, Chen_Chacón_2014, Chen_Chacón_2015} have developed structure-preserving particle pushers for neoclassical transport in the Vlasov equations, derived from Crank--Nicolson integrators. We show these too can can derive from a FET interpretation, similarly offering potential extensions to higher-order-in-time particle pushers. The FET formulation is used also to consider how the stochastic drift terms can be incorporated into the pushers. Stochastic gyrokinetic expansions are also discussed.

        Different options for the numerical implementation of these schemes are considered.

        Due to the efficacy of FET in the development of SP timesteppers for both the fluid and kinetic component, we hope this approach will prove effective in the future for developing SP timesteppers for the full hybrid model. We hope this will give us the opportunity to incorporate previously inaccessible kinetic effects into the highly effective, modern, finite-element MHD models.
    \end{abstract}
    
    
    \newpage
    \tableofcontents
    
    
    \newpage
    \pagenumbering{arabic}
    %\linenumbers\renewcommand\thelinenumber{\color{black!50}\arabic{linenumber}}
            \input{0 - introduction/main.tex}
        \part{Research}
            \input{1 - low-noise PiC models/main.tex}
            \input{2 - kinetic component/main.tex}
            \input{3 - fluid component/main.tex}
            \input{4 - numerical implementation/main.tex}
        \part{Project Overview}
            \input{5 - research plan/main.tex}
            \input{6 - summary/main.tex}
    
    
    %\section{}
    \newpage
    \pagenumbering{gobble}
        \printbibliography


    \newpage
    \pagenumbering{roman}
    \appendix
        \part{Appendices}
            \input{8 - Hilbert complexes/main.tex}
            \input{9 - weak conservation proofs/main.tex}
\end{document}

        \part{Research}
            \documentclass[12pt, a4paper]{report}

\input{template/main.tex}

\title{\BA{Title in Progress...}}
\author{Boris Andrews}
\affil{Mathematical Institute, University of Oxford}
\date{\today}


\begin{document}
    \pagenumbering{gobble}
    \maketitle
    
    
    \begin{abstract}
        Magnetic confinement reactors---in particular tokamaks---offer one of the most promising options for achieving practical nuclear fusion, with the potential to provide virtually limitless, clean energy. The theoretical and numerical modeling of tokamak plasmas is simultaneously an essential component of effective reactor design, and a great research barrier. Tokamak operational conditions exhibit comparatively low Knudsen numbers. Kinetic effects, including kinetic waves and instabilities, Landau damping, bump-on-tail instabilities and more, are therefore highly influential in tokamak plasma dynamics. Purely fluid models are inherently incapable of capturing these effects, whereas the high dimensionality in purely kinetic models render them practically intractable for most relevant purposes.

        We consider a $\delta\!f$ decomposition model, with a macroscopic fluid background and microscopic kinetic correction, both fully coupled to each other. A similar manner of discretization is proposed to that used in the recent \texttt{STRUPHY} code \cite{Holderied_Possanner_Wang_2021, Holderied_2022, Li_et_al_2023} with a finite-element model for the background and a pseudo-particle/PiC model for the correction.

        The fluid background satisfies the full, non-linear, resistive, compressible, Hall MHD equations. \cite{Laakmann_Hu_Farrell_2022} introduces finite-element(-in-space) implicit timesteppers for the incompressible analogue to this system with structure-preserving (SP) properties in the ideal case, alongside parameter-robust preconditioners. We show that these timesteppers can derive from a finite-element-in-time (FET) (and finite-element-in-space) interpretation. The benefits of this reformulation are discussed, including the derivation of timesteppers that are higher order in time, and the quantifiable dissipative SP properties in the non-ideal, resistive case.
        
        We discuss possible options for extending this FET approach to timesteppers for the compressible case.

        The kinetic corrections satisfy linearized Boltzmann equations. Using a Lénard--Bernstein collision operator, these take Fokker--Planck-like forms \cite{Fokker_1914, Planck_1917} wherein pseudo-particles in the numerical model obey the neoclassical transport equations, with particle-independent Brownian drift terms. This offers a rigorous methodology for incorporating collisions into the particle transport model, without coupling the equations of motions for each particle.
        
        Works by Chen, Chacón et al. \cite{Chen_Chacón_Barnes_2011, Chacón_Chen_Barnes_2013, Chen_Chacón_2014, Chen_Chacón_2015} have developed structure-preserving particle pushers for neoclassical transport in the Vlasov equations, derived from Crank--Nicolson integrators. We show these too can can derive from a FET interpretation, similarly offering potential extensions to higher-order-in-time particle pushers. The FET formulation is used also to consider how the stochastic drift terms can be incorporated into the pushers. Stochastic gyrokinetic expansions are also discussed.

        Different options for the numerical implementation of these schemes are considered.

        Due to the efficacy of FET in the development of SP timesteppers for both the fluid and kinetic component, we hope this approach will prove effective in the future for developing SP timesteppers for the full hybrid model. We hope this will give us the opportunity to incorporate previously inaccessible kinetic effects into the highly effective, modern, finite-element MHD models.
    \end{abstract}
    
    
    \newpage
    \tableofcontents
    
    
    \newpage
    \pagenumbering{arabic}
    %\linenumbers\renewcommand\thelinenumber{\color{black!50}\arabic{linenumber}}
            \input{0 - introduction/main.tex}
        \part{Research}
            \input{1 - low-noise PiC models/main.tex}
            \input{2 - kinetic component/main.tex}
            \input{3 - fluid component/main.tex}
            \input{4 - numerical implementation/main.tex}
        \part{Project Overview}
            \input{5 - research plan/main.tex}
            \input{6 - summary/main.tex}
    
    
    %\section{}
    \newpage
    \pagenumbering{gobble}
        \printbibliography


    \newpage
    \pagenumbering{roman}
    \appendix
        \part{Appendices}
            \input{8 - Hilbert complexes/main.tex}
            \input{9 - weak conservation proofs/main.tex}
\end{document}

            \documentclass[12pt, a4paper]{report}

\input{template/main.tex}

\title{\BA{Title in Progress...}}
\author{Boris Andrews}
\affil{Mathematical Institute, University of Oxford}
\date{\today}


\begin{document}
    \pagenumbering{gobble}
    \maketitle
    
    
    \begin{abstract}
        Magnetic confinement reactors---in particular tokamaks---offer one of the most promising options for achieving practical nuclear fusion, with the potential to provide virtually limitless, clean energy. The theoretical and numerical modeling of tokamak plasmas is simultaneously an essential component of effective reactor design, and a great research barrier. Tokamak operational conditions exhibit comparatively low Knudsen numbers. Kinetic effects, including kinetic waves and instabilities, Landau damping, bump-on-tail instabilities and more, are therefore highly influential in tokamak plasma dynamics. Purely fluid models are inherently incapable of capturing these effects, whereas the high dimensionality in purely kinetic models render them practically intractable for most relevant purposes.

        We consider a $\delta\!f$ decomposition model, with a macroscopic fluid background and microscopic kinetic correction, both fully coupled to each other. A similar manner of discretization is proposed to that used in the recent \texttt{STRUPHY} code \cite{Holderied_Possanner_Wang_2021, Holderied_2022, Li_et_al_2023} with a finite-element model for the background and a pseudo-particle/PiC model for the correction.

        The fluid background satisfies the full, non-linear, resistive, compressible, Hall MHD equations. \cite{Laakmann_Hu_Farrell_2022} introduces finite-element(-in-space) implicit timesteppers for the incompressible analogue to this system with structure-preserving (SP) properties in the ideal case, alongside parameter-robust preconditioners. We show that these timesteppers can derive from a finite-element-in-time (FET) (and finite-element-in-space) interpretation. The benefits of this reformulation are discussed, including the derivation of timesteppers that are higher order in time, and the quantifiable dissipative SP properties in the non-ideal, resistive case.
        
        We discuss possible options for extending this FET approach to timesteppers for the compressible case.

        The kinetic corrections satisfy linearized Boltzmann equations. Using a Lénard--Bernstein collision operator, these take Fokker--Planck-like forms \cite{Fokker_1914, Planck_1917} wherein pseudo-particles in the numerical model obey the neoclassical transport equations, with particle-independent Brownian drift terms. This offers a rigorous methodology for incorporating collisions into the particle transport model, without coupling the equations of motions for each particle.
        
        Works by Chen, Chacón et al. \cite{Chen_Chacón_Barnes_2011, Chacón_Chen_Barnes_2013, Chen_Chacón_2014, Chen_Chacón_2015} have developed structure-preserving particle pushers for neoclassical transport in the Vlasov equations, derived from Crank--Nicolson integrators. We show these too can can derive from a FET interpretation, similarly offering potential extensions to higher-order-in-time particle pushers. The FET formulation is used also to consider how the stochastic drift terms can be incorporated into the pushers. Stochastic gyrokinetic expansions are also discussed.

        Different options for the numerical implementation of these schemes are considered.

        Due to the efficacy of FET in the development of SP timesteppers for both the fluid and kinetic component, we hope this approach will prove effective in the future for developing SP timesteppers for the full hybrid model. We hope this will give us the opportunity to incorporate previously inaccessible kinetic effects into the highly effective, modern, finite-element MHD models.
    \end{abstract}
    
    
    \newpage
    \tableofcontents
    
    
    \newpage
    \pagenumbering{arabic}
    %\linenumbers\renewcommand\thelinenumber{\color{black!50}\arabic{linenumber}}
            \input{0 - introduction/main.tex}
        \part{Research}
            \input{1 - low-noise PiC models/main.tex}
            \input{2 - kinetic component/main.tex}
            \input{3 - fluid component/main.tex}
            \input{4 - numerical implementation/main.tex}
        \part{Project Overview}
            \input{5 - research plan/main.tex}
            \input{6 - summary/main.tex}
    
    
    %\section{}
    \newpage
    \pagenumbering{gobble}
        \printbibliography


    \newpage
    \pagenumbering{roman}
    \appendix
        \part{Appendices}
            \input{8 - Hilbert complexes/main.tex}
            \input{9 - weak conservation proofs/main.tex}
\end{document}

            \documentclass[12pt, a4paper]{report}

\input{template/main.tex}

\title{\BA{Title in Progress...}}
\author{Boris Andrews}
\affil{Mathematical Institute, University of Oxford}
\date{\today}


\begin{document}
    \pagenumbering{gobble}
    \maketitle
    
    
    \begin{abstract}
        Magnetic confinement reactors---in particular tokamaks---offer one of the most promising options for achieving practical nuclear fusion, with the potential to provide virtually limitless, clean energy. The theoretical and numerical modeling of tokamak plasmas is simultaneously an essential component of effective reactor design, and a great research barrier. Tokamak operational conditions exhibit comparatively low Knudsen numbers. Kinetic effects, including kinetic waves and instabilities, Landau damping, bump-on-tail instabilities and more, are therefore highly influential in tokamak plasma dynamics. Purely fluid models are inherently incapable of capturing these effects, whereas the high dimensionality in purely kinetic models render them practically intractable for most relevant purposes.

        We consider a $\delta\!f$ decomposition model, with a macroscopic fluid background and microscopic kinetic correction, both fully coupled to each other. A similar manner of discretization is proposed to that used in the recent \texttt{STRUPHY} code \cite{Holderied_Possanner_Wang_2021, Holderied_2022, Li_et_al_2023} with a finite-element model for the background and a pseudo-particle/PiC model for the correction.

        The fluid background satisfies the full, non-linear, resistive, compressible, Hall MHD equations. \cite{Laakmann_Hu_Farrell_2022} introduces finite-element(-in-space) implicit timesteppers for the incompressible analogue to this system with structure-preserving (SP) properties in the ideal case, alongside parameter-robust preconditioners. We show that these timesteppers can derive from a finite-element-in-time (FET) (and finite-element-in-space) interpretation. The benefits of this reformulation are discussed, including the derivation of timesteppers that are higher order in time, and the quantifiable dissipative SP properties in the non-ideal, resistive case.
        
        We discuss possible options for extending this FET approach to timesteppers for the compressible case.

        The kinetic corrections satisfy linearized Boltzmann equations. Using a Lénard--Bernstein collision operator, these take Fokker--Planck-like forms \cite{Fokker_1914, Planck_1917} wherein pseudo-particles in the numerical model obey the neoclassical transport equations, with particle-independent Brownian drift terms. This offers a rigorous methodology for incorporating collisions into the particle transport model, without coupling the equations of motions for each particle.
        
        Works by Chen, Chacón et al. \cite{Chen_Chacón_Barnes_2011, Chacón_Chen_Barnes_2013, Chen_Chacón_2014, Chen_Chacón_2015} have developed structure-preserving particle pushers for neoclassical transport in the Vlasov equations, derived from Crank--Nicolson integrators. We show these too can can derive from a FET interpretation, similarly offering potential extensions to higher-order-in-time particle pushers. The FET formulation is used also to consider how the stochastic drift terms can be incorporated into the pushers. Stochastic gyrokinetic expansions are also discussed.

        Different options for the numerical implementation of these schemes are considered.

        Due to the efficacy of FET in the development of SP timesteppers for both the fluid and kinetic component, we hope this approach will prove effective in the future for developing SP timesteppers for the full hybrid model. We hope this will give us the opportunity to incorporate previously inaccessible kinetic effects into the highly effective, modern, finite-element MHD models.
    \end{abstract}
    
    
    \newpage
    \tableofcontents
    
    
    \newpage
    \pagenumbering{arabic}
    %\linenumbers\renewcommand\thelinenumber{\color{black!50}\arabic{linenumber}}
            \input{0 - introduction/main.tex}
        \part{Research}
            \input{1 - low-noise PiC models/main.tex}
            \input{2 - kinetic component/main.tex}
            \input{3 - fluid component/main.tex}
            \input{4 - numerical implementation/main.tex}
        \part{Project Overview}
            \input{5 - research plan/main.tex}
            \input{6 - summary/main.tex}
    
    
    %\section{}
    \newpage
    \pagenumbering{gobble}
        \printbibliography


    \newpage
    \pagenumbering{roman}
    \appendix
        \part{Appendices}
            \input{8 - Hilbert complexes/main.tex}
            \input{9 - weak conservation proofs/main.tex}
\end{document}

            \documentclass[12pt, a4paper]{report}

\input{template/main.tex}

\title{\BA{Title in Progress...}}
\author{Boris Andrews}
\affil{Mathematical Institute, University of Oxford}
\date{\today}


\begin{document}
    \pagenumbering{gobble}
    \maketitle
    
    
    \begin{abstract}
        Magnetic confinement reactors---in particular tokamaks---offer one of the most promising options for achieving practical nuclear fusion, with the potential to provide virtually limitless, clean energy. The theoretical and numerical modeling of tokamak plasmas is simultaneously an essential component of effective reactor design, and a great research barrier. Tokamak operational conditions exhibit comparatively low Knudsen numbers. Kinetic effects, including kinetic waves and instabilities, Landau damping, bump-on-tail instabilities and more, are therefore highly influential in tokamak plasma dynamics. Purely fluid models are inherently incapable of capturing these effects, whereas the high dimensionality in purely kinetic models render them practically intractable for most relevant purposes.

        We consider a $\delta\!f$ decomposition model, with a macroscopic fluid background and microscopic kinetic correction, both fully coupled to each other. A similar manner of discretization is proposed to that used in the recent \texttt{STRUPHY} code \cite{Holderied_Possanner_Wang_2021, Holderied_2022, Li_et_al_2023} with a finite-element model for the background and a pseudo-particle/PiC model for the correction.

        The fluid background satisfies the full, non-linear, resistive, compressible, Hall MHD equations. \cite{Laakmann_Hu_Farrell_2022} introduces finite-element(-in-space) implicit timesteppers for the incompressible analogue to this system with structure-preserving (SP) properties in the ideal case, alongside parameter-robust preconditioners. We show that these timesteppers can derive from a finite-element-in-time (FET) (and finite-element-in-space) interpretation. The benefits of this reformulation are discussed, including the derivation of timesteppers that are higher order in time, and the quantifiable dissipative SP properties in the non-ideal, resistive case.
        
        We discuss possible options for extending this FET approach to timesteppers for the compressible case.

        The kinetic corrections satisfy linearized Boltzmann equations. Using a Lénard--Bernstein collision operator, these take Fokker--Planck-like forms \cite{Fokker_1914, Planck_1917} wherein pseudo-particles in the numerical model obey the neoclassical transport equations, with particle-independent Brownian drift terms. This offers a rigorous methodology for incorporating collisions into the particle transport model, without coupling the equations of motions for each particle.
        
        Works by Chen, Chacón et al. \cite{Chen_Chacón_Barnes_2011, Chacón_Chen_Barnes_2013, Chen_Chacón_2014, Chen_Chacón_2015} have developed structure-preserving particle pushers for neoclassical transport in the Vlasov equations, derived from Crank--Nicolson integrators. We show these too can can derive from a FET interpretation, similarly offering potential extensions to higher-order-in-time particle pushers. The FET formulation is used also to consider how the stochastic drift terms can be incorporated into the pushers. Stochastic gyrokinetic expansions are also discussed.

        Different options for the numerical implementation of these schemes are considered.

        Due to the efficacy of FET in the development of SP timesteppers for both the fluid and kinetic component, we hope this approach will prove effective in the future for developing SP timesteppers for the full hybrid model. We hope this will give us the opportunity to incorporate previously inaccessible kinetic effects into the highly effective, modern, finite-element MHD models.
    \end{abstract}
    
    
    \newpage
    \tableofcontents
    
    
    \newpage
    \pagenumbering{arabic}
    %\linenumbers\renewcommand\thelinenumber{\color{black!50}\arabic{linenumber}}
            \input{0 - introduction/main.tex}
        \part{Research}
            \input{1 - low-noise PiC models/main.tex}
            \input{2 - kinetic component/main.tex}
            \input{3 - fluid component/main.tex}
            \input{4 - numerical implementation/main.tex}
        \part{Project Overview}
            \input{5 - research plan/main.tex}
            \input{6 - summary/main.tex}
    
    
    %\section{}
    \newpage
    \pagenumbering{gobble}
        \printbibliography


    \newpage
    \pagenumbering{roman}
    \appendix
        \part{Appendices}
            \input{8 - Hilbert complexes/main.tex}
            \input{9 - weak conservation proofs/main.tex}
\end{document}

        \part{Project Overview}
            \documentclass[12pt, a4paper]{report}

\input{template/main.tex}

\title{\BA{Title in Progress...}}
\author{Boris Andrews}
\affil{Mathematical Institute, University of Oxford}
\date{\today}


\begin{document}
    \pagenumbering{gobble}
    \maketitle
    
    
    \begin{abstract}
        Magnetic confinement reactors---in particular tokamaks---offer one of the most promising options for achieving practical nuclear fusion, with the potential to provide virtually limitless, clean energy. The theoretical and numerical modeling of tokamak plasmas is simultaneously an essential component of effective reactor design, and a great research barrier. Tokamak operational conditions exhibit comparatively low Knudsen numbers. Kinetic effects, including kinetic waves and instabilities, Landau damping, bump-on-tail instabilities and more, are therefore highly influential in tokamak plasma dynamics. Purely fluid models are inherently incapable of capturing these effects, whereas the high dimensionality in purely kinetic models render them practically intractable for most relevant purposes.

        We consider a $\delta\!f$ decomposition model, with a macroscopic fluid background and microscopic kinetic correction, both fully coupled to each other. A similar manner of discretization is proposed to that used in the recent \texttt{STRUPHY} code \cite{Holderied_Possanner_Wang_2021, Holderied_2022, Li_et_al_2023} with a finite-element model for the background and a pseudo-particle/PiC model for the correction.

        The fluid background satisfies the full, non-linear, resistive, compressible, Hall MHD equations. \cite{Laakmann_Hu_Farrell_2022} introduces finite-element(-in-space) implicit timesteppers for the incompressible analogue to this system with structure-preserving (SP) properties in the ideal case, alongside parameter-robust preconditioners. We show that these timesteppers can derive from a finite-element-in-time (FET) (and finite-element-in-space) interpretation. The benefits of this reformulation are discussed, including the derivation of timesteppers that are higher order in time, and the quantifiable dissipative SP properties in the non-ideal, resistive case.
        
        We discuss possible options for extending this FET approach to timesteppers for the compressible case.

        The kinetic corrections satisfy linearized Boltzmann equations. Using a Lénard--Bernstein collision operator, these take Fokker--Planck-like forms \cite{Fokker_1914, Planck_1917} wherein pseudo-particles in the numerical model obey the neoclassical transport equations, with particle-independent Brownian drift terms. This offers a rigorous methodology for incorporating collisions into the particle transport model, without coupling the equations of motions for each particle.
        
        Works by Chen, Chacón et al. \cite{Chen_Chacón_Barnes_2011, Chacón_Chen_Barnes_2013, Chen_Chacón_2014, Chen_Chacón_2015} have developed structure-preserving particle pushers for neoclassical transport in the Vlasov equations, derived from Crank--Nicolson integrators. We show these too can can derive from a FET interpretation, similarly offering potential extensions to higher-order-in-time particle pushers. The FET formulation is used also to consider how the stochastic drift terms can be incorporated into the pushers. Stochastic gyrokinetic expansions are also discussed.

        Different options for the numerical implementation of these schemes are considered.

        Due to the efficacy of FET in the development of SP timesteppers for both the fluid and kinetic component, we hope this approach will prove effective in the future for developing SP timesteppers for the full hybrid model. We hope this will give us the opportunity to incorporate previously inaccessible kinetic effects into the highly effective, modern, finite-element MHD models.
    \end{abstract}
    
    
    \newpage
    \tableofcontents
    
    
    \newpage
    \pagenumbering{arabic}
    %\linenumbers\renewcommand\thelinenumber{\color{black!50}\arabic{linenumber}}
            \input{0 - introduction/main.tex}
        \part{Research}
            \input{1 - low-noise PiC models/main.tex}
            \input{2 - kinetic component/main.tex}
            \input{3 - fluid component/main.tex}
            \input{4 - numerical implementation/main.tex}
        \part{Project Overview}
            \input{5 - research plan/main.tex}
            \input{6 - summary/main.tex}
    
    
    %\section{}
    \newpage
    \pagenumbering{gobble}
        \printbibliography


    \newpage
    \pagenumbering{roman}
    \appendix
        \part{Appendices}
            \input{8 - Hilbert complexes/main.tex}
            \input{9 - weak conservation proofs/main.tex}
\end{document}

            \documentclass[12pt, a4paper]{report}

\input{template/main.tex}

\title{\BA{Title in Progress...}}
\author{Boris Andrews}
\affil{Mathematical Institute, University of Oxford}
\date{\today}


\begin{document}
    \pagenumbering{gobble}
    \maketitle
    
    
    \begin{abstract}
        Magnetic confinement reactors---in particular tokamaks---offer one of the most promising options for achieving practical nuclear fusion, with the potential to provide virtually limitless, clean energy. The theoretical and numerical modeling of tokamak plasmas is simultaneously an essential component of effective reactor design, and a great research barrier. Tokamak operational conditions exhibit comparatively low Knudsen numbers. Kinetic effects, including kinetic waves and instabilities, Landau damping, bump-on-tail instabilities and more, are therefore highly influential in tokamak plasma dynamics. Purely fluid models are inherently incapable of capturing these effects, whereas the high dimensionality in purely kinetic models render them practically intractable for most relevant purposes.

        We consider a $\delta\!f$ decomposition model, with a macroscopic fluid background and microscopic kinetic correction, both fully coupled to each other. A similar manner of discretization is proposed to that used in the recent \texttt{STRUPHY} code \cite{Holderied_Possanner_Wang_2021, Holderied_2022, Li_et_al_2023} with a finite-element model for the background and a pseudo-particle/PiC model for the correction.

        The fluid background satisfies the full, non-linear, resistive, compressible, Hall MHD equations. \cite{Laakmann_Hu_Farrell_2022} introduces finite-element(-in-space) implicit timesteppers for the incompressible analogue to this system with structure-preserving (SP) properties in the ideal case, alongside parameter-robust preconditioners. We show that these timesteppers can derive from a finite-element-in-time (FET) (and finite-element-in-space) interpretation. The benefits of this reformulation are discussed, including the derivation of timesteppers that are higher order in time, and the quantifiable dissipative SP properties in the non-ideal, resistive case.
        
        We discuss possible options for extending this FET approach to timesteppers for the compressible case.

        The kinetic corrections satisfy linearized Boltzmann equations. Using a Lénard--Bernstein collision operator, these take Fokker--Planck-like forms \cite{Fokker_1914, Planck_1917} wherein pseudo-particles in the numerical model obey the neoclassical transport equations, with particle-independent Brownian drift terms. This offers a rigorous methodology for incorporating collisions into the particle transport model, without coupling the equations of motions for each particle.
        
        Works by Chen, Chacón et al. \cite{Chen_Chacón_Barnes_2011, Chacón_Chen_Barnes_2013, Chen_Chacón_2014, Chen_Chacón_2015} have developed structure-preserving particle pushers for neoclassical transport in the Vlasov equations, derived from Crank--Nicolson integrators. We show these too can can derive from a FET interpretation, similarly offering potential extensions to higher-order-in-time particle pushers. The FET formulation is used also to consider how the stochastic drift terms can be incorporated into the pushers. Stochastic gyrokinetic expansions are also discussed.

        Different options for the numerical implementation of these schemes are considered.

        Due to the efficacy of FET in the development of SP timesteppers for both the fluid and kinetic component, we hope this approach will prove effective in the future for developing SP timesteppers for the full hybrid model. We hope this will give us the opportunity to incorporate previously inaccessible kinetic effects into the highly effective, modern, finite-element MHD models.
    \end{abstract}
    
    
    \newpage
    \tableofcontents
    
    
    \newpage
    \pagenumbering{arabic}
    %\linenumbers\renewcommand\thelinenumber{\color{black!50}\arabic{linenumber}}
            \input{0 - introduction/main.tex}
        \part{Research}
            \input{1 - low-noise PiC models/main.tex}
            \input{2 - kinetic component/main.tex}
            \input{3 - fluid component/main.tex}
            \input{4 - numerical implementation/main.tex}
        \part{Project Overview}
            \input{5 - research plan/main.tex}
            \input{6 - summary/main.tex}
    
    
    %\section{}
    \newpage
    \pagenumbering{gobble}
        \printbibliography


    \newpage
    \pagenumbering{roman}
    \appendix
        \part{Appendices}
            \input{8 - Hilbert complexes/main.tex}
            \input{9 - weak conservation proofs/main.tex}
\end{document}

    
    
    %\section{}
    \newpage
    \pagenumbering{gobble}
        \printbibliography


    \newpage
    \pagenumbering{roman}
    \appendix
        \part{Appendices}
            \documentclass[12pt, a4paper]{report}

\input{template/main.tex}

\title{\BA{Title in Progress...}}
\author{Boris Andrews}
\affil{Mathematical Institute, University of Oxford}
\date{\today}


\begin{document}
    \pagenumbering{gobble}
    \maketitle
    
    
    \begin{abstract}
        Magnetic confinement reactors---in particular tokamaks---offer one of the most promising options for achieving practical nuclear fusion, with the potential to provide virtually limitless, clean energy. The theoretical and numerical modeling of tokamak plasmas is simultaneously an essential component of effective reactor design, and a great research barrier. Tokamak operational conditions exhibit comparatively low Knudsen numbers. Kinetic effects, including kinetic waves and instabilities, Landau damping, bump-on-tail instabilities and more, are therefore highly influential in tokamak plasma dynamics. Purely fluid models are inherently incapable of capturing these effects, whereas the high dimensionality in purely kinetic models render them practically intractable for most relevant purposes.

        We consider a $\delta\!f$ decomposition model, with a macroscopic fluid background and microscopic kinetic correction, both fully coupled to each other. A similar manner of discretization is proposed to that used in the recent \texttt{STRUPHY} code \cite{Holderied_Possanner_Wang_2021, Holderied_2022, Li_et_al_2023} with a finite-element model for the background and a pseudo-particle/PiC model for the correction.

        The fluid background satisfies the full, non-linear, resistive, compressible, Hall MHD equations. \cite{Laakmann_Hu_Farrell_2022} introduces finite-element(-in-space) implicit timesteppers for the incompressible analogue to this system with structure-preserving (SP) properties in the ideal case, alongside parameter-robust preconditioners. We show that these timesteppers can derive from a finite-element-in-time (FET) (and finite-element-in-space) interpretation. The benefits of this reformulation are discussed, including the derivation of timesteppers that are higher order in time, and the quantifiable dissipative SP properties in the non-ideal, resistive case.
        
        We discuss possible options for extending this FET approach to timesteppers for the compressible case.

        The kinetic corrections satisfy linearized Boltzmann equations. Using a Lénard--Bernstein collision operator, these take Fokker--Planck-like forms \cite{Fokker_1914, Planck_1917} wherein pseudo-particles in the numerical model obey the neoclassical transport equations, with particle-independent Brownian drift terms. This offers a rigorous methodology for incorporating collisions into the particle transport model, without coupling the equations of motions for each particle.
        
        Works by Chen, Chacón et al. \cite{Chen_Chacón_Barnes_2011, Chacón_Chen_Barnes_2013, Chen_Chacón_2014, Chen_Chacón_2015} have developed structure-preserving particle pushers for neoclassical transport in the Vlasov equations, derived from Crank--Nicolson integrators. We show these too can can derive from a FET interpretation, similarly offering potential extensions to higher-order-in-time particle pushers. The FET formulation is used also to consider how the stochastic drift terms can be incorporated into the pushers. Stochastic gyrokinetic expansions are also discussed.

        Different options for the numerical implementation of these schemes are considered.

        Due to the efficacy of FET in the development of SP timesteppers for both the fluid and kinetic component, we hope this approach will prove effective in the future for developing SP timesteppers for the full hybrid model. We hope this will give us the opportunity to incorporate previously inaccessible kinetic effects into the highly effective, modern, finite-element MHD models.
    \end{abstract}
    
    
    \newpage
    \tableofcontents
    
    
    \newpage
    \pagenumbering{arabic}
    %\linenumbers\renewcommand\thelinenumber{\color{black!50}\arabic{linenumber}}
            \input{0 - introduction/main.tex}
        \part{Research}
            \input{1 - low-noise PiC models/main.tex}
            \input{2 - kinetic component/main.tex}
            \input{3 - fluid component/main.tex}
            \input{4 - numerical implementation/main.tex}
        \part{Project Overview}
            \input{5 - research plan/main.tex}
            \input{6 - summary/main.tex}
    
    
    %\section{}
    \newpage
    \pagenumbering{gobble}
        \printbibliography


    \newpage
    \pagenumbering{roman}
    \appendix
        \part{Appendices}
            \input{8 - Hilbert complexes/main.tex}
            \input{9 - weak conservation proofs/main.tex}
\end{document}

            \documentclass[12pt, a4paper]{report}

\input{template/main.tex}

\title{\BA{Title in Progress...}}
\author{Boris Andrews}
\affil{Mathematical Institute, University of Oxford}
\date{\today}


\begin{document}
    \pagenumbering{gobble}
    \maketitle
    
    
    \begin{abstract}
        Magnetic confinement reactors---in particular tokamaks---offer one of the most promising options for achieving practical nuclear fusion, with the potential to provide virtually limitless, clean energy. The theoretical and numerical modeling of tokamak plasmas is simultaneously an essential component of effective reactor design, and a great research barrier. Tokamak operational conditions exhibit comparatively low Knudsen numbers. Kinetic effects, including kinetic waves and instabilities, Landau damping, bump-on-tail instabilities and more, are therefore highly influential in tokamak plasma dynamics. Purely fluid models are inherently incapable of capturing these effects, whereas the high dimensionality in purely kinetic models render them practically intractable for most relevant purposes.

        We consider a $\delta\!f$ decomposition model, with a macroscopic fluid background and microscopic kinetic correction, both fully coupled to each other. A similar manner of discretization is proposed to that used in the recent \texttt{STRUPHY} code \cite{Holderied_Possanner_Wang_2021, Holderied_2022, Li_et_al_2023} with a finite-element model for the background and a pseudo-particle/PiC model for the correction.

        The fluid background satisfies the full, non-linear, resistive, compressible, Hall MHD equations. \cite{Laakmann_Hu_Farrell_2022} introduces finite-element(-in-space) implicit timesteppers for the incompressible analogue to this system with structure-preserving (SP) properties in the ideal case, alongside parameter-robust preconditioners. We show that these timesteppers can derive from a finite-element-in-time (FET) (and finite-element-in-space) interpretation. The benefits of this reformulation are discussed, including the derivation of timesteppers that are higher order in time, and the quantifiable dissipative SP properties in the non-ideal, resistive case.
        
        We discuss possible options for extending this FET approach to timesteppers for the compressible case.

        The kinetic corrections satisfy linearized Boltzmann equations. Using a Lénard--Bernstein collision operator, these take Fokker--Planck-like forms \cite{Fokker_1914, Planck_1917} wherein pseudo-particles in the numerical model obey the neoclassical transport equations, with particle-independent Brownian drift terms. This offers a rigorous methodology for incorporating collisions into the particle transport model, without coupling the equations of motions for each particle.
        
        Works by Chen, Chacón et al. \cite{Chen_Chacón_Barnes_2011, Chacón_Chen_Barnes_2013, Chen_Chacón_2014, Chen_Chacón_2015} have developed structure-preserving particle pushers for neoclassical transport in the Vlasov equations, derived from Crank--Nicolson integrators. We show these too can can derive from a FET interpretation, similarly offering potential extensions to higher-order-in-time particle pushers. The FET formulation is used also to consider how the stochastic drift terms can be incorporated into the pushers. Stochastic gyrokinetic expansions are also discussed.

        Different options for the numerical implementation of these schemes are considered.

        Due to the efficacy of FET in the development of SP timesteppers for both the fluid and kinetic component, we hope this approach will prove effective in the future for developing SP timesteppers for the full hybrid model. We hope this will give us the opportunity to incorporate previously inaccessible kinetic effects into the highly effective, modern, finite-element MHD models.
    \end{abstract}
    
    
    \newpage
    \tableofcontents
    
    
    \newpage
    \pagenumbering{arabic}
    %\linenumbers\renewcommand\thelinenumber{\color{black!50}\arabic{linenumber}}
            \input{0 - introduction/main.tex}
        \part{Research}
            \input{1 - low-noise PiC models/main.tex}
            \input{2 - kinetic component/main.tex}
            \input{3 - fluid component/main.tex}
            \input{4 - numerical implementation/main.tex}
        \part{Project Overview}
            \input{5 - research plan/main.tex}
            \input{6 - summary/main.tex}
    
    
    %\section{}
    \newpage
    \pagenumbering{gobble}
        \printbibliography


    \newpage
    \pagenumbering{roman}
    \appendix
        \part{Appendices}
            \input{8 - Hilbert complexes/main.tex}
            \input{9 - weak conservation proofs/main.tex}
\end{document}

\end{document}

            \documentclass[12pt, a4paper]{report}

\documentclass[12pt, a4paper]{report}

\input{template/main.tex}

\title{\BA{Title in Progress...}}
\author{Boris Andrews}
\affil{Mathematical Institute, University of Oxford}
\date{\today}


\begin{document}
    \pagenumbering{gobble}
    \maketitle
    
    
    \begin{abstract}
        Magnetic confinement reactors---in particular tokamaks---offer one of the most promising options for achieving practical nuclear fusion, with the potential to provide virtually limitless, clean energy. The theoretical and numerical modeling of tokamak plasmas is simultaneously an essential component of effective reactor design, and a great research barrier. Tokamak operational conditions exhibit comparatively low Knudsen numbers. Kinetic effects, including kinetic waves and instabilities, Landau damping, bump-on-tail instabilities and more, are therefore highly influential in tokamak plasma dynamics. Purely fluid models are inherently incapable of capturing these effects, whereas the high dimensionality in purely kinetic models render them practically intractable for most relevant purposes.

        We consider a $\delta\!f$ decomposition model, with a macroscopic fluid background and microscopic kinetic correction, both fully coupled to each other. A similar manner of discretization is proposed to that used in the recent \texttt{STRUPHY} code \cite{Holderied_Possanner_Wang_2021, Holderied_2022, Li_et_al_2023} with a finite-element model for the background and a pseudo-particle/PiC model for the correction.

        The fluid background satisfies the full, non-linear, resistive, compressible, Hall MHD equations. \cite{Laakmann_Hu_Farrell_2022} introduces finite-element(-in-space) implicit timesteppers for the incompressible analogue to this system with structure-preserving (SP) properties in the ideal case, alongside parameter-robust preconditioners. We show that these timesteppers can derive from a finite-element-in-time (FET) (and finite-element-in-space) interpretation. The benefits of this reformulation are discussed, including the derivation of timesteppers that are higher order in time, and the quantifiable dissipative SP properties in the non-ideal, resistive case.
        
        We discuss possible options for extending this FET approach to timesteppers for the compressible case.

        The kinetic corrections satisfy linearized Boltzmann equations. Using a Lénard--Bernstein collision operator, these take Fokker--Planck-like forms \cite{Fokker_1914, Planck_1917} wherein pseudo-particles in the numerical model obey the neoclassical transport equations, with particle-independent Brownian drift terms. This offers a rigorous methodology for incorporating collisions into the particle transport model, without coupling the equations of motions for each particle.
        
        Works by Chen, Chacón et al. \cite{Chen_Chacón_Barnes_2011, Chacón_Chen_Barnes_2013, Chen_Chacón_2014, Chen_Chacón_2015} have developed structure-preserving particle pushers for neoclassical transport in the Vlasov equations, derived from Crank--Nicolson integrators. We show these too can can derive from a FET interpretation, similarly offering potential extensions to higher-order-in-time particle pushers. The FET formulation is used also to consider how the stochastic drift terms can be incorporated into the pushers. Stochastic gyrokinetic expansions are also discussed.

        Different options for the numerical implementation of these schemes are considered.

        Due to the efficacy of FET in the development of SP timesteppers for both the fluid and kinetic component, we hope this approach will prove effective in the future for developing SP timesteppers for the full hybrid model. We hope this will give us the opportunity to incorporate previously inaccessible kinetic effects into the highly effective, modern, finite-element MHD models.
    \end{abstract}
    
    
    \newpage
    \tableofcontents
    
    
    \newpage
    \pagenumbering{arabic}
    %\linenumbers\renewcommand\thelinenumber{\color{black!50}\arabic{linenumber}}
            \input{0 - introduction/main.tex}
        \part{Research}
            \input{1 - low-noise PiC models/main.tex}
            \input{2 - kinetic component/main.tex}
            \input{3 - fluid component/main.tex}
            \input{4 - numerical implementation/main.tex}
        \part{Project Overview}
            \input{5 - research plan/main.tex}
            \input{6 - summary/main.tex}
    
    
    %\section{}
    \newpage
    \pagenumbering{gobble}
        \printbibliography


    \newpage
    \pagenumbering{roman}
    \appendix
        \part{Appendices}
            \input{8 - Hilbert complexes/main.tex}
            \input{9 - weak conservation proofs/main.tex}
\end{document}


\title{\BA{Title in Progress...}}
\author{Boris Andrews}
\affil{Mathematical Institute, University of Oxford}
\date{\today}


\begin{document}
    \pagenumbering{gobble}
    \maketitle
    
    
    \begin{abstract}
        Magnetic confinement reactors---in particular tokamaks---offer one of the most promising options for achieving practical nuclear fusion, with the potential to provide virtually limitless, clean energy. The theoretical and numerical modeling of tokamak plasmas is simultaneously an essential component of effective reactor design, and a great research barrier. Tokamak operational conditions exhibit comparatively low Knudsen numbers. Kinetic effects, including kinetic waves and instabilities, Landau damping, bump-on-tail instabilities and more, are therefore highly influential in tokamak plasma dynamics. Purely fluid models are inherently incapable of capturing these effects, whereas the high dimensionality in purely kinetic models render them practically intractable for most relevant purposes.

        We consider a $\delta\!f$ decomposition model, with a macroscopic fluid background and microscopic kinetic correction, both fully coupled to each other. A similar manner of discretization is proposed to that used in the recent \texttt{STRUPHY} code \cite{Holderied_Possanner_Wang_2021, Holderied_2022, Li_et_al_2023} with a finite-element model for the background and a pseudo-particle/PiC model for the correction.

        The fluid background satisfies the full, non-linear, resistive, compressible, Hall MHD equations. \cite{Laakmann_Hu_Farrell_2022} introduces finite-element(-in-space) implicit timesteppers for the incompressible analogue to this system with structure-preserving (SP) properties in the ideal case, alongside parameter-robust preconditioners. We show that these timesteppers can derive from a finite-element-in-time (FET) (and finite-element-in-space) interpretation. The benefits of this reformulation are discussed, including the derivation of timesteppers that are higher order in time, and the quantifiable dissipative SP properties in the non-ideal, resistive case.
        
        We discuss possible options for extending this FET approach to timesteppers for the compressible case.

        The kinetic corrections satisfy linearized Boltzmann equations. Using a Lénard--Bernstein collision operator, these take Fokker--Planck-like forms \cite{Fokker_1914, Planck_1917} wherein pseudo-particles in the numerical model obey the neoclassical transport equations, with particle-independent Brownian drift terms. This offers a rigorous methodology for incorporating collisions into the particle transport model, without coupling the equations of motions for each particle.
        
        Works by Chen, Chacón et al. \cite{Chen_Chacón_Barnes_2011, Chacón_Chen_Barnes_2013, Chen_Chacón_2014, Chen_Chacón_2015} have developed structure-preserving particle pushers for neoclassical transport in the Vlasov equations, derived from Crank--Nicolson integrators. We show these too can can derive from a FET interpretation, similarly offering potential extensions to higher-order-in-time particle pushers. The FET formulation is used also to consider how the stochastic drift terms can be incorporated into the pushers. Stochastic gyrokinetic expansions are also discussed.

        Different options for the numerical implementation of these schemes are considered.

        Due to the efficacy of FET in the development of SP timesteppers for both the fluid and kinetic component, we hope this approach will prove effective in the future for developing SP timesteppers for the full hybrid model. We hope this will give us the opportunity to incorporate previously inaccessible kinetic effects into the highly effective, modern, finite-element MHD models.
    \end{abstract}
    
    
    \newpage
    \tableofcontents
    
    
    \newpage
    \pagenumbering{arabic}
    %\linenumbers\renewcommand\thelinenumber{\color{black!50}\arabic{linenumber}}
            \documentclass[12pt, a4paper]{report}

\input{template/main.tex}

\title{\BA{Title in Progress...}}
\author{Boris Andrews}
\affil{Mathematical Institute, University of Oxford}
\date{\today}


\begin{document}
    \pagenumbering{gobble}
    \maketitle
    
    
    \begin{abstract}
        Magnetic confinement reactors---in particular tokamaks---offer one of the most promising options for achieving practical nuclear fusion, with the potential to provide virtually limitless, clean energy. The theoretical and numerical modeling of tokamak plasmas is simultaneously an essential component of effective reactor design, and a great research barrier. Tokamak operational conditions exhibit comparatively low Knudsen numbers. Kinetic effects, including kinetic waves and instabilities, Landau damping, bump-on-tail instabilities and more, are therefore highly influential in tokamak plasma dynamics. Purely fluid models are inherently incapable of capturing these effects, whereas the high dimensionality in purely kinetic models render them practically intractable for most relevant purposes.

        We consider a $\delta\!f$ decomposition model, with a macroscopic fluid background and microscopic kinetic correction, both fully coupled to each other. A similar manner of discretization is proposed to that used in the recent \texttt{STRUPHY} code \cite{Holderied_Possanner_Wang_2021, Holderied_2022, Li_et_al_2023} with a finite-element model for the background and a pseudo-particle/PiC model for the correction.

        The fluid background satisfies the full, non-linear, resistive, compressible, Hall MHD equations. \cite{Laakmann_Hu_Farrell_2022} introduces finite-element(-in-space) implicit timesteppers for the incompressible analogue to this system with structure-preserving (SP) properties in the ideal case, alongside parameter-robust preconditioners. We show that these timesteppers can derive from a finite-element-in-time (FET) (and finite-element-in-space) interpretation. The benefits of this reformulation are discussed, including the derivation of timesteppers that are higher order in time, and the quantifiable dissipative SP properties in the non-ideal, resistive case.
        
        We discuss possible options for extending this FET approach to timesteppers for the compressible case.

        The kinetic corrections satisfy linearized Boltzmann equations. Using a Lénard--Bernstein collision operator, these take Fokker--Planck-like forms \cite{Fokker_1914, Planck_1917} wherein pseudo-particles in the numerical model obey the neoclassical transport equations, with particle-independent Brownian drift terms. This offers a rigorous methodology for incorporating collisions into the particle transport model, without coupling the equations of motions for each particle.
        
        Works by Chen, Chacón et al. \cite{Chen_Chacón_Barnes_2011, Chacón_Chen_Barnes_2013, Chen_Chacón_2014, Chen_Chacón_2015} have developed structure-preserving particle pushers for neoclassical transport in the Vlasov equations, derived from Crank--Nicolson integrators. We show these too can can derive from a FET interpretation, similarly offering potential extensions to higher-order-in-time particle pushers. The FET formulation is used also to consider how the stochastic drift terms can be incorporated into the pushers. Stochastic gyrokinetic expansions are also discussed.

        Different options for the numerical implementation of these schemes are considered.

        Due to the efficacy of FET in the development of SP timesteppers for both the fluid and kinetic component, we hope this approach will prove effective in the future for developing SP timesteppers for the full hybrid model. We hope this will give us the opportunity to incorporate previously inaccessible kinetic effects into the highly effective, modern, finite-element MHD models.
    \end{abstract}
    
    
    \newpage
    \tableofcontents
    
    
    \newpage
    \pagenumbering{arabic}
    %\linenumbers\renewcommand\thelinenumber{\color{black!50}\arabic{linenumber}}
            \input{0 - introduction/main.tex}
        \part{Research}
            \input{1 - low-noise PiC models/main.tex}
            \input{2 - kinetic component/main.tex}
            \input{3 - fluid component/main.tex}
            \input{4 - numerical implementation/main.tex}
        \part{Project Overview}
            \input{5 - research plan/main.tex}
            \input{6 - summary/main.tex}
    
    
    %\section{}
    \newpage
    \pagenumbering{gobble}
        \printbibliography


    \newpage
    \pagenumbering{roman}
    \appendix
        \part{Appendices}
            \input{8 - Hilbert complexes/main.tex}
            \input{9 - weak conservation proofs/main.tex}
\end{document}

        \part{Research}
            \documentclass[12pt, a4paper]{report}

\input{template/main.tex}

\title{\BA{Title in Progress...}}
\author{Boris Andrews}
\affil{Mathematical Institute, University of Oxford}
\date{\today}


\begin{document}
    \pagenumbering{gobble}
    \maketitle
    
    
    \begin{abstract}
        Magnetic confinement reactors---in particular tokamaks---offer one of the most promising options for achieving practical nuclear fusion, with the potential to provide virtually limitless, clean energy. The theoretical and numerical modeling of tokamak plasmas is simultaneously an essential component of effective reactor design, and a great research barrier. Tokamak operational conditions exhibit comparatively low Knudsen numbers. Kinetic effects, including kinetic waves and instabilities, Landau damping, bump-on-tail instabilities and more, are therefore highly influential in tokamak plasma dynamics. Purely fluid models are inherently incapable of capturing these effects, whereas the high dimensionality in purely kinetic models render them practically intractable for most relevant purposes.

        We consider a $\delta\!f$ decomposition model, with a macroscopic fluid background and microscopic kinetic correction, both fully coupled to each other. A similar manner of discretization is proposed to that used in the recent \texttt{STRUPHY} code \cite{Holderied_Possanner_Wang_2021, Holderied_2022, Li_et_al_2023} with a finite-element model for the background and a pseudo-particle/PiC model for the correction.

        The fluid background satisfies the full, non-linear, resistive, compressible, Hall MHD equations. \cite{Laakmann_Hu_Farrell_2022} introduces finite-element(-in-space) implicit timesteppers for the incompressible analogue to this system with structure-preserving (SP) properties in the ideal case, alongside parameter-robust preconditioners. We show that these timesteppers can derive from a finite-element-in-time (FET) (and finite-element-in-space) interpretation. The benefits of this reformulation are discussed, including the derivation of timesteppers that are higher order in time, and the quantifiable dissipative SP properties in the non-ideal, resistive case.
        
        We discuss possible options for extending this FET approach to timesteppers for the compressible case.

        The kinetic corrections satisfy linearized Boltzmann equations. Using a Lénard--Bernstein collision operator, these take Fokker--Planck-like forms \cite{Fokker_1914, Planck_1917} wherein pseudo-particles in the numerical model obey the neoclassical transport equations, with particle-independent Brownian drift terms. This offers a rigorous methodology for incorporating collisions into the particle transport model, without coupling the equations of motions for each particle.
        
        Works by Chen, Chacón et al. \cite{Chen_Chacón_Barnes_2011, Chacón_Chen_Barnes_2013, Chen_Chacón_2014, Chen_Chacón_2015} have developed structure-preserving particle pushers for neoclassical transport in the Vlasov equations, derived from Crank--Nicolson integrators. We show these too can can derive from a FET interpretation, similarly offering potential extensions to higher-order-in-time particle pushers. The FET formulation is used also to consider how the stochastic drift terms can be incorporated into the pushers. Stochastic gyrokinetic expansions are also discussed.

        Different options for the numerical implementation of these schemes are considered.

        Due to the efficacy of FET in the development of SP timesteppers for both the fluid and kinetic component, we hope this approach will prove effective in the future for developing SP timesteppers for the full hybrid model. We hope this will give us the opportunity to incorporate previously inaccessible kinetic effects into the highly effective, modern, finite-element MHD models.
    \end{abstract}
    
    
    \newpage
    \tableofcontents
    
    
    \newpage
    \pagenumbering{arabic}
    %\linenumbers\renewcommand\thelinenumber{\color{black!50}\arabic{linenumber}}
            \input{0 - introduction/main.tex}
        \part{Research}
            \input{1 - low-noise PiC models/main.tex}
            \input{2 - kinetic component/main.tex}
            \input{3 - fluid component/main.tex}
            \input{4 - numerical implementation/main.tex}
        \part{Project Overview}
            \input{5 - research plan/main.tex}
            \input{6 - summary/main.tex}
    
    
    %\section{}
    \newpage
    \pagenumbering{gobble}
        \printbibliography


    \newpage
    \pagenumbering{roman}
    \appendix
        \part{Appendices}
            \input{8 - Hilbert complexes/main.tex}
            \input{9 - weak conservation proofs/main.tex}
\end{document}

            \documentclass[12pt, a4paper]{report}

\input{template/main.tex}

\title{\BA{Title in Progress...}}
\author{Boris Andrews}
\affil{Mathematical Institute, University of Oxford}
\date{\today}


\begin{document}
    \pagenumbering{gobble}
    \maketitle
    
    
    \begin{abstract}
        Magnetic confinement reactors---in particular tokamaks---offer one of the most promising options for achieving practical nuclear fusion, with the potential to provide virtually limitless, clean energy. The theoretical and numerical modeling of tokamak plasmas is simultaneously an essential component of effective reactor design, and a great research barrier. Tokamak operational conditions exhibit comparatively low Knudsen numbers. Kinetic effects, including kinetic waves and instabilities, Landau damping, bump-on-tail instabilities and more, are therefore highly influential in tokamak plasma dynamics. Purely fluid models are inherently incapable of capturing these effects, whereas the high dimensionality in purely kinetic models render them practically intractable for most relevant purposes.

        We consider a $\delta\!f$ decomposition model, with a macroscopic fluid background and microscopic kinetic correction, both fully coupled to each other. A similar manner of discretization is proposed to that used in the recent \texttt{STRUPHY} code \cite{Holderied_Possanner_Wang_2021, Holderied_2022, Li_et_al_2023} with a finite-element model for the background and a pseudo-particle/PiC model for the correction.

        The fluid background satisfies the full, non-linear, resistive, compressible, Hall MHD equations. \cite{Laakmann_Hu_Farrell_2022} introduces finite-element(-in-space) implicit timesteppers for the incompressible analogue to this system with structure-preserving (SP) properties in the ideal case, alongside parameter-robust preconditioners. We show that these timesteppers can derive from a finite-element-in-time (FET) (and finite-element-in-space) interpretation. The benefits of this reformulation are discussed, including the derivation of timesteppers that are higher order in time, and the quantifiable dissipative SP properties in the non-ideal, resistive case.
        
        We discuss possible options for extending this FET approach to timesteppers for the compressible case.

        The kinetic corrections satisfy linearized Boltzmann equations. Using a Lénard--Bernstein collision operator, these take Fokker--Planck-like forms \cite{Fokker_1914, Planck_1917} wherein pseudo-particles in the numerical model obey the neoclassical transport equations, with particle-independent Brownian drift terms. This offers a rigorous methodology for incorporating collisions into the particle transport model, without coupling the equations of motions for each particle.
        
        Works by Chen, Chacón et al. \cite{Chen_Chacón_Barnes_2011, Chacón_Chen_Barnes_2013, Chen_Chacón_2014, Chen_Chacón_2015} have developed structure-preserving particle pushers for neoclassical transport in the Vlasov equations, derived from Crank--Nicolson integrators. We show these too can can derive from a FET interpretation, similarly offering potential extensions to higher-order-in-time particle pushers. The FET formulation is used also to consider how the stochastic drift terms can be incorporated into the pushers. Stochastic gyrokinetic expansions are also discussed.

        Different options for the numerical implementation of these schemes are considered.

        Due to the efficacy of FET in the development of SP timesteppers for both the fluid and kinetic component, we hope this approach will prove effective in the future for developing SP timesteppers for the full hybrid model. We hope this will give us the opportunity to incorporate previously inaccessible kinetic effects into the highly effective, modern, finite-element MHD models.
    \end{abstract}
    
    
    \newpage
    \tableofcontents
    
    
    \newpage
    \pagenumbering{arabic}
    %\linenumbers\renewcommand\thelinenumber{\color{black!50}\arabic{linenumber}}
            \input{0 - introduction/main.tex}
        \part{Research}
            \input{1 - low-noise PiC models/main.tex}
            \input{2 - kinetic component/main.tex}
            \input{3 - fluid component/main.tex}
            \input{4 - numerical implementation/main.tex}
        \part{Project Overview}
            \input{5 - research plan/main.tex}
            \input{6 - summary/main.tex}
    
    
    %\section{}
    \newpage
    \pagenumbering{gobble}
        \printbibliography


    \newpage
    \pagenumbering{roman}
    \appendix
        \part{Appendices}
            \input{8 - Hilbert complexes/main.tex}
            \input{9 - weak conservation proofs/main.tex}
\end{document}

            \documentclass[12pt, a4paper]{report}

\input{template/main.tex}

\title{\BA{Title in Progress...}}
\author{Boris Andrews}
\affil{Mathematical Institute, University of Oxford}
\date{\today}


\begin{document}
    \pagenumbering{gobble}
    \maketitle
    
    
    \begin{abstract}
        Magnetic confinement reactors---in particular tokamaks---offer one of the most promising options for achieving practical nuclear fusion, with the potential to provide virtually limitless, clean energy. The theoretical and numerical modeling of tokamak plasmas is simultaneously an essential component of effective reactor design, and a great research barrier. Tokamak operational conditions exhibit comparatively low Knudsen numbers. Kinetic effects, including kinetic waves and instabilities, Landau damping, bump-on-tail instabilities and more, are therefore highly influential in tokamak plasma dynamics. Purely fluid models are inherently incapable of capturing these effects, whereas the high dimensionality in purely kinetic models render them practically intractable for most relevant purposes.

        We consider a $\delta\!f$ decomposition model, with a macroscopic fluid background and microscopic kinetic correction, both fully coupled to each other. A similar manner of discretization is proposed to that used in the recent \texttt{STRUPHY} code \cite{Holderied_Possanner_Wang_2021, Holderied_2022, Li_et_al_2023} with a finite-element model for the background and a pseudo-particle/PiC model for the correction.

        The fluid background satisfies the full, non-linear, resistive, compressible, Hall MHD equations. \cite{Laakmann_Hu_Farrell_2022} introduces finite-element(-in-space) implicit timesteppers for the incompressible analogue to this system with structure-preserving (SP) properties in the ideal case, alongside parameter-robust preconditioners. We show that these timesteppers can derive from a finite-element-in-time (FET) (and finite-element-in-space) interpretation. The benefits of this reformulation are discussed, including the derivation of timesteppers that are higher order in time, and the quantifiable dissipative SP properties in the non-ideal, resistive case.
        
        We discuss possible options for extending this FET approach to timesteppers for the compressible case.

        The kinetic corrections satisfy linearized Boltzmann equations. Using a Lénard--Bernstein collision operator, these take Fokker--Planck-like forms \cite{Fokker_1914, Planck_1917} wherein pseudo-particles in the numerical model obey the neoclassical transport equations, with particle-independent Brownian drift terms. This offers a rigorous methodology for incorporating collisions into the particle transport model, without coupling the equations of motions for each particle.
        
        Works by Chen, Chacón et al. \cite{Chen_Chacón_Barnes_2011, Chacón_Chen_Barnes_2013, Chen_Chacón_2014, Chen_Chacón_2015} have developed structure-preserving particle pushers for neoclassical transport in the Vlasov equations, derived from Crank--Nicolson integrators. We show these too can can derive from a FET interpretation, similarly offering potential extensions to higher-order-in-time particle pushers. The FET formulation is used also to consider how the stochastic drift terms can be incorporated into the pushers. Stochastic gyrokinetic expansions are also discussed.

        Different options for the numerical implementation of these schemes are considered.

        Due to the efficacy of FET in the development of SP timesteppers for both the fluid and kinetic component, we hope this approach will prove effective in the future for developing SP timesteppers for the full hybrid model. We hope this will give us the opportunity to incorporate previously inaccessible kinetic effects into the highly effective, modern, finite-element MHD models.
    \end{abstract}
    
    
    \newpage
    \tableofcontents
    
    
    \newpage
    \pagenumbering{arabic}
    %\linenumbers\renewcommand\thelinenumber{\color{black!50}\arabic{linenumber}}
            \input{0 - introduction/main.tex}
        \part{Research}
            \input{1 - low-noise PiC models/main.tex}
            \input{2 - kinetic component/main.tex}
            \input{3 - fluid component/main.tex}
            \input{4 - numerical implementation/main.tex}
        \part{Project Overview}
            \input{5 - research plan/main.tex}
            \input{6 - summary/main.tex}
    
    
    %\section{}
    \newpage
    \pagenumbering{gobble}
        \printbibliography


    \newpage
    \pagenumbering{roman}
    \appendix
        \part{Appendices}
            \input{8 - Hilbert complexes/main.tex}
            \input{9 - weak conservation proofs/main.tex}
\end{document}

            \documentclass[12pt, a4paper]{report}

\input{template/main.tex}

\title{\BA{Title in Progress...}}
\author{Boris Andrews}
\affil{Mathematical Institute, University of Oxford}
\date{\today}


\begin{document}
    \pagenumbering{gobble}
    \maketitle
    
    
    \begin{abstract}
        Magnetic confinement reactors---in particular tokamaks---offer one of the most promising options for achieving practical nuclear fusion, with the potential to provide virtually limitless, clean energy. The theoretical and numerical modeling of tokamak plasmas is simultaneously an essential component of effective reactor design, and a great research barrier. Tokamak operational conditions exhibit comparatively low Knudsen numbers. Kinetic effects, including kinetic waves and instabilities, Landau damping, bump-on-tail instabilities and more, are therefore highly influential in tokamak plasma dynamics. Purely fluid models are inherently incapable of capturing these effects, whereas the high dimensionality in purely kinetic models render them practically intractable for most relevant purposes.

        We consider a $\delta\!f$ decomposition model, with a macroscopic fluid background and microscopic kinetic correction, both fully coupled to each other. A similar manner of discretization is proposed to that used in the recent \texttt{STRUPHY} code \cite{Holderied_Possanner_Wang_2021, Holderied_2022, Li_et_al_2023} with a finite-element model for the background and a pseudo-particle/PiC model for the correction.

        The fluid background satisfies the full, non-linear, resistive, compressible, Hall MHD equations. \cite{Laakmann_Hu_Farrell_2022} introduces finite-element(-in-space) implicit timesteppers for the incompressible analogue to this system with structure-preserving (SP) properties in the ideal case, alongside parameter-robust preconditioners. We show that these timesteppers can derive from a finite-element-in-time (FET) (and finite-element-in-space) interpretation. The benefits of this reformulation are discussed, including the derivation of timesteppers that are higher order in time, and the quantifiable dissipative SP properties in the non-ideal, resistive case.
        
        We discuss possible options for extending this FET approach to timesteppers for the compressible case.

        The kinetic corrections satisfy linearized Boltzmann equations. Using a Lénard--Bernstein collision operator, these take Fokker--Planck-like forms \cite{Fokker_1914, Planck_1917} wherein pseudo-particles in the numerical model obey the neoclassical transport equations, with particle-independent Brownian drift terms. This offers a rigorous methodology for incorporating collisions into the particle transport model, without coupling the equations of motions for each particle.
        
        Works by Chen, Chacón et al. \cite{Chen_Chacón_Barnes_2011, Chacón_Chen_Barnes_2013, Chen_Chacón_2014, Chen_Chacón_2015} have developed structure-preserving particle pushers for neoclassical transport in the Vlasov equations, derived from Crank--Nicolson integrators. We show these too can can derive from a FET interpretation, similarly offering potential extensions to higher-order-in-time particle pushers. The FET formulation is used also to consider how the stochastic drift terms can be incorporated into the pushers. Stochastic gyrokinetic expansions are also discussed.

        Different options for the numerical implementation of these schemes are considered.

        Due to the efficacy of FET in the development of SP timesteppers for both the fluid and kinetic component, we hope this approach will prove effective in the future for developing SP timesteppers for the full hybrid model. We hope this will give us the opportunity to incorporate previously inaccessible kinetic effects into the highly effective, modern, finite-element MHD models.
    \end{abstract}
    
    
    \newpage
    \tableofcontents
    
    
    \newpage
    \pagenumbering{arabic}
    %\linenumbers\renewcommand\thelinenumber{\color{black!50}\arabic{linenumber}}
            \input{0 - introduction/main.tex}
        \part{Research}
            \input{1 - low-noise PiC models/main.tex}
            \input{2 - kinetic component/main.tex}
            \input{3 - fluid component/main.tex}
            \input{4 - numerical implementation/main.tex}
        \part{Project Overview}
            \input{5 - research plan/main.tex}
            \input{6 - summary/main.tex}
    
    
    %\section{}
    \newpage
    \pagenumbering{gobble}
        \printbibliography


    \newpage
    \pagenumbering{roman}
    \appendix
        \part{Appendices}
            \input{8 - Hilbert complexes/main.tex}
            \input{9 - weak conservation proofs/main.tex}
\end{document}

        \part{Project Overview}
            \documentclass[12pt, a4paper]{report}

\input{template/main.tex}

\title{\BA{Title in Progress...}}
\author{Boris Andrews}
\affil{Mathematical Institute, University of Oxford}
\date{\today}


\begin{document}
    \pagenumbering{gobble}
    \maketitle
    
    
    \begin{abstract}
        Magnetic confinement reactors---in particular tokamaks---offer one of the most promising options for achieving practical nuclear fusion, with the potential to provide virtually limitless, clean energy. The theoretical and numerical modeling of tokamak plasmas is simultaneously an essential component of effective reactor design, and a great research barrier. Tokamak operational conditions exhibit comparatively low Knudsen numbers. Kinetic effects, including kinetic waves and instabilities, Landau damping, bump-on-tail instabilities and more, are therefore highly influential in tokamak plasma dynamics. Purely fluid models are inherently incapable of capturing these effects, whereas the high dimensionality in purely kinetic models render them practically intractable for most relevant purposes.

        We consider a $\delta\!f$ decomposition model, with a macroscopic fluid background and microscopic kinetic correction, both fully coupled to each other. A similar manner of discretization is proposed to that used in the recent \texttt{STRUPHY} code \cite{Holderied_Possanner_Wang_2021, Holderied_2022, Li_et_al_2023} with a finite-element model for the background and a pseudo-particle/PiC model for the correction.

        The fluid background satisfies the full, non-linear, resistive, compressible, Hall MHD equations. \cite{Laakmann_Hu_Farrell_2022} introduces finite-element(-in-space) implicit timesteppers for the incompressible analogue to this system with structure-preserving (SP) properties in the ideal case, alongside parameter-robust preconditioners. We show that these timesteppers can derive from a finite-element-in-time (FET) (and finite-element-in-space) interpretation. The benefits of this reformulation are discussed, including the derivation of timesteppers that are higher order in time, and the quantifiable dissipative SP properties in the non-ideal, resistive case.
        
        We discuss possible options for extending this FET approach to timesteppers for the compressible case.

        The kinetic corrections satisfy linearized Boltzmann equations. Using a Lénard--Bernstein collision operator, these take Fokker--Planck-like forms \cite{Fokker_1914, Planck_1917} wherein pseudo-particles in the numerical model obey the neoclassical transport equations, with particle-independent Brownian drift terms. This offers a rigorous methodology for incorporating collisions into the particle transport model, without coupling the equations of motions for each particle.
        
        Works by Chen, Chacón et al. \cite{Chen_Chacón_Barnes_2011, Chacón_Chen_Barnes_2013, Chen_Chacón_2014, Chen_Chacón_2015} have developed structure-preserving particle pushers for neoclassical transport in the Vlasov equations, derived from Crank--Nicolson integrators. We show these too can can derive from a FET interpretation, similarly offering potential extensions to higher-order-in-time particle pushers. The FET formulation is used also to consider how the stochastic drift terms can be incorporated into the pushers. Stochastic gyrokinetic expansions are also discussed.

        Different options for the numerical implementation of these schemes are considered.

        Due to the efficacy of FET in the development of SP timesteppers for both the fluid and kinetic component, we hope this approach will prove effective in the future for developing SP timesteppers for the full hybrid model. We hope this will give us the opportunity to incorporate previously inaccessible kinetic effects into the highly effective, modern, finite-element MHD models.
    \end{abstract}
    
    
    \newpage
    \tableofcontents
    
    
    \newpage
    \pagenumbering{arabic}
    %\linenumbers\renewcommand\thelinenumber{\color{black!50}\arabic{linenumber}}
            \input{0 - introduction/main.tex}
        \part{Research}
            \input{1 - low-noise PiC models/main.tex}
            \input{2 - kinetic component/main.tex}
            \input{3 - fluid component/main.tex}
            \input{4 - numerical implementation/main.tex}
        \part{Project Overview}
            \input{5 - research plan/main.tex}
            \input{6 - summary/main.tex}
    
    
    %\section{}
    \newpage
    \pagenumbering{gobble}
        \printbibliography


    \newpage
    \pagenumbering{roman}
    \appendix
        \part{Appendices}
            \input{8 - Hilbert complexes/main.tex}
            \input{9 - weak conservation proofs/main.tex}
\end{document}

            \documentclass[12pt, a4paper]{report}

\input{template/main.tex}

\title{\BA{Title in Progress...}}
\author{Boris Andrews}
\affil{Mathematical Institute, University of Oxford}
\date{\today}


\begin{document}
    \pagenumbering{gobble}
    \maketitle
    
    
    \begin{abstract}
        Magnetic confinement reactors---in particular tokamaks---offer one of the most promising options for achieving practical nuclear fusion, with the potential to provide virtually limitless, clean energy. The theoretical and numerical modeling of tokamak plasmas is simultaneously an essential component of effective reactor design, and a great research barrier. Tokamak operational conditions exhibit comparatively low Knudsen numbers. Kinetic effects, including kinetic waves and instabilities, Landau damping, bump-on-tail instabilities and more, are therefore highly influential in tokamak plasma dynamics. Purely fluid models are inherently incapable of capturing these effects, whereas the high dimensionality in purely kinetic models render them practically intractable for most relevant purposes.

        We consider a $\delta\!f$ decomposition model, with a macroscopic fluid background and microscopic kinetic correction, both fully coupled to each other. A similar manner of discretization is proposed to that used in the recent \texttt{STRUPHY} code \cite{Holderied_Possanner_Wang_2021, Holderied_2022, Li_et_al_2023} with a finite-element model for the background and a pseudo-particle/PiC model for the correction.

        The fluid background satisfies the full, non-linear, resistive, compressible, Hall MHD equations. \cite{Laakmann_Hu_Farrell_2022} introduces finite-element(-in-space) implicit timesteppers for the incompressible analogue to this system with structure-preserving (SP) properties in the ideal case, alongside parameter-robust preconditioners. We show that these timesteppers can derive from a finite-element-in-time (FET) (and finite-element-in-space) interpretation. The benefits of this reformulation are discussed, including the derivation of timesteppers that are higher order in time, and the quantifiable dissipative SP properties in the non-ideal, resistive case.
        
        We discuss possible options for extending this FET approach to timesteppers for the compressible case.

        The kinetic corrections satisfy linearized Boltzmann equations. Using a Lénard--Bernstein collision operator, these take Fokker--Planck-like forms \cite{Fokker_1914, Planck_1917} wherein pseudo-particles in the numerical model obey the neoclassical transport equations, with particle-independent Brownian drift terms. This offers a rigorous methodology for incorporating collisions into the particle transport model, without coupling the equations of motions for each particle.
        
        Works by Chen, Chacón et al. \cite{Chen_Chacón_Barnes_2011, Chacón_Chen_Barnes_2013, Chen_Chacón_2014, Chen_Chacón_2015} have developed structure-preserving particle pushers for neoclassical transport in the Vlasov equations, derived from Crank--Nicolson integrators. We show these too can can derive from a FET interpretation, similarly offering potential extensions to higher-order-in-time particle pushers. The FET formulation is used also to consider how the stochastic drift terms can be incorporated into the pushers. Stochastic gyrokinetic expansions are also discussed.

        Different options for the numerical implementation of these schemes are considered.

        Due to the efficacy of FET in the development of SP timesteppers for both the fluid and kinetic component, we hope this approach will prove effective in the future for developing SP timesteppers for the full hybrid model. We hope this will give us the opportunity to incorporate previously inaccessible kinetic effects into the highly effective, modern, finite-element MHD models.
    \end{abstract}
    
    
    \newpage
    \tableofcontents
    
    
    \newpage
    \pagenumbering{arabic}
    %\linenumbers\renewcommand\thelinenumber{\color{black!50}\arabic{linenumber}}
            \input{0 - introduction/main.tex}
        \part{Research}
            \input{1 - low-noise PiC models/main.tex}
            \input{2 - kinetic component/main.tex}
            \input{3 - fluid component/main.tex}
            \input{4 - numerical implementation/main.tex}
        \part{Project Overview}
            \input{5 - research plan/main.tex}
            \input{6 - summary/main.tex}
    
    
    %\section{}
    \newpage
    \pagenumbering{gobble}
        \printbibliography


    \newpage
    \pagenumbering{roman}
    \appendix
        \part{Appendices}
            \input{8 - Hilbert complexes/main.tex}
            \input{9 - weak conservation proofs/main.tex}
\end{document}

    
    
    %\section{}
    \newpage
    \pagenumbering{gobble}
        \printbibliography


    \newpage
    \pagenumbering{roman}
    \appendix
        \part{Appendices}
            \documentclass[12pt, a4paper]{report}

\input{template/main.tex}

\title{\BA{Title in Progress...}}
\author{Boris Andrews}
\affil{Mathematical Institute, University of Oxford}
\date{\today}


\begin{document}
    \pagenumbering{gobble}
    \maketitle
    
    
    \begin{abstract}
        Magnetic confinement reactors---in particular tokamaks---offer one of the most promising options for achieving practical nuclear fusion, with the potential to provide virtually limitless, clean energy. The theoretical and numerical modeling of tokamak plasmas is simultaneously an essential component of effective reactor design, and a great research barrier. Tokamak operational conditions exhibit comparatively low Knudsen numbers. Kinetic effects, including kinetic waves and instabilities, Landau damping, bump-on-tail instabilities and more, are therefore highly influential in tokamak plasma dynamics. Purely fluid models are inherently incapable of capturing these effects, whereas the high dimensionality in purely kinetic models render them practically intractable for most relevant purposes.

        We consider a $\delta\!f$ decomposition model, with a macroscopic fluid background and microscopic kinetic correction, both fully coupled to each other. A similar manner of discretization is proposed to that used in the recent \texttt{STRUPHY} code \cite{Holderied_Possanner_Wang_2021, Holderied_2022, Li_et_al_2023} with a finite-element model for the background and a pseudo-particle/PiC model for the correction.

        The fluid background satisfies the full, non-linear, resistive, compressible, Hall MHD equations. \cite{Laakmann_Hu_Farrell_2022} introduces finite-element(-in-space) implicit timesteppers for the incompressible analogue to this system with structure-preserving (SP) properties in the ideal case, alongside parameter-robust preconditioners. We show that these timesteppers can derive from a finite-element-in-time (FET) (and finite-element-in-space) interpretation. The benefits of this reformulation are discussed, including the derivation of timesteppers that are higher order in time, and the quantifiable dissipative SP properties in the non-ideal, resistive case.
        
        We discuss possible options for extending this FET approach to timesteppers for the compressible case.

        The kinetic corrections satisfy linearized Boltzmann equations. Using a Lénard--Bernstein collision operator, these take Fokker--Planck-like forms \cite{Fokker_1914, Planck_1917} wherein pseudo-particles in the numerical model obey the neoclassical transport equations, with particle-independent Brownian drift terms. This offers a rigorous methodology for incorporating collisions into the particle transport model, without coupling the equations of motions for each particle.
        
        Works by Chen, Chacón et al. \cite{Chen_Chacón_Barnes_2011, Chacón_Chen_Barnes_2013, Chen_Chacón_2014, Chen_Chacón_2015} have developed structure-preserving particle pushers for neoclassical transport in the Vlasov equations, derived from Crank--Nicolson integrators. We show these too can can derive from a FET interpretation, similarly offering potential extensions to higher-order-in-time particle pushers. The FET formulation is used also to consider how the stochastic drift terms can be incorporated into the pushers. Stochastic gyrokinetic expansions are also discussed.

        Different options for the numerical implementation of these schemes are considered.

        Due to the efficacy of FET in the development of SP timesteppers for both the fluid and kinetic component, we hope this approach will prove effective in the future for developing SP timesteppers for the full hybrid model. We hope this will give us the opportunity to incorporate previously inaccessible kinetic effects into the highly effective, modern, finite-element MHD models.
    \end{abstract}
    
    
    \newpage
    \tableofcontents
    
    
    \newpage
    \pagenumbering{arabic}
    %\linenumbers\renewcommand\thelinenumber{\color{black!50}\arabic{linenumber}}
            \input{0 - introduction/main.tex}
        \part{Research}
            \input{1 - low-noise PiC models/main.tex}
            \input{2 - kinetic component/main.tex}
            \input{3 - fluid component/main.tex}
            \input{4 - numerical implementation/main.tex}
        \part{Project Overview}
            \input{5 - research plan/main.tex}
            \input{6 - summary/main.tex}
    
    
    %\section{}
    \newpage
    \pagenumbering{gobble}
        \printbibliography


    \newpage
    \pagenumbering{roman}
    \appendix
        \part{Appendices}
            \input{8 - Hilbert complexes/main.tex}
            \input{9 - weak conservation proofs/main.tex}
\end{document}

            \documentclass[12pt, a4paper]{report}

\input{template/main.tex}

\title{\BA{Title in Progress...}}
\author{Boris Andrews}
\affil{Mathematical Institute, University of Oxford}
\date{\today}


\begin{document}
    \pagenumbering{gobble}
    \maketitle
    
    
    \begin{abstract}
        Magnetic confinement reactors---in particular tokamaks---offer one of the most promising options for achieving practical nuclear fusion, with the potential to provide virtually limitless, clean energy. The theoretical and numerical modeling of tokamak plasmas is simultaneously an essential component of effective reactor design, and a great research barrier. Tokamak operational conditions exhibit comparatively low Knudsen numbers. Kinetic effects, including kinetic waves and instabilities, Landau damping, bump-on-tail instabilities and more, are therefore highly influential in tokamak plasma dynamics. Purely fluid models are inherently incapable of capturing these effects, whereas the high dimensionality in purely kinetic models render them practically intractable for most relevant purposes.

        We consider a $\delta\!f$ decomposition model, with a macroscopic fluid background and microscopic kinetic correction, both fully coupled to each other. A similar manner of discretization is proposed to that used in the recent \texttt{STRUPHY} code \cite{Holderied_Possanner_Wang_2021, Holderied_2022, Li_et_al_2023} with a finite-element model for the background and a pseudo-particle/PiC model for the correction.

        The fluid background satisfies the full, non-linear, resistive, compressible, Hall MHD equations. \cite{Laakmann_Hu_Farrell_2022} introduces finite-element(-in-space) implicit timesteppers for the incompressible analogue to this system with structure-preserving (SP) properties in the ideal case, alongside parameter-robust preconditioners. We show that these timesteppers can derive from a finite-element-in-time (FET) (and finite-element-in-space) interpretation. The benefits of this reformulation are discussed, including the derivation of timesteppers that are higher order in time, and the quantifiable dissipative SP properties in the non-ideal, resistive case.
        
        We discuss possible options for extending this FET approach to timesteppers for the compressible case.

        The kinetic corrections satisfy linearized Boltzmann equations. Using a Lénard--Bernstein collision operator, these take Fokker--Planck-like forms \cite{Fokker_1914, Planck_1917} wherein pseudo-particles in the numerical model obey the neoclassical transport equations, with particle-independent Brownian drift terms. This offers a rigorous methodology for incorporating collisions into the particle transport model, without coupling the equations of motions for each particle.
        
        Works by Chen, Chacón et al. \cite{Chen_Chacón_Barnes_2011, Chacón_Chen_Barnes_2013, Chen_Chacón_2014, Chen_Chacón_2015} have developed structure-preserving particle pushers for neoclassical transport in the Vlasov equations, derived from Crank--Nicolson integrators. We show these too can can derive from a FET interpretation, similarly offering potential extensions to higher-order-in-time particle pushers. The FET formulation is used also to consider how the stochastic drift terms can be incorporated into the pushers. Stochastic gyrokinetic expansions are also discussed.

        Different options for the numerical implementation of these schemes are considered.

        Due to the efficacy of FET in the development of SP timesteppers for both the fluid and kinetic component, we hope this approach will prove effective in the future for developing SP timesteppers for the full hybrid model. We hope this will give us the opportunity to incorporate previously inaccessible kinetic effects into the highly effective, modern, finite-element MHD models.
    \end{abstract}
    
    
    \newpage
    \tableofcontents
    
    
    \newpage
    \pagenumbering{arabic}
    %\linenumbers\renewcommand\thelinenumber{\color{black!50}\arabic{linenumber}}
            \input{0 - introduction/main.tex}
        \part{Research}
            \input{1 - low-noise PiC models/main.tex}
            \input{2 - kinetic component/main.tex}
            \input{3 - fluid component/main.tex}
            \input{4 - numerical implementation/main.tex}
        \part{Project Overview}
            \input{5 - research plan/main.tex}
            \input{6 - summary/main.tex}
    
    
    %\section{}
    \newpage
    \pagenumbering{gobble}
        \printbibliography


    \newpage
    \pagenumbering{roman}
    \appendix
        \part{Appendices}
            \input{8 - Hilbert complexes/main.tex}
            \input{9 - weak conservation proofs/main.tex}
\end{document}

\end{document}

            \documentclass[12pt, a4paper]{report}

\documentclass[12pt, a4paper]{report}

\input{template/main.tex}

\title{\BA{Title in Progress...}}
\author{Boris Andrews}
\affil{Mathematical Institute, University of Oxford}
\date{\today}


\begin{document}
    \pagenumbering{gobble}
    \maketitle
    
    
    \begin{abstract}
        Magnetic confinement reactors---in particular tokamaks---offer one of the most promising options for achieving practical nuclear fusion, with the potential to provide virtually limitless, clean energy. The theoretical and numerical modeling of tokamak plasmas is simultaneously an essential component of effective reactor design, and a great research barrier. Tokamak operational conditions exhibit comparatively low Knudsen numbers. Kinetic effects, including kinetic waves and instabilities, Landau damping, bump-on-tail instabilities and more, are therefore highly influential in tokamak plasma dynamics. Purely fluid models are inherently incapable of capturing these effects, whereas the high dimensionality in purely kinetic models render them practically intractable for most relevant purposes.

        We consider a $\delta\!f$ decomposition model, with a macroscopic fluid background and microscopic kinetic correction, both fully coupled to each other. A similar manner of discretization is proposed to that used in the recent \texttt{STRUPHY} code \cite{Holderied_Possanner_Wang_2021, Holderied_2022, Li_et_al_2023} with a finite-element model for the background and a pseudo-particle/PiC model for the correction.

        The fluid background satisfies the full, non-linear, resistive, compressible, Hall MHD equations. \cite{Laakmann_Hu_Farrell_2022} introduces finite-element(-in-space) implicit timesteppers for the incompressible analogue to this system with structure-preserving (SP) properties in the ideal case, alongside parameter-robust preconditioners. We show that these timesteppers can derive from a finite-element-in-time (FET) (and finite-element-in-space) interpretation. The benefits of this reformulation are discussed, including the derivation of timesteppers that are higher order in time, and the quantifiable dissipative SP properties in the non-ideal, resistive case.
        
        We discuss possible options for extending this FET approach to timesteppers for the compressible case.

        The kinetic corrections satisfy linearized Boltzmann equations. Using a Lénard--Bernstein collision operator, these take Fokker--Planck-like forms \cite{Fokker_1914, Planck_1917} wherein pseudo-particles in the numerical model obey the neoclassical transport equations, with particle-independent Brownian drift terms. This offers a rigorous methodology for incorporating collisions into the particle transport model, without coupling the equations of motions for each particle.
        
        Works by Chen, Chacón et al. \cite{Chen_Chacón_Barnes_2011, Chacón_Chen_Barnes_2013, Chen_Chacón_2014, Chen_Chacón_2015} have developed structure-preserving particle pushers for neoclassical transport in the Vlasov equations, derived from Crank--Nicolson integrators. We show these too can can derive from a FET interpretation, similarly offering potential extensions to higher-order-in-time particle pushers. The FET formulation is used also to consider how the stochastic drift terms can be incorporated into the pushers. Stochastic gyrokinetic expansions are also discussed.

        Different options for the numerical implementation of these schemes are considered.

        Due to the efficacy of FET in the development of SP timesteppers for both the fluid and kinetic component, we hope this approach will prove effective in the future for developing SP timesteppers for the full hybrid model. We hope this will give us the opportunity to incorporate previously inaccessible kinetic effects into the highly effective, modern, finite-element MHD models.
    \end{abstract}
    
    
    \newpage
    \tableofcontents
    
    
    \newpage
    \pagenumbering{arabic}
    %\linenumbers\renewcommand\thelinenumber{\color{black!50}\arabic{linenumber}}
            \input{0 - introduction/main.tex}
        \part{Research}
            \input{1 - low-noise PiC models/main.tex}
            \input{2 - kinetic component/main.tex}
            \input{3 - fluid component/main.tex}
            \input{4 - numerical implementation/main.tex}
        \part{Project Overview}
            \input{5 - research plan/main.tex}
            \input{6 - summary/main.tex}
    
    
    %\section{}
    \newpage
    \pagenumbering{gobble}
        \printbibliography


    \newpage
    \pagenumbering{roman}
    \appendix
        \part{Appendices}
            \input{8 - Hilbert complexes/main.tex}
            \input{9 - weak conservation proofs/main.tex}
\end{document}


\title{\BA{Title in Progress...}}
\author{Boris Andrews}
\affil{Mathematical Institute, University of Oxford}
\date{\today}


\begin{document}
    \pagenumbering{gobble}
    \maketitle
    
    
    \begin{abstract}
        Magnetic confinement reactors---in particular tokamaks---offer one of the most promising options for achieving practical nuclear fusion, with the potential to provide virtually limitless, clean energy. The theoretical and numerical modeling of tokamak plasmas is simultaneously an essential component of effective reactor design, and a great research barrier. Tokamak operational conditions exhibit comparatively low Knudsen numbers. Kinetic effects, including kinetic waves and instabilities, Landau damping, bump-on-tail instabilities and more, are therefore highly influential in tokamak plasma dynamics. Purely fluid models are inherently incapable of capturing these effects, whereas the high dimensionality in purely kinetic models render them practically intractable for most relevant purposes.

        We consider a $\delta\!f$ decomposition model, with a macroscopic fluid background and microscopic kinetic correction, both fully coupled to each other. A similar manner of discretization is proposed to that used in the recent \texttt{STRUPHY} code \cite{Holderied_Possanner_Wang_2021, Holderied_2022, Li_et_al_2023} with a finite-element model for the background and a pseudo-particle/PiC model for the correction.

        The fluid background satisfies the full, non-linear, resistive, compressible, Hall MHD equations. \cite{Laakmann_Hu_Farrell_2022} introduces finite-element(-in-space) implicit timesteppers for the incompressible analogue to this system with structure-preserving (SP) properties in the ideal case, alongside parameter-robust preconditioners. We show that these timesteppers can derive from a finite-element-in-time (FET) (and finite-element-in-space) interpretation. The benefits of this reformulation are discussed, including the derivation of timesteppers that are higher order in time, and the quantifiable dissipative SP properties in the non-ideal, resistive case.
        
        We discuss possible options for extending this FET approach to timesteppers for the compressible case.

        The kinetic corrections satisfy linearized Boltzmann equations. Using a Lénard--Bernstein collision operator, these take Fokker--Planck-like forms \cite{Fokker_1914, Planck_1917} wherein pseudo-particles in the numerical model obey the neoclassical transport equations, with particle-independent Brownian drift terms. This offers a rigorous methodology for incorporating collisions into the particle transport model, without coupling the equations of motions for each particle.
        
        Works by Chen, Chacón et al. \cite{Chen_Chacón_Barnes_2011, Chacón_Chen_Barnes_2013, Chen_Chacón_2014, Chen_Chacón_2015} have developed structure-preserving particle pushers for neoclassical transport in the Vlasov equations, derived from Crank--Nicolson integrators. We show these too can can derive from a FET interpretation, similarly offering potential extensions to higher-order-in-time particle pushers. The FET formulation is used also to consider how the stochastic drift terms can be incorporated into the pushers. Stochastic gyrokinetic expansions are also discussed.

        Different options for the numerical implementation of these schemes are considered.

        Due to the efficacy of FET in the development of SP timesteppers for both the fluid and kinetic component, we hope this approach will prove effective in the future for developing SP timesteppers for the full hybrid model. We hope this will give us the opportunity to incorporate previously inaccessible kinetic effects into the highly effective, modern, finite-element MHD models.
    \end{abstract}
    
    
    \newpage
    \tableofcontents
    
    
    \newpage
    \pagenumbering{arabic}
    %\linenumbers\renewcommand\thelinenumber{\color{black!50}\arabic{linenumber}}
            \documentclass[12pt, a4paper]{report}

\input{template/main.tex}

\title{\BA{Title in Progress...}}
\author{Boris Andrews}
\affil{Mathematical Institute, University of Oxford}
\date{\today}


\begin{document}
    \pagenumbering{gobble}
    \maketitle
    
    
    \begin{abstract}
        Magnetic confinement reactors---in particular tokamaks---offer one of the most promising options for achieving practical nuclear fusion, with the potential to provide virtually limitless, clean energy. The theoretical and numerical modeling of tokamak plasmas is simultaneously an essential component of effective reactor design, and a great research barrier. Tokamak operational conditions exhibit comparatively low Knudsen numbers. Kinetic effects, including kinetic waves and instabilities, Landau damping, bump-on-tail instabilities and more, are therefore highly influential in tokamak plasma dynamics. Purely fluid models are inherently incapable of capturing these effects, whereas the high dimensionality in purely kinetic models render them practically intractable for most relevant purposes.

        We consider a $\delta\!f$ decomposition model, with a macroscopic fluid background and microscopic kinetic correction, both fully coupled to each other. A similar manner of discretization is proposed to that used in the recent \texttt{STRUPHY} code \cite{Holderied_Possanner_Wang_2021, Holderied_2022, Li_et_al_2023} with a finite-element model for the background and a pseudo-particle/PiC model for the correction.

        The fluid background satisfies the full, non-linear, resistive, compressible, Hall MHD equations. \cite{Laakmann_Hu_Farrell_2022} introduces finite-element(-in-space) implicit timesteppers for the incompressible analogue to this system with structure-preserving (SP) properties in the ideal case, alongside parameter-robust preconditioners. We show that these timesteppers can derive from a finite-element-in-time (FET) (and finite-element-in-space) interpretation. The benefits of this reformulation are discussed, including the derivation of timesteppers that are higher order in time, and the quantifiable dissipative SP properties in the non-ideal, resistive case.
        
        We discuss possible options for extending this FET approach to timesteppers for the compressible case.

        The kinetic corrections satisfy linearized Boltzmann equations. Using a Lénard--Bernstein collision operator, these take Fokker--Planck-like forms \cite{Fokker_1914, Planck_1917} wherein pseudo-particles in the numerical model obey the neoclassical transport equations, with particle-independent Brownian drift terms. This offers a rigorous methodology for incorporating collisions into the particle transport model, without coupling the equations of motions for each particle.
        
        Works by Chen, Chacón et al. \cite{Chen_Chacón_Barnes_2011, Chacón_Chen_Barnes_2013, Chen_Chacón_2014, Chen_Chacón_2015} have developed structure-preserving particle pushers for neoclassical transport in the Vlasov equations, derived from Crank--Nicolson integrators. We show these too can can derive from a FET interpretation, similarly offering potential extensions to higher-order-in-time particle pushers. The FET formulation is used also to consider how the stochastic drift terms can be incorporated into the pushers. Stochastic gyrokinetic expansions are also discussed.

        Different options for the numerical implementation of these schemes are considered.

        Due to the efficacy of FET in the development of SP timesteppers for both the fluid and kinetic component, we hope this approach will prove effective in the future for developing SP timesteppers for the full hybrid model. We hope this will give us the opportunity to incorporate previously inaccessible kinetic effects into the highly effective, modern, finite-element MHD models.
    \end{abstract}
    
    
    \newpage
    \tableofcontents
    
    
    \newpage
    \pagenumbering{arabic}
    %\linenumbers\renewcommand\thelinenumber{\color{black!50}\arabic{linenumber}}
            \input{0 - introduction/main.tex}
        \part{Research}
            \input{1 - low-noise PiC models/main.tex}
            \input{2 - kinetic component/main.tex}
            \input{3 - fluid component/main.tex}
            \input{4 - numerical implementation/main.tex}
        \part{Project Overview}
            \input{5 - research plan/main.tex}
            \input{6 - summary/main.tex}
    
    
    %\section{}
    \newpage
    \pagenumbering{gobble}
        \printbibliography


    \newpage
    \pagenumbering{roman}
    \appendix
        \part{Appendices}
            \input{8 - Hilbert complexes/main.tex}
            \input{9 - weak conservation proofs/main.tex}
\end{document}

        \part{Research}
            \documentclass[12pt, a4paper]{report}

\input{template/main.tex}

\title{\BA{Title in Progress...}}
\author{Boris Andrews}
\affil{Mathematical Institute, University of Oxford}
\date{\today}


\begin{document}
    \pagenumbering{gobble}
    \maketitle
    
    
    \begin{abstract}
        Magnetic confinement reactors---in particular tokamaks---offer one of the most promising options for achieving practical nuclear fusion, with the potential to provide virtually limitless, clean energy. The theoretical and numerical modeling of tokamak plasmas is simultaneously an essential component of effective reactor design, and a great research barrier. Tokamak operational conditions exhibit comparatively low Knudsen numbers. Kinetic effects, including kinetic waves and instabilities, Landau damping, bump-on-tail instabilities and more, are therefore highly influential in tokamak plasma dynamics. Purely fluid models are inherently incapable of capturing these effects, whereas the high dimensionality in purely kinetic models render them practically intractable for most relevant purposes.

        We consider a $\delta\!f$ decomposition model, with a macroscopic fluid background and microscopic kinetic correction, both fully coupled to each other. A similar manner of discretization is proposed to that used in the recent \texttt{STRUPHY} code \cite{Holderied_Possanner_Wang_2021, Holderied_2022, Li_et_al_2023} with a finite-element model for the background and a pseudo-particle/PiC model for the correction.

        The fluid background satisfies the full, non-linear, resistive, compressible, Hall MHD equations. \cite{Laakmann_Hu_Farrell_2022} introduces finite-element(-in-space) implicit timesteppers for the incompressible analogue to this system with structure-preserving (SP) properties in the ideal case, alongside parameter-robust preconditioners. We show that these timesteppers can derive from a finite-element-in-time (FET) (and finite-element-in-space) interpretation. The benefits of this reformulation are discussed, including the derivation of timesteppers that are higher order in time, and the quantifiable dissipative SP properties in the non-ideal, resistive case.
        
        We discuss possible options for extending this FET approach to timesteppers for the compressible case.

        The kinetic corrections satisfy linearized Boltzmann equations. Using a Lénard--Bernstein collision operator, these take Fokker--Planck-like forms \cite{Fokker_1914, Planck_1917} wherein pseudo-particles in the numerical model obey the neoclassical transport equations, with particle-independent Brownian drift terms. This offers a rigorous methodology for incorporating collisions into the particle transport model, without coupling the equations of motions for each particle.
        
        Works by Chen, Chacón et al. \cite{Chen_Chacón_Barnes_2011, Chacón_Chen_Barnes_2013, Chen_Chacón_2014, Chen_Chacón_2015} have developed structure-preserving particle pushers for neoclassical transport in the Vlasov equations, derived from Crank--Nicolson integrators. We show these too can can derive from a FET interpretation, similarly offering potential extensions to higher-order-in-time particle pushers. The FET formulation is used also to consider how the stochastic drift terms can be incorporated into the pushers. Stochastic gyrokinetic expansions are also discussed.

        Different options for the numerical implementation of these schemes are considered.

        Due to the efficacy of FET in the development of SP timesteppers for both the fluid and kinetic component, we hope this approach will prove effective in the future for developing SP timesteppers for the full hybrid model. We hope this will give us the opportunity to incorporate previously inaccessible kinetic effects into the highly effective, modern, finite-element MHD models.
    \end{abstract}
    
    
    \newpage
    \tableofcontents
    
    
    \newpage
    \pagenumbering{arabic}
    %\linenumbers\renewcommand\thelinenumber{\color{black!50}\arabic{linenumber}}
            \input{0 - introduction/main.tex}
        \part{Research}
            \input{1 - low-noise PiC models/main.tex}
            \input{2 - kinetic component/main.tex}
            \input{3 - fluid component/main.tex}
            \input{4 - numerical implementation/main.tex}
        \part{Project Overview}
            \input{5 - research plan/main.tex}
            \input{6 - summary/main.tex}
    
    
    %\section{}
    \newpage
    \pagenumbering{gobble}
        \printbibliography


    \newpage
    \pagenumbering{roman}
    \appendix
        \part{Appendices}
            \input{8 - Hilbert complexes/main.tex}
            \input{9 - weak conservation proofs/main.tex}
\end{document}

            \documentclass[12pt, a4paper]{report}

\input{template/main.tex}

\title{\BA{Title in Progress...}}
\author{Boris Andrews}
\affil{Mathematical Institute, University of Oxford}
\date{\today}


\begin{document}
    \pagenumbering{gobble}
    \maketitle
    
    
    \begin{abstract}
        Magnetic confinement reactors---in particular tokamaks---offer one of the most promising options for achieving practical nuclear fusion, with the potential to provide virtually limitless, clean energy. The theoretical and numerical modeling of tokamak plasmas is simultaneously an essential component of effective reactor design, and a great research barrier. Tokamak operational conditions exhibit comparatively low Knudsen numbers. Kinetic effects, including kinetic waves and instabilities, Landau damping, bump-on-tail instabilities and more, are therefore highly influential in tokamak plasma dynamics. Purely fluid models are inherently incapable of capturing these effects, whereas the high dimensionality in purely kinetic models render them practically intractable for most relevant purposes.

        We consider a $\delta\!f$ decomposition model, with a macroscopic fluid background and microscopic kinetic correction, both fully coupled to each other. A similar manner of discretization is proposed to that used in the recent \texttt{STRUPHY} code \cite{Holderied_Possanner_Wang_2021, Holderied_2022, Li_et_al_2023} with a finite-element model for the background and a pseudo-particle/PiC model for the correction.

        The fluid background satisfies the full, non-linear, resistive, compressible, Hall MHD equations. \cite{Laakmann_Hu_Farrell_2022} introduces finite-element(-in-space) implicit timesteppers for the incompressible analogue to this system with structure-preserving (SP) properties in the ideal case, alongside parameter-robust preconditioners. We show that these timesteppers can derive from a finite-element-in-time (FET) (and finite-element-in-space) interpretation. The benefits of this reformulation are discussed, including the derivation of timesteppers that are higher order in time, and the quantifiable dissipative SP properties in the non-ideal, resistive case.
        
        We discuss possible options for extending this FET approach to timesteppers for the compressible case.

        The kinetic corrections satisfy linearized Boltzmann equations. Using a Lénard--Bernstein collision operator, these take Fokker--Planck-like forms \cite{Fokker_1914, Planck_1917} wherein pseudo-particles in the numerical model obey the neoclassical transport equations, with particle-independent Brownian drift terms. This offers a rigorous methodology for incorporating collisions into the particle transport model, without coupling the equations of motions for each particle.
        
        Works by Chen, Chacón et al. \cite{Chen_Chacón_Barnes_2011, Chacón_Chen_Barnes_2013, Chen_Chacón_2014, Chen_Chacón_2015} have developed structure-preserving particle pushers for neoclassical transport in the Vlasov equations, derived from Crank--Nicolson integrators. We show these too can can derive from a FET interpretation, similarly offering potential extensions to higher-order-in-time particle pushers. The FET formulation is used also to consider how the stochastic drift terms can be incorporated into the pushers. Stochastic gyrokinetic expansions are also discussed.

        Different options for the numerical implementation of these schemes are considered.

        Due to the efficacy of FET in the development of SP timesteppers for both the fluid and kinetic component, we hope this approach will prove effective in the future for developing SP timesteppers for the full hybrid model. We hope this will give us the opportunity to incorporate previously inaccessible kinetic effects into the highly effective, modern, finite-element MHD models.
    \end{abstract}
    
    
    \newpage
    \tableofcontents
    
    
    \newpage
    \pagenumbering{arabic}
    %\linenumbers\renewcommand\thelinenumber{\color{black!50}\arabic{linenumber}}
            \input{0 - introduction/main.tex}
        \part{Research}
            \input{1 - low-noise PiC models/main.tex}
            \input{2 - kinetic component/main.tex}
            \input{3 - fluid component/main.tex}
            \input{4 - numerical implementation/main.tex}
        \part{Project Overview}
            \input{5 - research plan/main.tex}
            \input{6 - summary/main.tex}
    
    
    %\section{}
    \newpage
    \pagenumbering{gobble}
        \printbibliography


    \newpage
    \pagenumbering{roman}
    \appendix
        \part{Appendices}
            \input{8 - Hilbert complexes/main.tex}
            \input{9 - weak conservation proofs/main.tex}
\end{document}

            \documentclass[12pt, a4paper]{report}

\input{template/main.tex}

\title{\BA{Title in Progress...}}
\author{Boris Andrews}
\affil{Mathematical Institute, University of Oxford}
\date{\today}


\begin{document}
    \pagenumbering{gobble}
    \maketitle
    
    
    \begin{abstract}
        Magnetic confinement reactors---in particular tokamaks---offer one of the most promising options for achieving practical nuclear fusion, with the potential to provide virtually limitless, clean energy. The theoretical and numerical modeling of tokamak plasmas is simultaneously an essential component of effective reactor design, and a great research barrier. Tokamak operational conditions exhibit comparatively low Knudsen numbers. Kinetic effects, including kinetic waves and instabilities, Landau damping, bump-on-tail instabilities and more, are therefore highly influential in tokamak plasma dynamics. Purely fluid models are inherently incapable of capturing these effects, whereas the high dimensionality in purely kinetic models render them practically intractable for most relevant purposes.

        We consider a $\delta\!f$ decomposition model, with a macroscopic fluid background and microscopic kinetic correction, both fully coupled to each other. A similar manner of discretization is proposed to that used in the recent \texttt{STRUPHY} code \cite{Holderied_Possanner_Wang_2021, Holderied_2022, Li_et_al_2023} with a finite-element model for the background and a pseudo-particle/PiC model for the correction.

        The fluid background satisfies the full, non-linear, resistive, compressible, Hall MHD equations. \cite{Laakmann_Hu_Farrell_2022} introduces finite-element(-in-space) implicit timesteppers for the incompressible analogue to this system with structure-preserving (SP) properties in the ideal case, alongside parameter-robust preconditioners. We show that these timesteppers can derive from a finite-element-in-time (FET) (and finite-element-in-space) interpretation. The benefits of this reformulation are discussed, including the derivation of timesteppers that are higher order in time, and the quantifiable dissipative SP properties in the non-ideal, resistive case.
        
        We discuss possible options for extending this FET approach to timesteppers for the compressible case.

        The kinetic corrections satisfy linearized Boltzmann equations. Using a Lénard--Bernstein collision operator, these take Fokker--Planck-like forms \cite{Fokker_1914, Planck_1917} wherein pseudo-particles in the numerical model obey the neoclassical transport equations, with particle-independent Brownian drift terms. This offers a rigorous methodology for incorporating collisions into the particle transport model, without coupling the equations of motions for each particle.
        
        Works by Chen, Chacón et al. \cite{Chen_Chacón_Barnes_2011, Chacón_Chen_Barnes_2013, Chen_Chacón_2014, Chen_Chacón_2015} have developed structure-preserving particle pushers for neoclassical transport in the Vlasov equations, derived from Crank--Nicolson integrators. We show these too can can derive from a FET interpretation, similarly offering potential extensions to higher-order-in-time particle pushers. The FET formulation is used also to consider how the stochastic drift terms can be incorporated into the pushers. Stochastic gyrokinetic expansions are also discussed.

        Different options for the numerical implementation of these schemes are considered.

        Due to the efficacy of FET in the development of SP timesteppers for both the fluid and kinetic component, we hope this approach will prove effective in the future for developing SP timesteppers for the full hybrid model. We hope this will give us the opportunity to incorporate previously inaccessible kinetic effects into the highly effective, modern, finite-element MHD models.
    \end{abstract}
    
    
    \newpage
    \tableofcontents
    
    
    \newpage
    \pagenumbering{arabic}
    %\linenumbers\renewcommand\thelinenumber{\color{black!50}\arabic{linenumber}}
            \input{0 - introduction/main.tex}
        \part{Research}
            \input{1 - low-noise PiC models/main.tex}
            \input{2 - kinetic component/main.tex}
            \input{3 - fluid component/main.tex}
            \input{4 - numerical implementation/main.tex}
        \part{Project Overview}
            \input{5 - research plan/main.tex}
            \input{6 - summary/main.tex}
    
    
    %\section{}
    \newpage
    \pagenumbering{gobble}
        \printbibliography


    \newpage
    \pagenumbering{roman}
    \appendix
        \part{Appendices}
            \input{8 - Hilbert complexes/main.tex}
            \input{9 - weak conservation proofs/main.tex}
\end{document}

            \documentclass[12pt, a4paper]{report}

\input{template/main.tex}

\title{\BA{Title in Progress...}}
\author{Boris Andrews}
\affil{Mathematical Institute, University of Oxford}
\date{\today}


\begin{document}
    \pagenumbering{gobble}
    \maketitle
    
    
    \begin{abstract}
        Magnetic confinement reactors---in particular tokamaks---offer one of the most promising options for achieving practical nuclear fusion, with the potential to provide virtually limitless, clean energy. The theoretical and numerical modeling of tokamak plasmas is simultaneously an essential component of effective reactor design, and a great research barrier. Tokamak operational conditions exhibit comparatively low Knudsen numbers. Kinetic effects, including kinetic waves and instabilities, Landau damping, bump-on-tail instabilities and more, are therefore highly influential in tokamak plasma dynamics. Purely fluid models are inherently incapable of capturing these effects, whereas the high dimensionality in purely kinetic models render them practically intractable for most relevant purposes.

        We consider a $\delta\!f$ decomposition model, with a macroscopic fluid background and microscopic kinetic correction, both fully coupled to each other. A similar manner of discretization is proposed to that used in the recent \texttt{STRUPHY} code \cite{Holderied_Possanner_Wang_2021, Holderied_2022, Li_et_al_2023} with a finite-element model for the background and a pseudo-particle/PiC model for the correction.

        The fluid background satisfies the full, non-linear, resistive, compressible, Hall MHD equations. \cite{Laakmann_Hu_Farrell_2022} introduces finite-element(-in-space) implicit timesteppers for the incompressible analogue to this system with structure-preserving (SP) properties in the ideal case, alongside parameter-robust preconditioners. We show that these timesteppers can derive from a finite-element-in-time (FET) (and finite-element-in-space) interpretation. The benefits of this reformulation are discussed, including the derivation of timesteppers that are higher order in time, and the quantifiable dissipative SP properties in the non-ideal, resistive case.
        
        We discuss possible options for extending this FET approach to timesteppers for the compressible case.

        The kinetic corrections satisfy linearized Boltzmann equations. Using a Lénard--Bernstein collision operator, these take Fokker--Planck-like forms \cite{Fokker_1914, Planck_1917} wherein pseudo-particles in the numerical model obey the neoclassical transport equations, with particle-independent Brownian drift terms. This offers a rigorous methodology for incorporating collisions into the particle transport model, without coupling the equations of motions for each particle.
        
        Works by Chen, Chacón et al. \cite{Chen_Chacón_Barnes_2011, Chacón_Chen_Barnes_2013, Chen_Chacón_2014, Chen_Chacón_2015} have developed structure-preserving particle pushers for neoclassical transport in the Vlasov equations, derived from Crank--Nicolson integrators. We show these too can can derive from a FET interpretation, similarly offering potential extensions to higher-order-in-time particle pushers. The FET formulation is used also to consider how the stochastic drift terms can be incorporated into the pushers. Stochastic gyrokinetic expansions are also discussed.

        Different options for the numerical implementation of these schemes are considered.

        Due to the efficacy of FET in the development of SP timesteppers for both the fluid and kinetic component, we hope this approach will prove effective in the future for developing SP timesteppers for the full hybrid model. We hope this will give us the opportunity to incorporate previously inaccessible kinetic effects into the highly effective, modern, finite-element MHD models.
    \end{abstract}
    
    
    \newpage
    \tableofcontents
    
    
    \newpage
    \pagenumbering{arabic}
    %\linenumbers\renewcommand\thelinenumber{\color{black!50}\arabic{linenumber}}
            \input{0 - introduction/main.tex}
        \part{Research}
            \input{1 - low-noise PiC models/main.tex}
            \input{2 - kinetic component/main.tex}
            \input{3 - fluid component/main.tex}
            \input{4 - numerical implementation/main.tex}
        \part{Project Overview}
            \input{5 - research plan/main.tex}
            \input{6 - summary/main.tex}
    
    
    %\section{}
    \newpage
    \pagenumbering{gobble}
        \printbibliography


    \newpage
    \pagenumbering{roman}
    \appendix
        \part{Appendices}
            \input{8 - Hilbert complexes/main.tex}
            \input{9 - weak conservation proofs/main.tex}
\end{document}

        \part{Project Overview}
            \documentclass[12pt, a4paper]{report}

\input{template/main.tex}

\title{\BA{Title in Progress...}}
\author{Boris Andrews}
\affil{Mathematical Institute, University of Oxford}
\date{\today}


\begin{document}
    \pagenumbering{gobble}
    \maketitle
    
    
    \begin{abstract}
        Magnetic confinement reactors---in particular tokamaks---offer one of the most promising options for achieving practical nuclear fusion, with the potential to provide virtually limitless, clean energy. The theoretical and numerical modeling of tokamak plasmas is simultaneously an essential component of effective reactor design, and a great research barrier. Tokamak operational conditions exhibit comparatively low Knudsen numbers. Kinetic effects, including kinetic waves and instabilities, Landau damping, bump-on-tail instabilities and more, are therefore highly influential in tokamak plasma dynamics. Purely fluid models are inherently incapable of capturing these effects, whereas the high dimensionality in purely kinetic models render them practically intractable for most relevant purposes.

        We consider a $\delta\!f$ decomposition model, with a macroscopic fluid background and microscopic kinetic correction, both fully coupled to each other. A similar manner of discretization is proposed to that used in the recent \texttt{STRUPHY} code \cite{Holderied_Possanner_Wang_2021, Holderied_2022, Li_et_al_2023} with a finite-element model for the background and a pseudo-particle/PiC model for the correction.

        The fluid background satisfies the full, non-linear, resistive, compressible, Hall MHD equations. \cite{Laakmann_Hu_Farrell_2022} introduces finite-element(-in-space) implicit timesteppers for the incompressible analogue to this system with structure-preserving (SP) properties in the ideal case, alongside parameter-robust preconditioners. We show that these timesteppers can derive from a finite-element-in-time (FET) (and finite-element-in-space) interpretation. The benefits of this reformulation are discussed, including the derivation of timesteppers that are higher order in time, and the quantifiable dissipative SP properties in the non-ideal, resistive case.
        
        We discuss possible options for extending this FET approach to timesteppers for the compressible case.

        The kinetic corrections satisfy linearized Boltzmann equations. Using a Lénard--Bernstein collision operator, these take Fokker--Planck-like forms \cite{Fokker_1914, Planck_1917} wherein pseudo-particles in the numerical model obey the neoclassical transport equations, with particle-independent Brownian drift terms. This offers a rigorous methodology for incorporating collisions into the particle transport model, without coupling the equations of motions for each particle.
        
        Works by Chen, Chacón et al. \cite{Chen_Chacón_Barnes_2011, Chacón_Chen_Barnes_2013, Chen_Chacón_2014, Chen_Chacón_2015} have developed structure-preserving particle pushers for neoclassical transport in the Vlasov equations, derived from Crank--Nicolson integrators. We show these too can can derive from a FET interpretation, similarly offering potential extensions to higher-order-in-time particle pushers. The FET formulation is used also to consider how the stochastic drift terms can be incorporated into the pushers. Stochastic gyrokinetic expansions are also discussed.

        Different options for the numerical implementation of these schemes are considered.

        Due to the efficacy of FET in the development of SP timesteppers for both the fluid and kinetic component, we hope this approach will prove effective in the future for developing SP timesteppers for the full hybrid model. We hope this will give us the opportunity to incorporate previously inaccessible kinetic effects into the highly effective, modern, finite-element MHD models.
    \end{abstract}
    
    
    \newpage
    \tableofcontents
    
    
    \newpage
    \pagenumbering{arabic}
    %\linenumbers\renewcommand\thelinenumber{\color{black!50}\arabic{linenumber}}
            \input{0 - introduction/main.tex}
        \part{Research}
            \input{1 - low-noise PiC models/main.tex}
            \input{2 - kinetic component/main.tex}
            \input{3 - fluid component/main.tex}
            \input{4 - numerical implementation/main.tex}
        \part{Project Overview}
            \input{5 - research plan/main.tex}
            \input{6 - summary/main.tex}
    
    
    %\section{}
    \newpage
    \pagenumbering{gobble}
        \printbibliography


    \newpage
    \pagenumbering{roman}
    \appendix
        \part{Appendices}
            \input{8 - Hilbert complexes/main.tex}
            \input{9 - weak conservation proofs/main.tex}
\end{document}

            \documentclass[12pt, a4paper]{report}

\input{template/main.tex}

\title{\BA{Title in Progress...}}
\author{Boris Andrews}
\affil{Mathematical Institute, University of Oxford}
\date{\today}


\begin{document}
    \pagenumbering{gobble}
    \maketitle
    
    
    \begin{abstract}
        Magnetic confinement reactors---in particular tokamaks---offer one of the most promising options for achieving practical nuclear fusion, with the potential to provide virtually limitless, clean energy. The theoretical and numerical modeling of tokamak plasmas is simultaneously an essential component of effective reactor design, and a great research barrier. Tokamak operational conditions exhibit comparatively low Knudsen numbers. Kinetic effects, including kinetic waves and instabilities, Landau damping, bump-on-tail instabilities and more, are therefore highly influential in tokamak plasma dynamics. Purely fluid models are inherently incapable of capturing these effects, whereas the high dimensionality in purely kinetic models render them practically intractable for most relevant purposes.

        We consider a $\delta\!f$ decomposition model, with a macroscopic fluid background and microscopic kinetic correction, both fully coupled to each other. A similar manner of discretization is proposed to that used in the recent \texttt{STRUPHY} code \cite{Holderied_Possanner_Wang_2021, Holderied_2022, Li_et_al_2023} with a finite-element model for the background and a pseudo-particle/PiC model for the correction.

        The fluid background satisfies the full, non-linear, resistive, compressible, Hall MHD equations. \cite{Laakmann_Hu_Farrell_2022} introduces finite-element(-in-space) implicit timesteppers for the incompressible analogue to this system with structure-preserving (SP) properties in the ideal case, alongside parameter-robust preconditioners. We show that these timesteppers can derive from a finite-element-in-time (FET) (and finite-element-in-space) interpretation. The benefits of this reformulation are discussed, including the derivation of timesteppers that are higher order in time, and the quantifiable dissipative SP properties in the non-ideal, resistive case.
        
        We discuss possible options for extending this FET approach to timesteppers for the compressible case.

        The kinetic corrections satisfy linearized Boltzmann equations. Using a Lénard--Bernstein collision operator, these take Fokker--Planck-like forms \cite{Fokker_1914, Planck_1917} wherein pseudo-particles in the numerical model obey the neoclassical transport equations, with particle-independent Brownian drift terms. This offers a rigorous methodology for incorporating collisions into the particle transport model, without coupling the equations of motions for each particle.
        
        Works by Chen, Chacón et al. \cite{Chen_Chacón_Barnes_2011, Chacón_Chen_Barnes_2013, Chen_Chacón_2014, Chen_Chacón_2015} have developed structure-preserving particle pushers for neoclassical transport in the Vlasov equations, derived from Crank--Nicolson integrators. We show these too can can derive from a FET interpretation, similarly offering potential extensions to higher-order-in-time particle pushers. The FET formulation is used also to consider how the stochastic drift terms can be incorporated into the pushers. Stochastic gyrokinetic expansions are also discussed.

        Different options for the numerical implementation of these schemes are considered.

        Due to the efficacy of FET in the development of SP timesteppers for both the fluid and kinetic component, we hope this approach will prove effective in the future for developing SP timesteppers for the full hybrid model. We hope this will give us the opportunity to incorporate previously inaccessible kinetic effects into the highly effective, modern, finite-element MHD models.
    \end{abstract}
    
    
    \newpage
    \tableofcontents
    
    
    \newpage
    \pagenumbering{arabic}
    %\linenumbers\renewcommand\thelinenumber{\color{black!50}\arabic{linenumber}}
            \input{0 - introduction/main.tex}
        \part{Research}
            \input{1 - low-noise PiC models/main.tex}
            \input{2 - kinetic component/main.tex}
            \input{3 - fluid component/main.tex}
            \input{4 - numerical implementation/main.tex}
        \part{Project Overview}
            \input{5 - research plan/main.tex}
            \input{6 - summary/main.tex}
    
    
    %\section{}
    \newpage
    \pagenumbering{gobble}
        \printbibliography


    \newpage
    \pagenumbering{roman}
    \appendix
        \part{Appendices}
            \input{8 - Hilbert complexes/main.tex}
            \input{9 - weak conservation proofs/main.tex}
\end{document}

    
    
    %\section{}
    \newpage
    \pagenumbering{gobble}
        \printbibliography


    \newpage
    \pagenumbering{roman}
    \appendix
        \part{Appendices}
            \documentclass[12pt, a4paper]{report}

\input{template/main.tex}

\title{\BA{Title in Progress...}}
\author{Boris Andrews}
\affil{Mathematical Institute, University of Oxford}
\date{\today}


\begin{document}
    \pagenumbering{gobble}
    \maketitle
    
    
    \begin{abstract}
        Magnetic confinement reactors---in particular tokamaks---offer one of the most promising options for achieving practical nuclear fusion, with the potential to provide virtually limitless, clean energy. The theoretical and numerical modeling of tokamak plasmas is simultaneously an essential component of effective reactor design, and a great research barrier. Tokamak operational conditions exhibit comparatively low Knudsen numbers. Kinetic effects, including kinetic waves and instabilities, Landau damping, bump-on-tail instabilities and more, are therefore highly influential in tokamak plasma dynamics. Purely fluid models are inherently incapable of capturing these effects, whereas the high dimensionality in purely kinetic models render them practically intractable for most relevant purposes.

        We consider a $\delta\!f$ decomposition model, with a macroscopic fluid background and microscopic kinetic correction, both fully coupled to each other. A similar manner of discretization is proposed to that used in the recent \texttt{STRUPHY} code \cite{Holderied_Possanner_Wang_2021, Holderied_2022, Li_et_al_2023} with a finite-element model for the background and a pseudo-particle/PiC model for the correction.

        The fluid background satisfies the full, non-linear, resistive, compressible, Hall MHD equations. \cite{Laakmann_Hu_Farrell_2022} introduces finite-element(-in-space) implicit timesteppers for the incompressible analogue to this system with structure-preserving (SP) properties in the ideal case, alongside parameter-robust preconditioners. We show that these timesteppers can derive from a finite-element-in-time (FET) (and finite-element-in-space) interpretation. The benefits of this reformulation are discussed, including the derivation of timesteppers that are higher order in time, and the quantifiable dissipative SP properties in the non-ideal, resistive case.
        
        We discuss possible options for extending this FET approach to timesteppers for the compressible case.

        The kinetic corrections satisfy linearized Boltzmann equations. Using a Lénard--Bernstein collision operator, these take Fokker--Planck-like forms \cite{Fokker_1914, Planck_1917} wherein pseudo-particles in the numerical model obey the neoclassical transport equations, with particle-independent Brownian drift terms. This offers a rigorous methodology for incorporating collisions into the particle transport model, without coupling the equations of motions for each particle.
        
        Works by Chen, Chacón et al. \cite{Chen_Chacón_Barnes_2011, Chacón_Chen_Barnes_2013, Chen_Chacón_2014, Chen_Chacón_2015} have developed structure-preserving particle pushers for neoclassical transport in the Vlasov equations, derived from Crank--Nicolson integrators. We show these too can can derive from a FET interpretation, similarly offering potential extensions to higher-order-in-time particle pushers. The FET formulation is used also to consider how the stochastic drift terms can be incorporated into the pushers. Stochastic gyrokinetic expansions are also discussed.

        Different options for the numerical implementation of these schemes are considered.

        Due to the efficacy of FET in the development of SP timesteppers for both the fluid and kinetic component, we hope this approach will prove effective in the future for developing SP timesteppers for the full hybrid model. We hope this will give us the opportunity to incorporate previously inaccessible kinetic effects into the highly effective, modern, finite-element MHD models.
    \end{abstract}
    
    
    \newpage
    \tableofcontents
    
    
    \newpage
    \pagenumbering{arabic}
    %\linenumbers\renewcommand\thelinenumber{\color{black!50}\arabic{linenumber}}
            \input{0 - introduction/main.tex}
        \part{Research}
            \input{1 - low-noise PiC models/main.tex}
            \input{2 - kinetic component/main.tex}
            \input{3 - fluid component/main.tex}
            \input{4 - numerical implementation/main.tex}
        \part{Project Overview}
            \input{5 - research plan/main.tex}
            \input{6 - summary/main.tex}
    
    
    %\section{}
    \newpage
    \pagenumbering{gobble}
        \printbibliography


    \newpage
    \pagenumbering{roman}
    \appendix
        \part{Appendices}
            \input{8 - Hilbert complexes/main.tex}
            \input{9 - weak conservation proofs/main.tex}
\end{document}

            \documentclass[12pt, a4paper]{report}

\input{template/main.tex}

\title{\BA{Title in Progress...}}
\author{Boris Andrews}
\affil{Mathematical Institute, University of Oxford}
\date{\today}


\begin{document}
    \pagenumbering{gobble}
    \maketitle
    
    
    \begin{abstract}
        Magnetic confinement reactors---in particular tokamaks---offer one of the most promising options for achieving practical nuclear fusion, with the potential to provide virtually limitless, clean energy. The theoretical and numerical modeling of tokamak plasmas is simultaneously an essential component of effective reactor design, and a great research barrier. Tokamak operational conditions exhibit comparatively low Knudsen numbers. Kinetic effects, including kinetic waves and instabilities, Landau damping, bump-on-tail instabilities and more, are therefore highly influential in tokamak plasma dynamics. Purely fluid models are inherently incapable of capturing these effects, whereas the high dimensionality in purely kinetic models render them practically intractable for most relevant purposes.

        We consider a $\delta\!f$ decomposition model, with a macroscopic fluid background and microscopic kinetic correction, both fully coupled to each other. A similar manner of discretization is proposed to that used in the recent \texttt{STRUPHY} code \cite{Holderied_Possanner_Wang_2021, Holderied_2022, Li_et_al_2023} with a finite-element model for the background and a pseudo-particle/PiC model for the correction.

        The fluid background satisfies the full, non-linear, resistive, compressible, Hall MHD equations. \cite{Laakmann_Hu_Farrell_2022} introduces finite-element(-in-space) implicit timesteppers for the incompressible analogue to this system with structure-preserving (SP) properties in the ideal case, alongside parameter-robust preconditioners. We show that these timesteppers can derive from a finite-element-in-time (FET) (and finite-element-in-space) interpretation. The benefits of this reformulation are discussed, including the derivation of timesteppers that are higher order in time, and the quantifiable dissipative SP properties in the non-ideal, resistive case.
        
        We discuss possible options for extending this FET approach to timesteppers for the compressible case.

        The kinetic corrections satisfy linearized Boltzmann equations. Using a Lénard--Bernstein collision operator, these take Fokker--Planck-like forms \cite{Fokker_1914, Planck_1917} wherein pseudo-particles in the numerical model obey the neoclassical transport equations, with particle-independent Brownian drift terms. This offers a rigorous methodology for incorporating collisions into the particle transport model, without coupling the equations of motions for each particle.
        
        Works by Chen, Chacón et al. \cite{Chen_Chacón_Barnes_2011, Chacón_Chen_Barnes_2013, Chen_Chacón_2014, Chen_Chacón_2015} have developed structure-preserving particle pushers for neoclassical transport in the Vlasov equations, derived from Crank--Nicolson integrators. We show these too can can derive from a FET interpretation, similarly offering potential extensions to higher-order-in-time particle pushers. The FET formulation is used also to consider how the stochastic drift terms can be incorporated into the pushers. Stochastic gyrokinetic expansions are also discussed.

        Different options for the numerical implementation of these schemes are considered.

        Due to the efficacy of FET in the development of SP timesteppers for both the fluid and kinetic component, we hope this approach will prove effective in the future for developing SP timesteppers for the full hybrid model. We hope this will give us the opportunity to incorporate previously inaccessible kinetic effects into the highly effective, modern, finite-element MHD models.
    \end{abstract}
    
    
    \newpage
    \tableofcontents
    
    
    \newpage
    \pagenumbering{arabic}
    %\linenumbers\renewcommand\thelinenumber{\color{black!50}\arabic{linenumber}}
            \input{0 - introduction/main.tex}
        \part{Research}
            \input{1 - low-noise PiC models/main.tex}
            \input{2 - kinetic component/main.tex}
            \input{3 - fluid component/main.tex}
            \input{4 - numerical implementation/main.tex}
        \part{Project Overview}
            \input{5 - research plan/main.tex}
            \input{6 - summary/main.tex}
    
    
    %\section{}
    \newpage
    \pagenumbering{gobble}
        \printbibliography


    \newpage
    \pagenumbering{roman}
    \appendix
        \part{Appendices}
            \input{8 - Hilbert complexes/main.tex}
            \input{9 - weak conservation proofs/main.tex}
\end{document}

\end{document}

            \documentclass[12pt, a4paper]{report}

\documentclass[12pt, a4paper]{report}

\input{template/main.tex}

\title{\BA{Title in Progress...}}
\author{Boris Andrews}
\affil{Mathematical Institute, University of Oxford}
\date{\today}


\begin{document}
    \pagenumbering{gobble}
    \maketitle
    
    
    \begin{abstract}
        Magnetic confinement reactors---in particular tokamaks---offer one of the most promising options for achieving practical nuclear fusion, with the potential to provide virtually limitless, clean energy. The theoretical and numerical modeling of tokamak plasmas is simultaneously an essential component of effective reactor design, and a great research barrier. Tokamak operational conditions exhibit comparatively low Knudsen numbers. Kinetic effects, including kinetic waves and instabilities, Landau damping, bump-on-tail instabilities and more, are therefore highly influential in tokamak plasma dynamics. Purely fluid models are inherently incapable of capturing these effects, whereas the high dimensionality in purely kinetic models render them practically intractable for most relevant purposes.

        We consider a $\delta\!f$ decomposition model, with a macroscopic fluid background and microscopic kinetic correction, both fully coupled to each other. A similar manner of discretization is proposed to that used in the recent \texttt{STRUPHY} code \cite{Holderied_Possanner_Wang_2021, Holderied_2022, Li_et_al_2023} with a finite-element model for the background and a pseudo-particle/PiC model for the correction.

        The fluid background satisfies the full, non-linear, resistive, compressible, Hall MHD equations. \cite{Laakmann_Hu_Farrell_2022} introduces finite-element(-in-space) implicit timesteppers for the incompressible analogue to this system with structure-preserving (SP) properties in the ideal case, alongside parameter-robust preconditioners. We show that these timesteppers can derive from a finite-element-in-time (FET) (and finite-element-in-space) interpretation. The benefits of this reformulation are discussed, including the derivation of timesteppers that are higher order in time, and the quantifiable dissipative SP properties in the non-ideal, resistive case.
        
        We discuss possible options for extending this FET approach to timesteppers for the compressible case.

        The kinetic corrections satisfy linearized Boltzmann equations. Using a Lénard--Bernstein collision operator, these take Fokker--Planck-like forms \cite{Fokker_1914, Planck_1917} wherein pseudo-particles in the numerical model obey the neoclassical transport equations, with particle-independent Brownian drift terms. This offers a rigorous methodology for incorporating collisions into the particle transport model, without coupling the equations of motions for each particle.
        
        Works by Chen, Chacón et al. \cite{Chen_Chacón_Barnes_2011, Chacón_Chen_Barnes_2013, Chen_Chacón_2014, Chen_Chacón_2015} have developed structure-preserving particle pushers for neoclassical transport in the Vlasov equations, derived from Crank--Nicolson integrators. We show these too can can derive from a FET interpretation, similarly offering potential extensions to higher-order-in-time particle pushers. The FET formulation is used also to consider how the stochastic drift terms can be incorporated into the pushers. Stochastic gyrokinetic expansions are also discussed.

        Different options for the numerical implementation of these schemes are considered.

        Due to the efficacy of FET in the development of SP timesteppers for both the fluid and kinetic component, we hope this approach will prove effective in the future for developing SP timesteppers for the full hybrid model. We hope this will give us the opportunity to incorporate previously inaccessible kinetic effects into the highly effective, modern, finite-element MHD models.
    \end{abstract}
    
    
    \newpage
    \tableofcontents
    
    
    \newpage
    \pagenumbering{arabic}
    %\linenumbers\renewcommand\thelinenumber{\color{black!50}\arabic{linenumber}}
            \input{0 - introduction/main.tex}
        \part{Research}
            \input{1 - low-noise PiC models/main.tex}
            \input{2 - kinetic component/main.tex}
            \input{3 - fluid component/main.tex}
            \input{4 - numerical implementation/main.tex}
        \part{Project Overview}
            \input{5 - research plan/main.tex}
            \input{6 - summary/main.tex}
    
    
    %\section{}
    \newpage
    \pagenumbering{gobble}
        \printbibliography


    \newpage
    \pagenumbering{roman}
    \appendix
        \part{Appendices}
            \input{8 - Hilbert complexes/main.tex}
            \input{9 - weak conservation proofs/main.tex}
\end{document}


\title{\BA{Title in Progress...}}
\author{Boris Andrews}
\affil{Mathematical Institute, University of Oxford}
\date{\today}


\begin{document}
    \pagenumbering{gobble}
    \maketitle
    
    
    \begin{abstract}
        Magnetic confinement reactors---in particular tokamaks---offer one of the most promising options for achieving practical nuclear fusion, with the potential to provide virtually limitless, clean energy. The theoretical and numerical modeling of tokamak plasmas is simultaneously an essential component of effective reactor design, and a great research barrier. Tokamak operational conditions exhibit comparatively low Knudsen numbers. Kinetic effects, including kinetic waves and instabilities, Landau damping, bump-on-tail instabilities and more, are therefore highly influential in tokamak plasma dynamics. Purely fluid models are inherently incapable of capturing these effects, whereas the high dimensionality in purely kinetic models render them practically intractable for most relevant purposes.

        We consider a $\delta\!f$ decomposition model, with a macroscopic fluid background and microscopic kinetic correction, both fully coupled to each other. A similar manner of discretization is proposed to that used in the recent \texttt{STRUPHY} code \cite{Holderied_Possanner_Wang_2021, Holderied_2022, Li_et_al_2023} with a finite-element model for the background and a pseudo-particle/PiC model for the correction.

        The fluid background satisfies the full, non-linear, resistive, compressible, Hall MHD equations. \cite{Laakmann_Hu_Farrell_2022} introduces finite-element(-in-space) implicit timesteppers for the incompressible analogue to this system with structure-preserving (SP) properties in the ideal case, alongside parameter-robust preconditioners. We show that these timesteppers can derive from a finite-element-in-time (FET) (and finite-element-in-space) interpretation. The benefits of this reformulation are discussed, including the derivation of timesteppers that are higher order in time, and the quantifiable dissipative SP properties in the non-ideal, resistive case.
        
        We discuss possible options for extending this FET approach to timesteppers for the compressible case.

        The kinetic corrections satisfy linearized Boltzmann equations. Using a Lénard--Bernstein collision operator, these take Fokker--Planck-like forms \cite{Fokker_1914, Planck_1917} wherein pseudo-particles in the numerical model obey the neoclassical transport equations, with particle-independent Brownian drift terms. This offers a rigorous methodology for incorporating collisions into the particle transport model, without coupling the equations of motions for each particle.
        
        Works by Chen, Chacón et al. \cite{Chen_Chacón_Barnes_2011, Chacón_Chen_Barnes_2013, Chen_Chacón_2014, Chen_Chacón_2015} have developed structure-preserving particle pushers for neoclassical transport in the Vlasov equations, derived from Crank--Nicolson integrators. We show these too can can derive from a FET interpretation, similarly offering potential extensions to higher-order-in-time particle pushers. The FET formulation is used also to consider how the stochastic drift terms can be incorporated into the pushers. Stochastic gyrokinetic expansions are also discussed.

        Different options for the numerical implementation of these schemes are considered.

        Due to the efficacy of FET in the development of SP timesteppers for both the fluid and kinetic component, we hope this approach will prove effective in the future for developing SP timesteppers for the full hybrid model. We hope this will give us the opportunity to incorporate previously inaccessible kinetic effects into the highly effective, modern, finite-element MHD models.
    \end{abstract}
    
    
    \newpage
    \tableofcontents
    
    
    \newpage
    \pagenumbering{arabic}
    %\linenumbers\renewcommand\thelinenumber{\color{black!50}\arabic{linenumber}}
            \documentclass[12pt, a4paper]{report}

\input{template/main.tex}

\title{\BA{Title in Progress...}}
\author{Boris Andrews}
\affil{Mathematical Institute, University of Oxford}
\date{\today}


\begin{document}
    \pagenumbering{gobble}
    \maketitle
    
    
    \begin{abstract}
        Magnetic confinement reactors---in particular tokamaks---offer one of the most promising options for achieving practical nuclear fusion, with the potential to provide virtually limitless, clean energy. The theoretical and numerical modeling of tokamak plasmas is simultaneously an essential component of effective reactor design, and a great research barrier. Tokamak operational conditions exhibit comparatively low Knudsen numbers. Kinetic effects, including kinetic waves and instabilities, Landau damping, bump-on-tail instabilities and more, are therefore highly influential in tokamak plasma dynamics. Purely fluid models are inherently incapable of capturing these effects, whereas the high dimensionality in purely kinetic models render them practically intractable for most relevant purposes.

        We consider a $\delta\!f$ decomposition model, with a macroscopic fluid background and microscopic kinetic correction, both fully coupled to each other. A similar manner of discretization is proposed to that used in the recent \texttt{STRUPHY} code \cite{Holderied_Possanner_Wang_2021, Holderied_2022, Li_et_al_2023} with a finite-element model for the background and a pseudo-particle/PiC model for the correction.

        The fluid background satisfies the full, non-linear, resistive, compressible, Hall MHD equations. \cite{Laakmann_Hu_Farrell_2022} introduces finite-element(-in-space) implicit timesteppers for the incompressible analogue to this system with structure-preserving (SP) properties in the ideal case, alongside parameter-robust preconditioners. We show that these timesteppers can derive from a finite-element-in-time (FET) (and finite-element-in-space) interpretation. The benefits of this reformulation are discussed, including the derivation of timesteppers that are higher order in time, and the quantifiable dissipative SP properties in the non-ideal, resistive case.
        
        We discuss possible options for extending this FET approach to timesteppers for the compressible case.

        The kinetic corrections satisfy linearized Boltzmann equations. Using a Lénard--Bernstein collision operator, these take Fokker--Planck-like forms \cite{Fokker_1914, Planck_1917} wherein pseudo-particles in the numerical model obey the neoclassical transport equations, with particle-independent Brownian drift terms. This offers a rigorous methodology for incorporating collisions into the particle transport model, without coupling the equations of motions for each particle.
        
        Works by Chen, Chacón et al. \cite{Chen_Chacón_Barnes_2011, Chacón_Chen_Barnes_2013, Chen_Chacón_2014, Chen_Chacón_2015} have developed structure-preserving particle pushers for neoclassical transport in the Vlasov equations, derived from Crank--Nicolson integrators. We show these too can can derive from a FET interpretation, similarly offering potential extensions to higher-order-in-time particle pushers. The FET formulation is used also to consider how the stochastic drift terms can be incorporated into the pushers. Stochastic gyrokinetic expansions are also discussed.

        Different options for the numerical implementation of these schemes are considered.

        Due to the efficacy of FET in the development of SP timesteppers for both the fluid and kinetic component, we hope this approach will prove effective in the future for developing SP timesteppers for the full hybrid model. We hope this will give us the opportunity to incorporate previously inaccessible kinetic effects into the highly effective, modern, finite-element MHD models.
    \end{abstract}
    
    
    \newpage
    \tableofcontents
    
    
    \newpage
    \pagenumbering{arabic}
    %\linenumbers\renewcommand\thelinenumber{\color{black!50}\arabic{linenumber}}
            \input{0 - introduction/main.tex}
        \part{Research}
            \input{1 - low-noise PiC models/main.tex}
            \input{2 - kinetic component/main.tex}
            \input{3 - fluid component/main.tex}
            \input{4 - numerical implementation/main.tex}
        \part{Project Overview}
            \input{5 - research plan/main.tex}
            \input{6 - summary/main.tex}
    
    
    %\section{}
    \newpage
    \pagenumbering{gobble}
        \printbibliography


    \newpage
    \pagenumbering{roman}
    \appendix
        \part{Appendices}
            \input{8 - Hilbert complexes/main.tex}
            \input{9 - weak conservation proofs/main.tex}
\end{document}

        \part{Research}
            \documentclass[12pt, a4paper]{report}

\input{template/main.tex}

\title{\BA{Title in Progress...}}
\author{Boris Andrews}
\affil{Mathematical Institute, University of Oxford}
\date{\today}


\begin{document}
    \pagenumbering{gobble}
    \maketitle
    
    
    \begin{abstract}
        Magnetic confinement reactors---in particular tokamaks---offer one of the most promising options for achieving practical nuclear fusion, with the potential to provide virtually limitless, clean energy. The theoretical and numerical modeling of tokamak plasmas is simultaneously an essential component of effective reactor design, and a great research barrier. Tokamak operational conditions exhibit comparatively low Knudsen numbers. Kinetic effects, including kinetic waves and instabilities, Landau damping, bump-on-tail instabilities and more, are therefore highly influential in tokamak plasma dynamics. Purely fluid models are inherently incapable of capturing these effects, whereas the high dimensionality in purely kinetic models render them practically intractable for most relevant purposes.

        We consider a $\delta\!f$ decomposition model, with a macroscopic fluid background and microscopic kinetic correction, both fully coupled to each other. A similar manner of discretization is proposed to that used in the recent \texttt{STRUPHY} code \cite{Holderied_Possanner_Wang_2021, Holderied_2022, Li_et_al_2023} with a finite-element model for the background and a pseudo-particle/PiC model for the correction.

        The fluid background satisfies the full, non-linear, resistive, compressible, Hall MHD equations. \cite{Laakmann_Hu_Farrell_2022} introduces finite-element(-in-space) implicit timesteppers for the incompressible analogue to this system with structure-preserving (SP) properties in the ideal case, alongside parameter-robust preconditioners. We show that these timesteppers can derive from a finite-element-in-time (FET) (and finite-element-in-space) interpretation. The benefits of this reformulation are discussed, including the derivation of timesteppers that are higher order in time, and the quantifiable dissipative SP properties in the non-ideal, resistive case.
        
        We discuss possible options for extending this FET approach to timesteppers for the compressible case.

        The kinetic corrections satisfy linearized Boltzmann equations. Using a Lénard--Bernstein collision operator, these take Fokker--Planck-like forms \cite{Fokker_1914, Planck_1917} wherein pseudo-particles in the numerical model obey the neoclassical transport equations, with particle-independent Brownian drift terms. This offers a rigorous methodology for incorporating collisions into the particle transport model, without coupling the equations of motions for each particle.
        
        Works by Chen, Chacón et al. \cite{Chen_Chacón_Barnes_2011, Chacón_Chen_Barnes_2013, Chen_Chacón_2014, Chen_Chacón_2015} have developed structure-preserving particle pushers for neoclassical transport in the Vlasov equations, derived from Crank--Nicolson integrators. We show these too can can derive from a FET interpretation, similarly offering potential extensions to higher-order-in-time particle pushers. The FET formulation is used also to consider how the stochastic drift terms can be incorporated into the pushers. Stochastic gyrokinetic expansions are also discussed.

        Different options for the numerical implementation of these schemes are considered.

        Due to the efficacy of FET in the development of SP timesteppers for both the fluid and kinetic component, we hope this approach will prove effective in the future for developing SP timesteppers for the full hybrid model. We hope this will give us the opportunity to incorporate previously inaccessible kinetic effects into the highly effective, modern, finite-element MHD models.
    \end{abstract}
    
    
    \newpage
    \tableofcontents
    
    
    \newpage
    \pagenumbering{arabic}
    %\linenumbers\renewcommand\thelinenumber{\color{black!50}\arabic{linenumber}}
            \input{0 - introduction/main.tex}
        \part{Research}
            \input{1 - low-noise PiC models/main.tex}
            \input{2 - kinetic component/main.tex}
            \input{3 - fluid component/main.tex}
            \input{4 - numerical implementation/main.tex}
        \part{Project Overview}
            \input{5 - research plan/main.tex}
            \input{6 - summary/main.tex}
    
    
    %\section{}
    \newpage
    \pagenumbering{gobble}
        \printbibliography


    \newpage
    \pagenumbering{roman}
    \appendix
        \part{Appendices}
            \input{8 - Hilbert complexes/main.tex}
            \input{9 - weak conservation proofs/main.tex}
\end{document}

            \documentclass[12pt, a4paper]{report}

\input{template/main.tex}

\title{\BA{Title in Progress...}}
\author{Boris Andrews}
\affil{Mathematical Institute, University of Oxford}
\date{\today}


\begin{document}
    \pagenumbering{gobble}
    \maketitle
    
    
    \begin{abstract}
        Magnetic confinement reactors---in particular tokamaks---offer one of the most promising options for achieving practical nuclear fusion, with the potential to provide virtually limitless, clean energy. The theoretical and numerical modeling of tokamak plasmas is simultaneously an essential component of effective reactor design, and a great research barrier. Tokamak operational conditions exhibit comparatively low Knudsen numbers. Kinetic effects, including kinetic waves and instabilities, Landau damping, bump-on-tail instabilities and more, are therefore highly influential in tokamak plasma dynamics. Purely fluid models are inherently incapable of capturing these effects, whereas the high dimensionality in purely kinetic models render them practically intractable for most relevant purposes.

        We consider a $\delta\!f$ decomposition model, with a macroscopic fluid background and microscopic kinetic correction, both fully coupled to each other. A similar manner of discretization is proposed to that used in the recent \texttt{STRUPHY} code \cite{Holderied_Possanner_Wang_2021, Holderied_2022, Li_et_al_2023} with a finite-element model for the background and a pseudo-particle/PiC model for the correction.

        The fluid background satisfies the full, non-linear, resistive, compressible, Hall MHD equations. \cite{Laakmann_Hu_Farrell_2022} introduces finite-element(-in-space) implicit timesteppers for the incompressible analogue to this system with structure-preserving (SP) properties in the ideal case, alongside parameter-robust preconditioners. We show that these timesteppers can derive from a finite-element-in-time (FET) (and finite-element-in-space) interpretation. The benefits of this reformulation are discussed, including the derivation of timesteppers that are higher order in time, and the quantifiable dissipative SP properties in the non-ideal, resistive case.
        
        We discuss possible options for extending this FET approach to timesteppers for the compressible case.

        The kinetic corrections satisfy linearized Boltzmann equations. Using a Lénard--Bernstein collision operator, these take Fokker--Planck-like forms \cite{Fokker_1914, Planck_1917} wherein pseudo-particles in the numerical model obey the neoclassical transport equations, with particle-independent Brownian drift terms. This offers a rigorous methodology for incorporating collisions into the particle transport model, without coupling the equations of motions for each particle.
        
        Works by Chen, Chacón et al. \cite{Chen_Chacón_Barnes_2011, Chacón_Chen_Barnes_2013, Chen_Chacón_2014, Chen_Chacón_2015} have developed structure-preserving particle pushers for neoclassical transport in the Vlasov equations, derived from Crank--Nicolson integrators. We show these too can can derive from a FET interpretation, similarly offering potential extensions to higher-order-in-time particle pushers. The FET formulation is used also to consider how the stochastic drift terms can be incorporated into the pushers. Stochastic gyrokinetic expansions are also discussed.

        Different options for the numerical implementation of these schemes are considered.

        Due to the efficacy of FET in the development of SP timesteppers for both the fluid and kinetic component, we hope this approach will prove effective in the future for developing SP timesteppers for the full hybrid model. We hope this will give us the opportunity to incorporate previously inaccessible kinetic effects into the highly effective, modern, finite-element MHD models.
    \end{abstract}
    
    
    \newpage
    \tableofcontents
    
    
    \newpage
    \pagenumbering{arabic}
    %\linenumbers\renewcommand\thelinenumber{\color{black!50}\arabic{linenumber}}
            \input{0 - introduction/main.tex}
        \part{Research}
            \input{1 - low-noise PiC models/main.tex}
            \input{2 - kinetic component/main.tex}
            \input{3 - fluid component/main.tex}
            \input{4 - numerical implementation/main.tex}
        \part{Project Overview}
            \input{5 - research plan/main.tex}
            \input{6 - summary/main.tex}
    
    
    %\section{}
    \newpage
    \pagenumbering{gobble}
        \printbibliography


    \newpage
    \pagenumbering{roman}
    \appendix
        \part{Appendices}
            \input{8 - Hilbert complexes/main.tex}
            \input{9 - weak conservation proofs/main.tex}
\end{document}

            \documentclass[12pt, a4paper]{report}

\input{template/main.tex}

\title{\BA{Title in Progress...}}
\author{Boris Andrews}
\affil{Mathematical Institute, University of Oxford}
\date{\today}


\begin{document}
    \pagenumbering{gobble}
    \maketitle
    
    
    \begin{abstract}
        Magnetic confinement reactors---in particular tokamaks---offer one of the most promising options for achieving practical nuclear fusion, with the potential to provide virtually limitless, clean energy. The theoretical and numerical modeling of tokamak plasmas is simultaneously an essential component of effective reactor design, and a great research barrier. Tokamak operational conditions exhibit comparatively low Knudsen numbers. Kinetic effects, including kinetic waves and instabilities, Landau damping, bump-on-tail instabilities and more, are therefore highly influential in tokamak plasma dynamics. Purely fluid models are inherently incapable of capturing these effects, whereas the high dimensionality in purely kinetic models render them practically intractable for most relevant purposes.

        We consider a $\delta\!f$ decomposition model, with a macroscopic fluid background and microscopic kinetic correction, both fully coupled to each other. A similar manner of discretization is proposed to that used in the recent \texttt{STRUPHY} code \cite{Holderied_Possanner_Wang_2021, Holderied_2022, Li_et_al_2023} with a finite-element model for the background and a pseudo-particle/PiC model for the correction.

        The fluid background satisfies the full, non-linear, resistive, compressible, Hall MHD equations. \cite{Laakmann_Hu_Farrell_2022} introduces finite-element(-in-space) implicit timesteppers for the incompressible analogue to this system with structure-preserving (SP) properties in the ideal case, alongside parameter-robust preconditioners. We show that these timesteppers can derive from a finite-element-in-time (FET) (and finite-element-in-space) interpretation. The benefits of this reformulation are discussed, including the derivation of timesteppers that are higher order in time, and the quantifiable dissipative SP properties in the non-ideal, resistive case.
        
        We discuss possible options for extending this FET approach to timesteppers for the compressible case.

        The kinetic corrections satisfy linearized Boltzmann equations. Using a Lénard--Bernstein collision operator, these take Fokker--Planck-like forms \cite{Fokker_1914, Planck_1917} wherein pseudo-particles in the numerical model obey the neoclassical transport equations, with particle-independent Brownian drift terms. This offers a rigorous methodology for incorporating collisions into the particle transport model, without coupling the equations of motions for each particle.
        
        Works by Chen, Chacón et al. \cite{Chen_Chacón_Barnes_2011, Chacón_Chen_Barnes_2013, Chen_Chacón_2014, Chen_Chacón_2015} have developed structure-preserving particle pushers for neoclassical transport in the Vlasov equations, derived from Crank--Nicolson integrators. We show these too can can derive from a FET interpretation, similarly offering potential extensions to higher-order-in-time particle pushers. The FET formulation is used also to consider how the stochastic drift terms can be incorporated into the pushers. Stochastic gyrokinetic expansions are also discussed.

        Different options for the numerical implementation of these schemes are considered.

        Due to the efficacy of FET in the development of SP timesteppers for both the fluid and kinetic component, we hope this approach will prove effective in the future for developing SP timesteppers for the full hybrid model. We hope this will give us the opportunity to incorporate previously inaccessible kinetic effects into the highly effective, modern, finite-element MHD models.
    \end{abstract}
    
    
    \newpage
    \tableofcontents
    
    
    \newpage
    \pagenumbering{arabic}
    %\linenumbers\renewcommand\thelinenumber{\color{black!50}\arabic{linenumber}}
            \input{0 - introduction/main.tex}
        \part{Research}
            \input{1 - low-noise PiC models/main.tex}
            \input{2 - kinetic component/main.tex}
            \input{3 - fluid component/main.tex}
            \input{4 - numerical implementation/main.tex}
        \part{Project Overview}
            \input{5 - research plan/main.tex}
            \input{6 - summary/main.tex}
    
    
    %\section{}
    \newpage
    \pagenumbering{gobble}
        \printbibliography


    \newpage
    \pagenumbering{roman}
    \appendix
        \part{Appendices}
            \input{8 - Hilbert complexes/main.tex}
            \input{9 - weak conservation proofs/main.tex}
\end{document}

            \documentclass[12pt, a4paper]{report}

\input{template/main.tex}

\title{\BA{Title in Progress...}}
\author{Boris Andrews}
\affil{Mathematical Institute, University of Oxford}
\date{\today}


\begin{document}
    \pagenumbering{gobble}
    \maketitle
    
    
    \begin{abstract}
        Magnetic confinement reactors---in particular tokamaks---offer one of the most promising options for achieving practical nuclear fusion, with the potential to provide virtually limitless, clean energy. The theoretical and numerical modeling of tokamak plasmas is simultaneously an essential component of effective reactor design, and a great research barrier. Tokamak operational conditions exhibit comparatively low Knudsen numbers. Kinetic effects, including kinetic waves and instabilities, Landau damping, bump-on-tail instabilities and more, are therefore highly influential in tokamak plasma dynamics. Purely fluid models are inherently incapable of capturing these effects, whereas the high dimensionality in purely kinetic models render them practically intractable for most relevant purposes.

        We consider a $\delta\!f$ decomposition model, with a macroscopic fluid background and microscopic kinetic correction, both fully coupled to each other. A similar manner of discretization is proposed to that used in the recent \texttt{STRUPHY} code \cite{Holderied_Possanner_Wang_2021, Holderied_2022, Li_et_al_2023} with a finite-element model for the background and a pseudo-particle/PiC model for the correction.

        The fluid background satisfies the full, non-linear, resistive, compressible, Hall MHD equations. \cite{Laakmann_Hu_Farrell_2022} introduces finite-element(-in-space) implicit timesteppers for the incompressible analogue to this system with structure-preserving (SP) properties in the ideal case, alongside parameter-robust preconditioners. We show that these timesteppers can derive from a finite-element-in-time (FET) (and finite-element-in-space) interpretation. The benefits of this reformulation are discussed, including the derivation of timesteppers that are higher order in time, and the quantifiable dissipative SP properties in the non-ideal, resistive case.
        
        We discuss possible options for extending this FET approach to timesteppers for the compressible case.

        The kinetic corrections satisfy linearized Boltzmann equations. Using a Lénard--Bernstein collision operator, these take Fokker--Planck-like forms \cite{Fokker_1914, Planck_1917} wherein pseudo-particles in the numerical model obey the neoclassical transport equations, with particle-independent Brownian drift terms. This offers a rigorous methodology for incorporating collisions into the particle transport model, without coupling the equations of motions for each particle.
        
        Works by Chen, Chacón et al. \cite{Chen_Chacón_Barnes_2011, Chacón_Chen_Barnes_2013, Chen_Chacón_2014, Chen_Chacón_2015} have developed structure-preserving particle pushers for neoclassical transport in the Vlasov equations, derived from Crank--Nicolson integrators. We show these too can can derive from a FET interpretation, similarly offering potential extensions to higher-order-in-time particle pushers. The FET formulation is used also to consider how the stochastic drift terms can be incorporated into the pushers. Stochastic gyrokinetic expansions are also discussed.

        Different options for the numerical implementation of these schemes are considered.

        Due to the efficacy of FET in the development of SP timesteppers for both the fluid and kinetic component, we hope this approach will prove effective in the future for developing SP timesteppers for the full hybrid model. We hope this will give us the opportunity to incorporate previously inaccessible kinetic effects into the highly effective, modern, finite-element MHD models.
    \end{abstract}
    
    
    \newpage
    \tableofcontents
    
    
    \newpage
    \pagenumbering{arabic}
    %\linenumbers\renewcommand\thelinenumber{\color{black!50}\arabic{linenumber}}
            \input{0 - introduction/main.tex}
        \part{Research}
            \input{1 - low-noise PiC models/main.tex}
            \input{2 - kinetic component/main.tex}
            \input{3 - fluid component/main.tex}
            \input{4 - numerical implementation/main.tex}
        \part{Project Overview}
            \input{5 - research plan/main.tex}
            \input{6 - summary/main.tex}
    
    
    %\section{}
    \newpage
    \pagenumbering{gobble}
        \printbibliography


    \newpage
    \pagenumbering{roman}
    \appendix
        \part{Appendices}
            \input{8 - Hilbert complexes/main.tex}
            \input{9 - weak conservation proofs/main.tex}
\end{document}

        \part{Project Overview}
            \documentclass[12pt, a4paper]{report}

\input{template/main.tex}

\title{\BA{Title in Progress...}}
\author{Boris Andrews}
\affil{Mathematical Institute, University of Oxford}
\date{\today}


\begin{document}
    \pagenumbering{gobble}
    \maketitle
    
    
    \begin{abstract}
        Magnetic confinement reactors---in particular tokamaks---offer one of the most promising options for achieving practical nuclear fusion, with the potential to provide virtually limitless, clean energy. The theoretical and numerical modeling of tokamak plasmas is simultaneously an essential component of effective reactor design, and a great research barrier. Tokamak operational conditions exhibit comparatively low Knudsen numbers. Kinetic effects, including kinetic waves and instabilities, Landau damping, bump-on-tail instabilities and more, are therefore highly influential in tokamak plasma dynamics. Purely fluid models are inherently incapable of capturing these effects, whereas the high dimensionality in purely kinetic models render them practically intractable for most relevant purposes.

        We consider a $\delta\!f$ decomposition model, with a macroscopic fluid background and microscopic kinetic correction, both fully coupled to each other. A similar manner of discretization is proposed to that used in the recent \texttt{STRUPHY} code \cite{Holderied_Possanner_Wang_2021, Holderied_2022, Li_et_al_2023} with a finite-element model for the background and a pseudo-particle/PiC model for the correction.

        The fluid background satisfies the full, non-linear, resistive, compressible, Hall MHD equations. \cite{Laakmann_Hu_Farrell_2022} introduces finite-element(-in-space) implicit timesteppers for the incompressible analogue to this system with structure-preserving (SP) properties in the ideal case, alongside parameter-robust preconditioners. We show that these timesteppers can derive from a finite-element-in-time (FET) (and finite-element-in-space) interpretation. The benefits of this reformulation are discussed, including the derivation of timesteppers that are higher order in time, and the quantifiable dissipative SP properties in the non-ideal, resistive case.
        
        We discuss possible options for extending this FET approach to timesteppers for the compressible case.

        The kinetic corrections satisfy linearized Boltzmann equations. Using a Lénard--Bernstein collision operator, these take Fokker--Planck-like forms \cite{Fokker_1914, Planck_1917} wherein pseudo-particles in the numerical model obey the neoclassical transport equations, with particle-independent Brownian drift terms. This offers a rigorous methodology for incorporating collisions into the particle transport model, without coupling the equations of motions for each particle.
        
        Works by Chen, Chacón et al. \cite{Chen_Chacón_Barnes_2011, Chacón_Chen_Barnes_2013, Chen_Chacón_2014, Chen_Chacón_2015} have developed structure-preserving particle pushers for neoclassical transport in the Vlasov equations, derived from Crank--Nicolson integrators. We show these too can can derive from a FET interpretation, similarly offering potential extensions to higher-order-in-time particle pushers. The FET formulation is used also to consider how the stochastic drift terms can be incorporated into the pushers. Stochastic gyrokinetic expansions are also discussed.

        Different options for the numerical implementation of these schemes are considered.

        Due to the efficacy of FET in the development of SP timesteppers for both the fluid and kinetic component, we hope this approach will prove effective in the future for developing SP timesteppers for the full hybrid model. We hope this will give us the opportunity to incorporate previously inaccessible kinetic effects into the highly effective, modern, finite-element MHD models.
    \end{abstract}
    
    
    \newpage
    \tableofcontents
    
    
    \newpage
    \pagenumbering{arabic}
    %\linenumbers\renewcommand\thelinenumber{\color{black!50}\arabic{linenumber}}
            \input{0 - introduction/main.tex}
        \part{Research}
            \input{1 - low-noise PiC models/main.tex}
            \input{2 - kinetic component/main.tex}
            \input{3 - fluid component/main.tex}
            \input{4 - numerical implementation/main.tex}
        \part{Project Overview}
            \input{5 - research plan/main.tex}
            \input{6 - summary/main.tex}
    
    
    %\section{}
    \newpage
    \pagenumbering{gobble}
        \printbibliography


    \newpage
    \pagenumbering{roman}
    \appendix
        \part{Appendices}
            \input{8 - Hilbert complexes/main.tex}
            \input{9 - weak conservation proofs/main.tex}
\end{document}

            \documentclass[12pt, a4paper]{report}

\input{template/main.tex}

\title{\BA{Title in Progress...}}
\author{Boris Andrews}
\affil{Mathematical Institute, University of Oxford}
\date{\today}


\begin{document}
    \pagenumbering{gobble}
    \maketitle
    
    
    \begin{abstract}
        Magnetic confinement reactors---in particular tokamaks---offer one of the most promising options for achieving practical nuclear fusion, with the potential to provide virtually limitless, clean energy. The theoretical and numerical modeling of tokamak plasmas is simultaneously an essential component of effective reactor design, and a great research barrier. Tokamak operational conditions exhibit comparatively low Knudsen numbers. Kinetic effects, including kinetic waves and instabilities, Landau damping, bump-on-tail instabilities and more, are therefore highly influential in tokamak plasma dynamics. Purely fluid models are inherently incapable of capturing these effects, whereas the high dimensionality in purely kinetic models render them practically intractable for most relevant purposes.

        We consider a $\delta\!f$ decomposition model, with a macroscopic fluid background and microscopic kinetic correction, both fully coupled to each other. A similar manner of discretization is proposed to that used in the recent \texttt{STRUPHY} code \cite{Holderied_Possanner_Wang_2021, Holderied_2022, Li_et_al_2023} with a finite-element model for the background and a pseudo-particle/PiC model for the correction.

        The fluid background satisfies the full, non-linear, resistive, compressible, Hall MHD equations. \cite{Laakmann_Hu_Farrell_2022} introduces finite-element(-in-space) implicit timesteppers for the incompressible analogue to this system with structure-preserving (SP) properties in the ideal case, alongside parameter-robust preconditioners. We show that these timesteppers can derive from a finite-element-in-time (FET) (and finite-element-in-space) interpretation. The benefits of this reformulation are discussed, including the derivation of timesteppers that are higher order in time, and the quantifiable dissipative SP properties in the non-ideal, resistive case.
        
        We discuss possible options for extending this FET approach to timesteppers for the compressible case.

        The kinetic corrections satisfy linearized Boltzmann equations. Using a Lénard--Bernstein collision operator, these take Fokker--Planck-like forms \cite{Fokker_1914, Planck_1917} wherein pseudo-particles in the numerical model obey the neoclassical transport equations, with particle-independent Brownian drift terms. This offers a rigorous methodology for incorporating collisions into the particle transport model, without coupling the equations of motions for each particle.
        
        Works by Chen, Chacón et al. \cite{Chen_Chacón_Barnes_2011, Chacón_Chen_Barnes_2013, Chen_Chacón_2014, Chen_Chacón_2015} have developed structure-preserving particle pushers for neoclassical transport in the Vlasov equations, derived from Crank--Nicolson integrators. We show these too can can derive from a FET interpretation, similarly offering potential extensions to higher-order-in-time particle pushers. The FET formulation is used also to consider how the stochastic drift terms can be incorporated into the pushers. Stochastic gyrokinetic expansions are also discussed.

        Different options for the numerical implementation of these schemes are considered.

        Due to the efficacy of FET in the development of SP timesteppers for both the fluid and kinetic component, we hope this approach will prove effective in the future for developing SP timesteppers for the full hybrid model. We hope this will give us the opportunity to incorporate previously inaccessible kinetic effects into the highly effective, modern, finite-element MHD models.
    \end{abstract}
    
    
    \newpage
    \tableofcontents
    
    
    \newpage
    \pagenumbering{arabic}
    %\linenumbers\renewcommand\thelinenumber{\color{black!50}\arabic{linenumber}}
            \input{0 - introduction/main.tex}
        \part{Research}
            \input{1 - low-noise PiC models/main.tex}
            \input{2 - kinetic component/main.tex}
            \input{3 - fluid component/main.tex}
            \input{4 - numerical implementation/main.tex}
        \part{Project Overview}
            \input{5 - research plan/main.tex}
            \input{6 - summary/main.tex}
    
    
    %\section{}
    \newpage
    \pagenumbering{gobble}
        \printbibliography


    \newpage
    \pagenumbering{roman}
    \appendix
        \part{Appendices}
            \input{8 - Hilbert complexes/main.tex}
            \input{9 - weak conservation proofs/main.tex}
\end{document}

    
    
    %\section{}
    \newpage
    \pagenumbering{gobble}
        \printbibliography


    \newpage
    \pagenumbering{roman}
    \appendix
        \part{Appendices}
            \documentclass[12pt, a4paper]{report}

\input{template/main.tex}

\title{\BA{Title in Progress...}}
\author{Boris Andrews}
\affil{Mathematical Institute, University of Oxford}
\date{\today}


\begin{document}
    \pagenumbering{gobble}
    \maketitle
    
    
    \begin{abstract}
        Magnetic confinement reactors---in particular tokamaks---offer one of the most promising options for achieving practical nuclear fusion, with the potential to provide virtually limitless, clean energy. The theoretical and numerical modeling of tokamak plasmas is simultaneously an essential component of effective reactor design, and a great research barrier. Tokamak operational conditions exhibit comparatively low Knudsen numbers. Kinetic effects, including kinetic waves and instabilities, Landau damping, bump-on-tail instabilities and more, are therefore highly influential in tokamak plasma dynamics. Purely fluid models are inherently incapable of capturing these effects, whereas the high dimensionality in purely kinetic models render them practically intractable for most relevant purposes.

        We consider a $\delta\!f$ decomposition model, with a macroscopic fluid background and microscopic kinetic correction, both fully coupled to each other. A similar manner of discretization is proposed to that used in the recent \texttt{STRUPHY} code \cite{Holderied_Possanner_Wang_2021, Holderied_2022, Li_et_al_2023} with a finite-element model for the background and a pseudo-particle/PiC model for the correction.

        The fluid background satisfies the full, non-linear, resistive, compressible, Hall MHD equations. \cite{Laakmann_Hu_Farrell_2022} introduces finite-element(-in-space) implicit timesteppers for the incompressible analogue to this system with structure-preserving (SP) properties in the ideal case, alongside parameter-robust preconditioners. We show that these timesteppers can derive from a finite-element-in-time (FET) (and finite-element-in-space) interpretation. The benefits of this reformulation are discussed, including the derivation of timesteppers that are higher order in time, and the quantifiable dissipative SP properties in the non-ideal, resistive case.
        
        We discuss possible options for extending this FET approach to timesteppers for the compressible case.

        The kinetic corrections satisfy linearized Boltzmann equations. Using a Lénard--Bernstein collision operator, these take Fokker--Planck-like forms \cite{Fokker_1914, Planck_1917} wherein pseudo-particles in the numerical model obey the neoclassical transport equations, with particle-independent Brownian drift terms. This offers a rigorous methodology for incorporating collisions into the particle transport model, without coupling the equations of motions for each particle.
        
        Works by Chen, Chacón et al. \cite{Chen_Chacón_Barnes_2011, Chacón_Chen_Barnes_2013, Chen_Chacón_2014, Chen_Chacón_2015} have developed structure-preserving particle pushers for neoclassical transport in the Vlasov equations, derived from Crank--Nicolson integrators. We show these too can can derive from a FET interpretation, similarly offering potential extensions to higher-order-in-time particle pushers. The FET formulation is used also to consider how the stochastic drift terms can be incorporated into the pushers. Stochastic gyrokinetic expansions are also discussed.

        Different options for the numerical implementation of these schemes are considered.

        Due to the efficacy of FET in the development of SP timesteppers for both the fluid and kinetic component, we hope this approach will prove effective in the future for developing SP timesteppers for the full hybrid model. We hope this will give us the opportunity to incorporate previously inaccessible kinetic effects into the highly effective, modern, finite-element MHD models.
    \end{abstract}
    
    
    \newpage
    \tableofcontents
    
    
    \newpage
    \pagenumbering{arabic}
    %\linenumbers\renewcommand\thelinenumber{\color{black!50}\arabic{linenumber}}
            \input{0 - introduction/main.tex}
        \part{Research}
            \input{1 - low-noise PiC models/main.tex}
            \input{2 - kinetic component/main.tex}
            \input{3 - fluid component/main.tex}
            \input{4 - numerical implementation/main.tex}
        \part{Project Overview}
            \input{5 - research plan/main.tex}
            \input{6 - summary/main.tex}
    
    
    %\section{}
    \newpage
    \pagenumbering{gobble}
        \printbibliography


    \newpage
    \pagenumbering{roman}
    \appendix
        \part{Appendices}
            \input{8 - Hilbert complexes/main.tex}
            \input{9 - weak conservation proofs/main.tex}
\end{document}

            \documentclass[12pt, a4paper]{report}

\input{template/main.tex}

\title{\BA{Title in Progress...}}
\author{Boris Andrews}
\affil{Mathematical Institute, University of Oxford}
\date{\today}


\begin{document}
    \pagenumbering{gobble}
    \maketitle
    
    
    \begin{abstract}
        Magnetic confinement reactors---in particular tokamaks---offer one of the most promising options for achieving practical nuclear fusion, with the potential to provide virtually limitless, clean energy. The theoretical and numerical modeling of tokamak plasmas is simultaneously an essential component of effective reactor design, and a great research barrier. Tokamak operational conditions exhibit comparatively low Knudsen numbers. Kinetic effects, including kinetic waves and instabilities, Landau damping, bump-on-tail instabilities and more, are therefore highly influential in tokamak plasma dynamics. Purely fluid models are inherently incapable of capturing these effects, whereas the high dimensionality in purely kinetic models render them practically intractable for most relevant purposes.

        We consider a $\delta\!f$ decomposition model, with a macroscopic fluid background and microscopic kinetic correction, both fully coupled to each other. A similar manner of discretization is proposed to that used in the recent \texttt{STRUPHY} code \cite{Holderied_Possanner_Wang_2021, Holderied_2022, Li_et_al_2023} with a finite-element model for the background and a pseudo-particle/PiC model for the correction.

        The fluid background satisfies the full, non-linear, resistive, compressible, Hall MHD equations. \cite{Laakmann_Hu_Farrell_2022} introduces finite-element(-in-space) implicit timesteppers for the incompressible analogue to this system with structure-preserving (SP) properties in the ideal case, alongside parameter-robust preconditioners. We show that these timesteppers can derive from a finite-element-in-time (FET) (and finite-element-in-space) interpretation. The benefits of this reformulation are discussed, including the derivation of timesteppers that are higher order in time, and the quantifiable dissipative SP properties in the non-ideal, resistive case.
        
        We discuss possible options for extending this FET approach to timesteppers for the compressible case.

        The kinetic corrections satisfy linearized Boltzmann equations. Using a Lénard--Bernstein collision operator, these take Fokker--Planck-like forms \cite{Fokker_1914, Planck_1917} wherein pseudo-particles in the numerical model obey the neoclassical transport equations, with particle-independent Brownian drift terms. This offers a rigorous methodology for incorporating collisions into the particle transport model, without coupling the equations of motions for each particle.
        
        Works by Chen, Chacón et al. \cite{Chen_Chacón_Barnes_2011, Chacón_Chen_Barnes_2013, Chen_Chacón_2014, Chen_Chacón_2015} have developed structure-preserving particle pushers for neoclassical transport in the Vlasov equations, derived from Crank--Nicolson integrators. We show these too can can derive from a FET interpretation, similarly offering potential extensions to higher-order-in-time particle pushers. The FET formulation is used also to consider how the stochastic drift terms can be incorporated into the pushers. Stochastic gyrokinetic expansions are also discussed.

        Different options for the numerical implementation of these schemes are considered.

        Due to the efficacy of FET in the development of SP timesteppers for both the fluid and kinetic component, we hope this approach will prove effective in the future for developing SP timesteppers for the full hybrid model. We hope this will give us the opportunity to incorporate previously inaccessible kinetic effects into the highly effective, modern, finite-element MHD models.
    \end{abstract}
    
    
    \newpage
    \tableofcontents
    
    
    \newpage
    \pagenumbering{arabic}
    %\linenumbers\renewcommand\thelinenumber{\color{black!50}\arabic{linenumber}}
            \input{0 - introduction/main.tex}
        \part{Research}
            \input{1 - low-noise PiC models/main.tex}
            \input{2 - kinetic component/main.tex}
            \input{3 - fluid component/main.tex}
            \input{4 - numerical implementation/main.tex}
        \part{Project Overview}
            \input{5 - research plan/main.tex}
            \input{6 - summary/main.tex}
    
    
    %\section{}
    \newpage
    \pagenumbering{gobble}
        \printbibliography


    \newpage
    \pagenumbering{roman}
    \appendix
        \part{Appendices}
            \input{8 - Hilbert complexes/main.tex}
            \input{9 - weak conservation proofs/main.tex}
\end{document}

\end{document}

        \part{Project Overview}
            \documentclass[12pt, a4paper]{report}

\documentclass[12pt, a4paper]{report}

\input{template/main.tex}

\title{\BA{Title in Progress...}}
\author{Boris Andrews}
\affil{Mathematical Institute, University of Oxford}
\date{\today}


\begin{document}
    \pagenumbering{gobble}
    \maketitle
    
    
    \begin{abstract}
        Magnetic confinement reactors---in particular tokamaks---offer one of the most promising options for achieving practical nuclear fusion, with the potential to provide virtually limitless, clean energy. The theoretical and numerical modeling of tokamak plasmas is simultaneously an essential component of effective reactor design, and a great research barrier. Tokamak operational conditions exhibit comparatively low Knudsen numbers. Kinetic effects, including kinetic waves and instabilities, Landau damping, bump-on-tail instabilities and more, are therefore highly influential in tokamak plasma dynamics. Purely fluid models are inherently incapable of capturing these effects, whereas the high dimensionality in purely kinetic models render them practically intractable for most relevant purposes.

        We consider a $\delta\!f$ decomposition model, with a macroscopic fluid background and microscopic kinetic correction, both fully coupled to each other. A similar manner of discretization is proposed to that used in the recent \texttt{STRUPHY} code \cite{Holderied_Possanner_Wang_2021, Holderied_2022, Li_et_al_2023} with a finite-element model for the background and a pseudo-particle/PiC model for the correction.

        The fluid background satisfies the full, non-linear, resistive, compressible, Hall MHD equations. \cite{Laakmann_Hu_Farrell_2022} introduces finite-element(-in-space) implicit timesteppers for the incompressible analogue to this system with structure-preserving (SP) properties in the ideal case, alongside parameter-robust preconditioners. We show that these timesteppers can derive from a finite-element-in-time (FET) (and finite-element-in-space) interpretation. The benefits of this reformulation are discussed, including the derivation of timesteppers that are higher order in time, and the quantifiable dissipative SP properties in the non-ideal, resistive case.
        
        We discuss possible options for extending this FET approach to timesteppers for the compressible case.

        The kinetic corrections satisfy linearized Boltzmann equations. Using a Lénard--Bernstein collision operator, these take Fokker--Planck-like forms \cite{Fokker_1914, Planck_1917} wherein pseudo-particles in the numerical model obey the neoclassical transport equations, with particle-independent Brownian drift terms. This offers a rigorous methodology for incorporating collisions into the particle transport model, without coupling the equations of motions for each particle.
        
        Works by Chen, Chacón et al. \cite{Chen_Chacón_Barnes_2011, Chacón_Chen_Barnes_2013, Chen_Chacón_2014, Chen_Chacón_2015} have developed structure-preserving particle pushers for neoclassical transport in the Vlasov equations, derived from Crank--Nicolson integrators. We show these too can can derive from a FET interpretation, similarly offering potential extensions to higher-order-in-time particle pushers. The FET formulation is used also to consider how the stochastic drift terms can be incorporated into the pushers. Stochastic gyrokinetic expansions are also discussed.

        Different options for the numerical implementation of these schemes are considered.

        Due to the efficacy of FET in the development of SP timesteppers for both the fluid and kinetic component, we hope this approach will prove effective in the future for developing SP timesteppers for the full hybrid model. We hope this will give us the opportunity to incorporate previously inaccessible kinetic effects into the highly effective, modern, finite-element MHD models.
    \end{abstract}
    
    
    \newpage
    \tableofcontents
    
    
    \newpage
    \pagenumbering{arabic}
    %\linenumbers\renewcommand\thelinenumber{\color{black!50}\arabic{linenumber}}
            \input{0 - introduction/main.tex}
        \part{Research}
            \input{1 - low-noise PiC models/main.tex}
            \input{2 - kinetic component/main.tex}
            \input{3 - fluid component/main.tex}
            \input{4 - numerical implementation/main.tex}
        \part{Project Overview}
            \input{5 - research plan/main.tex}
            \input{6 - summary/main.tex}
    
    
    %\section{}
    \newpage
    \pagenumbering{gobble}
        \printbibliography


    \newpage
    \pagenumbering{roman}
    \appendix
        \part{Appendices}
            \input{8 - Hilbert complexes/main.tex}
            \input{9 - weak conservation proofs/main.tex}
\end{document}


\title{\BA{Title in Progress...}}
\author{Boris Andrews}
\affil{Mathematical Institute, University of Oxford}
\date{\today}


\begin{document}
    \pagenumbering{gobble}
    \maketitle
    
    
    \begin{abstract}
        Magnetic confinement reactors---in particular tokamaks---offer one of the most promising options for achieving practical nuclear fusion, with the potential to provide virtually limitless, clean energy. The theoretical and numerical modeling of tokamak plasmas is simultaneously an essential component of effective reactor design, and a great research barrier. Tokamak operational conditions exhibit comparatively low Knudsen numbers. Kinetic effects, including kinetic waves and instabilities, Landau damping, bump-on-tail instabilities and more, are therefore highly influential in tokamak plasma dynamics. Purely fluid models are inherently incapable of capturing these effects, whereas the high dimensionality in purely kinetic models render them practically intractable for most relevant purposes.

        We consider a $\delta\!f$ decomposition model, with a macroscopic fluid background and microscopic kinetic correction, both fully coupled to each other. A similar manner of discretization is proposed to that used in the recent \texttt{STRUPHY} code \cite{Holderied_Possanner_Wang_2021, Holderied_2022, Li_et_al_2023} with a finite-element model for the background and a pseudo-particle/PiC model for the correction.

        The fluid background satisfies the full, non-linear, resistive, compressible, Hall MHD equations. \cite{Laakmann_Hu_Farrell_2022} introduces finite-element(-in-space) implicit timesteppers for the incompressible analogue to this system with structure-preserving (SP) properties in the ideal case, alongside parameter-robust preconditioners. We show that these timesteppers can derive from a finite-element-in-time (FET) (and finite-element-in-space) interpretation. The benefits of this reformulation are discussed, including the derivation of timesteppers that are higher order in time, and the quantifiable dissipative SP properties in the non-ideal, resistive case.
        
        We discuss possible options for extending this FET approach to timesteppers for the compressible case.

        The kinetic corrections satisfy linearized Boltzmann equations. Using a Lénard--Bernstein collision operator, these take Fokker--Planck-like forms \cite{Fokker_1914, Planck_1917} wherein pseudo-particles in the numerical model obey the neoclassical transport equations, with particle-independent Brownian drift terms. This offers a rigorous methodology for incorporating collisions into the particle transport model, without coupling the equations of motions for each particle.
        
        Works by Chen, Chacón et al. \cite{Chen_Chacón_Barnes_2011, Chacón_Chen_Barnes_2013, Chen_Chacón_2014, Chen_Chacón_2015} have developed structure-preserving particle pushers for neoclassical transport in the Vlasov equations, derived from Crank--Nicolson integrators. We show these too can can derive from a FET interpretation, similarly offering potential extensions to higher-order-in-time particle pushers. The FET formulation is used also to consider how the stochastic drift terms can be incorporated into the pushers. Stochastic gyrokinetic expansions are also discussed.

        Different options for the numerical implementation of these schemes are considered.

        Due to the efficacy of FET in the development of SP timesteppers for both the fluid and kinetic component, we hope this approach will prove effective in the future for developing SP timesteppers for the full hybrid model. We hope this will give us the opportunity to incorporate previously inaccessible kinetic effects into the highly effective, modern, finite-element MHD models.
    \end{abstract}
    
    
    \newpage
    \tableofcontents
    
    
    \newpage
    \pagenumbering{arabic}
    %\linenumbers\renewcommand\thelinenumber{\color{black!50}\arabic{linenumber}}
            \documentclass[12pt, a4paper]{report}

\input{template/main.tex}

\title{\BA{Title in Progress...}}
\author{Boris Andrews}
\affil{Mathematical Institute, University of Oxford}
\date{\today}


\begin{document}
    \pagenumbering{gobble}
    \maketitle
    
    
    \begin{abstract}
        Magnetic confinement reactors---in particular tokamaks---offer one of the most promising options for achieving practical nuclear fusion, with the potential to provide virtually limitless, clean energy. The theoretical and numerical modeling of tokamak plasmas is simultaneously an essential component of effective reactor design, and a great research barrier. Tokamak operational conditions exhibit comparatively low Knudsen numbers. Kinetic effects, including kinetic waves and instabilities, Landau damping, bump-on-tail instabilities and more, are therefore highly influential in tokamak plasma dynamics. Purely fluid models are inherently incapable of capturing these effects, whereas the high dimensionality in purely kinetic models render them practically intractable for most relevant purposes.

        We consider a $\delta\!f$ decomposition model, with a macroscopic fluid background and microscopic kinetic correction, both fully coupled to each other. A similar manner of discretization is proposed to that used in the recent \texttt{STRUPHY} code \cite{Holderied_Possanner_Wang_2021, Holderied_2022, Li_et_al_2023} with a finite-element model for the background and a pseudo-particle/PiC model for the correction.

        The fluid background satisfies the full, non-linear, resistive, compressible, Hall MHD equations. \cite{Laakmann_Hu_Farrell_2022} introduces finite-element(-in-space) implicit timesteppers for the incompressible analogue to this system with structure-preserving (SP) properties in the ideal case, alongside parameter-robust preconditioners. We show that these timesteppers can derive from a finite-element-in-time (FET) (and finite-element-in-space) interpretation. The benefits of this reformulation are discussed, including the derivation of timesteppers that are higher order in time, and the quantifiable dissipative SP properties in the non-ideal, resistive case.
        
        We discuss possible options for extending this FET approach to timesteppers for the compressible case.

        The kinetic corrections satisfy linearized Boltzmann equations. Using a Lénard--Bernstein collision operator, these take Fokker--Planck-like forms \cite{Fokker_1914, Planck_1917} wherein pseudo-particles in the numerical model obey the neoclassical transport equations, with particle-independent Brownian drift terms. This offers a rigorous methodology for incorporating collisions into the particle transport model, without coupling the equations of motions for each particle.
        
        Works by Chen, Chacón et al. \cite{Chen_Chacón_Barnes_2011, Chacón_Chen_Barnes_2013, Chen_Chacón_2014, Chen_Chacón_2015} have developed structure-preserving particle pushers for neoclassical transport in the Vlasov equations, derived from Crank--Nicolson integrators. We show these too can can derive from a FET interpretation, similarly offering potential extensions to higher-order-in-time particle pushers. The FET formulation is used also to consider how the stochastic drift terms can be incorporated into the pushers. Stochastic gyrokinetic expansions are also discussed.

        Different options for the numerical implementation of these schemes are considered.

        Due to the efficacy of FET in the development of SP timesteppers for both the fluid and kinetic component, we hope this approach will prove effective in the future for developing SP timesteppers for the full hybrid model. We hope this will give us the opportunity to incorporate previously inaccessible kinetic effects into the highly effective, modern, finite-element MHD models.
    \end{abstract}
    
    
    \newpage
    \tableofcontents
    
    
    \newpage
    \pagenumbering{arabic}
    %\linenumbers\renewcommand\thelinenumber{\color{black!50}\arabic{linenumber}}
            \input{0 - introduction/main.tex}
        \part{Research}
            \input{1 - low-noise PiC models/main.tex}
            \input{2 - kinetic component/main.tex}
            \input{3 - fluid component/main.tex}
            \input{4 - numerical implementation/main.tex}
        \part{Project Overview}
            \input{5 - research plan/main.tex}
            \input{6 - summary/main.tex}
    
    
    %\section{}
    \newpage
    \pagenumbering{gobble}
        \printbibliography


    \newpage
    \pagenumbering{roman}
    \appendix
        \part{Appendices}
            \input{8 - Hilbert complexes/main.tex}
            \input{9 - weak conservation proofs/main.tex}
\end{document}

        \part{Research}
            \documentclass[12pt, a4paper]{report}

\input{template/main.tex}

\title{\BA{Title in Progress...}}
\author{Boris Andrews}
\affil{Mathematical Institute, University of Oxford}
\date{\today}


\begin{document}
    \pagenumbering{gobble}
    \maketitle
    
    
    \begin{abstract}
        Magnetic confinement reactors---in particular tokamaks---offer one of the most promising options for achieving practical nuclear fusion, with the potential to provide virtually limitless, clean energy. The theoretical and numerical modeling of tokamak plasmas is simultaneously an essential component of effective reactor design, and a great research barrier. Tokamak operational conditions exhibit comparatively low Knudsen numbers. Kinetic effects, including kinetic waves and instabilities, Landau damping, bump-on-tail instabilities and more, are therefore highly influential in tokamak plasma dynamics. Purely fluid models are inherently incapable of capturing these effects, whereas the high dimensionality in purely kinetic models render them practically intractable for most relevant purposes.

        We consider a $\delta\!f$ decomposition model, with a macroscopic fluid background and microscopic kinetic correction, both fully coupled to each other. A similar manner of discretization is proposed to that used in the recent \texttt{STRUPHY} code \cite{Holderied_Possanner_Wang_2021, Holderied_2022, Li_et_al_2023} with a finite-element model for the background and a pseudo-particle/PiC model for the correction.

        The fluid background satisfies the full, non-linear, resistive, compressible, Hall MHD equations. \cite{Laakmann_Hu_Farrell_2022} introduces finite-element(-in-space) implicit timesteppers for the incompressible analogue to this system with structure-preserving (SP) properties in the ideal case, alongside parameter-robust preconditioners. We show that these timesteppers can derive from a finite-element-in-time (FET) (and finite-element-in-space) interpretation. The benefits of this reformulation are discussed, including the derivation of timesteppers that are higher order in time, and the quantifiable dissipative SP properties in the non-ideal, resistive case.
        
        We discuss possible options for extending this FET approach to timesteppers for the compressible case.

        The kinetic corrections satisfy linearized Boltzmann equations. Using a Lénard--Bernstein collision operator, these take Fokker--Planck-like forms \cite{Fokker_1914, Planck_1917} wherein pseudo-particles in the numerical model obey the neoclassical transport equations, with particle-independent Brownian drift terms. This offers a rigorous methodology for incorporating collisions into the particle transport model, without coupling the equations of motions for each particle.
        
        Works by Chen, Chacón et al. \cite{Chen_Chacón_Barnes_2011, Chacón_Chen_Barnes_2013, Chen_Chacón_2014, Chen_Chacón_2015} have developed structure-preserving particle pushers for neoclassical transport in the Vlasov equations, derived from Crank--Nicolson integrators. We show these too can can derive from a FET interpretation, similarly offering potential extensions to higher-order-in-time particle pushers. The FET formulation is used also to consider how the stochastic drift terms can be incorporated into the pushers. Stochastic gyrokinetic expansions are also discussed.

        Different options for the numerical implementation of these schemes are considered.

        Due to the efficacy of FET in the development of SP timesteppers for both the fluid and kinetic component, we hope this approach will prove effective in the future for developing SP timesteppers for the full hybrid model. We hope this will give us the opportunity to incorporate previously inaccessible kinetic effects into the highly effective, modern, finite-element MHD models.
    \end{abstract}
    
    
    \newpage
    \tableofcontents
    
    
    \newpage
    \pagenumbering{arabic}
    %\linenumbers\renewcommand\thelinenumber{\color{black!50}\arabic{linenumber}}
            \input{0 - introduction/main.tex}
        \part{Research}
            \input{1 - low-noise PiC models/main.tex}
            \input{2 - kinetic component/main.tex}
            \input{3 - fluid component/main.tex}
            \input{4 - numerical implementation/main.tex}
        \part{Project Overview}
            \input{5 - research plan/main.tex}
            \input{6 - summary/main.tex}
    
    
    %\section{}
    \newpage
    \pagenumbering{gobble}
        \printbibliography


    \newpage
    \pagenumbering{roman}
    \appendix
        \part{Appendices}
            \input{8 - Hilbert complexes/main.tex}
            \input{9 - weak conservation proofs/main.tex}
\end{document}

            \documentclass[12pt, a4paper]{report}

\input{template/main.tex}

\title{\BA{Title in Progress...}}
\author{Boris Andrews}
\affil{Mathematical Institute, University of Oxford}
\date{\today}


\begin{document}
    \pagenumbering{gobble}
    \maketitle
    
    
    \begin{abstract}
        Magnetic confinement reactors---in particular tokamaks---offer one of the most promising options for achieving practical nuclear fusion, with the potential to provide virtually limitless, clean energy. The theoretical and numerical modeling of tokamak plasmas is simultaneously an essential component of effective reactor design, and a great research barrier. Tokamak operational conditions exhibit comparatively low Knudsen numbers. Kinetic effects, including kinetic waves and instabilities, Landau damping, bump-on-tail instabilities and more, are therefore highly influential in tokamak plasma dynamics. Purely fluid models are inherently incapable of capturing these effects, whereas the high dimensionality in purely kinetic models render them practically intractable for most relevant purposes.

        We consider a $\delta\!f$ decomposition model, with a macroscopic fluid background and microscopic kinetic correction, both fully coupled to each other. A similar manner of discretization is proposed to that used in the recent \texttt{STRUPHY} code \cite{Holderied_Possanner_Wang_2021, Holderied_2022, Li_et_al_2023} with a finite-element model for the background and a pseudo-particle/PiC model for the correction.

        The fluid background satisfies the full, non-linear, resistive, compressible, Hall MHD equations. \cite{Laakmann_Hu_Farrell_2022} introduces finite-element(-in-space) implicit timesteppers for the incompressible analogue to this system with structure-preserving (SP) properties in the ideal case, alongside parameter-robust preconditioners. We show that these timesteppers can derive from a finite-element-in-time (FET) (and finite-element-in-space) interpretation. The benefits of this reformulation are discussed, including the derivation of timesteppers that are higher order in time, and the quantifiable dissipative SP properties in the non-ideal, resistive case.
        
        We discuss possible options for extending this FET approach to timesteppers for the compressible case.

        The kinetic corrections satisfy linearized Boltzmann equations. Using a Lénard--Bernstein collision operator, these take Fokker--Planck-like forms \cite{Fokker_1914, Planck_1917} wherein pseudo-particles in the numerical model obey the neoclassical transport equations, with particle-independent Brownian drift terms. This offers a rigorous methodology for incorporating collisions into the particle transport model, without coupling the equations of motions for each particle.
        
        Works by Chen, Chacón et al. \cite{Chen_Chacón_Barnes_2011, Chacón_Chen_Barnes_2013, Chen_Chacón_2014, Chen_Chacón_2015} have developed structure-preserving particle pushers for neoclassical transport in the Vlasov equations, derived from Crank--Nicolson integrators. We show these too can can derive from a FET interpretation, similarly offering potential extensions to higher-order-in-time particle pushers. The FET formulation is used also to consider how the stochastic drift terms can be incorporated into the pushers. Stochastic gyrokinetic expansions are also discussed.

        Different options for the numerical implementation of these schemes are considered.

        Due to the efficacy of FET in the development of SP timesteppers for both the fluid and kinetic component, we hope this approach will prove effective in the future for developing SP timesteppers for the full hybrid model. We hope this will give us the opportunity to incorporate previously inaccessible kinetic effects into the highly effective, modern, finite-element MHD models.
    \end{abstract}
    
    
    \newpage
    \tableofcontents
    
    
    \newpage
    \pagenumbering{arabic}
    %\linenumbers\renewcommand\thelinenumber{\color{black!50}\arabic{linenumber}}
            \input{0 - introduction/main.tex}
        \part{Research}
            \input{1 - low-noise PiC models/main.tex}
            \input{2 - kinetic component/main.tex}
            \input{3 - fluid component/main.tex}
            \input{4 - numerical implementation/main.tex}
        \part{Project Overview}
            \input{5 - research plan/main.tex}
            \input{6 - summary/main.tex}
    
    
    %\section{}
    \newpage
    \pagenumbering{gobble}
        \printbibliography


    \newpage
    \pagenumbering{roman}
    \appendix
        \part{Appendices}
            \input{8 - Hilbert complexes/main.tex}
            \input{9 - weak conservation proofs/main.tex}
\end{document}

            \documentclass[12pt, a4paper]{report}

\input{template/main.tex}

\title{\BA{Title in Progress...}}
\author{Boris Andrews}
\affil{Mathematical Institute, University of Oxford}
\date{\today}


\begin{document}
    \pagenumbering{gobble}
    \maketitle
    
    
    \begin{abstract}
        Magnetic confinement reactors---in particular tokamaks---offer one of the most promising options for achieving practical nuclear fusion, with the potential to provide virtually limitless, clean energy. The theoretical and numerical modeling of tokamak plasmas is simultaneously an essential component of effective reactor design, and a great research barrier. Tokamak operational conditions exhibit comparatively low Knudsen numbers. Kinetic effects, including kinetic waves and instabilities, Landau damping, bump-on-tail instabilities and more, are therefore highly influential in tokamak plasma dynamics. Purely fluid models are inherently incapable of capturing these effects, whereas the high dimensionality in purely kinetic models render them practically intractable for most relevant purposes.

        We consider a $\delta\!f$ decomposition model, with a macroscopic fluid background and microscopic kinetic correction, both fully coupled to each other. A similar manner of discretization is proposed to that used in the recent \texttt{STRUPHY} code \cite{Holderied_Possanner_Wang_2021, Holderied_2022, Li_et_al_2023} with a finite-element model for the background and a pseudo-particle/PiC model for the correction.

        The fluid background satisfies the full, non-linear, resistive, compressible, Hall MHD equations. \cite{Laakmann_Hu_Farrell_2022} introduces finite-element(-in-space) implicit timesteppers for the incompressible analogue to this system with structure-preserving (SP) properties in the ideal case, alongside parameter-robust preconditioners. We show that these timesteppers can derive from a finite-element-in-time (FET) (and finite-element-in-space) interpretation. The benefits of this reformulation are discussed, including the derivation of timesteppers that are higher order in time, and the quantifiable dissipative SP properties in the non-ideal, resistive case.
        
        We discuss possible options for extending this FET approach to timesteppers for the compressible case.

        The kinetic corrections satisfy linearized Boltzmann equations. Using a Lénard--Bernstein collision operator, these take Fokker--Planck-like forms \cite{Fokker_1914, Planck_1917} wherein pseudo-particles in the numerical model obey the neoclassical transport equations, with particle-independent Brownian drift terms. This offers a rigorous methodology for incorporating collisions into the particle transport model, without coupling the equations of motions for each particle.
        
        Works by Chen, Chacón et al. \cite{Chen_Chacón_Barnes_2011, Chacón_Chen_Barnes_2013, Chen_Chacón_2014, Chen_Chacón_2015} have developed structure-preserving particle pushers for neoclassical transport in the Vlasov equations, derived from Crank--Nicolson integrators. We show these too can can derive from a FET interpretation, similarly offering potential extensions to higher-order-in-time particle pushers. The FET formulation is used also to consider how the stochastic drift terms can be incorporated into the pushers. Stochastic gyrokinetic expansions are also discussed.

        Different options for the numerical implementation of these schemes are considered.

        Due to the efficacy of FET in the development of SP timesteppers for both the fluid and kinetic component, we hope this approach will prove effective in the future for developing SP timesteppers for the full hybrid model. We hope this will give us the opportunity to incorporate previously inaccessible kinetic effects into the highly effective, modern, finite-element MHD models.
    \end{abstract}
    
    
    \newpage
    \tableofcontents
    
    
    \newpage
    \pagenumbering{arabic}
    %\linenumbers\renewcommand\thelinenumber{\color{black!50}\arabic{linenumber}}
            \input{0 - introduction/main.tex}
        \part{Research}
            \input{1 - low-noise PiC models/main.tex}
            \input{2 - kinetic component/main.tex}
            \input{3 - fluid component/main.tex}
            \input{4 - numerical implementation/main.tex}
        \part{Project Overview}
            \input{5 - research plan/main.tex}
            \input{6 - summary/main.tex}
    
    
    %\section{}
    \newpage
    \pagenumbering{gobble}
        \printbibliography


    \newpage
    \pagenumbering{roman}
    \appendix
        \part{Appendices}
            \input{8 - Hilbert complexes/main.tex}
            \input{9 - weak conservation proofs/main.tex}
\end{document}

            \documentclass[12pt, a4paper]{report}

\input{template/main.tex}

\title{\BA{Title in Progress...}}
\author{Boris Andrews}
\affil{Mathematical Institute, University of Oxford}
\date{\today}


\begin{document}
    \pagenumbering{gobble}
    \maketitle
    
    
    \begin{abstract}
        Magnetic confinement reactors---in particular tokamaks---offer one of the most promising options for achieving practical nuclear fusion, with the potential to provide virtually limitless, clean energy. The theoretical and numerical modeling of tokamak plasmas is simultaneously an essential component of effective reactor design, and a great research barrier. Tokamak operational conditions exhibit comparatively low Knudsen numbers. Kinetic effects, including kinetic waves and instabilities, Landau damping, bump-on-tail instabilities and more, are therefore highly influential in tokamak plasma dynamics. Purely fluid models are inherently incapable of capturing these effects, whereas the high dimensionality in purely kinetic models render them practically intractable for most relevant purposes.

        We consider a $\delta\!f$ decomposition model, with a macroscopic fluid background and microscopic kinetic correction, both fully coupled to each other. A similar manner of discretization is proposed to that used in the recent \texttt{STRUPHY} code \cite{Holderied_Possanner_Wang_2021, Holderied_2022, Li_et_al_2023} with a finite-element model for the background and a pseudo-particle/PiC model for the correction.

        The fluid background satisfies the full, non-linear, resistive, compressible, Hall MHD equations. \cite{Laakmann_Hu_Farrell_2022} introduces finite-element(-in-space) implicit timesteppers for the incompressible analogue to this system with structure-preserving (SP) properties in the ideal case, alongside parameter-robust preconditioners. We show that these timesteppers can derive from a finite-element-in-time (FET) (and finite-element-in-space) interpretation. The benefits of this reformulation are discussed, including the derivation of timesteppers that are higher order in time, and the quantifiable dissipative SP properties in the non-ideal, resistive case.
        
        We discuss possible options for extending this FET approach to timesteppers for the compressible case.

        The kinetic corrections satisfy linearized Boltzmann equations. Using a Lénard--Bernstein collision operator, these take Fokker--Planck-like forms \cite{Fokker_1914, Planck_1917} wherein pseudo-particles in the numerical model obey the neoclassical transport equations, with particle-independent Brownian drift terms. This offers a rigorous methodology for incorporating collisions into the particle transport model, without coupling the equations of motions for each particle.
        
        Works by Chen, Chacón et al. \cite{Chen_Chacón_Barnes_2011, Chacón_Chen_Barnes_2013, Chen_Chacón_2014, Chen_Chacón_2015} have developed structure-preserving particle pushers for neoclassical transport in the Vlasov equations, derived from Crank--Nicolson integrators. We show these too can can derive from a FET interpretation, similarly offering potential extensions to higher-order-in-time particle pushers. The FET formulation is used also to consider how the stochastic drift terms can be incorporated into the pushers. Stochastic gyrokinetic expansions are also discussed.

        Different options for the numerical implementation of these schemes are considered.

        Due to the efficacy of FET in the development of SP timesteppers for both the fluid and kinetic component, we hope this approach will prove effective in the future for developing SP timesteppers for the full hybrid model. We hope this will give us the opportunity to incorporate previously inaccessible kinetic effects into the highly effective, modern, finite-element MHD models.
    \end{abstract}
    
    
    \newpage
    \tableofcontents
    
    
    \newpage
    \pagenumbering{arabic}
    %\linenumbers\renewcommand\thelinenumber{\color{black!50}\arabic{linenumber}}
            \input{0 - introduction/main.tex}
        \part{Research}
            \input{1 - low-noise PiC models/main.tex}
            \input{2 - kinetic component/main.tex}
            \input{3 - fluid component/main.tex}
            \input{4 - numerical implementation/main.tex}
        \part{Project Overview}
            \input{5 - research plan/main.tex}
            \input{6 - summary/main.tex}
    
    
    %\section{}
    \newpage
    \pagenumbering{gobble}
        \printbibliography


    \newpage
    \pagenumbering{roman}
    \appendix
        \part{Appendices}
            \input{8 - Hilbert complexes/main.tex}
            \input{9 - weak conservation proofs/main.tex}
\end{document}

        \part{Project Overview}
            \documentclass[12pt, a4paper]{report}

\input{template/main.tex}

\title{\BA{Title in Progress...}}
\author{Boris Andrews}
\affil{Mathematical Institute, University of Oxford}
\date{\today}


\begin{document}
    \pagenumbering{gobble}
    \maketitle
    
    
    \begin{abstract}
        Magnetic confinement reactors---in particular tokamaks---offer one of the most promising options for achieving practical nuclear fusion, with the potential to provide virtually limitless, clean energy. The theoretical and numerical modeling of tokamak plasmas is simultaneously an essential component of effective reactor design, and a great research barrier. Tokamak operational conditions exhibit comparatively low Knudsen numbers. Kinetic effects, including kinetic waves and instabilities, Landau damping, bump-on-tail instabilities and more, are therefore highly influential in tokamak plasma dynamics. Purely fluid models are inherently incapable of capturing these effects, whereas the high dimensionality in purely kinetic models render them practically intractable for most relevant purposes.

        We consider a $\delta\!f$ decomposition model, with a macroscopic fluid background and microscopic kinetic correction, both fully coupled to each other. A similar manner of discretization is proposed to that used in the recent \texttt{STRUPHY} code \cite{Holderied_Possanner_Wang_2021, Holderied_2022, Li_et_al_2023} with a finite-element model for the background and a pseudo-particle/PiC model for the correction.

        The fluid background satisfies the full, non-linear, resistive, compressible, Hall MHD equations. \cite{Laakmann_Hu_Farrell_2022} introduces finite-element(-in-space) implicit timesteppers for the incompressible analogue to this system with structure-preserving (SP) properties in the ideal case, alongside parameter-robust preconditioners. We show that these timesteppers can derive from a finite-element-in-time (FET) (and finite-element-in-space) interpretation. The benefits of this reformulation are discussed, including the derivation of timesteppers that are higher order in time, and the quantifiable dissipative SP properties in the non-ideal, resistive case.
        
        We discuss possible options for extending this FET approach to timesteppers for the compressible case.

        The kinetic corrections satisfy linearized Boltzmann equations. Using a Lénard--Bernstein collision operator, these take Fokker--Planck-like forms \cite{Fokker_1914, Planck_1917} wherein pseudo-particles in the numerical model obey the neoclassical transport equations, with particle-independent Brownian drift terms. This offers a rigorous methodology for incorporating collisions into the particle transport model, without coupling the equations of motions for each particle.
        
        Works by Chen, Chacón et al. \cite{Chen_Chacón_Barnes_2011, Chacón_Chen_Barnes_2013, Chen_Chacón_2014, Chen_Chacón_2015} have developed structure-preserving particle pushers for neoclassical transport in the Vlasov equations, derived from Crank--Nicolson integrators. We show these too can can derive from a FET interpretation, similarly offering potential extensions to higher-order-in-time particle pushers. The FET formulation is used also to consider how the stochastic drift terms can be incorporated into the pushers. Stochastic gyrokinetic expansions are also discussed.

        Different options for the numerical implementation of these schemes are considered.

        Due to the efficacy of FET in the development of SP timesteppers for both the fluid and kinetic component, we hope this approach will prove effective in the future for developing SP timesteppers for the full hybrid model. We hope this will give us the opportunity to incorporate previously inaccessible kinetic effects into the highly effective, modern, finite-element MHD models.
    \end{abstract}
    
    
    \newpage
    \tableofcontents
    
    
    \newpage
    \pagenumbering{arabic}
    %\linenumbers\renewcommand\thelinenumber{\color{black!50}\arabic{linenumber}}
            \input{0 - introduction/main.tex}
        \part{Research}
            \input{1 - low-noise PiC models/main.tex}
            \input{2 - kinetic component/main.tex}
            \input{3 - fluid component/main.tex}
            \input{4 - numerical implementation/main.tex}
        \part{Project Overview}
            \input{5 - research plan/main.tex}
            \input{6 - summary/main.tex}
    
    
    %\section{}
    \newpage
    \pagenumbering{gobble}
        \printbibliography


    \newpage
    \pagenumbering{roman}
    \appendix
        \part{Appendices}
            \input{8 - Hilbert complexes/main.tex}
            \input{9 - weak conservation proofs/main.tex}
\end{document}

            \documentclass[12pt, a4paper]{report}

\input{template/main.tex}

\title{\BA{Title in Progress...}}
\author{Boris Andrews}
\affil{Mathematical Institute, University of Oxford}
\date{\today}


\begin{document}
    \pagenumbering{gobble}
    \maketitle
    
    
    \begin{abstract}
        Magnetic confinement reactors---in particular tokamaks---offer one of the most promising options for achieving practical nuclear fusion, with the potential to provide virtually limitless, clean energy. The theoretical and numerical modeling of tokamak plasmas is simultaneously an essential component of effective reactor design, and a great research barrier. Tokamak operational conditions exhibit comparatively low Knudsen numbers. Kinetic effects, including kinetic waves and instabilities, Landau damping, bump-on-tail instabilities and more, are therefore highly influential in tokamak plasma dynamics. Purely fluid models are inherently incapable of capturing these effects, whereas the high dimensionality in purely kinetic models render them practically intractable for most relevant purposes.

        We consider a $\delta\!f$ decomposition model, with a macroscopic fluid background and microscopic kinetic correction, both fully coupled to each other. A similar manner of discretization is proposed to that used in the recent \texttt{STRUPHY} code \cite{Holderied_Possanner_Wang_2021, Holderied_2022, Li_et_al_2023} with a finite-element model for the background and a pseudo-particle/PiC model for the correction.

        The fluid background satisfies the full, non-linear, resistive, compressible, Hall MHD equations. \cite{Laakmann_Hu_Farrell_2022} introduces finite-element(-in-space) implicit timesteppers for the incompressible analogue to this system with structure-preserving (SP) properties in the ideal case, alongside parameter-robust preconditioners. We show that these timesteppers can derive from a finite-element-in-time (FET) (and finite-element-in-space) interpretation. The benefits of this reformulation are discussed, including the derivation of timesteppers that are higher order in time, and the quantifiable dissipative SP properties in the non-ideal, resistive case.
        
        We discuss possible options for extending this FET approach to timesteppers for the compressible case.

        The kinetic corrections satisfy linearized Boltzmann equations. Using a Lénard--Bernstein collision operator, these take Fokker--Planck-like forms \cite{Fokker_1914, Planck_1917} wherein pseudo-particles in the numerical model obey the neoclassical transport equations, with particle-independent Brownian drift terms. This offers a rigorous methodology for incorporating collisions into the particle transport model, without coupling the equations of motions for each particle.
        
        Works by Chen, Chacón et al. \cite{Chen_Chacón_Barnes_2011, Chacón_Chen_Barnes_2013, Chen_Chacón_2014, Chen_Chacón_2015} have developed structure-preserving particle pushers for neoclassical transport in the Vlasov equations, derived from Crank--Nicolson integrators. We show these too can can derive from a FET interpretation, similarly offering potential extensions to higher-order-in-time particle pushers. The FET formulation is used also to consider how the stochastic drift terms can be incorporated into the pushers. Stochastic gyrokinetic expansions are also discussed.

        Different options for the numerical implementation of these schemes are considered.

        Due to the efficacy of FET in the development of SP timesteppers for both the fluid and kinetic component, we hope this approach will prove effective in the future for developing SP timesteppers for the full hybrid model. We hope this will give us the opportunity to incorporate previously inaccessible kinetic effects into the highly effective, modern, finite-element MHD models.
    \end{abstract}
    
    
    \newpage
    \tableofcontents
    
    
    \newpage
    \pagenumbering{arabic}
    %\linenumbers\renewcommand\thelinenumber{\color{black!50}\arabic{linenumber}}
            \input{0 - introduction/main.tex}
        \part{Research}
            \input{1 - low-noise PiC models/main.tex}
            \input{2 - kinetic component/main.tex}
            \input{3 - fluid component/main.tex}
            \input{4 - numerical implementation/main.tex}
        \part{Project Overview}
            \input{5 - research plan/main.tex}
            \input{6 - summary/main.tex}
    
    
    %\section{}
    \newpage
    \pagenumbering{gobble}
        \printbibliography


    \newpage
    \pagenumbering{roman}
    \appendix
        \part{Appendices}
            \input{8 - Hilbert complexes/main.tex}
            \input{9 - weak conservation proofs/main.tex}
\end{document}

    
    
    %\section{}
    \newpage
    \pagenumbering{gobble}
        \printbibliography


    \newpage
    \pagenumbering{roman}
    \appendix
        \part{Appendices}
            \documentclass[12pt, a4paper]{report}

\input{template/main.tex}

\title{\BA{Title in Progress...}}
\author{Boris Andrews}
\affil{Mathematical Institute, University of Oxford}
\date{\today}


\begin{document}
    \pagenumbering{gobble}
    \maketitle
    
    
    \begin{abstract}
        Magnetic confinement reactors---in particular tokamaks---offer one of the most promising options for achieving practical nuclear fusion, with the potential to provide virtually limitless, clean energy. The theoretical and numerical modeling of tokamak plasmas is simultaneously an essential component of effective reactor design, and a great research barrier. Tokamak operational conditions exhibit comparatively low Knudsen numbers. Kinetic effects, including kinetic waves and instabilities, Landau damping, bump-on-tail instabilities and more, are therefore highly influential in tokamak plasma dynamics. Purely fluid models are inherently incapable of capturing these effects, whereas the high dimensionality in purely kinetic models render them practically intractable for most relevant purposes.

        We consider a $\delta\!f$ decomposition model, with a macroscopic fluid background and microscopic kinetic correction, both fully coupled to each other. A similar manner of discretization is proposed to that used in the recent \texttt{STRUPHY} code \cite{Holderied_Possanner_Wang_2021, Holderied_2022, Li_et_al_2023} with a finite-element model for the background and a pseudo-particle/PiC model for the correction.

        The fluid background satisfies the full, non-linear, resistive, compressible, Hall MHD equations. \cite{Laakmann_Hu_Farrell_2022} introduces finite-element(-in-space) implicit timesteppers for the incompressible analogue to this system with structure-preserving (SP) properties in the ideal case, alongside parameter-robust preconditioners. We show that these timesteppers can derive from a finite-element-in-time (FET) (and finite-element-in-space) interpretation. The benefits of this reformulation are discussed, including the derivation of timesteppers that are higher order in time, and the quantifiable dissipative SP properties in the non-ideal, resistive case.
        
        We discuss possible options for extending this FET approach to timesteppers for the compressible case.

        The kinetic corrections satisfy linearized Boltzmann equations. Using a Lénard--Bernstein collision operator, these take Fokker--Planck-like forms \cite{Fokker_1914, Planck_1917} wherein pseudo-particles in the numerical model obey the neoclassical transport equations, with particle-independent Brownian drift terms. This offers a rigorous methodology for incorporating collisions into the particle transport model, without coupling the equations of motions for each particle.
        
        Works by Chen, Chacón et al. \cite{Chen_Chacón_Barnes_2011, Chacón_Chen_Barnes_2013, Chen_Chacón_2014, Chen_Chacón_2015} have developed structure-preserving particle pushers for neoclassical transport in the Vlasov equations, derived from Crank--Nicolson integrators. We show these too can can derive from a FET interpretation, similarly offering potential extensions to higher-order-in-time particle pushers. The FET formulation is used also to consider how the stochastic drift terms can be incorporated into the pushers. Stochastic gyrokinetic expansions are also discussed.

        Different options for the numerical implementation of these schemes are considered.

        Due to the efficacy of FET in the development of SP timesteppers for both the fluid and kinetic component, we hope this approach will prove effective in the future for developing SP timesteppers for the full hybrid model. We hope this will give us the opportunity to incorporate previously inaccessible kinetic effects into the highly effective, modern, finite-element MHD models.
    \end{abstract}
    
    
    \newpage
    \tableofcontents
    
    
    \newpage
    \pagenumbering{arabic}
    %\linenumbers\renewcommand\thelinenumber{\color{black!50}\arabic{linenumber}}
            \input{0 - introduction/main.tex}
        \part{Research}
            \input{1 - low-noise PiC models/main.tex}
            \input{2 - kinetic component/main.tex}
            \input{3 - fluid component/main.tex}
            \input{4 - numerical implementation/main.tex}
        \part{Project Overview}
            \input{5 - research plan/main.tex}
            \input{6 - summary/main.tex}
    
    
    %\section{}
    \newpage
    \pagenumbering{gobble}
        \printbibliography


    \newpage
    \pagenumbering{roman}
    \appendix
        \part{Appendices}
            \input{8 - Hilbert complexes/main.tex}
            \input{9 - weak conservation proofs/main.tex}
\end{document}

            \documentclass[12pt, a4paper]{report}

\input{template/main.tex}

\title{\BA{Title in Progress...}}
\author{Boris Andrews}
\affil{Mathematical Institute, University of Oxford}
\date{\today}


\begin{document}
    \pagenumbering{gobble}
    \maketitle
    
    
    \begin{abstract}
        Magnetic confinement reactors---in particular tokamaks---offer one of the most promising options for achieving practical nuclear fusion, with the potential to provide virtually limitless, clean energy. The theoretical and numerical modeling of tokamak plasmas is simultaneously an essential component of effective reactor design, and a great research barrier. Tokamak operational conditions exhibit comparatively low Knudsen numbers. Kinetic effects, including kinetic waves and instabilities, Landau damping, bump-on-tail instabilities and more, are therefore highly influential in tokamak plasma dynamics. Purely fluid models are inherently incapable of capturing these effects, whereas the high dimensionality in purely kinetic models render them practically intractable for most relevant purposes.

        We consider a $\delta\!f$ decomposition model, with a macroscopic fluid background and microscopic kinetic correction, both fully coupled to each other. A similar manner of discretization is proposed to that used in the recent \texttt{STRUPHY} code \cite{Holderied_Possanner_Wang_2021, Holderied_2022, Li_et_al_2023} with a finite-element model for the background and a pseudo-particle/PiC model for the correction.

        The fluid background satisfies the full, non-linear, resistive, compressible, Hall MHD equations. \cite{Laakmann_Hu_Farrell_2022} introduces finite-element(-in-space) implicit timesteppers for the incompressible analogue to this system with structure-preserving (SP) properties in the ideal case, alongside parameter-robust preconditioners. We show that these timesteppers can derive from a finite-element-in-time (FET) (and finite-element-in-space) interpretation. The benefits of this reformulation are discussed, including the derivation of timesteppers that are higher order in time, and the quantifiable dissipative SP properties in the non-ideal, resistive case.
        
        We discuss possible options for extending this FET approach to timesteppers for the compressible case.

        The kinetic corrections satisfy linearized Boltzmann equations. Using a Lénard--Bernstein collision operator, these take Fokker--Planck-like forms \cite{Fokker_1914, Planck_1917} wherein pseudo-particles in the numerical model obey the neoclassical transport equations, with particle-independent Brownian drift terms. This offers a rigorous methodology for incorporating collisions into the particle transport model, without coupling the equations of motions for each particle.
        
        Works by Chen, Chacón et al. \cite{Chen_Chacón_Barnes_2011, Chacón_Chen_Barnes_2013, Chen_Chacón_2014, Chen_Chacón_2015} have developed structure-preserving particle pushers for neoclassical transport in the Vlasov equations, derived from Crank--Nicolson integrators. We show these too can can derive from a FET interpretation, similarly offering potential extensions to higher-order-in-time particle pushers. The FET formulation is used also to consider how the stochastic drift terms can be incorporated into the pushers. Stochastic gyrokinetic expansions are also discussed.

        Different options for the numerical implementation of these schemes are considered.

        Due to the efficacy of FET in the development of SP timesteppers for both the fluid and kinetic component, we hope this approach will prove effective in the future for developing SP timesteppers for the full hybrid model. We hope this will give us the opportunity to incorporate previously inaccessible kinetic effects into the highly effective, modern, finite-element MHD models.
    \end{abstract}
    
    
    \newpage
    \tableofcontents
    
    
    \newpage
    \pagenumbering{arabic}
    %\linenumbers\renewcommand\thelinenumber{\color{black!50}\arabic{linenumber}}
            \input{0 - introduction/main.tex}
        \part{Research}
            \input{1 - low-noise PiC models/main.tex}
            \input{2 - kinetic component/main.tex}
            \input{3 - fluid component/main.tex}
            \input{4 - numerical implementation/main.tex}
        \part{Project Overview}
            \input{5 - research plan/main.tex}
            \input{6 - summary/main.tex}
    
    
    %\section{}
    \newpage
    \pagenumbering{gobble}
        \printbibliography


    \newpage
    \pagenumbering{roman}
    \appendix
        \part{Appendices}
            \input{8 - Hilbert complexes/main.tex}
            \input{9 - weak conservation proofs/main.tex}
\end{document}

\end{document}

            \documentclass[12pt, a4paper]{report}

\documentclass[12pt, a4paper]{report}

\input{template/main.tex}

\title{\BA{Title in Progress...}}
\author{Boris Andrews}
\affil{Mathematical Institute, University of Oxford}
\date{\today}


\begin{document}
    \pagenumbering{gobble}
    \maketitle
    
    
    \begin{abstract}
        Magnetic confinement reactors---in particular tokamaks---offer one of the most promising options for achieving practical nuclear fusion, with the potential to provide virtually limitless, clean energy. The theoretical and numerical modeling of tokamak plasmas is simultaneously an essential component of effective reactor design, and a great research barrier. Tokamak operational conditions exhibit comparatively low Knudsen numbers. Kinetic effects, including kinetic waves and instabilities, Landau damping, bump-on-tail instabilities and more, are therefore highly influential in tokamak plasma dynamics. Purely fluid models are inherently incapable of capturing these effects, whereas the high dimensionality in purely kinetic models render them practically intractable for most relevant purposes.

        We consider a $\delta\!f$ decomposition model, with a macroscopic fluid background and microscopic kinetic correction, both fully coupled to each other. A similar manner of discretization is proposed to that used in the recent \texttt{STRUPHY} code \cite{Holderied_Possanner_Wang_2021, Holderied_2022, Li_et_al_2023} with a finite-element model for the background and a pseudo-particle/PiC model for the correction.

        The fluid background satisfies the full, non-linear, resistive, compressible, Hall MHD equations. \cite{Laakmann_Hu_Farrell_2022} introduces finite-element(-in-space) implicit timesteppers for the incompressible analogue to this system with structure-preserving (SP) properties in the ideal case, alongside parameter-robust preconditioners. We show that these timesteppers can derive from a finite-element-in-time (FET) (and finite-element-in-space) interpretation. The benefits of this reformulation are discussed, including the derivation of timesteppers that are higher order in time, and the quantifiable dissipative SP properties in the non-ideal, resistive case.
        
        We discuss possible options for extending this FET approach to timesteppers for the compressible case.

        The kinetic corrections satisfy linearized Boltzmann equations. Using a Lénard--Bernstein collision operator, these take Fokker--Planck-like forms \cite{Fokker_1914, Planck_1917} wherein pseudo-particles in the numerical model obey the neoclassical transport equations, with particle-independent Brownian drift terms. This offers a rigorous methodology for incorporating collisions into the particle transport model, without coupling the equations of motions for each particle.
        
        Works by Chen, Chacón et al. \cite{Chen_Chacón_Barnes_2011, Chacón_Chen_Barnes_2013, Chen_Chacón_2014, Chen_Chacón_2015} have developed structure-preserving particle pushers for neoclassical transport in the Vlasov equations, derived from Crank--Nicolson integrators. We show these too can can derive from a FET interpretation, similarly offering potential extensions to higher-order-in-time particle pushers. The FET formulation is used also to consider how the stochastic drift terms can be incorporated into the pushers. Stochastic gyrokinetic expansions are also discussed.

        Different options for the numerical implementation of these schemes are considered.

        Due to the efficacy of FET in the development of SP timesteppers for both the fluid and kinetic component, we hope this approach will prove effective in the future for developing SP timesteppers for the full hybrid model. We hope this will give us the opportunity to incorporate previously inaccessible kinetic effects into the highly effective, modern, finite-element MHD models.
    \end{abstract}
    
    
    \newpage
    \tableofcontents
    
    
    \newpage
    \pagenumbering{arabic}
    %\linenumbers\renewcommand\thelinenumber{\color{black!50}\arabic{linenumber}}
            \input{0 - introduction/main.tex}
        \part{Research}
            \input{1 - low-noise PiC models/main.tex}
            \input{2 - kinetic component/main.tex}
            \input{3 - fluid component/main.tex}
            \input{4 - numerical implementation/main.tex}
        \part{Project Overview}
            \input{5 - research plan/main.tex}
            \input{6 - summary/main.tex}
    
    
    %\section{}
    \newpage
    \pagenumbering{gobble}
        \printbibliography


    \newpage
    \pagenumbering{roman}
    \appendix
        \part{Appendices}
            \input{8 - Hilbert complexes/main.tex}
            \input{9 - weak conservation proofs/main.tex}
\end{document}


\title{\BA{Title in Progress...}}
\author{Boris Andrews}
\affil{Mathematical Institute, University of Oxford}
\date{\today}


\begin{document}
    \pagenumbering{gobble}
    \maketitle
    
    
    \begin{abstract}
        Magnetic confinement reactors---in particular tokamaks---offer one of the most promising options for achieving practical nuclear fusion, with the potential to provide virtually limitless, clean energy. The theoretical and numerical modeling of tokamak plasmas is simultaneously an essential component of effective reactor design, and a great research barrier. Tokamak operational conditions exhibit comparatively low Knudsen numbers. Kinetic effects, including kinetic waves and instabilities, Landau damping, bump-on-tail instabilities and more, are therefore highly influential in tokamak plasma dynamics. Purely fluid models are inherently incapable of capturing these effects, whereas the high dimensionality in purely kinetic models render them practically intractable for most relevant purposes.

        We consider a $\delta\!f$ decomposition model, with a macroscopic fluid background and microscopic kinetic correction, both fully coupled to each other. A similar manner of discretization is proposed to that used in the recent \texttt{STRUPHY} code \cite{Holderied_Possanner_Wang_2021, Holderied_2022, Li_et_al_2023} with a finite-element model for the background and a pseudo-particle/PiC model for the correction.

        The fluid background satisfies the full, non-linear, resistive, compressible, Hall MHD equations. \cite{Laakmann_Hu_Farrell_2022} introduces finite-element(-in-space) implicit timesteppers for the incompressible analogue to this system with structure-preserving (SP) properties in the ideal case, alongside parameter-robust preconditioners. We show that these timesteppers can derive from a finite-element-in-time (FET) (and finite-element-in-space) interpretation. The benefits of this reformulation are discussed, including the derivation of timesteppers that are higher order in time, and the quantifiable dissipative SP properties in the non-ideal, resistive case.
        
        We discuss possible options for extending this FET approach to timesteppers for the compressible case.

        The kinetic corrections satisfy linearized Boltzmann equations. Using a Lénard--Bernstein collision operator, these take Fokker--Planck-like forms \cite{Fokker_1914, Planck_1917} wherein pseudo-particles in the numerical model obey the neoclassical transport equations, with particle-independent Brownian drift terms. This offers a rigorous methodology for incorporating collisions into the particle transport model, without coupling the equations of motions for each particle.
        
        Works by Chen, Chacón et al. \cite{Chen_Chacón_Barnes_2011, Chacón_Chen_Barnes_2013, Chen_Chacón_2014, Chen_Chacón_2015} have developed structure-preserving particle pushers for neoclassical transport in the Vlasov equations, derived from Crank--Nicolson integrators. We show these too can can derive from a FET interpretation, similarly offering potential extensions to higher-order-in-time particle pushers. The FET formulation is used also to consider how the stochastic drift terms can be incorporated into the pushers. Stochastic gyrokinetic expansions are also discussed.

        Different options for the numerical implementation of these schemes are considered.

        Due to the efficacy of FET in the development of SP timesteppers for both the fluid and kinetic component, we hope this approach will prove effective in the future for developing SP timesteppers for the full hybrid model. We hope this will give us the opportunity to incorporate previously inaccessible kinetic effects into the highly effective, modern, finite-element MHD models.
    \end{abstract}
    
    
    \newpage
    \tableofcontents
    
    
    \newpage
    \pagenumbering{arabic}
    %\linenumbers\renewcommand\thelinenumber{\color{black!50}\arabic{linenumber}}
            \documentclass[12pt, a4paper]{report}

\input{template/main.tex}

\title{\BA{Title in Progress...}}
\author{Boris Andrews}
\affil{Mathematical Institute, University of Oxford}
\date{\today}


\begin{document}
    \pagenumbering{gobble}
    \maketitle
    
    
    \begin{abstract}
        Magnetic confinement reactors---in particular tokamaks---offer one of the most promising options for achieving practical nuclear fusion, with the potential to provide virtually limitless, clean energy. The theoretical and numerical modeling of tokamak plasmas is simultaneously an essential component of effective reactor design, and a great research barrier. Tokamak operational conditions exhibit comparatively low Knudsen numbers. Kinetic effects, including kinetic waves and instabilities, Landau damping, bump-on-tail instabilities and more, are therefore highly influential in tokamak plasma dynamics. Purely fluid models are inherently incapable of capturing these effects, whereas the high dimensionality in purely kinetic models render them practically intractable for most relevant purposes.

        We consider a $\delta\!f$ decomposition model, with a macroscopic fluid background and microscopic kinetic correction, both fully coupled to each other. A similar manner of discretization is proposed to that used in the recent \texttt{STRUPHY} code \cite{Holderied_Possanner_Wang_2021, Holderied_2022, Li_et_al_2023} with a finite-element model for the background and a pseudo-particle/PiC model for the correction.

        The fluid background satisfies the full, non-linear, resistive, compressible, Hall MHD equations. \cite{Laakmann_Hu_Farrell_2022} introduces finite-element(-in-space) implicit timesteppers for the incompressible analogue to this system with structure-preserving (SP) properties in the ideal case, alongside parameter-robust preconditioners. We show that these timesteppers can derive from a finite-element-in-time (FET) (and finite-element-in-space) interpretation. The benefits of this reformulation are discussed, including the derivation of timesteppers that are higher order in time, and the quantifiable dissipative SP properties in the non-ideal, resistive case.
        
        We discuss possible options for extending this FET approach to timesteppers for the compressible case.

        The kinetic corrections satisfy linearized Boltzmann equations. Using a Lénard--Bernstein collision operator, these take Fokker--Planck-like forms \cite{Fokker_1914, Planck_1917} wherein pseudo-particles in the numerical model obey the neoclassical transport equations, with particle-independent Brownian drift terms. This offers a rigorous methodology for incorporating collisions into the particle transport model, without coupling the equations of motions for each particle.
        
        Works by Chen, Chacón et al. \cite{Chen_Chacón_Barnes_2011, Chacón_Chen_Barnes_2013, Chen_Chacón_2014, Chen_Chacón_2015} have developed structure-preserving particle pushers for neoclassical transport in the Vlasov equations, derived from Crank--Nicolson integrators. We show these too can can derive from a FET interpretation, similarly offering potential extensions to higher-order-in-time particle pushers. The FET formulation is used also to consider how the stochastic drift terms can be incorporated into the pushers. Stochastic gyrokinetic expansions are also discussed.

        Different options for the numerical implementation of these schemes are considered.

        Due to the efficacy of FET in the development of SP timesteppers for both the fluid and kinetic component, we hope this approach will prove effective in the future for developing SP timesteppers for the full hybrid model. We hope this will give us the opportunity to incorporate previously inaccessible kinetic effects into the highly effective, modern, finite-element MHD models.
    \end{abstract}
    
    
    \newpage
    \tableofcontents
    
    
    \newpage
    \pagenumbering{arabic}
    %\linenumbers\renewcommand\thelinenumber{\color{black!50}\arabic{linenumber}}
            \input{0 - introduction/main.tex}
        \part{Research}
            \input{1 - low-noise PiC models/main.tex}
            \input{2 - kinetic component/main.tex}
            \input{3 - fluid component/main.tex}
            \input{4 - numerical implementation/main.tex}
        \part{Project Overview}
            \input{5 - research plan/main.tex}
            \input{6 - summary/main.tex}
    
    
    %\section{}
    \newpage
    \pagenumbering{gobble}
        \printbibliography


    \newpage
    \pagenumbering{roman}
    \appendix
        \part{Appendices}
            \input{8 - Hilbert complexes/main.tex}
            \input{9 - weak conservation proofs/main.tex}
\end{document}

        \part{Research}
            \documentclass[12pt, a4paper]{report}

\input{template/main.tex}

\title{\BA{Title in Progress...}}
\author{Boris Andrews}
\affil{Mathematical Institute, University of Oxford}
\date{\today}


\begin{document}
    \pagenumbering{gobble}
    \maketitle
    
    
    \begin{abstract}
        Magnetic confinement reactors---in particular tokamaks---offer one of the most promising options for achieving practical nuclear fusion, with the potential to provide virtually limitless, clean energy. The theoretical and numerical modeling of tokamak plasmas is simultaneously an essential component of effective reactor design, and a great research barrier. Tokamak operational conditions exhibit comparatively low Knudsen numbers. Kinetic effects, including kinetic waves and instabilities, Landau damping, bump-on-tail instabilities and more, are therefore highly influential in tokamak plasma dynamics. Purely fluid models are inherently incapable of capturing these effects, whereas the high dimensionality in purely kinetic models render them practically intractable for most relevant purposes.

        We consider a $\delta\!f$ decomposition model, with a macroscopic fluid background and microscopic kinetic correction, both fully coupled to each other. A similar manner of discretization is proposed to that used in the recent \texttt{STRUPHY} code \cite{Holderied_Possanner_Wang_2021, Holderied_2022, Li_et_al_2023} with a finite-element model for the background and a pseudo-particle/PiC model for the correction.

        The fluid background satisfies the full, non-linear, resistive, compressible, Hall MHD equations. \cite{Laakmann_Hu_Farrell_2022} introduces finite-element(-in-space) implicit timesteppers for the incompressible analogue to this system with structure-preserving (SP) properties in the ideal case, alongside parameter-robust preconditioners. We show that these timesteppers can derive from a finite-element-in-time (FET) (and finite-element-in-space) interpretation. The benefits of this reformulation are discussed, including the derivation of timesteppers that are higher order in time, and the quantifiable dissipative SP properties in the non-ideal, resistive case.
        
        We discuss possible options for extending this FET approach to timesteppers for the compressible case.

        The kinetic corrections satisfy linearized Boltzmann equations. Using a Lénard--Bernstein collision operator, these take Fokker--Planck-like forms \cite{Fokker_1914, Planck_1917} wherein pseudo-particles in the numerical model obey the neoclassical transport equations, with particle-independent Brownian drift terms. This offers a rigorous methodology for incorporating collisions into the particle transport model, without coupling the equations of motions for each particle.
        
        Works by Chen, Chacón et al. \cite{Chen_Chacón_Barnes_2011, Chacón_Chen_Barnes_2013, Chen_Chacón_2014, Chen_Chacón_2015} have developed structure-preserving particle pushers for neoclassical transport in the Vlasov equations, derived from Crank--Nicolson integrators. We show these too can can derive from a FET interpretation, similarly offering potential extensions to higher-order-in-time particle pushers. The FET formulation is used also to consider how the stochastic drift terms can be incorporated into the pushers. Stochastic gyrokinetic expansions are also discussed.

        Different options for the numerical implementation of these schemes are considered.

        Due to the efficacy of FET in the development of SP timesteppers for both the fluid and kinetic component, we hope this approach will prove effective in the future for developing SP timesteppers for the full hybrid model. We hope this will give us the opportunity to incorporate previously inaccessible kinetic effects into the highly effective, modern, finite-element MHD models.
    \end{abstract}
    
    
    \newpage
    \tableofcontents
    
    
    \newpage
    \pagenumbering{arabic}
    %\linenumbers\renewcommand\thelinenumber{\color{black!50}\arabic{linenumber}}
            \input{0 - introduction/main.tex}
        \part{Research}
            \input{1 - low-noise PiC models/main.tex}
            \input{2 - kinetic component/main.tex}
            \input{3 - fluid component/main.tex}
            \input{4 - numerical implementation/main.tex}
        \part{Project Overview}
            \input{5 - research plan/main.tex}
            \input{6 - summary/main.tex}
    
    
    %\section{}
    \newpage
    \pagenumbering{gobble}
        \printbibliography


    \newpage
    \pagenumbering{roman}
    \appendix
        \part{Appendices}
            \input{8 - Hilbert complexes/main.tex}
            \input{9 - weak conservation proofs/main.tex}
\end{document}

            \documentclass[12pt, a4paper]{report}

\input{template/main.tex}

\title{\BA{Title in Progress...}}
\author{Boris Andrews}
\affil{Mathematical Institute, University of Oxford}
\date{\today}


\begin{document}
    \pagenumbering{gobble}
    \maketitle
    
    
    \begin{abstract}
        Magnetic confinement reactors---in particular tokamaks---offer one of the most promising options for achieving practical nuclear fusion, with the potential to provide virtually limitless, clean energy. The theoretical and numerical modeling of tokamak plasmas is simultaneously an essential component of effective reactor design, and a great research barrier. Tokamak operational conditions exhibit comparatively low Knudsen numbers. Kinetic effects, including kinetic waves and instabilities, Landau damping, bump-on-tail instabilities and more, are therefore highly influential in tokamak plasma dynamics. Purely fluid models are inherently incapable of capturing these effects, whereas the high dimensionality in purely kinetic models render them practically intractable for most relevant purposes.

        We consider a $\delta\!f$ decomposition model, with a macroscopic fluid background and microscopic kinetic correction, both fully coupled to each other. A similar manner of discretization is proposed to that used in the recent \texttt{STRUPHY} code \cite{Holderied_Possanner_Wang_2021, Holderied_2022, Li_et_al_2023} with a finite-element model for the background and a pseudo-particle/PiC model for the correction.

        The fluid background satisfies the full, non-linear, resistive, compressible, Hall MHD equations. \cite{Laakmann_Hu_Farrell_2022} introduces finite-element(-in-space) implicit timesteppers for the incompressible analogue to this system with structure-preserving (SP) properties in the ideal case, alongside parameter-robust preconditioners. We show that these timesteppers can derive from a finite-element-in-time (FET) (and finite-element-in-space) interpretation. The benefits of this reformulation are discussed, including the derivation of timesteppers that are higher order in time, and the quantifiable dissipative SP properties in the non-ideal, resistive case.
        
        We discuss possible options for extending this FET approach to timesteppers for the compressible case.

        The kinetic corrections satisfy linearized Boltzmann equations. Using a Lénard--Bernstein collision operator, these take Fokker--Planck-like forms \cite{Fokker_1914, Planck_1917} wherein pseudo-particles in the numerical model obey the neoclassical transport equations, with particle-independent Brownian drift terms. This offers a rigorous methodology for incorporating collisions into the particle transport model, without coupling the equations of motions for each particle.
        
        Works by Chen, Chacón et al. \cite{Chen_Chacón_Barnes_2011, Chacón_Chen_Barnes_2013, Chen_Chacón_2014, Chen_Chacón_2015} have developed structure-preserving particle pushers for neoclassical transport in the Vlasov equations, derived from Crank--Nicolson integrators. We show these too can can derive from a FET interpretation, similarly offering potential extensions to higher-order-in-time particle pushers. The FET formulation is used also to consider how the stochastic drift terms can be incorporated into the pushers. Stochastic gyrokinetic expansions are also discussed.

        Different options for the numerical implementation of these schemes are considered.

        Due to the efficacy of FET in the development of SP timesteppers for both the fluid and kinetic component, we hope this approach will prove effective in the future for developing SP timesteppers for the full hybrid model. We hope this will give us the opportunity to incorporate previously inaccessible kinetic effects into the highly effective, modern, finite-element MHD models.
    \end{abstract}
    
    
    \newpage
    \tableofcontents
    
    
    \newpage
    \pagenumbering{arabic}
    %\linenumbers\renewcommand\thelinenumber{\color{black!50}\arabic{linenumber}}
            \input{0 - introduction/main.tex}
        \part{Research}
            \input{1 - low-noise PiC models/main.tex}
            \input{2 - kinetic component/main.tex}
            \input{3 - fluid component/main.tex}
            \input{4 - numerical implementation/main.tex}
        \part{Project Overview}
            \input{5 - research plan/main.tex}
            \input{6 - summary/main.tex}
    
    
    %\section{}
    \newpage
    \pagenumbering{gobble}
        \printbibliography


    \newpage
    \pagenumbering{roman}
    \appendix
        \part{Appendices}
            \input{8 - Hilbert complexes/main.tex}
            \input{9 - weak conservation proofs/main.tex}
\end{document}

            \documentclass[12pt, a4paper]{report}

\input{template/main.tex}

\title{\BA{Title in Progress...}}
\author{Boris Andrews}
\affil{Mathematical Institute, University of Oxford}
\date{\today}


\begin{document}
    \pagenumbering{gobble}
    \maketitle
    
    
    \begin{abstract}
        Magnetic confinement reactors---in particular tokamaks---offer one of the most promising options for achieving practical nuclear fusion, with the potential to provide virtually limitless, clean energy. The theoretical and numerical modeling of tokamak plasmas is simultaneously an essential component of effective reactor design, and a great research barrier. Tokamak operational conditions exhibit comparatively low Knudsen numbers. Kinetic effects, including kinetic waves and instabilities, Landau damping, bump-on-tail instabilities and more, are therefore highly influential in tokamak plasma dynamics. Purely fluid models are inherently incapable of capturing these effects, whereas the high dimensionality in purely kinetic models render them practically intractable for most relevant purposes.

        We consider a $\delta\!f$ decomposition model, with a macroscopic fluid background and microscopic kinetic correction, both fully coupled to each other. A similar manner of discretization is proposed to that used in the recent \texttt{STRUPHY} code \cite{Holderied_Possanner_Wang_2021, Holderied_2022, Li_et_al_2023} with a finite-element model for the background and a pseudo-particle/PiC model for the correction.

        The fluid background satisfies the full, non-linear, resistive, compressible, Hall MHD equations. \cite{Laakmann_Hu_Farrell_2022} introduces finite-element(-in-space) implicit timesteppers for the incompressible analogue to this system with structure-preserving (SP) properties in the ideal case, alongside parameter-robust preconditioners. We show that these timesteppers can derive from a finite-element-in-time (FET) (and finite-element-in-space) interpretation. The benefits of this reformulation are discussed, including the derivation of timesteppers that are higher order in time, and the quantifiable dissipative SP properties in the non-ideal, resistive case.
        
        We discuss possible options for extending this FET approach to timesteppers for the compressible case.

        The kinetic corrections satisfy linearized Boltzmann equations. Using a Lénard--Bernstein collision operator, these take Fokker--Planck-like forms \cite{Fokker_1914, Planck_1917} wherein pseudo-particles in the numerical model obey the neoclassical transport equations, with particle-independent Brownian drift terms. This offers a rigorous methodology for incorporating collisions into the particle transport model, without coupling the equations of motions for each particle.
        
        Works by Chen, Chacón et al. \cite{Chen_Chacón_Barnes_2011, Chacón_Chen_Barnes_2013, Chen_Chacón_2014, Chen_Chacón_2015} have developed structure-preserving particle pushers for neoclassical transport in the Vlasov equations, derived from Crank--Nicolson integrators. We show these too can can derive from a FET interpretation, similarly offering potential extensions to higher-order-in-time particle pushers. The FET formulation is used also to consider how the stochastic drift terms can be incorporated into the pushers. Stochastic gyrokinetic expansions are also discussed.

        Different options for the numerical implementation of these schemes are considered.

        Due to the efficacy of FET in the development of SP timesteppers for both the fluid and kinetic component, we hope this approach will prove effective in the future for developing SP timesteppers for the full hybrid model. We hope this will give us the opportunity to incorporate previously inaccessible kinetic effects into the highly effective, modern, finite-element MHD models.
    \end{abstract}
    
    
    \newpage
    \tableofcontents
    
    
    \newpage
    \pagenumbering{arabic}
    %\linenumbers\renewcommand\thelinenumber{\color{black!50}\arabic{linenumber}}
            \input{0 - introduction/main.tex}
        \part{Research}
            \input{1 - low-noise PiC models/main.tex}
            \input{2 - kinetic component/main.tex}
            \input{3 - fluid component/main.tex}
            \input{4 - numerical implementation/main.tex}
        \part{Project Overview}
            \input{5 - research plan/main.tex}
            \input{6 - summary/main.tex}
    
    
    %\section{}
    \newpage
    \pagenumbering{gobble}
        \printbibliography


    \newpage
    \pagenumbering{roman}
    \appendix
        \part{Appendices}
            \input{8 - Hilbert complexes/main.tex}
            \input{9 - weak conservation proofs/main.tex}
\end{document}

            \documentclass[12pt, a4paper]{report}

\input{template/main.tex}

\title{\BA{Title in Progress...}}
\author{Boris Andrews}
\affil{Mathematical Institute, University of Oxford}
\date{\today}


\begin{document}
    \pagenumbering{gobble}
    \maketitle
    
    
    \begin{abstract}
        Magnetic confinement reactors---in particular tokamaks---offer one of the most promising options for achieving practical nuclear fusion, with the potential to provide virtually limitless, clean energy. The theoretical and numerical modeling of tokamak plasmas is simultaneously an essential component of effective reactor design, and a great research barrier. Tokamak operational conditions exhibit comparatively low Knudsen numbers. Kinetic effects, including kinetic waves and instabilities, Landau damping, bump-on-tail instabilities and more, are therefore highly influential in tokamak plasma dynamics. Purely fluid models are inherently incapable of capturing these effects, whereas the high dimensionality in purely kinetic models render them practically intractable for most relevant purposes.

        We consider a $\delta\!f$ decomposition model, with a macroscopic fluid background and microscopic kinetic correction, both fully coupled to each other. A similar manner of discretization is proposed to that used in the recent \texttt{STRUPHY} code \cite{Holderied_Possanner_Wang_2021, Holderied_2022, Li_et_al_2023} with a finite-element model for the background and a pseudo-particle/PiC model for the correction.

        The fluid background satisfies the full, non-linear, resistive, compressible, Hall MHD equations. \cite{Laakmann_Hu_Farrell_2022} introduces finite-element(-in-space) implicit timesteppers for the incompressible analogue to this system with structure-preserving (SP) properties in the ideal case, alongside parameter-robust preconditioners. We show that these timesteppers can derive from a finite-element-in-time (FET) (and finite-element-in-space) interpretation. The benefits of this reformulation are discussed, including the derivation of timesteppers that are higher order in time, and the quantifiable dissipative SP properties in the non-ideal, resistive case.
        
        We discuss possible options for extending this FET approach to timesteppers for the compressible case.

        The kinetic corrections satisfy linearized Boltzmann equations. Using a Lénard--Bernstein collision operator, these take Fokker--Planck-like forms \cite{Fokker_1914, Planck_1917} wherein pseudo-particles in the numerical model obey the neoclassical transport equations, with particle-independent Brownian drift terms. This offers a rigorous methodology for incorporating collisions into the particle transport model, without coupling the equations of motions for each particle.
        
        Works by Chen, Chacón et al. \cite{Chen_Chacón_Barnes_2011, Chacón_Chen_Barnes_2013, Chen_Chacón_2014, Chen_Chacón_2015} have developed structure-preserving particle pushers for neoclassical transport in the Vlasov equations, derived from Crank--Nicolson integrators. We show these too can can derive from a FET interpretation, similarly offering potential extensions to higher-order-in-time particle pushers. The FET formulation is used also to consider how the stochastic drift terms can be incorporated into the pushers. Stochastic gyrokinetic expansions are also discussed.

        Different options for the numerical implementation of these schemes are considered.

        Due to the efficacy of FET in the development of SP timesteppers for both the fluid and kinetic component, we hope this approach will prove effective in the future for developing SP timesteppers for the full hybrid model. We hope this will give us the opportunity to incorporate previously inaccessible kinetic effects into the highly effective, modern, finite-element MHD models.
    \end{abstract}
    
    
    \newpage
    \tableofcontents
    
    
    \newpage
    \pagenumbering{arabic}
    %\linenumbers\renewcommand\thelinenumber{\color{black!50}\arabic{linenumber}}
            \input{0 - introduction/main.tex}
        \part{Research}
            \input{1 - low-noise PiC models/main.tex}
            \input{2 - kinetic component/main.tex}
            \input{3 - fluid component/main.tex}
            \input{4 - numerical implementation/main.tex}
        \part{Project Overview}
            \input{5 - research plan/main.tex}
            \input{6 - summary/main.tex}
    
    
    %\section{}
    \newpage
    \pagenumbering{gobble}
        \printbibliography


    \newpage
    \pagenumbering{roman}
    \appendix
        \part{Appendices}
            \input{8 - Hilbert complexes/main.tex}
            \input{9 - weak conservation proofs/main.tex}
\end{document}

        \part{Project Overview}
            \documentclass[12pt, a4paper]{report}

\input{template/main.tex}

\title{\BA{Title in Progress...}}
\author{Boris Andrews}
\affil{Mathematical Institute, University of Oxford}
\date{\today}


\begin{document}
    \pagenumbering{gobble}
    \maketitle
    
    
    \begin{abstract}
        Magnetic confinement reactors---in particular tokamaks---offer one of the most promising options for achieving practical nuclear fusion, with the potential to provide virtually limitless, clean energy. The theoretical and numerical modeling of tokamak plasmas is simultaneously an essential component of effective reactor design, and a great research barrier. Tokamak operational conditions exhibit comparatively low Knudsen numbers. Kinetic effects, including kinetic waves and instabilities, Landau damping, bump-on-tail instabilities and more, are therefore highly influential in tokamak plasma dynamics. Purely fluid models are inherently incapable of capturing these effects, whereas the high dimensionality in purely kinetic models render them practically intractable for most relevant purposes.

        We consider a $\delta\!f$ decomposition model, with a macroscopic fluid background and microscopic kinetic correction, both fully coupled to each other. A similar manner of discretization is proposed to that used in the recent \texttt{STRUPHY} code \cite{Holderied_Possanner_Wang_2021, Holderied_2022, Li_et_al_2023} with a finite-element model for the background and a pseudo-particle/PiC model for the correction.

        The fluid background satisfies the full, non-linear, resistive, compressible, Hall MHD equations. \cite{Laakmann_Hu_Farrell_2022} introduces finite-element(-in-space) implicit timesteppers for the incompressible analogue to this system with structure-preserving (SP) properties in the ideal case, alongside parameter-robust preconditioners. We show that these timesteppers can derive from a finite-element-in-time (FET) (and finite-element-in-space) interpretation. The benefits of this reformulation are discussed, including the derivation of timesteppers that are higher order in time, and the quantifiable dissipative SP properties in the non-ideal, resistive case.
        
        We discuss possible options for extending this FET approach to timesteppers for the compressible case.

        The kinetic corrections satisfy linearized Boltzmann equations. Using a Lénard--Bernstein collision operator, these take Fokker--Planck-like forms \cite{Fokker_1914, Planck_1917} wherein pseudo-particles in the numerical model obey the neoclassical transport equations, with particle-independent Brownian drift terms. This offers a rigorous methodology for incorporating collisions into the particle transport model, without coupling the equations of motions for each particle.
        
        Works by Chen, Chacón et al. \cite{Chen_Chacón_Barnes_2011, Chacón_Chen_Barnes_2013, Chen_Chacón_2014, Chen_Chacón_2015} have developed structure-preserving particle pushers for neoclassical transport in the Vlasov equations, derived from Crank--Nicolson integrators. We show these too can can derive from a FET interpretation, similarly offering potential extensions to higher-order-in-time particle pushers. The FET formulation is used also to consider how the stochastic drift terms can be incorporated into the pushers. Stochastic gyrokinetic expansions are also discussed.

        Different options for the numerical implementation of these schemes are considered.

        Due to the efficacy of FET in the development of SP timesteppers for both the fluid and kinetic component, we hope this approach will prove effective in the future for developing SP timesteppers for the full hybrid model. We hope this will give us the opportunity to incorporate previously inaccessible kinetic effects into the highly effective, modern, finite-element MHD models.
    \end{abstract}
    
    
    \newpage
    \tableofcontents
    
    
    \newpage
    \pagenumbering{arabic}
    %\linenumbers\renewcommand\thelinenumber{\color{black!50}\arabic{linenumber}}
            \input{0 - introduction/main.tex}
        \part{Research}
            \input{1 - low-noise PiC models/main.tex}
            \input{2 - kinetic component/main.tex}
            \input{3 - fluid component/main.tex}
            \input{4 - numerical implementation/main.tex}
        \part{Project Overview}
            \input{5 - research plan/main.tex}
            \input{6 - summary/main.tex}
    
    
    %\section{}
    \newpage
    \pagenumbering{gobble}
        \printbibliography


    \newpage
    \pagenumbering{roman}
    \appendix
        \part{Appendices}
            \input{8 - Hilbert complexes/main.tex}
            \input{9 - weak conservation proofs/main.tex}
\end{document}

            \documentclass[12pt, a4paper]{report}

\input{template/main.tex}

\title{\BA{Title in Progress...}}
\author{Boris Andrews}
\affil{Mathematical Institute, University of Oxford}
\date{\today}


\begin{document}
    \pagenumbering{gobble}
    \maketitle
    
    
    \begin{abstract}
        Magnetic confinement reactors---in particular tokamaks---offer one of the most promising options for achieving practical nuclear fusion, with the potential to provide virtually limitless, clean energy. The theoretical and numerical modeling of tokamak plasmas is simultaneously an essential component of effective reactor design, and a great research barrier. Tokamak operational conditions exhibit comparatively low Knudsen numbers. Kinetic effects, including kinetic waves and instabilities, Landau damping, bump-on-tail instabilities and more, are therefore highly influential in tokamak plasma dynamics. Purely fluid models are inherently incapable of capturing these effects, whereas the high dimensionality in purely kinetic models render them practically intractable for most relevant purposes.

        We consider a $\delta\!f$ decomposition model, with a macroscopic fluid background and microscopic kinetic correction, both fully coupled to each other. A similar manner of discretization is proposed to that used in the recent \texttt{STRUPHY} code \cite{Holderied_Possanner_Wang_2021, Holderied_2022, Li_et_al_2023} with a finite-element model for the background and a pseudo-particle/PiC model for the correction.

        The fluid background satisfies the full, non-linear, resistive, compressible, Hall MHD equations. \cite{Laakmann_Hu_Farrell_2022} introduces finite-element(-in-space) implicit timesteppers for the incompressible analogue to this system with structure-preserving (SP) properties in the ideal case, alongside parameter-robust preconditioners. We show that these timesteppers can derive from a finite-element-in-time (FET) (and finite-element-in-space) interpretation. The benefits of this reformulation are discussed, including the derivation of timesteppers that are higher order in time, and the quantifiable dissipative SP properties in the non-ideal, resistive case.
        
        We discuss possible options for extending this FET approach to timesteppers for the compressible case.

        The kinetic corrections satisfy linearized Boltzmann equations. Using a Lénard--Bernstein collision operator, these take Fokker--Planck-like forms \cite{Fokker_1914, Planck_1917} wherein pseudo-particles in the numerical model obey the neoclassical transport equations, with particle-independent Brownian drift terms. This offers a rigorous methodology for incorporating collisions into the particle transport model, without coupling the equations of motions for each particle.
        
        Works by Chen, Chacón et al. \cite{Chen_Chacón_Barnes_2011, Chacón_Chen_Barnes_2013, Chen_Chacón_2014, Chen_Chacón_2015} have developed structure-preserving particle pushers for neoclassical transport in the Vlasov equations, derived from Crank--Nicolson integrators. We show these too can can derive from a FET interpretation, similarly offering potential extensions to higher-order-in-time particle pushers. The FET formulation is used also to consider how the stochastic drift terms can be incorporated into the pushers. Stochastic gyrokinetic expansions are also discussed.

        Different options for the numerical implementation of these schemes are considered.

        Due to the efficacy of FET in the development of SP timesteppers for both the fluid and kinetic component, we hope this approach will prove effective in the future for developing SP timesteppers for the full hybrid model. We hope this will give us the opportunity to incorporate previously inaccessible kinetic effects into the highly effective, modern, finite-element MHD models.
    \end{abstract}
    
    
    \newpage
    \tableofcontents
    
    
    \newpage
    \pagenumbering{arabic}
    %\linenumbers\renewcommand\thelinenumber{\color{black!50}\arabic{linenumber}}
            \input{0 - introduction/main.tex}
        \part{Research}
            \input{1 - low-noise PiC models/main.tex}
            \input{2 - kinetic component/main.tex}
            \input{3 - fluid component/main.tex}
            \input{4 - numerical implementation/main.tex}
        \part{Project Overview}
            \input{5 - research plan/main.tex}
            \input{6 - summary/main.tex}
    
    
    %\section{}
    \newpage
    \pagenumbering{gobble}
        \printbibliography


    \newpage
    \pagenumbering{roman}
    \appendix
        \part{Appendices}
            \input{8 - Hilbert complexes/main.tex}
            \input{9 - weak conservation proofs/main.tex}
\end{document}

    
    
    %\section{}
    \newpage
    \pagenumbering{gobble}
        \printbibliography


    \newpage
    \pagenumbering{roman}
    \appendix
        \part{Appendices}
            \documentclass[12pt, a4paper]{report}

\input{template/main.tex}

\title{\BA{Title in Progress...}}
\author{Boris Andrews}
\affil{Mathematical Institute, University of Oxford}
\date{\today}


\begin{document}
    \pagenumbering{gobble}
    \maketitle
    
    
    \begin{abstract}
        Magnetic confinement reactors---in particular tokamaks---offer one of the most promising options for achieving practical nuclear fusion, with the potential to provide virtually limitless, clean energy. The theoretical and numerical modeling of tokamak plasmas is simultaneously an essential component of effective reactor design, and a great research barrier. Tokamak operational conditions exhibit comparatively low Knudsen numbers. Kinetic effects, including kinetic waves and instabilities, Landau damping, bump-on-tail instabilities and more, are therefore highly influential in tokamak plasma dynamics. Purely fluid models are inherently incapable of capturing these effects, whereas the high dimensionality in purely kinetic models render them practically intractable for most relevant purposes.

        We consider a $\delta\!f$ decomposition model, with a macroscopic fluid background and microscopic kinetic correction, both fully coupled to each other. A similar manner of discretization is proposed to that used in the recent \texttt{STRUPHY} code \cite{Holderied_Possanner_Wang_2021, Holderied_2022, Li_et_al_2023} with a finite-element model for the background and a pseudo-particle/PiC model for the correction.

        The fluid background satisfies the full, non-linear, resistive, compressible, Hall MHD equations. \cite{Laakmann_Hu_Farrell_2022} introduces finite-element(-in-space) implicit timesteppers for the incompressible analogue to this system with structure-preserving (SP) properties in the ideal case, alongside parameter-robust preconditioners. We show that these timesteppers can derive from a finite-element-in-time (FET) (and finite-element-in-space) interpretation. The benefits of this reformulation are discussed, including the derivation of timesteppers that are higher order in time, and the quantifiable dissipative SP properties in the non-ideal, resistive case.
        
        We discuss possible options for extending this FET approach to timesteppers for the compressible case.

        The kinetic corrections satisfy linearized Boltzmann equations. Using a Lénard--Bernstein collision operator, these take Fokker--Planck-like forms \cite{Fokker_1914, Planck_1917} wherein pseudo-particles in the numerical model obey the neoclassical transport equations, with particle-independent Brownian drift terms. This offers a rigorous methodology for incorporating collisions into the particle transport model, without coupling the equations of motions for each particle.
        
        Works by Chen, Chacón et al. \cite{Chen_Chacón_Barnes_2011, Chacón_Chen_Barnes_2013, Chen_Chacón_2014, Chen_Chacón_2015} have developed structure-preserving particle pushers for neoclassical transport in the Vlasov equations, derived from Crank--Nicolson integrators. We show these too can can derive from a FET interpretation, similarly offering potential extensions to higher-order-in-time particle pushers. The FET formulation is used also to consider how the stochastic drift terms can be incorporated into the pushers. Stochastic gyrokinetic expansions are also discussed.

        Different options for the numerical implementation of these schemes are considered.

        Due to the efficacy of FET in the development of SP timesteppers for both the fluid and kinetic component, we hope this approach will prove effective in the future for developing SP timesteppers for the full hybrid model. We hope this will give us the opportunity to incorporate previously inaccessible kinetic effects into the highly effective, modern, finite-element MHD models.
    \end{abstract}
    
    
    \newpage
    \tableofcontents
    
    
    \newpage
    \pagenumbering{arabic}
    %\linenumbers\renewcommand\thelinenumber{\color{black!50}\arabic{linenumber}}
            \input{0 - introduction/main.tex}
        \part{Research}
            \input{1 - low-noise PiC models/main.tex}
            \input{2 - kinetic component/main.tex}
            \input{3 - fluid component/main.tex}
            \input{4 - numerical implementation/main.tex}
        \part{Project Overview}
            \input{5 - research plan/main.tex}
            \input{6 - summary/main.tex}
    
    
    %\section{}
    \newpage
    \pagenumbering{gobble}
        \printbibliography


    \newpage
    \pagenumbering{roman}
    \appendix
        \part{Appendices}
            \input{8 - Hilbert complexes/main.tex}
            \input{9 - weak conservation proofs/main.tex}
\end{document}

            \documentclass[12pt, a4paper]{report}

\input{template/main.tex}

\title{\BA{Title in Progress...}}
\author{Boris Andrews}
\affil{Mathematical Institute, University of Oxford}
\date{\today}


\begin{document}
    \pagenumbering{gobble}
    \maketitle
    
    
    \begin{abstract}
        Magnetic confinement reactors---in particular tokamaks---offer one of the most promising options for achieving practical nuclear fusion, with the potential to provide virtually limitless, clean energy. The theoretical and numerical modeling of tokamak plasmas is simultaneously an essential component of effective reactor design, and a great research barrier. Tokamak operational conditions exhibit comparatively low Knudsen numbers. Kinetic effects, including kinetic waves and instabilities, Landau damping, bump-on-tail instabilities and more, are therefore highly influential in tokamak plasma dynamics. Purely fluid models are inherently incapable of capturing these effects, whereas the high dimensionality in purely kinetic models render them practically intractable for most relevant purposes.

        We consider a $\delta\!f$ decomposition model, with a macroscopic fluid background and microscopic kinetic correction, both fully coupled to each other. A similar manner of discretization is proposed to that used in the recent \texttt{STRUPHY} code \cite{Holderied_Possanner_Wang_2021, Holderied_2022, Li_et_al_2023} with a finite-element model for the background and a pseudo-particle/PiC model for the correction.

        The fluid background satisfies the full, non-linear, resistive, compressible, Hall MHD equations. \cite{Laakmann_Hu_Farrell_2022} introduces finite-element(-in-space) implicit timesteppers for the incompressible analogue to this system with structure-preserving (SP) properties in the ideal case, alongside parameter-robust preconditioners. We show that these timesteppers can derive from a finite-element-in-time (FET) (and finite-element-in-space) interpretation. The benefits of this reformulation are discussed, including the derivation of timesteppers that are higher order in time, and the quantifiable dissipative SP properties in the non-ideal, resistive case.
        
        We discuss possible options for extending this FET approach to timesteppers for the compressible case.

        The kinetic corrections satisfy linearized Boltzmann equations. Using a Lénard--Bernstein collision operator, these take Fokker--Planck-like forms \cite{Fokker_1914, Planck_1917} wherein pseudo-particles in the numerical model obey the neoclassical transport equations, with particle-independent Brownian drift terms. This offers a rigorous methodology for incorporating collisions into the particle transport model, without coupling the equations of motions for each particle.
        
        Works by Chen, Chacón et al. \cite{Chen_Chacón_Barnes_2011, Chacón_Chen_Barnes_2013, Chen_Chacón_2014, Chen_Chacón_2015} have developed structure-preserving particle pushers for neoclassical transport in the Vlasov equations, derived from Crank--Nicolson integrators. We show these too can can derive from a FET interpretation, similarly offering potential extensions to higher-order-in-time particle pushers. The FET formulation is used also to consider how the stochastic drift terms can be incorporated into the pushers. Stochastic gyrokinetic expansions are also discussed.

        Different options for the numerical implementation of these schemes are considered.

        Due to the efficacy of FET in the development of SP timesteppers for both the fluid and kinetic component, we hope this approach will prove effective in the future for developing SP timesteppers for the full hybrid model. We hope this will give us the opportunity to incorporate previously inaccessible kinetic effects into the highly effective, modern, finite-element MHD models.
    \end{abstract}
    
    
    \newpage
    \tableofcontents
    
    
    \newpage
    \pagenumbering{arabic}
    %\linenumbers\renewcommand\thelinenumber{\color{black!50}\arabic{linenumber}}
            \input{0 - introduction/main.tex}
        \part{Research}
            \input{1 - low-noise PiC models/main.tex}
            \input{2 - kinetic component/main.tex}
            \input{3 - fluid component/main.tex}
            \input{4 - numerical implementation/main.tex}
        \part{Project Overview}
            \input{5 - research plan/main.tex}
            \input{6 - summary/main.tex}
    
    
    %\section{}
    \newpage
    \pagenumbering{gobble}
        \printbibliography


    \newpage
    \pagenumbering{roman}
    \appendix
        \part{Appendices}
            \input{8 - Hilbert complexes/main.tex}
            \input{9 - weak conservation proofs/main.tex}
\end{document}

\end{document}

    
    
    %\section{}
    \newpage
    \pagenumbering{gobble}
        \printbibliography


    \newpage
    \pagenumbering{roman}
    \appendix
        \part{Appendices}
            \documentclass[12pt, a4paper]{report}

\documentclass[12pt, a4paper]{report}

\input{template/main.tex}

\title{\BA{Title in Progress...}}
\author{Boris Andrews}
\affil{Mathematical Institute, University of Oxford}
\date{\today}


\begin{document}
    \pagenumbering{gobble}
    \maketitle
    
    
    \begin{abstract}
        Magnetic confinement reactors---in particular tokamaks---offer one of the most promising options for achieving practical nuclear fusion, with the potential to provide virtually limitless, clean energy. The theoretical and numerical modeling of tokamak plasmas is simultaneously an essential component of effective reactor design, and a great research barrier. Tokamak operational conditions exhibit comparatively low Knudsen numbers. Kinetic effects, including kinetic waves and instabilities, Landau damping, bump-on-tail instabilities and more, are therefore highly influential in tokamak plasma dynamics. Purely fluid models are inherently incapable of capturing these effects, whereas the high dimensionality in purely kinetic models render them practically intractable for most relevant purposes.

        We consider a $\delta\!f$ decomposition model, with a macroscopic fluid background and microscopic kinetic correction, both fully coupled to each other. A similar manner of discretization is proposed to that used in the recent \texttt{STRUPHY} code \cite{Holderied_Possanner_Wang_2021, Holderied_2022, Li_et_al_2023} with a finite-element model for the background and a pseudo-particle/PiC model for the correction.

        The fluid background satisfies the full, non-linear, resistive, compressible, Hall MHD equations. \cite{Laakmann_Hu_Farrell_2022} introduces finite-element(-in-space) implicit timesteppers for the incompressible analogue to this system with structure-preserving (SP) properties in the ideal case, alongside parameter-robust preconditioners. We show that these timesteppers can derive from a finite-element-in-time (FET) (and finite-element-in-space) interpretation. The benefits of this reformulation are discussed, including the derivation of timesteppers that are higher order in time, and the quantifiable dissipative SP properties in the non-ideal, resistive case.
        
        We discuss possible options for extending this FET approach to timesteppers for the compressible case.

        The kinetic corrections satisfy linearized Boltzmann equations. Using a Lénard--Bernstein collision operator, these take Fokker--Planck-like forms \cite{Fokker_1914, Planck_1917} wherein pseudo-particles in the numerical model obey the neoclassical transport equations, with particle-independent Brownian drift terms. This offers a rigorous methodology for incorporating collisions into the particle transport model, without coupling the equations of motions for each particle.
        
        Works by Chen, Chacón et al. \cite{Chen_Chacón_Barnes_2011, Chacón_Chen_Barnes_2013, Chen_Chacón_2014, Chen_Chacón_2015} have developed structure-preserving particle pushers for neoclassical transport in the Vlasov equations, derived from Crank--Nicolson integrators. We show these too can can derive from a FET interpretation, similarly offering potential extensions to higher-order-in-time particle pushers. The FET formulation is used also to consider how the stochastic drift terms can be incorporated into the pushers. Stochastic gyrokinetic expansions are also discussed.

        Different options for the numerical implementation of these schemes are considered.

        Due to the efficacy of FET in the development of SP timesteppers for both the fluid and kinetic component, we hope this approach will prove effective in the future for developing SP timesteppers for the full hybrid model. We hope this will give us the opportunity to incorporate previously inaccessible kinetic effects into the highly effective, modern, finite-element MHD models.
    \end{abstract}
    
    
    \newpage
    \tableofcontents
    
    
    \newpage
    \pagenumbering{arabic}
    %\linenumbers\renewcommand\thelinenumber{\color{black!50}\arabic{linenumber}}
            \input{0 - introduction/main.tex}
        \part{Research}
            \input{1 - low-noise PiC models/main.tex}
            \input{2 - kinetic component/main.tex}
            \input{3 - fluid component/main.tex}
            \input{4 - numerical implementation/main.tex}
        \part{Project Overview}
            \input{5 - research plan/main.tex}
            \input{6 - summary/main.tex}
    
    
    %\section{}
    \newpage
    \pagenumbering{gobble}
        \printbibliography


    \newpage
    \pagenumbering{roman}
    \appendix
        \part{Appendices}
            \input{8 - Hilbert complexes/main.tex}
            \input{9 - weak conservation proofs/main.tex}
\end{document}


\title{\BA{Title in Progress...}}
\author{Boris Andrews}
\affil{Mathematical Institute, University of Oxford}
\date{\today}


\begin{document}
    \pagenumbering{gobble}
    \maketitle
    
    
    \begin{abstract}
        Magnetic confinement reactors---in particular tokamaks---offer one of the most promising options for achieving practical nuclear fusion, with the potential to provide virtually limitless, clean energy. The theoretical and numerical modeling of tokamak plasmas is simultaneously an essential component of effective reactor design, and a great research barrier. Tokamak operational conditions exhibit comparatively low Knudsen numbers. Kinetic effects, including kinetic waves and instabilities, Landau damping, bump-on-tail instabilities and more, are therefore highly influential in tokamak plasma dynamics. Purely fluid models are inherently incapable of capturing these effects, whereas the high dimensionality in purely kinetic models render them practically intractable for most relevant purposes.

        We consider a $\delta\!f$ decomposition model, with a macroscopic fluid background and microscopic kinetic correction, both fully coupled to each other. A similar manner of discretization is proposed to that used in the recent \texttt{STRUPHY} code \cite{Holderied_Possanner_Wang_2021, Holderied_2022, Li_et_al_2023} with a finite-element model for the background and a pseudo-particle/PiC model for the correction.

        The fluid background satisfies the full, non-linear, resistive, compressible, Hall MHD equations. \cite{Laakmann_Hu_Farrell_2022} introduces finite-element(-in-space) implicit timesteppers for the incompressible analogue to this system with structure-preserving (SP) properties in the ideal case, alongside parameter-robust preconditioners. We show that these timesteppers can derive from a finite-element-in-time (FET) (and finite-element-in-space) interpretation. The benefits of this reformulation are discussed, including the derivation of timesteppers that are higher order in time, and the quantifiable dissipative SP properties in the non-ideal, resistive case.
        
        We discuss possible options for extending this FET approach to timesteppers for the compressible case.

        The kinetic corrections satisfy linearized Boltzmann equations. Using a Lénard--Bernstein collision operator, these take Fokker--Planck-like forms \cite{Fokker_1914, Planck_1917} wherein pseudo-particles in the numerical model obey the neoclassical transport equations, with particle-independent Brownian drift terms. This offers a rigorous methodology for incorporating collisions into the particle transport model, without coupling the equations of motions for each particle.
        
        Works by Chen, Chacón et al. \cite{Chen_Chacón_Barnes_2011, Chacón_Chen_Barnes_2013, Chen_Chacón_2014, Chen_Chacón_2015} have developed structure-preserving particle pushers for neoclassical transport in the Vlasov equations, derived from Crank--Nicolson integrators. We show these too can can derive from a FET interpretation, similarly offering potential extensions to higher-order-in-time particle pushers. The FET formulation is used also to consider how the stochastic drift terms can be incorporated into the pushers. Stochastic gyrokinetic expansions are also discussed.

        Different options for the numerical implementation of these schemes are considered.

        Due to the efficacy of FET in the development of SP timesteppers for both the fluid and kinetic component, we hope this approach will prove effective in the future for developing SP timesteppers for the full hybrid model. We hope this will give us the opportunity to incorporate previously inaccessible kinetic effects into the highly effective, modern, finite-element MHD models.
    \end{abstract}
    
    
    \newpage
    \tableofcontents
    
    
    \newpage
    \pagenumbering{arabic}
    %\linenumbers\renewcommand\thelinenumber{\color{black!50}\arabic{linenumber}}
            \documentclass[12pt, a4paper]{report}

\input{template/main.tex}

\title{\BA{Title in Progress...}}
\author{Boris Andrews}
\affil{Mathematical Institute, University of Oxford}
\date{\today}


\begin{document}
    \pagenumbering{gobble}
    \maketitle
    
    
    \begin{abstract}
        Magnetic confinement reactors---in particular tokamaks---offer one of the most promising options for achieving practical nuclear fusion, with the potential to provide virtually limitless, clean energy. The theoretical and numerical modeling of tokamak plasmas is simultaneously an essential component of effective reactor design, and a great research barrier. Tokamak operational conditions exhibit comparatively low Knudsen numbers. Kinetic effects, including kinetic waves and instabilities, Landau damping, bump-on-tail instabilities and more, are therefore highly influential in tokamak plasma dynamics. Purely fluid models are inherently incapable of capturing these effects, whereas the high dimensionality in purely kinetic models render them practically intractable for most relevant purposes.

        We consider a $\delta\!f$ decomposition model, with a macroscopic fluid background and microscopic kinetic correction, both fully coupled to each other. A similar manner of discretization is proposed to that used in the recent \texttt{STRUPHY} code \cite{Holderied_Possanner_Wang_2021, Holderied_2022, Li_et_al_2023} with a finite-element model for the background and a pseudo-particle/PiC model for the correction.

        The fluid background satisfies the full, non-linear, resistive, compressible, Hall MHD equations. \cite{Laakmann_Hu_Farrell_2022} introduces finite-element(-in-space) implicit timesteppers for the incompressible analogue to this system with structure-preserving (SP) properties in the ideal case, alongside parameter-robust preconditioners. We show that these timesteppers can derive from a finite-element-in-time (FET) (and finite-element-in-space) interpretation. The benefits of this reformulation are discussed, including the derivation of timesteppers that are higher order in time, and the quantifiable dissipative SP properties in the non-ideal, resistive case.
        
        We discuss possible options for extending this FET approach to timesteppers for the compressible case.

        The kinetic corrections satisfy linearized Boltzmann equations. Using a Lénard--Bernstein collision operator, these take Fokker--Planck-like forms \cite{Fokker_1914, Planck_1917} wherein pseudo-particles in the numerical model obey the neoclassical transport equations, with particle-independent Brownian drift terms. This offers a rigorous methodology for incorporating collisions into the particle transport model, without coupling the equations of motions for each particle.
        
        Works by Chen, Chacón et al. \cite{Chen_Chacón_Barnes_2011, Chacón_Chen_Barnes_2013, Chen_Chacón_2014, Chen_Chacón_2015} have developed structure-preserving particle pushers for neoclassical transport in the Vlasov equations, derived from Crank--Nicolson integrators. We show these too can can derive from a FET interpretation, similarly offering potential extensions to higher-order-in-time particle pushers. The FET formulation is used also to consider how the stochastic drift terms can be incorporated into the pushers. Stochastic gyrokinetic expansions are also discussed.

        Different options for the numerical implementation of these schemes are considered.

        Due to the efficacy of FET in the development of SP timesteppers for both the fluid and kinetic component, we hope this approach will prove effective in the future for developing SP timesteppers for the full hybrid model. We hope this will give us the opportunity to incorporate previously inaccessible kinetic effects into the highly effective, modern, finite-element MHD models.
    \end{abstract}
    
    
    \newpage
    \tableofcontents
    
    
    \newpage
    \pagenumbering{arabic}
    %\linenumbers\renewcommand\thelinenumber{\color{black!50}\arabic{linenumber}}
            \input{0 - introduction/main.tex}
        \part{Research}
            \input{1 - low-noise PiC models/main.tex}
            \input{2 - kinetic component/main.tex}
            \input{3 - fluid component/main.tex}
            \input{4 - numerical implementation/main.tex}
        \part{Project Overview}
            \input{5 - research plan/main.tex}
            \input{6 - summary/main.tex}
    
    
    %\section{}
    \newpage
    \pagenumbering{gobble}
        \printbibliography


    \newpage
    \pagenumbering{roman}
    \appendix
        \part{Appendices}
            \input{8 - Hilbert complexes/main.tex}
            \input{9 - weak conservation proofs/main.tex}
\end{document}

        \part{Research}
            \documentclass[12pt, a4paper]{report}

\input{template/main.tex}

\title{\BA{Title in Progress...}}
\author{Boris Andrews}
\affil{Mathematical Institute, University of Oxford}
\date{\today}


\begin{document}
    \pagenumbering{gobble}
    \maketitle
    
    
    \begin{abstract}
        Magnetic confinement reactors---in particular tokamaks---offer one of the most promising options for achieving practical nuclear fusion, with the potential to provide virtually limitless, clean energy. The theoretical and numerical modeling of tokamak plasmas is simultaneously an essential component of effective reactor design, and a great research barrier. Tokamak operational conditions exhibit comparatively low Knudsen numbers. Kinetic effects, including kinetic waves and instabilities, Landau damping, bump-on-tail instabilities and more, are therefore highly influential in tokamak plasma dynamics. Purely fluid models are inherently incapable of capturing these effects, whereas the high dimensionality in purely kinetic models render them practically intractable for most relevant purposes.

        We consider a $\delta\!f$ decomposition model, with a macroscopic fluid background and microscopic kinetic correction, both fully coupled to each other. A similar manner of discretization is proposed to that used in the recent \texttt{STRUPHY} code \cite{Holderied_Possanner_Wang_2021, Holderied_2022, Li_et_al_2023} with a finite-element model for the background and a pseudo-particle/PiC model for the correction.

        The fluid background satisfies the full, non-linear, resistive, compressible, Hall MHD equations. \cite{Laakmann_Hu_Farrell_2022} introduces finite-element(-in-space) implicit timesteppers for the incompressible analogue to this system with structure-preserving (SP) properties in the ideal case, alongside parameter-robust preconditioners. We show that these timesteppers can derive from a finite-element-in-time (FET) (and finite-element-in-space) interpretation. The benefits of this reformulation are discussed, including the derivation of timesteppers that are higher order in time, and the quantifiable dissipative SP properties in the non-ideal, resistive case.
        
        We discuss possible options for extending this FET approach to timesteppers for the compressible case.

        The kinetic corrections satisfy linearized Boltzmann equations. Using a Lénard--Bernstein collision operator, these take Fokker--Planck-like forms \cite{Fokker_1914, Planck_1917} wherein pseudo-particles in the numerical model obey the neoclassical transport equations, with particle-independent Brownian drift terms. This offers a rigorous methodology for incorporating collisions into the particle transport model, without coupling the equations of motions for each particle.
        
        Works by Chen, Chacón et al. \cite{Chen_Chacón_Barnes_2011, Chacón_Chen_Barnes_2013, Chen_Chacón_2014, Chen_Chacón_2015} have developed structure-preserving particle pushers for neoclassical transport in the Vlasov equations, derived from Crank--Nicolson integrators. We show these too can can derive from a FET interpretation, similarly offering potential extensions to higher-order-in-time particle pushers. The FET formulation is used also to consider how the stochastic drift terms can be incorporated into the pushers. Stochastic gyrokinetic expansions are also discussed.

        Different options for the numerical implementation of these schemes are considered.

        Due to the efficacy of FET in the development of SP timesteppers for both the fluid and kinetic component, we hope this approach will prove effective in the future for developing SP timesteppers for the full hybrid model. We hope this will give us the opportunity to incorporate previously inaccessible kinetic effects into the highly effective, modern, finite-element MHD models.
    \end{abstract}
    
    
    \newpage
    \tableofcontents
    
    
    \newpage
    \pagenumbering{arabic}
    %\linenumbers\renewcommand\thelinenumber{\color{black!50}\arabic{linenumber}}
            \input{0 - introduction/main.tex}
        \part{Research}
            \input{1 - low-noise PiC models/main.tex}
            \input{2 - kinetic component/main.tex}
            \input{3 - fluid component/main.tex}
            \input{4 - numerical implementation/main.tex}
        \part{Project Overview}
            \input{5 - research plan/main.tex}
            \input{6 - summary/main.tex}
    
    
    %\section{}
    \newpage
    \pagenumbering{gobble}
        \printbibliography


    \newpage
    \pagenumbering{roman}
    \appendix
        \part{Appendices}
            \input{8 - Hilbert complexes/main.tex}
            \input{9 - weak conservation proofs/main.tex}
\end{document}

            \documentclass[12pt, a4paper]{report}

\input{template/main.tex}

\title{\BA{Title in Progress...}}
\author{Boris Andrews}
\affil{Mathematical Institute, University of Oxford}
\date{\today}


\begin{document}
    \pagenumbering{gobble}
    \maketitle
    
    
    \begin{abstract}
        Magnetic confinement reactors---in particular tokamaks---offer one of the most promising options for achieving practical nuclear fusion, with the potential to provide virtually limitless, clean energy. The theoretical and numerical modeling of tokamak plasmas is simultaneously an essential component of effective reactor design, and a great research barrier. Tokamak operational conditions exhibit comparatively low Knudsen numbers. Kinetic effects, including kinetic waves and instabilities, Landau damping, bump-on-tail instabilities and more, are therefore highly influential in tokamak plasma dynamics. Purely fluid models are inherently incapable of capturing these effects, whereas the high dimensionality in purely kinetic models render them practically intractable for most relevant purposes.

        We consider a $\delta\!f$ decomposition model, with a macroscopic fluid background and microscopic kinetic correction, both fully coupled to each other. A similar manner of discretization is proposed to that used in the recent \texttt{STRUPHY} code \cite{Holderied_Possanner_Wang_2021, Holderied_2022, Li_et_al_2023} with a finite-element model for the background and a pseudo-particle/PiC model for the correction.

        The fluid background satisfies the full, non-linear, resistive, compressible, Hall MHD equations. \cite{Laakmann_Hu_Farrell_2022} introduces finite-element(-in-space) implicit timesteppers for the incompressible analogue to this system with structure-preserving (SP) properties in the ideal case, alongside parameter-robust preconditioners. We show that these timesteppers can derive from a finite-element-in-time (FET) (and finite-element-in-space) interpretation. The benefits of this reformulation are discussed, including the derivation of timesteppers that are higher order in time, and the quantifiable dissipative SP properties in the non-ideal, resistive case.
        
        We discuss possible options for extending this FET approach to timesteppers for the compressible case.

        The kinetic corrections satisfy linearized Boltzmann equations. Using a Lénard--Bernstein collision operator, these take Fokker--Planck-like forms \cite{Fokker_1914, Planck_1917} wherein pseudo-particles in the numerical model obey the neoclassical transport equations, with particle-independent Brownian drift terms. This offers a rigorous methodology for incorporating collisions into the particle transport model, without coupling the equations of motions for each particle.
        
        Works by Chen, Chacón et al. \cite{Chen_Chacón_Barnes_2011, Chacón_Chen_Barnes_2013, Chen_Chacón_2014, Chen_Chacón_2015} have developed structure-preserving particle pushers for neoclassical transport in the Vlasov equations, derived from Crank--Nicolson integrators. We show these too can can derive from a FET interpretation, similarly offering potential extensions to higher-order-in-time particle pushers. The FET formulation is used also to consider how the stochastic drift terms can be incorporated into the pushers. Stochastic gyrokinetic expansions are also discussed.

        Different options for the numerical implementation of these schemes are considered.

        Due to the efficacy of FET in the development of SP timesteppers for both the fluid and kinetic component, we hope this approach will prove effective in the future for developing SP timesteppers for the full hybrid model. We hope this will give us the opportunity to incorporate previously inaccessible kinetic effects into the highly effective, modern, finite-element MHD models.
    \end{abstract}
    
    
    \newpage
    \tableofcontents
    
    
    \newpage
    \pagenumbering{arabic}
    %\linenumbers\renewcommand\thelinenumber{\color{black!50}\arabic{linenumber}}
            \input{0 - introduction/main.tex}
        \part{Research}
            \input{1 - low-noise PiC models/main.tex}
            \input{2 - kinetic component/main.tex}
            \input{3 - fluid component/main.tex}
            \input{4 - numerical implementation/main.tex}
        \part{Project Overview}
            \input{5 - research plan/main.tex}
            \input{6 - summary/main.tex}
    
    
    %\section{}
    \newpage
    \pagenumbering{gobble}
        \printbibliography


    \newpage
    \pagenumbering{roman}
    \appendix
        \part{Appendices}
            \input{8 - Hilbert complexes/main.tex}
            \input{9 - weak conservation proofs/main.tex}
\end{document}

            \documentclass[12pt, a4paper]{report}

\input{template/main.tex}

\title{\BA{Title in Progress...}}
\author{Boris Andrews}
\affil{Mathematical Institute, University of Oxford}
\date{\today}


\begin{document}
    \pagenumbering{gobble}
    \maketitle
    
    
    \begin{abstract}
        Magnetic confinement reactors---in particular tokamaks---offer one of the most promising options for achieving practical nuclear fusion, with the potential to provide virtually limitless, clean energy. The theoretical and numerical modeling of tokamak plasmas is simultaneously an essential component of effective reactor design, and a great research barrier. Tokamak operational conditions exhibit comparatively low Knudsen numbers. Kinetic effects, including kinetic waves and instabilities, Landau damping, bump-on-tail instabilities and more, are therefore highly influential in tokamak plasma dynamics. Purely fluid models are inherently incapable of capturing these effects, whereas the high dimensionality in purely kinetic models render them practically intractable for most relevant purposes.

        We consider a $\delta\!f$ decomposition model, with a macroscopic fluid background and microscopic kinetic correction, both fully coupled to each other. A similar manner of discretization is proposed to that used in the recent \texttt{STRUPHY} code \cite{Holderied_Possanner_Wang_2021, Holderied_2022, Li_et_al_2023} with a finite-element model for the background and a pseudo-particle/PiC model for the correction.

        The fluid background satisfies the full, non-linear, resistive, compressible, Hall MHD equations. \cite{Laakmann_Hu_Farrell_2022} introduces finite-element(-in-space) implicit timesteppers for the incompressible analogue to this system with structure-preserving (SP) properties in the ideal case, alongside parameter-robust preconditioners. We show that these timesteppers can derive from a finite-element-in-time (FET) (and finite-element-in-space) interpretation. The benefits of this reformulation are discussed, including the derivation of timesteppers that are higher order in time, and the quantifiable dissipative SP properties in the non-ideal, resistive case.
        
        We discuss possible options for extending this FET approach to timesteppers for the compressible case.

        The kinetic corrections satisfy linearized Boltzmann equations. Using a Lénard--Bernstein collision operator, these take Fokker--Planck-like forms \cite{Fokker_1914, Planck_1917} wherein pseudo-particles in the numerical model obey the neoclassical transport equations, with particle-independent Brownian drift terms. This offers a rigorous methodology for incorporating collisions into the particle transport model, without coupling the equations of motions for each particle.
        
        Works by Chen, Chacón et al. \cite{Chen_Chacón_Barnes_2011, Chacón_Chen_Barnes_2013, Chen_Chacón_2014, Chen_Chacón_2015} have developed structure-preserving particle pushers for neoclassical transport in the Vlasov equations, derived from Crank--Nicolson integrators. We show these too can can derive from a FET interpretation, similarly offering potential extensions to higher-order-in-time particle pushers. The FET formulation is used also to consider how the stochastic drift terms can be incorporated into the pushers. Stochastic gyrokinetic expansions are also discussed.

        Different options for the numerical implementation of these schemes are considered.

        Due to the efficacy of FET in the development of SP timesteppers for both the fluid and kinetic component, we hope this approach will prove effective in the future for developing SP timesteppers for the full hybrid model. We hope this will give us the opportunity to incorporate previously inaccessible kinetic effects into the highly effective, modern, finite-element MHD models.
    \end{abstract}
    
    
    \newpage
    \tableofcontents
    
    
    \newpage
    \pagenumbering{arabic}
    %\linenumbers\renewcommand\thelinenumber{\color{black!50}\arabic{linenumber}}
            \input{0 - introduction/main.tex}
        \part{Research}
            \input{1 - low-noise PiC models/main.tex}
            \input{2 - kinetic component/main.tex}
            \input{3 - fluid component/main.tex}
            \input{4 - numerical implementation/main.tex}
        \part{Project Overview}
            \input{5 - research plan/main.tex}
            \input{6 - summary/main.tex}
    
    
    %\section{}
    \newpage
    \pagenumbering{gobble}
        \printbibliography


    \newpage
    \pagenumbering{roman}
    \appendix
        \part{Appendices}
            \input{8 - Hilbert complexes/main.tex}
            \input{9 - weak conservation proofs/main.tex}
\end{document}

            \documentclass[12pt, a4paper]{report}

\input{template/main.tex}

\title{\BA{Title in Progress...}}
\author{Boris Andrews}
\affil{Mathematical Institute, University of Oxford}
\date{\today}


\begin{document}
    \pagenumbering{gobble}
    \maketitle
    
    
    \begin{abstract}
        Magnetic confinement reactors---in particular tokamaks---offer one of the most promising options for achieving practical nuclear fusion, with the potential to provide virtually limitless, clean energy. The theoretical and numerical modeling of tokamak plasmas is simultaneously an essential component of effective reactor design, and a great research barrier. Tokamak operational conditions exhibit comparatively low Knudsen numbers. Kinetic effects, including kinetic waves and instabilities, Landau damping, bump-on-tail instabilities and more, are therefore highly influential in tokamak plasma dynamics. Purely fluid models are inherently incapable of capturing these effects, whereas the high dimensionality in purely kinetic models render them practically intractable for most relevant purposes.

        We consider a $\delta\!f$ decomposition model, with a macroscopic fluid background and microscopic kinetic correction, both fully coupled to each other. A similar manner of discretization is proposed to that used in the recent \texttt{STRUPHY} code \cite{Holderied_Possanner_Wang_2021, Holderied_2022, Li_et_al_2023} with a finite-element model for the background and a pseudo-particle/PiC model for the correction.

        The fluid background satisfies the full, non-linear, resistive, compressible, Hall MHD equations. \cite{Laakmann_Hu_Farrell_2022} introduces finite-element(-in-space) implicit timesteppers for the incompressible analogue to this system with structure-preserving (SP) properties in the ideal case, alongside parameter-robust preconditioners. We show that these timesteppers can derive from a finite-element-in-time (FET) (and finite-element-in-space) interpretation. The benefits of this reformulation are discussed, including the derivation of timesteppers that are higher order in time, and the quantifiable dissipative SP properties in the non-ideal, resistive case.
        
        We discuss possible options for extending this FET approach to timesteppers for the compressible case.

        The kinetic corrections satisfy linearized Boltzmann equations. Using a Lénard--Bernstein collision operator, these take Fokker--Planck-like forms \cite{Fokker_1914, Planck_1917} wherein pseudo-particles in the numerical model obey the neoclassical transport equations, with particle-independent Brownian drift terms. This offers a rigorous methodology for incorporating collisions into the particle transport model, without coupling the equations of motions for each particle.
        
        Works by Chen, Chacón et al. \cite{Chen_Chacón_Barnes_2011, Chacón_Chen_Barnes_2013, Chen_Chacón_2014, Chen_Chacón_2015} have developed structure-preserving particle pushers for neoclassical transport in the Vlasov equations, derived from Crank--Nicolson integrators. We show these too can can derive from a FET interpretation, similarly offering potential extensions to higher-order-in-time particle pushers. The FET formulation is used also to consider how the stochastic drift terms can be incorporated into the pushers. Stochastic gyrokinetic expansions are also discussed.

        Different options for the numerical implementation of these schemes are considered.

        Due to the efficacy of FET in the development of SP timesteppers for both the fluid and kinetic component, we hope this approach will prove effective in the future for developing SP timesteppers for the full hybrid model. We hope this will give us the opportunity to incorporate previously inaccessible kinetic effects into the highly effective, modern, finite-element MHD models.
    \end{abstract}
    
    
    \newpage
    \tableofcontents
    
    
    \newpage
    \pagenumbering{arabic}
    %\linenumbers\renewcommand\thelinenumber{\color{black!50}\arabic{linenumber}}
            \input{0 - introduction/main.tex}
        \part{Research}
            \input{1 - low-noise PiC models/main.tex}
            \input{2 - kinetic component/main.tex}
            \input{3 - fluid component/main.tex}
            \input{4 - numerical implementation/main.tex}
        \part{Project Overview}
            \input{5 - research plan/main.tex}
            \input{6 - summary/main.tex}
    
    
    %\section{}
    \newpage
    \pagenumbering{gobble}
        \printbibliography


    \newpage
    \pagenumbering{roman}
    \appendix
        \part{Appendices}
            \input{8 - Hilbert complexes/main.tex}
            \input{9 - weak conservation proofs/main.tex}
\end{document}

        \part{Project Overview}
            \documentclass[12pt, a4paper]{report}

\input{template/main.tex}

\title{\BA{Title in Progress...}}
\author{Boris Andrews}
\affil{Mathematical Institute, University of Oxford}
\date{\today}


\begin{document}
    \pagenumbering{gobble}
    \maketitle
    
    
    \begin{abstract}
        Magnetic confinement reactors---in particular tokamaks---offer one of the most promising options for achieving practical nuclear fusion, with the potential to provide virtually limitless, clean energy. The theoretical and numerical modeling of tokamak plasmas is simultaneously an essential component of effective reactor design, and a great research barrier. Tokamak operational conditions exhibit comparatively low Knudsen numbers. Kinetic effects, including kinetic waves and instabilities, Landau damping, bump-on-tail instabilities and more, are therefore highly influential in tokamak plasma dynamics. Purely fluid models are inherently incapable of capturing these effects, whereas the high dimensionality in purely kinetic models render them practically intractable for most relevant purposes.

        We consider a $\delta\!f$ decomposition model, with a macroscopic fluid background and microscopic kinetic correction, both fully coupled to each other. A similar manner of discretization is proposed to that used in the recent \texttt{STRUPHY} code \cite{Holderied_Possanner_Wang_2021, Holderied_2022, Li_et_al_2023} with a finite-element model for the background and a pseudo-particle/PiC model for the correction.

        The fluid background satisfies the full, non-linear, resistive, compressible, Hall MHD equations. \cite{Laakmann_Hu_Farrell_2022} introduces finite-element(-in-space) implicit timesteppers for the incompressible analogue to this system with structure-preserving (SP) properties in the ideal case, alongside parameter-robust preconditioners. We show that these timesteppers can derive from a finite-element-in-time (FET) (and finite-element-in-space) interpretation. The benefits of this reformulation are discussed, including the derivation of timesteppers that are higher order in time, and the quantifiable dissipative SP properties in the non-ideal, resistive case.
        
        We discuss possible options for extending this FET approach to timesteppers for the compressible case.

        The kinetic corrections satisfy linearized Boltzmann equations. Using a Lénard--Bernstein collision operator, these take Fokker--Planck-like forms \cite{Fokker_1914, Planck_1917} wherein pseudo-particles in the numerical model obey the neoclassical transport equations, with particle-independent Brownian drift terms. This offers a rigorous methodology for incorporating collisions into the particle transport model, without coupling the equations of motions for each particle.
        
        Works by Chen, Chacón et al. \cite{Chen_Chacón_Barnes_2011, Chacón_Chen_Barnes_2013, Chen_Chacón_2014, Chen_Chacón_2015} have developed structure-preserving particle pushers for neoclassical transport in the Vlasov equations, derived from Crank--Nicolson integrators. We show these too can can derive from a FET interpretation, similarly offering potential extensions to higher-order-in-time particle pushers. The FET formulation is used also to consider how the stochastic drift terms can be incorporated into the pushers. Stochastic gyrokinetic expansions are also discussed.

        Different options for the numerical implementation of these schemes are considered.

        Due to the efficacy of FET in the development of SP timesteppers for both the fluid and kinetic component, we hope this approach will prove effective in the future for developing SP timesteppers for the full hybrid model. We hope this will give us the opportunity to incorporate previously inaccessible kinetic effects into the highly effective, modern, finite-element MHD models.
    \end{abstract}
    
    
    \newpage
    \tableofcontents
    
    
    \newpage
    \pagenumbering{arabic}
    %\linenumbers\renewcommand\thelinenumber{\color{black!50}\arabic{linenumber}}
            \input{0 - introduction/main.tex}
        \part{Research}
            \input{1 - low-noise PiC models/main.tex}
            \input{2 - kinetic component/main.tex}
            \input{3 - fluid component/main.tex}
            \input{4 - numerical implementation/main.tex}
        \part{Project Overview}
            \input{5 - research plan/main.tex}
            \input{6 - summary/main.tex}
    
    
    %\section{}
    \newpage
    \pagenumbering{gobble}
        \printbibliography


    \newpage
    \pagenumbering{roman}
    \appendix
        \part{Appendices}
            \input{8 - Hilbert complexes/main.tex}
            \input{9 - weak conservation proofs/main.tex}
\end{document}

            \documentclass[12pt, a4paper]{report}

\input{template/main.tex}

\title{\BA{Title in Progress...}}
\author{Boris Andrews}
\affil{Mathematical Institute, University of Oxford}
\date{\today}


\begin{document}
    \pagenumbering{gobble}
    \maketitle
    
    
    \begin{abstract}
        Magnetic confinement reactors---in particular tokamaks---offer one of the most promising options for achieving practical nuclear fusion, with the potential to provide virtually limitless, clean energy. The theoretical and numerical modeling of tokamak plasmas is simultaneously an essential component of effective reactor design, and a great research barrier. Tokamak operational conditions exhibit comparatively low Knudsen numbers. Kinetic effects, including kinetic waves and instabilities, Landau damping, bump-on-tail instabilities and more, are therefore highly influential in tokamak plasma dynamics. Purely fluid models are inherently incapable of capturing these effects, whereas the high dimensionality in purely kinetic models render them practically intractable for most relevant purposes.

        We consider a $\delta\!f$ decomposition model, with a macroscopic fluid background and microscopic kinetic correction, both fully coupled to each other. A similar manner of discretization is proposed to that used in the recent \texttt{STRUPHY} code \cite{Holderied_Possanner_Wang_2021, Holderied_2022, Li_et_al_2023} with a finite-element model for the background and a pseudo-particle/PiC model for the correction.

        The fluid background satisfies the full, non-linear, resistive, compressible, Hall MHD equations. \cite{Laakmann_Hu_Farrell_2022} introduces finite-element(-in-space) implicit timesteppers for the incompressible analogue to this system with structure-preserving (SP) properties in the ideal case, alongside parameter-robust preconditioners. We show that these timesteppers can derive from a finite-element-in-time (FET) (and finite-element-in-space) interpretation. The benefits of this reformulation are discussed, including the derivation of timesteppers that are higher order in time, and the quantifiable dissipative SP properties in the non-ideal, resistive case.
        
        We discuss possible options for extending this FET approach to timesteppers for the compressible case.

        The kinetic corrections satisfy linearized Boltzmann equations. Using a Lénard--Bernstein collision operator, these take Fokker--Planck-like forms \cite{Fokker_1914, Planck_1917} wherein pseudo-particles in the numerical model obey the neoclassical transport equations, with particle-independent Brownian drift terms. This offers a rigorous methodology for incorporating collisions into the particle transport model, without coupling the equations of motions for each particle.
        
        Works by Chen, Chacón et al. \cite{Chen_Chacón_Barnes_2011, Chacón_Chen_Barnes_2013, Chen_Chacón_2014, Chen_Chacón_2015} have developed structure-preserving particle pushers for neoclassical transport in the Vlasov equations, derived from Crank--Nicolson integrators. We show these too can can derive from a FET interpretation, similarly offering potential extensions to higher-order-in-time particle pushers. The FET formulation is used also to consider how the stochastic drift terms can be incorporated into the pushers. Stochastic gyrokinetic expansions are also discussed.

        Different options for the numerical implementation of these schemes are considered.

        Due to the efficacy of FET in the development of SP timesteppers for both the fluid and kinetic component, we hope this approach will prove effective in the future for developing SP timesteppers for the full hybrid model. We hope this will give us the opportunity to incorporate previously inaccessible kinetic effects into the highly effective, modern, finite-element MHD models.
    \end{abstract}
    
    
    \newpage
    \tableofcontents
    
    
    \newpage
    \pagenumbering{arabic}
    %\linenumbers\renewcommand\thelinenumber{\color{black!50}\arabic{linenumber}}
            \input{0 - introduction/main.tex}
        \part{Research}
            \input{1 - low-noise PiC models/main.tex}
            \input{2 - kinetic component/main.tex}
            \input{3 - fluid component/main.tex}
            \input{4 - numerical implementation/main.tex}
        \part{Project Overview}
            \input{5 - research plan/main.tex}
            \input{6 - summary/main.tex}
    
    
    %\section{}
    \newpage
    \pagenumbering{gobble}
        \printbibliography


    \newpage
    \pagenumbering{roman}
    \appendix
        \part{Appendices}
            \input{8 - Hilbert complexes/main.tex}
            \input{9 - weak conservation proofs/main.tex}
\end{document}

    
    
    %\section{}
    \newpage
    \pagenumbering{gobble}
        \printbibliography


    \newpage
    \pagenumbering{roman}
    \appendix
        \part{Appendices}
            \documentclass[12pt, a4paper]{report}

\input{template/main.tex}

\title{\BA{Title in Progress...}}
\author{Boris Andrews}
\affil{Mathematical Institute, University of Oxford}
\date{\today}


\begin{document}
    \pagenumbering{gobble}
    \maketitle
    
    
    \begin{abstract}
        Magnetic confinement reactors---in particular tokamaks---offer one of the most promising options for achieving practical nuclear fusion, with the potential to provide virtually limitless, clean energy. The theoretical and numerical modeling of tokamak plasmas is simultaneously an essential component of effective reactor design, and a great research barrier. Tokamak operational conditions exhibit comparatively low Knudsen numbers. Kinetic effects, including kinetic waves and instabilities, Landau damping, bump-on-tail instabilities and more, are therefore highly influential in tokamak plasma dynamics. Purely fluid models are inherently incapable of capturing these effects, whereas the high dimensionality in purely kinetic models render them practically intractable for most relevant purposes.

        We consider a $\delta\!f$ decomposition model, with a macroscopic fluid background and microscopic kinetic correction, both fully coupled to each other. A similar manner of discretization is proposed to that used in the recent \texttt{STRUPHY} code \cite{Holderied_Possanner_Wang_2021, Holderied_2022, Li_et_al_2023} with a finite-element model for the background and a pseudo-particle/PiC model for the correction.

        The fluid background satisfies the full, non-linear, resistive, compressible, Hall MHD equations. \cite{Laakmann_Hu_Farrell_2022} introduces finite-element(-in-space) implicit timesteppers for the incompressible analogue to this system with structure-preserving (SP) properties in the ideal case, alongside parameter-robust preconditioners. We show that these timesteppers can derive from a finite-element-in-time (FET) (and finite-element-in-space) interpretation. The benefits of this reformulation are discussed, including the derivation of timesteppers that are higher order in time, and the quantifiable dissipative SP properties in the non-ideal, resistive case.
        
        We discuss possible options for extending this FET approach to timesteppers for the compressible case.

        The kinetic corrections satisfy linearized Boltzmann equations. Using a Lénard--Bernstein collision operator, these take Fokker--Planck-like forms \cite{Fokker_1914, Planck_1917} wherein pseudo-particles in the numerical model obey the neoclassical transport equations, with particle-independent Brownian drift terms. This offers a rigorous methodology for incorporating collisions into the particle transport model, without coupling the equations of motions for each particle.
        
        Works by Chen, Chacón et al. \cite{Chen_Chacón_Barnes_2011, Chacón_Chen_Barnes_2013, Chen_Chacón_2014, Chen_Chacón_2015} have developed structure-preserving particle pushers for neoclassical transport in the Vlasov equations, derived from Crank--Nicolson integrators. We show these too can can derive from a FET interpretation, similarly offering potential extensions to higher-order-in-time particle pushers. The FET formulation is used also to consider how the stochastic drift terms can be incorporated into the pushers. Stochastic gyrokinetic expansions are also discussed.

        Different options for the numerical implementation of these schemes are considered.

        Due to the efficacy of FET in the development of SP timesteppers for both the fluid and kinetic component, we hope this approach will prove effective in the future for developing SP timesteppers for the full hybrid model. We hope this will give us the opportunity to incorporate previously inaccessible kinetic effects into the highly effective, modern, finite-element MHD models.
    \end{abstract}
    
    
    \newpage
    \tableofcontents
    
    
    \newpage
    \pagenumbering{arabic}
    %\linenumbers\renewcommand\thelinenumber{\color{black!50}\arabic{linenumber}}
            \input{0 - introduction/main.tex}
        \part{Research}
            \input{1 - low-noise PiC models/main.tex}
            \input{2 - kinetic component/main.tex}
            \input{3 - fluid component/main.tex}
            \input{4 - numerical implementation/main.tex}
        \part{Project Overview}
            \input{5 - research plan/main.tex}
            \input{6 - summary/main.tex}
    
    
    %\section{}
    \newpage
    \pagenumbering{gobble}
        \printbibliography


    \newpage
    \pagenumbering{roman}
    \appendix
        \part{Appendices}
            \input{8 - Hilbert complexes/main.tex}
            \input{9 - weak conservation proofs/main.tex}
\end{document}

            \documentclass[12pt, a4paper]{report}

\input{template/main.tex}

\title{\BA{Title in Progress...}}
\author{Boris Andrews}
\affil{Mathematical Institute, University of Oxford}
\date{\today}


\begin{document}
    \pagenumbering{gobble}
    \maketitle
    
    
    \begin{abstract}
        Magnetic confinement reactors---in particular tokamaks---offer one of the most promising options for achieving practical nuclear fusion, with the potential to provide virtually limitless, clean energy. The theoretical and numerical modeling of tokamak plasmas is simultaneously an essential component of effective reactor design, and a great research barrier. Tokamak operational conditions exhibit comparatively low Knudsen numbers. Kinetic effects, including kinetic waves and instabilities, Landau damping, bump-on-tail instabilities and more, are therefore highly influential in tokamak plasma dynamics. Purely fluid models are inherently incapable of capturing these effects, whereas the high dimensionality in purely kinetic models render them practically intractable for most relevant purposes.

        We consider a $\delta\!f$ decomposition model, with a macroscopic fluid background and microscopic kinetic correction, both fully coupled to each other. A similar manner of discretization is proposed to that used in the recent \texttt{STRUPHY} code \cite{Holderied_Possanner_Wang_2021, Holderied_2022, Li_et_al_2023} with a finite-element model for the background and a pseudo-particle/PiC model for the correction.

        The fluid background satisfies the full, non-linear, resistive, compressible, Hall MHD equations. \cite{Laakmann_Hu_Farrell_2022} introduces finite-element(-in-space) implicit timesteppers for the incompressible analogue to this system with structure-preserving (SP) properties in the ideal case, alongside parameter-robust preconditioners. We show that these timesteppers can derive from a finite-element-in-time (FET) (and finite-element-in-space) interpretation. The benefits of this reformulation are discussed, including the derivation of timesteppers that are higher order in time, and the quantifiable dissipative SP properties in the non-ideal, resistive case.
        
        We discuss possible options for extending this FET approach to timesteppers for the compressible case.

        The kinetic corrections satisfy linearized Boltzmann equations. Using a Lénard--Bernstein collision operator, these take Fokker--Planck-like forms \cite{Fokker_1914, Planck_1917} wherein pseudo-particles in the numerical model obey the neoclassical transport equations, with particle-independent Brownian drift terms. This offers a rigorous methodology for incorporating collisions into the particle transport model, without coupling the equations of motions for each particle.
        
        Works by Chen, Chacón et al. \cite{Chen_Chacón_Barnes_2011, Chacón_Chen_Barnes_2013, Chen_Chacón_2014, Chen_Chacón_2015} have developed structure-preserving particle pushers for neoclassical transport in the Vlasov equations, derived from Crank--Nicolson integrators. We show these too can can derive from a FET interpretation, similarly offering potential extensions to higher-order-in-time particle pushers. The FET formulation is used also to consider how the stochastic drift terms can be incorporated into the pushers. Stochastic gyrokinetic expansions are also discussed.

        Different options for the numerical implementation of these schemes are considered.

        Due to the efficacy of FET in the development of SP timesteppers for both the fluid and kinetic component, we hope this approach will prove effective in the future for developing SP timesteppers for the full hybrid model. We hope this will give us the opportunity to incorporate previously inaccessible kinetic effects into the highly effective, modern, finite-element MHD models.
    \end{abstract}
    
    
    \newpage
    \tableofcontents
    
    
    \newpage
    \pagenumbering{arabic}
    %\linenumbers\renewcommand\thelinenumber{\color{black!50}\arabic{linenumber}}
            \input{0 - introduction/main.tex}
        \part{Research}
            \input{1 - low-noise PiC models/main.tex}
            \input{2 - kinetic component/main.tex}
            \input{3 - fluid component/main.tex}
            \input{4 - numerical implementation/main.tex}
        \part{Project Overview}
            \input{5 - research plan/main.tex}
            \input{6 - summary/main.tex}
    
    
    %\section{}
    \newpage
    \pagenumbering{gobble}
        \printbibliography


    \newpage
    \pagenumbering{roman}
    \appendix
        \part{Appendices}
            \input{8 - Hilbert complexes/main.tex}
            \input{9 - weak conservation proofs/main.tex}
\end{document}

\end{document}

            \documentclass[12pt, a4paper]{report}

\documentclass[12pt, a4paper]{report}

\input{template/main.tex}

\title{\BA{Title in Progress...}}
\author{Boris Andrews}
\affil{Mathematical Institute, University of Oxford}
\date{\today}


\begin{document}
    \pagenumbering{gobble}
    \maketitle
    
    
    \begin{abstract}
        Magnetic confinement reactors---in particular tokamaks---offer one of the most promising options for achieving practical nuclear fusion, with the potential to provide virtually limitless, clean energy. The theoretical and numerical modeling of tokamak plasmas is simultaneously an essential component of effective reactor design, and a great research barrier. Tokamak operational conditions exhibit comparatively low Knudsen numbers. Kinetic effects, including kinetic waves and instabilities, Landau damping, bump-on-tail instabilities and more, are therefore highly influential in tokamak plasma dynamics. Purely fluid models are inherently incapable of capturing these effects, whereas the high dimensionality in purely kinetic models render them practically intractable for most relevant purposes.

        We consider a $\delta\!f$ decomposition model, with a macroscopic fluid background and microscopic kinetic correction, both fully coupled to each other. A similar manner of discretization is proposed to that used in the recent \texttt{STRUPHY} code \cite{Holderied_Possanner_Wang_2021, Holderied_2022, Li_et_al_2023} with a finite-element model for the background and a pseudo-particle/PiC model for the correction.

        The fluid background satisfies the full, non-linear, resistive, compressible, Hall MHD equations. \cite{Laakmann_Hu_Farrell_2022} introduces finite-element(-in-space) implicit timesteppers for the incompressible analogue to this system with structure-preserving (SP) properties in the ideal case, alongside parameter-robust preconditioners. We show that these timesteppers can derive from a finite-element-in-time (FET) (and finite-element-in-space) interpretation. The benefits of this reformulation are discussed, including the derivation of timesteppers that are higher order in time, and the quantifiable dissipative SP properties in the non-ideal, resistive case.
        
        We discuss possible options for extending this FET approach to timesteppers for the compressible case.

        The kinetic corrections satisfy linearized Boltzmann equations. Using a Lénard--Bernstein collision operator, these take Fokker--Planck-like forms \cite{Fokker_1914, Planck_1917} wherein pseudo-particles in the numerical model obey the neoclassical transport equations, with particle-independent Brownian drift terms. This offers a rigorous methodology for incorporating collisions into the particle transport model, without coupling the equations of motions for each particle.
        
        Works by Chen, Chacón et al. \cite{Chen_Chacón_Barnes_2011, Chacón_Chen_Barnes_2013, Chen_Chacón_2014, Chen_Chacón_2015} have developed structure-preserving particle pushers for neoclassical transport in the Vlasov equations, derived from Crank--Nicolson integrators. We show these too can can derive from a FET interpretation, similarly offering potential extensions to higher-order-in-time particle pushers. The FET formulation is used also to consider how the stochastic drift terms can be incorporated into the pushers. Stochastic gyrokinetic expansions are also discussed.

        Different options for the numerical implementation of these schemes are considered.

        Due to the efficacy of FET in the development of SP timesteppers for both the fluid and kinetic component, we hope this approach will prove effective in the future for developing SP timesteppers for the full hybrid model. We hope this will give us the opportunity to incorporate previously inaccessible kinetic effects into the highly effective, modern, finite-element MHD models.
    \end{abstract}
    
    
    \newpage
    \tableofcontents
    
    
    \newpage
    \pagenumbering{arabic}
    %\linenumbers\renewcommand\thelinenumber{\color{black!50}\arabic{linenumber}}
            \input{0 - introduction/main.tex}
        \part{Research}
            \input{1 - low-noise PiC models/main.tex}
            \input{2 - kinetic component/main.tex}
            \input{3 - fluid component/main.tex}
            \input{4 - numerical implementation/main.tex}
        \part{Project Overview}
            \input{5 - research plan/main.tex}
            \input{6 - summary/main.tex}
    
    
    %\section{}
    \newpage
    \pagenumbering{gobble}
        \printbibliography


    \newpage
    \pagenumbering{roman}
    \appendix
        \part{Appendices}
            \input{8 - Hilbert complexes/main.tex}
            \input{9 - weak conservation proofs/main.tex}
\end{document}


\title{\BA{Title in Progress...}}
\author{Boris Andrews}
\affil{Mathematical Institute, University of Oxford}
\date{\today}


\begin{document}
    \pagenumbering{gobble}
    \maketitle
    
    
    \begin{abstract}
        Magnetic confinement reactors---in particular tokamaks---offer one of the most promising options for achieving practical nuclear fusion, with the potential to provide virtually limitless, clean energy. The theoretical and numerical modeling of tokamak plasmas is simultaneously an essential component of effective reactor design, and a great research barrier. Tokamak operational conditions exhibit comparatively low Knudsen numbers. Kinetic effects, including kinetic waves and instabilities, Landau damping, bump-on-tail instabilities and more, are therefore highly influential in tokamak plasma dynamics. Purely fluid models are inherently incapable of capturing these effects, whereas the high dimensionality in purely kinetic models render them practically intractable for most relevant purposes.

        We consider a $\delta\!f$ decomposition model, with a macroscopic fluid background and microscopic kinetic correction, both fully coupled to each other. A similar manner of discretization is proposed to that used in the recent \texttt{STRUPHY} code \cite{Holderied_Possanner_Wang_2021, Holderied_2022, Li_et_al_2023} with a finite-element model for the background and a pseudo-particle/PiC model for the correction.

        The fluid background satisfies the full, non-linear, resistive, compressible, Hall MHD equations. \cite{Laakmann_Hu_Farrell_2022} introduces finite-element(-in-space) implicit timesteppers for the incompressible analogue to this system with structure-preserving (SP) properties in the ideal case, alongside parameter-robust preconditioners. We show that these timesteppers can derive from a finite-element-in-time (FET) (and finite-element-in-space) interpretation. The benefits of this reformulation are discussed, including the derivation of timesteppers that are higher order in time, and the quantifiable dissipative SP properties in the non-ideal, resistive case.
        
        We discuss possible options for extending this FET approach to timesteppers for the compressible case.

        The kinetic corrections satisfy linearized Boltzmann equations. Using a Lénard--Bernstein collision operator, these take Fokker--Planck-like forms \cite{Fokker_1914, Planck_1917} wherein pseudo-particles in the numerical model obey the neoclassical transport equations, with particle-independent Brownian drift terms. This offers a rigorous methodology for incorporating collisions into the particle transport model, without coupling the equations of motions for each particle.
        
        Works by Chen, Chacón et al. \cite{Chen_Chacón_Barnes_2011, Chacón_Chen_Barnes_2013, Chen_Chacón_2014, Chen_Chacón_2015} have developed structure-preserving particle pushers for neoclassical transport in the Vlasov equations, derived from Crank--Nicolson integrators. We show these too can can derive from a FET interpretation, similarly offering potential extensions to higher-order-in-time particle pushers. The FET formulation is used also to consider how the stochastic drift terms can be incorporated into the pushers. Stochastic gyrokinetic expansions are also discussed.

        Different options for the numerical implementation of these schemes are considered.

        Due to the efficacy of FET in the development of SP timesteppers for both the fluid and kinetic component, we hope this approach will prove effective in the future for developing SP timesteppers for the full hybrid model. We hope this will give us the opportunity to incorporate previously inaccessible kinetic effects into the highly effective, modern, finite-element MHD models.
    \end{abstract}
    
    
    \newpage
    \tableofcontents
    
    
    \newpage
    \pagenumbering{arabic}
    %\linenumbers\renewcommand\thelinenumber{\color{black!50}\arabic{linenumber}}
            \documentclass[12pt, a4paper]{report}

\input{template/main.tex}

\title{\BA{Title in Progress...}}
\author{Boris Andrews}
\affil{Mathematical Institute, University of Oxford}
\date{\today}


\begin{document}
    \pagenumbering{gobble}
    \maketitle
    
    
    \begin{abstract}
        Magnetic confinement reactors---in particular tokamaks---offer one of the most promising options for achieving practical nuclear fusion, with the potential to provide virtually limitless, clean energy. The theoretical and numerical modeling of tokamak plasmas is simultaneously an essential component of effective reactor design, and a great research barrier. Tokamak operational conditions exhibit comparatively low Knudsen numbers. Kinetic effects, including kinetic waves and instabilities, Landau damping, bump-on-tail instabilities and more, are therefore highly influential in tokamak plasma dynamics. Purely fluid models are inherently incapable of capturing these effects, whereas the high dimensionality in purely kinetic models render them practically intractable for most relevant purposes.

        We consider a $\delta\!f$ decomposition model, with a macroscopic fluid background and microscopic kinetic correction, both fully coupled to each other. A similar manner of discretization is proposed to that used in the recent \texttt{STRUPHY} code \cite{Holderied_Possanner_Wang_2021, Holderied_2022, Li_et_al_2023} with a finite-element model for the background and a pseudo-particle/PiC model for the correction.

        The fluid background satisfies the full, non-linear, resistive, compressible, Hall MHD equations. \cite{Laakmann_Hu_Farrell_2022} introduces finite-element(-in-space) implicit timesteppers for the incompressible analogue to this system with structure-preserving (SP) properties in the ideal case, alongside parameter-robust preconditioners. We show that these timesteppers can derive from a finite-element-in-time (FET) (and finite-element-in-space) interpretation. The benefits of this reformulation are discussed, including the derivation of timesteppers that are higher order in time, and the quantifiable dissipative SP properties in the non-ideal, resistive case.
        
        We discuss possible options for extending this FET approach to timesteppers for the compressible case.

        The kinetic corrections satisfy linearized Boltzmann equations. Using a Lénard--Bernstein collision operator, these take Fokker--Planck-like forms \cite{Fokker_1914, Planck_1917} wherein pseudo-particles in the numerical model obey the neoclassical transport equations, with particle-independent Brownian drift terms. This offers a rigorous methodology for incorporating collisions into the particle transport model, without coupling the equations of motions for each particle.
        
        Works by Chen, Chacón et al. \cite{Chen_Chacón_Barnes_2011, Chacón_Chen_Barnes_2013, Chen_Chacón_2014, Chen_Chacón_2015} have developed structure-preserving particle pushers for neoclassical transport in the Vlasov equations, derived from Crank--Nicolson integrators. We show these too can can derive from a FET interpretation, similarly offering potential extensions to higher-order-in-time particle pushers. The FET formulation is used also to consider how the stochastic drift terms can be incorporated into the pushers. Stochastic gyrokinetic expansions are also discussed.

        Different options for the numerical implementation of these schemes are considered.

        Due to the efficacy of FET in the development of SP timesteppers for both the fluid and kinetic component, we hope this approach will prove effective in the future for developing SP timesteppers for the full hybrid model. We hope this will give us the opportunity to incorporate previously inaccessible kinetic effects into the highly effective, modern, finite-element MHD models.
    \end{abstract}
    
    
    \newpage
    \tableofcontents
    
    
    \newpage
    \pagenumbering{arabic}
    %\linenumbers\renewcommand\thelinenumber{\color{black!50}\arabic{linenumber}}
            \input{0 - introduction/main.tex}
        \part{Research}
            \input{1 - low-noise PiC models/main.tex}
            \input{2 - kinetic component/main.tex}
            \input{3 - fluid component/main.tex}
            \input{4 - numerical implementation/main.tex}
        \part{Project Overview}
            \input{5 - research plan/main.tex}
            \input{6 - summary/main.tex}
    
    
    %\section{}
    \newpage
    \pagenumbering{gobble}
        \printbibliography


    \newpage
    \pagenumbering{roman}
    \appendix
        \part{Appendices}
            \input{8 - Hilbert complexes/main.tex}
            \input{9 - weak conservation proofs/main.tex}
\end{document}

        \part{Research}
            \documentclass[12pt, a4paper]{report}

\input{template/main.tex}

\title{\BA{Title in Progress...}}
\author{Boris Andrews}
\affil{Mathematical Institute, University of Oxford}
\date{\today}


\begin{document}
    \pagenumbering{gobble}
    \maketitle
    
    
    \begin{abstract}
        Magnetic confinement reactors---in particular tokamaks---offer one of the most promising options for achieving practical nuclear fusion, with the potential to provide virtually limitless, clean energy. The theoretical and numerical modeling of tokamak plasmas is simultaneously an essential component of effective reactor design, and a great research barrier. Tokamak operational conditions exhibit comparatively low Knudsen numbers. Kinetic effects, including kinetic waves and instabilities, Landau damping, bump-on-tail instabilities and more, are therefore highly influential in tokamak plasma dynamics. Purely fluid models are inherently incapable of capturing these effects, whereas the high dimensionality in purely kinetic models render them practically intractable for most relevant purposes.

        We consider a $\delta\!f$ decomposition model, with a macroscopic fluid background and microscopic kinetic correction, both fully coupled to each other. A similar manner of discretization is proposed to that used in the recent \texttt{STRUPHY} code \cite{Holderied_Possanner_Wang_2021, Holderied_2022, Li_et_al_2023} with a finite-element model for the background and a pseudo-particle/PiC model for the correction.

        The fluid background satisfies the full, non-linear, resistive, compressible, Hall MHD equations. \cite{Laakmann_Hu_Farrell_2022} introduces finite-element(-in-space) implicit timesteppers for the incompressible analogue to this system with structure-preserving (SP) properties in the ideal case, alongside parameter-robust preconditioners. We show that these timesteppers can derive from a finite-element-in-time (FET) (and finite-element-in-space) interpretation. The benefits of this reformulation are discussed, including the derivation of timesteppers that are higher order in time, and the quantifiable dissipative SP properties in the non-ideal, resistive case.
        
        We discuss possible options for extending this FET approach to timesteppers for the compressible case.

        The kinetic corrections satisfy linearized Boltzmann equations. Using a Lénard--Bernstein collision operator, these take Fokker--Planck-like forms \cite{Fokker_1914, Planck_1917} wherein pseudo-particles in the numerical model obey the neoclassical transport equations, with particle-independent Brownian drift terms. This offers a rigorous methodology for incorporating collisions into the particle transport model, without coupling the equations of motions for each particle.
        
        Works by Chen, Chacón et al. \cite{Chen_Chacón_Barnes_2011, Chacón_Chen_Barnes_2013, Chen_Chacón_2014, Chen_Chacón_2015} have developed structure-preserving particle pushers for neoclassical transport in the Vlasov equations, derived from Crank--Nicolson integrators. We show these too can can derive from a FET interpretation, similarly offering potential extensions to higher-order-in-time particle pushers. The FET formulation is used also to consider how the stochastic drift terms can be incorporated into the pushers. Stochastic gyrokinetic expansions are also discussed.

        Different options for the numerical implementation of these schemes are considered.

        Due to the efficacy of FET in the development of SP timesteppers for both the fluid and kinetic component, we hope this approach will prove effective in the future for developing SP timesteppers for the full hybrid model. We hope this will give us the opportunity to incorporate previously inaccessible kinetic effects into the highly effective, modern, finite-element MHD models.
    \end{abstract}
    
    
    \newpage
    \tableofcontents
    
    
    \newpage
    \pagenumbering{arabic}
    %\linenumbers\renewcommand\thelinenumber{\color{black!50}\arabic{linenumber}}
            \input{0 - introduction/main.tex}
        \part{Research}
            \input{1 - low-noise PiC models/main.tex}
            \input{2 - kinetic component/main.tex}
            \input{3 - fluid component/main.tex}
            \input{4 - numerical implementation/main.tex}
        \part{Project Overview}
            \input{5 - research plan/main.tex}
            \input{6 - summary/main.tex}
    
    
    %\section{}
    \newpage
    \pagenumbering{gobble}
        \printbibliography


    \newpage
    \pagenumbering{roman}
    \appendix
        \part{Appendices}
            \input{8 - Hilbert complexes/main.tex}
            \input{9 - weak conservation proofs/main.tex}
\end{document}

            \documentclass[12pt, a4paper]{report}

\input{template/main.tex}

\title{\BA{Title in Progress...}}
\author{Boris Andrews}
\affil{Mathematical Institute, University of Oxford}
\date{\today}


\begin{document}
    \pagenumbering{gobble}
    \maketitle
    
    
    \begin{abstract}
        Magnetic confinement reactors---in particular tokamaks---offer one of the most promising options for achieving practical nuclear fusion, with the potential to provide virtually limitless, clean energy. The theoretical and numerical modeling of tokamak plasmas is simultaneously an essential component of effective reactor design, and a great research barrier. Tokamak operational conditions exhibit comparatively low Knudsen numbers. Kinetic effects, including kinetic waves and instabilities, Landau damping, bump-on-tail instabilities and more, are therefore highly influential in tokamak plasma dynamics. Purely fluid models are inherently incapable of capturing these effects, whereas the high dimensionality in purely kinetic models render them practically intractable for most relevant purposes.

        We consider a $\delta\!f$ decomposition model, with a macroscopic fluid background and microscopic kinetic correction, both fully coupled to each other. A similar manner of discretization is proposed to that used in the recent \texttt{STRUPHY} code \cite{Holderied_Possanner_Wang_2021, Holderied_2022, Li_et_al_2023} with a finite-element model for the background and a pseudo-particle/PiC model for the correction.

        The fluid background satisfies the full, non-linear, resistive, compressible, Hall MHD equations. \cite{Laakmann_Hu_Farrell_2022} introduces finite-element(-in-space) implicit timesteppers for the incompressible analogue to this system with structure-preserving (SP) properties in the ideal case, alongside parameter-robust preconditioners. We show that these timesteppers can derive from a finite-element-in-time (FET) (and finite-element-in-space) interpretation. The benefits of this reformulation are discussed, including the derivation of timesteppers that are higher order in time, and the quantifiable dissipative SP properties in the non-ideal, resistive case.
        
        We discuss possible options for extending this FET approach to timesteppers for the compressible case.

        The kinetic corrections satisfy linearized Boltzmann equations. Using a Lénard--Bernstein collision operator, these take Fokker--Planck-like forms \cite{Fokker_1914, Planck_1917} wherein pseudo-particles in the numerical model obey the neoclassical transport equations, with particle-independent Brownian drift terms. This offers a rigorous methodology for incorporating collisions into the particle transport model, without coupling the equations of motions for each particle.
        
        Works by Chen, Chacón et al. \cite{Chen_Chacón_Barnes_2011, Chacón_Chen_Barnes_2013, Chen_Chacón_2014, Chen_Chacón_2015} have developed structure-preserving particle pushers for neoclassical transport in the Vlasov equations, derived from Crank--Nicolson integrators. We show these too can can derive from a FET interpretation, similarly offering potential extensions to higher-order-in-time particle pushers. The FET formulation is used also to consider how the stochastic drift terms can be incorporated into the pushers. Stochastic gyrokinetic expansions are also discussed.

        Different options for the numerical implementation of these schemes are considered.

        Due to the efficacy of FET in the development of SP timesteppers for both the fluid and kinetic component, we hope this approach will prove effective in the future for developing SP timesteppers for the full hybrid model. We hope this will give us the opportunity to incorporate previously inaccessible kinetic effects into the highly effective, modern, finite-element MHD models.
    \end{abstract}
    
    
    \newpage
    \tableofcontents
    
    
    \newpage
    \pagenumbering{arabic}
    %\linenumbers\renewcommand\thelinenumber{\color{black!50}\arabic{linenumber}}
            \input{0 - introduction/main.tex}
        \part{Research}
            \input{1 - low-noise PiC models/main.tex}
            \input{2 - kinetic component/main.tex}
            \input{3 - fluid component/main.tex}
            \input{4 - numerical implementation/main.tex}
        \part{Project Overview}
            \input{5 - research plan/main.tex}
            \input{6 - summary/main.tex}
    
    
    %\section{}
    \newpage
    \pagenumbering{gobble}
        \printbibliography


    \newpage
    \pagenumbering{roman}
    \appendix
        \part{Appendices}
            \input{8 - Hilbert complexes/main.tex}
            \input{9 - weak conservation proofs/main.tex}
\end{document}

            \documentclass[12pt, a4paper]{report}

\input{template/main.tex}

\title{\BA{Title in Progress...}}
\author{Boris Andrews}
\affil{Mathematical Institute, University of Oxford}
\date{\today}


\begin{document}
    \pagenumbering{gobble}
    \maketitle
    
    
    \begin{abstract}
        Magnetic confinement reactors---in particular tokamaks---offer one of the most promising options for achieving practical nuclear fusion, with the potential to provide virtually limitless, clean energy. The theoretical and numerical modeling of tokamak plasmas is simultaneously an essential component of effective reactor design, and a great research barrier. Tokamak operational conditions exhibit comparatively low Knudsen numbers. Kinetic effects, including kinetic waves and instabilities, Landau damping, bump-on-tail instabilities and more, are therefore highly influential in tokamak plasma dynamics. Purely fluid models are inherently incapable of capturing these effects, whereas the high dimensionality in purely kinetic models render them practically intractable for most relevant purposes.

        We consider a $\delta\!f$ decomposition model, with a macroscopic fluid background and microscopic kinetic correction, both fully coupled to each other. A similar manner of discretization is proposed to that used in the recent \texttt{STRUPHY} code \cite{Holderied_Possanner_Wang_2021, Holderied_2022, Li_et_al_2023} with a finite-element model for the background and a pseudo-particle/PiC model for the correction.

        The fluid background satisfies the full, non-linear, resistive, compressible, Hall MHD equations. \cite{Laakmann_Hu_Farrell_2022} introduces finite-element(-in-space) implicit timesteppers for the incompressible analogue to this system with structure-preserving (SP) properties in the ideal case, alongside parameter-robust preconditioners. We show that these timesteppers can derive from a finite-element-in-time (FET) (and finite-element-in-space) interpretation. The benefits of this reformulation are discussed, including the derivation of timesteppers that are higher order in time, and the quantifiable dissipative SP properties in the non-ideal, resistive case.
        
        We discuss possible options for extending this FET approach to timesteppers for the compressible case.

        The kinetic corrections satisfy linearized Boltzmann equations. Using a Lénard--Bernstein collision operator, these take Fokker--Planck-like forms \cite{Fokker_1914, Planck_1917} wherein pseudo-particles in the numerical model obey the neoclassical transport equations, with particle-independent Brownian drift terms. This offers a rigorous methodology for incorporating collisions into the particle transport model, without coupling the equations of motions for each particle.
        
        Works by Chen, Chacón et al. \cite{Chen_Chacón_Barnes_2011, Chacón_Chen_Barnes_2013, Chen_Chacón_2014, Chen_Chacón_2015} have developed structure-preserving particle pushers for neoclassical transport in the Vlasov equations, derived from Crank--Nicolson integrators. We show these too can can derive from a FET interpretation, similarly offering potential extensions to higher-order-in-time particle pushers. The FET formulation is used also to consider how the stochastic drift terms can be incorporated into the pushers. Stochastic gyrokinetic expansions are also discussed.

        Different options for the numerical implementation of these schemes are considered.

        Due to the efficacy of FET in the development of SP timesteppers for both the fluid and kinetic component, we hope this approach will prove effective in the future for developing SP timesteppers for the full hybrid model. We hope this will give us the opportunity to incorporate previously inaccessible kinetic effects into the highly effective, modern, finite-element MHD models.
    \end{abstract}
    
    
    \newpage
    \tableofcontents
    
    
    \newpage
    \pagenumbering{arabic}
    %\linenumbers\renewcommand\thelinenumber{\color{black!50}\arabic{linenumber}}
            \input{0 - introduction/main.tex}
        \part{Research}
            \input{1 - low-noise PiC models/main.tex}
            \input{2 - kinetic component/main.tex}
            \input{3 - fluid component/main.tex}
            \input{4 - numerical implementation/main.tex}
        \part{Project Overview}
            \input{5 - research plan/main.tex}
            \input{6 - summary/main.tex}
    
    
    %\section{}
    \newpage
    \pagenumbering{gobble}
        \printbibliography


    \newpage
    \pagenumbering{roman}
    \appendix
        \part{Appendices}
            \input{8 - Hilbert complexes/main.tex}
            \input{9 - weak conservation proofs/main.tex}
\end{document}

            \documentclass[12pt, a4paper]{report}

\input{template/main.tex}

\title{\BA{Title in Progress...}}
\author{Boris Andrews}
\affil{Mathematical Institute, University of Oxford}
\date{\today}


\begin{document}
    \pagenumbering{gobble}
    \maketitle
    
    
    \begin{abstract}
        Magnetic confinement reactors---in particular tokamaks---offer one of the most promising options for achieving practical nuclear fusion, with the potential to provide virtually limitless, clean energy. The theoretical and numerical modeling of tokamak plasmas is simultaneously an essential component of effective reactor design, and a great research barrier. Tokamak operational conditions exhibit comparatively low Knudsen numbers. Kinetic effects, including kinetic waves and instabilities, Landau damping, bump-on-tail instabilities and more, are therefore highly influential in tokamak plasma dynamics. Purely fluid models are inherently incapable of capturing these effects, whereas the high dimensionality in purely kinetic models render them practically intractable for most relevant purposes.

        We consider a $\delta\!f$ decomposition model, with a macroscopic fluid background and microscopic kinetic correction, both fully coupled to each other. A similar manner of discretization is proposed to that used in the recent \texttt{STRUPHY} code \cite{Holderied_Possanner_Wang_2021, Holderied_2022, Li_et_al_2023} with a finite-element model for the background and a pseudo-particle/PiC model for the correction.

        The fluid background satisfies the full, non-linear, resistive, compressible, Hall MHD equations. \cite{Laakmann_Hu_Farrell_2022} introduces finite-element(-in-space) implicit timesteppers for the incompressible analogue to this system with structure-preserving (SP) properties in the ideal case, alongside parameter-robust preconditioners. We show that these timesteppers can derive from a finite-element-in-time (FET) (and finite-element-in-space) interpretation. The benefits of this reformulation are discussed, including the derivation of timesteppers that are higher order in time, and the quantifiable dissipative SP properties in the non-ideal, resistive case.
        
        We discuss possible options for extending this FET approach to timesteppers for the compressible case.

        The kinetic corrections satisfy linearized Boltzmann equations. Using a Lénard--Bernstein collision operator, these take Fokker--Planck-like forms \cite{Fokker_1914, Planck_1917} wherein pseudo-particles in the numerical model obey the neoclassical transport equations, with particle-independent Brownian drift terms. This offers a rigorous methodology for incorporating collisions into the particle transport model, without coupling the equations of motions for each particle.
        
        Works by Chen, Chacón et al. \cite{Chen_Chacón_Barnes_2011, Chacón_Chen_Barnes_2013, Chen_Chacón_2014, Chen_Chacón_2015} have developed structure-preserving particle pushers for neoclassical transport in the Vlasov equations, derived from Crank--Nicolson integrators. We show these too can can derive from a FET interpretation, similarly offering potential extensions to higher-order-in-time particle pushers. The FET formulation is used also to consider how the stochastic drift terms can be incorporated into the pushers. Stochastic gyrokinetic expansions are also discussed.

        Different options for the numerical implementation of these schemes are considered.

        Due to the efficacy of FET in the development of SP timesteppers for both the fluid and kinetic component, we hope this approach will prove effective in the future for developing SP timesteppers for the full hybrid model. We hope this will give us the opportunity to incorporate previously inaccessible kinetic effects into the highly effective, modern, finite-element MHD models.
    \end{abstract}
    
    
    \newpage
    \tableofcontents
    
    
    \newpage
    \pagenumbering{arabic}
    %\linenumbers\renewcommand\thelinenumber{\color{black!50}\arabic{linenumber}}
            \input{0 - introduction/main.tex}
        \part{Research}
            \input{1 - low-noise PiC models/main.tex}
            \input{2 - kinetic component/main.tex}
            \input{3 - fluid component/main.tex}
            \input{4 - numerical implementation/main.tex}
        \part{Project Overview}
            \input{5 - research plan/main.tex}
            \input{6 - summary/main.tex}
    
    
    %\section{}
    \newpage
    \pagenumbering{gobble}
        \printbibliography


    \newpage
    \pagenumbering{roman}
    \appendix
        \part{Appendices}
            \input{8 - Hilbert complexes/main.tex}
            \input{9 - weak conservation proofs/main.tex}
\end{document}

        \part{Project Overview}
            \documentclass[12pt, a4paper]{report}

\input{template/main.tex}

\title{\BA{Title in Progress...}}
\author{Boris Andrews}
\affil{Mathematical Institute, University of Oxford}
\date{\today}


\begin{document}
    \pagenumbering{gobble}
    \maketitle
    
    
    \begin{abstract}
        Magnetic confinement reactors---in particular tokamaks---offer one of the most promising options for achieving practical nuclear fusion, with the potential to provide virtually limitless, clean energy. The theoretical and numerical modeling of tokamak plasmas is simultaneously an essential component of effective reactor design, and a great research barrier. Tokamak operational conditions exhibit comparatively low Knudsen numbers. Kinetic effects, including kinetic waves and instabilities, Landau damping, bump-on-tail instabilities and more, are therefore highly influential in tokamak plasma dynamics. Purely fluid models are inherently incapable of capturing these effects, whereas the high dimensionality in purely kinetic models render them practically intractable for most relevant purposes.

        We consider a $\delta\!f$ decomposition model, with a macroscopic fluid background and microscopic kinetic correction, both fully coupled to each other. A similar manner of discretization is proposed to that used in the recent \texttt{STRUPHY} code \cite{Holderied_Possanner_Wang_2021, Holderied_2022, Li_et_al_2023} with a finite-element model for the background and a pseudo-particle/PiC model for the correction.

        The fluid background satisfies the full, non-linear, resistive, compressible, Hall MHD equations. \cite{Laakmann_Hu_Farrell_2022} introduces finite-element(-in-space) implicit timesteppers for the incompressible analogue to this system with structure-preserving (SP) properties in the ideal case, alongside parameter-robust preconditioners. We show that these timesteppers can derive from a finite-element-in-time (FET) (and finite-element-in-space) interpretation. The benefits of this reformulation are discussed, including the derivation of timesteppers that are higher order in time, and the quantifiable dissipative SP properties in the non-ideal, resistive case.
        
        We discuss possible options for extending this FET approach to timesteppers for the compressible case.

        The kinetic corrections satisfy linearized Boltzmann equations. Using a Lénard--Bernstein collision operator, these take Fokker--Planck-like forms \cite{Fokker_1914, Planck_1917} wherein pseudo-particles in the numerical model obey the neoclassical transport equations, with particle-independent Brownian drift terms. This offers a rigorous methodology for incorporating collisions into the particle transport model, without coupling the equations of motions for each particle.
        
        Works by Chen, Chacón et al. \cite{Chen_Chacón_Barnes_2011, Chacón_Chen_Barnes_2013, Chen_Chacón_2014, Chen_Chacón_2015} have developed structure-preserving particle pushers for neoclassical transport in the Vlasov equations, derived from Crank--Nicolson integrators. We show these too can can derive from a FET interpretation, similarly offering potential extensions to higher-order-in-time particle pushers. The FET formulation is used also to consider how the stochastic drift terms can be incorporated into the pushers. Stochastic gyrokinetic expansions are also discussed.

        Different options for the numerical implementation of these schemes are considered.

        Due to the efficacy of FET in the development of SP timesteppers for both the fluid and kinetic component, we hope this approach will prove effective in the future for developing SP timesteppers for the full hybrid model. We hope this will give us the opportunity to incorporate previously inaccessible kinetic effects into the highly effective, modern, finite-element MHD models.
    \end{abstract}
    
    
    \newpage
    \tableofcontents
    
    
    \newpage
    \pagenumbering{arabic}
    %\linenumbers\renewcommand\thelinenumber{\color{black!50}\arabic{linenumber}}
            \input{0 - introduction/main.tex}
        \part{Research}
            \input{1 - low-noise PiC models/main.tex}
            \input{2 - kinetic component/main.tex}
            \input{3 - fluid component/main.tex}
            \input{4 - numerical implementation/main.tex}
        \part{Project Overview}
            \input{5 - research plan/main.tex}
            \input{6 - summary/main.tex}
    
    
    %\section{}
    \newpage
    \pagenumbering{gobble}
        \printbibliography


    \newpage
    \pagenumbering{roman}
    \appendix
        \part{Appendices}
            \input{8 - Hilbert complexes/main.tex}
            \input{9 - weak conservation proofs/main.tex}
\end{document}

            \documentclass[12pt, a4paper]{report}

\input{template/main.tex}

\title{\BA{Title in Progress...}}
\author{Boris Andrews}
\affil{Mathematical Institute, University of Oxford}
\date{\today}


\begin{document}
    \pagenumbering{gobble}
    \maketitle
    
    
    \begin{abstract}
        Magnetic confinement reactors---in particular tokamaks---offer one of the most promising options for achieving practical nuclear fusion, with the potential to provide virtually limitless, clean energy. The theoretical and numerical modeling of tokamak plasmas is simultaneously an essential component of effective reactor design, and a great research barrier. Tokamak operational conditions exhibit comparatively low Knudsen numbers. Kinetic effects, including kinetic waves and instabilities, Landau damping, bump-on-tail instabilities and more, are therefore highly influential in tokamak plasma dynamics. Purely fluid models are inherently incapable of capturing these effects, whereas the high dimensionality in purely kinetic models render them practically intractable for most relevant purposes.

        We consider a $\delta\!f$ decomposition model, with a macroscopic fluid background and microscopic kinetic correction, both fully coupled to each other. A similar manner of discretization is proposed to that used in the recent \texttt{STRUPHY} code \cite{Holderied_Possanner_Wang_2021, Holderied_2022, Li_et_al_2023} with a finite-element model for the background and a pseudo-particle/PiC model for the correction.

        The fluid background satisfies the full, non-linear, resistive, compressible, Hall MHD equations. \cite{Laakmann_Hu_Farrell_2022} introduces finite-element(-in-space) implicit timesteppers for the incompressible analogue to this system with structure-preserving (SP) properties in the ideal case, alongside parameter-robust preconditioners. We show that these timesteppers can derive from a finite-element-in-time (FET) (and finite-element-in-space) interpretation. The benefits of this reformulation are discussed, including the derivation of timesteppers that are higher order in time, and the quantifiable dissipative SP properties in the non-ideal, resistive case.
        
        We discuss possible options for extending this FET approach to timesteppers for the compressible case.

        The kinetic corrections satisfy linearized Boltzmann equations. Using a Lénard--Bernstein collision operator, these take Fokker--Planck-like forms \cite{Fokker_1914, Planck_1917} wherein pseudo-particles in the numerical model obey the neoclassical transport equations, with particle-independent Brownian drift terms. This offers a rigorous methodology for incorporating collisions into the particle transport model, without coupling the equations of motions for each particle.
        
        Works by Chen, Chacón et al. \cite{Chen_Chacón_Barnes_2011, Chacón_Chen_Barnes_2013, Chen_Chacón_2014, Chen_Chacón_2015} have developed structure-preserving particle pushers for neoclassical transport in the Vlasov equations, derived from Crank--Nicolson integrators. We show these too can can derive from a FET interpretation, similarly offering potential extensions to higher-order-in-time particle pushers. The FET formulation is used also to consider how the stochastic drift terms can be incorporated into the pushers. Stochastic gyrokinetic expansions are also discussed.

        Different options for the numerical implementation of these schemes are considered.

        Due to the efficacy of FET in the development of SP timesteppers for both the fluid and kinetic component, we hope this approach will prove effective in the future for developing SP timesteppers for the full hybrid model. We hope this will give us the opportunity to incorporate previously inaccessible kinetic effects into the highly effective, modern, finite-element MHD models.
    \end{abstract}
    
    
    \newpage
    \tableofcontents
    
    
    \newpage
    \pagenumbering{arabic}
    %\linenumbers\renewcommand\thelinenumber{\color{black!50}\arabic{linenumber}}
            \input{0 - introduction/main.tex}
        \part{Research}
            \input{1 - low-noise PiC models/main.tex}
            \input{2 - kinetic component/main.tex}
            \input{3 - fluid component/main.tex}
            \input{4 - numerical implementation/main.tex}
        \part{Project Overview}
            \input{5 - research plan/main.tex}
            \input{6 - summary/main.tex}
    
    
    %\section{}
    \newpage
    \pagenumbering{gobble}
        \printbibliography


    \newpage
    \pagenumbering{roman}
    \appendix
        \part{Appendices}
            \input{8 - Hilbert complexes/main.tex}
            \input{9 - weak conservation proofs/main.tex}
\end{document}

    
    
    %\section{}
    \newpage
    \pagenumbering{gobble}
        \printbibliography


    \newpage
    \pagenumbering{roman}
    \appendix
        \part{Appendices}
            \documentclass[12pt, a4paper]{report}

\input{template/main.tex}

\title{\BA{Title in Progress...}}
\author{Boris Andrews}
\affil{Mathematical Institute, University of Oxford}
\date{\today}


\begin{document}
    \pagenumbering{gobble}
    \maketitle
    
    
    \begin{abstract}
        Magnetic confinement reactors---in particular tokamaks---offer one of the most promising options for achieving practical nuclear fusion, with the potential to provide virtually limitless, clean energy. The theoretical and numerical modeling of tokamak plasmas is simultaneously an essential component of effective reactor design, and a great research barrier. Tokamak operational conditions exhibit comparatively low Knudsen numbers. Kinetic effects, including kinetic waves and instabilities, Landau damping, bump-on-tail instabilities and more, are therefore highly influential in tokamak plasma dynamics. Purely fluid models are inherently incapable of capturing these effects, whereas the high dimensionality in purely kinetic models render them practically intractable for most relevant purposes.

        We consider a $\delta\!f$ decomposition model, with a macroscopic fluid background and microscopic kinetic correction, both fully coupled to each other. A similar manner of discretization is proposed to that used in the recent \texttt{STRUPHY} code \cite{Holderied_Possanner_Wang_2021, Holderied_2022, Li_et_al_2023} with a finite-element model for the background and a pseudo-particle/PiC model for the correction.

        The fluid background satisfies the full, non-linear, resistive, compressible, Hall MHD equations. \cite{Laakmann_Hu_Farrell_2022} introduces finite-element(-in-space) implicit timesteppers for the incompressible analogue to this system with structure-preserving (SP) properties in the ideal case, alongside parameter-robust preconditioners. We show that these timesteppers can derive from a finite-element-in-time (FET) (and finite-element-in-space) interpretation. The benefits of this reformulation are discussed, including the derivation of timesteppers that are higher order in time, and the quantifiable dissipative SP properties in the non-ideal, resistive case.
        
        We discuss possible options for extending this FET approach to timesteppers for the compressible case.

        The kinetic corrections satisfy linearized Boltzmann equations. Using a Lénard--Bernstein collision operator, these take Fokker--Planck-like forms \cite{Fokker_1914, Planck_1917} wherein pseudo-particles in the numerical model obey the neoclassical transport equations, with particle-independent Brownian drift terms. This offers a rigorous methodology for incorporating collisions into the particle transport model, without coupling the equations of motions for each particle.
        
        Works by Chen, Chacón et al. \cite{Chen_Chacón_Barnes_2011, Chacón_Chen_Barnes_2013, Chen_Chacón_2014, Chen_Chacón_2015} have developed structure-preserving particle pushers for neoclassical transport in the Vlasov equations, derived from Crank--Nicolson integrators. We show these too can can derive from a FET interpretation, similarly offering potential extensions to higher-order-in-time particle pushers. The FET formulation is used also to consider how the stochastic drift terms can be incorporated into the pushers. Stochastic gyrokinetic expansions are also discussed.

        Different options for the numerical implementation of these schemes are considered.

        Due to the efficacy of FET in the development of SP timesteppers for both the fluid and kinetic component, we hope this approach will prove effective in the future for developing SP timesteppers for the full hybrid model. We hope this will give us the opportunity to incorporate previously inaccessible kinetic effects into the highly effective, modern, finite-element MHD models.
    \end{abstract}
    
    
    \newpage
    \tableofcontents
    
    
    \newpage
    \pagenumbering{arabic}
    %\linenumbers\renewcommand\thelinenumber{\color{black!50}\arabic{linenumber}}
            \input{0 - introduction/main.tex}
        \part{Research}
            \input{1 - low-noise PiC models/main.tex}
            \input{2 - kinetic component/main.tex}
            \input{3 - fluid component/main.tex}
            \input{4 - numerical implementation/main.tex}
        \part{Project Overview}
            \input{5 - research plan/main.tex}
            \input{6 - summary/main.tex}
    
    
    %\section{}
    \newpage
    \pagenumbering{gobble}
        \printbibliography


    \newpage
    \pagenumbering{roman}
    \appendix
        \part{Appendices}
            \input{8 - Hilbert complexes/main.tex}
            \input{9 - weak conservation proofs/main.tex}
\end{document}

            \documentclass[12pt, a4paper]{report}

\input{template/main.tex}

\title{\BA{Title in Progress...}}
\author{Boris Andrews}
\affil{Mathematical Institute, University of Oxford}
\date{\today}


\begin{document}
    \pagenumbering{gobble}
    \maketitle
    
    
    \begin{abstract}
        Magnetic confinement reactors---in particular tokamaks---offer one of the most promising options for achieving practical nuclear fusion, with the potential to provide virtually limitless, clean energy. The theoretical and numerical modeling of tokamak plasmas is simultaneously an essential component of effective reactor design, and a great research barrier. Tokamak operational conditions exhibit comparatively low Knudsen numbers. Kinetic effects, including kinetic waves and instabilities, Landau damping, bump-on-tail instabilities and more, are therefore highly influential in tokamak plasma dynamics. Purely fluid models are inherently incapable of capturing these effects, whereas the high dimensionality in purely kinetic models render them practically intractable for most relevant purposes.

        We consider a $\delta\!f$ decomposition model, with a macroscopic fluid background and microscopic kinetic correction, both fully coupled to each other. A similar manner of discretization is proposed to that used in the recent \texttt{STRUPHY} code \cite{Holderied_Possanner_Wang_2021, Holderied_2022, Li_et_al_2023} with a finite-element model for the background and a pseudo-particle/PiC model for the correction.

        The fluid background satisfies the full, non-linear, resistive, compressible, Hall MHD equations. \cite{Laakmann_Hu_Farrell_2022} introduces finite-element(-in-space) implicit timesteppers for the incompressible analogue to this system with structure-preserving (SP) properties in the ideal case, alongside parameter-robust preconditioners. We show that these timesteppers can derive from a finite-element-in-time (FET) (and finite-element-in-space) interpretation. The benefits of this reformulation are discussed, including the derivation of timesteppers that are higher order in time, and the quantifiable dissipative SP properties in the non-ideal, resistive case.
        
        We discuss possible options for extending this FET approach to timesteppers for the compressible case.

        The kinetic corrections satisfy linearized Boltzmann equations. Using a Lénard--Bernstein collision operator, these take Fokker--Planck-like forms \cite{Fokker_1914, Planck_1917} wherein pseudo-particles in the numerical model obey the neoclassical transport equations, with particle-independent Brownian drift terms. This offers a rigorous methodology for incorporating collisions into the particle transport model, without coupling the equations of motions for each particle.
        
        Works by Chen, Chacón et al. \cite{Chen_Chacón_Barnes_2011, Chacón_Chen_Barnes_2013, Chen_Chacón_2014, Chen_Chacón_2015} have developed structure-preserving particle pushers for neoclassical transport in the Vlasov equations, derived from Crank--Nicolson integrators. We show these too can can derive from a FET interpretation, similarly offering potential extensions to higher-order-in-time particle pushers. The FET formulation is used also to consider how the stochastic drift terms can be incorporated into the pushers. Stochastic gyrokinetic expansions are also discussed.

        Different options for the numerical implementation of these schemes are considered.

        Due to the efficacy of FET in the development of SP timesteppers for both the fluid and kinetic component, we hope this approach will prove effective in the future for developing SP timesteppers for the full hybrid model. We hope this will give us the opportunity to incorporate previously inaccessible kinetic effects into the highly effective, modern, finite-element MHD models.
    \end{abstract}
    
    
    \newpage
    \tableofcontents
    
    
    \newpage
    \pagenumbering{arabic}
    %\linenumbers\renewcommand\thelinenumber{\color{black!50}\arabic{linenumber}}
            \input{0 - introduction/main.tex}
        \part{Research}
            \input{1 - low-noise PiC models/main.tex}
            \input{2 - kinetic component/main.tex}
            \input{3 - fluid component/main.tex}
            \input{4 - numerical implementation/main.tex}
        \part{Project Overview}
            \input{5 - research plan/main.tex}
            \input{6 - summary/main.tex}
    
    
    %\section{}
    \newpage
    \pagenumbering{gobble}
        \printbibliography


    \newpage
    \pagenumbering{roman}
    \appendix
        \part{Appendices}
            \input{8 - Hilbert complexes/main.tex}
            \input{9 - weak conservation proofs/main.tex}
\end{document}

\end{document}

\end{document}

    
    
    %\section{}
    \newpage
    \pagenumbering{gobble}
        \printbibliography


    \newpage
    \pagenumbering{roman}
    \appendix
        \part{Appendices}
            \documentclass[12pt, a4paper]{report}

\documentclass[12pt, a4paper]{report}

\documentclass[12pt, a4paper]{report}

\input{template/main.tex}

\title{\BA{Title in Progress...}}
\author{Boris Andrews}
\affil{Mathematical Institute, University of Oxford}
\date{\today}


\begin{document}
    \pagenumbering{gobble}
    \maketitle
    
    
    \begin{abstract}
        Magnetic confinement reactors---in particular tokamaks---offer one of the most promising options for achieving practical nuclear fusion, with the potential to provide virtually limitless, clean energy. The theoretical and numerical modeling of tokamak plasmas is simultaneously an essential component of effective reactor design, and a great research barrier. Tokamak operational conditions exhibit comparatively low Knudsen numbers. Kinetic effects, including kinetic waves and instabilities, Landau damping, bump-on-tail instabilities and more, are therefore highly influential in tokamak plasma dynamics. Purely fluid models are inherently incapable of capturing these effects, whereas the high dimensionality in purely kinetic models render them practically intractable for most relevant purposes.

        We consider a $\delta\!f$ decomposition model, with a macroscopic fluid background and microscopic kinetic correction, both fully coupled to each other. A similar manner of discretization is proposed to that used in the recent \texttt{STRUPHY} code \cite{Holderied_Possanner_Wang_2021, Holderied_2022, Li_et_al_2023} with a finite-element model for the background and a pseudo-particle/PiC model for the correction.

        The fluid background satisfies the full, non-linear, resistive, compressible, Hall MHD equations. \cite{Laakmann_Hu_Farrell_2022} introduces finite-element(-in-space) implicit timesteppers for the incompressible analogue to this system with structure-preserving (SP) properties in the ideal case, alongside parameter-robust preconditioners. We show that these timesteppers can derive from a finite-element-in-time (FET) (and finite-element-in-space) interpretation. The benefits of this reformulation are discussed, including the derivation of timesteppers that are higher order in time, and the quantifiable dissipative SP properties in the non-ideal, resistive case.
        
        We discuss possible options for extending this FET approach to timesteppers for the compressible case.

        The kinetic corrections satisfy linearized Boltzmann equations. Using a Lénard--Bernstein collision operator, these take Fokker--Planck-like forms \cite{Fokker_1914, Planck_1917} wherein pseudo-particles in the numerical model obey the neoclassical transport equations, with particle-independent Brownian drift terms. This offers a rigorous methodology for incorporating collisions into the particle transport model, without coupling the equations of motions for each particle.
        
        Works by Chen, Chacón et al. \cite{Chen_Chacón_Barnes_2011, Chacón_Chen_Barnes_2013, Chen_Chacón_2014, Chen_Chacón_2015} have developed structure-preserving particle pushers for neoclassical transport in the Vlasov equations, derived from Crank--Nicolson integrators. We show these too can can derive from a FET interpretation, similarly offering potential extensions to higher-order-in-time particle pushers. The FET formulation is used also to consider how the stochastic drift terms can be incorporated into the pushers. Stochastic gyrokinetic expansions are also discussed.

        Different options for the numerical implementation of these schemes are considered.

        Due to the efficacy of FET in the development of SP timesteppers for both the fluid and kinetic component, we hope this approach will prove effective in the future for developing SP timesteppers for the full hybrid model. We hope this will give us the opportunity to incorporate previously inaccessible kinetic effects into the highly effective, modern, finite-element MHD models.
    \end{abstract}
    
    
    \newpage
    \tableofcontents
    
    
    \newpage
    \pagenumbering{arabic}
    %\linenumbers\renewcommand\thelinenumber{\color{black!50}\arabic{linenumber}}
            \input{0 - introduction/main.tex}
        \part{Research}
            \input{1 - low-noise PiC models/main.tex}
            \input{2 - kinetic component/main.tex}
            \input{3 - fluid component/main.tex}
            \input{4 - numerical implementation/main.tex}
        \part{Project Overview}
            \input{5 - research plan/main.tex}
            \input{6 - summary/main.tex}
    
    
    %\section{}
    \newpage
    \pagenumbering{gobble}
        \printbibliography


    \newpage
    \pagenumbering{roman}
    \appendix
        \part{Appendices}
            \input{8 - Hilbert complexes/main.tex}
            \input{9 - weak conservation proofs/main.tex}
\end{document}


\title{\BA{Title in Progress...}}
\author{Boris Andrews}
\affil{Mathematical Institute, University of Oxford}
\date{\today}


\begin{document}
    \pagenumbering{gobble}
    \maketitle
    
    
    \begin{abstract}
        Magnetic confinement reactors---in particular tokamaks---offer one of the most promising options for achieving practical nuclear fusion, with the potential to provide virtually limitless, clean energy. The theoretical and numerical modeling of tokamak plasmas is simultaneously an essential component of effective reactor design, and a great research barrier. Tokamak operational conditions exhibit comparatively low Knudsen numbers. Kinetic effects, including kinetic waves and instabilities, Landau damping, bump-on-tail instabilities and more, are therefore highly influential in tokamak plasma dynamics. Purely fluid models are inherently incapable of capturing these effects, whereas the high dimensionality in purely kinetic models render them practically intractable for most relevant purposes.

        We consider a $\delta\!f$ decomposition model, with a macroscopic fluid background and microscopic kinetic correction, both fully coupled to each other. A similar manner of discretization is proposed to that used in the recent \texttt{STRUPHY} code \cite{Holderied_Possanner_Wang_2021, Holderied_2022, Li_et_al_2023} with a finite-element model for the background and a pseudo-particle/PiC model for the correction.

        The fluid background satisfies the full, non-linear, resistive, compressible, Hall MHD equations. \cite{Laakmann_Hu_Farrell_2022} introduces finite-element(-in-space) implicit timesteppers for the incompressible analogue to this system with structure-preserving (SP) properties in the ideal case, alongside parameter-robust preconditioners. We show that these timesteppers can derive from a finite-element-in-time (FET) (and finite-element-in-space) interpretation. The benefits of this reformulation are discussed, including the derivation of timesteppers that are higher order in time, and the quantifiable dissipative SP properties in the non-ideal, resistive case.
        
        We discuss possible options for extending this FET approach to timesteppers for the compressible case.

        The kinetic corrections satisfy linearized Boltzmann equations. Using a Lénard--Bernstein collision operator, these take Fokker--Planck-like forms \cite{Fokker_1914, Planck_1917} wherein pseudo-particles in the numerical model obey the neoclassical transport equations, with particle-independent Brownian drift terms. This offers a rigorous methodology for incorporating collisions into the particle transport model, without coupling the equations of motions for each particle.
        
        Works by Chen, Chacón et al. \cite{Chen_Chacón_Barnes_2011, Chacón_Chen_Barnes_2013, Chen_Chacón_2014, Chen_Chacón_2015} have developed structure-preserving particle pushers for neoclassical transport in the Vlasov equations, derived from Crank--Nicolson integrators. We show these too can can derive from a FET interpretation, similarly offering potential extensions to higher-order-in-time particle pushers. The FET formulation is used also to consider how the stochastic drift terms can be incorporated into the pushers. Stochastic gyrokinetic expansions are also discussed.

        Different options for the numerical implementation of these schemes are considered.

        Due to the efficacy of FET in the development of SP timesteppers for both the fluid and kinetic component, we hope this approach will prove effective in the future for developing SP timesteppers for the full hybrid model. We hope this will give us the opportunity to incorporate previously inaccessible kinetic effects into the highly effective, modern, finite-element MHD models.
    \end{abstract}
    
    
    \newpage
    \tableofcontents
    
    
    \newpage
    \pagenumbering{arabic}
    %\linenumbers\renewcommand\thelinenumber{\color{black!50}\arabic{linenumber}}
            \documentclass[12pt, a4paper]{report}

\input{template/main.tex}

\title{\BA{Title in Progress...}}
\author{Boris Andrews}
\affil{Mathematical Institute, University of Oxford}
\date{\today}


\begin{document}
    \pagenumbering{gobble}
    \maketitle
    
    
    \begin{abstract}
        Magnetic confinement reactors---in particular tokamaks---offer one of the most promising options for achieving practical nuclear fusion, with the potential to provide virtually limitless, clean energy. The theoretical and numerical modeling of tokamak plasmas is simultaneously an essential component of effective reactor design, and a great research barrier. Tokamak operational conditions exhibit comparatively low Knudsen numbers. Kinetic effects, including kinetic waves and instabilities, Landau damping, bump-on-tail instabilities and more, are therefore highly influential in tokamak plasma dynamics. Purely fluid models are inherently incapable of capturing these effects, whereas the high dimensionality in purely kinetic models render them practically intractable for most relevant purposes.

        We consider a $\delta\!f$ decomposition model, with a macroscopic fluid background and microscopic kinetic correction, both fully coupled to each other. A similar manner of discretization is proposed to that used in the recent \texttt{STRUPHY} code \cite{Holderied_Possanner_Wang_2021, Holderied_2022, Li_et_al_2023} with a finite-element model for the background and a pseudo-particle/PiC model for the correction.

        The fluid background satisfies the full, non-linear, resistive, compressible, Hall MHD equations. \cite{Laakmann_Hu_Farrell_2022} introduces finite-element(-in-space) implicit timesteppers for the incompressible analogue to this system with structure-preserving (SP) properties in the ideal case, alongside parameter-robust preconditioners. We show that these timesteppers can derive from a finite-element-in-time (FET) (and finite-element-in-space) interpretation. The benefits of this reformulation are discussed, including the derivation of timesteppers that are higher order in time, and the quantifiable dissipative SP properties in the non-ideal, resistive case.
        
        We discuss possible options for extending this FET approach to timesteppers for the compressible case.

        The kinetic corrections satisfy linearized Boltzmann equations. Using a Lénard--Bernstein collision operator, these take Fokker--Planck-like forms \cite{Fokker_1914, Planck_1917} wherein pseudo-particles in the numerical model obey the neoclassical transport equations, with particle-independent Brownian drift terms. This offers a rigorous methodology for incorporating collisions into the particle transport model, without coupling the equations of motions for each particle.
        
        Works by Chen, Chacón et al. \cite{Chen_Chacón_Barnes_2011, Chacón_Chen_Barnes_2013, Chen_Chacón_2014, Chen_Chacón_2015} have developed structure-preserving particle pushers for neoclassical transport in the Vlasov equations, derived from Crank--Nicolson integrators. We show these too can can derive from a FET interpretation, similarly offering potential extensions to higher-order-in-time particle pushers. The FET formulation is used also to consider how the stochastic drift terms can be incorporated into the pushers. Stochastic gyrokinetic expansions are also discussed.

        Different options for the numerical implementation of these schemes are considered.

        Due to the efficacy of FET in the development of SP timesteppers for both the fluid and kinetic component, we hope this approach will prove effective in the future for developing SP timesteppers for the full hybrid model. We hope this will give us the opportunity to incorporate previously inaccessible kinetic effects into the highly effective, modern, finite-element MHD models.
    \end{abstract}
    
    
    \newpage
    \tableofcontents
    
    
    \newpage
    \pagenumbering{arabic}
    %\linenumbers\renewcommand\thelinenumber{\color{black!50}\arabic{linenumber}}
            \input{0 - introduction/main.tex}
        \part{Research}
            \input{1 - low-noise PiC models/main.tex}
            \input{2 - kinetic component/main.tex}
            \input{3 - fluid component/main.tex}
            \input{4 - numerical implementation/main.tex}
        \part{Project Overview}
            \input{5 - research plan/main.tex}
            \input{6 - summary/main.tex}
    
    
    %\section{}
    \newpage
    \pagenumbering{gobble}
        \printbibliography


    \newpage
    \pagenumbering{roman}
    \appendix
        \part{Appendices}
            \input{8 - Hilbert complexes/main.tex}
            \input{9 - weak conservation proofs/main.tex}
\end{document}

        \part{Research}
            \documentclass[12pt, a4paper]{report}

\input{template/main.tex}

\title{\BA{Title in Progress...}}
\author{Boris Andrews}
\affil{Mathematical Institute, University of Oxford}
\date{\today}


\begin{document}
    \pagenumbering{gobble}
    \maketitle
    
    
    \begin{abstract}
        Magnetic confinement reactors---in particular tokamaks---offer one of the most promising options for achieving practical nuclear fusion, with the potential to provide virtually limitless, clean energy. The theoretical and numerical modeling of tokamak plasmas is simultaneously an essential component of effective reactor design, and a great research barrier. Tokamak operational conditions exhibit comparatively low Knudsen numbers. Kinetic effects, including kinetic waves and instabilities, Landau damping, bump-on-tail instabilities and more, are therefore highly influential in tokamak plasma dynamics. Purely fluid models are inherently incapable of capturing these effects, whereas the high dimensionality in purely kinetic models render them practically intractable for most relevant purposes.

        We consider a $\delta\!f$ decomposition model, with a macroscopic fluid background and microscopic kinetic correction, both fully coupled to each other. A similar manner of discretization is proposed to that used in the recent \texttt{STRUPHY} code \cite{Holderied_Possanner_Wang_2021, Holderied_2022, Li_et_al_2023} with a finite-element model for the background and a pseudo-particle/PiC model for the correction.

        The fluid background satisfies the full, non-linear, resistive, compressible, Hall MHD equations. \cite{Laakmann_Hu_Farrell_2022} introduces finite-element(-in-space) implicit timesteppers for the incompressible analogue to this system with structure-preserving (SP) properties in the ideal case, alongside parameter-robust preconditioners. We show that these timesteppers can derive from a finite-element-in-time (FET) (and finite-element-in-space) interpretation. The benefits of this reformulation are discussed, including the derivation of timesteppers that are higher order in time, and the quantifiable dissipative SP properties in the non-ideal, resistive case.
        
        We discuss possible options for extending this FET approach to timesteppers for the compressible case.

        The kinetic corrections satisfy linearized Boltzmann equations. Using a Lénard--Bernstein collision operator, these take Fokker--Planck-like forms \cite{Fokker_1914, Planck_1917} wherein pseudo-particles in the numerical model obey the neoclassical transport equations, with particle-independent Brownian drift terms. This offers a rigorous methodology for incorporating collisions into the particle transport model, without coupling the equations of motions for each particle.
        
        Works by Chen, Chacón et al. \cite{Chen_Chacón_Barnes_2011, Chacón_Chen_Barnes_2013, Chen_Chacón_2014, Chen_Chacón_2015} have developed structure-preserving particle pushers for neoclassical transport in the Vlasov equations, derived from Crank--Nicolson integrators. We show these too can can derive from a FET interpretation, similarly offering potential extensions to higher-order-in-time particle pushers. The FET formulation is used also to consider how the stochastic drift terms can be incorporated into the pushers. Stochastic gyrokinetic expansions are also discussed.

        Different options for the numerical implementation of these schemes are considered.

        Due to the efficacy of FET in the development of SP timesteppers for both the fluid and kinetic component, we hope this approach will prove effective in the future for developing SP timesteppers for the full hybrid model. We hope this will give us the opportunity to incorporate previously inaccessible kinetic effects into the highly effective, modern, finite-element MHD models.
    \end{abstract}
    
    
    \newpage
    \tableofcontents
    
    
    \newpage
    \pagenumbering{arabic}
    %\linenumbers\renewcommand\thelinenumber{\color{black!50}\arabic{linenumber}}
            \input{0 - introduction/main.tex}
        \part{Research}
            \input{1 - low-noise PiC models/main.tex}
            \input{2 - kinetic component/main.tex}
            \input{3 - fluid component/main.tex}
            \input{4 - numerical implementation/main.tex}
        \part{Project Overview}
            \input{5 - research plan/main.tex}
            \input{6 - summary/main.tex}
    
    
    %\section{}
    \newpage
    \pagenumbering{gobble}
        \printbibliography


    \newpage
    \pagenumbering{roman}
    \appendix
        \part{Appendices}
            \input{8 - Hilbert complexes/main.tex}
            \input{9 - weak conservation proofs/main.tex}
\end{document}

            \documentclass[12pt, a4paper]{report}

\input{template/main.tex}

\title{\BA{Title in Progress...}}
\author{Boris Andrews}
\affil{Mathematical Institute, University of Oxford}
\date{\today}


\begin{document}
    \pagenumbering{gobble}
    \maketitle
    
    
    \begin{abstract}
        Magnetic confinement reactors---in particular tokamaks---offer one of the most promising options for achieving practical nuclear fusion, with the potential to provide virtually limitless, clean energy. The theoretical and numerical modeling of tokamak plasmas is simultaneously an essential component of effective reactor design, and a great research barrier. Tokamak operational conditions exhibit comparatively low Knudsen numbers. Kinetic effects, including kinetic waves and instabilities, Landau damping, bump-on-tail instabilities and more, are therefore highly influential in tokamak plasma dynamics. Purely fluid models are inherently incapable of capturing these effects, whereas the high dimensionality in purely kinetic models render them practically intractable for most relevant purposes.

        We consider a $\delta\!f$ decomposition model, with a macroscopic fluid background and microscopic kinetic correction, both fully coupled to each other. A similar manner of discretization is proposed to that used in the recent \texttt{STRUPHY} code \cite{Holderied_Possanner_Wang_2021, Holderied_2022, Li_et_al_2023} with a finite-element model for the background and a pseudo-particle/PiC model for the correction.

        The fluid background satisfies the full, non-linear, resistive, compressible, Hall MHD equations. \cite{Laakmann_Hu_Farrell_2022} introduces finite-element(-in-space) implicit timesteppers for the incompressible analogue to this system with structure-preserving (SP) properties in the ideal case, alongside parameter-robust preconditioners. We show that these timesteppers can derive from a finite-element-in-time (FET) (and finite-element-in-space) interpretation. The benefits of this reformulation are discussed, including the derivation of timesteppers that are higher order in time, and the quantifiable dissipative SP properties in the non-ideal, resistive case.
        
        We discuss possible options for extending this FET approach to timesteppers for the compressible case.

        The kinetic corrections satisfy linearized Boltzmann equations. Using a Lénard--Bernstein collision operator, these take Fokker--Planck-like forms \cite{Fokker_1914, Planck_1917} wherein pseudo-particles in the numerical model obey the neoclassical transport equations, with particle-independent Brownian drift terms. This offers a rigorous methodology for incorporating collisions into the particle transport model, without coupling the equations of motions for each particle.
        
        Works by Chen, Chacón et al. \cite{Chen_Chacón_Barnes_2011, Chacón_Chen_Barnes_2013, Chen_Chacón_2014, Chen_Chacón_2015} have developed structure-preserving particle pushers for neoclassical transport in the Vlasov equations, derived from Crank--Nicolson integrators. We show these too can can derive from a FET interpretation, similarly offering potential extensions to higher-order-in-time particle pushers. The FET formulation is used also to consider how the stochastic drift terms can be incorporated into the pushers. Stochastic gyrokinetic expansions are also discussed.

        Different options for the numerical implementation of these schemes are considered.

        Due to the efficacy of FET in the development of SP timesteppers for both the fluid and kinetic component, we hope this approach will prove effective in the future for developing SP timesteppers for the full hybrid model. We hope this will give us the opportunity to incorporate previously inaccessible kinetic effects into the highly effective, modern, finite-element MHD models.
    \end{abstract}
    
    
    \newpage
    \tableofcontents
    
    
    \newpage
    \pagenumbering{arabic}
    %\linenumbers\renewcommand\thelinenumber{\color{black!50}\arabic{linenumber}}
            \input{0 - introduction/main.tex}
        \part{Research}
            \input{1 - low-noise PiC models/main.tex}
            \input{2 - kinetic component/main.tex}
            \input{3 - fluid component/main.tex}
            \input{4 - numerical implementation/main.tex}
        \part{Project Overview}
            \input{5 - research plan/main.tex}
            \input{6 - summary/main.tex}
    
    
    %\section{}
    \newpage
    \pagenumbering{gobble}
        \printbibliography


    \newpage
    \pagenumbering{roman}
    \appendix
        \part{Appendices}
            \input{8 - Hilbert complexes/main.tex}
            \input{9 - weak conservation proofs/main.tex}
\end{document}

            \documentclass[12pt, a4paper]{report}

\input{template/main.tex}

\title{\BA{Title in Progress...}}
\author{Boris Andrews}
\affil{Mathematical Institute, University of Oxford}
\date{\today}


\begin{document}
    \pagenumbering{gobble}
    \maketitle
    
    
    \begin{abstract}
        Magnetic confinement reactors---in particular tokamaks---offer one of the most promising options for achieving practical nuclear fusion, with the potential to provide virtually limitless, clean energy. The theoretical and numerical modeling of tokamak plasmas is simultaneously an essential component of effective reactor design, and a great research barrier. Tokamak operational conditions exhibit comparatively low Knudsen numbers. Kinetic effects, including kinetic waves and instabilities, Landau damping, bump-on-tail instabilities and more, are therefore highly influential in tokamak plasma dynamics. Purely fluid models are inherently incapable of capturing these effects, whereas the high dimensionality in purely kinetic models render them practically intractable for most relevant purposes.

        We consider a $\delta\!f$ decomposition model, with a macroscopic fluid background and microscopic kinetic correction, both fully coupled to each other. A similar manner of discretization is proposed to that used in the recent \texttt{STRUPHY} code \cite{Holderied_Possanner_Wang_2021, Holderied_2022, Li_et_al_2023} with a finite-element model for the background and a pseudo-particle/PiC model for the correction.

        The fluid background satisfies the full, non-linear, resistive, compressible, Hall MHD equations. \cite{Laakmann_Hu_Farrell_2022} introduces finite-element(-in-space) implicit timesteppers for the incompressible analogue to this system with structure-preserving (SP) properties in the ideal case, alongside parameter-robust preconditioners. We show that these timesteppers can derive from a finite-element-in-time (FET) (and finite-element-in-space) interpretation. The benefits of this reformulation are discussed, including the derivation of timesteppers that are higher order in time, and the quantifiable dissipative SP properties in the non-ideal, resistive case.
        
        We discuss possible options for extending this FET approach to timesteppers for the compressible case.

        The kinetic corrections satisfy linearized Boltzmann equations. Using a Lénard--Bernstein collision operator, these take Fokker--Planck-like forms \cite{Fokker_1914, Planck_1917} wherein pseudo-particles in the numerical model obey the neoclassical transport equations, with particle-independent Brownian drift terms. This offers a rigorous methodology for incorporating collisions into the particle transport model, without coupling the equations of motions for each particle.
        
        Works by Chen, Chacón et al. \cite{Chen_Chacón_Barnes_2011, Chacón_Chen_Barnes_2013, Chen_Chacón_2014, Chen_Chacón_2015} have developed structure-preserving particle pushers for neoclassical transport in the Vlasov equations, derived from Crank--Nicolson integrators. We show these too can can derive from a FET interpretation, similarly offering potential extensions to higher-order-in-time particle pushers. The FET formulation is used also to consider how the stochastic drift terms can be incorporated into the pushers. Stochastic gyrokinetic expansions are also discussed.

        Different options for the numerical implementation of these schemes are considered.

        Due to the efficacy of FET in the development of SP timesteppers for both the fluid and kinetic component, we hope this approach will prove effective in the future for developing SP timesteppers for the full hybrid model. We hope this will give us the opportunity to incorporate previously inaccessible kinetic effects into the highly effective, modern, finite-element MHD models.
    \end{abstract}
    
    
    \newpage
    \tableofcontents
    
    
    \newpage
    \pagenumbering{arabic}
    %\linenumbers\renewcommand\thelinenumber{\color{black!50}\arabic{linenumber}}
            \input{0 - introduction/main.tex}
        \part{Research}
            \input{1 - low-noise PiC models/main.tex}
            \input{2 - kinetic component/main.tex}
            \input{3 - fluid component/main.tex}
            \input{4 - numerical implementation/main.tex}
        \part{Project Overview}
            \input{5 - research plan/main.tex}
            \input{6 - summary/main.tex}
    
    
    %\section{}
    \newpage
    \pagenumbering{gobble}
        \printbibliography


    \newpage
    \pagenumbering{roman}
    \appendix
        \part{Appendices}
            \input{8 - Hilbert complexes/main.tex}
            \input{9 - weak conservation proofs/main.tex}
\end{document}

            \documentclass[12pt, a4paper]{report}

\input{template/main.tex}

\title{\BA{Title in Progress...}}
\author{Boris Andrews}
\affil{Mathematical Institute, University of Oxford}
\date{\today}


\begin{document}
    \pagenumbering{gobble}
    \maketitle
    
    
    \begin{abstract}
        Magnetic confinement reactors---in particular tokamaks---offer one of the most promising options for achieving practical nuclear fusion, with the potential to provide virtually limitless, clean energy. The theoretical and numerical modeling of tokamak plasmas is simultaneously an essential component of effective reactor design, and a great research barrier. Tokamak operational conditions exhibit comparatively low Knudsen numbers. Kinetic effects, including kinetic waves and instabilities, Landau damping, bump-on-tail instabilities and more, are therefore highly influential in tokamak plasma dynamics. Purely fluid models are inherently incapable of capturing these effects, whereas the high dimensionality in purely kinetic models render them practically intractable for most relevant purposes.

        We consider a $\delta\!f$ decomposition model, with a macroscopic fluid background and microscopic kinetic correction, both fully coupled to each other. A similar manner of discretization is proposed to that used in the recent \texttt{STRUPHY} code \cite{Holderied_Possanner_Wang_2021, Holderied_2022, Li_et_al_2023} with a finite-element model for the background and a pseudo-particle/PiC model for the correction.

        The fluid background satisfies the full, non-linear, resistive, compressible, Hall MHD equations. \cite{Laakmann_Hu_Farrell_2022} introduces finite-element(-in-space) implicit timesteppers for the incompressible analogue to this system with structure-preserving (SP) properties in the ideal case, alongside parameter-robust preconditioners. We show that these timesteppers can derive from a finite-element-in-time (FET) (and finite-element-in-space) interpretation. The benefits of this reformulation are discussed, including the derivation of timesteppers that are higher order in time, and the quantifiable dissipative SP properties in the non-ideal, resistive case.
        
        We discuss possible options for extending this FET approach to timesteppers for the compressible case.

        The kinetic corrections satisfy linearized Boltzmann equations. Using a Lénard--Bernstein collision operator, these take Fokker--Planck-like forms \cite{Fokker_1914, Planck_1917} wherein pseudo-particles in the numerical model obey the neoclassical transport equations, with particle-independent Brownian drift terms. This offers a rigorous methodology for incorporating collisions into the particle transport model, without coupling the equations of motions for each particle.
        
        Works by Chen, Chacón et al. \cite{Chen_Chacón_Barnes_2011, Chacón_Chen_Barnes_2013, Chen_Chacón_2014, Chen_Chacón_2015} have developed structure-preserving particle pushers for neoclassical transport in the Vlasov equations, derived from Crank--Nicolson integrators. We show these too can can derive from a FET interpretation, similarly offering potential extensions to higher-order-in-time particle pushers. The FET formulation is used also to consider how the stochastic drift terms can be incorporated into the pushers. Stochastic gyrokinetic expansions are also discussed.

        Different options for the numerical implementation of these schemes are considered.

        Due to the efficacy of FET in the development of SP timesteppers for both the fluid and kinetic component, we hope this approach will prove effective in the future for developing SP timesteppers for the full hybrid model. We hope this will give us the opportunity to incorporate previously inaccessible kinetic effects into the highly effective, modern, finite-element MHD models.
    \end{abstract}
    
    
    \newpage
    \tableofcontents
    
    
    \newpage
    \pagenumbering{arabic}
    %\linenumbers\renewcommand\thelinenumber{\color{black!50}\arabic{linenumber}}
            \input{0 - introduction/main.tex}
        \part{Research}
            \input{1 - low-noise PiC models/main.tex}
            \input{2 - kinetic component/main.tex}
            \input{3 - fluid component/main.tex}
            \input{4 - numerical implementation/main.tex}
        \part{Project Overview}
            \input{5 - research plan/main.tex}
            \input{6 - summary/main.tex}
    
    
    %\section{}
    \newpage
    \pagenumbering{gobble}
        \printbibliography


    \newpage
    \pagenumbering{roman}
    \appendix
        \part{Appendices}
            \input{8 - Hilbert complexes/main.tex}
            \input{9 - weak conservation proofs/main.tex}
\end{document}

        \part{Project Overview}
            \documentclass[12pt, a4paper]{report}

\input{template/main.tex}

\title{\BA{Title in Progress...}}
\author{Boris Andrews}
\affil{Mathematical Institute, University of Oxford}
\date{\today}


\begin{document}
    \pagenumbering{gobble}
    \maketitle
    
    
    \begin{abstract}
        Magnetic confinement reactors---in particular tokamaks---offer one of the most promising options for achieving practical nuclear fusion, with the potential to provide virtually limitless, clean energy. The theoretical and numerical modeling of tokamak plasmas is simultaneously an essential component of effective reactor design, and a great research barrier. Tokamak operational conditions exhibit comparatively low Knudsen numbers. Kinetic effects, including kinetic waves and instabilities, Landau damping, bump-on-tail instabilities and more, are therefore highly influential in tokamak plasma dynamics. Purely fluid models are inherently incapable of capturing these effects, whereas the high dimensionality in purely kinetic models render them practically intractable for most relevant purposes.

        We consider a $\delta\!f$ decomposition model, with a macroscopic fluid background and microscopic kinetic correction, both fully coupled to each other. A similar manner of discretization is proposed to that used in the recent \texttt{STRUPHY} code \cite{Holderied_Possanner_Wang_2021, Holderied_2022, Li_et_al_2023} with a finite-element model for the background and a pseudo-particle/PiC model for the correction.

        The fluid background satisfies the full, non-linear, resistive, compressible, Hall MHD equations. \cite{Laakmann_Hu_Farrell_2022} introduces finite-element(-in-space) implicit timesteppers for the incompressible analogue to this system with structure-preserving (SP) properties in the ideal case, alongside parameter-robust preconditioners. We show that these timesteppers can derive from a finite-element-in-time (FET) (and finite-element-in-space) interpretation. The benefits of this reformulation are discussed, including the derivation of timesteppers that are higher order in time, and the quantifiable dissipative SP properties in the non-ideal, resistive case.
        
        We discuss possible options for extending this FET approach to timesteppers for the compressible case.

        The kinetic corrections satisfy linearized Boltzmann equations. Using a Lénard--Bernstein collision operator, these take Fokker--Planck-like forms \cite{Fokker_1914, Planck_1917} wherein pseudo-particles in the numerical model obey the neoclassical transport equations, with particle-independent Brownian drift terms. This offers a rigorous methodology for incorporating collisions into the particle transport model, without coupling the equations of motions for each particle.
        
        Works by Chen, Chacón et al. \cite{Chen_Chacón_Barnes_2011, Chacón_Chen_Barnes_2013, Chen_Chacón_2014, Chen_Chacón_2015} have developed structure-preserving particle pushers for neoclassical transport in the Vlasov equations, derived from Crank--Nicolson integrators. We show these too can can derive from a FET interpretation, similarly offering potential extensions to higher-order-in-time particle pushers. The FET formulation is used also to consider how the stochastic drift terms can be incorporated into the pushers. Stochastic gyrokinetic expansions are also discussed.

        Different options for the numerical implementation of these schemes are considered.

        Due to the efficacy of FET in the development of SP timesteppers for both the fluid and kinetic component, we hope this approach will prove effective in the future for developing SP timesteppers for the full hybrid model. We hope this will give us the opportunity to incorporate previously inaccessible kinetic effects into the highly effective, modern, finite-element MHD models.
    \end{abstract}
    
    
    \newpage
    \tableofcontents
    
    
    \newpage
    \pagenumbering{arabic}
    %\linenumbers\renewcommand\thelinenumber{\color{black!50}\arabic{linenumber}}
            \input{0 - introduction/main.tex}
        \part{Research}
            \input{1 - low-noise PiC models/main.tex}
            \input{2 - kinetic component/main.tex}
            \input{3 - fluid component/main.tex}
            \input{4 - numerical implementation/main.tex}
        \part{Project Overview}
            \input{5 - research plan/main.tex}
            \input{6 - summary/main.tex}
    
    
    %\section{}
    \newpage
    \pagenumbering{gobble}
        \printbibliography


    \newpage
    \pagenumbering{roman}
    \appendix
        \part{Appendices}
            \input{8 - Hilbert complexes/main.tex}
            \input{9 - weak conservation proofs/main.tex}
\end{document}

            \documentclass[12pt, a4paper]{report}

\input{template/main.tex}

\title{\BA{Title in Progress...}}
\author{Boris Andrews}
\affil{Mathematical Institute, University of Oxford}
\date{\today}


\begin{document}
    \pagenumbering{gobble}
    \maketitle
    
    
    \begin{abstract}
        Magnetic confinement reactors---in particular tokamaks---offer one of the most promising options for achieving practical nuclear fusion, with the potential to provide virtually limitless, clean energy. The theoretical and numerical modeling of tokamak plasmas is simultaneously an essential component of effective reactor design, and a great research barrier. Tokamak operational conditions exhibit comparatively low Knudsen numbers. Kinetic effects, including kinetic waves and instabilities, Landau damping, bump-on-tail instabilities and more, are therefore highly influential in tokamak plasma dynamics. Purely fluid models are inherently incapable of capturing these effects, whereas the high dimensionality in purely kinetic models render them practically intractable for most relevant purposes.

        We consider a $\delta\!f$ decomposition model, with a macroscopic fluid background and microscopic kinetic correction, both fully coupled to each other. A similar manner of discretization is proposed to that used in the recent \texttt{STRUPHY} code \cite{Holderied_Possanner_Wang_2021, Holderied_2022, Li_et_al_2023} with a finite-element model for the background and a pseudo-particle/PiC model for the correction.

        The fluid background satisfies the full, non-linear, resistive, compressible, Hall MHD equations. \cite{Laakmann_Hu_Farrell_2022} introduces finite-element(-in-space) implicit timesteppers for the incompressible analogue to this system with structure-preserving (SP) properties in the ideal case, alongside parameter-robust preconditioners. We show that these timesteppers can derive from a finite-element-in-time (FET) (and finite-element-in-space) interpretation. The benefits of this reformulation are discussed, including the derivation of timesteppers that are higher order in time, and the quantifiable dissipative SP properties in the non-ideal, resistive case.
        
        We discuss possible options for extending this FET approach to timesteppers for the compressible case.

        The kinetic corrections satisfy linearized Boltzmann equations. Using a Lénard--Bernstein collision operator, these take Fokker--Planck-like forms \cite{Fokker_1914, Planck_1917} wherein pseudo-particles in the numerical model obey the neoclassical transport equations, with particle-independent Brownian drift terms. This offers a rigorous methodology for incorporating collisions into the particle transport model, without coupling the equations of motions for each particle.
        
        Works by Chen, Chacón et al. \cite{Chen_Chacón_Barnes_2011, Chacón_Chen_Barnes_2013, Chen_Chacón_2014, Chen_Chacón_2015} have developed structure-preserving particle pushers for neoclassical transport in the Vlasov equations, derived from Crank--Nicolson integrators. We show these too can can derive from a FET interpretation, similarly offering potential extensions to higher-order-in-time particle pushers. The FET formulation is used also to consider how the stochastic drift terms can be incorporated into the pushers. Stochastic gyrokinetic expansions are also discussed.

        Different options for the numerical implementation of these schemes are considered.

        Due to the efficacy of FET in the development of SP timesteppers for both the fluid and kinetic component, we hope this approach will prove effective in the future for developing SP timesteppers for the full hybrid model. We hope this will give us the opportunity to incorporate previously inaccessible kinetic effects into the highly effective, modern, finite-element MHD models.
    \end{abstract}
    
    
    \newpage
    \tableofcontents
    
    
    \newpage
    \pagenumbering{arabic}
    %\linenumbers\renewcommand\thelinenumber{\color{black!50}\arabic{linenumber}}
            \input{0 - introduction/main.tex}
        \part{Research}
            \input{1 - low-noise PiC models/main.tex}
            \input{2 - kinetic component/main.tex}
            \input{3 - fluid component/main.tex}
            \input{4 - numerical implementation/main.tex}
        \part{Project Overview}
            \input{5 - research plan/main.tex}
            \input{6 - summary/main.tex}
    
    
    %\section{}
    \newpage
    \pagenumbering{gobble}
        \printbibliography


    \newpage
    \pagenumbering{roman}
    \appendix
        \part{Appendices}
            \input{8 - Hilbert complexes/main.tex}
            \input{9 - weak conservation proofs/main.tex}
\end{document}

    
    
    %\section{}
    \newpage
    \pagenumbering{gobble}
        \printbibliography


    \newpage
    \pagenumbering{roman}
    \appendix
        \part{Appendices}
            \documentclass[12pt, a4paper]{report}

\input{template/main.tex}

\title{\BA{Title in Progress...}}
\author{Boris Andrews}
\affil{Mathematical Institute, University of Oxford}
\date{\today}


\begin{document}
    \pagenumbering{gobble}
    \maketitle
    
    
    \begin{abstract}
        Magnetic confinement reactors---in particular tokamaks---offer one of the most promising options for achieving practical nuclear fusion, with the potential to provide virtually limitless, clean energy. The theoretical and numerical modeling of tokamak plasmas is simultaneously an essential component of effective reactor design, and a great research barrier. Tokamak operational conditions exhibit comparatively low Knudsen numbers. Kinetic effects, including kinetic waves and instabilities, Landau damping, bump-on-tail instabilities and more, are therefore highly influential in tokamak plasma dynamics. Purely fluid models are inherently incapable of capturing these effects, whereas the high dimensionality in purely kinetic models render them practically intractable for most relevant purposes.

        We consider a $\delta\!f$ decomposition model, with a macroscopic fluid background and microscopic kinetic correction, both fully coupled to each other. A similar manner of discretization is proposed to that used in the recent \texttt{STRUPHY} code \cite{Holderied_Possanner_Wang_2021, Holderied_2022, Li_et_al_2023} with a finite-element model for the background and a pseudo-particle/PiC model for the correction.

        The fluid background satisfies the full, non-linear, resistive, compressible, Hall MHD equations. \cite{Laakmann_Hu_Farrell_2022} introduces finite-element(-in-space) implicit timesteppers for the incompressible analogue to this system with structure-preserving (SP) properties in the ideal case, alongside parameter-robust preconditioners. We show that these timesteppers can derive from a finite-element-in-time (FET) (and finite-element-in-space) interpretation. The benefits of this reformulation are discussed, including the derivation of timesteppers that are higher order in time, and the quantifiable dissipative SP properties in the non-ideal, resistive case.
        
        We discuss possible options for extending this FET approach to timesteppers for the compressible case.

        The kinetic corrections satisfy linearized Boltzmann equations. Using a Lénard--Bernstein collision operator, these take Fokker--Planck-like forms \cite{Fokker_1914, Planck_1917} wherein pseudo-particles in the numerical model obey the neoclassical transport equations, with particle-independent Brownian drift terms. This offers a rigorous methodology for incorporating collisions into the particle transport model, without coupling the equations of motions for each particle.
        
        Works by Chen, Chacón et al. \cite{Chen_Chacón_Barnes_2011, Chacón_Chen_Barnes_2013, Chen_Chacón_2014, Chen_Chacón_2015} have developed structure-preserving particle pushers for neoclassical transport in the Vlasov equations, derived from Crank--Nicolson integrators. We show these too can can derive from a FET interpretation, similarly offering potential extensions to higher-order-in-time particle pushers. The FET formulation is used also to consider how the stochastic drift terms can be incorporated into the pushers. Stochastic gyrokinetic expansions are also discussed.

        Different options for the numerical implementation of these schemes are considered.

        Due to the efficacy of FET in the development of SP timesteppers for both the fluid and kinetic component, we hope this approach will prove effective in the future for developing SP timesteppers for the full hybrid model. We hope this will give us the opportunity to incorporate previously inaccessible kinetic effects into the highly effective, modern, finite-element MHD models.
    \end{abstract}
    
    
    \newpage
    \tableofcontents
    
    
    \newpage
    \pagenumbering{arabic}
    %\linenumbers\renewcommand\thelinenumber{\color{black!50}\arabic{linenumber}}
            \input{0 - introduction/main.tex}
        \part{Research}
            \input{1 - low-noise PiC models/main.tex}
            \input{2 - kinetic component/main.tex}
            \input{3 - fluid component/main.tex}
            \input{4 - numerical implementation/main.tex}
        \part{Project Overview}
            \input{5 - research plan/main.tex}
            \input{6 - summary/main.tex}
    
    
    %\section{}
    \newpage
    \pagenumbering{gobble}
        \printbibliography


    \newpage
    \pagenumbering{roman}
    \appendix
        \part{Appendices}
            \input{8 - Hilbert complexes/main.tex}
            \input{9 - weak conservation proofs/main.tex}
\end{document}

            \documentclass[12pt, a4paper]{report}

\input{template/main.tex}

\title{\BA{Title in Progress...}}
\author{Boris Andrews}
\affil{Mathematical Institute, University of Oxford}
\date{\today}


\begin{document}
    \pagenumbering{gobble}
    \maketitle
    
    
    \begin{abstract}
        Magnetic confinement reactors---in particular tokamaks---offer one of the most promising options for achieving practical nuclear fusion, with the potential to provide virtually limitless, clean energy. The theoretical and numerical modeling of tokamak plasmas is simultaneously an essential component of effective reactor design, and a great research barrier. Tokamak operational conditions exhibit comparatively low Knudsen numbers. Kinetic effects, including kinetic waves and instabilities, Landau damping, bump-on-tail instabilities and more, are therefore highly influential in tokamak plasma dynamics. Purely fluid models are inherently incapable of capturing these effects, whereas the high dimensionality in purely kinetic models render them practically intractable for most relevant purposes.

        We consider a $\delta\!f$ decomposition model, with a macroscopic fluid background and microscopic kinetic correction, both fully coupled to each other. A similar manner of discretization is proposed to that used in the recent \texttt{STRUPHY} code \cite{Holderied_Possanner_Wang_2021, Holderied_2022, Li_et_al_2023} with a finite-element model for the background and a pseudo-particle/PiC model for the correction.

        The fluid background satisfies the full, non-linear, resistive, compressible, Hall MHD equations. \cite{Laakmann_Hu_Farrell_2022} introduces finite-element(-in-space) implicit timesteppers for the incompressible analogue to this system with structure-preserving (SP) properties in the ideal case, alongside parameter-robust preconditioners. We show that these timesteppers can derive from a finite-element-in-time (FET) (and finite-element-in-space) interpretation. The benefits of this reformulation are discussed, including the derivation of timesteppers that are higher order in time, and the quantifiable dissipative SP properties in the non-ideal, resistive case.
        
        We discuss possible options for extending this FET approach to timesteppers for the compressible case.

        The kinetic corrections satisfy linearized Boltzmann equations. Using a Lénard--Bernstein collision operator, these take Fokker--Planck-like forms \cite{Fokker_1914, Planck_1917} wherein pseudo-particles in the numerical model obey the neoclassical transport equations, with particle-independent Brownian drift terms. This offers a rigorous methodology for incorporating collisions into the particle transport model, without coupling the equations of motions for each particle.
        
        Works by Chen, Chacón et al. \cite{Chen_Chacón_Barnes_2011, Chacón_Chen_Barnes_2013, Chen_Chacón_2014, Chen_Chacón_2015} have developed structure-preserving particle pushers for neoclassical transport in the Vlasov equations, derived from Crank--Nicolson integrators. We show these too can can derive from a FET interpretation, similarly offering potential extensions to higher-order-in-time particle pushers. The FET formulation is used also to consider how the stochastic drift terms can be incorporated into the pushers. Stochastic gyrokinetic expansions are also discussed.

        Different options for the numerical implementation of these schemes are considered.

        Due to the efficacy of FET in the development of SP timesteppers for both the fluid and kinetic component, we hope this approach will prove effective in the future for developing SP timesteppers for the full hybrid model. We hope this will give us the opportunity to incorporate previously inaccessible kinetic effects into the highly effective, modern, finite-element MHD models.
    \end{abstract}
    
    
    \newpage
    \tableofcontents
    
    
    \newpage
    \pagenumbering{arabic}
    %\linenumbers\renewcommand\thelinenumber{\color{black!50}\arabic{linenumber}}
            \input{0 - introduction/main.tex}
        \part{Research}
            \input{1 - low-noise PiC models/main.tex}
            \input{2 - kinetic component/main.tex}
            \input{3 - fluid component/main.tex}
            \input{4 - numerical implementation/main.tex}
        \part{Project Overview}
            \input{5 - research plan/main.tex}
            \input{6 - summary/main.tex}
    
    
    %\section{}
    \newpage
    \pagenumbering{gobble}
        \printbibliography


    \newpage
    \pagenumbering{roman}
    \appendix
        \part{Appendices}
            \input{8 - Hilbert complexes/main.tex}
            \input{9 - weak conservation proofs/main.tex}
\end{document}

\end{document}


\title{\BA{Title in Progress...}}
\author{Boris Andrews}
\affil{Mathematical Institute, University of Oxford}
\date{\today}


\begin{document}
    \pagenumbering{gobble}
    \maketitle
    
    
    \begin{abstract}
        Magnetic confinement reactors---in particular tokamaks---offer one of the most promising options for achieving practical nuclear fusion, with the potential to provide virtually limitless, clean energy. The theoretical and numerical modeling of tokamak plasmas is simultaneously an essential component of effective reactor design, and a great research barrier. Tokamak operational conditions exhibit comparatively low Knudsen numbers. Kinetic effects, including kinetic waves and instabilities, Landau damping, bump-on-tail instabilities and more, are therefore highly influential in tokamak plasma dynamics. Purely fluid models are inherently incapable of capturing these effects, whereas the high dimensionality in purely kinetic models render them practically intractable for most relevant purposes.

        We consider a $\delta\!f$ decomposition model, with a macroscopic fluid background and microscopic kinetic correction, both fully coupled to each other. A similar manner of discretization is proposed to that used in the recent \texttt{STRUPHY} code \cite{Holderied_Possanner_Wang_2021, Holderied_2022, Li_et_al_2023} with a finite-element model for the background and a pseudo-particle/PiC model for the correction.

        The fluid background satisfies the full, non-linear, resistive, compressible, Hall MHD equations. \cite{Laakmann_Hu_Farrell_2022} introduces finite-element(-in-space) implicit timesteppers for the incompressible analogue to this system with structure-preserving (SP) properties in the ideal case, alongside parameter-robust preconditioners. We show that these timesteppers can derive from a finite-element-in-time (FET) (and finite-element-in-space) interpretation. The benefits of this reformulation are discussed, including the derivation of timesteppers that are higher order in time, and the quantifiable dissipative SP properties in the non-ideal, resistive case.
        
        We discuss possible options for extending this FET approach to timesteppers for the compressible case.

        The kinetic corrections satisfy linearized Boltzmann equations. Using a Lénard--Bernstein collision operator, these take Fokker--Planck-like forms \cite{Fokker_1914, Planck_1917} wherein pseudo-particles in the numerical model obey the neoclassical transport equations, with particle-independent Brownian drift terms. This offers a rigorous methodology for incorporating collisions into the particle transport model, without coupling the equations of motions for each particle.
        
        Works by Chen, Chacón et al. \cite{Chen_Chacón_Barnes_2011, Chacón_Chen_Barnes_2013, Chen_Chacón_2014, Chen_Chacón_2015} have developed structure-preserving particle pushers for neoclassical transport in the Vlasov equations, derived from Crank--Nicolson integrators. We show these too can can derive from a FET interpretation, similarly offering potential extensions to higher-order-in-time particle pushers. The FET formulation is used also to consider how the stochastic drift terms can be incorporated into the pushers. Stochastic gyrokinetic expansions are also discussed.

        Different options for the numerical implementation of these schemes are considered.

        Due to the efficacy of FET in the development of SP timesteppers for both the fluid and kinetic component, we hope this approach will prove effective in the future for developing SP timesteppers for the full hybrid model. We hope this will give us the opportunity to incorporate previously inaccessible kinetic effects into the highly effective, modern, finite-element MHD models.
    \end{abstract}
    
    
    \newpage
    \tableofcontents
    
    
    \newpage
    \pagenumbering{arabic}
    %\linenumbers\renewcommand\thelinenumber{\color{black!50}\arabic{linenumber}}
            \documentclass[12pt, a4paper]{report}

\documentclass[12pt, a4paper]{report}

\input{template/main.tex}

\title{\BA{Title in Progress...}}
\author{Boris Andrews}
\affil{Mathematical Institute, University of Oxford}
\date{\today}


\begin{document}
    \pagenumbering{gobble}
    \maketitle
    
    
    \begin{abstract}
        Magnetic confinement reactors---in particular tokamaks---offer one of the most promising options for achieving practical nuclear fusion, with the potential to provide virtually limitless, clean energy. The theoretical and numerical modeling of tokamak plasmas is simultaneously an essential component of effective reactor design, and a great research barrier. Tokamak operational conditions exhibit comparatively low Knudsen numbers. Kinetic effects, including kinetic waves and instabilities, Landau damping, bump-on-tail instabilities and more, are therefore highly influential in tokamak plasma dynamics. Purely fluid models are inherently incapable of capturing these effects, whereas the high dimensionality in purely kinetic models render them practically intractable for most relevant purposes.

        We consider a $\delta\!f$ decomposition model, with a macroscopic fluid background and microscopic kinetic correction, both fully coupled to each other. A similar manner of discretization is proposed to that used in the recent \texttt{STRUPHY} code \cite{Holderied_Possanner_Wang_2021, Holderied_2022, Li_et_al_2023} with a finite-element model for the background and a pseudo-particle/PiC model for the correction.

        The fluid background satisfies the full, non-linear, resistive, compressible, Hall MHD equations. \cite{Laakmann_Hu_Farrell_2022} introduces finite-element(-in-space) implicit timesteppers for the incompressible analogue to this system with structure-preserving (SP) properties in the ideal case, alongside parameter-robust preconditioners. We show that these timesteppers can derive from a finite-element-in-time (FET) (and finite-element-in-space) interpretation. The benefits of this reformulation are discussed, including the derivation of timesteppers that are higher order in time, and the quantifiable dissipative SP properties in the non-ideal, resistive case.
        
        We discuss possible options for extending this FET approach to timesteppers for the compressible case.

        The kinetic corrections satisfy linearized Boltzmann equations. Using a Lénard--Bernstein collision operator, these take Fokker--Planck-like forms \cite{Fokker_1914, Planck_1917} wherein pseudo-particles in the numerical model obey the neoclassical transport equations, with particle-independent Brownian drift terms. This offers a rigorous methodology for incorporating collisions into the particle transport model, without coupling the equations of motions for each particle.
        
        Works by Chen, Chacón et al. \cite{Chen_Chacón_Barnes_2011, Chacón_Chen_Barnes_2013, Chen_Chacón_2014, Chen_Chacón_2015} have developed structure-preserving particle pushers for neoclassical transport in the Vlasov equations, derived from Crank--Nicolson integrators. We show these too can can derive from a FET interpretation, similarly offering potential extensions to higher-order-in-time particle pushers. The FET formulation is used also to consider how the stochastic drift terms can be incorporated into the pushers. Stochastic gyrokinetic expansions are also discussed.

        Different options for the numerical implementation of these schemes are considered.

        Due to the efficacy of FET in the development of SP timesteppers for both the fluid and kinetic component, we hope this approach will prove effective in the future for developing SP timesteppers for the full hybrid model. We hope this will give us the opportunity to incorporate previously inaccessible kinetic effects into the highly effective, modern, finite-element MHD models.
    \end{abstract}
    
    
    \newpage
    \tableofcontents
    
    
    \newpage
    \pagenumbering{arabic}
    %\linenumbers\renewcommand\thelinenumber{\color{black!50}\arabic{linenumber}}
            \input{0 - introduction/main.tex}
        \part{Research}
            \input{1 - low-noise PiC models/main.tex}
            \input{2 - kinetic component/main.tex}
            \input{3 - fluid component/main.tex}
            \input{4 - numerical implementation/main.tex}
        \part{Project Overview}
            \input{5 - research plan/main.tex}
            \input{6 - summary/main.tex}
    
    
    %\section{}
    \newpage
    \pagenumbering{gobble}
        \printbibliography


    \newpage
    \pagenumbering{roman}
    \appendix
        \part{Appendices}
            \input{8 - Hilbert complexes/main.tex}
            \input{9 - weak conservation proofs/main.tex}
\end{document}


\title{\BA{Title in Progress...}}
\author{Boris Andrews}
\affil{Mathematical Institute, University of Oxford}
\date{\today}


\begin{document}
    \pagenumbering{gobble}
    \maketitle
    
    
    \begin{abstract}
        Magnetic confinement reactors---in particular tokamaks---offer one of the most promising options for achieving practical nuclear fusion, with the potential to provide virtually limitless, clean energy. The theoretical and numerical modeling of tokamak plasmas is simultaneously an essential component of effective reactor design, and a great research barrier. Tokamak operational conditions exhibit comparatively low Knudsen numbers. Kinetic effects, including kinetic waves and instabilities, Landau damping, bump-on-tail instabilities and more, are therefore highly influential in tokamak plasma dynamics. Purely fluid models are inherently incapable of capturing these effects, whereas the high dimensionality in purely kinetic models render them practically intractable for most relevant purposes.

        We consider a $\delta\!f$ decomposition model, with a macroscopic fluid background and microscopic kinetic correction, both fully coupled to each other. A similar manner of discretization is proposed to that used in the recent \texttt{STRUPHY} code \cite{Holderied_Possanner_Wang_2021, Holderied_2022, Li_et_al_2023} with a finite-element model for the background and a pseudo-particle/PiC model for the correction.

        The fluid background satisfies the full, non-linear, resistive, compressible, Hall MHD equations. \cite{Laakmann_Hu_Farrell_2022} introduces finite-element(-in-space) implicit timesteppers for the incompressible analogue to this system with structure-preserving (SP) properties in the ideal case, alongside parameter-robust preconditioners. We show that these timesteppers can derive from a finite-element-in-time (FET) (and finite-element-in-space) interpretation. The benefits of this reformulation are discussed, including the derivation of timesteppers that are higher order in time, and the quantifiable dissipative SP properties in the non-ideal, resistive case.
        
        We discuss possible options for extending this FET approach to timesteppers for the compressible case.

        The kinetic corrections satisfy linearized Boltzmann equations. Using a Lénard--Bernstein collision operator, these take Fokker--Planck-like forms \cite{Fokker_1914, Planck_1917} wherein pseudo-particles in the numerical model obey the neoclassical transport equations, with particle-independent Brownian drift terms. This offers a rigorous methodology for incorporating collisions into the particle transport model, without coupling the equations of motions for each particle.
        
        Works by Chen, Chacón et al. \cite{Chen_Chacón_Barnes_2011, Chacón_Chen_Barnes_2013, Chen_Chacón_2014, Chen_Chacón_2015} have developed structure-preserving particle pushers for neoclassical transport in the Vlasov equations, derived from Crank--Nicolson integrators. We show these too can can derive from a FET interpretation, similarly offering potential extensions to higher-order-in-time particle pushers. The FET formulation is used also to consider how the stochastic drift terms can be incorporated into the pushers. Stochastic gyrokinetic expansions are also discussed.

        Different options for the numerical implementation of these schemes are considered.

        Due to the efficacy of FET in the development of SP timesteppers for both the fluid and kinetic component, we hope this approach will prove effective in the future for developing SP timesteppers for the full hybrid model. We hope this will give us the opportunity to incorporate previously inaccessible kinetic effects into the highly effective, modern, finite-element MHD models.
    \end{abstract}
    
    
    \newpage
    \tableofcontents
    
    
    \newpage
    \pagenumbering{arabic}
    %\linenumbers\renewcommand\thelinenumber{\color{black!50}\arabic{linenumber}}
            \documentclass[12pt, a4paper]{report}

\input{template/main.tex}

\title{\BA{Title in Progress...}}
\author{Boris Andrews}
\affil{Mathematical Institute, University of Oxford}
\date{\today}


\begin{document}
    \pagenumbering{gobble}
    \maketitle
    
    
    \begin{abstract}
        Magnetic confinement reactors---in particular tokamaks---offer one of the most promising options for achieving practical nuclear fusion, with the potential to provide virtually limitless, clean energy. The theoretical and numerical modeling of tokamak plasmas is simultaneously an essential component of effective reactor design, and a great research barrier. Tokamak operational conditions exhibit comparatively low Knudsen numbers. Kinetic effects, including kinetic waves and instabilities, Landau damping, bump-on-tail instabilities and more, are therefore highly influential in tokamak plasma dynamics. Purely fluid models are inherently incapable of capturing these effects, whereas the high dimensionality in purely kinetic models render them practically intractable for most relevant purposes.

        We consider a $\delta\!f$ decomposition model, with a macroscopic fluid background and microscopic kinetic correction, both fully coupled to each other. A similar manner of discretization is proposed to that used in the recent \texttt{STRUPHY} code \cite{Holderied_Possanner_Wang_2021, Holderied_2022, Li_et_al_2023} with a finite-element model for the background and a pseudo-particle/PiC model for the correction.

        The fluid background satisfies the full, non-linear, resistive, compressible, Hall MHD equations. \cite{Laakmann_Hu_Farrell_2022} introduces finite-element(-in-space) implicit timesteppers for the incompressible analogue to this system with structure-preserving (SP) properties in the ideal case, alongside parameter-robust preconditioners. We show that these timesteppers can derive from a finite-element-in-time (FET) (and finite-element-in-space) interpretation. The benefits of this reformulation are discussed, including the derivation of timesteppers that are higher order in time, and the quantifiable dissipative SP properties in the non-ideal, resistive case.
        
        We discuss possible options for extending this FET approach to timesteppers for the compressible case.

        The kinetic corrections satisfy linearized Boltzmann equations. Using a Lénard--Bernstein collision operator, these take Fokker--Planck-like forms \cite{Fokker_1914, Planck_1917} wherein pseudo-particles in the numerical model obey the neoclassical transport equations, with particle-independent Brownian drift terms. This offers a rigorous methodology for incorporating collisions into the particle transport model, without coupling the equations of motions for each particle.
        
        Works by Chen, Chacón et al. \cite{Chen_Chacón_Barnes_2011, Chacón_Chen_Barnes_2013, Chen_Chacón_2014, Chen_Chacón_2015} have developed structure-preserving particle pushers for neoclassical transport in the Vlasov equations, derived from Crank--Nicolson integrators. We show these too can can derive from a FET interpretation, similarly offering potential extensions to higher-order-in-time particle pushers. The FET formulation is used also to consider how the stochastic drift terms can be incorporated into the pushers. Stochastic gyrokinetic expansions are also discussed.

        Different options for the numerical implementation of these schemes are considered.

        Due to the efficacy of FET in the development of SP timesteppers for both the fluid and kinetic component, we hope this approach will prove effective in the future for developing SP timesteppers for the full hybrid model. We hope this will give us the opportunity to incorporate previously inaccessible kinetic effects into the highly effective, modern, finite-element MHD models.
    \end{abstract}
    
    
    \newpage
    \tableofcontents
    
    
    \newpage
    \pagenumbering{arabic}
    %\linenumbers\renewcommand\thelinenumber{\color{black!50}\arabic{linenumber}}
            \input{0 - introduction/main.tex}
        \part{Research}
            \input{1 - low-noise PiC models/main.tex}
            \input{2 - kinetic component/main.tex}
            \input{3 - fluid component/main.tex}
            \input{4 - numerical implementation/main.tex}
        \part{Project Overview}
            \input{5 - research plan/main.tex}
            \input{6 - summary/main.tex}
    
    
    %\section{}
    \newpage
    \pagenumbering{gobble}
        \printbibliography


    \newpage
    \pagenumbering{roman}
    \appendix
        \part{Appendices}
            \input{8 - Hilbert complexes/main.tex}
            \input{9 - weak conservation proofs/main.tex}
\end{document}

        \part{Research}
            \documentclass[12pt, a4paper]{report}

\input{template/main.tex}

\title{\BA{Title in Progress...}}
\author{Boris Andrews}
\affil{Mathematical Institute, University of Oxford}
\date{\today}


\begin{document}
    \pagenumbering{gobble}
    \maketitle
    
    
    \begin{abstract}
        Magnetic confinement reactors---in particular tokamaks---offer one of the most promising options for achieving practical nuclear fusion, with the potential to provide virtually limitless, clean energy. The theoretical and numerical modeling of tokamak plasmas is simultaneously an essential component of effective reactor design, and a great research barrier. Tokamak operational conditions exhibit comparatively low Knudsen numbers. Kinetic effects, including kinetic waves and instabilities, Landau damping, bump-on-tail instabilities and more, are therefore highly influential in tokamak plasma dynamics. Purely fluid models are inherently incapable of capturing these effects, whereas the high dimensionality in purely kinetic models render them practically intractable for most relevant purposes.

        We consider a $\delta\!f$ decomposition model, with a macroscopic fluid background and microscopic kinetic correction, both fully coupled to each other. A similar manner of discretization is proposed to that used in the recent \texttt{STRUPHY} code \cite{Holderied_Possanner_Wang_2021, Holderied_2022, Li_et_al_2023} with a finite-element model for the background and a pseudo-particle/PiC model for the correction.

        The fluid background satisfies the full, non-linear, resistive, compressible, Hall MHD equations. \cite{Laakmann_Hu_Farrell_2022} introduces finite-element(-in-space) implicit timesteppers for the incompressible analogue to this system with structure-preserving (SP) properties in the ideal case, alongside parameter-robust preconditioners. We show that these timesteppers can derive from a finite-element-in-time (FET) (and finite-element-in-space) interpretation. The benefits of this reformulation are discussed, including the derivation of timesteppers that are higher order in time, and the quantifiable dissipative SP properties in the non-ideal, resistive case.
        
        We discuss possible options for extending this FET approach to timesteppers for the compressible case.

        The kinetic corrections satisfy linearized Boltzmann equations. Using a Lénard--Bernstein collision operator, these take Fokker--Planck-like forms \cite{Fokker_1914, Planck_1917} wherein pseudo-particles in the numerical model obey the neoclassical transport equations, with particle-independent Brownian drift terms. This offers a rigorous methodology for incorporating collisions into the particle transport model, without coupling the equations of motions for each particle.
        
        Works by Chen, Chacón et al. \cite{Chen_Chacón_Barnes_2011, Chacón_Chen_Barnes_2013, Chen_Chacón_2014, Chen_Chacón_2015} have developed structure-preserving particle pushers for neoclassical transport in the Vlasov equations, derived from Crank--Nicolson integrators. We show these too can can derive from a FET interpretation, similarly offering potential extensions to higher-order-in-time particle pushers. The FET formulation is used also to consider how the stochastic drift terms can be incorporated into the pushers. Stochastic gyrokinetic expansions are also discussed.

        Different options for the numerical implementation of these schemes are considered.

        Due to the efficacy of FET in the development of SP timesteppers for both the fluid and kinetic component, we hope this approach will prove effective in the future for developing SP timesteppers for the full hybrid model. We hope this will give us the opportunity to incorporate previously inaccessible kinetic effects into the highly effective, modern, finite-element MHD models.
    \end{abstract}
    
    
    \newpage
    \tableofcontents
    
    
    \newpage
    \pagenumbering{arabic}
    %\linenumbers\renewcommand\thelinenumber{\color{black!50}\arabic{linenumber}}
            \input{0 - introduction/main.tex}
        \part{Research}
            \input{1 - low-noise PiC models/main.tex}
            \input{2 - kinetic component/main.tex}
            \input{3 - fluid component/main.tex}
            \input{4 - numerical implementation/main.tex}
        \part{Project Overview}
            \input{5 - research plan/main.tex}
            \input{6 - summary/main.tex}
    
    
    %\section{}
    \newpage
    \pagenumbering{gobble}
        \printbibliography


    \newpage
    \pagenumbering{roman}
    \appendix
        \part{Appendices}
            \input{8 - Hilbert complexes/main.tex}
            \input{9 - weak conservation proofs/main.tex}
\end{document}

            \documentclass[12pt, a4paper]{report}

\input{template/main.tex}

\title{\BA{Title in Progress...}}
\author{Boris Andrews}
\affil{Mathematical Institute, University of Oxford}
\date{\today}


\begin{document}
    \pagenumbering{gobble}
    \maketitle
    
    
    \begin{abstract}
        Magnetic confinement reactors---in particular tokamaks---offer one of the most promising options for achieving practical nuclear fusion, with the potential to provide virtually limitless, clean energy. The theoretical and numerical modeling of tokamak plasmas is simultaneously an essential component of effective reactor design, and a great research barrier. Tokamak operational conditions exhibit comparatively low Knudsen numbers. Kinetic effects, including kinetic waves and instabilities, Landau damping, bump-on-tail instabilities and more, are therefore highly influential in tokamak plasma dynamics. Purely fluid models are inherently incapable of capturing these effects, whereas the high dimensionality in purely kinetic models render them practically intractable for most relevant purposes.

        We consider a $\delta\!f$ decomposition model, with a macroscopic fluid background and microscopic kinetic correction, both fully coupled to each other. A similar manner of discretization is proposed to that used in the recent \texttt{STRUPHY} code \cite{Holderied_Possanner_Wang_2021, Holderied_2022, Li_et_al_2023} with a finite-element model for the background and a pseudo-particle/PiC model for the correction.

        The fluid background satisfies the full, non-linear, resistive, compressible, Hall MHD equations. \cite{Laakmann_Hu_Farrell_2022} introduces finite-element(-in-space) implicit timesteppers for the incompressible analogue to this system with structure-preserving (SP) properties in the ideal case, alongside parameter-robust preconditioners. We show that these timesteppers can derive from a finite-element-in-time (FET) (and finite-element-in-space) interpretation. The benefits of this reformulation are discussed, including the derivation of timesteppers that are higher order in time, and the quantifiable dissipative SP properties in the non-ideal, resistive case.
        
        We discuss possible options for extending this FET approach to timesteppers for the compressible case.

        The kinetic corrections satisfy linearized Boltzmann equations. Using a Lénard--Bernstein collision operator, these take Fokker--Planck-like forms \cite{Fokker_1914, Planck_1917} wherein pseudo-particles in the numerical model obey the neoclassical transport equations, with particle-independent Brownian drift terms. This offers a rigorous methodology for incorporating collisions into the particle transport model, without coupling the equations of motions for each particle.
        
        Works by Chen, Chacón et al. \cite{Chen_Chacón_Barnes_2011, Chacón_Chen_Barnes_2013, Chen_Chacón_2014, Chen_Chacón_2015} have developed structure-preserving particle pushers for neoclassical transport in the Vlasov equations, derived from Crank--Nicolson integrators. We show these too can can derive from a FET interpretation, similarly offering potential extensions to higher-order-in-time particle pushers. The FET formulation is used also to consider how the stochastic drift terms can be incorporated into the pushers. Stochastic gyrokinetic expansions are also discussed.

        Different options for the numerical implementation of these schemes are considered.

        Due to the efficacy of FET in the development of SP timesteppers for both the fluid and kinetic component, we hope this approach will prove effective in the future for developing SP timesteppers for the full hybrid model. We hope this will give us the opportunity to incorporate previously inaccessible kinetic effects into the highly effective, modern, finite-element MHD models.
    \end{abstract}
    
    
    \newpage
    \tableofcontents
    
    
    \newpage
    \pagenumbering{arabic}
    %\linenumbers\renewcommand\thelinenumber{\color{black!50}\arabic{linenumber}}
            \input{0 - introduction/main.tex}
        \part{Research}
            \input{1 - low-noise PiC models/main.tex}
            \input{2 - kinetic component/main.tex}
            \input{3 - fluid component/main.tex}
            \input{4 - numerical implementation/main.tex}
        \part{Project Overview}
            \input{5 - research plan/main.tex}
            \input{6 - summary/main.tex}
    
    
    %\section{}
    \newpage
    \pagenumbering{gobble}
        \printbibliography


    \newpage
    \pagenumbering{roman}
    \appendix
        \part{Appendices}
            \input{8 - Hilbert complexes/main.tex}
            \input{9 - weak conservation proofs/main.tex}
\end{document}

            \documentclass[12pt, a4paper]{report}

\input{template/main.tex}

\title{\BA{Title in Progress...}}
\author{Boris Andrews}
\affil{Mathematical Institute, University of Oxford}
\date{\today}


\begin{document}
    \pagenumbering{gobble}
    \maketitle
    
    
    \begin{abstract}
        Magnetic confinement reactors---in particular tokamaks---offer one of the most promising options for achieving practical nuclear fusion, with the potential to provide virtually limitless, clean energy. The theoretical and numerical modeling of tokamak plasmas is simultaneously an essential component of effective reactor design, and a great research barrier. Tokamak operational conditions exhibit comparatively low Knudsen numbers. Kinetic effects, including kinetic waves and instabilities, Landau damping, bump-on-tail instabilities and more, are therefore highly influential in tokamak plasma dynamics. Purely fluid models are inherently incapable of capturing these effects, whereas the high dimensionality in purely kinetic models render them practically intractable for most relevant purposes.

        We consider a $\delta\!f$ decomposition model, with a macroscopic fluid background and microscopic kinetic correction, both fully coupled to each other. A similar manner of discretization is proposed to that used in the recent \texttt{STRUPHY} code \cite{Holderied_Possanner_Wang_2021, Holderied_2022, Li_et_al_2023} with a finite-element model for the background and a pseudo-particle/PiC model for the correction.

        The fluid background satisfies the full, non-linear, resistive, compressible, Hall MHD equations. \cite{Laakmann_Hu_Farrell_2022} introduces finite-element(-in-space) implicit timesteppers for the incompressible analogue to this system with structure-preserving (SP) properties in the ideal case, alongside parameter-robust preconditioners. We show that these timesteppers can derive from a finite-element-in-time (FET) (and finite-element-in-space) interpretation. The benefits of this reformulation are discussed, including the derivation of timesteppers that are higher order in time, and the quantifiable dissipative SP properties in the non-ideal, resistive case.
        
        We discuss possible options for extending this FET approach to timesteppers for the compressible case.

        The kinetic corrections satisfy linearized Boltzmann equations. Using a Lénard--Bernstein collision operator, these take Fokker--Planck-like forms \cite{Fokker_1914, Planck_1917} wherein pseudo-particles in the numerical model obey the neoclassical transport equations, with particle-independent Brownian drift terms. This offers a rigorous methodology for incorporating collisions into the particle transport model, without coupling the equations of motions for each particle.
        
        Works by Chen, Chacón et al. \cite{Chen_Chacón_Barnes_2011, Chacón_Chen_Barnes_2013, Chen_Chacón_2014, Chen_Chacón_2015} have developed structure-preserving particle pushers for neoclassical transport in the Vlasov equations, derived from Crank--Nicolson integrators. We show these too can can derive from a FET interpretation, similarly offering potential extensions to higher-order-in-time particle pushers. The FET formulation is used also to consider how the stochastic drift terms can be incorporated into the pushers. Stochastic gyrokinetic expansions are also discussed.

        Different options for the numerical implementation of these schemes are considered.

        Due to the efficacy of FET in the development of SP timesteppers for both the fluid and kinetic component, we hope this approach will prove effective in the future for developing SP timesteppers for the full hybrid model. We hope this will give us the opportunity to incorporate previously inaccessible kinetic effects into the highly effective, modern, finite-element MHD models.
    \end{abstract}
    
    
    \newpage
    \tableofcontents
    
    
    \newpage
    \pagenumbering{arabic}
    %\linenumbers\renewcommand\thelinenumber{\color{black!50}\arabic{linenumber}}
            \input{0 - introduction/main.tex}
        \part{Research}
            \input{1 - low-noise PiC models/main.tex}
            \input{2 - kinetic component/main.tex}
            \input{3 - fluid component/main.tex}
            \input{4 - numerical implementation/main.tex}
        \part{Project Overview}
            \input{5 - research plan/main.tex}
            \input{6 - summary/main.tex}
    
    
    %\section{}
    \newpage
    \pagenumbering{gobble}
        \printbibliography


    \newpage
    \pagenumbering{roman}
    \appendix
        \part{Appendices}
            \input{8 - Hilbert complexes/main.tex}
            \input{9 - weak conservation proofs/main.tex}
\end{document}

            \documentclass[12pt, a4paper]{report}

\input{template/main.tex}

\title{\BA{Title in Progress...}}
\author{Boris Andrews}
\affil{Mathematical Institute, University of Oxford}
\date{\today}


\begin{document}
    \pagenumbering{gobble}
    \maketitle
    
    
    \begin{abstract}
        Magnetic confinement reactors---in particular tokamaks---offer one of the most promising options for achieving practical nuclear fusion, with the potential to provide virtually limitless, clean energy. The theoretical and numerical modeling of tokamak plasmas is simultaneously an essential component of effective reactor design, and a great research barrier. Tokamak operational conditions exhibit comparatively low Knudsen numbers. Kinetic effects, including kinetic waves and instabilities, Landau damping, bump-on-tail instabilities and more, are therefore highly influential in tokamak plasma dynamics. Purely fluid models are inherently incapable of capturing these effects, whereas the high dimensionality in purely kinetic models render them practically intractable for most relevant purposes.

        We consider a $\delta\!f$ decomposition model, with a macroscopic fluid background and microscopic kinetic correction, both fully coupled to each other. A similar manner of discretization is proposed to that used in the recent \texttt{STRUPHY} code \cite{Holderied_Possanner_Wang_2021, Holderied_2022, Li_et_al_2023} with a finite-element model for the background and a pseudo-particle/PiC model for the correction.

        The fluid background satisfies the full, non-linear, resistive, compressible, Hall MHD equations. \cite{Laakmann_Hu_Farrell_2022} introduces finite-element(-in-space) implicit timesteppers for the incompressible analogue to this system with structure-preserving (SP) properties in the ideal case, alongside parameter-robust preconditioners. We show that these timesteppers can derive from a finite-element-in-time (FET) (and finite-element-in-space) interpretation. The benefits of this reformulation are discussed, including the derivation of timesteppers that are higher order in time, and the quantifiable dissipative SP properties in the non-ideal, resistive case.
        
        We discuss possible options for extending this FET approach to timesteppers for the compressible case.

        The kinetic corrections satisfy linearized Boltzmann equations. Using a Lénard--Bernstein collision operator, these take Fokker--Planck-like forms \cite{Fokker_1914, Planck_1917} wherein pseudo-particles in the numerical model obey the neoclassical transport equations, with particle-independent Brownian drift terms. This offers a rigorous methodology for incorporating collisions into the particle transport model, without coupling the equations of motions for each particle.
        
        Works by Chen, Chacón et al. \cite{Chen_Chacón_Barnes_2011, Chacón_Chen_Barnes_2013, Chen_Chacón_2014, Chen_Chacón_2015} have developed structure-preserving particle pushers for neoclassical transport in the Vlasov equations, derived from Crank--Nicolson integrators. We show these too can can derive from a FET interpretation, similarly offering potential extensions to higher-order-in-time particle pushers. The FET formulation is used also to consider how the stochastic drift terms can be incorporated into the pushers. Stochastic gyrokinetic expansions are also discussed.

        Different options for the numerical implementation of these schemes are considered.

        Due to the efficacy of FET in the development of SP timesteppers for both the fluid and kinetic component, we hope this approach will prove effective in the future for developing SP timesteppers for the full hybrid model. We hope this will give us the opportunity to incorporate previously inaccessible kinetic effects into the highly effective, modern, finite-element MHD models.
    \end{abstract}
    
    
    \newpage
    \tableofcontents
    
    
    \newpage
    \pagenumbering{arabic}
    %\linenumbers\renewcommand\thelinenumber{\color{black!50}\arabic{linenumber}}
            \input{0 - introduction/main.tex}
        \part{Research}
            \input{1 - low-noise PiC models/main.tex}
            \input{2 - kinetic component/main.tex}
            \input{3 - fluid component/main.tex}
            \input{4 - numerical implementation/main.tex}
        \part{Project Overview}
            \input{5 - research plan/main.tex}
            \input{6 - summary/main.tex}
    
    
    %\section{}
    \newpage
    \pagenumbering{gobble}
        \printbibliography


    \newpage
    \pagenumbering{roman}
    \appendix
        \part{Appendices}
            \input{8 - Hilbert complexes/main.tex}
            \input{9 - weak conservation proofs/main.tex}
\end{document}

        \part{Project Overview}
            \documentclass[12pt, a4paper]{report}

\input{template/main.tex}

\title{\BA{Title in Progress...}}
\author{Boris Andrews}
\affil{Mathematical Institute, University of Oxford}
\date{\today}


\begin{document}
    \pagenumbering{gobble}
    \maketitle
    
    
    \begin{abstract}
        Magnetic confinement reactors---in particular tokamaks---offer one of the most promising options for achieving practical nuclear fusion, with the potential to provide virtually limitless, clean energy. The theoretical and numerical modeling of tokamak plasmas is simultaneously an essential component of effective reactor design, and a great research barrier. Tokamak operational conditions exhibit comparatively low Knudsen numbers. Kinetic effects, including kinetic waves and instabilities, Landau damping, bump-on-tail instabilities and more, are therefore highly influential in tokamak plasma dynamics. Purely fluid models are inherently incapable of capturing these effects, whereas the high dimensionality in purely kinetic models render them practically intractable for most relevant purposes.

        We consider a $\delta\!f$ decomposition model, with a macroscopic fluid background and microscopic kinetic correction, both fully coupled to each other. A similar manner of discretization is proposed to that used in the recent \texttt{STRUPHY} code \cite{Holderied_Possanner_Wang_2021, Holderied_2022, Li_et_al_2023} with a finite-element model for the background and a pseudo-particle/PiC model for the correction.

        The fluid background satisfies the full, non-linear, resistive, compressible, Hall MHD equations. \cite{Laakmann_Hu_Farrell_2022} introduces finite-element(-in-space) implicit timesteppers for the incompressible analogue to this system with structure-preserving (SP) properties in the ideal case, alongside parameter-robust preconditioners. We show that these timesteppers can derive from a finite-element-in-time (FET) (and finite-element-in-space) interpretation. The benefits of this reformulation are discussed, including the derivation of timesteppers that are higher order in time, and the quantifiable dissipative SP properties in the non-ideal, resistive case.
        
        We discuss possible options for extending this FET approach to timesteppers for the compressible case.

        The kinetic corrections satisfy linearized Boltzmann equations. Using a Lénard--Bernstein collision operator, these take Fokker--Planck-like forms \cite{Fokker_1914, Planck_1917} wherein pseudo-particles in the numerical model obey the neoclassical transport equations, with particle-independent Brownian drift terms. This offers a rigorous methodology for incorporating collisions into the particle transport model, without coupling the equations of motions for each particle.
        
        Works by Chen, Chacón et al. \cite{Chen_Chacón_Barnes_2011, Chacón_Chen_Barnes_2013, Chen_Chacón_2014, Chen_Chacón_2015} have developed structure-preserving particle pushers for neoclassical transport in the Vlasov equations, derived from Crank--Nicolson integrators. We show these too can can derive from a FET interpretation, similarly offering potential extensions to higher-order-in-time particle pushers. The FET formulation is used also to consider how the stochastic drift terms can be incorporated into the pushers. Stochastic gyrokinetic expansions are also discussed.

        Different options for the numerical implementation of these schemes are considered.

        Due to the efficacy of FET in the development of SP timesteppers for both the fluid and kinetic component, we hope this approach will prove effective in the future for developing SP timesteppers for the full hybrid model. We hope this will give us the opportunity to incorporate previously inaccessible kinetic effects into the highly effective, modern, finite-element MHD models.
    \end{abstract}
    
    
    \newpage
    \tableofcontents
    
    
    \newpage
    \pagenumbering{arabic}
    %\linenumbers\renewcommand\thelinenumber{\color{black!50}\arabic{linenumber}}
            \input{0 - introduction/main.tex}
        \part{Research}
            \input{1 - low-noise PiC models/main.tex}
            \input{2 - kinetic component/main.tex}
            \input{3 - fluid component/main.tex}
            \input{4 - numerical implementation/main.tex}
        \part{Project Overview}
            \input{5 - research plan/main.tex}
            \input{6 - summary/main.tex}
    
    
    %\section{}
    \newpage
    \pagenumbering{gobble}
        \printbibliography


    \newpage
    \pagenumbering{roman}
    \appendix
        \part{Appendices}
            \input{8 - Hilbert complexes/main.tex}
            \input{9 - weak conservation proofs/main.tex}
\end{document}

            \documentclass[12pt, a4paper]{report}

\input{template/main.tex}

\title{\BA{Title in Progress...}}
\author{Boris Andrews}
\affil{Mathematical Institute, University of Oxford}
\date{\today}


\begin{document}
    \pagenumbering{gobble}
    \maketitle
    
    
    \begin{abstract}
        Magnetic confinement reactors---in particular tokamaks---offer one of the most promising options for achieving practical nuclear fusion, with the potential to provide virtually limitless, clean energy. The theoretical and numerical modeling of tokamak plasmas is simultaneously an essential component of effective reactor design, and a great research barrier. Tokamak operational conditions exhibit comparatively low Knudsen numbers. Kinetic effects, including kinetic waves and instabilities, Landau damping, bump-on-tail instabilities and more, are therefore highly influential in tokamak plasma dynamics. Purely fluid models are inherently incapable of capturing these effects, whereas the high dimensionality in purely kinetic models render them practically intractable for most relevant purposes.

        We consider a $\delta\!f$ decomposition model, with a macroscopic fluid background and microscopic kinetic correction, both fully coupled to each other. A similar manner of discretization is proposed to that used in the recent \texttt{STRUPHY} code \cite{Holderied_Possanner_Wang_2021, Holderied_2022, Li_et_al_2023} with a finite-element model for the background and a pseudo-particle/PiC model for the correction.

        The fluid background satisfies the full, non-linear, resistive, compressible, Hall MHD equations. \cite{Laakmann_Hu_Farrell_2022} introduces finite-element(-in-space) implicit timesteppers for the incompressible analogue to this system with structure-preserving (SP) properties in the ideal case, alongside parameter-robust preconditioners. We show that these timesteppers can derive from a finite-element-in-time (FET) (and finite-element-in-space) interpretation. The benefits of this reformulation are discussed, including the derivation of timesteppers that are higher order in time, and the quantifiable dissipative SP properties in the non-ideal, resistive case.
        
        We discuss possible options for extending this FET approach to timesteppers for the compressible case.

        The kinetic corrections satisfy linearized Boltzmann equations. Using a Lénard--Bernstein collision operator, these take Fokker--Planck-like forms \cite{Fokker_1914, Planck_1917} wherein pseudo-particles in the numerical model obey the neoclassical transport equations, with particle-independent Brownian drift terms. This offers a rigorous methodology for incorporating collisions into the particle transport model, without coupling the equations of motions for each particle.
        
        Works by Chen, Chacón et al. \cite{Chen_Chacón_Barnes_2011, Chacón_Chen_Barnes_2013, Chen_Chacón_2014, Chen_Chacón_2015} have developed structure-preserving particle pushers for neoclassical transport in the Vlasov equations, derived from Crank--Nicolson integrators. We show these too can can derive from a FET interpretation, similarly offering potential extensions to higher-order-in-time particle pushers. The FET formulation is used also to consider how the stochastic drift terms can be incorporated into the pushers. Stochastic gyrokinetic expansions are also discussed.

        Different options for the numerical implementation of these schemes are considered.

        Due to the efficacy of FET in the development of SP timesteppers for both the fluid and kinetic component, we hope this approach will prove effective in the future for developing SP timesteppers for the full hybrid model. We hope this will give us the opportunity to incorporate previously inaccessible kinetic effects into the highly effective, modern, finite-element MHD models.
    \end{abstract}
    
    
    \newpage
    \tableofcontents
    
    
    \newpage
    \pagenumbering{arabic}
    %\linenumbers\renewcommand\thelinenumber{\color{black!50}\arabic{linenumber}}
            \input{0 - introduction/main.tex}
        \part{Research}
            \input{1 - low-noise PiC models/main.tex}
            \input{2 - kinetic component/main.tex}
            \input{3 - fluid component/main.tex}
            \input{4 - numerical implementation/main.tex}
        \part{Project Overview}
            \input{5 - research plan/main.tex}
            \input{6 - summary/main.tex}
    
    
    %\section{}
    \newpage
    \pagenumbering{gobble}
        \printbibliography


    \newpage
    \pagenumbering{roman}
    \appendix
        \part{Appendices}
            \input{8 - Hilbert complexes/main.tex}
            \input{9 - weak conservation proofs/main.tex}
\end{document}

    
    
    %\section{}
    \newpage
    \pagenumbering{gobble}
        \printbibliography


    \newpage
    \pagenumbering{roman}
    \appendix
        \part{Appendices}
            \documentclass[12pt, a4paper]{report}

\input{template/main.tex}

\title{\BA{Title in Progress...}}
\author{Boris Andrews}
\affil{Mathematical Institute, University of Oxford}
\date{\today}


\begin{document}
    \pagenumbering{gobble}
    \maketitle
    
    
    \begin{abstract}
        Magnetic confinement reactors---in particular tokamaks---offer one of the most promising options for achieving practical nuclear fusion, with the potential to provide virtually limitless, clean energy. The theoretical and numerical modeling of tokamak plasmas is simultaneously an essential component of effective reactor design, and a great research barrier. Tokamak operational conditions exhibit comparatively low Knudsen numbers. Kinetic effects, including kinetic waves and instabilities, Landau damping, bump-on-tail instabilities and more, are therefore highly influential in tokamak plasma dynamics. Purely fluid models are inherently incapable of capturing these effects, whereas the high dimensionality in purely kinetic models render them practically intractable for most relevant purposes.

        We consider a $\delta\!f$ decomposition model, with a macroscopic fluid background and microscopic kinetic correction, both fully coupled to each other. A similar manner of discretization is proposed to that used in the recent \texttt{STRUPHY} code \cite{Holderied_Possanner_Wang_2021, Holderied_2022, Li_et_al_2023} with a finite-element model for the background and a pseudo-particle/PiC model for the correction.

        The fluid background satisfies the full, non-linear, resistive, compressible, Hall MHD equations. \cite{Laakmann_Hu_Farrell_2022} introduces finite-element(-in-space) implicit timesteppers for the incompressible analogue to this system with structure-preserving (SP) properties in the ideal case, alongside parameter-robust preconditioners. We show that these timesteppers can derive from a finite-element-in-time (FET) (and finite-element-in-space) interpretation. The benefits of this reformulation are discussed, including the derivation of timesteppers that are higher order in time, and the quantifiable dissipative SP properties in the non-ideal, resistive case.
        
        We discuss possible options for extending this FET approach to timesteppers for the compressible case.

        The kinetic corrections satisfy linearized Boltzmann equations. Using a Lénard--Bernstein collision operator, these take Fokker--Planck-like forms \cite{Fokker_1914, Planck_1917} wherein pseudo-particles in the numerical model obey the neoclassical transport equations, with particle-independent Brownian drift terms. This offers a rigorous methodology for incorporating collisions into the particle transport model, without coupling the equations of motions for each particle.
        
        Works by Chen, Chacón et al. \cite{Chen_Chacón_Barnes_2011, Chacón_Chen_Barnes_2013, Chen_Chacón_2014, Chen_Chacón_2015} have developed structure-preserving particle pushers for neoclassical transport in the Vlasov equations, derived from Crank--Nicolson integrators. We show these too can can derive from a FET interpretation, similarly offering potential extensions to higher-order-in-time particle pushers. The FET formulation is used also to consider how the stochastic drift terms can be incorporated into the pushers. Stochastic gyrokinetic expansions are also discussed.

        Different options for the numerical implementation of these schemes are considered.

        Due to the efficacy of FET in the development of SP timesteppers for both the fluid and kinetic component, we hope this approach will prove effective in the future for developing SP timesteppers for the full hybrid model. We hope this will give us the opportunity to incorporate previously inaccessible kinetic effects into the highly effective, modern, finite-element MHD models.
    \end{abstract}
    
    
    \newpage
    \tableofcontents
    
    
    \newpage
    \pagenumbering{arabic}
    %\linenumbers\renewcommand\thelinenumber{\color{black!50}\arabic{linenumber}}
            \input{0 - introduction/main.tex}
        \part{Research}
            \input{1 - low-noise PiC models/main.tex}
            \input{2 - kinetic component/main.tex}
            \input{3 - fluid component/main.tex}
            \input{4 - numerical implementation/main.tex}
        \part{Project Overview}
            \input{5 - research plan/main.tex}
            \input{6 - summary/main.tex}
    
    
    %\section{}
    \newpage
    \pagenumbering{gobble}
        \printbibliography


    \newpage
    \pagenumbering{roman}
    \appendix
        \part{Appendices}
            \input{8 - Hilbert complexes/main.tex}
            \input{9 - weak conservation proofs/main.tex}
\end{document}

            \documentclass[12pt, a4paper]{report}

\input{template/main.tex}

\title{\BA{Title in Progress...}}
\author{Boris Andrews}
\affil{Mathematical Institute, University of Oxford}
\date{\today}


\begin{document}
    \pagenumbering{gobble}
    \maketitle
    
    
    \begin{abstract}
        Magnetic confinement reactors---in particular tokamaks---offer one of the most promising options for achieving practical nuclear fusion, with the potential to provide virtually limitless, clean energy. The theoretical and numerical modeling of tokamak plasmas is simultaneously an essential component of effective reactor design, and a great research barrier. Tokamak operational conditions exhibit comparatively low Knudsen numbers. Kinetic effects, including kinetic waves and instabilities, Landau damping, bump-on-tail instabilities and more, are therefore highly influential in tokamak plasma dynamics. Purely fluid models are inherently incapable of capturing these effects, whereas the high dimensionality in purely kinetic models render them practically intractable for most relevant purposes.

        We consider a $\delta\!f$ decomposition model, with a macroscopic fluid background and microscopic kinetic correction, both fully coupled to each other. A similar manner of discretization is proposed to that used in the recent \texttt{STRUPHY} code \cite{Holderied_Possanner_Wang_2021, Holderied_2022, Li_et_al_2023} with a finite-element model for the background and a pseudo-particle/PiC model for the correction.

        The fluid background satisfies the full, non-linear, resistive, compressible, Hall MHD equations. \cite{Laakmann_Hu_Farrell_2022} introduces finite-element(-in-space) implicit timesteppers for the incompressible analogue to this system with structure-preserving (SP) properties in the ideal case, alongside parameter-robust preconditioners. We show that these timesteppers can derive from a finite-element-in-time (FET) (and finite-element-in-space) interpretation. The benefits of this reformulation are discussed, including the derivation of timesteppers that are higher order in time, and the quantifiable dissipative SP properties in the non-ideal, resistive case.
        
        We discuss possible options for extending this FET approach to timesteppers for the compressible case.

        The kinetic corrections satisfy linearized Boltzmann equations. Using a Lénard--Bernstein collision operator, these take Fokker--Planck-like forms \cite{Fokker_1914, Planck_1917} wherein pseudo-particles in the numerical model obey the neoclassical transport equations, with particle-independent Brownian drift terms. This offers a rigorous methodology for incorporating collisions into the particle transport model, without coupling the equations of motions for each particle.
        
        Works by Chen, Chacón et al. \cite{Chen_Chacón_Barnes_2011, Chacón_Chen_Barnes_2013, Chen_Chacón_2014, Chen_Chacón_2015} have developed structure-preserving particle pushers for neoclassical transport in the Vlasov equations, derived from Crank--Nicolson integrators. We show these too can can derive from a FET interpretation, similarly offering potential extensions to higher-order-in-time particle pushers. The FET formulation is used also to consider how the stochastic drift terms can be incorporated into the pushers. Stochastic gyrokinetic expansions are also discussed.

        Different options for the numerical implementation of these schemes are considered.

        Due to the efficacy of FET in the development of SP timesteppers for both the fluid and kinetic component, we hope this approach will prove effective in the future for developing SP timesteppers for the full hybrid model. We hope this will give us the opportunity to incorporate previously inaccessible kinetic effects into the highly effective, modern, finite-element MHD models.
    \end{abstract}
    
    
    \newpage
    \tableofcontents
    
    
    \newpage
    \pagenumbering{arabic}
    %\linenumbers\renewcommand\thelinenumber{\color{black!50}\arabic{linenumber}}
            \input{0 - introduction/main.tex}
        \part{Research}
            \input{1 - low-noise PiC models/main.tex}
            \input{2 - kinetic component/main.tex}
            \input{3 - fluid component/main.tex}
            \input{4 - numerical implementation/main.tex}
        \part{Project Overview}
            \input{5 - research plan/main.tex}
            \input{6 - summary/main.tex}
    
    
    %\section{}
    \newpage
    \pagenumbering{gobble}
        \printbibliography


    \newpage
    \pagenumbering{roman}
    \appendix
        \part{Appendices}
            \input{8 - Hilbert complexes/main.tex}
            \input{9 - weak conservation proofs/main.tex}
\end{document}

\end{document}

        \part{Research}
            \documentclass[12pt, a4paper]{report}

\documentclass[12pt, a4paper]{report}

\input{template/main.tex}

\title{\BA{Title in Progress...}}
\author{Boris Andrews}
\affil{Mathematical Institute, University of Oxford}
\date{\today}


\begin{document}
    \pagenumbering{gobble}
    \maketitle
    
    
    \begin{abstract}
        Magnetic confinement reactors---in particular tokamaks---offer one of the most promising options for achieving practical nuclear fusion, with the potential to provide virtually limitless, clean energy. The theoretical and numerical modeling of tokamak plasmas is simultaneously an essential component of effective reactor design, and a great research barrier. Tokamak operational conditions exhibit comparatively low Knudsen numbers. Kinetic effects, including kinetic waves and instabilities, Landau damping, bump-on-tail instabilities and more, are therefore highly influential in tokamak plasma dynamics. Purely fluid models are inherently incapable of capturing these effects, whereas the high dimensionality in purely kinetic models render them practically intractable for most relevant purposes.

        We consider a $\delta\!f$ decomposition model, with a macroscopic fluid background and microscopic kinetic correction, both fully coupled to each other. A similar manner of discretization is proposed to that used in the recent \texttt{STRUPHY} code \cite{Holderied_Possanner_Wang_2021, Holderied_2022, Li_et_al_2023} with a finite-element model for the background and a pseudo-particle/PiC model for the correction.

        The fluid background satisfies the full, non-linear, resistive, compressible, Hall MHD equations. \cite{Laakmann_Hu_Farrell_2022} introduces finite-element(-in-space) implicit timesteppers for the incompressible analogue to this system with structure-preserving (SP) properties in the ideal case, alongside parameter-robust preconditioners. We show that these timesteppers can derive from a finite-element-in-time (FET) (and finite-element-in-space) interpretation. The benefits of this reformulation are discussed, including the derivation of timesteppers that are higher order in time, and the quantifiable dissipative SP properties in the non-ideal, resistive case.
        
        We discuss possible options for extending this FET approach to timesteppers for the compressible case.

        The kinetic corrections satisfy linearized Boltzmann equations. Using a Lénard--Bernstein collision operator, these take Fokker--Planck-like forms \cite{Fokker_1914, Planck_1917} wherein pseudo-particles in the numerical model obey the neoclassical transport equations, with particle-independent Brownian drift terms. This offers a rigorous methodology for incorporating collisions into the particle transport model, without coupling the equations of motions for each particle.
        
        Works by Chen, Chacón et al. \cite{Chen_Chacón_Barnes_2011, Chacón_Chen_Barnes_2013, Chen_Chacón_2014, Chen_Chacón_2015} have developed structure-preserving particle pushers for neoclassical transport in the Vlasov equations, derived from Crank--Nicolson integrators. We show these too can can derive from a FET interpretation, similarly offering potential extensions to higher-order-in-time particle pushers. The FET formulation is used also to consider how the stochastic drift terms can be incorporated into the pushers. Stochastic gyrokinetic expansions are also discussed.

        Different options for the numerical implementation of these schemes are considered.

        Due to the efficacy of FET in the development of SP timesteppers for both the fluid and kinetic component, we hope this approach will prove effective in the future for developing SP timesteppers for the full hybrid model. We hope this will give us the opportunity to incorporate previously inaccessible kinetic effects into the highly effective, modern, finite-element MHD models.
    \end{abstract}
    
    
    \newpage
    \tableofcontents
    
    
    \newpage
    \pagenumbering{arabic}
    %\linenumbers\renewcommand\thelinenumber{\color{black!50}\arabic{linenumber}}
            \input{0 - introduction/main.tex}
        \part{Research}
            \input{1 - low-noise PiC models/main.tex}
            \input{2 - kinetic component/main.tex}
            \input{3 - fluid component/main.tex}
            \input{4 - numerical implementation/main.tex}
        \part{Project Overview}
            \input{5 - research plan/main.tex}
            \input{6 - summary/main.tex}
    
    
    %\section{}
    \newpage
    \pagenumbering{gobble}
        \printbibliography


    \newpage
    \pagenumbering{roman}
    \appendix
        \part{Appendices}
            \input{8 - Hilbert complexes/main.tex}
            \input{9 - weak conservation proofs/main.tex}
\end{document}


\title{\BA{Title in Progress...}}
\author{Boris Andrews}
\affil{Mathematical Institute, University of Oxford}
\date{\today}


\begin{document}
    \pagenumbering{gobble}
    \maketitle
    
    
    \begin{abstract}
        Magnetic confinement reactors---in particular tokamaks---offer one of the most promising options for achieving practical nuclear fusion, with the potential to provide virtually limitless, clean energy. The theoretical and numerical modeling of tokamak plasmas is simultaneously an essential component of effective reactor design, and a great research barrier. Tokamak operational conditions exhibit comparatively low Knudsen numbers. Kinetic effects, including kinetic waves and instabilities, Landau damping, bump-on-tail instabilities and more, are therefore highly influential in tokamak plasma dynamics. Purely fluid models are inherently incapable of capturing these effects, whereas the high dimensionality in purely kinetic models render them practically intractable for most relevant purposes.

        We consider a $\delta\!f$ decomposition model, with a macroscopic fluid background and microscopic kinetic correction, both fully coupled to each other. A similar manner of discretization is proposed to that used in the recent \texttt{STRUPHY} code \cite{Holderied_Possanner_Wang_2021, Holderied_2022, Li_et_al_2023} with a finite-element model for the background and a pseudo-particle/PiC model for the correction.

        The fluid background satisfies the full, non-linear, resistive, compressible, Hall MHD equations. \cite{Laakmann_Hu_Farrell_2022} introduces finite-element(-in-space) implicit timesteppers for the incompressible analogue to this system with structure-preserving (SP) properties in the ideal case, alongside parameter-robust preconditioners. We show that these timesteppers can derive from a finite-element-in-time (FET) (and finite-element-in-space) interpretation. The benefits of this reformulation are discussed, including the derivation of timesteppers that are higher order in time, and the quantifiable dissipative SP properties in the non-ideal, resistive case.
        
        We discuss possible options for extending this FET approach to timesteppers for the compressible case.

        The kinetic corrections satisfy linearized Boltzmann equations. Using a Lénard--Bernstein collision operator, these take Fokker--Planck-like forms \cite{Fokker_1914, Planck_1917} wherein pseudo-particles in the numerical model obey the neoclassical transport equations, with particle-independent Brownian drift terms. This offers a rigorous methodology for incorporating collisions into the particle transport model, without coupling the equations of motions for each particle.
        
        Works by Chen, Chacón et al. \cite{Chen_Chacón_Barnes_2011, Chacón_Chen_Barnes_2013, Chen_Chacón_2014, Chen_Chacón_2015} have developed structure-preserving particle pushers for neoclassical transport in the Vlasov equations, derived from Crank--Nicolson integrators. We show these too can can derive from a FET interpretation, similarly offering potential extensions to higher-order-in-time particle pushers. The FET formulation is used also to consider how the stochastic drift terms can be incorporated into the pushers. Stochastic gyrokinetic expansions are also discussed.

        Different options for the numerical implementation of these schemes are considered.

        Due to the efficacy of FET in the development of SP timesteppers for both the fluid and kinetic component, we hope this approach will prove effective in the future for developing SP timesteppers for the full hybrid model. We hope this will give us the opportunity to incorporate previously inaccessible kinetic effects into the highly effective, modern, finite-element MHD models.
    \end{abstract}
    
    
    \newpage
    \tableofcontents
    
    
    \newpage
    \pagenumbering{arabic}
    %\linenumbers\renewcommand\thelinenumber{\color{black!50}\arabic{linenumber}}
            \documentclass[12pt, a4paper]{report}

\input{template/main.tex}

\title{\BA{Title in Progress...}}
\author{Boris Andrews}
\affil{Mathematical Institute, University of Oxford}
\date{\today}


\begin{document}
    \pagenumbering{gobble}
    \maketitle
    
    
    \begin{abstract}
        Magnetic confinement reactors---in particular tokamaks---offer one of the most promising options for achieving practical nuclear fusion, with the potential to provide virtually limitless, clean energy. The theoretical and numerical modeling of tokamak plasmas is simultaneously an essential component of effective reactor design, and a great research barrier. Tokamak operational conditions exhibit comparatively low Knudsen numbers. Kinetic effects, including kinetic waves and instabilities, Landau damping, bump-on-tail instabilities and more, are therefore highly influential in tokamak plasma dynamics. Purely fluid models are inherently incapable of capturing these effects, whereas the high dimensionality in purely kinetic models render them practically intractable for most relevant purposes.

        We consider a $\delta\!f$ decomposition model, with a macroscopic fluid background and microscopic kinetic correction, both fully coupled to each other. A similar manner of discretization is proposed to that used in the recent \texttt{STRUPHY} code \cite{Holderied_Possanner_Wang_2021, Holderied_2022, Li_et_al_2023} with a finite-element model for the background and a pseudo-particle/PiC model for the correction.

        The fluid background satisfies the full, non-linear, resistive, compressible, Hall MHD equations. \cite{Laakmann_Hu_Farrell_2022} introduces finite-element(-in-space) implicit timesteppers for the incompressible analogue to this system with structure-preserving (SP) properties in the ideal case, alongside parameter-robust preconditioners. We show that these timesteppers can derive from a finite-element-in-time (FET) (and finite-element-in-space) interpretation. The benefits of this reformulation are discussed, including the derivation of timesteppers that are higher order in time, and the quantifiable dissipative SP properties in the non-ideal, resistive case.
        
        We discuss possible options for extending this FET approach to timesteppers for the compressible case.

        The kinetic corrections satisfy linearized Boltzmann equations. Using a Lénard--Bernstein collision operator, these take Fokker--Planck-like forms \cite{Fokker_1914, Planck_1917} wherein pseudo-particles in the numerical model obey the neoclassical transport equations, with particle-independent Brownian drift terms. This offers a rigorous methodology for incorporating collisions into the particle transport model, without coupling the equations of motions for each particle.
        
        Works by Chen, Chacón et al. \cite{Chen_Chacón_Barnes_2011, Chacón_Chen_Barnes_2013, Chen_Chacón_2014, Chen_Chacón_2015} have developed structure-preserving particle pushers for neoclassical transport in the Vlasov equations, derived from Crank--Nicolson integrators. We show these too can can derive from a FET interpretation, similarly offering potential extensions to higher-order-in-time particle pushers. The FET formulation is used also to consider how the stochastic drift terms can be incorporated into the pushers. Stochastic gyrokinetic expansions are also discussed.

        Different options for the numerical implementation of these schemes are considered.

        Due to the efficacy of FET in the development of SP timesteppers for both the fluid and kinetic component, we hope this approach will prove effective in the future for developing SP timesteppers for the full hybrid model. We hope this will give us the opportunity to incorporate previously inaccessible kinetic effects into the highly effective, modern, finite-element MHD models.
    \end{abstract}
    
    
    \newpage
    \tableofcontents
    
    
    \newpage
    \pagenumbering{arabic}
    %\linenumbers\renewcommand\thelinenumber{\color{black!50}\arabic{linenumber}}
            \input{0 - introduction/main.tex}
        \part{Research}
            \input{1 - low-noise PiC models/main.tex}
            \input{2 - kinetic component/main.tex}
            \input{3 - fluid component/main.tex}
            \input{4 - numerical implementation/main.tex}
        \part{Project Overview}
            \input{5 - research plan/main.tex}
            \input{6 - summary/main.tex}
    
    
    %\section{}
    \newpage
    \pagenumbering{gobble}
        \printbibliography


    \newpage
    \pagenumbering{roman}
    \appendix
        \part{Appendices}
            \input{8 - Hilbert complexes/main.tex}
            \input{9 - weak conservation proofs/main.tex}
\end{document}

        \part{Research}
            \documentclass[12pt, a4paper]{report}

\input{template/main.tex}

\title{\BA{Title in Progress...}}
\author{Boris Andrews}
\affil{Mathematical Institute, University of Oxford}
\date{\today}


\begin{document}
    \pagenumbering{gobble}
    \maketitle
    
    
    \begin{abstract}
        Magnetic confinement reactors---in particular tokamaks---offer one of the most promising options for achieving practical nuclear fusion, with the potential to provide virtually limitless, clean energy. The theoretical and numerical modeling of tokamak plasmas is simultaneously an essential component of effective reactor design, and a great research barrier. Tokamak operational conditions exhibit comparatively low Knudsen numbers. Kinetic effects, including kinetic waves and instabilities, Landau damping, bump-on-tail instabilities and more, are therefore highly influential in tokamak plasma dynamics. Purely fluid models are inherently incapable of capturing these effects, whereas the high dimensionality in purely kinetic models render them practically intractable for most relevant purposes.

        We consider a $\delta\!f$ decomposition model, with a macroscopic fluid background and microscopic kinetic correction, both fully coupled to each other. A similar manner of discretization is proposed to that used in the recent \texttt{STRUPHY} code \cite{Holderied_Possanner_Wang_2021, Holderied_2022, Li_et_al_2023} with a finite-element model for the background and a pseudo-particle/PiC model for the correction.

        The fluid background satisfies the full, non-linear, resistive, compressible, Hall MHD equations. \cite{Laakmann_Hu_Farrell_2022} introduces finite-element(-in-space) implicit timesteppers for the incompressible analogue to this system with structure-preserving (SP) properties in the ideal case, alongside parameter-robust preconditioners. We show that these timesteppers can derive from a finite-element-in-time (FET) (and finite-element-in-space) interpretation. The benefits of this reformulation are discussed, including the derivation of timesteppers that are higher order in time, and the quantifiable dissipative SP properties in the non-ideal, resistive case.
        
        We discuss possible options for extending this FET approach to timesteppers for the compressible case.

        The kinetic corrections satisfy linearized Boltzmann equations. Using a Lénard--Bernstein collision operator, these take Fokker--Planck-like forms \cite{Fokker_1914, Planck_1917} wherein pseudo-particles in the numerical model obey the neoclassical transport equations, with particle-independent Brownian drift terms. This offers a rigorous methodology for incorporating collisions into the particle transport model, without coupling the equations of motions for each particle.
        
        Works by Chen, Chacón et al. \cite{Chen_Chacón_Barnes_2011, Chacón_Chen_Barnes_2013, Chen_Chacón_2014, Chen_Chacón_2015} have developed structure-preserving particle pushers for neoclassical transport in the Vlasov equations, derived from Crank--Nicolson integrators. We show these too can can derive from a FET interpretation, similarly offering potential extensions to higher-order-in-time particle pushers. The FET formulation is used also to consider how the stochastic drift terms can be incorporated into the pushers. Stochastic gyrokinetic expansions are also discussed.

        Different options for the numerical implementation of these schemes are considered.

        Due to the efficacy of FET in the development of SP timesteppers for both the fluid and kinetic component, we hope this approach will prove effective in the future for developing SP timesteppers for the full hybrid model. We hope this will give us the opportunity to incorporate previously inaccessible kinetic effects into the highly effective, modern, finite-element MHD models.
    \end{abstract}
    
    
    \newpage
    \tableofcontents
    
    
    \newpage
    \pagenumbering{arabic}
    %\linenumbers\renewcommand\thelinenumber{\color{black!50}\arabic{linenumber}}
            \input{0 - introduction/main.tex}
        \part{Research}
            \input{1 - low-noise PiC models/main.tex}
            \input{2 - kinetic component/main.tex}
            \input{3 - fluid component/main.tex}
            \input{4 - numerical implementation/main.tex}
        \part{Project Overview}
            \input{5 - research plan/main.tex}
            \input{6 - summary/main.tex}
    
    
    %\section{}
    \newpage
    \pagenumbering{gobble}
        \printbibliography


    \newpage
    \pagenumbering{roman}
    \appendix
        \part{Appendices}
            \input{8 - Hilbert complexes/main.tex}
            \input{9 - weak conservation proofs/main.tex}
\end{document}

            \documentclass[12pt, a4paper]{report}

\input{template/main.tex}

\title{\BA{Title in Progress...}}
\author{Boris Andrews}
\affil{Mathematical Institute, University of Oxford}
\date{\today}


\begin{document}
    \pagenumbering{gobble}
    \maketitle
    
    
    \begin{abstract}
        Magnetic confinement reactors---in particular tokamaks---offer one of the most promising options for achieving practical nuclear fusion, with the potential to provide virtually limitless, clean energy. The theoretical and numerical modeling of tokamak plasmas is simultaneously an essential component of effective reactor design, and a great research barrier. Tokamak operational conditions exhibit comparatively low Knudsen numbers. Kinetic effects, including kinetic waves and instabilities, Landau damping, bump-on-tail instabilities and more, are therefore highly influential in tokamak plasma dynamics. Purely fluid models are inherently incapable of capturing these effects, whereas the high dimensionality in purely kinetic models render them practically intractable for most relevant purposes.

        We consider a $\delta\!f$ decomposition model, with a macroscopic fluid background and microscopic kinetic correction, both fully coupled to each other. A similar manner of discretization is proposed to that used in the recent \texttt{STRUPHY} code \cite{Holderied_Possanner_Wang_2021, Holderied_2022, Li_et_al_2023} with a finite-element model for the background and a pseudo-particle/PiC model for the correction.

        The fluid background satisfies the full, non-linear, resistive, compressible, Hall MHD equations. \cite{Laakmann_Hu_Farrell_2022} introduces finite-element(-in-space) implicit timesteppers for the incompressible analogue to this system with structure-preserving (SP) properties in the ideal case, alongside parameter-robust preconditioners. We show that these timesteppers can derive from a finite-element-in-time (FET) (and finite-element-in-space) interpretation. The benefits of this reformulation are discussed, including the derivation of timesteppers that are higher order in time, and the quantifiable dissipative SP properties in the non-ideal, resistive case.
        
        We discuss possible options for extending this FET approach to timesteppers for the compressible case.

        The kinetic corrections satisfy linearized Boltzmann equations. Using a Lénard--Bernstein collision operator, these take Fokker--Planck-like forms \cite{Fokker_1914, Planck_1917} wherein pseudo-particles in the numerical model obey the neoclassical transport equations, with particle-independent Brownian drift terms. This offers a rigorous methodology for incorporating collisions into the particle transport model, without coupling the equations of motions for each particle.
        
        Works by Chen, Chacón et al. \cite{Chen_Chacón_Barnes_2011, Chacón_Chen_Barnes_2013, Chen_Chacón_2014, Chen_Chacón_2015} have developed structure-preserving particle pushers for neoclassical transport in the Vlasov equations, derived from Crank--Nicolson integrators. We show these too can can derive from a FET interpretation, similarly offering potential extensions to higher-order-in-time particle pushers. The FET formulation is used also to consider how the stochastic drift terms can be incorporated into the pushers. Stochastic gyrokinetic expansions are also discussed.

        Different options for the numerical implementation of these schemes are considered.

        Due to the efficacy of FET in the development of SP timesteppers for both the fluid and kinetic component, we hope this approach will prove effective in the future for developing SP timesteppers for the full hybrid model. We hope this will give us the opportunity to incorporate previously inaccessible kinetic effects into the highly effective, modern, finite-element MHD models.
    \end{abstract}
    
    
    \newpage
    \tableofcontents
    
    
    \newpage
    \pagenumbering{arabic}
    %\linenumbers\renewcommand\thelinenumber{\color{black!50}\arabic{linenumber}}
            \input{0 - introduction/main.tex}
        \part{Research}
            \input{1 - low-noise PiC models/main.tex}
            \input{2 - kinetic component/main.tex}
            \input{3 - fluid component/main.tex}
            \input{4 - numerical implementation/main.tex}
        \part{Project Overview}
            \input{5 - research plan/main.tex}
            \input{6 - summary/main.tex}
    
    
    %\section{}
    \newpage
    \pagenumbering{gobble}
        \printbibliography


    \newpage
    \pagenumbering{roman}
    \appendix
        \part{Appendices}
            \input{8 - Hilbert complexes/main.tex}
            \input{9 - weak conservation proofs/main.tex}
\end{document}

            \documentclass[12pt, a4paper]{report}

\input{template/main.tex}

\title{\BA{Title in Progress...}}
\author{Boris Andrews}
\affil{Mathematical Institute, University of Oxford}
\date{\today}


\begin{document}
    \pagenumbering{gobble}
    \maketitle
    
    
    \begin{abstract}
        Magnetic confinement reactors---in particular tokamaks---offer one of the most promising options for achieving practical nuclear fusion, with the potential to provide virtually limitless, clean energy. The theoretical and numerical modeling of tokamak plasmas is simultaneously an essential component of effective reactor design, and a great research barrier. Tokamak operational conditions exhibit comparatively low Knudsen numbers. Kinetic effects, including kinetic waves and instabilities, Landau damping, bump-on-tail instabilities and more, are therefore highly influential in tokamak plasma dynamics. Purely fluid models are inherently incapable of capturing these effects, whereas the high dimensionality in purely kinetic models render them practically intractable for most relevant purposes.

        We consider a $\delta\!f$ decomposition model, with a macroscopic fluid background and microscopic kinetic correction, both fully coupled to each other. A similar manner of discretization is proposed to that used in the recent \texttt{STRUPHY} code \cite{Holderied_Possanner_Wang_2021, Holderied_2022, Li_et_al_2023} with a finite-element model for the background and a pseudo-particle/PiC model for the correction.

        The fluid background satisfies the full, non-linear, resistive, compressible, Hall MHD equations. \cite{Laakmann_Hu_Farrell_2022} introduces finite-element(-in-space) implicit timesteppers for the incompressible analogue to this system with structure-preserving (SP) properties in the ideal case, alongside parameter-robust preconditioners. We show that these timesteppers can derive from a finite-element-in-time (FET) (and finite-element-in-space) interpretation. The benefits of this reformulation are discussed, including the derivation of timesteppers that are higher order in time, and the quantifiable dissipative SP properties in the non-ideal, resistive case.
        
        We discuss possible options for extending this FET approach to timesteppers for the compressible case.

        The kinetic corrections satisfy linearized Boltzmann equations. Using a Lénard--Bernstein collision operator, these take Fokker--Planck-like forms \cite{Fokker_1914, Planck_1917} wherein pseudo-particles in the numerical model obey the neoclassical transport equations, with particle-independent Brownian drift terms. This offers a rigorous methodology for incorporating collisions into the particle transport model, without coupling the equations of motions for each particle.
        
        Works by Chen, Chacón et al. \cite{Chen_Chacón_Barnes_2011, Chacón_Chen_Barnes_2013, Chen_Chacón_2014, Chen_Chacón_2015} have developed structure-preserving particle pushers for neoclassical transport in the Vlasov equations, derived from Crank--Nicolson integrators. We show these too can can derive from a FET interpretation, similarly offering potential extensions to higher-order-in-time particle pushers. The FET formulation is used also to consider how the stochastic drift terms can be incorporated into the pushers. Stochastic gyrokinetic expansions are also discussed.

        Different options for the numerical implementation of these schemes are considered.

        Due to the efficacy of FET in the development of SP timesteppers for both the fluid and kinetic component, we hope this approach will prove effective in the future for developing SP timesteppers for the full hybrid model. We hope this will give us the opportunity to incorporate previously inaccessible kinetic effects into the highly effective, modern, finite-element MHD models.
    \end{abstract}
    
    
    \newpage
    \tableofcontents
    
    
    \newpage
    \pagenumbering{arabic}
    %\linenumbers\renewcommand\thelinenumber{\color{black!50}\arabic{linenumber}}
            \input{0 - introduction/main.tex}
        \part{Research}
            \input{1 - low-noise PiC models/main.tex}
            \input{2 - kinetic component/main.tex}
            \input{3 - fluid component/main.tex}
            \input{4 - numerical implementation/main.tex}
        \part{Project Overview}
            \input{5 - research plan/main.tex}
            \input{6 - summary/main.tex}
    
    
    %\section{}
    \newpage
    \pagenumbering{gobble}
        \printbibliography


    \newpage
    \pagenumbering{roman}
    \appendix
        \part{Appendices}
            \input{8 - Hilbert complexes/main.tex}
            \input{9 - weak conservation proofs/main.tex}
\end{document}

            \documentclass[12pt, a4paper]{report}

\input{template/main.tex}

\title{\BA{Title in Progress...}}
\author{Boris Andrews}
\affil{Mathematical Institute, University of Oxford}
\date{\today}


\begin{document}
    \pagenumbering{gobble}
    \maketitle
    
    
    \begin{abstract}
        Magnetic confinement reactors---in particular tokamaks---offer one of the most promising options for achieving practical nuclear fusion, with the potential to provide virtually limitless, clean energy. The theoretical and numerical modeling of tokamak plasmas is simultaneously an essential component of effective reactor design, and a great research barrier. Tokamak operational conditions exhibit comparatively low Knudsen numbers. Kinetic effects, including kinetic waves and instabilities, Landau damping, bump-on-tail instabilities and more, are therefore highly influential in tokamak plasma dynamics. Purely fluid models are inherently incapable of capturing these effects, whereas the high dimensionality in purely kinetic models render them practically intractable for most relevant purposes.

        We consider a $\delta\!f$ decomposition model, with a macroscopic fluid background and microscopic kinetic correction, both fully coupled to each other. A similar manner of discretization is proposed to that used in the recent \texttt{STRUPHY} code \cite{Holderied_Possanner_Wang_2021, Holderied_2022, Li_et_al_2023} with a finite-element model for the background and a pseudo-particle/PiC model for the correction.

        The fluid background satisfies the full, non-linear, resistive, compressible, Hall MHD equations. \cite{Laakmann_Hu_Farrell_2022} introduces finite-element(-in-space) implicit timesteppers for the incompressible analogue to this system with structure-preserving (SP) properties in the ideal case, alongside parameter-robust preconditioners. We show that these timesteppers can derive from a finite-element-in-time (FET) (and finite-element-in-space) interpretation. The benefits of this reformulation are discussed, including the derivation of timesteppers that are higher order in time, and the quantifiable dissipative SP properties in the non-ideal, resistive case.
        
        We discuss possible options for extending this FET approach to timesteppers for the compressible case.

        The kinetic corrections satisfy linearized Boltzmann equations. Using a Lénard--Bernstein collision operator, these take Fokker--Planck-like forms \cite{Fokker_1914, Planck_1917} wherein pseudo-particles in the numerical model obey the neoclassical transport equations, with particle-independent Brownian drift terms. This offers a rigorous methodology for incorporating collisions into the particle transport model, without coupling the equations of motions for each particle.
        
        Works by Chen, Chacón et al. \cite{Chen_Chacón_Barnes_2011, Chacón_Chen_Barnes_2013, Chen_Chacón_2014, Chen_Chacón_2015} have developed structure-preserving particle pushers for neoclassical transport in the Vlasov equations, derived from Crank--Nicolson integrators. We show these too can can derive from a FET interpretation, similarly offering potential extensions to higher-order-in-time particle pushers. The FET formulation is used also to consider how the stochastic drift terms can be incorporated into the pushers. Stochastic gyrokinetic expansions are also discussed.

        Different options for the numerical implementation of these schemes are considered.

        Due to the efficacy of FET in the development of SP timesteppers for both the fluid and kinetic component, we hope this approach will prove effective in the future for developing SP timesteppers for the full hybrid model. We hope this will give us the opportunity to incorporate previously inaccessible kinetic effects into the highly effective, modern, finite-element MHD models.
    \end{abstract}
    
    
    \newpage
    \tableofcontents
    
    
    \newpage
    \pagenumbering{arabic}
    %\linenumbers\renewcommand\thelinenumber{\color{black!50}\arabic{linenumber}}
            \input{0 - introduction/main.tex}
        \part{Research}
            \input{1 - low-noise PiC models/main.tex}
            \input{2 - kinetic component/main.tex}
            \input{3 - fluid component/main.tex}
            \input{4 - numerical implementation/main.tex}
        \part{Project Overview}
            \input{5 - research plan/main.tex}
            \input{6 - summary/main.tex}
    
    
    %\section{}
    \newpage
    \pagenumbering{gobble}
        \printbibliography


    \newpage
    \pagenumbering{roman}
    \appendix
        \part{Appendices}
            \input{8 - Hilbert complexes/main.tex}
            \input{9 - weak conservation proofs/main.tex}
\end{document}

        \part{Project Overview}
            \documentclass[12pt, a4paper]{report}

\input{template/main.tex}

\title{\BA{Title in Progress...}}
\author{Boris Andrews}
\affil{Mathematical Institute, University of Oxford}
\date{\today}


\begin{document}
    \pagenumbering{gobble}
    \maketitle
    
    
    \begin{abstract}
        Magnetic confinement reactors---in particular tokamaks---offer one of the most promising options for achieving practical nuclear fusion, with the potential to provide virtually limitless, clean energy. The theoretical and numerical modeling of tokamak plasmas is simultaneously an essential component of effective reactor design, and a great research barrier. Tokamak operational conditions exhibit comparatively low Knudsen numbers. Kinetic effects, including kinetic waves and instabilities, Landau damping, bump-on-tail instabilities and more, are therefore highly influential in tokamak plasma dynamics. Purely fluid models are inherently incapable of capturing these effects, whereas the high dimensionality in purely kinetic models render them practically intractable for most relevant purposes.

        We consider a $\delta\!f$ decomposition model, with a macroscopic fluid background and microscopic kinetic correction, both fully coupled to each other. A similar manner of discretization is proposed to that used in the recent \texttt{STRUPHY} code \cite{Holderied_Possanner_Wang_2021, Holderied_2022, Li_et_al_2023} with a finite-element model for the background and a pseudo-particle/PiC model for the correction.

        The fluid background satisfies the full, non-linear, resistive, compressible, Hall MHD equations. \cite{Laakmann_Hu_Farrell_2022} introduces finite-element(-in-space) implicit timesteppers for the incompressible analogue to this system with structure-preserving (SP) properties in the ideal case, alongside parameter-robust preconditioners. We show that these timesteppers can derive from a finite-element-in-time (FET) (and finite-element-in-space) interpretation. The benefits of this reformulation are discussed, including the derivation of timesteppers that are higher order in time, and the quantifiable dissipative SP properties in the non-ideal, resistive case.
        
        We discuss possible options for extending this FET approach to timesteppers for the compressible case.

        The kinetic corrections satisfy linearized Boltzmann equations. Using a Lénard--Bernstein collision operator, these take Fokker--Planck-like forms \cite{Fokker_1914, Planck_1917} wherein pseudo-particles in the numerical model obey the neoclassical transport equations, with particle-independent Brownian drift terms. This offers a rigorous methodology for incorporating collisions into the particle transport model, without coupling the equations of motions for each particle.
        
        Works by Chen, Chacón et al. \cite{Chen_Chacón_Barnes_2011, Chacón_Chen_Barnes_2013, Chen_Chacón_2014, Chen_Chacón_2015} have developed structure-preserving particle pushers for neoclassical transport in the Vlasov equations, derived from Crank--Nicolson integrators. We show these too can can derive from a FET interpretation, similarly offering potential extensions to higher-order-in-time particle pushers. The FET formulation is used also to consider how the stochastic drift terms can be incorporated into the pushers. Stochastic gyrokinetic expansions are also discussed.

        Different options for the numerical implementation of these schemes are considered.

        Due to the efficacy of FET in the development of SP timesteppers for both the fluid and kinetic component, we hope this approach will prove effective in the future for developing SP timesteppers for the full hybrid model. We hope this will give us the opportunity to incorporate previously inaccessible kinetic effects into the highly effective, modern, finite-element MHD models.
    \end{abstract}
    
    
    \newpage
    \tableofcontents
    
    
    \newpage
    \pagenumbering{arabic}
    %\linenumbers\renewcommand\thelinenumber{\color{black!50}\arabic{linenumber}}
            \input{0 - introduction/main.tex}
        \part{Research}
            \input{1 - low-noise PiC models/main.tex}
            \input{2 - kinetic component/main.tex}
            \input{3 - fluid component/main.tex}
            \input{4 - numerical implementation/main.tex}
        \part{Project Overview}
            \input{5 - research plan/main.tex}
            \input{6 - summary/main.tex}
    
    
    %\section{}
    \newpage
    \pagenumbering{gobble}
        \printbibliography


    \newpage
    \pagenumbering{roman}
    \appendix
        \part{Appendices}
            \input{8 - Hilbert complexes/main.tex}
            \input{9 - weak conservation proofs/main.tex}
\end{document}

            \documentclass[12pt, a4paper]{report}

\input{template/main.tex}

\title{\BA{Title in Progress...}}
\author{Boris Andrews}
\affil{Mathematical Institute, University of Oxford}
\date{\today}


\begin{document}
    \pagenumbering{gobble}
    \maketitle
    
    
    \begin{abstract}
        Magnetic confinement reactors---in particular tokamaks---offer one of the most promising options for achieving practical nuclear fusion, with the potential to provide virtually limitless, clean energy. The theoretical and numerical modeling of tokamak plasmas is simultaneously an essential component of effective reactor design, and a great research barrier. Tokamak operational conditions exhibit comparatively low Knudsen numbers. Kinetic effects, including kinetic waves and instabilities, Landau damping, bump-on-tail instabilities and more, are therefore highly influential in tokamak plasma dynamics. Purely fluid models are inherently incapable of capturing these effects, whereas the high dimensionality in purely kinetic models render them practically intractable for most relevant purposes.

        We consider a $\delta\!f$ decomposition model, with a macroscopic fluid background and microscopic kinetic correction, both fully coupled to each other. A similar manner of discretization is proposed to that used in the recent \texttt{STRUPHY} code \cite{Holderied_Possanner_Wang_2021, Holderied_2022, Li_et_al_2023} with a finite-element model for the background and a pseudo-particle/PiC model for the correction.

        The fluid background satisfies the full, non-linear, resistive, compressible, Hall MHD equations. \cite{Laakmann_Hu_Farrell_2022} introduces finite-element(-in-space) implicit timesteppers for the incompressible analogue to this system with structure-preserving (SP) properties in the ideal case, alongside parameter-robust preconditioners. We show that these timesteppers can derive from a finite-element-in-time (FET) (and finite-element-in-space) interpretation. The benefits of this reformulation are discussed, including the derivation of timesteppers that are higher order in time, and the quantifiable dissipative SP properties in the non-ideal, resistive case.
        
        We discuss possible options for extending this FET approach to timesteppers for the compressible case.

        The kinetic corrections satisfy linearized Boltzmann equations. Using a Lénard--Bernstein collision operator, these take Fokker--Planck-like forms \cite{Fokker_1914, Planck_1917} wherein pseudo-particles in the numerical model obey the neoclassical transport equations, with particle-independent Brownian drift terms. This offers a rigorous methodology for incorporating collisions into the particle transport model, without coupling the equations of motions for each particle.
        
        Works by Chen, Chacón et al. \cite{Chen_Chacón_Barnes_2011, Chacón_Chen_Barnes_2013, Chen_Chacón_2014, Chen_Chacón_2015} have developed structure-preserving particle pushers for neoclassical transport in the Vlasov equations, derived from Crank--Nicolson integrators. We show these too can can derive from a FET interpretation, similarly offering potential extensions to higher-order-in-time particle pushers. The FET formulation is used also to consider how the stochastic drift terms can be incorporated into the pushers. Stochastic gyrokinetic expansions are also discussed.

        Different options for the numerical implementation of these schemes are considered.

        Due to the efficacy of FET in the development of SP timesteppers for both the fluid and kinetic component, we hope this approach will prove effective in the future for developing SP timesteppers for the full hybrid model. We hope this will give us the opportunity to incorporate previously inaccessible kinetic effects into the highly effective, modern, finite-element MHD models.
    \end{abstract}
    
    
    \newpage
    \tableofcontents
    
    
    \newpage
    \pagenumbering{arabic}
    %\linenumbers\renewcommand\thelinenumber{\color{black!50}\arabic{linenumber}}
            \input{0 - introduction/main.tex}
        \part{Research}
            \input{1 - low-noise PiC models/main.tex}
            \input{2 - kinetic component/main.tex}
            \input{3 - fluid component/main.tex}
            \input{4 - numerical implementation/main.tex}
        \part{Project Overview}
            \input{5 - research plan/main.tex}
            \input{6 - summary/main.tex}
    
    
    %\section{}
    \newpage
    \pagenumbering{gobble}
        \printbibliography


    \newpage
    \pagenumbering{roman}
    \appendix
        \part{Appendices}
            \input{8 - Hilbert complexes/main.tex}
            \input{9 - weak conservation proofs/main.tex}
\end{document}

    
    
    %\section{}
    \newpage
    \pagenumbering{gobble}
        \printbibliography


    \newpage
    \pagenumbering{roman}
    \appendix
        \part{Appendices}
            \documentclass[12pt, a4paper]{report}

\input{template/main.tex}

\title{\BA{Title in Progress...}}
\author{Boris Andrews}
\affil{Mathematical Institute, University of Oxford}
\date{\today}


\begin{document}
    \pagenumbering{gobble}
    \maketitle
    
    
    \begin{abstract}
        Magnetic confinement reactors---in particular tokamaks---offer one of the most promising options for achieving practical nuclear fusion, with the potential to provide virtually limitless, clean energy. The theoretical and numerical modeling of tokamak plasmas is simultaneously an essential component of effective reactor design, and a great research barrier. Tokamak operational conditions exhibit comparatively low Knudsen numbers. Kinetic effects, including kinetic waves and instabilities, Landau damping, bump-on-tail instabilities and more, are therefore highly influential in tokamak plasma dynamics. Purely fluid models are inherently incapable of capturing these effects, whereas the high dimensionality in purely kinetic models render them practically intractable for most relevant purposes.

        We consider a $\delta\!f$ decomposition model, with a macroscopic fluid background and microscopic kinetic correction, both fully coupled to each other. A similar manner of discretization is proposed to that used in the recent \texttt{STRUPHY} code \cite{Holderied_Possanner_Wang_2021, Holderied_2022, Li_et_al_2023} with a finite-element model for the background and a pseudo-particle/PiC model for the correction.

        The fluid background satisfies the full, non-linear, resistive, compressible, Hall MHD equations. \cite{Laakmann_Hu_Farrell_2022} introduces finite-element(-in-space) implicit timesteppers for the incompressible analogue to this system with structure-preserving (SP) properties in the ideal case, alongside parameter-robust preconditioners. We show that these timesteppers can derive from a finite-element-in-time (FET) (and finite-element-in-space) interpretation. The benefits of this reformulation are discussed, including the derivation of timesteppers that are higher order in time, and the quantifiable dissipative SP properties in the non-ideal, resistive case.
        
        We discuss possible options for extending this FET approach to timesteppers for the compressible case.

        The kinetic corrections satisfy linearized Boltzmann equations. Using a Lénard--Bernstein collision operator, these take Fokker--Planck-like forms \cite{Fokker_1914, Planck_1917} wherein pseudo-particles in the numerical model obey the neoclassical transport equations, with particle-independent Brownian drift terms. This offers a rigorous methodology for incorporating collisions into the particle transport model, without coupling the equations of motions for each particle.
        
        Works by Chen, Chacón et al. \cite{Chen_Chacón_Barnes_2011, Chacón_Chen_Barnes_2013, Chen_Chacón_2014, Chen_Chacón_2015} have developed structure-preserving particle pushers for neoclassical transport in the Vlasov equations, derived from Crank--Nicolson integrators. We show these too can can derive from a FET interpretation, similarly offering potential extensions to higher-order-in-time particle pushers. The FET formulation is used also to consider how the stochastic drift terms can be incorporated into the pushers. Stochastic gyrokinetic expansions are also discussed.

        Different options for the numerical implementation of these schemes are considered.

        Due to the efficacy of FET in the development of SP timesteppers for both the fluid and kinetic component, we hope this approach will prove effective in the future for developing SP timesteppers for the full hybrid model. We hope this will give us the opportunity to incorporate previously inaccessible kinetic effects into the highly effective, modern, finite-element MHD models.
    \end{abstract}
    
    
    \newpage
    \tableofcontents
    
    
    \newpage
    \pagenumbering{arabic}
    %\linenumbers\renewcommand\thelinenumber{\color{black!50}\arabic{linenumber}}
            \input{0 - introduction/main.tex}
        \part{Research}
            \input{1 - low-noise PiC models/main.tex}
            \input{2 - kinetic component/main.tex}
            \input{3 - fluid component/main.tex}
            \input{4 - numerical implementation/main.tex}
        \part{Project Overview}
            \input{5 - research plan/main.tex}
            \input{6 - summary/main.tex}
    
    
    %\section{}
    \newpage
    \pagenumbering{gobble}
        \printbibliography


    \newpage
    \pagenumbering{roman}
    \appendix
        \part{Appendices}
            \input{8 - Hilbert complexes/main.tex}
            \input{9 - weak conservation proofs/main.tex}
\end{document}

            \documentclass[12pt, a4paper]{report}

\input{template/main.tex}

\title{\BA{Title in Progress...}}
\author{Boris Andrews}
\affil{Mathematical Institute, University of Oxford}
\date{\today}


\begin{document}
    \pagenumbering{gobble}
    \maketitle
    
    
    \begin{abstract}
        Magnetic confinement reactors---in particular tokamaks---offer one of the most promising options for achieving practical nuclear fusion, with the potential to provide virtually limitless, clean energy. The theoretical and numerical modeling of tokamak plasmas is simultaneously an essential component of effective reactor design, and a great research barrier. Tokamak operational conditions exhibit comparatively low Knudsen numbers. Kinetic effects, including kinetic waves and instabilities, Landau damping, bump-on-tail instabilities and more, are therefore highly influential in tokamak plasma dynamics. Purely fluid models are inherently incapable of capturing these effects, whereas the high dimensionality in purely kinetic models render them practically intractable for most relevant purposes.

        We consider a $\delta\!f$ decomposition model, with a macroscopic fluid background and microscopic kinetic correction, both fully coupled to each other. A similar manner of discretization is proposed to that used in the recent \texttt{STRUPHY} code \cite{Holderied_Possanner_Wang_2021, Holderied_2022, Li_et_al_2023} with a finite-element model for the background and a pseudo-particle/PiC model for the correction.

        The fluid background satisfies the full, non-linear, resistive, compressible, Hall MHD equations. \cite{Laakmann_Hu_Farrell_2022} introduces finite-element(-in-space) implicit timesteppers for the incompressible analogue to this system with structure-preserving (SP) properties in the ideal case, alongside parameter-robust preconditioners. We show that these timesteppers can derive from a finite-element-in-time (FET) (and finite-element-in-space) interpretation. The benefits of this reformulation are discussed, including the derivation of timesteppers that are higher order in time, and the quantifiable dissipative SP properties in the non-ideal, resistive case.
        
        We discuss possible options for extending this FET approach to timesteppers for the compressible case.

        The kinetic corrections satisfy linearized Boltzmann equations. Using a Lénard--Bernstein collision operator, these take Fokker--Planck-like forms \cite{Fokker_1914, Planck_1917} wherein pseudo-particles in the numerical model obey the neoclassical transport equations, with particle-independent Brownian drift terms. This offers a rigorous methodology for incorporating collisions into the particle transport model, without coupling the equations of motions for each particle.
        
        Works by Chen, Chacón et al. \cite{Chen_Chacón_Barnes_2011, Chacón_Chen_Barnes_2013, Chen_Chacón_2014, Chen_Chacón_2015} have developed structure-preserving particle pushers for neoclassical transport in the Vlasov equations, derived from Crank--Nicolson integrators. We show these too can can derive from a FET interpretation, similarly offering potential extensions to higher-order-in-time particle pushers. The FET formulation is used also to consider how the stochastic drift terms can be incorporated into the pushers. Stochastic gyrokinetic expansions are also discussed.

        Different options for the numerical implementation of these schemes are considered.

        Due to the efficacy of FET in the development of SP timesteppers for both the fluid and kinetic component, we hope this approach will prove effective in the future for developing SP timesteppers for the full hybrid model. We hope this will give us the opportunity to incorporate previously inaccessible kinetic effects into the highly effective, modern, finite-element MHD models.
    \end{abstract}
    
    
    \newpage
    \tableofcontents
    
    
    \newpage
    \pagenumbering{arabic}
    %\linenumbers\renewcommand\thelinenumber{\color{black!50}\arabic{linenumber}}
            \input{0 - introduction/main.tex}
        \part{Research}
            \input{1 - low-noise PiC models/main.tex}
            \input{2 - kinetic component/main.tex}
            \input{3 - fluid component/main.tex}
            \input{4 - numerical implementation/main.tex}
        \part{Project Overview}
            \input{5 - research plan/main.tex}
            \input{6 - summary/main.tex}
    
    
    %\section{}
    \newpage
    \pagenumbering{gobble}
        \printbibliography


    \newpage
    \pagenumbering{roman}
    \appendix
        \part{Appendices}
            \input{8 - Hilbert complexes/main.tex}
            \input{9 - weak conservation proofs/main.tex}
\end{document}

\end{document}

            \documentclass[12pt, a4paper]{report}

\documentclass[12pt, a4paper]{report}

\input{template/main.tex}

\title{\BA{Title in Progress...}}
\author{Boris Andrews}
\affil{Mathematical Institute, University of Oxford}
\date{\today}


\begin{document}
    \pagenumbering{gobble}
    \maketitle
    
    
    \begin{abstract}
        Magnetic confinement reactors---in particular tokamaks---offer one of the most promising options for achieving practical nuclear fusion, with the potential to provide virtually limitless, clean energy. The theoretical and numerical modeling of tokamak plasmas is simultaneously an essential component of effective reactor design, and a great research barrier. Tokamak operational conditions exhibit comparatively low Knudsen numbers. Kinetic effects, including kinetic waves and instabilities, Landau damping, bump-on-tail instabilities and more, are therefore highly influential in tokamak plasma dynamics. Purely fluid models are inherently incapable of capturing these effects, whereas the high dimensionality in purely kinetic models render them practically intractable for most relevant purposes.

        We consider a $\delta\!f$ decomposition model, with a macroscopic fluid background and microscopic kinetic correction, both fully coupled to each other. A similar manner of discretization is proposed to that used in the recent \texttt{STRUPHY} code \cite{Holderied_Possanner_Wang_2021, Holderied_2022, Li_et_al_2023} with a finite-element model for the background and a pseudo-particle/PiC model for the correction.

        The fluid background satisfies the full, non-linear, resistive, compressible, Hall MHD equations. \cite{Laakmann_Hu_Farrell_2022} introduces finite-element(-in-space) implicit timesteppers for the incompressible analogue to this system with structure-preserving (SP) properties in the ideal case, alongside parameter-robust preconditioners. We show that these timesteppers can derive from a finite-element-in-time (FET) (and finite-element-in-space) interpretation. The benefits of this reformulation are discussed, including the derivation of timesteppers that are higher order in time, and the quantifiable dissipative SP properties in the non-ideal, resistive case.
        
        We discuss possible options for extending this FET approach to timesteppers for the compressible case.

        The kinetic corrections satisfy linearized Boltzmann equations. Using a Lénard--Bernstein collision operator, these take Fokker--Planck-like forms \cite{Fokker_1914, Planck_1917} wherein pseudo-particles in the numerical model obey the neoclassical transport equations, with particle-independent Brownian drift terms. This offers a rigorous methodology for incorporating collisions into the particle transport model, without coupling the equations of motions for each particle.
        
        Works by Chen, Chacón et al. \cite{Chen_Chacón_Barnes_2011, Chacón_Chen_Barnes_2013, Chen_Chacón_2014, Chen_Chacón_2015} have developed structure-preserving particle pushers for neoclassical transport in the Vlasov equations, derived from Crank--Nicolson integrators. We show these too can can derive from a FET interpretation, similarly offering potential extensions to higher-order-in-time particle pushers. The FET formulation is used also to consider how the stochastic drift terms can be incorporated into the pushers. Stochastic gyrokinetic expansions are also discussed.

        Different options for the numerical implementation of these schemes are considered.

        Due to the efficacy of FET in the development of SP timesteppers for both the fluid and kinetic component, we hope this approach will prove effective in the future for developing SP timesteppers for the full hybrid model. We hope this will give us the opportunity to incorporate previously inaccessible kinetic effects into the highly effective, modern, finite-element MHD models.
    \end{abstract}
    
    
    \newpage
    \tableofcontents
    
    
    \newpage
    \pagenumbering{arabic}
    %\linenumbers\renewcommand\thelinenumber{\color{black!50}\arabic{linenumber}}
            \input{0 - introduction/main.tex}
        \part{Research}
            \input{1 - low-noise PiC models/main.tex}
            \input{2 - kinetic component/main.tex}
            \input{3 - fluid component/main.tex}
            \input{4 - numerical implementation/main.tex}
        \part{Project Overview}
            \input{5 - research plan/main.tex}
            \input{6 - summary/main.tex}
    
    
    %\section{}
    \newpage
    \pagenumbering{gobble}
        \printbibliography


    \newpage
    \pagenumbering{roman}
    \appendix
        \part{Appendices}
            \input{8 - Hilbert complexes/main.tex}
            \input{9 - weak conservation proofs/main.tex}
\end{document}


\title{\BA{Title in Progress...}}
\author{Boris Andrews}
\affil{Mathematical Institute, University of Oxford}
\date{\today}


\begin{document}
    \pagenumbering{gobble}
    \maketitle
    
    
    \begin{abstract}
        Magnetic confinement reactors---in particular tokamaks---offer one of the most promising options for achieving practical nuclear fusion, with the potential to provide virtually limitless, clean energy. The theoretical and numerical modeling of tokamak plasmas is simultaneously an essential component of effective reactor design, and a great research barrier. Tokamak operational conditions exhibit comparatively low Knudsen numbers. Kinetic effects, including kinetic waves and instabilities, Landau damping, bump-on-tail instabilities and more, are therefore highly influential in tokamak plasma dynamics. Purely fluid models are inherently incapable of capturing these effects, whereas the high dimensionality in purely kinetic models render them practically intractable for most relevant purposes.

        We consider a $\delta\!f$ decomposition model, with a macroscopic fluid background and microscopic kinetic correction, both fully coupled to each other. A similar manner of discretization is proposed to that used in the recent \texttt{STRUPHY} code \cite{Holderied_Possanner_Wang_2021, Holderied_2022, Li_et_al_2023} with a finite-element model for the background and a pseudo-particle/PiC model for the correction.

        The fluid background satisfies the full, non-linear, resistive, compressible, Hall MHD equations. \cite{Laakmann_Hu_Farrell_2022} introduces finite-element(-in-space) implicit timesteppers for the incompressible analogue to this system with structure-preserving (SP) properties in the ideal case, alongside parameter-robust preconditioners. We show that these timesteppers can derive from a finite-element-in-time (FET) (and finite-element-in-space) interpretation. The benefits of this reformulation are discussed, including the derivation of timesteppers that are higher order in time, and the quantifiable dissipative SP properties in the non-ideal, resistive case.
        
        We discuss possible options for extending this FET approach to timesteppers for the compressible case.

        The kinetic corrections satisfy linearized Boltzmann equations. Using a Lénard--Bernstein collision operator, these take Fokker--Planck-like forms \cite{Fokker_1914, Planck_1917} wherein pseudo-particles in the numerical model obey the neoclassical transport equations, with particle-independent Brownian drift terms. This offers a rigorous methodology for incorporating collisions into the particle transport model, without coupling the equations of motions for each particle.
        
        Works by Chen, Chacón et al. \cite{Chen_Chacón_Barnes_2011, Chacón_Chen_Barnes_2013, Chen_Chacón_2014, Chen_Chacón_2015} have developed structure-preserving particle pushers for neoclassical transport in the Vlasov equations, derived from Crank--Nicolson integrators. We show these too can can derive from a FET interpretation, similarly offering potential extensions to higher-order-in-time particle pushers. The FET formulation is used also to consider how the stochastic drift terms can be incorporated into the pushers. Stochastic gyrokinetic expansions are also discussed.

        Different options for the numerical implementation of these schemes are considered.

        Due to the efficacy of FET in the development of SP timesteppers for both the fluid and kinetic component, we hope this approach will prove effective in the future for developing SP timesteppers for the full hybrid model. We hope this will give us the opportunity to incorporate previously inaccessible kinetic effects into the highly effective, modern, finite-element MHD models.
    \end{abstract}
    
    
    \newpage
    \tableofcontents
    
    
    \newpage
    \pagenumbering{arabic}
    %\linenumbers\renewcommand\thelinenumber{\color{black!50}\arabic{linenumber}}
            \documentclass[12pt, a4paper]{report}

\input{template/main.tex}

\title{\BA{Title in Progress...}}
\author{Boris Andrews}
\affil{Mathematical Institute, University of Oxford}
\date{\today}


\begin{document}
    \pagenumbering{gobble}
    \maketitle
    
    
    \begin{abstract}
        Magnetic confinement reactors---in particular tokamaks---offer one of the most promising options for achieving practical nuclear fusion, with the potential to provide virtually limitless, clean energy. The theoretical and numerical modeling of tokamak plasmas is simultaneously an essential component of effective reactor design, and a great research barrier. Tokamak operational conditions exhibit comparatively low Knudsen numbers. Kinetic effects, including kinetic waves and instabilities, Landau damping, bump-on-tail instabilities and more, are therefore highly influential in tokamak plasma dynamics. Purely fluid models are inherently incapable of capturing these effects, whereas the high dimensionality in purely kinetic models render them practically intractable for most relevant purposes.

        We consider a $\delta\!f$ decomposition model, with a macroscopic fluid background and microscopic kinetic correction, both fully coupled to each other. A similar manner of discretization is proposed to that used in the recent \texttt{STRUPHY} code \cite{Holderied_Possanner_Wang_2021, Holderied_2022, Li_et_al_2023} with a finite-element model for the background and a pseudo-particle/PiC model for the correction.

        The fluid background satisfies the full, non-linear, resistive, compressible, Hall MHD equations. \cite{Laakmann_Hu_Farrell_2022} introduces finite-element(-in-space) implicit timesteppers for the incompressible analogue to this system with structure-preserving (SP) properties in the ideal case, alongside parameter-robust preconditioners. We show that these timesteppers can derive from a finite-element-in-time (FET) (and finite-element-in-space) interpretation. The benefits of this reformulation are discussed, including the derivation of timesteppers that are higher order in time, and the quantifiable dissipative SP properties in the non-ideal, resistive case.
        
        We discuss possible options for extending this FET approach to timesteppers for the compressible case.

        The kinetic corrections satisfy linearized Boltzmann equations. Using a Lénard--Bernstein collision operator, these take Fokker--Planck-like forms \cite{Fokker_1914, Planck_1917} wherein pseudo-particles in the numerical model obey the neoclassical transport equations, with particle-independent Brownian drift terms. This offers a rigorous methodology for incorporating collisions into the particle transport model, without coupling the equations of motions for each particle.
        
        Works by Chen, Chacón et al. \cite{Chen_Chacón_Barnes_2011, Chacón_Chen_Barnes_2013, Chen_Chacón_2014, Chen_Chacón_2015} have developed structure-preserving particle pushers for neoclassical transport in the Vlasov equations, derived from Crank--Nicolson integrators. We show these too can can derive from a FET interpretation, similarly offering potential extensions to higher-order-in-time particle pushers. The FET formulation is used also to consider how the stochastic drift terms can be incorporated into the pushers. Stochastic gyrokinetic expansions are also discussed.

        Different options for the numerical implementation of these schemes are considered.

        Due to the efficacy of FET in the development of SP timesteppers for both the fluid and kinetic component, we hope this approach will prove effective in the future for developing SP timesteppers for the full hybrid model. We hope this will give us the opportunity to incorporate previously inaccessible kinetic effects into the highly effective, modern, finite-element MHD models.
    \end{abstract}
    
    
    \newpage
    \tableofcontents
    
    
    \newpage
    \pagenumbering{arabic}
    %\linenumbers\renewcommand\thelinenumber{\color{black!50}\arabic{linenumber}}
            \input{0 - introduction/main.tex}
        \part{Research}
            \input{1 - low-noise PiC models/main.tex}
            \input{2 - kinetic component/main.tex}
            \input{3 - fluid component/main.tex}
            \input{4 - numerical implementation/main.tex}
        \part{Project Overview}
            \input{5 - research plan/main.tex}
            \input{6 - summary/main.tex}
    
    
    %\section{}
    \newpage
    \pagenumbering{gobble}
        \printbibliography


    \newpage
    \pagenumbering{roman}
    \appendix
        \part{Appendices}
            \input{8 - Hilbert complexes/main.tex}
            \input{9 - weak conservation proofs/main.tex}
\end{document}

        \part{Research}
            \documentclass[12pt, a4paper]{report}

\input{template/main.tex}

\title{\BA{Title in Progress...}}
\author{Boris Andrews}
\affil{Mathematical Institute, University of Oxford}
\date{\today}


\begin{document}
    \pagenumbering{gobble}
    \maketitle
    
    
    \begin{abstract}
        Magnetic confinement reactors---in particular tokamaks---offer one of the most promising options for achieving practical nuclear fusion, with the potential to provide virtually limitless, clean energy. The theoretical and numerical modeling of tokamak plasmas is simultaneously an essential component of effective reactor design, and a great research barrier. Tokamak operational conditions exhibit comparatively low Knudsen numbers. Kinetic effects, including kinetic waves and instabilities, Landau damping, bump-on-tail instabilities and more, are therefore highly influential in tokamak plasma dynamics. Purely fluid models are inherently incapable of capturing these effects, whereas the high dimensionality in purely kinetic models render them practically intractable for most relevant purposes.

        We consider a $\delta\!f$ decomposition model, with a macroscopic fluid background and microscopic kinetic correction, both fully coupled to each other. A similar manner of discretization is proposed to that used in the recent \texttt{STRUPHY} code \cite{Holderied_Possanner_Wang_2021, Holderied_2022, Li_et_al_2023} with a finite-element model for the background and a pseudo-particle/PiC model for the correction.

        The fluid background satisfies the full, non-linear, resistive, compressible, Hall MHD equations. \cite{Laakmann_Hu_Farrell_2022} introduces finite-element(-in-space) implicit timesteppers for the incompressible analogue to this system with structure-preserving (SP) properties in the ideal case, alongside parameter-robust preconditioners. We show that these timesteppers can derive from a finite-element-in-time (FET) (and finite-element-in-space) interpretation. The benefits of this reformulation are discussed, including the derivation of timesteppers that are higher order in time, and the quantifiable dissipative SP properties in the non-ideal, resistive case.
        
        We discuss possible options for extending this FET approach to timesteppers for the compressible case.

        The kinetic corrections satisfy linearized Boltzmann equations. Using a Lénard--Bernstein collision operator, these take Fokker--Planck-like forms \cite{Fokker_1914, Planck_1917} wherein pseudo-particles in the numerical model obey the neoclassical transport equations, with particle-independent Brownian drift terms. This offers a rigorous methodology for incorporating collisions into the particle transport model, without coupling the equations of motions for each particle.
        
        Works by Chen, Chacón et al. \cite{Chen_Chacón_Barnes_2011, Chacón_Chen_Barnes_2013, Chen_Chacón_2014, Chen_Chacón_2015} have developed structure-preserving particle pushers for neoclassical transport in the Vlasov equations, derived from Crank--Nicolson integrators. We show these too can can derive from a FET interpretation, similarly offering potential extensions to higher-order-in-time particle pushers. The FET formulation is used also to consider how the stochastic drift terms can be incorporated into the pushers. Stochastic gyrokinetic expansions are also discussed.

        Different options for the numerical implementation of these schemes are considered.

        Due to the efficacy of FET in the development of SP timesteppers for both the fluid and kinetic component, we hope this approach will prove effective in the future for developing SP timesteppers for the full hybrid model. We hope this will give us the opportunity to incorporate previously inaccessible kinetic effects into the highly effective, modern, finite-element MHD models.
    \end{abstract}
    
    
    \newpage
    \tableofcontents
    
    
    \newpage
    \pagenumbering{arabic}
    %\linenumbers\renewcommand\thelinenumber{\color{black!50}\arabic{linenumber}}
            \input{0 - introduction/main.tex}
        \part{Research}
            \input{1 - low-noise PiC models/main.tex}
            \input{2 - kinetic component/main.tex}
            \input{3 - fluid component/main.tex}
            \input{4 - numerical implementation/main.tex}
        \part{Project Overview}
            \input{5 - research plan/main.tex}
            \input{6 - summary/main.tex}
    
    
    %\section{}
    \newpage
    \pagenumbering{gobble}
        \printbibliography


    \newpage
    \pagenumbering{roman}
    \appendix
        \part{Appendices}
            \input{8 - Hilbert complexes/main.tex}
            \input{9 - weak conservation proofs/main.tex}
\end{document}

            \documentclass[12pt, a4paper]{report}

\input{template/main.tex}

\title{\BA{Title in Progress...}}
\author{Boris Andrews}
\affil{Mathematical Institute, University of Oxford}
\date{\today}


\begin{document}
    \pagenumbering{gobble}
    \maketitle
    
    
    \begin{abstract}
        Magnetic confinement reactors---in particular tokamaks---offer one of the most promising options for achieving practical nuclear fusion, with the potential to provide virtually limitless, clean energy. The theoretical and numerical modeling of tokamak plasmas is simultaneously an essential component of effective reactor design, and a great research barrier. Tokamak operational conditions exhibit comparatively low Knudsen numbers. Kinetic effects, including kinetic waves and instabilities, Landau damping, bump-on-tail instabilities and more, are therefore highly influential in tokamak plasma dynamics. Purely fluid models are inherently incapable of capturing these effects, whereas the high dimensionality in purely kinetic models render them practically intractable for most relevant purposes.

        We consider a $\delta\!f$ decomposition model, with a macroscopic fluid background and microscopic kinetic correction, both fully coupled to each other. A similar manner of discretization is proposed to that used in the recent \texttt{STRUPHY} code \cite{Holderied_Possanner_Wang_2021, Holderied_2022, Li_et_al_2023} with a finite-element model for the background and a pseudo-particle/PiC model for the correction.

        The fluid background satisfies the full, non-linear, resistive, compressible, Hall MHD equations. \cite{Laakmann_Hu_Farrell_2022} introduces finite-element(-in-space) implicit timesteppers for the incompressible analogue to this system with structure-preserving (SP) properties in the ideal case, alongside parameter-robust preconditioners. We show that these timesteppers can derive from a finite-element-in-time (FET) (and finite-element-in-space) interpretation. The benefits of this reformulation are discussed, including the derivation of timesteppers that are higher order in time, and the quantifiable dissipative SP properties in the non-ideal, resistive case.
        
        We discuss possible options for extending this FET approach to timesteppers for the compressible case.

        The kinetic corrections satisfy linearized Boltzmann equations. Using a Lénard--Bernstein collision operator, these take Fokker--Planck-like forms \cite{Fokker_1914, Planck_1917} wherein pseudo-particles in the numerical model obey the neoclassical transport equations, with particle-independent Brownian drift terms. This offers a rigorous methodology for incorporating collisions into the particle transport model, without coupling the equations of motions for each particle.
        
        Works by Chen, Chacón et al. \cite{Chen_Chacón_Barnes_2011, Chacón_Chen_Barnes_2013, Chen_Chacón_2014, Chen_Chacón_2015} have developed structure-preserving particle pushers for neoclassical transport in the Vlasov equations, derived from Crank--Nicolson integrators. We show these too can can derive from a FET interpretation, similarly offering potential extensions to higher-order-in-time particle pushers. The FET formulation is used also to consider how the stochastic drift terms can be incorporated into the pushers. Stochastic gyrokinetic expansions are also discussed.

        Different options for the numerical implementation of these schemes are considered.

        Due to the efficacy of FET in the development of SP timesteppers for both the fluid and kinetic component, we hope this approach will prove effective in the future for developing SP timesteppers for the full hybrid model. We hope this will give us the opportunity to incorporate previously inaccessible kinetic effects into the highly effective, modern, finite-element MHD models.
    \end{abstract}
    
    
    \newpage
    \tableofcontents
    
    
    \newpage
    \pagenumbering{arabic}
    %\linenumbers\renewcommand\thelinenumber{\color{black!50}\arabic{linenumber}}
            \input{0 - introduction/main.tex}
        \part{Research}
            \input{1 - low-noise PiC models/main.tex}
            \input{2 - kinetic component/main.tex}
            \input{3 - fluid component/main.tex}
            \input{4 - numerical implementation/main.tex}
        \part{Project Overview}
            \input{5 - research plan/main.tex}
            \input{6 - summary/main.tex}
    
    
    %\section{}
    \newpage
    \pagenumbering{gobble}
        \printbibliography


    \newpage
    \pagenumbering{roman}
    \appendix
        \part{Appendices}
            \input{8 - Hilbert complexes/main.tex}
            \input{9 - weak conservation proofs/main.tex}
\end{document}

            \documentclass[12pt, a4paper]{report}

\input{template/main.tex}

\title{\BA{Title in Progress...}}
\author{Boris Andrews}
\affil{Mathematical Institute, University of Oxford}
\date{\today}


\begin{document}
    \pagenumbering{gobble}
    \maketitle
    
    
    \begin{abstract}
        Magnetic confinement reactors---in particular tokamaks---offer one of the most promising options for achieving practical nuclear fusion, with the potential to provide virtually limitless, clean energy. The theoretical and numerical modeling of tokamak plasmas is simultaneously an essential component of effective reactor design, and a great research barrier. Tokamak operational conditions exhibit comparatively low Knudsen numbers. Kinetic effects, including kinetic waves and instabilities, Landau damping, bump-on-tail instabilities and more, are therefore highly influential in tokamak plasma dynamics. Purely fluid models are inherently incapable of capturing these effects, whereas the high dimensionality in purely kinetic models render them practically intractable for most relevant purposes.

        We consider a $\delta\!f$ decomposition model, with a macroscopic fluid background and microscopic kinetic correction, both fully coupled to each other. A similar manner of discretization is proposed to that used in the recent \texttt{STRUPHY} code \cite{Holderied_Possanner_Wang_2021, Holderied_2022, Li_et_al_2023} with a finite-element model for the background and a pseudo-particle/PiC model for the correction.

        The fluid background satisfies the full, non-linear, resistive, compressible, Hall MHD equations. \cite{Laakmann_Hu_Farrell_2022} introduces finite-element(-in-space) implicit timesteppers for the incompressible analogue to this system with structure-preserving (SP) properties in the ideal case, alongside parameter-robust preconditioners. We show that these timesteppers can derive from a finite-element-in-time (FET) (and finite-element-in-space) interpretation. The benefits of this reformulation are discussed, including the derivation of timesteppers that are higher order in time, and the quantifiable dissipative SP properties in the non-ideal, resistive case.
        
        We discuss possible options for extending this FET approach to timesteppers for the compressible case.

        The kinetic corrections satisfy linearized Boltzmann equations. Using a Lénard--Bernstein collision operator, these take Fokker--Planck-like forms \cite{Fokker_1914, Planck_1917} wherein pseudo-particles in the numerical model obey the neoclassical transport equations, with particle-independent Brownian drift terms. This offers a rigorous methodology for incorporating collisions into the particle transport model, without coupling the equations of motions for each particle.
        
        Works by Chen, Chacón et al. \cite{Chen_Chacón_Barnes_2011, Chacón_Chen_Barnes_2013, Chen_Chacón_2014, Chen_Chacón_2015} have developed structure-preserving particle pushers for neoclassical transport in the Vlasov equations, derived from Crank--Nicolson integrators. We show these too can can derive from a FET interpretation, similarly offering potential extensions to higher-order-in-time particle pushers. The FET formulation is used also to consider how the stochastic drift terms can be incorporated into the pushers. Stochastic gyrokinetic expansions are also discussed.

        Different options for the numerical implementation of these schemes are considered.

        Due to the efficacy of FET in the development of SP timesteppers for both the fluid and kinetic component, we hope this approach will prove effective in the future for developing SP timesteppers for the full hybrid model. We hope this will give us the opportunity to incorporate previously inaccessible kinetic effects into the highly effective, modern, finite-element MHD models.
    \end{abstract}
    
    
    \newpage
    \tableofcontents
    
    
    \newpage
    \pagenumbering{arabic}
    %\linenumbers\renewcommand\thelinenumber{\color{black!50}\arabic{linenumber}}
            \input{0 - introduction/main.tex}
        \part{Research}
            \input{1 - low-noise PiC models/main.tex}
            \input{2 - kinetic component/main.tex}
            \input{3 - fluid component/main.tex}
            \input{4 - numerical implementation/main.tex}
        \part{Project Overview}
            \input{5 - research plan/main.tex}
            \input{6 - summary/main.tex}
    
    
    %\section{}
    \newpage
    \pagenumbering{gobble}
        \printbibliography


    \newpage
    \pagenumbering{roman}
    \appendix
        \part{Appendices}
            \input{8 - Hilbert complexes/main.tex}
            \input{9 - weak conservation proofs/main.tex}
\end{document}

            \documentclass[12pt, a4paper]{report}

\input{template/main.tex}

\title{\BA{Title in Progress...}}
\author{Boris Andrews}
\affil{Mathematical Institute, University of Oxford}
\date{\today}


\begin{document}
    \pagenumbering{gobble}
    \maketitle
    
    
    \begin{abstract}
        Magnetic confinement reactors---in particular tokamaks---offer one of the most promising options for achieving practical nuclear fusion, with the potential to provide virtually limitless, clean energy. The theoretical and numerical modeling of tokamak plasmas is simultaneously an essential component of effective reactor design, and a great research barrier. Tokamak operational conditions exhibit comparatively low Knudsen numbers. Kinetic effects, including kinetic waves and instabilities, Landau damping, bump-on-tail instabilities and more, are therefore highly influential in tokamak plasma dynamics. Purely fluid models are inherently incapable of capturing these effects, whereas the high dimensionality in purely kinetic models render them practically intractable for most relevant purposes.

        We consider a $\delta\!f$ decomposition model, with a macroscopic fluid background and microscopic kinetic correction, both fully coupled to each other. A similar manner of discretization is proposed to that used in the recent \texttt{STRUPHY} code \cite{Holderied_Possanner_Wang_2021, Holderied_2022, Li_et_al_2023} with a finite-element model for the background and a pseudo-particle/PiC model for the correction.

        The fluid background satisfies the full, non-linear, resistive, compressible, Hall MHD equations. \cite{Laakmann_Hu_Farrell_2022} introduces finite-element(-in-space) implicit timesteppers for the incompressible analogue to this system with structure-preserving (SP) properties in the ideal case, alongside parameter-robust preconditioners. We show that these timesteppers can derive from a finite-element-in-time (FET) (and finite-element-in-space) interpretation. The benefits of this reformulation are discussed, including the derivation of timesteppers that are higher order in time, and the quantifiable dissipative SP properties in the non-ideal, resistive case.
        
        We discuss possible options for extending this FET approach to timesteppers for the compressible case.

        The kinetic corrections satisfy linearized Boltzmann equations. Using a Lénard--Bernstein collision operator, these take Fokker--Planck-like forms \cite{Fokker_1914, Planck_1917} wherein pseudo-particles in the numerical model obey the neoclassical transport equations, with particle-independent Brownian drift terms. This offers a rigorous methodology for incorporating collisions into the particle transport model, without coupling the equations of motions for each particle.
        
        Works by Chen, Chacón et al. \cite{Chen_Chacón_Barnes_2011, Chacón_Chen_Barnes_2013, Chen_Chacón_2014, Chen_Chacón_2015} have developed structure-preserving particle pushers for neoclassical transport in the Vlasov equations, derived from Crank--Nicolson integrators. We show these too can can derive from a FET interpretation, similarly offering potential extensions to higher-order-in-time particle pushers. The FET formulation is used also to consider how the stochastic drift terms can be incorporated into the pushers. Stochastic gyrokinetic expansions are also discussed.

        Different options for the numerical implementation of these schemes are considered.

        Due to the efficacy of FET in the development of SP timesteppers for both the fluid and kinetic component, we hope this approach will prove effective in the future for developing SP timesteppers for the full hybrid model. We hope this will give us the opportunity to incorporate previously inaccessible kinetic effects into the highly effective, modern, finite-element MHD models.
    \end{abstract}
    
    
    \newpage
    \tableofcontents
    
    
    \newpage
    \pagenumbering{arabic}
    %\linenumbers\renewcommand\thelinenumber{\color{black!50}\arabic{linenumber}}
            \input{0 - introduction/main.tex}
        \part{Research}
            \input{1 - low-noise PiC models/main.tex}
            \input{2 - kinetic component/main.tex}
            \input{3 - fluid component/main.tex}
            \input{4 - numerical implementation/main.tex}
        \part{Project Overview}
            \input{5 - research plan/main.tex}
            \input{6 - summary/main.tex}
    
    
    %\section{}
    \newpage
    \pagenumbering{gobble}
        \printbibliography


    \newpage
    \pagenumbering{roman}
    \appendix
        \part{Appendices}
            \input{8 - Hilbert complexes/main.tex}
            \input{9 - weak conservation proofs/main.tex}
\end{document}

        \part{Project Overview}
            \documentclass[12pt, a4paper]{report}

\input{template/main.tex}

\title{\BA{Title in Progress...}}
\author{Boris Andrews}
\affil{Mathematical Institute, University of Oxford}
\date{\today}


\begin{document}
    \pagenumbering{gobble}
    \maketitle
    
    
    \begin{abstract}
        Magnetic confinement reactors---in particular tokamaks---offer one of the most promising options for achieving practical nuclear fusion, with the potential to provide virtually limitless, clean energy. The theoretical and numerical modeling of tokamak plasmas is simultaneously an essential component of effective reactor design, and a great research barrier. Tokamak operational conditions exhibit comparatively low Knudsen numbers. Kinetic effects, including kinetic waves and instabilities, Landau damping, bump-on-tail instabilities and more, are therefore highly influential in tokamak plasma dynamics. Purely fluid models are inherently incapable of capturing these effects, whereas the high dimensionality in purely kinetic models render them practically intractable for most relevant purposes.

        We consider a $\delta\!f$ decomposition model, with a macroscopic fluid background and microscopic kinetic correction, both fully coupled to each other. A similar manner of discretization is proposed to that used in the recent \texttt{STRUPHY} code \cite{Holderied_Possanner_Wang_2021, Holderied_2022, Li_et_al_2023} with a finite-element model for the background and a pseudo-particle/PiC model for the correction.

        The fluid background satisfies the full, non-linear, resistive, compressible, Hall MHD equations. \cite{Laakmann_Hu_Farrell_2022} introduces finite-element(-in-space) implicit timesteppers for the incompressible analogue to this system with structure-preserving (SP) properties in the ideal case, alongside parameter-robust preconditioners. We show that these timesteppers can derive from a finite-element-in-time (FET) (and finite-element-in-space) interpretation. The benefits of this reformulation are discussed, including the derivation of timesteppers that are higher order in time, and the quantifiable dissipative SP properties in the non-ideal, resistive case.
        
        We discuss possible options for extending this FET approach to timesteppers for the compressible case.

        The kinetic corrections satisfy linearized Boltzmann equations. Using a Lénard--Bernstein collision operator, these take Fokker--Planck-like forms \cite{Fokker_1914, Planck_1917} wherein pseudo-particles in the numerical model obey the neoclassical transport equations, with particle-independent Brownian drift terms. This offers a rigorous methodology for incorporating collisions into the particle transport model, without coupling the equations of motions for each particle.
        
        Works by Chen, Chacón et al. \cite{Chen_Chacón_Barnes_2011, Chacón_Chen_Barnes_2013, Chen_Chacón_2014, Chen_Chacón_2015} have developed structure-preserving particle pushers for neoclassical transport in the Vlasov equations, derived from Crank--Nicolson integrators. We show these too can can derive from a FET interpretation, similarly offering potential extensions to higher-order-in-time particle pushers. The FET formulation is used also to consider how the stochastic drift terms can be incorporated into the pushers. Stochastic gyrokinetic expansions are also discussed.

        Different options for the numerical implementation of these schemes are considered.

        Due to the efficacy of FET in the development of SP timesteppers for both the fluid and kinetic component, we hope this approach will prove effective in the future for developing SP timesteppers for the full hybrid model. We hope this will give us the opportunity to incorporate previously inaccessible kinetic effects into the highly effective, modern, finite-element MHD models.
    \end{abstract}
    
    
    \newpage
    \tableofcontents
    
    
    \newpage
    \pagenumbering{arabic}
    %\linenumbers\renewcommand\thelinenumber{\color{black!50}\arabic{linenumber}}
            \input{0 - introduction/main.tex}
        \part{Research}
            \input{1 - low-noise PiC models/main.tex}
            \input{2 - kinetic component/main.tex}
            \input{3 - fluid component/main.tex}
            \input{4 - numerical implementation/main.tex}
        \part{Project Overview}
            \input{5 - research plan/main.tex}
            \input{6 - summary/main.tex}
    
    
    %\section{}
    \newpage
    \pagenumbering{gobble}
        \printbibliography


    \newpage
    \pagenumbering{roman}
    \appendix
        \part{Appendices}
            \input{8 - Hilbert complexes/main.tex}
            \input{9 - weak conservation proofs/main.tex}
\end{document}

            \documentclass[12pt, a4paper]{report}

\input{template/main.tex}

\title{\BA{Title in Progress...}}
\author{Boris Andrews}
\affil{Mathematical Institute, University of Oxford}
\date{\today}


\begin{document}
    \pagenumbering{gobble}
    \maketitle
    
    
    \begin{abstract}
        Magnetic confinement reactors---in particular tokamaks---offer one of the most promising options for achieving practical nuclear fusion, with the potential to provide virtually limitless, clean energy. The theoretical and numerical modeling of tokamak plasmas is simultaneously an essential component of effective reactor design, and a great research barrier. Tokamak operational conditions exhibit comparatively low Knudsen numbers. Kinetic effects, including kinetic waves and instabilities, Landau damping, bump-on-tail instabilities and more, are therefore highly influential in tokamak plasma dynamics. Purely fluid models are inherently incapable of capturing these effects, whereas the high dimensionality in purely kinetic models render them practically intractable for most relevant purposes.

        We consider a $\delta\!f$ decomposition model, with a macroscopic fluid background and microscopic kinetic correction, both fully coupled to each other. A similar manner of discretization is proposed to that used in the recent \texttt{STRUPHY} code \cite{Holderied_Possanner_Wang_2021, Holderied_2022, Li_et_al_2023} with a finite-element model for the background and a pseudo-particle/PiC model for the correction.

        The fluid background satisfies the full, non-linear, resistive, compressible, Hall MHD equations. \cite{Laakmann_Hu_Farrell_2022} introduces finite-element(-in-space) implicit timesteppers for the incompressible analogue to this system with structure-preserving (SP) properties in the ideal case, alongside parameter-robust preconditioners. We show that these timesteppers can derive from a finite-element-in-time (FET) (and finite-element-in-space) interpretation. The benefits of this reformulation are discussed, including the derivation of timesteppers that are higher order in time, and the quantifiable dissipative SP properties in the non-ideal, resistive case.
        
        We discuss possible options for extending this FET approach to timesteppers for the compressible case.

        The kinetic corrections satisfy linearized Boltzmann equations. Using a Lénard--Bernstein collision operator, these take Fokker--Planck-like forms \cite{Fokker_1914, Planck_1917} wherein pseudo-particles in the numerical model obey the neoclassical transport equations, with particle-independent Brownian drift terms. This offers a rigorous methodology for incorporating collisions into the particle transport model, without coupling the equations of motions for each particle.
        
        Works by Chen, Chacón et al. \cite{Chen_Chacón_Barnes_2011, Chacón_Chen_Barnes_2013, Chen_Chacón_2014, Chen_Chacón_2015} have developed structure-preserving particle pushers for neoclassical transport in the Vlasov equations, derived from Crank--Nicolson integrators. We show these too can can derive from a FET interpretation, similarly offering potential extensions to higher-order-in-time particle pushers. The FET formulation is used also to consider how the stochastic drift terms can be incorporated into the pushers. Stochastic gyrokinetic expansions are also discussed.

        Different options for the numerical implementation of these schemes are considered.

        Due to the efficacy of FET in the development of SP timesteppers for both the fluid and kinetic component, we hope this approach will prove effective in the future for developing SP timesteppers for the full hybrid model. We hope this will give us the opportunity to incorporate previously inaccessible kinetic effects into the highly effective, modern, finite-element MHD models.
    \end{abstract}
    
    
    \newpage
    \tableofcontents
    
    
    \newpage
    \pagenumbering{arabic}
    %\linenumbers\renewcommand\thelinenumber{\color{black!50}\arabic{linenumber}}
            \input{0 - introduction/main.tex}
        \part{Research}
            \input{1 - low-noise PiC models/main.tex}
            \input{2 - kinetic component/main.tex}
            \input{3 - fluid component/main.tex}
            \input{4 - numerical implementation/main.tex}
        \part{Project Overview}
            \input{5 - research plan/main.tex}
            \input{6 - summary/main.tex}
    
    
    %\section{}
    \newpage
    \pagenumbering{gobble}
        \printbibliography


    \newpage
    \pagenumbering{roman}
    \appendix
        \part{Appendices}
            \input{8 - Hilbert complexes/main.tex}
            \input{9 - weak conservation proofs/main.tex}
\end{document}

    
    
    %\section{}
    \newpage
    \pagenumbering{gobble}
        \printbibliography


    \newpage
    \pagenumbering{roman}
    \appendix
        \part{Appendices}
            \documentclass[12pt, a4paper]{report}

\input{template/main.tex}

\title{\BA{Title in Progress...}}
\author{Boris Andrews}
\affil{Mathematical Institute, University of Oxford}
\date{\today}


\begin{document}
    \pagenumbering{gobble}
    \maketitle
    
    
    \begin{abstract}
        Magnetic confinement reactors---in particular tokamaks---offer one of the most promising options for achieving practical nuclear fusion, with the potential to provide virtually limitless, clean energy. The theoretical and numerical modeling of tokamak plasmas is simultaneously an essential component of effective reactor design, and a great research barrier. Tokamak operational conditions exhibit comparatively low Knudsen numbers. Kinetic effects, including kinetic waves and instabilities, Landau damping, bump-on-tail instabilities and more, are therefore highly influential in tokamak plasma dynamics. Purely fluid models are inherently incapable of capturing these effects, whereas the high dimensionality in purely kinetic models render them practically intractable for most relevant purposes.

        We consider a $\delta\!f$ decomposition model, with a macroscopic fluid background and microscopic kinetic correction, both fully coupled to each other. A similar manner of discretization is proposed to that used in the recent \texttt{STRUPHY} code \cite{Holderied_Possanner_Wang_2021, Holderied_2022, Li_et_al_2023} with a finite-element model for the background and a pseudo-particle/PiC model for the correction.

        The fluid background satisfies the full, non-linear, resistive, compressible, Hall MHD equations. \cite{Laakmann_Hu_Farrell_2022} introduces finite-element(-in-space) implicit timesteppers for the incompressible analogue to this system with structure-preserving (SP) properties in the ideal case, alongside parameter-robust preconditioners. We show that these timesteppers can derive from a finite-element-in-time (FET) (and finite-element-in-space) interpretation. The benefits of this reformulation are discussed, including the derivation of timesteppers that are higher order in time, and the quantifiable dissipative SP properties in the non-ideal, resistive case.
        
        We discuss possible options for extending this FET approach to timesteppers for the compressible case.

        The kinetic corrections satisfy linearized Boltzmann equations. Using a Lénard--Bernstein collision operator, these take Fokker--Planck-like forms \cite{Fokker_1914, Planck_1917} wherein pseudo-particles in the numerical model obey the neoclassical transport equations, with particle-independent Brownian drift terms. This offers a rigorous methodology for incorporating collisions into the particle transport model, without coupling the equations of motions for each particle.
        
        Works by Chen, Chacón et al. \cite{Chen_Chacón_Barnes_2011, Chacón_Chen_Barnes_2013, Chen_Chacón_2014, Chen_Chacón_2015} have developed structure-preserving particle pushers for neoclassical transport in the Vlasov equations, derived from Crank--Nicolson integrators. We show these too can can derive from a FET interpretation, similarly offering potential extensions to higher-order-in-time particle pushers. The FET formulation is used also to consider how the stochastic drift terms can be incorporated into the pushers. Stochastic gyrokinetic expansions are also discussed.

        Different options for the numerical implementation of these schemes are considered.

        Due to the efficacy of FET in the development of SP timesteppers for both the fluid and kinetic component, we hope this approach will prove effective in the future for developing SP timesteppers for the full hybrid model. We hope this will give us the opportunity to incorporate previously inaccessible kinetic effects into the highly effective, modern, finite-element MHD models.
    \end{abstract}
    
    
    \newpage
    \tableofcontents
    
    
    \newpage
    \pagenumbering{arabic}
    %\linenumbers\renewcommand\thelinenumber{\color{black!50}\arabic{linenumber}}
            \input{0 - introduction/main.tex}
        \part{Research}
            \input{1 - low-noise PiC models/main.tex}
            \input{2 - kinetic component/main.tex}
            \input{3 - fluid component/main.tex}
            \input{4 - numerical implementation/main.tex}
        \part{Project Overview}
            \input{5 - research plan/main.tex}
            \input{6 - summary/main.tex}
    
    
    %\section{}
    \newpage
    \pagenumbering{gobble}
        \printbibliography


    \newpage
    \pagenumbering{roman}
    \appendix
        \part{Appendices}
            \input{8 - Hilbert complexes/main.tex}
            \input{9 - weak conservation proofs/main.tex}
\end{document}

            \documentclass[12pt, a4paper]{report}

\input{template/main.tex}

\title{\BA{Title in Progress...}}
\author{Boris Andrews}
\affil{Mathematical Institute, University of Oxford}
\date{\today}


\begin{document}
    \pagenumbering{gobble}
    \maketitle
    
    
    \begin{abstract}
        Magnetic confinement reactors---in particular tokamaks---offer one of the most promising options for achieving practical nuclear fusion, with the potential to provide virtually limitless, clean energy. The theoretical and numerical modeling of tokamak plasmas is simultaneously an essential component of effective reactor design, and a great research barrier. Tokamak operational conditions exhibit comparatively low Knudsen numbers. Kinetic effects, including kinetic waves and instabilities, Landau damping, bump-on-tail instabilities and more, are therefore highly influential in tokamak plasma dynamics. Purely fluid models are inherently incapable of capturing these effects, whereas the high dimensionality in purely kinetic models render them practically intractable for most relevant purposes.

        We consider a $\delta\!f$ decomposition model, with a macroscopic fluid background and microscopic kinetic correction, both fully coupled to each other. A similar manner of discretization is proposed to that used in the recent \texttt{STRUPHY} code \cite{Holderied_Possanner_Wang_2021, Holderied_2022, Li_et_al_2023} with a finite-element model for the background and a pseudo-particle/PiC model for the correction.

        The fluid background satisfies the full, non-linear, resistive, compressible, Hall MHD equations. \cite{Laakmann_Hu_Farrell_2022} introduces finite-element(-in-space) implicit timesteppers for the incompressible analogue to this system with structure-preserving (SP) properties in the ideal case, alongside parameter-robust preconditioners. We show that these timesteppers can derive from a finite-element-in-time (FET) (and finite-element-in-space) interpretation. The benefits of this reformulation are discussed, including the derivation of timesteppers that are higher order in time, and the quantifiable dissipative SP properties in the non-ideal, resistive case.
        
        We discuss possible options for extending this FET approach to timesteppers for the compressible case.

        The kinetic corrections satisfy linearized Boltzmann equations. Using a Lénard--Bernstein collision operator, these take Fokker--Planck-like forms \cite{Fokker_1914, Planck_1917} wherein pseudo-particles in the numerical model obey the neoclassical transport equations, with particle-independent Brownian drift terms. This offers a rigorous methodology for incorporating collisions into the particle transport model, without coupling the equations of motions for each particle.
        
        Works by Chen, Chacón et al. \cite{Chen_Chacón_Barnes_2011, Chacón_Chen_Barnes_2013, Chen_Chacón_2014, Chen_Chacón_2015} have developed structure-preserving particle pushers for neoclassical transport in the Vlasov equations, derived from Crank--Nicolson integrators. We show these too can can derive from a FET interpretation, similarly offering potential extensions to higher-order-in-time particle pushers. The FET formulation is used also to consider how the stochastic drift terms can be incorporated into the pushers. Stochastic gyrokinetic expansions are also discussed.

        Different options for the numerical implementation of these schemes are considered.

        Due to the efficacy of FET in the development of SP timesteppers for both the fluid and kinetic component, we hope this approach will prove effective in the future for developing SP timesteppers for the full hybrid model. We hope this will give us the opportunity to incorporate previously inaccessible kinetic effects into the highly effective, modern, finite-element MHD models.
    \end{abstract}
    
    
    \newpage
    \tableofcontents
    
    
    \newpage
    \pagenumbering{arabic}
    %\linenumbers\renewcommand\thelinenumber{\color{black!50}\arabic{linenumber}}
            \input{0 - introduction/main.tex}
        \part{Research}
            \input{1 - low-noise PiC models/main.tex}
            \input{2 - kinetic component/main.tex}
            \input{3 - fluid component/main.tex}
            \input{4 - numerical implementation/main.tex}
        \part{Project Overview}
            \input{5 - research plan/main.tex}
            \input{6 - summary/main.tex}
    
    
    %\section{}
    \newpage
    \pagenumbering{gobble}
        \printbibliography


    \newpage
    \pagenumbering{roman}
    \appendix
        \part{Appendices}
            \input{8 - Hilbert complexes/main.tex}
            \input{9 - weak conservation proofs/main.tex}
\end{document}

\end{document}

            \documentclass[12pt, a4paper]{report}

\documentclass[12pt, a4paper]{report}

\input{template/main.tex}

\title{\BA{Title in Progress...}}
\author{Boris Andrews}
\affil{Mathematical Institute, University of Oxford}
\date{\today}


\begin{document}
    \pagenumbering{gobble}
    \maketitle
    
    
    \begin{abstract}
        Magnetic confinement reactors---in particular tokamaks---offer one of the most promising options for achieving practical nuclear fusion, with the potential to provide virtually limitless, clean energy. The theoretical and numerical modeling of tokamak plasmas is simultaneously an essential component of effective reactor design, and a great research barrier. Tokamak operational conditions exhibit comparatively low Knudsen numbers. Kinetic effects, including kinetic waves and instabilities, Landau damping, bump-on-tail instabilities and more, are therefore highly influential in tokamak plasma dynamics. Purely fluid models are inherently incapable of capturing these effects, whereas the high dimensionality in purely kinetic models render them practically intractable for most relevant purposes.

        We consider a $\delta\!f$ decomposition model, with a macroscopic fluid background and microscopic kinetic correction, both fully coupled to each other. A similar manner of discretization is proposed to that used in the recent \texttt{STRUPHY} code \cite{Holderied_Possanner_Wang_2021, Holderied_2022, Li_et_al_2023} with a finite-element model for the background and a pseudo-particle/PiC model for the correction.

        The fluid background satisfies the full, non-linear, resistive, compressible, Hall MHD equations. \cite{Laakmann_Hu_Farrell_2022} introduces finite-element(-in-space) implicit timesteppers for the incompressible analogue to this system with structure-preserving (SP) properties in the ideal case, alongside parameter-robust preconditioners. We show that these timesteppers can derive from a finite-element-in-time (FET) (and finite-element-in-space) interpretation. The benefits of this reformulation are discussed, including the derivation of timesteppers that are higher order in time, and the quantifiable dissipative SP properties in the non-ideal, resistive case.
        
        We discuss possible options for extending this FET approach to timesteppers for the compressible case.

        The kinetic corrections satisfy linearized Boltzmann equations. Using a Lénard--Bernstein collision operator, these take Fokker--Planck-like forms \cite{Fokker_1914, Planck_1917} wherein pseudo-particles in the numerical model obey the neoclassical transport equations, with particle-independent Brownian drift terms. This offers a rigorous methodology for incorporating collisions into the particle transport model, without coupling the equations of motions for each particle.
        
        Works by Chen, Chacón et al. \cite{Chen_Chacón_Barnes_2011, Chacón_Chen_Barnes_2013, Chen_Chacón_2014, Chen_Chacón_2015} have developed structure-preserving particle pushers for neoclassical transport in the Vlasov equations, derived from Crank--Nicolson integrators. We show these too can can derive from a FET interpretation, similarly offering potential extensions to higher-order-in-time particle pushers. The FET formulation is used also to consider how the stochastic drift terms can be incorporated into the pushers. Stochastic gyrokinetic expansions are also discussed.

        Different options for the numerical implementation of these schemes are considered.

        Due to the efficacy of FET in the development of SP timesteppers for both the fluid and kinetic component, we hope this approach will prove effective in the future for developing SP timesteppers for the full hybrid model. We hope this will give us the opportunity to incorporate previously inaccessible kinetic effects into the highly effective, modern, finite-element MHD models.
    \end{abstract}
    
    
    \newpage
    \tableofcontents
    
    
    \newpage
    \pagenumbering{arabic}
    %\linenumbers\renewcommand\thelinenumber{\color{black!50}\arabic{linenumber}}
            \input{0 - introduction/main.tex}
        \part{Research}
            \input{1 - low-noise PiC models/main.tex}
            \input{2 - kinetic component/main.tex}
            \input{3 - fluid component/main.tex}
            \input{4 - numerical implementation/main.tex}
        \part{Project Overview}
            \input{5 - research plan/main.tex}
            \input{6 - summary/main.tex}
    
    
    %\section{}
    \newpage
    \pagenumbering{gobble}
        \printbibliography


    \newpage
    \pagenumbering{roman}
    \appendix
        \part{Appendices}
            \input{8 - Hilbert complexes/main.tex}
            \input{9 - weak conservation proofs/main.tex}
\end{document}


\title{\BA{Title in Progress...}}
\author{Boris Andrews}
\affil{Mathematical Institute, University of Oxford}
\date{\today}


\begin{document}
    \pagenumbering{gobble}
    \maketitle
    
    
    \begin{abstract}
        Magnetic confinement reactors---in particular tokamaks---offer one of the most promising options for achieving practical nuclear fusion, with the potential to provide virtually limitless, clean energy. The theoretical and numerical modeling of tokamak plasmas is simultaneously an essential component of effective reactor design, and a great research barrier. Tokamak operational conditions exhibit comparatively low Knudsen numbers. Kinetic effects, including kinetic waves and instabilities, Landau damping, bump-on-tail instabilities and more, are therefore highly influential in tokamak plasma dynamics. Purely fluid models are inherently incapable of capturing these effects, whereas the high dimensionality in purely kinetic models render them practically intractable for most relevant purposes.

        We consider a $\delta\!f$ decomposition model, with a macroscopic fluid background and microscopic kinetic correction, both fully coupled to each other. A similar manner of discretization is proposed to that used in the recent \texttt{STRUPHY} code \cite{Holderied_Possanner_Wang_2021, Holderied_2022, Li_et_al_2023} with a finite-element model for the background and a pseudo-particle/PiC model for the correction.

        The fluid background satisfies the full, non-linear, resistive, compressible, Hall MHD equations. \cite{Laakmann_Hu_Farrell_2022} introduces finite-element(-in-space) implicit timesteppers for the incompressible analogue to this system with structure-preserving (SP) properties in the ideal case, alongside parameter-robust preconditioners. We show that these timesteppers can derive from a finite-element-in-time (FET) (and finite-element-in-space) interpretation. The benefits of this reformulation are discussed, including the derivation of timesteppers that are higher order in time, and the quantifiable dissipative SP properties in the non-ideal, resistive case.
        
        We discuss possible options for extending this FET approach to timesteppers for the compressible case.

        The kinetic corrections satisfy linearized Boltzmann equations. Using a Lénard--Bernstein collision operator, these take Fokker--Planck-like forms \cite{Fokker_1914, Planck_1917} wherein pseudo-particles in the numerical model obey the neoclassical transport equations, with particle-independent Brownian drift terms. This offers a rigorous methodology for incorporating collisions into the particle transport model, without coupling the equations of motions for each particle.
        
        Works by Chen, Chacón et al. \cite{Chen_Chacón_Barnes_2011, Chacón_Chen_Barnes_2013, Chen_Chacón_2014, Chen_Chacón_2015} have developed structure-preserving particle pushers for neoclassical transport in the Vlasov equations, derived from Crank--Nicolson integrators. We show these too can can derive from a FET interpretation, similarly offering potential extensions to higher-order-in-time particle pushers. The FET formulation is used also to consider how the stochastic drift terms can be incorporated into the pushers. Stochastic gyrokinetic expansions are also discussed.

        Different options for the numerical implementation of these schemes are considered.

        Due to the efficacy of FET in the development of SP timesteppers for both the fluid and kinetic component, we hope this approach will prove effective in the future for developing SP timesteppers for the full hybrid model. We hope this will give us the opportunity to incorporate previously inaccessible kinetic effects into the highly effective, modern, finite-element MHD models.
    \end{abstract}
    
    
    \newpage
    \tableofcontents
    
    
    \newpage
    \pagenumbering{arabic}
    %\linenumbers\renewcommand\thelinenumber{\color{black!50}\arabic{linenumber}}
            \documentclass[12pt, a4paper]{report}

\input{template/main.tex}

\title{\BA{Title in Progress...}}
\author{Boris Andrews}
\affil{Mathematical Institute, University of Oxford}
\date{\today}


\begin{document}
    \pagenumbering{gobble}
    \maketitle
    
    
    \begin{abstract}
        Magnetic confinement reactors---in particular tokamaks---offer one of the most promising options for achieving practical nuclear fusion, with the potential to provide virtually limitless, clean energy. The theoretical and numerical modeling of tokamak plasmas is simultaneously an essential component of effective reactor design, and a great research barrier. Tokamak operational conditions exhibit comparatively low Knudsen numbers. Kinetic effects, including kinetic waves and instabilities, Landau damping, bump-on-tail instabilities and more, are therefore highly influential in tokamak plasma dynamics. Purely fluid models are inherently incapable of capturing these effects, whereas the high dimensionality in purely kinetic models render them practically intractable for most relevant purposes.

        We consider a $\delta\!f$ decomposition model, with a macroscopic fluid background and microscopic kinetic correction, both fully coupled to each other. A similar manner of discretization is proposed to that used in the recent \texttt{STRUPHY} code \cite{Holderied_Possanner_Wang_2021, Holderied_2022, Li_et_al_2023} with a finite-element model for the background and a pseudo-particle/PiC model for the correction.

        The fluid background satisfies the full, non-linear, resistive, compressible, Hall MHD equations. \cite{Laakmann_Hu_Farrell_2022} introduces finite-element(-in-space) implicit timesteppers for the incompressible analogue to this system with structure-preserving (SP) properties in the ideal case, alongside parameter-robust preconditioners. We show that these timesteppers can derive from a finite-element-in-time (FET) (and finite-element-in-space) interpretation. The benefits of this reformulation are discussed, including the derivation of timesteppers that are higher order in time, and the quantifiable dissipative SP properties in the non-ideal, resistive case.
        
        We discuss possible options for extending this FET approach to timesteppers for the compressible case.

        The kinetic corrections satisfy linearized Boltzmann equations. Using a Lénard--Bernstein collision operator, these take Fokker--Planck-like forms \cite{Fokker_1914, Planck_1917} wherein pseudo-particles in the numerical model obey the neoclassical transport equations, with particle-independent Brownian drift terms. This offers a rigorous methodology for incorporating collisions into the particle transport model, without coupling the equations of motions for each particle.
        
        Works by Chen, Chacón et al. \cite{Chen_Chacón_Barnes_2011, Chacón_Chen_Barnes_2013, Chen_Chacón_2014, Chen_Chacón_2015} have developed structure-preserving particle pushers for neoclassical transport in the Vlasov equations, derived from Crank--Nicolson integrators. We show these too can can derive from a FET interpretation, similarly offering potential extensions to higher-order-in-time particle pushers. The FET formulation is used also to consider how the stochastic drift terms can be incorporated into the pushers. Stochastic gyrokinetic expansions are also discussed.

        Different options for the numerical implementation of these schemes are considered.

        Due to the efficacy of FET in the development of SP timesteppers for both the fluid and kinetic component, we hope this approach will prove effective in the future for developing SP timesteppers for the full hybrid model. We hope this will give us the opportunity to incorporate previously inaccessible kinetic effects into the highly effective, modern, finite-element MHD models.
    \end{abstract}
    
    
    \newpage
    \tableofcontents
    
    
    \newpage
    \pagenumbering{arabic}
    %\linenumbers\renewcommand\thelinenumber{\color{black!50}\arabic{linenumber}}
            \input{0 - introduction/main.tex}
        \part{Research}
            \input{1 - low-noise PiC models/main.tex}
            \input{2 - kinetic component/main.tex}
            \input{3 - fluid component/main.tex}
            \input{4 - numerical implementation/main.tex}
        \part{Project Overview}
            \input{5 - research plan/main.tex}
            \input{6 - summary/main.tex}
    
    
    %\section{}
    \newpage
    \pagenumbering{gobble}
        \printbibliography


    \newpage
    \pagenumbering{roman}
    \appendix
        \part{Appendices}
            \input{8 - Hilbert complexes/main.tex}
            \input{9 - weak conservation proofs/main.tex}
\end{document}

        \part{Research}
            \documentclass[12pt, a4paper]{report}

\input{template/main.tex}

\title{\BA{Title in Progress...}}
\author{Boris Andrews}
\affil{Mathematical Institute, University of Oxford}
\date{\today}


\begin{document}
    \pagenumbering{gobble}
    \maketitle
    
    
    \begin{abstract}
        Magnetic confinement reactors---in particular tokamaks---offer one of the most promising options for achieving practical nuclear fusion, with the potential to provide virtually limitless, clean energy. The theoretical and numerical modeling of tokamak plasmas is simultaneously an essential component of effective reactor design, and a great research barrier. Tokamak operational conditions exhibit comparatively low Knudsen numbers. Kinetic effects, including kinetic waves and instabilities, Landau damping, bump-on-tail instabilities and more, are therefore highly influential in tokamak plasma dynamics. Purely fluid models are inherently incapable of capturing these effects, whereas the high dimensionality in purely kinetic models render them practically intractable for most relevant purposes.

        We consider a $\delta\!f$ decomposition model, with a macroscopic fluid background and microscopic kinetic correction, both fully coupled to each other. A similar manner of discretization is proposed to that used in the recent \texttt{STRUPHY} code \cite{Holderied_Possanner_Wang_2021, Holderied_2022, Li_et_al_2023} with a finite-element model for the background and a pseudo-particle/PiC model for the correction.

        The fluid background satisfies the full, non-linear, resistive, compressible, Hall MHD equations. \cite{Laakmann_Hu_Farrell_2022} introduces finite-element(-in-space) implicit timesteppers for the incompressible analogue to this system with structure-preserving (SP) properties in the ideal case, alongside parameter-robust preconditioners. We show that these timesteppers can derive from a finite-element-in-time (FET) (and finite-element-in-space) interpretation. The benefits of this reformulation are discussed, including the derivation of timesteppers that are higher order in time, and the quantifiable dissipative SP properties in the non-ideal, resistive case.
        
        We discuss possible options for extending this FET approach to timesteppers for the compressible case.

        The kinetic corrections satisfy linearized Boltzmann equations. Using a Lénard--Bernstein collision operator, these take Fokker--Planck-like forms \cite{Fokker_1914, Planck_1917} wherein pseudo-particles in the numerical model obey the neoclassical transport equations, with particle-independent Brownian drift terms. This offers a rigorous methodology for incorporating collisions into the particle transport model, without coupling the equations of motions for each particle.
        
        Works by Chen, Chacón et al. \cite{Chen_Chacón_Barnes_2011, Chacón_Chen_Barnes_2013, Chen_Chacón_2014, Chen_Chacón_2015} have developed structure-preserving particle pushers for neoclassical transport in the Vlasov equations, derived from Crank--Nicolson integrators. We show these too can can derive from a FET interpretation, similarly offering potential extensions to higher-order-in-time particle pushers. The FET formulation is used also to consider how the stochastic drift terms can be incorporated into the pushers. Stochastic gyrokinetic expansions are also discussed.

        Different options for the numerical implementation of these schemes are considered.

        Due to the efficacy of FET in the development of SP timesteppers for both the fluid and kinetic component, we hope this approach will prove effective in the future for developing SP timesteppers for the full hybrid model. We hope this will give us the opportunity to incorporate previously inaccessible kinetic effects into the highly effective, modern, finite-element MHD models.
    \end{abstract}
    
    
    \newpage
    \tableofcontents
    
    
    \newpage
    \pagenumbering{arabic}
    %\linenumbers\renewcommand\thelinenumber{\color{black!50}\arabic{linenumber}}
            \input{0 - introduction/main.tex}
        \part{Research}
            \input{1 - low-noise PiC models/main.tex}
            \input{2 - kinetic component/main.tex}
            \input{3 - fluid component/main.tex}
            \input{4 - numerical implementation/main.tex}
        \part{Project Overview}
            \input{5 - research plan/main.tex}
            \input{6 - summary/main.tex}
    
    
    %\section{}
    \newpage
    \pagenumbering{gobble}
        \printbibliography


    \newpage
    \pagenumbering{roman}
    \appendix
        \part{Appendices}
            \input{8 - Hilbert complexes/main.tex}
            \input{9 - weak conservation proofs/main.tex}
\end{document}

            \documentclass[12pt, a4paper]{report}

\input{template/main.tex}

\title{\BA{Title in Progress...}}
\author{Boris Andrews}
\affil{Mathematical Institute, University of Oxford}
\date{\today}


\begin{document}
    \pagenumbering{gobble}
    \maketitle
    
    
    \begin{abstract}
        Magnetic confinement reactors---in particular tokamaks---offer one of the most promising options for achieving practical nuclear fusion, with the potential to provide virtually limitless, clean energy. The theoretical and numerical modeling of tokamak plasmas is simultaneously an essential component of effective reactor design, and a great research barrier. Tokamak operational conditions exhibit comparatively low Knudsen numbers. Kinetic effects, including kinetic waves and instabilities, Landau damping, bump-on-tail instabilities and more, are therefore highly influential in tokamak plasma dynamics. Purely fluid models are inherently incapable of capturing these effects, whereas the high dimensionality in purely kinetic models render them practically intractable for most relevant purposes.

        We consider a $\delta\!f$ decomposition model, with a macroscopic fluid background and microscopic kinetic correction, both fully coupled to each other. A similar manner of discretization is proposed to that used in the recent \texttt{STRUPHY} code \cite{Holderied_Possanner_Wang_2021, Holderied_2022, Li_et_al_2023} with a finite-element model for the background and a pseudo-particle/PiC model for the correction.

        The fluid background satisfies the full, non-linear, resistive, compressible, Hall MHD equations. \cite{Laakmann_Hu_Farrell_2022} introduces finite-element(-in-space) implicit timesteppers for the incompressible analogue to this system with structure-preserving (SP) properties in the ideal case, alongside parameter-robust preconditioners. We show that these timesteppers can derive from a finite-element-in-time (FET) (and finite-element-in-space) interpretation. The benefits of this reformulation are discussed, including the derivation of timesteppers that are higher order in time, and the quantifiable dissipative SP properties in the non-ideal, resistive case.
        
        We discuss possible options for extending this FET approach to timesteppers for the compressible case.

        The kinetic corrections satisfy linearized Boltzmann equations. Using a Lénard--Bernstein collision operator, these take Fokker--Planck-like forms \cite{Fokker_1914, Planck_1917} wherein pseudo-particles in the numerical model obey the neoclassical transport equations, with particle-independent Brownian drift terms. This offers a rigorous methodology for incorporating collisions into the particle transport model, without coupling the equations of motions for each particle.
        
        Works by Chen, Chacón et al. \cite{Chen_Chacón_Barnes_2011, Chacón_Chen_Barnes_2013, Chen_Chacón_2014, Chen_Chacón_2015} have developed structure-preserving particle pushers for neoclassical transport in the Vlasov equations, derived from Crank--Nicolson integrators. We show these too can can derive from a FET interpretation, similarly offering potential extensions to higher-order-in-time particle pushers. The FET formulation is used also to consider how the stochastic drift terms can be incorporated into the pushers. Stochastic gyrokinetic expansions are also discussed.

        Different options for the numerical implementation of these schemes are considered.

        Due to the efficacy of FET in the development of SP timesteppers for both the fluid and kinetic component, we hope this approach will prove effective in the future for developing SP timesteppers for the full hybrid model. We hope this will give us the opportunity to incorporate previously inaccessible kinetic effects into the highly effective, modern, finite-element MHD models.
    \end{abstract}
    
    
    \newpage
    \tableofcontents
    
    
    \newpage
    \pagenumbering{arabic}
    %\linenumbers\renewcommand\thelinenumber{\color{black!50}\arabic{linenumber}}
            \input{0 - introduction/main.tex}
        \part{Research}
            \input{1 - low-noise PiC models/main.tex}
            \input{2 - kinetic component/main.tex}
            \input{3 - fluid component/main.tex}
            \input{4 - numerical implementation/main.tex}
        \part{Project Overview}
            \input{5 - research plan/main.tex}
            \input{6 - summary/main.tex}
    
    
    %\section{}
    \newpage
    \pagenumbering{gobble}
        \printbibliography


    \newpage
    \pagenumbering{roman}
    \appendix
        \part{Appendices}
            \input{8 - Hilbert complexes/main.tex}
            \input{9 - weak conservation proofs/main.tex}
\end{document}

            \documentclass[12pt, a4paper]{report}

\input{template/main.tex}

\title{\BA{Title in Progress...}}
\author{Boris Andrews}
\affil{Mathematical Institute, University of Oxford}
\date{\today}


\begin{document}
    \pagenumbering{gobble}
    \maketitle
    
    
    \begin{abstract}
        Magnetic confinement reactors---in particular tokamaks---offer one of the most promising options for achieving practical nuclear fusion, with the potential to provide virtually limitless, clean energy. The theoretical and numerical modeling of tokamak plasmas is simultaneously an essential component of effective reactor design, and a great research barrier. Tokamak operational conditions exhibit comparatively low Knudsen numbers. Kinetic effects, including kinetic waves and instabilities, Landau damping, bump-on-tail instabilities and more, are therefore highly influential in tokamak plasma dynamics. Purely fluid models are inherently incapable of capturing these effects, whereas the high dimensionality in purely kinetic models render them practically intractable for most relevant purposes.

        We consider a $\delta\!f$ decomposition model, with a macroscopic fluid background and microscopic kinetic correction, both fully coupled to each other. A similar manner of discretization is proposed to that used in the recent \texttt{STRUPHY} code \cite{Holderied_Possanner_Wang_2021, Holderied_2022, Li_et_al_2023} with a finite-element model for the background and a pseudo-particle/PiC model for the correction.

        The fluid background satisfies the full, non-linear, resistive, compressible, Hall MHD equations. \cite{Laakmann_Hu_Farrell_2022} introduces finite-element(-in-space) implicit timesteppers for the incompressible analogue to this system with structure-preserving (SP) properties in the ideal case, alongside parameter-robust preconditioners. We show that these timesteppers can derive from a finite-element-in-time (FET) (and finite-element-in-space) interpretation. The benefits of this reformulation are discussed, including the derivation of timesteppers that are higher order in time, and the quantifiable dissipative SP properties in the non-ideal, resistive case.
        
        We discuss possible options for extending this FET approach to timesteppers for the compressible case.

        The kinetic corrections satisfy linearized Boltzmann equations. Using a Lénard--Bernstein collision operator, these take Fokker--Planck-like forms \cite{Fokker_1914, Planck_1917} wherein pseudo-particles in the numerical model obey the neoclassical transport equations, with particle-independent Brownian drift terms. This offers a rigorous methodology for incorporating collisions into the particle transport model, without coupling the equations of motions for each particle.
        
        Works by Chen, Chacón et al. \cite{Chen_Chacón_Barnes_2011, Chacón_Chen_Barnes_2013, Chen_Chacón_2014, Chen_Chacón_2015} have developed structure-preserving particle pushers for neoclassical transport in the Vlasov equations, derived from Crank--Nicolson integrators. We show these too can can derive from a FET interpretation, similarly offering potential extensions to higher-order-in-time particle pushers. The FET formulation is used also to consider how the stochastic drift terms can be incorporated into the pushers. Stochastic gyrokinetic expansions are also discussed.

        Different options for the numerical implementation of these schemes are considered.

        Due to the efficacy of FET in the development of SP timesteppers for both the fluid and kinetic component, we hope this approach will prove effective in the future for developing SP timesteppers for the full hybrid model. We hope this will give us the opportunity to incorporate previously inaccessible kinetic effects into the highly effective, modern, finite-element MHD models.
    \end{abstract}
    
    
    \newpage
    \tableofcontents
    
    
    \newpage
    \pagenumbering{arabic}
    %\linenumbers\renewcommand\thelinenumber{\color{black!50}\arabic{linenumber}}
            \input{0 - introduction/main.tex}
        \part{Research}
            \input{1 - low-noise PiC models/main.tex}
            \input{2 - kinetic component/main.tex}
            \input{3 - fluid component/main.tex}
            \input{4 - numerical implementation/main.tex}
        \part{Project Overview}
            \input{5 - research plan/main.tex}
            \input{6 - summary/main.tex}
    
    
    %\section{}
    \newpage
    \pagenumbering{gobble}
        \printbibliography


    \newpage
    \pagenumbering{roman}
    \appendix
        \part{Appendices}
            \input{8 - Hilbert complexes/main.tex}
            \input{9 - weak conservation proofs/main.tex}
\end{document}

            \documentclass[12pt, a4paper]{report}

\input{template/main.tex}

\title{\BA{Title in Progress...}}
\author{Boris Andrews}
\affil{Mathematical Institute, University of Oxford}
\date{\today}


\begin{document}
    \pagenumbering{gobble}
    \maketitle
    
    
    \begin{abstract}
        Magnetic confinement reactors---in particular tokamaks---offer one of the most promising options for achieving practical nuclear fusion, with the potential to provide virtually limitless, clean energy. The theoretical and numerical modeling of tokamak plasmas is simultaneously an essential component of effective reactor design, and a great research barrier. Tokamak operational conditions exhibit comparatively low Knudsen numbers. Kinetic effects, including kinetic waves and instabilities, Landau damping, bump-on-tail instabilities and more, are therefore highly influential in tokamak plasma dynamics. Purely fluid models are inherently incapable of capturing these effects, whereas the high dimensionality in purely kinetic models render them practically intractable for most relevant purposes.

        We consider a $\delta\!f$ decomposition model, with a macroscopic fluid background and microscopic kinetic correction, both fully coupled to each other. A similar manner of discretization is proposed to that used in the recent \texttt{STRUPHY} code \cite{Holderied_Possanner_Wang_2021, Holderied_2022, Li_et_al_2023} with a finite-element model for the background and a pseudo-particle/PiC model for the correction.

        The fluid background satisfies the full, non-linear, resistive, compressible, Hall MHD equations. \cite{Laakmann_Hu_Farrell_2022} introduces finite-element(-in-space) implicit timesteppers for the incompressible analogue to this system with structure-preserving (SP) properties in the ideal case, alongside parameter-robust preconditioners. We show that these timesteppers can derive from a finite-element-in-time (FET) (and finite-element-in-space) interpretation. The benefits of this reformulation are discussed, including the derivation of timesteppers that are higher order in time, and the quantifiable dissipative SP properties in the non-ideal, resistive case.
        
        We discuss possible options for extending this FET approach to timesteppers for the compressible case.

        The kinetic corrections satisfy linearized Boltzmann equations. Using a Lénard--Bernstein collision operator, these take Fokker--Planck-like forms \cite{Fokker_1914, Planck_1917} wherein pseudo-particles in the numerical model obey the neoclassical transport equations, with particle-independent Brownian drift terms. This offers a rigorous methodology for incorporating collisions into the particle transport model, without coupling the equations of motions for each particle.
        
        Works by Chen, Chacón et al. \cite{Chen_Chacón_Barnes_2011, Chacón_Chen_Barnes_2013, Chen_Chacón_2014, Chen_Chacón_2015} have developed structure-preserving particle pushers for neoclassical transport in the Vlasov equations, derived from Crank--Nicolson integrators. We show these too can can derive from a FET interpretation, similarly offering potential extensions to higher-order-in-time particle pushers. The FET formulation is used also to consider how the stochastic drift terms can be incorporated into the pushers. Stochastic gyrokinetic expansions are also discussed.

        Different options for the numerical implementation of these schemes are considered.

        Due to the efficacy of FET in the development of SP timesteppers for both the fluid and kinetic component, we hope this approach will prove effective in the future for developing SP timesteppers for the full hybrid model. We hope this will give us the opportunity to incorporate previously inaccessible kinetic effects into the highly effective, modern, finite-element MHD models.
    \end{abstract}
    
    
    \newpage
    \tableofcontents
    
    
    \newpage
    \pagenumbering{arabic}
    %\linenumbers\renewcommand\thelinenumber{\color{black!50}\arabic{linenumber}}
            \input{0 - introduction/main.tex}
        \part{Research}
            \input{1 - low-noise PiC models/main.tex}
            \input{2 - kinetic component/main.tex}
            \input{3 - fluid component/main.tex}
            \input{4 - numerical implementation/main.tex}
        \part{Project Overview}
            \input{5 - research plan/main.tex}
            \input{6 - summary/main.tex}
    
    
    %\section{}
    \newpage
    \pagenumbering{gobble}
        \printbibliography


    \newpage
    \pagenumbering{roman}
    \appendix
        \part{Appendices}
            \input{8 - Hilbert complexes/main.tex}
            \input{9 - weak conservation proofs/main.tex}
\end{document}

        \part{Project Overview}
            \documentclass[12pt, a4paper]{report}

\input{template/main.tex}

\title{\BA{Title in Progress...}}
\author{Boris Andrews}
\affil{Mathematical Institute, University of Oxford}
\date{\today}


\begin{document}
    \pagenumbering{gobble}
    \maketitle
    
    
    \begin{abstract}
        Magnetic confinement reactors---in particular tokamaks---offer one of the most promising options for achieving practical nuclear fusion, with the potential to provide virtually limitless, clean energy. The theoretical and numerical modeling of tokamak plasmas is simultaneously an essential component of effective reactor design, and a great research barrier. Tokamak operational conditions exhibit comparatively low Knudsen numbers. Kinetic effects, including kinetic waves and instabilities, Landau damping, bump-on-tail instabilities and more, are therefore highly influential in tokamak plasma dynamics. Purely fluid models are inherently incapable of capturing these effects, whereas the high dimensionality in purely kinetic models render them practically intractable for most relevant purposes.

        We consider a $\delta\!f$ decomposition model, with a macroscopic fluid background and microscopic kinetic correction, both fully coupled to each other. A similar manner of discretization is proposed to that used in the recent \texttt{STRUPHY} code \cite{Holderied_Possanner_Wang_2021, Holderied_2022, Li_et_al_2023} with a finite-element model for the background and a pseudo-particle/PiC model for the correction.

        The fluid background satisfies the full, non-linear, resistive, compressible, Hall MHD equations. \cite{Laakmann_Hu_Farrell_2022} introduces finite-element(-in-space) implicit timesteppers for the incompressible analogue to this system with structure-preserving (SP) properties in the ideal case, alongside parameter-robust preconditioners. We show that these timesteppers can derive from a finite-element-in-time (FET) (and finite-element-in-space) interpretation. The benefits of this reformulation are discussed, including the derivation of timesteppers that are higher order in time, and the quantifiable dissipative SP properties in the non-ideal, resistive case.
        
        We discuss possible options for extending this FET approach to timesteppers for the compressible case.

        The kinetic corrections satisfy linearized Boltzmann equations. Using a Lénard--Bernstein collision operator, these take Fokker--Planck-like forms \cite{Fokker_1914, Planck_1917} wherein pseudo-particles in the numerical model obey the neoclassical transport equations, with particle-independent Brownian drift terms. This offers a rigorous methodology for incorporating collisions into the particle transport model, without coupling the equations of motions for each particle.
        
        Works by Chen, Chacón et al. \cite{Chen_Chacón_Barnes_2011, Chacón_Chen_Barnes_2013, Chen_Chacón_2014, Chen_Chacón_2015} have developed structure-preserving particle pushers for neoclassical transport in the Vlasov equations, derived from Crank--Nicolson integrators. We show these too can can derive from a FET interpretation, similarly offering potential extensions to higher-order-in-time particle pushers. The FET formulation is used also to consider how the stochastic drift terms can be incorporated into the pushers. Stochastic gyrokinetic expansions are also discussed.

        Different options for the numerical implementation of these schemes are considered.

        Due to the efficacy of FET in the development of SP timesteppers for both the fluid and kinetic component, we hope this approach will prove effective in the future for developing SP timesteppers for the full hybrid model. We hope this will give us the opportunity to incorporate previously inaccessible kinetic effects into the highly effective, modern, finite-element MHD models.
    \end{abstract}
    
    
    \newpage
    \tableofcontents
    
    
    \newpage
    \pagenumbering{arabic}
    %\linenumbers\renewcommand\thelinenumber{\color{black!50}\arabic{linenumber}}
            \input{0 - introduction/main.tex}
        \part{Research}
            \input{1 - low-noise PiC models/main.tex}
            \input{2 - kinetic component/main.tex}
            \input{3 - fluid component/main.tex}
            \input{4 - numerical implementation/main.tex}
        \part{Project Overview}
            \input{5 - research plan/main.tex}
            \input{6 - summary/main.tex}
    
    
    %\section{}
    \newpage
    \pagenumbering{gobble}
        \printbibliography


    \newpage
    \pagenumbering{roman}
    \appendix
        \part{Appendices}
            \input{8 - Hilbert complexes/main.tex}
            \input{9 - weak conservation proofs/main.tex}
\end{document}

            \documentclass[12pt, a4paper]{report}

\input{template/main.tex}

\title{\BA{Title in Progress...}}
\author{Boris Andrews}
\affil{Mathematical Institute, University of Oxford}
\date{\today}


\begin{document}
    \pagenumbering{gobble}
    \maketitle
    
    
    \begin{abstract}
        Magnetic confinement reactors---in particular tokamaks---offer one of the most promising options for achieving practical nuclear fusion, with the potential to provide virtually limitless, clean energy. The theoretical and numerical modeling of tokamak plasmas is simultaneously an essential component of effective reactor design, and a great research barrier. Tokamak operational conditions exhibit comparatively low Knudsen numbers. Kinetic effects, including kinetic waves and instabilities, Landau damping, bump-on-tail instabilities and more, are therefore highly influential in tokamak plasma dynamics. Purely fluid models are inherently incapable of capturing these effects, whereas the high dimensionality in purely kinetic models render them practically intractable for most relevant purposes.

        We consider a $\delta\!f$ decomposition model, with a macroscopic fluid background and microscopic kinetic correction, both fully coupled to each other. A similar manner of discretization is proposed to that used in the recent \texttt{STRUPHY} code \cite{Holderied_Possanner_Wang_2021, Holderied_2022, Li_et_al_2023} with a finite-element model for the background and a pseudo-particle/PiC model for the correction.

        The fluid background satisfies the full, non-linear, resistive, compressible, Hall MHD equations. \cite{Laakmann_Hu_Farrell_2022} introduces finite-element(-in-space) implicit timesteppers for the incompressible analogue to this system with structure-preserving (SP) properties in the ideal case, alongside parameter-robust preconditioners. We show that these timesteppers can derive from a finite-element-in-time (FET) (and finite-element-in-space) interpretation. The benefits of this reformulation are discussed, including the derivation of timesteppers that are higher order in time, and the quantifiable dissipative SP properties in the non-ideal, resistive case.
        
        We discuss possible options for extending this FET approach to timesteppers for the compressible case.

        The kinetic corrections satisfy linearized Boltzmann equations. Using a Lénard--Bernstein collision operator, these take Fokker--Planck-like forms \cite{Fokker_1914, Planck_1917} wherein pseudo-particles in the numerical model obey the neoclassical transport equations, with particle-independent Brownian drift terms. This offers a rigorous methodology for incorporating collisions into the particle transport model, without coupling the equations of motions for each particle.
        
        Works by Chen, Chacón et al. \cite{Chen_Chacón_Barnes_2011, Chacón_Chen_Barnes_2013, Chen_Chacón_2014, Chen_Chacón_2015} have developed structure-preserving particle pushers for neoclassical transport in the Vlasov equations, derived from Crank--Nicolson integrators. We show these too can can derive from a FET interpretation, similarly offering potential extensions to higher-order-in-time particle pushers. The FET formulation is used also to consider how the stochastic drift terms can be incorporated into the pushers. Stochastic gyrokinetic expansions are also discussed.

        Different options for the numerical implementation of these schemes are considered.

        Due to the efficacy of FET in the development of SP timesteppers for both the fluid and kinetic component, we hope this approach will prove effective in the future for developing SP timesteppers for the full hybrid model. We hope this will give us the opportunity to incorporate previously inaccessible kinetic effects into the highly effective, modern, finite-element MHD models.
    \end{abstract}
    
    
    \newpage
    \tableofcontents
    
    
    \newpage
    \pagenumbering{arabic}
    %\linenumbers\renewcommand\thelinenumber{\color{black!50}\arabic{linenumber}}
            \input{0 - introduction/main.tex}
        \part{Research}
            \input{1 - low-noise PiC models/main.tex}
            \input{2 - kinetic component/main.tex}
            \input{3 - fluid component/main.tex}
            \input{4 - numerical implementation/main.tex}
        \part{Project Overview}
            \input{5 - research plan/main.tex}
            \input{6 - summary/main.tex}
    
    
    %\section{}
    \newpage
    \pagenumbering{gobble}
        \printbibliography


    \newpage
    \pagenumbering{roman}
    \appendix
        \part{Appendices}
            \input{8 - Hilbert complexes/main.tex}
            \input{9 - weak conservation proofs/main.tex}
\end{document}

    
    
    %\section{}
    \newpage
    \pagenumbering{gobble}
        \printbibliography


    \newpage
    \pagenumbering{roman}
    \appendix
        \part{Appendices}
            \documentclass[12pt, a4paper]{report}

\input{template/main.tex}

\title{\BA{Title in Progress...}}
\author{Boris Andrews}
\affil{Mathematical Institute, University of Oxford}
\date{\today}


\begin{document}
    \pagenumbering{gobble}
    \maketitle
    
    
    \begin{abstract}
        Magnetic confinement reactors---in particular tokamaks---offer one of the most promising options for achieving practical nuclear fusion, with the potential to provide virtually limitless, clean energy. The theoretical and numerical modeling of tokamak plasmas is simultaneously an essential component of effective reactor design, and a great research barrier. Tokamak operational conditions exhibit comparatively low Knudsen numbers. Kinetic effects, including kinetic waves and instabilities, Landau damping, bump-on-tail instabilities and more, are therefore highly influential in tokamak plasma dynamics. Purely fluid models are inherently incapable of capturing these effects, whereas the high dimensionality in purely kinetic models render them practically intractable for most relevant purposes.

        We consider a $\delta\!f$ decomposition model, with a macroscopic fluid background and microscopic kinetic correction, both fully coupled to each other. A similar manner of discretization is proposed to that used in the recent \texttt{STRUPHY} code \cite{Holderied_Possanner_Wang_2021, Holderied_2022, Li_et_al_2023} with a finite-element model for the background and a pseudo-particle/PiC model for the correction.

        The fluid background satisfies the full, non-linear, resistive, compressible, Hall MHD equations. \cite{Laakmann_Hu_Farrell_2022} introduces finite-element(-in-space) implicit timesteppers for the incompressible analogue to this system with structure-preserving (SP) properties in the ideal case, alongside parameter-robust preconditioners. We show that these timesteppers can derive from a finite-element-in-time (FET) (and finite-element-in-space) interpretation. The benefits of this reformulation are discussed, including the derivation of timesteppers that are higher order in time, and the quantifiable dissipative SP properties in the non-ideal, resistive case.
        
        We discuss possible options for extending this FET approach to timesteppers for the compressible case.

        The kinetic corrections satisfy linearized Boltzmann equations. Using a Lénard--Bernstein collision operator, these take Fokker--Planck-like forms \cite{Fokker_1914, Planck_1917} wherein pseudo-particles in the numerical model obey the neoclassical transport equations, with particle-independent Brownian drift terms. This offers a rigorous methodology for incorporating collisions into the particle transport model, without coupling the equations of motions for each particle.
        
        Works by Chen, Chacón et al. \cite{Chen_Chacón_Barnes_2011, Chacón_Chen_Barnes_2013, Chen_Chacón_2014, Chen_Chacón_2015} have developed structure-preserving particle pushers for neoclassical transport in the Vlasov equations, derived from Crank--Nicolson integrators. We show these too can can derive from a FET interpretation, similarly offering potential extensions to higher-order-in-time particle pushers. The FET formulation is used also to consider how the stochastic drift terms can be incorporated into the pushers. Stochastic gyrokinetic expansions are also discussed.

        Different options for the numerical implementation of these schemes are considered.

        Due to the efficacy of FET in the development of SP timesteppers for both the fluid and kinetic component, we hope this approach will prove effective in the future for developing SP timesteppers for the full hybrid model. We hope this will give us the opportunity to incorporate previously inaccessible kinetic effects into the highly effective, modern, finite-element MHD models.
    \end{abstract}
    
    
    \newpage
    \tableofcontents
    
    
    \newpage
    \pagenumbering{arabic}
    %\linenumbers\renewcommand\thelinenumber{\color{black!50}\arabic{linenumber}}
            \input{0 - introduction/main.tex}
        \part{Research}
            \input{1 - low-noise PiC models/main.tex}
            \input{2 - kinetic component/main.tex}
            \input{3 - fluid component/main.tex}
            \input{4 - numerical implementation/main.tex}
        \part{Project Overview}
            \input{5 - research plan/main.tex}
            \input{6 - summary/main.tex}
    
    
    %\section{}
    \newpage
    \pagenumbering{gobble}
        \printbibliography


    \newpage
    \pagenumbering{roman}
    \appendix
        \part{Appendices}
            \input{8 - Hilbert complexes/main.tex}
            \input{9 - weak conservation proofs/main.tex}
\end{document}

            \documentclass[12pt, a4paper]{report}

\input{template/main.tex}

\title{\BA{Title in Progress...}}
\author{Boris Andrews}
\affil{Mathematical Institute, University of Oxford}
\date{\today}


\begin{document}
    \pagenumbering{gobble}
    \maketitle
    
    
    \begin{abstract}
        Magnetic confinement reactors---in particular tokamaks---offer one of the most promising options for achieving practical nuclear fusion, with the potential to provide virtually limitless, clean energy. The theoretical and numerical modeling of tokamak plasmas is simultaneously an essential component of effective reactor design, and a great research barrier. Tokamak operational conditions exhibit comparatively low Knudsen numbers. Kinetic effects, including kinetic waves and instabilities, Landau damping, bump-on-tail instabilities and more, are therefore highly influential in tokamak plasma dynamics. Purely fluid models are inherently incapable of capturing these effects, whereas the high dimensionality in purely kinetic models render them practically intractable for most relevant purposes.

        We consider a $\delta\!f$ decomposition model, with a macroscopic fluid background and microscopic kinetic correction, both fully coupled to each other. A similar manner of discretization is proposed to that used in the recent \texttt{STRUPHY} code \cite{Holderied_Possanner_Wang_2021, Holderied_2022, Li_et_al_2023} with a finite-element model for the background and a pseudo-particle/PiC model for the correction.

        The fluid background satisfies the full, non-linear, resistive, compressible, Hall MHD equations. \cite{Laakmann_Hu_Farrell_2022} introduces finite-element(-in-space) implicit timesteppers for the incompressible analogue to this system with structure-preserving (SP) properties in the ideal case, alongside parameter-robust preconditioners. We show that these timesteppers can derive from a finite-element-in-time (FET) (and finite-element-in-space) interpretation. The benefits of this reformulation are discussed, including the derivation of timesteppers that are higher order in time, and the quantifiable dissipative SP properties in the non-ideal, resistive case.
        
        We discuss possible options for extending this FET approach to timesteppers for the compressible case.

        The kinetic corrections satisfy linearized Boltzmann equations. Using a Lénard--Bernstein collision operator, these take Fokker--Planck-like forms \cite{Fokker_1914, Planck_1917} wherein pseudo-particles in the numerical model obey the neoclassical transport equations, with particle-independent Brownian drift terms. This offers a rigorous methodology for incorporating collisions into the particle transport model, without coupling the equations of motions for each particle.
        
        Works by Chen, Chacón et al. \cite{Chen_Chacón_Barnes_2011, Chacón_Chen_Barnes_2013, Chen_Chacón_2014, Chen_Chacón_2015} have developed structure-preserving particle pushers for neoclassical transport in the Vlasov equations, derived from Crank--Nicolson integrators. We show these too can can derive from a FET interpretation, similarly offering potential extensions to higher-order-in-time particle pushers. The FET formulation is used also to consider how the stochastic drift terms can be incorporated into the pushers. Stochastic gyrokinetic expansions are also discussed.

        Different options for the numerical implementation of these schemes are considered.

        Due to the efficacy of FET in the development of SP timesteppers for both the fluid and kinetic component, we hope this approach will prove effective in the future for developing SP timesteppers for the full hybrid model. We hope this will give us the opportunity to incorporate previously inaccessible kinetic effects into the highly effective, modern, finite-element MHD models.
    \end{abstract}
    
    
    \newpage
    \tableofcontents
    
    
    \newpage
    \pagenumbering{arabic}
    %\linenumbers\renewcommand\thelinenumber{\color{black!50}\arabic{linenumber}}
            \input{0 - introduction/main.tex}
        \part{Research}
            \input{1 - low-noise PiC models/main.tex}
            \input{2 - kinetic component/main.tex}
            \input{3 - fluid component/main.tex}
            \input{4 - numerical implementation/main.tex}
        \part{Project Overview}
            \input{5 - research plan/main.tex}
            \input{6 - summary/main.tex}
    
    
    %\section{}
    \newpage
    \pagenumbering{gobble}
        \printbibliography


    \newpage
    \pagenumbering{roman}
    \appendix
        \part{Appendices}
            \input{8 - Hilbert complexes/main.tex}
            \input{9 - weak conservation proofs/main.tex}
\end{document}

\end{document}

            \documentclass[12pt, a4paper]{report}

\documentclass[12pt, a4paper]{report}

\input{template/main.tex}

\title{\BA{Title in Progress...}}
\author{Boris Andrews}
\affil{Mathematical Institute, University of Oxford}
\date{\today}


\begin{document}
    \pagenumbering{gobble}
    \maketitle
    
    
    \begin{abstract}
        Magnetic confinement reactors---in particular tokamaks---offer one of the most promising options for achieving practical nuclear fusion, with the potential to provide virtually limitless, clean energy. The theoretical and numerical modeling of tokamak plasmas is simultaneously an essential component of effective reactor design, and a great research barrier. Tokamak operational conditions exhibit comparatively low Knudsen numbers. Kinetic effects, including kinetic waves and instabilities, Landau damping, bump-on-tail instabilities and more, are therefore highly influential in tokamak plasma dynamics. Purely fluid models are inherently incapable of capturing these effects, whereas the high dimensionality in purely kinetic models render them practically intractable for most relevant purposes.

        We consider a $\delta\!f$ decomposition model, with a macroscopic fluid background and microscopic kinetic correction, both fully coupled to each other. A similar manner of discretization is proposed to that used in the recent \texttt{STRUPHY} code \cite{Holderied_Possanner_Wang_2021, Holderied_2022, Li_et_al_2023} with a finite-element model for the background and a pseudo-particle/PiC model for the correction.

        The fluid background satisfies the full, non-linear, resistive, compressible, Hall MHD equations. \cite{Laakmann_Hu_Farrell_2022} introduces finite-element(-in-space) implicit timesteppers for the incompressible analogue to this system with structure-preserving (SP) properties in the ideal case, alongside parameter-robust preconditioners. We show that these timesteppers can derive from a finite-element-in-time (FET) (and finite-element-in-space) interpretation. The benefits of this reformulation are discussed, including the derivation of timesteppers that are higher order in time, and the quantifiable dissipative SP properties in the non-ideal, resistive case.
        
        We discuss possible options for extending this FET approach to timesteppers for the compressible case.

        The kinetic corrections satisfy linearized Boltzmann equations. Using a Lénard--Bernstein collision operator, these take Fokker--Planck-like forms \cite{Fokker_1914, Planck_1917} wherein pseudo-particles in the numerical model obey the neoclassical transport equations, with particle-independent Brownian drift terms. This offers a rigorous methodology for incorporating collisions into the particle transport model, without coupling the equations of motions for each particle.
        
        Works by Chen, Chacón et al. \cite{Chen_Chacón_Barnes_2011, Chacón_Chen_Barnes_2013, Chen_Chacón_2014, Chen_Chacón_2015} have developed structure-preserving particle pushers for neoclassical transport in the Vlasov equations, derived from Crank--Nicolson integrators. We show these too can can derive from a FET interpretation, similarly offering potential extensions to higher-order-in-time particle pushers. The FET formulation is used also to consider how the stochastic drift terms can be incorporated into the pushers. Stochastic gyrokinetic expansions are also discussed.

        Different options for the numerical implementation of these schemes are considered.

        Due to the efficacy of FET in the development of SP timesteppers for both the fluid and kinetic component, we hope this approach will prove effective in the future for developing SP timesteppers for the full hybrid model. We hope this will give us the opportunity to incorporate previously inaccessible kinetic effects into the highly effective, modern, finite-element MHD models.
    \end{abstract}
    
    
    \newpage
    \tableofcontents
    
    
    \newpage
    \pagenumbering{arabic}
    %\linenumbers\renewcommand\thelinenumber{\color{black!50}\arabic{linenumber}}
            \input{0 - introduction/main.tex}
        \part{Research}
            \input{1 - low-noise PiC models/main.tex}
            \input{2 - kinetic component/main.tex}
            \input{3 - fluid component/main.tex}
            \input{4 - numerical implementation/main.tex}
        \part{Project Overview}
            \input{5 - research plan/main.tex}
            \input{6 - summary/main.tex}
    
    
    %\section{}
    \newpage
    \pagenumbering{gobble}
        \printbibliography


    \newpage
    \pagenumbering{roman}
    \appendix
        \part{Appendices}
            \input{8 - Hilbert complexes/main.tex}
            \input{9 - weak conservation proofs/main.tex}
\end{document}


\title{\BA{Title in Progress...}}
\author{Boris Andrews}
\affil{Mathematical Institute, University of Oxford}
\date{\today}


\begin{document}
    \pagenumbering{gobble}
    \maketitle
    
    
    \begin{abstract}
        Magnetic confinement reactors---in particular tokamaks---offer one of the most promising options for achieving practical nuclear fusion, with the potential to provide virtually limitless, clean energy. The theoretical and numerical modeling of tokamak plasmas is simultaneously an essential component of effective reactor design, and a great research barrier. Tokamak operational conditions exhibit comparatively low Knudsen numbers. Kinetic effects, including kinetic waves and instabilities, Landau damping, bump-on-tail instabilities and more, are therefore highly influential in tokamak plasma dynamics. Purely fluid models are inherently incapable of capturing these effects, whereas the high dimensionality in purely kinetic models render them practically intractable for most relevant purposes.

        We consider a $\delta\!f$ decomposition model, with a macroscopic fluid background and microscopic kinetic correction, both fully coupled to each other. A similar manner of discretization is proposed to that used in the recent \texttt{STRUPHY} code \cite{Holderied_Possanner_Wang_2021, Holderied_2022, Li_et_al_2023} with a finite-element model for the background and a pseudo-particle/PiC model for the correction.

        The fluid background satisfies the full, non-linear, resistive, compressible, Hall MHD equations. \cite{Laakmann_Hu_Farrell_2022} introduces finite-element(-in-space) implicit timesteppers for the incompressible analogue to this system with structure-preserving (SP) properties in the ideal case, alongside parameter-robust preconditioners. We show that these timesteppers can derive from a finite-element-in-time (FET) (and finite-element-in-space) interpretation. The benefits of this reformulation are discussed, including the derivation of timesteppers that are higher order in time, and the quantifiable dissipative SP properties in the non-ideal, resistive case.
        
        We discuss possible options for extending this FET approach to timesteppers for the compressible case.

        The kinetic corrections satisfy linearized Boltzmann equations. Using a Lénard--Bernstein collision operator, these take Fokker--Planck-like forms \cite{Fokker_1914, Planck_1917} wherein pseudo-particles in the numerical model obey the neoclassical transport equations, with particle-independent Brownian drift terms. This offers a rigorous methodology for incorporating collisions into the particle transport model, without coupling the equations of motions for each particle.
        
        Works by Chen, Chacón et al. \cite{Chen_Chacón_Barnes_2011, Chacón_Chen_Barnes_2013, Chen_Chacón_2014, Chen_Chacón_2015} have developed structure-preserving particle pushers for neoclassical transport in the Vlasov equations, derived from Crank--Nicolson integrators. We show these too can can derive from a FET interpretation, similarly offering potential extensions to higher-order-in-time particle pushers. The FET formulation is used also to consider how the stochastic drift terms can be incorporated into the pushers. Stochastic gyrokinetic expansions are also discussed.

        Different options for the numerical implementation of these schemes are considered.

        Due to the efficacy of FET in the development of SP timesteppers for both the fluid and kinetic component, we hope this approach will prove effective in the future for developing SP timesteppers for the full hybrid model. We hope this will give us the opportunity to incorporate previously inaccessible kinetic effects into the highly effective, modern, finite-element MHD models.
    \end{abstract}
    
    
    \newpage
    \tableofcontents
    
    
    \newpage
    \pagenumbering{arabic}
    %\linenumbers\renewcommand\thelinenumber{\color{black!50}\arabic{linenumber}}
            \documentclass[12pt, a4paper]{report}

\input{template/main.tex}

\title{\BA{Title in Progress...}}
\author{Boris Andrews}
\affil{Mathematical Institute, University of Oxford}
\date{\today}


\begin{document}
    \pagenumbering{gobble}
    \maketitle
    
    
    \begin{abstract}
        Magnetic confinement reactors---in particular tokamaks---offer one of the most promising options for achieving practical nuclear fusion, with the potential to provide virtually limitless, clean energy. The theoretical and numerical modeling of tokamak plasmas is simultaneously an essential component of effective reactor design, and a great research barrier. Tokamak operational conditions exhibit comparatively low Knudsen numbers. Kinetic effects, including kinetic waves and instabilities, Landau damping, bump-on-tail instabilities and more, are therefore highly influential in tokamak plasma dynamics. Purely fluid models are inherently incapable of capturing these effects, whereas the high dimensionality in purely kinetic models render them practically intractable for most relevant purposes.

        We consider a $\delta\!f$ decomposition model, with a macroscopic fluid background and microscopic kinetic correction, both fully coupled to each other. A similar manner of discretization is proposed to that used in the recent \texttt{STRUPHY} code \cite{Holderied_Possanner_Wang_2021, Holderied_2022, Li_et_al_2023} with a finite-element model for the background and a pseudo-particle/PiC model for the correction.

        The fluid background satisfies the full, non-linear, resistive, compressible, Hall MHD equations. \cite{Laakmann_Hu_Farrell_2022} introduces finite-element(-in-space) implicit timesteppers for the incompressible analogue to this system with structure-preserving (SP) properties in the ideal case, alongside parameter-robust preconditioners. We show that these timesteppers can derive from a finite-element-in-time (FET) (and finite-element-in-space) interpretation. The benefits of this reformulation are discussed, including the derivation of timesteppers that are higher order in time, and the quantifiable dissipative SP properties in the non-ideal, resistive case.
        
        We discuss possible options for extending this FET approach to timesteppers for the compressible case.

        The kinetic corrections satisfy linearized Boltzmann equations. Using a Lénard--Bernstein collision operator, these take Fokker--Planck-like forms \cite{Fokker_1914, Planck_1917} wherein pseudo-particles in the numerical model obey the neoclassical transport equations, with particle-independent Brownian drift terms. This offers a rigorous methodology for incorporating collisions into the particle transport model, without coupling the equations of motions for each particle.
        
        Works by Chen, Chacón et al. \cite{Chen_Chacón_Barnes_2011, Chacón_Chen_Barnes_2013, Chen_Chacón_2014, Chen_Chacón_2015} have developed structure-preserving particle pushers for neoclassical transport in the Vlasov equations, derived from Crank--Nicolson integrators. We show these too can can derive from a FET interpretation, similarly offering potential extensions to higher-order-in-time particle pushers. The FET formulation is used also to consider how the stochastic drift terms can be incorporated into the pushers. Stochastic gyrokinetic expansions are also discussed.

        Different options for the numerical implementation of these schemes are considered.

        Due to the efficacy of FET in the development of SP timesteppers for both the fluid and kinetic component, we hope this approach will prove effective in the future for developing SP timesteppers for the full hybrid model. We hope this will give us the opportunity to incorporate previously inaccessible kinetic effects into the highly effective, modern, finite-element MHD models.
    \end{abstract}
    
    
    \newpage
    \tableofcontents
    
    
    \newpage
    \pagenumbering{arabic}
    %\linenumbers\renewcommand\thelinenumber{\color{black!50}\arabic{linenumber}}
            \input{0 - introduction/main.tex}
        \part{Research}
            \input{1 - low-noise PiC models/main.tex}
            \input{2 - kinetic component/main.tex}
            \input{3 - fluid component/main.tex}
            \input{4 - numerical implementation/main.tex}
        \part{Project Overview}
            \input{5 - research plan/main.tex}
            \input{6 - summary/main.tex}
    
    
    %\section{}
    \newpage
    \pagenumbering{gobble}
        \printbibliography


    \newpage
    \pagenumbering{roman}
    \appendix
        \part{Appendices}
            \input{8 - Hilbert complexes/main.tex}
            \input{9 - weak conservation proofs/main.tex}
\end{document}

        \part{Research}
            \documentclass[12pt, a4paper]{report}

\input{template/main.tex}

\title{\BA{Title in Progress...}}
\author{Boris Andrews}
\affil{Mathematical Institute, University of Oxford}
\date{\today}


\begin{document}
    \pagenumbering{gobble}
    \maketitle
    
    
    \begin{abstract}
        Magnetic confinement reactors---in particular tokamaks---offer one of the most promising options for achieving practical nuclear fusion, with the potential to provide virtually limitless, clean energy. The theoretical and numerical modeling of tokamak plasmas is simultaneously an essential component of effective reactor design, and a great research barrier. Tokamak operational conditions exhibit comparatively low Knudsen numbers. Kinetic effects, including kinetic waves and instabilities, Landau damping, bump-on-tail instabilities and more, are therefore highly influential in tokamak plasma dynamics. Purely fluid models are inherently incapable of capturing these effects, whereas the high dimensionality in purely kinetic models render them practically intractable for most relevant purposes.

        We consider a $\delta\!f$ decomposition model, with a macroscopic fluid background and microscopic kinetic correction, both fully coupled to each other. A similar manner of discretization is proposed to that used in the recent \texttt{STRUPHY} code \cite{Holderied_Possanner_Wang_2021, Holderied_2022, Li_et_al_2023} with a finite-element model for the background and a pseudo-particle/PiC model for the correction.

        The fluid background satisfies the full, non-linear, resistive, compressible, Hall MHD equations. \cite{Laakmann_Hu_Farrell_2022} introduces finite-element(-in-space) implicit timesteppers for the incompressible analogue to this system with structure-preserving (SP) properties in the ideal case, alongside parameter-robust preconditioners. We show that these timesteppers can derive from a finite-element-in-time (FET) (and finite-element-in-space) interpretation. The benefits of this reformulation are discussed, including the derivation of timesteppers that are higher order in time, and the quantifiable dissipative SP properties in the non-ideal, resistive case.
        
        We discuss possible options for extending this FET approach to timesteppers for the compressible case.

        The kinetic corrections satisfy linearized Boltzmann equations. Using a Lénard--Bernstein collision operator, these take Fokker--Planck-like forms \cite{Fokker_1914, Planck_1917} wherein pseudo-particles in the numerical model obey the neoclassical transport equations, with particle-independent Brownian drift terms. This offers a rigorous methodology for incorporating collisions into the particle transport model, without coupling the equations of motions for each particle.
        
        Works by Chen, Chacón et al. \cite{Chen_Chacón_Barnes_2011, Chacón_Chen_Barnes_2013, Chen_Chacón_2014, Chen_Chacón_2015} have developed structure-preserving particle pushers for neoclassical transport in the Vlasov equations, derived from Crank--Nicolson integrators. We show these too can can derive from a FET interpretation, similarly offering potential extensions to higher-order-in-time particle pushers. The FET formulation is used also to consider how the stochastic drift terms can be incorporated into the pushers. Stochastic gyrokinetic expansions are also discussed.

        Different options for the numerical implementation of these schemes are considered.

        Due to the efficacy of FET in the development of SP timesteppers for both the fluid and kinetic component, we hope this approach will prove effective in the future for developing SP timesteppers for the full hybrid model. We hope this will give us the opportunity to incorporate previously inaccessible kinetic effects into the highly effective, modern, finite-element MHD models.
    \end{abstract}
    
    
    \newpage
    \tableofcontents
    
    
    \newpage
    \pagenumbering{arabic}
    %\linenumbers\renewcommand\thelinenumber{\color{black!50}\arabic{linenumber}}
            \input{0 - introduction/main.tex}
        \part{Research}
            \input{1 - low-noise PiC models/main.tex}
            \input{2 - kinetic component/main.tex}
            \input{3 - fluid component/main.tex}
            \input{4 - numerical implementation/main.tex}
        \part{Project Overview}
            \input{5 - research plan/main.tex}
            \input{6 - summary/main.tex}
    
    
    %\section{}
    \newpage
    \pagenumbering{gobble}
        \printbibliography


    \newpage
    \pagenumbering{roman}
    \appendix
        \part{Appendices}
            \input{8 - Hilbert complexes/main.tex}
            \input{9 - weak conservation proofs/main.tex}
\end{document}

            \documentclass[12pt, a4paper]{report}

\input{template/main.tex}

\title{\BA{Title in Progress...}}
\author{Boris Andrews}
\affil{Mathematical Institute, University of Oxford}
\date{\today}


\begin{document}
    \pagenumbering{gobble}
    \maketitle
    
    
    \begin{abstract}
        Magnetic confinement reactors---in particular tokamaks---offer one of the most promising options for achieving practical nuclear fusion, with the potential to provide virtually limitless, clean energy. The theoretical and numerical modeling of tokamak plasmas is simultaneously an essential component of effective reactor design, and a great research barrier. Tokamak operational conditions exhibit comparatively low Knudsen numbers. Kinetic effects, including kinetic waves and instabilities, Landau damping, bump-on-tail instabilities and more, are therefore highly influential in tokamak plasma dynamics. Purely fluid models are inherently incapable of capturing these effects, whereas the high dimensionality in purely kinetic models render them practically intractable for most relevant purposes.

        We consider a $\delta\!f$ decomposition model, with a macroscopic fluid background and microscopic kinetic correction, both fully coupled to each other. A similar manner of discretization is proposed to that used in the recent \texttt{STRUPHY} code \cite{Holderied_Possanner_Wang_2021, Holderied_2022, Li_et_al_2023} with a finite-element model for the background and a pseudo-particle/PiC model for the correction.

        The fluid background satisfies the full, non-linear, resistive, compressible, Hall MHD equations. \cite{Laakmann_Hu_Farrell_2022} introduces finite-element(-in-space) implicit timesteppers for the incompressible analogue to this system with structure-preserving (SP) properties in the ideal case, alongside parameter-robust preconditioners. We show that these timesteppers can derive from a finite-element-in-time (FET) (and finite-element-in-space) interpretation. The benefits of this reformulation are discussed, including the derivation of timesteppers that are higher order in time, and the quantifiable dissipative SP properties in the non-ideal, resistive case.
        
        We discuss possible options for extending this FET approach to timesteppers for the compressible case.

        The kinetic corrections satisfy linearized Boltzmann equations. Using a Lénard--Bernstein collision operator, these take Fokker--Planck-like forms \cite{Fokker_1914, Planck_1917} wherein pseudo-particles in the numerical model obey the neoclassical transport equations, with particle-independent Brownian drift terms. This offers a rigorous methodology for incorporating collisions into the particle transport model, without coupling the equations of motions for each particle.
        
        Works by Chen, Chacón et al. \cite{Chen_Chacón_Barnes_2011, Chacón_Chen_Barnes_2013, Chen_Chacón_2014, Chen_Chacón_2015} have developed structure-preserving particle pushers for neoclassical transport in the Vlasov equations, derived from Crank--Nicolson integrators. We show these too can can derive from a FET interpretation, similarly offering potential extensions to higher-order-in-time particle pushers. The FET formulation is used also to consider how the stochastic drift terms can be incorporated into the pushers. Stochastic gyrokinetic expansions are also discussed.

        Different options for the numerical implementation of these schemes are considered.

        Due to the efficacy of FET in the development of SP timesteppers for both the fluid and kinetic component, we hope this approach will prove effective in the future for developing SP timesteppers for the full hybrid model. We hope this will give us the opportunity to incorporate previously inaccessible kinetic effects into the highly effective, modern, finite-element MHD models.
    \end{abstract}
    
    
    \newpage
    \tableofcontents
    
    
    \newpage
    \pagenumbering{arabic}
    %\linenumbers\renewcommand\thelinenumber{\color{black!50}\arabic{linenumber}}
            \input{0 - introduction/main.tex}
        \part{Research}
            \input{1 - low-noise PiC models/main.tex}
            \input{2 - kinetic component/main.tex}
            \input{3 - fluid component/main.tex}
            \input{4 - numerical implementation/main.tex}
        \part{Project Overview}
            \input{5 - research plan/main.tex}
            \input{6 - summary/main.tex}
    
    
    %\section{}
    \newpage
    \pagenumbering{gobble}
        \printbibliography


    \newpage
    \pagenumbering{roman}
    \appendix
        \part{Appendices}
            \input{8 - Hilbert complexes/main.tex}
            \input{9 - weak conservation proofs/main.tex}
\end{document}

            \documentclass[12pt, a4paper]{report}

\input{template/main.tex}

\title{\BA{Title in Progress...}}
\author{Boris Andrews}
\affil{Mathematical Institute, University of Oxford}
\date{\today}


\begin{document}
    \pagenumbering{gobble}
    \maketitle
    
    
    \begin{abstract}
        Magnetic confinement reactors---in particular tokamaks---offer one of the most promising options for achieving practical nuclear fusion, with the potential to provide virtually limitless, clean energy. The theoretical and numerical modeling of tokamak plasmas is simultaneously an essential component of effective reactor design, and a great research barrier. Tokamak operational conditions exhibit comparatively low Knudsen numbers. Kinetic effects, including kinetic waves and instabilities, Landau damping, bump-on-tail instabilities and more, are therefore highly influential in tokamak plasma dynamics. Purely fluid models are inherently incapable of capturing these effects, whereas the high dimensionality in purely kinetic models render them practically intractable for most relevant purposes.

        We consider a $\delta\!f$ decomposition model, with a macroscopic fluid background and microscopic kinetic correction, both fully coupled to each other. A similar manner of discretization is proposed to that used in the recent \texttt{STRUPHY} code \cite{Holderied_Possanner_Wang_2021, Holderied_2022, Li_et_al_2023} with a finite-element model for the background and a pseudo-particle/PiC model for the correction.

        The fluid background satisfies the full, non-linear, resistive, compressible, Hall MHD equations. \cite{Laakmann_Hu_Farrell_2022} introduces finite-element(-in-space) implicit timesteppers for the incompressible analogue to this system with structure-preserving (SP) properties in the ideal case, alongside parameter-robust preconditioners. We show that these timesteppers can derive from a finite-element-in-time (FET) (and finite-element-in-space) interpretation. The benefits of this reformulation are discussed, including the derivation of timesteppers that are higher order in time, and the quantifiable dissipative SP properties in the non-ideal, resistive case.
        
        We discuss possible options for extending this FET approach to timesteppers for the compressible case.

        The kinetic corrections satisfy linearized Boltzmann equations. Using a Lénard--Bernstein collision operator, these take Fokker--Planck-like forms \cite{Fokker_1914, Planck_1917} wherein pseudo-particles in the numerical model obey the neoclassical transport equations, with particle-independent Brownian drift terms. This offers a rigorous methodology for incorporating collisions into the particle transport model, without coupling the equations of motions for each particle.
        
        Works by Chen, Chacón et al. \cite{Chen_Chacón_Barnes_2011, Chacón_Chen_Barnes_2013, Chen_Chacón_2014, Chen_Chacón_2015} have developed structure-preserving particle pushers for neoclassical transport in the Vlasov equations, derived from Crank--Nicolson integrators. We show these too can can derive from a FET interpretation, similarly offering potential extensions to higher-order-in-time particle pushers. The FET formulation is used also to consider how the stochastic drift terms can be incorporated into the pushers. Stochastic gyrokinetic expansions are also discussed.

        Different options for the numerical implementation of these schemes are considered.

        Due to the efficacy of FET in the development of SP timesteppers for both the fluid and kinetic component, we hope this approach will prove effective in the future for developing SP timesteppers for the full hybrid model. We hope this will give us the opportunity to incorporate previously inaccessible kinetic effects into the highly effective, modern, finite-element MHD models.
    \end{abstract}
    
    
    \newpage
    \tableofcontents
    
    
    \newpage
    \pagenumbering{arabic}
    %\linenumbers\renewcommand\thelinenumber{\color{black!50}\arabic{linenumber}}
            \input{0 - introduction/main.tex}
        \part{Research}
            \input{1 - low-noise PiC models/main.tex}
            \input{2 - kinetic component/main.tex}
            \input{3 - fluid component/main.tex}
            \input{4 - numerical implementation/main.tex}
        \part{Project Overview}
            \input{5 - research plan/main.tex}
            \input{6 - summary/main.tex}
    
    
    %\section{}
    \newpage
    \pagenumbering{gobble}
        \printbibliography


    \newpage
    \pagenumbering{roman}
    \appendix
        \part{Appendices}
            \input{8 - Hilbert complexes/main.tex}
            \input{9 - weak conservation proofs/main.tex}
\end{document}

            \documentclass[12pt, a4paper]{report}

\input{template/main.tex}

\title{\BA{Title in Progress...}}
\author{Boris Andrews}
\affil{Mathematical Institute, University of Oxford}
\date{\today}


\begin{document}
    \pagenumbering{gobble}
    \maketitle
    
    
    \begin{abstract}
        Magnetic confinement reactors---in particular tokamaks---offer one of the most promising options for achieving practical nuclear fusion, with the potential to provide virtually limitless, clean energy. The theoretical and numerical modeling of tokamak plasmas is simultaneously an essential component of effective reactor design, and a great research barrier. Tokamak operational conditions exhibit comparatively low Knudsen numbers. Kinetic effects, including kinetic waves and instabilities, Landau damping, bump-on-tail instabilities and more, are therefore highly influential in tokamak plasma dynamics. Purely fluid models are inherently incapable of capturing these effects, whereas the high dimensionality in purely kinetic models render them practically intractable for most relevant purposes.

        We consider a $\delta\!f$ decomposition model, with a macroscopic fluid background and microscopic kinetic correction, both fully coupled to each other. A similar manner of discretization is proposed to that used in the recent \texttt{STRUPHY} code \cite{Holderied_Possanner_Wang_2021, Holderied_2022, Li_et_al_2023} with a finite-element model for the background and a pseudo-particle/PiC model for the correction.

        The fluid background satisfies the full, non-linear, resistive, compressible, Hall MHD equations. \cite{Laakmann_Hu_Farrell_2022} introduces finite-element(-in-space) implicit timesteppers for the incompressible analogue to this system with structure-preserving (SP) properties in the ideal case, alongside parameter-robust preconditioners. We show that these timesteppers can derive from a finite-element-in-time (FET) (and finite-element-in-space) interpretation. The benefits of this reformulation are discussed, including the derivation of timesteppers that are higher order in time, and the quantifiable dissipative SP properties in the non-ideal, resistive case.
        
        We discuss possible options for extending this FET approach to timesteppers for the compressible case.

        The kinetic corrections satisfy linearized Boltzmann equations. Using a Lénard--Bernstein collision operator, these take Fokker--Planck-like forms \cite{Fokker_1914, Planck_1917} wherein pseudo-particles in the numerical model obey the neoclassical transport equations, with particle-independent Brownian drift terms. This offers a rigorous methodology for incorporating collisions into the particle transport model, without coupling the equations of motions for each particle.
        
        Works by Chen, Chacón et al. \cite{Chen_Chacón_Barnes_2011, Chacón_Chen_Barnes_2013, Chen_Chacón_2014, Chen_Chacón_2015} have developed structure-preserving particle pushers for neoclassical transport in the Vlasov equations, derived from Crank--Nicolson integrators. We show these too can can derive from a FET interpretation, similarly offering potential extensions to higher-order-in-time particle pushers. The FET formulation is used also to consider how the stochastic drift terms can be incorporated into the pushers. Stochastic gyrokinetic expansions are also discussed.

        Different options for the numerical implementation of these schemes are considered.

        Due to the efficacy of FET in the development of SP timesteppers for both the fluid and kinetic component, we hope this approach will prove effective in the future for developing SP timesteppers for the full hybrid model. We hope this will give us the opportunity to incorporate previously inaccessible kinetic effects into the highly effective, modern, finite-element MHD models.
    \end{abstract}
    
    
    \newpage
    \tableofcontents
    
    
    \newpage
    \pagenumbering{arabic}
    %\linenumbers\renewcommand\thelinenumber{\color{black!50}\arabic{linenumber}}
            \input{0 - introduction/main.tex}
        \part{Research}
            \input{1 - low-noise PiC models/main.tex}
            \input{2 - kinetic component/main.tex}
            \input{3 - fluid component/main.tex}
            \input{4 - numerical implementation/main.tex}
        \part{Project Overview}
            \input{5 - research plan/main.tex}
            \input{6 - summary/main.tex}
    
    
    %\section{}
    \newpage
    \pagenumbering{gobble}
        \printbibliography


    \newpage
    \pagenumbering{roman}
    \appendix
        \part{Appendices}
            \input{8 - Hilbert complexes/main.tex}
            \input{9 - weak conservation proofs/main.tex}
\end{document}

        \part{Project Overview}
            \documentclass[12pt, a4paper]{report}

\input{template/main.tex}

\title{\BA{Title in Progress...}}
\author{Boris Andrews}
\affil{Mathematical Institute, University of Oxford}
\date{\today}


\begin{document}
    \pagenumbering{gobble}
    \maketitle
    
    
    \begin{abstract}
        Magnetic confinement reactors---in particular tokamaks---offer one of the most promising options for achieving practical nuclear fusion, with the potential to provide virtually limitless, clean energy. The theoretical and numerical modeling of tokamak plasmas is simultaneously an essential component of effective reactor design, and a great research barrier. Tokamak operational conditions exhibit comparatively low Knudsen numbers. Kinetic effects, including kinetic waves and instabilities, Landau damping, bump-on-tail instabilities and more, are therefore highly influential in tokamak plasma dynamics. Purely fluid models are inherently incapable of capturing these effects, whereas the high dimensionality in purely kinetic models render them practically intractable for most relevant purposes.

        We consider a $\delta\!f$ decomposition model, with a macroscopic fluid background and microscopic kinetic correction, both fully coupled to each other. A similar manner of discretization is proposed to that used in the recent \texttt{STRUPHY} code \cite{Holderied_Possanner_Wang_2021, Holderied_2022, Li_et_al_2023} with a finite-element model for the background and a pseudo-particle/PiC model for the correction.

        The fluid background satisfies the full, non-linear, resistive, compressible, Hall MHD equations. \cite{Laakmann_Hu_Farrell_2022} introduces finite-element(-in-space) implicit timesteppers for the incompressible analogue to this system with structure-preserving (SP) properties in the ideal case, alongside parameter-robust preconditioners. We show that these timesteppers can derive from a finite-element-in-time (FET) (and finite-element-in-space) interpretation. The benefits of this reformulation are discussed, including the derivation of timesteppers that are higher order in time, and the quantifiable dissipative SP properties in the non-ideal, resistive case.
        
        We discuss possible options for extending this FET approach to timesteppers for the compressible case.

        The kinetic corrections satisfy linearized Boltzmann equations. Using a Lénard--Bernstein collision operator, these take Fokker--Planck-like forms \cite{Fokker_1914, Planck_1917} wherein pseudo-particles in the numerical model obey the neoclassical transport equations, with particle-independent Brownian drift terms. This offers a rigorous methodology for incorporating collisions into the particle transport model, without coupling the equations of motions for each particle.
        
        Works by Chen, Chacón et al. \cite{Chen_Chacón_Barnes_2011, Chacón_Chen_Barnes_2013, Chen_Chacón_2014, Chen_Chacón_2015} have developed structure-preserving particle pushers for neoclassical transport in the Vlasov equations, derived from Crank--Nicolson integrators. We show these too can can derive from a FET interpretation, similarly offering potential extensions to higher-order-in-time particle pushers. The FET formulation is used also to consider how the stochastic drift terms can be incorporated into the pushers. Stochastic gyrokinetic expansions are also discussed.

        Different options for the numerical implementation of these schemes are considered.

        Due to the efficacy of FET in the development of SP timesteppers for both the fluid and kinetic component, we hope this approach will prove effective in the future for developing SP timesteppers for the full hybrid model. We hope this will give us the opportunity to incorporate previously inaccessible kinetic effects into the highly effective, modern, finite-element MHD models.
    \end{abstract}
    
    
    \newpage
    \tableofcontents
    
    
    \newpage
    \pagenumbering{arabic}
    %\linenumbers\renewcommand\thelinenumber{\color{black!50}\arabic{linenumber}}
            \input{0 - introduction/main.tex}
        \part{Research}
            \input{1 - low-noise PiC models/main.tex}
            \input{2 - kinetic component/main.tex}
            \input{3 - fluid component/main.tex}
            \input{4 - numerical implementation/main.tex}
        \part{Project Overview}
            \input{5 - research plan/main.tex}
            \input{6 - summary/main.tex}
    
    
    %\section{}
    \newpage
    \pagenumbering{gobble}
        \printbibliography


    \newpage
    \pagenumbering{roman}
    \appendix
        \part{Appendices}
            \input{8 - Hilbert complexes/main.tex}
            \input{9 - weak conservation proofs/main.tex}
\end{document}

            \documentclass[12pt, a4paper]{report}

\input{template/main.tex}

\title{\BA{Title in Progress...}}
\author{Boris Andrews}
\affil{Mathematical Institute, University of Oxford}
\date{\today}


\begin{document}
    \pagenumbering{gobble}
    \maketitle
    
    
    \begin{abstract}
        Magnetic confinement reactors---in particular tokamaks---offer one of the most promising options for achieving practical nuclear fusion, with the potential to provide virtually limitless, clean energy. The theoretical and numerical modeling of tokamak plasmas is simultaneously an essential component of effective reactor design, and a great research barrier. Tokamak operational conditions exhibit comparatively low Knudsen numbers. Kinetic effects, including kinetic waves and instabilities, Landau damping, bump-on-tail instabilities and more, are therefore highly influential in tokamak plasma dynamics. Purely fluid models are inherently incapable of capturing these effects, whereas the high dimensionality in purely kinetic models render them practically intractable for most relevant purposes.

        We consider a $\delta\!f$ decomposition model, with a macroscopic fluid background and microscopic kinetic correction, both fully coupled to each other. A similar manner of discretization is proposed to that used in the recent \texttt{STRUPHY} code \cite{Holderied_Possanner_Wang_2021, Holderied_2022, Li_et_al_2023} with a finite-element model for the background and a pseudo-particle/PiC model for the correction.

        The fluid background satisfies the full, non-linear, resistive, compressible, Hall MHD equations. \cite{Laakmann_Hu_Farrell_2022} introduces finite-element(-in-space) implicit timesteppers for the incompressible analogue to this system with structure-preserving (SP) properties in the ideal case, alongside parameter-robust preconditioners. We show that these timesteppers can derive from a finite-element-in-time (FET) (and finite-element-in-space) interpretation. The benefits of this reformulation are discussed, including the derivation of timesteppers that are higher order in time, and the quantifiable dissipative SP properties in the non-ideal, resistive case.
        
        We discuss possible options for extending this FET approach to timesteppers for the compressible case.

        The kinetic corrections satisfy linearized Boltzmann equations. Using a Lénard--Bernstein collision operator, these take Fokker--Planck-like forms \cite{Fokker_1914, Planck_1917} wherein pseudo-particles in the numerical model obey the neoclassical transport equations, with particle-independent Brownian drift terms. This offers a rigorous methodology for incorporating collisions into the particle transport model, without coupling the equations of motions for each particle.
        
        Works by Chen, Chacón et al. \cite{Chen_Chacón_Barnes_2011, Chacón_Chen_Barnes_2013, Chen_Chacón_2014, Chen_Chacón_2015} have developed structure-preserving particle pushers for neoclassical transport in the Vlasov equations, derived from Crank--Nicolson integrators. We show these too can can derive from a FET interpretation, similarly offering potential extensions to higher-order-in-time particle pushers. The FET formulation is used also to consider how the stochastic drift terms can be incorporated into the pushers. Stochastic gyrokinetic expansions are also discussed.

        Different options for the numerical implementation of these schemes are considered.

        Due to the efficacy of FET in the development of SP timesteppers for both the fluid and kinetic component, we hope this approach will prove effective in the future for developing SP timesteppers for the full hybrid model. We hope this will give us the opportunity to incorporate previously inaccessible kinetic effects into the highly effective, modern, finite-element MHD models.
    \end{abstract}
    
    
    \newpage
    \tableofcontents
    
    
    \newpage
    \pagenumbering{arabic}
    %\linenumbers\renewcommand\thelinenumber{\color{black!50}\arabic{linenumber}}
            \input{0 - introduction/main.tex}
        \part{Research}
            \input{1 - low-noise PiC models/main.tex}
            \input{2 - kinetic component/main.tex}
            \input{3 - fluid component/main.tex}
            \input{4 - numerical implementation/main.tex}
        \part{Project Overview}
            \input{5 - research plan/main.tex}
            \input{6 - summary/main.tex}
    
    
    %\section{}
    \newpage
    \pagenumbering{gobble}
        \printbibliography


    \newpage
    \pagenumbering{roman}
    \appendix
        \part{Appendices}
            \input{8 - Hilbert complexes/main.tex}
            \input{9 - weak conservation proofs/main.tex}
\end{document}

    
    
    %\section{}
    \newpage
    \pagenumbering{gobble}
        \printbibliography


    \newpage
    \pagenumbering{roman}
    \appendix
        \part{Appendices}
            \documentclass[12pt, a4paper]{report}

\input{template/main.tex}

\title{\BA{Title in Progress...}}
\author{Boris Andrews}
\affil{Mathematical Institute, University of Oxford}
\date{\today}


\begin{document}
    \pagenumbering{gobble}
    \maketitle
    
    
    \begin{abstract}
        Magnetic confinement reactors---in particular tokamaks---offer one of the most promising options for achieving practical nuclear fusion, with the potential to provide virtually limitless, clean energy. The theoretical and numerical modeling of tokamak plasmas is simultaneously an essential component of effective reactor design, and a great research barrier. Tokamak operational conditions exhibit comparatively low Knudsen numbers. Kinetic effects, including kinetic waves and instabilities, Landau damping, bump-on-tail instabilities and more, are therefore highly influential in tokamak plasma dynamics. Purely fluid models are inherently incapable of capturing these effects, whereas the high dimensionality in purely kinetic models render them practically intractable for most relevant purposes.

        We consider a $\delta\!f$ decomposition model, with a macroscopic fluid background and microscopic kinetic correction, both fully coupled to each other. A similar manner of discretization is proposed to that used in the recent \texttt{STRUPHY} code \cite{Holderied_Possanner_Wang_2021, Holderied_2022, Li_et_al_2023} with a finite-element model for the background and a pseudo-particle/PiC model for the correction.

        The fluid background satisfies the full, non-linear, resistive, compressible, Hall MHD equations. \cite{Laakmann_Hu_Farrell_2022} introduces finite-element(-in-space) implicit timesteppers for the incompressible analogue to this system with structure-preserving (SP) properties in the ideal case, alongside parameter-robust preconditioners. We show that these timesteppers can derive from a finite-element-in-time (FET) (and finite-element-in-space) interpretation. The benefits of this reformulation are discussed, including the derivation of timesteppers that are higher order in time, and the quantifiable dissipative SP properties in the non-ideal, resistive case.
        
        We discuss possible options for extending this FET approach to timesteppers for the compressible case.

        The kinetic corrections satisfy linearized Boltzmann equations. Using a Lénard--Bernstein collision operator, these take Fokker--Planck-like forms \cite{Fokker_1914, Planck_1917} wherein pseudo-particles in the numerical model obey the neoclassical transport equations, with particle-independent Brownian drift terms. This offers a rigorous methodology for incorporating collisions into the particle transport model, without coupling the equations of motions for each particle.
        
        Works by Chen, Chacón et al. \cite{Chen_Chacón_Barnes_2011, Chacón_Chen_Barnes_2013, Chen_Chacón_2014, Chen_Chacón_2015} have developed structure-preserving particle pushers for neoclassical transport in the Vlasov equations, derived from Crank--Nicolson integrators. We show these too can can derive from a FET interpretation, similarly offering potential extensions to higher-order-in-time particle pushers. The FET formulation is used also to consider how the stochastic drift terms can be incorporated into the pushers. Stochastic gyrokinetic expansions are also discussed.

        Different options for the numerical implementation of these schemes are considered.

        Due to the efficacy of FET in the development of SP timesteppers for both the fluid and kinetic component, we hope this approach will prove effective in the future for developing SP timesteppers for the full hybrid model. We hope this will give us the opportunity to incorporate previously inaccessible kinetic effects into the highly effective, modern, finite-element MHD models.
    \end{abstract}
    
    
    \newpage
    \tableofcontents
    
    
    \newpage
    \pagenumbering{arabic}
    %\linenumbers\renewcommand\thelinenumber{\color{black!50}\arabic{linenumber}}
            \input{0 - introduction/main.tex}
        \part{Research}
            \input{1 - low-noise PiC models/main.tex}
            \input{2 - kinetic component/main.tex}
            \input{3 - fluid component/main.tex}
            \input{4 - numerical implementation/main.tex}
        \part{Project Overview}
            \input{5 - research plan/main.tex}
            \input{6 - summary/main.tex}
    
    
    %\section{}
    \newpage
    \pagenumbering{gobble}
        \printbibliography


    \newpage
    \pagenumbering{roman}
    \appendix
        \part{Appendices}
            \input{8 - Hilbert complexes/main.tex}
            \input{9 - weak conservation proofs/main.tex}
\end{document}

            \documentclass[12pt, a4paper]{report}

\input{template/main.tex}

\title{\BA{Title in Progress...}}
\author{Boris Andrews}
\affil{Mathematical Institute, University of Oxford}
\date{\today}


\begin{document}
    \pagenumbering{gobble}
    \maketitle
    
    
    \begin{abstract}
        Magnetic confinement reactors---in particular tokamaks---offer one of the most promising options for achieving practical nuclear fusion, with the potential to provide virtually limitless, clean energy. The theoretical and numerical modeling of tokamak plasmas is simultaneously an essential component of effective reactor design, and a great research barrier. Tokamak operational conditions exhibit comparatively low Knudsen numbers. Kinetic effects, including kinetic waves and instabilities, Landau damping, bump-on-tail instabilities and more, are therefore highly influential in tokamak plasma dynamics. Purely fluid models are inherently incapable of capturing these effects, whereas the high dimensionality in purely kinetic models render them practically intractable for most relevant purposes.

        We consider a $\delta\!f$ decomposition model, with a macroscopic fluid background and microscopic kinetic correction, both fully coupled to each other. A similar manner of discretization is proposed to that used in the recent \texttt{STRUPHY} code \cite{Holderied_Possanner_Wang_2021, Holderied_2022, Li_et_al_2023} with a finite-element model for the background and a pseudo-particle/PiC model for the correction.

        The fluid background satisfies the full, non-linear, resistive, compressible, Hall MHD equations. \cite{Laakmann_Hu_Farrell_2022} introduces finite-element(-in-space) implicit timesteppers for the incompressible analogue to this system with structure-preserving (SP) properties in the ideal case, alongside parameter-robust preconditioners. We show that these timesteppers can derive from a finite-element-in-time (FET) (and finite-element-in-space) interpretation. The benefits of this reformulation are discussed, including the derivation of timesteppers that are higher order in time, and the quantifiable dissipative SP properties in the non-ideal, resistive case.
        
        We discuss possible options for extending this FET approach to timesteppers for the compressible case.

        The kinetic corrections satisfy linearized Boltzmann equations. Using a Lénard--Bernstein collision operator, these take Fokker--Planck-like forms \cite{Fokker_1914, Planck_1917} wherein pseudo-particles in the numerical model obey the neoclassical transport equations, with particle-independent Brownian drift terms. This offers a rigorous methodology for incorporating collisions into the particle transport model, without coupling the equations of motions for each particle.
        
        Works by Chen, Chacón et al. \cite{Chen_Chacón_Barnes_2011, Chacón_Chen_Barnes_2013, Chen_Chacón_2014, Chen_Chacón_2015} have developed structure-preserving particle pushers for neoclassical transport in the Vlasov equations, derived from Crank--Nicolson integrators. We show these too can can derive from a FET interpretation, similarly offering potential extensions to higher-order-in-time particle pushers. The FET formulation is used also to consider how the stochastic drift terms can be incorporated into the pushers. Stochastic gyrokinetic expansions are also discussed.

        Different options for the numerical implementation of these schemes are considered.

        Due to the efficacy of FET in the development of SP timesteppers for both the fluid and kinetic component, we hope this approach will prove effective in the future for developing SP timesteppers for the full hybrid model. We hope this will give us the opportunity to incorporate previously inaccessible kinetic effects into the highly effective, modern, finite-element MHD models.
    \end{abstract}
    
    
    \newpage
    \tableofcontents
    
    
    \newpage
    \pagenumbering{arabic}
    %\linenumbers\renewcommand\thelinenumber{\color{black!50}\arabic{linenumber}}
            \input{0 - introduction/main.tex}
        \part{Research}
            \input{1 - low-noise PiC models/main.tex}
            \input{2 - kinetic component/main.tex}
            \input{3 - fluid component/main.tex}
            \input{4 - numerical implementation/main.tex}
        \part{Project Overview}
            \input{5 - research plan/main.tex}
            \input{6 - summary/main.tex}
    
    
    %\section{}
    \newpage
    \pagenumbering{gobble}
        \printbibliography


    \newpage
    \pagenumbering{roman}
    \appendix
        \part{Appendices}
            \input{8 - Hilbert complexes/main.tex}
            \input{9 - weak conservation proofs/main.tex}
\end{document}

\end{document}

        \part{Project Overview}
            \documentclass[12pt, a4paper]{report}

\documentclass[12pt, a4paper]{report}

\input{template/main.tex}

\title{\BA{Title in Progress...}}
\author{Boris Andrews}
\affil{Mathematical Institute, University of Oxford}
\date{\today}


\begin{document}
    \pagenumbering{gobble}
    \maketitle
    
    
    \begin{abstract}
        Magnetic confinement reactors---in particular tokamaks---offer one of the most promising options for achieving practical nuclear fusion, with the potential to provide virtually limitless, clean energy. The theoretical and numerical modeling of tokamak plasmas is simultaneously an essential component of effective reactor design, and a great research barrier. Tokamak operational conditions exhibit comparatively low Knudsen numbers. Kinetic effects, including kinetic waves and instabilities, Landau damping, bump-on-tail instabilities and more, are therefore highly influential in tokamak plasma dynamics. Purely fluid models are inherently incapable of capturing these effects, whereas the high dimensionality in purely kinetic models render them practically intractable for most relevant purposes.

        We consider a $\delta\!f$ decomposition model, with a macroscopic fluid background and microscopic kinetic correction, both fully coupled to each other. A similar manner of discretization is proposed to that used in the recent \texttt{STRUPHY} code \cite{Holderied_Possanner_Wang_2021, Holderied_2022, Li_et_al_2023} with a finite-element model for the background and a pseudo-particle/PiC model for the correction.

        The fluid background satisfies the full, non-linear, resistive, compressible, Hall MHD equations. \cite{Laakmann_Hu_Farrell_2022} introduces finite-element(-in-space) implicit timesteppers for the incompressible analogue to this system with structure-preserving (SP) properties in the ideal case, alongside parameter-robust preconditioners. We show that these timesteppers can derive from a finite-element-in-time (FET) (and finite-element-in-space) interpretation. The benefits of this reformulation are discussed, including the derivation of timesteppers that are higher order in time, and the quantifiable dissipative SP properties in the non-ideal, resistive case.
        
        We discuss possible options for extending this FET approach to timesteppers for the compressible case.

        The kinetic corrections satisfy linearized Boltzmann equations. Using a Lénard--Bernstein collision operator, these take Fokker--Planck-like forms \cite{Fokker_1914, Planck_1917} wherein pseudo-particles in the numerical model obey the neoclassical transport equations, with particle-independent Brownian drift terms. This offers a rigorous methodology for incorporating collisions into the particle transport model, without coupling the equations of motions for each particle.
        
        Works by Chen, Chacón et al. \cite{Chen_Chacón_Barnes_2011, Chacón_Chen_Barnes_2013, Chen_Chacón_2014, Chen_Chacón_2015} have developed structure-preserving particle pushers for neoclassical transport in the Vlasov equations, derived from Crank--Nicolson integrators. We show these too can can derive from a FET interpretation, similarly offering potential extensions to higher-order-in-time particle pushers. The FET formulation is used also to consider how the stochastic drift terms can be incorporated into the pushers. Stochastic gyrokinetic expansions are also discussed.

        Different options for the numerical implementation of these schemes are considered.

        Due to the efficacy of FET in the development of SP timesteppers for both the fluid and kinetic component, we hope this approach will prove effective in the future for developing SP timesteppers for the full hybrid model. We hope this will give us the opportunity to incorporate previously inaccessible kinetic effects into the highly effective, modern, finite-element MHD models.
    \end{abstract}
    
    
    \newpage
    \tableofcontents
    
    
    \newpage
    \pagenumbering{arabic}
    %\linenumbers\renewcommand\thelinenumber{\color{black!50}\arabic{linenumber}}
            \input{0 - introduction/main.tex}
        \part{Research}
            \input{1 - low-noise PiC models/main.tex}
            \input{2 - kinetic component/main.tex}
            \input{3 - fluid component/main.tex}
            \input{4 - numerical implementation/main.tex}
        \part{Project Overview}
            \input{5 - research plan/main.tex}
            \input{6 - summary/main.tex}
    
    
    %\section{}
    \newpage
    \pagenumbering{gobble}
        \printbibliography


    \newpage
    \pagenumbering{roman}
    \appendix
        \part{Appendices}
            \input{8 - Hilbert complexes/main.tex}
            \input{9 - weak conservation proofs/main.tex}
\end{document}


\title{\BA{Title in Progress...}}
\author{Boris Andrews}
\affil{Mathematical Institute, University of Oxford}
\date{\today}


\begin{document}
    \pagenumbering{gobble}
    \maketitle
    
    
    \begin{abstract}
        Magnetic confinement reactors---in particular tokamaks---offer one of the most promising options for achieving practical nuclear fusion, with the potential to provide virtually limitless, clean energy. The theoretical and numerical modeling of tokamak plasmas is simultaneously an essential component of effective reactor design, and a great research barrier. Tokamak operational conditions exhibit comparatively low Knudsen numbers. Kinetic effects, including kinetic waves and instabilities, Landau damping, bump-on-tail instabilities and more, are therefore highly influential in tokamak plasma dynamics. Purely fluid models are inherently incapable of capturing these effects, whereas the high dimensionality in purely kinetic models render them practically intractable for most relevant purposes.

        We consider a $\delta\!f$ decomposition model, with a macroscopic fluid background and microscopic kinetic correction, both fully coupled to each other. A similar manner of discretization is proposed to that used in the recent \texttt{STRUPHY} code \cite{Holderied_Possanner_Wang_2021, Holderied_2022, Li_et_al_2023} with a finite-element model for the background and a pseudo-particle/PiC model for the correction.

        The fluid background satisfies the full, non-linear, resistive, compressible, Hall MHD equations. \cite{Laakmann_Hu_Farrell_2022} introduces finite-element(-in-space) implicit timesteppers for the incompressible analogue to this system with structure-preserving (SP) properties in the ideal case, alongside parameter-robust preconditioners. We show that these timesteppers can derive from a finite-element-in-time (FET) (and finite-element-in-space) interpretation. The benefits of this reformulation are discussed, including the derivation of timesteppers that are higher order in time, and the quantifiable dissipative SP properties in the non-ideal, resistive case.
        
        We discuss possible options for extending this FET approach to timesteppers for the compressible case.

        The kinetic corrections satisfy linearized Boltzmann equations. Using a Lénard--Bernstein collision operator, these take Fokker--Planck-like forms \cite{Fokker_1914, Planck_1917} wherein pseudo-particles in the numerical model obey the neoclassical transport equations, with particle-independent Brownian drift terms. This offers a rigorous methodology for incorporating collisions into the particle transport model, without coupling the equations of motions for each particle.
        
        Works by Chen, Chacón et al. \cite{Chen_Chacón_Barnes_2011, Chacón_Chen_Barnes_2013, Chen_Chacón_2014, Chen_Chacón_2015} have developed structure-preserving particle pushers for neoclassical transport in the Vlasov equations, derived from Crank--Nicolson integrators. We show these too can can derive from a FET interpretation, similarly offering potential extensions to higher-order-in-time particle pushers. The FET formulation is used also to consider how the stochastic drift terms can be incorporated into the pushers. Stochastic gyrokinetic expansions are also discussed.

        Different options for the numerical implementation of these schemes are considered.

        Due to the efficacy of FET in the development of SP timesteppers for both the fluid and kinetic component, we hope this approach will prove effective in the future for developing SP timesteppers for the full hybrid model. We hope this will give us the opportunity to incorporate previously inaccessible kinetic effects into the highly effective, modern, finite-element MHD models.
    \end{abstract}
    
    
    \newpage
    \tableofcontents
    
    
    \newpage
    \pagenumbering{arabic}
    %\linenumbers\renewcommand\thelinenumber{\color{black!50}\arabic{linenumber}}
            \documentclass[12pt, a4paper]{report}

\input{template/main.tex}

\title{\BA{Title in Progress...}}
\author{Boris Andrews}
\affil{Mathematical Institute, University of Oxford}
\date{\today}


\begin{document}
    \pagenumbering{gobble}
    \maketitle
    
    
    \begin{abstract}
        Magnetic confinement reactors---in particular tokamaks---offer one of the most promising options for achieving practical nuclear fusion, with the potential to provide virtually limitless, clean energy. The theoretical and numerical modeling of tokamak plasmas is simultaneously an essential component of effective reactor design, and a great research barrier. Tokamak operational conditions exhibit comparatively low Knudsen numbers. Kinetic effects, including kinetic waves and instabilities, Landau damping, bump-on-tail instabilities and more, are therefore highly influential in tokamak plasma dynamics. Purely fluid models are inherently incapable of capturing these effects, whereas the high dimensionality in purely kinetic models render them practically intractable for most relevant purposes.

        We consider a $\delta\!f$ decomposition model, with a macroscopic fluid background and microscopic kinetic correction, both fully coupled to each other. A similar manner of discretization is proposed to that used in the recent \texttt{STRUPHY} code \cite{Holderied_Possanner_Wang_2021, Holderied_2022, Li_et_al_2023} with a finite-element model for the background and a pseudo-particle/PiC model for the correction.

        The fluid background satisfies the full, non-linear, resistive, compressible, Hall MHD equations. \cite{Laakmann_Hu_Farrell_2022} introduces finite-element(-in-space) implicit timesteppers for the incompressible analogue to this system with structure-preserving (SP) properties in the ideal case, alongside parameter-robust preconditioners. We show that these timesteppers can derive from a finite-element-in-time (FET) (and finite-element-in-space) interpretation. The benefits of this reformulation are discussed, including the derivation of timesteppers that are higher order in time, and the quantifiable dissipative SP properties in the non-ideal, resistive case.
        
        We discuss possible options for extending this FET approach to timesteppers for the compressible case.

        The kinetic corrections satisfy linearized Boltzmann equations. Using a Lénard--Bernstein collision operator, these take Fokker--Planck-like forms \cite{Fokker_1914, Planck_1917} wherein pseudo-particles in the numerical model obey the neoclassical transport equations, with particle-independent Brownian drift terms. This offers a rigorous methodology for incorporating collisions into the particle transport model, without coupling the equations of motions for each particle.
        
        Works by Chen, Chacón et al. \cite{Chen_Chacón_Barnes_2011, Chacón_Chen_Barnes_2013, Chen_Chacón_2014, Chen_Chacón_2015} have developed structure-preserving particle pushers for neoclassical transport in the Vlasov equations, derived from Crank--Nicolson integrators. We show these too can can derive from a FET interpretation, similarly offering potential extensions to higher-order-in-time particle pushers. The FET formulation is used also to consider how the stochastic drift terms can be incorporated into the pushers. Stochastic gyrokinetic expansions are also discussed.

        Different options for the numerical implementation of these schemes are considered.

        Due to the efficacy of FET in the development of SP timesteppers for both the fluid and kinetic component, we hope this approach will prove effective in the future for developing SP timesteppers for the full hybrid model. We hope this will give us the opportunity to incorporate previously inaccessible kinetic effects into the highly effective, modern, finite-element MHD models.
    \end{abstract}
    
    
    \newpage
    \tableofcontents
    
    
    \newpage
    \pagenumbering{arabic}
    %\linenumbers\renewcommand\thelinenumber{\color{black!50}\arabic{linenumber}}
            \input{0 - introduction/main.tex}
        \part{Research}
            \input{1 - low-noise PiC models/main.tex}
            \input{2 - kinetic component/main.tex}
            \input{3 - fluid component/main.tex}
            \input{4 - numerical implementation/main.tex}
        \part{Project Overview}
            \input{5 - research plan/main.tex}
            \input{6 - summary/main.tex}
    
    
    %\section{}
    \newpage
    \pagenumbering{gobble}
        \printbibliography


    \newpage
    \pagenumbering{roman}
    \appendix
        \part{Appendices}
            \input{8 - Hilbert complexes/main.tex}
            \input{9 - weak conservation proofs/main.tex}
\end{document}

        \part{Research}
            \documentclass[12pt, a4paper]{report}

\input{template/main.tex}

\title{\BA{Title in Progress...}}
\author{Boris Andrews}
\affil{Mathematical Institute, University of Oxford}
\date{\today}


\begin{document}
    \pagenumbering{gobble}
    \maketitle
    
    
    \begin{abstract}
        Magnetic confinement reactors---in particular tokamaks---offer one of the most promising options for achieving practical nuclear fusion, with the potential to provide virtually limitless, clean energy. The theoretical and numerical modeling of tokamak plasmas is simultaneously an essential component of effective reactor design, and a great research barrier. Tokamak operational conditions exhibit comparatively low Knudsen numbers. Kinetic effects, including kinetic waves and instabilities, Landau damping, bump-on-tail instabilities and more, are therefore highly influential in tokamak plasma dynamics. Purely fluid models are inherently incapable of capturing these effects, whereas the high dimensionality in purely kinetic models render them practically intractable for most relevant purposes.

        We consider a $\delta\!f$ decomposition model, with a macroscopic fluid background and microscopic kinetic correction, both fully coupled to each other. A similar manner of discretization is proposed to that used in the recent \texttt{STRUPHY} code \cite{Holderied_Possanner_Wang_2021, Holderied_2022, Li_et_al_2023} with a finite-element model for the background and a pseudo-particle/PiC model for the correction.

        The fluid background satisfies the full, non-linear, resistive, compressible, Hall MHD equations. \cite{Laakmann_Hu_Farrell_2022} introduces finite-element(-in-space) implicit timesteppers for the incompressible analogue to this system with structure-preserving (SP) properties in the ideal case, alongside parameter-robust preconditioners. We show that these timesteppers can derive from a finite-element-in-time (FET) (and finite-element-in-space) interpretation. The benefits of this reformulation are discussed, including the derivation of timesteppers that are higher order in time, and the quantifiable dissipative SP properties in the non-ideal, resistive case.
        
        We discuss possible options for extending this FET approach to timesteppers for the compressible case.

        The kinetic corrections satisfy linearized Boltzmann equations. Using a Lénard--Bernstein collision operator, these take Fokker--Planck-like forms \cite{Fokker_1914, Planck_1917} wherein pseudo-particles in the numerical model obey the neoclassical transport equations, with particle-independent Brownian drift terms. This offers a rigorous methodology for incorporating collisions into the particle transport model, without coupling the equations of motions for each particle.
        
        Works by Chen, Chacón et al. \cite{Chen_Chacón_Barnes_2011, Chacón_Chen_Barnes_2013, Chen_Chacón_2014, Chen_Chacón_2015} have developed structure-preserving particle pushers for neoclassical transport in the Vlasov equations, derived from Crank--Nicolson integrators. We show these too can can derive from a FET interpretation, similarly offering potential extensions to higher-order-in-time particle pushers. The FET formulation is used also to consider how the stochastic drift terms can be incorporated into the pushers. Stochastic gyrokinetic expansions are also discussed.

        Different options for the numerical implementation of these schemes are considered.

        Due to the efficacy of FET in the development of SP timesteppers for both the fluid and kinetic component, we hope this approach will prove effective in the future for developing SP timesteppers for the full hybrid model. We hope this will give us the opportunity to incorporate previously inaccessible kinetic effects into the highly effective, modern, finite-element MHD models.
    \end{abstract}
    
    
    \newpage
    \tableofcontents
    
    
    \newpage
    \pagenumbering{arabic}
    %\linenumbers\renewcommand\thelinenumber{\color{black!50}\arabic{linenumber}}
            \input{0 - introduction/main.tex}
        \part{Research}
            \input{1 - low-noise PiC models/main.tex}
            \input{2 - kinetic component/main.tex}
            \input{3 - fluid component/main.tex}
            \input{4 - numerical implementation/main.tex}
        \part{Project Overview}
            \input{5 - research plan/main.tex}
            \input{6 - summary/main.tex}
    
    
    %\section{}
    \newpage
    \pagenumbering{gobble}
        \printbibliography


    \newpage
    \pagenumbering{roman}
    \appendix
        \part{Appendices}
            \input{8 - Hilbert complexes/main.tex}
            \input{9 - weak conservation proofs/main.tex}
\end{document}

            \documentclass[12pt, a4paper]{report}

\input{template/main.tex}

\title{\BA{Title in Progress...}}
\author{Boris Andrews}
\affil{Mathematical Institute, University of Oxford}
\date{\today}


\begin{document}
    \pagenumbering{gobble}
    \maketitle
    
    
    \begin{abstract}
        Magnetic confinement reactors---in particular tokamaks---offer one of the most promising options for achieving practical nuclear fusion, with the potential to provide virtually limitless, clean energy. The theoretical and numerical modeling of tokamak plasmas is simultaneously an essential component of effective reactor design, and a great research barrier. Tokamak operational conditions exhibit comparatively low Knudsen numbers. Kinetic effects, including kinetic waves and instabilities, Landau damping, bump-on-tail instabilities and more, are therefore highly influential in tokamak plasma dynamics. Purely fluid models are inherently incapable of capturing these effects, whereas the high dimensionality in purely kinetic models render them practically intractable for most relevant purposes.

        We consider a $\delta\!f$ decomposition model, with a macroscopic fluid background and microscopic kinetic correction, both fully coupled to each other. A similar manner of discretization is proposed to that used in the recent \texttt{STRUPHY} code \cite{Holderied_Possanner_Wang_2021, Holderied_2022, Li_et_al_2023} with a finite-element model for the background and a pseudo-particle/PiC model for the correction.

        The fluid background satisfies the full, non-linear, resistive, compressible, Hall MHD equations. \cite{Laakmann_Hu_Farrell_2022} introduces finite-element(-in-space) implicit timesteppers for the incompressible analogue to this system with structure-preserving (SP) properties in the ideal case, alongside parameter-robust preconditioners. We show that these timesteppers can derive from a finite-element-in-time (FET) (and finite-element-in-space) interpretation. The benefits of this reformulation are discussed, including the derivation of timesteppers that are higher order in time, and the quantifiable dissipative SP properties in the non-ideal, resistive case.
        
        We discuss possible options for extending this FET approach to timesteppers for the compressible case.

        The kinetic corrections satisfy linearized Boltzmann equations. Using a Lénard--Bernstein collision operator, these take Fokker--Planck-like forms \cite{Fokker_1914, Planck_1917} wherein pseudo-particles in the numerical model obey the neoclassical transport equations, with particle-independent Brownian drift terms. This offers a rigorous methodology for incorporating collisions into the particle transport model, without coupling the equations of motions for each particle.
        
        Works by Chen, Chacón et al. \cite{Chen_Chacón_Barnes_2011, Chacón_Chen_Barnes_2013, Chen_Chacón_2014, Chen_Chacón_2015} have developed structure-preserving particle pushers for neoclassical transport in the Vlasov equations, derived from Crank--Nicolson integrators. We show these too can can derive from a FET interpretation, similarly offering potential extensions to higher-order-in-time particle pushers. The FET formulation is used also to consider how the stochastic drift terms can be incorporated into the pushers. Stochastic gyrokinetic expansions are also discussed.

        Different options for the numerical implementation of these schemes are considered.

        Due to the efficacy of FET in the development of SP timesteppers for both the fluid and kinetic component, we hope this approach will prove effective in the future for developing SP timesteppers for the full hybrid model. We hope this will give us the opportunity to incorporate previously inaccessible kinetic effects into the highly effective, modern, finite-element MHD models.
    \end{abstract}
    
    
    \newpage
    \tableofcontents
    
    
    \newpage
    \pagenumbering{arabic}
    %\linenumbers\renewcommand\thelinenumber{\color{black!50}\arabic{linenumber}}
            \input{0 - introduction/main.tex}
        \part{Research}
            \input{1 - low-noise PiC models/main.tex}
            \input{2 - kinetic component/main.tex}
            \input{3 - fluid component/main.tex}
            \input{4 - numerical implementation/main.tex}
        \part{Project Overview}
            \input{5 - research plan/main.tex}
            \input{6 - summary/main.tex}
    
    
    %\section{}
    \newpage
    \pagenumbering{gobble}
        \printbibliography


    \newpage
    \pagenumbering{roman}
    \appendix
        \part{Appendices}
            \input{8 - Hilbert complexes/main.tex}
            \input{9 - weak conservation proofs/main.tex}
\end{document}

            \documentclass[12pt, a4paper]{report}

\input{template/main.tex}

\title{\BA{Title in Progress...}}
\author{Boris Andrews}
\affil{Mathematical Institute, University of Oxford}
\date{\today}


\begin{document}
    \pagenumbering{gobble}
    \maketitle
    
    
    \begin{abstract}
        Magnetic confinement reactors---in particular tokamaks---offer one of the most promising options for achieving practical nuclear fusion, with the potential to provide virtually limitless, clean energy. The theoretical and numerical modeling of tokamak plasmas is simultaneously an essential component of effective reactor design, and a great research barrier. Tokamak operational conditions exhibit comparatively low Knudsen numbers. Kinetic effects, including kinetic waves and instabilities, Landau damping, bump-on-tail instabilities and more, are therefore highly influential in tokamak plasma dynamics. Purely fluid models are inherently incapable of capturing these effects, whereas the high dimensionality in purely kinetic models render them practically intractable for most relevant purposes.

        We consider a $\delta\!f$ decomposition model, with a macroscopic fluid background and microscopic kinetic correction, both fully coupled to each other. A similar manner of discretization is proposed to that used in the recent \texttt{STRUPHY} code \cite{Holderied_Possanner_Wang_2021, Holderied_2022, Li_et_al_2023} with a finite-element model for the background and a pseudo-particle/PiC model for the correction.

        The fluid background satisfies the full, non-linear, resistive, compressible, Hall MHD equations. \cite{Laakmann_Hu_Farrell_2022} introduces finite-element(-in-space) implicit timesteppers for the incompressible analogue to this system with structure-preserving (SP) properties in the ideal case, alongside parameter-robust preconditioners. We show that these timesteppers can derive from a finite-element-in-time (FET) (and finite-element-in-space) interpretation. The benefits of this reformulation are discussed, including the derivation of timesteppers that are higher order in time, and the quantifiable dissipative SP properties in the non-ideal, resistive case.
        
        We discuss possible options for extending this FET approach to timesteppers for the compressible case.

        The kinetic corrections satisfy linearized Boltzmann equations. Using a Lénard--Bernstein collision operator, these take Fokker--Planck-like forms \cite{Fokker_1914, Planck_1917} wherein pseudo-particles in the numerical model obey the neoclassical transport equations, with particle-independent Brownian drift terms. This offers a rigorous methodology for incorporating collisions into the particle transport model, without coupling the equations of motions for each particle.
        
        Works by Chen, Chacón et al. \cite{Chen_Chacón_Barnes_2011, Chacón_Chen_Barnes_2013, Chen_Chacón_2014, Chen_Chacón_2015} have developed structure-preserving particle pushers for neoclassical transport in the Vlasov equations, derived from Crank--Nicolson integrators. We show these too can can derive from a FET interpretation, similarly offering potential extensions to higher-order-in-time particle pushers. The FET formulation is used also to consider how the stochastic drift terms can be incorporated into the pushers. Stochastic gyrokinetic expansions are also discussed.

        Different options for the numerical implementation of these schemes are considered.

        Due to the efficacy of FET in the development of SP timesteppers for both the fluid and kinetic component, we hope this approach will prove effective in the future for developing SP timesteppers for the full hybrid model. We hope this will give us the opportunity to incorporate previously inaccessible kinetic effects into the highly effective, modern, finite-element MHD models.
    \end{abstract}
    
    
    \newpage
    \tableofcontents
    
    
    \newpage
    \pagenumbering{arabic}
    %\linenumbers\renewcommand\thelinenumber{\color{black!50}\arabic{linenumber}}
            \input{0 - introduction/main.tex}
        \part{Research}
            \input{1 - low-noise PiC models/main.tex}
            \input{2 - kinetic component/main.tex}
            \input{3 - fluid component/main.tex}
            \input{4 - numerical implementation/main.tex}
        \part{Project Overview}
            \input{5 - research plan/main.tex}
            \input{6 - summary/main.tex}
    
    
    %\section{}
    \newpage
    \pagenumbering{gobble}
        \printbibliography


    \newpage
    \pagenumbering{roman}
    \appendix
        \part{Appendices}
            \input{8 - Hilbert complexes/main.tex}
            \input{9 - weak conservation proofs/main.tex}
\end{document}

            \documentclass[12pt, a4paper]{report}

\input{template/main.tex}

\title{\BA{Title in Progress...}}
\author{Boris Andrews}
\affil{Mathematical Institute, University of Oxford}
\date{\today}


\begin{document}
    \pagenumbering{gobble}
    \maketitle
    
    
    \begin{abstract}
        Magnetic confinement reactors---in particular tokamaks---offer one of the most promising options for achieving practical nuclear fusion, with the potential to provide virtually limitless, clean energy. The theoretical and numerical modeling of tokamak plasmas is simultaneously an essential component of effective reactor design, and a great research barrier. Tokamak operational conditions exhibit comparatively low Knudsen numbers. Kinetic effects, including kinetic waves and instabilities, Landau damping, bump-on-tail instabilities and more, are therefore highly influential in tokamak plasma dynamics. Purely fluid models are inherently incapable of capturing these effects, whereas the high dimensionality in purely kinetic models render them practically intractable for most relevant purposes.

        We consider a $\delta\!f$ decomposition model, with a macroscopic fluid background and microscopic kinetic correction, both fully coupled to each other. A similar manner of discretization is proposed to that used in the recent \texttt{STRUPHY} code \cite{Holderied_Possanner_Wang_2021, Holderied_2022, Li_et_al_2023} with a finite-element model for the background and a pseudo-particle/PiC model for the correction.

        The fluid background satisfies the full, non-linear, resistive, compressible, Hall MHD equations. \cite{Laakmann_Hu_Farrell_2022} introduces finite-element(-in-space) implicit timesteppers for the incompressible analogue to this system with structure-preserving (SP) properties in the ideal case, alongside parameter-robust preconditioners. We show that these timesteppers can derive from a finite-element-in-time (FET) (and finite-element-in-space) interpretation. The benefits of this reformulation are discussed, including the derivation of timesteppers that are higher order in time, and the quantifiable dissipative SP properties in the non-ideal, resistive case.
        
        We discuss possible options for extending this FET approach to timesteppers for the compressible case.

        The kinetic corrections satisfy linearized Boltzmann equations. Using a Lénard--Bernstein collision operator, these take Fokker--Planck-like forms \cite{Fokker_1914, Planck_1917} wherein pseudo-particles in the numerical model obey the neoclassical transport equations, with particle-independent Brownian drift terms. This offers a rigorous methodology for incorporating collisions into the particle transport model, without coupling the equations of motions for each particle.
        
        Works by Chen, Chacón et al. \cite{Chen_Chacón_Barnes_2011, Chacón_Chen_Barnes_2013, Chen_Chacón_2014, Chen_Chacón_2015} have developed structure-preserving particle pushers for neoclassical transport in the Vlasov equations, derived from Crank--Nicolson integrators. We show these too can can derive from a FET interpretation, similarly offering potential extensions to higher-order-in-time particle pushers. The FET formulation is used also to consider how the stochastic drift terms can be incorporated into the pushers. Stochastic gyrokinetic expansions are also discussed.

        Different options for the numerical implementation of these schemes are considered.

        Due to the efficacy of FET in the development of SP timesteppers for both the fluid and kinetic component, we hope this approach will prove effective in the future for developing SP timesteppers for the full hybrid model. We hope this will give us the opportunity to incorporate previously inaccessible kinetic effects into the highly effective, modern, finite-element MHD models.
    \end{abstract}
    
    
    \newpage
    \tableofcontents
    
    
    \newpage
    \pagenumbering{arabic}
    %\linenumbers\renewcommand\thelinenumber{\color{black!50}\arabic{linenumber}}
            \input{0 - introduction/main.tex}
        \part{Research}
            \input{1 - low-noise PiC models/main.tex}
            \input{2 - kinetic component/main.tex}
            \input{3 - fluid component/main.tex}
            \input{4 - numerical implementation/main.tex}
        \part{Project Overview}
            \input{5 - research plan/main.tex}
            \input{6 - summary/main.tex}
    
    
    %\section{}
    \newpage
    \pagenumbering{gobble}
        \printbibliography


    \newpage
    \pagenumbering{roman}
    \appendix
        \part{Appendices}
            \input{8 - Hilbert complexes/main.tex}
            \input{9 - weak conservation proofs/main.tex}
\end{document}

        \part{Project Overview}
            \documentclass[12pt, a4paper]{report}

\input{template/main.tex}

\title{\BA{Title in Progress...}}
\author{Boris Andrews}
\affil{Mathematical Institute, University of Oxford}
\date{\today}


\begin{document}
    \pagenumbering{gobble}
    \maketitle
    
    
    \begin{abstract}
        Magnetic confinement reactors---in particular tokamaks---offer one of the most promising options for achieving practical nuclear fusion, with the potential to provide virtually limitless, clean energy. The theoretical and numerical modeling of tokamak plasmas is simultaneously an essential component of effective reactor design, and a great research barrier. Tokamak operational conditions exhibit comparatively low Knudsen numbers. Kinetic effects, including kinetic waves and instabilities, Landau damping, bump-on-tail instabilities and more, are therefore highly influential in tokamak plasma dynamics. Purely fluid models are inherently incapable of capturing these effects, whereas the high dimensionality in purely kinetic models render them practically intractable for most relevant purposes.

        We consider a $\delta\!f$ decomposition model, with a macroscopic fluid background and microscopic kinetic correction, both fully coupled to each other. A similar manner of discretization is proposed to that used in the recent \texttt{STRUPHY} code \cite{Holderied_Possanner_Wang_2021, Holderied_2022, Li_et_al_2023} with a finite-element model for the background and a pseudo-particle/PiC model for the correction.

        The fluid background satisfies the full, non-linear, resistive, compressible, Hall MHD equations. \cite{Laakmann_Hu_Farrell_2022} introduces finite-element(-in-space) implicit timesteppers for the incompressible analogue to this system with structure-preserving (SP) properties in the ideal case, alongside parameter-robust preconditioners. We show that these timesteppers can derive from a finite-element-in-time (FET) (and finite-element-in-space) interpretation. The benefits of this reformulation are discussed, including the derivation of timesteppers that are higher order in time, and the quantifiable dissipative SP properties in the non-ideal, resistive case.
        
        We discuss possible options for extending this FET approach to timesteppers for the compressible case.

        The kinetic corrections satisfy linearized Boltzmann equations. Using a Lénard--Bernstein collision operator, these take Fokker--Planck-like forms \cite{Fokker_1914, Planck_1917} wherein pseudo-particles in the numerical model obey the neoclassical transport equations, with particle-independent Brownian drift terms. This offers a rigorous methodology for incorporating collisions into the particle transport model, without coupling the equations of motions for each particle.
        
        Works by Chen, Chacón et al. \cite{Chen_Chacón_Barnes_2011, Chacón_Chen_Barnes_2013, Chen_Chacón_2014, Chen_Chacón_2015} have developed structure-preserving particle pushers for neoclassical transport in the Vlasov equations, derived from Crank--Nicolson integrators. We show these too can can derive from a FET interpretation, similarly offering potential extensions to higher-order-in-time particle pushers. The FET formulation is used also to consider how the stochastic drift terms can be incorporated into the pushers. Stochastic gyrokinetic expansions are also discussed.

        Different options for the numerical implementation of these schemes are considered.

        Due to the efficacy of FET in the development of SP timesteppers for both the fluid and kinetic component, we hope this approach will prove effective in the future for developing SP timesteppers for the full hybrid model. We hope this will give us the opportunity to incorporate previously inaccessible kinetic effects into the highly effective, modern, finite-element MHD models.
    \end{abstract}
    
    
    \newpage
    \tableofcontents
    
    
    \newpage
    \pagenumbering{arabic}
    %\linenumbers\renewcommand\thelinenumber{\color{black!50}\arabic{linenumber}}
            \input{0 - introduction/main.tex}
        \part{Research}
            \input{1 - low-noise PiC models/main.tex}
            \input{2 - kinetic component/main.tex}
            \input{3 - fluid component/main.tex}
            \input{4 - numerical implementation/main.tex}
        \part{Project Overview}
            \input{5 - research plan/main.tex}
            \input{6 - summary/main.tex}
    
    
    %\section{}
    \newpage
    \pagenumbering{gobble}
        \printbibliography


    \newpage
    \pagenumbering{roman}
    \appendix
        \part{Appendices}
            \input{8 - Hilbert complexes/main.tex}
            \input{9 - weak conservation proofs/main.tex}
\end{document}

            \documentclass[12pt, a4paper]{report}

\input{template/main.tex}

\title{\BA{Title in Progress...}}
\author{Boris Andrews}
\affil{Mathematical Institute, University of Oxford}
\date{\today}


\begin{document}
    \pagenumbering{gobble}
    \maketitle
    
    
    \begin{abstract}
        Magnetic confinement reactors---in particular tokamaks---offer one of the most promising options for achieving practical nuclear fusion, with the potential to provide virtually limitless, clean energy. The theoretical and numerical modeling of tokamak plasmas is simultaneously an essential component of effective reactor design, and a great research barrier. Tokamak operational conditions exhibit comparatively low Knudsen numbers. Kinetic effects, including kinetic waves and instabilities, Landau damping, bump-on-tail instabilities and more, are therefore highly influential in tokamak plasma dynamics. Purely fluid models are inherently incapable of capturing these effects, whereas the high dimensionality in purely kinetic models render them practically intractable for most relevant purposes.

        We consider a $\delta\!f$ decomposition model, with a macroscopic fluid background and microscopic kinetic correction, both fully coupled to each other. A similar manner of discretization is proposed to that used in the recent \texttt{STRUPHY} code \cite{Holderied_Possanner_Wang_2021, Holderied_2022, Li_et_al_2023} with a finite-element model for the background and a pseudo-particle/PiC model for the correction.

        The fluid background satisfies the full, non-linear, resistive, compressible, Hall MHD equations. \cite{Laakmann_Hu_Farrell_2022} introduces finite-element(-in-space) implicit timesteppers for the incompressible analogue to this system with structure-preserving (SP) properties in the ideal case, alongside parameter-robust preconditioners. We show that these timesteppers can derive from a finite-element-in-time (FET) (and finite-element-in-space) interpretation. The benefits of this reformulation are discussed, including the derivation of timesteppers that are higher order in time, and the quantifiable dissipative SP properties in the non-ideal, resistive case.
        
        We discuss possible options for extending this FET approach to timesteppers for the compressible case.

        The kinetic corrections satisfy linearized Boltzmann equations. Using a Lénard--Bernstein collision operator, these take Fokker--Planck-like forms \cite{Fokker_1914, Planck_1917} wherein pseudo-particles in the numerical model obey the neoclassical transport equations, with particle-independent Brownian drift terms. This offers a rigorous methodology for incorporating collisions into the particle transport model, without coupling the equations of motions for each particle.
        
        Works by Chen, Chacón et al. \cite{Chen_Chacón_Barnes_2011, Chacón_Chen_Barnes_2013, Chen_Chacón_2014, Chen_Chacón_2015} have developed structure-preserving particle pushers for neoclassical transport in the Vlasov equations, derived from Crank--Nicolson integrators. We show these too can can derive from a FET interpretation, similarly offering potential extensions to higher-order-in-time particle pushers. The FET formulation is used also to consider how the stochastic drift terms can be incorporated into the pushers. Stochastic gyrokinetic expansions are also discussed.

        Different options for the numerical implementation of these schemes are considered.

        Due to the efficacy of FET in the development of SP timesteppers for both the fluid and kinetic component, we hope this approach will prove effective in the future for developing SP timesteppers for the full hybrid model. We hope this will give us the opportunity to incorporate previously inaccessible kinetic effects into the highly effective, modern, finite-element MHD models.
    \end{abstract}
    
    
    \newpage
    \tableofcontents
    
    
    \newpage
    \pagenumbering{arabic}
    %\linenumbers\renewcommand\thelinenumber{\color{black!50}\arabic{linenumber}}
            \input{0 - introduction/main.tex}
        \part{Research}
            \input{1 - low-noise PiC models/main.tex}
            \input{2 - kinetic component/main.tex}
            \input{3 - fluid component/main.tex}
            \input{4 - numerical implementation/main.tex}
        \part{Project Overview}
            \input{5 - research plan/main.tex}
            \input{6 - summary/main.tex}
    
    
    %\section{}
    \newpage
    \pagenumbering{gobble}
        \printbibliography


    \newpage
    \pagenumbering{roman}
    \appendix
        \part{Appendices}
            \input{8 - Hilbert complexes/main.tex}
            \input{9 - weak conservation proofs/main.tex}
\end{document}

    
    
    %\section{}
    \newpage
    \pagenumbering{gobble}
        \printbibliography


    \newpage
    \pagenumbering{roman}
    \appendix
        \part{Appendices}
            \documentclass[12pt, a4paper]{report}

\input{template/main.tex}

\title{\BA{Title in Progress...}}
\author{Boris Andrews}
\affil{Mathematical Institute, University of Oxford}
\date{\today}


\begin{document}
    \pagenumbering{gobble}
    \maketitle
    
    
    \begin{abstract}
        Magnetic confinement reactors---in particular tokamaks---offer one of the most promising options for achieving practical nuclear fusion, with the potential to provide virtually limitless, clean energy. The theoretical and numerical modeling of tokamak plasmas is simultaneously an essential component of effective reactor design, and a great research barrier. Tokamak operational conditions exhibit comparatively low Knudsen numbers. Kinetic effects, including kinetic waves and instabilities, Landau damping, bump-on-tail instabilities and more, are therefore highly influential in tokamak plasma dynamics. Purely fluid models are inherently incapable of capturing these effects, whereas the high dimensionality in purely kinetic models render them practically intractable for most relevant purposes.

        We consider a $\delta\!f$ decomposition model, with a macroscopic fluid background and microscopic kinetic correction, both fully coupled to each other. A similar manner of discretization is proposed to that used in the recent \texttt{STRUPHY} code \cite{Holderied_Possanner_Wang_2021, Holderied_2022, Li_et_al_2023} with a finite-element model for the background and a pseudo-particle/PiC model for the correction.

        The fluid background satisfies the full, non-linear, resistive, compressible, Hall MHD equations. \cite{Laakmann_Hu_Farrell_2022} introduces finite-element(-in-space) implicit timesteppers for the incompressible analogue to this system with structure-preserving (SP) properties in the ideal case, alongside parameter-robust preconditioners. We show that these timesteppers can derive from a finite-element-in-time (FET) (and finite-element-in-space) interpretation. The benefits of this reformulation are discussed, including the derivation of timesteppers that are higher order in time, and the quantifiable dissipative SP properties in the non-ideal, resistive case.
        
        We discuss possible options for extending this FET approach to timesteppers for the compressible case.

        The kinetic corrections satisfy linearized Boltzmann equations. Using a Lénard--Bernstein collision operator, these take Fokker--Planck-like forms \cite{Fokker_1914, Planck_1917} wherein pseudo-particles in the numerical model obey the neoclassical transport equations, with particle-independent Brownian drift terms. This offers a rigorous methodology for incorporating collisions into the particle transport model, without coupling the equations of motions for each particle.
        
        Works by Chen, Chacón et al. \cite{Chen_Chacón_Barnes_2011, Chacón_Chen_Barnes_2013, Chen_Chacón_2014, Chen_Chacón_2015} have developed structure-preserving particle pushers for neoclassical transport in the Vlasov equations, derived from Crank--Nicolson integrators. We show these too can can derive from a FET interpretation, similarly offering potential extensions to higher-order-in-time particle pushers. The FET formulation is used also to consider how the stochastic drift terms can be incorporated into the pushers. Stochastic gyrokinetic expansions are also discussed.

        Different options for the numerical implementation of these schemes are considered.

        Due to the efficacy of FET in the development of SP timesteppers for both the fluid and kinetic component, we hope this approach will prove effective in the future for developing SP timesteppers for the full hybrid model. We hope this will give us the opportunity to incorporate previously inaccessible kinetic effects into the highly effective, modern, finite-element MHD models.
    \end{abstract}
    
    
    \newpage
    \tableofcontents
    
    
    \newpage
    \pagenumbering{arabic}
    %\linenumbers\renewcommand\thelinenumber{\color{black!50}\arabic{linenumber}}
            \input{0 - introduction/main.tex}
        \part{Research}
            \input{1 - low-noise PiC models/main.tex}
            \input{2 - kinetic component/main.tex}
            \input{3 - fluid component/main.tex}
            \input{4 - numerical implementation/main.tex}
        \part{Project Overview}
            \input{5 - research plan/main.tex}
            \input{6 - summary/main.tex}
    
    
    %\section{}
    \newpage
    \pagenumbering{gobble}
        \printbibliography


    \newpage
    \pagenumbering{roman}
    \appendix
        \part{Appendices}
            \input{8 - Hilbert complexes/main.tex}
            \input{9 - weak conservation proofs/main.tex}
\end{document}

            \documentclass[12pt, a4paper]{report}

\input{template/main.tex}

\title{\BA{Title in Progress...}}
\author{Boris Andrews}
\affil{Mathematical Institute, University of Oxford}
\date{\today}


\begin{document}
    \pagenumbering{gobble}
    \maketitle
    
    
    \begin{abstract}
        Magnetic confinement reactors---in particular tokamaks---offer one of the most promising options for achieving practical nuclear fusion, with the potential to provide virtually limitless, clean energy. The theoretical and numerical modeling of tokamak plasmas is simultaneously an essential component of effective reactor design, and a great research barrier. Tokamak operational conditions exhibit comparatively low Knudsen numbers. Kinetic effects, including kinetic waves and instabilities, Landau damping, bump-on-tail instabilities and more, are therefore highly influential in tokamak plasma dynamics. Purely fluid models are inherently incapable of capturing these effects, whereas the high dimensionality in purely kinetic models render them practically intractable for most relevant purposes.

        We consider a $\delta\!f$ decomposition model, with a macroscopic fluid background and microscopic kinetic correction, both fully coupled to each other. A similar manner of discretization is proposed to that used in the recent \texttt{STRUPHY} code \cite{Holderied_Possanner_Wang_2021, Holderied_2022, Li_et_al_2023} with a finite-element model for the background and a pseudo-particle/PiC model for the correction.

        The fluid background satisfies the full, non-linear, resistive, compressible, Hall MHD equations. \cite{Laakmann_Hu_Farrell_2022} introduces finite-element(-in-space) implicit timesteppers for the incompressible analogue to this system with structure-preserving (SP) properties in the ideal case, alongside parameter-robust preconditioners. We show that these timesteppers can derive from a finite-element-in-time (FET) (and finite-element-in-space) interpretation. The benefits of this reformulation are discussed, including the derivation of timesteppers that are higher order in time, and the quantifiable dissipative SP properties in the non-ideal, resistive case.
        
        We discuss possible options for extending this FET approach to timesteppers for the compressible case.

        The kinetic corrections satisfy linearized Boltzmann equations. Using a Lénard--Bernstein collision operator, these take Fokker--Planck-like forms \cite{Fokker_1914, Planck_1917} wherein pseudo-particles in the numerical model obey the neoclassical transport equations, with particle-independent Brownian drift terms. This offers a rigorous methodology for incorporating collisions into the particle transport model, without coupling the equations of motions for each particle.
        
        Works by Chen, Chacón et al. \cite{Chen_Chacón_Barnes_2011, Chacón_Chen_Barnes_2013, Chen_Chacón_2014, Chen_Chacón_2015} have developed structure-preserving particle pushers for neoclassical transport in the Vlasov equations, derived from Crank--Nicolson integrators. We show these too can can derive from a FET interpretation, similarly offering potential extensions to higher-order-in-time particle pushers. The FET formulation is used also to consider how the stochastic drift terms can be incorporated into the pushers. Stochastic gyrokinetic expansions are also discussed.

        Different options for the numerical implementation of these schemes are considered.

        Due to the efficacy of FET in the development of SP timesteppers for both the fluid and kinetic component, we hope this approach will prove effective in the future for developing SP timesteppers for the full hybrid model. We hope this will give us the opportunity to incorporate previously inaccessible kinetic effects into the highly effective, modern, finite-element MHD models.
    \end{abstract}
    
    
    \newpage
    \tableofcontents
    
    
    \newpage
    \pagenumbering{arabic}
    %\linenumbers\renewcommand\thelinenumber{\color{black!50}\arabic{linenumber}}
            \input{0 - introduction/main.tex}
        \part{Research}
            \input{1 - low-noise PiC models/main.tex}
            \input{2 - kinetic component/main.tex}
            \input{3 - fluid component/main.tex}
            \input{4 - numerical implementation/main.tex}
        \part{Project Overview}
            \input{5 - research plan/main.tex}
            \input{6 - summary/main.tex}
    
    
    %\section{}
    \newpage
    \pagenumbering{gobble}
        \printbibliography


    \newpage
    \pagenumbering{roman}
    \appendix
        \part{Appendices}
            \input{8 - Hilbert complexes/main.tex}
            \input{9 - weak conservation proofs/main.tex}
\end{document}

\end{document}

            \documentclass[12pt, a4paper]{report}

\documentclass[12pt, a4paper]{report}

\input{template/main.tex}

\title{\BA{Title in Progress...}}
\author{Boris Andrews}
\affil{Mathematical Institute, University of Oxford}
\date{\today}


\begin{document}
    \pagenumbering{gobble}
    \maketitle
    
    
    \begin{abstract}
        Magnetic confinement reactors---in particular tokamaks---offer one of the most promising options for achieving practical nuclear fusion, with the potential to provide virtually limitless, clean energy. The theoretical and numerical modeling of tokamak plasmas is simultaneously an essential component of effective reactor design, and a great research barrier. Tokamak operational conditions exhibit comparatively low Knudsen numbers. Kinetic effects, including kinetic waves and instabilities, Landau damping, bump-on-tail instabilities and more, are therefore highly influential in tokamak plasma dynamics. Purely fluid models are inherently incapable of capturing these effects, whereas the high dimensionality in purely kinetic models render them practically intractable for most relevant purposes.

        We consider a $\delta\!f$ decomposition model, with a macroscopic fluid background and microscopic kinetic correction, both fully coupled to each other. A similar manner of discretization is proposed to that used in the recent \texttt{STRUPHY} code \cite{Holderied_Possanner_Wang_2021, Holderied_2022, Li_et_al_2023} with a finite-element model for the background and a pseudo-particle/PiC model for the correction.

        The fluid background satisfies the full, non-linear, resistive, compressible, Hall MHD equations. \cite{Laakmann_Hu_Farrell_2022} introduces finite-element(-in-space) implicit timesteppers for the incompressible analogue to this system with structure-preserving (SP) properties in the ideal case, alongside parameter-robust preconditioners. We show that these timesteppers can derive from a finite-element-in-time (FET) (and finite-element-in-space) interpretation. The benefits of this reformulation are discussed, including the derivation of timesteppers that are higher order in time, and the quantifiable dissipative SP properties in the non-ideal, resistive case.
        
        We discuss possible options for extending this FET approach to timesteppers for the compressible case.

        The kinetic corrections satisfy linearized Boltzmann equations. Using a Lénard--Bernstein collision operator, these take Fokker--Planck-like forms \cite{Fokker_1914, Planck_1917} wherein pseudo-particles in the numerical model obey the neoclassical transport equations, with particle-independent Brownian drift terms. This offers a rigorous methodology for incorporating collisions into the particle transport model, without coupling the equations of motions for each particle.
        
        Works by Chen, Chacón et al. \cite{Chen_Chacón_Barnes_2011, Chacón_Chen_Barnes_2013, Chen_Chacón_2014, Chen_Chacón_2015} have developed structure-preserving particle pushers for neoclassical transport in the Vlasov equations, derived from Crank--Nicolson integrators. We show these too can can derive from a FET interpretation, similarly offering potential extensions to higher-order-in-time particle pushers. The FET formulation is used also to consider how the stochastic drift terms can be incorporated into the pushers. Stochastic gyrokinetic expansions are also discussed.

        Different options for the numerical implementation of these schemes are considered.

        Due to the efficacy of FET in the development of SP timesteppers for both the fluid and kinetic component, we hope this approach will prove effective in the future for developing SP timesteppers for the full hybrid model. We hope this will give us the opportunity to incorporate previously inaccessible kinetic effects into the highly effective, modern, finite-element MHD models.
    \end{abstract}
    
    
    \newpage
    \tableofcontents
    
    
    \newpage
    \pagenumbering{arabic}
    %\linenumbers\renewcommand\thelinenumber{\color{black!50}\arabic{linenumber}}
            \input{0 - introduction/main.tex}
        \part{Research}
            \input{1 - low-noise PiC models/main.tex}
            \input{2 - kinetic component/main.tex}
            \input{3 - fluid component/main.tex}
            \input{4 - numerical implementation/main.tex}
        \part{Project Overview}
            \input{5 - research plan/main.tex}
            \input{6 - summary/main.tex}
    
    
    %\section{}
    \newpage
    \pagenumbering{gobble}
        \printbibliography


    \newpage
    \pagenumbering{roman}
    \appendix
        \part{Appendices}
            \input{8 - Hilbert complexes/main.tex}
            \input{9 - weak conservation proofs/main.tex}
\end{document}


\title{\BA{Title in Progress...}}
\author{Boris Andrews}
\affil{Mathematical Institute, University of Oxford}
\date{\today}


\begin{document}
    \pagenumbering{gobble}
    \maketitle
    
    
    \begin{abstract}
        Magnetic confinement reactors---in particular tokamaks---offer one of the most promising options for achieving practical nuclear fusion, with the potential to provide virtually limitless, clean energy. The theoretical and numerical modeling of tokamak plasmas is simultaneously an essential component of effective reactor design, and a great research barrier. Tokamak operational conditions exhibit comparatively low Knudsen numbers. Kinetic effects, including kinetic waves and instabilities, Landau damping, bump-on-tail instabilities and more, are therefore highly influential in tokamak plasma dynamics. Purely fluid models are inherently incapable of capturing these effects, whereas the high dimensionality in purely kinetic models render them practically intractable for most relevant purposes.

        We consider a $\delta\!f$ decomposition model, with a macroscopic fluid background and microscopic kinetic correction, both fully coupled to each other. A similar manner of discretization is proposed to that used in the recent \texttt{STRUPHY} code \cite{Holderied_Possanner_Wang_2021, Holderied_2022, Li_et_al_2023} with a finite-element model for the background and a pseudo-particle/PiC model for the correction.

        The fluid background satisfies the full, non-linear, resistive, compressible, Hall MHD equations. \cite{Laakmann_Hu_Farrell_2022} introduces finite-element(-in-space) implicit timesteppers for the incompressible analogue to this system with structure-preserving (SP) properties in the ideal case, alongside parameter-robust preconditioners. We show that these timesteppers can derive from a finite-element-in-time (FET) (and finite-element-in-space) interpretation. The benefits of this reformulation are discussed, including the derivation of timesteppers that are higher order in time, and the quantifiable dissipative SP properties in the non-ideal, resistive case.
        
        We discuss possible options for extending this FET approach to timesteppers for the compressible case.

        The kinetic corrections satisfy linearized Boltzmann equations. Using a Lénard--Bernstein collision operator, these take Fokker--Planck-like forms \cite{Fokker_1914, Planck_1917} wherein pseudo-particles in the numerical model obey the neoclassical transport equations, with particle-independent Brownian drift terms. This offers a rigorous methodology for incorporating collisions into the particle transport model, without coupling the equations of motions for each particle.
        
        Works by Chen, Chacón et al. \cite{Chen_Chacón_Barnes_2011, Chacón_Chen_Barnes_2013, Chen_Chacón_2014, Chen_Chacón_2015} have developed structure-preserving particle pushers for neoclassical transport in the Vlasov equations, derived from Crank--Nicolson integrators. We show these too can can derive from a FET interpretation, similarly offering potential extensions to higher-order-in-time particle pushers. The FET formulation is used also to consider how the stochastic drift terms can be incorporated into the pushers. Stochastic gyrokinetic expansions are also discussed.

        Different options for the numerical implementation of these schemes are considered.

        Due to the efficacy of FET in the development of SP timesteppers for both the fluid and kinetic component, we hope this approach will prove effective in the future for developing SP timesteppers for the full hybrid model. We hope this will give us the opportunity to incorporate previously inaccessible kinetic effects into the highly effective, modern, finite-element MHD models.
    \end{abstract}
    
    
    \newpage
    \tableofcontents
    
    
    \newpage
    \pagenumbering{arabic}
    %\linenumbers\renewcommand\thelinenumber{\color{black!50}\arabic{linenumber}}
            \documentclass[12pt, a4paper]{report}

\input{template/main.tex}

\title{\BA{Title in Progress...}}
\author{Boris Andrews}
\affil{Mathematical Institute, University of Oxford}
\date{\today}


\begin{document}
    \pagenumbering{gobble}
    \maketitle
    
    
    \begin{abstract}
        Magnetic confinement reactors---in particular tokamaks---offer one of the most promising options for achieving practical nuclear fusion, with the potential to provide virtually limitless, clean energy. The theoretical and numerical modeling of tokamak plasmas is simultaneously an essential component of effective reactor design, and a great research barrier. Tokamak operational conditions exhibit comparatively low Knudsen numbers. Kinetic effects, including kinetic waves and instabilities, Landau damping, bump-on-tail instabilities and more, are therefore highly influential in tokamak plasma dynamics. Purely fluid models are inherently incapable of capturing these effects, whereas the high dimensionality in purely kinetic models render them practically intractable for most relevant purposes.

        We consider a $\delta\!f$ decomposition model, with a macroscopic fluid background and microscopic kinetic correction, both fully coupled to each other. A similar manner of discretization is proposed to that used in the recent \texttt{STRUPHY} code \cite{Holderied_Possanner_Wang_2021, Holderied_2022, Li_et_al_2023} with a finite-element model for the background and a pseudo-particle/PiC model for the correction.

        The fluid background satisfies the full, non-linear, resistive, compressible, Hall MHD equations. \cite{Laakmann_Hu_Farrell_2022} introduces finite-element(-in-space) implicit timesteppers for the incompressible analogue to this system with structure-preserving (SP) properties in the ideal case, alongside parameter-robust preconditioners. We show that these timesteppers can derive from a finite-element-in-time (FET) (and finite-element-in-space) interpretation. The benefits of this reformulation are discussed, including the derivation of timesteppers that are higher order in time, and the quantifiable dissipative SP properties in the non-ideal, resistive case.
        
        We discuss possible options for extending this FET approach to timesteppers for the compressible case.

        The kinetic corrections satisfy linearized Boltzmann equations. Using a Lénard--Bernstein collision operator, these take Fokker--Planck-like forms \cite{Fokker_1914, Planck_1917} wherein pseudo-particles in the numerical model obey the neoclassical transport equations, with particle-independent Brownian drift terms. This offers a rigorous methodology for incorporating collisions into the particle transport model, without coupling the equations of motions for each particle.
        
        Works by Chen, Chacón et al. \cite{Chen_Chacón_Barnes_2011, Chacón_Chen_Barnes_2013, Chen_Chacón_2014, Chen_Chacón_2015} have developed structure-preserving particle pushers for neoclassical transport in the Vlasov equations, derived from Crank--Nicolson integrators. We show these too can can derive from a FET interpretation, similarly offering potential extensions to higher-order-in-time particle pushers. The FET formulation is used also to consider how the stochastic drift terms can be incorporated into the pushers. Stochastic gyrokinetic expansions are also discussed.

        Different options for the numerical implementation of these schemes are considered.

        Due to the efficacy of FET in the development of SP timesteppers for both the fluid and kinetic component, we hope this approach will prove effective in the future for developing SP timesteppers for the full hybrid model. We hope this will give us the opportunity to incorporate previously inaccessible kinetic effects into the highly effective, modern, finite-element MHD models.
    \end{abstract}
    
    
    \newpage
    \tableofcontents
    
    
    \newpage
    \pagenumbering{arabic}
    %\linenumbers\renewcommand\thelinenumber{\color{black!50}\arabic{linenumber}}
            \input{0 - introduction/main.tex}
        \part{Research}
            \input{1 - low-noise PiC models/main.tex}
            \input{2 - kinetic component/main.tex}
            \input{3 - fluid component/main.tex}
            \input{4 - numerical implementation/main.tex}
        \part{Project Overview}
            \input{5 - research plan/main.tex}
            \input{6 - summary/main.tex}
    
    
    %\section{}
    \newpage
    \pagenumbering{gobble}
        \printbibliography


    \newpage
    \pagenumbering{roman}
    \appendix
        \part{Appendices}
            \input{8 - Hilbert complexes/main.tex}
            \input{9 - weak conservation proofs/main.tex}
\end{document}

        \part{Research}
            \documentclass[12pt, a4paper]{report}

\input{template/main.tex}

\title{\BA{Title in Progress...}}
\author{Boris Andrews}
\affil{Mathematical Institute, University of Oxford}
\date{\today}


\begin{document}
    \pagenumbering{gobble}
    \maketitle
    
    
    \begin{abstract}
        Magnetic confinement reactors---in particular tokamaks---offer one of the most promising options for achieving practical nuclear fusion, with the potential to provide virtually limitless, clean energy. The theoretical and numerical modeling of tokamak plasmas is simultaneously an essential component of effective reactor design, and a great research barrier. Tokamak operational conditions exhibit comparatively low Knudsen numbers. Kinetic effects, including kinetic waves and instabilities, Landau damping, bump-on-tail instabilities and more, are therefore highly influential in tokamak plasma dynamics. Purely fluid models are inherently incapable of capturing these effects, whereas the high dimensionality in purely kinetic models render them practically intractable for most relevant purposes.

        We consider a $\delta\!f$ decomposition model, with a macroscopic fluid background and microscopic kinetic correction, both fully coupled to each other. A similar manner of discretization is proposed to that used in the recent \texttt{STRUPHY} code \cite{Holderied_Possanner_Wang_2021, Holderied_2022, Li_et_al_2023} with a finite-element model for the background and a pseudo-particle/PiC model for the correction.

        The fluid background satisfies the full, non-linear, resistive, compressible, Hall MHD equations. \cite{Laakmann_Hu_Farrell_2022} introduces finite-element(-in-space) implicit timesteppers for the incompressible analogue to this system with structure-preserving (SP) properties in the ideal case, alongside parameter-robust preconditioners. We show that these timesteppers can derive from a finite-element-in-time (FET) (and finite-element-in-space) interpretation. The benefits of this reformulation are discussed, including the derivation of timesteppers that are higher order in time, and the quantifiable dissipative SP properties in the non-ideal, resistive case.
        
        We discuss possible options for extending this FET approach to timesteppers for the compressible case.

        The kinetic corrections satisfy linearized Boltzmann equations. Using a Lénard--Bernstein collision operator, these take Fokker--Planck-like forms \cite{Fokker_1914, Planck_1917} wherein pseudo-particles in the numerical model obey the neoclassical transport equations, with particle-independent Brownian drift terms. This offers a rigorous methodology for incorporating collisions into the particle transport model, without coupling the equations of motions for each particle.
        
        Works by Chen, Chacón et al. \cite{Chen_Chacón_Barnes_2011, Chacón_Chen_Barnes_2013, Chen_Chacón_2014, Chen_Chacón_2015} have developed structure-preserving particle pushers for neoclassical transport in the Vlasov equations, derived from Crank--Nicolson integrators. We show these too can can derive from a FET interpretation, similarly offering potential extensions to higher-order-in-time particle pushers. The FET formulation is used also to consider how the stochastic drift terms can be incorporated into the pushers. Stochastic gyrokinetic expansions are also discussed.

        Different options for the numerical implementation of these schemes are considered.

        Due to the efficacy of FET in the development of SP timesteppers for both the fluid and kinetic component, we hope this approach will prove effective in the future for developing SP timesteppers for the full hybrid model. We hope this will give us the opportunity to incorporate previously inaccessible kinetic effects into the highly effective, modern, finite-element MHD models.
    \end{abstract}
    
    
    \newpage
    \tableofcontents
    
    
    \newpage
    \pagenumbering{arabic}
    %\linenumbers\renewcommand\thelinenumber{\color{black!50}\arabic{linenumber}}
            \input{0 - introduction/main.tex}
        \part{Research}
            \input{1 - low-noise PiC models/main.tex}
            \input{2 - kinetic component/main.tex}
            \input{3 - fluid component/main.tex}
            \input{4 - numerical implementation/main.tex}
        \part{Project Overview}
            \input{5 - research plan/main.tex}
            \input{6 - summary/main.tex}
    
    
    %\section{}
    \newpage
    \pagenumbering{gobble}
        \printbibliography


    \newpage
    \pagenumbering{roman}
    \appendix
        \part{Appendices}
            \input{8 - Hilbert complexes/main.tex}
            \input{9 - weak conservation proofs/main.tex}
\end{document}

            \documentclass[12pt, a4paper]{report}

\input{template/main.tex}

\title{\BA{Title in Progress...}}
\author{Boris Andrews}
\affil{Mathematical Institute, University of Oxford}
\date{\today}


\begin{document}
    \pagenumbering{gobble}
    \maketitle
    
    
    \begin{abstract}
        Magnetic confinement reactors---in particular tokamaks---offer one of the most promising options for achieving practical nuclear fusion, with the potential to provide virtually limitless, clean energy. The theoretical and numerical modeling of tokamak plasmas is simultaneously an essential component of effective reactor design, and a great research barrier. Tokamak operational conditions exhibit comparatively low Knudsen numbers. Kinetic effects, including kinetic waves and instabilities, Landau damping, bump-on-tail instabilities and more, are therefore highly influential in tokamak plasma dynamics. Purely fluid models are inherently incapable of capturing these effects, whereas the high dimensionality in purely kinetic models render them practically intractable for most relevant purposes.

        We consider a $\delta\!f$ decomposition model, with a macroscopic fluid background and microscopic kinetic correction, both fully coupled to each other. A similar manner of discretization is proposed to that used in the recent \texttt{STRUPHY} code \cite{Holderied_Possanner_Wang_2021, Holderied_2022, Li_et_al_2023} with a finite-element model for the background and a pseudo-particle/PiC model for the correction.

        The fluid background satisfies the full, non-linear, resistive, compressible, Hall MHD equations. \cite{Laakmann_Hu_Farrell_2022} introduces finite-element(-in-space) implicit timesteppers for the incompressible analogue to this system with structure-preserving (SP) properties in the ideal case, alongside parameter-robust preconditioners. We show that these timesteppers can derive from a finite-element-in-time (FET) (and finite-element-in-space) interpretation. The benefits of this reformulation are discussed, including the derivation of timesteppers that are higher order in time, and the quantifiable dissipative SP properties in the non-ideal, resistive case.
        
        We discuss possible options for extending this FET approach to timesteppers for the compressible case.

        The kinetic corrections satisfy linearized Boltzmann equations. Using a Lénard--Bernstein collision operator, these take Fokker--Planck-like forms \cite{Fokker_1914, Planck_1917} wherein pseudo-particles in the numerical model obey the neoclassical transport equations, with particle-independent Brownian drift terms. This offers a rigorous methodology for incorporating collisions into the particle transport model, without coupling the equations of motions for each particle.
        
        Works by Chen, Chacón et al. \cite{Chen_Chacón_Barnes_2011, Chacón_Chen_Barnes_2013, Chen_Chacón_2014, Chen_Chacón_2015} have developed structure-preserving particle pushers for neoclassical transport in the Vlasov equations, derived from Crank--Nicolson integrators. We show these too can can derive from a FET interpretation, similarly offering potential extensions to higher-order-in-time particle pushers. The FET formulation is used also to consider how the stochastic drift terms can be incorporated into the pushers. Stochastic gyrokinetic expansions are also discussed.

        Different options for the numerical implementation of these schemes are considered.

        Due to the efficacy of FET in the development of SP timesteppers for both the fluid and kinetic component, we hope this approach will prove effective in the future for developing SP timesteppers for the full hybrid model. We hope this will give us the opportunity to incorporate previously inaccessible kinetic effects into the highly effective, modern, finite-element MHD models.
    \end{abstract}
    
    
    \newpage
    \tableofcontents
    
    
    \newpage
    \pagenumbering{arabic}
    %\linenumbers\renewcommand\thelinenumber{\color{black!50}\arabic{linenumber}}
            \input{0 - introduction/main.tex}
        \part{Research}
            \input{1 - low-noise PiC models/main.tex}
            \input{2 - kinetic component/main.tex}
            \input{3 - fluid component/main.tex}
            \input{4 - numerical implementation/main.tex}
        \part{Project Overview}
            \input{5 - research plan/main.tex}
            \input{6 - summary/main.tex}
    
    
    %\section{}
    \newpage
    \pagenumbering{gobble}
        \printbibliography


    \newpage
    \pagenumbering{roman}
    \appendix
        \part{Appendices}
            \input{8 - Hilbert complexes/main.tex}
            \input{9 - weak conservation proofs/main.tex}
\end{document}

            \documentclass[12pt, a4paper]{report}

\input{template/main.tex}

\title{\BA{Title in Progress...}}
\author{Boris Andrews}
\affil{Mathematical Institute, University of Oxford}
\date{\today}


\begin{document}
    \pagenumbering{gobble}
    \maketitle
    
    
    \begin{abstract}
        Magnetic confinement reactors---in particular tokamaks---offer one of the most promising options for achieving practical nuclear fusion, with the potential to provide virtually limitless, clean energy. The theoretical and numerical modeling of tokamak plasmas is simultaneously an essential component of effective reactor design, and a great research barrier. Tokamak operational conditions exhibit comparatively low Knudsen numbers. Kinetic effects, including kinetic waves and instabilities, Landau damping, bump-on-tail instabilities and more, are therefore highly influential in tokamak plasma dynamics. Purely fluid models are inherently incapable of capturing these effects, whereas the high dimensionality in purely kinetic models render them practically intractable for most relevant purposes.

        We consider a $\delta\!f$ decomposition model, with a macroscopic fluid background and microscopic kinetic correction, both fully coupled to each other. A similar manner of discretization is proposed to that used in the recent \texttt{STRUPHY} code \cite{Holderied_Possanner_Wang_2021, Holderied_2022, Li_et_al_2023} with a finite-element model for the background and a pseudo-particle/PiC model for the correction.

        The fluid background satisfies the full, non-linear, resistive, compressible, Hall MHD equations. \cite{Laakmann_Hu_Farrell_2022} introduces finite-element(-in-space) implicit timesteppers for the incompressible analogue to this system with structure-preserving (SP) properties in the ideal case, alongside parameter-robust preconditioners. We show that these timesteppers can derive from a finite-element-in-time (FET) (and finite-element-in-space) interpretation. The benefits of this reformulation are discussed, including the derivation of timesteppers that are higher order in time, and the quantifiable dissipative SP properties in the non-ideal, resistive case.
        
        We discuss possible options for extending this FET approach to timesteppers for the compressible case.

        The kinetic corrections satisfy linearized Boltzmann equations. Using a Lénard--Bernstein collision operator, these take Fokker--Planck-like forms \cite{Fokker_1914, Planck_1917} wherein pseudo-particles in the numerical model obey the neoclassical transport equations, with particle-independent Brownian drift terms. This offers a rigorous methodology for incorporating collisions into the particle transport model, without coupling the equations of motions for each particle.
        
        Works by Chen, Chacón et al. \cite{Chen_Chacón_Barnes_2011, Chacón_Chen_Barnes_2013, Chen_Chacón_2014, Chen_Chacón_2015} have developed structure-preserving particle pushers for neoclassical transport in the Vlasov equations, derived from Crank--Nicolson integrators. We show these too can can derive from a FET interpretation, similarly offering potential extensions to higher-order-in-time particle pushers. The FET formulation is used also to consider how the stochastic drift terms can be incorporated into the pushers. Stochastic gyrokinetic expansions are also discussed.

        Different options for the numerical implementation of these schemes are considered.

        Due to the efficacy of FET in the development of SP timesteppers for both the fluid and kinetic component, we hope this approach will prove effective in the future for developing SP timesteppers for the full hybrid model. We hope this will give us the opportunity to incorporate previously inaccessible kinetic effects into the highly effective, modern, finite-element MHD models.
    \end{abstract}
    
    
    \newpage
    \tableofcontents
    
    
    \newpage
    \pagenumbering{arabic}
    %\linenumbers\renewcommand\thelinenumber{\color{black!50}\arabic{linenumber}}
            \input{0 - introduction/main.tex}
        \part{Research}
            \input{1 - low-noise PiC models/main.tex}
            \input{2 - kinetic component/main.tex}
            \input{3 - fluid component/main.tex}
            \input{4 - numerical implementation/main.tex}
        \part{Project Overview}
            \input{5 - research plan/main.tex}
            \input{6 - summary/main.tex}
    
    
    %\section{}
    \newpage
    \pagenumbering{gobble}
        \printbibliography


    \newpage
    \pagenumbering{roman}
    \appendix
        \part{Appendices}
            \input{8 - Hilbert complexes/main.tex}
            \input{9 - weak conservation proofs/main.tex}
\end{document}

            \documentclass[12pt, a4paper]{report}

\input{template/main.tex}

\title{\BA{Title in Progress...}}
\author{Boris Andrews}
\affil{Mathematical Institute, University of Oxford}
\date{\today}


\begin{document}
    \pagenumbering{gobble}
    \maketitle
    
    
    \begin{abstract}
        Magnetic confinement reactors---in particular tokamaks---offer one of the most promising options for achieving practical nuclear fusion, with the potential to provide virtually limitless, clean energy. The theoretical and numerical modeling of tokamak plasmas is simultaneously an essential component of effective reactor design, and a great research barrier. Tokamak operational conditions exhibit comparatively low Knudsen numbers. Kinetic effects, including kinetic waves and instabilities, Landau damping, bump-on-tail instabilities and more, are therefore highly influential in tokamak plasma dynamics. Purely fluid models are inherently incapable of capturing these effects, whereas the high dimensionality in purely kinetic models render them practically intractable for most relevant purposes.

        We consider a $\delta\!f$ decomposition model, with a macroscopic fluid background and microscopic kinetic correction, both fully coupled to each other. A similar manner of discretization is proposed to that used in the recent \texttt{STRUPHY} code \cite{Holderied_Possanner_Wang_2021, Holderied_2022, Li_et_al_2023} with a finite-element model for the background and a pseudo-particle/PiC model for the correction.

        The fluid background satisfies the full, non-linear, resistive, compressible, Hall MHD equations. \cite{Laakmann_Hu_Farrell_2022} introduces finite-element(-in-space) implicit timesteppers for the incompressible analogue to this system with structure-preserving (SP) properties in the ideal case, alongside parameter-robust preconditioners. We show that these timesteppers can derive from a finite-element-in-time (FET) (and finite-element-in-space) interpretation. The benefits of this reformulation are discussed, including the derivation of timesteppers that are higher order in time, and the quantifiable dissipative SP properties in the non-ideal, resistive case.
        
        We discuss possible options for extending this FET approach to timesteppers for the compressible case.

        The kinetic corrections satisfy linearized Boltzmann equations. Using a Lénard--Bernstein collision operator, these take Fokker--Planck-like forms \cite{Fokker_1914, Planck_1917} wherein pseudo-particles in the numerical model obey the neoclassical transport equations, with particle-independent Brownian drift terms. This offers a rigorous methodology for incorporating collisions into the particle transport model, without coupling the equations of motions for each particle.
        
        Works by Chen, Chacón et al. \cite{Chen_Chacón_Barnes_2011, Chacón_Chen_Barnes_2013, Chen_Chacón_2014, Chen_Chacón_2015} have developed structure-preserving particle pushers for neoclassical transport in the Vlasov equations, derived from Crank--Nicolson integrators. We show these too can can derive from a FET interpretation, similarly offering potential extensions to higher-order-in-time particle pushers. The FET formulation is used also to consider how the stochastic drift terms can be incorporated into the pushers. Stochastic gyrokinetic expansions are also discussed.

        Different options for the numerical implementation of these schemes are considered.

        Due to the efficacy of FET in the development of SP timesteppers for both the fluid and kinetic component, we hope this approach will prove effective in the future for developing SP timesteppers for the full hybrid model. We hope this will give us the opportunity to incorporate previously inaccessible kinetic effects into the highly effective, modern, finite-element MHD models.
    \end{abstract}
    
    
    \newpage
    \tableofcontents
    
    
    \newpage
    \pagenumbering{arabic}
    %\linenumbers\renewcommand\thelinenumber{\color{black!50}\arabic{linenumber}}
            \input{0 - introduction/main.tex}
        \part{Research}
            \input{1 - low-noise PiC models/main.tex}
            \input{2 - kinetic component/main.tex}
            \input{3 - fluid component/main.tex}
            \input{4 - numerical implementation/main.tex}
        \part{Project Overview}
            \input{5 - research plan/main.tex}
            \input{6 - summary/main.tex}
    
    
    %\section{}
    \newpage
    \pagenumbering{gobble}
        \printbibliography


    \newpage
    \pagenumbering{roman}
    \appendix
        \part{Appendices}
            \input{8 - Hilbert complexes/main.tex}
            \input{9 - weak conservation proofs/main.tex}
\end{document}

        \part{Project Overview}
            \documentclass[12pt, a4paper]{report}

\input{template/main.tex}

\title{\BA{Title in Progress...}}
\author{Boris Andrews}
\affil{Mathematical Institute, University of Oxford}
\date{\today}


\begin{document}
    \pagenumbering{gobble}
    \maketitle
    
    
    \begin{abstract}
        Magnetic confinement reactors---in particular tokamaks---offer one of the most promising options for achieving practical nuclear fusion, with the potential to provide virtually limitless, clean energy. The theoretical and numerical modeling of tokamak plasmas is simultaneously an essential component of effective reactor design, and a great research barrier. Tokamak operational conditions exhibit comparatively low Knudsen numbers. Kinetic effects, including kinetic waves and instabilities, Landau damping, bump-on-tail instabilities and more, are therefore highly influential in tokamak plasma dynamics. Purely fluid models are inherently incapable of capturing these effects, whereas the high dimensionality in purely kinetic models render them practically intractable for most relevant purposes.

        We consider a $\delta\!f$ decomposition model, with a macroscopic fluid background and microscopic kinetic correction, both fully coupled to each other. A similar manner of discretization is proposed to that used in the recent \texttt{STRUPHY} code \cite{Holderied_Possanner_Wang_2021, Holderied_2022, Li_et_al_2023} with a finite-element model for the background and a pseudo-particle/PiC model for the correction.

        The fluid background satisfies the full, non-linear, resistive, compressible, Hall MHD equations. \cite{Laakmann_Hu_Farrell_2022} introduces finite-element(-in-space) implicit timesteppers for the incompressible analogue to this system with structure-preserving (SP) properties in the ideal case, alongside parameter-robust preconditioners. We show that these timesteppers can derive from a finite-element-in-time (FET) (and finite-element-in-space) interpretation. The benefits of this reformulation are discussed, including the derivation of timesteppers that are higher order in time, and the quantifiable dissipative SP properties in the non-ideal, resistive case.
        
        We discuss possible options for extending this FET approach to timesteppers for the compressible case.

        The kinetic corrections satisfy linearized Boltzmann equations. Using a Lénard--Bernstein collision operator, these take Fokker--Planck-like forms \cite{Fokker_1914, Planck_1917} wherein pseudo-particles in the numerical model obey the neoclassical transport equations, with particle-independent Brownian drift terms. This offers a rigorous methodology for incorporating collisions into the particle transport model, without coupling the equations of motions for each particle.
        
        Works by Chen, Chacón et al. \cite{Chen_Chacón_Barnes_2011, Chacón_Chen_Barnes_2013, Chen_Chacón_2014, Chen_Chacón_2015} have developed structure-preserving particle pushers for neoclassical transport in the Vlasov equations, derived from Crank--Nicolson integrators. We show these too can can derive from a FET interpretation, similarly offering potential extensions to higher-order-in-time particle pushers. The FET formulation is used also to consider how the stochastic drift terms can be incorporated into the pushers. Stochastic gyrokinetic expansions are also discussed.

        Different options for the numerical implementation of these schemes are considered.

        Due to the efficacy of FET in the development of SP timesteppers for both the fluid and kinetic component, we hope this approach will prove effective in the future for developing SP timesteppers for the full hybrid model. We hope this will give us the opportunity to incorporate previously inaccessible kinetic effects into the highly effective, modern, finite-element MHD models.
    \end{abstract}
    
    
    \newpage
    \tableofcontents
    
    
    \newpage
    \pagenumbering{arabic}
    %\linenumbers\renewcommand\thelinenumber{\color{black!50}\arabic{linenumber}}
            \input{0 - introduction/main.tex}
        \part{Research}
            \input{1 - low-noise PiC models/main.tex}
            \input{2 - kinetic component/main.tex}
            \input{3 - fluid component/main.tex}
            \input{4 - numerical implementation/main.tex}
        \part{Project Overview}
            \input{5 - research plan/main.tex}
            \input{6 - summary/main.tex}
    
    
    %\section{}
    \newpage
    \pagenumbering{gobble}
        \printbibliography


    \newpage
    \pagenumbering{roman}
    \appendix
        \part{Appendices}
            \input{8 - Hilbert complexes/main.tex}
            \input{9 - weak conservation proofs/main.tex}
\end{document}

            \documentclass[12pt, a4paper]{report}

\input{template/main.tex}

\title{\BA{Title in Progress...}}
\author{Boris Andrews}
\affil{Mathematical Institute, University of Oxford}
\date{\today}


\begin{document}
    \pagenumbering{gobble}
    \maketitle
    
    
    \begin{abstract}
        Magnetic confinement reactors---in particular tokamaks---offer one of the most promising options for achieving practical nuclear fusion, with the potential to provide virtually limitless, clean energy. The theoretical and numerical modeling of tokamak plasmas is simultaneously an essential component of effective reactor design, and a great research barrier. Tokamak operational conditions exhibit comparatively low Knudsen numbers. Kinetic effects, including kinetic waves and instabilities, Landau damping, bump-on-tail instabilities and more, are therefore highly influential in tokamak plasma dynamics. Purely fluid models are inherently incapable of capturing these effects, whereas the high dimensionality in purely kinetic models render them practically intractable for most relevant purposes.

        We consider a $\delta\!f$ decomposition model, with a macroscopic fluid background and microscopic kinetic correction, both fully coupled to each other. A similar manner of discretization is proposed to that used in the recent \texttt{STRUPHY} code \cite{Holderied_Possanner_Wang_2021, Holderied_2022, Li_et_al_2023} with a finite-element model for the background and a pseudo-particle/PiC model for the correction.

        The fluid background satisfies the full, non-linear, resistive, compressible, Hall MHD equations. \cite{Laakmann_Hu_Farrell_2022} introduces finite-element(-in-space) implicit timesteppers for the incompressible analogue to this system with structure-preserving (SP) properties in the ideal case, alongside parameter-robust preconditioners. We show that these timesteppers can derive from a finite-element-in-time (FET) (and finite-element-in-space) interpretation. The benefits of this reformulation are discussed, including the derivation of timesteppers that are higher order in time, and the quantifiable dissipative SP properties in the non-ideal, resistive case.
        
        We discuss possible options for extending this FET approach to timesteppers for the compressible case.

        The kinetic corrections satisfy linearized Boltzmann equations. Using a Lénard--Bernstein collision operator, these take Fokker--Planck-like forms \cite{Fokker_1914, Planck_1917} wherein pseudo-particles in the numerical model obey the neoclassical transport equations, with particle-independent Brownian drift terms. This offers a rigorous methodology for incorporating collisions into the particle transport model, without coupling the equations of motions for each particle.
        
        Works by Chen, Chacón et al. \cite{Chen_Chacón_Barnes_2011, Chacón_Chen_Barnes_2013, Chen_Chacón_2014, Chen_Chacón_2015} have developed structure-preserving particle pushers for neoclassical transport in the Vlasov equations, derived from Crank--Nicolson integrators. We show these too can can derive from a FET interpretation, similarly offering potential extensions to higher-order-in-time particle pushers. The FET formulation is used also to consider how the stochastic drift terms can be incorporated into the pushers. Stochastic gyrokinetic expansions are also discussed.

        Different options for the numerical implementation of these schemes are considered.

        Due to the efficacy of FET in the development of SP timesteppers for both the fluid and kinetic component, we hope this approach will prove effective in the future for developing SP timesteppers for the full hybrid model. We hope this will give us the opportunity to incorporate previously inaccessible kinetic effects into the highly effective, modern, finite-element MHD models.
    \end{abstract}
    
    
    \newpage
    \tableofcontents
    
    
    \newpage
    \pagenumbering{arabic}
    %\linenumbers\renewcommand\thelinenumber{\color{black!50}\arabic{linenumber}}
            \input{0 - introduction/main.tex}
        \part{Research}
            \input{1 - low-noise PiC models/main.tex}
            \input{2 - kinetic component/main.tex}
            \input{3 - fluid component/main.tex}
            \input{4 - numerical implementation/main.tex}
        \part{Project Overview}
            \input{5 - research plan/main.tex}
            \input{6 - summary/main.tex}
    
    
    %\section{}
    \newpage
    \pagenumbering{gobble}
        \printbibliography


    \newpage
    \pagenumbering{roman}
    \appendix
        \part{Appendices}
            \input{8 - Hilbert complexes/main.tex}
            \input{9 - weak conservation proofs/main.tex}
\end{document}

    
    
    %\section{}
    \newpage
    \pagenumbering{gobble}
        \printbibliography


    \newpage
    \pagenumbering{roman}
    \appendix
        \part{Appendices}
            \documentclass[12pt, a4paper]{report}

\input{template/main.tex}

\title{\BA{Title in Progress...}}
\author{Boris Andrews}
\affil{Mathematical Institute, University of Oxford}
\date{\today}


\begin{document}
    \pagenumbering{gobble}
    \maketitle
    
    
    \begin{abstract}
        Magnetic confinement reactors---in particular tokamaks---offer one of the most promising options for achieving practical nuclear fusion, with the potential to provide virtually limitless, clean energy. The theoretical and numerical modeling of tokamak plasmas is simultaneously an essential component of effective reactor design, and a great research barrier. Tokamak operational conditions exhibit comparatively low Knudsen numbers. Kinetic effects, including kinetic waves and instabilities, Landau damping, bump-on-tail instabilities and more, are therefore highly influential in tokamak plasma dynamics. Purely fluid models are inherently incapable of capturing these effects, whereas the high dimensionality in purely kinetic models render them practically intractable for most relevant purposes.

        We consider a $\delta\!f$ decomposition model, with a macroscopic fluid background and microscopic kinetic correction, both fully coupled to each other. A similar manner of discretization is proposed to that used in the recent \texttt{STRUPHY} code \cite{Holderied_Possanner_Wang_2021, Holderied_2022, Li_et_al_2023} with a finite-element model for the background and a pseudo-particle/PiC model for the correction.

        The fluid background satisfies the full, non-linear, resistive, compressible, Hall MHD equations. \cite{Laakmann_Hu_Farrell_2022} introduces finite-element(-in-space) implicit timesteppers for the incompressible analogue to this system with structure-preserving (SP) properties in the ideal case, alongside parameter-robust preconditioners. We show that these timesteppers can derive from a finite-element-in-time (FET) (and finite-element-in-space) interpretation. The benefits of this reformulation are discussed, including the derivation of timesteppers that are higher order in time, and the quantifiable dissipative SP properties in the non-ideal, resistive case.
        
        We discuss possible options for extending this FET approach to timesteppers for the compressible case.

        The kinetic corrections satisfy linearized Boltzmann equations. Using a Lénard--Bernstein collision operator, these take Fokker--Planck-like forms \cite{Fokker_1914, Planck_1917} wherein pseudo-particles in the numerical model obey the neoclassical transport equations, with particle-independent Brownian drift terms. This offers a rigorous methodology for incorporating collisions into the particle transport model, without coupling the equations of motions for each particle.
        
        Works by Chen, Chacón et al. \cite{Chen_Chacón_Barnes_2011, Chacón_Chen_Barnes_2013, Chen_Chacón_2014, Chen_Chacón_2015} have developed structure-preserving particle pushers for neoclassical transport in the Vlasov equations, derived from Crank--Nicolson integrators. We show these too can can derive from a FET interpretation, similarly offering potential extensions to higher-order-in-time particle pushers. The FET formulation is used also to consider how the stochastic drift terms can be incorporated into the pushers. Stochastic gyrokinetic expansions are also discussed.

        Different options for the numerical implementation of these schemes are considered.

        Due to the efficacy of FET in the development of SP timesteppers for both the fluid and kinetic component, we hope this approach will prove effective in the future for developing SP timesteppers for the full hybrid model. We hope this will give us the opportunity to incorporate previously inaccessible kinetic effects into the highly effective, modern, finite-element MHD models.
    \end{abstract}
    
    
    \newpage
    \tableofcontents
    
    
    \newpage
    \pagenumbering{arabic}
    %\linenumbers\renewcommand\thelinenumber{\color{black!50}\arabic{linenumber}}
            \input{0 - introduction/main.tex}
        \part{Research}
            \input{1 - low-noise PiC models/main.tex}
            \input{2 - kinetic component/main.tex}
            \input{3 - fluid component/main.tex}
            \input{4 - numerical implementation/main.tex}
        \part{Project Overview}
            \input{5 - research plan/main.tex}
            \input{6 - summary/main.tex}
    
    
    %\section{}
    \newpage
    \pagenumbering{gobble}
        \printbibliography


    \newpage
    \pagenumbering{roman}
    \appendix
        \part{Appendices}
            \input{8 - Hilbert complexes/main.tex}
            \input{9 - weak conservation proofs/main.tex}
\end{document}

            \documentclass[12pt, a4paper]{report}

\input{template/main.tex}

\title{\BA{Title in Progress...}}
\author{Boris Andrews}
\affil{Mathematical Institute, University of Oxford}
\date{\today}


\begin{document}
    \pagenumbering{gobble}
    \maketitle
    
    
    \begin{abstract}
        Magnetic confinement reactors---in particular tokamaks---offer one of the most promising options for achieving practical nuclear fusion, with the potential to provide virtually limitless, clean energy. The theoretical and numerical modeling of tokamak plasmas is simultaneously an essential component of effective reactor design, and a great research barrier. Tokamak operational conditions exhibit comparatively low Knudsen numbers. Kinetic effects, including kinetic waves and instabilities, Landau damping, bump-on-tail instabilities and more, are therefore highly influential in tokamak plasma dynamics. Purely fluid models are inherently incapable of capturing these effects, whereas the high dimensionality in purely kinetic models render them practically intractable for most relevant purposes.

        We consider a $\delta\!f$ decomposition model, with a macroscopic fluid background and microscopic kinetic correction, both fully coupled to each other. A similar manner of discretization is proposed to that used in the recent \texttt{STRUPHY} code \cite{Holderied_Possanner_Wang_2021, Holderied_2022, Li_et_al_2023} with a finite-element model for the background and a pseudo-particle/PiC model for the correction.

        The fluid background satisfies the full, non-linear, resistive, compressible, Hall MHD equations. \cite{Laakmann_Hu_Farrell_2022} introduces finite-element(-in-space) implicit timesteppers for the incompressible analogue to this system with structure-preserving (SP) properties in the ideal case, alongside parameter-robust preconditioners. We show that these timesteppers can derive from a finite-element-in-time (FET) (and finite-element-in-space) interpretation. The benefits of this reformulation are discussed, including the derivation of timesteppers that are higher order in time, and the quantifiable dissipative SP properties in the non-ideal, resistive case.
        
        We discuss possible options for extending this FET approach to timesteppers for the compressible case.

        The kinetic corrections satisfy linearized Boltzmann equations. Using a Lénard--Bernstein collision operator, these take Fokker--Planck-like forms \cite{Fokker_1914, Planck_1917} wherein pseudo-particles in the numerical model obey the neoclassical transport equations, with particle-independent Brownian drift terms. This offers a rigorous methodology for incorporating collisions into the particle transport model, without coupling the equations of motions for each particle.
        
        Works by Chen, Chacón et al. \cite{Chen_Chacón_Barnes_2011, Chacón_Chen_Barnes_2013, Chen_Chacón_2014, Chen_Chacón_2015} have developed structure-preserving particle pushers for neoclassical transport in the Vlasov equations, derived from Crank--Nicolson integrators. We show these too can can derive from a FET interpretation, similarly offering potential extensions to higher-order-in-time particle pushers. The FET formulation is used also to consider how the stochastic drift terms can be incorporated into the pushers. Stochastic gyrokinetic expansions are also discussed.

        Different options for the numerical implementation of these schemes are considered.

        Due to the efficacy of FET in the development of SP timesteppers for both the fluid and kinetic component, we hope this approach will prove effective in the future for developing SP timesteppers for the full hybrid model. We hope this will give us the opportunity to incorporate previously inaccessible kinetic effects into the highly effective, modern, finite-element MHD models.
    \end{abstract}
    
    
    \newpage
    \tableofcontents
    
    
    \newpage
    \pagenumbering{arabic}
    %\linenumbers\renewcommand\thelinenumber{\color{black!50}\arabic{linenumber}}
            \input{0 - introduction/main.tex}
        \part{Research}
            \input{1 - low-noise PiC models/main.tex}
            \input{2 - kinetic component/main.tex}
            \input{3 - fluid component/main.tex}
            \input{4 - numerical implementation/main.tex}
        \part{Project Overview}
            \input{5 - research plan/main.tex}
            \input{6 - summary/main.tex}
    
    
    %\section{}
    \newpage
    \pagenumbering{gobble}
        \printbibliography


    \newpage
    \pagenumbering{roman}
    \appendix
        \part{Appendices}
            \input{8 - Hilbert complexes/main.tex}
            \input{9 - weak conservation proofs/main.tex}
\end{document}

\end{document}

    
    
    %\section{}
    \newpage
    \pagenumbering{gobble}
        \printbibliography


    \newpage
    \pagenumbering{roman}
    \appendix
        \part{Appendices}
            \documentclass[12pt, a4paper]{report}

\documentclass[12pt, a4paper]{report}

\input{template/main.tex}

\title{\BA{Title in Progress...}}
\author{Boris Andrews}
\affil{Mathematical Institute, University of Oxford}
\date{\today}


\begin{document}
    \pagenumbering{gobble}
    \maketitle
    
    
    \begin{abstract}
        Magnetic confinement reactors---in particular tokamaks---offer one of the most promising options for achieving practical nuclear fusion, with the potential to provide virtually limitless, clean energy. The theoretical and numerical modeling of tokamak plasmas is simultaneously an essential component of effective reactor design, and a great research barrier. Tokamak operational conditions exhibit comparatively low Knudsen numbers. Kinetic effects, including kinetic waves and instabilities, Landau damping, bump-on-tail instabilities and more, are therefore highly influential in tokamak plasma dynamics. Purely fluid models are inherently incapable of capturing these effects, whereas the high dimensionality in purely kinetic models render them practically intractable for most relevant purposes.

        We consider a $\delta\!f$ decomposition model, with a macroscopic fluid background and microscopic kinetic correction, both fully coupled to each other. A similar manner of discretization is proposed to that used in the recent \texttt{STRUPHY} code \cite{Holderied_Possanner_Wang_2021, Holderied_2022, Li_et_al_2023} with a finite-element model for the background and a pseudo-particle/PiC model for the correction.

        The fluid background satisfies the full, non-linear, resistive, compressible, Hall MHD equations. \cite{Laakmann_Hu_Farrell_2022} introduces finite-element(-in-space) implicit timesteppers for the incompressible analogue to this system with structure-preserving (SP) properties in the ideal case, alongside parameter-robust preconditioners. We show that these timesteppers can derive from a finite-element-in-time (FET) (and finite-element-in-space) interpretation. The benefits of this reformulation are discussed, including the derivation of timesteppers that are higher order in time, and the quantifiable dissipative SP properties in the non-ideal, resistive case.
        
        We discuss possible options for extending this FET approach to timesteppers for the compressible case.

        The kinetic corrections satisfy linearized Boltzmann equations. Using a Lénard--Bernstein collision operator, these take Fokker--Planck-like forms \cite{Fokker_1914, Planck_1917} wherein pseudo-particles in the numerical model obey the neoclassical transport equations, with particle-independent Brownian drift terms. This offers a rigorous methodology for incorporating collisions into the particle transport model, without coupling the equations of motions for each particle.
        
        Works by Chen, Chacón et al. \cite{Chen_Chacón_Barnes_2011, Chacón_Chen_Barnes_2013, Chen_Chacón_2014, Chen_Chacón_2015} have developed structure-preserving particle pushers for neoclassical transport in the Vlasov equations, derived from Crank--Nicolson integrators. We show these too can can derive from a FET interpretation, similarly offering potential extensions to higher-order-in-time particle pushers. The FET formulation is used also to consider how the stochastic drift terms can be incorporated into the pushers. Stochastic gyrokinetic expansions are also discussed.

        Different options for the numerical implementation of these schemes are considered.

        Due to the efficacy of FET in the development of SP timesteppers for both the fluid and kinetic component, we hope this approach will prove effective in the future for developing SP timesteppers for the full hybrid model. We hope this will give us the opportunity to incorporate previously inaccessible kinetic effects into the highly effective, modern, finite-element MHD models.
    \end{abstract}
    
    
    \newpage
    \tableofcontents
    
    
    \newpage
    \pagenumbering{arabic}
    %\linenumbers\renewcommand\thelinenumber{\color{black!50}\arabic{linenumber}}
            \input{0 - introduction/main.tex}
        \part{Research}
            \input{1 - low-noise PiC models/main.tex}
            \input{2 - kinetic component/main.tex}
            \input{3 - fluid component/main.tex}
            \input{4 - numerical implementation/main.tex}
        \part{Project Overview}
            \input{5 - research plan/main.tex}
            \input{6 - summary/main.tex}
    
    
    %\section{}
    \newpage
    \pagenumbering{gobble}
        \printbibliography


    \newpage
    \pagenumbering{roman}
    \appendix
        \part{Appendices}
            \input{8 - Hilbert complexes/main.tex}
            \input{9 - weak conservation proofs/main.tex}
\end{document}


\title{\BA{Title in Progress...}}
\author{Boris Andrews}
\affil{Mathematical Institute, University of Oxford}
\date{\today}


\begin{document}
    \pagenumbering{gobble}
    \maketitle
    
    
    \begin{abstract}
        Magnetic confinement reactors---in particular tokamaks---offer one of the most promising options for achieving practical nuclear fusion, with the potential to provide virtually limitless, clean energy. The theoretical and numerical modeling of tokamak plasmas is simultaneously an essential component of effective reactor design, and a great research barrier. Tokamak operational conditions exhibit comparatively low Knudsen numbers. Kinetic effects, including kinetic waves and instabilities, Landau damping, bump-on-tail instabilities and more, are therefore highly influential in tokamak plasma dynamics. Purely fluid models are inherently incapable of capturing these effects, whereas the high dimensionality in purely kinetic models render them practically intractable for most relevant purposes.

        We consider a $\delta\!f$ decomposition model, with a macroscopic fluid background and microscopic kinetic correction, both fully coupled to each other. A similar manner of discretization is proposed to that used in the recent \texttt{STRUPHY} code \cite{Holderied_Possanner_Wang_2021, Holderied_2022, Li_et_al_2023} with a finite-element model for the background and a pseudo-particle/PiC model for the correction.

        The fluid background satisfies the full, non-linear, resistive, compressible, Hall MHD equations. \cite{Laakmann_Hu_Farrell_2022} introduces finite-element(-in-space) implicit timesteppers for the incompressible analogue to this system with structure-preserving (SP) properties in the ideal case, alongside parameter-robust preconditioners. We show that these timesteppers can derive from a finite-element-in-time (FET) (and finite-element-in-space) interpretation. The benefits of this reformulation are discussed, including the derivation of timesteppers that are higher order in time, and the quantifiable dissipative SP properties in the non-ideal, resistive case.
        
        We discuss possible options for extending this FET approach to timesteppers for the compressible case.

        The kinetic corrections satisfy linearized Boltzmann equations. Using a Lénard--Bernstein collision operator, these take Fokker--Planck-like forms \cite{Fokker_1914, Planck_1917} wherein pseudo-particles in the numerical model obey the neoclassical transport equations, with particle-independent Brownian drift terms. This offers a rigorous methodology for incorporating collisions into the particle transport model, without coupling the equations of motions for each particle.
        
        Works by Chen, Chacón et al. \cite{Chen_Chacón_Barnes_2011, Chacón_Chen_Barnes_2013, Chen_Chacón_2014, Chen_Chacón_2015} have developed structure-preserving particle pushers for neoclassical transport in the Vlasov equations, derived from Crank--Nicolson integrators. We show these too can can derive from a FET interpretation, similarly offering potential extensions to higher-order-in-time particle pushers. The FET formulation is used also to consider how the stochastic drift terms can be incorporated into the pushers. Stochastic gyrokinetic expansions are also discussed.

        Different options for the numerical implementation of these schemes are considered.

        Due to the efficacy of FET in the development of SP timesteppers for both the fluid and kinetic component, we hope this approach will prove effective in the future for developing SP timesteppers for the full hybrid model. We hope this will give us the opportunity to incorporate previously inaccessible kinetic effects into the highly effective, modern, finite-element MHD models.
    \end{abstract}
    
    
    \newpage
    \tableofcontents
    
    
    \newpage
    \pagenumbering{arabic}
    %\linenumbers\renewcommand\thelinenumber{\color{black!50}\arabic{linenumber}}
            \documentclass[12pt, a4paper]{report}

\input{template/main.tex}

\title{\BA{Title in Progress...}}
\author{Boris Andrews}
\affil{Mathematical Institute, University of Oxford}
\date{\today}


\begin{document}
    \pagenumbering{gobble}
    \maketitle
    
    
    \begin{abstract}
        Magnetic confinement reactors---in particular tokamaks---offer one of the most promising options for achieving practical nuclear fusion, with the potential to provide virtually limitless, clean energy. The theoretical and numerical modeling of tokamak plasmas is simultaneously an essential component of effective reactor design, and a great research barrier. Tokamak operational conditions exhibit comparatively low Knudsen numbers. Kinetic effects, including kinetic waves and instabilities, Landau damping, bump-on-tail instabilities and more, are therefore highly influential in tokamak plasma dynamics. Purely fluid models are inherently incapable of capturing these effects, whereas the high dimensionality in purely kinetic models render them practically intractable for most relevant purposes.

        We consider a $\delta\!f$ decomposition model, with a macroscopic fluid background and microscopic kinetic correction, both fully coupled to each other. A similar manner of discretization is proposed to that used in the recent \texttt{STRUPHY} code \cite{Holderied_Possanner_Wang_2021, Holderied_2022, Li_et_al_2023} with a finite-element model for the background and a pseudo-particle/PiC model for the correction.

        The fluid background satisfies the full, non-linear, resistive, compressible, Hall MHD equations. \cite{Laakmann_Hu_Farrell_2022} introduces finite-element(-in-space) implicit timesteppers for the incompressible analogue to this system with structure-preserving (SP) properties in the ideal case, alongside parameter-robust preconditioners. We show that these timesteppers can derive from a finite-element-in-time (FET) (and finite-element-in-space) interpretation. The benefits of this reformulation are discussed, including the derivation of timesteppers that are higher order in time, and the quantifiable dissipative SP properties in the non-ideal, resistive case.
        
        We discuss possible options for extending this FET approach to timesteppers for the compressible case.

        The kinetic corrections satisfy linearized Boltzmann equations. Using a Lénard--Bernstein collision operator, these take Fokker--Planck-like forms \cite{Fokker_1914, Planck_1917} wherein pseudo-particles in the numerical model obey the neoclassical transport equations, with particle-independent Brownian drift terms. This offers a rigorous methodology for incorporating collisions into the particle transport model, without coupling the equations of motions for each particle.
        
        Works by Chen, Chacón et al. \cite{Chen_Chacón_Barnes_2011, Chacón_Chen_Barnes_2013, Chen_Chacón_2014, Chen_Chacón_2015} have developed structure-preserving particle pushers for neoclassical transport in the Vlasov equations, derived from Crank--Nicolson integrators. We show these too can can derive from a FET interpretation, similarly offering potential extensions to higher-order-in-time particle pushers. The FET formulation is used also to consider how the stochastic drift terms can be incorporated into the pushers. Stochastic gyrokinetic expansions are also discussed.

        Different options for the numerical implementation of these schemes are considered.

        Due to the efficacy of FET in the development of SP timesteppers for both the fluid and kinetic component, we hope this approach will prove effective in the future for developing SP timesteppers for the full hybrid model. We hope this will give us the opportunity to incorporate previously inaccessible kinetic effects into the highly effective, modern, finite-element MHD models.
    \end{abstract}
    
    
    \newpage
    \tableofcontents
    
    
    \newpage
    \pagenumbering{arabic}
    %\linenumbers\renewcommand\thelinenumber{\color{black!50}\arabic{linenumber}}
            \input{0 - introduction/main.tex}
        \part{Research}
            \input{1 - low-noise PiC models/main.tex}
            \input{2 - kinetic component/main.tex}
            \input{3 - fluid component/main.tex}
            \input{4 - numerical implementation/main.tex}
        \part{Project Overview}
            \input{5 - research plan/main.tex}
            \input{6 - summary/main.tex}
    
    
    %\section{}
    \newpage
    \pagenumbering{gobble}
        \printbibliography


    \newpage
    \pagenumbering{roman}
    \appendix
        \part{Appendices}
            \input{8 - Hilbert complexes/main.tex}
            \input{9 - weak conservation proofs/main.tex}
\end{document}

        \part{Research}
            \documentclass[12pt, a4paper]{report}

\input{template/main.tex}

\title{\BA{Title in Progress...}}
\author{Boris Andrews}
\affil{Mathematical Institute, University of Oxford}
\date{\today}


\begin{document}
    \pagenumbering{gobble}
    \maketitle
    
    
    \begin{abstract}
        Magnetic confinement reactors---in particular tokamaks---offer one of the most promising options for achieving practical nuclear fusion, with the potential to provide virtually limitless, clean energy. The theoretical and numerical modeling of tokamak plasmas is simultaneously an essential component of effective reactor design, and a great research barrier. Tokamak operational conditions exhibit comparatively low Knudsen numbers. Kinetic effects, including kinetic waves and instabilities, Landau damping, bump-on-tail instabilities and more, are therefore highly influential in tokamak plasma dynamics. Purely fluid models are inherently incapable of capturing these effects, whereas the high dimensionality in purely kinetic models render them practically intractable for most relevant purposes.

        We consider a $\delta\!f$ decomposition model, with a macroscopic fluid background and microscopic kinetic correction, both fully coupled to each other. A similar manner of discretization is proposed to that used in the recent \texttt{STRUPHY} code \cite{Holderied_Possanner_Wang_2021, Holderied_2022, Li_et_al_2023} with a finite-element model for the background and a pseudo-particle/PiC model for the correction.

        The fluid background satisfies the full, non-linear, resistive, compressible, Hall MHD equations. \cite{Laakmann_Hu_Farrell_2022} introduces finite-element(-in-space) implicit timesteppers for the incompressible analogue to this system with structure-preserving (SP) properties in the ideal case, alongside parameter-robust preconditioners. We show that these timesteppers can derive from a finite-element-in-time (FET) (and finite-element-in-space) interpretation. The benefits of this reformulation are discussed, including the derivation of timesteppers that are higher order in time, and the quantifiable dissipative SP properties in the non-ideal, resistive case.
        
        We discuss possible options for extending this FET approach to timesteppers for the compressible case.

        The kinetic corrections satisfy linearized Boltzmann equations. Using a Lénard--Bernstein collision operator, these take Fokker--Planck-like forms \cite{Fokker_1914, Planck_1917} wherein pseudo-particles in the numerical model obey the neoclassical transport equations, with particle-independent Brownian drift terms. This offers a rigorous methodology for incorporating collisions into the particle transport model, without coupling the equations of motions for each particle.
        
        Works by Chen, Chacón et al. \cite{Chen_Chacón_Barnes_2011, Chacón_Chen_Barnes_2013, Chen_Chacón_2014, Chen_Chacón_2015} have developed structure-preserving particle pushers for neoclassical transport in the Vlasov equations, derived from Crank--Nicolson integrators. We show these too can can derive from a FET interpretation, similarly offering potential extensions to higher-order-in-time particle pushers. The FET formulation is used also to consider how the stochastic drift terms can be incorporated into the pushers. Stochastic gyrokinetic expansions are also discussed.

        Different options for the numerical implementation of these schemes are considered.

        Due to the efficacy of FET in the development of SP timesteppers for both the fluid and kinetic component, we hope this approach will prove effective in the future for developing SP timesteppers for the full hybrid model. We hope this will give us the opportunity to incorporate previously inaccessible kinetic effects into the highly effective, modern, finite-element MHD models.
    \end{abstract}
    
    
    \newpage
    \tableofcontents
    
    
    \newpage
    \pagenumbering{arabic}
    %\linenumbers\renewcommand\thelinenumber{\color{black!50}\arabic{linenumber}}
            \input{0 - introduction/main.tex}
        \part{Research}
            \input{1 - low-noise PiC models/main.tex}
            \input{2 - kinetic component/main.tex}
            \input{3 - fluid component/main.tex}
            \input{4 - numerical implementation/main.tex}
        \part{Project Overview}
            \input{5 - research plan/main.tex}
            \input{6 - summary/main.tex}
    
    
    %\section{}
    \newpage
    \pagenumbering{gobble}
        \printbibliography


    \newpage
    \pagenumbering{roman}
    \appendix
        \part{Appendices}
            \input{8 - Hilbert complexes/main.tex}
            \input{9 - weak conservation proofs/main.tex}
\end{document}

            \documentclass[12pt, a4paper]{report}

\input{template/main.tex}

\title{\BA{Title in Progress...}}
\author{Boris Andrews}
\affil{Mathematical Institute, University of Oxford}
\date{\today}


\begin{document}
    \pagenumbering{gobble}
    \maketitle
    
    
    \begin{abstract}
        Magnetic confinement reactors---in particular tokamaks---offer one of the most promising options for achieving practical nuclear fusion, with the potential to provide virtually limitless, clean energy. The theoretical and numerical modeling of tokamak plasmas is simultaneously an essential component of effective reactor design, and a great research barrier. Tokamak operational conditions exhibit comparatively low Knudsen numbers. Kinetic effects, including kinetic waves and instabilities, Landau damping, bump-on-tail instabilities and more, are therefore highly influential in tokamak plasma dynamics. Purely fluid models are inherently incapable of capturing these effects, whereas the high dimensionality in purely kinetic models render them practically intractable for most relevant purposes.

        We consider a $\delta\!f$ decomposition model, with a macroscopic fluid background and microscopic kinetic correction, both fully coupled to each other. A similar manner of discretization is proposed to that used in the recent \texttt{STRUPHY} code \cite{Holderied_Possanner_Wang_2021, Holderied_2022, Li_et_al_2023} with a finite-element model for the background and a pseudo-particle/PiC model for the correction.

        The fluid background satisfies the full, non-linear, resistive, compressible, Hall MHD equations. \cite{Laakmann_Hu_Farrell_2022} introduces finite-element(-in-space) implicit timesteppers for the incompressible analogue to this system with structure-preserving (SP) properties in the ideal case, alongside parameter-robust preconditioners. We show that these timesteppers can derive from a finite-element-in-time (FET) (and finite-element-in-space) interpretation. The benefits of this reformulation are discussed, including the derivation of timesteppers that are higher order in time, and the quantifiable dissipative SP properties in the non-ideal, resistive case.
        
        We discuss possible options for extending this FET approach to timesteppers for the compressible case.

        The kinetic corrections satisfy linearized Boltzmann equations. Using a Lénard--Bernstein collision operator, these take Fokker--Planck-like forms \cite{Fokker_1914, Planck_1917} wherein pseudo-particles in the numerical model obey the neoclassical transport equations, with particle-independent Brownian drift terms. This offers a rigorous methodology for incorporating collisions into the particle transport model, without coupling the equations of motions for each particle.
        
        Works by Chen, Chacón et al. \cite{Chen_Chacón_Barnes_2011, Chacón_Chen_Barnes_2013, Chen_Chacón_2014, Chen_Chacón_2015} have developed structure-preserving particle pushers for neoclassical transport in the Vlasov equations, derived from Crank--Nicolson integrators. We show these too can can derive from a FET interpretation, similarly offering potential extensions to higher-order-in-time particle pushers. The FET formulation is used also to consider how the stochastic drift terms can be incorporated into the pushers. Stochastic gyrokinetic expansions are also discussed.

        Different options for the numerical implementation of these schemes are considered.

        Due to the efficacy of FET in the development of SP timesteppers for both the fluid and kinetic component, we hope this approach will prove effective in the future for developing SP timesteppers for the full hybrid model. We hope this will give us the opportunity to incorporate previously inaccessible kinetic effects into the highly effective, modern, finite-element MHD models.
    \end{abstract}
    
    
    \newpage
    \tableofcontents
    
    
    \newpage
    \pagenumbering{arabic}
    %\linenumbers\renewcommand\thelinenumber{\color{black!50}\arabic{linenumber}}
            \input{0 - introduction/main.tex}
        \part{Research}
            \input{1 - low-noise PiC models/main.tex}
            \input{2 - kinetic component/main.tex}
            \input{3 - fluid component/main.tex}
            \input{4 - numerical implementation/main.tex}
        \part{Project Overview}
            \input{5 - research plan/main.tex}
            \input{6 - summary/main.tex}
    
    
    %\section{}
    \newpage
    \pagenumbering{gobble}
        \printbibliography


    \newpage
    \pagenumbering{roman}
    \appendix
        \part{Appendices}
            \input{8 - Hilbert complexes/main.tex}
            \input{9 - weak conservation proofs/main.tex}
\end{document}

            \documentclass[12pt, a4paper]{report}

\input{template/main.tex}

\title{\BA{Title in Progress...}}
\author{Boris Andrews}
\affil{Mathematical Institute, University of Oxford}
\date{\today}


\begin{document}
    \pagenumbering{gobble}
    \maketitle
    
    
    \begin{abstract}
        Magnetic confinement reactors---in particular tokamaks---offer one of the most promising options for achieving practical nuclear fusion, with the potential to provide virtually limitless, clean energy. The theoretical and numerical modeling of tokamak plasmas is simultaneously an essential component of effective reactor design, and a great research barrier. Tokamak operational conditions exhibit comparatively low Knudsen numbers. Kinetic effects, including kinetic waves and instabilities, Landau damping, bump-on-tail instabilities and more, are therefore highly influential in tokamak plasma dynamics. Purely fluid models are inherently incapable of capturing these effects, whereas the high dimensionality in purely kinetic models render them practically intractable for most relevant purposes.

        We consider a $\delta\!f$ decomposition model, with a macroscopic fluid background and microscopic kinetic correction, both fully coupled to each other. A similar manner of discretization is proposed to that used in the recent \texttt{STRUPHY} code \cite{Holderied_Possanner_Wang_2021, Holderied_2022, Li_et_al_2023} with a finite-element model for the background and a pseudo-particle/PiC model for the correction.

        The fluid background satisfies the full, non-linear, resistive, compressible, Hall MHD equations. \cite{Laakmann_Hu_Farrell_2022} introduces finite-element(-in-space) implicit timesteppers for the incompressible analogue to this system with structure-preserving (SP) properties in the ideal case, alongside parameter-robust preconditioners. We show that these timesteppers can derive from a finite-element-in-time (FET) (and finite-element-in-space) interpretation. The benefits of this reformulation are discussed, including the derivation of timesteppers that are higher order in time, and the quantifiable dissipative SP properties in the non-ideal, resistive case.
        
        We discuss possible options for extending this FET approach to timesteppers for the compressible case.

        The kinetic corrections satisfy linearized Boltzmann equations. Using a Lénard--Bernstein collision operator, these take Fokker--Planck-like forms \cite{Fokker_1914, Planck_1917} wherein pseudo-particles in the numerical model obey the neoclassical transport equations, with particle-independent Brownian drift terms. This offers a rigorous methodology for incorporating collisions into the particle transport model, without coupling the equations of motions for each particle.
        
        Works by Chen, Chacón et al. \cite{Chen_Chacón_Barnes_2011, Chacón_Chen_Barnes_2013, Chen_Chacón_2014, Chen_Chacón_2015} have developed structure-preserving particle pushers for neoclassical transport in the Vlasov equations, derived from Crank--Nicolson integrators. We show these too can can derive from a FET interpretation, similarly offering potential extensions to higher-order-in-time particle pushers. The FET formulation is used also to consider how the stochastic drift terms can be incorporated into the pushers. Stochastic gyrokinetic expansions are also discussed.

        Different options for the numerical implementation of these schemes are considered.

        Due to the efficacy of FET in the development of SP timesteppers for both the fluid and kinetic component, we hope this approach will prove effective in the future for developing SP timesteppers for the full hybrid model. We hope this will give us the opportunity to incorporate previously inaccessible kinetic effects into the highly effective, modern, finite-element MHD models.
    \end{abstract}
    
    
    \newpage
    \tableofcontents
    
    
    \newpage
    \pagenumbering{arabic}
    %\linenumbers\renewcommand\thelinenumber{\color{black!50}\arabic{linenumber}}
            \input{0 - introduction/main.tex}
        \part{Research}
            \input{1 - low-noise PiC models/main.tex}
            \input{2 - kinetic component/main.tex}
            \input{3 - fluid component/main.tex}
            \input{4 - numerical implementation/main.tex}
        \part{Project Overview}
            \input{5 - research plan/main.tex}
            \input{6 - summary/main.tex}
    
    
    %\section{}
    \newpage
    \pagenumbering{gobble}
        \printbibliography


    \newpage
    \pagenumbering{roman}
    \appendix
        \part{Appendices}
            \input{8 - Hilbert complexes/main.tex}
            \input{9 - weak conservation proofs/main.tex}
\end{document}

            \documentclass[12pt, a4paper]{report}

\input{template/main.tex}

\title{\BA{Title in Progress...}}
\author{Boris Andrews}
\affil{Mathematical Institute, University of Oxford}
\date{\today}


\begin{document}
    \pagenumbering{gobble}
    \maketitle
    
    
    \begin{abstract}
        Magnetic confinement reactors---in particular tokamaks---offer one of the most promising options for achieving practical nuclear fusion, with the potential to provide virtually limitless, clean energy. The theoretical and numerical modeling of tokamak plasmas is simultaneously an essential component of effective reactor design, and a great research barrier. Tokamak operational conditions exhibit comparatively low Knudsen numbers. Kinetic effects, including kinetic waves and instabilities, Landau damping, bump-on-tail instabilities and more, are therefore highly influential in tokamak plasma dynamics. Purely fluid models are inherently incapable of capturing these effects, whereas the high dimensionality in purely kinetic models render them practically intractable for most relevant purposes.

        We consider a $\delta\!f$ decomposition model, with a macroscopic fluid background and microscopic kinetic correction, both fully coupled to each other. A similar manner of discretization is proposed to that used in the recent \texttt{STRUPHY} code \cite{Holderied_Possanner_Wang_2021, Holderied_2022, Li_et_al_2023} with a finite-element model for the background and a pseudo-particle/PiC model for the correction.

        The fluid background satisfies the full, non-linear, resistive, compressible, Hall MHD equations. \cite{Laakmann_Hu_Farrell_2022} introduces finite-element(-in-space) implicit timesteppers for the incompressible analogue to this system with structure-preserving (SP) properties in the ideal case, alongside parameter-robust preconditioners. We show that these timesteppers can derive from a finite-element-in-time (FET) (and finite-element-in-space) interpretation. The benefits of this reformulation are discussed, including the derivation of timesteppers that are higher order in time, and the quantifiable dissipative SP properties in the non-ideal, resistive case.
        
        We discuss possible options for extending this FET approach to timesteppers for the compressible case.

        The kinetic corrections satisfy linearized Boltzmann equations. Using a Lénard--Bernstein collision operator, these take Fokker--Planck-like forms \cite{Fokker_1914, Planck_1917} wherein pseudo-particles in the numerical model obey the neoclassical transport equations, with particle-independent Brownian drift terms. This offers a rigorous methodology for incorporating collisions into the particle transport model, without coupling the equations of motions for each particle.
        
        Works by Chen, Chacón et al. \cite{Chen_Chacón_Barnes_2011, Chacón_Chen_Barnes_2013, Chen_Chacón_2014, Chen_Chacón_2015} have developed structure-preserving particle pushers for neoclassical transport in the Vlasov equations, derived from Crank--Nicolson integrators. We show these too can can derive from a FET interpretation, similarly offering potential extensions to higher-order-in-time particle pushers. The FET formulation is used also to consider how the stochastic drift terms can be incorporated into the pushers. Stochastic gyrokinetic expansions are also discussed.

        Different options for the numerical implementation of these schemes are considered.

        Due to the efficacy of FET in the development of SP timesteppers for both the fluid and kinetic component, we hope this approach will prove effective in the future for developing SP timesteppers for the full hybrid model. We hope this will give us the opportunity to incorporate previously inaccessible kinetic effects into the highly effective, modern, finite-element MHD models.
    \end{abstract}
    
    
    \newpage
    \tableofcontents
    
    
    \newpage
    \pagenumbering{arabic}
    %\linenumbers\renewcommand\thelinenumber{\color{black!50}\arabic{linenumber}}
            \input{0 - introduction/main.tex}
        \part{Research}
            \input{1 - low-noise PiC models/main.tex}
            \input{2 - kinetic component/main.tex}
            \input{3 - fluid component/main.tex}
            \input{4 - numerical implementation/main.tex}
        \part{Project Overview}
            \input{5 - research plan/main.tex}
            \input{6 - summary/main.tex}
    
    
    %\section{}
    \newpage
    \pagenumbering{gobble}
        \printbibliography


    \newpage
    \pagenumbering{roman}
    \appendix
        \part{Appendices}
            \input{8 - Hilbert complexes/main.tex}
            \input{9 - weak conservation proofs/main.tex}
\end{document}

        \part{Project Overview}
            \documentclass[12pt, a4paper]{report}

\input{template/main.tex}

\title{\BA{Title in Progress...}}
\author{Boris Andrews}
\affil{Mathematical Institute, University of Oxford}
\date{\today}


\begin{document}
    \pagenumbering{gobble}
    \maketitle
    
    
    \begin{abstract}
        Magnetic confinement reactors---in particular tokamaks---offer one of the most promising options for achieving practical nuclear fusion, with the potential to provide virtually limitless, clean energy. The theoretical and numerical modeling of tokamak plasmas is simultaneously an essential component of effective reactor design, and a great research barrier. Tokamak operational conditions exhibit comparatively low Knudsen numbers. Kinetic effects, including kinetic waves and instabilities, Landau damping, bump-on-tail instabilities and more, are therefore highly influential in tokamak plasma dynamics. Purely fluid models are inherently incapable of capturing these effects, whereas the high dimensionality in purely kinetic models render them practically intractable for most relevant purposes.

        We consider a $\delta\!f$ decomposition model, with a macroscopic fluid background and microscopic kinetic correction, both fully coupled to each other. A similar manner of discretization is proposed to that used in the recent \texttt{STRUPHY} code \cite{Holderied_Possanner_Wang_2021, Holderied_2022, Li_et_al_2023} with a finite-element model for the background and a pseudo-particle/PiC model for the correction.

        The fluid background satisfies the full, non-linear, resistive, compressible, Hall MHD equations. \cite{Laakmann_Hu_Farrell_2022} introduces finite-element(-in-space) implicit timesteppers for the incompressible analogue to this system with structure-preserving (SP) properties in the ideal case, alongside parameter-robust preconditioners. We show that these timesteppers can derive from a finite-element-in-time (FET) (and finite-element-in-space) interpretation. The benefits of this reformulation are discussed, including the derivation of timesteppers that are higher order in time, and the quantifiable dissipative SP properties in the non-ideal, resistive case.
        
        We discuss possible options for extending this FET approach to timesteppers for the compressible case.

        The kinetic corrections satisfy linearized Boltzmann equations. Using a Lénard--Bernstein collision operator, these take Fokker--Planck-like forms \cite{Fokker_1914, Planck_1917} wherein pseudo-particles in the numerical model obey the neoclassical transport equations, with particle-independent Brownian drift terms. This offers a rigorous methodology for incorporating collisions into the particle transport model, without coupling the equations of motions for each particle.
        
        Works by Chen, Chacón et al. \cite{Chen_Chacón_Barnes_2011, Chacón_Chen_Barnes_2013, Chen_Chacón_2014, Chen_Chacón_2015} have developed structure-preserving particle pushers for neoclassical transport in the Vlasov equations, derived from Crank--Nicolson integrators. We show these too can can derive from a FET interpretation, similarly offering potential extensions to higher-order-in-time particle pushers. The FET formulation is used also to consider how the stochastic drift terms can be incorporated into the pushers. Stochastic gyrokinetic expansions are also discussed.

        Different options for the numerical implementation of these schemes are considered.

        Due to the efficacy of FET in the development of SP timesteppers for both the fluid and kinetic component, we hope this approach will prove effective in the future for developing SP timesteppers for the full hybrid model. We hope this will give us the opportunity to incorporate previously inaccessible kinetic effects into the highly effective, modern, finite-element MHD models.
    \end{abstract}
    
    
    \newpage
    \tableofcontents
    
    
    \newpage
    \pagenumbering{arabic}
    %\linenumbers\renewcommand\thelinenumber{\color{black!50}\arabic{linenumber}}
            \input{0 - introduction/main.tex}
        \part{Research}
            \input{1 - low-noise PiC models/main.tex}
            \input{2 - kinetic component/main.tex}
            \input{3 - fluid component/main.tex}
            \input{4 - numerical implementation/main.tex}
        \part{Project Overview}
            \input{5 - research plan/main.tex}
            \input{6 - summary/main.tex}
    
    
    %\section{}
    \newpage
    \pagenumbering{gobble}
        \printbibliography


    \newpage
    \pagenumbering{roman}
    \appendix
        \part{Appendices}
            \input{8 - Hilbert complexes/main.tex}
            \input{9 - weak conservation proofs/main.tex}
\end{document}

            \documentclass[12pt, a4paper]{report}

\input{template/main.tex}

\title{\BA{Title in Progress...}}
\author{Boris Andrews}
\affil{Mathematical Institute, University of Oxford}
\date{\today}


\begin{document}
    \pagenumbering{gobble}
    \maketitle
    
    
    \begin{abstract}
        Magnetic confinement reactors---in particular tokamaks---offer one of the most promising options for achieving practical nuclear fusion, with the potential to provide virtually limitless, clean energy. The theoretical and numerical modeling of tokamak plasmas is simultaneously an essential component of effective reactor design, and a great research barrier. Tokamak operational conditions exhibit comparatively low Knudsen numbers. Kinetic effects, including kinetic waves and instabilities, Landau damping, bump-on-tail instabilities and more, are therefore highly influential in tokamak plasma dynamics. Purely fluid models are inherently incapable of capturing these effects, whereas the high dimensionality in purely kinetic models render them practically intractable for most relevant purposes.

        We consider a $\delta\!f$ decomposition model, with a macroscopic fluid background and microscopic kinetic correction, both fully coupled to each other. A similar manner of discretization is proposed to that used in the recent \texttt{STRUPHY} code \cite{Holderied_Possanner_Wang_2021, Holderied_2022, Li_et_al_2023} with a finite-element model for the background and a pseudo-particle/PiC model for the correction.

        The fluid background satisfies the full, non-linear, resistive, compressible, Hall MHD equations. \cite{Laakmann_Hu_Farrell_2022} introduces finite-element(-in-space) implicit timesteppers for the incompressible analogue to this system with structure-preserving (SP) properties in the ideal case, alongside parameter-robust preconditioners. We show that these timesteppers can derive from a finite-element-in-time (FET) (and finite-element-in-space) interpretation. The benefits of this reformulation are discussed, including the derivation of timesteppers that are higher order in time, and the quantifiable dissipative SP properties in the non-ideal, resistive case.
        
        We discuss possible options for extending this FET approach to timesteppers for the compressible case.

        The kinetic corrections satisfy linearized Boltzmann equations. Using a Lénard--Bernstein collision operator, these take Fokker--Planck-like forms \cite{Fokker_1914, Planck_1917} wherein pseudo-particles in the numerical model obey the neoclassical transport equations, with particle-independent Brownian drift terms. This offers a rigorous methodology for incorporating collisions into the particle transport model, without coupling the equations of motions for each particle.
        
        Works by Chen, Chacón et al. \cite{Chen_Chacón_Barnes_2011, Chacón_Chen_Barnes_2013, Chen_Chacón_2014, Chen_Chacón_2015} have developed structure-preserving particle pushers for neoclassical transport in the Vlasov equations, derived from Crank--Nicolson integrators. We show these too can can derive from a FET interpretation, similarly offering potential extensions to higher-order-in-time particle pushers. The FET formulation is used also to consider how the stochastic drift terms can be incorporated into the pushers. Stochastic gyrokinetic expansions are also discussed.

        Different options for the numerical implementation of these schemes are considered.

        Due to the efficacy of FET in the development of SP timesteppers for both the fluid and kinetic component, we hope this approach will prove effective in the future for developing SP timesteppers for the full hybrid model. We hope this will give us the opportunity to incorporate previously inaccessible kinetic effects into the highly effective, modern, finite-element MHD models.
    \end{abstract}
    
    
    \newpage
    \tableofcontents
    
    
    \newpage
    \pagenumbering{arabic}
    %\linenumbers\renewcommand\thelinenumber{\color{black!50}\arabic{linenumber}}
            \input{0 - introduction/main.tex}
        \part{Research}
            \input{1 - low-noise PiC models/main.tex}
            \input{2 - kinetic component/main.tex}
            \input{3 - fluid component/main.tex}
            \input{4 - numerical implementation/main.tex}
        \part{Project Overview}
            \input{5 - research plan/main.tex}
            \input{6 - summary/main.tex}
    
    
    %\section{}
    \newpage
    \pagenumbering{gobble}
        \printbibliography


    \newpage
    \pagenumbering{roman}
    \appendix
        \part{Appendices}
            \input{8 - Hilbert complexes/main.tex}
            \input{9 - weak conservation proofs/main.tex}
\end{document}

    
    
    %\section{}
    \newpage
    \pagenumbering{gobble}
        \printbibliography


    \newpage
    \pagenumbering{roman}
    \appendix
        \part{Appendices}
            \documentclass[12pt, a4paper]{report}

\input{template/main.tex}

\title{\BA{Title in Progress...}}
\author{Boris Andrews}
\affil{Mathematical Institute, University of Oxford}
\date{\today}


\begin{document}
    \pagenumbering{gobble}
    \maketitle
    
    
    \begin{abstract}
        Magnetic confinement reactors---in particular tokamaks---offer one of the most promising options for achieving practical nuclear fusion, with the potential to provide virtually limitless, clean energy. The theoretical and numerical modeling of tokamak plasmas is simultaneously an essential component of effective reactor design, and a great research barrier. Tokamak operational conditions exhibit comparatively low Knudsen numbers. Kinetic effects, including kinetic waves and instabilities, Landau damping, bump-on-tail instabilities and more, are therefore highly influential in tokamak plasma dynamics. Purely fluid models are inherently incapable of capturing these effects, whereas the high dimensionality in purely kinetic models render them practically intractable for most relevant purposes.

        We consider a $\delta\!f$ decomposition model, with a macroscopic fluid background and microscopic kinetic correction, both fully coupled to each other. A similar manner of discretization is proposed to that used in the recent \texttt{STRUPHY} code \cite{Holderied_Possanner_Wang_2021, Holderied_2022, Li_et_al_2023} with a finite-element model for the background and a pseudo-particle/PiC model for the correction.

        The fluid background satisfies the full, non-linear, resistive, compressible, Hall MHD equations. \cite{Laakmann_Hu_Farrell_2022} introduces finite-element(-in-space) implicit timesteppers for the incompressible analogue to this system with structure-preserving (SP) properties in the ideal case, alongside parameter-robust preconditioners. We show that these timesteppers can derive from a finite-element-in-time (FET) (and finite-element-in-space) interpretation. The benefits of this reformulation are discussed, including the derivation of timesteppers that are higher order in time, and the quantifiable dissipative SP properties in the non-ideal, resistive case.
        
        We discuss possible options for extending this FET approach to timesteppers for the compressible case.

        The kinetic corrections satisfy linearized Boltzmann equations. Using a Lénard--Bernstein collision operator, these take Fokker--Planck-like forms \cite{Fokker_1914, Planck_1917} wherein pseudo-particles in the numerical model obey the neoclassical transport equations, with particle-independent Brownian drift terms. This offers a rigorous methodology for incorporating collisions into the particle transport model, without coupling the equations of motions for each particle.
        
        Works by Chen, Chacón et al. \cite{Chen_Chacón_Barnes_2011, Chacón_Chen_Barnes_2013, Chen_Chacón_2014, Chen_Chacón_2015} have developed structure-preserving particle pushers for neoclassical transport in the Vlasov equations, derived from Crank--Nicolson integrators. We show these too can can derive from a FET interpretation, similarly offering potential extensions to higher-order-in-time particle pushers. The FET formulation is used also to consider how the stochastic drift terms can be incorporated into the pushers. Stochastic gyrokinetic expansions are also discussed.

        Different options for the numerical implementation of these schemes are considered.

        Due to the efficacy of FET in the development of SP timesteppers for both the fluid and kinetic component, we hope this approach will prove effective in the future for developing SP timesteppers for the full hybrid model. We hope this will give us the opportunity to incorporate previously inaccessible kinetic effects into the highly effective, modern, finite-element MHD models.
    \end{abstract}
    
    
    \newpage
    \tableofcontents
    
    
    \newpage
    \pagenumbering{arabic}
    %\linenumbers\renewcommand\thelinenumber{\color{black!50}\arabic{linenumber}}
            \input{0 - introduction/main.tex}
        \part{Research}
            \input{1 - low-noise PiC models/main.tex}
            \input{2 - kinetic component/main.tex}
            \input{3 - fluid component/main.tex}
            \input{4 - numerical implementation/main.tex}
        \part{Project Overview}
            \input{5 - research plan/main.tex}
            \input{6 - summary/main.tex}
    
    
    %\section{}
    \newpage
    \pagenumbering{gobble}
        \printbibliography


    \newpage
    \pagenumbering{roman}
    \appendix
        \part{Appendices}
            \input{8 - Hilbert complexes/main.tex}
            \input{9 - weak conservation proofs/main.tex}
\end{document}

            \documentclass[12pt, a4paper]{report}

\input{template/main.tex}

\title{\BA{Title in Progress...}}
\author{Boris Andrews}
\affil{Mathematical Institute, University of Oxford}
\date{\today}


\begin{document}
    \pagenumbering{gobble}
    \maketitle
    
    
    \begin{abstract}
        Magnetic confinement reactors---in particular tokamaks---offer one of the most promising options for achieving practical nuclear fusion, with the potential to provide virtually limitless, clean energy. The theoretical and numerical modeling of tokamak plasmas is simultaneously an essential component of effective reactor design, and a great research barrier. Tokamak operational conditions exhibit comparatively low Knudsen numbers. Kinetic effects, including kinetic waves and instabilities, Landau damping, bump-on-tail instabilities and more, are therefore highly influential in tokamak plasma dynamics. Purely fluid models are inherently incapable of capturing these effects, whereas the high dimensionality in purely kinetic models render them practically intractable for most relevant purposes.

        We consider a $\delta\!f$ decomposition model, with a macroscopic fluid background and microscopic kinetic correction, both fully coupled to each other. A similar manner of discretization is proposed to that used in the recent \texttt{STRUPHY} code \cite{Holderied_Possanner_Wang_2021, Holderied_2022, Li_et_al_2023} with a finite-element model for the background and a pseudo-particle/PiC model for the correction.

        The fluid background satisfies the full, non-linear, resistive, compressible, Hall MHD equations. \cite{Laakmann_Hu_Farrell_2022} introduces finite-element(-in-space) implicit timesteppers for the incompressible analogue to this system with structure-preserving (SP) properties in the ideal case, alongside parameter-robust preconditioners. We show that these timesteppers can derive from a finite-element-in-time (FET) (and finite-element-in-space) interpretation. The benefits of this reformulation are discussed, including the derivation of timesteppers that are higher order in time, and the quantifiable dissipative SP properties in the non-ideal, resistive case.
        
        We discuss possible options for extending this FET approach to timesteppers for the compressible case.

        The kinetic corrections satisfy linearized Boltzmann equations. Using a Lénard--Bernstein collision operator, these take Fokker--Planck-like forms \cite{Fokker_1914, Planck_1917} wherein pseudo-particles in the numerical model obey the neoclassical transport equations, with particle-independent Brownian drift terms. This offers a rigorous methodology for incorporating collisions into the particle transport model, without coupling the equations of motions for each particle.
        
        Works by Chen, Chacón et al. \cite{Chen_Chacón_Barnes_2011, Chacón_Chen_Barnes_2013, Chen_Chacón_2014, Chen_Chacón_2015} have developed structure-preserving particle pushers for neoclassical transport in the Vlasov equations, derived from Crank--Nicolson integrators. We show these too can can derive from a FET interpretation, similarly offering potential extensions to higher-order-in-time particle pushers. The FET formulation is used also to consider how the stochastic drift terms can be incorporated into the pushers. Stochastic gyrokinetic expansions are also discussed.

        Different options for the numerical implementation of these schemes are considered.

        Due to the efficacy of FET in the development of SP timesteppers for both the fluid and kinetic component, we hope this approach will prove effective in the future for developing SP timesteppers for the full hybrid model. We hope this will give us the opportunity to incorporate previously inaccessible kinetic effects into the highly effective, modern, finite-element MHD models.
    \end{abstract}
    
    
    \newpage
    \tableofcontents
    
    
    \newpage
    \pagenumbering{arabic}
    %\linenumbers\renewcommand\thelinenumber{\color{black!50}\arabic{linenumber}}
            \input{0 - introduction/main.tex}
        \part{Research}
            \input{1 - low-noise PiC models/main.tex}
            \input{2 - kinetic component/main.tex}
            \input{3 - fluid component/main.tex}
            \input{4 - numerical implementation/main.tex}
        \part{Project Overview}
            \input{5 - research plan/main.tex}
            \input{6 - summary/main.tex}
    
    
    %\section{}
    \newpage
    \pagenumbering{gobble}
        \printbibliography


    \newpage
    \pagenumbering{roman}
    \appendix
        \part{Appendices}
            \input{8 - Hilbert complexes/main.tex}
            \input{9 - weak conservation proofs/main.tex}
\end{document}

\end{document}

            \documentclass[12pt, a4paper]{report}

\documentclass[12pt, a4paper]{report}

\input{template/main.tex}

\title{\BA{Title in Progress...}}
\author{Boris Andrews}
\affil{Mathematical Institute, University of Oxford}
\date{\today}


\begin{document}
    \pagenumbering{gobble}
    \maketitle
    
    
    \begin{abstract}
        Magnetic confinement reactors---in particular tokamaks---offer one of the most promising options for achieving practical nuclear fusion, with the potential to provide virtually limitless, clean energy. The theoretical and numerical modeling of tokamak plasmas is simultaneously an essential component of effective reactor design, and a great research barrier. Tokamak operational conditions exhibit comparatively low Knudsen numbers. Kinetic effects, including kinetic waves and instabilities, Landau damping, bump-on-tail instabilities and more, are therefore highly influential in tokamak plasma dynamics. Purely fluid models are inherently incapable of capturing these effects, whereas the high dimensionality in purely kinetic models render them practically intractable for most relevant purposes.

        We consider a $\delta\!f$ decomposition model, with a macroscopic fluid background and microscopic kinetic correction, both fully coupled to each other. A similar manner of discretization is proposed to that used in the recent \texttt{STRUPHY} code \cite{Holderied_Possanner_Wang_2021, Holderied_2022, Li_et_al_2023} with a finite-element model for the background and a pseudo-particle/PiC model for the correction.

        The fluid background satisfies the full, non-linear, resistive, compressible, Hall MHD equations. \cite{Laakmann_Hu_Farrell_2022} introduces finite-element(-in-space) implicit timesteppers for the incompressible analogue to this system with structure-preserving (SP) properties in the ideal case, alongside parameter-robust preconditioners. We show that these timesteppers can derive from a finite-element-in-time (FET) (and finite-element-in-space) interpretation. The benefits of this reformulation are discussed, including the derivation of timesteppers that are higher order in time, and the quantifiable dissipative SP properties in the non-ideal, resistive case.
        
        We discuss possible options for extending this FET approach to timesteppers for the compressible case.

        The kinetic corrections satisfy linearized Boltzmann equations. Using a Lénard--Bernstein collision operator, these take Fokker--Planck-like forms \cite{Fokker_1914, Planck_1917} wherein pseudo-particles in the numerical model obey the neoclassical transport equations, with particle-independent Brownian drift terms. This offers a rigorous methodology for incorporating collisions into the particle transport model, without coupling the equations of motions for each particle.
        
        Works by Chen, Chacón et al. \cite{Chen_Chacón_Barnes_2011, Chacón_Chen_Barnes_2013, Chen_Chacón_2014, Chen_Chacón_2015} have developed structure-preserving particle pushers for neoclassical transport in the Vlasov equations, derived from Crank--Nicolson integrators. We show these too can can derive from a FET interpretation, similarly offering potential extensions to higher-order-in-time particle pushers. The FET formulation is used also to consider how the stochastic drift terms can be incorporated into the pushers. Stochastic gyrokinetic expansions are also discussed.

        Different options for the numerical implementation of these schemes are considered.

        Due to the efficacy of FET in the development of SP timesteppers for both the fluid and kinetic component, we hope this approach will prove effective in the future for developing SP timesteppers for the full hybrid model. We hope this will give us the opportunity to incorporate previously inaccessible kinetic effects into the highly effective, modern, finite-element MHD models.
    \end{abstract}
    
    
    \newpage
    \tableofcontents
    
    
    \newpage
    \pagenumbering{arabic}
    %\linenumbers\renewcommand\thelinenumber{\color{black!50}\arabic{linenumber}}
            \input{0 - introduction/main.tex}
        \part{Research}
            \input{1 - low-noise PiC models/main.tex}
            \input{2 - kinetic component/main.tex}
            \input{3 - fluid component/main.tex}
            \input{4 - numerical implementation/main.tex}
        \part{Project Overview}
            \input{5 - research plan/main.tex}
            \input{6 - summary/main.tex}
    
    
    %\section{}
    \newpage
    \pagenumbering{gobble}
        \printbibliography


    \newpage
    \pagenumbering{roman}
    \appendix
        \part{Appendices}
            \input{8 - Hilbert complexes/main.tex}
            \input{9 - weak conservation proofs/main.tex}
\end{document}


\title{\BA{Title in Progress...}}
\author{Boris Andrews}
\affil{Mathematical Institute, University of Oxford}
\date{\today}


\begin{document}
    \pagenumbering{gobble}
    \maketitle
    
    
    \begin{abstract}
        Magnetic confinement reactors---in particular tokamaks---offer one of the most promising options for achieving practical nuclear fusion, with the potential to provide virtually limitless, clean energy. The theoretical and numerical modeling of tokamak plasmas is simultaneously an essential component of effective reactor design, and a great research barrier. Tokamak operational conditions exhibit comparatively low Knudsen numbers. Kinetic effects, including kinetic waves and instabilities, Landau damping, bump-on-tail instabilities and more, are therefore highly influential in tokamak plasma dynamics. Purely fluid models are inherently incapable of capturing these effects, whereas the high dimensionality in purely kinetic models render them practically intractable for most relevant purposes.

        We consider a $\delta\!f$ decomposition model, with a macroscopic fluid background and microscopic kinetic correction, both fully coupled to each other. A similar manner of discretization is proposed to that used in the recent \texttt{STRUPHY} code \cite{Holderied_Possanner_Wang_2021, Holderied_2022, Li_et_al_2023} with a finite-element model for the background and a pseudo-particle/PiC model for the correction.

        The fluid background satisfies the full, non-linear, resistive, compressible, Hall MHD equations. \cite{Laakmann_Hu_Farrell_2022} introduces finite-element(-in-space) implicit timesteppers for the incompressible analogue to this system with structure-preserving (SP) properties in the ideal case, alongside parameter-robust preconditioners. We show that these timesteppers can derive from a finite-element-in-time (FET) (and finite-element-in-space) interpretation. The benefits of this reformulation are discussed, including the derivation of timesteppers that are higher order in time, and the quantifiable dissipative SP properties in the non-ideal, resistive case.
        
        We discuss possible options for extending this FET approach to timesteppers for the compressible case.

        The kinetic corrections satisfy linearized Boltzmann equations. Using a Lénard--Bernstein collision operator, these take Fokker--Planck-like forms \cite{Fokker_1914, Planck_1917} wherein pseudo-particles in the numerical model obey the neoclassical transport equations, with particle-independent Brownian drift terms. This offers a rigorous methodology for incorporating collisions into the particle transport model, without coupling the equations of motions for each particle.
        
        Works by Chen, Chacón et al. \cite{Chen_Chacón_Barnes_2011, Chacón_Chen_Barnes_2013, Chen_Chacón_2014, Chen_Chacón_2015} have developed structure-preserving particle pushers for neoclassical transport in the Vlasov equations, derived from Crank--Nicolson integrators. We show these too can can derive from a FET interpretation, similarly offering potential extensions to higher-order-in-time particle pushers. The FET formulation is used also to consider how the stochastic drift terms can be incorporated into the pushers. Stochastic gyrokinetic expansions are also discussed.

        Different options for the numerical implementation of these schemes are considered.

        Due to the efficacy of FET in the development of SP timesteppers for both the fluid and kinetic component, we hope this approach will prove effective in the future for developing SP timesteppers for the full hybrid model. We hope this will give us the opportunity to incorporate previously inaccessible kinetic effects into the highly effective, modern, finite-element MHD models.
    \end{abstract}
    
    
    \newpage
    \tableofcontents
    
    
    \newpage
    \pagenumbering{arabic}
    %\linenumbers\renewcommand\thelinenumber{\color{black!50}\arabic{linenumber}}
            \documentclass[12pt, a4paper]{report}

\input{template/main.tex}

\title{\BA{Title in Progress...}}
\author{Boris Andrews}
\affil{Mathematical Institute, University of Oxford}
\date{\today}


\begin{document}
    \pagenumbering{gobble}
    \maketitle
    
    
    \begin{abstract}
        Magnetic confinement reactors---in particular tokamaks---offer one of the most promising options for achieving practical nuclear fusion, with the potential to provide virtually limitless, clean energy. The theoretical and numerical modeling of tokamak plasmas is simultaneously an essential component of effective reactor design, and a great research barrier. Tokamak operational conditions exhibit comparatively low Knudsen numbers. Kinetic effects, including kinetic waves and instabilities, Landau damping, bump-on-tail instabilities and more, are therefore highly influential in tokamak plasma dynamics. Purely fluid models are inherently incapable of capturing these effects, whereas the high dimensionality in purely kinetic models render them practically intractable for most relevant purposes.

        We consider a $\delta\!f$ decomposition model, with a macroscopic fluid background and microscopic kinetic correction, both fully coupled to each other. A similar manner of discretization is proposed to that used in the recent \texttt{STRUPHY} code \cite{Holderied_Possanner_Wang_2021, Holderied_2022, Li_et_al_2023} with a finite-element model for the background and a pseudo-particle/PiC model for the correction.

        The fluid background satisfies the full, non-linear, resistive, compressible, Hall MHD equations. \cite{Laakmann_Hu_Farrell_2022} introduces finite-element(-in-space) implicit timesteppers for the incompressible analogue to this system with structure-preserving (SP) properties in the ideal case, alongside parameter-robust preconditioners. We show that these timesteppers can derive from a finite-element-in-time (FET) (and finite-element-in-space) interpretation. The benefits of this reformulation are discussed, including the derivation of timesteppers that are higher order in time, and the quantifiable dissipative SP properties in the non-ideal, resistive case.
        
        We discuss possible options for extending this FET approach to timesteppers for the compressible case.

        The kinetic corrections satisfy linearized Boltzmann equations. Using a Lénard--Bernstein collision operator, these take Fokker--Planck-like forms \cite{Fokker_1914, Planck_1917} wherein pseudo-particles in the numerical model obey the neoclassical transport equations, with particle-independent Brownian drift terms. This offers a rigorous methodology for incorporating collisions into the particle transport model, without coupling the equations of motions for each particle.
        
        Works by Chen, Chacón et al. \cite{Chen_Chacón_Barnes_2011, Chacón_Chen_Barnes_2013, Chen_Chacón_2014, Chen_Chacón_2015} have developed structure-preserving particle pushers for neoclassical transport in the Vlasov equations, derived from Crank--Nicolson integrators. We show these too can can derive from a FET interpretation, similarly offering potential extensions to higher-order-in-time particle pushers. The FET formulation is used also to consider how the stochastic drift terms can be incorporated into the pushers. Stochastic gyrokinetic expansions are also discussed.

        Different options for the numerical implementation of these schemes are considered.

        Due to the efficacy of FET in the development of SP timesteppers for both the fluid and kinetic component, we hope this approach will prove effective in the future for developing SP timesteppers for the full hybrid model. We hope this will give us the opportunity to incorporate previously inaccessible kinetic effects into the highly effective, modern, finite-element MHD models.
    \end{abstract}
    
    
    \newpage
    \tableofcontents
    
    
    \newpage
    \pagenumbering{arabic}
    %\linenumbers\renewcommand\thelinenumber{\color{black!50}\arabic{linenumber}}
            \input{0 - introduction/main.tex}
        \part{Research}
            \input{1 - low-noise PiC models/main.tex}
            \input{2 - kinetic component/main.tex}
            \input{3 - fluid component/main.tex}
            \input{4 - numerical implementation/main.tex}
        \part{Project Overview}
            \input{5 - research plan/main.tex}
            \input{6 - summary/main.tex}
    
    
    %\section{}
    \newpage
    \pagenumbering{gobble}
        \printbibliography


    \newpage
    \pagenumbering{roman}
    \appendix
        \part{Appendices}
            \input{8 - Hilbert complexes/main.tex}
            \input{9 - weak conservation proofs/main.tex}
\end{document}

        \part{Research}
            \documentclass[12pt, a4paper]{report}

\input{template/main.tex}

\title{\BA{Title in Progress...}}
\author{Boris Andrews}
\affil{Mathematical Institute, University of Oxford}
\date{\today}


\begin{document}
    \pagenumbering{gobble}
    \maketitle
    
    
    \begin{abstract}
        Magnetic confinement reactors---in particular tokamaks---offer one of the most promising options for achieving practical nuclear fusion, with the potential to provide virtually limitless, clean energy. The theoretical and numerical modeling of tokamak plasmas is simultaneously an essential component of effective reactor design, and a great research barrier. Tokamak operational conditions exhibit comparatively low Knudsen numbers. Kinetic effects, including kinetic waves and instabilities, Landau damping, bump-on-tail instabilities and more, are therefore highly influential in tokamak plasma dynamics. Purely fluid models are inherently incapable of capturing these effects, whereas the high dimensionality in purely kinetic models render them practically intractable for most relevant purposes.

        We consider a $\delta\!f$ decomposition model, with a macroscopic fluid background and microscopic kinetic correction, both fully coupled to each other. A similar manner of discretization is proposed to that used in the recent \texttt{STRUPHY} code \cite{Holderied_Possanner_Wang_2021, Holderied_2022, Li_et_al_2023} with a finite-element model for the background and a pseudo-particle/PiC model for the correction.

        The fluid background satisfies the full, non-linear, resistive, compressible, Hall MHD equations. \cite{Laakmann_Hu_Farrell_2022} introduces finite-element(-in-space) implicit timesteppers for the incompressible analogue to this system with structure-preserving (SP) properties in the ideal case, alongside parameter-robust preconditioners. We show that these timesteppers can derive from a finite-element-in-time (FET) (and finite-element-in-space) interpretation. The benefits of this reformulation are discussed, including the derivation of timesteppers that are higher order in time, and the quantifiable dissipative SP properties in the non-ideal, resistive case.
        
        We discuss possible options for extending this FET approach to timesteppers for the compressible case.

        The kinetic corrections satisfy linearized Boltzmann equations. Using a Lénard--Bernstein collision operator, these take Fokker--Planck-like forms \cite{Fokker_1914, Planck_1917} wherein pseudo-particles in the numerical model obey the neoclassical transport equations, with particle-independent Brownian drift terms. This offers a rigorous methodology for incorporating collisions into the particle transport model, without coupling the equations of motions for each particle.
        
        Works by Chen, Chacón et al. \cite{Chen_Chacón_Barnes_2011, Chacón_Chen_Barnes_2013, Chen_Chacón_2014, Chen_Chacón_2015} have developed structure-preserving particle pushers for neoclassical transport in the Vlasov equations, derived from Crank--Nicolson integrators. We show these too can can derive from a FET interpretation, similarly offering potential extensions to higher-order-in-time particle pushers. The FET formulation is used also to consider how the stochastic drift terms can be incorporated into the pushers. Stochastic gyrokinetic expansions are also discussed.

        Different options for the numerical implementation of these schemes are considered.

        Due to the efficacy of FET in the development of SP timesteppers for both the fluid and kinetic component, we hope this approach will prove effective in the future for developing SP timesteppers for the full hybrid model. We hope this will give us the opportunity to incorporate previously inaccessible kinetic effects into the highly effective, modern, finite-element MHD models.
    \end{abstract}
    
    
    \newpage
    \tableofcontents
    
    
    \newpage
    \pagenumbering{arabic}
    %\linenumbers\renewcommand\thelinenumber{\color{black!50}\arabic{linenumber}}
            \input{0 - introduction/main.tex}
        \part{Research}
            \input{1 - low-noise PiC models/main.tex}
            \input{2 - kinetic component/main.tex}
            \input{3 - fluid component/main.tex}
            \input{4 - numerical implementation/main.tex}
        \part{Project Overview}
            \input{5 - research plan/main.tex}
            \input{6 - summary/main.tex}
    
    
    %\section{}
    \newpage
    \pagenumbering{gobble}
        \printbibliography


    \newpage
    \pagenumbering{roman}
    \appendix
        \part{Appendices}
            \input{8 - Hilbert complexes/main.tex}
            \input{9 - weak conservation proofs/main.tex}
\end{document}

            \documentclass[12pt, a4paper]{report}

\input{template/main.tex}

\title{\BA{Title in Progress...}}
\author{Boris Andrews}
\affil{Mathematical Institute, University of Oxford}
\date{\today}


\begin{document}
    \pagenumbering{gobble}
    \maketitle
    
    
    \begin{abstract}
        Magnetic confinement reactors---in particular tokamaks---offer one of the most promising options for achieving practical nuclear fusion, with the potential to provide virtually limitless, clean energy. The theoretical and numerical modeling of tokamak plasmas is simultaneously an essential component of effective reactor design, and a great research barrier. Tokamak operational conditions exhibit comparatively low Knudsen numbers. Kinetic effects, including kinetic waves and instabilities, Landau damping, bump-on-tail instabilities and more, are therefore highly influential in tokamak plasma dynamics. Purely fluid models are inherently incapable of capturing these effects, whereas the high dimensionality in purely kinetic models render them practically intractable for most relevant purposes.

        We consider a $\delta\!f$ decomposition model, with a macroscopic fluid background and microscopic kinetic correction, both fully coupled to each other. A similar manner of discretization is proposed to that used in the recent \texttt{STRUPHY} code \cite{Holderied_Possanner_Wang_2021, Holderied_2022, Li_et_al_2023} with a finite-element model for the background and a pseudo-particle/PiC model for the correction.

        The fluid background satisfies the full, non-linear, resistive, compressible, Hall MHD equations. \cite{Laakmann_Hu_Farrell_2022} introduces finite-element(-in-space) implicit timesteppers for the incompressible analogue to this system with structure-preserving (SP) properties in the ideal case, alongside parameter-robust preconditioners. We show that these timesteppers can derive from a finite-element-in-time (FET) (and finite-element-in-space) interpretation. The benefits of this reformulation are discussed, including the derivation of timesteppers that are higher order in time, and the quantifiable dissipative SP properties in the non-ideal, resistive case.
        
        We discuss possible options for extending this FET approach to timesteppers for the compressible case.

        The kinetic corrections satisfy linearized Boltzmann equations. Using a Lénard--Bernstein collision operator, these take Fokker--Planck-like forms \cite{Fokker_1914, Planck_1917} wherein pseudo-particles in the numerical model obey the neoclassical transport equations, with particle-independent Brownian drift terms. This offers a rigorous methodology for incorporating collisions into the particle transport model, without coupling the equations of motions for each particle.
        
        Works by Chen, Chacón et al. \cite{Chen_Chacón_Barnes_2011, Chacón_Chen_Barnes_2013, Chen_Chacón_2014, Chen_Chacón_2015} have developed structure-preserving particle pushers for neoclassical transport in the Vlasov equations, derived from Crank--Nicolson integrators. We show these too can can derive from a FET interpretation, similarly offering potential extensions to higher-order-in-time particle pushers. The FET formulation is used also to consider how the stochastic drift terms can be incorporated into the pushers. Stochastic gyrokinetic expansions are also discussed.

        Different options for the numerical implementation of these schemes are considered.

        Due to the efficacy of FET in the development of SP timesteppers for both the fluid and kinetic component, we hope this approach will prove effective in the future for developing SP timesteppers for the full hybrid model. We hope this will give us the opportunity to incorporate previously inaccessible kinetic effects into the highly effective, modern, finite-element MHD models.
    \end{abstract}
    
    
    \newpage
    \tableofcontents
    
    
    \newpage
    \pagenumbering{arabic}
    %\linenumbers\renewcommand\thelinenumber{\color{black!50}\arabic{linenumber}}
            \input{0 - introduction/main.tex}
        \part{Research}
            \input{1 - low-noise PiC models/main.tex}
            \input{2 - kinetic component/main.tex}
            \input{3 - fluid component/main.tex}
            \input{4 - numerical implementation/main.tex}
        \part{Project Overview}
            \input{5 - research plan/main.tex}
            \input{6 - summary/main.tex}
    
    
    %\section{}
    \newpage
    \pagenumbering{gobble}
        \printbibliography


    \newpage
    \pagenumbering{roman}
    \appendix
        \part{Appendices}
            \input{8 - Hilbert complexes/main.tex}
            \input{9 - weak conservation proofs/main.tex}
\end{document}

            \documentclass[12pt, a4paper]{report}

\input{template/main.tex}

\title{\BA{Title in Progress...}}
\author{Boris Andrews}
\affil{Mathematical Institute, University of Oxford}
\date{\today}


\begin{document}
    \pagenumbering{gobble}
    \maketitle
    
    
    \begin{abstract}
        Magnetic confinement reactors---in particular tokamaks---offer one of the most promising options for achieving practical nuclear fusion, with the potential to provide virtually limitless, clean energy. The theoretical and numerical modeling of tokamak plasmas is simultaneously an essential component of effective reactor design, and a great research barrier. Tokamak operational conditions exhibit comparatively low Knudsen numbers. Kinetic effects, including kinetic waves and instabilities, Landau damping, bump-on-tail instabilities and more, are therefore highly influential in tokamak plasma dynamics. Purely fluid models are inherently incapable of capturing these effects, whereas the high dimensionality in purely kinetic models render them practically intractable for most relevant purposes.

        We consider a $\delta\!f$ decomposition model, with a macroscopic fluid background and microscopic kinetic correction, both fully coupled to each other. A similar manner of discretization is proposed to that used in the recent \texttt{STRUPHY} code \cite{Holderied_Possanner_Wang_2021, Holderied_2022, Li_et_al_2023} with a finite-element model for the background and a pseudo-particle/PiC model for the correction.

        The fluid background satisfies the full, non-linear, resistive, compressible, Hall MHD equations. \cite{Laakmann_Hu_Farrell_2022} introduces finite-element(-in-space) implicit timesteppers for the incompressible analogue to this system with structure-preserving (SP) properties in the ideal case, alongside parameter-robust preconditioners. We show that these timesteppers can derive from a finite-element-in-time (FET) (and finite-element-in-space) interpretation. The benefits of this reformulation are discussed, including the derivation of timesteppers that are higher order in time, and the quantifiable dissipative SP properties in the non-ideal, resistive case.
        
        We discuss possible options for extending this FET approach to timesteppers for the compressible case.

        The kinetic corrections satisfy linearized Boltzmann equations. Using a Lénard--Bernstein collision operator, these take Fokker--Planck-like forms \cite{Fokker_1914, Planck_1917} wherein pseudo-particles in the numerical model obey the neoclassical transport equations, with particle-independent Brownian drift terms. This offers a rigorous methodology for incorporating collisions into the particle transport model, without coupling the equations of motions for each particle.
        
        Works by Chen, Chacón et al. \cite{Chen_Chacón_Barnes_2011, Chacón_Chen_Barnes_2013, Chen_Chacón_2014, Chen_Chacón_2015} have developed structure-preserving particle pushers for neoclassical transport in the Vlasov equations, derived from Crank--Nicolson integrators. We show these too can can derive from a FET interpretation, similarly offering potential extensions to higher-order-in-time particle pushers. The FET formulation is used also to consider how the stochastic drift terms can be incorporated into the pushers. Stochastic gyrokinetic expansions are also discussed.

        Different options for the numerical implementation of these schemes are considered.

        Due to the efficacy of FET in the development of SP timesteppers for both the fluid and kinetic component, we hope this approach will prove effective in the future for developing SP timesteppers for the full hybrid model. We hope this will give us the opportunity to incorporate previously inaccessible kinetic effects into the highly effective, modern, finite-element MHD models.
    \end{abstract}
    
    
    \newpage
    \tableofcontents
    
    
    \newpage
    \pagenumbering{arabic}
    %\linenumbers\renewcommand\thelinenumber{\color{black!50}\arabic{linenumber}}
            \input{0 - introduction/main.tex}
        \part{Research}
            \input{1 - low-noise PiC models/main.tex}
            \input{2 - kinetic component/main.tex}
            \input{3 - fluid component/main.tex}
            \input{4 - numerical implementation/main.tex}
        \part{Project Overview}
            \input{5 - research plan/main.tex}
            \input{6 - summary/main.tex}
    
    
    %\section{}
    \newpage
    \pagenumbering{gobble}
        \printbibliography


    \newpage
    \pagenumbering{roman}
    \appendix
        \part{Appendices}
            \input{8 - Hilbert complexes/main.tex}
            \input{9 - weak conservation proofs/main.tex}
\end{document}

            \documentclass[12pt, a4paper]{report}

\input{template/main.tex}

\title{\BA{Title in Progress...}}
\author{Boris Andrews}
\affil{Mathematical Institute, University of Oxford}
\date{\today}


\begin{document}
    \pagenumbering{gobble}
    \maketitle
    
    
    \begin{abstract}
        Magnetic confinement reactors---in particular tokamaks---offer one of the most promising options for achieving practical nuclear fusion, with the potential to provide virtually limitless, clean energy. The theoretical and numerical modeling of tokamak plasmas is simultaneously an essential component of effective reactor design, and a great research barrier. Tokamak operational conditions exhibit comparatively low Knudsen numbers. Kinetic effects, including kinetic waves and instabilities, Landau damping, bump-on-tail instabilities and more, are therefore highly influential in tokamak plasma dynamics. Purely fluid models are inherently incapable of capturing these effects, whereas the high dimensionality in purely kinetic models render them practically intractable for most relevant purposes.

        We consider a $\delta\!f$ decomposition model, with a macroscopic fluid background and microscopic kinetic correction, both fully coupled to each other. A similar manner of discretization is proposed to that used in the recent \texttt{STRUPHY} code \cite{Holderied_Possanner_Wang_2021, Holderied_2022, Li_et_al_2023} with a finite-element model for the background and a pseudo-particle/PiC model for the correction.

        The fluid background satisfies the full, non-linear, resistive, compressible, Hall MHD equations. \cite{Laakmann_Hu_Farrell_2022} introduces finite-element(-in-space) implicit timesteppers for the incompressible analogue to this system with structure-preserving (SP) properties in the ideal case, alongside parameter-robust preconditioners. We show that these timesteppers can derive from a finite-element-in-time (FET) (and finite-element-in-space) interpretation. The benefits of this reformulation are discussed, including the derivation of timesteppers that are higher order in time, and the quantifiable dissipative SP properties in the non-ideal, resistive case.
        
        We discuss possible options for extending this FET approach to timesteppers for the compressible case.

        The kinetic corrections satisfy linearized Boltzmann equations. Using a Lénard--Bernstein collision operator, these take Fokker--Planck-like forms \cite{Fokker_1914, Planck_1917} wherein pseudo-particles in the numerical model obey the neoclassical transport equations, with particle-independent Brownian drift terms. This offers a rigorous methodology for incorporating collisions into the particle transport model, without coupling the equations of motions for each particle.
        
        Works by Chen, Chacón et al. \cite{Chen_Chacón_Barnes_2011, Chacón_Chen_Barnes_2013, Chen_Chacón_2014, Chen_Chacón_2015} have developed structure-preserving particle pushers for neoclassical transport in the Vlasov equations, derived from Crank--Nicolson integrators. We show these too can can derive from a FET interpretation, similarly offering potential extensions to higher-order-in-time particle pushers. The FET formulation is used also to consider how the stochastic drift terms can be incorporated into the pushers. Stochastic gyrokinetic expansions are also discussed.

        Different options for the numerical implementation of these schemes are considered.

        Due to the efficacy of FET in the development of SP timesteppers for both the fluid and kinetic component, we hope this approach will prove effective in the future for developing SP timesteppers for the full hybrid model. We hope this will give us the opportunity to incorporate previously inaccessible kinetic effects into the highly effective, modern, finite-element MHD models.
    \end{abstract}
    
    
    \newpage
    \tableofcontents
    
    
    \newpage
    \pagenumbering{arabic}
    %\linenumbers\renewcommand\thelinenumber{\color{black!50}\arabic{linenumber}}
            \input{0 - introduction/main.tex}
        \part{Research}
            \input{1 - low-noise PiC models/main.tex}
            \input{2 - kinetic component/main.tex}
            \input{3 - fluid component/main.tex}
            \input{4 - numerical implementation/main.tex}
        \part{Project Overview}
            \input{5 - research plan/main.tex}
            \input{6 - summary/main.tex}
    
    
    %\section{}
    \newpage
    \pagenumbering{gobble}
        \printbibliography


    \newpage
    \pagenumbering{roman}
    \appendix
        \part{Appendices}
            \input{8 - Hilbert complexes/main.tex}
            \input{9 - weak conservation proofs/main.tex}
\end{document}

        \part{Project Overview}
            \documentclass[12pt, a4paper]{report}

\input{template/main.tex}

\title{\BA{Title in Progress...}}
\author{Boris Andrews}
\affil{Mathematical Institute, University of Oxford}
\date{\today}


\begin{document}
    \pagenumbering{gobble}
    \maketitle
    
    
    \begin{abstract}
        Magnetic confinement reactors---in particular tokamaks---offer one of the most promising options for achieving practical nuclear fusion, with the potential to provide virtually limitless, clean energy. The theoretical and numerical modeling of tokamak plasmas is simultaneously an essential component of effective reactor design, and a great research barrier. Tokamak operational conditions exhibit comparatively low Knudsen numbers. Kinetic effects, including kinetic waves and instabilities, Landau damping, bump-on-tail instabilities and more, are therefore highly influential in tokamak plasma dynamics. Purely fluid models are inherently incapable of capturing these effects, whereas the high dimensionality in purely kinetic models render them practically intractable for most relevant purposes.

        We consider a $\delta\!f$ decomposition model, with a macroscopic fluid background and microscopic kinetic correction, both fully coupled to each other. A similar manner of discretization is proposed to that used in the recent \texttt{STRUPHY} code \cite{Holderied_Possanner_Wang_2021, Holderied_2022, Li_et_al_2023} with a finite-element model for the background and a pseudo-particle/PiC model for the correction.

        The fluid background satisfies the full, non-linear, resistive, compressible, Hall MHD equations. \cite{Laakmann_Hu_Farrell_2022} introduces finite-element(-in-space) implicit timesteppers for the incompressible analogue to this system with structure-preserving (SP) properties in the ideal case, alongside parameter-robust preconditioners. We show that these timesteppers can derive from a finite-element-in-time (FET) (and finite-element-in-space) interpretation. The benefits of this reformulation are discussed, including the derivation of timesteppers that are higher order in time, and the quantifiable dissipative SP properties in the non-ideal, resistive case.
        
        We discuss possible options for extending this FET approach to timesteppers for the compressible case.

        The kinetic corrections satisfy linearized Boltzmann equations. Using a Lénard--Bernstein collision operator, these take Fokker--Planck-like forms \cite{Fokker_1914, Planck_1917} wherein pseudo-particles in the numerical model obey the neoclassical transport equations, with particle-independent Brownian drift terms. This offers a rigorous methodology for incorporating collisions into the particle transport model, without coupling the equations of motions for each particle.
        
        Works by Chen, Chacón et al. \cite{Chen_Chacón_Barnes_2011, Chacón_Chen_Barnes_2013, Chen_Chacón_2014, Chen_Chacón_2015} have developed structure-preserving particle pushers for neoclassical transport in the Vlasov equations, derived from Crank--Nicolson integrators. We show these too can can derive from a FET interpretation, similarly offering potential extensions to higher-order-in-time particle pushers. The FET formulation is used also to consider how the stochastic drift terms can be incorporated into the pushers. Stochastic gyrokinetic expansions are also discussed.

        Different options for the numerical implementation of these schemes are considered.

        Due to the efficacy of FET in the development of SP timesteppers for both the fluid and kinetic component, we hope this approach will prove effective in the future for developing SP timesteppers for the full hybrid model. We hope this will give us the opportunity to incorporate previously inaccessible kinetic effects into the highly effective, modern, finite-element MHD models.
    \end{abstract}
    
    
    \newpage
    \tableofcontents
    
    
    \newpage
    \pagenumbering{arabic}
    %\linenumbers\renewcommand\thelinenumber{\color{black!50}\arabic{linenumber}}
            \input{0 - introduction/main.tex}
        \part{Research}
            \input{1 - low-noise PiC models/main.tex}
            \input{2 - kinetic component/main.tex}
            \input{3 - fluid component/main.tex}
            \input{4 - numerical implementation/main.tex}
        \part{Project Overview}
            \input{5 - research plan/main.tex}
            \input{6 - summary/main.tex}
    
    
    %\section{}
    \newpage
    \pagenumbering{gobble}
        \printbibliography


    \newpage
    \pagenumbering{roman}
    \appendix
        \part{Appendices}
            \input{8 - Hilbert complexes/main.tex}
            \input{9 - weak conservation proofs/main.tex}
\end{document}

            \documentclass[12pt, a4paper]{report}

\input{template/main.tex}

\title{\BA{Title in Progress...}}
\author{Boris Andrews}
\affil{Mathematical Institute, University of Oxford}
\date{\today}


\begin{document}
    \pagenumbering{gobble}
    \maketitle
    
    
    \begin{abstract}
        Magnetic confinement reactors---in particular tokamaks---offer one of the most promising options for achieving practical nuclear fusion, with the potential to provide virtually limitless, clean energy. The theoretical and numerical modeling of tokamak plasmas is simultaneously an essential component of effective reactor design, and a great research barrier. Tokamak operational conditions exhibit comparatively low Knudsen numbers. Kinetic effects, including kinetic waves and instabilities, Landau damping, bump-on-tail instabilities and more, are therefore highly influential in tokamak plasma dynamics. Purely fluid models are inherently incapable of capturing these effects, whereas the high dimensionality in purely kinetic models render them practically intractable for most relevant purposes.

        We consider a $\delta\!f$ decomposition model, with a macroscopic fluid background and microscopic kinetic correction, both fully coupled to each other. A similar manner of discretization is proposed to that used in the recent \texttt{STRUPHY} code \cite{Holderied_Possanner_Wang_2021, Holderied_2022, Li_et_al_2023} with a finite-element model for the background and a pseudo-particle/PiC model for the correction.

        The fluid background satisfies the full, non-linear, resistive, compressible, Hall MHD equations. \cite{Laakmann_Hu_Farrell_2022} introduces finite-element(-in-space) implicit timesteppers for the incompressible analogue to this system with structure-preserving (SP) properties in the ideal case, alongside parameter-robust preconditioners. We show that these timesteppers can derive from a finite-element-in-time (FET) (and finite-element-in-space) interpretation. The benefits of this reformulation are discussed, including the derivation of timesteppers that are higher order in time, and the quantifiable dissipative SP properties in the non-ideal, resistive case.
        
        We discuss possible options for extending this FET approach to timesteppers for the compressible case.

        The kinetic corrections satisfy linearized Boltzmann equations. Using a Lénard--Bernstein collision operator, these take Fokker--Planck-like forms \cite{Fokker_1914, Planck_1917} wherein pseudo-particles in the numerical model obey the neoclassical transport equations, with particle-independent Brownian drift terms. This offers a rigorous methodology for incorporating collisions into the particle transport model, without coupling the equations of motions for each particle.
        
        Works by Chen, Chacón et al. \cite{Chen_Chacón_Barnes_2011, Chacón_Chen_Barnes_2013, Chen_Chacón_2014, Chen_Chacón_2015} have developed structure-preserving particle pushers for neoclassical transport in the Vlasov equations, derived from Crank--Nicolson integrators. We show these too can can derive from a FET interpretation, similarly offering potential extensions to higher-order-in-time particle pushers. The FET formulation is used also to consider how the stochastic drift terms can be incorporated into the pushers. Stochastic gyrokinetic expansions are also discussed.

        Different options for the numerical implementation of these schemes are considered.

        Due to the efficacy of FET in the development of SP timesteppers for both the fluid and kinetic component, we hope this approach will prove effective in the future for developing SP timesteppers for the full hybrid model. We hope this will give us the opportunity to incorporate previously inaccessible kinetic effects into the highly effective, modern, finite-element MHD models.
    \end{abstract}
    
    
    \newpage
    \tableofcontents
    
    
    \newpage
    \pagenumbering{arabic}
    %\linenumbers\renewcommand\thelinenumber{\color{black!50}\arabic{linenumber}}
            \input{0 - introduction/main.tex}
        \part{Research}
            \input{1 - low-noise PiC models/main.tex}
            \input{2 - kinetic component/main.tex}
            \input{3 - fluid component/main.tex}
            \input{4 - numerical implementation/main.tex}
        \part{Project Overview}
            \input{5 - research plan/main.tex}
            \input{6 - summary/main.tex}
    
    
    %\section{}
    \newpage
    \pagenumbering{gobble}
        \printbibliography


    \newpage
    \pagenumbering{roman}
    \appendix
        \part{Appendices}
            \input{8 - Hilbert complexes/main.tex}
            \input{9 - weak conservation proofs/main.tex}
\end{document}

    
    
    %\section{}
    \newpage
    \pagenumbering{gobble}
        \printbibliography


    \newpage
    \pagenumbering{roman}
    \appendix
        \part{Appendices}
            \documentclass[12pt, a4paper]{report}

\input{template/main.tex}

\title{\BA{Title in Progress...}}
\author{Boris Andrews}
\affil{Mathematical Institute, University of Oxford}
\date{\today}


\begin{document}
    \pagenumbering{gobble}
    \maketitle
    
    
    \begin{abstract}
        Magnetic confinement reactors---in particular tokamaks---offer one of the most promising options for achieving practical nuclear fusion, with the potential to provide virtually limitless, clean energy. The theoretical and numerical modeling of tokamak plasmas is simultaneously an essential component of effective reactor design, and a great research barrier. Tokamak operational conditions exhibit comparatively low Knudsen numbers. Kinetic effects, including kinetic waves and instabilities, Landau damping, bump-on-tail instabilities and more, are therefore highly influential in tokamak plasma dynamics. Purely fluid models are inherently incapable of capturing these effects, whereas the high dimensionality in purely kinetic models render them practically intractable for most relevant purposes.

        We consider a $\delta\!f$ decomposition model, with a macroscopic fluid background and microscopic kinetic correction, both fully coupled to each other. A similar manner of discretization is proposed to that used in the recent \texttt{STRUPHY} code \cite{Holderied_Possanner_Wang_2021, Holderied_2022, Li_et_al_2023} with a finite-element model for the background and a pseudo-particle/PiC model for the correction.

        The fluid background satisfies the full, non-linear, resistive, compressible, Hall MHD equations. \cite{Laakmann_Hu_Farrell_2022} introduces finite-element(-in-space) implicit timesteppers for the incompressible analogue to this system with structure-preserving (SP) properties in the ideal case, alongside parameter-robust preconditioners. We show that these timesteppers can derive from a finite-element-in-time (FET) (and finite-element-in-space) interpretation. The benefits of this reformulation are discussed, including the derivation of timesteppers that are higher order in time, and the quantifiable dissipative SP properties in the non-ideal, resistive case.
        
        We discuss possible options for extending this FET approach to timesteppers for the compressible case.

        The kinetic corrections satisfy linearized Boltzmann equations. Using a Lénard--Bernstein collision operator, these take Fokker--Planck-like forms \cite{Fokker_1914, Planck_1917} wherein pseudo-particles in the numerical model obey the neoclassical transport equations, with particle-independent Brownian drift terms. This offers a rigorous methodology for incorporating collisions into the particle transport model, without coupling the equations of motions for each particle.
        
        Works by Chen, Chacón et al. \cite{Chen_Chacón_Barnes_2011, Chacón_Chen_Barnes_2013, Chen_Chacón_2014, Chen_Chacón_2015} have developed structure-preserving particle pushers for neoclassical transport in the Vlasov equations, derived from Crank--Nicolson integrators. We show these too can can derive from a FET interpretation, similarly offering potential extensions to higher-order-in-time particle pushers. The FET formulation is used also to consider how the stochastic drift terms can be incorporated into the pushers. Stochastic gyrokinetic expansions are also discussed.

        Different options for the numerical implementation of these schemes are considered.

        Due to the efficacy of FET in the development of SP timesteppers for both the fluid and kinetic component, we hope this approach will prove effective in the future for developing SP timesteppers for the full hybrid model. We hope this will give us the opportunity to incorporate previously inaccessible kinetic effects into the highly effective, modern, finite-element MHD models.
    \end{abstract}
    
    
    \newpage
    \tableofcontents
    
    
    \newpage
    \pagenumbering{arabic}
    %\linenumbers\renewcommand\thelinenumber{\color{black!50}\arabic{linenumber}}
            \input{0 - introduction/main.tex}
        \part{Research}
            \input{1 - low-noise PiC models/main.tex}
            \input{2 - kinetic component/main.tex}
            \input{3 - fluid component/main.tex}
            \input{4 - numerical implementation/main.tex}
        \part{Project Overview}
            \input{5 - research plan/main.tex}
            \input{6 - summary/main.tex}
    
    
    %\section{}
    \newpage
    \pagenumbering{gobble}
        \printbibliography


    \newpage
    \pagenumbering{roman}
    \appendix
        \part{Appendices}
            \input{8 - Hilbert complexes/main.tex}
            \input{9 - weak conservation proofs/main.tex}
\end{document}

            \documentclass[12pt, a4paper]{report}

\input{template/main.tex}

\title{\BA{Title in Progress...}}
\author{Boris Andrews}
\affil{Mathematical Institute, University of Oxford}
\date{\today}


\begin{document}
    \pagenumbering{gobble}
    \maketitle
    
    
    \begin{abstract}
        Magnetic confinement reactors---in particular tokamaks---offer one of the most promising options for achieving practical nuclear fusion, with the potential to provide virtually limitless, clean energy. The theoretical and numerical modeling of tokamak plasmas is simultaneously an essential component of effective reactor design, and a great research barrier. Tokamak operational conditions exhibit comparatively low Knudsen numbers. Kinetic effects, including kinetic waves and instabilities, Landau damping, bump-on-tail instabilities and more, are therefore highly influential in tokamak plasma dynamics. Purely fluid models are inherently incapable of capturing these effects, whereas the high dimensionality in purely kinetic models render them practically intractable for most relevant purposes.

        We consider a $\delta\!f$ decomposition model, with a macroscopic fluid background and microscopic kinetic correction, both fully coupled to each other. A similar manner of discretization is proposed to that used in the recent \texttt{STRUPHY} code \cite{Holderied_Possanner_Wang_2021, Holderied_2022, Li_et_al_2023} with a finite-element model for the background and a pseudo-particle/PiC model for the correction.

        The fluid background satisfies the full, non-linear, resistive, compressible, Hall MHD equations. \cite{Laakmann_Hu_Farrell_2022} introduces finite-element(-in-space) implicit timesteppers for the incompressible analogue to this system with structure-preserving (SP) properties in the ideal case, alongside parameter-robust preconditioners. We show that these timesteppers can derive from a finite-element-in-time (FET) (and finite-element-in-space) interpretation. The benefits of this reformulation are discussed, including the derivation of timesteppers that are higher order in time, and the quantifiable dissipative SP properties in the non-ideal, resistive case.
        
        We discuss possible options for extending this FET approach to timesteppers for the compressible case.

        The kinetic corrections satisfy linearized Boltzmann equations. Using a Lénard--Bernstein collision operator, these take Fokker--Planck-like forms \cite{Fokker_1914, Planck_1917} wherein pseudo-particles in the numerical model obey the neoclassical transport equations, with particle-independent Brownian drift terms. This offers a rigorous methodology for incorporating collisions into the particle transport model, without coupling the equations of motions for each particle.
        
        Works by Chen, Chacón et al. \cite{Chen_Chacón_Barnes_2011, Chacón_Chen_Barnes_2013, Chen_Chacón_2014, Chen_Chacón_2015} have developed structure-preserving particle pushers for neoclassical transport in the Vlasov equations, derived from Crank--Nicolson integrators. We show these too can can derive from a FET interpretation, similarly offering potential extensions to higher-order-in-time particle pushers. The FET formulation is used also to consider how the stochastic drift terms can be incorporated into the pushers. Stochastic gyrokinetic expansions are also discussed.

        Different options for the numerical implementation of these schemes are considered.

        Due to the efficacy of FET in the development of SP timesteppers for both the fluid and kinetic component, we hope this approach will prove effective in the future for developing SP timesteppers for the full hybrid model. We hope this will give us the opportunity to incorporate previously inaccessible kinetic effects into the highly effective, modern, finite-element MHD models.
    \end{abstract}
    
    
    \newpage
    \tableofcontents
    
    
    \newpage
    \pagenumbering{arabic}
    %\linenumbers\renewcommand\thelinenumber{\color{black!50}\arabic{linenumber}}
            \input{0 - introduction/main.tex}
        \part{Research}
            \input{1 - low-noise PiC models/main.tex}
            \input{2 - kinetic component/main.tex}
            \input{3 - fluid component/main.tex}
            \input{4 - numerical implementation/main.tex}
        \part{Project Overview}
            \input{5 - research plan/main.tex}
            \input{6 - summary/main.tex}
    
    
    %\section{}
    \newpage
    \pagenumbering{gobble}
        \printbibliography


    \newpage
    \pagenumbering{roman}
    \appendix
        \part{Appendices}
            \input{8 - Hilbert complexes/main.tex}
            \input{9 - weak conservation proofs/main.tex}
\end{document}

\end{document}

\end{document}

            \documentclass[12pt, a4paper]{report}

\documentclass[12pt, a4paper]{report}

\documentclass[12pt, a4paper]{report}

\input{template/main.tex}

\title{\BA{Title in Progress...}}
\author{Boris Andrews}
\affil{Mathematical Institute, University of Oxford}
\date{\today}


\begin{document}
    \pagenumbering{gobble}
    \maketitle
    
    
    \begin{abstract}
        Magnetic confinement reactors---in particular tokamaks---offer one of the most promising options for achieving practical nuclear fusion, with the potential to provide virtually limitless, clean energy. The theoretical and numerical modeling of tokamak plasmas is simultaneously an essential component of effective reactor design, and a great research barrier. Tokamak operational conditions exhibit comparatively low Knudsen numbers. Kinetic effects, including kinetic waves and instabilities, Landau damping, bump-on-tail instabilities and more, are therefore highly influential in tokamak plasma dynamics. Purely fluid models are inherently incapable of capturing these effects, whereas the high dimensionality in purely kinetic models render them practically intractable for most relevant purposes.

        We consider a $\delta\!f$ decomposition model, with a macroscopic fluid background and microscopic kinetic correction, both fully coupled to each other. A similar manner of discretization is proposed to that used in the recent \texttt{STRUPHY} code \cite{Holderied_Possanner_Wang_2021, Holderied_2022, Li_et_al_2023} with a finite-element model for the background and a pseudo-particle/PiC model for the correction.

        The fluid background satisfies the full, non-linear, resistive, compressible, Hall MHD equations. \cite{Laakmann_Hu_Farrell_2022} introduces finite-element(-in-space) implicit timesteppers for the incompressible analogue to this system with structure-preserving (SP) properties in the ideal case, alongside parameter-robust preconditioners. We show that these timesteppers can derive from a finite-element-in-time (FET) (and finite-element-in-space) interpretation. The benefits of this reformulation are discussed, including the derivation of timesteppers that are higher order in time, and the quantifiable dissipative SP properties in the non-ideal, resistive case.
        
        We discuss possible options for extending this FET approach to timesteppers for the compressible case.

        The kinetic corrections satisfy linearized Boltzmann equations. Using a Lénard--Bernstein collision operator, these take Fokker--Planck-like forms \cite{Fokker_1914, Planck_1917} wherein pseudo-particles in the numerical model obey the neoclassical transport equations, with particle-independent Brownian drift terms. This offers a rigorous methodology for incorporating collisions into the particle transport model, without coupling the equations of motions for each particle.
        
        Works by Chen, Chacón et al. \cite{Chen_Chacón_Barnes_2011, Chacón_Chen_Barnes_2013, Chen_Chacón_2014, Chen_Chacón_2015} have developed structure-preserving particle pushers for neoclassical transport in the Vlasov equations, derived from Crank--Nicolson integrators. We show these too can can derive from a FET interpretation, similarly offering potential extensions to higher-order-in-time particle pushers. The FET formulation is used also to consider how the stochastic drift terms can be incorporated into the pushers. Stochastic gyrokinetic expansions are also discussed.

        Different options for the numerical implementation of these schemes are considered.

        Due to the efficacy of FET in the development of SP timesteppers for both the fluid and kinetic component, we hope this approach will prove effective in the future for developing SP timesteppers for the full hybrid model. We hope this will give us the opportunity to incorporate previously inaccessible kinetic effects into the highly effective, modern, finite-element MHD models.
    \end{abstract}
    
    
    \newpage
    \tableofcontents
    
    
    \newpage
    \pagenumbering{arabic}
    %\linenumbers\renewcommand\thelinenumber{\color{black!50}\arabic{linenumber}}
            \input{0 - introduction/main.tex}
        \part{Research}
            \input{1 - low-noise PiC models/main.tex}
            \input{2 - kinetic component/main.tex}
            \input{3 - fluid component/main.tex}
            \input{4 - numerical implementation/main.tex}
        \part{Project Overview}
            \input{5 - research plan/main.tex}
            \input{6 - summary/main.tex}
    
    
    %\section{}
    \newpage
    \pagenumbering{gobble}
        \printbibliography


    \newpage
    \pagenumbering{roman}
    \appendix
        \part{Appendices}
            \input{8 - Hilbert complexes/main.tex}
            \input{9 - weak conservation proofs/main.tex}
\end{document}


\title{\BA{Title in Progress...}}
\author{Boris Andrews}
\affil{Mathematical Institute, University of Oxford}
\date{\today}


\begin{document}
    \pagenumbering{gobble}
    \maketitle
    
    
    \begin{abstract}
        Magnetic confinement reactors---in particular tokamaks---offer one of the most promising options for achieving practical nuclear fusion, with the potential to provide virtually limitless, clean energy. The theoretical and numerical modeling of tokamak plasmas is simultaneously an essential component of effective reactor design, and a great research barrier. Tokamak operational conditions exhibit comparatively low Knudsen numbers. Kinetic effects, including kinetic waves and instabilities, Landau damping, bump-on-tail instabilities and more, are therefore highly influential in tokamak plasma dynamics. Purely fluid models are inherently incapable of capturing these effects, whereas the high dimensionality in purely kinetic models render them practically intractable for most relevant purposes.

        We consider a $\delta\!f$ decomposition model, with a macroscopic fluid background and microscopic kinetic correction, both fully coupled to each other. A similar manner of discretization is proposed to that used in the recent \texttt{STRUPHY} code \cite{Holderied_Possanner_Wang_2021, Holderied_2022, Li_et_al_2023} with a finite-element model for the background and a pseudo-particle/PiC model for the correction.

        The fluid background satisfies the full, non-linear, resistive, compressible, Hall MHD equations. \cite{Laakmann_Hu_Farrell_2022} introduces finite-element(-in-space) implicit timesteppers for the incompressible analogue to this system with structure-preserving (SP) properties in the ideal case, alongside parameter-robust preconditioners. We show that these timesteppers can derive from a finite-element-in-time (FET) (and finite-element-in-space) interpretation. The benefits of this reformulation are discussed, including the derivation of timesteppers that are higher order in time, and the quantifiable dissipative SP properties in the non-ideal, resistive case.
        
        We discuss possible options for extending this FET approach to timesteppers for the compressible case.

        The kinetic corrections satisfy linearized Boltzmann equations. Using a Lénard--Bernstein collision operator, these take Fokker--Planck-like forms \cite{Fokker_1914, Planck_1917} wherein pseudo-particles in the numerical model obey the neoclassical transport equations, with particle-independent Brownian drift terms. This offers a rigorous methodology for incorporating collisions into the particle transport model, without coupling the equations of motions for each particle.
        
        Works by Chen, Chacón et al. \cite{Chen_Chacón_Barnes_2011, Chacón_Chen_Barnes_2013, Chen_Chacón_2014, Chen_Chacón_2015} have developed structure-preserving particle pushers for neoclassical transport in the Vlasov equations, derived from Crank--Nicolson integrators. We show these too can can derive from a FET interpretation, similarly offering potential extensions to higher-order-in-time particle pushers. The FET formulation is used also to consider how the stochastic drift terms can be incorporated into the pushers. Stochastic gyrokinetic expansions are also discussed.

        Different options for the numerical implementation of these schemes are considered.

        Due to the efficacy of FET in the development of SP timesteppers for both the fluid and kinetic component, we hope this approach will prove effective in the future for developing SP timesteppers for the full hybrid model. We hope this will give us the opportunity to incorporate previously inaccessible kinetic effects into the highly effective, modern, finite-element MHD models.
    \end{abstract}
    
    
    \newpage
    \tableofcontents
    
    
    \newpage
    \pagenumbering{arabic}
    %\linenumbers\renewcommand\thelinenumber{\color{black!50}\arabic{linenumber}}
            \documentclass[12pt, a4paper]{report}

\input{template/main.tex}

\title{\BA{Title in Progress...}}
\author{Boris Andrews}
\affil{Mathematical Institute, University of Oxford}
\date{\today}


\begin{document}
    \pagenumbering{gobble}
    \maketitle
    
    
    \begin{abstract}
        Magnetic confinement reactors---in particular tokamaks---offer one of the most promising options for achieving practical nuclear fusion, with the potential to provide virtually limitless, clean energy. The theoretical and numerical modeling of tokamak plasmas is simultaneously an essential component of effective reactor design, and a great research barrier. Tokamak operational conditions exhibit comparatively low Knudsen numbers. Kinetic effects, including kinetic waves and instabilities, Landau damping, bump-on-tail instabilities and more, are therefore highly influential in tokamak plasma dynamics. Purely fluid models are inherently incapable of capturing these effects, whereas the high dimensionality in purely kinetic models render them practically intractable for most relevant purposes.

        We consider a $\delta\!f$ decomposition model, with a macroscopic fluid background and microscopic kinetic correction, both fully coupled to each other. A similar manner of discretization is proposed to that used in the recent \texttt{STRUPHY} code \cite{Holderied_Possanner_Wang_2021, Holderied_2022, Li_et_al_2023} with a finite-element model for the background and a pseudo-particle/PiC model for the correction.

        The fluid background satisfies the full, non-linear, resistive, compressible, Hall MHD equations. \cite{Laakmann_Hu_Farrell_2022} introduces finite-element(-in-space) implicit timesteppers for the incompressible analogue to this system with structure-preserving (SP) properties in the ideal case, alongside parameter-robust preconditioners. We show that these timesteppers can derive from a finite-element-in-time (FET) (and finite-element-in-space) interpretation. The benefits of this reformulation are discussed, including the derivation of timesteppers that are higher order in time, and the quantifiable dissipative SP properties in the non-ideal, resistive case.
        
        We discuss possible options for extending this FET approach to timesteppers for the compressible case.

        The kinetic corrections satisfy linearized Boltzmann equations. Using a Lénard--Bernstein collision operator, these take Fokker--Planck-like forms \cite{Fokker_1914, Planck_1917} wherein pseudo-particles in the numerical model obey the neoclassical transport equations, with particle-independent Brownian drift terms. This offers a rigorous methodology for incorporating collisions into the particle transport model, without coupling the equations of motions for each particle.
        
        Works by Chen, Chacón et al. \cite{Chen_Chacón_Barnes_2011, Chacón_Chen_Barnes_2013, Chen_Chacón_2014, Chen_Chacón_2015} have developed structure-preserving particle pushers for neoclassical transport in the Vlasov equations, derived from Crank--Nicolson integrators. We show these too can can derive from a FET interpretation, similarly offering potential extensions to higher-order-in-time particle pushers. The FET formulation is used also to consider how the stochastic drift terms can be incorporated into the pushers. Stochastic gyrokinetic expansions are also discussed.

        Different options for the numerical implementation of these schemes are considered.

        Due to the efficacy of FET in the development of SP timesteppers for both the fluid and kinetic component, we hope this approach will prove effective in the future for developing SP timesteppers for the full hybrid model. We hope this will give us the opportunity to incorporate previously inaccessible kinetic effects into the highly effective, modern, finite-element MHD models.
    \end{abstract}
    
    
    \newpage
    \tableofcontents
    
    
    \newpage
    \pagenumbering{arabic}
    %\linenumbers\renewcommand\thelinenumber{\color{black!50}\arabic{linenumber}}
            \input{0 - introduction/main.tex}
        \part{Research}
            \input{1 - low-noise PiC models/main.tex}
            \input{2 - kinetic component/main.tex}
            \input{3 - fluid component/main.tex}
            \input{4 - numerical implementation/main.tex}
        \part{Project Overview}
            \input{5 - research plan/main.tex}
            \input{6 - summary/main.tex}
    
    
    %\section{}
    \newpage
    \pagenumbering{gobble}
        \printbibliography


    \newpage
    \pagenumbering{roman}
    \appendix
        \part{Appendices}
            \input{8 - Hilbert complexes/main.tex}
            \input{9 - weak conservation proofs/main.tex}
\end{document}

        \part{Research}
            \documentclass[12pt, a4paper]{report}

\input{template/main.tex}

\title{\BA{Title in Progress...}}
\author{Boris Andrews}
\affil{Mathematical Institute, University of Oxford}
\date{\today}


\begin{document}
    \pagenumbering{gobble}
    \maketitle
    
    
    \begin{abstract}
        Magnetic confinement reactors---in particular tokamaks---offer one of the most promising options for achieving practical nuclear fusion, with the potential to provide virtually limitless, clean energy. The theoretical and numerical modeling of tokamak plasmas is simultaneously an essential component of effective reactor design, and a great research barrier. Tokamak operational conditions exhibit comparatively low Knudsen numbers. Kinetic effects, including kinetic waves and instabilities, Landau damping, bump-on-tail instabilities and more, are therefore highly influential in tokamak plasma dynamics. Purely fluid models are inherently incapable of capturing these effects, whereas the high dimensionality in purely kinetic models render them practically intractable for most relevant purposes.

        We consider a $\delta\!f$ decomposition model, with a macroscopic fluid background and microscopic kinetic correction, both fully coupled to each other. A similar manner of discretization is proposed to that used in the recent \texttt{STRUPHY} code \cite{Holderied_Possanner_Wang_2021, Holderied_2022, Li_et_al_2023} with a finite-element model for the background and a pseudo-particle/PiC model for the correction.

        The fluid background satisfies the full, non-linear, resistive, compressible, Hall MHD equations. \cite{Laakmann_Hu_Farrell_2022} introduces finite-element(-in-space) implicit timesteppers for the incompressible analogue to this system with structure-preserving (SP) properties in the ideal case, alongside parameter-robust preconditioners. We show that these timesteppers can derive from a finite-element-in-time (FET) (and finite-element-in-space) interpretation. The benefits of this reformulation are discussed, including the derivation of timesteppers that are higher order in time, and the quantifiable dissipative SP properties in the non-ideal, resistive case.
        
        We discuss possible options for extending this FET approach to timesteppers for the compressible case.

        The kinetic corrections satisfy linearized Boltzmann equations. Using a Lénard--Bernstein collision operator, these take Fokker--Planck-like forms \cite{Fokker_1914, Planck_1917} wherein pseudo-particles in the numerical model obey the neoclassical transport equations, with particle-independent Brownian drift terms. This offers a rigorous methodology for incorporating collisions into the particle transport model, without coupling the equations of motions for each particle.
        
        Works by Chen, Chacón et al. \cite{Chen_Chacón_Barnes_2011, Chacón_Chen_Barnes_2013, Chen_Chacón_2014, Chen_Chacón_2015} have developed structure-preserving particle pushers for neoclassical transport in the Vlasov equations, derived from Crank--Nicolson integrators. We show these too can can derive from a FET interpretation, similarly offering potential extensions to higher-order-in-time particle pushers. The FET formulation is used also to consider how the stochastic drift terms can be incorporated into the pushers. Stochastic gyrokinetic expansions are also discussed.

        Different options for the numerical implementation of these schemes are considered.

        Due to the efficacy of FET in the development of SP timesteppers for both the fluid and kinetic component, we hope this approach will prove effective in the future for developing SP timesteppers for the full hybrid model. We hope this will give us the opportunity to incorporate previously inaccessible kinetic effects into the highly effective, modern, finite-element MHD models.
    \end{abstract}
    
    
    \newpage
    \tableofcontents
    
    
    \newpage
    \pagenumbering{arabic}
    %\linenumbers\renewcommand\thelinenumber{\color{black!50}\arabic{linenumber}}
            \input{0 - introduction/main.tex}
        \part{Research}
            \input{1 - low-noise PiC models/main.tex}
            \input{2 - kinetic component/main.tex}
            \input{3 - fluid component/main.tex}
            \input{4 - numerical implementation/main.tex}
        \part{Project Overview}
            \input{5 - research plan/main.tex}
            \input{6 - summary/main.tex}
    
    
    %\section{}
    \newpage
    \pagenumbering{gobble}
        \printbibliography


    \newpage
    \pagenumbering{roman}
    \appendix
        \part{Appendices}
            \input{8 - Hilbert complexes/main.tex}
            \input{9 - weak conservation proofs/main.tex}
\end{document}

            \documentclass[12pt, a4paper]{report}

\input{template/main.tex}

\title{\BA{Title in Progress...}}
\author{Boris Andrews}
\affil{Mathematical Institute, University of Oxford}
\date{\today}


\begin{document}
    \pagenumbering{gobble}
    \maketitle
    
    
    \begin{abstract}
        Magnetic confinement reactors---in particular tokamaks---offer one of the most promising options for achieving practical nuclear fusion, with the potential to provide virtually limitless, clean energy. The theoretical and numerical modeling of tokamak plasmas is simultaneously an essential component of effective reactor design, and a great research barrier. Tokamak operational conditions exhibit comparatively low Knudsen numbers. Kinetic effects, including kinetic waves and instabilities, Landau damping, bump-on-tail instabilities and more, are therefore highly influential in tokamak plasma dynamics. Purely fluid models are inherently incapable of capturing these effects, whereas the high dimensionality in purely kinetic models render them practically intractable for most relevant purposes.

        We consider a $\delta\!f$ decomposition model, with a macroscopic fluid background and microscopic kinetic correction, both fully coupled to each other. A similar manner of discretization is proposed to that used in the recent \texttt{STRUPHY} code \cite{Holderied_Possanner_Wang_2021, Holderied_2022, Li_et_al_2023} with a finite-element model for the background and a pseudo-particle/PiC model for the correction.

        The fluid background satisfies the full, non-linear, resistive, compressible, Hall MHD equations. \cite{Laakmann_Hu_Farrell_2022} introduces finite-element(-in-space) implicit timesteppers for the incompressible analogue to this system with structure-preserving (SP) properties in the ideal case, alongside parameter-robust preconditioners. We show that these timesteppers can derive from a finite-element-in-time (FET) (and finite-element-in-space) interpretation. The benefits of this reformulation are discussed, including the derivation of timesteppers that are higher order in time, and the quantifiable dissipative SP properties in the non-ideal, resistive case.
        
        We discuss possible options for extending this FET approach to timesteppers for the compressible case.

        The kinetic corrections satisfy linearized Boltzmann equations. Using a Lénard--Bernstein collision operator, these take Fokker--Planck-like forms \cite{Fokker_1914, Planck_1917} wherein pseudo-particles in the numerical model obey the neoclassical transport equations, with particle-independent Brownian drift terms. This offers a rigorous methodology for incorporating collisions into the particle transport model, without coupling the equations of motions for each particle.
        
        Works by Chen, Chacón et al. \cite{Chen_Chacón_Barnes_2011, Chacón_Chen_Barnes_2013, Chen_Chacón_2014, Chen_Chacón_2015} have developed structure-preserving particle pushers for neoclassical transport in the Vlasov equations, derived from Crank--Nicolson integrators. We show these too can can derive from a FET interpretation, similarly offering potential extensions to higher-order-in-time particle pushers. The FET formulation is used also to consider how the stochastic drift terms can be incorporated into the pushers. Stochastic gyrokinetic expansions are also discussed.

        Different options for the numerical implementation of these schemes are considered.

        Due to the efficacy of FET in the development of SP timesteppers for both the fluid and kinetic component, we hope this approach will prove effective in the future for developing SP timesteppers for the full hybrid model. We hope this will give us the opportunity to incorporate previously inaccessible kinetic effects into the highly effective, modern, finite-element MHD models.
    \end{abstract}
    
    
    \newpage
    \tableofcontents
    
    
    \newpage
    \pagenumbering{arabic}
    %\linenumbers\renewcommand\thelinenumber{\color{black!50}\arabic{linenumber}}
            \input{0 - introduction/main.tex}
        \part{Research}
            \input{1 - low-noise PiC models/main.tex}
            \input{2 - kinetic component/main.tex}
            \input{3 - fluid component/main.tex}
            \input{4 - numerical implementation/main.tex}
        \part{Project Overview}
            \input{5 - research plan/main.tex}
            \input{6 - summary/main.tex}
    
    
    %\section{}
    \newpage
    \pagenumbering{gobble}
        \printbibliography


    \newpage
    \pagenumbering{roman}
    \appendix
        \part{Appendices}
            \input{8 - Hilbert complexes/main.tex}
            \input{9 - weak conservation proofs/main.tex}
\end{document}

            \documentclass[12pt, a4paper]{report}

\input{template/main.tex}

\title{\BA{Title in Progress...}}
\author{Boris Andrews}
\affil{Mathematical Institute, University of Oxford}
\date{\today}


\begin{document}
    \pagenumbering{gobble}
    \maketitle
    
    
    \begin{abstract}
        Magnetic confinement reactors---in particular tokamaks---offer one of the most promising options for achieving practical nuclear fusion, with the potential to provide virtually limitless, clean energy. The theoretical and numerical modeling of tokamak plasmas is simultaneously an essential component of effective reactor design, and a great research barrier. Tokamak operational conditions exhibit comparatively low Knudsen numbers. Kinetic effects, including kinetic waves and instabilities, Landau damping, bump-on-tail instabilities and more, are therefore highly influential in tokamak plasma dynamics. Purely fluid models are inherently incapable of capturing these effects, whereas the high dimensionality in purely kinetic models render them practically intractable for most relevant purposes.

        We consider a $\delta\!f$ decomposition model, with a macroscopic fluid background and microscopic kinetic correction, both fully coupled to each other. A similar manner of discretization is proposed to that used in the recent \texttt{STRUPHY} code \cite{Holderied_Possanner_Wang_2021, Holderied_2022, Li_et_al_2023} with a finite-element model for the background and a pseudo-particle/PiC model for the correction.

        The fluid background satisfies the full, non-linear, resistive, compressible, Hall MHD equations. \cite{Laakmann_Hu_Farrell_2022} introduces finite-element(-in-space) implicit timesteppers for the incompressible analogue to this system with structure-preserving (SP) properties in the ideal case, alongside parameter-robust preconditioners. We show that these timesteppers can derive from a finite-element-in-time (FET) (and finite-element-in-space) interpretation. The benefits of this reformulation are discussed, including the derivation of timesteppers that are higher order in time, and the quantifiable dissipative SP properties in the non-ideal, resistive case.
        
        We discuss possible options for extending this FET approach to timesteppers for the compressible case.

        The kinetic corrections satisfy linearized Boltzmann equations. Using a Lénard--Bernstein collision operator, these take Fokker--Planck-like forms \cite{Fokker_1914, Planck_1917} wherein pseudo-particles in the numerical model obey the neoclassical transport equations, with particle-independent Brownian drift terms. This offers a rigorous methodology for incorporating collisions into the particle transport model, without coupling the equations of motions for each particle.
        
        Works by Chen, Chacón et al. \cite{Chen_Chacón_Barnes_2011, Chacón_Chen_Barnes_2013, Chen_Chacón_2014, Chen_Chacón_2015} have developed structure-preserving particle pushers for neoclassical transport in the Vlasov equations, derived from Crank--Nicolson integrators. We show these too can can derive from a FET interpretation, similarly offering potential extensions to higher-order-in-time particle pushers. The FET formulation is used also to consider how the stochastic drift terms can be incorporated into the pushers. Stochastic gyrokinetic expansions are also discussed.

        Different options for the numerical implementation of these schemes are considered.

        Due to the efficacy of FET in the development of SP timesteppers for both the fluid and kinetic component, we hope this approach will prove effective in the future for developing SP timesteppers for the full hybrid model. We hope this will give us the opportunity to incorporate previously inaccessible kinetic effects into the highly effective, modern, finite-element MHD models.
    \end{abstract}
    
    
    \newpage
    \tableofcontents
    
    
    \newpage
    \pagenumbering{arabic}
    %\linenumbers\renewcommand\thelinenumber{\color{black!50}\arabic{linenumber}}
            \input{0 - introduction/main.tex}
        \part{Research}
            \input{1 - low-noise PiC models/main.tex}
            \input{2 - kinetic component/main.tex}
            \input{3 - fluid component/main.tex}
            \input{4 - numerical implementation/main.tex}
        \part{Project Overview}
            \input{5 - research plan/main.tex}
            \input{6 - summary/main.tex}
    
    
    %\section{}
    \newpage
    \pagenumbering{gobble}
        \printbibliography


    \newpage
    \pagenumbering{roman}
    \appendix
        \part{Appendices}
            \input{8 - Hilbert complexes/main.tex}
            \input{9 - weak conservation proofs/main.tex}
\end{document}

            \documentclass[12pt, a4paper]{report}

\input{template/main.tex}

\title{\BA{Title in Progress...}}
\author{Boris Andrews}
\affil{Mathematical Institute, University of Oxford}
\date{\today}


\begin{document}
    \pagenumbering{gobble}
    \maketitle
    
    
    \begin{abstract}
        Magnetic confinement reactors---in particular tokamaks---offer one of the most promising options for achieving practical nuclear fusion, with the potential to provide virtually limitless, clean energy. The theoretical and numerical modeling of tokamak plasmas is simultaneously an essential component of effective reactor design, and a great research barrier. Tokamak operational conditions exhibit comparatively low Knudsen numbers. Kinetic effects, including kinetic waves and instabilities, Landau damping, bump-on-tail instabilities and more, are therefore highly influential in tokamak plasma dynamics. Purely fluid models are inherently incapable of capturing these effects, whereas the high dimensionality in purely kinetic models render them practically intractable for most relevant purposes.

        We consider a $\delta\!f$ decomposition model, with a macroscopic fluid background and microscopic kinetic correction, both fully coupled to each other. A similar manner of discretization is proposed to that used in the recent \texttt{STRUPHY} code \cite{Holderied_Possanner_Wang_2021, Holderied_2022, Li_et_al_2023} with a finite-element model for the background and a pseudo-particle/PiC model for the correction.

        The fluid background satisfies the full, non-linear, resistive, compressible, Hall MHD equations. \cite{Laakmann_Hu_Farrell_2022} introduces finite-element(-in-space) implicit timesteppers for the incompressible analogue to this system with structure-preserving (SP) properties in the ideal case, alongside parameter-robust preconditioners. We show that these timesteppers can derive from a finite-element-in-time (FET) (and finite-element-in-space) interpretation. The benefits of this reformulation are discussed, including the derivation of timesteppers that are higher order in time, and the quantifiable dissipative SP properties in the non-ideal, resistive case.
        
        We discuss possible options for extending this FET approach to timesteppers for the compressible case.

        The kinetic corrections satisfy linearized Boltzmann equations. Using a Lénard--Bernstein collision operator, these take Fokker--Planck-like forms \cite{Fokker_1914, Planck_1917} wherein pseudo-particles in the numerical model obey the neoclassical transport equations, with particle-independent Brownian drift terms. This offers a rigorous methodology for incorporating collisions into the particle transport model, without coupling the equations of motions for each particle.
        
        Works by Chen, Chacón et al. \cite{Chen_Chacón_Barnes_2011, Chacón_Chen_Barnes_2013, Chen_Chacón_2014, Chen_Chacón_2015} have developed structure-preserving particle pushers for neoclassical transport in the Vlasov equations, derived from Crank--Nicolson integrators. We show these too can can derive from a FET interpretation, similarly offering potential extensions to higher-order-in-time particle pushers. The FET formulation is used also to consider how the stochastic drift terms can be incorporated into the pushers. Stochastic gyrokinetic expansions are also discussed.

        Different options for the numerical implementation of these schemes are considered.

        Due to the efficacy of FET in the development of SP timesteppers for both the fluid and kinetic component, we hope this approach will prove effective in the future for developing SP timesteppers for the full hybrid model. We hope this will give us the opportunity to incorporate previously inaccessible kinetic effects into the highly effective, modern, finite-element MHD models.
    \end{abstract}
    
    
    \newpage
    \tableofcontents
    
    
    \newpage
    \pagenumbering{arabic}
    %\linenumbers\renewcommand\thelinenumber{\color{black!50}\arabic{linenumber}}
            \input{0 - introduction/main.tex}
        \part{Research}
            \input{1 - low-noise PiC models/main.tex}
            \input{2 - kinetic component/main.tex}
            \input{3 - fluid component/main.tex}
            \input{4 - numerical implementation/main.tex}
        \part{Project Overview}
            \input{5 - research plan/main.tex}
            \input{6 - summary/main.tex}
    
    
    %\section{}
    \newpage
    \pagenumbering{gobble}
        \printbibliography


    \newpage
    \pagenumbering{roman}
    \appendix
        \part{Appendices}
            \input{8 - Hilbert complexes/main.tex}
            \input{9 - weak conservation proofs/main.tex}
\end{document}

        \part{Project Overview}
            \documentclass[12pt, a4paper]{report}

\input{template/main.tex}

\title{\BA{Title in Progress...}}
\author{Boris Andrews}
\affil{Mathematical Institute, University of Oxford}
\date{\today}


\begin{document}
    \pagenumbering{gobble}
    \maketitle
    
    
    \begin{abstract}
        Magnetic confinement reactors---in particular tokamaks---offer one of the most promising options for achieving practical nuclear fusion, with the potential to provide virtually limitless, clean energy. The theoretical and numerical modeling of tokamak plasmas is simultaneously an essential component of effective reactor design, and a great research barrier. Tokamak operational conditions exhibit comparatively low Knudsen numbers. Kinetic effects, including kinetic waves and instabilities, Landau damping, bump-on-tail instabilities and more, are therefore highly influential in tokamak plasma dynamics. Purely fluid models are inherently incapable of capturing these effects, whereas the high dimensionality in purely kinetic models render them practically intractable for most relevant purposes.

        We consider a $\delta\!f$ decomposition model, with a macroscopic fluid background and microscopic kinetic correction, both fully coupled to each other. A similar manner of discretization is proposed to that used in the recent \texttt{STRUPHY} code \cite{Holderied_Possanner_Wang_2021, Holderied_2022, Li_et_al_2023} with a finite-element model for the background and a pseudo-particle/PiC model for the correction.

        The fluid background satisfies the full, non-linear, resistive, compressible, Hall MHD equations. \cite{Laakmann_Hu_Farrell_2022} introduces finite-element(-in-space) implicit timesteppers for the incompressible analogue to this system with structure-preserving (SP) properties in the ideal case, alongside parameter-robust preconditioners. We show that these timesteppers can derive from a finite-element-in-time (FET) (and finite-element-in-space) interpretation. The benefits of this reformulation are discussed, including the derivation of timesteppers that are higher order in time, and the quantifiable dissipative SP properties in the non-ideal, resistive case.
        
        We discuss possible options for extending this FET approach to timesteppers for the compressible case.

        The kinetic corrections satisfy linearized Boltzmann equations. Using a Lénard--Bernstein collision operator, these take Fokker--Planck-like forms \cite{Fokker_1914, Planck_1917} wherein pseudo-particles in the numerical model obey the neoclassical transport equations, with particle-independent Brownian drift terms. This offers a rigorous methodology for incorporating collisions into the particle transport model, without coupling the equations of motions for each particle.
        
        Works by Chen, Chacón et al. \cite{Chen_Chacón_Barnes_2011, Chacón_Chen_Barnes_2013, Chen_Chacón_2014, Chen_Chacón_2015} have developed structure-preserving particle pushers for neoclassical transport in the Vlasov equations, derived from Crank--Nicolson integrators. We show these too can can derive from a FET interpretation, similarly offering potential extensions to higher-order-in-time particle pushers. The FET formulation is used also to consider how the stochastic drift terms can be incorporated into the pushers. Stochastic gyrokinetic expansions are also discussed.

        Different options for the numerical implementation of these schemes are considered.

        Due to the efficacy of FET in the development of SP timesteppers for both the fluid and kinetic component, we hope this approach will prove effective in the future for developing SP timesteppers for the full hybrid model. We hope this will give us the opportunity to incorporate previously inaccessible kinetic effects into the highly effective, modern, finite-element MHD models.
    \end{abstract}
    
    
    \newpage
    \tableofcontents
    
    
    \newpage
    \pagenumbering{arabic}
    %\linenumbers\renewcommand\thelinenumber{\color{black!50}\arabic{linenumber}}
            \input{0 - introduction/main.tex}
        \part{Research}
            \input{1 - low-noise PiC models/main.tex}
            \input{2 - kinetic component/main.tex}
            \input{3 - fluid component/main.tex}
            \input{4 - numerical implementation/main.tex}
        \part{Project Overview}
            \input{5 - research plan/main.tex}
            \input{6 - summary/main.tex}
    
    
    %\section{}
    \newpage
    \pagenumbering{gobble}
        \printbibliography


    \newpage
    \pagenumbering{roman}
    \appendix
        \part{Appendices}
            \input{8 - Hilbert complexes/main.tex}
            \input{9 - weak conservation proofs/main.tex}
\end{document}

            \documentclass[12pt, a4paper]{report}

\input{template/main.tex}

\title{\BA{Title in Progress...}}
\author{Boris Andrews}
\affil{Mathematical Institute, University of Oxford}
\date{\today}


\begin{document}
    \pagenumbering{gobble}
    \maketitle
    
    
    \begin{abstract}
        Magnetic confinement reactors---in particular tokamaks---offer one of the most promising options for achieving practical nuclear fusion, with the potential to provide virtually limitless, clean energy. The theoretical and numerical modeling of tokamak plasmas is simultaneously an essential component of effective reactor design, and a great research barrier. Tokamak operational conditions exhibit comparatively low Knudsen numbers. Kinetic effects, including kinetic waves and instabilities, Landau damping, bump-on-tail instabilities and more, are therefore highly influential in tokamak plasma dynamics. Purely fluid models are inherently incapable of capturing these effects, whereas the high dimensionality in purely kinetic models render them practically intractable for most relevant purposes.

        We consider a $\delta\!f$ decomposition model, with a macroscopic fluid background and microscopic kinetic correction, both fully coupled to each other. A similar manner of discretization is proposed to that used in the recent \texttt{STRUPHY} code \cite{Holderied_Possanner_Wang_2021, Holderied_2022, Li_et_al_2023} with a finite-element model for the background and a pseudo-particle/PiC model for the correction.

        The fluid background satisfies the full, non-linear, resistive, compressible, Hall MHD equations. \cite{Laakmann_Hu_Farrell_2022} introduces finite-element(-in-space) implicit timesteppers for the incompressible analogue to this system with structure-preserving (SP) properties in the ideal case, alongside parameter-robust preconditioners. We show that these timesteppers can derive from a finite-element-in-time (FET) (and finite-element-in-space) interpretation. The benefits of this reformulation are discussed, including the derivation of timesteppers that are higher order in time, and the quantifiable dissipative SP properties in the non-ideal, resistive case.
        
        We discuss possible options for extending this FET approach to timesteppers for the compressible case.

        The kinetic corrections satisfy linearized Boltzmann equations. Using a Lénard--Bernstein collision operator, these take Fokker--Planck-like forms \cite{Fokker_1914, Planck_1917} wherein pseudo-particles in the numerical model obey the neoclassical transport equations, with particle-independent Brownian drift terms. This offers a rigorous methodology for incorporating collisions into the particle transport model, without coupling the equations of motions for each particle.
        
        Works by Chen, Chacón et al. \cite{Chen_Chacón_Barnes_2011, Chacón_Chen_Barnes_2013, Chen_Chacón_2014, Chen_Chacón_2015} have developed structure-preserving particle pushers for neoclassical transport in the Vlasov equations, derived from Crank--Nicolson integrators. We show these too can can derive from a FET interpretation, similarly offering potential extensions to higher-order-in-time particle pushers. The FET formulation is used also to consider how the stochastic drift terms can be incorporated into the pushers. Stochastic gyrokinetic expansions are also discussed.

        Different options for the numerical implementation of these schemes are considered.

        Due to the efficacy of FET in the development of SP timesteppers for both the fluid and kinetic component, we hope this approach will prove effective in the future for developing SP timesteppers for the full hybrid model. We hope this will give us the opportunity to incorporate previously inaccessible kinetic effects into the highly effective, modern, finite-element MHD models.
    \end{abstract}
    
    
    \newpage
    \tableofcontents
    
    
    \newpage
    \pagenumbering{arabic}
    %\linenumbers\renewcommand\thelinenumber{\color{black!50}\arabic{linenumber}}
            \input{0 - introduction/main.tex}
        \part{Research}
            \input{1 - low-noise PiC models/main.tex}
            \input{2 - kinetic component/main.tex}
            \input{3 - fluid component/main.tex}
            \input{4 - numerical implementation/main.tex}
        \part{Project Overview}
            \input{5 - research plan/main.tex}
            \input{6 - summary/main.tex}
    
    
    %\section{}
    \newpage
    \pagenumbering{gobble}
        \printbibliography


    \newpage
    \pagenumbering{roman}
    \appendix
        \part{Appendices}
            \input{8 - Hilbert complexes/main.tex}
            \input{9 - weak conservation proofs/main.tex}
\end{document}

    
    
    %\section{}
    \newpage
    \pagenumbering{gobble}
        \printbibliography


    \newpage
    \pagenumbering{roman}
    \appendix
        \part{Appendices}
            \documentclass[12pt, a4paper]{report}

\input{template/main.tex}

\title{\BA{Title in Progress...}}
\author{Boris Andrews}
\affil{Mathematical Institute, University of Oxford}
\date{\today}


\begin{document}
    \pagenumbering{gobble}
    \maketitle
    
    
    \begin{abstract}
        Magnetic confinement reactors---in particular tokamaks---offer one of the most promising options for achieving practical nuclear fusion, with the potential to provide virtually limitless, clean energy. The theoretical and numerical modeling of tokamak plasmas is simultaneously an essential component of effective reactor design, and a great research barrier. Tokamak operational conditions exhibit comparatively low Knudsen numbers. Kinetic effects, including kinetic waves and instabilities, Landau damping, bump-on-tail instabilities and more, are therefore highly influential in tokamak plasma dynamics. Purely fluid models are inherently incapable of capturing these effects, whereas the high dimensionality in purely kinetic models render them practically intractable for most relevant purposes.

        We consider a $\delta\!f$ decomposition model, with a macroscopic fluid background and microscopic kinetic correction, both fully coupled to each other. A similar manner of discretization is proposed to that used in the recent \texttt{STRUPHY} code \cite{Holderied_Possanner_Wang_2021, Holderied_2022, Li_et_al_2023} with a finite-element model for the background and a pseudo-particle/PiC model for the correction.

        The fluid background satisfies the full, non-linear, resistive, compressible, Hall MHD equations. \cite{Laakmann_Hu_Farrell_2022} introduces finite-element(-in-space) implicit timesteppers for the incompressible analogue to this system with structure-preserving (SP) properties in the ideal case, alongside parameter-robust preconditioners. We show that these timesteppers can derive from a finite-element-in-time (FET) (and finite-element-in-space) interpretation. The benefits of this reformulation are discussed, including the derivation of timesteppers that are higher order in time, and the quantifiable dissipative SP properties in the non-ideal, resistive case.
        
        We discuss possible options for extending this FET approach to timesteppers for the compressible case.

        The kinetic corrections satisfy linearized Boltzmann equations. Using a Lénard--Bernstein collision operator, these take Fokker--Planck-like forms \cite{Fokker_1914, Planck_1917} wherein pseudo-particles in the numerical model obey the neoclassical transport equations, with particle-independent Brownian drift terms. This offers a rigorous methodology for incorporating collisions into the particle transport model, without coupling the equations of motions for each particle.
        
        Works by Chen, Chacón et al. \cite{Chen_Chacón_Barnes_2011, Chacón_Chen_Barnes_2013, Chen_Chacón_2014, Chen_Chacón_2015} have developed structure-preserving particle pushers for neoclassical transport in the Vlasov equations, derived from Crank--Nicolson integrators. We show these too can can derive from a FET interpretation, similarly offering potential extensions to higher-order-in-time particle pushers. The FET formulation is used also to consider how the stochastic drift terms can be incorporated into the pushers. Stochastic gyrokinetic expansions are also discussed.

        Different options for the numerical implementation of these schemes are considered.

        Due to the efficacy of FET in the development of SP timesteppers for both the fluid and kinetic component, we hope this approach will prove effective in the future for developing SP timesteppers for the full hybrid model. We hope this will give us the opportunity to incorporate previously inaccessible kinetic effects into the highly effective, modern, finite-element MHD models.
    \end{abstract}
    
    
    \newpage
    \tableofcontents
    
    
    \newpage
    \pagenumbering{arabic}
    %\linenumbers\renewcommand\thelinenumber{\color{black!50}\arabic{linenumber}}
            \input{0 - introduction/main.tex}
        \part{Research}
            \input{1 - low-noise PiC models/main.tex}
            \input{2 - kinetic component/main.tex}
            \input{3 - fluid component/main.tex}
            \input{4 - numerical implementation/main.tex}
        \part{Project Overview}
            \input{5 - research plan/main.tex}
            \input{6 - summary/main.tex}
    
    
    %\section{}
    \newpage
    \pagenumbering{gobble}
        \printbibliography


    \newpage
    \pagenumbering{roman}
    \appendix
        \part{Appendices}
            \input{8 - Hilbert complexes/main.tex}
            \input{9 - weak conservation proofs/main.tex}
\end{document}

            \documentclass[12pt, a4paper]{report}

\input{template/main.tex}

\title{\BA{Title in Progress...}}
\author{Boris Andrews}
\affil{Mathematical Institute, University of Oxford}
\date{\today}


\begin{document}
    \pagenumbering{gobble}
    \maketitle
    
    
    \begin{abstract}
        Magnetic confinement reactors---in particular tokamaks---offer one of the most promising options for achieving practical nuclear fusion, with the potential to provide virtually limitless, clean energy. The theoretical and numerical modeling of tokamak plasmas is simultaneously an essential component of effective reactor design, and a great research barrier. Tokamak operational conditions exhibit comparatively low Knudsen numbers. Kinetic effects, including kinetic waves and instabilities, Landau damping, bump-on-tail instabilities and more, are therefore highly influential in tokamak plasma dynamics. Purely fluid models are inherently incapable of capturing these effects, whereas the high dimensionality in purely kinetic models render them practically intractable for most relevant purposes.

        We consider a $\delta\!f$ decomposition model, with a macroscopic fluid background and microscopic kinetic correction, both fully coupled to each other. A similar manner of discretization is proposed to that used in the recent \texttt{STRUPHY} code \cite{Holderied_Possanner_Wang_2021, Holderied_2022, Li_et_al_2023} with a finite-element model for the background and a pseudo-particle/PiC model for the correction.

        The fluid background satisfies the full, non-linear, resistive, compressible, Hall MHD equations. \cite{Laakmann_Hu_Farrell_2022} introduces finite-element(-in-space) implicit timesteppers for the incompressible analogue to this system with structure-preserving (SP) properties in the ideal case, alongside parameter-robust preconditioners. We show that these timesteppers can derive from a finite-element-in-time (FET) (and finite-element-in-space) interpretation. The benefits of this reformulation are discussed, including the derivation of timesteppers that are higher order in time, and the quantifiable dissipative SP properties in the non-ideal, resistive case.
        
        We discuss possible options for extending this FET approach to timesteppers for the compressible case.

        The kinetic corrections satisfy linearized Boltzmann equations. Using a Lénard--Bernstein collision operator, these take Fokker--Planck-like forms \cite{Fokker_1914, Planck_1917} wherein pseudo-particles in the numerical model obey the neoclassical transport equations, with particle-independent Brownian drift terms. This offers a rigorous methodology for incorporating collisions into the particle transport model, without coupling the equations of motions for each particle.
        
        Works by Chen, Chacón et al. \cite{Chen_Chacón_Barnes_2011, Chacón_Chen_Barnes_2013, Chen_Chacón_2014, Chen_Chacón_2015} have developed structure-preserving particle pushers for neoclassical transport in the Vlasov equations, derived from Crank--Nicolson integrators. We show these too can can derive from a FET interpretation, similarly offering potential extensions to higher-order-in-time particle pushers. The FET formulation is used also to consider how the stochastic drift terms can be incorporated into the pushers. Stochastic gyrokinetic expansions are also discussed.

        Different options for the numerical implementation of these schemes are considered.

        Due to the efficacy of FET in the development of SP timesteppers for both the fluid and kinetic component, we hope this approach will prove effective in the future for developing SP timesteppers for the full hybrid model. We hope this will give us the opportunity to incorporate previously inaccessible kinetic effects into the highly effective, modern, finite-element MHD models.
    \end{abstract}
    
    
    \newpage
    \tableofcontents
    
    
    \newpage
    \pagenumbering{arabic}
    %\linenumbers\renewcommand\thelinenumber{\color{black!50}\arabic{linenumber}}
            \input{0 - introduction/main.tex}
        \part{Research}
            \input{1 - low-noise PiC models/main.tex}
            \input{2 - kinetic component/main.tex}
            \input{3 - fluid component/main.tex}
            \input{4 - numerical implementation/main.tex}
        \part{Project Overview}
            \input{5 - research plan/main.tex}
            \input{6 - summary/main.tex}
    
    
    %\section{}
    \newpage
    \pagenumbering{gobble}
        \printbibliography


    \newpage
    \pagenumbering{roman}
    \appendix
        \part{Appendices}
            \input{8 - Hilbert complexes/main.tex}
            \input{9 - weak conservation proofs/main.tex}
\end{document}

\end{document}


\title{\BA{Title in Progress...}}
\author{Boris Andrews}
\affil{Mathematical Institute, University of Oxford}
\date{\today}


\begin{document}
    \pagenumbering{gobble}
    \maketitle
    
    
    \begin{abstract}
        Magnetic confinement reactors---in particular tokamaks---offer one of the most promising options for achieving practical nuclear fusion, with the potential to provide virtually limitless, clean energy. The theoretical and numerical modeling of tokamak plasmas is simultaneously an essential component of effective reactor design, and a great research barrier. Tokamak operational conditions exhibit comparatively low Knudsen numbers. Kinetic effects, including kinetic waves and instabilities, Landau damping, bump-on-tail instabilities and more, are therefore highly influential in tokamak plasma dynamics. Purely fluid models are inherently incapable of capturing these effects, whereas the high dimensionality in purely kinetic models render them practically intractable for most relevant purposes.

        We consider a $\delta\!f$ decomposition model, with a macroscopic fluid background and microscopic kinetic correction, both fully coupled to each other. A similar manner of discretization is proposed to that used in the recent \texttt{STRUPHY} code \cite{Holderied_Possanner_Wang_2021, Holderied_2022, Li_et_al_2023} with a finite-element model for the background and a pseudo-particle/PiC model for the correction.

        The fluid background satisfies the full, non-linear, resistive, compressible, Hall MHD equations. \cite{Laakmann_Hu_Farrell_2022} introduces finite-element(-in-space) implicit timesteppers for the incompressible analogue to this system with structure-preserving (SP) properties in the ideal case, alongside parameter-robust preconditioners. We show that these timesteppers can derive from a finite-element-in-time (FET) (and finite-element-in-space) interpretation. The benefits of this reformulation are discussed, including the derivation of timesteppers that are higher order in time, and the quantifiable dissipative SP properties in the non-ideal, resistive case.
        
        We discuss possible options for extending this FET approach to timesteppers for the compressible case.

        The kinetic corrections satisfy linearized Boltzmann equations. Using a Lénard--Bernstein collision operator, these take Fokker--Planck-like forms \cite{Fokker_1914, Planck_1917} wherein pseudo-particles in the numerical model obey the neoclassical transport equations, with particle-independent Brownian drift terms. This offers a rigorous methodology for incorporating collisions into the particle transport model, without coupling the equations of motions for each particle.
        
        Works by Chen, Chacón et al. \cite{Chen_Chacón_Barnes_2011, Chacón_Chen_Barnes_2013, Chen_Chacón_2014, Chen_Chacón_2015} have developed structure-preserving particle pushers for neoclassical transport in the Vlasov equations, derived from Crank--Nicolson integrators. We show these too can can derive from a FET interpretation, similarly offering potential extensions to higher-order-in-time particle pushers. The FET formulation is used also to consider how the stochastic drift terms can be incorporated into the pushers. Stochastic gyrokinetic expansions are also discussed.

        Different options for the numerical implementation of these schemes are considered.

        Due to the efficacy of FET in the development of SP timesteppers for both the fluid and kinetic component, we hope this approach will prove effective in the future for developing SP timesteppers for the full hybrid model. We hope this will give us the opportunity to incorporate previously inaccessible kinetic effects into the highly effective, modern, finite-element MHD models.
    \end{abstract}
    
    
    \newpage
    \tableofcontents
    
    
    \newpage
    \pagenumbering{arabic}
    %\linenumbers\renewcommand\thelinenumber{\color{black!50}\arabic{linenumber}}
            \documentclass[12pt, a4paper]{report}

\documentclass[12pt, a4paper]{report}

\input{template/main.tex}

\title{\BA{Title in Progress...}}
\author{Boris Andrews}
\affil{Mathematical Institute, University of Oxford}
\date{\today}


\begin{document}
    \pagenumbering{gobble}
    \maketitle
    
    
    \begin{abstract}
        Magnetic confinement reactors---in particular tokamaks---offer one of the most promising options for achieving practical nuclear fusion, with the potential to provide virtually limitless, clean energy. The theoretical and numerical modeling of tokamak plasmas is simultaneously an essential component of effective reactor design, and a great research barrier. Tokamak operational conditions exhibit comparatively low Knudsen numbers. Kinetic effects, including kinetic waves and instabilities, Landau damping, bump-on-tail instabilities and more, are therefore highly influential in tokamak plasma dynamics. Purely fluid models are inherently incapable of capturing these effects, whereas the high dimensionality in purely kinetic models render them practically intractable for most relevant purposes.

        We consider a $\delta\!f$ decomposition model, with a macroscopic fluid background and microscopic kinetic correction, both fully coupled to each other. A similar manner of discretization is proposed to that used in the recent \texttt{STRUPHY} code \cite{Holderied_Possanner_Wang_2021, Holderied_2022, Li_et_al_2023} with a finite-element model for the background and a pseudo-particle/PiC model for the correction.

        The fluid background satisfies the full, non-linear, resistive, compressible, Hall MHD equations. \cite{Laakmann_Hu_Farrell_2022} introduces finite-element(-in-space) implicit timesteppers for the incompressible analogue to this system with structure-preserving (SP) properties in the ideal case, alongside parameter-robust preconditioners. We show that these timesteppers can derive from a finite-element-in-time (FET) (and finite-element-in-space) interpretation. The benefits of this reformulation are discussed, including the derivation of timesteppers that are higher order in time, and the quantifiable dissipative SP properties in the non-ideal, resistive case.
        
        We discuss possible options for extending this FET approach to timesteppers for the compressible case.

        The kinetic corrections satisfy linearized Boltzmann equations. Using a Lénard--Bernstein collision operator, these take Fokker--Planck-like forms \cite{Fokker_1914, Planck_1917} wherein pseudo-particles in the numerical model obey the neoclassical transport equations, with particle-independent Brownian drift terms. This offers a rigorous methodology for incorporating collisions into the particle transport model, without coupling the equations of motions for each particle.
        
        Works by Chen, Chacón et al. \cite{Chen_Chacón_Barnes_2011, Chacón_Chen_Barnes_2013, Chen_Chacón_2014, Chen_Chacón_2015} have developed structure-preserving particle pushers for neoclassical transport in the Vlasov equations, derived from Crank--Nicolson integrators. We show these too can can derive from a FET interpretation, similarly offering potential extensions to higher-order-in-time particle pushers. The FET formulation is used also to consider how the stochastic drift terms can be incorporated into the pushers. Stochastic gyrokinetic expansions are also discussed.

        Different options for the numerical implementation of these schemes are considered.

        Due to the efficacy of FET in the development of SP timesteppers for both the fluid and kinetic component, we hope this approach will prove effective in the future for developing SP timesteppers for the full hybrid model. We hope this will give us the opportunity to incorporate previously inaccessible kinetic effects into the highly effective, modern, finite-element MHD models.
    \end{abstract}
    
    
    \newpage
    \tableofcontents
    
    
    \newpage
    \pagenumbering{arabic}
    %\linenumbers\renewcommand\thelinenumber{\color{black!50}\arabic{linenumber}}
            \input{0 - introduction/main.tex}
        \part{Research}
            \input{1 - low-noise PiC models/main.tex}
            \input{2 - kinetic component/main.tex}
            \input{3 - fluid component/main.tex}
            \input{4 - numerical implementation/main.tex}
        \part{Project Overview}
            \input{5 - research plan/main.tex}
            \input{6 - summary/main.tex}
    
    
    %\section{}
    \newpage
    \pagenumbering{gobble}
        \printbibliography


    \newpage
    \pagenumbering{roman}
    \appendix
        \part{Appendices}
            \input{8 - Hilbert complexes/main.tex}
            \input{9 - weak conservation proofs/main.tex}
\end{document}


\title{\BA{Title in Progress...}}
\author{Boris Andrews}
\affil{Mathematical Institute, University of Oxford}
\date{\today}


\begin{document}
    \pagenumbering{gobble}
    \maketitle
    
    
    \begin{abstract}
        Magnetic confinement reactors---in particular tokamaks---offer one of the most promising options for achieving practical nuclear fusion, with the potential to provide virtually limitless, clean energy. The theoretical and numerical modeling of tokamak plasmas is simultaneously an essential component of effective reactor design, and a great research barrier. Tokamak operational conditions exhibit comparatively low Knudsen numbers. Kinetic effects, including kinetic waves and instabilities, Landau damping, bump-on-tail instabilities and more, are therefore highly influential in tokamak plasma dynamics. Purely fluid models are inherently incapable of capturing these effects, whereas the high dimensionality in purely kinetic models render them practically intractable for most relevant purposes.

        We consider a $\delta\!f$ decomposition model, with a macroscopic fluid background and microscopic kinetic correction, both fully coupled to each other. A similar manner of discretization is proposed to that used in the recent \texttt{STRUPHY} code \cite{Holderied_Possanner_Wang_2021, Holderied_2022, Li_et_al_2023} with a finite-element model for the background and a pseudo-particle/PiC model for the correction.

        The fluid background satisfies the full, non-linear, resistive, compressible, Hall MHD equations. \cite{Laakmann_Hu_Farrell_2022} introduces finite-element(-in-space) implicit timesteppers for the incompressible analogue to this system with structure-preserving (SP) properties in the ideal case, alongside parameter-robust preconditioners. We show that these timesteppers can derive from a finite-element-in-time (FET) (and finite-element-in-space) interpretation. The benefits of this reformulation are discussed, including the derivation of timesteppers that are higher order in time, and the quantifiable dissipative SP properties in the non-ideal, resistive case.
        
        We discuss possible options for extending this FET approach to timesteppers for the compressible case.

        The kinetic corrections satisfy linearized Boltzmann equations. Using a Lénard--Bernstein collision operator, these take Fokker--Planck-like forms \cite{Fokker_1914, Planck_1917} wherein pseudo-particles in the numerical model obey the neoclassical transport equations, with particle-independent Brownian drift terms. This offers a rigorous methodology for incorporating collisions into the particle transport model, without coupling the equations of motions for each particle.
        
        Works by Chen, Chacón et al. \cite{Chen_Chacón_Barnes_2011, Chacón_Chen_Barnes_2013, Chen_Chacón_2014, Chen_Chacón_2015} have developed structure-preserving particle pushers for neoclassical transport in the Vlasov equations, derived from Crank--Nicolson integrators. We show these too can can derive from a FET interpretation, similarly offering potential extensions to higher-order-in-time particle pushers. The FET formulation is used also to consider how the stochastic drift terms can be incorporated into the pushers. Stochastic gyrokinetic expansions are also discussed.

        Different options for the numerical implementation of these schemes are considered.

        Due to the efficacy of FET in the development of SP timesteppers for both the fluid and kinetic component, we hope this approach will prove effective in the future for developing SP timesteppers for the full hybrid model. We hope this will give us the opportunity to incorporate previously inaccessible kinetic effects into the highly effective, modern, finite-element MHD models.
    \end{abstract}
    
    
    \newpage
    \tableofcontents
    
    
    \newpage
    \pagenumbering{arabic}
    %\linenumbers\renewcommand\thelinenumber{\color{black!50}\arabic{linenumber}}
            \documentclass[12pt, a4paper]{report}

\input{template/main.tex}

\title{\BA{Title in Progress...}}
\author{Boris Andrews}
\affil{Mathematical Institute, University of Oxford}
\date{\today}


\begin{document}
    \pagenumbering{gobble}
    \maketitle
    
    
    \begin{abstract}
        Magnetic confinement reactors---in particular tokamaks---offer one of the most promising options for achieving practical nuclear fusion, with the potential to provide virtually limitless, clean energy. The theoretical and numerical modeling of tokamak plasmas is simultaneously an essential component of effective reactor design, and a great research barrier. Tokamak operational conditions exhibit comparatively low Knudsen numbers. Kinetic effects, including kinetic waves and instabilities, Landau damping, bump-on-tail instabilities and more, are therefore highly influential in tokamak plasma dynamics. Purely fluid models are inherently incapable of capturing these effects, whereas the high dimensionality in purely kinetic models render them practically intractable for most relevant purposes.

        We consider a $\delta\!f$ decomposition model, with a macroscopic fluid background and microscopic kinetic correction, both fully coupled to each other. A similar manner of discretization is proposed to that used in the recent \texttt{STRUPHY} code \cite{Holderied_Possanner_Wang_2021, Holderied_2022, Li_et_al_2023} with a finite-element model for the background and a pseudo-particle/PiC model for the correction.

        The fluid background satisfies the full, non-linear, resistive, compressible, Hall MHD equations. \cite{Laakmann_Hu_Farrell_2022} introduces finite-element(-in-space) implicit timesteppers for the incompressible analogue to this system with structure-preserving (SP) properties in the ideal case, alongside parameter-robust preconditioners. We show that these timesteppers can derive from a finite-element-in-time (FET) (and finite-element-in-space) interpretation. The benefits of this reformulation are discussed, including the derivation of timesteppers that are higher order in time, and the quantifiable dissipative SP properties in the non-ideal, resistive case.
        
        We discuss possible options for extending this FET approach to timesteppers for the compressible case.

        The kinetic corrections satisfy linearized Boltzmann equations. Using a Lénard--Bernstein collision operator, these take Fokker--Planck-like forms \cite{Fokker_1914, Planck_1917} wherein pseudo-particles in the numerical model obey the neoclassical transport equations, with particle-independent Brownian drift terms. This offers a rigorous methodology for incorporating collisions into the particle transport model, without coupling the equations of motions for each particle.
        
        Works by Chen, Chacón et al. \cite{Chen_Chacón_Barnes_2011, Chacón_Chen_Barnes_2013, Chen_Chacón_2014, Chen_Chacón_2015} have developed structure-preserving particle pushers for neoclassical transport in the Vlasov equations, derived from Crank--Nicolson integrators. We show these too can can derive from a FET interpretation, similarly offering potential extensions to higher-order-in-time particle pushers. The FET formulation is used also to consider how the stochastic drift terms can be incorporated into the pushers. Stochastic gyrokinetic expansions are also discussed.

        Different options for the numerical implementation of these schemes are considered.

        Due to the efficacy of FET in the development of SP timesteppers for both the fluid and kinetic component, we hope this approach will prove effective in the future for developing SP timesteppers for the full hybrid model. We hope this will give us the opportunity to incorporate previously inaccessible kinetic effects into the highly effective, modern, finite-element MHD models.
    \end{abstract}
    
    
    \newpage
    \tableofcontents
    
    
    \newpage
    \pagenumbering{arabic}
    %\linenumbers\renewcommand\thelinenumber{\color{black!50}\arabic{linenumber}}
            \input{0 - introduction/main.tex}
        \part{Research}
            \input{1 - low-noise PiC models/main.tex}
            \input{2 - kinetic component/main.tex}
            \input{3 - fluid component/main.tex}
            \input{4 - numerical implementation/main.tex}
        \part{Project Overview}
            \input{5 - research plan/main.tex}
            \input{6 - summary/main.tex}
    
    
    %\section{}
    \newpage
    \pagenumbering{gobble}
        \printbibliography


    \newpage
    \pagenumbering{roman}
    \appendix
        \part{Appendices}
            \input{8 - Hilbert complexes/main.tex}
            \input{9 - weak conservation proofs/main.tex}
\end{document}

        \part{Research}
            \documentclass[12pt, a4paper]{report}

\input{template/main.tex}

\title{\BA{Title in Progress...}}
\author{Boris Andrews}
\affil{Mathematical Institute, University of Oxford}
\date{\today}


\begin{document}
    \pagenumbering{gobble}
    \maketitle
    
    
    \begin{abstract}
        Magnetic confinement reactors---in particular tokamaks---offer one of the most promising options for achieving practical nuclear fusion, with the potential to provide virtually limitless, clean energy. The theoretical and numerical modeling of tokamak plasmas is simultaneously an essential component of effective reactor design, and a great research barrier. Tokamak operational conditions exhibit comparatively low Knudsen numbers. Kinetic effects, including kinetic waves and instabilities, Landau damping, bump-on-tail instabilities and more, are therefore highly influential in tokamak plasma dynamics. Purely fluid models are inherently incapable of capturing these effects, whereas the high dimensionality in purely kinetic models render them practically intractable for most relevant purposes.

        We consider a $\delta\!f$ decomposition model, with a macroscopic fluid background and microscopic kinetic correction, both fully coupled to each other. A similar manner of discretization is proposed to that used in the recent \texttt{STRUPHY} code \cite{Holderied_Possanner_Wang_2021, Holderied_2022, Li_et_al_2023} with a finite-element model for the background and a pseudo-particle/PiC model for the correction.

        The fluid background satisfies the full, non-linear, resistive, compressible, Hall MHD equations. \cite{Laakmann_Hu_Farrell_2022} introduces finite-element(-in-space) implicit timesteppers for the incompressible analogue to this system with structure-preserving (SP) properties in the ideal case, alongside parameter-robust preconditioners. We show that these timesteppers can derive from a finite-element-in-time (FET) (and finite-element-in-space) interpretation. The benefits of this reformulation are discussed, including the derivation of timesteppers that are higher order in time, and the quantifiable dissipative SP properties in the non-ideal, resistive case.
        
        We discuss possible options for extending this FET approach to timesteppers for the compressible case.

        The kinetic corrections satisfy linearized Boltzmann equations. Using a Lénard--Bernstein collision operator, these take Fokker--Planck-like forms \cite{Fokker_1914, Planck_1917} wherein pseudo-particles in the numerical model obey the neoclassical transport equations, with particle-independent Brownian drift terms. This offers a rigorous methodology for incorporating collisions into the particle transport model, without coupling the equations of motions for each particle.
        
        Works by Chen, Chacón et al. \cite{Chen_Chacón_Barnes_2011, Chacón_Chen_Barnes_2013, Chen_Chacón_2014, Chen_Chacón_2015} have developed structure-preserving particle pushers for neoclassical transport in the Vlasov equations, derived from Crank--Nicolson integrators. We show these too can can derive from a FET interpretation, similarly offering potential extensions to higher-order-in-time particle pushers. The FET formulation is used also to consider how the stochastic drift terms can be incorporated into the pushers. Stochastic gyrokinetic expansions are also discussed.

        Different options for the numerical implementation of these schemes are considered.

        Due to the efficacy of FET in the development of SP timesteppers for both the fluid and kinetic component, we hope this approach will prove effective in the future for developing SP timesteppers for the full hybrid model. We hope this will give us the opportunity to incorporate previously inaccessible kinetic effects into the highly effective, modern, finite-element MHD models.
    \end{abstract}
    
    
    \newpage
    \tableofcontents
    
    
    \newpage
    \pagenumbering{arabic}
    %\linenumbers\renewcommand\thelinenumber{\color{black!50}\arabic{linenumber}}
            \input{0 - introduction/main.tex}
        \part{Research}
            \input{1 - low-noise PiC models/main.tex}
            \input{2 - kinetic component/main.tex}
            \input{3 - fluid component/main.tex}
            \input{4 - numerical implementation/main.tex}
        \part{Project Overview}
            \input{5 - research plan/main.tex}
            \input{6 - summary/main.tex}
    
    
    %\section{}
    \newpage
    \pagenumbering{gobble}
        \printbibliography


    \newpage
    \pagenumbering{roman}
    \appendix
        \part{Appendices}
            \input{8 - Hilbert complexes/main.tex}
            \input{9 - weak conservation proofs/main.tex}
\end{document}

            \documentclass[12pt, a4paper]{report}

\input{template/main.tex}

\title{\BA{Title in Progress...}}
\author{Boris Andrews}
\affil{Mathematical Institute, University of Oxford}
\date{\today}


\begin{document}
    \pagenumbering{gobble}
    \maketitle
    
    
    \begin{abstract}
        Magnetic confinement reactors---in particular tokamaks---offer one of the most promising options for achieving practical nuclear fusion, with the potential to provide virtually limitless, clean energy. The theoretical and numerical modeling of tokamak plasmas is simultaneously an essential component of effective reactor design, and a great research barrier. Tokamak operational conditions exhibit comparatively low Knudsen numbers. Kinetic effects, including kinetic waves and instabilities, Landau damping, bump-on-tail instabilities and more, are therefore highly influential in tokamak plasma dynamics. Purely fluid models are inherently incapable of capturing these effects, whereas the high dimensionality in purely kinetic models render them practically intractable for most relevant purposes.

        We consider a $\delta\!f$ decomposition model, with a macroscopic fluid background and microscopic kinetic correction, both fully coupled to each other. A similar manner of discretization is proposed to that used in the recent \texttt{STRUPHY} code \cite{Holderied_Possanner_Wang_2021, Holderied_2022, Li_et_al_2023} with a finite-element model for the background and a pseudo-particle/PiC model for the correction.

        The fluid background satisfies the full, non-linear, resistive, compressible, Hall MHD equations. \cite{Laakmann_Hu_Farrell_2022} introduces finite-element(-in-space) implicit timesteppers for the incompressible analogue to this system with structure-preserving (SP) properties in the ideal case, alongside parameter-robust preconditioners. We show that these timesteppers can derive from a finite-element-in-time (FET) (and finite-element-in-space) interpretation. The benefits of this reformulation are discussed, including the derivation of timesteppers that are higher order in time, and the quantifiable dissipative SP properties in the non-ideal, resistive case.
        
        We discuss possible options for extending this FET approach to timesteppers for the compressible case.

        The kinetic corrections satisfy linearized Boltzmann equations. Using a Lénard--Bernstein collision operator, these take Fokker--Planck-like forms \cite{Fokker_1914, Planck_1917} wherein pseudo-particles in the numerical model obey the neoclassical transport equations, with particle-independent Brownian drift terms. This offers a rigorous methodology for incorporating collisions into the particle transport model, without coupling the equations of motions for each particle.
        
        Works by Chen, Chacón et al. \cite{Chen_Chacón_Barnes_2011, Chacón_Chen_Barnes_2013, Chen_Chacón_2014, Chen_Chacón_2015} have developed structure-preserving particle pushers for neoclassical transport in the Vlasov equations, derived from Crank--Nicolson integrators. We show these too can can derive from a FET interpretation, similarly offering potential extensions to higher-order-in-time particle pushers. The FET formulation is used also to consider how the stochastic drift terms can be incorporated into the pushers. Stochastic gyrokinetic expansions are also discussed.

        Different options for the numerical implementation of these schemes are considered.

        Due to the efficacy of FET in the development of SP timesteppers for both the fluid and kinetic component, we hope this approach will prove effective in the future for developing SP timesteppers for the full hybrid model. We hope this will give us the opportunity to incorporate previously inaccessible kinetic effects into the highly effective, modern, finite-element MHD models.
    \end{abstract}
    
    
    \newpage
    \tableofcontents
    
    
    \newpage
    \pagenumbering{arabic}
    %\linenumbers\renewcommand\thelinenumber{\color{black!50}\arabic{linenumber}}
            \input{0 - introduction/main.tex}
        \part{Research}
            \input{1 - low-noise PiC models/main.tex}
            \input{2 - kinetic component/main.tex}
            \input{3 - fluid component/main.tex}
            \input{4 - numerical implementation/main.tex}
        \part{Project Overview}
            \input{5 - research plan/main.tex}
            \input{6 - summary/main.tex}
    
    
    %\section{}
    \newpage
    \pagenumbering{gobble}
        \printbibliography


    \newpage
    \pagenumbering{roman}
    \appendix
        \part{Appendices}
            \input{8 - Hilbert complexes/main.tex}
            \input{9 - weak conservation proofs/main.tex}
\end{document}

            \documentclass[12pt, a4paper]{report}

\input{template/main.tex}

\title{\BA{Title in Progress...}}
\author{Boris Andrews}
\affil{Mathematical Institute, University of Oxford}
\date{\today}


\begin{document}
    \pagenumbering{gobble}
    \maketitle
    
    
    \begin{abstract}
        Magnetic confinement reactors---in particular tokamaks---offer one of the most promising options for achieving practical nuclear fusion, with the potential to provide virtually limitless, clean energy. The theoretical and numerical modeling of tokamak plasmas is simultaneously an essential component of effective reactor design, and a great research barrier. Tokamak operational conditions exhibit comparatively low Knudsen numbers. Kinetic effects, including kinetic waves and instabilities, Landau damping, bump-on-tail instabilities and more, are therefore highly influential in tokamak plasma dynamics. Purely fluid models are inherently incapable of capturing these effects, whereas the high dimensionality in purely kinetic models render them practically intractable for most relevant purposes.

        We consider a $\delta\!f$ decomposition model, with a macroscopic fluid background and microscopic kinetic correction, both fully coupled to each other. A similar manner of discretization is proposed to that used in the recent \texttt{STRUPHY} code \cite{Holderied_Possanner_Wang_2021, Holderied_2022, Li_et_al_2023} with a finite-element model for the background and a pseudo-particle/PiC model for the correction.

        The fluid background satisfies the full, non-linear, resistive, compressible, Hall MHD equations. \cite{Laakmann_Hu_Farrell_2022} introduces finite-element(-in-space) implicit timesteppers for the incompressible analogue to this system with structure-preserving (SP) properties in the ideal case, alongside parameter-robust preconditioners. We show that these timesteppers can derive from a finite-element-in-time (FET) (and finite-element-in-space) interpretation. The benefits of this reformulation are discussed, including the derivation of timesteppers that are higher order in time, and the quantifiable dissipative SP properties in the non-ideal, resistive case.
        
        We discuss possible options for extending this FET approach to timesteppers for the compressible case.

        The kinetic corrections satisfy linearized Boltzmann equations. Using a Lénard--Bernstein collision operator, these take Fokker--Planck-like forms \cite{Fokker_1914, Planck_1917} wherein pseudo-particles in the numerical model obey the neoclassical transport equations, with particle-independent Brownian drift terms. This offers a rigorous methodology for incorporating collisions into the particle transport model, without coupling the equations of motions for each particle.
        
        Works by Chen, Chacón et al. \cite{Chen_Chacón_Barnes_2011, Chacón_Chen_Barnes_2013, Chen_Chacón_2014, Chen_Chacón_2015} have developed structure-preserving particle pushers for neoclassical transport in the Vlasov equations, derived from Crank--Nicolson integrators. We show these too can can derive from a FET interpretation, similarly offering potential extensions to higher-order-in-time particle pushers. The FET formulation is used also to consider how the stochastic drift terms can be incorporated into the pushers. Stochastic gyrokinetic expansions are also discussed.

        Different options for the numerical implementation of these schemes are considered.

        Due to the efficacy of FET in the development of SP timesteppers for both the fluid and kinetic component, we hope this approach will prove effective in the future for developing SP timesteppers for the full hybrid model. We hope this will give us the opportunity to incorporate previously inaccessible kinetic effects into the highly effective, modern, finite-element MHD models.
    \end{abstract}
    
    
    \newpage
    \tableofcontents
    
    
    \newpage
    \pagenumbering{arabic}
    %\linenumbers\renewcommand\thelinenumber{\color{black!50}\arabic{linenumber}}
            \input{0 - introduction/main.tex}
        \part{Research}
            \input{1 - low-noise PiC models/main.tex}
            \input{2 - kinetic component/main.tex}
            \input{3 - fluid component/main.tex}
            \input{4 - numerical implementation/main.tex}
        \part{Project Overview}
            \input{5 - research plan/main.tex}
            \input{6 - summary/main.tex}
    
    
    %\section{}
    \newpage
    \pagenumbering{gobble}
        \printbibliography


    \newpage
    \pagenumbering{roman}
    \appendix
        \part{Appendices}
            \input{8 - Hilbert complexes/main.tex}
            \input{9 - weak conservation proofs/main.tex}
\end{document}

            \documentclass[12pt, a4paper]{report}

\input{template/main.tex}

\title{\BA{Title in Progress...}}
\author{Boris Andrews}
\affil{Mathematical Institute, University of Oxford}
\date{\today}


\begin{document}
    \pagenumbering{gobble}
    \maketitle
    
    
    \begin{abstract}
        Magnetic confinement reactors---in particular tokamaks---offer one of the most promising options for achieving practical nuclear fusion, with the potential to provide virtually limitless, clean energy. The theoretical and numerical modeling of tokamak plasmas is simultaneously an essential component of effective reactor design, and a great research barrier. Tokamak operational conditions exhibit comparatively low Knudsen numbers. Kinetic effects, including kinetic waves and instabilities, Landau damping, bump-on-tail instabilities and more, are therefore highly influential in tokamak plasma dynamics. Purely fluid models are inherently incapable of capturing these effects, whereas the high dimensionality in purely kinetic models render them practically intractable for most relevant purposes.

        We consider a $\delta\!f$ decomposition model, with a macroscopic fluid background and microscopic kinetic correction, both fully coupled to each other. A similar manner of discretization is proposed to that used in the recent \texttt{STRUPHY} code \cite{Holderied_Possanner_Wang_2021, Holderied_2022, Li_et_al_2023} with a finite-element model for the background and a pseudo-particle/PiC model for the correction.

        The fluid background satisfies the full, non-linear, resistive, compressible, Hall MHD equations. \cite{Laakmann_Hu_Farrell_2022} introduces finite-element(-in-space) implicit timesteppers for the incompressible analogue to this system with structure-preserving (SP) properties in the ideal case, alongside parameter-robust preconditioners. We show that these timesteppers can derive from a finite-element-in-time (FET) (and finite-element-in-space) interpretation. The benefits of this reformulation are discussed, including the derivation of timesteppers that are higher order in time, and the quantifiable dissipative SP properties in the non-ideal, resistive case.
        
        We discuss possible options for extending this FET approach to timesteppers for the compressible case.

        The kinetic corrections satisfy linearized Boltzmann equations. Using a Lénard--Bernstein collision operator, these take Fokker--Planck-like forms \cite{Fokker_1914, Planck_1917} wherein pseudo-particles in the numerical model obey the neoclassical transport equations, with particle-independent Brownian drift terms. This offers a rigorous methodology for incorporating collisions into the particle transport model, without coupling the equations of motions for each particle.
        
        Works by Chen, Chacón et al. \cite{Chen_Chacón_Barnes_2011, Chacón_Chen_Barnes_2013, Chen_Chacón_2014, Chen_Chacón_2015} have developed structure-preserving particle pushers for neoclassical transport in the Vlasov equations, derived from Crank--Nicolson integrators. We show these too can can derive from a FET interpretation, similarly offering potential extensions to higher-order-in-time particle pushers. The FET formulation is used also to consider how the stochastic drift terms can be incorporated into the pushers. Stochastic gyrokinetic expansions are also discussed.

        Different options for the numerical implementation of these schemes are considered.

        Due to the efficacy of FET in the development of SP timesteppers for both the fluid and kinetic component, we hope this approach will prove effective in the future for developing SP timesteppers for the full hybrid model. We hope this will give us the opportunity to incorporate previously inaccessible kinetic effects into the highly effective, modern, finite-element MHD models.
    \end{abstract}
    
    
    \newpage
    \tableofcontents
    
    
    \newpage
    \pagenumbering{arabic}
    %\linenumbers\renewcommand\thelinenumber{\color{black!50}\arabic{linenumber}}
            \input{0 - introduction/main.tex}
        \part{Research}
            \input{1 - low-noise PiC models/main.tex}
            \input{2 - kinetic component/main.tex}
            \input{3 - fluid component/main.tex}
            \input{4 - numerical implementation/main.tex}
        \part{Project Overview}
            \input{5 - research plan/main.tex}
            \input{6 - summary/main.tex}
    
    
    %\section{}
    \newpage
    \pagenumbering{gobble}
        \printbibliography


    \newpage
    \pagenumbering{roman}
    \appendix
        \part{Appendices}
            \input{8 - Hilbert complexes/main.tex}
            \input{9 - weak conservation proofs/main.tex}
\end{document}

        \part{Project Overview}
            \documentclass[12pt, a4paper]{report}

\input{template/main.tex}

\title{\BA{Title in Progress...}}
\author{Boris Andrews}
\affil{Mathematical Institute, University of Oxford}
\date{\today}


\begin{document}
    \pagenumbering{gobble}
    \maketitle
    
    
    \begin{abstract}
        Magnetic confinement reactors---in particular tokamaks---offer one of the most promising options for achieving practical nuclear fusion, with the potential to provide virtually limitless, clean energy. The theoretical and numerical modeling of tokamak plasmas is simultaneously an essential component of effective reactor design, and a great research barrier. Tokamak operational conditions exhibit comparatively low Knudsen numbers. Kinetic effects, including kinetic waves and instabilities, Landau damping, bump-on-tail instabilities and more, are therefore highly influential in tokamak plasma dynamics. Purely fluid models are inherently incapable of capturing these effects, whereas the high dimensionality in purely kinetic models render them practically intractable for most relevant purposes.

        We consider a $\delta\!f$ decomposition model, with a macroscopic fluid background and microscopic kinetic correction, both fully coupled to each other. A similar manner of discretization is proposed to that used in the recent \texttt{STRUPHY} code \cite{Holderied_Possanner_Wang_2021, Holderied_2022, Li_et_al_2023} with a finite-element model for the background and a pseudo-particle/PiC model for the correction.

        The fluid background satisfies the full, non-linear, resistive, compressible, Hall MHD equations. \cite{Laakmann_Hu_Farrell_2022} introduces finite-element(-in-space) implicit timesteppers for the incompressible analogue to this system with structure-preserving (SP) properties in the ideal case, alongside parameter-robust preconditioners. We show that these timesteppers can derive from a finite-element-in-time (FET) (and finite-element-in-space) interpretation. The benefits of this reformulation are discussed, including the derivation of timesteppers that are higher order in time, and the quantifiable dissipative SP properties in the non-ideal, resistive case.
        
        We discuss possible options for extending this FET approach to timesteppers for the compressible case.

        The kinetic corrections satisfy linearized Boltzmann equations. Using a Lénard--Bernstein collision operator, these take Fokker--Planck-like forms \cite{Fokker_1914, Planck_1917} wherein pseudo-particles in the numerical model obey the neoclassical transport equations, with particle-independent Brownian drift terms. This offers a rigorous methodology for incorporating collisions into the particle transport model, without coupling the equations of motions for each particle.
        
        Works by Chen, Chacón et al. \cite{Chen_Chacón_Barnes_2011, Chacón_Chen_Barnes_2013, Chen_Chacón_2014, Chen_Chacón_2015} have developed structure-preserving particle pushers for neoclassical transport in the Vlasov equations, derived from Crank--Nicolson integrators. We show these too can can derive from a FET interpretation, similarly offering potential extensions to higher-order-in-time particle pushers. The FET formulation is used also to consider how the stochastic drift terms can be incorporated into the pushers. Stochastic gyrokinetic expansions are also discussed.

        Different options for the numerical implementation of these schemes are considered.

        Due to the efficacy of FET in the development of SP timesteppers for both the fluid and kinetic component, we hope this approach will prove effective in the future for developing SP timesteppers for the full hybrid model. We hope this will give us the opportunity to incorporate previously inaccessible kinetic effects into the highly effective, modern, finite-element MHD models.
    \end{abstract}
    
    
    \newpage
    \tableofcontents
    
    
    \newpage
    \pagenumbering{arabic}
    %\linenumbers\renewcommand\thelinenumber{\color{black!50}\arabic{linenumber}}
            \input{0 - introduction/main.tex}
        \part{Research}
            \input{1 - low-noise PiC models/main.tex}
            \input{2 - kinetic component/main.tex}
            \input{3 - fluid component/main.tex}
            \input{4 - numerical implementation/main.tex}
        \part{Project Overview}
            \input{5 - research plan/main.tex}
            \input{6 - summary/main.tex}
    
    
    %\section{}
    \newpage
    \pagenumbering{gobble}
        \printbibliography


    \newpage
    \pagenumbering{roman}
    \appendix
        \part{Appendices}
            \input{8 - Hilbert complexes/main.tex}
            \input{9 - weak conservation proofs/main.tex}
\end{document}

            \documentclass[12pt, a4paper]{report}

\input{template/main.tex}

\title{\BA{Title in Progress...}}
\author{Boris Andrews}
\affil{Mathematical Institute, University of Oxford}
\date{\today}


\begin{document}
    \pagenumbering{gobble}
    \maketitle
    
    
    \begin{abstract}
        Magnetic confinement reactors---in particular tokamaks---offer one of the most promising options for achieving practical nuclear fusion, with the potential to provide virtually limitless, clean energy. The theoretical and numerical modeling of tokamak plasmas is simultaneously an essential component of effective reactor design, and a great research barrier. Tokamak operational conditions exhibit comparatively low Knudsen numbers. Kinetic effects, including kinetic waves and instabilities, Landau damping, bump-on-tail instabilities and more, are therefore highly influential in tokamak plasma dynamics. Purely fluid models are inherently incapable of capturing these effects, whereas the high dimensionality in purely kinetic models render them practically intractable for most relevant purposes.

        We consider a $\delta\!f$ decomposition model, with a macroscopic fluid background and microscopic kinetic correction, both fully coupled to each other. A similar manner of discretization is proposed to that used in the recent \texttt{STRUPHY} code \cite{Holderied_Possanner_Wang_2021, Holderied_2022, Li_et_al_2023} with a finite-element model for the background and a pseudo-particle/PiC model for the correction.

        The fluid background satisfies the full, non-linear, resistive, compressible, Hall MHD equations. \cite{Laakmann_Hu_Farrell_2022} introduces finite-element(-in-space) implicit timesteppers for the incompressible analogue to this system with structure-preserving (SP) properties in the ideal case, alongside parameter-robust preconditioners. We show that these timesteppers can derive from a finite-element-in-time (FET) (and finite-element-in-space) interpretation. The benefits of this reformulation are discussed, including the derivation of timesteppers that are higher order in time, and the quantifiable dissipative SP properties in the non-ideal, resistive case.
        
        We discuss possible options for extending this FET approach to timesteppers for the compressible case.

        The kinetic corrections satisfy linearized Boltzmann equations. Using a Lénard--Bernstein collision operator, these take Fokker--Planck-like forms \cite{Fokker_1914, Planck_1917} wherein pseudo-particles in the numerical model obey the neoclassical transport equations, with particle-independent Brownian drift terms. This offers a rigorous methodology for incorporating collisions into the particle transport model, without coupling the equations of motions for each particle.
        
        Works by Chen, Chacón et al. \cite{Chen_Chacón_Barnes_2011, Chacón_Chen_Barnes_2013, Chen_Chacón_2014, Chen_Chacón_2015} have developed structure-preserving particle pushers for neoclassical transport in the Vlasov equations, derived from Crank--Nicolson integrators. We show these too can can derive from a FET interpretation, similarly offering potential extensions to higher-order-in-time particle pushers. The FET formulation is used also to consider how the stochastic drift terms can be incorporated into the pushers. Stochastic gyrokinetic expansions are also discussed.

        Different options for the numerical implementation of these schemes are considered.

        Due to the efficacy of FET in the development of SP timesteppers for both the fluid and kinetic component, we hope this approach will prove effective in the future for developing SP timesteppers for the full hybrid model. We hope this will give us the opportunity to incorporate previously inaccessible kinetic effects into the highly effective, modern, finite-element MHD models.
    \end{abstract}
    
    
    \newpage
    \tableofcontents
    
    
    \newpage
    \pagenumbering{arabic}
    %\linenumbers\renewcommand\thelinenumber{\color{black!50}\arabic{linenumber}}
            \input{0 - introduction/main.tex}
        \part{Research}
            \input{1 - low-noise PiC models/main.tex}
            \input{2 - kinetic component/main.tex}
            \input{3 - fluid component/main.tex}
            \input{4 - numerical implementation/main.tex}
        \part{Project Overview}
            \input{5 - research plan/main.tex}
            \input{6 - summary/main.tex}
    
    
    %\section{}
    \newpage
    \pagenumbering{gobble}
        \printbibliography


    \newpage
    \pagenumbering{roman}
    \appendix
        \part{Appendices}
            \input{8 - Hilbert complexes/main.tex}
            \input{9 - weak conservation proofs/main.tex}
\end{document}

    
    
    %\section{}
    \newpage
    \pagenumbering{gobble}
        \printbibliography


    \newpage
    \pagenumbering{roman}
    \appendix
        \part{Appendices}
            \documentclass[12pt, a4paper]{report}

\input{template/main.tex}

\title{\BA{Title in Progress...}}
\author{Boris Andrews}
\affil{Mathematical Institute, University of Oxford}
\date{\today}


\begin{document}
    \pagenumbering{gobble}
    \maketitle
    
    
    \begin{abstract}
        Magnetic confinement reactors---in particular tokamaks---offer one of the most promising options for achieving practical nuclear fusion, with the potential to provide virtually limitless, clean energy. The theoretical and numerical modeling of tokamak plasmas is simultaneously an essential component of effective reactor design, and a great research barrier. Tokamak operational conditions exhibit comparatively low Knudsen numbers. Kinetic effects, including kinetic waves and instabilities, Landau damping, bump-on-tail instabilities and more, are therefore highly influential in tokamak plasma dynamics. Purely fluid models are inherently incapable of capturing these effects, whereas the high dimensionality in purely kinetic models render them practically intractable for most relevant purposes.

        We consider a $\delta\!f$ decomposition model, with a macroscopic fluid background and microscopic kinetic correction, both fully coupled to each other. A similar manner of discretization is proposed to that used in the recent \texttt{STRUPHY} code \cite{Holderied_Possanner_Wang_2021, Holderied_2022, Li_et_al_2023} with a finite-element model for the background and a pseudo-particle/PiC model for the correction.

        The fluid background satisfies the full, non-linear, resistive, compressible, Hall MHD equations. \cite{Laakmann_Hu_Farrell_2022} introduces finite-element(-in-space) implicit timesteppers for the incompressible analogue to this system with structure-preserving (SP) properties in the ideal case, alongside parameter-robust preconditioners. We show that these timesteppers can derive from a finite-element-in-time (FET) (and finite-element-in-space) interpretation. The benefits of this reformulation are discussed, including the derivation of timesteppers that are higher order in time, and the quantifiable dissipative SP properties in the non-ideal, resistive case.
        
        We discuss possible options for extending this FET approach to timesteppers for the compressible case.

        The kinetic corrections satisfy linearized Boltzmann equations. Using a Lénard--Bernstein collision operator, these take Fokker--Planck-like forms \cite{Fokker_1914, Planck_1917} wherein pseudo-particles in the numerical model obey the neoclassical transport equations, with particle-independent Brownian drift terms. This offers a rigorous methodology for incorporating collisions into the particle transport model, without coupling the equations of motions for each particle.
        
        Works by Chen, Chacón et al. \cite{Chen_Chacón_Barnes_2011, Chacón_Chen_Barnes_2013, Chen_Chacón_2014, Chen_Chacón_2015} have developed structure-preserving particle pushers for neoclassical transport in the Vlasov equations, derived from Crank--Nicolson integrators. We show these too can can derive from a FET interpretation, similarly offering potential extensions to higher-order-in-time particle pushers. The FET formulation is used also to consider how the stochastic drift terms can be incorporated into the pushers. Stochastic gyrokinetic expansions are also discussed.

        Different options for the numerical implementation of these schemes are considered.

        Due to the efficacy of FET in the development of SP timesteppers for both the fluid and kinetic component, we hope this approach will prove effective in the future for developing SP timesteppers for the full hybrid model. We hope this will give us the opportunity to incorporate previously inaccessible kinetic effects into the highly effective, modern, finite-element MHD models.
    \end{abstract}
    
    
    \newpage
    \tableofcontents
    
    
    \newpage
    \pagenumbering{arabic}
    %\linenumbers\renewcommand\thelinenumber{\color{black!50}\arabic{linenumber}}
            \input{0 - introduction/main.tex}
        \part{Research}
            \input{1 - low-noise PiC models/main.tex}
            \input{2 - kinetic component/main.tex}
            \input{3 - fluid component/main.tex}
            \input{4 - numerical implementation/main.tex}
        \part{Project Overview}
            \input{5 - research plan/main.tex}
            \input{6 - summary/main.tex}
    
    
    %\section{}
    \newpage
    \pagenumbering{gobble}
        \printbibliography


    \newpage
    \pagenumbering{roman}
    \appendix
        \part{Appendices}
            \input{8 - Hilbert complexes/main.tex}
            \input{9 - weak conservation proofs/main.tex}
\end{document}

            \documentclass[12pt, a4paper]{report}

\input{template/main.tex}

\title{\BA{Title in Progress...}}
\author{Boris Andrews}
\affil{Mathematical Institute, University of Oxford}
\date{\today}


\begin{document}
    \pagenumbering{gobble}
    \maketitle
    
    
    \begin{abstract}
        Magnetic confinement reactors---in particular tokamaks---offer one of the most promising options for achieving practical nuclear fusion, with the potential to provide virtually limitless, clean energy. The theoretical and numerical modeling of tokamak plasmas is simultaneously an essential component of effective reactor design, and a great research barrier. Tokamak operational conditions exhibit comparatively low Knudsen numbers. Kinetic effects, including kinetic waves and instabilities, Landau damping, bump-on-tail instabilities and more, are therefore highly influential in tokamak plasma dynamics. Purely fluid models are inherently incapable of capturing these effects, whereas the high dimensionality in purely kinetic models render them practically intractable for most relevant purposes.

        We consider a $\delta\!f$ decomposition model, with a macroscopic fluid background and microscopic kinetic correction, both fully coupled to each other. A similar manner of discretization is proposed to that used in the recent \texttt{STRUPHY} code \cite{Holderied_Possanner_Wang_2021, Holderied_2022, Li_et_al_2023} with a finite-element model for the background and a pseudo-particle/PiC model for the correction.

        The fluid background satisfies the full, non-linear, resistive, compressible, Hall MHD equations. \cite{Laakmann_Hu_Farrell_2022} introduces finite-element(-in-space) implicit timesteppers for the incompressible analogue to this system with structure-preserving (SP) properties in the ideal case, alongside parameter-robust preconditioners. We show that these timesteppers can derive from a finite-element-in-time (FET) (and finite-element-in-space) interpretation. The benefits of this reformulation are discussed, including the derivation of timesteppers that are higher order in time, and the quantifiable dissipative SP properties in the non-ideal, resistive case.
        
        We discuss possible options for extending this FET approach to timesteppers for the compressible case.

        The kinetic corrections satisfy linearized Boltzmann equations. Using a Lénard--Bernstein collision operator, these take Fokker--Planck-like forms \cite{Fokker_1914, Planck_1917} wherein pseudo-particles in the numerical model obey the neoclassical transport equations, with particle-independent Brownian drift terms. This offers a rigorous methodology for incorporating collisions into the particle transport model, without coupling the equations of motions for each particle.
        
        Works by Chen, Chacón et al. \cite{Chen_Chacón_Barnes_2011, Chacón_Chen_Barnes_2013, Chen_Chacón_2014, Chen_Chacón_2015} have developed structure-preserving particle pushers for neoclassical transport in the Vlasov equations, derived from Crank--Nicolson integrators. We show these too can can derive from a FET interpretation, similarly offering potential extensions to higher-order-in-time particle pushers. The FET formulation is used also to consider how the stochastic drift terms can be incorporated into the pushers. Stochastic gyrokinetic expansions are also discussed.

        Different options for the numerical implementation of these schemes are considered.

        Due to the efficacy of FET in the development of SP timesteppers for both the fluid and kinetic component, we hope this approach will prove effective in the future for developing SP timesteppers for the full hybrid model. We hope this will give us the opportunity to incorporate previously inaccessible kinetic effects into the highly effective, modern, finite-element MHD models.
    \end{abstract}
    
    
    \newpage
    \tableofcontents
    
    
    \newpage
    \pagenumbering{arabic}
    %\linenumbers\renewcommand\thelinenumber{\color{black!50}\arabic{linenumber}}
            \input{0 - introduction/main.tex}
        \part{Research}
            \input{1 - low-noise PiC models/main.tex}
            \input{2 - kinetic component/main.tex}
            \input{3 - fluid component/main.tex}
            \input{4 - numerical implementation/main.tex}
        \part{Project Overview}
            \input{5 - research plan/main.tex}
            \input{6 - summary/main.tex}
    
    
    %\section{}
    \newpage
    \pagenumbering{gobble}
        \printbibliography


    \newpage
    \pagenumbering{roman}
    \appendix
        \part{Appendices}
            \input{8 - Hilbert complexes/main.tex}
            \input{9 - weak conservation proofs/main.tex}
\end{document}

\end{document}

        \part{Research}
            \documentclass[12pt, a4paper]{report}

\documentclass[12pt, a4paper]{report}

\input{template/main.tex}

\title{\BA{Title in Progress...}}
\author{Boris Andrews}
\affil{Mathematical Institute, University of Oxford}
\date{\today}


\begin{document}
    \pagenumbering{gobble}
    \maketitle
    
    
    \begin{abstract}
        Magnetic confinement reactors---in particular tokamaks---offer one of the most promising options for achieving practical nuclear fusion, with the potential to provide virtually limitless, clean energy. The theoretical and numerical modeling of tokamak plasmas is simultaneously an essential component of effective reactor design, and a great research barrier. Tokamak operational conditions exhibit comparatively low Knudsen numbers. Kinetic effects, including kinetic waves and instabilities, Landau damping, bump-on-tail instabilities and more, are therefore highly influential in tokamak plasma dynamics. Purely fluid models are inherently incapable of capturing these effects, whereas the high dimensionality in purely kinetic models render them practically intractable for most relevant purposes.

        We consider a $\delta\!f$ decomposition model, with a macroscopic fluid background and microscopic kinetic correction, both fully coupled to each other. A similar manner of discretization is proposed to that used in the recent \texttt{STRUPHY} code \cite{Holderied_Possanner_Wang_2021, Holderied_2022, Li_et_al_2023} with a finite-element model for the background and a pseudo-particle/PiC model for the correction.

        The fluid background satisfies the full, non-linear, resistive, compressible, Hall MHD equations. \cite{Laakmann_Hu_Farrell_2022} introduces finite-element(-in-space) implicit timesteppers for the incompressible analogue to this system with structure-preserving (SP) properties in the ideal case, alongside parameter-robust preconditioners. We show that these timesteppers can derive from a finite-element-in-time (FET) (and finite-element-in-space) interpretation. The benefits of this reformulation are discussed, including the derivation of timesteppers that are higher order in time, and the quantifiable dissipative SP properties in the non-ideal, resistive case.
        
        We discuss possible options for extending this FET approach to timesteppers for the compressible case.

        The kinetic corrections satisfy linearized Boltzmann equations. Using a Lénard--Bernstein collision operator, these take Fokker--Planck-like forms \cite{Fokker_1914, Planck_1917} wherein pseudo-particles in the numerical model obey the neoclassical transport equations, with particle-independent Brownian drift terms. This offers a rigorous methodology for incorporating collisions into the particle transport model, without coupling the equations of motions for each particle.
        
        Works by Chen, Chacón et al. \cite{Chen_Chacón_Barnes_2011, Chacón_Chen_Barnes_2013, Chen_Chacón_2014, Chen_Chacón_2015} have developed structure-preserving particle pushers for neoclassical transport in the Vlasov equations, derived from Crank--Nicolson integrators. We show these too can can derive from a FET interpretation, similarly offering potential extensions to higher-order-in-time particle pushers. The FET formulation is used also to consider how the stochastic drift terms can be incorporated into the pushers. Stochastic gyrokinetic expansions are also discussed.

        Different options for the numerical implementation of these schemes are considered.

        Due to the efficacy of FET in the development of SP timesteppers for both the fluid and kinetic component, we hope this approach will prove effective in the future for developing SP timesteppers for the full hybrid model. We hope this will give us the opportunity to incorporate previously inaccessible kinetic effects into the highly effective, modern, finite-element MHD models.
    \end{abstract}
    
    
    \newpage
    \tableofcontents
    
    
    \newpage
    \pagenumbering{arabic}
    %\linenumbers\renewcommand\thelinenumber{\color{black!50}\arabic{linenumber}}
            \input{0 - introduction/main.tex}
        \part{Research}
            \input{1 - low-noise PiC models/main.tex}
            \input{2 - kinetic component/main.tex}
            \input{3 - fluid component/main.tex}
            \input{4 - numerical implementation/main.tex}
        \part{Project Overview}
            \input{5 - research plan/main.tex}
            \input{6 - summary/main.tex}
    
    
    %\section{}
    \newpage
    \pagenumbering{gobble}
        \printbibliography


    \newpage
    \pagenumbering{roman}
    \appendix
        \part{Appendices}
            \input{8 - Hilbert complexes/main.tex}
            \input{9 - weak conservation proofs/main.tex}
\end{document}


\title{\BA{Title in Progress...}}
\author{Boris Andrews}
\affil{Mathematical Institute, University of Oxford}
\date{\today}


\begin{document}
    \pagenumbering{gobble}
    \maketitle
    
    
    \begin{abstract}
        Magnetic confinement reactors---in particular tokamaks---offer one of the most promising options for achieving practical nuclear fusion, with the potential to provide virtually limitless, clean energy. The theoretical and numerical modeling of tokamak plasmas is simultaneously an essential component of effective reactor design, and a great research barrier. Tokamak operational conditions exhibit comparatively low Knudsen numbers. Kinetic effects, including kinetic waves and instabilities, Landau damping, bump-on-tail instabilities and more, are therefore highly influential in tokamak plasma dynamics. Purely fluid models are inherently incapable of capturing these effects, whereas the high dimensionality in purely kinetic models render them practically intractable for most relevant purposes.

        We consider a $\delta\!f$ decomposition model, with a macroscopic fluid background and microscopic kinetic correction, both fully coupled to each other. A similar manner of discretization is proposed to that used in the recent \texttt{STRUPHY} code \cite{Holderied_Possanner_Wang_2021, Holderied_2022, Li_et_al_2023} with a finite-element model for the background and a pseudo-particle/PiC model for the correction.

        The fluid background satisfies the full, non-linear, resistive, compressible, Hall MHD equations. \cite{Laakmann_Hu_Farrell_2022} introduces finite-element(-in-space) implicit timesteppers for the incompressible analogue to this system with structure-preserving (SP) properties in the ideal case, alongside parameter-robust preconditioners. We show that these timesteppers can derive from a finite-element-in-time (FET) (and finite-element-in-space) interpretation. The benefits of this reformulation are discussed, including the derivation of timesteppers that are higher order in time, and the quantifiable dissipative SP properties in the non-ideal, resistive case.
        
        We discuss possible options for extending this FET approach to timesteppers for the compressible case.

        The kinetic corrections satisfy linearized Boltzmann equations. Using a Lénard--Bernstein collision operator, these take Fokker--Planck-like forms \cite{Fokker_1914, Planck_1917} wherein pseudo-particles in the numerical model obey the neoclassical transport equations, with particle-independent Brownian drift terms. This offers a rigorous methodology for incorporating collisions into the particle transport model, without coupling the equations of motions for each particle.
        
        Works by Chen, Chacón et al. \cite{Chen_Chacón_Barnes_2011, Chacón_Chen_Barnes_2013, Chen_Chacón_2014, Chen_Chacón_2015} have developed structure-preserving particle pushers for neoclassical transport in the Vlasov equations, derived from Crank--Nicolson integrators. We show these too can can derive from a FET interpretation, similarly offering potential extensions to higher-order-in-time particle pushers. The FET formulation is used also to consider how the stochastic drift terms can be incorporated into the pushers. Stochastic gyrokinetic expansions are also discussed.

        Different options for the numerical implementation of these schemes are considered.

        Due to the efficacy of FET in the development of SP timesteppers for both the fluid and kinetic component, we hope this approach will prove effective in the future for developing SP timesteppers for the full hybrid model. We hope this will give us the opportunity to incorporate previously inaccessible kinetic effects into the highly effective, modern, finite-element MHD models.
    \end{abstract}
    
    
    \newpage
    \tableofcontents
    
    
    \newpage
    \pagenumbering{arabic}
    %\linenumbers\renewcommand\thelinenumber{\color{black!50}\arabic{linenumber}}
            \documentclass[12pt, a4paper]{report}

\input{template/main.tex}

\title{\BA{Title in Progress...}}
\author{Boris Andrews}
\affil{Mathematical Institute, University of Oxford}
\date{\today}


\begin{document}
    \pagenumbering{gobble}
    \maketitle
    
    
    \begin{abstract}
        Magnetic confinement reactors---in particular tokamaks---offer one of the most promising options for achieving practical nuclear fusion, with the potential to provide virtually limitless, clean energy. The theoretical and numerical modeling of tokamak plasmas is simultaneously an essential component of effective reactor design, and a great research barrier. Tokamak operational conditions exhibit comparatively low Knudsen numbers. Kinetic effects, including kinetic waves and instabilities, Landau damping, bump-on-tail instabilities and more, are therefore highly influential in tokamak plasma dynamics. Purely fluid models are inherently incapable of capturing these effects, whereas the high dimensionality in purely kinetic models render them practically intractable for most relevant purposes.

        We consider a $\delta\!f$ decomposition model, with a macroscopic fluid background and microscopic kinetic correction, both fully coupled to each other. A similar manner of discretization is proposed to that used in the recent \texttt{STRUPHY} code \cite{Holderied_Possanner_Wang_2021, Holderied_2022, Li_et_al_2023} with a finite-element model for the background and a pseudo-particle/PiC model for the correction.

        The fluid background satisfies the full, non-linear, resistive, compressible, Hall MHD equations. \cite{Laakmann_Hu_Farrell_2022} introduces finite-element(-in-space) implicit timesteppers for the incompressible analogue to this system with structure-preserving (SP) properties in the ideal case, alongside parameter-robust preconditioners. We show that these timesteppers can derive from a finite-element-in-time (FET) (and finite-element-in-space) interpretation. The benefits of this reformulation are discussed, including the derivation of timesteppers that are higher order in time, and the quantifiable dissipative SP properties in the non-ideal, resistive case.
        
        We discuss possible options for extending this FET approach to timesteppers for the compressible case.

        The kinetic corrections satisfy linearized Boltzmann equations. Using a Lénard--Bernstein collision operator, these take Fokker--Planck-like forms \cite{Fokker_1914, Planck_1917} wherein pseudo-particles in the numerical model obey the neoclassical transport equations, with particle-independent Brownian drift terms. This offers a rigorous methodology for incorporating collisions into the particle transport model, without coupling the equations of motions for each particle.
        
        Works by Chen, Chacón et al. \cite{Chen_Chacón_Barnes_2011, Chacón_Chen_Barnes_2013, Chen_Chacón_2014, Chen_Chacón_2015} have developed structure-preserving particle pushers for neoclassical transport in the Vlasov equations, derived from Crank--Nicolson integrators. We show these too can can derive from a FET interpretation, similarly offering potential extensions to higher-order-in-time particle pushers. The FET formulation is used also to consider how the stochastic drift terms can be incorporated into the pushers. Stochastic gyrokinetic expansions are also discussed.

        Different options for the numerical implementation of these schemes are considered.

        Due to the efficacy of FET in the development of SP timesteppers for both the fluid and kinetic component, we hope this approach will prove effective in the future for developing SP timesteppers for the full hybrid model. We hope this will give us the opportunity to incorporate previously inaccessible kinetic effects into the highly effective, modern, finite-element MHD models.
    \end{abstract}
    
    
    \newpage
    \tableofcontents
    
    
    \newpage
    \pagenumbering{arabic}
    %\linenumbers\renewcommand\thelinenumber{\color{black!50}\arabic{linenumber}}
            \input{0 - introduction/main.tex}
        \part{Research}
            \input{1 - low-noise PiC models/main.tex}
            \input{2 - kinetic component/main.tex}
            \input{3 - fluid component/main.tex}
            \input{4 - numerical implementation/main.tex}
        \part{Project Overview}
            \input{5 - research plan/main.tex}
            \input{6 - summary/main.tex}
    
    
    %\section{}
    \newpage
    \pagenumbering{gobble}
        \printbibliography


    \newpage
    \pagenumbering{roman}
    \appendix
        \part{Appendices}
            \input{8 - Hilbert complexes/main.tex}
            \input{9 - weak conservation proofs/main.tex}
\end{document}

        \part{Research}
            \documentclass[12pt, a4paper]{report}

\input{template/main.tex}

\title{\BA{Title in Progress...}}
\author{Boris Andrews}
\affil{Mathematical Institute, University of Oxford}
\date{\today}


\begin{document}
    \pagenumbering{gobble}
    \maketitle
    
    
    \begin{abstract}
        Magnetic confinement reactors---in particular tokamaks---offer one of the most promising options for achieving practical nuclear fusion, with the potential to provide virtually limitless, clean energy. The theoretical and numerical modeling of tokamak plasmas is simultaneously an essential component of effective reactor design, and a great research barrier. Tokamak operational conditions exhibit comparatively low Knudsen numbers. Kinetic effects, including kinetic waves and instabilities, Landau damping, bump-on-tail instabilities and more, are therefore highly influential in tokamak plasma dynamics. Purely fluid models are inherently incapable of capturing these effects, whereas the high dimensionality in purely kinetic models render them practically intractable for most relevant purposes.

        We consider a $\delta\!f$ decomposition model, with a macroscopic fluid background and microscopic kinetic correction, both fully coupled to each other. A similar manner of discretization is proposed to that used in the recent \texttt{STRUPHY} code \cite{Holderied_Possanner_Wang_2021, Holderied_2022, Li_et_al_2023} with a finite-element model for the background and a pseudo-particle/PiC model for the correction.

        The fluid background satisfies the full, non-linear, resistive, compressible, Hall MHD equations. \cite{Laakmann_Hu_Farrell_2022} introduces finite-element(-in-space) implicit timesteppers for the incompressible analogue to this system with structure-preserving (SP) properties in the ideal case, alongside parameter-robust preconditioners. We show that these timesteppers can derive from a finite-element-in-time (FET) (and finite-element-in-space) interpretation. The benefits of this reformulation are discussed, including the derivation of timesteppers that are higher order in time, and the quantifiable dissipative SP properties in the non-ideal, resistive case.
        
        We discuss possible options for extending this FET approach to timesteppers for the compressible case.

        The kinetic corrections satisfy linearized Boltzmann equations. Using a Lénard--Bernstein collision operator, these take Fokker--Planck-like forms \cite{Fokker_1914, Planck_1917} wherein pseudo-particles in the numerical model obey the neoclassical transport equations, with particle-independent Brownian drift terms. This offers a rigorous methodology for incorporating collisions into the particle transport model, without coupling the equations of motions for each particle.
        
        Works by Chen, Chacón et al. \cite{Chen_Chacón_Barnes_2011, Chacón_Chen_Barnes_2013, Chen_Chacón_2014, Chen_Chacón_2015} have developed structure-preserving particle pushers for neoclassical transport in the Vlasov equations, derived from Crank--Nicolson integrators. We show these too can can derive from a FET interpretation, similarly offering potential extensions to higher-order-in-time particle pushers. The FET formulation is used also to consider how the stochastic drift terms can be incorporated into the pushers. Stochastic gyrokinetic expansions are also discussed.

        Different options for the numerical implementation of these schemes are considered.

        Due to the efficacy of FET in the development of SP timesteppers for both the fluid and kinetic component, we hope this approach will prove effective in the future for developing SP timesteppers for the full hybrid model. We hope this will give us the opportunity to incorporate previously inaccessible kinetic effects into the highly effective, modern, finite-element MHD models.
    \end{abstract}
    
    
    \newpage
    \tableofcontents
    
    
    \newpage
    \pagenumbering{arabic}
    %\linenumbers\renewcommand\thelinenumber{\color{black!50}\arabic{linenumber}}
            \input{0 - introduction/main.tex}
        \part{Research}
            \input{1 - low-noise PiC models/main.tex}
            \input{2 - kinetic component/main.tex}
            \input{3 - fluid component/main.tex}
            \input{4 - numerical implementation/main.tex}
        \part{Project Overview}
            \input{5 - research plan/main.tex}
            \input{6 - summary/main.tex}
    
    
    %\section{}
    \newpage
    \pagenumbering{gobble}
        \printbibliography


    \newpage
    \pagenumbering{roman}
    \appendix
        \part{Appendices}
            \input{8 - Hilbert complexes/main.tex}
            \input{9 - weak conservation proofs/main.tex}
\end{document}

            \documentclass[12pt, a4paper]{report}

\input{template/main.tex}

\title{\BA{Title in Progress...}}
\author{Boris Andrews}
\affil{Mathematical Institute, University of Oxford}
\date{\today}


\begin{document}
    \pagenumbering{gobble}
    \maketitle
    
    
    \begin{abstract}
        Magnetic confinement reactors---in particular tokamaks---offer one of the most promising options for achieving practical nuclear fusion, with the potential to provide virtually limitless, clean energy. The theoretical and numerical modeling of tokamak plasmas is simultaneously an essential component of effective reactor design, and a great research barrier. Tokamak operational conditions exhibit comparatively low Knudsen numbers. Kinetic effects, including kinetic waves and instabilities, Landau damping, bump-on-tail instabilities and more, are therefore highly influential in tokamak plasma dynamics. Purely fluid models are inherently incapable of capturing these effects, whereas the high dimensionality in purely kinetic models render them practically intractable for most relevant purposes.

        We consider a $\delta\!f$ decomposition model, with a macroscopic fluid background and microscopic kinetic correction, both fully coupled to each other. A similar manner of discretization is proposed to that used in the recent \texttt{STRUPHY} code \cite{Holderied_Possanner_Wang_2021, Holderied_2022, Li_et_al_2023} with a finite-element model for the background and a pseudo-particle/PiC model for the correction.

        The fluid background satisfies the full, non-linear, resistive, compressible, Hall MHD equations. \cite{Laakmann_Hu_Farrell_2022} introduces finite-element(-in-space) implicit timesteppers for the incompressible analogue to this system with structure-preserving (SP) properties in the ideal case, alongside parameter-robust preconditioners. We show that these timesteppers can derive from a finite-element-in-time (FET) (and finite-element-in-space) interpretation. The benefits of this reformulation are discussed, including the derivation of timesteppers that are higher order in time, and the quantifiable dissipative SP properties in the non-ideal, resistive case.
        
        We discuss possible options for extending this FET approach to timesteppers for the compressible case.

        The kinetic corrections satisfy linearized Boltzmann equations. Using a Lénard--Bernstein collision operator, these take Fokker--Planck-like forms \cite{Fokker_1914, Planck_1917} wherein pseudo-particles in the numerical model obey the neoclassical transport equations, with particle-independent Brownian drift terms. This offers a rigorous methodology for incorporating collisions into the particle transport model, without coupling the equations of motions for each particle.
        
        Works by Chen, Chacón et al. \cite{Chen_Chacón_Barnes_2011, Chacón_Chen_Barnes_2013, Chen_Chacón_2014, Chen_Chacón_2015} have developed structure-preserving particle pushers for neoclassical transport in the Vlasov equations, derived from Crank--Nicolson integrators. We show these too can can derive from a FET interpretation, similarly offering potential extensions to higher-order-in-time particle pushers. The FET formulation is used also to consider how the stochastic drift terms can be incorporated into the pushers. Stochastic gyrokinetic expansions are also discussed.

        Different options for the numerical implementation of these schemes are considered.

        Due to the efficacy of FET in the development of SP timesteppers for both the fluid and kinetic component, we hope this approach will prove effective in the future for developing SP timesteppers for the full hybrid model. We hope this will give us the opportunity to incorporate previously inaccessible kinetic effects into the highly effective, modern, finite-element MHD models.
    \end{abstract}
    
    
    \newpage
    \tableofcontents
    
    
    \newpage
    \pagenumbering{arabic}
    %\linenumbers\renewcommand\thelinenumber{\color{black!50}\arabic{linenumber}}
            \input{0 - introduction/main.tex}
        \part{Research}
            \input{1 - low-noise PiC models/main.tex}
            \input{2 - kinetic component/main.tex}
            \input{3 - fluid component/main.tex}
            \input{4 - numerical implementation/main.tex}
        \part{Project Overview}
            \input{5 - research plan/main.tex}
            \input{6 - summary/main.tex}
    
    
    %\section{}
    \newpage
    \pagenumbering{gobble}
        \printbibliography


    \newpage
    \pagenumbering{roman}
    \appendix
        \part{Appendices}
            \input{8 - Hilbert complexes/main.tex}
            \input{9 - weak conservation proofs/main.tex}
\end{document}

            \documentclass[12pt, a4paper]{report}

\input{template/main.tex}

\title{\BA{Title in Progress...}}
\author{Boris Andrews}
\affil{Mathematical Institute, University of Oxford}
\date{\today}


\begin{document}
    \pagenumbering{gobble}
    \maketitle
    
    
    \begin{abstract}
        Magnetic confinement reactors---in particular tokamaks---offer one of the most promising options for achieving practical nuclear fusion, with the potential to provide virtually limitless, clean energy. The theoretical and numerical modeling of tokamak plasmas is simultaneously an essential component of effective reactor design, and a great research barrier. Tokamak operational conditions exhibit comparatively low Knudsen numbers. Kinetic effects, including kinetic waves and instabilities, Landau damping, bump-on-tail instabilities and more, are therefore highly influential in tokamak plasma dynamics. Purely fluid models are inherently incapable of capturing these effects, whereas the high dimensionality in purely kinetic models render them practically intractable for most relevant purposes.

        We consider a $\delta\!f$ decomposition model, with a macroscopic fluid background and microscopic kinetic correction, both fully coupled to each other. A similar manner of discretization is proposed to that used in the recent \texttt{STRUPHY} code \cite{Holderied_Possanner_Wang_2021, Holderied_2022, Li_et_al_2023} with a finite-element model for the background and a pseudo-particle/PiC model for the correction.

        The fluid background satisfies the full, non-linear, resistive, compressible, Hall MHD equations. \cite{Laakmann_Hu_Farrell_2022} introduces finite-element(-in-space) implicit timesteppers for the incompressible analogue to this system with structure-preserving (SP) properties in the ideal case, alongside parameter-robust preconditioners. We show that these timesteppers can derive from a finite-element-in-time (FET) (and finite-element-in-space) interpretation. The benefits of this reformulation are discussed, including the derivation of timesteppers that are higher order in time, and the quantifiable dissipative SP properties in the non-ideal, resistive case.
        
        We discuss possible options for extending this FET approach to timesteppers for the compressible case.

        The kinetic corrections satisfy linearized Boltzmann equations. Using a Lénard--Bernstein collision operator, these take Fokker--Planck-like forms \cite{Fokker_1914, Planck_1917} wherein pseudo-particles in the numerical model obey the neoclassical transport equations, with particle-independent Brownian drift terms. This offers a rigorous methodology for incorporating collisions into the particle transport model, without coupling the equations of motions for each particle.
        
        Works by Chen, Chacón et al. \cite{Chen_Chacón_Barnes_2011, Chacón_Chen_Barnes_2013, Chen_Chacón_2014, Chen_Chacón_2015} have developed structure-preserving particle pushers for neoclassical transport in the Vlasov equations, derived from Crank--Nicolson integrators. We show these too can can derive from a FET interpretation, similarly offering potential extensions to higher-order-in-time particle pushers. The FET formulation is used also to consider how the stochastic drift terms can be incorporated into the pushers. Stochastic gyrokinetic expansions are also discussed.

        Different options for the numerical implementation of these schemes are considered.

        Due to the efficacy of FET in the development of SP timesteppers for both the fluid and kinetic component, we hope this approach will prove effective in the future for developing SP timesteppers for the full hybrid model. We hope this will give us the opportunity to incorporate previously inaccessible kinetic effects into the highly effective, modern, finite-element MHD models.
    \end{abstract}
    
    
    \newpage
    \tableofcontents
    
    
    \newpage
    \pagenumbering{arabic}
    %\linenumbers\renewcommand\thelinenumber{\color{black!50}\arabic{linenumber}}
            \input{0 - introduction/main.tex}
        \part{Research}
            \input{1 - low-noise PiC models/main.tex}
            \input{2 - kinetic component/main.tex}
            \input{3 - fluid component/main.tex}
            \input{4 - numerical implementation/main.tex}
        \part{Project Overview}
            \input{5 - research plan/main.tex}
            \input{6 - summary/main.tex}
    
    
    %\section{}
    \newpage
    \pagenumbering{gobble}
        \printbibliography


    \newpage
    \pagenumbering{roman}
    \appendix
        \part{Appendices}
            \input{8 - Hilbert complexes/main.tex}
            \input{9 - weak conservation proofs/main.tex}
\end{document}

            \documentclass[12pt, a4paper]{report}

\input{template/main.tex}

\title{\BA{Title in Progress...}}
\author{Boris Andrews}
\affil{Mathematical Institute, University of Oxford}
\date{\today}


\begin{document}
    \pagenumbering{gobble}
    \maketitle
    
    
    \begin{abstract}
        Magnetic confinement reactors---in particular tokamaks---offer one of the most promising options for achieving practical nuclear fusion, with the potential to provide virtually limitless, clean energy. The theoretical and numerical modeling of tokamak plasmas is simultaneously an essential component of effective reactor design, and a great research barrier. Tokamak operational conditions exhibit comparatively low Knudsen numbers. Kinetic effects, including kinetic waves and instabilities, Landau damping, bump-on-tail instabilities and more, are therefore highly influential in tokamak plasma dynamics. Purely fluid models are inherently incapable of capturing these effects, whereas the high dimensionality in purely kinetic models render them practically intractable for most relevant purposes.

        We consider a $\delta\!f$ decomposition model, with a macroscopic fluid background and microscopic kinetic correction, both fully coupled to each other. A similar manner of discretization is proposed to that used in the recent \texttt{STRUPHY} code \cite{Holderied_Possanner_Wang_2021, Holderied_2022, Li_et_al_2023} with a finite-element model for the background and a pseudo-particle/PiC model for the correction.

        The fluid background satisfies the full, non-linear, resistive, compressible, Hall MHD equations. \cite{Laakmann_Hu_Farrell_2022} introduces finite-element(-in-space) implicit timesteppers for the incompressible analogue to this system with structure-preserving (SP) properties in the ideal case, alongside parameter-robust preconditioners. We show that these timesteppers can derive from a finite-element-in-time (FET) (and finite-element-in-space) interpretation. The benefits of this reformulation are discussed, including the derivation of timesteppers that are higher order in time, and the quantifiable dissipative SP properties in the non-ideal, resistive case.
        
        We discuss possible options for extending this FET approach to timesteppers for the compressible case.

        The kinetic corrections satisfy linearized Boltzmann equations. Using a Lénard--Bernstein collision operator, these take Fokker--Planck-like forms \cite{Fokker_1914, Planck_1917} wherein pseudo-particles in the numerical model obey the neoclassical transport equations, with particle-independent Brownian drift terms. This offers a rigorous methodology for incorporating collisions into the particle transport model, without coupling the equations of motions for each particle.
        
        Works by Chen, Chacón et al. \cite{Chen_Chacón_Barnes_2011, Chacón_Chen_Barnes_2013, Chen_Chacón_2014, Chen_Chacón_2015} have developed structure-preserving particle pushers for neoclassical transport in the Vlasov equations, derived from Crank--Nicolson integrators. We show these too can can derive from a FET interpretation, similarly offering potential extensions to higher-order-in-time particle pushers. The FET formulation is used also to consider how the stochastic drift terms can be incorporated into the pushers. Stochastic gyrokinetic expansions are also discussed.

        Different options for the numerical implementation of these schemes are considered.

        Due to the efficacy of FET in the development of SP timesteppers for both the fluid and kinetic component, we hope this approach will prove effective in the future for developing SP timesteppers for the full hybrid model. We hope this will give us the opportunity to incorporate previously inaccessible kinetic effects into the highly effective, modern, finite-element MHD models.
    \end{abstract}
    
    
    \newpage
    \tableofcontents
    
    
    \newpage
    \pagenumbering{arabic}
    %\linenumbers\renewcommand\thelinenumber{\color{black!50}\arabic{linenumber}}
            \input{0 - introduction/main.tex}
        \part{Research}
            \input{1 - low-noise PiC models/main.tex}
            \input{2 - kinetic component/main.tex}
            \input{3 - fluid component/main.tex}
            \input{4 - numerical implementation/main.tex}
        \part{Project Overview}
            \input{5 - research plan/main.tex}
            \input{6 - summary/main.tex}
    
    
    %\section{}
    \newpage
    \pagenumbering{gobble}
        \printbibliography


    \newpage
    \pagenumbering{roman}
    \appendix
        \part{Appendices}
            \input{8 - Hilbert complexes/main.tex}
            \input{9 - weak conservation proofs/main.tex}
\end{document}

        \part{Project Overview}
            \documentclass[12pt, a4paper]{report}

\input{template/main.tex}

\title{\BA{Title in Progress...}}
\author{Boris Andrews}
\affil{Mathematical Institute, University of Oxford}
\date{\today}


\begin{document}
    \pagenumbering{gobble}
    \maketitle
    
    
    \begin{abstract}
        Magnetic confinement reactors---in particular tokamaks---offer one of the most promising options for achieving practical nuclear fusion, with the potential to provide virtually limitless, clean energy. The theoretical and numerical modeling of tokamak plasmas is simultaneously an essential component of effective reactor design, and a great research barrier. Tokamak operational conditions exhibit comparatively low Knudsen numbers. Kinetic effects, including kinetic waves and instabilities, Landau damping, bump-on-tail instabilities and more, are therefore highly influential in tokamak plasma dynamics. Purely fluid models are inherently incapable of capturing these effects, whereas the high dimensionality in purely kinetic models render them practically intractable for most relevant purposes.

        We consider a $\delta\!f$ decomposition model, with a macroscopic fluid background and microscopic kinetic correction, both fully coupled to each other. A similar manner of discretization is proposed to that used in the recent \texttt{STRUPHY} code \cite{Holderied_Possanner_Wang_2021, Holderied_2022, Li_et_al_2023} with a finite-element model for the background and a pseudo-particle/PiC model for the correction.

        The fluid background satisfies the full, non-linear, resistive, compressible, Hall MHD equations. \cite{Laakmann_Hu_Farrell_2022} introduces finite-element(-in-space) implicit timesteppers for the incompressible analogue to this system with structure-preserving (SP) properties in the ideal case, alongside parameter-robust preconditioners. We show that these timesteppers can derive from a finite-element-in-time (FET) (and finite-element-in-space) interpretation. The benefits of this reformulation are discussed, including the derivation of timesteppers that are higher order in time, and the quantifiable dissipative SP properties in the non-ideal, resistive case.
        
        We discuss possible options for extending this FET approach to timesteppers for the compressible case.

        The kinetic corrections satisfy linearized Boltzmann equations. Using a Lénard--Bernstein collision operator, these take Fokker--Planck-like forms \cite{Fokker_1914, Planck_1917} wherein pseudo-particles in the numerical model obey the neoclassical transport equations, with particle-independent Brownian drift terms. This offers a rigorous methodology for incorporating collisions into the particle transport model, without coupling the equations of motions for each particle.
        
        Works by Chen, Chacón et al. \cite{Chen_Chacón_Barnes_2011, Chacón_Chen_Barnes_2013, Chen_Chacón_2014, Chen_Chacón_2015} have developed structure-preserving particle pushers for neoclassical transport in the Vlasov equations, derived from Crank--Nicolson integrators. We show these too can can derive from a FET interpretation, similarly offering potential extensions to higher-order-in-time particle pushers. The FET formulation is used also to consider how the stochastic drift terms can be incorporated into the pushers. Stochastic gyrokinetic expansions are also discussed.

        Different options for the numerical implementation of these schemes are considered.

        Due to the efficacy of FET in the development of SP timesteppers for both the fluid and kinetic component, we hope this approach will prove effective in the future for developing SP timesteppers for the full hybrid model. We hope this will give us the opportunity to incorporate previously inaccessible kinetic effects into the highly effective, modern, finite-element MHD models.
    \end{abstract}
    
    
    \newpage
    \tableofcontents
    
    
    \newpage
    \pagenumbering{arabic}
    %\linenumbers\renewcommand\thelinenumber{\color{black!50}\arabic{linenumber}}
            \input{0 - introduction/main.tex}
        \part{Research}
            \input{1 - low-noise PiC models/main.tex}
            \input{2 - kinetic component/main.tex}
            \input{3 - fluid component/main.tex}
            \input{4 - numerical implementation/main.tex}
        \part{Project Overview}
            \input{5 - research plan/main.tex}
            \input{6 - summary/main.tex}
    
    
    %\section{}
    \newpage
    \pagenumbering{gobble}
        \printbibliography


    \newpage
    \pagenumbering{roman}
    \appendix
        \part{Appendices}
            \input{8 - Hilbert complexes/main.tex}
            \input{9 - weak conservation proofs/main.tex}
\end{document}

            \documentclass[12pt, a4paper]{report}

\input{template/main.tex}

\title{\BA{Title in Progress...}}
\author{Boris Andrews}
\affil{Mathematical Institute, University of Oxford}
\date{\today}


\begin{document}
    \pagenumbering{gobble}
    \maketitle
    
    
    \begin{abstract}
        Magnetic confinement reactors---in particular tokamaks---offer one of the most promising options for achieving practical nuclear fusion, with the potential to provide virtually limitless, clean energy. The theoretical and numerical modeling of tokamak plasmas is simultaneously an essential component of effective reactor design, and a great research barrier. Tokamak operational conditions exhibit comparatively low Knudsen numbers. Kinetic effects, including kinetic waves and instabilities, Landau damping, bump-on-tail instabilities and more, are therefore highly influential in tokamak plasma dynamics. Purely fluid models are inherently incapable of capturing these effects, whereas the high dimensionality in purely kinetic models render them practically intractable for most relevant purposes.

        We consider a $\delta\!f$ decomposition model, with a macroscopic fluid background and microscopic kinetic correction, both fully coupled to each other. A similar manner of discretization is proposed to that used in the recent \texttt{STRUPHY} code \cite{Holderied_Possanner_Wang_2021, Holderied_2022, Li_et_al_2023} with a finite-element model for the background and a pseudo-particle/PiC model for the correction.

        The fluid background satisfies the full, non-linear, resistive, compressible, Hall MHD equations. \cite{Laakmann_Hu_Farrell_2022} introduces finite-element(-in-space) implicit timesteppers for the incompressible analogue to this system with structure-preserving (SP) properties in the ideal case, alongside parameter-robust preconditioners. We show that these timesteppers can derive from a finite-element-in-time (FET) (and finite-element-in-space) interpretation. The benefits of this reformulation are discussed, including the derivation of timesteppers that are higher order in time, and the quantifiable dissipative SP properties in the non-ideal, resistive case.
        
        We discuss possible options for extending this FET approach to timesteppers for the compressible case.

        The kinetic corrections satisfy linearized Boltzmann equations. Using a Lénard--Bernstein collision operator, these take Fokker--Planck-like forms \cite{Fokker_1914, Planck_1917} wherein pseudo-particles in the numerical model obey the neoclassical transport equations, with particle-independent Brownian drift terms. This offers a rigorous methodology for incorporating collisions into the particle transport model, without coupling the equations of motions for each particle.
        
        Works by Chen, Chacón et al. \cite{Chen_Chacón_Barnes_2011, Chacón_Chen_Barnes_2013, Chen_Chacón_2014, Chen_Chacón_2015} have developed structure-preserving particle pushers for neoclassical transport in the Vlasov equations, derived from Crank--Nicolson integrators. We show these too can can derive from a FET interpretation, similarly offering potential extensions to higher-order-in-time particle pushers. The FET formulation is used also to consider how the stochastic drift terms can be incorporated into the pushers. Stochastic gyrokinetic expansions are also discussed.

        Different options for the numerical implementation of these schemes are considered.

        Due to the efficacy of FET in the development of SP timesteppers for both the fluid and kinetic component, we hope this approach will prove effective in the future for developing SP timesteppers for the full hybrid model. We hope this will give us the opportunity to incorporate previously inaccessible kinetic effects into the highly effective, modern, finite-element MHD models.
    \end{abstract}
    
    
    \newpage
    \tableofcontents
    
    
    \newpage
    \pagenumbering{arabic}
    %\linenumbers\renewcommand\thelinenumber{\color{black!50}\arabic{linenumber}}
            \input{0 - introduction/main.tex}
        \part{Research}
            \input{1 - low-noise PiC models/main.tex}
            \input{2 - kinetic component/main.tex}
            \input{3 - fluid component/main.tex}
            \input{4 - numerical implementation/main.tex}
        \part{Project Overview}
            \input{5 - research plan/main.tex}
            \input{6 - summary/main.tex}
    
    
    %\section{}
    \newpage
    \pagenumbering{gobble}
        \printbibliography


    \newpage
    \pagenumbering{roman}
    \appendix
        \part{Appendices}
            \input{8 - Hilbert complexes/main.tex}
            \input{9 - weak conservation proofs/main.tex}
\end{document}

    
    
    %\section{}
    \newpage
    \pagenumbering{gobble}
        \printbibliography


    \newpage
    \pagenumbering{roman}
    \appendix
        \part{Appendices}
            \documentclass[12pt, a4paper]{report}

\input{template/main.tex}

\title{\BA{Title in Progress...}}
\author{Boris Andrews}
\affil{Mathematical Institute, University of Oxford}
\date{\today}


\begin{document}
    \pagenumbering{gobble}
    \maketitle
    
    
    \begin{abstract}
        Magnetic confinement reactors---in particular tokamaks---offer one of the most promising options for achieving practical nuclear fusion, with the potential to provide virtually limitless, clean energy. The theoretical and numerical modeling of tokamak plasmas is simultaneously an essential component of effective reactor design, and a great research barrier. Tokamak operational conditions exhibit comparatively low Knudsen numbers. Kinetic effects, including kinetic waves and instabilities, Landau damping, bump-on-tail instabilities and more, are therefore highly influential in tokamak plasma dynamics. Purely fluid models are inherently incapable of capturing these effects, whereas the high dimensionality in purely kinetic models render them practically intractable for most relevant purposes.

        We consider a $\delta\!f$ decomposition model, with a macroscopic fluid background and microscopic kinetic correction, both fully coupled to each other. A similar manner of discretization is proposed to that used in the recent \texttt{STRUPHY} code \cite{Holderied_Possanner_Wang_2021, Holderied_2022, Li_et_al_2023} with a finite-element model for the background and a pseudo-particle/PiC model for the correction.

        The fluid background satisfies the full, non-linear, resistive, compressible, Hall MHD equations. \cite{Laakmann_Hu_Farrell_2022} introduces finite-element(-in-space) implicit timesteppers for the incompressible analogue to this system with structure-preserving (SP) properties in the ideal case, alongside parameter-robust preconditioners. We show that these timesteppers can derive from a finite-element-in-time (FET) (and finite-element-in-space) interpretation. The benefits of this reformulation are discussed, including the derivation of timesteppers that are higher order in time, and the quantifiable dissipative SP properties in the non-ideal, resistive case.
        
        We discuss possible options for extending this FET approach to timesteppers for the compressible case.

        The kinetic corrections satisfy linearized Boltzmann equations. Using a Lénard--Bernstein collision operator, these take Fokker--Planck-like forms \cite{Fokker_1914, Planck_1917} wherein pseudo-particles in the numerical model obey the neoclassical transport equations, with particle-independent Brownian drift terms. This offers a rigorous methodology for incorporating collisions into the particle transport model, without coupling the equations of motions for each particle.
        
        Works by Chen, Chacón et al. \cite{Chen_Chacón_Barnes_2011, Chacón_Chen_Barnes_2013, Chen_Chacón_2014, Chen_Chacón_2015} have developed structure-preserving particle pushers for neoclassical transport in the Vlasov equations, derived from Crank--Nicolson integrators. We show these too can can derive from a FET interpretation, similarly offering potential extensions to higher-order-in-time particle pushers. The FET formulation is used also to consider how the stochastic drift terms can be incorporated into the pushers. Stochastic gyrokinetic expansions are also discussed.

        Different options for the numerical implementation of these schemes are considered.

        Due to the efficacy of FET in the development of SP timesteppers for both the fluid and kinetic component, we hope this approach will prove effective in the future for developing SP timesteppers for the full hybrid model. We hope this will give us the opportunity to incorporate previously inaccessible kinetic effects into the highly effective, modern, finite-element MHD models.
    \end{abstract}
    
    
    \newpage
    \tableofcontents
    
    
    \newpage
    \pagenumbering{arabic}
    %\linenumbers\renewcommand\thelinenumber{\color{black!50}\arabic{linenumber}}
            \input{0 - introduction/main.tex}
        \part{Research}
            \input{1 - low-noise PiC models/main.tex}
            \input{2 - kinetic component/main.tex}
            \input{3 - fluid component/main.tex}
            \input{4 - numerical implementation/main.tex}
        \part{Project Overview}
            \input{5 - research plan/main.tex}
            \input{6 - summary/main.tex}
    
    
    %\section{}
    \newpage
    \pagenumbering{gobble}
        \printbibliography


    \newpage
    \pagenumbering{roman}
    \appendix
        \part{Appendices}
            \input{8 - Hilbert complexes/main.tex}
            \input{9 - weak conservation proofs/main.tex}
\end{document}

            \documentclass[12pt, a4paper]{report}

\input{template/main.tex}

\title{\BA{Title in Progress...}}
\author{Boris Andrews}
\affil{Mathematical Institute, University of Oxford}
\date{\today}


\begin{document}
    \pagenumbering{gobble}
    \maketitle
    
    
    \begin{abstract}
        Magnetic confinement reactors---in particular tokamaks---offer one of the most promising options for achieving practical nuclear fusion, with the potential to provide virtually limitless, clean energy. The theoretical and numerical modeling of tokamak plasmas is simultaneously an essential component of effective reactor design, and a great research barrier. Tokamak operational conditions exhibit comparatively low Knudsen numbers. Kinetic effects, including kinetic waves and instabilities, Landau damping, bump-on-tail instabilities and more, are therefore highly influential in tokamak plasma dynamics. Purely fluid models are inherently incapable of capturing these effects, whereas the high dimensionality in purely kinetic models render them practically intractable for most relevant purposes.

        We consider a $\delta\!f$ decomposition model, with a macroscopic fluid background and microscopic kinetic correction, both fully coupled to each other. A similar manner of discretization is proposed to that used in the recent \texttt{STRUPHY} code \cite{Holderied_Possanner_Wang_2021, Holderied_2022, Li_et_al_2023} with a finite-element model for the background and a pseudo-particle/PiC model for the correction.

        The fluid background satisfies the full, non-linear, resistive, compressible, Hall MHD equations. \cite{Laakmann_Hu_Farrell_2022} introduces finite-element(-in-space) implicit timesteppers for the incompressible analogue to this system with structure-preserving (SP) properties in the ideal case, alongside parameter-robust preconditioners. We show that these timesteppers can derive from a finite-element-in-time (FET) (and finite-element-in-space) interpretation. The benefits of this reformulation are discussed, including the derivation of timesteppers that are higher order in time, and the quantifiable dissipative SP properties in the non-ideal, resistive case.
        
        We discuss possible options for extending this FET approach to timesteppers for the compressible case.

        The kinetic corrections satisfy linearized Boltzmann equations. Using a Lénard--Bernstein collision operator, these take Fokker--Planck-like forms \cite{Fokker_1914, Planck_1917} wherein pseudo-particles in the numerical model obey the neoclassical transport equations, with particle-independent Brownian drift terms. This offers a rigorous methodology for incorporating collisions into the particle transport model, without coupling the equations of motions for each particle.
        
        Works by Chen, Chacón et al. \cite{Chen_Chacón_Barnes_2011, Chacón_Chen_Barnes_2013, Chen_Chacón_2014, Chen_Chacón_2015} have developed structure-preserving particle pushers for neoclassical transport in the Vlasov equations, derived from Crank--Nicolson integrators. We show these too can can derive from a FET interpretation, similarly offering potential extensions to higher-order-in-time particle pushers. The FET formulation is used also to consider how the stochastic drift terms can be incorporated into the pushers. Stochastic gyrokinetic expansions are also discussed.

        Different options for the numerical implementation of these schemes are considered.

        Due to the efficacy of FET in the development of SP timesteppers for both the fluid and kinetic component, we hope this approach will prove effective in the future for developing SP timesteppers for the full hybrid model. We hope this will give us the opportunity to incorporate previously inaccessible kinetic effects into the highly effective, modern, finite-element MHD models.
    \end{abstract}
    
    
    \newpage
    \tableofcontents
    
    
    \newpage
    \pagenumbering{arabic}
    %\linenumbers\renewcommand\thelinenumber{\color{black!50}\arabic{linenumber}}
            \input{0 - introduction/main.tex}
        \part{Research}
            \input{1 - low-noise PiC models/main.tex}
            \input{2 - kinetic component/main.tex}
            \input{3 - fluid component/main.tex}
            \input{4 - numerical implementation/main.tex}
        \part{Project Overview}
            \input{5 - research plan/main.tex}
            \input{6 - summary/main.tex}
    
    
    %\section{}
    \newpage
    \pagenumbering{gobble}
        \printbibliography


    \newpage
    \pagenumbering{roman}
    \appendix
        \part{Appendices}
            \input{8 - Hilbert complexes/main.tex}
            \input{9 - weak conservation proofs/main.tex}
\end{document}

\end{document}

            \documentclass[12pt, a4paper]{report}

\documentclass[12pt, a4paper]{report}

\input{template/main.tex}

\title{\BA{Title in Progress...}}
\author{Boris Andrews}
\affil{Mathematical Institute, University of Oxford}
\date{\today}


\begin{document}
    \pagenumbering{gobble}
    \maketitle
    
    
    \begin{abstract}
        Magnetic confinement reactors---in particular tokamaks---offer one of the most promising options for achieving practical nuclear fusion, with the potential to provide virtually limitless, clean energy. The theoretical and numerical modeling of tokamak plasmas is simultaneously an essential component of effective reactor design, and a great research barrier. Tokamak operational conditions exhibit comparatively low Knudsen numbers. Kinetic effects, including kinetic waves and instabilities, Landau damping, bump-on-tail instabilities and more, are therefore highly influential in tokamak plasma dynamics. Purely fluid models are inherently incapable of capturing these effects, whereas the high dimensionality in purely kinetic models render them practically intractable for most relevant purposes.

        We consider a $\delta\!f$ decomposition model, with a macroscopic fluid background and microscopic kinetic correction, both fully coupled to each other. A similar manner of discretization is proposed to that used in the recent \texttt{STRUPHY} code \cite{Holderied_Possanner_Wang_2021, Holderied_2022, Li_et_al_2023} with a finite-element model for the background and a pseudo-particle/PiC model for the correction.

        The fluid background satisfies the full, non-linear, resistive, compressible, Hall MHD equations. \cite{Laakmann_Hu_Farrell_2022} introduces finite-element(-in-space) implicit timesteppers for the incompressible analogue to this system with structure-preserving (SP) properties in the ideal case, alongside parameter-robust preconditioners. We show that these timesteppers can derive from a finite-element-in-time (FET) (and finite-element-in-space) interpretation. The benefits of this reformulation are discussed, including the derivation of timesteppers that are higher order in time, and the quantifiable dissipative SP properties in the non-ideal, resistive case.
        
        We discuss possible options for extending this FET approach to timesteppers for the compressible case.

        The kinetic corrections satisfy linearized Boltzmann equations. Using a Lénard--Bernstein collision operator, these take Fokker--Planck-like forms \cite{Fokker_1914, Planck_1917} wherein pseudo-particles in the numerical model obey the neoclassical transport equations, with particle-independent Brownian drift terms. This offers a rigorous methodology for incorporating collisions into the particle transport model, without coupling the equations of motions for each particle.
        
        Works by Chen, Chacón et al. \cite{Chen_Chacón_Barnes_2011, Chacón_Chen_Barnes_2013, Chen_Chacón_2014, Chen_Chacón_2015} have developed structure-preserving particle pushers for neoclassical transport in the Vlasov equations, derived from Crank--Nicolson integrators. We show these too can can derive from a FET interpretation, similarly offering potential extensions to higher-order-in-time particle pushers. The FET formulation is used also to consider how the stochastic drift terms can be incorporated into the pushers. Stochastic gyrokinetic expansions are also discussed.

        Different options for the numerical implementation of these schemes are considered.

        Due to the efficacy of FET in the development of SP timesteppers for both the fluid and kinetic component, we hope this approach will prove effective in the future for developing SP timesteppers for the full hybrid model. We hope this will give us the opportunity to incorporate previously inaccessible kinetic effects into the highly effective, modern, finite-element MHD models.
    \end{abstract}
    
    
    \newpage
    \tableofcontents
    
    
    \newpage
    \pagenumbering{arabic}
    %\linenumbers\renewcommand\thelinenumber{\color{black!50}\arabic{linenumber}}
            \input{0 - introduction/main.tex}
        \part{Research}
            \input{1 - low-noise PiC models/main.tex}
            \input{2 - kinetic component/main.tex}
            \input{3 - fluid component/main.tex}
            \input{4 - numerical implementation/main.tex}
        \part{Project Overview}
            \input{5 - research plan/main.tex}
            \input{6 - summary/main.tex}
    
    
    %\section{}
    \newpage
    \pagenumbering{gobble}
        \printbibliography


    \newpage
    \pagenumbering{roman}
    \appendix
        \part{Appendices}
            \input{8 - Hilbert complexes/main.tex}
            \input{9 - weak conservation proofs/main.tex}
\end{document}


\title{\BA{Title in Progress...}}
\author{Boris Andrews}
\affil{Mathematical Institute, University of Oxford}
\date{\today}


\begin{document}
    \pagenumbering{gobble}
    \maketitle
    
    
    \begin{abstract}
        Magnetic confinement reactors---in particular tokamaks---offer one of the most promising options for achieving practical nuclear fusion, with the potential to provide virtually limitless, clean energy. The theoretical and numerical modeling of tokamak plasmas is simultaneously an essential component of effective reactor design, and a great research barrier. Tokamak operational conditions exhibit comparatively low Knudsen numbers. Kinetic effects, including kinetic waves and instabilities, Landau damping, bump-on-tail instabilities and more, are therefore highly influential in tokamak plasma dynamics. Purely fluid models are inherently incapable of capturing these effects, whereas the high dimensionality in purely kinetic models render them practically intractable for most relevant purposes.

        We consider a $\delta\!f$ decomposition model, with a macroscopic fluid background and microscopic kinetic correction, both fully coupled to each other. A similar manner of discretization is proposed to that used in the recent \texttt{STRUPHY} code \cite{Holderied_Possanner_Wang_2021, Holderied_2022, Li_et_al_2023} with a finite-element model for the background and a pseudo-particle/PiC model for the correction.

        The fluid background satisfies the full, non-linear, resistive, compressible, Hall MHD equations. \cite{Laakmann_Hu_Farrell_2022} introduces finite-element(-in-space) implicit timesteppers for the incompressible analogue to this system with structure-preserving (SP) properties in the ideal case, alongside parameter-robust preconditioners. We show that these timesteppers can derive from a finite-element-in-time (FET) (and finite-element-in-space) interpretation. The benefits of this reformulation are discussed, including the derivation of timesteppers that are higher order in time, and the quantifiable dissipative SP properties in the non-ideal, resistive case.
        
        We discuss possible options for extending this FET approach to timesteppers for the compressible case.

        The kinetic corrections satisfy linearized Boltzmann equations. Using a Lénard--Bernstein collision operator, these take Fokker--Planck-like forms \cite{Fokker_1914, Planck_1917} wherein pseudo-particles in the numerical model obey the neoclassical transport equations, with particle-independent Brownian drift terms. This offers a rigorous methodology for incorporating collisions into the particle transport model, without coupling the equations of motions for each particle.
        
        Works by Chen, Chacón et al. \cite{Chen_Chacón_Barnes_2011, Chacón_Chen_Barnes_2013, Chen_Chacón_2014, Chen_Chacón_2015} have developed structure-preserving particle pushers for neoclassical transport in the Vlasov equations, derived from Crank--Nicolson integrators. We show these too can can derive from a FET interpretation, similarly offering potential extensions to higher-order-in-time particle pushers. The FET formulation is used also to consider how the stochastic drift terms can be incorporated into the pushers. Stochastic gyrokinetic expansions are also discussed.

        Different options for the numerical implementation of these schemes are considered.

        Due to the efficacy of FET in the development of SP timesteppers for both the fluid and kinetic component, we hope this approach will prove effective in the future for developing SP timesteppers for the full hybrid model. We hope this will give us the opportunity to incorporate previously inaccessible kinetic effects into the highly effective, modern, finite-element MHD models.
    \end{abstract}
    
    
    \newpage
    \tableofcontents
    
    
    \newpage
    \pagenumbering{arabic}
    %\linenumbers\renewcommand\thelinenumber{\color{black!50}\arabic{linenumber}}
            \documentclass[12pt, a4paper]{report}

\input{template/main.tex}

\title{\BA{Title in Progress...}}
\author{Boris Andrews}
\affil{Mathematical Institute, University of Oxford}
\date{\today}


\begin{document}
    \pagenumbering{gobble}
    \maketitle
    
    
    \begin{abstract}
        Magnetic confinement reactors---in particular tokamaks---offer one of the most promising options for achieving practical nuclear fusion, with the potential to provide virtually limitless, clean energy. The theoretical and numerical modeling of tokamak plasmas is simultaneously an essential component of effective reactor design, and a great research barrier. Tokamak operational conditions exhibit comparatively low Knudsen numbers. Kinetic effects, including kinetic waves and instabilities, Landau damping, bump-on-tail instabilities and more, are therefore highly influential in tokamak plasma dynamics. Purely fluid models are inherently incapable of capturing these effects, whereas the high dimensionality in purely kinetic models render them practically intractable for most relevant purposes.

        We consider a $\delta\!f$ decomposition model, with a macroscopic fluid background and microscopic kinetic correction, both fully coupled to each other. A similar manner of discretization is proposed to that used in the recent \texttt{STRUPHY} code \cite{Holderied_Possanner_Wang_2021, Holderied_2022, Li_et_al_2023} with a finite-element model for the background and a pseudo-particle/PiC model for the correction.

        The fluid background satisfies the full, non-linear, resistive, compressible, Hall MHD equations. \cite{Laakmann_Hu_Farrell_2022} introduces finite-element(-in-space) implicit timesteppers for the incompressible analogue to this system with structure-preserving (SP) properties in the ideal case, alongside parameter-robust preconditioners. We show that these timesteppers can derive from a finite-element-in-time (FET) (and finite-element-in-space) interpretation. The benefits of this reformulation are discussed, including the derivation of timesteppers that are higher order in time, and the quantifiable dissipative SP properties in the non-ideal, resistive case.
        
        We discuss possible options for extending this FET approach to timesteppers for the compressible case.

        The kinetic corrections satisfy linearized Boltzmann equations. Using a Lénard--Bernstein collision operator, these take Fokker--Planck-like forms \cite{Fokker_1914, Planck_1917} wherein pseudo-particles in the numerical model obey the neoclassical transport equations, with particle-independent Brownian drift terms. This offers a rigorous methodology for incorporating collisions into the particle transport model, without coupling the equations of motions for each particle.
        
        Works by Chen, Chacón et al. \cite{Chen_Chacón_Barnes_2011, Chacón_Chen_Barnes_2013, Chen_Chacón_2014, Chen_Chacón_2015} have developed structure-preserving particle pushers for neoclassical transport in the Vlasov equations, derived from Crank--Nicolson integrators. We show these too can can derive from a FET interpretation, similarly offering potential extensions to higher-order-in-time particle pushers. The FET formulation is used also to consider how the stochastic drift terms can be incorporated into the pushers. Stochastic gyrokinetic expansions are also discussed.

        Different options for the numerical implementation of these schemes are considered.

        Due to the efficacy of FET in the development of SP timesteppers for both the fluid and kinetic component, we hope this approach will prove effective in the future for developing SP timesteppers for the full hybrid model. We hope this will give us the opportunity to incorporate previously inaccessible kinetic effects into the highly effective, modern, finite-element MHD models.
    \end{abstract}
    
    
    \newpage
    \tableofcontents
    
    
    \newpage
    \pagenumbering{arabic}
    %\linenumbers\renewcommand\thelinenumber{\color{black!50}\arabic{linenumber}}
            \input{0 - introduction/main.tex}
        \part{Research}
            \input{1 - low-noise PiC models/main.tex}
            \input{2 - kinetic component/main.tex}
            \input{3 - fluid component/main.tex}
            \input{4 - numerical implementation/main.tex}
        \part{Project Overview}
            \input{5 - research plan/main.tex}
            \input{6 - summary/main.tex}
    
    
    %\section{}
    \newpage
    \pagenumbering{gobble}
        \printbibliography


    \newpage
    \pagenumbering{roman}
    \appendix
        \part{Appendices}
            \input{8 - Hilbert complexes/main.tex}
            \input{9 - weak conservation proofs/main.tex}
\end{document}

        \part{Research}
            \documentclass[12pt, a4paper]{report}

\input{template/main.tex}

\title{\BA{Title in Progress...}}
\author{Boris Andrews}
\affil{Mathematical Institute, University of Oxford}
\date{\today}


\begin{document}
    \pagenumbering{gobble}
    \maketitle
    
    
    \begin{abstract}
        Magnetic confinement reactors---in particular tokamaks---offer one of the most promising options for achieving practical nuclear fusion, with the potential to provide virtually limitless, clean energy. The theoretical and numerical modeling of tokamak plasmas is simultaneously an essential component of effective reactor design, and a great research barrier. Tokamak operational conditions exhibit comparatively low Knudsen numbers. Kinetic effects, including kinetic waves and instabilities, Landau damping, bump-on-tail instabilities and more, are therefore highly influential in tokamak plasma dynamics. Purely fluid models are inherently incapable of capturing these effects, whereas the high dimensionality in purely kinetic models render them practically intractable for most relevant purposes.

        We consider a $\delta\!f$ decomposition model, with a macroscopic fluid background and microscopic kinetic correction, both fully coupled to each other. A similar manner of discretization is proposed to that used in the recent \texttt{STRUPHY} code \cite{Holderied_Possanner_Wang_2021, Holderied_2022, Li_et_al_2023} with a finite-element model for the background and a pseudo-particle/PiC model for the correction.

        The fluid background satisfies the full, non-linear, resistive, compressible, Hall MHD equations. \cite{Laakmann_Hu_Farrell_2022} introduces finite-element(-in-space) implicit timesteppers for the incompressible analogue to this system with structure-preserving (SP) properties in the ideal case, alongside parameter-robust preconditioners. We show that these timesteppers can derive from a finite-element-in-time (FET) (and finite-element-in-space) interpretation. The benefits of this reformulation are discussed, including the derivation of timesteppers that are higher order in time, and the quantifiable dissipative SP properties in the non-ideal, resistive case.
        
        We discuss possible options for extending this FET approach to timesteppers for the compressible case.

        The kinetic corrections satisfy linearized Boltzmann equations. Using a Lénard--Bernstein collision operator, these take Fokker--Planck-like forms \cite{Fokker_1914, Planck_1917} wherein pseudo-particles in the numerical model obey the neoclassical transport equations, with particle-independent Brownian drift terms. This offers a rigorous methodology for incorporating collisions into the particle transport model, without coupling the equations of motions for each particle.
        
        Works by Chen, Chacón et al. \cite{Chen_Chacón_Barnes_2011, Chacón_Chen_Barnes_2013, Chen_Chacón_2014, Chen_Chacón_2015} have developed structure-preserving particle pushers for neoclassical transport in the Vlasov equations, derived from Crank--Nicolson integrators. We show these too can can derive from a FET interpretation, similarly offering potential extensions to higher-order-in-time particle pushers. The FET formulation is used also to consider how the stochastic drift terms can be incorporated into the pushers. Stochastic gyrokinetic expansions are also discussed.

        Different options for the numerical implementation of these schemes are considered.

        Due to the efficacy of FET in the development of SP timesteppers for both the fluid and kinetic component, we hope this approach will prove effective in the future for developing SP timesteppers for the full hybrid model. We hope this will give us the opportunity to incorporate previously inaccessible kinetic effects into the highly effective, modern, finite-element MHD models.
    \end{abstract}
    
    
    \newpage
    \tableofcontents
    
    
    \newpage
    \pagenumbering{arabic}
    %\linenumbers\renewcommand\thelinenumber{\color{black!50}\arabic{linenumber}}
            \input{0 - introduction/main.tex}
        \part{Research}
            \input{1 - low-noise PiC models/main.tex}
            \input{2 - kinetic component/main.tex}
            \input{3 - fluid component/main.tex}
            \input{4 - numerical implementation/main.tex}
        \part{Project Overview}
            \input{5 - research plan/main.tex}
            \input{6 - summary/main.tex}
    
    
    %\section{}
    \newpage
    \pagenumbering{gobble}
        \printbibliography


    \newpage
    \pagenumbering{roman}
    \appendix
        \part{Appendices}
            \input{8 - Hilbert complexes/main.tex}
            \input{9 - weak conservation proofs/main.tex}
\end{document}

            \documentclass[12pt, a4paper]{report}

\input{template/main.tex}

\title{\BA{Title in Progress...}}
\author{Boris Andrews}
\affil{Mathematical Institute, University of Oxford}
\date{\today}


\begin{document}
    \pagenumbering{gobble}
    \maketitle
    
    
    \begin{abstract}
        Magnetic confinement reactors---in particular tokamaks---offer one of the most promising options for achieving practical nuclear fusion, with the potential to provide virtually limitless, clean energy. The theoretical and numerical modeling of tokamak plasmas is simultaneously an essential component of effective reactor design, and a great research barrier. Tokamak operational conditions exhibit comparatively low Knudsen numbers. Kinetic effects, including kinetic waves and instabilities, Landau damping, bump-on-tail instabilities and more, are therefore highly influential in tokamak plasma dynamics. Purely fluid models are inherently incapable of capturing these effects, whereas the high dimensionality in purely kinetic models render them practically intractable for most relevant purposes.

        We consider a $\delta\!f$ decomposition model, with a macroscopic fluid background and microscopic kinetic correction, both fully coupled to each other. A similar manner of discretization is proposed to that used in the recent \texttt{STRUPHY} code \cite{Holderied_Possanner_Wang_2021, Holderied_2022, Li_et_al_2023} with a finite-element model for the background and a pseudo-particle/PiC model for the correction.

        The fluid background satisfies the full, non-linear, resistive, compressible, Hall MHD equations. \cite{Laakmann_Hu_Farrell_2022} introduces finite-element(-in-space) implicit timesteppers for the incompressible analogue to this system with structure-preserving (SP) properties in the ideal case, alongside parameter-robust preconditioners. We show that these timesteppers can derive from a finite-element-in-time (FET) (and finite-element-in-space) interpretation. The benefits of this reformulation are discussed, including the derivation of timesteppers that are higher order in time, and the quantifiable dissipative SP properties in the non-ideal, resistive case.
        
        We discuss possible options for extending this FET approach to timesteppers for the compressible case.

        The kinetic corrections satisfy linearized Boltzmann equations. Using a Lénard--Bernstein collision operator, these take Fokker--Planck-like forms \cite{Fokker_1914, Planck_1917} wherein pseudo-particles in the numerical model obey the neoclassical transport equations, with particle-independent Brownian drift terms. This offers a rigorous methodology for incorporating collisions into the particle transport model, without coupling the equations of motions for each particle.
        
        Works by Chen, Chacón et al. \cite{Chen_Chacón_Barnes_2011, Chacón_Chen_Barnes_2013, Chen_Chacón_2014, Chen_Chacón_2015} have developed structure-preserving particle pushers for neoclassical transport in the Vlasov equations, derived from Crank--Nicolson integrators. We show these too can can derive from a FET interpretation, similarly offering potential extensions to higher-order-in-time particle pushers. The FET formulation is used also to consider how the stochastic drift terms can be incorporated into the pushers. Stochastic gyrokinetic expansions are also discussed.

        Different options for the numerical implementation of these schemes are considered.

        Due to the efficacy of FET in the development of SP timesteppers for both the fluid and kinetic component, we hope this approach will prove effective in the future for developing SP timesteppers for the full hybrid model. We hope this will give us the opportunity to incorporate previously inaccessible kinetic effects into the highly effective, modern, finite-element MHD models.
    \end{abstract}
    
    
    \newpage
    \tableofcontents
    
    
    \newpage
    \pagenumbering{arabic}
    %\linenumbers\renewcommand\thelinenumber{\color{black!50}\arabic{linenumber}}
            \input{0 - introduction/main.tex}
        \part{Research}
            \input{1 - low-noise PiC models/main.tex}
            \input{2 - kinetic component/main.tex}
            \input{3 - fluid component/main.tex}
            \input{4 - numerical implementation/main.tex}
        \part{Project Overview}
            \input{5 - research plan/main.tex}
            \input{6 - summary/main.tex}
    
    
    %\section{}
    \newpage
    \pagenumbering{gobble}
        \printbibliography


    \newpage
    \pagenumbering{roman}
    \appendix
        \part{Appendices}
            \input{8 - Hilbert complexes/main.tex}
            \input{9 - weak conservation proofs/main.tex}
\end{document}

            \documentclass[12pt, a4paper]{report}

\input{template/main.tex}

\title{\BA{Title in Progress...}}
\author{Boris Andrews}
\affil{Mathematical Institute, University of Oxford}
\date{\today}


\begin{document}
    \pagenumbering{gobble}
    \maketitle
    
    
    \begin{abstract}
        Magnetic confinement reactors---in particular tokamaks---offer one of the most promising options for achieving practical nuclear fusion, with the potential to provide virtually limitless, clean energy. The theoretical and numerical modeling of tokamak plasmas is simultaneously an essential component of effective reactor design, and a great research barrier. Tokamak operational conditions exhibit comparatively low Knudsen numbers. Kinetic effects, including kinetic waves and instabilities, Landau damping, bump-on-tail instabilities and more, are therefore highly influential in tokamak plasma dynamics. Purely fluid models are inherently incapable of capturing these effects, whereas the high dimensionality in purely kinetic models render them practically intractable for most relevant purposes.

        We consider a $\delta\!f$ decomposition model, with a macroscopic fluid background and microscopic kinetic correction, both fully coupled to each other. A similar manner of discretization is proposed to that used in the recent \texttt{STRUPHY} code \cite{Holderied_Possanner_Wang_2021, Holderied_2022, Li_et_al_2023} with a finite-element model for the background and a pseudo-particle/PiC model for the correction.

        The fluid background satisfies the full, non-linear, resistive, compressible, Hall MHD equations. \cite{Laakmann_Hu_Farrell_2022} introduces finite-element(-in-space) implicit timesteppers for the incompressible analogue to this system with structure-preserving (SP) properties in the ideal case, alongside parameter-robust preconditioners. We show that these timesteppers can derive from a finite-element-in-time (FET) (and finite-element-in-space) interpretation. The benefits of this reformulation are discussed, including the derivation of timesteppers that are higher order in time, and the quantifiable dissipative SP properties in the non-ideal, resistive case.
        
        We discuss possible options for extending this FET approach to timesteppers for the compressible case.

        The kinetic corrections satisfy linearized Boltzmann equations. Using a Lénard--Bernstein collision operator, these take Fokker--Planck-like forms \cite{Fokker_1914, Planck_1917} wherein pseudo-particles in the numerical model obey the neoclassical transport equations, with particle-independent Brownian drift terms. This offers a rigorous methodology for incorporating collisions into the particle transport model, without coupling the equations of motions for each particle.
        
        Works by Chen, Chacón et al. \cite{Chen_Chacón_Barnes_2011, Chacón_Chen_Barnes_2013, Chen_Chacón_2014, Chen_Chacón_2015} have developed structure-preserving particle pushers for neoclassical transport in the Vlasov equations, derived from Crank--Nicolson integrators. We show these too can can derive from a FET interpretation, similarly offering potential extensions to higher-order-in-time particle pushers. The FET formulation is used also to consider how the stochastic drift terms can be incorporated into the pushers. Stochastic gyrokinetic expansions are also discussed.

        Different options for the numerical implementation of these schemes are considered.

        Due to the efficacy of FET in the development of SP timesteppers for both the fluid and kinetic component, we hope this approach will prove effective in the future for developing SP timesteppers for the full hybrid model. We hope this will give us the opportunity to incorporate previously inaccessible kinetic effects into the highly effective, modern, finite-element MHD models.
    \end{abstract}
    
    
    \newpage
    \tableofcontents
    
    
    \newpage
    \pagenumbering{arabic}
    %\linenumbers\renewcommand\thelinenumber{\color{black!50}\arabic{linenumber}}
            \input{0 - introduction/main.tex}
        \part{Research}
            \input{1 - low-noise PiC models/main.tex}
            \input{2 - kinetic component/main.tex}
            \input{3 - fluid component/main.tex}
            \input{4 - numerical implementation/main.tex}
        \part{Project Overview}
            \input{5 - research plan/main.tex}
            \input{6 - summary/main.tex}
    
    
    %\section{}
    \newpage
    \pagenumbering{gobble}
        \printbibliography


    \newpage
    \pagenumbering{roman}
    \appendix
        \part{Appendices}
            \input{8 - Hilbert complexes/main.tex}
            \input{9 - weak conservation proofs/main.tex}
\end{document}

            \documentclass[12pt, a4paper]{report}

\input{template/main.tex}

\title{\BA{Title in Progress...}}
\author{Boris Andrews}
\affil{Mathematical Institute, University of Oxford}
\date{\today}


\begin{document}
    \pagenumbering{gobble}
    \maketitle
    
    
    \begin{abstract}
        Magnetic confinement reactors---in particular tokamaks---offer one of the most promising options for achieving practical nuclear fusion, with the potential to provide virtually limitless, clean energy. The theoretical and numerical modeling of tokamak plasmas is simultaneously an essential component of effective reactor design, and a great research barrier. Tokamak operational conditions exhibit comparatively low Knudsen numbers. Kinetic effects, including kinetic waves and instabilities, Landau damping, bump-on-tail instabilities and more, are therefore highly influential in tokamak plasma dynamics. Purely fluid models are inherently incapable of capturing these effects, whereas the high dimensionality in purely kinetic models render them practically intractable for most relevant purposes.

        We consider a $\delta\!f$ decomposition model, with a macroscopic fluid background and microscopic kinetic correction, both fully coupled to each other. A similar manner of discretization is proposed to that used in the recent \texttt{STRUPHY} code \cite{Holderied_Possanner_Wang_2021, Holderied_2022, Li_et_al_2023} with a finite-element model for the background and a pseudo-particle/PiC model for the correction.

        The fluid background satisfies the full, non-linear, resistive, compressible, Hall MHD equations. \cite{Laakmann_Hu_Farrell_2022} introduces finite-element(-in-space) implicit timesteppers for the incompressible analogue to this system with structure-preserving (SP) properties in the ideal case, alongside parameter-robust preconditioners. We show that these timesteppers can derive from a finite-element-in-time (FET) (and finite-element-in-space) interpretation. The benefits of this reformulation are discussed, including the derivation of timesteppers that are higher order in time, and the quantifiable dissipative SP properties in the non-ideal, resistive case.
        
        We discuss possible options for extending this FET approach to timesteppers for the compressible case.

        The kinetic corrections satisfy linearized Boltzmann equations. Using a Lénard--Bernstein collision operator, these take Fokker--Planck-like forms \cite{Fokker_1914, Planck_1917} wherein pseudo-particles in the numerical model obey the neoclassical transport equations, with particle-independent Brownian drift terms. This offers a rigorous methodology for incorporating collisions into the particle transport model, without coupling the equations of motions for each particle.
        
        Works by Chen, Chacón et al. \cite{Chen_Chacón_Barnes_2011, Chacón_Chen_Barnes_2013, Chen_Chacón_2014, Chen_Chacón_2015} have developed structure-preserving particle pushers for neoclassical transport in the Vlasov equations, derived from Crank--Nicolson integrators. We show these too can can derive from a FET interpretation, similarly offering potential extensions to higher-order-in-time particle pushers. The FET formulation is used also to consider how the stochastic drift terms can be incorporated into the pushers. Stochastic gyrokinetic expansions are also discussed.

        Different options for the numerical implementation of these schemes are considered.

        Due to the efficacy of FET in the development of SP timesteppers for both the fluid and kinetic component, we hope this approach will prove effective in the future for developing SP timesteppers for the full hybrid model. We hope this will give us the opportunity to incorporate previously inaccessible kinetic effects into the highly effective, modern, finite-element MHD models.
    \end{abstract}
    
    
    \newpage
    \tableofcontents
    
    
    \newpage
    \pagenumbering{arabic}
    %\linenumbers\renewcommand\thelinenumber{\color{black!50}\arabic{linenumber}}
            \input{0 - introduction/main.tex}
        \part{Research}
            \input{1 - low-noise PiC models/main.tex}
            \input{2 - kinetic component/main.tex}
            \input{3 - fluid component/main.tex}
            \input{4 - numerical implementation/main.tex}
        \part{Project Overview}
            \input{5 - research plan/main.tex}
            \input{6 - summary/main.tex}
    
    
    %\section{}
    \newpage
    \pagenumbering{gobble}
        \printbibliography


    \newpage
    \pagenumbering{roman}
    \appendix
        \part{Appendices}
            \input{8 - Hilbert complexes/main.tex}
            \input{9 - weak conservation proofs/main.tex}
\end{document}

        \part{Project Overview}
            \documentclass[12pt, a4paper]{report}

\input{template/main.tex}

\title{\BA{Title in Progress...}}
\author{Boris Andrews}
\affil{Mathematical Institute, University of Oxford}
\date{\today}


\begin{document}
    \pagenumbering{gobble}
    \maketitle
    
    
    \begin{abstract}
        Magnetic confinement reactors---in particular tokamaks---offer one of the most promising options for achieving practical nuclear fusion, with the potential to provide virtually limitless, clean energy. The theoretical and numerical modeling of tokamak plasmas is simultaneously an essential component of effective reactor design, and a great research barrier. Tokamak operational conditions exhibit comparatively low Knudsen numbers. Kinetic effects, including kinetic waves and instabilities, Landau damping, bump-on-tail instabilities and more, are therefore highly influential in tokamak plasma dynamics. Purely fluid models are inherently incapable of capturing these effects, whereas the high dimensionality in purely kinetic models render them practically intractable for most relevant purposes.

        We consider a $\delta\!f$ decomposition model, with a macroscopic fluid background and microscopic kinetic correction, both fully coupled to each other. A similar manner of discretization is proposed to that used in the recent \texttt{STRUPHY} code \cite{Holderied_Possanner_Wang_2021, Holderied_2022, Li_et_al_2023} with a finite-element model for the background and a pseudo-particle/PiC model for the correction.

        The fluid background satisfies the full, non-linear, resistive, compressible, Hall MHD equations. \cite{Laakmann_Hu_Farrell_2022} introduces finite-element(-in-space) implicit timesteppers for the incompressible analogue to this system with structure-preserving (SP) properties in the ideal case, alongside parameter-robust preconditioners. We show that these timesteppers can derive from a finite-element-in-time (FET) (and finite-element-in-space) interpretation. The benefits of this reformulation are discussed, including the derivation of timesteppers that are higher order in time, and the quantifiable dissipative SP properties in the non-ideal, resistive case.
        
        We discuss possible options for extending this FET approach to timesteppers for the compressible case.

        The kinetic corrections satisfy linearized Boltzmann equations. Using a Lénard--Bernstein collision operator, these take Fokker--Planck-like forms \cite{Fokker_1914, Planck_1917} wherein pseudo-particles in the numerical model obey the neoclassical transport equations, with particle-independent Brownian drift terms. This offers a rigorous methodology for incorporating collisions into the particle transport model, without coupling the equations of motions for each particle.
        
        Works by Chen, Chacón et al. \cite{Chen_Chacón_Barnes_2011, Chacón_Chen_Barnes_2013, Chen_Chacón_2014, Chen_Chacón_2015} have developed structure-preserving particle pushers for neoclassical transport in the Vlasov equations, derived from Crank--Nicolson integrators. We show these too can can derive from a FET interpretation, similarly offering potential extensions to higher-order-in-time particle pushers. The FET formulation is used also to consider how the stochastic drift terms can be incorporated into the pushers. Stochastic gyrokinetic expansions are also discussed.

        Different options for the numerical implementation of these schemes are considered.

        Due to the efficacy of FET in the development of SP timesteppers for both the fluid and kinetic component, we hope this approach will prove effective in the future for developing SP timesteppers for the full hybrid model. We hope this will give us the opportunity to incorporate previously inaccessible kinetic effects into the highly effective, modern, finite-element MHD models.
    \end{abstract}
    
    
    \newpage
    \tableofcontents
    
    
    \newpage
    \pagenumbering{arabic}
    %\linenumbers\renewcommand\thelinenumber{\color{black!50}\arabic{linenumber}}
            \input{0 - introduction/main.tex}
        \part{Research}
            \input{1 - low-noise PiC models/main.tex}
            \input{2 - kinetic component/main.tex}
            \input{3 - fluid component/main.tex}
            \input{4 - numerical implementation/main.tex}
        \part{Project Overview}
            \input{5 - research plan/main.tex}
            \input{6 - summary/main.tex}
    
    
    %\section{}
    \newpage
    \pagenumbering{gobble}
        \printbibliography


    \newpage
    \pagenumbering{roman}
    \appendix
        \part{Appendices}
            \input{8 - Hilbert complexes/main.tex}
            \input{9 - weak conservation proofs/main.tex}
\end{document}

            \documentclass[12pt, a4paper]{report}

\input{template/main.tex}

\title{\BA{Title in Progress...}}
\author{Boris Andrews}
\affil{Mathematical Institute, University of Oxford}
\date{\today}


\begin{document}
    \pagenumbering{gobble}
    \maketitle
    
    
    \begin{abstract}
        Magnetic confinement reactors---in particular tokamaks---offer one of the most promising options for achieving practical nuclear fusion, with the potential to provide virtually limitless, clean energy. The theoretical and numerical modeling of tokamak plasmas is simultaneously an essential component of effective reactor design, and a great research barrier. Tokamak operational conditions exhibit comparatively low Knudsen numbers. Kinetic effects, including kinetic waves and instabilities, Landau damping, bump-on-tail instabilities and more, are therefore highly influential in tokamak plasma dynamics. Purely fluid models are inherently incapable of capturing these effects, whereas the high dimensionality in purely kinetic models render them practically intractable for most relevant purposes.

        We consider a $\delta\!f$ decomposition model, with a macroscopic fluid background and microscopic kinetic correction, both fully coupled to each other. A similar manner of discretization is proposed to that used in the recent \texttt{STRUPHY} code \cite{Holderied_Possanner_Wang_2021, Holderied_2022, Li_et_al_2023} with a finite-element model for the background and a pseudo-particle/PiC model for the correction.

        The fluid background satisfies the full, non-linear, resistive, compressible, Hall MHD equations. \cite{Laakmann_Hu_Farrell_2022} introduces finite-element(-in-space) implicit timesteppers for the incompressible analogue to this system with structure-preserving (SP) properties in the ideal case, alongside parameter-robust preconditioners. We show that these timesteppers can derive from a finite-element-in-time (FET) (and finite-element-in-space) interpretation. The benefits of this reformulation are discussed, including the derivation of timesteppers that are higher order in time, and the quantifiable dissipative SP properties in the non-ideal, resistive case.
        
        We discuss possible options for extending this FET approach to timesteppers for the compressible case.

        The kinetic corrections satisfy linearized Boltzmann equations. Using a Lénard--Bernstein collision operator, these take Fokker--Planck-like forms \cite{Fokker_1914, Planck_1917} wherein pseudo-particles in the numerical model obey the neoclassical transport equations, with particle-independent Brownian drift terms. This offers a rigorous methodology for incorporating collisions into the particle transport model, without coupling the equations of motions for each particle.
        
        Works by Chen, Chacón et al. \cite{Chen_Chacón_Barnes_2011, Chacón_Chen_Barnes_2013, Chen_Chacón_2014, Chen_Chacón_2015} have developed structure-preserving particle pushers for neoclassical transport in the Vlasov equations, derived from Crank--Nicolson integrators. We show these too can can derive from a FET interpretation, similarly offering potential extensions to higher-order-in-time particle pushers. The FET formulation is used also to consider how the stochastic drift terms can be incorporated into the pushers. Stochastic gyrokinetic expansions are also discussed.

        Different options for the numerical implementation of these schemes are considered.

        Due to the efficacy of FET in the development of SP timesteppers for both the fluid and kinetic component, we hope this approach will prove effective in the future for developing SP timesteppers for the full hybrid model. We hope this will give us the opportunity to incorporate previously inaccessible kinetic effects into the highly effective, modern, finite-element MHD models.
    \end{abstract}
    
    
    \newpage
    \tableofcontents
    
    
    \newpage
    \pagenumbering{arabic}
    %\linenumbers\renewcommand\thelinenumber{\color{black!50}\arabic{linenumber}}
            \input{0 - introduction/main.tex}
        \part{Research}
            \input{1 - low-noise PiC models/main.tex}
            \input{2 - kinetic component/main.tex}
            \input{3 - fluid component/main.tex}
            \input{4 - numerical implementation/main.tex}
        \part{Project Overview}
            \input{5 - research plan/main.tex}
            \input{6 - summary/main.tex}
    
    
    %\section{}
    \newpage
    \pagenumbering{gobble}
        \printbibliography


    \newpage
    \pagenumbering{roman}
    \appendix
        \part{Appendices}
            \input{8 - Hilbert complexes/main.tex}
            \input{9 - weak conservation proofs/main.tex}
\end{document}

    
    
    %\section{}
    \newpage
    \pagenumbering{gobble}
        \printbibliography


    \newpage
    \pagenumbering{roman}
    \appendix
        \part{Appendices}
            \documentclass[12pt, a4paper]{report}

\input{template/main.tex}

\title{\BA{Title in Progress...}}
\author{Boris Andrews}
\affil{Mathematical Institute, University of Oxford}
\date{\today}


\begin{document}
    \pagenumbering{gobble}
    \maketitle
    
    
    \begin{abstract}
        Magnetic confinement reactors---in particular tokamaks---offer one of the most promising options for achieving practical nuclear fusion, with the potential to provide virtually limitless, clean energy. The theoretical and numerical modeling of tokamak plasmas is simultaneously an essential component of effective reactor design, and a great research barrier. Tokamak operational conditions exhibit comparatively low Knudsen numbers. Kinetic effects, including kinetic waves and instabilities, Landau damping, bump-on-tail instabilities and more, are therefore highly influential in tokamak plasma dynamics. Purely fluid models are inherently incapable of capturing these effects, whereas the high dimensionality in purely kinetic models render them practically intractable for most relevant purposes.

        We consider a $\delta\!f$ decomposition model, with a macroscopic fluid background and microscopic kinetic correction, both fully coupled to each other. A similar manner of discretization is proposed to that used in the recent \texttt{STRUPHY} code \cite{Holderied_Possanner_Wang_2021, Holderied_2022, Li_et_al_2023} with a finite-element model for the background and a pseudo-particle/PiC model for the correction.

        The fluid background satisfies the full, non-linear, resistive, compressible, Hall MHD equations. \cite{Laakmann_Hu_Farrell_2022} introduces finite-element(-in-space) implicit timesteppers for the incompressible analogue to this system with structure-preserving (SP) properties in the ideal case, alongside parameter-robust preconditioners. We show that these timesteppers can derive from a finite-element-in-time (FET) (and finite-element-in-space) interpretation. The benefits of this reformulation are discussed, including the derivation of timesteppers that are higher order in time, and the quantifiable dissipative SP properties in the non-ideal, resistive case.
        
        We discuss possible options for extending this FET approach to timesteppers for the compressible case.

        The kinetic corrections satisfy linearized Boltzmann equations. Using a Lénard--Bernstein collision operator, these take Fokker--Planck-like forms \cite{Fokker_1914, Planck_1917} wherein pseudo-particles in the numerical model obey the neoclassical transport equations, with particle-independent Brownian drift terms. This offers a rigorous methodology for incorporating collisions into the particle transport model, without coupling the equations of motions for each particle.
        
        Works by Chen, Chacón et al. \cite{Chen_Chacón_Barnes_2011, Chacón_Chen_Barnes_2013, Chen_Chacón_2014, Chen_Chacón_2015} have developed structure-preserving particle pushers for neoclassical transport in the Vlasov equations, derived from Crank--Nicolson integrators. We show these too can can derive from a FET interpretation, similarly offering potential extensions to higher-order-in-time particle pushers. The FET formulation is used also to consider how the stochastic drift terms can be incorporated into the pushers. Stochastic gyrokinetic expansions are also discussed.

        Different options for the numerical implementation of these schemes are considered.

        Due to the efficacy of FET in the development of SP timesteppers for both the fluid and kinetic component, we hope this approach will prove effective in the future for developing SP timesteppers for the full hybrid model. We hope this will give us the opportunity to incorporate previously inaccessible kinetic effects into the highly effective, modern, finite-element MHD models.
    \end{abstract}
    
    
    \newpage
    \tableofcontents
    
    
    \newpage
    \pagenumbering{arabic}
    %\linenumbers\renewcommand\thelinenumber{\color{black!50}\arabic{linenumber}}
            \input{0 - introduction/main.tex}
        \part{Research}
            \input{1 - low-noise PiC models/main.tex}
            \input{2 - kinetic component/main.tex}
            \input{3 - fluid component/main.tex}
            \input{4 - numerical implementation/main.tex}
        \part{Project Overview}
            \input{5 - research plan/main.tex}
            \input{6 - summary/main.tex}
    
    
    %\section{}
    \newpage
    \pagenumbering{gobble}
        \printbibliography


    \newpage
    \pagenumbering{roman}
    \appendix
        \part{Appendices}
            \input{8 - Hilbert complexes/main.tex}
            \input{9 - weak conservation proofs/main.tex}
\end{document}

            \documentclass[12pt, a4paper]{report}

\input{template/main.tex}

\title{\BA{Title in Progress...}}
\author{Boris Andrews}
\affil{Mathematical Institute, University of Oxford}
\date{\today}


\begin{document}
    \pagenumbering{gobble}
    \maketitle
    
    
    \begin{abstract}
        Magnetic confinement reactors---in particular tokamaks---offer one of the most promising options for achieving practical nuclear fusion, with the potential to provide virtually limitless, clean energy. The theoretical and numerical modeling of tokamak plasmas is simultaneously an essential component of effective reactor design, and a great research barrier. Tokamak operational conditions exhibit comparatively low Knudsen numbers. Kinetic effects, including kinetic waves and instabilities, Landau damping, bump-on-tail instabilities and more, are therefore highly influential in tokamak plasma dynamics. Purely fluid models are inherently incapable of capturing these effects, whereas the high dimensionality in purely kinetic models render them practically intractable for most relevant purposes.

        We consider a $\delta\!f$ decomposition model, with a macroscopic fluid background and microscopic kinetic correction, both fully coupled to each other. A similar manner of discretization is proposed to that used in the recent \texttt{STRUPHY} code \cite{Holderied_Possanner_Wang_2021, Holderied_2022, Li_et_al_2023} with a finite-element model for the background and a pseudo-particle/PiC model for the correction.

        The fluid background satisfies the full, non-linear, resistive, compressible, Hall MHD equations. \cite{Laakmann_Hu_Farrell_2022} introduces finite-element(-in-space) implicit timesteppers for the incompressible analogue to this system with structure-preserving (SP) properties in the ideal case, alongside parameter-robust preconditioners. We show that these timesteppers can derive from a finite-element-in-time (FET) (and finite-element-in-space) interpretation. The benefits of this reformulation are discussed, including the derivation of timesteppers that are higher order in time, and the quantifiable dissipative SP properties in the non-ideal, resistive case.
        
        We discuss possible options for extending this FET approach to timesteppers for the compressible case.

        The kinetic corrections satisfy linearized Boltzmann equations. Using a Lénard--Bernstein collision operator, these take Fokker--Planck-like forms \cite{Fokker_1914, Planck_1917} wherein pseudo-particles in the numerical model obey the neoclassical transport equations, with particle-independent Brownian drift terms. This offers a rigorous methodology for incorporating collisions into the particle transport model, without coupling the equations of motions for each particle.
        
        Works by Chen, Chacón et al. \cite{Chen_Chacón_Barnes_2011, Chacón_Chen_Barnes_2013, Chen_Chacón_2014, Chen_Chacón_2015} have developed structure-preserving particle pushers for neoclassical transport in the Vlasov equations, derived from Crank--Nicolson integrators. We show these too can can derive from a FET interpretation, similarly offering potential extensions to higher-order-in-time particle pushers. The FET formulation is used also to consider how the stochastic drift terms can be incorporated into the pushers. Stochastic gyrokinetic expansions are also discussed.

        Different options for the numerical implementation of these schemes are considered.

        Due to the efficacy of FET in the development of SP timesteppers for both the fluid and kinetic component, we hope this approach will prove effective in the future for developing SP timesteppers for the full hybrid model. We hope this will give us the opportunity to incorporate previously inaccessible kinetic effects into the highly effective, modern, finite-element MHD models.
    \end{abstract}
    
    
    \newpage
    \tableofcontents
    
    
    \newpage
    \pagenumbering{arabic}
    %\linenumbers\renewcommand\thelinenumber{\color{black!50}\arabic{linenumber}}
            \input{0 - introduction/main.tex}
        \part{Research}
            \input{1 - low-noise PiC models/main.tex}
            \input{2 - kinetic component/main.tex}
            \input{3 - fluid component/main.tex}
            \input{4 - numerical implementation/main.tex}
        \part{Project Overview}
            \input{5 - research plan/main.tex}
            \input{6 - summary/main.tex}
    
    
    %\section{}
    \newpage
    \pagenumbering{gobble}
        \printbibliography


    \newpage
    \pagenumbering{roman}
    \appendix
        \part{Appendices}
            \input{8 - Hilbert complexes/main.tex}
            \input{9 - weak conservation proofs/main.tex}
\end{document}

\end{document}

            \documentclass[12pt, a4paper]{report}

\documentclass[12pt, a4paper]{report}

\input{template/main.tex}

\title{\BA{Title in Progress...}}
\author{Boris Andrews}
\affil{Mathematical Institute, University of Oxford}
\date{\today}


\begin{document}
    \pagenumbering{gobble}
    \maketitle
    
    
    \begin{abstract}
        Magnetic confinement reactors---in particular tokamaks---offer one of the most promising options for achieving practical nuclear fusion, with the potential to provide virtually limitless, clean energy. The theoretical and numerical modeling of tokamak plasmas is simultaneously an essential component of effective reactor design, and a great research barrier. Tokamak operational conditions exhibit comparatively low Knudsen numbers. Kinetic effects, including kinetic waves and instabilities, Landau damping, bump-on-tail instabilities and more, are therefore highly influential in tokamak plasma dynamics. Purely fluid models are inherently incapable of capturing these effects, whereas the high dimensionality in purely kinetic models render them practically intractable for most relevant purposes.

        We consider a $\delta\!f$ decomposition model, with a macroscopic fluid background and microscopic kinetic correction, both fully coupled to each other. A similar manner of discretization is proposed to that used in the recent \texttt{STRUPHY} code \cite{Holderied_Possanner_Wang_2021, Holderied_2022, Li_et_al_2023} with a finite-element model for the background and a pseudo-particle/PiC model for the correction.

        The fluid background satisfies the full, non-linear, resistive, compressible, Hall MHD equations. \cite{Laakmann_Hu_Farrell_2022} introduces finite-element(-in-space) implicit timesteppers for the incompressible analogue to this system with structure-preserving (SP) properties in the ideal case, alongside parameter-robust preconditioners. We show that these timesteppers can derive from a finite-element-in-time (FET) (and finite-element-in-space) interpretation. The benefits of this reformulation are discussed, including the derivation of timesteppers that are higher order in time, and the quantifiable dissipative SP properties in the non-ideal, resistive case.
        
        We discuss possible options for extending this FET approach to timesteppers for the compressible case.

        The kinetic corrections satisfy linearized Boltzmann equations. Using a Lénard--Bernstein collision operator, these take Fokker--Planck-like forms \cite{Fokker_1914, Planck_1917} wherein pseudo-particles in the numerical model obey the neoclassical transport equations, with particle-independent Brownian drift terms. This offers a rigorous methodology for incorporating collisions into the particle transport model, without coupling the equations of motions for each particle.
        
        Works by Chen, Chacón et al. \cite{Chen_Chacón_Barnes_2011, Chacón_Chen_Barnes_2013, Chen_Chacón_2014, Chen_Chacón_2015} have developed structure-preserving particle pushers for neoclassical transport in the Vlasov equations, derived from Crank--Nicolson integrators. We show these too can can derive from a FET interpretation, similarly offering potential extensions to higher-order-in-time particle pushers. The FET formulation is used also to consider how the stochastic drift terms can be incorporated into the pushers. Stochastic gyrokinetic expansions are also discussed.

        Different options for the numerical implementation of these schemes are considered.

        Due to the efficacy of FET in the development of SP timesteppers for both the fluid and kinetic component, we hope this approach will prove effective in the future for developing SP timesteppers for the full hybrid model. We hope this will give us the opportunity to incorporate previously inaccessible kinetic effects into the highly effective, modern, finite-element MHD models.
    \end{abstract}
    
    
    \newpage
    \tableofcontents
    
    
    \newpage
    \pagenumbering{arabic}
    %\linenumbers\renewcommand\thelinenumber{\color{black!50}\arabic{linenumber}}
            \input{0 - introduction/main.tex}
        \part{Research}
            \input{1 - low-noise PiC models/main.tex}
            \input{2 - kinetic component/main.tex}
            \input{3 - fluid component/main.tex}
            \input{4 - numerical implementation/main.tex}
        \part{Project Overview}
            \input{5 - research plan/main.tex}
            \input{6 - summary/main.tex}
    
    
    %\section{}
    \newpage
    \pagenumbering{gobble}
        \printbibliography


    \newpage
    \pagenumbering{roman}
    \appendix
        \part{Appendices}
            \input{8 - Hilbert complexes/main.tex}
            \input{9 - weak conservation proofs/main.tex}
\end{document}


\title{\BA{Title in Progress...}}
\author{Boris Andrews}
\affil{Mathematical Institute, University of Oxford}
\date{\today}


\begin{document}
    \pagenumbering{gobble}
    \maketitle
    
    
    \begin{abstract}
        Magnetic confinement reactors---in particular tokamaks---offer one of the most promising options for achieving practical nuclear fusion, with the potential to provide virtually limitless, clean energy. The theoretical and numerical modeling of tokamak plasmas is simultaneously an essential component of effective reactor design, and a great research barrier. Tokamak operational conditions exhibit comparatively low Knudsen numbers. Kinetic effects, including kinetic waves and instabilities, Landau damping, bump-on-tail instabilities and more, are therefore highly influential in tokamak plasma dynamics. Purely fluid models are inherently incapable of capturing these effects, whereas the high dimensionality in purely kinetic models render them practically intractable for most relevant purposes.

        We consider a $\delta\!f$ decomposition model, with a macroscopic fluid background and microscopic kinetic correction, both fully coupled to each other. A similar manner of discretization is proposed to that used in the recent \texttt{STRUPHY} code \cite{Holderied_Possanner_Wang_2021, Holderied_2022, Li_et_al_2023} with a finite-element model for the background and a pseudo-particle/PiC model for the correction.

        The fluid background satisfies the full, non-linear, resistive, compressible, Hall MHD equations. \cite{Laakmann_Hu_Farrell_2022} introduces finite-element(-in-space) implicit timesteppers for the incompressible analogue to this system with structure-preserving (SP) properties in the ideal case, alongside parameter-robust preconditioners. We show that these timesteppers can derive from a finite-element-in-time (FET) (and finite-element-in-space) interpretation. The benefits of this reformulation are discussed, including the derivation of timesteppers that are higher order in time, and the quantifiable dissipative SP properties in the non-ideal, resistive case.
        
        We discuss possible options for extending this FET approach to timesteppers for the compressible case.

        The kinetic corrections satisfy linearized Boltzmann equations. Using a Lénard--Bernstein collision operator, these take Fokker--Planck-like forms \cite{Fokker_1914, Planck_1917} wherein pseudo-particles in the numerical model obey the neoclassical transport equations, with particle-independent Brownian drift terms. This offers a rigorous methodology for incorporating collisions into the particle transport model, without coupling the equations of motions for each particle.
        
        Works by Chen, Chacón et al. \cite{Chen_Chacón_Barnes_2011, Chacón_Chen_Barnes_2013, Chen_Chacón_2014, Chen_Chacón_2015} have developed structure-preserving particle pushers for neoclassical transport in the Vlasov equations, derived from Crank--Nicolson integrators. We show these too can can derive from a FET interpretation, similarly offering potential extensions to higher-order-in-time particle pushers. The FET formulation is used also to consider how the stochastic drift terms can be incorporated into the pushers. Stochastic gyrokinetic expansions are also discussed.

        Different options for the numerical implementation of these schemes are considered.

        Due to the efficacy of FET in the development of SP timesteppers for both the fluid and kinetic component, we hope this approach will prove effective in the future for developing SP timesteppers for the full hybrid model. We hope this will give us the opportunity to incorporate previously inaccessible kinetic effects into the highly effective, modern, finite-element MHD models.
    \end{abstract}
    
    
    \newpage
    \tableofcontents
    
    
    \newpage
    \pagenumbering{arabic}
    %\linenumbers\renewcommand\thelinenumber{\color{black!50}\arabic{linenumber}}
            \documentclass[12pt, a4paper]{report}

\input{template/main.tex}

\title{\BA{Title in Progress...}}
\author{Boris Andrews}
\affil{Mathematical Institute, University of Oxford}
\date{\today}


\begin{document}
    \pagenumbering{gobble}
    \maketitle
    
    
    \begin{abstract}
        Magnetic confinement reactors---in particular tokamaks---offer one of the most promising options for achieving practical nuclear fusion, with the potential to provide virtually limitless, clean energy. The theoretical and numerical modeling of tokamak plasmas is simultaneously an essential component of effective reactor design, and a great research barrier. Tokamak operational conditions exhibit comparatively low Knudsen numbers. Kinetic effects, including kinetic waves and instabilities, Landau damping, bump-on-tail instabilities and more, are therefore highly influential in tokamak plasma dynamics. Purely fluid models are inherently incapable of capturing these effects, whereas the high dimensionality in purely kinetic models render them practically intractable for most relevant purposes.

        We consider a $\delta\!f$ decomposition model, with a macroscopic fluid background and microscopic kinetic correction, both fully coupled to each other. A similar manner of discretization is proposed to that used in the recent \texttt{STRUPHY} code \cite{Holderied_Possanner_Wang_2021, Holderied_2022, Li_et_al_2023} with a finite-element model for the background and a pseudo-particle/PiC model for the correction.

        The fluid background satisfies the full, non-linear, resistive, compressible, Hall MHD equations. \cite{Laakmann_Hu_Farrell_2022} introduces finite-element(-in-space) implicit timesteppers for the incompressible analogue to this system with structure-preserving (SP) properties in the ideal case, alongside parameter-robust preconditioners. We show that these timesteppers can derive from a finite-element-in-time (FET) (and finite-element-in-space) interpretation. The benefits of this reformulation are discussed, including the derivation of timesteppers that are higher order in time, and the quantifiable dissipative SP properties in the non-ideal, resistive case.
        
        We discuss possible options for extending this FET approach to timesteppers for the compressible case.

        The kinetic corrections satisfy linearized Boltzmann equations. Using a Lénard--Bernstein collision operator, these take Fokker--Planck-like forms \cite{Fokker_1914, Planck_1917} wherein pseudo-particles in the numerical model obey the neoclassical transport equations, with particle-independent Brownian drift terms. This offers a rigorous methodology for incorporating collisions into the particle transport model, without coupling the equations of motions for each particle.
        
        Works by Chen, Chacón et al. \cite{Chen_Chacón_Barnes_2011, Chacón_Chen_Barnes_2013, Chen_Chacón_2014, Chen_Chacón_2015} have developed structure-preserving particle pushers for neoclassical transport in the Vlasov equations, derived from Crank--Nicolson integrators. We show these too can can derive from a FET interpretation, similarly offering potential extensions to higher-order-in-time particle pushers. The FET formulation is used also to consider how the stochastic drift terms can be incorporated into the pushers. Stochastic gyrokinetic expansions are also discussed.

        Different options for the numerical implementation of these schemes are considered.

        Due to the efficacy of FET in the development of SP timesteppers for both the fluid and kinetic component, we hope this approach will prove effective in the future for developing SP timesteppers for the full hybrid model. We hope this will give us the opportunity to incorporate previously inaccessible kinetic effects into the highly effective, modern, finite-element MHD models.
    \end{abstract}
    
    
    \newpage
    \tableofcontents
    
    
    \newpage
    \pagenumbering{arabic}
    %\linenumbers\renewcommand\thelinenumber{\color{black!50}\arabic{linenumber}}
            \input{0 - introduction/main.tex}
        \part{Research}
            \input{1 - low-noise PiC models/main.tex}
            \input{2 - kinetic component/main.tex}
            \input{3 - fluid component/main.tex}
            \input{4 - numerical implementation/main.tex}
        \part{Project Overview}
            \input{5 - research plan/main.tex}
            \input{6 - summary/main.tex}
    
    
    %\section{}
    \newpage
    \pagenumbering{gobble}
        \printbibliography


    \newpage
    \pagenumbering{roman}
    \appendix
        \part{Appendices}
            \input{8 - Hilbert complexes/main.tex}
            \input{9 - weak conservation proofs/main.tex}
\end{document}

        \part{Research}
            \documentclass[12pt, a4paper]{report}

\input{template/main.tex}

\title{\BA{Title in Progress...}}
\author{Boris Andrews}
\affil{Mathematical Institute, University of Oxford}
\date{\today}


\begin{document}
    \pagenumbering{gobble}
    \maketitle
    
    
    \begin{abstract}
        Magnetic confinement reactors---in particular tokamaks---offer one of the most promising options for achieving practical nuclear fusion, with the potential to provide virtually limitless, clean energy. The theoretical and numerical modeling of tokamak plasmas is simultaneously an essential component of effective reactor design, and a great research barrier. Tokamak operational conditions exhibit comparatively low Knudsen numbers. Kinetic effects, including kinetic waves and instabilities, Landau damping, bump-on-tail instabilities and more, are therefore highly influential in tokamak plasma dynamics. Purely fluid models are inherently incapable of capturing these effects, whereas the high dimensionality in purely kinetic models render them practically intractable for most relevant purposes.

        We consider a $\delta\!f$ decomposition model, with a macroscopic fluid background and microscopic kinetic correction, both fully coupled to each other. A similar manner of discretization is proposed to that used in the recent \texttt{STRUPHY} code \cite{Holderied_Possanner_Wang_2021, Holderied_2022, Li_et_al_2023} with a finite-element model for the background and a pseudo-particle/PiC model for the correction.

        The fluid background satisfies the full, non-linear, resistive, compressible, Hall MHD equations. \cite{Laakmann_Hu_Farrell_2022} introduces finite-element(-in-space) implicit timesteppers for the incompressible analogue to this system with structure-preserving (SP) properties in the ideal case, alongside parameter-robust preconditioners. We show that these timesteppers can derive from a finite-element-in-time (FET) (and finite-element-in-space) interpretation. The benefits of this reformulation are discussed, including the derivation of timesteppers that are higher order in time, and the quantifiable dissipative SP properties in the non-ideal, resistive case.
        
        We discuss possible options for extending this FET approach to timesteppers for the compressible case.

        The kinetic corrections satisfy linearized Boltzmann equations. Using a Lénard--Bernstein collision operator, these take Fokker--Planck-like forms \cite{Fokker_1914, Planck_1917} wherein pseudo-particles in the numerical model obey the neoclassical transport equations, with particle-independent Brownian drift terms. This offers a rigorous methodology for incorporating collisions into the particle transport model, without coupling the equations of motions for each particle.
        
        Works by Chen, Chacón et al. \cite{Chen_Chacón_Barnes_2011, Chacón_Chen_Barnes_2013, Chen_Chacón_2014, Chen_Chacón_2015} have developed structure-preserving particle pushers for neoclassical transport in the Vlasov equations, derived from Crank--Nicolson integrators. We show these too can can derive from a FET interpretation, similarly offering potential extensions to higher-order-in-time particle pushers. The FET formulation is used also to consider how the stochastic drift terms can be incorporated into the pushers. Stochastic gyrokinetic expansions are also discussed.

        Different options for the numerical implementation of these schemes are considered.

        Due to the efficacy of FET in the development of SP timesteppers for both the fluid and kinetic component, we hope this approach will prove effective in the future for developing SP timesteppers for the full hybrid model. We hope this will give us the opportunity to incorporate previously inaccessible kinetic effects into the highly effective, modern, finite-element MHD models.
    \end{abstract}
    
    
    \newpage
    \tableofcontents
    
    
    \newpage
    \pagenumbering{arabic}
    %\linenumbers\renewcommand\thelinenumber{\color{black!50}\arabic{linenumber}}
            \input{0 - introduction/main.tex}
        \part{Research}
            \input{1 - low-noise PiC models/main.tex}
            \input{2 - kinetic component/main.tex}
            \input{3 - fluid component/main.tex}
            \input{4 - numerical implementation/main.tex}
        \part{Project Overview}
            \input{5 - research plan/main.tex}
            \input{6 - summary/main.tex}
    
    
    %\section{}
    \newpage
    \pagenumbering{gobble}
        \printbibliography


    \newpage
    \pagenumbering{roman}
    \appendix
        \part{Appendices}
            \input{8 - Hilbert complexes/main.tex}
            \input{9 - weak conservation proofs/main.tex}
\end{document}

            \documentclass[12pt, a4paper]{report}

\input{template/main.tex}

\title{\BA{Title in Progress...}}
\author{Boris Andrews}
\affil{Mathematical Institute, University of Oxford}
\date{\today}


\begin{document}
    \pagenumbering{gobble}
    \maketitle
    
    
    \begin{abstract}
        Magnetic confinement reactors---in particular tokamaks---offer one of the most promising options for achieving practical nuclear fusion, with the potential to provide virtually limitless, clean energy. The theoretical and numerical modeling of tokamak plasmas is simultaneously an essential component of effective reactor design, and a great research barrier. Tokamak operational conditions exhibit comparatively low Knudsen numbers. Kinetic effects, including kinetic waves and instabilities, Landau damping, bump-on-tail instabilities and more, are therefore highly influential in tokamak plasma dynamics. Purely fluid models are inherently incapable of capturing these effects, whereas the high dimensionality in purely kinetic models render them practically intractable for most relevant purposes.

        We consider a $\delta\!f$ decomposition model, with a macroscopic fluid background and microscopic kinetic correction, both fully coupled to each other. A similar manner of discretization is proposed to that used in the recent \texttt{STRUPHY} code \cite{Holderied_Possanner_Wang_2021, Holderied_2022, Li_et_al_2023} with a finite-element model for the background and a pseudo-particle/PiC model for the correction.

        The fluid background satisfies the full, non-linear, resistive, compressible, Hall MHD equations. \cite{Laakmann_Hu_Farrell_2022} introduces finite-element(-in-space) implicit timesteppers for the incompressible analogue to this system with structure-preserving (SP) properties in the ideal case, alongside parameter-robust preconditioners. We show that these timesteppers can derive from a finite-element-in-time (FET) (and finite-element-in-space) interpretation. The benefits of this reformulation are discussed, including the derivation of timesteppers that are higher order in time, and the quantifiable dissipative SP properties in the non-ideal, resistive case.
        
        We discuss possible options for extending this FET approach to timesteppers for the compressible case.

        The kinetic corrections satisfy linearized Boltzmann equations. Using a Lénard--Bernstein collision operator, these take Fokker--Planck-like forms \cite{Fokker_1914, Planck_1917} wherein pseudo-particles in the numerical model obey the neoclassical transport equations, with particle-independent Brownian drift terms. This offers a rigorous methodology for incorporating collisions into the particle transport model, without coupling the equations of motions for each particle.
        
        Works by Chen, Chacón et al. \cite{Chen_Chacón_Barnes_2011, Chacón_Chen_Barnes_2013, Chen_Chacón_2014, Chen_Chacón_2015} have developed structure-preserving particle pushers for neoclassical transport in the Vlasov equations, derived from Crank--Nicolson integrators. We show these too can can derive from a FET interpretation, similarly offering potential extensions to higher-order-in-time particle pushers. The FET formulation is used also to consider how the stochastic drift terms can be incorporated into the pushers. Stochastic gyrokinetic expansions are also discussed.

        Different options for the numerical implementation of these schemes are considered.

        Due to the efficacy of FET in the development of SP timesteppers for both the fluid and kinetic component, we hope this approach will prove effective in the future for developing SP timesteppers for the full hybrid model. We hope this will give us the opportunity to incorporate previously inaccessible kinetic effects into the highly effective, modern, finite-element MHD models.
    \end{abstract}
    
    
    \newpage
    \tableofcontents
    
    
    \newpage
    \pagenumbering{arabic}
    %\linenumbers\renewcommand\thelinenumber{\color{black!50}\arabic{linenumber}}
            \input{0 - introduction/main.tex}
        \part{Research}
            \input{1 - low-noise PiC models/main.tex}
            \input{2 - kinetic component/main.tex}
            \input{3 - fluid component/main.tex}
            \input{4 - numerical implementation/main.tex}
        \part{Project Overview}
            \input{5 - research plan/main.tex}
            \input{6 - summary/main.tex}
    
    
    %\section{}
    \newpage
    \pagenumbering{gobble}
        \printbibliography


    \newpage
    \pagenumbering{roman}
    \appendix
        \part{Appendices}
            \input{8 - Hilbert complexes/main.tex}
            \input{9 - weak conservation proofs/main.tex}
\end{document}

            \documentclass[12pt, a4paper]{report}

\input{template/main.tex}

\title{\BA{Title in Progress...}}
\author{Boris Andrews}
\affil{Mathematical Institute, University of Oxford}
\date{\today}


\begin{document}
    \pagenumbering{gobble}
    \maketitle
    
    
    \begin{abstract}
        Magnetic confinement reactors---in particular tokamaks---offer one of the most promising options for achieving practical nuclear fusion, with the potential to provide virtually limitless, clean energy. The theoretical and numerical modeling of tokamak plasmas is simultaneously an essential component of effective reactor design, and a great research barrier. Tokamak operational conditions exhibit comparatively low Knudsen numbers. Kinetic effects, including kinetic waves and instabilities, Landau damping, bump-on-tail instabilities and more, are therefore highly influential in tokamak plasma dynamics. Purely fluid models are inherently incapable of capturing these effects, whereas the high dimensionality in purely kinetic models render them practically intractable for most relevant purposes.

        We consider a $\delta\!f$ decomposition model, with a macroscopic fluid background and microscopic kinetic correction, both fully coupled to each other. A similar manner of discretization is proposed to that used in the recent \texttt{STRUPHY} code \cite{Holderied_Possanner_Wang_2021, Holderied_2022, Li_et_al_2023} with a finite-element model for the background and a pseudo-particle/PiC model for the correction.

        The fluid background satisfies the full, non-linear, resistive, compressible, Hall MHD equations. \cite{Laakmann_Hu_Farrell_2022} introduces finite-element(-in-space) implicit timesteppers for the incompressible analogue to this system with structure-preserving (SP) properties in the ideal case, alongside parameter-robust preconditioners. We show that these timesteppers can derive from a finite-element-in-time (FET) (and finite-element-in-space) interpretation. The benefits of this reformulation are discussed, including the derivation of timesteppers that are higher order in time, and the quantifiable dissipative SP properties in the non-ideal, resistive case.
        
        We discuss possible options for extending this FET approach to timesteppers for the compressible case.

        The kinetic corrections satisfy linearized Boltzmann equations. Using a Lénard--Bernstein collision operator, these take Fokker--Planck-like forms \cite{Fokker_1914, Planck_1917} wherein pseudo-particles in the numerical model obey the neoclassical transport equations, with particle-independent Brownian drift terms. This offers a rigorous methodology for incorporating collisions into the particle transport model, without coupling the equations of motions for each particle.
        
        Works by Chen, Chacón et al. \cite{Chen_Chacón_Barnes_2011, Chacón_Chen_Barnes_2013, Chen_Chacón_2014, Chen_Chacón_2015} have developed structure-preserving particle pushers for neoclassical transport in the Vlasov equations, derived from Crank--Nicolson integrators. We show these too can can derive from a FET interpretation, similarly offering potential extensions to higher-order-in-time particle pushers. The FET formulation is used also to consider how the stochastic drift terms can be incorporated into the pushers. Stochastic gyrokinetic expansions are also discussed.

        Different options for the numerical implementation of these schemes are considered.

        Due to the efficacy of FET in the development of SP timesteppers for both the fluid and kinetic component, we hope this approach will prove effective in the future for developing SP timesteppers for the full hybrid model. We hope this will give us the opportunity to incorporate previously inaccessible kinetic effects into the highly effective, modern, finite-element MHD models.
    \end{abstract}
    
    
    \newpage
    \tableofcontents
    
    
    \newpage
    \pagenumbering{arabic}
    %\linenumbers\renewcommand\thelinenumber{\color{black!50}\arabic{linenumber}}
            \input{0 - introduction/main.tex}
        \part{Research}
            \input{1 - low-noise PiC models/main.tex}
            \input{2 - kinetic component/main.tex}
            \input{3 - fluid component/main.tex}
            \input{4 - numerical implementation/main.tex}
        \part{Project Overview}
            \input{5 - research plan/main.tex}
            \input{6 - summary/main.tex}
    
    
    %\section{}
    \newpage
    \pagenumbering{gobble}
        \printbibliography


    \newpage
    \pagenumbering{roman}
    \appendix
        \part{Appendices}
            \input{8 - Hilbert complexes/main.tex}
            \input{9 - weak conservation proofs/main.tex}
\end{document}

            \documentclass[12pt, a4paper]{report}

\input{template/main.tex}

\title{\BA{Title in Progress...}}
\author{Boris Andrews}
\affil{Mathematical Institute, University of Oxford}
\date{\today}


\begin{document}
    \pagenumbering{gobble}
    \maketitle
    
    
    \begin{abstract}
        Magnetic confinement reactors---in particular tokamaks---offer one of the most promising options for achieving practical nuclear fusion, with the potential to provide virtually limitless, clean energy. The theoretical and numerical modeling of tokamak plasmas is simultaneously an essential component of effective reactor design, and a great research barrier. Tokamak operational conditions exhibit comparatively low Knudsen numbers. Kinetic effects, including kinetic waves and instabilities, Landau damping, bump-on-tail instabilities and more, are therefore highly influential in tokamak plasma dynamics. Purely fluid models are inherently incapable of capturing these effects, whereas the high dimensionality in purely kinetic models render them practically intractable for most relevant purposes.

        We consider a $\delta\!f$ decomposition model, with a macroscopic fluid background and microscopic kinetic correction, both fully coupled to each other. A similar manner of discretization is proposed to that used in the recent \texttt{STRUPHY} code \cite{Holderied_Possanner_Wang_2021, Holderied_2022, Li_et_al_2023} with a finite-element model for the background and a pseudo-particle/PiC model for the correction.

        The fluid background satisfies the full, non-linear, resistive, compressible, Hall MHD equations. \cite{Laakmann_Hu_Farrell_2022} introduces finite-element(-in-space) implicit timesteppers for the incompressible analogue to this system with structure-preserving (SP) properties in the ideal case, alongside parameter-robust preconditioners. We show that these timesteppers can derive from a finite-element-in-time (FET) (and finite-element-in-space) interpretation. The benefits of this reformulation are discussed, including the derivation of timesteppers that are higher order in time, and the quantifiable dissipative SP properties in the non-ideal, resistive case.
        
        We discuss possible options for extending this FET approach to timesteppers for the compressible case.

        The kinetic corrections satisfy linearized Boltzmann equations. Using a Lénard--Bernstein collision operator, these take Fokker--Planck-like forms \cite{Fokker_1914, Planck_1917} wherein pseudo-particles in the numerical model obey the neoclassical transport equations, with particle-independent Brownian drift terms. This offers a rigorous methodology for incorporating collisions into the particle transport model, without coupling the equations of motions for each particle.
        
        Works by Chen, Chacón et al. \cite{Chen_Chacón_Barnes_2011, Chacón_Chen_Barnes_2013, Chen_Chacón_2014, Chen_Chacón_2015} have developed structure-preserving particle pushers for neoclassical transport in the Vlasov equations, derived from Crank--Nicolson integrators. We show these too can can derive from a FET interpretation, similarly offering potential extensions to higher-order-in-time particle pushers. The FET formulation is used also to consider how the stochastic drift terms can be incorporated into the pushers. Stochastic gyrokinetic expansions are also discussed.

        Different options for the numerical implementation of these schemes are considered.

        Due to the efficacy of FET in the development of SP timesteppers for both the fluid and kinetic component, we hope this approach will prove effective in the future for developing SP timesteppers for the full hybrid model. We hope this will give us the opportunity to incorporate previously inaccessible kinetic effects into the highly effective, modern, finite-element MHD models.
    \end{abstract}
    
    
    \newpage
    \tableofcontents
    
    
    \newpage
    \pagenumbering{arabic}
    %\linenumbers\renewcommand\thelinenumber{\color{black!50}\arabic{linenumber}}
            \input{0 - introduction/main.tex}
        \part{Research}
            \input{1 - low-noise PiC models/main.tex}
            \input{2 - kinetic component/main.tex}
            \input{3 - fluid component/main.tex}
            \input{4 - numerical implementation/main.tex}
        \part{Project Overview}
            \input{5 - research plan/main.tex}
            \input{6 - summary/main.tex}
    
    
    %\section{}
    \newpage
    \pagenumbering{gobble}
        \printbibliography


    \newpage
    \pagenumbering{roman}
    \appendix
        \part{Appendices}
            \input{8 - Hilbert complexes/main.tex}
            \input{9 - weak conservation proofs/main.tex}
\end{document}

        \part{Project Overview}
            \documentclass[12pt, a4paper]{report}

\input{template/main.tex}

\title{\BA{Title in Progress...}}
\author{Boris Andrews}
\affil{Mathematical Institute, University of Oxford}
\date{\today}


\begin{document}
    \pagenumbering{gobble}
    \maketitle
    
    
    \begin{abstract}
        Magnetic confinement reactors---in particular tokamaks---offer one of the most promising options for achieving practical nuclear fusion, with the potential to provide virtually limitless, clean energy. The theoretical and numerical modeling of tokamak plasmas is simultaneously an essential component of effective reactor design, and a great research barrier. Tokamak operational conditions exhibit comparatively low Knudsen numbers. Kinetic effects, including kinetic waves and instabilities, Landau damping, bump-on-tail instabilities and more, are therefore highly influential in tokamak plasma dynamics. Purely fluid models are inherently incapable of capturing these effects, whereas the high dimensionality in purely kinetic models render them practically intractable for most relevant purposes.

        We consider a $\delta\!f$ decomposition model, with a macroscopic fluid background and microscopic kinetic correction, both fully coupled to each other. A similar manner of discretization is proposed to that used in the recent \texttt{STRUPHY} code \cite{Holderied_Possanner_Wang_2021, Holderied_2022, Li_et_al_2023} with a finite-element model for the background and a pseudo-particle/PiC model for the correction.

        The fluid background satisfies the full, non-linear, resistive, compressible, Hall MHD equations. \cite{Laakmann_Hu_Farrell_2022} introduces finite-element(-in-space) implicit timesteppers for the incompressible analogue to this system with structure-preserving (SP) properties in the ideal case, alongside parameter-robust preconditioners. We show that these timesteppers can derive from a finite-element-in-time (FET) (and finite-element-in-space) interpretation. The benefits of this reformulation are discussed, including the derivation of timesteppers that are higher order in time, and the quantifiable dissipative SP properties in the non-ideal, resistive case.
        
        We discuss possible options for extending this FET approach to timesteppers for the compressible case.

        The kinetic corrections satisfy linearized Boltzmann equations. Using a Lénard--Bernstein collision operator, these take Fokker--Planck-like forms \cite{Fokker_1914, Planck_1917} wherein pseudo-particles in the numerical model obey the neoclassical transport equations, with particle-independent Brownian drift terms. This offers a rigorous methodology for incorporating collisions into the particle transport model, without coupling the equations of motions for each particle.
        
        Works by Chen, Chacón et al. \cite{Chen_Chacón_Barnes_2011, Chacón_Chen_Barnes_2013, Chen_Chacón_2014, Chen_Chacón_2015} have developed structure-preserving particle pushers for neoclassical transport in the Vlasov equations, derived from Crank--Nicolson integrators. We show these too can can derive from a FET interpretation, similarly offering potential extensions to higher-order-in-time particle pushers. The FET formulation is used also to consider how the stochastic drift terms can be incorporated into the pushers. Stochastic gyrokinetic expansions are also discussed.

        Different options for the numerical implementation of these schemes are considered.

        Due to the efficacy of FET in the development of SP timesteppers for both the fluid and kinetic component, we hope this approach will prove effective in the future for developing SP timesteppers for the full hybrid model. We hope this will give us the opportunity to incorporate previously inaccessible kinetic effects into the highly effective, modern, finite-element MHD models.
    \end{abstract}
    
    
    \newpage
    \tableofcontents
    
    
    \newpage
    \pagenumbering{arabic}
    %\linenumbers\renewcommand\thelinenumber{\color{black!50}\arabic{linenumber}}
            \input{0 - introduction/main.tex}
        \part{Research}
            \input{1 - low-noise PiC models/main.tex}
            \input{2 - kinetic component/main.tex}
            \input{3 - fluid component/main.tex}
            \input{4 - numerical implementation/main.tex}
        \part{Project Overview}
            \input{5 - research plan/main.tex}
            \input{6 - summary/main.tex}
    
    
    %\section{}
    \newpage
    \pagenumbering{gobble}
        \printbibliography


    \newpage
    \pagenumbering{roman}
    \appendix
        \part{Appendices}
            \input{8 - Hilbert complexes/main.tex}
            \input{9 - weak conservation proofs/main.tex}
\end{document}

            \documentclass[12pt, a4paper]{report}

\input{template/main.tex}

\title{\BA{Title in Progress...}}
\author{Boris Andrews}
\affil{Mathematical Institute, University of Oxford}
\date{\today}


\begin{document}
    \pagenumbering{gobble}
    \maketitle
    
    
    \begin{abstract}
        Magnetic confinement reactors---in particular tokamaks---offer one of the most promising options for achieving practical nuclear fusion, with the potential to provide virtually limitless, clean energy. The theoretical and numerical modeling of tokamak plasmas is simultaneously an essential component of effective reactor design, and a great research barrier. Tokamak operational conditions exhibit comparatively low Knudsen numbers. Kinetic effects, including kinetic waves and instabilities, Landau damping, bump-on-tail instabilities and more, are therefore highly influential in tokamak plasma dynamics. Purely fluid models are inherently incapable of capturing these effects, whereas the high dimensionality in purely kinetic models render them practically intractable for most relevant purposes.

        We consider a $\delta\!f$ decomposition model, with a macroscopic fluid background and microscopic kinetic correction, both fully coupled to each other. A similar manner of discretization is proposed to that used in the recent \texttt{STRUPHY} code \cite{Holderied_Possanner_Wang_2021, Holderied_2022, Li_et_al_2023} with a finite-element model for the background and a pseudo-particle/PiC model for the correction.

        The fluid background satisfies the full, non-linear, resistive, compressible, Hall MHD equations. \cite{Laakmann_Hu_Farrell_2022} introduces finite-element(-in-space) implicit timesteppers for the incompressible analogue to this system with structure-preserving (SP) properties in the ideal case, alongside parameter-robust preconditioners. We show that these timesteppers can derive from a finite-element-in-time (FET) (and finite-element-in-space) interpretation. The benefits of this reformulation are discussed, including the derivation of timesteppers that are higher order in time, and the quantifiable dissipative SP properties in the non-ideal, resistive case.
        
        We discuss possible options for extending this FET approach to timesteppers for the compressible case.

        The kinetic corrections satisfy linearized Boltzmann equations. Using a Lénard--Bernstein collision operator, these take Fokker--Planck-like forms \cite{Fokker_1914, Planck_1917} wherein pseudo-particles in the numerical model obey the neoclassical transport equations, with particle-independent Brownian drift terms. This offers a rigorous methodology for incorporating collisions into the particle transport model, without coupling the equations of motions for each particle.
        
        Works by Chen, Chacón et al. \cite{Chen_Chacón_Barnes_2011, Chacón_Chen_Barnes_2013, Chen_Chacón_2014, Chen_Chacón_2015} have developed structure-preserving particle pushers for neoclassical transport in the Vlasov equations, derived from Crank--Nicolson integrators. We show these too can can derive from a FET interpretation, similarly offering potential extensions to higher-order-in-time particle pushers. The FET formulation is used also to consider how the stochastic drift terms can be incorporated into the pushers. Stochastic gyrokinetic expansions are also discussed.

        Different options for the numerical implementation of these schemes are considered.

        Due to the efficacy of FET in the development of SP timesteppers for both the fluid and kinetic component, we hope this approach will prove effective in the future for developing SP timesteppers for the full hybrid model. We hope this will give us the opportunity to incorporate previously inaccessible kinetic effects into the highly effective, modern, finite-element MHD models.
    \end{abstract}
    
    
    \newpage
    \tableofcontents
    
    
    \newpage
    \pagenumbering{arabic}
    %\linenumbers\renewcommand\thelinenumber{\color{black!50}\arabic{linenumber}}
            \input{0 - introduction/main.tex}
        \part{Research}
            \input{1 - low-noise PiC models/main.tex}
            \input{2 - kinetic component/main.tex}
            \input{3 - fluid component/main.tex}
            \input{4 - numerical implementation/main.tex}
        \part{Project Overview}
            \input{5 - research plan/main.tex}
            \input{6 - summary/main.tex}
    
    
    %\section{}
    \newpage
    \pagenumbering{gobble}
        \printbibliography


    \newpage
    \pagenumbering{roman}
    \appendix
        \part{Appendices}
            \input{8 - Hilbert complexes/main.tex}
            \input{9 - weak conservation proofs/main.tex}
\end{document}

    
    
    %\section{}
    \newpage
    \pagenumbering{gobble}
        \printbibliography


    \newpage
    \pagenumbering{roman}
    \appendix
        \part{Appendices}
            \documentclass[12pt, a4paper]{report}

\input{template/main.tex}

\title{\BA{Title in Progress...}}
\author{Boris Andrews}
\affil{Mathematical Institute, University of Oxford}
\date{\today}


\begin{document}
    \pagenumbering{gobble}
    \maketitle
    
    
    \begin{abstract}
        Magnetic confinement reactors---in particular tokamaks---offer one of the most promising options for achieving practical nuclear fusion, with the potential to provide virtually limitless, clean energy. The theoretical and numerical modeling of tokamak plasmas is simultaneously an essential component of effective reactor design, and a great research barrier. Tokamak operational conditions exhibit comparatively low Knudsen numbers. Kinetic effects, including kinetic waves and instabilities, Landau damping, bump-on-tail instabilities and more, are therefore highly influential in tokamak plasma dynamics. Purely fluid models are inherently incapable of capturing these effects, whereas the high dimensionality in purely kinetic models render them practically intractable for most relevant purposes.

        We consider a $\delta\!f$ decomposition model, with a macroscopic fluid background and microscopic kinetic correction, both fully coupled to each other. A similar manner of discretization is proposed to that used in the recent \texttt{STRUPHY} code \cite{Holderied_Possanner_Wang_2021, Holderied_2022, Li_et_al_2023} with a finite-element model for the background and a pseudo-particle/PiC model for the correction.

        The fluid background satisfies the full, non-linear, resistive, compressible, Hall MHD equations. \cite{Laakmann_Hu_Farrell_2022} introduces finite-element(-in-space) implicit timesteppers for the incompressible analogue to this system with structure-preserving (SP) properties in the ideal case, alongside parameter-robust preconditioners. We show that these timesteppers can derive from a finite-element-in-time (FET) (and finite-element-in-space) interpretation. The benefits of this reformulation are discussed, including the derivation of timesteppers that are higher order in time, and the quantifiable dissipative SP properties in the non-ideal, resistive case.
        
        We discuss possible options for extending this FET approach to timesteppers for the compressible case.

        The kinetic corrections satisfy linearized Boltzmann equations. Using a Lénard--Bernstein collision operator, these take Fokker--Planck-like forms \cite{Fokker_1914, Planck_1917} wherein pseudo-particles in the numerical model obey the neoclassical transport equations, with particle-independent Brownian drift terms. This offers a rigorous methodology for incorporating collisions into the particle transport model, without coupling the equations of motions for each particle.
        
        Works by Chen, Chacón et al. \cite{Chen_Chacón_Barnes_2011, Chacón_Chen_Barnes_2013, Chen_Chacón_2014, Chen_Chacón_2015} have developed structure-preserving particle pushers for neoclassical transport in the Vlasov equations, derived from Crank--Nicolson integrators. We show these too can can derive from a FET interpretation, similarly offering potential extensions to higher-order-in-time particle pushers. The FET formulation is used also to consider how the stochastic drift terms can be incorporated into the pushers. Stochastic gyrokinetic expansions are also discussed.

        Different options for the numerical implementation of these schemes are considered.

        Due to the efficacy of FET in the development of SP timesteppers for both the fluid and kinetic component, we hope this approach will prove effective in the future for developing SP timesteppers for the full hybrid model. We hope this will give us the opportunity to incorporate previously inaccessible kinetic effects into the highly effective, modern, finite-element MHD models.
    \end{abstract}
    
    
    \newpage
    \tableofcontents
    
    
    \newpage
    \pagenumbering{arabic}
    %\linenumbers\renewcommand\thelinenumber{\color{black!50}\arabic{linenumber}}
            \input{0 - introduction/main.tex}
        \part{Research}
            \input{1 - low-noise PiC models/main.tex}
            \input{2 - kinetic component/main.tex}
            \input{3 - fluid component/main.tex}
            \input{4 - numerical implementation/main.tex}
        \part{Project Overview}
            \input{5 - research plan/main.tex}
            \input{6 - summary/main.tex}
    
    
    %\section{}
    \newpage
    \pagenumbering{gobble}
        \printbibliography


    \newpage
    \pagenumbering{roman}
    \appendix
        \part{Appendices}
            \input{8 - Hilbert complexes/main.tex}
            \input{9 - weak conservation proofs/main.tex}
\end{document}

            \documentclass[12pt, a4paper]{report}

\input{template/main.tex}

\title{\BA{Title in Progress...}}
\author{Boris Andrews}
\affil{Mathematical Institute, University of Oxford}
\date{\today}


\begin{document}
    \pagenumbering{gobble}
    \maketitle
    
    
    \begin{abstract}
        Magnetic confinement reactors---in particular tokamaks---offer one of the most promising options for achieving practical nuclear fusion, with the potential to provide virtually limitless, clean energy. The theoretical and numerical modeling of tokamak plasmas is simultaneously an essential component of effective reactor design, and a great research barrier. Tokamak operational conditions exhibit comparatively low Knudsen numbers. Kinetic effects, including kinetic waves and instabilities, Landau damping, bump-on-tail instabilities and more, are therefore highly influential in tokamak plasma dynamics. Purely fluid models are inherently incapable of capturing these effects, whereas the high dimensionality in purely kinetic models render them practically intractable for most relevant purposes.

        We consider a $\delta\!f$ decomposition model, with a macroscopic fluid background and microscopic kinetic correction, both fully coupled to each other. A similar manner of discretization is proposed to that used in the recent \texttt{STRUPHY} code \cite{Holderied_Possanner_Wang_2021, Holderied_2022, Li_et_al_2023} with a finite-element model for the background and a pseudo-particle/PiC model for the correction.

        The fluid background satisfies the full, non-linear, resistive, compressible, Hall MHD equations. \cite{Laakmann_Hu_Farrell_2022} introduces finite-element(-in-space) implicit timesteppers for the incompressible analogue to this system with structure-preserving (SP) properties in the ideal case, alongside parameter-robust preconditioners. We show that these timesteppers can derive from a finite-element-in-time (FET) (and finite-element-in-space) interpretation. The benefits of this reformulation are discussed, including the derivation of timesteppers that are higher order in time, and the quantifiable dissipative SP properties in the non-ideal, resistive case.
        
        We discuss possible options for extending this FET approach to timesteppers for the compressible case.

        The kinetic corrections satisfy linearized Boltzmann equations. Using a Lénard--Bernstein collision operator, these take Fokker--Planck-like forms \cite{Fokker_1914, Planck_1917} wherein pseudo-particles in the numerical model obey the neoclassical transport equations, with particle-independent Brownian drift terms. This offers a rigorous methodology for incorporating collisions into the particle transport model, without coupling the equations of motions for each particle.
        
        Works by Chen, Chacón et al. \cite{Chen_Chacón_Barnes_2011, Chacón_Chen_Barnes_2013, Chen_Chacón_2014, Chen_Chacón_2015} have developed structure-preserving particle pushers for neoclassical transport in the Vlasov equations, derived from Crank--Nicolson integrators. We show these too can can derive from a FET interpretation, similarly offering potential extensions to higher-order-in-time particle pushers. The FET formulation is used also to consider how the stochastic drift terms can be incorporated into the pushers. Stochastic gyrokinetic expansions are also discussed.

        Different options for the numerical implementation of these schemes are considered.

        Due to the efficacy of FET in the development of SP timesteppers for both the fluid and kinetic component, we hope this approach will prove effective in the future for developing SP timesteppers for the full hybrid model. We hope this will give us the opportunity to incorporate previously inaccessible kinetic effects into the highly effective, modern, finite-element MHD models.
    \end{abstract}
    
    
    \newpage
    \tableofcontents
    
    
    \newpage
    \pagenumbering{arabic}
    %\linenumbers\renewcommand\thelinenumber{\color{black!50}\arabic{linenumber}}
            \input{0 - introduction/main.tex}
        \part{Research}
            \input{1 - low-noise PiC models/main.tex}
            \input{2 - kinetic component/main.tex}
            \input{3 - fluid component/main.tex}
            \input{4 - numerical implementation/main.tex}
        \part{Project Overview}
            \input{5 - research plan/main.tex}
            \input{6 - summary/main.tex}
    
    
    %\section{}
    \newpage
    \pagenumbering{gobble}
        \printbibliography


    \newpage
    \pagenumbering{roman}
    \appendix
        \part{Appendices}
            \input{8 - Hilbert complexes/main.tex}
            \input{9 - weak conservation proofs/main.tex}
\end{document}

\end{document}

            \documentclass[12pt, a4paper]{report}

\documentclass[12pt, a4paper]{report}

\input{template/main.tex}

\title{\BA{Title in Progress...}}
\author{Boris Andrews}
\affil{Mathematical Institute, University of Oxford}
\date{\today}


\begin{document}
    \pagenumbering{gobble}
    \maketitle
    
    
    \begin{abstract}
        Magnetic confinement reactors---in particular tokamaks---offer one of the most promising options for achieving practical nuclear fusion, with the potential to provide virtually limitless, clean energy. The theoretical and numerical modeling of tokamak plasmas is simultaneously an essential component of effective reactor design, and a great research barrier. Tokamak operational conditions exhibit comparatively low Knudsen numbers. Kinetic effects, including kinetic waves and instabilities, Landau damping, bump-on-tail instabilities and more, are therefore highly influential in tokamak plasma dynamics. Purely fluid models are inherently incapable of capturing these effects, whereas the high dimensionality in purely kinetic models render them practically intractable for most relevant purposes.

        We consider a $\delta\!f$ decomposition model, with a macroscopic fluid background and microscopic kinetic correction, both fully coupled to each other. A similar manner of discretization is proposed to that used in the recent \texttt{STRUPHY} code \cite{Holderied_Possanner_Wang_2021, Holderied_2022, Li_et_al_2023} with a finite-element model for the background and a pseudo-particle/PiC model for the correction.

        The fluid background satisfies the full, non-linear, resistive, compressible, Hall MHD equations. \cite{Laakmann_Hu_Farrell_2022} introduces finite-element(-in-space) implicit timesteppers for the incompressible analogue to this system with structure-preserving (SP) properties in the ideal case, alongside parameter-robust preconditioners. We show that these timesteppers can derive from a finite-element-in-time (FET) (and finite-element-in-space) interpretation. The benefits of this reformulation are discussed, including the derivation of timesteppers that are higher order in time, and the quantifiable dissipative SP properties in the non-ideal, resistive case.
        
        We discuss possible options for extending this FET approach to timesteppers for the compressible case.

        The kinetic corrections satisfy linearized Boltzmann equations. Using a Lénard--Bernstein collision operator, these take Fokker--Planck-like forms \cite{Fokker_1914, Planck_1917} wherein pseudo-particles in the numerical model obey the neoclassical transport equations, with particle-independent Brownian drift terms. This offers a rigorous methodology for incorporating collisions into the particle transport model, without coupling the equations of motions for each particle.
        
        Works by Chen, Chacón et al. \cite{Chen_Chacón_Barnes_2011, Chacón_Chen_Barnes_2013, Chen_Chacón_2014, Chen_Chacón_2015} have developed structure-preserving particle pushers for neoclassical transport in the Vlasov equations, derived from Crank--Nicolson integrators. We show these too can can derive from a FET interpretation, similarly offering potential extensions to higher-order-in-time particle pushers. The FET formulation is used also to consider how the stochastic drift terms can be incorporated into the pushers. Stochastic gyrokinetic expansions are also discussed.

        Different options for the numerical implementation of these schemes are considered.

        Due to the efficacy of FET in the development of SP timesteppers for both the fluid and kinetic component, we hope this approach will prove effective in the future for developing SP timesteppers for the full hybrid model. We hope this will give us the opportunity to incorporate previously inaccessible kinetic effects into the highly effective, modern, finite-element MHD models.
    \end{abstract}
    
    
    \newpage
    \tableofcontents
    
    
    \newpage
    \pagenumbering{arabic}
    %\linenumbers\renewcommand\thelinenumber{\color{black!50}\arabic{linenumber}}
            \input{0 - introduction/main.tex}
        \part{Research}
            \input{1 - low-noise PiC models/main.tex}
            \input{2 - kinetic component/main.tex}
            \input{3 - fluid component/main.tex}
            \input{4 - numerical implementation/main.tex}
        \part{Project Overview}
            \input{5 - research plan/main.tex}
            \input{6 - summary/main.tex}
    
    
    %\section{}
    \newpage
    \pagenumbering{gobble}
        \printbibliography


    \newpage
    \pagenumbering{roman}
    \appendix
        \part{Appendices}
            \input{8 - Hilbert complexes/main.tex}
            \input{9 - weak conservation proofs/main.tex}
\end{document}


\title{\BA{Title in Progress...}}
\author{Boris Andrews}
\affil{Mathematical Institute, University of Oxford}
\date{\today}


\begin{document}
    \pagenumbering{gobble}
    \maketitle
    
    
    \begin{abstract}
        Magnetic confinement reactors---in particular tokamaks---offer one of the most promising options for achieving practical nuclear fusion, with the potential to provide virtually limitless, clean energy. The theoretical and numerical modeling of tokamak plasmas is simultaneously an essential component of effective reactor design, and a great research barrier. Tokamak operational conditions exhibit comparatively low Knudsen numbers. Kinetic effects, including kinetic waves and instabilities, Landau damping, bump-on-tail instabilities and more, are therefore highly influential in tokamak plasma dynamics. Purely fluid models are inherently incapable of capturing these effects, whereas the high dimensionality in purely kinetic models render them practically intractable for most relevant purposes.

        We consider a $\delta\!f$ decomposition model, with a macroscopic fluid background and microscopic kinetic correction, both fully coupled to each other. A similar manner of discretization is proposed to that used in the recent \texttt{STRUPHY} code \cite{Holderied_Possanner_Wang_2021, Holderied_2022, Li_et_al_2023} with a finite-element model for the background and a pseudo-particle/PiC model for the correction.

        The fluid background satisfies the full, non-linear, resistive, compressible, Hall MHD equations. \cite{Laakmann_Hu_Farrell_2022} introduces finite-element(-in-space) implicit timesteppers for the incompressible analogue to this system with structure-preserving (SP) properties in the ideal case, alongside parameter-robust preconditioners. We show that these timesteppers can derive from a finite-element-in-time (FET) (and finite-element-in-space) interpretation. The benefits of this reformulation are discussed, including the derivation of timesteppers that are higher order in time, and the quantifiable dissipative SP properties in the non-ideal, resistive case.
        
        We discuss possible options for extending this FET approach to timesteppers for the compressible case.

        The kinetic corrections satisfy linearized Boltzmann equations. Using a Lénard--Bernstein collision operator, these take Fokker--Planck-like forms \cite{Fokker_1914, Planck_1917} wherein pseudo-particles in the numerical model obey the neoclassical transport equations, with particle-independent Brownian drift terms. This offers a rigorous methodology for incorporating collisions into the particle transport model, without coupling the equations of motions for each particle.
        
        Works by Chen, Chacón et al. \cite{Chen_Chacón_Barnes_2011, Chacón_Chen_Barnes_2013, Chen_Chacón_2014, Chen_Chacón_2015} have developed structure-preserving particle pushers for neoclassical transport in the Vlasov equations, derived from Crank--Nicolson integrators. We show these too can can derive from a FET interpretation, similarly offering potential extensions to higher-order-in-time particle pushers. The FET formulation is used also to consider how the stochastic drift terms can be incorporated into the pushers. Stochastic gyrokinetic expansions are also discussed.

        Different options for the numerical implementation of these schemes are considered.

        Due to the efficacy of FET in the development of SP timesteppers for both the fluid and kinetic component, we hope this approach will prove effective in the future for developing SP timesteppers for the full hybrid model. We hope this will give us the opportunity to incorporate previously inaccessible kinetic effects into the highly effective, modern, finite-element MHD models.
    \end{abstract}
    
    
    \newpage
    \tableofcontents
    
    
    \newpage
    \pagenumbering{arabic}
    %\linenumbers\renewcommand\thelinenumber{\color{black!50}\arabic{linenumber}}
            \documentclass[12pt, a4paper]{report}

\input{template/main.tex}

\title{\BA{Title in Progress...}}
\author{Boris Andrews}
\affil{Mathematical Institute, University of Oxford}
\date{\today}


\begin{document}
    \pagenumbering{gobble}
    \maketitle
    
    
    \begin{abstract}
        Magnetic confinement reactors---in particular tokamaks---offer one of the most promising options for achieving practical nuclear fusion, with the potential to provide virtually limitless, clean energy. The theoretical and numerical modeling of tokamak plasmas is simultaneously an essential component of effective reactor design, and a great research barrier. Tokamak operational conditions exhibit comparatively low Knudsen numbers. Kinetic effects, including kinetic waves and instabilities, Landau damping, bump-on-tail instabilities and more, are therefore highly influential in tokamak plasma dynamics. Purely fluid models are inherently incapable of capturing these effects, whereas the high dimensionality in purely kinetic models render them practically intractable for most relevant purposes.

        We consider a $\delta\!f$ decomposition model, with a macroscopic fluid background and microscopic kinetic correction, both fully coupled to each other. A similar manner of discretization is proposed to that used in the recent \texttt{STRUPHY} code \cite{Holderied_Possanner_Wang_2021, Holderied_2022, Li_et_al_2023} with a finite-element model for the background and a pseudo-particle/PiC model for the correction.

        The fluid background satisfies the full, non-linear, resistive, compressible, Hall MHD equations. \cite{Laakmann_Hu_Farrell_2022} introduces finite-element(-in-space) implicit timesteppers for the incompressible analogue to this system with structure-preserving (SP) properties in the ideal case, alongside parameter-robust preconditioners. We show that these timesteppers can derive from a finite-element-in-time (FET) (and finite-element-in-space) interpretation. The benefits of this reformulation are discussed, including the derivation of timesteppers that are higher order in time, and the quantifiable dissipative SP properties in the non-ideal, resistive case.
        
        We discuss possible options for extending this FET approach to timesteppers for the compressible case.

        The kinetic corrections satisfy linearized Boltzmann equations. Using a Lénard--Bernstein collision operator, these take Fokker--Planck-like forms \cite{Fokker_1914, Planck_1917} wherein pseudo-particles in the numerical model obey the neoclassical transport equations, with particle-independent Brownian drift terms. This offers a rigorous methodology for incorporating collisions into the particle transport model, without coupling the equations of motions for each particle.
        
        Works by Chen, Chacón et al. \cite{Chen_Chacón_Barnes_2011, Chacón_Chen_Barnes_2013, Chen_Chacón_2014, Chen_Chacón_2015} have developed structure-preserving particle pushers for neoclassical transport in the Vlasov equations, derived from Crank--Nicolson integrators. We show these too can can derive from a FET interpretation, similarly offering potential extensions to higher-order-in-time particle pushers. The FET formulation is used also to consider how the stochastic drift terms can be incorporated into the pushers. Stochastic gyrokinetic expansions are also discussed.

        Different options for the numerical implementation of these schemes are considered.

        Due to the efficacy of FET in the development of SP timesteppers for both the fluid and kinetic component, we hope this approach will prove effective in the future for developing SP timesteppers for the full hybrid model. We hope this will give us the opportunity to incorporate previously inaccessible kinetic effects into the highly effective, modern, finite-element MHD models.
    \end{abstract}
    
    
    \newpage
    \tableofcontents
    
    
    \newpage
    \pagenumbering{arabic}
    %\linenumbers\renewcommand\thelinenumber{\color{black!50}\arabic{linenumber}}
            \input{0 - introduction/main.tex}
        \part{Research}
            \input{1 - low-noise PiC models/main.tex}
            \input{2 - kinetic component/main.tex}
            \input{3 - fluid component/main.tex}
            \input{4 - numerical implementation/main.tex}
        \part{Project Overview}
            \input{5 - research plan/main.tex}
            \input{6 - summary/main.tex}
    
    
    %\section{}
    \newpage
    \pagenumbering{gobble}
        \printbibliography


    \newpage
    \pagenumbering{roman}
    \appendix
        \part{Appendices}
            \input{8 - Hilbert complexes/main.tex}
            \input{9 - weak conservation proofs/main.tex}
\end{document}

        \part{Research}
            \documentclass[12pt, a4paper]{report}

\input{template/main.tex}

\title{\BA{Title in Progress...}}
\author{Boris Andrews}
\affil{Mathematical Institute, University of Oxford}
\date{\today}


\begin{document}
    \pagenumbering{gobble}
    \maketitle
    
    
    \begin{abstract}
        Magnetic confinement reactors---in particular tokamaks---offer one of the most promising options for achieving practical nuclear fusion, with the potential to provide virtually limitless, clean energy. The theoretical and numerical modeling of tokamak plasmas is simultaneously an essential component of effective reactor design, and a great research barrier. Tokamak operational conditions exhibit comparatively low Knudsen numbers. Kinetic effects, including kinetic waves and instabilities, Landau damping, bump-on-tail instabilities and more, are therefore highly influential in tokamak plasma dynamics. Purely fluid models are inherently incapable of capturing these effects, whereas the high dimensionality in purely kinetic models render them practically intractable for most relevant purposes.

        We consider a $\delta\!f$ decomposition model, with a macroscopic fluid background and microscopic kinetic correction, both fully coupled to each other. A similar manner of discretization is proposed to that used in the recent \texttt{STRUPHY} code \cite{Holderied_Possanner_Wang_2021, Holderied_2022, Li_et_al_2023} with a finite-element model for the background and a pseudo-particle/PiC model for the correction.

        The fluid background satisfies the full, non-linear, resistive, compressible, Hall MHD equations. \cite{Laakmann_Hu_Farrell_2022} introduces finite-element(-in-space) implicit timesteppers for the incompressible analogue to this system with structure-preserving (SP) properties in the ideal case, alongside parameter-robust preconditioners. We show that these timesteppers can derive from a finite-element-in-time (FET) (and finite-element-in-space) interpretation. The benefits of this reformulation are discussed, including the derivation of timesteppers that are higher order in time, and the quantifiable dissipative SP properties in the non-ideal, resistive case.
        
        We discuss possible options for extending this FET approach to timesteppers for the compressible case.

        The kinetic corrections satisfy linearized Boltzmann equations. Using a Lénard--Bernstein collision operator, these take Fokker--Planck-like forms \cite{Fokker_1914, Planck_1917} wherein pseudo-particles in the numerical model obey the neoclassical transport equations, with particle-independent Brownian drift terms. This offers a rigorous methodology for incorporating collisions into the particle transport model, without coupling the equations of motions for each particle.
        
        Works by Chen, Chacón et al. \cite{Chen_Chacón_Barnes_2011, Chacón_Chen_Barnes_2013, Chen_Chacón_2014, Chen_Chacón_2015} have developed structure-preserving particle pushers for neoclassical transport in the Vlasov equations, derived from Crank--Nicolson integrators. We show these too can can derive from a FET interpretation, similarly offering potential extensions to higher-order-in-time particle pushers. The FET formulation is used also to consider how the stochastic drift terms can be incorporated into the pushers. Stochastic gyrokinetic expansions are also discussed.

        Different options for the numerical implementation of these schemes are considered.

        Due to the efficacy of FET in the development of SP timesteppers for both the fluid and kinetic component, we hope this approach will prove effective in the future for developing SP timesteppers for the full hybrid model. We hope this will give us the opportunity to incorporate previously inaccessible kinetic effects into the highly effective, modern, finite-element MHD models.
    \end{abstract}
    
    
    \newpage
    \tableofcontents
    
    
    \newpage
    \pagenumbering{arabic}
    %\linenumbers\renewcommand\thelinenumber{\color{black!50}\arabic{linenumber}}
            \input{0 - introduction/main.tex}
        \part{Research}
            \input{1 - low-noise PiC models/main.tex}
            \input{2 - kinetic component/main.tex}
            \input{3 - fluid component/main.tex}
            \input{4 - numerical implementation/main.tex}
        \part{Project Overview}
            \input{5 - research plan/main.tex}
            \input{6 - summary/main.tex}
    
    
    %\section{}
    \newpage
    \pagenumbering{gobble}
        \printbibliography


    \newpage
    \pagenumbering{roman}
    \appendix
        \part{Appendices}
            \input{8 - Hilbert complexes/main.tex}
            \input{9 - weak conservation proofs/main.tex}
\end{document}

            \documentclass[12pt, a4paper]{report}

\input{template/main.tex}

\title{\BA{Title in Progress...}}
\author{Boris Andrews}
\affil{Mathematical Institute, University of Oxford}
\date{\today}


\begin{document}
    \pagenumbering{gobble}
    \maketitle
    
    
    \begin{abstract}
        Magnetic confinement reactors---in particular tokamaks---offer one of the most promising options for achieving practical nuclear fusion, with the potential to provide virtually limitless, clean energy. The theoretical and numerical modeling of tokamak plasmas is simultaneously an essential component of effective reactor design, and a great research barrier. Tokamak operational conditions exhibit comparatively low Knudsen numbers. Kinetic effects, including kinetic waves and instabilities, Landau damping, bump-on-tail instabilities and more, are therefore highly influential in tokamak plasma dynamics. Purely fluid models are inherently incapable of capturing these effects, whereas the high dimensionality in purely kinetic models render them practically intractable for most relevant purposes.

        We consider a $\delta\!f$ decomposition model, with a macroscopic fluid background and microscopic kinetic correction, both fully coupled to each other. A similar manner of discretization is proposed to that used in the recent \texttt{STRUPHY} code \cite{Holderied_Possanner_Wang_2021, Holderied_2022, Li_et_al_2023} with a finite-element model for the background and a pseudo-particle/PiC model for the correction.

        The fluid background satisfies the full, non-linear, resistive, compressible, Hall MHD equations. \cite{Laakmann_Hu_Farrell_2022} introduces finite-element(-in-space) implicit timesteppers for the incompressible analogue to this system with structure-preserving (SP) properties in the ideal case, alongside parameter-robust preconditioners. We show that these timesteppers can derive from a finite-element-in-time (FET) (and finite-element-in-space) interpretation. The benefits of this reformulation are discussed, including the derivation of timesteppers that are higher order in time, and the quantifiable dissipative SP properties in the non-ideal, resistive case.
        
        We discuss possible options for extending this FET approach to timesteppers for the compressible case.

        The kinetic corrections satisfy linearized Boltzmann equations. Using a Lénard--Bernstein collision operator, these take Fokker--Planck-like forms \cite{Fokker_1914, Planck_1917} wherein pseudo-particles in the numerical model obey the neoclassical transport equations, with particle-independent Brownian drift terms. This offers a rigorous methodology for incorporating collisions into the particle transport model, without coupling the equations of motions for each particle.
        
        Works by Chen, Chacón et al. \cite{Chen_Chacón_Barnes_2011, Chacón_Chen_Barnes_2013, Chen_Chacón_2014, Chen_Chacón_2015} have developed structure-preserving particle pushers for neoclassical transport in the Vlasov equations, derived from Crank--Nicolson integrators. We show these too can can derive from a FET interpretation, similarly offering potential extensions to higher-order-in-time particle pushers. The FET formulation is used also to consider how the stochastic drift terms can be incorporated into the pushers. Stochastic gyrokinetic expansions are also discussed.

        Different options for the numerical implementation of these schemes are considered.

        Due to the efficacy of FET in the development of SP timesteppers for both the fluid and kinetic component, we hope this approach will prove effective in the future for developing SP timesteppers for the full hybrid model. We hope this will give us the opportunity to incorporate previously inaccessible kinetic effects into the highly effective, modern, finite-element MHD models.
    \end{abstract}
    
    
    \newpage
    \tableofcontents
    
    
    \newpage
    \pagenumbering{arabic}
    %\linenumbers\renewcommand\thelinenumber{\color{black!50}\arabic{linenumber}}
            \input{0 - introduction/main.tex}
        \part{Research}
            \input{1 - low-noise PiC models/main.tex}
            \input{2 - kinetic component/main.tex}
            \input{3 - fluid component/main.tex}
            \input{4 - numerical implementation/main.tex}
        \part{Project Overview}
            \input{5 - research plan/main.tex}
            \input{6 - summary/main.tex}
    
    
    %\section{}
    \newpage
    \pagenumbering{gobble}
        \printbibliography


    \newpage
    \pagenumbering{roman}
    \appendix
        \part{Appendices}
            \input{8 - Hilbert complexes/main.tex}
            \input{9 - weak conservation proofs/main.tex}
\end{document}

            \documentclass[12pt, a4paper]{report}

\input{template/main.tex}

\title{\BA{Title in Progress...}}
\author{Boris Andrews}
\affil{Mathematical Institute, University of Oxford}
\date{\today}


\begin{document}
    \pagenumbering{gobble}
    \maketitle
    
    
    \begin{abstract}
        Magnetic confinement reactors---in particular tokamaks---offer one of the most promising options for achieving practical nuclear fusion, with the potential to provide virtually limitless, clean energy. The theoretical and numerical modeling of tokamak plasmas is simultaneously an essential component of effective reactor design, and a great research barrier. Tokamak operational conditions exhibit comparatively low Knudsen numbers. Kinetic effects, including kinetic waves and instabilities, Landau damping, bump-on-tail instabilities and more, are therefore highly influential in tokamak plasma dynamics. Purely fluid models are inherently incapable of capturing these effects, whereas the high dimensionality in purely kinetic models render them practically intractable for most relevant purposes.

        We consider a $\delta\!f$ decomposition model, with a macroscopic fluid background and microscopic kinetic correction, both fully coupled to each other. A similar manner of discretization is proposed to that used in the recent \texttt{STRUPHY} code \cite{Holderied_Possanner_Wang_2021, Holderied_2022, Li_et_al_2023} with a finite-element model for the background and a pseudo-particle/PiC model for the correction.

        The fluid background satisfies the full, non-linear, resistive, compressible, Hall MHD equations. \cite{Laakmann_Hu_Farrell_2022} introduces finite-element(-in-space) implicit timesteppers for the incompressible analogue to this system with structure-preserving (SP) properties in the ideal case, alongside parameter-robust preconditioners. We show that these timesteppers can derive from a finite-element-in-time (FET) (and finite-element-in-space) interpretation. The benefits of this reformulation are discussed, including the derivation of timesteppers that are higher order in time, and the quantifiable dissipative SP properties in the non-ideal, resistive case.
        
        We discuss possible options for extending this FET approach to timesteppers for the compressible case.

        The kinetic corrections satisfy linearized Boltzmann equations. Using a Lénard--Bernstein collision operator, these take Fokker--Planck-like forms \cite{Fokker_1914, Planck_1917} wherein pseudo-particles in the numerical model obey the neoclassical transport equations, with particle-independent Brownian drift terms. This offers a rigorous methodology for incorporating collisions into the particle transport model, without coupling the equations of motions for each particle.
        
        Works by Chen, Chacón et al. \cite{Chen_Chacón_Barnes_2011, Chacón_Chen_Barnes_2013, Chen_Chacón_2014, Chen_Chacón_2015} have developed structure-preserving particle pushers for neoclassical transport in the Vlasov equations, derived from Crank--Nicolson integrators. We show these too can can derive from a FET interpretation, similarly offering potential extensions to higher-order-in-time particle pushers. The FET formulation is used also to consider how the stochastic drift terms can be incorporated into the pushers. Stochastic gyrokinetic expansions are also discussed.

        Different options for the numerical implementation of these schemes are considered.

        Due to the efficacy of FET in the development of SP timesteppers for both the fluid and kinetic component, we hope this approach will prove effective in the future for developing SP timesteppers for the full hybrid model. We hope this will give us the opportunity to incorporate previously inaccessible kinetic effects into the highly effective, modern, finite-element MHD models.
    \end{abstract}
    
    
    \newpage
    \tableofcontents
    
    
    \newpage
    \pagenumbering{arabic}
    %\linenumbers\renewcommand\thelinenumber{\color{black!50}\arabic{linenumber}}
            \input{0 - introduction/main.tex}
        \part{Research}
            \input{1 - low-noise PiC models/main.tex}
            \input{2 - kinetic component/main.tex}
            \input{3 - fluid component/main.tex}
            \input{4 - numerical implementation/main.tex}
        \part{Project Overview}
            \input{5 - research plan/main.tex}
            \input{6 - summary/main.tex}
    
    
    %\section{}
    \newpage
    \pagenumbering{gobble}
        \printbibliography


    \newpage
    \pagenumbering{roman}
    \appendix
        \part{Appendices}
            \input{8 - Hilbert complexes/main.tex}
            \input{9 - weak conservation proofs/main.tex}
\end{document}

            \documentclass[12pt, a4paper]{report}

\input{template/main.tex}

\title{\BA{Title in Progress...}}
\author{Boris Andrews}
\affil{Mathematical Institute, University of Oxford}
\date{\today}


\begin{document}
    \pagenumbering{gobble}
    \maketitle
    
    
    \begin{abstract}
        Magnetic confinement reactors---in particular tokamaks---offer one of the most promising options for achieving practical nuclear fusion, with the potential to provide virtually limitless, clean energy. The theoretical and numerical modeling of tokamak plasmas is simultaneously an essential component of effective reactor design, and a great research barrier. Tokamak operational conditions exhibit comparatively low Knudsen numbers. Kinetic effects, including kinetic waves and instabilities, Landau damping, bump-on-tail instabilities and more, are therefore highly influential in tokamak plasma dynamics. Purely fluid models are inherently incapable of capturing these effects, whereas the high dimensionality in purely kinetic models render them practically intractable for most relevant purposes.

        We consider a $\delta\!f$ decomposition model, with a macroscopic fluid background and microscopic kinetic correction, both fully coupled to each other. A similar manner of discretization is proposed to that used in the recent \texttt{STRUPHY} code \cite{Holderied_Possanner_Wang_2021, Holderied_2022, Li_et_al_2023} with a finite-element model for the background and a pseudo-particle/PiC model for the correction.

        The fluid background satisfies the full, non-linear, resistive, compressible, Hall MHD equations. \cite{Laakmann_Hu_Farrell_2022} introduces finite-element(-in-space) implicit timesteppers for the incompressible analogue to this system with structure-preserving (SP) properties in the ideal case, alongside parameter-robust preconditioners. We show that these timesteppers can derive from a finite-element-in-time (FET) (and finite-element-in-space) interpretation. The benefits of this reformulation are discussed, including the derivation of timesteppers that are higher order in time, and the quantifiable dissipative SP properties in the non-ideal, resistive case.
        
        We discuss possible options for extending this FET approach to timesteppers for the compressible case.

        The kinetic corrections satisfy linearized Boltzmann equations. Using a Lénard--Bernstein collision operator, these take Fokker--Planck-like forms \cite{Fokker_1914, Planck_1917} wherein pseudo-particles in the numerical model obey the neoclassical transport equations, with particle-independent Brownian drift terms. This offers a rigorous methodology for incorporating collisions into the particle transport model, without coupling the equations of motions for each particle.
        
        Works by Chen, Chacón et al. \cite{Chen_Chacón_Barnes_2011, Chacón_Chen_Barnes_2013, Chen_Chacón_2014, Chen_Chacón_2015} have developed structure-preserving particle pushers for neoclassical transport in the Vlasov equations, derived from Crank--Nicolson integrators. We show these too can can derive from a FET interpretation, similarly offering potential extensions to higher-order-in-time particle pushers. The FET formulation is used also to consider how the stochastic drift terms can be incorporated into the pushers. Stochastic gyrokinetic expansions are also discussed.

        Different options for the numerical implementation of these schemes are considered.

        Due to the efficacy of FET in the development of SP timesteppers for both the fluid and kinetic component, we hope this approach will prove effective in the future for developing SP timesteppers for the full hybrid model. We hope this will give us the opportunity to incorporate previously inaccessible kinetic effects into the highly effective, modern, finite-element MHD models.
    \end{abstract}
    
    
    \newpage
    \tableofcontents
    
    
    \newpage
    \pagenumbering{arabic}
    %\linenumbers\renewcommand\thelinenumber{\color{black!50}\arabic{linenumber}}
            \input{0 - introduction/main.tex}
        \part{Research}
            \input{1 - low-noise PiC models/main.tex}
            \input{2 - kinetic component/main.tex}
            \input{3 - fluid component/main.tex}
            \input{4 - numerical implementation/main.tex}
        \part{Project Overview}
            \input{5 - research plan/main.tex}
            \input{6 - summary/main.tex}
    
    
    %\section{}
    \newpage
    \pagenumbering{gobble}
        \printbibliography


    \newpage
    \pagenumbering{roman}
    \appendix
        \part{Appendices}
            \input{8 - Hilbert complexes/main.tex}
            \input{9 - weak conservation proofs/main.tex}
\end{document}

        \part{Project Overview}
            \documentclass[12pt, a4paper]{report}

\input{template/main.tex}

\title{\BA{Title in Progress...}}
\author{Boris Andrews}
\affil{Mathematical Institute, University of Oxford}
\date{\today}


\begin{document}
    \pagenumbering{gobble}
    \maketitle
    
    
    \begin{abstract}
        Magnetic confinement reactors---in particular tokamaks---offer one of the most promising options for achieving practical nuclear fusion, with the potential to provide virtually limitless, clean energy. The theoretical and numerical modeling of tokamak plasmas is simultaneously an essential component of effective reactor design, and a great research barrier. Tokamak operational conditions exhibit comparatively low Knudsen numbers. Kinetic effects, including kinetic waves and instabilities, Landau damping, bump-on-tail instabilities and more, are therefore highly influential in tokamak plasma dynamics. Purely fluid models are inherently incapable of capturing these effects, whereas the high dimensionality in purely kinetic models render them practically intractable for most relevant purposes.

        We consider a $\delta\!f$ decomposition model, with a macroscopic fluid background and microscopic kinetic correction, both fully coupled to each other. A similar manner of discretization is proposed to that used in the recent \texttt{STRUPHY} code \cite{Holderied_Possanner_Wang_2021, Holderied_2022, Li_et_al_2023} with a finite-element model for the background and a pseudo-particle/PiC model for the correction.

        The fluid background satisfies the full, non-linear, resistive, compressible, Hall MHD equations. \cite{Laakmann_Hu_Farrell_2022} introduces finite-element(-in-space) implicit timesteppers for the incompressible analogue to this system with structure-preserving (SP) properties in the ideal case, alongside parameter-robust preconditioners. We show that these timesteppers can derive from a finite-element-in-time (FET) (and finite-element-in-space) interpretation. The benefits of this reformulation are discussed, including the derivation of timesteppers that are higher order in time, and the quantifiable dissipative SP properties in the non-ideal, resistive case.
        
        We discuss possible options for extending this FET approach to timesteppers for the compressible case.

        The kinetic corrections satisfy linearized Boltzmann equations. Using a Lénard--Bernstein collision operator, these take Fokker--Planck-like forms \cite{Fokker_1914, Planck_1917} wherein pseudo-particles in the numerical model obey the neoclassical transport equations, with particle-independent Brownian drift terms. This offers a rigorous methodology for incorporating collisions into the particle transport model, without coupling the equations of motions for each particle.
        
        Works by Chen, Chacón et al. \cite{Chen_Chacón_Barnes_2011, Chacón_Chen_Barnes_2013, Chen_Chacón_2014, Chen_Chacón_2015} have developed structure-preserving particle pushers for neoclassical transport in the Vlasov equations, derived from Crank--Nicolson integrators. We show these too can can derive from a FET interpretation, similarly offering potential extensions to higher-order-in-time particle pushers. The FET formulation is used also to consider how the stochastic drift terms can be incorporated into the pushers. Stochastic gyrokinetic expansions are also discussed.

        Different options for the numerical implementation of these schemes are considered.

        Due to the efficacy of FET in the development of SP timesteppers for both the fluid and kinetic component, we hope this approach will prove effective in the future for developing SP timesteppers for the full hybrid model. We hope this will give us the opportunity to incorporate previously inaccessible kinetic effects into the highly effective, modern, finite-element MHD models.
    \end{abstract}
    
    
    \newpage
    \tableofcontents
    
    
    \newpage
    \pagenumbering{arabic}
    %\linenumbers\renewcommand\thelinenumber{\color{black!50}\arabic{linenumber}}
            \input{0 - introduction/main.tex}
        \part{Research}
            \input{1 - low-noise PiC models/main.tex}
            \input{2 - kinetic component/main.tex}
            \input{3 - fluid component/main.tex}
            \input{4 - numerical implementation/main.tex}
        \part{Project Overview}
            \input{5 - research plan/main.tex}
            \input{6 - summary/main.tex}
    
    
    %\section{}
    \newpage
    \pagenumbering{gobble}
        \printbibliography


    \newpage
    \pagenumbering{roman}
    \appendix
        \part{Appendices}
            \input{8 - Hilbert complexes/main.tex}
            \input{9 - weak conservation proofs/main.tex}
\end{document}

            \documentclass[12pt, a4paper]{report}

\input{template/main.tex}

\title{\BA{Title in Progress...}}
\author{Boris Andrews}
\affil{Mathematical Institute, University of Oxford}
\date{\today}


\begin{document}
    \pagenumbering{gobble}
    \maketitle
    
    
    \begin{abstract}
        Magnetic confinement reactors---in particular tokamaks---offer one of the most promising options for achieving practical nuclear fusion, with the potential to provide virtually limitless, clean energy. The theoretical and numerical modeling of tokamak plasmas is simultaneously an essential component of effective reactor design, and a great research barrier. Tokamak operational conditions exhibit comparatively low Knudsen numbers. Kinetic effects, including kinetic waves and instabilities, Landau damping, bump-on-tail instabilities and more, are therefore highly influential in tokamak plasma dynamics. Purely fluid models are inherently incapable of capturing these effects, whereas the high dimensionality in purely kinetic models render them practically intractable for most relevant purposes.

        We consider a $\delta\!f$ decomposition model, with a macroscopic fluid background and microscopic kinetic correction, both fully coupled to each other. A similar manner of discretization is proposed to that used in the recent \texttt{STRUPHY} code \cite{Holderied_Possanner_Wang_2021, Holderied_2022, Li_et_al_2023} with a finite-element model for the background and a pseudo-particle/PiC model for the correction.

        The fluid background satisfies the full, non-linear, resistive, compressible, Hall MHD equations. \cite{Laakmann_Hu_Farrell_2022} introduces finite-element(-in-space) implicit timesteppers for the incompressible analogue to this system with structure-preserving (SP) properties in the ideal case, alongside parameter-robust preconditioners. We show that these timesteppers can derive from a finite-element-in-time (FET) (and finite-element-in-space) interpretation. The benefits of this reformulation are discussed, including the derivation of timesteppers that are higher order in time, and the quantifiable dissipative SP properties in the non-ideal, resistive case.
        
        We discuss possible options for extending this FET approach to timesteppers for the compressible case.

        The kinetic corrections satisfy linearized Boltzmann equations. Using a Lénard--Bernstein collision operator, these take Fokker--Planck-like forms \cite{Fokker_1914, Planck_1917} wherein pseudo-particles in the numerical model obey the neoclassical transport equations, with particle-independent Brownian drift terms. This offers a rigorous methodology for incorporating collisions into the particle transport model, without coupling the equations of motions for each particle.
        
        Works by Chen, Chacón et al. \cite{Chen_Chacón_Barnes_2011, Chacón_Chen_Barnes_2013, Chen_Chacón_2014, Chen_Chacón_2015} have developed structure-preserving particle pushers for neoclassical transport in the Vlasov equations, derived from Crank--Nicolson integrators. We show these too can can derive from a FET interpretation, similarly offering potential extensions to higher-order-in-time particle pushers. The FET formulation is used also to consider how the stochastic drift terms can be incorporated into the pushers. Stochastic gyrokinetic expansions are also discussed.

        Different options for the numerical implementation of these schemes are considered.

        Due to the efficacy of FET in the development of SP timesteppers for both the fluid and kinetic component, we hope this approach will prove effective in the future for developing SP timesteppers for the full hybrid model. We hope this will give us the opportunity to incorporate previously inaccessible kinetic effects into the highly effective, modern, finite-element MHD models.
    \end{abstract}
    
    
    \newpage
    \tableofcontents
    
    
    \newpage
    \pagenumbering{arabic}
    %\linenumbers\renewcommand\thelinenumber{\color{black!50}\arabic{linenumber}}
            \input{0 - introduction/main.tex}
        \part{Research}
            \input{1 - low-noise PiC models/main.tex}
            \input{2 - kinetic component/main.tex}
            \input{3 - fluid component/main.tex}
            \input{4 - numerical implementation/main.tex}
        \part{Project Overview}
            \input{5 - research plan/main.tex}
            \input{6 - summary/main.tex}
    
    
    %\section{}
    \newpage
    \pagenumbering{gobble}
        \printbibliography


    \newpage
    \pagenumbering{roman}
    \appendix
        \part{Appendices}
            \input{8 - Hilbert complexes/main.tex}
            \input{9 - weak conservation proofs/main.tex}
\end{document}

    
    
    %\section{}
    \newpage
    \pagenumbering{gobble}
        \printbibliography


    \newpage
    \pagenumbering{roman}
    \appendix
        \part{Appendices}
            \documentclass[12pt, a4paper]{report}

\input{template/main.tex}

\title{\BA{Title in Progress...}}
\author{Boris Andrews}
\affil{Mathematical Institute, University of Oxford}
\date{\today}


\begin{document}
    \pagenumbering{gobble}
    \maketitle
    
    
    \begin{abstract}
        Magnetic confinement reactors---in particular tokamaks---offer one of the most promising options for achieving practical nuclear fusion, with the potential to provide virtually limitless, clean energy. The theoretical and numerical modeling of tokamak plasmas is simultaneously an essential component of effective reactor design, and a great research barrier. Tokamak operational conditions exhibit comparatively low Knudsen numbers. Kinetic effects, including kinetic waves and instabilities, Landau damping, bump-on-tail instabilities and more, are therefore highly influential in tokamak plasma dynamics. Purely fluid models are inherently incapable of capturing these effects, whereas the high dimensionality in purely kinetic models render them practically intractable for most relevant purposes.

        We consider a $\delta\!f$ decomposition model, with a macroscopic fluid background and microscopic kinetic correction, both fully coupled to each other. A similar manner of discretization is proposed to that used in the recent \texttt{STRUPHY} code \cite{Holderied_Possanner_Wang_2021, Holderied_2022, Li_et_al_2023} with a finite-element model for the background and a pseudo-particle/PiC model for the correction.

        The fluid background satisfies the full, non-linear, resistive, compressible, Hall MHD equations. \cite{Laakmann_Hu_Farrell_2022} introduces finite-element(-in-space) implicit timesteppers for the incompressible analogue to this system with structure-preserving (SP) properties in the ideal case, alongside parameter-robust preconditioners. We show that these timesteppers can derive from a finite-element-in-time (FET) (and finite-element-in-space) interpretation. The benefits of this reformulation are discussed, including the derivation of timesteppers that are higher order in time, and the quantifiable dissipative SP properties in the non-ideal, resistive case.
        
        We discuss possible options for extending this FET approach to timesteppers for the compressible case.

        The kinetic corrections satisfy linearized Boltzmann equations. Using a Lénard--Bernstein collision operator, these take Fokker--Planck-like forms \cite{Fokker_1914, Planck_1917} wherein pseudo-particles in the numerical model obey the neoclassical transport equations, with particle-independent Brownian drift terms. This offers a rigorous methodology for incorporating collisions into the particle transport model, without coupling the equations of motions for each particle.
        
        Works by Chen, Chacón et al. \cite{Chen_Chacón_Barnes_2011, Chacón_Chen_Barnes_2013, Chen_Chacón_2014, Chen_Chacón_2015} have developed structure-preserving particle pushers for neoclassical transport in the Vlasov equations, derived from Crank--Nicolson integrators. We show these too can can derive from a FET interpretation, similarly offering potential extensions to higher-order-in-time particle pushers. The FET formulation is used also to consider how the stochastic drift terms can be incorporated into the pushers. Stochastic gyrokinetic expansions are also discussed.

        Different options for the numerical implementation of these schemes are considered.

        Due to the efficacy of FET in the development of SP timesteppers for both the fluid and kinetic component, we hope this approach will prove effective in the future for developing SP timesteppers for the full hybrid model. We hope this will give us the opportunity to incorporate previously inaccessible kinetic effects into the highly effective, modern, finite-element MHD models.
    \end{abstract}
    
    
    \newpage
    \tableofcontents
    
    
    \newpage
    \pagenumbering{arabic}
    %\linenumbers\renewcommand\thelinenumber{\color{black!50}\arabic{linenumber}}
            \input{0 - introduction/main.tex}
        \part{Research}
            \input{1 - low-noise PiC models/main.tex}
            \input{2 - kinetic component/main.tex}
            \input{3 - fluid component/main.tex}
            \input{4 - numerical implementation/main.tex}
        \part{Project Overview}
            \input{5 - research plan/main.tex}
            \input{6 - summary/main.tex}
    
    
    %\section{}
    \newpage
    \pagenumbering{gobble}
        \printbibliography


    \newpage
    \pagenumbering{roman}
    \appendix
        \part{Appendices}
            \input{8 - Hilbert complexes/main.tex}
            \input{9 - weak conservation proofs/main.tex}
\end{document}

            \documentclass[12pt, a4paper]{report}

\input{template/main.tex}

\title{\BA{Title in Progress...}}
\author{Boris Andrews}
\affil{Mathematical Institute, University of Oxford}
\date{\today}


\begin{document}
    \pagenumbering{gobble}
    \maketitle
    
    
    \begin{abstract}
        Magnetic confinement reactors---in particular tokamaks---offer one of the most promising options for achieving practical nuclear fusion, with the potential to provide virtually limitless, clean energy. The theoretical and numerical modeling of tokamak plasmas is simultaneously an essential component of effective reactor design, and a great research barrier. Tokamak operational conditions exhibit comparatively low Knudsen numbers. Kinetic effects, including kinetic waves and instabilities, Landau damping, bump-on-tail instabilities and more, are therefore highly influential in tokamak plasma dynamics. Purely fluid models are inherently incapable of capturing these effects, whereas the high dimensionality in purely kinetic models render them practically intractable for most relevant purposes.

        We consider a $\delta\!f$ decomposition model, with a macroscopic fluid background and microscopic kinetic correction, both fully coupled to each other. A similar manner of discretization is proposed to that used in the recent \texttt{STRUPHY} code \cite{Holderied_Possanner_Wang_2021, Holderied_2022, Li_et_al_2023} with a finite-element model for the background and a pseudo-particle/PiC model for the correction.

        The fluid background satisfies the full, non-linear, resistive, compressible, Hall MHD equations. \cite{Laakmann_Hu_Farrell_2022} introduces finite-element(-in-space) implicit timesteppers for the incompressible analogue to this system with structure-preserving (SP) properties in the ideal case, alongside parameter-robust preconditioners. We show that these timesteppers can derive from a finite-element-in-time (FET) (and finite-element-in-space) interpretation. The benefits of this reformulation are discussed, including the derivation of timesteppers that are higher order in time, and the quantifiable dissipative SP properties in the non-ideal, resistive case.
        
        We discuss possible options for extending this FET approach to timesteppers for the compressible case.

        The kinetic corrections satisfy linearized Boltzmann equations. Using a Lénard--Bernstein collision operator, these take Fokker--Planck-like forms \cite{Fokker_1914, Planck_1917} wherein pseudo-particles in the numerical model obey the neoclassical transport equations, with particle-independent Brownian drift terms. This offers a rigorous methodology for incorporating collisions into the particle transport model, without coupling the equations of motions for each particle.
        
        Works by Chen, Chacón et al. \cite{Chen_Chacón_Barnes_2011, Chacón_Chen_Barnes_2013, Chen_Chacón_2014, Chen_Chacón_2015} have developed structure-preserving particle pushers for neoclassical transport in the Vlasov equations, derived from Crank--Nicolson integrators. We show these too can can derive from a FET interpretation, similarly offering potential extensions to higher-order-in-time particle pushers. The FET formulation is used also to consider how the stochastic drift terms can be incorporated into the pushers. Stochastic gyrokinetic expansions are also discussed.

        Different options for the numerical implementation of these schemes are considered.

        Due to the efficacy of FET in the development of SP timesteppers for both the fluid and kinetic component, we hope this approach will prove effective in the future for developing SP timesteppers for the full hybrid model. We hope this will give us the opportunity to incorporate previously inaccessible kinetic effects into the highly effective, modern, finite-element MHD models.
    \end{abstract}
    
    
    \newpage
    \tableofcontents
    
    
    \newpage
    \pagenumbering{arabic}
    %\linenumbers\renewcommand\thelinenumber{\color{black!50}\arabic{linenumber}}
            \input{0 - introduction/main.tex}
        \part{Research}
            \input{1 - low-noise PiC models/main.tex}
            \input{2 - kinetic component/main.tex}
            \input{3 - fluid component/main.tex}
            \input{4 - numerical implementation/main.tex}
        \part{Project Overview}
            \input{5 - research plan/main.tex}
            \input{6 - summary/main.tex}
    
    
    %\section{}
    \newpage
    \pagenumbering{gobble}
        \printbibliography


    \newpage
    \pagenumbering{roman}
    \appendix
        \part{Appendices}
            \input{8 - Hilbert complexes/main.tex}
            \input{9 - weak conservation proofs/main.tex}
\end{document}

\end{document}

        \part{Project Overview}
            \documentclass[12pt, a4paper]{report}

\documentclass[12pt, a4paper]{report}

\input{template/main.tex}

\title{\BA{Title in Progress...}}
\author{Boris Andrews}
\affil{Mathematical Institute, University of Oxford}
\date{\today}


\begin{document}
    \pagenumbering{gobble}
    \maketitle
    
    
    \begin{abstract}
        Magnetic confinement reactors---in particular tokamaks---offer one of the most promising options for achieving practical nuclear fusion, with the potential to provide virtually limitless, clean energy. The theoretical and numerical modeling of tokamak plasmas is simultaneously an essential component of effective reactor design, and a great research barrier. Tokamak operational conditions exhibit comparatively low Knudsen numbers. Kinetic effects, including kinetic waves and instabilities, Landau damping, bump-on-tail instabilities and more, are therefore highly influential in tokamak plasma dynamics. Purely fluid models are inherently incapable of capturing these effects, whereas the high dimensionality in purely kinetic models render them practically intractable for most relevant purposes.

        We consider a $\delta\!f$ decomposition model, with a macroscopic fluid background and microscopic kinetic correction, both fully coupled to each other. A similar manner of discretization is proposed to that used in the recent \texttt{STRUPHY} code \cite{Holderied_Possanner_Wang_2021, Holderied_2022, Li_et_al_2023} with a finite-element model for the background and a pseudo-particle/PiC model for the correction.

        The fluid background satisfies the full, non-linear, resistive, compressible, Hall MHD equations. \cite{Laakmann_Hu_Farrell_2022} introduces finite-element(-in-space) implicit timesteppers for the incompressible analogue to this system with structure-preserving (SP) properties in the ideal case, alongside parameter-robust preconditioners. We show that these timesteppers can derive from a finite-element-in-time (FET) (and finite-element-in-space) interpretation. The benefits of this reformulation are discussed, including the derivation of timesteppers that are higher order in time, and the quantifiable dissipative SP properties in the non-ideal, resistive case.
        
        We discuss possible options for extending this FET approach to timesteppers for the compressible case.

        The kinetic corrections satisfy linearized Boltzmann equations. Using a Lénard--Bernstein collision operator, these take Fokker--Planck-like forms \cite{Fokker_1914, Planck_1917} wherein pseudo-particles in the numerical model obey the neoclassical transport equations, with particle-independent Brownian drift terms. This offers a rigorous methodology for incorporating collisions into the particle transport model, without coupling the equations of motions for each particle.
        
        Works by Chen, Chacón et al. \cite{Chen_Chacón_Barnes_2011, Chacón_Chen_Barnes_2013, Chen_Chacón_2014, Chen_Chacón_2015} have developed structure-preserving particle pushers for neoclassical transport in the Vlasov equations, derived from Crank--Nicolson integrators. We show these too can can derive from a FET interpretation, similarly offering potential extensions to higher-order-in-time particle pushers. The FET formulation is used also to consider how the stochastic drift terms can be incorporated into the pushers. Stochastic gyrokinetic expansions are also discussed.

        Different options for the numerical implementation of these schemes are considered.

        Due to the efficacy of FET in the development of SP timesteppers for both the fluid and kinetic component, we hope this approach will prove effective in the future for developing SP timesteppers for the full hybrid model. We hope this will give us the opportunity to incorporate previously inaccessible kinetic effects into the highly effective, modern, finite-element MHD models.
    \end{abstract}
    
    
    \newpage
    \tableofcontents
    
    
    \newpage
    \pagenumbering{arabic}
    %\linenumbers\renewcommand\thelinenumber{\color{black!50}\arabic{linenumber}}
            \input{0 - introduction/main.tex}
        \part{Research}
            \input{1 - low-noise PiC models/main.tex}
            \input{2 - kinetic component/main.tex}
            \input{3 - fluid component/main.tex}
            \input{4 - numerical implementation/main.tex}
        \part{Project Overview}
            \input{5 - research plan/main.tex}
            \input{6 - summary/main.tex}
    
    
    %\section{}
    \newpage
    \pagenumbering{gobble}
        \printbibliography


    \newpage
    \pagenumbering{roman}
    \appendix
        \part{Appendices}
            \input{8 - Hilbert complexes/main.tex}
            \input{9 - weak conservation proofs/main.tex}
\end{document}


\title{\BA{Title in Progress...}}
\author{Boris Andrews}
\affil{Mathematical Institute, University of Oxford}
\date{\today}


\begin{document}
    \pagenumbering{gobble}
    \maketitle
    
    
    \begin{abstract}
        Magnetic confinement reactors---in particular tokamaks---offer one of the most promising options for achieving practical nuclear fusion, with the potential to provide virtually limitless, clean energy. The theoretical and numerical modeling of tokamak plasmas is simultaneously an essential component of effective reactor design, and a great research barrier. Tokamak operational conditions exhibit comparatively low Knudsen numbers. Kinetic effects, including kinetic waves and instabilities, Landau damping, bump-on-tail instabilities and more, are therefore highly influential in tokamak plasma dynamics. Purely fluid models are inherently incapable of capturing these effects, whereas the high dimensionality in purely kinetic models render them practically intractable for most relevant purposes.

        We consider a $\delta\!f$ decomposition model, with a macroscopic fluid background and microscopic kinetic correction, both fully coupled to each other. A similar manner of discretization is proposed to that used in the recent \texttt{STRUPHY} code \cite{Holderied_Possanner_Wang_2021, Holderied_2022, Li_et_al_2023} with a finite-element model for the background and a pseudo-particle/PiC model for the correction.

        The fluid background satisfies the full, non-linear, resistive, compressible, Hall MHD equations. \cite{Laakmann_Hu_Farrell_2022} introduces finite-element(-in-space) implicit timesteppers for the incompressible analogue to this system with structure-preserving (SP) properties in the ideal case, alongside parameter-robust preconditioners. We show that these timesteppers can derive from a finite-element-in-time (FET) (and finite-element-in-space) interpretation. The benefits of this reformulation are discussed, including the derivation of timesteppers that are higher order in time, and the quantifiable dissipative SP properties in the non-ideal, resistive case.
        
        We discuss possible options for extending this FET approach to timesteppers for the compressible case.

        The kinetic corrections satisfy linearized Boltzmann equations. Using a Lénard--Bernstein collision operator, these take Fokker--Planck-like forms \cite{Fokker_1914, Planck_1917} wherein pseudo-particles in the numerical model obey the neoclassical transport equations, with particle-independent Brownian drift terms. This offers a rigorous methodology for incorporating collisions into the particle transport model, without coupling the equations of motions for each particle.
        
        Works by Chen, Chacón et al. \cite{Chen_Chacón_Barnes_2011, Chacón_Chen_Barnes_2013, Chen_Chacón_2014, Chen_Chacón_2015} have developed structure-preserving particle pushers for neoclassical transport in the Vlasov equations, derived from Crank--Nicolson integrators. We show these too can can derive from a FET interpretation, similarly offering potential extensions to higher-order-in-time particle pushers. The FET formulation is used also to consider how the stochastic drift terms can be incorporated into the pushers. Stochastic gyrokinetic expansions are also discussed.

        Different options for the numerical implementation of these schemes are considered.

        Due to the efficacy of FET in the development of SP timesteppers for both the fluid and kinetic component, we hope this approach will prove effective in the future for developing SP timesteppers for the full hybrid model. We hope this will give us the opportunity to incorporate previously inaccessible kinetic effects into the highly effective, modern, finite-element MHD models.
    \end{abstract}
    
    
    \newpage
    \tableofcontents
    
    
    \newpage
    \pagenumbering{arabic}
    %\linenumbers\renewcommand\thelinenumber{\color{black!50}\arabic{linenumber}}
            \documentclass[12pt, a4paper]{report}

\input{template/main.tex}

\title{\BA{Title in Progress...}}
\author{Boris Andrews}
\affil{Mathematical Institute, University of Oxford}
\date{\today}


\begin{document}
    \pagenumbering{gobble}
    \maketitle
    
    
    \begin{abstract}
        Magnetic confinement reactors---in particular tokamaks---offer one of the most promising options for achieving practical nuclear fusion, with the potential to provide virtually limitless, clean energy. The theoretical and numerical modeling of tokamak plasmas is simultaneously an essential component of effective reactor design, and a great research barrier. Tokamak operational conditions exhibit comparatively low Knudsen numbers. Kinetic effects, including kinetic waves and instabilities, Landau damping, bump-on-tail instabilities and more, are therefore highly influential in tokamak plasma dynamics. Purely fluid models are inherently incapable of capturing these effects, whereas the high dimensionality in purely kinetic models render them practically intractable for most relevant purposes.

        We consider a $\delta\!f$ decomposition model, with a macroscopic fluid background and microscopic kinetic correction, both fully coupled to each other. A similar manner of discretization is proposed to that used in the recent \texttt{STRUPHY} code \cite{Holderied_Possanner_Wang_2021, Holderied_2022, Li_et_al_2023} with a finite-element model for the background and a pseudo-particle/PiC model for the correction.

        The fluid background satisfies the full, non-linear, resistive, compressible, Hall MHD equations. \cite{Laakmann_Hu_Farrell_2022} introduces finite-element(-in-space) implicit timesteppers for the incompressible analogue to this system with structure-preserving (SP) properties in the ideal case, alongside parameter-robust preconditioners. We show that these timesteppers can derive from a finite-element-in-time (FET) (and finite-element-in-space) interpretation. The benefits of this reformulation are discussed, including the derivation of timesteppers that are higher order in time, and the quantifiable dissipative SP properties in the non-ideal, resistive case.
        
        We discuss possible options for extending this FET approach to timesteppers for the compressible case.

        The kinetic corrections satisfy linearized Boltzmann equations. Using a Lénard--Bernstein collision operator, these take Fokker--Planck-like forms \cite{Fokker_1914, Planck_1917} wherein pseudo-particles in the numerical model obey the neoclassical transport equations, with particle-independent Brownian drift terms. This offers a rigorous methodology for incorporating collisions into the particle transport model, without coupling the equations of motions for each particle.
        
        Works by Chen, Chacón et al. \cite{Chen_Chacón_Barnes_2011, Chacón_Chen_Barnes_2013, Chen_Chacón_2014, Chen_Chacón_2015} have developed structure-preserving particle pushers for neoclassical transport in the Vlasov equations, derived from Crank--Nicolson integrators. We show these too can can derive from a FET interpretation, similarly offering potential extensions to higher-order-in-time particle pushers. The FET formulation is used also to consider how the stochastic drift terms can be incorporated into the pushers. Stochastic gyrokinetic expansions are also discussed.

        Different options for the numerical implementation of these schemes are considered.

        Due to the efficacy of FET in the development of SP timesteppers for both the fluid and kinetic component, we hope this approach will prove effective in the future for developing SP timesteppers for the full hybrid model. We hope this will give us the opportunity to incorporate previously inaccessible kinetic effects into the highly effective, modern, finite-element MHD models.
    \end{abstract}
    
    
    \newpage
    \tableofcontents
    
    
    \newpage
    \pagenumbering{arabic}
    %\linenumbers\renewcommand\thelinenumber{\color{black!50}\arabic{linenumber}}
            \input{0 - introduction/main.tex}
        \part{Research}
            \input{1 - low-noise PiC models/main.tex}
            \input{2 - kinetic component/main.tex}
            \input{3 - fluid component/main.tex}
            \input{4 - numerical implementation/main.tex}
        \part{Project Overview}
            \input{5 - research plan/main.tex}
            \input{6 - summary/main.tex}
    
    
    %\section{}
    \newpage
    \pagenumbering{gobble}
        \printbibliography


    \newpage
    \pagenumbering{roman}
    \appendix
        \part{Appendices}
            \input{8 - Hilbert complexes/main.tex}
            \input{9 - weak conservation proofs/main.tex}
\end{document}

        \part{Research}
            \documentclass[12pt, a4paper]{report}

\input{template/main.tex}

\title{\BA{Title in Progress...}}
\author{Boris Andrews}
\affil{Mathematical Institute, University of Oxford}
\date{\today}


\begin{document}
    \pagenumbering{gobble}
    \maketitle
    
    
    \begin{abstract}
        Magnetic confinement reactors---in particular tokamaks---offer one of the most promising options for achieving practical nuclear fusion, with the potential to provide virtually limitless, clean energy. The theoretical and numerical modeling of tokamak plasmas is simultaneously an essential component of effective reactor design, and a great research barrier. Tokamak operational conditions exhibit comparatively low Knudsen numbers. Kinetic effects, including kinetic waves and instabilities, Landau damping, bump-on-tail instabilities and more, are therefore highly influential in tokamak plasma dynamics. Purely fluid models are inherently incapable of capturing these effects, whereas the high dimensionality in purely kinetic models render them practically intractable for most relevant purposes.

        We consider a $\delta\!f$ decomposition model, with a macroscopic fluid background and microscopic kinetic correction, both fully coupled to each other. A similar manner of discretization is proposed to that used in the recent \texttt{STRUPHY} code \cite{Holderied_Possanner_Wang_2021, Holderied_2022, Li_et_al_2023} with a finite-element model for the background and a pseudo-particle/PiC model for the correction.

        The fluid background satisfies the full, non-linear, resistive, compressible, Hall MHD equations. \cite{Laakmann_Hu_Farrell_2022} introduces finite-element(-in-space) implicit timesteppers for the incompressible analogue to this system with structure-preserving (SP) properties in the ideal case, alongside parameter-robust preconditioners. We show that these timesteppers can derive from a finite-element-in-time (FET) (and finite-element-in-space) interpretation. The benefits of this reformulation are discussed, including the derivation of timesteppers that are higher order in time, and the quantifiable dissipative SP properties in the non-ideal, resistive case.
        
        We discuss possible options for extending this FET approach to timesteppers for the compressible case.

        The kinetic corrections satisfy linearized Boltzmann equations. Using a Lénard--Bernstein collision operator, these take Fokker--Planck-like forms \cite{Fokker_1914, Planck_1917} wherein pseudo-particles in the numerical model obey the neoclassical transport equations, with particle-independent Brownian drift terms. This offers a rigorous methodology for incorporating collisions into the particle transport model, without coupling the equations of motions for each particle.
        
        Works by Chen, Chacón et al. \cite{Chen_Chacón_Barnes_2011, Chacón_Chen_Barnes_2013, Chen_Chacón_2014, Chen_Chacón_2015} have developed structure-preserving particle pushers for neoclassical transport in the Vlasov equations, derived from Crank--Nicolson integrators. We show these too can can derive from a FET interpretation, similarly offering potential extensions to higher-order-in-time particle pushers. The FET formulation is used also to consider how the stochastic drift terms can be incorporated into the pushers. Stochastic gyrokinetic expansions are also discussed.

        Different options for the numerical implementation of these schemes are considered.

        Due to the efficacy of FET in the development of SP timesteppers for both the fluid and kinetic component, we hope this approach will prove effective in the future for developing SP timesteppers for the full hybrid model. We hope this will give us the opportunity to incorporate previously inaccessible kinetic effects into the highly effective, modern, finite-element MHD models.
    \end{abstract}
    
    
    \newpage
    \tableofcontents
    
    
    \newpage
    \pagenumbering{arabic}
    %\linenumbers\renewcommand\thelinenumber{\color{black!50}\arabic{linenumber}}
            \input{0 - introduction/main.tex}
        \part{Research}
            \input{1 - low-noise PiC models/main.tex}
            \input{2 - kinetic component/main.tex}
            \input{3 - fluid component/main.tex}
            \input{4 - numerical implementation/main.tex}
        \part{Project Overview}
            \input{5 - research plan/main.tex}
            \input{6 - summary/main.tex}
    
    
    %\section{}
    \newpage
    \pagenumbering{gobble}
        \printbibliography


    \newpage
    \pagenumbering{roman}
    \appendix
        \part{Appendices}
            \input{8 - Hilbert complexes/main.tex}
            \input{9 - weak conservation proofs/main.tex}
\end{document}

            \documentclass[12pt, a4paper]{report}

\input{template/main.tex}

\title{\BA{Title in Progress...}}
\author{Boris Andrews}
\affil{Mathematical Institute, University of Oxford}
\date{\today}


\begin{document}
    \pagenumbering{gobble}
    \maketitle
    
    
    \begin{abstract}
        Magnetic confinement reactors---in particular tokamaks---offer one of the most promising options for achieving practical nuclear fusion, with the potential to provide virtually limitless, clean energy. The theoretical and numerical modeling of tokamak plasmas is simultaneously an essential component of effective reactor design, and a great research barrier. Tokamak operational conditions exhibit comparatively low Knudsen numbers. Kinetic effects, including kinetic waves and instabilities, Landau damping, bump-on-tail instabilities and more, are therefore highly influential in tokamak plasma dynamics. Purely fluid models are inherently incapable of capturing these effects, whereas the high dimensionality in purely kinetic models render them practically intractable for most relevant purposes.

        We consider a $\delta\!f$ decomposition model, with a macroscopic fluid background and microscopic kinetic correction, both fully coupled to each other. A similar manner of discretization is proposed to that used in the recent \texttt{STRUPHY} code \cite{Holderied_Possanner_Wang_2021, Holderied_2022, Li_et_al_2023} with a finite-element model for the background and a pseudo-particle/PiC model for the correction.

        The fluid background satisfies the full, non-linear, resistive, compressible, Hall MHD equations. \cite{Laakmann_Hu_Farrell_2022} introduces finite-element(-in-space) implicit timesteppers for the incompressible analogue to this system with structure-preserving (SP) properties in the ideal case, alongside parameter-robust preconditioners. We show that these timesteppers can derive from a finite-element-in-time (FET) (and finite-element-in-space) interpretation. The benefits of this reformulation are discussed, including the derivation of timesteppers that are higher order in time, and the quantifiable dissipative SP properties in the non-ideal, resistive case.
        
        We discuss possible options for extending this FET approach to timesteppers for the compressible case.

        The kinetic corrections satisfy linearized Boltzmann equations. Using a Lénard--Bernstein collision operator, these take Fokker--Planck-like forms \cite{Fokker_1914, Planck_1917} wherein pseudo-particles in the numerical model obey the neoclassical transport equations, with particle-independent Brownian drift terms. This offers a rigorous methodology for incorporating collisions into the particle transport model, without coupling the equations of motions for each particle.
        
        Works by Chen, Chacón et al. \cite{Chen_Chacón_Barnes_2011, Chacón_Chen_Barnes_2013, Chen_Chacón_2014, Chen_Chacón_2015} have developed structure-preserving particle pushers for neoclassical transport in the Vlasov equations, derived from Crank--Nicolson integrators. We show these too can can derive from a FET interpretation, similarly offering potential extensions to higher-order-in-time particle pushers. The FET formulation is used also to consider how the stochastic drift terms can be incorporated into the pushers. Stochastic gyrokinetic expansions are also discussed.

        Different options for the numerical implementation of these schemes are considered.

        Due to the efficacy of FET in the development of SP timesteppers for both the fluid and kinetic component, we hope this approach will prove effective in the future for developing SP timesteppers for the full hybrid model. We hope this will give us the opportunity to incorporate previously inaccessible kinetic effects into the highly effective, modern, finite-element MHD models.
    \end{abstract}
    
    
    \newpage
    \tableofcontents
    
    
    \newpage
    \pagenumbering{arabic}
    %\linenumbers\renewcommand\thelinenumber{\color{black!50}\arabic{linenumber}}
            \input{0 - introduction/main.tex}
        \part{Research}
            \input{1 - low-noise PiC models/main.tex}
            \input{2 - kinetic component/main.tex}
            \input{3 - fluid component/main.tex}
            \input{4 - numerical implementation/main.tex}
        \part{Project Overview}
            \input{5 - research plan/main.tex}
            \input{6 - summary/main.tex}
    
    
    %\section{}
    \newpage
    \pagenumbering{gobble}
        \printbibliography


    \newpage
    \pagenumbering{roman}
    \appendix
        \part{Appendices}
            \input{8 - Hilbert complexes/main.tex}
            \input{9 - weak conservation proofs/main.tex}
\end{document}

            \documentclass[12pt, a4paper]{report}

\input{template/main.tex}

\title{\BA{Title in Progress...}}
\author{Boris Andrews}
\affil{Mathematical Institute, University of Oxford}
\date{\today}


\begin{document}
    \pagenumbering{gobble}
    \maketitle
    
    
    \begin{abstract}
        Magnetic confinement reactors---in particular tokamaks---offer one of the most promising options for achieving practical nuclear fusion, with the potential to provide virtually limitless, clean energy. The theoretical and numerical modeling of tokamak plasmas is simultaneously an essential component of effective reactor design, and a great research barrier. Tokamak operational conditions exhibit comparatively low Knudsen numbers. Kinetic effects, including kinetic waves and instabilities, Landau damping, bump-on-tail instabilities and more, are therefore highly influential in tokamak plasma dynamics. Purely fluid models are inherently incapable of capturing these effects, whereas the high dimensionality in purely kinetic models render them practically intractable for most relevant purposes.

        We consider a $\delta\!f$ decomposition model, with a macroscopic fluid background and microscopic kinetic correction, both fully coupled to each other. A similar manner of discretization is proposed to that used in the recent \texttt{STRUPHY} code \cite{Holderied_Possanner_Wang_2021, Holderied_2022, Li_et_al_2023} with a finite-element model for the background and a pseudo-particle/PiC model for the correction.

        The fluid background satisfies the full, non-linear, resistive, compressible, Hall MHD equations. \cite{Laakmann_Hu_Farrell_2022} introduces finite-element(-in-space) implicit timesteppers for the incompressible analogue to this system with structure-preserving (SP) properties in the ideal case, alongside parameter-robust preconditioners. We show that these timesteppers can derive from a finite-element-in-time (FET) (and finite-element-in-space) interpretation. The benefits of this reformulation are discussed, including the derivation of timesteppers that are higher order in time, and the quantifiable dissipative SP properties in the non-ideal, resistive case.
        
        We discuss possible options for extending this FET approach to timesteppers for the compressible case.

        The kinetic corrections satisfy linearized Boltzmann equations. Using a Lénard--Bernstein collision operator, these take Fokker--Planck-like forms \cite{Fokker_1914, Planck_1917} wherein pseudo-particles in the numerical model obey the neoclassical transport equations, with particle-independent Brownian drift terms. This offers a rigorous methodology for incorporating collisions into the particle transport model, without coupling the equations of motions for each particle.
        
        Works by Chen, Chacón et al. \cite{Chen_Chacón_Barnes_2011, Chacón_Chen_Barnes_2013, Chen_Chacón_2014, Chen_Chacón_2015} have developed structure-preserving particle pushers for neoclassical transport in the Vlasov equations, derived from Crank--Nicolson integrators. We show these too can can derive from a FET interpretation, similarly offering potential extensions to higher-order-in-time particle pushers. The FET formulation is used also to consider how the stochastic drift terms can be incorporated into the pushers. Stochastic gyrokinetic expansions are also discussed.

        Different options for the numerical implementation of these schemes are considered.

        Due to the efficacy of FET in the development of SP timesteppers for both the fluid and kinetic component, we hope this approach will prove effective in the future for developing SP timesteppers for the full hybrid model. We hope this will give us the opportunity to incorporate previously inaccessible kinetic effects into the highly effective, modern, finite-element MHD models.
    \end{abstract}
    
    
    \newpage
    \tableofcontents
    
    
    \newpage
    \pagenumbering{arabic}
    %\linenumbers\renewcommand\thelinenumber{\color{black!50}\arabic{linenumber}}
            \input{0 - introduction/main.tex}
        \part{Research}
            \input{1 - low-noise PiC models/main.tex}
            \input{2 - kinetic component/main.tex}
            \input{3 - fluid component/main.tex}
            \input{4 - numerical implementation/main.tex}
        \part{Project Overview}
            \input{5 - research plan/main.tex}
            \input{6 - summary/main.tex}
    
    
    %\section{}
    \newpage
    \pagenumbering{gobble}
        \printbibliography


    \newpage
    \pagenumbering{roman}
    \appendix
        \part{Appendices}
            \input{8 - Hilbert complexes/main.tex}
            \input{9 - weak conservation proofs/main.tex}
\end{document}

            \documentclass[12pt, a4paper]{report}

\input{template/main.tex}

\title{\BA{Title in Progress...}}
\author{Boris Andrews}
\affil{Mathematical Institute, University of Oxford}
\date{\today}


\begin{document}
    \pagenumbering{gobble}
    \maketitle
    
    
    \begin{abstract}
        Magnetic confinement reactors---in particular tokamaks---offer one of the most promising options for achieving practical nuclear fusion, with the potential to provide virtually limitless, clean energy. The theoretical and numerical modeling of tokamak plasmas is simultaneously an essential component of effective reactor design, and a great research barrier. Tokamak operational conditions exhibit comparatively low Knudsen numbers. Kinetic effects, including kinetic waves and instabilities, Landau damping, bump-on-tail instabilities and more, are therefore highly influential in tokamak plasma dynamics. Purely fluid models are inherently incapable of capturing these effects, whereas the high dimensionality in purely kinetic models render them practically intractable for most relevant purposes.

        We consider a $\delta\!f$ decomposition model, with a macroscopic fluid background and microscopic kinetic correction, both fully coupled to each other. A similar manner of discretization is proposed to that used in the recent \texttt{STRUPHY} code \cite{Holderied_Possanner_Wang_2021, Holderied_2022, Li_et_al_2023} with a finite-element model for the background and a pseudo-particle/PiC model for the correction.

        The fluid background satisfies the full, non-linear, resistive, compressible, Hall MHD equations. \cite{Laakmann_Hu_Farrell_2022} introduces finite-element(-in-space) implicit timesteppers for the incompressible analogue to this system with structure-preserving (SP) properties in the ideal case, alongside parameter-robust preconditioners. We show that these timesteppers can derive from a finite-element-in-time (FET) (and finite-element-in-space) interpretation. The benefits of this reformulation are discussed, including the derivation of timesteppers that are higher order in time, and the quantifiable dissipative SP properties in the non-ideal, resistive case.
        
        We discuss possible options for extending this FET approach to timesteppers for the compressible case.

        The kinetic corrections satisfy linearized Boltzmann equations. Using a Lénard--Bernstein collision operator, these take Fokker--Planck-like forms \cite{Fokker_1914, Planck_1917} wherein pseudo-particles in the numerical model obey the neoclassical transport equations, with particle-independent Brownian drift terms. This offers a rigorous methodology for incorporating collisions into the particle transport model, without coupling the equations of motions for each particle.
        
        Works by Chen, Chacón et al. \cite{Chen_Chacón_Barnes_2011, Chacón_Chen_Barnes_2013, Chen_Chacón_2014, Chen_Chacón_2015} have developed structure-preserving particle pushers for neoclassical transport in the Vlasov equations, derived from Crank--Nicolson integrators. We show these too can can derive from a FET interpretation, similarly offering potential extensions to higher-order-in-time particle pushers. The FET formulation is used also to consider how the stochastic drift terms can be incorporated into the pushers. Stochastic gyrokinetic expansions are also discussed.

        Different options for the numerical implementation of these schemes are considered.

        Due to the efficacy of FET in the development of SP timesteppers for both the fluid and kinetic component, we hope this approach will prove effective in the future for developing SP timesteppers for the full hybrid model. We hope this will give us the opportunity to incorporate previously inaccessible kinetic effects into the highly effective, modern, finite-element MHD models.
    \end{abstract}
    
    
    \newpage
    \tableofcontents
    
    
    \newpage
    \pagenumbering{arabic}
    %\linenumbers\renewcommand\thelinenumber{\color{black!50}\arabic{linenumber}}
            \input{0 - introduction/main.tex}
        \part{Research}
            \input{1 - low-noise PiC models/main.tex}
            \input{2 - kinetic component/main.tex}
            \input{3 - fluid component/main.tex}
            \input{4 - numerical implementation/main.tex}
        \part{Project Overview}
            \input{5 - research plan/main.tex}
            \input{6 - summary/main.tex}
    
    
    %\section{}
    \newpage
    \pagenumbering{gobble}
        \printbibliography


    \newpage
    \pagenumbering{roman}
    \appendix
        \part{Appendices}
            \input{8 - Hilbert complexes/main.tex}
            \input{9 - weak conservation proofs/main.tex}
\end{document}

        \part{Project Overview}
            \documentclass[12pt, a4paper]{report}

\input{template/main.tex}

\title{\BA{Title in Progress...}}
\author{Boris Andrews}
\affil{Mathematical Institute, University of Oxford}
\date{\today}


\begin{document}
    \pagenumbering{gobble}
    \maketitle
    
    
    \begin{abstract}
        Magnetic confinement reactors---in particular tokamaks---offer one of the most promising options for achieving practical nuclear fusion, with the potential to provide virtually limitless, clean energy. The theoretical and numerical modeling of tokamak plasmas is simultaneously an essential component of effective reactor design, and a great research barrier. Tokamak operational conditions exhibit comparatively low Knudsen numbers. Kinetic effects, including kinetic waves and instabilities, Landau damping, bump-on-tail instabilities and more, are therefore highly influential in tokamak plasma dynamics. Purely fluid models are inherently incapable of capturing these effects, whereas the high dimensionality in purely kinetic models render them practically intractable for most relevant purposes.

        We consider a $\delta\!f$ decomposition model, with a macroscopic fluid background and microscopic kinetic correction, both fully coupled to each other. A similar manner of discretization is proposed to that used in the recent \texttt{STRUPHY} code \cite{Holderied_Possanner_Wang_2021, Holderied_2022, Li_et_al_2023} with a finite-element model for the background and a pseudo-particle/PiC model for the correction.

        The fluid background satisfies the full, non-linear, resistive, compressible, Hall MHD equations. \cite{Laakmann_Hu_Farrell_2022} introduces finite-element(-in-space) implicit timesteppers for the incompressible analogue to this system with structure-preserving (SP) properties in the ideal case, alongside parameter-robust preconditioners. We show that these timesteppers can derive from a finite-element-in-time (FET) (and finite-element-in-space) interpretation. The benefits of this reformulation are discussed, including the derivation of timesteppers that are higher order in time, and the quantifiable dissipative SP properties in the non-ideal, resistive case.
        
        We discuss possible options for extending this FET approach to timesteppers for the compressible case.

        The kinetic corrections satisfy linearized Boltzmann equations. Using a Lénard--Bernstein collision operator, these take Fokker--Planck-like forms \cite{Fokker_1914, Planck_1917} wherein pseudo-particles in the numerical model obey the neoclassical transport equations, with particle-independent Brownian drift terms. This offers a rigorous methodology for incorporating collisions into the particle transport model, without coupling the equations of motions for each particle.
        
        Works by Chen, Chacón et al. \cite{Chen_Chacón_Barnes_2011, Chacón_Chen_Barnes_2013, Chen_Chacón_2014, Chen_Chacón_2015} have developed structure-preserving particle pushers for neoclassical transport in the Vlasov equations, derived from Crank--Nicolson integrators. We show these too can can derive from a FET interpretation, similarly offering potential extensions to higher-order-in-time particle pushers. The FET formulation is used also to consider how the stochastic drift terms can be incorporated into the pushers. Stochastic gyrokinetic expansions are also discussed.

        Different options for the numerical implementation of these schemes are considered.

        Due to the efficacy of FET in the development of SP timesteppers for both the fluid and kinetic component, we hope this approach will prove effective in the future for developing SP timesteppers for the full hybrid model. We hope this will give us the opportunity to incorporate previously inaccessible kinetic effects into the highly effective, modern, finite-element MHD models.
    \end{abstract}
    
    
    \newpage
    \tableofcontents
    
    
    \newpage
    \pagenumbering{arabic}
    %\linenumbers\renewcommand\thelinenumber{\color{black!50}\arabic{linenumber}}
            \input{0 - introduction/main.tex}
        \part{Research}
            \input{1 - low-noise PiC models/main.tex}
            \input{2 - kinetic component/main.tex}
            \input{3 - fluid component/main.tex}
            \input{4 - numerical implementation/main.tex}
        \part{Project Overview}
            \input{5 - research plan/main.tex}
            \input{6 - summary/main.tex}
    
    
    %\section{}
    \newpage
    \pagenumbering{gobble}
        \printbibliography


    \newpage
    \pagenumbering{roman}
    \appendix
        \part{Appendices}
            \input{8 - Hilbert complexes/main.tex}
            \input{9 - weak conservation proofs/main.tex}
\end{document}

            \documentclass[12pt, a4paper]{report}

\input{template/main.tex}

\title{\BA{Title in Progress...}}
\author{Boris Andrews}
\affil{Mathematical Institute, University of Oxford}
\date{\today}


\begin{document}
    \pagenumbering{gobble}
    \maketitle
    
    
    \begin{abstract}
        Magnetic confinement reactors---in particular tokamaks---offer one of the most promising options for achieving practical nuclear fusion, with the potential to provide virtually limitless, clean energy. The theoretical and numerical modeling of tokamak plasmas is simultaneously an essential component of effective reactor design, and a great research barrier. Tokamak operational conditions exhibit comparatively low Knudsen numbers. Kinetic effects, including kinetic waves and instabilities, Landau damping, bump-on-tail instabilities and more, are therefore highly influential in tokamak plasma dynamics. Purely fluid models are inherently incapable of capturing these effects, whereas the high dimensionality in purely kinetic models render them practically intractable for most relevant purposes.

        We consider a $\delta\!f$ decomposition model, with a macroscopic fluid background and microscopic kinetic correction, both fully coupled to each other. A similar manner of discretization is proposed to that used in the recent \texttt{STRUPHY} code \cite{Holderied_Possanner_Wang_2021, Holderied_2022, Li_et_al_2023} with a finite-element model for the background and a pseudo-particle/PiC model for the correction.

        The fluid background satisfies the full, non-linear, resistive, compressible, Hall MHD equations. \cite{Laakmann_Hu_Farrell_2022} introduces finite-element(-in-space) implicit timesteppers for the incompressible analogue to this system with structure-preserving (SP) properties in the ideal case, alongside parameter-robust preconditioners. We show that these timesteppers can derive from a finite-element-in-time (FET) (and finite-element-in-space) interpretation. The benefits of this reformulation are discussed, including the derivation of timesteppers that are higher order in time, and the quantifiable dissipative SP properties in the non-ideal, resistive case.
        
        We discuss possible options for extending this FET approach to timesteppers for the compressible case.

        The kinetic corrections satisfy linearized Boltzmann equations. Using a Lénard--Bernstein collision operator, these take Fokker--Planck-like forms \cite{Fokker_1914, Planck_1917} wherein pseudo-particles in the numerical model obey the neoclassical transport equations, with particle-independent Brownian drift terms. This offers a rigorous methodology for incorporating collisions into the particle transport model, without coupling the equations of motions for each particle.
        
        Works by Chen, Chacón et al. \cite{Chen_Chacón_Barnes_2011, Chacón_Chen_Barnes_2013, Chen_Chacón_2014, Chen_Chacón_2015} have developed structure-preserving particle pushers for neoclassical transport in the Vlasov equations, derived from Crank--Nicolson integrators. We show these too can can derive from a FET interpretation, similarly offering potential extensions to higher-order-in-time particle pushers. The FET formulation is used also to consider how the stochastic drift terms can be incorporated into the pushers. Stochastic gyrokinetic expansions are also discussed.

        Different options for the numerical implementation of these schemes are considered.

        Due to the efficacy of FET in the development of SP timesteppers for both the fluid and kinetic component, we hope this approach will prove effective in the future for developing SP timesteppers for the full hybrid model. We hope this will give us the opportunity to incorporate previously inaccessible kinetic effects into the highly effective, modern, finite-element MHD models.
    \end{abstract}
    
    
    \newpage
    \tableofcontents
    
    
    \newpage
    \pagenumbering{arabic}
    %\linenumbers\renewcommand\thelinenumber{\color{black!50}\arabic{linenumber}}
            \input{0 - introduction/main.tex}
        \part{Research}
            \input{1 - low-noise PiC models/main.tex}
            \input{2 - kinetic component/main.tex}
            \input{3 - fluid component/main.tex}
            \input{4 - numerical implementation/main.tex}
        \part{Project Overview}
            \input{5 - research plan/main.tex}
            \input{6 - summary/main.tex}
    
    
    %\section{}
    \newpage
    \pagenumbering{gobble}
        \printbibliography


    \newpage
    \pagenumbering{roman}
    \appendix
        \part{Appendices}
            \input{8 - Hilbert complexes/main.tex}
            \input{9 - weak conservation proofs/main.tex}
\end{document}

    
    
    %\section{}
    \newpage
    \pagenumbering{gobble}
        \printbibliography


    \newpage
    \pagenumbering{roman}
    \appendix
        \part{Appendices}
            \documentclass[12pt, a4paper]{report}

\input{template/main.tex}

\title{\BA{Title in Progress...}}
\author{Boris Andrews}
\affil{Mathematical Institute, University of Oxford}
\date{\today}


\begin{document}
    \pagenumbering{gobble}
    \maketitle
    
    
    \begin{abstract}
        Magnetic confinement reactors---in particular tokamaks---offer one of the most promising options for achieving practical nuclear fusion, with the potential to provide virtually limitless, clean energy. The theoretical and numerical modeling of tokamak plasmas is simultaneously an essential component of effective reactor design, and a great research barrier. Tokamak operational conditions exhibit comparatively low Knudsen numbers. Kinetic effects, including kinetic waves and instabilities, Landau damping, bump-on-tail instabilities and more, are therefore highly influential in tokamak plasma dynamics. Purely fluid models are inherently incapable of capturing these effects, whereas the high dimensionality in purely kinetic models render them practically intractable for most relevant purposes.

        We consider a $\delta\!f$ decomposition model, with a macroscopic fluid background and microscopic kinetic correction, both fully coupled to each other. A similar manner of discretization is proposed to that used in the recent \texttt{STRUPHY} code \cite{Holderied_Possanner_Wang_2021, Holderied_2022, Li_et_al_2023} with a finite-element model for the background and a pseudo-particle/PiC model for the correction.

        The fluid background satisfies the full, non-linear, resistive, compressible, Hall MHD equations. \cite{Laakmann_Hu_Farrell_2022} introduces finite-element(-in-space) implicit timesteppers for the incompressible analogue to this system with structure-preserving (SP) properties in the ideal case, alongside parameter-robust preconditioners. We show that these timesteppers can derive from a finite-element-in-time (FET) (and finite-element-in-space) interpretation. The benefits of this reformulation are discussed, including the derivation of timesteppers that are higher order in time, and the quantifiable dissipative SP properties in the non-ideal, resistive case.
        
        We discuss possible options for extending this FET approach to timesteppers for the compressible case.

        The kinetic corrections satisfy linearized Boltzmann equations. Using a Lénard--Bernstein collision operator, these take Fokker--Planck-like forms \cite{Fokker_1914, Planck_1917} wherein pseudo-particles in the numerical model obey the neoclassical transport equations, with particle-independent Brownian drift terms. This offers a rigorous methodology for incorporating collisions into the particle transport model, without coupling the equations of motions for each particle.
        
        Works by Chen, Chacón et al. \cite{Chen_Chacón_Barnes_2011, Chacón_Chen_Barnes_2013, Chen_Chacón_2014, Chen_Chacón_2015} have developed structure-preserving particle pushers for neoclassical transport in the Vlasov equations, derived from Crank--Nicolson integrators. We show these too can can derive from a FET interpretation, similarly offering potential extensions to higher-order-in-time particle pushers. The FET formulation is used also to consider how the stochastic drift terms can be incorporated into the pushers. Stochastic gyrokinetic expansions are also discussed.

        Different options for the numerical implementation of these schemes are considered.

        Due to the efficacy of FET in the development of SP timesteppers for both the fluid and kinetic component, we hope this approach will prove effective in the future for developing SP timesteppers for the full hybrid model. We hope this will give us the opportunity to incorporate previously inaccessible kinetic effects into the highly effective, modern, finite-element MHD models.
    \end{abstract}
    
    
    \newpage
    \tableofcontents
    
    
    \newpage
    \pagenumbering{arabic}
    %\linenumbers\renewcommand\thelinenumber{\color{black!50}\arabic{linenumber}}
            \input{0 - introduction/main.tex}
        \part{Research}
            \input{1 - low-noise PiC models/main.tex}
            \input{2 - kinetic component/main.tex}
            \input{3 - fluid component/main.tex}
            \input{4 - numerical implementation/main.tex}
        \part{Project Overview}
            \input{5 - research plan/main.tex}
            \input{6 - summary/main.tex}
    
    
    %\section{}
    \newpage
    \pagenumbering{gobble}
        \printbibliography


    \newpage
    \pagenumbering{roman}
    \appendix
        \part{Appendices}
            \input{8 - Hilbert complexes/main.tex}
            \input{9 - weak conservation proofs/main.tex}
\end{document}

            \documentclass[12pt, a4paper]{report}

\input{template/main.tex}

\title{\BA{Title in Progress...}}
\author{Boris Andrews}
\affil{Mathematical Institute, University of Oxford}
\date{\today}


\begin{document}
    \pagenumbering{gobble}
    \maketitle
    
    
    \begin{abstract}
        Magnetic confinement reactors---in particular tokamaks---offer one of the most promising options for achieving practical nuclear fusion, with the potential to provide virtually limitless, clean energy. The theoretical and numerical modeling of tokamak plasmas is simultaneously an essential component of effective reactor design, and a great research barrier. Tokamak operational conditions exhibit comparatively low Knudsen numbers. Kinetic effects, including kinetic waves and instabilities, Landau damping, bump-on-tail instabilities and more, are therefore highly influential in tokamak plasma dynamics. Purely fluid models are inherently incapable of capturing these effects, whereas the high dimensionality in purely kinetic models render them practically intractable for most relevant purposes.

        We consider a $\delta\!f$ decomposition model, with a macroscopic fluid background and microscopic kinetic correction, both fully coupled to each other. A similar manner of discretization is proposed to that used in the recent \texttt{STRUPHY} code \cite{Holderied_Possanner_Wang_2021, Holderied_2022, Li_et_al_2023} with a finite-element model for the background and a pseudo-particle/PiC model for the correction.

        The fluid background satisfies the full, non-linear, resistive, compressible, Hall MHD equations. \cite{Laakmann_Hu_Farrell_2022} introduces finite-element(-in-space) implicit timesteppers for the incompressible analogue to this system with structure-preserving (SP) properties in the ideal case, alongside parameter-robust preconditioners. We show that these timesteppers can derive from a finite-element-in-time (FET) (and finite-element-in-space) interpretation. The benefits of this reformulation are discussed, including the derivation of timesteppers that are higher order in time, and the quantifiable dissipative SP properties in the non-ideal, resistive case.
        
        We discuss possible options for extending this FET approach to timesteppers for the compressible case.

        The kinetic corrections satisfy linearized Boltzmann equations. Using a Lénard--Bernstein collision operator, these take Fokker--Planck-like forms \cite{Fokker_1914, Planck_1917} wherein pseudo-particles in the numerical model obey the neoclassical transport equations, with particle-independent Brownian drift terms. This offers a rigorous methodology for incorporating collisions into the particle transport model, without coupling the equations of motions for each particle.
        
        Works by Chen, Chacón et al. \cite{Chen_Chacón_Barnes_2011, Chacón_Chen_Barnes_2013, Chen_Chacón_2014, Chen_Chacón_2015} have developed structure-preserving particle pushers for neoclassical transport in the Vlasov equations, derived from Crank--Nicolson integrators. We show these too can can derive from a FET interpretation, similarly offering potential extensions to higher-order-in-time particle pushers. The FET formulation is used also to consider how the stochastic drift terms can be incorporated into the pushers. Stochastic gyrokinetic expansions are also discussed.

        Different options for the numerical implementation of these schemes are considered.

        Due to the efficacy of FET in the development of SP timesteppers for both the fluid and kinetic component, we hope this approach will prove effective in the future for developing SP timesteppers for the full hybrid model. We hope this will give us the opportunity to incorporate previously inaccessible kinetic effects into the highly effective, modern, finite-element MHD models.
    \end{abstract}
    
    
    \newpage
    \tableofcontents
    
    
    \newpage
    \pagenumbering{arabic}
    %\linenumbers\renewcommand\thelinenumber{\color{black!50}\arabic{linenumber}}
            \input{0 - introduction/main.tex}
        \part{Research}
            \input{1 - low-noise PiC models/main.tex}
            \input{2 - kinetic component/main.tex}
            \input{3 - fluid component/main.tex}
            \input{4 - numerical implementation/main.tex}
        \part{Project Overview}
            \input{5 - research plan/main.tex}
            \input{6 - summary/main.tex}
    
    
    %\section{}
    \newpage
    \pagenumbering{gobble}
        \printbibliography


    \newpage
    \pagenumbering{roman}
    \appendix
        \part{Appendices}
            \input{8 - Hilbert complexes/main.tex}
            \input{9 - weak conservation proofs/main.tex}
\end{document}

\end{document}

            \documentclass[12pt, a4paper]{report}

\documentclass[12pt, a4paper]{report}

\input{template/main.tex}

\title{\BA{Title in Progress...}}
\author{Boris Andrews}
\affil{Mathematical Institute, University of Oxford}
\date{\today}


\begin{document}
    \pagenumbering{gobble}
    \maketitle
    
    
    \begin{abstract}
        Magnetic confinement reactors---in particular tokamaks---offer one of the most promising options for achieving practical nuclear fusion, with the potential to provide virtually limitless, clean energy. The theoretical and numerical modeling of tokamak plasmas is simultaneously an essential component of effective reactor design, and a great research barrier. Tokamak operational conditions exhibit comparatively low Knudsen numbers. Kinetic effects, including kinetic waves and instabilities, Landau damping, bump-on-tail instabilities and more, are therefore highly influential in tokamak plasma dynamics. Purely fluid models are inherently incapable of capturing these effects, whereas the high dimensionality in purely kinetic models render them practically intractable for most relevant purposes.

        We consider a $\delta\!f$ decomposition model, with a macroscopic fluid background and microscopic kinetic correction, both fully coupled to each other. A similar manner of discretization is proposed to that used in the recent \texttt{STRUPHY} code \cite{Holderied_Possanner_Wang_2021, Holderied_2022, Li_et_al_2023} with a finite-element model for the background and a pseudo-particle/PiC model for the correction.

        The fluid background satisfies the full, non-linear, resistive, compressible, Hall MHD equations. \cite{Laakmann_Hu_Farrell_2022} introduces finite-element(-in-space) implicit timesteppers for the incompressible analogue to this system with structure-preserving (SP) properties in the ideal case, alongside parameter-robust preconditioners. We show that these timesteppers can derive from a finite-element-in-time (FET) (and finite-element-in-space) interpretation. The benefits of this reformulation are discussed, including the derivation of timesteppers that are higher order in time, and the quantifiable dissipative SP properties in the non-ideal, resistive case.
        
        We discuss possible options for extending this FET approach to timesteppers for the compressible case.

        The kinetic corrections satisfy linearized Boltzmann equations. Using a Lénard--Bernstein collision operator, these take Fokker--Planck-like forms \cite{Fokker_1914, Planck_1917} wherein pseudo-particles in the numerical model obey the neoclassical transport equations, with particle-independent Brownian drift terms. This offers a rigorous methodology for incorporating collisions into the particle transport model, without coupling the equations of motions for each particle.
        
        Works by Chen, Chacón et al. \cite{Chen_Chacón_Barnes_2011, Chacón_Chen_Barnes_2013, Chen_Chacón_2014, Chen_Chacón_2015} have developed structure-preserving particle pushers for neoclassical transport in the Vlasov equations, derived from Crank--Nicolson integrators. We show these too can can derive from a FET interpretation, similarly offering potential extensions to higher-order-in-time particle pushers. The FET formulation is used also to consider how the stochastic drift terms can be incorporated into the pushers. Stochastic gyrokinetic expansions are also discussed.

        Different options for the numerical implementation of these schemes are considered.

        Due to the efficacy of FET in the development of SP timesteppers for both the fluid and kinetic component, we hope this approach will prove effective in the future for developing SP timesteppers for the full hybrid model. We hope this will give us the opportunity to incorporate previously inaccessible kinetic effects into the highly effective, modern, finite-element MHD models.
    \end{abstract}
    
    
    \newpage
    \tableofcontents
    
    
    \newpage
    \pagenumbering{arabic}
    %\linenumbers\renewcommand\thelinenumber{\color{black!50}\arabic{linenumber}}
            \input{0 - introduction/main.tex}
        \part{Research}
            \input{1 - low-noise PiC models/main.tex}
            \input{2 - kinetic component/main.tex}
            \input{3 - fluid component/main.tex}
            \input{4 - numerical implementation/main.tex}
        \part{Project Overview}
            \input{5 - research plan/main.tex}
            \input{6 - summary/main.tex}
    
    
    %\section{}
    \newpage
    \pagenumbering{gobble}
        \printbibliography


    \newpage
    \pagenumbering{roman}
    \appendix
        \part{Appendices}
            \input{8 - Hilbert complexes/main.tex}
            \input{9 - weak conservation proofs/main.tex}
\end{document}


\title{\BA{Title in Progress...}}
\author{Boris Andrews}
\affil{Mathematical Institute, University of Oxford}
\date{\today}


\begin{document}
    \pagenumbering{gobble}
    \maketitle
    
    
    \begin{abstract}
        Magnetic confinement reactors---in particular tokamaks---offer one of the most promising options for achieving practical nuclear fusion, with the potential to provide virtually limitless, clean energy. The theoretical and numerical modeling of tokamak plasmas is simultaneously an essential component of effective reactor design, and a great research barrier. Tokamak operational conditions exhibit comparatively low Knudsen numbers. Kinetic effects, including kinetic waves and instabilities, Landau damping, bump-on-tail instabilities and more, are therefore highly influential in tokamak plasma dynamics. Purely fluid models are inherently incapable of capturing these effects, whereas the high dimensionality in purely kinetic models render them practically intractable for most relevant purposes.

        We consider a $\delta\!f$ decomposition model, with a macroscopic fluid background and microscopic kinetic correction, both fully coupled to each other. A similar manner of discretization is proposed to that used in the recent \texttt{STRUPHY} code \cite{Holderied_Possanner_Wang_2021, Holderied_2022, Li_et_al_2023} with a finite-element model for the background and a pseudo-particle/PiC model for the correction.

        The fluid background satisfies the full, non-linear, resistive, compressible, Hall MHD equations. \cite{Laakmann_Hu_Farrell_2022} introduces finite-element(-in-space) implicit timesteppers for the incompressible analogue to this system with structure-preserving (SP) properties in the ideal case, alongside parameter-robust preconditioners. We show that these timesteppers can derive from a finite-element-in-time (FET) (and finite-element-in-space) interpretation. The benefits of this reformulation are discussed, including the derivation of timesteppers that are higher order in time, and the quantifiable dissipative SP properties in the non-ideal, resistive case.
        
        We discuss possible options for extending this FET approach to timesteppers for the compressible case.

        The kinetic corrections satisfy linearized Boltzmann equations. Using a Lénard--Bernstein collision operator, these take Fokker--Planck-like forms \cite{Fokker_1914, Planck_1917} wherein pseudo-particles in the numerical model obey the neoclassical transport equations, with particle-independent Brownian drift terms. This offers a rigorous methodology for incorporating collisions into the particle transport model, without coupling the equations of motions for each particle.
        
        Works by Chen, Chacón et al. \cite{Chen_Chacón_Barnes_2011, Chacón_Chen_Barnes_2013, Chen_Chacón_2014, Chen_Chacón_2015} have developed structure-preserving particle pushers for neoclassical transport in the Vlasov equations, derived from Crank--Nicolson integrators. We show these too can can derive from a FET interpretation, similarly offering potential extensions to higher-order-in-time particle pushers. The FET formulation is used also to consider how the stochastic drift terms can be incorporated into the pushers. Stochastic gyrokinetic expansions are also discussed.

        Different options for the numerical implementation of these schemes are considered.

        Due to the efficacy of FET in the development of SP timesteppers for both the fluid and kinetic component, we hope this approach will prove effective in the future for developing SP timesteppers for the full hybrid model. We hope this will give us the opportunity to incorporate previously inaccessible kinetic effects into the highly effective, modern, finite-element MHD models.
    \end{abstract}
    
    
    \newpage
    \tableofcontents
    
    
    \newpage
    \pagenumbering{arabic}
    %\linenumbers\renewcommand\thelinenumber{\color{black!50}\arabic{linenumber}}
            \documentclass[12pt, a4paper]{report}

\input{template/main.tex}

\title{\BA{Title in Progress...}}
\author{Boris Andrews}
\affil{Mathematical Institute, University of Oxford}
\date{\today}


\begin{document}
    \pagenumbering{gobble}
    \maketitle
    
    
    \begin{abstract}
        Magnetic confinement reactors---in particular tokamaks---offer one of the most promising options for achieving practical nuclear fusion, with the potential to provide virtually limitless, clean energy. The theoretical and numerical modeling of tokamak plasmas is simultaneously an essential component of effective reactor design, and a great research barrier. Tokamak operational conditions exhibit comparatively low Knudsen numbers. Kinetic effects, including kinetic waves and instabilities, Landau damping, bump-on-tail instabilities and more, are therefore highly influential in tokamak plasma dynamics. Purely fluid models are inherently incapable of capturing these effects, whereas the high dimensionality in purely kinetic models render them practically intractable for most relevant purposes.

        We consider a $\delta\!f$ decomposition model, with a macroscopic fluid background and microscopic kinetic correction, both fully coupled to each other. A similar manner of discretization is proposed to that used in the recent \texttt{STRUPHY} code \cite{Holderied_Possanner_Wang_2021, Holderied_2022, Li_et_al_2023} with a finite-element model for the background and a pseudo-particle/PiC model for the correction.

        The fluid background satisfies the full, non-linear, resistive, compressible, Hall MHD equations. \cite{Laakmann_Hu_Farrell_2022} introduces finite-element(-in-space) implicit timesteppers for the incompressible analogue to this system with structure-preserving (SP) properties in the ideal case, alongside parameter-robust preconditioners. We show that these timesteppers can derive from a finite-element-in-time (FET) (and finite-element-in-space) interpretation. The benefits of this reformulation are discussed, including the derivation of timesteppers that are higher order in time, and the quantifiable dissipative SP properties in the non-ideal, resistive case.
        
        We discuss possible options for extending this FET approach to timesteppers for the compressible case.

        The kinetic corrections satisfy linearized Boltzmann equations. Using a Lénard--Bernstein collision operator, these take Fokker--Planck-like forms \cite{Fokker_1914, Planck_1917} wherein pseudo-particles in the numerical model obey the neoclassical transport equations, with particle-independent Brownian drift terms. This offers a rigorous methodology for incorporating collisions into the particle transport model, without coupling the equations of motions for each particle.
        
        Works by Chen, Chacón et al. \cite{Chen_Chacón_Barnes_2011, Chacón_Chen_Barnes_2013, Chen_Chacón_2014, Chen_Chacón_2015} have developed structure-preserving particle pushers for neoclassical transport in the Vlasov equations, derived from Crank--Nicolson integrators. We show these too can can derive from a FET interpretation, similarly offering potential extensions to higher-order-in-time particle pushers. The FET formulation is used also to consider how the stochastic drift terms can be incorporated into the pushers. Stochastic gyrokinetic expansions are also discussed.

        Different options for the numerical implementation of these schemes are considered.

        Due to the efficacy of FET in the development of SP timesteppers for both the fluid and kinetic component, we hope this approach will prove effective in the future for developing SP timesteppers for the full hybrid model. We hope this will give us the opportunity to incorporate previously inaccessible kinetic effects into the highly effective, modern, finite-element MHD models.
    \end{abstract}
    
    
    \newpage
    \tableofcontents
    
    
    \newpage
    \pagenumbering{arabic}
    %\linenumbers\renewcommand\thelinenumber{\color{black!50}\arabic{linenumber}}
            \input{0 - introduction/main.tex}
        \part{Research}
            \input{1 - low-noise PiC models/main.tex}
            \input{2 - kinetic component/main.tex}
            \input{3 - fluid component/main.tex}
            \input{4 - numerical implementation/main.tex}
        \part{Project Overview}
            \input{5 - research plan/main.tex}
            \input{6 - summary/main.tex}
    
    
    %\section{}
    \newpage
    \pagenumbering{gobble}
        \printbibliography


    \newpage
    \pagenumbering{roman}
    \appendix
        \part{Appendices}
            \input{8 - Hilbert complexes/main.tex}
            \input{9 - weak conservation proofs/main.tex}
\end{document}

        \part{Research}
            \documentclass[12pt, a4paper]{report}

\input{template/main.tex}

\title{\BA{Title in Progress...}}
\author{Boris Andrews}
\affil{Mathematical Institute, University of Oxford}
\date{\today}


\begin{document}
    \pagenumbering{gobble}
    \maketitle
    
    
    \begin{abstract}
        Magnetic confinement reactors---in particular tokamaks---offer one of the most promising options for achieving practical nuclear fusion, with the potential to provide virtually limitless, clean energy. The theoretical and numerical modeling of tokamak plasmas is simultaneously an essential component of effective reactor design, and a great research barrier. Tokamak operational conditions exhibit comparatively low Knudsen numbers. Kinetic effects, including kinetic waves and instabilities, Landau damping, bump-on-tail instabilities and more, are therefore highly influential in tokamak plasma dynamics. Purely fluid models are inherently incapable of capturing these effects, whereas the high dimensionality in purely kinetic models render them practically intractable for most relevant purposes.

        We consider a $\delta\!f$ decomposition model, with a macroscopic fluid background and microscopic kinetic correction, both fully coupled to each other. A similar manner of discretization is proposed to that used in the recent \texttt{STRUPHY} code \cite{Holderied_Possanner_Wang_2021, Holderied_2022, Li_et_al_2023} with a finite-element model for the background and a pseudo-particle/PiC model for the correction.

        The fluid background satisfies the full, non-linear, resistive, compressible, Hall MHD equations. \cite{Laakmann_Hu_Farrell_2022} introduces finite-element(-in-space) implicit timesteppers for the incompressible analogue to this system with structure-preserving (SP) properties in the ideal case, alongside parameter-robust preconditioners. We show that these timesteppers can derive from a finite-element-in-time (FET) (and finite-element-in-space) interpretation. The benefits of this reformulation are discussed, including the derivation of timesteppers that are higher order in time, and the quantifiable dissipative SP properties in the non-ideal, resistive case.
        
        We discuss possible options for extending this FET approach to timesteppers for the compressible case.

        The kinetic corrections satisfy linearized Boltzmann equations. Using a Lénard--Bernstein collision operator, these take Fokker--Planck-like forms \cite{Fokker_1914, Planck_1917} wherein pseudo-particles in the numerical model obey the neoclassical transport equations, with particle-independent Brownian drift terms. This offers a rigorous methodology for incorporating collisions into the particle transport model, without coupling the equations of motions for each particle.
        
        Works by Chen, Chacón et al. \cite{Chen_Chacón_Barnes_2011, Chacón_Chen_Barnes_2013, Chen_Chacón_2014, Chen_Chacón_2015} have developed structure-preserving particle pushers for neoclassical transport in the Vlasov equations, derived from Crank--Nicolson integrators. We show these too can can derive from a FET interpretation, similarly offering potential extensions to higher-order-in-time particle pushers. The FET formulation is used also to consider how the stochastic drift terms can be incorporated into the pushers. Stochastic gyrokinetic expansions are also discussed.

        Different options for the numerical implementation of these schemes are considered.

        Due to the efficacy of FET in the development of SP timesteppers for both the fluid and kinetic component, we hope this approach will prove effective in the future for developing SP timesteppers for the full hybrid model. We hope this will give us the opportunity to incorporate previously inaccessible kinetic effects into the highly effective, modern, finite-element MHD models.
    \end{abstract}
    
    
    \newpage
    \tableofcontents
    
    
    \newpage
    \pagenumbering{arabic}
    %\linenumbers\renewcommand\thelinenumber{\color{black!50}\arabic{linenumber}}
            \input{0 - introduction/main.tex}
        \part{Research}
            \input{1 - low-noise PiC models/main.tex}
            \input{2 - kinetic component/main.tex}
            \input{3 - fluid component/main.tex}
            \input{4 - numerical implementation/main.tex}
        \part{Project Overview}
            \input{5 - research plan/main.tex}
            \input{6 - summary/main.tex}
    
    
    %\section{}
    \newpage
    \pagenumbering{gobble}
        \printbibliography


    \newpage
    \pagenumbering{roman}
    \appendix
        \part{Appendices}
            \input{8 - Hilbert complexes/main.tex}
            \input{9 - weak conservation proofs/main.tex}
\end{document}

            \documentclass[12pt, a4paper]{report}

\input{template/main.tex}

\title{\BA{Title in Progress...}}
\author{Boris Andrews}
\affil{Mathematical Institute, University of Oxford}
\date{\today}


\begin{document}
    \pagenumbering{gobble}
    \maketitle
    
    
    \begin{abstract}
        Magnetic confinement reactors---in particular tokamaks---offer one of the most promising options for achieving practical nuclear fusion, with the potential to provide virtually limitless, clean energy. The theoretical and numerical modeling of tokamak plasmas is simultaneously an essential component of effective reactor design, and a great research barrier. Tokamak operational conditions exhibit comparatively low Knudsen numbers. Kinetic effects, including kinetic waves and instabilities, Landau damping, bump-on-tail instabilities and more, are therefore highly influential in tokamak plasma dynamics. Purely fluid models are inherently incapable of capturing these effects, whereas the high dimensionality in purely kinetic models render them practically intractable for most relevant purposes.

        We consider a $\delta\!f$ decomposition model, with a macroscopic fluid background and microscopic kinetic correction, both fully coupled to each other. A similar manner of discretization is proposed to that used in the recent \texttt{STRUPHY} code \cite{Holderied_Possanner_Wang_2021, Holderied_2022, Li_et_al_2023} with a finite-element model for the background and a pseudo-particle/PiC model for the correction.

        The fluid background satisfies the full, non-linear, resistive, compressible, Hall MHD equations. \cite{Laakmann_Hu_Farrell_2022} introduces finite-element(-in-space) implicit timesteppers for the incompressible analogue to this system with structure-preserving (SP) properties in the ideal case, alongside parameter-robust preconditioners. We show that these timesteppers can derive from a finite-element-in-time (FET) (and finite-element-in-space) interpretation. The benefits of this reformulation are discussed, including the derivation of timesteppers that are higher order in time, and the quantifiable dissipative SP properties in the non-ideal, resistive case.
        
        We discuss possible options for extending this FET approach to timesteppers for the compressible case.

        The kinetic corrections satisfy linearized Boltzmann equations. Using a Lénard--Bernstein collision operator, these take Fokker--Planck-like forms \cite{Fokker_1914, Planck_1917} wherein pseudo-particles in the numerical model obey the neoclassical transport equations, with particle-independent Brownian drift terms. This offers a rigorous methodology for incorporating collisions into the particle transport model, without coupling the equations of motions for each particle.
        
        Works by Chen, Chacón et al. \cite{Chen_Chacón_Barnes_2011, Chacón_Chen_Barnes_2013, Chen_Chacón_2014, Chen_Chacón_2015} have developed structure-preserving particle pushers for neoclassical transport in the Vlasov equations, derived from Crank--Nicolson integrators. We show these too can can derive from a FET interpretation, similarly offering potential extensions to higher-order-in-time particle pushers. The FET formulation is used also to consider how the stochastic drift terms can be incorporated into the pushers. Stochastic gyrokinetic expansions are also discussed.

        Different options for the numerical implementation of these schemes are considered.

        Due to the efficacy of FET in the development of SP timesteppers for both the fluid and kinetic component, we hope this approach will prove effective in the future for developing SP timesteppers for the full hybrid model. We hope this will give us the opportunity to incorporate previously inaccessible kinetic effects into the highly effective, modern, finite-element MHD models.
    \end{abstract}
    
    
    \newpage
    \tableofcontents
    
    
    \newpage
    \pagenumbering{arabic}
    %\linenumbers\renewcommand\thelinenumber{\color{black!50}\arabic{linenumber}}
            \input{0 - introduction/main.tex}
        \part{Research}
            \input{1 - low-noise PiC models/main.tex}
            \input{2 - kinetic component/main.tex}
            \input{3 - fluid component/main.tex}
            \input{4 - numerical implementation/main.tex}
        \part{Project Overview}
            \input{5 - research plan/main.tex}
            \input{6 - summary/main.tex}
    
    
    %\section{}
    \newpage
    \pagenumbering{gobble}
        \printbibliography


    \newpage
    \pagenumbering{roman}
    \appendix
        \part{Appendices}
            \input{8 - Hilbert complexes/main.tex}
            \input{9 - weak conservation proofs/main.tex}
\end{document}

            \documentclass[12pt, a4paper]{report}

\input{template/main.tex}

\title{\BA{Title in Progress...}}
\author{Boris Andrews}
\affil{Mathematical Institute, University of Oxford}
\date{\today}


\begin{document}
    \pagenumbering{gobble}
    \maketitle
    
    
    \begin{abstract}
        Magnetic confinement reactors---in particular tokamaks---offer one of the most promising options for achieving practical nuclear fusion, with the potential to provide virtually limitless, clean energy. The theoretical and numerical modeling of tokamak plasmas is simultaneously an essential component of effective reactor design, and a great research barrier. Tokamak operational conditions exhibit comparatively low Knudsen numbers. Kinetic effects, including kinetic waves and instabilities, Landau damping, bump-on-tail instabilities and more, are therefore highly influential in tokamak plasma dynamics. Purely fluid models are inherently incapable of capturing these effects, whereas the high dimensionality in purely kinetic models render them practically intractable for most relevant purposes.

        We consider a $\delta\!f$ decomposition model, with a macroscopic fluid background and microscopic kinetic correction, both fully coupled to each other. A similar manner of discretization is proposed to that used in the recent \texttt{STRUPHY} code \cite{Holderied_Possanner_Wang_2021, Holderied_2022, Li_et_al_2023} with a finite-element model for the background and a pseudo-particle/PiC model for the correction.

        The fluid background satisfies the full, non-linear, resistive, compressible, Hall MHD equations. \cite{Laakmann_Hu_Farrell_2022} introduces finite-element(-in-space) implicit timesteppers for the incompressible analogue to this system with structure-preserving (SP) properties in the ideal case, alongside parameter-robust preconditioners. We show that these timesteppers can derive from a finite-element-in-time (FET) (and finite-element-in-space) interpretation. The benefits of this reformulation are discussed, including the derivation of timesteppers that are higher order in time, and the quantifiable dissipative SP properties in the non-ideal, resistive case.
        
        We discuss possible options for extending this FET approach to timesteppers for the compressible case.

        The kinetic corrections satisfy linearized Boltzmann equations. Using a Lénard--Bernstein collision operator, these take Fokker--Planck-like forms \cite{Fokker_1914, Planck_1917} wherein pseudo-particles in the numerical model obey the neoclassical transport equations, with particle-independent Brownian drift terms. This offers a rigorous methodology for incorporating collisions into the particle transport model, without coupling the equations of motions for each particle.
        
        Works by Chen, Chacón et al. \cite{Chen_Chacón_Barnes_2011, Chacón_Chen_Barnes_2013, Chen_Chacón_2014, Chen_Chacón_2015} have developed structure-preserving particle pushers for neoclassical transport in the Vlasov equations, derived from Crank--Nicolson integrators. We show these too can can derive from a FET interpretation, similarly offering potential extensions to higher-order-in-time particle pushers. The FET formulation is used also to consider how the stochastic drift terms can be incorporated into the pushers. Stochastic gyrokinetic expansions are also discussed.

        Different options for the numerical implementation of these schemes are considered.

        Due to the efficacy of FET in the development of SP timesteppers for both the fluid and kinetic component, we hope this approach will prove effective in the future for developing SP timesteppers for the full hybrid model. We hope this will give us the opportunity to incorporate previously inaccessible kinetic effects into the highly effective, modern, finite-element MHD models.
    \end{abstract}
    
    
    \newpage
    \tableofcontents
    
    
    \newpage
    \pagenumbering{arabic}
    %\linenumbers\renewcommand\thelinenumber{\color{black!50}\arabic{linenumber}}
            \input{0 - introduction/main.tex}
        \part{Research}
            \input{1 - low-noise PiC models/main.tex}
            \input{2 - kinetic component/main.tex}
            \input{3 - fluid component/main.tex}
            \input{4 - numerical implementation/main.tex}
        \part{Project Overview}
            \input{5 - research plan/main.tex}
            \input{6 - summary/main.tex}
    
    
    %\section{}
    \newpage
    \pagenumbering{gobble}
        \printbibliography


    \newpage
    \pagenumbering{roman}
    \appendix
        \part{Appendices}
            \input{8 - Hilbert complexes/main.tex}
            \input{9 - weak conservation proofs/main.tex}
\end{document}

            \documentclass[12pt, a4paper]{report}

\input{template/main.tex}

\title{\BA{Title in Progress...}}
\author{Boris Andrews}
\affil{Mathematical Institute, University of Oxford}
\date{\today}


\begin{document}
    \pagenumbering{gobble}
    \maketitle
    
    
    \begin{abstract}
        Magnetic confinement reactors---in particular tokamaks---offer one of the most promising options for achieving practical nuclear fusion, with the potential to provide virtually limitless, clean energy. The theoretical and numerical modeling of tokamak plasmas is simultaneously an essential component of effective reactor design, and a great research barrier. Tokamak operational conditions exhibit comparatively low Knudsen numbers. Kinetic effects, including kinetic waves and instabilities, Landau damping, bump-on-tail instabilities and more, are therefore highly influential in tokamak plasma dynamics. Purely fluid models are inherently incapable of capturing these effects, whereas the high dimensionality in purely kinetic models render them practically intractable for most relevant purposes.

        We consider a $\delta\!f$ decomposition model, with a macroscopic fluid background and microscopic kinetic correction, both fully coupled to each other. A similar manner of discretization is proposed to that used in the recent \texttt{STRUPHY} code \cite{Holderied_Possanner_Wang_2021, Holderied_2022, Li_et_al_2023} with a finite-element model for the background and a pseudo-particle/PiC model for the correction.

        The fluid background satisfies the full, non-linear, resistive, compressible, Hall MHD equations. \cite{Laakmann_Hu_Farrell_2022} introduces finite-element(-in-space) implicit timesteppers for the incompressible analogue to this system with structure-preserving (SP) properties in the ideal case, alongside parameter-robust preconditioners. We show that these timesteppers can derive from a finite-element-in-time (FET) (and finite-element-in-space) interpretation. The benefits of this reformulation are discussed, including the derivation of timesteppers that are higher order in time, and the quantifiable dissipative SP properties in the non-ideal, resistive case.
        
        We discuss possible options for extending this FET approach to timesteppers for the compressible case.

        The kinetic corrections satisfy linearized Boltzmann equations. Using a Lénard--Bernstein collision operator, these take Fokker--Planck-like forms \cite{Fokker_1914, Planck_1917} wherein pseudo-particles in the numerical model obey the neoclassical transport equations, with particle-independent Brownian drift terms. This offers a rigorous methodology for incorporating collisions into the particle transport model, without coupling the equations of motions for each particle.
        
        Works by Chen, Chacón et al. \cite{Chen_Chacón_Barnes_2011, Chacón_Chen_Barnes_2013, Chen_Chacón_2014, Chen_Chacón_2015} have developed structure-preserving particle pushers for neoclassical transport in the Vlasov equations, derived from Crank--Nicolson integrators. We show these too can can derive from a FET interpretation, similarly offering potential extensions to higher-order-in-time particle pushers. The FET formulation is used also to consider how the stochastic drift terms can be incorporated into the pushers. Stochastic gyrokinetic expansions are also discussed.

        Different options for the numerical implementation of these schemes are considered.

        Due to the efficacy of FET in the development of SP timesteppers for both the fluid and kinetic component, we hope this approach will prove effective in the future for developing SP timesteppers for the full hybrid model. We hope this will give us the opportunity to incorporate previously inaccessible kinetic effects into the highly effective, modern, finite-element MHD models.
    \end{abstract}
    
    
    \newpage
    \tableofcontents
    
    
    \newpage
    \pagenumbering{arabic}
    %\linenumbers\renewcommand\thelinenumber{\color{black!50}\arabic{linenumber}}
            \input{0 - introduction/main.tex}
        \part{Research}
            \input{1 - low-noise PiC models/main.tex}
            \input{2 - kinetic component/main.tex}
            \input{3 - fluid component/main.tex}
            \input{4 - numerical implementation/main.tex}
        \part{Project Overview}
            \input{5 - research plan/main.tex}
            \input{6 - summary/main.tex}
    
    
    %\section{}
    \newpage
    \pagenumbering{gobble}
        \printbibliography


    \newpage
    \pagenumbering{roman}
    \appendix
        \part{Appendices}
            \input{8 - Hilbert complexes/main.tex}
            \input{9 - weak conservation proofs/main.tex}
\end{document}

        \part{Project Overview}
            \documentclass[12pt, a4paper]{report}

\input{template/main.tex}

\title{\BA{Title in Progress...}}
\author{Boris Andrews}
\affil{Mathematical Institute, University of Oxford}
\date{\today}


\begin{document}
    \pagenumbering{gobble}
    \maketitle
    
    
    \begin{abstract}
        Magnetic confinement reactors---in particular tokamaks---offer one of the most promising options for achieving practical nuclear fusion, with the potential to provide virtually limitless, clean energy. The theoretical and numerical modeling of tokamak plasmas is simultaneously an essential component of effective reactor design, and a great research barrier. Tokamak operational conditions exhibit comparatively low Knudsen numbers. Kinetic effects, including kinetic waves and instabilities, Landau damping, bump-on-tail instabilities and more, are therefore highly influential in tokamak plasma dynamics. Purely fluid models are inherently incapable of capturing these effects, whereas the high dimensionality in purely kinetic models render them practically intractable for most relevant purposes.

        We consider a $\delta\!f$ decomposition model, with a macroscopic fluid background and microscopic kinetic correction, both fully coupled to each other. A similar manner of discretization is proposed to that used in the recent \texttt{STRUPHY} code \cite{Holderied_Possanner_Wang_2021, Holderied_2022, Li_et_al_2023} with a finite-element model for the background and a pseudo-particle/PiC model for the correction.

        The fluid background satisfies the full, non-linear, resistive, compressible, Hall MHD equations. \cite{Laakmann_Hu_Farrell_2022} introduces finite-element(-in-space) implicit timesteppers for the incompressible analogue to this system with structure-preserving (SP) properties in the ideal case, alongside parameter-robust preconditioners. We show that these timesteppers can derive from a finite-element-in-time (FET) (and finite-element-in-space) interpretation. The benefits of this reformulation are discussed, including the derivation of timesteppers that are higher order in time, and the quantifiable dissipative SP properties in the non-ideal, resistive case.
        
        We discuss possible options for extending this FET approach to timesteppers for the compressible case.

        The kinetic corrections satisfy linearized Boltzmann equations. Using a Lénard--Bernstein collision operator, these take Fokker--Planck-like forms \cite{Fokker_1914, Planck_1917} wherein pseudo-particles in the numerical model obey the neoclassical transport equations, with particle-independent Brownian drift terms. This offers a rigorous methodology for incorporating collisions into the particle transport model, without coupling the equations of motions for each particle.
        
        Works by Chen, Chacón et al. \cite{Chen_Chacón_Barnes_2011, Chacón_Chen_Barnes_2013, Chen_Chacón_2014, Chen_Chacón_2015} have developed structure-preserving particle pushers for neoclassical transport in the Vlasov equations, derived from Crank--Nicolson integrators. We show these too can can derive from a FET interpretation, similarly offering potential extensions to higher-order-in-time particle pushers. The FET formulation is used also to consider how the stochastic drift terms can be incorporated into the pushers. Stochastic gyrokinetic expansions are also discussed.

        Different options for the numerical implementation of these schemes are considered.

        Due to the efficacy of FET in the development of SP timesteppers for both the fluid and kinetic component, we hope this approach will prove effective in the future for developing SP timesteppers for the full hybrid model. We hope this will give us the opportunity to incorporate previously inaccessible kinetic effects into the highly effective, modern, finite-element MHD models.
    \end{abstract}
    
    
    \newpage
    \tableofcontents
    
    
    \newpage
    \pagenumbering{arabic}
    %\linenumbers\renewcommand\thelinenumber{\color{black!50}\arabic{linenumber}}
            \input{0 - introduction/main.tex}
        \part{Research}
            \input{1 - low-noise PiC models/main.tex}
            \input{2 - kinetic component/main.tex}
            \input{3 - fluid component/main.tex}
            \input{4 - numerical implementation/main.tex}
        \part{Project Overview}
            \input{5 - research plan/main.tex}
            \input{6 - summary/main.tex}
    
    
    %\section{}
    \newpage
    \pagenumbering{gobble}
        \printbibliography


    \newpage
    \pagenumbering{roman}
    \appendix
        \part{Appendices}
            \input{8 - Hilbert complexes/main.tex}
            \input{9 - weak conservation proofs/main.tex}
\end{document}

            \documentclass[12pt, a4paper]{report}

\input{template/main.tex}

\title{\BA{Title in Progress...}}
\author{Boris Andrews}
\affil{Mathematical Institute, University of Oxford}
\date{\today}


\begin{document}
    \pagenumbering{gobble}
    \maketitle
    
    
    \begin{abstract}
        Magnetic confinement reactors---in particular tokamaks---offer one of the most promising options for achieving practical nuclear fusion, with the potential to provide virtually limitless, clean energy. The theoretical and numerical modeling of tokamak plasmas is simultaneously an essential component of effective reactor design, and a great research barrier. Tokamak operational conditions exhibit comparatively low Knudsen numbers. Kinetic effects, including kinetic waves and instabilities, Landau damping, bump-on-tail instabilities and more, are therefore highly influential in tokamak plasma dynamics. Purely fluid models are inherently incapable of capturing these effects, whereas the high dimensionality in purely kinetic models render them practically intractable for most relevant purposes.

        We consider a $\delta\!f$ decomposition model, with a macroscopic fluid background and microscopic kinetic correction, both fully coupled to each other. A similar manner of discretization is proposed to that used in the recent \texttt{STRUPHY} code \cite{Holderied_Possanner_Wang_2021, Holderied_2022, Li_et_al_2023} with a finite-element model for the background and a pseudo-particle/PiC model for the correction.

        The fluid background satisfies the full, non-linear, resistive, compressible, Hall MHD equations. \cite{Laakmann_Hu_Farrell_2022} introduces finite-element(-in-space) implicit timesteppers for the incompressible analogue to this system with structure-preserving (SP) properties in the ideal case, alongside parameter-robust preconditioners. We show that these timesteppers can derive from a finite-element-in-time (FET) (and finite-element-in-space) interpretation. The benefits of this reformulation are discussed, including the derivation of timesteppers that are higher order in time, and the quantifiable dissipative SP properties in the non-ideal, resistive case.
        
        We discuss possible options for extending this FET approach to timesteppers for the compressible case.

        The kinetic corrections satisfy linearized Boltzmann equations. Using a Lénard--Bernstein collision operator, these take Fokker--Planck-like forms \cite{Fokker_1914, Planck_1917} wherein pseudo-particles in the numerical model obey the neoclassical transport equations, with particle-independent Brownian drift terms. This offers a rigorous methodology for incorporating collisions into the particle transport model, without coupling the equations of motions for each particle.
        
        Works by Chen, Chacón et al. \cite{Chen_Chacón_Barnes_2011, Chacón_Chen_Barnes_2013, Chen_Chacón_2014, Chen_Chacón_2015} have developed structure-preserving particle pushers for neoclassical transport in the Vlasov equations, derived from Crank--Nicolson integrators. We show these too can can derive from a FET interpretation, similarly offering potential extensions to higher-order-in-time particle pushers. The FET formulation is used also to consider how the stochastic drift terms can be incorporated into the pushers. Stochastic gyrokinetic expansions are also discussed.

        Different options for the numerical implementation of these schemes are considered.

        Due to the efficacy of FET in the development of SP timesteppers for both the fluid and kinetic component, we hope this approach will prove effective in the future for developing SP timesteppers for the full hybrid model. We hope this will give us the opportunity to incorporate previously inaccessible kinetic effects into the highly effective, modern, finite-element MHD models.
    \end{abstract}
    
    
    \newpage
    \tableofcontents
    
    
    \newpage
    \pagenumbering{arabic}
    %\linenumbers\renewcommand\thelinenumber{\color{black!50}\arabic{linenumber}}
            \input{0 - introduction/main.tex}
        \part{Research}
            \input{1 - low-noise PiC models/main.tex}
            \input{2 - kinetic component/main.tex}
            \input{3 - fluid component/main.tex}
            \input{4 - numerical implementation/main.tex}
        \part{Project Overview}
            \input{5 - research plan/main.tex}
            \input{6 - summary/main.tex}
    
    
    %\section{}
    \newpage
    \pagenumbering{gobble}
        \printbibliography


    \newpage
    \pagenumbering{roman}
    \appendix
        \part{Appendices}
            \input{8 - Hilbert complexes/main.tex}
            \input{9 - weak conservation proofs/main.tex}
\end{document}

    
    
    %\section{}
    \newpage
    \pagenumbering{gobble}
        \printbibliography


    \newpage
    \pagenumbering{roman}
    \appendix
        \part{Appendices}
            \documentclass[12pt, a4paper]{report}

\input{template/main.tex}

\title{\BA{Title in Progress...}}
\author{Boris Andrews}
\affil{Mathematical Institute, University of Oxford}
\date{\today}


\begin{document}
    \pagenumbering{gobble}
    \maketitle
    
    
    \begin{abstract}
        Magnetic confinement reactors---in particular tokamaks---offer one of the most promising options for achieving practical nuclear fusion, with the potential to provide virtually limitless, clean energy. The theoretical and numerical modeling of tokamak plasmas is simultaneously an essential component of effective reactor design, and a great research barrier. Tokamak operational conditions exhibit comparatively low Knudsen numbers. Kinetic effects, including kinetic waves and instabilities, Landau damping, bump-on-tail instabilities and more, are therefore highly influential in tokamak plasma dynamics. Purely fluid models are inherently incapable of capturing these effects, whereas the high dimensionality in purely kinetic models render them practically intractable for most relevant purposes.

        We consider a $\delta\!f$ decomposition model, with a macroscopic fluid background and microscopic kinetic correction, both fully coupled to each other. A similar manner of discretization is proposed to that used in the recent \texttt{STRUPHY} code \cite{Holderied_Possanner_Wang_2021, Holderied_2022, Li_et_al_2023} with a finite-element model for the background and a pseudo-particle/PiC model for the correction.

        The fluid background satisfies the full, non-linear, resistive, compressible, Hall MHD equations. \cite{Laakmann_Hu_Farrell_2022} introduces finite-element(-in-space) implicit timesteppers for the incompressible analogue to this system with structure-preserving (SP) properties in the ideal case, alongside parameter-robust preconditioners. We show that these timesteppers can derive from a finite-element-in-time (FET) (and finite-element-in-space) interpretation. The benefits of this reformulation are discussed, including the derivation of timesteppers that are higher order in time, and the quantifiable dissipative SP properties in the non-ideal, resistive case.
        
        We discuss possible options for extending this FET approach to timesteppers for the compressible case.

        The kinetic corrections satisfy linearized Boltzmann equations. Using a Lénard--Bernstein collision operator, these take Fokker--Planck-like forms \cite{Fokker_1914, Planck_1917} wherein pseudo-particles in the numerical model obey the neoclassical transport equations, with particle-independent Brownian drift terms. This offers a rigorous methodology for incorporating collisions into the particle transport model, without coupling the equations of motions for each particle.
        
        Works by Chen, Chacón et al. \cite{Chen_Chacón_Barnes_2011, Chacón_Chen_Barnes_2013, Chen_Chacón_2014, Chen_Chacón_2015} have developed structure-preserving particle pushers for neoclassical transport in the Vlasov equations, derived from Crank--Nicolson integrators. We show these too can can derive from a FET interpretation, similarly offering potential extensions to higher-order-in-time particle pushers. The FET formulation is used also to consider how the stochastic drift terms can be incorporated into the pushers. Stochastic gyrokinetic expansions are also discussed.

        Different options for the numerical implementation of these schemes are considered.

        Due to the efficacy of FET in the development of SP timesteppers for both the fluid and kinetic component, we hope this approach will prove effective in the future for developing SP timesteppers for the full hybrid model. We hope this will give us the opportunity to incorporate previously inaccessible kinetic effects into the highly effective, modern, finite-element MHD models.
    \end{abstract}
    
    
    \newpage
    \tableofcontents
    
    
    \newpage
    \pagenumbering{arabic}
    %\linenumbers\renewcommand\thelinenumber{\color{black!50}\arabic{linenumber}}
            \input{0 - introduction/main.tex}
        \part{Research}
            \input{1 - low-noise PiC models/main.tex}
            \input{2 - kinetic component/main.tex}
            \input{3 - fluid component/main.tex}
            \input{4 - numerical implementation/main.tex}
        \part{Project Overview}
            \input{5 - research plan/main.tex}
            \input{6 - summary/main.tex}
    
    
    %\section{}
    \newpage
    \pagenumbering{gobble}
        \printbibliography


    \newpage
    \pagenumbering{roman}
    \appendix
        \part{Appendices}
            \input{8 - Hilbert complexes/main.tex}
            \input{9 - weak conservation proofs/main.tex}
\end{document}

            \documentclass[12pt, a4paper]{report}

\input{template/main.tex}

\title{\BA{Title in Progress...}}
\author{Boris Andrews}
\affil{Mathematical Institute, University of Oxford}
\date{\today}


\begin{document}
    \pagenumbering{gobble}
    \maketitle
    
    
    \begin{abstract}
        Magnetic confinement reactors---in particular tokamaks---offer one of the most promising options for achieving practical nuclear fusion, with the potential to provide virtually limitless, clean energy. The theoretical and numerical modeling of tokamak plasmas is simultaneously an essential component of effective reactor design, and a great research barrier. Tokamak operational conditions exhibit comparatively low Knudsen numbers. Kinetic effects, including kinetic waves and instabilities, Landau damping, bump-on-tail instabilities and more, are therefore highly influential in tokamak plasma dynamics. Purely fluid models are inherently incapable of capturing these effects, whereas the high dimensionality in purely kinetic models render them practically intractable for most relevant purposes.

        We consider a $\delta\!f$ decomposition model, with a macroscopic fluid background and microscopic kinetic correction, both fully coupled to each other. A similar manner of discretization is proposed to that used in the recent \texttt{STRUPHY} code \cite{Holderied_Possanner_Wang_2021, Holderied_2022, Li_et_al_2023} with a finite-element model for the background and a pseudo-particle/PiC model for the correction.

        The fluid background satisfies the full, non-linear, resistive, compressible, Hall MHD equations. \cite{Laakmann_Hu_Farrell_2022} introduces finite-element(-in-space) implicit timesteppers for the incompressible analogue to this system with structure-preserving (SP) properties in the ideal case, alongside parameter-robust preconditioners. We show that these timesteppers can derive from a finite-element-in-time (FET) (and finite-element-in-space) interpretation. The benefits of this reformulation are discussed, including the derivation of timesteppers that are higher order in time, and the quantifiable dissipative SP properties in the non-ideal, resistive case.
        
        We discuss possible options for extending this FET approach to timesteppers for the compressible case.

        The kinetic corrections satisfy linearized Boltzmann equations. Using a Lénard--Bernstein collision operator, these take Fokker--Planck-like forms \cite{Fokker_1914, Planck_1917} wherein pseudo-particles in the numerical model obey the neoclassical transport equations, with particle-independent Brownian drift terms. This offers a rigorous methodology for incorporating collisions into the particle transport model, without coupling the equations of motions for each particle.
        
        Works by Chen, Chacón et al. \cite{Chen_Chacón_Barnes_2011, Chacón_Chen_Barnes_2013, Chen_Chacón_2014, Chen_Chacón_2015} have developed structure-preserving particle pushers for neoclassical transport in the Vlasov equations, derived from Crank--Nicolson integrators. We show these too can can derive from a FET interpretation, similarly offering potential extensions to higher-order-in-time particle pushers. The FET formulation is used also to consider how the stochastic drift terms can be incorporated into the pushers. Stochastic gyrokinetic expansions are also discussed.

        Different options for the numerical implementation of these schemes are considered.

        Due to the efficacy of FET in the development of SP timesteppers for both the fluid and kinetic component, we hope this approach will prove effective in the future for developing SP timesteppers for the full hybrid model. We hope this will give us the opportunity to incorporate previously inaccessible kinetic effects into the highly effective, modern, finite-element MHD models.
    \end{abstract}
    
    
    \newpage
    \tableofcontents
    
    
    \newpage
    \pagenumbering{arabic}
    %\linenumbers\renewcommand\thelinenumber{\color{black!50}\arabic{linenumber}}
            \input{0 - introduction/main.tex}
        \part{Research}
            \input{1 - low-noise PiC models/main.tex}
            \input{2 - kinetic component/main.tex}
            \input{3 - fluid component/main.tex}
            \input{4 - numerical implementation/main.tex}
        \part{Project Overview}
            \input{5 - research plan/main.tex}
            \input{6 - summary/main.tex}
    
    
    %\section{}
    \newpage
    \pagenumbering{gobble}
        \printbibliography


    \newpage
    \pagenumbering{roman}
    \appendix
        \part{Appendices}
            \input{8 - Hilbert complexes/main.tex}
            \input{9 - weak conservation proofs/main.tex}
\end{document}

\end{document}

    
    
    %\section{}
    \newpage
    \pagenumbering{gobble}
        \printbibliography


    \newpage
    \pagenumbering{roman}
    \appendix
        \part{Appendices}
            \documentclass[12pt, a4paper]{report}

\documentclass[12pt, a4paper]{report}

\input{template/main.tex}

\title{\BA{Title in Progress...}}
\author{Boris Andrews}
\affil{Mathematical Institute, University of Oxford}
\date{\today}


\begin{document}
    \pagenumbering{gobble}
    \maketitle
    
    
    \begin{abstract}
        Magnetic confinement reactors---in particular tokamaks---offer one of the most promising options for achieving practical nuclear fusion, with the potential to provide virtually limitless, clean energy. The theoretical and numerical modeling of tokamak plasmas is simultaneously an essential component of effective reactor design, and a great research barrier. Tokamak operational conditions exhibit comparatively low Knudsen numbers. Kinetic effects, including kinetic waves and instabilities, Landau damping, bump-on-tail instabilities and more, are therefore highly influential in tokamak plasma dynamics. Purely fluid models are inherently incapable of capturing these effects, whereas the high dimensionality in purely kinetic models render them practically intractable for most relevant purposes.

        We consider a $\delta\!f$ decomposition model, with a macroscopic fluid background and microscopic kinetic correction, both fully coupled to each other. A similar manner of discretization is proposed to that used in the recent \texttt{STRUPHY} code \cite{Holderied_Possanner_Wang_2021, Holderied_2022, Li_et_al_2023} with a finite-element model for the background and a pseudo-particle/PiC model for the correction.

        The fluid background satisfies the full, non-linear, resistive, compressible, Hall MHD equations. \cite{Laakmann_Hu_Farrell_2022} introduces finite-element(-in-space) implicit timesteppers for the incompressible analogue to this system with structure-preserving (SP) properties in the ideal case, alongside parameter-robust preconditioners. We show that these timesteppers can derive from a finite-element-in-time (FET) (and finite-element-in-space) interpretation. The benefits of this reformulation are discussed, including the derivation of timesteppers that are higher order in time, and the quantifiable dissipative SP properties in the non-ideal, resistive case.
        
        We discuss possible options for extending this FET approach to timesteppers for the compressible case.

        The kinetic corrections satisfy linearized Boltzmann equations. Using a Lénard--Bernstein collision operator, these take Fokker--Planck-like forms \cite{Fokker_1914, Planck_1917} wherein pseudo-particles in the numerical model obey the neoclassical transport equations, with particle-independent Brownian drift terms. This offers a rigorous methodology for incorporating collisions into the particle transport model, without coupling the equations of motions for each particle.
        
        Works by Chen, Chacón et al. \cite{Chen_Chacón_Barnes_2011, Chacón_Chen_Barnes_2013, Chen_Chacón_2014, Chen_Chacón_2015} have developed structure-preserving particle pushers for neoclassical transport in the Vlasov equations, derived from Crank--Nicolson integrators. We show these too can can derive from a FET interpretation, similarly offering potential extensions to higher-order-in-time particle pushers. The FET formulation is used also to consider how the stochastic drift terms can be incorporated into the pushers. Stochastic gyrokinetic expansions are also discussed.

        Different options for the numerical implementation of these schemes are considered.

        Due to the efficacy of FET in the development of SP timesteppers for both the fluid and kinetic component, we hope this approach will prove effective in the future for developing SP timesteppers for the full hybrid model. We hope this will give us the opportunity to incorporate previously inaccessible kinetic effects into the highly effective, modern, finite-element MHD models.
    \end{abstract}
    
    
    \newpage
    \tableofcontents
    
    
    \newpage
    \pagenumbering{arabic}
    %\linenumbers\renewcommand\thelinenumber{\color{black!50}\arabic{linenumber}}
            \input{0 - introduction/main.tex}
        \part{Research}
            \input{1 - low-noise PiC models/main.tex}
            \input{2 - kinetic component/main.tex}
            \input{3 - fluid component/main.tex}
            \input{4 - numerical implementation/main.tex}
        \part{Project Overview}
            \input{5 - research plan/main.tex}
            \input{6 - summary/main.tex}
    
    
    %\section{}
    \newpage
    \pagenumbering{gobble}
        \printbibliography


    \newpage
    \pagenumbering{roman}
    \appendix
        \part{Appendices}
            \input{8 - Hilbert complexes/main.tex}
            \input{9 - weak conservation proofs/main.tex}
\end{document}


\title{\BA{Title in Progress...}}
\author{Boris Andrews}
\affil{Mathematical Institute, University of Oxford}
\date{\today}


\begin{document}
    \pagenumbering{gobble}
    \maketitle
    
    
    \begin{abstract}
        Magnetic confinement reactors---in particular tokamaks---offer one of the most promising options for achieving practical nuclear fusion, with the potential to provide virtually limitless, clean energy. The theoretical and numerical modeling of tokamak plasmas is simultaneously an essential component of effective reactor design, and a great research barrier. Tokamak operational conditions exhibit comparatively low Knudsen numbers. Kinetic effects, including kinetic waves and instabilities, Landau damping, bump-on-tail instabilities and more, are therefore highly influential in tokamak plasma dynamics. Purely fluid models are inherently incapable of capturing these effects, whereas the high dimensionality in purely kinetic models render them practically intractable for most relevant purposes.

        We consider a $\delta\!f$ decomposition model, with a macroscopic fluid background and microscopic kinetic correction, both fully coupled to each other. A similar manner of discretization is proposed to that used in the recent \texttt{STRUPHY} code \cite{Holderied_Possanner_Wang_2021, Holderied_2022, Li_et_al_2023} with a finite-element model for the background and a pseudo-particle/PiC model for the correction.

        The fluid background satisfies the full, non-linear, resistive, compressible, Hall MHD equations. \cite{Laakmann_Hu_Farrell_2022} introduces finite-element(-in-space) implicit timesteppers for the incompressible analogue to this system with structure-preserving (SP) properties in the ideal case, alongside parameter-robust preconditioners. We show that these timesteppers can derive from a finite-element-in-time (FET) (and finite-element-in-space) interpretation. The benefits of this reformulation are discussed, including the derivation of timesteppers that are higher order in time, and the quantifiable dissipative SP properties in the non-ideal, resistive case.
        
        We discuss possible options for extending this FET approach to timesteppers for the compressible case.

        The kinetic corrections satisfy linearized Boltzmann equations. Using a Lénard--Bernstein collision operator, these take Fokker--Planck-like forms \cite{Fokker_1914, Planck_1917} wherein pseudo-particles in the numerical model obey the neoclassical transport equations, with particle-independent Brownian drift terms. This offers a rigorous methodology for incorporating collisions into the particle transport model, without coupling the equations of motions for each particle.
        
        Works by Chen, Chacón et al. \cite{Chen_Chacón_Barnes_2011, Chacón_Chen_Barnes_2013, Chen_Chacón_2014, Chen_Chacón_2015} have developed structure-preserving particle pushers for neoclassical transport in the Vlasov equations, derived from Crank--Nicolson integrators. We show these too can can derive from a FET interpretation, similarly offering potential extensions to higher-order-in-time particle pushers. The FET formulation is used also to consider how the stochastic drift terms can be incorporated into the pushers. Stochastic gyrokinetic expansions are also discussed.

        Different options for the numerical implementation of these schemes are considered.

        Due to the efficacy of FET in the development of SP timesteppers for both the fluid and kinetic component, we hope this approach will prove effective in the future for developing SP timesteppers for the full hybrid model. We hope this will give us the opportunity to incorporate previously inaccessible kinetic effects into the highly effective, modern, finite-element MHD models.
    \end{abstract}
    
    
    \newpage
    \tableofcontents
    
    
    \newpage
    \pagenumbering{arabic}
    %\linenumbers\renewcommand\thelinenumber{\color{black!50}\arabic{linenumber}}
            \documentclass[12pt, a4paper]{report}

\input{template/main.tex}

\title{\BA{Title in Progress...}}
\author{Boris Andrews}
\affil{Mathematical Institute, University of Oxford}
\date{\today}


\begin{document}
    \pagenumbering{gobble}
    \maketitle
    
    
    \begin{abstract}
        Magnetic confinement reactors---in particular tokamaks---offer one of the most promising options for achieving practical nuclear fusion, with the potential to provide virtually limitless, clean energy. The theoretical and numerical modeling of tokamak plasmas is simultaneously an essential component of effective reactor design, and a great research barrier. Tokamak operational conditions exhibit comparatively low Knudsen numbers. Kinetic effects, including kinetic waves and instabilities, Landau damping, bump-on-tail instabilities and more, are therefore highly influential in tokamak plasma dynamics. Purely fluid models are inherently incapable of capturing these effects, whereas the high dimensionality in purely kinetic models render them practically intractable for most relevant purposes.

        We consider a $\delta\!f$ decomposition model, with a macroscopic fluid background and microscopic kinetic correction, both fully coupled to each other. A similar manner of discretization is proposed to that used in the recent \texttt{STRUPHY} code \cite{Holderied_Possanner_Wang_2021, Holderied_2022, Li_et_al_2023} with a finite-element model for the background and a pseudo-particle/PiC model for the correction.

        The fluid background satisfies the full, non-linear, resistive, compressible, Hall MHD equations. \cite{Laakmann_Hu_Farrell_2022} introduces finite-element(-in-space) implicit timesteppers for the incompressible analogue to this system with structure-preserving (SP) properties in the ideal case, alongside parameter-robust preconditioners. We show that these timesteppers can derive from a finite-element-in-time (FET) (and finite-element-in-space) interpretation. The benefits of this reformulation are discussed, including the derivation of timesteppers that are higher order in time, and the quantifiable dissipative SP properties in the non-ideal, resistive case.
        
        We discuss possible options for extending this FET approach to timesteppers for the compressible case.

        The kinetic corrections satisfy linearized Boltzmann equations. Using a Lénard--Bernstein collision operator, these take Fokker--Planck-like forms \cite{Fokker_1914, Planck_1917} wherein pseudo-particles in the numerical model obey the neoclassical transport equations, with particle-independent Brownian drift terms. This offers a rigorous methodology for incorporating collisions into the particle transport model, without coupling the equations of motions for each particle.
        
        Works by Chen, Chacón et al. \cite{Chen_Chacón_Barnes_2011, Chacón_Chen_Barnes_2013, Chen_Chacón_2014, Chen_Chacón_2015} have developed structure-preserving particle pushers for neoclassical transport in the Vlasov equations, derived from Crank--Nicolson integrators. We show these too can can derive from a FET interpretation, similarly offering potential extensions to higher-order-in-time particle pushers. The FET formulation is used also to consider how the stochastic drift terms can be incorporated into the pushers. Stochastic gyrokinetic expansions are also discussed.

        Different options for the numerical implementation of these schemes are considered.

        Due to the efficacy of FET in the development of SP timesteppers for both the fluid and kinetic component, we hope this approach will prove effective in the future for developing SP timesteppers for the full hybrid model. We hope this will give us the opportunity to incorporate previously inaccessible kinetic effects into the highly effective, modern, finite-element MHD models.
    \end{abstract}
    
    
    \newpage
    \tableofcontents
    
    
    \newpage
    \pagenumbering{arabic}
    %\linenumbers\renewcommand\thelinenumber{\color{black!50}\arabic{linenumber}}
            \input{0 - introduction/main.tex}
        \part{Research}
            \input{1 - low-noise PiC models/main.tex}
            \input{2 - kinetic component/main.tex}
            \input{3 - fluid component/main.tex}
            \input{4 - numerical implementation/main.tex}
        \part{Project Overview}
            \input{5 - research plan/main.tex}
            \input{6 - summary/main.tex}
    
    
    %\section{}
    \newpage
    \pagenumbering{gobble}
        \printbibliography


    \newpage
    \pagenumbering{roman}
    \appendix
        \part{Appendices}
            \input{8 - Hilbert complexes/main.tex}
            \input{9 - weak conservation proofs/main.tex}
\end{document}

        \part{Research}
            \documentclass[12pt, a4paper]{report}

\input{template/main.tex}

\title{\BA{Title in Progress...}}
\author{Boris Andrews}
\affil{Mathematical Institute, University of Oxford}
\date{\today}


\begin{document}
    \pagenumbering{gobble}
    \maketitle
    
    
    \begin{abstract}
        Magnetic confinement reactors---in particular tokamaks---offer one of the most promising options for achieving practical nuclear fusion, with the potential to provide virtually limitless, clean energy. The theoretical and numerical modeling of tokamak plasmas is simultaneously an essential component of effective reactor design, and a great research barrier. Tokamak operational conditions exhibit comparatively low Knudsen numbers. Kinetic effects, including kinetic waves and instabilities, Landau damping, bump-on-tail instabilities and more, are therefore highly influential in tokamak plasma dynamics. Purely fluid models are inherently incapable of capturing these effects, whereas the high dimensionality in purely kinetic models render them practically intractable for most relevant purposes.

        We consider a $\delta\!f$ decomposition model, with a macroscopic fluid background and microscopic kinetic correction, both fully coupled to each other. A similar manner of discretization is proposed to that used in the recent \texttt{STRUPHY} code \cite{Holderied_Possanner_Wang_2021, Holderied_2022, Li_et_al_2023} with a finite-element model for the background and a pseudo-particle/PiC model for the correction.

        The fluid background satisfies the full, non-linear, resistive, compressible, Hall MHD equations. \cite{Laakmann_Hu_Farrell_2022} introduces finite-element(-in-space) implicit timesteppers for the incompressible analogue to this system with structure-preserving (SP) properties in the ideal case, alongside parameter-robust preconditioners. We show that these timesteppers can derive from a finite-element-in-time (FET) (and finite-element-in-space) interpretation. The benefits of this reformulation are discussed, including the derivation of timesteppers that are higher order in time, and the quantifiable dissipative SP properties in the non-ideal, resistive case.
        
        We discuss possible options for extending this FET approach to timesteppers for the compressible case.

        The kinetic corrections satisfy linearized Boltzmann equations. Using a Lénard--Bernstein collision operator, these take Fokker--Planck-like forms \cite{Fokker_1914, Planck_1917} wherein pseudo-particles in the numerical model obey the neoclassical transport equations, with particle-independent Brownian drift terms. This offers a rigorous methodology for incorporating collisions into the particle transport model, without coupling the equations of motions for each particle.
        
        Works by Chen, Chacón et al. \cite{Chen_Chacón_Barnes_2011, Chacón_Chen_Barnes_2013, Chen_Chacón_2014, Chen_Chacón_2015} have developed structure-preserving particle pushers for neoclassical transport in the Vlasov equations, derived from Crank--Nicolson integrators. We show these too can can derive from a FET interpretation, similarly offering potential extensions to higher-order-in-time particle pushers. The FET formulation is used also to consider how the stochastic drift terms can be incorporated into the pushers. Stochastic gyrokinetic expansions are also discussed.

        Different options for the numerical implementation of these schemes are considered.

        Due to the efficacy of FET in the development of SP timesteppers for both the fluid and kinetic component, we hope this approach will prove effective in the future for developing SP timesteppers for the full hybrid model. We hope this will give us the opportunity to incorporate previously inaccessible kinetic effects into the highly effective, modern, finite-element MHD models.
    \end{abstract}
    
    
    \newpage
    \tableofcontents
    
    
    \newpage
    \pagenumbering{arabic}
    %\linenumbers\renewcommand\thelinenumber{\color{black!50}\arabic{linenumber}}
            \input{0 - introduction/main.tex}
        \part{Research}
            \input{1 - low-noise PiC models/main.tex}
            \input{2 - kinetic component/main.tex}
            \input{3 - fluid component/main.tex}
            \input{4 - numerical implementation/main.tex}
        \part{Project Overview}
            \input{5 - research plan/main.tex}
            \input{6 - summary/main.tex}
    
    
    %\section{}
    \newpage
    \pagenumbering{gobble}
        \printbibliography


    \newpage
    \pagenumbering{roman}
    \appendix
        \part{Appendices}
            \input{8 - Hilbert complexes/main.tex}
            \input{9 - weak conservation proofs/main.tex}
\end{document}

            \documentclass[12pt, a4paper]{report}

\input{template/main.tex}

\title{\BA{Title in Progress...}}
\author{Boris Andrews}
\affil{Mathematical Institute, University of Oxford}
\date{\today}


\begin{document}
    \pagenumbering{gobble}
    \maketitle
    
    
    \begin{abstract}
        Magnetic confinement reactors---in particular tokamaks---offer one of the most promising options for achieving practical nuclear fusion, with the potential to provide virtually limitless, clean energy. The theoretical and numerical modeling of tokamak plasmas is simultaneously an essential component of effective reactor design, and a great research barrier. Tokamak operational conditions exhibit comparatively low Knudsen numbers. Kinetic effects, including kinetic waves and instabilities, Landau damping, bump-on-tail instabilities and more, are therefore highly influential in tokamak plasma dynamics. Purely fluid models are inherently incapable of capturing these effects, whereas the high dimensionality in purely kinetic models render them practically intractable for most relevant purposes.

        We consider a $\delta\!f$ decomposition model, with a macroscopic fluid background and microscopic kinetic correction, both fully coupled to each other. A similar manner of discretization is proposed to that used in the recent \texttt{STRUPHY} code \cite{Holderied_Possanner_Wang_2021, Holderied_2022, Li_et_al_2023} with a finite-element model for the background and a pseudo-particle/PiC model for the correction.

        The fluid background satisfies the full, non-linear, resistive, compressible, Hall MHD equations. \cite{Laakmann_Hu_Farrell_2022} introduces finite-element(-in-space) implicit timesteppers for the incompressible analogue to this system with structure-preserving (SP) properties in the ideal case, alongside parameter-robust preconditioners. We show that these timesteppers can derive from a finite-element-in-time (FET) (and finite-element-in-space) interpretation. The benefits of this reformulation are discussed, including the derivation of timesteppers that are higher order in time, and the quantifiable dissipative SP properties in the non-ideal, resistive case.
        
        We discuss possible options for extending this FET approach to timesteppers for the compressible case.

        The kinetic corrections satisfy linearized Boltzmann equations. Using a Lénard--Bernstein collision operator, these take Fokker--Planck-like forms \cite{Fokker_1914, Planck_1917} wherein pseudo-particles in the numerical model obey the neoclassical transport equations, with particle-independent Brownian drift terms. This offers a rigorous methodology for incorporating collisions into the particle transport model, without coupling the equations of motions for each particle.
        
        Works by Chen, Chacón et al. \cite{Chen_Chacón_Barnes_2011, Chacón_Chen_Barnes_2013, Chen_Chacón_2014, Chen_Chacón_2015} have developed structure-preserving particle pushers for neoclassical transport in the Vlasov equations, derived from Crank--Nicolson integrators. We show these too can can derive from a FET interpretation, similarly offering potential extensions to higher-order-in-time particle pushers. The FET formulation is used also to consider how the stochastic drift terms can be incorporated into the pushers. Stochastic gyrokinetic expansions are also discussed.

        Different options for the numerical implementation of these schemes are considered.

        Due to the efficacy of FET in the development of SP timesteppers for both the fluid and kinetic component, we hope this approach will prove effective in the future for developing SP timesteppers for the full hybrid model. We hope this will give us the opportunity to incorporate previously inaccessible kinetic effects into the highly effective, modern, finite-element MHD models.
    \end{abstract}
    
    
    \newpage
    \tableofcontents
    
    
    \newpage
    \pagenumbering{arabic}
    %\linenumbers\renewcommand\thelinenumber{\color{black!50}\arabic{linenumber}}
            \input{0 - introduction/main.tex}
        \part{Research}
            \input{1 - low-noise PiC models/main.tex}
            \input{2 - kinetic component/main.tex}
            \input{3 - fluid component/main.tex}
            \input{4 - numerical implementation/main.tex}
        \part{Project Overview}
            \input{5 - research plan/main.tex}
            \input{6 - summary/main.tex}
    
    
    %\section{}
    \newpage
    \pagenumbering{gobble}
        \printbibliography


    \newpage
    \pagenumbering{roman}
    \appendix
        \part{Appendices}
            \input{8 - Hilbert complexes/main.tex}
            \input{9 - weak conservation proofs/main.tex}
\end{document}

            \documentclass[12pt, a4paper]{report}

\input{template/main.tex}

\title{\BA{Title in Progress...}}
\author{Boris Andrews}
\affil{Mathematical Institute, University of Oxford}
\date{\today}


\begin{document}
    \pagenumbering{gobble}
    \maketitle
    
    
    \begin{abstract}
        Magnetic confinement reactors---in particular tokamaks---offer one of the most promising options for achieving practical nuclear fusion, with the potential to provide virtually limitless, clean energy. The theoretical and numerical modeling of tokamak plasmas is simultaneously an essential component of effective reactor design, and a great research barrier. Tokamak operational conditions exhibit comparatively low Knudsen numbers. Kinetic effects, including kinetic waves and instabilities, Landau damping, bump-on-tail instabilities and more, are therefore highly influential in tokamak plasma dynamics. Purely fluid models are inherently incapable of capturing these effects, whereas the high dimensionality in purely kinetic models render them practically intractable for most relevant purposes.

        We consider a $\delta\!f$ decomposition model, with a macroscopic fluid background and microscopic kinetic correction, both fully coupled to each other. A similar manner of discretization is proposed to that used in the recent \texttt{STRUPHY} code \cite{Holderied_Possanner_Wang_2021, Holderied_2022, Li_et_al_2023} with a finite-element model for the background and a pseudo-particle/PiC model for the correction.

        The fluid background satisfies the full, non-linear, resistive, compressible, Hall MHD equations. \cite{Laakmann_Hu_Farrell_2022} introduces finite-element(-in-space) implicit timesteppers for the incompressible analogue to this system with structure-preserving (SP) properties in the ideal case, alongside parameter-robust preconditioners. We show that these timesteppers can derive from a finite-element-in-time (FET) (and finite-element-in-space) interpretation. The benefits of this reformulation are discussed, including the derivation of timesteppers that are higher order in time, and the quantifiable dissipative SP properties in the non-ideal, resistive case.
        
        We discuss possible options for extending this FET approach to timesteppers for the compressible case.

        The kinetic corrections satisfy linearized Boltzmann equations. Using a Lénard--Bernstein collision operator, these take Fokker--Planck-like forms \cite{Fokker_1914, Planck_1917} wherein pseudo-particles in the numerical model obey the neoclassical transport equations, with particle-independent Brownian drift terms. This offers a rigorous methodology for incorporating collisions into the particle transport model, without coupling the equations of motions for each particle.
        
        Works by Chen, Chacón et al. \cite{Chen_Chacón_Barnes_2011, Chacón_Chen_Barnes_2013, Chen_Chacón_2014, Chen_Chacón_2015} have developed structure-preserving particle pushers for neoclassical transport in the Vlasov equations, derived from Crank--Nicolson integrators. We show these too can can derive from a FET interpretation, similarly offering potential extensions to higher-order-in-time particle pushers. The FET formulation is used also to consider how the stochastic drift terms can be incorporated into the pushers. Stochastic gyrokinetic expansions are also discussed.

        Different options for the numerical implementation of these schemes are considered.

        Due to the efficacy of FET in the development of SP timesteppers for both the fluid and kinetic component, we hope this approach will prove effective in the future for developing SP timesteppers for the full hybrid model. We hope this will give us the opportunity to incorporate previously inaccessible kinetic effects into the highly effective, modern, finite-element MHD models.
    \end{abstract}
    
    
    \newpage
    \tableofcontents
    
    
    \newpage
    \pagenumbering{arabic}
    %\linenumbers\renewcommand\thelinenumber{\color{black!50}\arabic{linenumber}}
            \input{0 - introduction/main.tex}
        \part{Research}
            \input{1 - low-noise PiC models/main.tex}
            \input{2 - kinetic component/main.tex}
            \input{3 - fluid component/main.tex}
            \input{4 - numerical implementation/main.tex}
        \part{Project Overview}
            \input{5 - research plan/main.tex}
            \input{6 - summary/main.tex}
    
    
    %\section{}
    \newpage
    \pagenumbering{gobble}
        \printbibliography


    \newpage
    \pagenumbering{roman}
    \appendix
        \part{Appendices}
            \input{8 - Hilbert complexes/main.tex}
            \input{9 - weak conservation proofs/main.tex}
\end{document}

            \documentclass[12pt, a4paper]{report}

\input{template/main.tex}

\title{\BA{Title in Progress...}}
\author{Boris Andrews}
\affil{Mathematical Institute, University of Oxford}
\date{\today}


\begin{document}
    \pagenumbering{gobble}
    \maketitle
    
    
    \begin{abstract}
        Magnetic confinement reactors---in particular tokamaks---offer one of the most promising options for achieving practical nuclear fusion, with the potential to provide virtually limitless, clean energy. The theoretical and numerical modeling of tokamak plasmas is simultaneously an essential component of effective reactor design, and a great research barrier. Tokamak operational conditions exhibit comparatively low Knudsen numbers. Kinetic effects, including kinetic waves and instabilities, Landau damping, bump-on-tail instabilities and more, are therefore highly influential in tokamak plasma dynamics. Purely fluid models are inherently incapable of capturing these effects, whereas the high dimensionality in purely kinetic models render them practically intractable for most relevant purposes.

        We consider a $\delta\!f$ decomposition model, with a macroscopic fluid background and microscopic kinetic correction, both fully coupled to each other. A similar manner of discretization is proposed to that used in the recent \texttt{STRUPHY} code \cite{Holderied_Possanner_Wang_2021, Holderied_2022, Li_et_al_2023} with a finite-element model for the background and a pseudo-particle/PiC model for the correction.

        The fluid background satisfies the full, non-linear, resistive, compressible, Hall MHD equations. \cite{Laakmann_Hu_Farrell_2022} introduces finite-element(-in-space) implicit timesteppers for the incompressible analogue to this system with structure-preserving (SP) properties in the ideal case, alongside parameter-robust preconditioners. We show that these timesteppers can derive from a finite-element-in-time (FET) (and finite-element-in-space) interpretation. The benefits of this reformulation are discussed, including the derivation of timesteppers that are higher order in time, and the quantifiable dissipative SP properties in the non-ideal, resistive case.
        
        We discuss possible options for extending this FET approach to timesteppers for the compressible case.

        The kinetic corrections satisfy linearized Boltzmann equations. Using a Lénard--Bernstein collision operator, these take Fokker--Planck-like forms \cite{Fokker_1914, Planck_1917} wherein pseudo-particles in the numerical model obey the neoclassical transport equations, with particle-independent Brownian drift terms. This offers a rigorous methodology for incorporating collisions into the particle transport model, without coupling the equations of motions for each particle.
        
        Works by Chen, Chacón et al. \cite{Chen_Chacón_Barnes_2011, Chacón_Chen_Barnes_2013, Chen_Chacón_2014, Chen_Chacón_2015} have developed structure-preserving particle pushers for neoclassical transport in the Vlasov equations, derived from Crank--Nicolson integrators. We show these too can can derive from a FET interpretation, similarly offering potential extensions to higher-order-in-time particle pushers. The FET formulation is used also to consider how the stochastic drift terms can be incorporated into the pushers. Stochastic gyrokinetic expansions are also discussed.

        Different options for the numerical implementation of these schemes are considered.

        Due to the efficacy of FET in the development of SP timesteppers for both the fluid and kinetic component, we hope this approach will prove effective in the future for developing SP timesteppers for the full hybrid model. We hope this will give us the opportunity to incorporate previously inaccessible kinetic effects into the highly effective, modern, finite-element MHD models.
    \end{abstract}
    
    
    \newpage
    \tableofcontents
    
    
    \newpage
    \pagenumbering{arabic}
    %\linenumbers\renewcommand\thelinenumber{\color{black!50}\arabic{linenumber}}
            \input{0 - introduction/main.tex}
        \part{Research}
            \input{1 - low-noise PiC models/main.tex}
            \input{2 - kinetic component/main.tex}
            \input{3 - fluid component/main.tex}
            \input{4 - numerical implementation/main.tex}
        \part{Project Overview}
            \input{5 - research plan/main.tex}
            \input{6 - summary/main.tex}
    
    
    %\section{}
    \newpage
    \pagenumbering{gobble}
        \printbibliography


    \newpage
    \pagenumbering{roman}
    \appendix
        \part{Appendices}
            \input{8 - Hilbert complexes/main.tex}
            \input{9 - weak conservation proofs/main.tex}
\end{document}

        \part{Project Overview}
            \documentclass[12pt, a4paper]{report}

\input{template/main.tex}

\title{\BA{Title in Progress...}}
\author{Boris Andrews}
\affil{Mathematical Institute, University of Oxford}
\date{\today}


\begin{document}
    \pagenumbering{gobble}
    \maketitle
    
    
    \begin{abstract}
        Magnetic confinement reactors---in particular tokamaks---offer one of the most promising options for achieving practical nuclear fusion, with the potential to provide virtually limitless, clean energy. The theoretical and numerical modeling of tokamak plasmas is simultaneously an essential component of effective reactor design, and a great research barrier. Tokamak operational conditions exhibit comparatively low Knudsen numbers. Kinetic effects, including kinetic waves and instabilities, Landau damping, bump-on-tail instabilities and more, are therefore highly influential in tokamak plasma dynamics. Purely fluid models are inherently incapable of capturing these effects, whereas the high dimensionality in purely kinetic models render them practically intractable for most relevant purposes.

        We consider a $\delta\!f$ decomposition model, with a macroscopic fluid background and microscopic kinetic correction, both fully coupled to each other. A similar manner of discretization is proposed to that used in the recent \texttt{STRUPHY} code \cite{Holderied_Possanner_Wang_2021, Holderied_2022, Li_et_al_2023} with a finite-element model for the background and a pseudo-particle/PiC model for the correction.

        The fluid background satisfies the full, non-linear, resistive, compressible, Hall MHD equations. \cite{Laakmann_Hu_Farrell_2022} introduces finite-element(-in-space) implicit timesteppers for the incompressible analogue to this system with structure-preserving (SP) properties in the ideal case, alongside parameter-robust preconditioners. We show that these timesteppers can derive from a finite-element-in-time (FET) (and finite-element-in-space) interpretation. The benefits of this reformulation are discussed, including the derivation of timesteppers that are higher order in time, and the quantifiable dissipative SP properties in the non-ideal, resistive case.
        
        We discuss possible options for extending this FET approach to timesteppers for the compressible case.

        The kinetic corrections satisfy linearized Boltzmann equations. Using a Lénard--Bernstein collision operator, these take Fokker--Planck-like forms \cite{Fokker_1914, Planck_1917} wherein pseudo-particles in the numerical model obey the neoclassical transport equations, with particle-independent Brownian drift terms. This offers a rigorous methodology for incorporating collisions into the particle transport model, without coupling the equations of motions for each particle.
        
        Works by Chen, Chacón et al. \cite{Chen_Chacón_Barnes_2011, Chacón_Chen_Barnes_2013, Chen_Chacón_2014, Chen_Chacón_2015} have developed structure-preserving particle pushers for neoclassical transport in the Vlasov equations, derived from Crank--Nicolson integrators. We show these too can can derive from a FET interpretation, similarly offering potential extensions to higher-order-in-time particle pushers. The FET formulation is used also to consider how the stochastic drift terms can be incorporated into the pushers. Stochastic gyrokinetic expansions are also discussed.

        Different options for the numerical implementation of these schemes are considered.

        Due to the efficacy of FET in the development of SP timesteppers for both the fluid and kinetic component, we hope this approach will prove effective in the future for developing SP timesteppers for the full hybrid model. We hope this will give us the opportunity to incorporate previously inaccessible kinetic effects into the highly effective, modern, finite-element MHD models.
    \end{abstract}
    
    
    \newpage
    \tableofcontents
    
    
    \newpage
    \pagenumbering{arabic}
    %\linenumbers\renewcommand\thelinenumber{\color{black!50}\arabic{linenumber}}
            \input{0 - introduction/main.tex}
        \part{Research}
            \input{1 - low-noise PiC models/main.tex}
            \input{2 - kinetic component/main.tex}
            \input{3 - fluid component/main.tex}
            \input{4 - numerical implementation/main.tex}
        \part{Project Overview}
            \input{5 - research plan/main.tex}
            \input{6 - summary/main.tex}
    
    
    %\section{}
    \newpage
    \pagenumbering{gobble}
        \printbibliography


    \newpage
    \pagenumbering{roman}
    \appendix
        \part{Appendices}
            \input{8 - Hilbert complexes/main.tex}
            \input{9 - weak conservation proofs/main.tex}
\end{document}

            \documentclass[12pt, a4paper]{report}

\input{template/main.tex}

\title{\BA{Title in Progress...}}
\author{Boris Andrews}
\affil{Mathematical Institute, University of Oxford}
\date{\today}


\begin{document}
    \pagenumbering{gobble}
    \maketitle
    
    
    \begin{abstract}
        Magnetic confinement reactors---in particular tokamaks---offer one of the most promising options for achieving practical nuclear fusion, with the potential to provide virtually limitless, clean energy. The theoretical and numerical modeling of tokamak plasmas is simultaneously an essential component of effective reactor design, and a great research barrier. Tokamak operational conditions exhibit comparatively low Knudsen numbers. Kinetic effects, including kinetic waves and instabilities, Landau damping, bump-on-tail instabilities and more, are therefore highly influential in tokamak plasma dynamics. Purely fluid models are inherently incapable of capturing these effects, whereas the high dimensionality in purely kinetic models render them practically intractable for most relevant purposes.

        We consider a $\delta\!f$ decomposition model, with a macroscopic fluid background and microscopic kinetic correction, both fully coupled to each other. A similar manner of discretization is proposed to that used in the recent \texttt{STRUPHY} code \cite{Holderied_Possanner_Wang_2021, Holderied_2022, Li_et_al_2023} with a finite-element model for the background and a pseudo-particle/PiC model for the correction.

        The fluid background satisfies the full, non-linear, resistive, compressible, Hall MHD equations. \cite{Laakmann_Hu_Farrell_2022} introduces finite-element(-in-space) implicit timesteppers for the incompressible analogue to this system with structure-preserving (SP) properties in the ideal case, alongside parameter-robust preconditioners. We show that these timesteppers can derive from a finite-element-in-time (FET) (and finite-element-in-space) interpretation. The benefits of this reformulation are discussed, including the derivation of timesteppers that are higher order in time, and the quantifiable dissipative SP properties in the non-ideal, resistive case.
        
        We discuss possible options for extending this FET approach to timesteppers for the compressible case.

        The kinetic corrections satisfy linearized Boltzmann equations. Using a Lénard--Bernstein collision operator, these take Fokker--Planck-like forms \cite{Fokker_1914, Planck_1917} wherein pseudo-particles in the numerical model obey the neoclassical transport equations, with particle-independent Brownian drift terms. This offers a rigorous methodology for incorporating collisions into the particle transport model, without coupling the equations of motions for each particle.
        
        Works by Chen, Chacón et al. \cite{Chen_Chacón_Barnes_2011, Chacón_Chen_Barnes_2013, Chen_Chacón_2014, Chen_Chacón_2015} have developed structure-preserving particle pushers for neoclassical transport in the Vlasov equations, derived from Crank--Nicolson integrators. We show these too can can derive from a FET interpretation, similarly offering potential extensions to higher-order-in-time particle pushers. The FET formulation is used also to consider how the stochastic drift terms can be incorporated into the pushers. Stochastic gyrokinetic expansions are also discussed.

        Different options for the numerical implementation of these schemes are considered.

        Due to the efficacy of FET in the development of SP timesteppers for both the fluid and kinetic component, we hope this approach will prove effective in the future for developing SP timesteppers for the full hybrid model. We hope this will give us the opportunity to incorporate previously inaccessible kinetic effects into the highly effective, modern, finite-element MHD models.
    \end{abstract}
    
    
    \newpage
    \tableofcontents
    
    
    \newpage
    \pagenumbering{arabic}
    %\linenumbers\renewcommand\thelinenumber{\color{black!50}\arabic{linenumber}}
            \input{0 - introduction/main.tex}
        \part{Research}
            \input{1 - low-noise PiC models/main.tex}
            \input{2 - kinetic component/main.tex}
            \input{3 - fluid component/main.tex}
            \input{4 - numerical implementation/main.tex}
        \part{Project Overview}
            \input{5 - research plan/main.tex}
            \input{6 - summary/main.tex}
    
    
    %\section{}
    \newpage
    \pagenumbering{gobble}
        \printbibliography


    \newpage
    \pagenumbering{roman}
    \appendix
        \part{Appendices}
            \input{8 - Hilbert complexes/main.tex}
            \input{9 - weak conservation proofs/main.tex}
\end{document}

    
    
    %\section{}
    \newpage
    \pagenumbering{gobble}
        \printbibliography


    \newpage
    \pagenumbering{roman}
    \appendix
        \part{Appendices}
            \documentclass[12pt, a4paper]{report}

\input{template/main.tex}

\title{\BA{Title in Progress...}}
\author{Boris Andrews}
\affil{Mathematical Institute, University of Oxford}
\date{\today}


\begin{document}
    \pagenumbering{gobble}
    \maketitle
    
    
    \begin{abstract}
        Magnetic confinement reactors---in particular tokamaks---offer one of the most promising options for achieving practical nuclear fusion, with the potential to provide virtually limitless, clean energy. The theoretical and numerical modeling of tokamak plasmas is simultaneously an essential component of effective reactor design, and a great research barrier. Tokamak operational conditions exhibit comparatively low Knudsen numbers. Kinetic effects, including kinetic waves and instabilities, Landau damping, bump-on-tail instabilities and more, are therefore highly influential in tokamak plasma dynamics. Purely fluid models are inherently incapable of capturing these effects, whereas the high dimensionality in purely kinetic models render them practically intractable for most relevant purposes.

        We consider a $\delta\!f$ decomposition model, with a macroscopic fluid background and microscopic kinetic correction, both fully coupled to each other. A similar manner of discretization is proposed to that used in the recent \texttt{STRUPHY} code \cite{Holderied_Possanner_Wang_2021, Holderied_2022, Li_et_al_2023} with a finite-element model for the background and a pseudo-particle/PiC model for the correction.

        The fluid background satisfies the full, non-linear, resistive, compressible, Hall MHD equations. \cite{Laakmann_Hu_Farrell_2022} introduces finite-element(-in-space) implicit timesteppers for the incompressible analogue to this system with structure-preserving (SP) properties in the ideal case, alongside parameter-robust preconditioners. We show that these timesteppers can derive from a finite-element-in-time (FET) (and finite-element-in-space) interpretation. The benefits of this reformulation are discussed, including the derivation of timesteppers that are higher order in time, and the quantifiable dissipative SP properties in the non-ideal, resistive case.
        
        We discuss possible options for extending this FET approach to timesteppers for the compressible case.

        The kinetic corrections satisfy linearized Boltzmann equations. Using a Lénard--Bernstein collision operator, these take Fokker--Planck-like forms \cite{Fokker_1914, Planck_1917} wherein pseudo-particles in the numerical model obey the neoclassical transport equations, with particle-independent Brownian drift terms. This offers a rigorous methodology for incorporating collisions into the particle transport model, without coupling the equations of motions for each particle.
        
        Works by Chen, Chacón et al. \cite{Chen_Chacón_Barnes_2011, Chacón_Chen_Barnes_2013, Chen_Chacón_2014, Chen_Chacón_2015} have developed structure-preserving particle pushers for neoclassical transport in the Vlasov equations, derived from Crank--Nicolson integrators. We show these too can can derive from a FET interpretation, similarly offering potential extensions to higher-order-in-time particle pushers. The FET formulation is used also to consider how the stochastic drift terms can be incorporated into the pushers. Stochastic gyrokinetic expansions are also discussed.

        Different options for the numerical implementation of these schemes are considered.

        Due to the efficacy of FET in the development of SP timesteppers for both the fluid and kinetic component, we hope this approach will prove effective in the future for developing SP timesteppers for the full hybrid model. We hope this will give us the opportunity to incorporate previously inaccessible kinetic effects into the highly effective, modern, finite-element MHD models.
    \end{abstract}
    
    
    \newpage
    \tableofcontents
    
    
    \newpage
    \pagenumbering{arabic}
    %\linenumbers\renewcommand\thelinenumber{\color{black!50}\arabic{linenumber}}
            \input{0 - introduction/main.tex}
        \part{Research}
            \input{1 - low-noise PiC models/main.tex}
            \input{2 - kinetic component/main.tex}
            \input{3 - fluid component/main.tex}
            \input{4 - numerical implementation/main.tex}
        \part{Project Overview}
            \input{5 - research plan/main.tex}
            \input{6 - summary/main.tex}
    
    
    %\section{}
    \newpage
    \pagenumbering{gobble}
        \printbibliography


    \newpage
    \pagenumbering{roman}
    \appendix
        \part{Appendices}
            \input{8 - Hilbert complexes/main.tex}
            \input{9 - weak conservation proofs/main.tex}
\end{document}

            \documentclass[12pt, a4paper]{report}

\input{template/main.tex}

\title{\BA{Title in Progress...}}
\author{Boris Andrews}
\affil{Mathematical Institute, University of Oxford}
\date{\today}


\begin{document}
    \pagenumbering{gobble}
    \maketitle
    
    
    \begin{abstract}
        Magnetic confinement reactors---in particular tokamaks---offer one of the most promising options for achieving practical nuclear fusion, with the potential to provide virtually limitless, clean energy. The theoretical and numerical modeling of tokamak plasmas is simultaneously an essential component of effective reactor design, and a great research barrier. Tokamak operational conditions exhibit comparatively low Knudsen numbers. Kinetic effects, including kinetic waves and instabilities, Landau damping, bump-on-tail instabilities and more, are therefore highly influential in tokamak plasma dynamics. Purely fluid models are inherently incapable of capturing these effects, whereas the high dimensionality in purely kinetic models render them practically intractable for most relevant purposes.

        We consider a $\delta\!f$ decomposition model, with a macroscopic fluid background and microscopic kinetic correction, both fully coupled to each other. A similar manner of discretization is proposed to that used in the recent \texttt{STRUPHY} code \cite{Holderied_Possanner_Wang_2021, Holderied_2022, Li_et_al_2023} with a finite-element model for the background and a pseudo-particle/PiC model for the correction.

        The fluid background satisfies the full, non-linear, resistive, compressible, Hall MHD equations. \cite{Laakmann_Hu_Farrell_2022} introduces finite-element(-in-space) implicit timesteppers for the incompressible analogue to this system with structure-preserving (SP) properties in the ideal case, alongside parameter-robust preconditioners. We show that these timesteppers can derive from a finite-element-in-time (FET) (and finite-element-in-space) interpretation. The benefits of this reformulation are discussed, including the derivation of timesteppers that are higher order in time, and the quantifiable dissipative SP properties in the non-ideal, resistive case.
        
        We discuss possible options for extending this FET approach to timesteppers for the compressible case.

        The kinetic corrections satisfy linearized Boltzmann equations. Using a Lénard--Bernstein collision operator, these take Fokker--Planck-like forms \cite{Fokker_1914, Planck_1917} wherein pseudo-particles in the numerical model obey the neoclassical transport equations, with particle-independent Brownian drift terms. This offers a rigorous methodology for incorporating collisions into the particle transport model, without coupling the equations of motions for each particle.
        
        Works by Chen, Chacón et al. \cite{Chen_Chacón_Barnes_2011, Chacón_Chen_Barnes_2013, Chen_Chacón_2014, Chen_Chacón_2015} have developed structure-preserving particle pushers for neoclassical transport in the Vlasov equations, derived from Crank--Nicolson integrators. We show these too can can derive from a FET interpretation, similarly offering potential extensions to higher-order-in-time particle pushers. The FET formulation is used also to consider how the stochastic drift terms can be incorporated into the pushers. Stochastic gyrokinetic expansions are also discussed.

        Different options for the numerical implementation of these schemes are considered.

        Due to the efficacy of FET in the development of SP timesteppers for both the fluid and kinetic component, we hope this approach will prove effective in the future for developing SP timesteppers for the full hybrid model. We hope this will give us the opportunity to incorporate previously inaccessible kinetic effects into the highly effective, modern, finite-element MHD models.
    \end{abstract}
    
    
    \newpage
    \tableofcontents
    
    
    \newpage
    \pagenumbering{arabic}
    %\linenumbers\renewcommand\thelinenumber{\color{black!50}\arabic{linenumber}}
            \input{0 - introduction/main.tex}
        \part{Research}
            \input{1 - low-noise PiC models/main.tex}
            \input{2 - kinetic component/main.tex}
            \input{3 - fluid component/main.tex}
            \input{4 - numerical implementation/main.tex}
        \part{Project Overview}
            \input{5 - research plan/main.tex}
            \input{6 - summary/main.tex}
    
    
    %\section{}
    \newpage
    \pagenumbering{gobble}
        \printbibliography


    \newpage
    \pagenumbering{roman}
    \appendix
        \part{Appendices}
            \input{8 - Hilbert complexes/main.tex}
            \input{9 - weak conservation proofs/main.tex}
\end{document}

\end{document}

            \documentclass[12pt, a4paper]{report}

\documentclass[12pt, a4paper]{report}

\input{template/main.tex}

\title{\BA{Title in Progress...}}
\author{Boris Andrews}
\affil{Mathematical Institute, University of Oxford}
\date{\today}


\begin{document}
    \pagenumbering{gobble}
    \maketitle
    
    
    \begin{abstract}
        Magnetic confinement reactors---in particular tokamaks---offer one of the most promising options for achieving practical nuclear fusion, with the potential to provide virtually limitless, clean energy. The theoretical and numerical modeling of tokamak plasmas is simultaneously an essential component of effective reactor design, and a great research barrier. Tokamak operational conditions exhibit comparatively low Knudsen numbers. Kinetic effects, including kinetic waves and instabilities, Landau damping, bump-on-tail instabilities and more, are therefore highly influential in tokamak plasma dynamics. Purely fluid models are inherently incapable of capturing these effects, whereas the high dimensionality in purely kinetic models render them practically intractable for most relevant purposes.

        We consider a $\delta\!f$ decomposition model, with a macroscopic fluid background and microscopic kinetic correction, both fully coupled to each other. A similar manner of discretization is proposed to that used in the recent \texttt{STRUPHY} code \cite{Holderied_Possanner_Wang_2021, Holderied_2022, Li_et_al_2023} with a finite-element model for the background and a pseudo-particle/PiC model for the correction.

        The fluid background satisfies the full, non-linear, resistive, compressible, Hall MHD equations. \cite{Laakmann_Hu_Farrell_2022} introduces finite-element(-in-space) implicit timesteppers for the incompressible analogue to this system with structure-preserving (SP) properties in the ideal case, alongside parameter-robust preconditioners. We show that these timesteppers can derive from a finite-element-in-time (FET) (and finite-element-in-space) interpretation. The benefits of this reformulation are discussed, including the derivation of timesteppers that are higher order in time, and the quantifiable dissipative SP properties in the non-ideal, resistive case.
        
        We discuss possible options for extending this FET approach to timesteppers for the compressible case.

        The kinetic corrections satisfy linearized Boltzmann equations. Using a Lénard--Bernstein collision operator, these take Fokker--Planck-like forms \cite{Fokker_1914, Planck_1917} wherein pseudo-particles in the numerical model obey the neoclassical transport equations, with particle-independent Brownian drift terms. This offers a rigorous methodology for incorporating collisions into the particle transport model, without coupling the equations of motions for each particle.
        
        Works by Chen, Chacón et al. \cite{Chen_Chacón_Barnes_2011, Chacón_Chen_Barnes_2013, Chen_Chacón_2014, Chen_Chacón_2015} have developed structure-preserving particle pushers for neoclassical transport in the Vlasov equations, derived from Crank--Nicolson integrators. We show these too can can derive from a FET interpretation, similarly offering potential extensions to higher-order-in-time particle pushers. The FET formulation is used also to consider how the stochastic drift terms can be incorporated into the pushers. Stochastic gyrokinetic expansions are also discussed.

        Different options for the numerical implementation of these schemes are considered.

        Due to the efficacy of FET in the development of SP timesteppers for both the fluid and kinetic component, we hope this approach will prove effective in the future for developing SP timesteppers for the full hybrid model. We hope this will give us the opportunity to incorporate previously inaccessible kinetic effects into the highly effective, modern, finite-element MHD models.
    \end{abstract}
    
    
    \newpage
    \tableofcontents
    
    
    \newpage
    \pagenumbering{arabic}
    %\linenumbers\renewcommand\thelinenumber{\color{black!50}\arabic{linenumber}}
            \input{0 - introduction/main.tex}
        \part{Research}
            \input{1 - low-noise PiC models/main.tex}
            \input{2 - kinetic component/main.tex}
            \input{3 - fluid component/main.tex}
            \input{4 - numerical implementation/main.tex}
        \part{Project Overview}
            \input{5 - research plan/main.tex}
            \input{6 - summary/main.tex}
    
    
    %\section{}
    \newpage
    \pagenumbering{gobble}
        \printbibliography


    \newpage
    \pagenumbering{roman}
    \appendix
        \part{Appendices}
            \input{8 - Hilbert complexes/main.tex}
            \input{9 - weak conservation proofs/main.tex}
\end{document}


\title{\BA{Title in Progress...}}
\author{Boris Andrews}
\affil{Mathematical Institute, University of Oxford}
\date{\today}


\begin{document}
    \pagenumbering{gobble}
    \maketitle
    
    
    \begin{abstract}
        Magnetic confinement reactors---in particular tokamaks---offer one of the most promising options for achieving practical nuclear fusion, with the potential to provide virtually limitless, clean energy. The theoretical and numerical modeling of tokamak plasmas is simultaneously an essential component of effective reactor design, and a great research barrier. Tokamak operational conditions exhibit comparatively low Knudsen numbers. Kinetic effects, including kinetic waves and instabilities, Landau damping, bump-on-tail instabilities and more, are therefore highly influential in tokamak plasma dynamics. Purely fluid models are inherently incapable of capturing these effects, whereas the high dimensionality in purely kinetic models render them practically intractable for most relevant purposes.

        We consider a $\delta\!f$ decomposition model, with a macroscopic fluid background and microscopic kinetic correction, both fully coupled to each other. A similar manner of discretization is proposed to that used in the recent \texttt{STRUPHY} code \cite{Holderied_Possanner_Wang_2021, Holderied_2022, Li_et_al_2023} with a finite-element model for the background and a pseudo-particle/PiC model for the correction.

        The fluid background satisfies the full, non-linear, resistive, compressible, Hall MHD equations. \cite{Laakmann_Hu_Farrell_2022} introduces finite-element(-in-space) implicit timesteppers for the incompressible analogue to this system with structure-preserving (SP) properties in the ideal case, alongside parameter-robust preconditioners. We show that these timesteppers can derive from a finite-element-in-time (FET) (and finite-element-in-space) interpretation. The benefits of this reformulation are discussed, including the derivation of timesteppers that are higher order in time, and the quantifiable dissipative SP properties in the non-ideal, resistive case.
        
        We discuss possible options for extending this FET approach to timesteppers for the compressible case.

        The kinetic corrections satisfy linearized Boltzmann equations. Using a Lénard--Bernstein collision operator, these take Fokker--Planck-like forms \cite{Fokker_1914, Planck_1917} wherein pseudo-particles in the numerical model obey the neoclassical transport equations, with particle-independent Brownian drift terms. This offers a rigorous methodology for incorporating collisions into the particle transport model, without coupling the equations of motions for each particle.
        
        Works by Chen, Chacón et al. \cite{Chen_Chacón_Barnes_2011, Chacón_Chen_Barnes_2013, Chen_Chacón_2014, Chen_Chacón_2015} have developed structure-preserving particle pushers for neoclassical transport in the Vlasov equations, derived from Crank--Nicolson integrators. We show these too can can derive from a FET interpretation, similarly offering potential extensions to higher-order-in-time particle pushers. The FET formulation is used also to consider how the stochastic drift terms can be incorporated into the pushers. Stochastic gyrokinetic expansions are also discussed.

        Different options for the numerical implementation of these schemes are considered.

        Due to the efficacy of FET in the development of SP timesteppers for both the fluid and kinetic component, we hope this approach will prove effective in the future for developing SP timesteppers for the full hybrid model. We hope this will give us the opportunity to incorporate previously inaccessible kinetic effects into the highly effective, modern, finite-element MHD models.
    \end{abstract}
    
    
    \newpage
    \tableofcontents
    
    
    \newpage
    \pagenumbering{arabic}
    %\linenumbers\renewcommand\thelinenumber{\color{black!50}\arabic{linenumber}}
            \documentclass[12pt, a4paper]{report}

\input{template/main.tex}

\title{\BA{Title in Progress...}}
\author{Boris Andrews}
\affil{Mathematical Institute, University of Oxford}
\date{\today}


\begin{document}
    \pagenumbering{gobble}
    \maketitle
    
    
    \begin{abstract}
        Magnetic confinement reactors---in particular tokamaks---offer one of the most promising options for achieving practical nuclear fusion, with the potential to provide virtually limitless, clean energy. The theoretical and numerical modeling of tokamak plasmas is simultaneously an essential component of effective reactor design, and a great research barrier. Tokamak operational conditions exhibit comparatively low Knudsen numbers. Kinetic effects, including kinetic waves and instabilities, Landau damping, bump-on-tail instabilities and more, are therefore highly influential in tokamak plasma dynamics. Purely fluid models are inherently incapable of capturing these effects, whereas the high dimensionality in purely kinetic models render them practically intractable for most relevant purposes.

        We consider a $\delta\!f$ decomposition model, with a macroscopic fluid background and microscopic kinetic correction, both fully coupled to each other. A similar manner of discretization is proposed to that used in the recent \texttt{STRUPHY} code \cite{Holderied_Possanner_Wang_2021, Holderied_2022, Li_et_al_2023} with a finite-element model for the background and a pseudo-particle/PiC model for the correction.

        The fluid background satisfies the full, non-linear, resistive, compressible, Hall MHD equations. \cite{Laakmann_Hu_Farrell_2022} introduces finite-element(-in-space) implicit timesteppers for the incompressible analogue to this system with structure-preserving (SP) properties in the ideal case, alongside parameter-robust preconditioners. We show that these timesteppers can derive from a finite-element-in-time (FET) (and finite-element-in-space) interpretation. The benefits of this reformulation are discussed, including the derivation of timesteppers that are higher order in time, and the quantifiable dissipative SP properties in the non-ideal, resistive case.
        
        We discuss possible options for extending this FET approach to timesteppers for the compressible case.

        The kinetic corrections satisfy linearized Boltzmann equations. Using a Lénard--Bernstein collision operator, these take Fokker--Planck-like forms \cite{Fokker_1914, Planck_1917} wherein pseudo-particles in the numerical model obey the neoclassical transport equations, with particle-independent Brownian drift terms. This offers a rigorous methodology for incorporating collisions into the particle transport model, without coupling the equations of motions for each particle.
        
        Works by Chen, Chacón et al. \cite{Chen_Chacón_Barnes_2011, Chacón_Chen_Barnes_2013, Chen_Chacón_2014, Chen_Chacón_2015} have developed structure-preserving particle pushers for neoclassical transport in the Vlasov equations, derived from Crank--Nicolson integrators. We show these too can can derive from a FET interpretation, similarly offering potential extensions to higher-order-in-time particle pushers. The FET formulation is used also to consider how the stochastic drift terms can be incorporated into the pushers. Stochastic gyrokinetic expansions are also discussed.

        Different options for the numerical implementation of these schemes are considered.

        Due to the efficacy of FET in the development of SP timesteppers for both the fluid and kinetic component, we hope this approach will prove effective in the future for developing SP timesteppers for the full hybrid model. We hope this will give us the opportunity to incorporate previously inaccessible kinetic effects into the highly effective, modern, finite-element MHD models.
    \end{abstract}
    
    
    \newpage
    \tableofcontents
    
    
    \newpage
    \pagenumbering{arabic}
    %\linenumbers\renewcommand\thelinenumber{\color{black!50}\arabic{linenumber}}
            \input{0 - introduction/main.tex}
        \part{Research}
            \input{1 - low-noise PiC models/main.tex}
            \input{2 - kinetic component/main.tex}
            \input{3 - fluid component/main.tex}
            \input{4 - numerical implementation/main.tex}
        \part{Project Overview}
            \input{5 - research plan/main.tex}
            \input{6 - summary/main.tex}
    
    
    %\section{}
    \newpage
    \pagenumbering{gobble}
        \printbibliography


    \newpage
    \pagenumbering{roman}
    \appendix
        \part{Appendices}
            \input{8 - Hilbert complexes/main.tex}
            \input{9 - weak conservation proofs/main.tex}
\end{document}

        \part{Research}
            \documentclass[12pt, a4paper]{report}

\input{template/main.tex}

\title{\BA{Title in Progress...}}
\author{Boris Andrews}
\affil{Mathematical Institute, University of Oxford}
\date{\today}


\begin{document}
    \pagenumbering{gobble}
    \maketitle
    
    
    \begin{abstract}
        Magnetic confinement reactors---in particular tokamaks---offer one of the most promising options for achieving practical nuclear fusion, with the potential to provide virtually limitless, clean energy. The theoretical and numerical modeling of tokamak plasmas is simultaneously an essential component of effective reactor design, and a great research barrier. Tokamak operational conditions exhibit comparatively low Knudsen numbers. Kinetic effects, including kinetic waves and instabilities, Landau damping, bump-on-tail instabilities and more, are therefore highly influential in tokamak plasma dynamics. Purely fluid models are inherently incapable of capturing these effects, whereas the high dimensionality in purely kinetic models render them practically intractable for most relevant purposes.

        We consider a $\delta\!f$ decomposition model, with a macroscopic fluid background and microscopic kinetic correction, both fully coupled to each other. A similar manner of discretization is proposed to that used in the recent \texttt{STRUPHY} code \cite{Holderied_Possanner_Wang_2021, Holderied_2022, Li_et_al_2023} with a finite-element model for the background and a pseudo-particle/PiC model for the correction.

        The fluid background satisfies the full, non-linear, resistive, compressible, Hall MHD equations. \cite{Laakmann_Hu_Farrell_2022} introduces finite-element(-in-space) implicit timesteppers for the incompressible analogue to this system with structure-preserving (SP) properties in the ideal case, alongside parameter-robust preconditioners. We show that these timesteppers can derive from a finite-element-in-time (FET) (and finite-element-in-space) interpretation. The benefits of this reformulation are discussed, including the derivation of timesteppers that are higher order in time, and the quantifiable dissipative SP properties in the non-ideal, resistive case.
        
        We discuss possible options for extending this FET approach to timesteppers for the compressible case.

        The kinetic corrections satisfy linearized Boltzmann equations. Using a Lénard--Bernstein collision operator, these take Fokker--Planck-like forms \cite{Fokker_1914, Planck_1917} wherein pseudo-particles in the numerical model obey the neoclassical transport equations, with particle-independent Brownian drift terms. This offers a rigorous methodology for incorporating collisions into the particle transport model, without coupling the equations of motions for each particle.
        
        Works by Chen, Chacón et al. \cite{Chen_Chacón_Barnes_2011, Chacón_Chen_Barnes_2013, Chen_Chacón_2014, Chen_Chacón_2015} have developed structure-preserving particle pushers for neoclassical transport in the Vlasov equations, derived from Crank--Nicolson integrators. We show these too can can derive from a FET interpretation, similarly offering potential extensions to higher-order-in-time particle pushers. The FET formulation is used also to consider how the stochastic drift terms can be incorporated into the pushers. Stochastic gyrokinetic expansions are also discussed.

        Different options for the numerical implementation of these schemes are considered.

        Due to the efficacy of FET in the development of SP timesteppers for both the fluid and kinetic component, we hope this approach will prove effective in the future for developing SP timesteppers for the full hybrid model. We hope this will give us the opportunity to incorporate previously inaccessible kinetic effects into the highly effective, modern, finite-element MHD models.
    \end{abstract}
    
    
    \newpage
    \tableofcontents
    
    
    \newpage
    \pagenumbering{arabic}
    %\linenumbers\renewcommand\thelinenumber{\color{black!50}\arabic{linenumber}}
            \input{0 - introduction/main.tex}
        \part{Research}
            \input{1 - low-noise PiC models/main.tex}
            \input{2 - kinetic component/main.tex}
            \input{3 - fluid component/main.tex}
            \input{4 - numerical implementation/main.tex}
        \part{Project Overview}
            \input{5 - research plan/main.tex}
            \input{6 - summary/main.tex}
    
    
    %\section{}
    \newpage
    \pagenumbering{gobble}
        \printbibliography


    \newpage
    \pagenumbering{roman}
    \appendix
        \part{Appendices}
            \input{8 - Hilbert complexes/main.tex}
            \input{9 - weak conservation proofs/main.tex}
\end{document}

            \documentclass[12pt, a4paper]{report}

\input{template/main.tex}

\title{\BA{Title in Progress...}}
\author{Boris Andrews}
\affil{Mathematical Institute, University of Oxford}
\date{\today}


\begin{document}
    \pagenumbering{gobble}
    \maketitle
    
    
    \begin{abstract}
        Magnetic confinement reactors---in particular tokamaks---offer one of the most promising options for achieving practical nuclear fusion, with the potential to provide virtually limitless, clean energy. The theoretical and numerical modeling of tokamak plasmas is simultaneously an essential component of effective reactor design, and a great research barrier. Tokamak operational conditions exhibit comparatively low Knudsen numbers. Kinetic effects, including kinetic waves and instabilities, Landau damping, bump-on-tail instabilities and more, are therefore highly influential in tokamak plasma dynamics. Purely fluid models are inherently incapable of capturing these effects, whereas the high dimensionality in purely kinetic models render them practically intractable for most relevant purposes.

        We consider a $\delta\!f$ decomposition model, with a macroscopic fluid background and microscopic kinetic correction, both fully coupled to each other. A similar manner of discretization is proposed to that used in the recent \texttt{STRUPHY} code \cite{Holderied_Possanner_Wang_2021, Holderied_2022, Li_et_al_2023} with a finite-element model for the background and a pseudo-particle/PiC model for the correction.

        The fluid background satisfies the full, non-linear, resistive, compressible, Hall MHD equations. \cite{Laakmann_Hu_Farrell_2022} introduces finite-element(-in-space) implicit timesteppers for the incompressible analogue to this system with structure-preserving (SP) properties in the ideal case, alongside parameter-robust preconditioners. We show that these timesteppers can derive from a finite-element-in-time (FET) (and finite-element-in-space) interpretation. The benefits of this reformulation are discussed, including the derivation of timesteppers that are higher order in time, and the quantifiable dissipative SP properties in the non-ideal, resistive case.
        
        We discuss possible options for extending this FET approach to timesteppers for the compressible case.

        The kinetic corrections satisfy linearized Boltzmann equations. Using a Lénard--Bernstein collision operator, these take Fokker--Planck-like forms \cite{Fokker_1914, Planck_1917} wherein pseudo-particles in the numerical model obey the neoclassical transport equations, with particle-independent Brownian drift terms. This offers a rigorous methodology for incorporating collisions into the particle transport model, without coupling the equations of motions for each particle.
        
        Works by Chen, Chacón et al. \cite{Chen_Chacón_Barnes_2011, Chacón_Chen_Barnes_2013, Chen_Chacón_2014, Chen_Chacón_2015} have developed structure-preserving particle pushers for neoclassical transport in the Vlasov equations, derived from Crank--Nicolson integrators. We show these too can can derive from a FET interpretation, similarly offering potential extensions to higher-order-in-time particle pushers. The FET formulation is used also to consider how the stochastic drift terms can be incorporated into the pushers. Stochastic gyrokinetic expansions are also discussed.

        Different options for the numerical implementation of these schemes are considered.

        Due to the efficacy of FET in the development of SP timesteppers for both the fluid and kinetic component, we hope this approach will prove effective in the future for developing SP timesteppers for the full hybrid model. We hope this will give us the opportunity to incorporate previously inaccessible kinetic effects into the highly effective, modern, finite-element MHD models.
    \end{abstract}
    
    
    \newpage
    \tableofcontents
    
    
    \newpage
    \pagenumbering{arabic}
    %\linenumbers\renewcommand\thelinenumber{\color{black!50}\arabic{linenumber}}
            \input{0 - introduction/main.tex}
        \part{Research}
            \input{1 - low-noise PiC models/main.tex}
            \input{2 - kinetic component/main.tex}
            \input{3 - fluid component/main.tex}
            \input{4 - numerical implementation/main.tex}
        \part{Project Overview}
            \input{5 - research plan/main.tex}
            \input{6 - summary/main.tex}
    
    
    %\section{}
    \newpage
    \pagenumbering{gobble}
        \printbibliography


    \newpage
    \pagenumbering{roman}
    \appendix
        \part{Appendices}
            \input{8 - Hilbert complexes/main.tex}
            \input{9 - weak conservation proofs/main.tex}
\end{document}

            \documentclass[12pt, a4paper]{report}

\input{template/main.tex}

\title{\BA{Title in Progress...}}
\author{Boris Andrews}
\affil{Mathematical Institute, University of Oxford}
\date{\today}


\begin{document}
    \pagenumbering{gobble}
    \maketitle
    
    
    \begin{abstract}
        Magnetic confinement reactors---in particular tokamaks---offer one of the most promising options for achieving practical nuclear fusion, with the potential to provide virtually limitless, clean energy. The theoretical and numerical modeling of tokamak plasmas is simultaneously an essential component of effective reactor design, and a great research barrier. Tokamak operational conditions exhibit comparatively low Knudsen numbers. Kinetic effects, including kinetic waves and instabilities, Landau damping, bump-on-tail instabilities and more, are therefore highly influential in tokamak plasma dynamics. Purely fluid models are inherently incapable of capturing these effects, whereas the high dimensionality in purely kinetic models render them practically intractable for most relevant purposes.

        We consider a $\delta\!f$ decomposition model, with a macroscopic fluid background and microscopic kinetic correction, both fully coupled to each other. A similar manner of discretization is proposed to that used in the recent \texttt{STRUPHY} code \cite{Holderied_Possanner_Wang_2021, Holderied_2022, Li_et_al_2023} with a finite-element model for the background and a pseudo-particle/PiC model for the correction.

        The fluid background satisfies the full, non-linear, resistive, compressible, Hall MHD equations. \cite{Laakmann_Hu_Farrell_2022} introduces finite-element(-in-space) implicit timesteppers for the incompressible analogue to this system with structure-preserving (SP) properties in the ideal case, alongside parameter-robust preconditioners. We show that these timesteppers can derive from a finite-element-in-time (FET) (and finite-element-in-space) interpretation. The benefits of this reformulation are discussed, including the derivation of timesteppers that are higher order in time, and the quantifiable dissipative SP properties in the non-ideal, resistive case.
        
        We discuss possible options for extending this FET approach to timesteppers for the compressible case.

        The kinetic corrections satisfy linearized Boltzmann equations. Using a Lénard--Bernstein collision operator, these take Fokker--Planck-like forms \cite{Fokker_1914, Planck_1917} wherein pseudo-particles in the numerical model obey the neoclassical transport equations, with particle-independent Brownian drift terms. This offers a rigorous methodology for incorporating collisions into the particle transport model, without coupling the equations of motions for each particle.
        
        Works by Chen, Chacón et al. \cite{Chen_Chacón_Barnes_2011, Chacón_Chen_Barnes_2013, Chen_Chacón_2014, Chen_Chacón_2015} have developed structure-preserving particle pushers for neoclassical transport in the Vlasov equations, derived from Crank--Nicolson integrators. We show these too can can derive from a FET interpretation, similarly offering potential extensions to higher-order-in-time particle pushers. The FET formulation is used also to consider how the stochastic drift terms can be incorporated into the pushers. Stochastic gyrokinetic expansions are also discussed.

        Different options for the numerical implementation of these schemes are considered.

        Due to the efficacy of FET in the development of SP timesteppers for both the fluid and kinetic component, we hope this approach will prove effective in the future for developing SP timesteppers for the full hybrid model. We hope this will give us the opportunity to incorporate previously inaccessible kinetic effects into the highly effective, modern, finite-element MHD models.
    \end{abstract}
    
    
    \newpage
    \tableofcontents
    
    
    \newpage
    \pagenumbering{arabic}
    %\linenumbers\renewcommand\thelinenumber{\color{black!50}\arabic{linenumber}}
            \input{0 - introduction/main.tex}
        \part{Research}
            \input{1 - low-noise PiC models/main.tex}
            \input{2 - kinetic component/main.tex}
            \input{3 - fluid component/main.tex}
            \input{4 - numerical implementation/main.tex}
        \part{Project Overview}
            \input{5 - research plan/main.tex}
            \input{6 - summary/main.tex}
    
    
    %\section{}
    \newpage
    \pagenumbering{gobble}
        \printbibliography


    \newpage
    \pagenumbering{roman}
    \appendix
        \part{Appendices}
            \input{8 - Hilbert complexes/main.tex}
            \input{9 - weak conservation proofs/main.tex}
\end{document}

            \documentclass[12pt, a4paper]{report}

\input{template/main.tex}

\title{\BA{Title in Progress...}}
\author{Boris Andrews}
\affil{Mathematical Institute, University of Oxford}
\date{\today}


\begin{document}
    \pagenumbering{gobble}
    \maketitle
    
    
    \begin{abstract}
        Magnetic confinement reactors---in particular tokamaks---offer one of the most promising options for achieving practical nuclear fusion, with the potential to provide virtually limitless, clean energy. The theoretical and numerical modeling of tokamak plasmas is simultaneously an essential component of effective reactor design, and a great research barrier. Tokamak operational conditions exhibit comparatively low Knudsen numbers. Kinetic effects, including kinetic waves and instabilities, Landau damping, bump-on-tail instabilities and more, are therefore highly influential in tokamak plasma dynamics. Purely fluid models are inherently incapable of capturing these effects, whereas the high dimensionality in purely kinetic models render them practically intractable for most relevant purposes.

        We consider a $\delta\!f$ decomposition model, with a macroscopic fluid background and microscopic kinetic correction, both fully coupled to each other. A similar manner of discretization is proposed to that used in the recent \texttt{STRUPHY} code \cite{Holderied_Possanner_Wang_2021, Holderied_2022, Li_et_al_2023} with a finite-element model for the background and a pseudo-particle/PiC model for the correction.

        The fluid background satisfies the full, non-linear, resistive, compressible, Hall MHD equations. \cite{Laakmann_Hu_Farrell_2022} introduces finite-element(-in-space) implicit timesteppers for the incompressible analogue to this system with structure-preserving (SP) properties in the ideal case, alongside parameter-robust preconditioners. We show that these timesteppers can derive from a finite-element-in-time (FET) (and finite-element-in-space) interpretation. The benefits of this reformulation are discussed, including the derivation of timesteppers that are higher order in time, and the quantifiable dissipative SP properties in the non-ideal, resistive case.
        
        We discuss possible options for extending this FET approach to timesteppers for the compressible case.

        The kinetic corrections satisfy linearized Boltzmann equations. Using a Lénard--Bernstein collision operator, these take Fokker--Planck-like forms \cite{Fokker_1914, Planck_1917} wherein pseudo-particles in the numerical model obey the neoclassical transport equations, with particle-independent Brownian drift terms. This offers a rigorous methodology for incorporating collisions into the particle transport model, without coupling the equations of motions for each particle.
        
        Works by Chen, Chacón et al. \cite{Chen_Chacón_Barnes_2011, Chacón_Chen_Barnes_2013, Chen_Chacón_2014, Chen_Chacón_2015} have developed structure-preserving particle pushers for neoclassical transport in the Vlasov equations, derived from Crank--Nicolson integrators. We show these too can can derive from a FET interpretation, similarly offering potential extensions to higher-order-in-time particle pushers. The FET formulation is used also to consider how the stochastic drift terms can be incorporated into the pushers. Stochastic gyrokinetic expansions are also discussed.

        Different options for the numerical implementation of these schemes are considered.

        Due to the efficacy of FET in the development of SP timesteppers for both the fluid and kinetic component, we hope this approach will prove effective in the future for developing SP timesteppers for the full hybrid model. We hope this will give us the opportunity to incorporate previously inaccessible kinetic effects into the highly effective, modern, finite-element MHD models.
    \end{abstract}
    
    
    \newpage
    \tableofcontents
    
    
    \newpage
    \pagenumbering{arabic}
    %\linenumbers\renewcommand\thelinenumber{\color{black!50}\arabic{linenumber}}
            \input{0 - introduction/main.tex}
        \part{Research}
            \input{1 - low-noise PiC models/main.tex}
            \input{2 - kinetic component/main.tex}
            \input{3 - fluid component/main.tex}
            \input{4 - numerical implementation/main.tex}
        \part{Project Overview}
            \input{5 - research plan/main.tex}
            \input{6 - summary/main.tex}
    
    
    %\section{}
    \newpage
    \pagenumbering{gobble}
        \printbibliography


    \newpage
    \pagenumbering{roman}
    \appendix
        \part{Appendices}
            \input{8 - Hilbert complexes/main.tex}
            \input{9 - weak conservation proofs/main.tex}
\end{document}

        \part{Project Overview}
            \documentclass[12pt, a4paper]{report}

\input{template/main.tex}

\title{\BA{Title in Progress...}}
\author{Boris Andrews}
\affil{Mathematical Institute, University of Oxford}
\date{\today}


\begin{document}
    \pagenumbering{gobble}
    \maketitle
    
    
    \begin{abstract}
        Magnetic confinement reactors---in particular tokamaks---offer one of the most promising options for achieving practical nuclear fusion, with the potential to provide virtually limitless, clean energy. The theoretical and numerical modeling of tokamak plasmas is simultaneously an essential component of effective reactor design, and a great research barrier. Tokamak operational conditions exhibit comparatively low Knudsen numbers. Kinetic effects, including kinetic waves and instabilities, Landau damping, bump-on-tail instabilities and more, are therefore highly influential in tokamak plasma dynamics. Purely fluid models are inherently incapable of capturing these effects, whereas the high dimensionality in purely kinetic models render them practically intractable for most relevant purposes.

        We consider a $\delta\!f$ decomposition model, with a macroscopic fluid background and microscopic kinetic correction, both fully coupled to each other. A similar manner of discretization is proposed to that used in the recent \texttt{STRUPHY} code \cite{Holderied_Possanner_Wang_2021, Holderied_2022, Li_et_al_2023} with a finite-element model for the background and a pseudo-particle/PiC model for the correction.

        The fluid background satisfies the full, non-linear, resistive, compressible, Hall MHD equations. \cite{Laakmann_Hu_Farrell_2022} introduces finite-element(-in-space) implicit timesteppers for the incompressible analogue to this system with structure-preserving (SP) properties in the ideal case, alongside parameter-robust preconditioners. We show that these timesteppers can derive from a finite-element-in-time (FET) (and finite-element-in-space) interpretation. The benefits of this reformulation are discussed, including the derivation of timesteppers that are higher order in time, and the quantifiable dissipative SP properties in the non-ideal, resistive case.
        
        We discuss possible options for extending this FET approach to timesteppers for the compressible case.

        The kinetic corrections satisfy linearized Boltzmann equations. Using a Lénard--Bernstein collision operator, these take Fokker--Planck-like forms \cite{Fokker_1914, Planck_1917} wherein pseudo-particles in the numerical model obey the neoclassical transport equations, with particle-independent Brownian drift terms. This offers a rigorous methodology for incorporating collisions into the particle transport model, without coupling the equations of motions for each particle.
        
        Works by Chen, Chacón et al. \cite{Chen_Chacón_Barnes_2011, Chacón_Chen_Barnes_2013, Chen_Chacón_2014, Chen_Chacón_2015} have developed structure-preserving particle pushers for neoclassical transport in the Vlasov equations, derived from Crank--Nicolson integrators. We show these too can can derive from a FET interpretation, similarly offering potential extensions to higher-order-in-time particle pushers. The FET formulation is used also to consider how the stochastic drift terms can be incorporated into the pushers. Stochastic gyrokinetic expansions are also discussed.

        Different options for the numerical implementation of these schemes are considered.

        Due to the efficacy of FET in the development of SP timesteppers for both the fluid and kinetic component, we hope this approach will prove effective in the future for developing SP timesteppers for the full hybrid model. We hope this will give us the opportunity to incorporate previously inaccessible kinetic effects into the highly effective, modern, finite-element MHD models.
    \end{abstract}
    
    
    \newpage
    \tableofcontents
    
    
    \newpage
    \pagenumbering{arabic}
    %\linenumbers\renewcommand\thelinenumber{\color{black!50}\arabic{linenumber}}
            \input{0 - introduction/main.tex}
        \part{Research}
            \input{1 - low-noise PiC models/main.tex}
            \input{2 - kinetic component/main.tex}
            \input{3 - fluid component/main.tex}
            \input{4 - numerical implementation/main.tex}
        \part{Project Overview}
            \input{5 - research plan/main.tex}
            \input{6 - summary/main.tex}
    
    
    %\section{}
    \newpage
    \pagenumbering{gobble}
        \printbibliography


    \newpage
    \pagenumbering{roman}
    \appendix
        \part{Appendices}
            \input{8 - Hilbert complexes/main.tex}
            \input{9 - weak conservation proofs/main.tex}
\end{document}

            \documentclass[12pt, a4paper]{report}

\input{template/main.tex}

\title{\BA{Title in Progress...}}
\author{Boris Andrews}
\affil{Mathematical Institute, University of Oxford}
\date{\today}


\begin{document}
    \pagenumbering{gobble}
    \maketitle
    
    
    \begin{abstract}
        Magnetic confinement reactors---in particular tokamaks---offer one of the most promising options for achieving practical nuclear fusion, with the potential to provide virtually limitless, clean energy. The theoretical and numerical modeling of tokamak plasmas is simultaneously an essential component of effective reactor design, and a great research barrier. Tokamak operational conditions exhibit comparatively low Knudsen numbers. Kinetic effects, including kinetic waves and instabilities, Landau damping, bump-on-tail instabilities and more, are therefore highly influential in tokamak plasma dynamics. Purely fluid models are inherently incapable of capturing these effects, whereas the high dimensionality in purely kinetic models render them practically intractable for most relevant purposes.

        We consider a $\delta\!f$ decomposition model, with a macroscopic fluid background and microscopic kinetic correction, both fully coupled to each other. A similar manner of discretization is proposed to that used in the recent \texttt{STRUPHY} code \cite{Holderied_Possanner_Wang_2021, Holderied_2022, Li_et_al_2023} with a finite-element model for the background and a pseudo-particle/PiC model for the correction.

        The fluid background satisfies the full, non-linear, resistive, compressible, Hall MHD equations. \cite{Laakmann_Hu_Farrell_2022} introduces finite-element(-in-space) implicit timesteppers for the incompressible analogue to this system with structure-preserving (SP) properties in the ideal case, alongside parameter-robust preconditioners. We show that these timesteppers can derive from a finite-element-in-time (FET) (and finite-element-in-space) interpretation. The benefits of this reformulation are discussed, including the derivation of timesteppers that are higher order in time, and the quantifiable dissipative SP properties in the non-ideal, resistive case.
        
        We discuss possible options for extending this FET approach to timesteppers for the compressible case.

        The kinetic corrections satisfy linearized Boltzmann equations. Using a Lénard--Bernstein collision operator, these take Fokker--Planck-like forms \cite{Fokker_1914, Planck_1917} wherein pseudo-particles in the numerical model obey the neoclassical transport equations, with particle-independent Brownian drift terms. This offers a rigorous methodology for incorporating collisions into the particle transport model, without coupling the equations of motions for each particle.
        
        Works by Chen, Chacón et al. \cite{Chen_Chacón_Barnes_2011, Chacón_Chen_Barnes_2013, Chen_Chacón_2014, Chen_Chacón_2015} have developed structure-preserving particle pushers for neoclassical transport in the Vlasov equations, derived from Crank--Nicolson integrators. We show these too can can derive from a FET interpretation, similarly offering potential extensions to higher-order-in-time particle pushers. The FET formulation is used also to consider how the stochastic drift terms can be incorporated into the pushers. Stochastic gyrokinetic expansions are also discussed.

        Different options for the numerical implementation of these schemes are considered.

        Due to the efficacy of FET in the development of SP timesteppers for both the fluid and kinetic component, we hope this approach will prove effective in the future for developing SP timesteppers for the full hybrid model. We hope this will give us the opportunity to incorporate previously inaccessible kinetic effects into the highly effective, modern, finite-element MHD models.
    \end{abstract}
    
    
    \newpage
    \tableofcontents
    
    
    \newpage
    \pagenumbering{arabic}
    %\linenumbers\renewcommand\thelinenumber{\color{black!50}\arabic{linenumber}}
            \input{0 - introduction/main.tex}
        \part{Research}
            \input{1 - low-noise PiC models/main.tex}
            \input{2 - kinetic component/main.tex}
            \input{3 - fluid component/main.tex}
            \input{4 - numerical implementation/main.tex}
        \part{Project Overview}
            \input{5 - research plan/main.tex}
            \input{6 - summary/main.tex}
    
    
    %\section{}
    \newpage
    \pagenumbering{gobble}
        \printbibliography


    \newpage
    \pagenumbering{roman}
    \appendix
        \part{Appendices}
            \input{8 - Hilbert complexes/main.tex}
            \input{9 - weak conservation proofs/main.tex}
\end{document}

    
    
    %\section{}
    \newpage
    \pagenumbering{gobble}
        \printbibliography


    \newpage
    \pagenumbering{roman}
    \appendix
        \part{Appendices}
            \documentclass[12pt, a4paper]{report}

\input{template/main.tex}

\title{\BA{Title in Progress...}}
\author{Boris Andrews}
\affil{Mathematical Institute, University of Oxford}
\date{\today}


\begin{document}
    \pagenumbering{gobble}
    \maketitle
    
    
    \begin{abstract}
        Magnetic confinement reactors---in particular tokamaks---offer one of the most promising options for achieving practical nuclear fusion, with the potential to provide virtually limitless, clean energy. The theoretical and numerical modeling of tokamak plasmas is simultaneously an essential component of effective reactor design, and a great research barrier. Tokamak operational conditions exhibit comparatively low Knudsen numbers. Kinetic effects, including kinetic waves and instabilities, Landau damping, bump-on-tail instabilities and more, are therefore highly influential in tokamak plasma dynamics. Purely fluid models are inherently incapable of capturing these effects, whereas the high dimensionality in purely kinetic models render them practically intractable for most relevant purposes.

        We consider a $\delta\!f$ decomposition model, with a macroscopic fluid background and microscopic kinetic correction, both fully coupled to each other. A similar manner of discretization is proposed to that used in the recent \texttt{STRUPHY} code \cite{Holderied_Possanner_Wang_2021, Holderied_2022, Li_et_al_2023} with a finite-element model for the background and a pseudo-particle/PiC model for the correction.

        The fluid background satisfies the full, non-linear, resistive, compressible, Hall MHD equations. \cite{Laakmann_Hu_Farrell_2022} introduces finite-element(-in-space) implicit timesteppers for the incompressible analogue to this system with structure-preserving (SP) properties in the ideal case, alongside parameter-robust preconditioners. We show that these timesteppers can derive from a finite-element-in-time (FET) (and finite-element-in-space) interpretation. The benefits of this reformulation are discussed, including the derivation of timesteppers that are higher order in time, and the quantifiable dissipative SP properties in the non-ideal, resistive case.
        
        We discuss possible options for extending this FET approach to timesteppers for the compressible case.

        The kinetic corrections satisfy linearized Boltzmann equations. Using a Lénard--Bernstein collision operator, these take Fokker--Planck-like forms \cite{Fokker_1914, Planck_1917} wherein pseudo-particles in the numerical model obey the neoclassical transport equations, with particle-independent Brownian drift terms. This offers a rigorous methodology for incorporating collisions into the particle transport model, without coupling the equations of motions for each particle.
        
        Works by Chen, Chacón et al. \cite{Chen_Chacón_Barnes_2011, Chacón_Chen_Barnes_2013, Chen_Chacón_2014, Chen_Chacón_2015} have developed structure-preserving particle pushers for neoclassical transport in the Vlasov equations, derived from Crank--Nicolson integrators. We show these too can can derive from a FET interpretation, similarly offering potential extensions to higher-order-in-time particle pushers. The FET formulation is used also to consider how the stochastic drift terms can be incorporated into the pushers. Stochastic gyrokinetic expansions are also discussed.

        Different options for the numerical implementation of these schemes are considered.

        Due to the efficacy of FET in the development of SP timesteppers for both the fluid and kinetic component, we hope this approach will prove effective in the future for developing SP timesteppers for the full hybrid model. We hope this will give us the opportunity to incorporate previously inaccessible kinetic effects into the highly effective, modern, finite-element MHD models.
    \end{abstract}
    
    
    \newpage
    \tableofcontents
    
    
    \newpage
    \pagenumbering{arabic}
    %\linenumbers\renewcommand\thelinenumber{\color{black!50}\arabic{linenumber}}
            \input{0 - introduction/main.tex}
        \part{Research}
            \input{1 - low-noise PiC models/main.tex}
            \input{2 - kinetic component/main.tex}
            \input{3 - fluid component/main.tex}
            \input{4 - numerical implementation/main.tex}
        \part{Project Overview}
            \input{5 - research plan/main.tex}
            \input{6 - summary/main.tex}
    
    
    %\section{}
    \newpage
    \pagenumbering{gobble}
        \printbibliography


    \newpage
    \pagenumbering{roman}
    \appendix
        \part{Appendices}
            \input{8 - Hilbert complexes/main.tex}
            \input{9 - weak conservation proofs/main.tex}
\end{document}

            \documentclass[12pt, a4paper]{report}

\input{template/main.tex}

\title{\BA{Title in Progress...}}
\author{Boris Andrews}
\affil{Mathematical Institute, University of Oxford}
\date{\today}


\begin{document}
    \pagenumbering{gobble}
    \maketitle
    
    
    \begin{abstract}
        Magnetic confinement reactors---in particular tokamaks---offer one of the most promising options for achieving practical nuclear fusion, with the potential to provide virtually limitless, clean energy. The theoretical and numerical modeling of tokamak plasmas is simultaneously an essential component of effective reactor design, and a great research barrier. Tokamak operational conditions exhibit comparatively low Knudsen numbers. Kinetic effects, including kinetic waves and instabilities, Landau damping, bump-on-tail instabilities and more, are therefore highly influential in tokamak plasma dynamics. Purely fluid models are inherently incapable of capturing these effects, whereas the high dimensionality in purely kinetic models render them practically intractable for most relevant purposes.

        We consider a $\delta\!f$ decomposition model, with a macroscopic fluid background and microscopic kinetic correction, both fully coupled to each other. A similar manner of discretization is proposed to that used in the recent \texttt{STRUPHY} code \cite{Holderied_Possanner_Wang_2021, Holderied_2022, Li_et_al_2023} with a finite-element model for the background and a pseudo-particle/PiC model for the correction.

        The fluid background satisfies the full, non-linear, resistive, compressible, Hall MHD equations. \cite{Laakmann_Hu_Farrell_2022} introduces finite-element(-in-space) implicit timesteppers for the incompressible analogue to this system with structure-preserving (SP) properties in the ideal case, alongside parameter-robust preconditioners. We show that these timesteppers can derive from a finite-element-in-time (FET) (and finite-element-in-space) interpretation. The benefits of this reformulation are discussed, including the derivation of timesteppers that are higher order in time, and the quantifiable dissipative SP properties in the non-ideal, resistive case.
        
        We discuss possible options for extending this FET approach to timesteppers for the compressible case.

        The kinetic corrections satisfy linearized Boltzmann equations. Using a Lénard--Bernstein collision operator, these take Fokker--Planck-like forms \cite{Fokker_1914, Planck_1917} wherein pseudo-particles in the numerical model obey the neoclassical transport equations, with particle-independent Brownian drift terms. This offers a rigorous methodology for incorporating collisions into the particle transport model, without coupling the equations of motions for each particle.
        
        Works by Chen, Chacón et al. \cite{Chen_Chacón_Barnes_2011, Chacón_Chen_Barnes_2013, Chen_Chacón_2014, Chen_Chacón_2015} have developed structure-preserving particle pushers for neoclassical transport in the Vlasov equations, derived from Crank--Nicolson integrators. We show these too can can derive from a FET interpretation, similarly offering potential extensions to higher-order-in-time particle pushers. The FET formulation is used also to consider how the stochastic drift terms can be incorporated into the pushers. Stochastic gyrokinetic expansions are also discussed.

        Different options for the numerical implementation of these schemes are considered.

        Due to the efficacy of FET in the development of SP timesteppers for both the fluid and kinetic component, we hope this approach will prove effective in the future for developing SP timesteppers for the full hybrid model. We hope this will give us the opportunity to incorporate previously inaccessible kinetic effects into the highly effective, modern, finite-element MHD models.
    \end{abstract}
    
    
    \newpage
    \tableofcontents
    
    
    \newpage
    \pagenumbering{arabic}
    %\linenumbers\renewcommand\thelinenumber{\color{black!50}\arabic{linenumber}}
            \input{0 - introduction/main.tex}
        \part{Research}
            \input{1 - low-noise PiC models/main.tex}
            \input{2 - kinetic component/main.tex}
            \input{3 - fluid component/main.tex}
            \input{4 - numerical implementation/main.tex}
        \part{Project Overview}
            \input{5 - research plan/main.tex}
            \input{6 - summary/main.tex}
    
    
    %\section{}
    \newpage
    \pagenumbering{gobble}
        \printbibliography


    \newpage
    \pagenumbering{roman}
    \appendix
        \part{Appendices}
            \input{8 - Hilbert complexes/main.tex}
            \input{9 - weak conservation proofs/main.tex}
\end{document}

\end{document}

\end{document}

\end{document}
