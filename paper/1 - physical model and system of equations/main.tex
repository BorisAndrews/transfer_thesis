\chapter{Physical Model and System of Equations}
    \BA{Introduction.}
    
    \BA{Room for lots of pictures here.}
    
    \BA{What physics characterize a (magnetized) neutral plasma? Quasi-neutral mix of \emph{separated} electrically-charged phases. (Check out the \href{https://en.wikipedia.org/wiki/Plasma_(physics)}{plasma Wikipedia page}.)}
    
    \BA{Will make the assumptions:
    \begin{itemize}
        \item  Only 2 phases (positive and negative) i.e.:
        \begin{itemize}
            \item  The plasma is fully ionized, i.e. neutrals (or at least the effects thereof) are negligible. I've been led to believe this is generally the case in edge plasmas (i.e. outside the divertor). (A bold assumption?)
            \item  Dust and impurities (or at least the effects thereof) are negligible. (A bold assumption?)
        \end{itemize}
        \item  Only (thermalizing) Coulomb collisions are considered- these are generally dominant over the others in a tokamak. (N.B. No fusion.)
        \item  Relativistic effects are negligible. (Is observing that the velocity scale is much smaller than $c$ sufficient justification? Imperial guy at the NEPTUNE workshop appeared to think not.)
    \end{itemize}}
    
    
    \section{Preserved Structures}
    \BA{Introduction.}
    
    Consider first those quantities that are conserved by the transient system, so as to seek discretisations which better represent the physical behaviour of the system by \emph{also} conserved these quantities. 
    
    \cite{LHF22} considers conservation of the following 3 quantities, which the authors define in the incompressible case as: \BA{(Oops I've never defined $\bfA$! That should probably be in the introduction...)}
    \begin{center}\begin{tabular}{ c c c }
        Properties  &  Symbol  &  Definition  \\
        \hline\hline
        Energy  &  $\rmE$  &  $\int_{\bfOmega}\left[\frac{1}{\rmEu\rho}\|\bfp\|^{2} + p + \frac{1}{\beta}\|\bfB\|^{2}\right]$  \\
        Magnetic helicity  &  $\rmH_{\rmM}$  &  $\int_{\bfOmega}\bfA\cdot\bfB$  \\
        Hybrid helicity  &  $\rmH_{\rmH}$  &  $\int_{\bfOmega}(a\bfA + \bfp)\cdot(b\bfB + \nabla\wedge\bfp)$
    \end{tabular}\end{center}
    where $a$, $b$ satisfy the relation $a + b  =  \frac{4}{\beta\rmRH}$. \BA{(What do these represent \emph{physically}? Diagrams!)} Taking the derivatives of these quantities over time (still in the incompressible system) gives
    \begin{align}
        \frac{d\rmE}{dt}  &=  \BA{\cdots}  \\
        \frac{d\rmH_{\rmM}}{dt}  &=  \int_{\bfGamma}(- \varphi\bfB + \bfA\wedge\bfE)\cdot\bfn - \frac{2}{\rmRem}\int_{\bfOmega}\bfB\cdot\bfj  \\
        \frac{d\rmH_{\rmH}}{dt}  &=  \BA{\cdots} \\
    \end{align}

    \BA{Proven that in the \emph{compressible} case, $\frac{d\rmE}{dt}$ evaluates as
    {\small \begin{equation}
        \frac{d\rmE}{dt}  =  \int_{\bfGamma}\left[- \frac{1}{2\rmEu\rho}\|\bfp\|^{2}\bfp - \frac{p}{2\rho}\bfp + \frac{1}{\rmEu\rmRe_{f}}\nabla\left[\frac{1}{\rho}\bfp\right]\cdot\frac{1}{\rho}\bfp - \frac{p}{2\rho}\bfp + \frac{1}{2\rmPe}\nabla\left[\frac{p}{\rho} + \frac{1}{\beta}\bfB\wedge\bfE\right]\right]\cdot\bfn
    \end{equation}}}
    
    \section{Preserved Structures}
    \BA{Introduction.}
    
    Consider first those quantities that are conserved by the transient system, so as to seek discretisations which better represent the physical behaviour of the system by \emph{also} conserved these quantities. 
    
    \cite{LHF22} considers conservation of the following 3 quantities, which the authors define in the incompressible case as: \BA{(Oops I've never defined $\bfA$! That should probably be in the introduction...)}
    \begin{center}\begin{tabular}{ c c c }
        Properties  &  Symbol  &  Definition  \\
        \hline\hline
        Energy  &  $\rmE$  &  $\int_{\bfOmega}\left[\frac{1}{\rmEu\rho}\|\bfp\|^{2} + p + \frac{1}{\beta}\|\bfB\|^{2}\right]$  \\
        Magnetic helicity  &  $\rmH_{\rmM}$  &  $\int_{\bfOmega}\bfA\cdot\bfB$  \\
        Hybrid helicity  &  $\rmH_{\rmH}$  &  $\int_{\bfOmega}(a\bfA + \bfp)\cdot(b\bfB + \nabla\wedge\bfp)$
    \end{tabular}\end{center}
    where $a$, $b$ satisfy the relation $a + b  =  \frac{4}{\beta\rmRH}$. \BA{(What do these represent \emph{physically}? Diagrams!)} Taking the derivatives of these quantities over time (still in the incompressible system) gives
    \begin{align}
        \frac{d\rmE}{dt}  &=  \BA{\cdots}  \\
        \frac{d\rmH_{\rmM}}{dt}  &=  \int_{\bfGamma}(- \varphi\bfB + \bfA\wedge\bfE)\cdot\bfn - \frac{2}{\rmRem}\int_{\bfOmega}\bfB\cdot\bfj  \\
        \frac{d\rmH_{\rmH}}{dt}  &=  \BA{\cdots} \\
    \end{align}

    \BA{Proven that in the \emph{compressible} case, $\frac{d\rmE}{dt}$ evaluates as
    {\small \begin{equation}
        \frac{d\rmE}{dt}  =  \int_{\bfGamma}\left[- \frac{1}{2\rmEu\rho}\|\bfp\|^{2}\bfp - \frac{p}{2\rho}\bfp + \frac{1}{\rmEu\rmRe_{f}}\nabla\left[\frac{1}{\rho}\bfp\right]\cdot\frac{1}{\rho}\bfp - \frac{p}{2\rho}\bfp + \frac{1}{2\rmPe}\nabla\left[\frac{p}{\rho} + \frac{1}{\beta}\bfB\wedge\bfE\right]\right]\cdot\bfn
    \end{equation}}}
    
    \section{Preserved Structures}
    \BA{Introduction.}
    
    Consider first those quantities that are conserved by the transient system, so as to seek discretisations which better represent the physical behaviour of the system by \emph{also} conserved these quantities. 
    
    \cite{LHF22} considers conservation of the following 3 quantities, which the authors define in the incompressible case as: \BA{(Oops I've never defined $\bfA$! That should probably be in the introduction...)}
    \begin{center}\begin{tabular}{ c c c }
        Properties  &  Symbol  &  Definition  \\
        \hline\hline
        Energy  &  $\rmE$  &  $\int_{\bfOmega}\left[\frac{1}{\rmEu\rho}\|\bfp\|^{2} + p + \frac{1}{\beta}\|\bfB\|^{2}\right]$  \\
        Magnetic helicity  &  $\rmH_{\rmM}$  &  $\int_{\bfOmega}\bfA\cdot\bfB$  \\
        Hybrid helicity  &  $\rmH_{\rmH}$  &  $\int_{\bfOmega}(a\bfA + \bfp)\cdot(b\bfB + \nabla\wedge\bfp)$
    \end{tabular}\end{center}
    where $a$, $b$ satisfy the relation $a + b  =  \frac{4}{\beta\rmRH}$. \BA{(What do these represent \emph{physically}? Diagrams!)} Taking the derivatives of these quantities over time (still in the incompressible system) gives
    \begin{align}
        \frac{d\rmE}{dt}  &=  \BA{\cdots}  \\
        \frac{d\rmH_{\rmM}}{dt}  &=  \int_{\bfGamma}(- \varphi\bfB + \bfA\wedge\bfE)\cdot\bfn - \frac{2}{\rmRem}\int_{\bfOmega}\bfB\cdot\bfj  \\
        \frac{d\rmH_{\rmH}}{dt}  &=  \BA{\cdots} \\
    \end{align}

    \BA{Proven that in the \emph{compressible} case, $\frac{d\rmE}{dt}$ evaluates as
    {\small \begin{equation}
        \frac{d\rmE}{dt}  =  \int_{\bfGamma}\left[- \frac{1}{2\rmEu\rho}\|\bfp\|^{2}\bfp - \frac{p}{2\rho}\bfp + \frac{1}{\rmEu\rmRe_{f}}\nabla\left[\frac{1}{\rho}\bfp\right]\cdot\frac{1}{\rho}\bfp - \frac{p}{2\rho}\bfp + \frac{1}{2\rmPe}\nabla\left[\frac{p}{\rho} + \frac{1}{\beta}\bfB\wedge\bfE\right]\right]\cdot\bfn
    \end{equation}}}
    
    \section{Preserved Structures}
    \BA{Introduction.}
    
    Consider first those quantities that are conserved by the transient system, so as to seek discretisations which better represent the physical behaviour of the system by \emph{also} conserved these quantities. 
    
    \cite{LHF22} considers conservation of the following 3 quantities, which the authors define in the incompressible case as: \BA{(Oops I've never defined $\bfA$! That should probably be in the introduction...)}
    \begin{center}\begin{tabular}{ c c c }
        Properties  &  Symbol  &  Definition  \\
        \hline\hline
        Energy  &  $\rmE$  &  $\int_{\bfOmega}\left[\frac{1}{\rmEu\rho}\|\bfp\|^{2} + p + \frac{1}{\beta}\|\bfB\|^{2}\right]$  \\
        Magnetic helicity  &  $\rmH_{\rmM}$  &  $\int_{\bfOmega}\bfA\cdot\bfB$  \\
        Hybrid helicity  &  $\rmH_{\rmH}$  &  $\int_{\bfOmega}(a\bfA + \bfp)\cdot(b\bfB + \nabla\wedge\bfp)$
    \end{tabular}\end{center}
    where $a$, $b$ satisfy the relation $a + b  =  \frac{4}{\beta\rmRH}$. \BA{(What do these represent \emph{physically}? Diagrams!)} Taking the derivatives of these quantities over time (still in the incompressible system) gives
    \begin{align}
        \frac{d\rmE}{dt}  &=  \BA{\cdots}  \\
        \frac{d\rmH_{\rmM}}{dt}  &=  \int_{\bfGamma}(- \varphi\bfB + \bfA\wedge\bfE)\cdot\bfn - \frac{2}{\rmRem}\int_{\bfOmega}\bfB\cdot\bfj  \\
        \frac{d\rmH_{\rmH}}{dt}  &=  \BA{\cdots} \\
    \end{align}

    \BA{Proven that in the \emph{compressible} case, $\frac{d\rmE}{dt}$ evaluates as
    {\small \begin{equation}
        \frac{d\rmE}{dt}  =  \int_{\bfGamma}\left[- \frac{1}{2\rmEu\rho}\|\bfp\|^{2}\bfp - \frac{p}{2\rho}\bfp + \frac{1}{\rmEu\rmRe_{f}}\nabla\left[\frac{1}{\rho}\bfp\right]\cdot\frac{1}{\rho}\bfp - \frac{p}{2\rho}\bfp + \frac{1}{2\rmPe}\nabla\left[\frac{p}{\rho} + \frac{1}{\beta}\bfB\wedge\bfE\right]\right]\cdot\bfn
    \end{equation}}}
    
    

    \section*{Summary}
        \BA{Summary.}