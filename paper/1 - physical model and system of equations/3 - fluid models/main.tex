\section{Fluid Models: The \emph{High} Collisionality Limit}
    It is often the case that on the typical parameter scales of interest, the collisional terms massively dominate the kinetic Boltzmann equation. \BA{(Dimensional analysis here? Or at least some note saying that this is natural to expect from the typical \emph{very high} particle densities we encounter in daily life.)}
    
    \subsection{Single-Phase Fluids}
        When the collisional terms are dominant in the single-phase fluid Boltzmann equation (\ref{eqn:single-phase Boltzmann equation}) we find, up to leading order
        \begin{align}
            0  &\sim  \mu\nabla_{\bfv}\cdot[f(\bfv - \bfu)] + \frac{\sigma^{2}}{2}\Delta_{\bfv}[f]  \\
            f  &\sim  \frac{\rho}{m}\sqrt{\frac{m}{k_{B}T}}^{3}\exp\left(- \frac{m}{2k_{B}T}\|\bfv - \bfu\|^{2}\right)
        \end{align}
        where $\rho$, $\bfu$, $T$—functions of $\bfx$ (and $t$)—are moments characterising the conserved quantities of mass, momentum and energy:
        \begin{align}
            \rho  &=  \int fmd\bfv  \\
            \rho\bfu  &=  \int fm\bfv d\bfv  \\
            \frac{1}{2}\rho\left(\|\bfu\|^{2} + 3\frac{k_{B}T}{m}\right)  &=  \int f\frac{1}{2}m\|\bfv\|^{2}d\bfv
        \end{align}
        and, by its definition from the energy equation here, \BA{(Don't like the way that's phrased...)} $k_{B}  \approx  1.381\times10^{- 23}\rmJ\rmK^{- 1}$ is the Boltzmann constant.
    
    \subsection{Multi-Phase Fluids}
    \subsection{Tokamak Plasmas}