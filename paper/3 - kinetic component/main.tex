\chapter{Kinetic Component}\label{cha:kinetic component}
    Recall the kinetic component of the $\delta\!f$ decomposition, the linearized Boltzmann equations (\ref{eqn:linearized Boltzmann equation}), which we write in the form,
    \begin{equation}\label{eqn:linearized Boltzmann equation simplified 1}
        \partial_{t}\delta\!f_{\pm}
        + \nabla_{\bfx}\cdot[\delta\!f_{\pm}\bfv]
        + \rmCy\!_{\pm}\nabla_{\bfv}\cdot[\delta\!f_{\pm}(\bfE + \bfv\wedge\bfB)]
        - \nabla_{\bfv}\cdot\left[\rmKn_{\pm\pm}\partial\bfC_{\pm\pm}^{(0)} + \rmKn_{\pm\mp}\partial\bfC_{\pm\mp}^{(0)}\right]
        =  \calF_{\pm}
    \end{equation}
    neglecting $\calO[\rmCy\!_{\pm}/\rmCy\!_{+}, \rmCy\!_{+}/\rmRef]$ terms, where $\calF_{\pm}$ denotes the inhomogeneous contribution from the RHS of (\ref{eqn:linearized Boltzmann equation}),
    \begin{multline}\label{eqn:inhomogeneous RHS}
        \calF_{\pm}[\rho_{\pm}, \bfp, p](\bfx, \bfv; t)
        :=  - \rmCy\!_{+}\partial f_{\pm}^{(0)}[(\rho_{\rmM}, \rho_{\rmC}, \bfp, p); \partial_{t}[(\rho_{\rmM}, \rho_{\rmC}, \bfp, p)]]  \\
        - \rmCy\!_{+}\partial f_{\pm}^{(0)}[(\rho_{\rmM}, \rho_{\rmC}, \bfp, p); \nabla_{\bfx}\cdot[\bfv\otimes(\rho_{\rmM}, \rho_{\rmC}, \bfp, p)]]  \\
        + \frac{2\rmCy\!_{+}}{\beta}\cdot\frac{1}{\rho_{\rmM}}\nabla_{\bfv}\cdot\left[f_{\pm}^{(0)}\left(\rmM^{2}\rho_{\rmC}\bfE + \bfj^{(0)}\wedge\bfB\right) - \nabla_{\bfv}\left[f_{\pm}^{(0)}\left(\bfj^{(0)} - \rmM^{2}\rho_{\rmC}\bfu\right)\cdot\left(\bfE + \bfu\wedge\bfB\right)\right]\right].
    \end{multline}

    For the numerical solution of (\ref{eqn:linearized Boltzmann equation simplified 1}) a Monte Carlo pseudo-particle/PiC approximation is proposed, whereby as the simulation progresses in time, pseudo-particles:
    \begin{itemize}
        \item  Are generated, according to the inhomogeneous RHS of (\ref{eqn:linearized Boltzmann equation simplified 1}), $\calF_{\pm}$, and eliminated, according to some modification of the system.
        \item  Move through the simulation domain, via particle pushers determined by the homogeneous LHS of (\ref{eqn:linearized Boltzmann equation simplified 1}).
    \end{itemize}
    The latter of these 2 parts, the pseudo-particle motion, shall be discussed first.


    \section{Preserved Structures}
    \BA{Introduction.}
    
    Consider first those quantities that are conserved by the transient system, so as to seek discretisations which better represent the physical behaviour of the system by \emph{also} conserved these quantities. 
    
    \cite{LHF22} considers conservation of the following 3 quantities, which the authors define in the incompressible case as: \BA{(Oops I've never defined $\bfA$! That should probably be in the introduction...)}
    \begin{center}\begin{tabular}{ c c c }
        Properties  &  Symbol  &  Definition  \\
        \hline\hline
        Energy  &  $\rmE$  &  $\int_{\bfOmega}\left[\frac{1}{\rmEu\rho}\|\bfp\|^{2} + p + \frac{1}{\beta}\|\bfB\|^{2}\right]$  \\
        Magnetic helicity  &  $\rmH_{\rmM}$  &  $\int_{\bfOmega}\bfA\cdot\bfB$  \\
        Hybrid helicity  &  $\rmH_{\rmH}$  &  $\int_{\bfOmega}(a\bfA + \bfp)\cdot(b\bfB + \nabla\wedge\bfp)$
    \end{tabular}\end{center}
    where $a$, $b$ satisfy the relation $a + b  =  \frac{4}{\beta\rmRH}$. \BA{(What do these represent \emph{physically}? Diagrams!)} Taking the derivatives of these quantities over time (still in the incompressible system) gives
    \begin{align}
        \frac{d\rmE}{dt}  &=  \BA{\cdots}  \\
        \frac{d\rmH_{\rmM}}{dt}  &=  \int_{\bfGamma}(- \varphi\bfB + \bfA\wedge\bfE)\cdot\bfn - \frac{2}{\rmRem}\int_{\bfOmega}\bfB\cdot\bfj  \\
        \frac{d\rmH_{\rmH}}{dt}  &=  \BA{\cdots} \\
    \end{align}

    \BA{Proven that in the \emph{compressible} case, $\frac{d\rmE}{dt}$ evaluates as
    {\small \begin{equation}
        \frac{d\rmE}{dt}  =  \int_{\bfGamma}\left[- \frac{1}{2\rmEu\rho}\|\bfp\|^{2}\bfp - \frac{p}{2\rho}\bfp + \frac{1}{\rmEu\rmRe_{f}}\nabla\left[\frac{1}{\rho}\bfp\right]\cdot\frac{1}{\rho}\bfp - \frac{p}{2\rho}\bfp + \frac{1}{2\rmPe}\nabla\left[\frac{p}{\rho} + \frac{1}{\beta}\bfB\wedge\bfE\right]\right]\cdot\bfn
    \end{equation}}}
    
    \section{Preserved Structures}
    \BA{Introduction.}
    
    Consider first those quantities that are conserved by the transient system, so as to seek discretisations which better represent the physical behaviour of the system by \emph{also} conserved these quantities. 
    
    \cite{LHF22} considers conservation of the following 3 quantities, which the authors define in the incompressible case as: \BA{(Oops I've never defined $\bfA$! That should probably be in the introduction...)}
    \begin{center}\begin{tabular}{ c c c }
        Properties  &  Symbol  &  Definition  \\
        \hline\hline
        Energy  &  $\rmE$  &  $\int_{\bfOmega}\left[\frac{1}{\rmEu\rho}\|\bfp\|^{2} + p + \frac{1}{\beta}\|\bfB\|^{2}\right]$  \\
        Magnetic helicity  &  $\rmH_{\rmM}$  &  $\int_{\bfOmega}\bfA\cdot\bfB$  \\
        Hybrid helicity  &  $\rmH_{\rmH}$  &  $\int_{\bfOmega}(a\bfA + \bfp)\cdot(b\bfB + \nabla\wedge\bfp)$
    \end{tabular}\end{center}
    where $a$, $b$ satisfy the relation $a + b  =  \frac{4}{\beta\rmRH}$. \BA{(What do these represent \emph{physically}? Diagrams!)} Taking the derivatives of these quantities over time (still in the incompressible system) gives
    \begin{align}
        \frac{d\rmE}{dt}  &=  \BA{\cdots}  \\
        \frac{d\rmH_{\rmM}}{dt}  &=  \int_{\bfGamma}(- \varphi\bfB + \bfA\wedge\bfE)\cdot\bfn - \frac{2}{\rmRem}\int_{\bfOmega}\bfB\cdot\bfj  \\
        \frac{d\rmH_{\rmH}}{dt}  &=  \BA{\cdots} \\
    \end{align}

    \BA{Proven that in the \emph{compressible} case, $\frac{d\rmE}{dt}$ evaluates as
    {\small \begin{equation}
        \frac{d\rmE}{dt}  =  \int_{\bfGamma}\left[- \frac{1}{2\rmEu\rho}\|\bfp\|^{2}\bfp - \frac{p}{2\rho}\bfp + \frac{1}{\rmEu\rmRe_{f}}\nabla\left[\frac{1}{\rho}\bfp\right]\cdot\frac{1}{\rho}\bfp - \frac{p}{2\rho}\bfp + \frac{1}{2\rmPe}\nabla\left[\frac{p}{\rho} + \frac{1}{\beta}\bfB\wedge\bfE\right]\right]\cdot\bfn
    \end{equation}}}
    
    