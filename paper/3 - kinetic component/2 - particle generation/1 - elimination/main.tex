\subsection*{Elimination}
    \begin{definition}[Bhatnagar--Gross--Krook (BGK) collision operator]
        In \cite{Bhatnagar_Gross_Krook_1954} Bhatnagar, Gross and Krook propose a linearized approximation to the collision operator:
        \begin{align}
            \nabla_{\bfv}\cdot\bfC_{\pm\pm}^{\rm BGK}  =  \eta_{\pm}\left(f_{\pm}^{(0)} - f_{\pm}\right),  &&
            \nabla_{\bfv}\cdot\bfC_{\pm\mp}^{\rm BGK}  =  0,
        \end{align}
        for decay parameters, $\eta_{\pm}$.
    \end{definition}
    This can naturally be written within the $\delta\!f$ decomposition framework as
    \begin{equation}
        \nabla_{\bfv}\cdot\bfC_{\pm\pm}^{\rm BGK}  =  - \eta_{\pm}\delta\!f_{\pm}.
    \end{equation}

    \shortline

    Suppose a small BGK component is introduced into the collision operator approximation, with
    \begin{equation}
        \bfC_{\pm_{1}\pm_{2}}^{(0)}  \mapsto  \bfC_{\pm_{1}\pm_{2}}^{(0)} + \frac{1}{|\rmCy_{\pm}|}\bfC_{\pm_{1}\pm_{2}}^{\rm BGK},
    \end{equation}
    e.g. in the case of the LB collision operator approximation, one would have $\bfC_{\pm_{1}\pm_{2}}^{(0)} = \bfC_{\pm_{1}\pm_{2}}^{\rm LB} + 1/|\rmCy_{\pm}|\cdot\bfC_{\pm_{1}\pm_{2}}^{\rm BGK}$. The linearized Boltzmann equation (\ref{eqn:linearized Boltzmann equation simplified 1}) then states
    \begin{multline}
        \partial_{t}\delta\!f_{\pm}
        + \nabla_{\bfx}\cdot[\delta\!f_{\pm}\bfv]
        + \rmCy\!_{\pm}\nabla_{\bfv}\cdot[\delta\!f_{\pm}(\bfE + \bfv\wedge\bfB)]  \\
        - \nabla_{\bfv}\cdot\left[\rmKn_{\pm\pm}\partial\bfC_{\pm\pm}^{(0)} + \rmKn_{\pm\mp}\partial\bfC_{\pm\mp}^{(0)}\right]
        + \eta_{\pm}\delta\!f_{\pm}
        =  \calF_{\pm}
    \end{multline}
    When modeling the homogeneous LHS via the PiC simulation, this has the effect of introducing a decay/half-life effect to the pseudo-particles, whereby after a (potentially) random duration $D_{\pm}^{(i)}$ the pseudo-particle is removed from the simulation, with
    \begin{equation}
        \bbE\left[D_{\pm}^{(i)}\right] = \frac{1}{\eta_{\pm}}.
    \end{equation}
    After discretization of the system into a timestepper on intervals $T^{n} = \left(t^{n}, t^{n + 1}\right]$, it is computationally convenient if $D_{\pm}^{(i)}$ is the sum of interval durations, $\delta t^{n}$, such that the pseudo-particle ceases to require modeling at some $t^{n}$, at the end of a timestep. This can be achieved by removing each pseudo-particle from the simulation after the timestep $T^{n}$ with probability $1 - \exp\left[\eta_{\pm}\delta t^{n}\right]$, likely independent for each particle.
    