\section{Numerics}
    \BA{Introduction.}

    \BA{Would be a great opportunity to work in the ideas of gyrokinetic theory, especially the more mathematical aspects such as the Lie transformations overviewed in Lapillone's thesis (I am sure there's a more original source for these ideas of course).}
    
    \BA{A gyro-averaged model would shine here! Plus lots of opportunity for interesting (and actually accurate, in contrast to most physics gyrokinetic model papers) maths and multiscale analysis, especially with the addition of the stochastic Wiener process term. (Not sure I have ever seen a multiscale analysis with a stochastic term in before, I should ask the perturbation methods lecturer!)}

    \BA{Typing out my thoughts here in a emph{semi}-formal manner. Will re-phrase once I know what's in the previous subsections etc....}
    
    Recall the system of SDEs satisfied by the position, $\bfX$, and velocity, $\bfV$, of the charged particle as defined in \BA{("earlier subsection")}: \BA{(The following is after non-dimensionalisation of course...)}
    \begin{align}
        d\bfX  &=  \bfV dt  \\
        d\bfV  &=  \frac{1}{\varepsilon^{2}}([\bfE + \bfV\wedge\bfB] + \mu[(\bfu + \nabla_{\bfx}\cdot\bftau) - \bfV])dt + \frac{\sigma}{\varepsilon}d\bfW
    \end{align}
    $\bfV$ here varies on the timescale of $\calO(\varepsilon)$, at the cyclotron frequency, such that if we were to attempt to directly simulate the motion of such a charged particle, we would require a timestepping scheme with timestep frequency on the order of the cyclotron frequency, which would be potentially computationally prohibitive as $\varepsilon \rightarrow 0$ in the highly-magnetized limit. \BA{(Compare with the value of $\varepsilon$ in a typical tokamak and make some computational estimates.)}
    
    We can combat this using a multiscale analysis, with the transport on the $\calO(1)$ timescale, and the gyrations on the $\calO(\varepsilon)$ timescale. This is the assumption that is typically done in deriving the gyrokinetic model and resulting gyrocenter drifts. Defining $T = \frac{1}{\varepsilon}t$, write $\bfX$, $\bfV$ as each respectively the sum of a gyro-averaged value dependent only on the transport timescale $\overline{\bfX}(t; \varepsilon)$, $\overline{\bfV}(t; \varepsilon)$, and a $\calO(1)$ perturbation $\delta\bfX(t, T; \varepsilon)$, $\delta\bfV(t, T; \varepsilon)$ oscillating on the cyclotron timescale with amplitudes dependent on $t$:
    \begin{align}
          \label{eqn:multiscale position SDE}  \left[\overline{\bfX}_{t} + \left(\delta\bfX_{t} + \frac{1}{\varepsilon}\delta\bfX_{T}\right)\right]dt  &= 
 \left[\overline{\bfV} + \delta\bfV\right]dt  \\
        \begin{split}
            \left[\overline{\bfV}_{t} + \left(\delta\bfV_{t} + \frac{1}{\varepsilon}\delta\bfV_{T}\right)\right]dt  &=  \frac{1}{\varepsilon^{2}}([\bfE + \left[\overline{\bfV} + \delta\bfV\right]\wedge\bfB] + \mu[(\bfu + \nabla_{\bfx}\cdot\bftau)  \\
            &  \;\;\;\;\;\;\;\;\;\;\;\;\;\;\;\;\;\;\;\;\;\;\;\;\;\;\;\;\;\;\;\;\;\;\;\;\;\;\;\;\;\;\;\;\;\;\;\;  - \left[\overline{\bfV} + \delta\bfV\right]])dt + \frac{\sigma}{\varepsilon}d\bfW
        \end{split}
    \end{align}
    \BA{[Ref]} This is modified from the classical gyrokinetic theory by the inclusion of the stochastic collisional terms, giving a multiscale \emph{stochastic} system. Compare with the systems as defined in \BA{[pretty much any paper on multiscale methods for SDEs]}. We similarly expand the Wiener process $\bfW$ as the sum of a gyro-averaged $\overline{\bfW}$ and perturbation $\delta\bfW$. \BA{(Do this by splitting the Fourier series at frequencies $\calO\left(\frac{1}{\sqrt{\varepsilon}}\right)$!}
    \begin{multline}\label{eqn:multiscale velocity SDE}
        \left[\overline{\bfV}_{t} + \left(\delta\bfV_{t} + \frac{1}{\varepsilon}\delta\bfV_{T}\right)\right]dt  =  \frac{1}{\varepsilon^{2}}([\bfE + \left[\overline{\bfV} + \delta\bfV\right]\wedge\bfB] + \mu[(\bfu + \nabla_{\bfx}\cdot\bftau) -  \\
        \left[\overline{\bfV} + \delta\bfV\right]])dt + \frac{\sigma}{\varepsilon}\left[d\overline{\bfW} + d\delta\bfW\right]
    \end{multline}
    Writing $\bfX$, $\bfV$---and with it $\overline{\bfX}$, $\overline{\bfV}$ and $\delta\bfX$, $\delta\bfV$---as power series in powers of $\varepsilon$ \footnote{Note, in the classical case \emph{without} the collisional terms, these need only be powers in $\varepsilon^{2}$.} as
    \begin{align}
        *(\cdots; \varepsilon)  =  \sum_{i}\varepsilon^{i}*^{(i)}(\cdots)  \left(=  *^{(0)}(\cdots) + \varepsilon*^{(1)}(\cdots) + \varepsilon^{2}*^{(2)}(\cdots) + \cdots\right)
    \end{align}
    we solve each of (\ref{eqn:multiscale position SDE}) and (\ref{eqn:multiscale velocity SDE}) at each order in $\varepsilon$: \BA{(A lot of technicalities coming up, maybe good to put in an appendix?)}
    \begin{itemize}
        \item[(\ref{eqn:multiscale position SDE}).]  At order $\calO\left(\varepsilon^{(i)}\right)$
        \begin{equation}
            \overline{\bfX}^{(i)}_{t} + \left(\delta\bfX^{(i)}_{t} + \delta\bfX^{(i - 1)}_{T}\right)  =  \overline{\bfV}^{(i)} + \delta\bfV^{(i)}
        \end{equation}
        Separating into gyro-averaged and fluctuating components:
        \begin{align}
            \overline{\bfX}^{(i)}_{t}  &=  \overline{\bfV}^{(i)}  \\
            \delta\bfX^{(i)}_{t} + \delta\bfX^{(i - 1)}_{T}  &=  \delta\bfV^{(i)}
        \end{align}
        Resursively from $i = 0$, this gives:
        \begin{align}
            \delta\bfX^{(0)}  &=  \bfzero  \\
            \delta\bfV^{(i)}  &=  \delta\bfX^{(i)}_{t} + \delta\bfX^{(i - 1)}_{T}
        \end{align}
        
        \item[(\ref{eqn:multiscale velocity SDE}).]  ...
    \end{itemize}
    
    \BA{In hindsight, I ought to go directly by just figuring out the evolution equations for the energy/gyrocenter/magnetic moment. I can do this for all kinds of random motion too (Lévy flight etc.).}
    