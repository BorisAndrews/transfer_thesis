\paragraph*{Direct Discretization}
    Direct discretizations define timesteppers in the canonical variables of $\bfX_{\pm}^{(i)}$, $\bfV_{\pm}^{(i)}$. For clarity here we will drop the sub- and super-script $*_{\pm}^{(i)}$.

    \shortline

    In particular, we highlight works by Chacón, Chen et al. \cite{Chen_Chacón_Barnes_2011, Chacón_Chen_Barnes_2013, Chen_Chacón_2014, Chen_Chacón_2015} which
     have developed timesteppers directly on $\bfX_{\pm}^{(i)}$, $\bfV_{\pm}^{(i)}$ with SP properties for certain pseudo-particle structures highlighted in the gyrokinetic analysis. These particle pushers derive from Crank--Nicolson timestepping schemes. Using notation from the timesteppers in Chapter \ref{cha:fluid component} over the timestep $T^{n} = \left(t^{n}, t^{n + 1}\right]$ of duration $\delta t^{n}$:
    \begin{align}
        \bfdelta\bfX^{n}
            =  \bfV^{n + \frac{1}{2}},  &&
        \bfdelta\bfV^{n}
            =  \pm |\rmCy\!_{\pm}|\left(\bfE^{n + \frac{1}{2}} + \bfV^{n + \frac{1}{2}}\wedge\bfB^{n + \frac{1}{2}}\right),
    \end{align}
    where:
    \begin{align}
        \bfX^{n + \frac{1}{2}}  :=  \bfX|_{t^{n}} + \frac{1}{2}\bfdelta\bfX^{n},  &&
        \bfV^{n + \frac{1}{2}}  :=  \bfV|_{t^{n}} + \frac{1}{2}\bfdelta\bfV^{n},  &&
        t^{n + \frac{1}{2}}  :=  t^{n} + \frac{1}{2}\delta t^{n},
    \end{align}
    and:
    \begin{align}
        \bfE^{n + \frac{1}{2}}  :=  \bfE\left(\bfX^{n + \frac{1}{2}}; t^{n + \frac{1}{2}}\right),  &&
        \bfB^{n + \frac{1}{2}}  :=  \bfB\left(\bfX^{n + \frac{1}{2}}; t^{n + \frac{1}{2}}\right).
    \end{align}
    Updates at $t^{n + 1}$ are given as:
    \begin{align}
        \bfX|_{t^{n + 1}}  =  \bfX|_{t^{n}} + \delta t^{n}\bfdelta\bfX^{n},  &&
        \bfV|_{t^{n + 1}}  =  \bfV|_{t^{n}} + \delta t^{n}\bfdelta\bfV^{n}.
    \end{align}
    
    \shortline

    This Crank--Nicolson timestepper has SP properties. In particular:
    \begin{itemize}
        \item  {\bf Energy dissipation:}
        \begin{equation}
            \frac{1}{\delta t^{n}}\left(E|_{t^{n + 1}} - E|_{t^{n}}\right)
                =  \pm |\rmCy_{\pm}|\bfE^{n + \frac{1}{2}}\cdot\bfV^{n + \frac{1}{2}}
        \end{equation}
        \item  {\bf Gyrocenter motion:} Presuming sufficient regularity on $\bfB$,
        \begin{multline}
            \frac{1}{\delta t^{n}}\left(\bfxi|_{t^{n + 1}} - \bfxi|_{t^{n}}\right)
                =  V_{\parallel}^{n + \frac{1}{2}}\bfb_{\parallel}^{n + \frac{1}{2}}
                + \frac{1}{\left(B^{n + \frac{1}{2}}\right)\!^{2}}\bfE^{n + \frac{1}{2}}\wedge\bfB^{n + \frac{1}{2}}  \\
                \pm \frac{1}{|\rmCy_{\pm}|}\bfV^{n + \frac{1}{2}}\wedge\left(\frac{1}{\delta t^{n}}\left(\frac{1}{\left(B^{n + 1}\right)^{2}}\bfB^{n + 1} - \frac{1}{\left(B^{n}\right)^{2}}\bfB^{n}\right)\right) + \calO\left[\left(\delta t^{n}\right)^{2}\right],
        \end{multline}
        for $V_{\parallel}$, $\bfb_{\parallel}$ defined similarly to as was done in the gyrokinetic model. Thus, both the motion of the gyrocenter along the magnetic field lines, and the dominant $\bfE\wedge\bfB$ drift are captured, up to a notable $\calO\left[\left(\delta t^{n}\right)^{2}\right]$ error. The ability, or extent of ability, of this timestepper to capture the gyrocenter motion is referred to as \emph{``asymptotic preservation''} (AP).\footnote{``Asymptotic preservation'' is defined more rigorously in \cite{Ricketson_Chacón_2020}.}
    \end{itemize}

    \shortline

    \begin{remark}[Direct particle pushers from FET]
        Recall from Section \ref{cha:SP timesteppers} that the Crank--Nicolson is the lowest order GL timestepper, and from Sections \ref{cha:SP timesteppers} and \ref{cha:Navier--Stokes} that GL(-like) methods can be derived from CG-DG-in-time FET discretization, giving motivation for the strong SP properties. It strikes me that these Crank--Nicolson timesteppers could be expressed in the language of FET too, just as was done for the SP discretizations of the fluid component.

        Using the same notation as in Chapter \ref{cha:fluid component}, I have had some luck using weak formulations in time wherein one seeks $\bfX_{-} \in \calX_{-}$, $\bfV_{-} \in \calV_{-}$ such that:
        \begin{align}
            \left(\frac{\rmd}{\rmd t}\bfX
                =\right)  \bbQ_{\rmd\calX/\rmd t}\!\left[\frac{\rmd}{\rmd t}\bfX\right]
                &=  \bbQ_{\rmd\calX/\rmd t}\!\left[\left\|\widetilde{\bfb}_{\parallel}\right\|^{2}\bbQ_{\rmd\calV/\rmd t}[\bfV]\right]  \\
            \left(\frac{\rmd}{\rmd t}\bfV
                =\right)  \bbQ_{\rmd\calV/\rmd t}\!\left[\frac{\rmd}{\rmd t}\bfV\right]
                &=  \pm |\rmCy_{\pm}|\bbQ_{\rmd\calV/\rmd t}\!\left[\bfE + \bbQ_{\rmd\calV/\rmd t}[\bfV]\wedge B\widetilde{\bfb}_{\parallel}\right]
        \end{align}
        where
        \begin{equation}
            \widetilde{\bfb}_{\parallel}  =  B\bbQ_{\rmd\calV/\rmd t}\!\left[\frac{1}{B^{2}}\bfB\right].
        \end{equation}
        This exhibits great \emph{exact} SP(/AP) properties, with:
        \begin{align}
            \frac{1}{\delta t^{n}}\left(E|_{t^{n + 1}} - E|_{t^{n}}\right)
                &=  \pm |\rmCy_{\pm}|\frac{1}{\delta t^{n}}\int_{T^{n}}\bfE\cdot\bbQ_{\rmd\calV/\rmd t}[\bfV]  \\
            \frac{1}{\delta t^{n}}\left(\bfxi|_{t^{n + 1}} - \bfxi|_{t^{n}}\right)
                &=  \frac{1}{\delta t^{n}}\int_{T^{n}}\left[\widetilde{V}_{\parallel}\widetilde{\bfb}_{\parallel}
                + \frac{1}{B^{2}}\bfE\wedge B\widetilde{\bfb}_{\parallel}
                \pm \frac{1}{|\rmCy_{\pm}|}\bfV\wedge\frac{\rmd}{\rmd t}\!\left[\frac{1}{B^{2}}\bfB\right]\right]
        \end{align}
        for $\widetilde{V}_{\parallel}  :=  \widetilde{\bfb}_{\parallel}\cdot\bbQ_{\rmd\calV/\rmd t}[\bfV]$, the latter result holding provided $\{{T^{n}} \rightarrow 1\}\otimes\bbR^{3} \subseteq \rmd\calX/\rmd t$, i.e. $\rmd\calX/\rmd t$ contains as elements the constant-in-time basis functions for $\bbR^{3}$.

        Whether this weak formulation is computationally tractable is a different question. The corresponding variational form requires the exact evaluation of integrals over $T^{n}$. Using the trapezium rule to approximate these integrals returns us to the classical Crank--Nicolson timesteppers of Chacón, Chen et al., introducing the $\calO\left[\left(\delta t^{n}\right)^{2}\right]$ errors in the gyrocenter motion result. This is all very much work in progress.

        It would be great if these ideas of SP(/AP) particle pushers could \emph{also} derive from FET, as this would offer some hope that, in the future, FET could be used directly on the full hybrid model to derive hybrid FEM-PiC timesteppers with SP properties.
    \end{remark}

    \begin{remark}[Incorporating gyration-induced drifts within the FET framework]
        Certain lower-order gyrocenter drifts, such as $\nabla B$ drifts, are induced by varying magnetic fields throughout the particle(/pseudo-particle)'s gyration. When classical Crank--Nicolson-based timesteppers use timesteps on the kinetic/transport timescales, these gyrations are no longer present in the particle path. These gyration-induced drifts therefore are not preserved after discretization, and this structure is lost.
        
        In \cite{Ricketson_Chacón_2020}, Ricketson and Chacón build upon the Crank--Nicolson framework, by modifying the definition of $\bfV^{n + \frac{1}{2}}$ in such a way as to re-introduce the $\nabla B$ drift structure. (See also \cite{Koshkarov_et_al_2022})

        I have been thinking about whether a \emph{multifunction} interpretation for $\bfX$ and $\bfV$ would be able to have a similar effect, where at each time $t$, $\bfX$ and $\bfV$ are interpreted as multifunctions over the circular orbit that would be followed by a particle with velocity $\bfV$ at position $\bfX$. Again, this is left for further work.
    \end{remark}
