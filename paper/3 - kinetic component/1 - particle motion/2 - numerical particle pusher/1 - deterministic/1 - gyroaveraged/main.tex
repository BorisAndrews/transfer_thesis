\paragraph*{Gyroaveraged Discretization}
    Writing (\ref{eqn:deterministic neoclassical transport 1}--\ref{eqn:deterministic neoclassical transport 2}) as a system evolving in parameters that are invariant (up to leading order) over cyclotron-frequency oscillations is referred to as a gyrokinetic approach. This is equivalent in effect to performing a multiscale analysis \cite{Kevorkin_Cole_2012} and as such captures the transport-timescale dynamics induced by the high-frequency oscillations directly.
    
    We begin first with an overview of classical gyrokinetic theory. For a further exposition and discussion, see \cite{Woods_2006, Freidberg_2008, Chen_2015}.

    \shortline

    For magnetic field strength $B  :=  \|\bfB\|$, denote the total, $V_{\pm}^{(i)}$, parallel, $V_{\pm\parallel}^{(i)}$, and perpendicular, $V_{\pm\perp}^{(i)}$, particle speeds:
    \begin{align}
                 V_{\pm}^{(i)}  :=  \left\|\bfV_{\pm}^{(i)}\right\|,   &&
        V_{\pm\parallel}^{(i)}  :=  \frac{1}{B}\bfB\cdot\bfV_{\pm}^{(i)},  &&
            V_{\pm\perp}^{(i)}  :=  \frac{1}{B}\left\|\bfB\wedge\bfV_{\pm}^{(i)}\right\|.
    \end{align}
    Define the parallel, $\bfb_{\parallel}$, and perpendicular, $\bfb_{\pm\perp}^{(i)}$, magnetic field directions:
    \begin{align}
        \bfb_{\parallel}  :=  \frac{1}{B}\bfB,  &&
            \bfb_{\pm\perp}^{(i)}  :=  - \frac{1}{B^{2}}\left(\frac{1}{V_{\pm\perp}^{(i)}}\bfV_{\pm}^{(i)}\wedge\bfB\right)\wedge\bfB,
    \end{align}
    such that the velocity, $\bfV_{\pm}^{(i)}$, can be decomposed as $\bfV_{\pm}^{(i)}  =  V_{\pm\parallel}^{(i)}\bfb_{\parallel} + V_{\pm\perp}^{(i)}\bfb_{\pm\perp}^{(i)}$. This decomposition splits $\bfV_{\pm}^{(i)}$ into 3 terms that change largely on convective timescales, $V_{\pm\parallel}^{(i)}$, $V_{\pm\perp}^{(i)}$, $\bfb_{\parallel}$, and 1 that oscillates largely on cyclotron timescales, $\bfb_{\pm\perp}^{(i)}$, with $\bfb_{\parallel}\cdot\bfb_{\pm\perp}^{(i)}  =  0$. $\bfb_{\parallel}$, $\bfb_{\pm\perp}^{(i)}$ are completed as an orthonormal basis for $\bbR^{3}$ by
    \begin{equation}
        \bfb_{\pm*}^{(i)}  :=  \bfb_{\parallel}\wedge\bfb_{\pm\perp}^{(i)}.
    \end{equation}

    \begin{definition}[Pseudo-particle gyrocenter, magnetic moment and energy]
        These are defined as follows:
        \begin{itemize}
            \item  Gyrocenter, $\bfxi_{\pm}^{(i)}$:
            \begin{equation}
                \bfxi_{\pm}^{(i)}\!\left[\bfX_{\pm}^{(i)}, \bfV_{\pm}^{(i)}; \bfB\right]\!(t)  :=  \bfX_{\pm}^{(i)} \mp \frac{1}{|\rmCy\!_{\pm}|B}\bfb_{\pm*}^{(i)}
            \end{equation}
            \item  Magnetic moment, $\mu_{\pm}^{(i)}$:
            \begin{equation}
                \mu_{\pm}^{(i)}\!\left[\bfV_{\pm}^{(i)}; \bfB\right]\!(t)  :=  \frac{1}{2}\cdot\frac{1}{B}\left(V_{\pm\perp}^{(i)}\right)^{2}
            \end{equation}
            \item  Energy, $E_{\pm}^{(i)}$:
            \begin{equation}
                E\!\left[\bfV_{\pm}^{(i)}\right]\!(t)  :=  \frac{1}{2}\left\|\bfV_{\pm}^{(i)}\right\|^{2}.
            \end{equation}
        \end{itemize}
    \end{definition}

    $\bfxi_{\pm}^{(i)}$, $\mu_{\pm}^{(i)}$, $E_{\pm}^{(i)}$ evolve according to the equations:
    \begin{align}
        \rmd\bfxi_{\pm}^{(i)} 
            &=  \left(V_{\pm\parallel}^{(i)}\bfb_{\parallel} + \frac{1}{B^{2}}\bfE\wedge\bfB\right)\rmd t
            \pm \frac{1}{|\rmCy\!_{\pm}|}\cdot\frac{1}{B^{2}}\left(\bfV_{\pm}^{(i)}\wedge\rmd\bfB - 2V_{\pm\perp}^{(i)}\left(\bfb_{\parallel}\cdot\rmd\bfB\right)\bfb_{\pm*}^{(i)}\right)  \label{eqn:gyrocenter equation}  \\
        \rmd\mu_{\pm}^{(i)}
            &=  \pm |\rmCy\!_{\pm}|\frac{1}{B}V_{\pm\perp}^{(i)}\left(\bfE\cdot\bfb_{\pm\perp}^{(i)}\right)\rmd t
            - \frac{1}{B^{2}}\left(\frac{1}{2}\left(V_{\pm\perp}^{(i)}\right)^{2}\bfb_{\parallel} + V_{\pm\parallel}^{(i)}V_{\pm\perp}^{(i)}\bfb_{\pm\perp}^{(i)}\right)\cdot\rmd\bfB  \label{eqn:magnetic moment equation}  \\
        \rmd E_{\pm}^{(i)}
            &=  \pm|\rmCy\!_{\pm}|\left(\bfE\cdot\bfV_{\pm}^{(i)}\right)\rmd t
    \end{align}
    with
    \begin{equation}
        \rmd\bfB  =  \left(\bfV_{\pm}^{(i)}\cdot\nabla\bfB - \nabla\wedge\bfE\right)\rmd t
    \end{equation}

    \shortline

    \emph{``Gyroaveraging''} can be interpreted as averaging these equations over the high-frequency oscillations, over a duration of $2\pi/|\rmCy\!_{\pm}|B$. This eliminates any cyclotron frequency oscillations in the parameters, such as those in $\bfb_{\pm\perp}^{(i)}$.
    
    After gyroaveraging on (\ref{eqn:gyrocenter equation}--\ref{eqn:magnetic moment equation}) $\bfxi_{\pm}^{(i)}$, $\mu_{\pm}^{(i)}$ evolve according to the equations:
    \begin{align}
        \begin{split}
            \rmd\bfxi_{\pm}^{(i)} 
                &=  \left(V_{\pm\parallel}^{(i)}\bfb_{\parallel} + \frac{1}{B^{2}}\bfE\wedge\bfB\right)\rmd t  \\
                &\qquad\pm \frac{1}{|\rmCy\!_{\pm}|}\cdot\frac{1}{B^{2}}\left(\left(2E_{\pm}^{(i)} - B\mu_{\pm}^{(i)}\right)\bfb_{\parallel}\wedge\nabla B\tall\right.  \\
                &\qquad\qquad\left.+ \left(\left(2E_{\pm}^{(i)} - 2B\mu_{\pm}^{(i)}\right)\bfI - \left(2E_{\pm}^{(i)} - 3B\mu_{\pm}^{(i)}\right)\bfb_{\parallel}^{\otimes 2}\right)\cdot\bfj\tall\right)\rmd t + \calO\left[\frac{\rmd t}{(|\rmCy\!_{\pm}|B)^{2}}\right]
        \end{split}  \label{eqn:gyroaveraged gyrocenter equation}  \\
        \rmd\mu_{\pm}^{(i)}
            &=  - \frac{1}{B^{2}}\mu_{\pm}^{(i)}\{\bfE, \bfb_{\parallel}, \nabla B\}\rmd t + \calO\left[\frac{\rmd t}{|\rmCy\!_{\pm}|B}\right]  \\
        \rmd E_{\pm}^{(i)}
            &=  \pm|\rmCy\!_{\pm}|V_{\pm\parallel}^{(i)}(\bfE\cdot\bfb_{\parallel})\rmd t
            - \mu_{\pm}^{(i)}(\nabla\wedge\bfE)\cdot\bfb_{\parallel} + \calO\left[\frac{\rmd t}{|\rmCy\!_{\pm}|B}\right].  \label{eqn:gyroaveraged energy equation}
    \end{align}
    where the EM fields $\bfE$, $\bfB$, gradients thereof, and the current $\bfj$ are evaluated at the gyrocenter, $\bfxi_{\pm}^{(i)}$. Terms following $V_{\pm\parallel}^{(i)}\bfb_{\parallel}$ in the gyrocenter evolution equation (\ref{eqn:gyroaveraged gyrocenter equation}) are referred to as \emph{``gyrocenter drifts''}. \cite{Woods_2006, Freidberg_2008, Chen_2015}

    This can be made into a closed system in $\bfxi_{\pm}^{(i)}$, $\mu_{\pm}^{(i)}$, $E_{\pm}^{(i)}$ by defining $V_{\pm\parallel}^{(i)}$ as
    \begin{equation}
        V_{\pm\parallel}^{(i)}  =  \pm \sqrt{2\left(E_{\pm}^{(i)} - B\mu_{\pm}^{(i)}\right)},
    \end{equation}
    with the sign of the square root switching whenever $E_{\pm}^{(i)}  =  B\mu_{\pm}^{(i)}$ as gyrocenters oscillate along the magnetic field lines. \cite{Freidberg_2008} This eliminates the multiscale behavior of pseudo-particles in the Vlasov model (\ref{eqn:deterministic neoclassical transport 1}--\ref{eqn:deterministic neoclassical transport 2}). One can then model the pseudo-particle motion through a timestepper in $\bfxi_{\pm}^{(i)}$, $\mu_{\pm}^{(i)}$, $E_{\pm}^{(i)}$.

    \begin{remark}[Verification of the deterministic gyrokinetic model]
        I \emph{believe} (\ref{eqn:gyroaveraged gyrocenter equation}--\ref{eqn:gyroaveraged energy equation}) are correct. It's surprisingly hard to find these precise forms in the literature, only certain leading-order results under certain assumptions, such as negligible $\bfj$ or $\bfE$. The best way to confirm them I suspect would just be to run some cyclotron-frequency simulations and check the results line up.
    \end{remark}

    \shortline

    Gyrokinetic expansions and particle pushers can exhibit greater computational efficiency than direct approaches for multiple reasons:
    \begin{itemize}
        \item  The multiscale expansion improves the conditioning of the numerical problem to be solved. With large $|\rmCy_{\pm}|$, the ill conditioning exists (primarily) in the high-frequency oscillations, which need not be modeled in a gyrokinetic discretization.
        \item  Direct discretizations must often be implicit to give SP properties. This is naturally not the case for gyrokinetic particle pushers. 
        \item  They reduce the number of variables to model from $6$ to $5$.
    \end{itemize}

    A major drawback however comes in knowing only the pseudo-particle path's gyrocenter at any given time, and not the pseudo-particle itself's exact location. This presents two main problems:
    \begin{itemize}
        \item  It is unclear how to handle boundary collisions, since it is unclear when a particle collides with the boundary.
        \item  In the fluid component of the hybrid model, one must ``test'' against the $\delta$ function representing for the pseudo-particle positions, when casting into a weak form. What this means when the particle location is unknown is up to interpretation.
    \end{itemize}
    There is also the drawback that the gyrokinetic expansion holds not just as $|\rmCy\!_{\pm}| \rightarrow \infty$, but as $|\rmCy\!_{\pm}|B \rightarrow \infty$. Even if it is true that $\rmCy\!_{\pm} \gg 1$, and $|\rmCy\!_{\pm}|B \gg 1$ in most locations, certain reactor layouts may feature regions of low magnetic field. If there exists a region within the tokamak where $B \not\gg 1/|\rmCy\!_{\pm}|$ then this assumption breaks down, as the pseudo-particle dynamics lose their multiscale nature, causing the gyrokinetic expansion to fail.
