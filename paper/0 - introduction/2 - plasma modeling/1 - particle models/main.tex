\subsection{Particle Models}
    Perhaps the most fundamental mathematical model for a plasma is the particle model.

    \line

    \begin{definition}[Particle model]\label{def:particle model}
        Here, ``particle'' models refer to those wherein every single particle is individually modeled.
    \end{definition}
    
    \line

    For a particle, indexed via the index $*_{i}$ in a phase indexed via the index $*_{s}$, denote the position $\bfx_{si}(t)$ and velocity $\bfv_{si}(t)$. Assuming the dominant forces acting on these particles are electromagnetic (EM) forces from either the background or other particles, $(\bfx_{si})_{si}$ and $(\bfv_{si})_{si}$ satisfy the systems of ODEs:
    \begin{align}
        \bfd\bfx_{si}  &=  \bfv_{si} dt  \label{eqn:particle motion}  \\
        \rmm_{s}\bfd\bfv_{si}  &=  \rmq_{s}[(\bfE - \bfE_{si}) + \bfv_{si}\wedge(\bfB - \bfB_{si})]dt  \label{eqn:particle forcing}
    \end{align}
    where $\bfE_{si}$, $\bfB_{si}$ refer respectively to the contributions to the total electric, $\bfE$, and magnetic, $\bfB$, fields generated by the particle indexed $*_{si}$, and $\rmq_{s}$, $\rmm_{s}$ respectively denote the particle charge and mass within the phase indexed $*_{s}$.

    \BA{Should maybe take note of background forces, i.e. gravity.}
    
    Naturally, this gives the most complete dynamical model for the plasma. On the tokamak scale however, with particle densities on the order of $10^{19}\rmm^{- 3}$, modeling each particle individually is computationally intractable.
    