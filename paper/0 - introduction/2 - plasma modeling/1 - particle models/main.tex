\subsection*{Particle Models}
    Perhaps the most fundamental mathematical model for a plasma is the particle model.
    \begin{definition}[Particle model]
        Here, ``particle'' models refers to those wherein every single particle is modelled individually.
    \end{definition}
    Supposing particle indexed via the index $i$ in a phase indexed via the index $s$, has position $\bfx_{si}(t)$ and velocity $\bfv_{si}(t)$, $(\bfx_{si})_{si}$ and $(\bfv_{si})_{si}$ satisfy the systems of ODEs:
    \begin{align}
        \bfd\bfx_{si}  &=  \bfv_{si} dt  \label{patricle motion}  \\
        m_{s}\bfd\bfv_{si}  &=  q_{s}(\bfE + \bfv_{si}\wedge\bfB)dt  \label{particle forcing}
    \end{align}
    \BA{(Obviously this is coupled with Maxwell... Should probably specify this/make it clear that $\bfE$ and $\bfB$ refer to the (exact) electric and magnetic fields.)} Naturally this leads to the most complete dynamical model, however on the tokamak scale with particle densities on the order of $10^{- 5}{\rm mol}^{- 1}$ this is computationally impossible. We must therefore begin to make certain assumptions to simplify the model to bring it within reach of analysis.
    