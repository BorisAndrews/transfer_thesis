\subsection*{Particle Models}
    Perhaps the most fundamental mathematical model for a plasma is the particle model.
    \begin{definition}[Particle model]
        Here, ``particle'' models refer to those wherein every single particle is modelled individually.
    \end{definition}
    Supposing particle indexed via the index $i$ in a phase indexed via the index $s$, has position $\bfx_{si}(t)$ and velocity $\bfv_{si}(t)$, $(\bfx_{si})_{si}$ and $(\bfv_{si})_{si}$ satisfy the systems of ODEs:
    \begin{align}
        \bfd\bfx_{si}  &=  \bfv_{si} dt  \label{eqn:particle motion}  \\
        m_{s}\bfd\bfv_{si}  &=  q_{s}[(\bfE - \bfE_{si}) + \bfv_{si}\wedge(\bfB - \bfB_{si})]dt  \label{eqn:particle forcing}
    \end{align}
    where $\bfE_{si}$, $\bfB_{si}$ refer to the contributions to the electric and magnetic fields (respectively) \emph{not} generated by the particle indexed $s$, $i$, satisfying Maxwell's equations:
    \begin{align*}
        \partial_{t}\bfE_{si}  &=  c^{2}\nabla\wedge\bfB_{si} - \frac{1}{\varepsilon_{0}}\sum_{s', i' : (s', i') \neq (s, i)}q_{s'}\delta^{3}(\bfx - \bfx_{s'i'})\bfv_{si},  &
        \partial_{t}\bfB_{si}  &=  - \nabla\wedge\bfE_{si}  \\
        \nabla\cdot\bfE_{si}  &=  \frac{1}{\varepsilon_{0}}\sum_{s', i' : (s', i') \neq (s, i)}q_{s'}\delta^{3}(\bfx - \bfx_{s'i'}),  &
        \nabla\cdot\bfB_{si}  &=  0
    \end{align*}
    Naturally this leads to the most complete dynamical model, however on the tokamak scale with particle densities on the order of $10^{- 5}{\rm mol}^{- 1}$ this is computationally impossible. We must therefore begin to make certain assumptions to simplify the model to bring it within reach of analysis.
    