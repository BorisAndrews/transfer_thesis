\subsubsection*{Assumption 3: Quasineutrality}
    Combining the assumptions (\ref{eqn:local collision operator}) and (\ref{eqn:leading-order Boltzmann equation}), the dominant component of the Boltzmann equations take the form
    \begin{equation}\label{eqn:local leading-order Boltzmann equation}
        \nabla_{\bfv}\cdot\left[f_{s}(\bfE + \bfv\wedge\bfB) - \sum_{s'}q_{s'}\bfC_{ss'}^{(0)}\right]  \sim  0.
    \end{equation}
    This leading-order PDE is crucially 0th-order in $\bfx$ and $t$, and can be solved pointwise in $\bfx$ and $t$.
    
    For this to be well-posed, the following 3 identites, for conservation of mass, momentum, and energy, must hold:
    \begin{align}
        \forall s,  \int_{\bfv}\nabla_{\bfv}\cdot\left[f_{s}(\bfE + \bfv\wedge\bfB) - \sum_{s'}q_{s'}\bfC_{ss'}^{(0)}\right]q_{s}  &\sim  0  \\
        \sum_{s}\int_{\bfv}\nabla_{\bfv}\cdot\left[f_{s}(\bfE + \bfv\wedge\bfB) - \sum_{s'}q_{s'}\bfC_{ss'}^{(0)}\right]q_{s}\bfv  &\sim  \bfzero  \\
        \sum_{s}\int_{\bfv}\nabla_{\bfv}\cdot\left[f_{s}(\bfE + \bfv\wedge\bfB) - \sum_{s'}q_{s'}\bfC_{ss'}^{(0)}\right]\frac{1}{2}q_{s}\|\bfv\|^{2}  &\sim  0
    \end{align}
    By Lemma \ref{lem:conservation on local collision operators}, these identities hold true provided,
    \begin{align}
        \overline{\rho}_{\rmC}  \ll \overline{n}_{s}\max_{s}\{\overline{q}_{s}\},  &&
        \overline{\bfj}  \ll  \overline{n}_{s}\max_{s}\{\overline{q}_{s}\}\overline{\bfv}.
    \end{align}
    That is to say, the total internal charge and current are comparatively small when compared to that in each phase. This is referred to as the ``quasineutrality'' assumption.
    