\subsubsection*{Assumption 3: Exact Thermalization}
    One can further the thermalization result (\ref{eqn:thermalization}) by the assumption of \emph{exact} thermalization:
    \begin{equation}
      f_{\pm}  =  f_{\pm}^{(0)}
    \end{equation}
    One can then derive evolution equations for $\rho_{\rmM}$, $\rho_{\rmC}$, $\bfp$, $\rmE$ by taking the corresponding moments of the original Boltzmann equations (\ref{eqn:Boltzmann equation non-dimensionalized}). This derives the corresponding fluid model.

    \begin{remark}
        Naturally, this assumption will (almost surely) never hold true.
        
        There exist other interpretations by which a fluid model can be derived that do \emph{not} rely so much on the exact thermalization assumption $f_{s}  =  f_{s}^{(0)}$ \BA{[Ref, Ref, Ref, ...]} however the fluid models here will be derived from this assumption, as the derivation is more concise, and better illustrates the motivation behind the kinds of models that will be considered in Chapter \ref{cha:delta f corrections}. 
    \end{remark}

    Let $\bfu$ denote the mass-averaged flow velocity,
    \begin{equation}
        \bfu  :=  \frac{1}{\rho_{\rmM}}\bfp.
    \end{equation}
    Assuming further that:
    \begin{itemize}
        \item  Local collisions $\bfC_{\pm_{1}\pm_{2}}^{(0)}$ and the resultant asymptotic distributions $f_{\pm}^{(0)}$ are isotropic, such that the 2$^{\text{nd}}$ and 3$^{\text{rd}}$ moments can be written in the forms:
        \begin{align}
            \int_{\bfv}f_{s}^{(0)}\bfv^{\otimes 2}  &\sim  \rho_{\rmM}\bfu^{\otimes 2} + p\sfI  \label{eqn:isotropic 2nd moment}  \\
            \int_{\bfv}f_{s}^{(0)}\bfv^{\otimes 3}  &\sim  \rho_{\rmM}\bfu^{\otimes 3} + p(\bfu\otimes\sfI + \cdots + \sfI\otimes\bfu)  \label{eqn:isotropic 3rd moment}
        \end{align}
        for a pressure $p$. \BA{(Is this a safe assumption? I've seen bimaxwellian background distributions being used that might imply not so much, but surely the collisions are still isotropic, no?)}
        \item  Each $f_{s}  =  o\left[\|\bfv\|^{- 5}\right]$ uniformly in $\bfv$ as $\|\bfv\|  \rightarrow  \infty$, such that integration by parts can be applied.
    \end{itemize}
    the following \emph{fluid model} is derived:
    \begin{align}
        \partial_{t}\rho_{\rmM} + \nabla_{\bfx}\cdot\bfp  &=  0,  \label{eqn:mass conservation introduction}  \\
        \rmM^{2}\partial_{t}\rho_{\rmC} + \nabla_{\bfx}\cdot\bfj  &=  0,
    \end{align}
    \vspace{-25pt}
    \begin{multline}
        \partial_{t}\bfp + \left(\nabla_{\bfx}\cdot\left[\rho_{\rmM}\bfu^{\otimes 2}\right] + \nabla_{\bfx}p\right) - \frac{2}{\beta}\left(\rmM^{2}\rho_{\rmC}\bfE + \bfj\wedge\bfB\right)  \\
        =  - \int_{\bfv}\left[\frac{1}{{\rmRef}_{++}}\bfdelta\bfC_{++} + \cdots + \frac{1}{{\rmRef}_{--}}\bfdelta\bfC_{--}\right] + \calO\left[\frac{\rmm_{-}}{\rmm_{-}}\right]
    \end{multline}
    \vspace{-15pt}
    \begin{multline}
        \partial_{t}E + \nabla_{\bfx}\cdot\left[\frac{1}{2}\rho_{\rmM}\|\bfu\|^{2}\bfu + \frac{5}{2}p\bfu\right] - \frac{2}{\beta}\bfj\cdot\bfE  \\
        =  - \int_{\bfv}\left[\frac{1}{{\rmRef}_{++}}\bfdelta\bfC_{++} + \cdots + \frac{1}{{\rmRef}_{--}}\bfdelta\bfC_{--}\right]\cdot\bfv + \calO\left[\frac{\rmm_{-}}{\rmm_{-}}\right]
    \end{multline}
    The energy equation can be reframed in terms of the pressure, $p$, as
    \begin{multline}
        \frac{3}{2}\partial_{t}p + \left(\frac{3}{2}\nabla_{\bfx}\cdot[p\bfu] + p\nabla_{\bfx}\cdot\bfu\right) - \frac{2}{\beta}\left(\bfj - \rmM^{2}\rho_{\rmC}\bfu\right)\cdot(\bfE + \bfu\wedge\bfB)  \\
        =  - \int_{\bfv}\left[\frac{1}{{\rmRef}_{++}}\bfdelta\bfC_{++} + \cdots + \frac{1}{{\rmRef}_{--}}\bfdelta\bfC_{--}\right]\cdot(\bfv - \bfu) + \calO\left[\frac{\rmm_{-}}{\rmm_{-}}\right].  \label{eqn:energy conservation introduction}
    \end{multline}
    