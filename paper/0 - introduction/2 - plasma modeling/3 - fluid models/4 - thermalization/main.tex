\subsubsection*{Assumption 4: Full thermalization}
    By the conservation of:
    \begin{itemize}
        \item  Mass $\implies$ $S$ parameters for $S$ phases
        \item  Momentum $\implies$ $3$ parameters from $3$ dimensions
        \item  Energy $\implies$ $1$ parameter
    \end{itemize}
    the solution to (\ref{eqn:local leading-order Boltzmann equation}) at each $\bfx$ and $t$ lies on a $(S + 4)$-dimensional manifold; that is to say that, in the limit, the solution for $(f_{s})_{s}$ can be written solely as a function of the:
    \begin{align}
        \text{Mass in each phase, }  \rho_{s}(\bfx; t)  &:=  \int_{\bfv}f_{s}m_{s}  \\
        \text{Total momentum, }  \bfp(\bfx; t)  &:=  \sum_{s}\int_{\bfv}f_{s}m_{s}\bfv  \\
        \text{Total energy, }  E(\bfx; t)  &:=  \sum_{s}\int_{\bfv}f_{s}\frac{1}{2}m_{s}\|\bfv\|^{2}
    \end{align}
    For each $s$, one can then write
    \begin{equation}
        f_{s}(\bfx, \bfv; t)  \sim  f_{s}^{(0)}[(\rho_{s})_{s}|_{\bfx; t}, \bfp|_{\bfx; t}, E|_{\bfx; t}](\bfx, \bfv; t).
    \end{equation}
    This limit is referred to as thermalization.
    
    A necessary assumption in the derivation of a fluid model is that the distribution function is \emph{fully} thermalized, in so far as $f_{s}  =  f_{s}^{(0)}$ \emph{exactly}, for each $s$. One can then derive evolution equations for $(\rho_{s})_{s}$, $\bfp$, $\rmE$ by taking the corresponding moments of the original Boltzmann equations (\ref{eqn:Boltzmann equation}).

    \begin{remark}[Flaws in the full thermalization assumption]
        The assumption that the distribution functions $(f_{s})_{s}$ are \emph{fully} thermalized, with $f_{s}  =  f_{s}^{(0)}$, is fundamentally flawed. Perturbations from the thermalized equilibrium distributions $(f_{s}^{(0)})_{s}$ are known to have an effect on the dynamics of the problem, as the MHD models presented here can \emph{not} be derived on the assumption of \emph{leading-order} thermalization $f_{s}  \sim  f_{s}^{(0)}$ alone. \BA{[Ref]} This is a flaw that pervades practically \emph{all} fluid models in kinetic theory.
    \end{remark}

    Let $\rho_{\rmM}$ denote the mass density,
    \begin{equation}
        \rho_{\rmM}  :=  \int_{\bfv}[f_{+}m_{+} + f_{-}m_{-}]  \left(\sim  \int_{\bfv}f_{+}m_{+}\right),
    \end{equation}
    and $\bfu$ denote the mass-averaged flow velocity,
    \begin{equation}
        \bfu  :=  \frac{1}{\rho_{M}}\bfp.
    \end{equation}
    Returning to the ion/electron model with the mass-dominant ion phase, assume:
    \begin{itemize}
        \item  Local collisions $(\bfC_{ss'}^{(0)})_{ss'}$ and the resultant asymptotic distributions $(f_{s}^{(0)})_{s}$ are isotropic, such that the 2$^{\text{nd}}$ and 3$^{\text{rd}}$ moments can be written in the forms:
        \begin{align}
            \sum_{s}\int_{\bfv}f_{s}^{(0)}m_{s}\bfv^{\otimes 2}  &\sim  \rho_{\rmM}\bfu^{\otimes 2} + p\sfI  \\
            \sum_{s}\int_{\bfv}f_{s}^{(0)}m_{s}\bfv^{\otimes 3}  &\sim  \rho_{\rmM}\bfu^{\otimes 3} + p(\bfu\otimes\sfI + \cdots + \sfI\otimes\bfu)
        \end{align}
        for a pressure $p$. \BA{(Is this a safe assumption? I've seen bimaxwellian background distributions being used that might imply not so much, but surely the collisions are still isotropic, no?)}
        \item  Each $f_{s}  =  o\left[\|\bfv\|^{- 5}\right]$ uniformly in $\bfv$ as $\|\bfv\|  \rightarrow  \infty$, such that integration by parts can be applied.
    \end{itemize}
    the following \emph{fluid model} is derived:
    \begin{align}
        \partial_{t}\rho_{\rmM} + \nabla\cdot\bfp  &=  0  \\
        \partial_{t}\rho_{\rmC} + \nabla\cdot\bfj  &=  0  \\
        \partial_{t}\bfp + \left(\nabla\cdot\left[\rho_{\rmM}\bfu^{\otimes 2}\right] + \nabla p\right) - \left(\rho_{c}\bfE + \bfj\wedge\bfB\right)  &\sim  - \sum_{s, s'}q_{s}q_{s'}\int_{\bfv}\bfdelta\bfC_{ss'}  \\
        \partial_{t}E + \nabla\cdot\left[\frac{1}{2}\rho_{\rmM}\|\bfu\|^{2}\bfu + \frac{5}{2}p\bfu\right] - \bfj\cdot\bfE  &\sim  - \sum_{s, s'}q_{s}q_{s'}\int_{\bfv}\bfdelta\bfC_{ss'}\cdot\bfv
    \end{align}
    or, in $p$,
    \begin{equation}
        \frac{3}{2}\partial_{t}p + \left(\frac{3}{2}\nabla\cdot[p\bfu] + p\nabla\cdot\bfu\right) - ((\bfj - \rho_{\rmC}\bfu)\cdot\bfE - \bfu\cdot(\bfj\wedge\bfB))  \sim  - \sum_{s, s'}q_{s}q_{s'}\int_{\bfv}\bfdelta\bfC_{ss'}\cdot\left(\bfv - \bfu\right),
    \end{equation}
    where $\nabla  =  \nabla_{\bfx}$ is henceforth assumed to be the derivative over $\bfx$, with $\bfv$ eliminated from the model.
    