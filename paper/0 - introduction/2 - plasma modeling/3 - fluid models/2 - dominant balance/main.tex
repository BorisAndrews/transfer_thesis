\subsubsection*{Assumption 2: Forcing vs. Collisions Dominant Balance}
    Consider now the dominant terms in the Boltzmann equation (\ref{eqn:Boltzmann equation}) in a tokamak environment. Denote the scale of a variable, $*$, with an overbar, $\overline{*}$. Suppose the following scales are given by the problem specification:
    \begin{align}
        \left(\overline{n}_{s}\right)_{s},         &&
        \overline{\bfx},                           &&
        \overline{\bfv},                           &&
        \overline{\bfB},
    \end{align}
    where $n_{s}$ denote the particle density of species $s$,
    \begin{equation}
        n_{s}  :=  \int_{\bfv}f_{s}.
    \end{equation}

    \BA{Also: $(q_{s})_{s}$, $(m_{s})_{s}$.}
    
    The remaining variable scales are defined by:
    \begin{itemize}
        \item  $\overline{t}           :=  \overline{\bfx}/\overline{\bfv}$, working on convective timescales.
        \item  $\overline{\bfE}        :=  \overline{\bfv}\overline{\bfB}$, balancing the electric and magnetic Lorentz forces. \BA{(Why would this be the case?)}
        \item  $\overline{f}_{s}       :=  \overline{n}_{s}/\overline{\bfv}^{3}$, by the definition of $n_{s}$.
        \item  $\overline{\bfC}_{ss'}  :=  \max_{s}\{\overline{n}_{s}\}\overline{\bfB}/\max_{s'}\{q_{s'}\}\overline{\bfv}^{2}$, giving dominant balance of electromagnetic and collisional forces from phase $s'$ in phase $s$. \BA{(Justification.)}
    \end{itemize}
    
    Terms in the Boltzmann equation (\ref{eqn:Boltzmann equation}) then have the scales:
    \begin{align}
        \overline{\partial_{t}f_{s} + \nabla_{\bfx}\cdot[f_{s}\bfv]}  &=  \frac{\overline{n}_{s}}{\overline{\bfx}\overline{\bfv}^{2}}  \label{eqn:convective scale}  \\
        \overline{\frac{q_{s}}{m_{s}}\nabla_{\bfv}\cdot[f_{s}(\bfE + \bfv\wedge\bfB)]}  &=  \frac{q_{s}}{m_{s}}\cdot\frac{\overline{n}_{s}\overline{\bfB}}{\overline{\bfv}^{3}}  \\
        \overline{\frac{q_{s}q_{s'}}{m_{s}}\nabla_{\bfv}\cdot\bfC_{ss'}}  &=  \frac{q_{s}q_{s'}}{m_{s}\max_{s}\{q_{s'}\}}\cdot\frac{\max_{s}\{\overline{n}_{s}\}\overline{\bfB}}{\overline{\bfv}^{3}}  \label{eqn:collisional scale}
    \end{align}
    Regarding the values of these scales within a tokamak environment, consider the conditions during a typical JET reactor pulse, for a predominant (positive) deuterium ion (indexed via $*_{+}$) and (negative) electron (indexed via $*_{-}$) phase, with physical parameters for the JET reactor as listed in Chapters 2 and 4 of \cite{Wes00}:
    \begin{itemize}
        \item  $\overline{n}_{\pm}  \approx  10^{19}\rmm^{- 3}$.
        \item  $\overline{\bfx}     \approx  2.5\rmm$, twice the minor radius of $1.25\rmm$.
        \item  $\overline{\bfv}     \approx  7.9\ldots\times 10^{5}\rmm\rms^{- 1}$, the thermal velocity $\sqrt{\rmk_{\rmB}T/\rmm_{+}}$ for the (mass/energy-dominant) ion phase at operational temperature $1.5\times 10^{8}\rmK$, in the middle of the range $1$–$2\times 10^{8}\rmK$.
        \item  $\overline{\bfB}     \approx  3.5\rmT$.
    \end{itemize}    
    The remaining scales then evaluate as:
    \begin{align*}
        \overline{t}           &\approx  3.2\times 10^{- 6}\rms              &
        \overline{\bfE}        &\approx  2.8\times 10^{6}\rmT\rmm\rms^{- 1}  \\
        \overline{f}_{\pm}     &\approx  2.1\times 10^{1}\rms^{3}\rmm^{- 6}  &
        \overline{\bfC}_{\pm_{1}\pm_{2}}  &\approx  3.5\times 10^{- 12}\rms^{3}\rmT^{2}\rmk\rmg^{- 1}\rmm^{- 1}
    \end{align*}
    
    \begin{remark}[Justification for the non-relativistic model]
        Up until now, a non-relativistic model has been assumed. Observing that $\overline{\bfv}  \ll  c$, we see that this is in general a fair assumption, although relativistic effects are known to have some effects on tokamak plasma dynamics due to their impact on the very-high energy tails of the distribution functions.
    \end{remark}
    
    The scales (\ref{eqn:convective scale}–\ref{eqn:collisional scale}) for each phase thus evaluate, in $s^{2}\rmm^{- 6}$, as:
    \begin{align}
        \overline{\partial_{t}f_{\pm} + \nabla_{\bfx}\cdot[f_{\pm}\bfv]}  &\approx  6.5\ldots\times 10^{6}  \\
        \overline{\frac{q_{+}}{m_{+}}\nabla_{\bfv}\cdot[f_{+}(\bfE + \bfv\wedge\bfB)]}  =  \overline{\frac{q_{+}q_{\pm}}{m_{+}}\nabla_{\bfv}\cdot\bfC_{+\pm}}  &\approx  3.4\ldots\times 10^{9}  \\
        \overline{\frac{q_{-}}{m_{-}}\nabla_{\bfv}\cdot[f_{-}(\bfE + \bfv\wedge\bfB)]}  =  \overline{\frac{q_{-}q_{\pm}}{m_{-}}\nabla_{\bfv}\cdot\bfC_{-\pm}}  &\approx  1.3\ldots\times 10^{13}
    \end{align}
    Under tokamak conditions therefore, the Boltzmann equations for either phase are dominated by the forcing terms, $\nabla_{\bfv}\cdot[f_{\pm}\left(\bfE + \bfv\wedge\bfB\right)]$, and the collisional terms $\nabla_{\bfv}\cdot\bfC_{\pm_{1}\pm_{2}}$, by a factor of approximately $1.9\ldots\times 10^{- 3}$ in the ion phase, and the stronger factor of approximately $5.2\ldots\times 10^{- 7}$ in the electron phase.

    \BA{(Think I should be including a separate force for ion-ion/electron-electron repulsion...)}

    Assuming it holds true therefore that the forcing and collisional terms achieve the dominant balance, the Boltzmann equations take the leading-order forms:
    \begin{align}\label{eqn:leading-order Boltzmann equation}
        \frac{q_{s}}{m_{s}}\nabla_{\bfv}\cdot[f_{s}(\bfE + \bfv\wedge\bfB)]  &\sim  \frac{q_{s}}{m_{s}}\sum_{s'}q_{s'}\nabla_{\bfv}\cdot\bfC_{ss'}  \\
        \nabla_{\bfv}\cdot\left[f_{s}(\bfE + \bfv\wedge\bfB) - \sum_{s'}q_{s'}\bfC_{ss'}\right]  &\sim  0
    \end{align}
    