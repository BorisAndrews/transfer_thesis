\subsubsection*{The Resultant Spatially-Local System}
    One seeks now to leverage these 2 assumptions to derive a fluid model in the following (non-dimensionalized) functions of $\bfx$ and $t$ only:
    \begin{align}
        \text{Mass density, }   \rho_{\rmM}(\bfx; t)  &:=  \int_{\bfv}\left[f_{+} + f_{-}\frac{\rmm_{-}}{\rmm_{-}}\right]  \\
        \text{Charge density, } \rho_{\rmC}(\bfx; t)  &:=  \frac{\beta\rmCy\!_{+}}{2\rmM^{2}}\int_{\bfv}[f_{+} - f_{-}]  \\
        \text{Total momentum, }        \bfp(\bfx; t)  &:=  \int_{\bfv}\left[f_{+} + f_{-}\frac{\rmm_{-}}{\rmm_{-}}\right]\bfv  \\
        \text{Total energy, }             E(\bfx; t)  &:=  \int_{\bfv}\left[f_{+} + f_{-}\frac{\rmm_{-}}{\rmm_{-}}\right]\frac{1}{2}\|\bfv\|^{2}
    \end{align}
    with $\beta$ defined: (See Figure \ref{fig:fluid dimensionless quantities})
    \begin{equation}
        \beta  :=  2\mu_{0}\rmm_{+}\cdot\frac{\overline{n}_{+}\overline{\bfv}^{2}}{\overline{\bfB}^{2}}
    \end{equation}
    With $\rmm_{-} \ll \rmm_{+}$, one can write:
    \begin{align}
      \rho_{\rmM}(\bfx; t)  &=  \int_{\bfv}f_{+} + \calO\left[\frac{\rmm_{-}}{\rmm_{-}}\right]  \\
             \bfp(\bfx; t)  &=  \int_{\bfv}f_{+}\bfv + \calO\left[\frac{\rmm_{-}}{\rmm_{-}}\right]  \\
                E(\bfx; t)  &=  \int_{\bfv}f_{+}\frac{1}{2}\|\bfv\|^{2} + \calO\left[\frac{\rmm_{-}}{\rmm_{-}}\right]
    \end{align}
    
    Combining the assumptions (\ref{eqn:local collision operator}) and (\ref{eqn:leading-order Boltzmann equation}), the dominant component of the Boltzmann equations take the form
    \begin{equation}
        \nabla_{\bfv}\cdot\left[\rmCy\!_{\pm}f_{\pm}(\bfE + \bfv\wedge\bfB) - \rmKn_{\pm\pm}\bfC_{\pm\pm}^{(0)} - \rmKn_{\pm\mp}\bfC_{\pm\mp}^{(0)}\right]  =  \calO[1]
    \end{equation}
    This leading-order PDE is crucially 0th-order in $\bfx$ and $t$, and can, in theory, be solved pointwise in $\bfx$ and $t$. For this system to be well-posed, it is necessary that the moments corresponding to mass, momentum and energy---as in Lemma \ref{lem:conservation on collision operators}---are 0. This can be guaranteed by the introduction of the following bulk forcing and heating/dissipation terms to the right-hand side (RHS) of each equation:
    \begin{multline}\label{eqn:local leading-order Boltzmann equation}
        \nabla_{\bfv}\cdot\left[\rmCy\!_{\pm}f_{\pm}(\bfE + \bfv\wedge\bfB) - \rmKn_{\pm\pm}\bfC_{\pm\pm}^{(0)} - \rmKn_{\pm\mp}\bfC_{\pm\mp}^{(0)}\right]  \\
        =  \frac{2}{\beta}\cdot\frac{1}{\rho_{\rmM}}\nabla_{\bfv}\cdot\left[f_{\pm}^{(0)}\left(\rmM^{2}\rho_{\rmC}\bfE + \bfj\wedge\bfB\right) - \nabla_{\bfv}\left[f_{\pm}^{(0)}\left(\bfj - \rmM^{2}\rho_{\rmC}\bfu\right)\cdot\left(\bfE + \bfu\wedge\bfB\right)\right]\right]  \\
        + \calO[1],
    \end{multline}
    Since, by definition, $\overline{\bfj}  \ngtr  \rmq_{\pm}\overline{\bfv}$, $\frac{2}{\beta}  \ngtr  |\rmCy_{\pm}|$, such that these new RHS term are no greater than the already existing left-hand side (LHS) terms.  Under the given typical tokamak operational conditions from \cite{Wes00} as above, $\beta$ takes a value of $\beta  \approx 4.249\ldots\times 10^{- 3}$, such that, as predicted, $\frac{2}{\beta}  \ngtr  |\rmCy_{\pm}|$.
    
    This leading-order PDE is crucially 0th-order in $\bfx$ and $t$, and can be solved pointwise in $\bfx$ and $t$. One can then write the solution to the system (\ref{eqn:local leading-order Boltzmann equation}) as
    \begin{equation}\label{eqn:thermalization}
        f_{\pm}(\bfx, \bfv; t)  =  f_{\pm}^{(0)}[\rho_{\rmM}|_{\bfx; t}, \rho_{\rmC}|_{\bfx; t}, \bfp|_{\bfx; t}, E|_{\bfx; t}](\bfx, \bfv; t) + \calO\left[\frac{1}{\rmCy_{+}}, \frac{1}{\rmCy_{+}\rmRef}\right].
    \end{equation}
    This limit is referred to as \emph{``thermalization''}. This motivates the creation of the fluid model.

    
