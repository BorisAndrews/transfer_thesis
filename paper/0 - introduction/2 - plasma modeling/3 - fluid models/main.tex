\subsection*{Fluid Models}\label{cha:fluid models}
    \line

    \begin{definition}[Fluid model]
        Here, ``fluid'' models refer to those wherein the system is reduced from one in \emph{both} position and velocity space (and time) to one in \emph{just} position space (and time) through some kind of approximation to the distribution function.
    \end{definition}

    \line
    
    Recall the Boltzmann equation (\ref{eqn:Boltzmann equation}):
    \begin{equation*}
        \partial_{t}f_{s} + \nabla_{\bfx}\cdot[f_{s}\bfv] + \frac{q_{s}}{m_{s}}\nabla_{\bfv}\cdot[f_{s}(\bfE + \bfv\wedge\bfB)]  =  \frac{q_{s}}{m_{s}}\sum_{s'}q_{s'}\nabla_{\bfv}\cdot\bfC_{ss'}.
    \end{equation*}
    To look at the dominant terms in this equation, one can non-dimensionalize. Denote the scale of a variable, $*$, with an overbar, $\overline{*}$. Suppose the following scales are given by the problem specification:
    \begin{align}
        \left(\overline{n}_{s}\right)_{s},         &&
        \overline{\bfx},                           &&
        \overline{\bfv},                           &&
        \overline{\bfB},
    \end{align}
    where $n_{s}$ denote the particle density of species $s$,
    \begin{equation}
        n_{s}  :=  \int_{\bfv}f_{s}.
    \end{equation}
    The remaining variable scales are given by:
    \begin{itemize}
        \item  $\overline{t}           :=  \overline{\bfx}/\overline{\bfv}$, working on convective timescales.
        \item  $\overline{\bfE}        :=  \overline{\bfv}\overline{\bfB}$, balancing the electric and magnetic Lorentz forces. \BA{(Why would this be the case?)}
        \item  $\overline{f}_{s}       :=  \overline{n}_{s}/\overline{\bfv}^{3}$, by the definition of $n_{s}$.
        \item  $\overline{\bfC}_{ss'}  :=  \max_{s}\{\overline{n}_{s}\}\overline{\bfB}/\max_{s'}\{q_{s'}\}\overline{\bfv}^{2}$, giving dominant balance of electromagnetic and collisional forces from phase $s'$ in phase $s$. \BA{(Justification.)}
    \end{itemize}
    
    The Boltzmann equation (\ref{eqn:Boltzmann equation}) then takes the non-dimensionalized form
    \begin{multline}\label{eqn:non-dim Boltzmann general}
        \left[\frac{m_{s}}{q_{s}}\cdot\frac{\overline{\bfv}}{\overline{\bfx}\overline{\bfB}}\right](\partial_{t}f_{s} + \nabla_{\bfx}\cdot[f_{s}\bfv]) + \nabla_{\bfv}\cdot[f_{s}(\bfE + \bfv\wedge\bfB)]  \\
        =  \left[\frac{\max_{s}\{\overline{n}_{s}\}}{\overline{n}_{s}}\right]\sum_{s'}\left[\frac{q_{s'}}{\max_{s'}\{q_{s'}\}}\right](\nabla_{\bfv}\cdot\bfC_{ss'}).
    \end{multline}
    
    Regarding the scale of these terms, consider the conditions during a typical JET reactor pulse, for a predominant (positive) deuterium \BA{(Should I be including tritium here?)} ion (indexed via $*_{+}$) and (negative) electron (indexed via $*_{-}$) phase, with physical parameters for the JET reactor as listed in Chapters 2 and 4 of \cite{Wes00}:
    \begin{itemize}
        \item  $\overline{n}_{\pm}  \approx  10^{19}\rmm^{- 3}$.
        \item  $\overline{\bfx}     \approx  2.5\rmm$, twice the minor radius of $1.25\rmm$.
        \item  $\overline{\bfv}     \approx  7.9\times 10^{5}\rmm\rms^{- 1}$, the thermal velocity $\sqrt{\rmk_{\rmB}T/\rmm_{+}}$ for the (mass/energy-dominant) ion phase at operational temperature $1.5\times 10^{8}\rmK$, in the middle of the range $1$–$2\times 10^{8}\rmK$.
        \item  $\overline{\bfB}     \approx  3.5\rmT$.
    \end{itemize}
    The remaining scales then evaluate as:
    \begin{align*}
        \overline{t}           &\approx  3.2\times 10^{- 6}\rms,              &
        \overline{\bfE}        &\approx  2.8\times 10^{6}\rmT\rmm\rms^{- 1},  \\
        \overline{f}_{\pm}     &\approx  2.1\times 10^{1}\rms^{3}\rmm^{- 6},  &
        \overline{\bfC}_{\pm_{1}\pm_{2}}  &\approx  3.5\times 10^{- 12}\rms^{3}\rmT^{2}\rmk\rmg^{- 1}\rmm^{- 1}.
    \end{align*}
    Thus, (\ref{eqn:non-dim Boltzmann general}) for each phase takes the non-dimensionalized form:
    \begin{align}
        1.9\ldots\times 10^{- 3}(\partial_{t}f_{+} + \nabla_{\bfx}\cdot[f_{+}\bfv]) + \nabla_{\bfv}\cdot[f_{+}(\bfE + \bfv\wedge\bfB)]  &=  \nabla_{\bfv}\cdot[\bfC_{++} + \bfC_{+-}],  \label{eqn:non-dim Boltzmann +}  \\
        5.2\ldots\times 10^{- 7}(\partial_{t}f_{-} + \nabla_{\bfx}\cdot[f_{-}\bfv]) - \nabla_{\bfv}\cdot[f_{-}(\bfE + \bfv\wedge\bfB)]  &=  \nabla_{\bfv}\cdot[\bfC_{-+} + \bfC_{--}].  \label{eqn:non-dim Boltzmann -}
    \end{align}
    These equations are dominated by the forcing terms, $\nabla_{\bfv}\cdot[f_{\pm}\left(\bfE + \bfv\wedge\bfB\right)]$, and the collisional terms $\nabla_{\bfv}\cdot\bfC_{\pm_{1}\pm_{2}}$.

    \BA{(Think I should be including a separate force for ion-ion/electron-electron repulsion...)}
    
    \line

    Fluid models extend this by assuming that the dominant collisional effects are local in space and time. \BA{(On what scales?)} When viewed as functionals on $(f_{\pm_{1}}, f_{\pm_{2}})$, it is assumed that $(\bfC_{\pm_{1}\pm_{2}})_{\pm_{1}\pm_{2}}$ are dominated by the local contributions of $(f_{\pm_{1}}, f_{\pm_{2}})$ at each $\bfx$ and $t$; \BA{(Not sure about this phrasing...)} that is to say that, up to leading order,
    \begin{equation}
        \bfC_{\pm_{1}\pm_{2}}[f_{\pm}, f_{\pm}](\bfx, \bfv; t)  \sim  \bfC_{\pm_{1}\pm_{2}}^{(0)}[f_{\pm}|_{\bfx; t}, f_{\pm}|_{\bfx; t}](\bfv)
    \end{equation}
    for some $(\bfC_{\pm_{1}\pm_{2}}^{(0)})_{\pm_{1}\pm_{2}}$. The leading-order components of the Boltzmann equations (\ref{eqn:non-dim Boltzmann +}–\ref{eqn:non-dim Boltzmann -}) therefore take the forms:
    \begin{align}
        \pm \nabla_{\bfv}\cdot[f_{\pm}(\bfE + \bfv\wedge\bfB)]  &\sim  \nabla_{\bfv}\cdot\left[\bfC_{\pm+}^{(0)} + \bfC_{\pm-}^{(0)}\right]  \\
        0  &\sim  \nabla_{\bfv}\cdot\left[\left(\bfC_{\pm+}^{(0)} + \bfC_{\pm-}^{(0)}\right) \mp f_{\pm}(\bfE + \bfv\wedge\bfB)\right]  \label{eqn:leading-order Boltzmann}
    \end{align}
    The subtitution of $(\bfC_{\pm_{1}\pm_{2}})_{\pm_{1}\pm_{2}}  \mapsto  (\bfC_{\pm_{1}\pm_{2}}^{(0)})_{\pm_{1}\pm_{2}}$ ensures these systems are 0$^{\text{th}}$-order in $\bfx$, implying that for any given $\bfx$, (\ref{eqn:leading-order Boltzmann}) is a system of PDEs in $(f_{\pm}|_{\bfx})_{\pm}$ that is dependent on $\bfv$ \emph{only}.

    \line

    \begin{lemma}
        Defining
        \begin{equation}
            \alpha_{\pm}  =  q_{\pm}\mu_{0}\cdot\frac{\overline{\bfx}\overline{\bfv}\overline{n}_{\pm}}{\overline{\bfB}},
        \end{equation}
        whereby for the scales assumed from \cite{Wes00},
        \begin{equation}
            \alpha_{\pm}  \approx  8.0\times 10^{- 6},
        \end{equation}
        provided the following moment conditions hold:
        \begin{align}
            0  &\sim  \int_{\bfv}\nabla_{\bfv}\cdot\left[\left(\bfC_{++}^{(0)} + \bfC_{+-}^{(0)}\right) - f_{+}(\bfE + \bfv\wedge\bfB)\right],  \label{eqn:leading-order + conservation}  \\
            0  &\sim  \int_{\bfv}\nabla_{\bfv}\cdot\left[\left(\bfC_{-+}^{(0)} + \bfC_{--}^{(0)}\right) + f_{-}(\bfE + \bfv\wedge\bfB)\right],  \label{eqn:leading-order - conservation}  \\
            \begin{split}
                0  &\sim  \int_{\bfv}\left(\nabla_{\bfv}\cdot\left[\left(\bfC_{++}^{(0)} + \bfC_{+-}^{(0)}\right) - f_{+}(\bfE + \bfv\wedge\bfB)\right]\right.  \\
                &\;\;\;\;\;\;\;\;\;\;\;\;\;\;\;\;\;\;\;\;\;\;\;\;\;\;\;\;\;\;\;\;\;\;\;\;\;\;\;\;\;\;\;\;\;\;\;\;+ \left.\nabla_{\bfv}\cdot\left[\left(\bfC_{-+}^{(0)} + \bfC_{--}^{(0)}\right) + f_{-}(\bfE + \bfv\wedge\bfB)\right]\right)\bfv,
            \end{split}  \label{eqn:leading-order momentum conservation}  \\
            \begin{split}
                0  &\sim  \int_{\bfv}\left(\nabla_{\bfv}\cdot\left[\left(\bfC_{\pm+}^{(0)} + \bfC_{\pm-}^{(0)}\right) - f_{\pm}(\bfE + \bfv\wedge\bfB)\right]\right.  \\
                &\;\;\;\;\;\;\;\;\;\;\;\;\;\;\;\;\;\;\;\;\;\;\;\;\;\;\;\;\;\;\;\;\;\;\;\;\;\;\;\;\;\;\;\;\;\;\;\;+ \left.\nabla_{\bfv}\cdot\left[\left(\bfC_{\pm+}^{(0)} + \bfC_{\pm-}^{(0)}\right) + f_{\pm}(\bfE + \bfv\wedge\bfB)\right]\right)\frac{1}{2}\|\bfv\|^{2},
            \end{split}  \label{eqn:leading-order energy conservation}
        \end{align}
        the leading-order Boltzmann system (\ref{eqn:leading-order Boltzmann}) has a 4-dimensional manifold of solutions. \BA{(Why...?)}
    \end{lemma}

    \line

    Assume $(f_{\pm})_{\pm} \rightarrow 0$ sufficiently fast as $\|\bfv\| \rightarrow \infty$, and define:
    \begin{align}
        \rho_{C}
    \end{align}
    (\ref{eqn:leading-order + conservation}–\ref{eqn:leading-order - conservation}) are immediate and exact. For (\ref{eqn:leading-order momentum conservation}),
    \begin{align}
        \begin{split}
            &\int_{\bfv}\left(\nabla_{\bfv}\cdot\left[\left(\bfC_{++}^{(0)} + \bfC_{+-}^{(0)}\right) - f_{+}(\bfE + \bfv\wedge\bfB)\right]\right.  \\
            &\;\;\;\;\;\;\;\;\;\;\;\;\;\;\;\;\;\;\;\;\;\;\;\;\;\;\;\;\;\;\;\;\;\;\;\;\;\;\;\;\;\;\;\;\;\;\;\;+ \left.\nabla_{\bfv}\cdot\left[\left(\bfC_{-+}^{(0)} + \bfC_{--}^{(0)}\right) + f_{-}(\bfE + \bfv\wedge\bfB)\right]\right)\bfv
        \end{split}  \\
        &\;\;\;\;\sim  \int_{\bfv}\nabla_{\bfv}\cdot\left[\left(\bfC_{++}^{(0)} + \bfC_{+-}^{(0)} + \bfC_{-+}^{(0)} + \bfC_{--}^{(0)}\right) - (f_{+} - f_{-})(\bfE + \bfv\wedge\bfB)\right]\bfv  \\
        &\;\;\;\;\sim  \int_{\bfv}(f_{+} - f_{-})(\bfE + \bfv\wedge\bfB)  \\
        &\;\;\;\;\rho_{C}(\bfE + \bfv\wedge\bfB)
    \end{align}

    Since, however, the leading-order collision operators $\left(\bfC_{s}^{(0)}\right)_{s}$ conserve both momentum and energy, i.e.
    \begin{align*}
        \bfzero  &=  \sum_{s, s'}\int_{\bfv}\left(\nabla_{\bfv}\cdot\bfC_{ss'}^{(0)}\right)m_{s}\bfv  &  0  &=  \sum_{s}\int_{\bfv}\left(\nabla_{\bfv}\cdot\bfC_{ss'}^{(0)}\right)\frac{1}{2}m_{s}\|\bfv\|^{2}  \\
        &=  - \sum_{s}\int_{\bfv}\bfC_{s}^{(0)}m_{s}  &  &=  - \sum_{s}\int_{\bfv}\bfC_{s}^{(0)}\cdot m_{s}\bfv
    \end{align*}
    \BA{(I only accounted for \emph{kinetic} energy here, I didn't include the \emph{magnetic potential} energy- not sure if I need to? Would be a nice opportunity to define $\bfA$ if I do.)} equation (\ref{eqn:Maxwellian equation}) is not (necessarily) well-posed, as taking moments for momentum and energy,
    \begin{align*}
        \bfzero  &=  \sum_{s}\int_{\bfv}\frac{q_{s}}{m_{s}}\nabla_{\bfv}\cdot\left[f_{s}^{(0)}(\bfE + \bfv\wedge\bfB)\right]m_{s}\bfv  &  0  &=  \sum_{s}\int_{\bfv}\frac{q_{s}}{m_{s}}\nabla_{\bfv}\cdot\left[f_{s}^{(0)}(\bfE + \bfv\wedge\bfB)\right]\frac{1}{2}m_{s}\|\bfv\|^{2}  \\
        &=  - \sum_{s}\int_{\bfv}f_{s}^{(0)}(\bfE + \bfv\wedge\bfB)q_{s}  &  &=  - \sum_{s}\int_{\bfv}f_{s}^{(0)}(\bfE + \bfv\wedge\bfB)\cdot q_{s}\bfv  \\
        &=  - \rho_{C}\bfE - \bfj\wedge\bfB  &  &=  - \bfj\cdot\bfE
    \end{align*}
    representing the bulk forces and heating on each of the phases, where the charge and current density, $\rho_{C}$ and $\bfj$ respectively, are defined:
    \begin{align}
        \rho_{C}  :=  \sum_{s}\int_{\bfv}f_{s}q_{s},  &&
        \bfj      :=  \sum_{s}\int_{\bfv}f_{s}q_{s}\bfv
    \end{align}
    functions of $\bfx$, $t$ only.
    
    This can be resolved by adding the terms:
    \begin{align}
        \nabla_{\bfv}\cdot\left[f_{s}\frac{1}{\rho_{Ms}}(\rho_{C}\bfE + \bfj\wedge\bfB)\right],  &&
        \Delta_{\bfv}\left[f_{s}\frac{1}{\rho_{M}}\left(\bfE\cdot\left(\frac{\rho_{C}}{\rho_{M}}\bfp - \bfj\right) + \bfB\cdot(\bfp\wedge\bfj)\right)\right]
    \end{align}
    to equation (\ref{eqn:Maxwellian equation}).
    
    Define also the mass and momentum density, $\rho_{M}$ and $\bfp$ respectively:
    \begin{align}
        \rho_{C}  :=  \sum_{s}\int_{\bfv}f_{s}m_{s},  &&
        \bfp      :=  \sum_{s}\int_{\bfv}f_{s}m_{s}\bfv
    \end{align}
    and velocity density, $\bfu$,
    \begin{equation}
        \bfu  :=  \frac{1}{\rho_{M}}\bfp
    \end{equation}
    all functions of $\bfx$, $t$ \emph{only}.
    
    Rewriting the Boltzmann equation in the form
    {\small \begin{multline}
        \partial_{t}f_{s} + \nabla_{\bfx}\cdot[f_{s}\bfv] + \nabla_{\bfv}\cdot\left[f_{s}\frac{1}{\rho_{M}}(\rho_{C}\bfE + \bfj\wedge\bfB)\right] + \Delta_{\bfv}\left[f_{s}\frac{1}{\rho_{M}}\left(\bfE\cdot\left(\frac{\rho_{C}}{\rho_{M}}\bfp - \bfj\right) + \bfB\cdot(\bfp\wedge\bfj)\right)\right]  \\
        + \nabla_{\bfv}\cdot\left[f_{s}\frac{q_{s}}{m_{s}}(\bfE + \bfv\wedge\bfB)\right] - \nabla_{\bfv}\cdot\left[f_{s}\frac{1}{\rho_{M}}(\rho_{C}\bfE + \bfj\wedge\bfB)\right]  \\
        - \Delta_{\bfv}\left[f_{s}\frac{1}{\rho_{M}}\left(\bfE\cdot\left(\frac{\rho_{C}}{\rho_{M}}\bfp - \bfj\right) + \bfB\cdot(\bfp\wedge\bfj)\right)\right]  =  \nabla_{\bfv}\cdot[\bfC_{s}((f_{s'})_{s'})]
    \end{multline}}
    the leading-order system
    \begin{multline}
        \nabla_{\bfv}\cdot\left[f_{s}\frac{q_{s}}{m_{s}}(\bfE + \bfv\wedge\bfB)\right] - \nabla_{\bfv}\cdot\left[f_{s}\frac{1}{\rho_{M}}(\rho_{C}\bfE + \bfj\wedge\bfB)\right]  \\
        - \Delta_{\bfv}\left[f_{s}\frac{1}{\rho_{M}}\left(\bfE\cdot\left(\frac{\rho_{C}}{\rho_{M}}\bfp - \bfj\right) + \bfB\cdot(\bfp\wedge\bfj)\right)\right]  =  \nabla_{\bfv}\cdot\left[\bfC_{s}^{(0)}((f_{s'}|_{\bfx})_{s'})\right]
    \end{multline}
    \emph{does} admit solutions, which can be written as a function of just the functions in $\bfx$, t conserved by the collision operator, the density for each phase, total momentum and total energy:
    \begin{align}
        \rho_{Ms}  :=  \int_{\bfv}f_{s}m_{s},  &&
        \bfp  :=  \sum_{s}\int_{\bfv}f_{s}m_{s}\bfv,  &&
        E  :=  \sum_{s}\int_{\bfv}f_{s}\frac{1}{2}m_{s}\|\bfv\|^{2}
    \end{align}
    Assuming $f_{s}  \sim  f_{s}^{(0)}$ (i.e. that the plasma has ``thermalised'') and taking the above moments of the Boltzmann equation, we retrieve a system in $(\rho_{Ms})_{s}$, $\bfp$, $E$ only. This techniques allows the reduction of the system from one in the 6 (or 7) dimensions of a full kinetic model, to a fluid mode in just 3 (or 4).
    
    The problem however with applying this technique directly to tokamak plasmas fundamentally lies in the assumption of collision-dominant dynamics. \BA{(Some estimates on the scale of these terms in the plasma/edge plasma- some nice parameter scale estimates/values in that gyrokinetics manuscript- Multiscale Gyrokinetics for Rotating Tokamak Plasmas: Fluctuations, Transport and Energy Flows.)} Many highly influential so-called ``kinetic'' effects are not captured by these MHD fluid models, including: \BA{([Ref, Ref, Ref, …])}
    \begin{itemize}
        \item  Most plasma waves
        \item  Most plasma/kinetic instabilities
        \item  Landau damping/Bump-on-tail instabilities
        \item  Leakage
        \item  Kinetic structures (Beams/Double layers)
        \item  Anisotropic pressures
    \end{itemize}
    Techniques for the numerical solution of the MHD equations have been very well developed over recent years however \BA{([Ref, Ref, Ref, Ref, Ref, …])}. The question therefore lies in how these more efficient techniques, can be reapplied to the more accurate kinetic models, when the two are so qualitatively different.

    \BA{...whereby through some approximation to the 2-particle distribution functions, the collision terms, $(\bfC_{ss'})_{ss'}$, are written in terms of the 1-particle distribution functions, $(f_{s})_{s}$.}
    
    \BA{(Check out \href{https://upload.wikimedia.org/wikipedia/commons/a/a9/A_Comparison_Chart_For_Modeling_Plasma2.png}{this} diagram off Wikipedia, or again the content under ``Mathematical Descriptions'' \href{https://en.wikipedia.org/wiki/Plasma_(physics)}{here}.)}

    \BA{Would like to consider an expansion of the collision operator of the form
    \begin{multline}
        \bfC_{s}  =  \left[f_{s}\rho_{s'}(\bfmu_{ss'}(\bfu_{s'} - \bfv) + \nabla_{\bfx}\cdot[\bftau_{ss'}({\bf sym}(\nabla_{\bfx}\bfu_{s}))] + \cdots)\right]  \\
        + \nabla_{\bfv}\cdot\left[f_{s}\rho_{s'}(\bfD_{ss'}(\bfu_{s'} - \bfv) + \cdots)\right] + \cdots
    \end{multline}}
    
    \BA{I have some inconsistency in notation here with the rest of the paper in how I donate integrals. I should make this consistent, by denoting $\int_{\bfx}  \mapsto  \int_{\bfOmega}$ and, say, $\int_{\bfv}  \mapsto  \int_{\bfXi}$.}
    
    \BA{Change my $\rho_{M}$'s and $\rho_{C}$'s to $\rho_{\rmM}$'s and $\rho_{\rmC}$'s.}
    
    \BA{Not mentioned at all here that I'm discounting relativistic effects, and looking at the velocity scales that'd be quite a poignant remark.}
    