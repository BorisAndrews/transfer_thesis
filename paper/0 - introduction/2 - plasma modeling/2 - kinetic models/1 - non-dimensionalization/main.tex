\subsubsection{Non-Dimensionalization}
    Consider now a non-dimensionalization of Maxwell's equation (\ref{eqn:Maxwell's equations transient}--\ref{eqn:Maxwell's equations steady-state}) and the Boltzmann equations (\ref{eqn:Boltzmann equation}).
    
    By the problem specification, the physical constants:
    \begin{align}
        (\rmq_{s})_{s},  &&
        (\rmm_{s})_{s},
    \end{align}
    are given by the parameters in the constituent phases. Further denoting an estimate for the general scale of a variable, $*$, with an overbar, $\overline{*}$, the following quantities are given too by the problem specification within the tokamak:
    \begin{align}
        \overline{t},                       &&
        \overline{\bfx},                    &&
        \left(\overline{n}_{s}\right)_{s},  &&
        \overline{\bfB},
    \end{align}
    where $n_{s}$ denote the particle density of phase $s$, $n_{s}  :=  \int_{\bfv}f_{s}$. From this definition,
    \begin{equation}
        \overline{f}_{s}  :=  \frac{\overline{n}_{s}}{\overline{\bfv}^{3}}.
    \end{equation}

    $\overline{\bfv}$ will scale to achieve dominant balance in the Boltzmann equations (\ref{eqn:Boltzmann equation}), however this shall be discussed in Subsection \ref{cha:fluid models}. For now, it shall be assumed known.

    Define also:
    \begin{itemize}
        \item  The cyclotron frequency in phase $s$,
        \begin{equation}
            \Omega_{s}  :=  \frac{\rmq_{s}}{\rmm_{s}}\cdot\overline{\bfB}.
        \end{equation}
        \item  The (effective) mean free for particles in phase $s$, between particles in phase $s'$
        \begin{equation}
            \lambda_{ss'}  :=  \frac{\rmm_{s}}{\rmq_{s}\rmq_{s'}}\cdot\frac{\overline{n}_{s}}{\overline{\bfv}\overline{\bfC}_{ss'}}.
        \end{equation}
    \end{itemize}
    These invoke the dimensionless quantities in Figure \ref{fig:kinetic dimensionless quantities}.

    \begin{figure}
        \centering
        \begin{tabular}{ c c c c }
            Name  &  Symbol  &  Value  &  Ratio  \\
            \hline\hline
            (Light) Mach number  &  $\rmM$  &  $\overline{\bfv}/\rmc$  &  (Particle : Light) speed  \\
            Strouhal number  &  $\rmSt$  &  $\overline{\bfx}/\overline{t}\overline{\bfv}$  &  (Reference : Kinetic) frequency  \\
            Cyclotron number(s)  &  $\rmCy\!_{s}$  &  $\Omega_{s}\overline{t}$  &  (Cyclotron : Reference) frequency  \\
            Knudsen number(s)  &  $\rmKn_{ss'}$  &  $\overline{\bfx}/\lambda_{ss'}$  &  (Reference : Mean free path) length
        \end{tabular}
        \caption{Dimensionless quantities in the Boltzmann--Maxwell system.}
        \label{fig:kinetic dimensionless quantities}
    \end{figure} 

    \shortline

    \paragraph*{Maxwell's equation (\ref{eqn:Maxwell's equations transient}--\ref{eqn:Maxwell's equations steady-state})} We suppose $\overline{\bfB}$ \emph{is} given by the problem specification while $\overline{\bfE}$ is \emph{not}, as the driving electromagnetic field within a tokamak is typically a magnetic field only, with the electric field induced internally through Maxwell's equations. As the magnetic field, $\bfB$, varies during a simulation, the electric field must change to balance this varying magnetic field in Faraday's law, $\partial_{t}\bfB  =  - \nabla\wedge\bfE$, such that the electric field scale, $\overline{\bfE}$, is given as
    \begin{equation}
        \overline{\bfE}  =  \rmSt\overline{\bfv}\overline{\bfB}.
    \end{equation}
    As such, up to variation in $\rmSt$, the electric and magnetic Lorentz forces are comparable in magnitude.

    Similarly, according to Ampère's law, $\frac{1}{\rmc^{2}}\partial_{t}\bfE  =  \nabla\wedge\bfB - \mu_{0}\bfj$, the curl in the applied magnetic field, $\nabla\wedge\bfB$, must be balanced by the induced current density, $\bfj$. Thus the current density scale, $\overline{\bfj}$, is given as
    \begin{equation}
        \overline{\bfj}  =  \frac{1}{\mu_{0}}\cdot\frac{\overline{\bfB}}{\overline{\bfx}}.
    \end{equation}

    Finally, according to Gauss's law, $\frac{1}{\rmc^{2}}\nabla\cdot\bfE  =  \mu_{0}\rho_{\rmC}$, the divergence in the induced electric field, $\nabla\cdot\bfE$, must be balanced by the induced charge density, $\rho_{\rmC}$. Thus the current density scale, $\overline{\rho}_{\rmC}$, is given as
    \begin{equation}
        \overline{\rho}_{\rmC}  =  \rmSt\rmM^{2}\cdot\frac{1}{\mu_{0}}\cdot\frac{\overline{\bfB}}{\overline{\bfx}\overline{\bfv}}.
    \end{equation}
    
    One can thus non-dimensionalize Maxwell's equations as:
    \begin{align}
        \rmM^{2}\partial_{t}\bfE  &=  \nabla\wedge\bfB - \bfj,  &
        \partial_{t}\bfB  &=  - \nabla\wedge\bfE,  \label{eqn:Maxwell's equations transient non-dimensionalized}  \\
        \nabla\cdot\bfE  &=  \rho_{\rmC},  &
        \nabla\cdot\bfB  &=  0.  \label{eqn:Maxwell's equations steady-state non-dimensionalized}
    \end{align}

    \paragraph*{Boltzmann equations (\ref{eqn:Boltzmann equation})} One can non-dimensionalize the Boltzmann equations (about the convective $\nabla_{\bfx}\cdot[f_{s}\bfv]$ terms) as
    \begin{equation}\label{eqn:Boltzmann equation non-dimensionalized}
        \rmSt\partial_{t}f_{s} + \nabla_{\bfx}\cdot[f_{s}\bfv] + \rmSt\rmCy\!_{s}\nabla_{\bfv}\cdot[f_{s}(\rmSt\bfE + \bfv\wedge\bfB)]  =   \sum_{s'}\rmKn_{ss'}\nabla_{\bfv}\cdot\bfC_{ss'}.
    \end{equation}
    
    \shortline

    From here, the non-dimensionalized system shall be assumed.
    
    \begin{lemma}[Momentum and energy conservation on $(\bfC_{ss'})_{ss'}$]\label{lem:conservation on collision operators}
        The following two identities hold on $(\bfC_{ss'})_{ss'}$:
        \begin{align}
            \int_{\bfx}\left[\sum_{s, s'}\rmKn_{ss'}\frac{\rmm_{s}\overline{n}_{s}}{\max_{s}\{\rmm_{s}\overline{n}_{s}\}}\int_{\bfv}\bfC_{ss'}\right]  =  \bfzero,  &&
            \int_{\bfx}\left[\sum_{s, s'}\rmKn_{ss'}\frac{\rmm_{s}\overline{n}_{s}}{\max_{s}\{\rmm_{s}\overline{n}_{s}\}}\int_{\bfv}\bfC_{ss'}\cdot\bfv\right]  =  0.
        \end{align}
    \end{lemma}
    \begin{proof}
        These results are immediate from conservation of momentum and energy over the whole domain.
    \end{proof}
    