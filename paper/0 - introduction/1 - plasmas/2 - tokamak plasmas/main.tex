\subsection*{What Distinguishes a Tokamak Plasma?}
    Certain properties characterize the plasma in a tokamak. The following figure are taken from \cite{Wes00}:
    \begin{itemize}
        \item  {\bf Very high temperature}: Plasma temperatures within a tokamak are on the order of $10^{8}\rmK$, an order of magnitude \emph{higher} than that in the center of the sun, at around $1.5\times10^{7}\rmK$ \BA{[Ref]}.
        \item  {\bf Very strong EM fields}: The EM fields used to ionize tokamak plasmas have strengths on the order $1\rmT$, with the world's most powerful magnets being those employed in the world's most powerful tokamaks \BA{[Ref]}.
        \item  {\bf Very low density}: Tokamak plasmas feature particle densities on the order of $10^{19}\rmm^{- 3}$. The quantity of hydrogen gas used during a Joint European Torus (JET) pulse is often likened to the mass of a postage stamp \BA{[Ref]}.
    \end{itemize}
    Engineering constraints on the vessel walls often imply too that these plasmas both border onto complex boundaries with complex boundary condition, and can acquire high-levels of impurities from the chamber wall. While these can both have a large impact on the plasma dynamics, they will not be subject of this thesis.
    