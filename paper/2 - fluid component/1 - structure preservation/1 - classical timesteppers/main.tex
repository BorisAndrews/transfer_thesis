\subsection*{Runge--Kutta (RK) Methods}\label{cha:RK methods}
    Traditional approaches for deriving numerical timesteppers from time-dependent PDEs employ a timestepping ``scheme'', typically a Runge--Kutta (RK) method. For an $s$-step method, the RK method of choice is characterized by some:
    \begin{align}
        \sfA  \in  \bbR^{s \times s},  &&
        \bfb  \in  \bbR^{s},  &&
        \bfc  \in  \bbR^{s}.
    \end{align}
    Suppose this RK method is applied over the interval $T^{n}$. Let $\bfu^{n}  =  \bfu|_{t^{n}}$ denote the value taken by $\bfu$ at the start of the interval. In the general case (\ref{eqn:general time-dependent PDE}) one considers the system of $s$ PDEs (or systems of PDEs) for each $i  \in  \{1, \cdots, s\}$, 
    \begin{equation}
        \bfzero  =  \bfF\!\left[\bfu|_{t^{n}} + \sum_{j}a_{ij}\bfdelta\bfu^{n}_{j}, \nabla\bfu|_{t^{n}} + \sum_{j}a_{ij}\nabla\bfdelta\bfu^{n}_{j}, \cdots; \bfdelta\bfu^{n}_{i}\right]\!\left(*, t^{n - 1} + c_{i}\delta t^{n}\right),
    \end{equation}
    where one intends to solve for $\bfdelta\bfu^{n}_{1}(\bfx)$, $\cdots$, $\bfdelta\bfu^{n}_{1}(\bfx)$. One then defines $\bfu|_{t^{n + 1}}$, the value taken by $\bfu$ at the \emph{end} of the interval as
    \begin{equation}
        \bfu|_{t^{n + 1}}  =  \bfu|_{t^{n}} + \delta t^{n}\sum_{j}b_{j}\bfdelta\bfu^{n}_{j}.
    \end{equation}

    \line
    
    \begin{definition}[Collocation methods]
        Collocation methods are a class of RK methods. For an $s$-step method, they are defined via:
        \begin{itemize}
            \item  A set of $s$ distinct ``collocation points'' $(c_{i})_{i}$, defining $\bfc$.
            \item  An $s$-dimensional function space of functions for $\partial_{t}\bfu$, defined on a reference timestep $(0, 1]$, typically polynomials, always containing the constant function $1$.
        \end{itemize}
        For a well-defined pair of collocation points and function space, there exists a basis of functions $(\phi_{j})_{j}$, such that $\phi_{j}(c_{i})  =  \delta_{ij}$. $\sfA$, $\bfb$ are then defined from $(\phi_{j})_{j}$ as:
        \begin{align}
            a_{ij}  :=  \int_{0}^{c_{i}}\phi_{j},  &&
            b_{j}   :=  \int_{0}^{1}\phi_{j}.
        \end{align}
    \end{definition}
    
    For linear problems, the FET-induced timesteppers discussed in the following subsection are often analogous to collocation methods. Consider the following examples:

    \begin{example}[Gauss--Legendre (GL) methods]
        GL methods use:
        \begin{itemize}
            \item  $(c_{i})_{i}$ GL points, roots of the Legendre polynomial $P_{s}$.
            \item  ${\rm Span}[(\phi_{j})_{j}]  =  \calP_{s - 1}$.
        \end{itemize}
        They are symplectic \cite{Lasagni_1988, Sanz--Serna_1988} and A-stable, with order $2s$.

        The lowest-order ($s  =  1$) GL method is the Crank--Nicolson/implicit-midpoint method.
    \end{example}

    \begin{example}[Radau-IIA methods]
        Radau-IIA methods use:
        \begin{itemize}
            \item  $(c_{i})_{i}$ roots of $\left(\frac{\rmd}{\rmd\rmt}\right)^{s - 1}\left[t^{s - 1}(t - 1)^{s}\right]$.
            \item  ${\rm Span}[(\phi_{j})_{j}]  =  \calP_{s - 1}$.
        \end{itemize}
        They are A-stable, with order $2s - 1$.

        The lowest-order ($s  =  1$) Radau-IIA method is the implicit-Euler method.
    \end{example}
    
    \line
    
    RK methods reduce the problem to one in space, $\bfx$, only. The spatial problem can then be discretized using classical spatial discretization techniques, e.g. via the FEM, giving the fully discretized problem.
    
    Typically supporting the discretization of problems in space, $\bfx$, only, traditional FE software is well-suited for the implementation of RK methods, where the RK method can be performed through either manually-implemented loops on each timestep, or specialist packages when available, such as the {\tt Irksome} package within {\tt Firedrake}. \cite{Farrell_Kirby_Marchena-Menéndez_2021}

    Careful choice of RK method can give timesteppers that preserve the conservative/dissipative structures of the continuous strong form.

    \line
    
    \begin{example}[Heat equation]
        GL methods are known to dissipate energy, $\rmE(t)  :=  \frac{1}{2}\int_{\bfOmega}u^{2}$, in the heat equation (\ref{eqn:heat equation}) under Dirichlet/Neumann boundary conditions.
        
        At lowest order ($s  =  1$) the Crank--Nicolson method has the exact dissipation
        \begin{equation}\label{eqn:heat equation dissipation implicit midpoint}
            \frac{\delta t^{n}}{\rmPe}\int_{\bfOmega}\left\|\nabla\left[\frac{1}{2}\left(u|_{t^{n + 1}} + u|_{t^{n}}\right)\right]\right\|^{2},
        \end{equation}
        over the timestep $T^{n}$, resembling the continuous result (\ref{eqn:heat equation dissipation}).
    \end{example}
    
    \begin{example}[Hamiltonian systems]
        Symplectic timesteppers/integrators not only preserve the Hamiltonian in Hamiltonian systems, but further the general symplectic structure. The study and creation of symplectic timesteppers/integrators are the subjects of a wide-ranging, independent field of study. See, for example, \cite{Hairer_Lubich_Wanner_2006}.
    \end{example}
    
    \line

    Most modern FE(-in-space) SP timesteppers in MHD rely on bespoke RK methods, in particular the Crank--Nicolson method or slight modifications thereof. This is the approach for time discretization taken by Hu et al. to date. \cite{Hu_Lee_Xu_2021, Laakmann_Hu_Farrell_2022}
    