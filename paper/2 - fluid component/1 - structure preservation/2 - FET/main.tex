\subsection*{Functions Spaces/Finite Elements in Time (FET)}\label{cha:FET}
    One can alternatively interpret (\ref{eqn:general time-dependent PDE}) directly over the full space-\emph{time} domain $\bfOmega\otimes T^{n}$. This is interpreted as a domain in the classical spatial sense, with $\bfOmega\otimes T^{n}  \subset  \bbR^{d + 1}$ when $\bfOmega  \subset  \bbR^{d}$. One can discretize the system over this domain using whatever classical technique is desired. In our case this is the finite-element method, or, since the domain is in space-\emph{time}, finite elements in time (FET). The value taken by $\bfu$ at the start of the interval, $\bfu|_{t^{n}}$, can be interpreted as one BC, with the value taken by $\bfu$ at the \emph{end} of the interval, $\bfu|_{t^{n + 1}}$, simply following the definition of the restriction of the solution to the final time.
    
    When using FET, the traditional independent choice of RK method when constructing a timestepper is transferred to one of the behavior of the function spaces and Galerkin discretizations in time, in their continuity, order, etc.
    
    A few key points on FET timesteppers:
    \begin{itemize}
        \item  Due to the tensor product nature of the space-time domain, $\bfOmega\otimes T^{n}$, the space-time trial and test function spaces in variational formulations can often be constructed via the tensor product of spaces on the spatial, $\bfOmega$, and time, $T^{n}$, domains without needing to construct new spaces on the entire space-time domain, $\bfOmega\otimes T^{n}$.
        \item  The test function spaces need not be identical to the trial function spaces (i.e. a \emph{Petrov}--Galerkin methods) and often won't be. This is natural with the odd order in time.
        \item  The intention of this approach is typically \emph{not} to solve an even higher dimension FE problem on all of $\bfOmega\otimes T^{n}$, but to pick discrete trial and test spaces in the Galerkin variational formulation that have a fixed, small number of DoFs in time---$\calO[s]$ in the sense of the number of steps of an RK method---such as $*\left(\bfOmega^{\rmh}\right)\otimes\bbP^{s}\left(T^{n}\right)$ or $*\left(\bfOmega^{\rmh}\right)\otimes\bbDP^{s - 1}\left(T^{n}\right)$.
    \end{itemize}
    FET interpretations provide certain benefits over traditional RK techniques:
    \begin{itemize}
        \item  Crucially, FET often provides variational formulations that align with those needed to prove structure-preserving properties directly.
        \item  FET interpretations inherit much of the useful results and techniques of traditional spatial FEM, including:
        \begin{itemize}
            \item  Finite-element exterior calculus (FEEC).
            \item  Parallel-in-time methods.
            \item  Well-studied interpolation and projection error estimates.
            \item  Language for the analysis and interpretation of non-confoming schemes.
            \item  Agency over bases for the time discretization with good orthogonality properties.
            \item  Easy extension to high order.
        \end{itemize}
        \item  Solutions provided by FET are meaningful at all times. As noted in Section \ref{cha:FEM vs. PiC} this will be particularly useful in our case when interacting with the kinetic correction in the future. 
        \item  FET can be the perfect language for solving problems that are already posed on a space-time domain, such as:
        \begin{itemize}
            \item  Searching for periodic solutions.
            \item  Problems with moving boundaries, where the space-time domain can \emph{not} be written in the tensor-product form $\bfOmega\otimes T^{n}$.
        \end{itemize}
    \end{itemize}
    
    \begin{example}[CG-DG-in-time dissipative timestepper for the heat equation]
      Consider the heat equation (\ref{eqn:heat equation}) on $\bfOmega\otimes T^{n}$. We take homogeneous Dirichlet/Neumann BCs, and suppose the ICs, $u|_{t^{n}}  =  \widehat{u}_{0}$ hold. In the strong form, the dissipation equation (\ref{eqn:heat equation dissipation}) holds.

      To cast into a chosen weak form \emph{in time}, define the solution space $\calU$, supported on $\bfOmega\otimes T^{n}$. To incorporate the ICs:
      \begin{enumerate}
          \item  Define $\calU_{-}  :=  \{u  \in  \calU  :  u|_{t^{n}}  =  0\}$ (such that $\calU_{-}  <  \calU$).
          \item  Let $u_{0}  \in  \calU$ be an extension of the ICs $\widehat{u}_{0}$ to be supported the space-time domain, $\bfOmega\otimes T^{n}$.
          \item  Decompose $u$ with a lifting of the ICs, as $u  :=  u_{0} + u_{-}$. When $u$ is written from here, it can be taken formally as a convenient shorthand for $u_{0} + u_{-}$.
      \end{enumerate}
      In weak form, one then seeks $u_{-}  \in  \calU_{-}$ such that,
      \begin{equation}\label{eqn:heat equation weak form}
          (\partial_{t}u  =)  \bbQ_{\partial_{t}\calU}[\partial_{t}u]  =  \frac{1}{\rmPe}\bbQ_{\partial_{t}\calU}[\Delta u].
      \end{equation}
      Casting into a variational formulation, one can seek $u_{-}  \in  \calU_{-}$ such that $\forall  v  \in  \partial_{t}\calU$,
      \begin{equation}\label{eqn:heat equation variational form}
          \langle v, \partial_{t}u\rangle_{\bfOmega\otimes T^{n}}  =   - \frac{1}{\rmPe}\langle\nabla v, \nabla u\rangle_{\bfOmega\otimes T^{n}},
      \end{equation}
      
      To create an order-$s$ CG-DG FET timestepper, define the Galerkin subspace
      \begin{equation}
          \bbU^{\rmh}  :=  \widehat{\calU}(\bfOmega)\otimes\bbP^{s}\left(T^{n}\right)  <  \calU.
      \end{equation}
      Consequently
      \begin{equation}
          \partial_{t}\bbU^{\rmh}  :=  \widehat{\calU}(\bfOmega)\otimes\bbDP^{s - 1}\left(T^{n}\right)  <  \partial_{t}\calU.
      \end{equation}
      Mapping $\calU  \mapsto  \bbU^{\rmh}$ in (\ref{eqn:heat equation variational form}) invokes the Petrov--Galerkin projection. Denote bases for these FE spaces:
      \begin{align}
          (\phi_{j})_{j}     \text{ for }  \bbP^{s}\left(T^{n}\right),         &&
          (\varphi_{i})_{i}  \text{ for }  \bbDP^{s - 1}\left(T^{n}\right),
      \end{align}
      with $(\phi_{j})_{j}$ defined such that:
      \begin{align}
          \phi_{j}\left(t^{n}\right)      =  \delta_{j0},  &&
          \phi_{j}\left(t^{n + 1}\right)  =  \delta_{js}.
      \end{align}
      $u^{\rmh}  \in  \bbU^{\rmh}$, $v^{\rmh}  \in  \partial_{t}\bbU^{\rmh}$ can be expressed in terms of these bases as:
      \begin{align}
          u^{\rmh}(\bfx; t)  =  \sum_{j}\widehat{u}_{j}(\bfx)\phi_{j}(t),  &&
          v^{\rmh}(\bfx; t)  =  \sum_{i}\widehat{v}_{i}(\bfx)\varphi_{i}(t),
      \end{align}
      for $\left(\widehat{u}_{j}\right)_{j}$, $\left(\widehat{v}_{i}\right)_{i}$ in $\widehat{\calU}$. The Petrov--Galerkin-projected variational formulational then seeks $\left(\widehat{u}_{j}\right)_{j \neq 0}$ in $\widehat{\calU}$ such that $\forall  \left(\widehat{v}_{i}\right)_{i}$ in $\widehat{\calU}$:
      \begin{align}
          \left\langle\sum_{i}\widehat{v}_{i}\varphi_{i}, \partial_{t}\left[\sum_{j}\widehat{u}_{j}\phi_{j}\right]\right\rangle_{\bfOmega\otimes T^{n}}  &=  - \frac{1}{\rmPe}\left\langle\nabla\left[\sum_{i}\widehat{v}_{i}\varphi_{i}\right], \nabla\left[\sum_{j}\widehat{u}_{j}\phi_{j}\right]\right\rangle_{\bfOmega\otimes T^{n}}  \\
          \sum_{i, j}\left\langle\widehat{v}_{i}, \widehat{u}_{j}\right\rangle_{\bfOmega}\langle\varphi_{i}, \partial_{t}\phi_{j}\rangle_{T^{n}}  &=  - \frac{1}{\rmPe}\sum_{i, j}\left\langle\nabla\widehat{v}_{i}, \nabla\widehat{u}_{j}\right\rangle_{\bfOmega}\langle\varphi_{i}, \phi_{j}\rangle_{T^{n}}
      \end{align}
      \vspace{-8mm}
      \begin{multline}
          \sum_{i, j \neq 0}\left(\left\langle\widehat{v}_{i}, \widehat{u}_{j}\right\rangle_{\bfOmega}\langle\varphi_{i}, \partial_{t}\phi_{j}\rangle_{T^{n}} + \frac{1}{\rmPe}\left\langle\nabla\widehat{v}_{i}, \nabla\widehat{u}_{j}\right\rangle_{\bfOmega}\langle\varphi_{i}, \phi_{j}\rangle_{T^{n}}\right)  \\
          =  - \sum_{i}\left(\left\langle\widehat{v}_{i}, \widehat{u}_{0}\right\rangle_{\bfOmega}\langle\varphi_{i}, \partial_{t}\phi_{0}\rangle_{T^{n}} + \frac{1}{\rmPe}\left\langle\nabla\widehat{v}_{i}, \nabla\widehat{u}_{0}\right\rangle_{\bfOmega}\langle\varphi_{i}, \phi_{0}\rangle_{T^{n}}\right)
      \end{multline}
      The final condition $u|_{t^{n + 1}}$ is then given simply as $u|_{t^{n + 1}}  =  \widehat{u}_{s}$. This quantifies the CG-DG-in-time timestepper.
    \end{example}

    In the case of $s  =  1$---the lowest-order method---one seeks $u|_{t^{n + 1}}  \in  \widehat{\calU}$, such that $\forall \widehat{v}  \in  \widehat{\calU}$,
    \begin{equation}
        \left\langle\widehat{v}, u|_{t^{n + 1}}\right\rangle_{\bfOmega} + \frac{\delta t^{n}}{2\rmPe}\left\langle\nabla\widehat{v}, \nabla u|_{t^{n + 1}}\right\rangle_{\bfOmega}  =  \left\langle\widehat{v}, u|_{t^{n}}\right\rangle_{\bfOmega} - \frac{\delta t^{n}}{2\rmPe}\left\langle\nabla\widehat{v}, \nabla u|_{t^{n}}\right\rangle_{\bfOmega}.
    \end{equation}
    By rearranging, this can be seen to be equivalent to the traditional Crank--Nicolson method,
    \begin{equation}
        \left\langle\widehat{v}, \frac{1}{\delta t}\left(u|_{t^{n + 1}} - u|_{t^{n}}\right)\right\rangle_{\bfOmega}  =  - \frac{1}{\rmPe}\left\langle\nabla\widehat{v}, \nabla\left[\frac{1}{2}\left(u|_{t^{n + 1}} + u|_{t^{n}}\right)\right]\right\rangle_{\bfOmega}.
    \end{equation}
    For $s  >  1$---at higher order---the resultant CG-DG-in-time timesteppers are here equivalent to the other higher-order GL methods. As stated above, these are all known to be exactly dissipative for the heat equation.
    
    Despite the fact that the use of FET implies $u$ is continuously defined over $\bfOmega\otimes T^{n}$, the dissipation in $\rmE$ is \emph{not} that which might be assumed from (\ref{eqn:heat equation dissipation}),
    \begin{equation}
        \frac{1}{\rmPe}\int_{\bfOmega\otimes T^{n}}\|\nabla u\|^{2},
    \end{equation}
    but a seemingly related value, e.g. as stated in (\ref{eqn:heat equation dissipation implicit midpoint}) when $s = 1$,
    \begin{equation}
        \frac{\delta t^{n}}{\rmPe}\int_{\bfOmega}\left\|\nabla\left[\frac{1}{2}\left(u|_{t^{n + 1}} + u|_{t^{n}}\right)\right]\right\|^{2}
    \end{equation}
    With sufficient regularity on $u$, these two dissipations can be seen to be asymptotically close, as
    \begin{equation}
      \frac{\delta t^{n}}{\rmPe}\int_{\bfOmega}\left\|\nabla\left[\frac{1}{2}\left(u|_{t^{n + 1}} + u|_{t^{n}}\right)\right]\right\|^{2}  =  \frac{1}{\rmPe}\int_{\bfOmega\otimes T^{n}}\|\nabla u\|^{2} + \calO\left[\frac{\left(\delta t^{n}\right)^{2}}{\rmPe}\right].
    \end{equation}
    In fact, again with sufficient regularity on $u$, the energy dissipation from the $s$-step GL method is off from the true value, $1/\rmPe\cdot\int_{\bfOmega\otimes T^{n}}\|\nabla u\|^{2}$, by factor on the order of $\calO\left[\left(\delta t^{n}\right)^{s + 1}\!\!/\rmPe\right]$.
    
    \shortline

    We can make sense of this through an auxiliary space reformulation, that will prove useful for the construction of SP timesteppers in the more complicated, non-linear systems yet to come.

    \shortline

    Let us first take a step back, and consider an alternative weak form of the heat equation, where one seeks $u_{-}  \in  \calU_{-}$ such that
    \begin{equation}\label{eqn:heat equation auxiliary weak form}
        (\partial_{t}u  =)  \bbQ_{\partial_{t}\calU}[\partial_{t}u]  =  \frac{1}{\rmPe}\bbQ_{\partial_{t}\calU}[\Delta\bbQ_{\partial_{t}\calU}[u]].
    \end{equation}
    This is modified slightly from (\ref{eqn:heat equation weak form}) by the introduction of the new $\bbQ_{\partial_{t}\calU}$ projection operator. This can be cast into a variational formulation by the introduction of an \emph{``auxiliary''} variable $\widetilde{u}  =  \bbQ_{\partial_{t}\calU}[u]$. One seeks $u_{-}  \in  \calU_{-}$, $\widetilde{u}  \in  \partial_{t}\calU$, such that:
    \begin{align}
        &\forall  \widetilde{v}  \in  \partial_{t}\calU,  &  \langle\widetilde{v}, \partial_{t}u\rangle  &=  - \frac{1}{\rmPe}\langle\nabla\widetilde{v}, \nabla\widetilde{u}\rangle  \label{eqn:heat equation auxiliary variational form 1}  \\
        &\forall              v  \in  \partial_{t}\calU,  &                                0  &=  \langle v, \widetilde{u}\rangle - \langle v, u\rangle  \label{eqn:heat equation auxiliary variational form 2}
    \end{align}
    where all inner products $\langle -, -\rangle  =  \langle -, -\rangle_{\bfOmega\otimes T^{n}}$ are taken over the space-time domain, $\bfOmega\otimes T^{n}$.
    
    When $\calU$ is a tensor product space of one in space and time, as is the case in the above example, (linear) operators in space, such as $\Delta$, and time, such as $\bbQ_{\partial_{t}\calU}$ when $\calU$ is such a tensor product space, commute. Thus when $\calU$ is such a tensor product space, (\ref{eqn:heat equation auxiliary weak form}) and (\ref{eqn:heat equation weak form}) are equivalent.

    One can use the weak form (\ref{eqn:heat equation auxiliary weak form}) directly to mimic the proof of energy dissipation in the continuous, strong formulation, to show this weak/variational formulations preserve the energy-dissipative structure of the continuous problem:
    \begin{align}
        \rmE\left(t^{n + 1}\right) - \rmE\left(t^{n}\right)  &=  \int_{T^{n}}\frac{\rmd}{\rmd t}\rmE  \\
        &=  \int_{T^{n}}\frac{\rmd}{\rmd t}\left[\frac{1}{2}\int_{\bfOmega}u^{2}\right]  \\
        &=  \int_{\bfOmega\otimes T^{n}}u\partial_{t}u  \\
        &=  \frac{1}{\rmPe}\int_{\bfOmega\otimes T^{n}}u\bbQ_{\partial_{t}\calU}[\Delta\bbQ_{\partial_{t}\calU}[u]]  \\
        &=  \frac{1}{\rmPe}\int_{\bfOmega\otimes T^{n}}\bbQ_{\partial_{t}\calU}[u]\Delta\bbQ_{\partial_{t}\calU}[u]  \\
        &=  - \frac{1}{\rmPe}\int_{\bfOmega\otimes T^{n}}\|\nabla\bbQ_{\partial_{t}\calU}[u]\|^{2}  \left(=  - \frac{1}{\rmPe}\int_{\bfOmega\otimes T^{n}}\|\nabla\widetilde{u}\|^{2}\right)  \leq  0
    \end{align}
    A similar proof follows from the equivalent variational formulation (\ref{eqn:heat equation auxiliary variational form 1}--\ref{eqn:heat equation auxiliary variational form 2}).

    This approach both:
    \begin{itemize}
        \item  Proves the energy-dissipation SP property both for the GL/CG-DG-in-time timestepper as constructed above, and a wider class of timesteppers for arbitrary $\calU$. 
        \item  Explains the discrepancy between the energy dissipation and $1/\rmPe\cdot\int_{\bfOmega\otimes T^{n}}\|\nabla u\|^{2}$, since the dissipation is exactly this (potentially semi-)norm on the \emph{auxiliary} function, $\widetilde{u}  =  \bbQ_{\partial_{t}\calU}[u]$. For tensor product $\calU$, one can in fact write
        \begin{equation}
            \rmE\left(t^{n + 1}\right) - \rmE\left(t^{n}\right)  =  - \frac{1}{\rmPe}\int_{\bfOmega\otimes T^{n}}\|\nabla u\|^{2} + \frac{1}{\rmPe}\int_{\bfOmega\otimes T^{n}}\|\nabla(\id - \bbQ_{\partial_{t}\calU})[u]\|^{2},
        \end{equation}
        such that as the order in time increases and (informally) $\bbQ_{\partial_{t}\calU}  \rightarrow  \id$,
        \begin{equation}
            \rmE\left(t^{n + 1}\right) - \rmE\left(t^{n}\right)  \rightarrow  - \frac{1}{\rmPe}\int_{\bfOmega\otimes T^{n}}\|\nabla u\|^{2}.
        \end{equation}
        One can thus approach the question of quantifying dissipation through the well-studied language of bounds on these projection operators.
    \end{itemize}
    