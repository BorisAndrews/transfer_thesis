\section{Ensuring Structure Preservation in Variational/Weak Formulations}
    \BA{Introduction.}

    Consider a time-dependent PDE (or system of PDEs) in the general strong form
    \begin{equation}\label{eqn:general time-dependent PDE}
        \bfzero  =  \bfF\left[\bfu|_{\bfx; t}, \nabla\bfu|_{\bfx; t}, \nabla^{2}\bfu|_{\bfx; t}, \cdots; \partial_{t}\bfu|_{\bfx; t}\right](\bfx; t),
    \end{equation}
    alongside appropriate boundary conditions. The physical origins of such systems often imply the existence of certain functionals $\rmE[\bfu|_{t}](t)$ that are conserved/dissipated for time, i.e. such that $\forall  t$ either $\frac{\rmd}{\rmd\rmt}\rmE  =  0$ or $\frac{\rmd}{\rmd\rmt}\rmE  \leq  0$.
    
    \line
    
    Other than the energies and helicities considered above for the Hall MHD system, other classical examples one might consider include:
    \begin{example}[Heat equation]
        The heat equation,
        \begin{equation}\label{eqn:heat equation}
            \partial_{t}u  =  \frac{1}{\rmPe}\Delta u,
        \end{equation}
        under homogeneous Dirichlet or Neumann boundary conditions, where the energy $\rmE(t)  :=  \int_{\bfOmega}u^{2}$ is dissipated according to
        \begin{equation}\label{eqn:heat equation dissipation}
            \frac{\rmd}{\rmd\rmt}\rmE  =  - \frac{1}{\rmPe}\|\nabla u\|^{2}.
        \end{equation}
    \end{example}
    
    \begin{example}[Hamiltonian systems]
        Hamiltonian systems with Hamiltonian $\calH[p|_{t}, q|_{t}](t)$:
        \begin{align}
            \partial_{t}p  =  - \nabla_{q|_{t}}\calH,  &&
            \partial_{t}q  =  + \nabla_{p|_{t}}\calH,
        \end{align}
        for $L^{2}$ functional derivatives $\nabla_{p|_{t}}$, $\nabla_{q|_{t}}$, on $\bbR^{d}$, which exactly conserve the Hamiltonian,
        \begin{equation}
            \frac{\rmd}{\rmd\rmt}\calH  =  0,
        \end{equation}
        such as:
        \begin{itemize}
            \item  The wave equation,
            \begin{align}
                \partial_{t}v  =  c^{2}\Delta u,  &&
                \partial_{t}u  =  v,
            \end{align}
            from the Hamiltonian
            \begin{equation}
                \calH(t)  :=  c^{2}\|\nabla u\|_{\bfOmega}^{2} + \|v\|_{\bfOmega}^{2}.
            \end{equation}
            \item  The Schrödinger equation,
            \begin{equation}
                i\hbar\partial_{t}\Psi  =  \left[- \frac{\hbar^{2}}{2m}\Delta + V\right]\Psi,
            \end{equation}
            or, from $\Psi  =  q + ip$,
            \begin{align}
                \partial_{t}p  =  - \frac{1}{\hbar}\left[- \frac{\hbar^{2}}{2m}\Delta + V\right]q,  &&
                \partial_{t}q  =  + \frac{1}{\hbar}\left[- \frac{\hbar^{2}}{2m}\Delta + V\right]p,
            \end{align}
            from the Hamiltonian
            \begin{equation}
                \calH(t)  :=  \frac{1}{\hbar}\int_{\bfOmega}\left[\frac{\hbar^{2}}{2m}\left(\|\nabla p\|^{2} + \|\nabla q\|^{2}\right) + V\left(p^{2} + q^{2}\right)\right],
            \end{equation}
            or
            \begin{equation}
                \calH(t)  :=  \frac{1}{\hbar}\int_{\bfOmega}\left[\frac{\hbar^{2}}{2m}\|\nabla\Psi\|^{2} + V|\Psi|^{2}\right].
            \end{equation}
        \end{itemize}
    \end{example}
    \line
    
    To apply the the finite-element method (FEM) to such a system, it must be cast into a variational/weak form.
    
    After discretizing via the Galerkin(/Petrov–Galerkin) method, the resulting discretization can be referred to as a ``timestepper'' if there exist timesteps $(0  <)  t^{1}  <  t^{2}  <  \cdots$ such that the following are equivalent for $m  <  n$:
    \begin{itemize}
        \item  The solution on the time interval $\left(0, t^{m}\right]$.
        \item  The solution on the time interval $\left(0, t^{n}\right]$, restricted to $\left(0, t^{m}\right]$.
    \end{itemize}
    That is to say, the discretization can be solved iteratively on each timestep $t^{1}$, $t^{2}$, etc. Naturally, this is preferable to solving a time-coupled system simultaneously over the higher-dimension space-time domain.
    
    When constructing timesteppers for systems with conserved/dissipated quantities, it is natural to seek ones that preserve these conservational/dissipative structures. That is to say that for $m  <  n$, with $\rmE^{n}  :=  \rmE[\bfu|_{t^{n}}]$, either $\rmE[\bfu|_{t^{n}}]  =  \rmE[\bfu|_{t^{m}}]$ or $\rmE[\bfu|_{t^{n}}]  \leq  \rmE[\bfu|_{t^{m}}]$, and potentially further with analogous, quantifiable differences in the dissipative case. There are many reasons one might seek to do this:
    \begin{itemize}
        \item  The physical origins for such quantities often imply informative qualitative results for the behavior of the system (such as the natural dissipation of energy in the heat equation). \BA{(Talk about the topological meaning of magnetic helicity in MHD. Particularly important for solar simulations apparently- Is that something to do with the high $\beta$?)}
        \item  \BA{Often $L^{2}$-norm-like nature often imply nice quantitative bounds, e.g. from—again—the energy in the heat equation. Useful for proving existence/well-posedness/convergence.}
        \item  \BA{More reasons! [Ref, ...]}
    \end{itemize}



    \section{Preserved Structures}
    \BA{Introduction.}
    
    Consider first those quantities that are conserved by the transient system, so as to seek discretisations which better represent the physical behaviour of the system by \emph{also} conserved these quantities. 
    
    \cite{LHF22} considers conservation of the following 3 quantities, which the authors define in the incompressible case as: \BA{(Oops I've never defined $\bfA$! That should probably be in the introduction...)}
    \begin{center}\begin{tabular}{ c c c }
        Properties  &  Symbol  &  Definition  \\
        \hline\hline
        Energy  &  $\rmE$  &  $\int_{\bfOmega}\left[\frac{1}{\rmEu\rho}\|\bfp\|^{2} + p + \frac{1}{\beta}\|\bfB\|^{2}\right]$  \\
        Magnetic helicity  &  $\rmH_{\rmM}$  &  $\int_{\bfOmega}\bfA\cdot\bfB$  \\
        Hybrid helicity  &  $\rmH_{\rmH}$  &  $\int_{\bfOmega}(a\bfA + \bfp)\cdot(b\bfB + \nabla\wedge\bfp)$
    \end{tabular}\end{center}
    where $a$, $b$ satisfy the relation $a + b  =  \frac{4}{\beta\rmRH}$. \BA{(What do these represent \emph{physically}? Diagrams!)} Taking the derivatives of these quantities over time (still in the incompressible system) gives
    \begin{align}
        \frac{d\rmE}{dt}  &=  \BA{\cdots}  \\
        \frac{d\rmH_{\rmM}}{dt}  &=  \int_{\bfGamma}(- \varphi\bfB + \bfA\wedge\bfE)\cdot\bfn - \frac{2}{\rmRem}\int_{\bfOmega}\bfB\cdot\bfj  \\
        \frac{d\rmH_{\rmH}}{dt}  &=  \BA{\cdots} \\
    \end{align}

    \BA{Proven that in the \emph{compressible} case, $\frac{d\rmE}{dt}$ evaluates as
    {\small \begin{equation}
        \frac{d\rmE}{dt}  =  \int_{\bfGamma}\left[- \frac{1}{2\rmEu\rho}\|\bfp\|^{2}\bfp - \frac{p}{2\rho}\bfp + \frac{1}{\rmEu\rmRe_{f}}\nabla\left[\frac{1}{\rho}\bfp\right]\cdot\frac{1}{\rho}\bfp - \frac{p}{2\rho}\bfp + \frac{1}{2\rmPe}\nabla\left[\frac{p}{\rho} + \frac{1}{\beta}\bfB\wedge\bfE\right]\right]\cdot\bfn
    \end{equation}}}
    
    \section{Preserved Structures}
    \BA{Introduction.}
    
    Consider first those quantities that are conserved by the transient system, so as to seek discretisations which better represent the physical behaviour of the system by \emph{also} conserved these quantities. 
    
    \cite{LHF22} considers conservation of the following 3 quantities, which the authors define in the incompressible case as: \BA{(Oops I've never defined $\bfA$! That should probably be in the introduction...)}
    \begin{center}\begin{tabular}{ c c c }
        Properties  &  Symbol  &  Definition  \\
        \hline\hline
        Energy  &  $\rmE$  &  $\int_{\bfOmega}\left[\frac{1}{\rmEu\rho}\|\bfp\|^{2} + p + \frac{1}{\beta}\|\bfB\|^{2}\right]$  \\
        Magnetic helicity  &  $\rmH_{\rmM}$  &  $\int_{\bfOmega}\bfA\cdot\bfB$  \\
        Hybrid helicity  &  $\rmH_{\rmH}$  &  $\int_{\bfOmega}(a\bfA + \bfp)\cdot(b\bfB + \nabla\wedge\bfp)$
    \end{tabular}\end{center}
    where $a$, $b$ satisfy the relation $a + b  =  \frac{4}{\beta\rmRH}$. \BA{(What do these represent \emph{physically}? Diagrams!)} Taking the derivatives of these quantities over time (still in the incompressible system) gives
    \begin{align}
        \frac{d\rmE}{dt}  &=  \BA{\cdots}  \\
        \frac{d\rmH_{\rmM}}{dt}  &=  \int_{\bfGamma}(- \varphi\bfB + \bfA\wedge\bfE)\cdot\bfn - \frac{2}{\rmRem}\int_{\bfOmega}\bfB\cdot\bfj  \\
        \frac{d\rmH_{\rmH}}{dt}  &=  \BA{\cdots} \\
    \end{align}

    \BA{Proven that in the \emph{compressible} case, $\frac{d\rmE}{dt}$ evaluates as
    {\small \begin{equation}
        \frac{d\rmE}{dt}  =  \int_{\bfGamma}\left[- \frac{1}{2\rmEu\rho}\|\bfp\|^{2}\bfp - \frac{p}{2\rho}\bfp + \frac{1}{\rmEu\rmRe_{f}}\nabla\left[\frac{1}{\rho}\bfp\right]\cdot\frac{1}{\rho}\bfp - \frac{p}{2\rho}\bfp + \frac{1}{2\rmPe}\nabla\left[\frac{p}{\rho} + \frac{1}{\beta}\bfB\wedge\bfE\right]\right]\cdot\bfn
    \end{equation}}}
    
