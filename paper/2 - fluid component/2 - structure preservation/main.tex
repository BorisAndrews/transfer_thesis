\section{Ensuring Structure Preservation in Weak Formulations}
    \BA{Introduction.}

    Consider a PDE (or potentially system of PDEs) in the general strong form
    \begin{equation}
        \bfzero  =  \bfF\left[\bfu|_{\bfx; t}, \nabla\bfu|_{\bfx; t}, \nabla^{2}\bfu|_{\bfx; t}, \cdots; \partial_{t}\bfu|_{\bfx; t}\right](\bfx; t),
    \end{equation}
    which, alongside certain boundary conditions, is known to conserve/dissipate certain functionals $\rmE_{i}[\bfu|_{t}](t)$, such that $\forall  t$ either $\frac{\rmd}{\rmd\rmt}\rmE_{i}  =  0$ or $\frac{\rmd}{\rmd\rmt}\rmE_{i}  \leq  0$.

    Other than the energies and helicities considered above for compressible Hall MHD system, other classical examples one might consider include:
    
    \begin{example}
        The heat equation,
        \begin{equation}
            \partial_{t}u  =  \frac{1}{\rmPe}\Delta u,
        \end{equation}
        under homogeneous Dirichlet or Neumann boundary conditions, where the energy $\rmE(t)  :=  \int_{\bfOmega}u^{2}$ is dissipated according to
        \begin{equation}
            \frac{\rmd}{\rmd\rmt}\rmE  =  - \frac{1}{\rmPe}\|\nabla u\|^{2}.
        \end{equation}
    \end{example}
    
    \begin{example}
        Hamiltonian systems with Hamiltonian $\calH[p|_{t}, q|_{t}](t)$:
        \begin{align}
            \partial_{t}p  =  - \nabla_{q|_{t}}\calH,  &&
            \partial_{t}q  =  + \nabla_{p|_{t}}\calH,
        \end{align}
        for $L^{2}$ functional derivatives $\nabla_{p|_{t}}$, $\nabla_{q|_{t}}$, on $\bbR^{d}$, which exactly conserve the Hamiltonian,
        \begin{equation}
            \frac{\rmd}{\rmd\rmt}\calH  =  0,
        \end{equation}
        such as:
        \begin{itemize}
            \item  The wave equation,
            \begin{align}
                \partial_{t}v  =  c^{2}\Delta u,  &&
                \partial_{t}u  =  v,
            \end{align}
            from the Hamiltonian
            \begin{equation}
                \calH(t)  :=  c^{2}\|\nabla u\|_{\bfOmega}^{2} + \|v\|_{\bfOmega}^{2}.
            \end{equation}
            \item  The Schrödinger equation,
            \begin{equation}
                i\hbar\partial_{t}\Psi  =  \left[- \frac{\hbar^{2}}{2m}\Delta + V\right]\Psi
            \end{equation}
            or, from $\Psi  =  q + ip$,
            \begin{align}
                \partial_{t}p  =  - \frac{1}{\hbar}\left[- \frac{\hbar^{2}}{2m}\Delta + V\right]q,  &&
                \partial_{t}q  =  + \frac{1}{\hbar}\left[- \frac{\hbar^{2}}{2m}\Delta + V\right]p,
            \end{align}
            from the Hamiltonian
            \begin{equation}
                \calH(t)  :=  \frac{1}{\hbar}\int_{\bfOmega}\left[\frac{\hbar^{2}}{2m}\left(\|\nabla p\|^{2} + \|\nabla q\|^{2}\right) + V\left(p^{2} + q^{2}\right)\right],
            \end{equation}
            or
            \begin{equation}
                \calH(t)  :=  \frac{1}{\hbar}\int_{\bfOmega}\left[\frac{\hbar^{2}}{2m}\|\nabla\Psi\|^{2} + V|\Psi|^{2}\right].
            \end{equation}
        \end{itemize}
    \end{example}


    \section{Preserved Structures}
    \BA{Introduction.}
    
    Consider first those quantities that are conserved by the transient system, so as to seek discretisations which better represent the physical behaviour of the system by \emph{also} conserved these quantities. 
    
    \cite{LHF22} considers conservation of the following 3 quantities, which the authors define in the incompressible case as: \BA{(Oops I've never defined $\bfA$! That should probably be in the introduction...)}
    \begin{center}\begin{tabular}{ c c c }
        Properties  &  Symbol  &  Definition  \\
        \hline\hline
        Energy  &  $\rmE$  &  $\int_{\bfOmega}\left[\frac{1}{\rmEu\rho}\|\bfp\|^{2} + p + \frac{1}{\beta}\|\bfB\|^{2}\right]$  \\
        Magnetic helicity  &  $\rmH_{\rmM}$  &  $\int_{\bfOmega}\bfA\cdot\bfB$  \\
        Hybrid helicity  &  $\rmH_{\rmH}$  &  $\int_{\bfOmega}(a\bfA + \bfp)\cdot(b\bfB + \nabla\wedge\bfp)$
    \end{tabular}\end{center}
    where $a$, $b$ satisfy the relation $a + b  =  \frac{4}{\beta\rmRH}$. \BA{(What do these represent \emph{physically}? Diagrams!)} Taking the derivatives of these quantities over time (still in the incompressible system) gives
    \begin{align}
        \frac{d\rmE}{dt}  &=  \BA{\cdots}  \\
        \frac{d\rmH_{\rmM}}{dt}  &=  \int_{\bfGamma}(- \varphi\bfB + \bfA\wedge\bfE)\cdot\bfn - \frac{2}{\rmRem}\int_{\bfOmega}\bfB\cdot\bfj  \\
        \frac{d\rmH_{\rmH}}{dt}  &=  \BA{\cdots} \\
    \end{align}

    \BA{Proven that in the \emph{compressible} case, $\frac{d\rmE}{dt}$ evaluates as
    {\small \begin{equation}
        \frac{d\rmE}{dt}  =  \int_{\bfGamma}\left[- \frac{1}{2\rmEu\rho}\|\bfp\|^{2}\bfp - \frac{p}{2\rho}\bfp + \frac{1}{\rmEu\rmRe_{f}}\nabla\left[\frac{1}{\rho}\bfp\right]\cdot\frac{1}{\rho}\bfp - \frac{p}{2\rho}\bfp + \frac{1}{2\rmPe}\nabla\left[\frac{p}{\rho} + \frac{1}{\beta}\bfB\wedge\bfE\right]\right]\cdot\bfn
    \end{equation}}}
    
    \section{Preserved Structures}
    \BA{Introduction.}
    
    Consider first those quantities that are conserved by the transient system, so as to seek discretisations which better represent the physical behaviour of the system by \emph{also} conserved these quantities. 
    
    \cite{LHF22} considers conservation of the following 3 quantities, which the authors define in the incompressible case as: \BA{(Oops I've never defined $\bfA$! That should probably be in the introduction...)}
    \begin{center}\begin{tabular}{ c c c }
        Properties  &  Symbol  &  Definition  \\
        \hline\hline
        Energy  &  $\rmE$  &  $\int_{\bfOmega}\left[\frac{1}{\rmEu\rho}\|\bfp\|^{2} + p + \frac{1}{\beta}\|\bfB\|^{2}\right]$  \\
        Magnetic helicity  &  $\rmH_{\rmM}$  &  $\int_{\bfOmega}\bfA\cdot\bfB$  \\
        Hybrid helicity  &  $\rmH_{\rmH}$  &  $\int_{\bfOmega}(a\bfA + \bfp)\cdot(b\bfB + \nabla\wedge\bfp)$
    \end{tabular}\end{center}
    where $a$, $b$ satisfy the relation $a + b  =  \frac{4}{\beta\rmRH}$. \BA{(What do these represent \emph{physically}? Diagrams!)} Taking the derivatives of these quantities over time (still in the incompressible system) gives
    \begin{align}
        \frac{d\rmE}{dt}  &=  \BA{\cdots}  \\
        \frac{d\rmH_{\rmM}}{dt}  &=  \int_{\bfGamma}(- \varphi\bfB + \bfA\wedge\bfE)\cdot\bfn - \frac{2}{\rmRem}\int_{\bfOmega}\bfB\cdot\bfj  \\
        \frac{d\rmH_{\rmH}}{dt}  &=  \BA{\cdots} \\
    \end{align}

    \BA{Proven that in the \emph{compressible} case, $\frac{d\rmE}{dt}$ evaluates as
    {\small \begin{equation}
        \frac{d\rmE}{dt}  =  \int_{\bfGamma}\left[- \frac{1}{2\rmEu\rho}\|\bfp\|^{2}\bfp - \frac{p}{2\rho}\bfp + \frac{1}{\rmEu\rmRe_{f}}\nabla\left[\frac{1}{\rho}\bfp\right]\cdot\frac{1}{\rho}\bfp - \frac{p}{2\rho}\bfp + \frac{1}{2\rmPe}\nabla\left[\frac{p}{\rho} + \frac{1}{\beta}\bfB\wedge\bfE\right]\right]\cdot\bfn
    \end{equation}}}
    
