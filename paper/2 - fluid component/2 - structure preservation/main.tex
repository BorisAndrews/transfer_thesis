\section{Ensuring Structure Preservation in Variational/Weak Formulations}
    \BA{Introduction.}

    Consider a time-dependent PDE (or system of PDEs) in the general strong form
    \begin{equation}\label{eqn:general time-dependent PDE}
        \bfzero  =  \bfF\left[\bfu|_{\bfx; t}, \nabla\bfu|_{\bfx; t}, \nabla^{2}\bfu|_{\bfx; t}, \cdots; \partial_{t}\bfu|_{\bfx; t}\right](\bfx; t),
    \end{equation}
    alongside appropriate boundary conditions. The physical origins of such systems often imply the existence of certain functionals $\rmE[\bfu|_{t}](t)$ that are conserved/dissipated for time, i.e. such that $\forall  t$ either $\frac{\rmd}{\rmd\rmt}\rmE  =  0$ or $\frac{\rmd}{\rmd\rmt}\rmE  \leq  0$.
    
    \line
    
    Other than the energies and helicities considered above for the Hall MHD system, other classical examples one might consider include:
    \begin{example}[Heat equation]
        The heat equation,
        \begin{equation}
            \partial_{t}u  =  \frac{1}{\rmPe}\Delta u,
        \end{equation}
        under homogeneous Dirichlet or Neumann boundary conditions, where the energy $\rmE(t)  :=  \int_{\bfOmega}u^{2}$ is dissipated according to
        \begin{equation}
            \frac{\rmd}{\rmd\rmt}\rmE  =  - \frac{1}{\rmPe}\|\nabla u\|^{2}.
        \end{equation}
    \end{example}
    
    \begin{example}[Hamiltonian systems]
        Hamiltonian systems with Hamiltonian $\calH[p|_{t}, q|_{t}](t)$:
        \begin{align}
            \partial_{t}p  =  - \nabla_{q|_{t}}\calH,  &&
            \partial_{t}q  =  + \nabla_{p|_{t}}\calH,
        \end{align}
        for $L^{2}$ functional derivatives $\nabla_{p|_{t}}$, $\nabla_{q|_{t}}$, on $\bbR^{d}$, which exactly conserve the Hamiltonian,
        \begin{equation}
            \frac{\rmd}{\rmd\rmt}\calH  =  0,
        \end{equation}
        such as:
        \begin{itemize}
            \item  The wave equation,
            \begin{align}
                \partial_{t}v  =  c^{2}\Delta u,  &&
                \partial_{t}u  =  v,
            \end{align}
            from the Hamiltonian
            \begin{equation}
                \calH(t)  :=  c^{2}\|\nabla u\|_{\bfOmega}^{2} + \|v\|_{\bfOmega}^{2}.
            \end{equation}
            \item  The Schrödinger equation,
            \begin{equation}
                i\hbar\partial_{t}\Psi  =  \left[- \frac{\hbar^{2}}{2m}\Delta + V\right]\Psi,
            \end{equation}
            or, from $\Psi  =  q + ip$,
            \begin{align}
                \partial_{t}p  =  - \frac{1}{\hbar}\left[- \frac{\hbar^{2}}{2m}\Delta + V\right]q,  &&
                \partial_{t}q  =  + \frac{1}{\hbar}\left[- \frac{\hbar^{2}}{2m}\Delta + V\right]p,
            \end{align}
            from the Hamiltonian
            \begin{equation}
                \calH(t)  :=  \frac{1}{\hbar}\int_{\bfOmega}\left[\frac{\hbar^{2}}{2m}\left(\|\nabla p\|^{2} + \|\nabla q\|^{2}\right) + V\left(p^{2} + q^{2}\right)\right],
            \end{equation}
            or
            \begin{equation}
                \calH(t)  :=  \frac{1}{\hbar}\int_{\bfOmega}\left[\frac{\hbar^{2}}{2m}\|\nabla\Psi\|^{2} + V|\Psi|^{2}\right].
            \end{equation}
        \end{itemize}
    \end{example}
    \line
    
    To apply the the finite-element method (FEM) to such a system, it must be cast into a variational/weak form.
    
    After discretizing via the Galerkin(/Petrov–Galerkin) method, the resulting discretization can be referred to as a ``timestepper'' if there exist timesteps $(0  <)  t^{1}  <  t^{2}  <  \cdots$ such that the following are equivalent for $m  <  n$:
    \begin{itemize}
        \item  The solution on the time interval $\left(0, t^{m}\right]$.
        \item  The solution on the time interval $\left(0, t^{n}\right]$, restricted to $\left(0, t^{m}\right]$.
    \end{itemize}
    That is to say, the discretization can be solved iteratively on each timestep $t^{1}$, $t^{2}$, etc. Naturally, this is preferable to solving a time-coupled system simultaneously over the higher-dimension space-time domain.
    
    When constructing timesteppers for systems with conserved/dissipated quantities, it is natural to seek ones that preserve these conservational/dissipative structures. That is to say that for $m  <  n$, with $\rmE^{n}  :=  \rmE[\bfu|_{t^{n}}]$, either $\rmE[\bfu|_{t^{n}}]  =  \rmE[\bfu|_{t^{m}}]$ or $\rmE[\bfu|_{t^{n}}]  \leq  \rmE[\bfu|_{t^{m}}]$, and potentially further with analogous, quantifiable differences in the dissipative case. There are many reasons one might seek to do this:
    \begin{itemize}
        \item  The physical origins for such quantities often imply informative qualitative results for the behavior of the system (such as the natural dissipation of energy in the heat equation). \BA{(Talk about the topological meaning of magnetic helicity in MHD. Particularly important for solar simulations apparently- Is that something to do with the high $\beta$?)}
        \item  \BA{Often $L^{2}$-norm-like nature often imply nice quantitative bounds, e.g. from—again—the energy in the heat equation. Useful for proving existence/well-posedness/convergence.}
        \item  \BA{More reasons! [Ref, ...]}
    \end{itemize}



    \documentclass[12pt, a4paper]{report}

\documentclass[12pt, a4paper]{report}

\documentclass[12pt, a4paper]{report}

\input{template/main.tex}

\title{\BA{Title in Progress...}}
\author{Boris Andrews}
\affil{Mathematical Institute, University of Oxford}
\date{\today}


\begin{document}
    \pagenumbering{gobble}
    \maketitle
    
    
    \begin{abstract}
        Magnetic confinement reactors---in particular tokamaks---offer one of the most promising options for achieving practical nuclear fusion, with the potential to provide virtually limitless, clean energy. The theoretical and numerical modeling of tokamak plasmas is simultaneously an essential component of effective reactor design, and a great research barrier. Tokamak operational conditions exhibit comparatively low Knudsen numbers. Kinetic effects, including kinetic waves and instabilities, Landau damping, bump-on-tail instabilities and more, are therefore highly influential in tokamak plasma dynamics. Purely fluid models are inherently incapable of capturing these effects, whereas the high dimensionality in purely kinetic models render them practically intractable for most relevant purposes.

        We consider a $\delta\!f$ decomposition model, with a macroscopic fluid background and microscopic kinetic correction, both fully coupled to each other. A similar manner of discretization is proposed to that used in the recent \texttt{STRUPHY} code \cite{Holderied_Possanner_Wang_2021, Holderied_2022, Li_et_al_2023} with a finite-element model for the background and a pseudo-particle/PiC model for the correction.

        The fluid background satisfies the full, non-linear, resistive, compressible, Hall MHD equations. \cite{Laakmann_Hu_Farrell_2022} introduces finite-element(-in-space) implicit timesteppers for the incompressible analogue to this system with structure-preserving (SP) properties in the ideal case, alongside parameter-robust preconditioners. We show that these timesteppers can derive from a finite-element-in-time (FET) (and finite-element-in-space) interpretation. The benefits of this reformulation are discussed, including the derivation of timesteppers that are higher order in time, and the quantifiable dissipative SP properties in the non-ideal, resistive case.
        
        We discuss possible options for extending this FET approach to timesteppers for the compressible case.

        The kinetic corrections satisfy linearized Boltzmann equations. Using a Lénard--Bernstein collision operator, these take Fokker--Planck-like forms \cite{Fokker_1914, Planck_1917} wherein pseudo-particles in the numerical model obey the neoclassical transport equations, with particle-independent Brownian drift terms. This offers a rigorous methodology for incorporating collisions into the particle transport model, without coupling the equations of motions for each particle.
        
        Works by Chen, Chacón et al. \cite{Chen_Chacón_Barnes_2011, Chacón_Chen_Barnes_2013, Chen_Chacón_2014, Chen_Chacón_2015} have developed structure-preserving particle pushers for neoclassical transport in the Vlasov equations, derived from Crank--Nicolson integrators. We show these too can can derive from a FET interpretation, similarly offering potential extensions to higher-order-in-time particle pushers. The FET formulation is used also to consider how the stochastic drift terms can be incorporated into the pushers. Stochastic gyrokinetic expansions are also discussed.

        Different options for the numerical implementation of these schemes are considered.

        Due to the efficacy of FET in the development of SP timesteppers for both the fluid and kinetic component, we hope this approach will prove effective in the future for developing SP timesteppers for the full hybrid model. We hope this will give us the opportunity to incorporate previously inaccessible kinetic effects into the highly effective, modern, finite-element MHD models.
    \end{abstract}
    
    
    \newpage
    \tableofcontents
    
    
    \newpage
    \pagenumbering{arabic}
    %\linenumbers\renewcommand\thelinenumber{\color{black!50}\arabic{linenumber}}
            \input{0 - introduction/main.tex}
        \part{Research}
            \input{1 - low-noise PiC models/main.tex}
            \input{2 - kinetic component/main.tex}
            \input{3 - fluid component/main.tex}
            \input{4 - numerical implementation/main.tex}
        \part{Project Overview}
            \input{5 - research plan/main.tex}
            \input{6 - summary/main.tex}
    
    
    %\section{}
    \newpage
    \pagenumbering{gobble}
        \printbibliography


    \newpage
    \pagenumbering{roman}
    \appendix
        \part{Appendices}
            \input{8 - Hilbert complexes/main.tex}
            \input{9 - weak conservation proofs/main.tex}
\end{document}


\title{\BA{Title in Progress...}}
\author{Boris Andrews}
\affil{Mathematical Institute, University of Oxford}
\date{\today}


\begin{document}
    \pagenumbering{gobble}
    \maketitle
    
    
    \begin{abstract}
        Magnetic confinement reactors---in particular tokamaks---offer one of the most promising options for achieving practical nuclear fusion, with the potential to provide virtually limitless, clean energy. The theoretical and numerical modeling of tokamak plasmas is simultaneously an essential component of effective reactor design, and a great research barrier. Tokamak operational conditions exhibit comparatively low Knudsen numbers. Kinetic effects, including kinetic waves and instabilities, Landau damping, bump-on-tail instabilities and more, are therefore highly influential in tokamak plasma dynamics. Purely fluid models are inherently incapable of capturing these effects, whereas the high dimensionality in purely kinetic models render them practically intractable for most relevant purposes.

        We consider a $\delta\!f$ decomposition model, with a macroscopic fluid background and microscopic kinetic correction, both fully coupled to each other. A similar manner of discretization is proposed to that used in the recent \texttt{STRUPHY} code \cite{Holderied_Possanner_Wang_2021, Holderied_2022, Li_et_al_2023} with a finite-element model for the background and a pseudo-particle/PiC model for the correction.

        The fluid background satisfies the full, non-linear, resistive, compressible, Hall MHD equations. \cite{Laakmann_Hu_Farrell_2022} introduces finite-element(-in-space) implicit timesteppers for the incompressible analogue to this system with structure-preserving (SP) properties in the ideal case, alongside parameter-robust preconditioners. We show that these timesteppers can derive from a finite-element-in-time (FET) (and finite-element-in-space) interpretation. The benefits of this reformulation are discussed, including the derivation of timesteppers that are higher order in time, and the quantifiable dissipative SP properties in the non-ideal, resistive case.
        
        We discuss possible options for extending this FET approach to timesteppers for the compressible case.

        The kinetic corrections satisfy linearized Boltzmann equations. Using a Lénard--Bernstein collision operator, these take Fokker--Planck-like forms \cite{Fokker_1914, Planck_1917} wherein pseudo-particles in the numerical model obey the neoclassical transport equations, with particle-independent Brownian drift terms. This offers a rigorous methodology for incorporating collisions into the particle transport model, without coupling the equations of motions for each particle.
        
        Works by Chen, Chacón et al. \cite{Chen_Chacón_Barnes_2011, Chacón_Chen_Barnes_2013, Chen_Chacón_2014, Chen_Chacón_2015} have developed structure-preserving particle pushers for neoclassical transport in the Vlasov equations, derived from Crank--Nicolson integrators. We show these too can can derive from a FET interpretation, similarly offering potential extensions to higher-order-in-time particle pushers. The FET formulation is used also to consider how the stochastic drift terms can be incorporated into the pushers. Stochastic gyrokinetic expansions are also discussed.

        Different options for the numerical implementation of these schemes are considered.

        Due to the efficacy of FET in the development of SP timesteppers for both the fluid and kinetic component, we hope this approach will prove effective in the future for developing SP timesteppers for the full hybrid model. We hope this will give us the opportunity to incorporate previously inaccessible kinetic effects into the highly effective, modern, finite-element MHD models.
    \end{abstract}
    
    
    \newpage
    \tableofcontents
    
    
    \newpage
    \pagenumbering{arabic}
    %\linenumbers\renewcommand\thelinenumber{\color{black!50}\arabic{linenumber}}
            \documentclass[12pt, a4paper]{report}

\input{template/main.tex}

\title{\BA{Title in Progress...}}
\author{Boris Andrews}
\affil{Mathematical Institute, University of Oxford}
\date{\today}


\begin{document}
    \pagenumbering{gobble}
    \maketitle
    
    
    \begin{abstract}
        Magnetic confinement reactors---in particular tokamaks---offer one of the most promising options for achieving practical nuclear fusion, with the potential to provide virtually limitless, clean energy. The theoretical and numerical modeling of tokamak plasmas is simultaneously an essential component of effective reactor design, and a great research barrier. Tokamak operational conditions exhibit comparatively low Knudsen numbers. Kinetic effects, including kinetic waves and instabilities, Landau damping, bump-on-tail instabilities and more, are therefore highly influential in tokamak plasma dynamics. Purely fluid models are inherently incapable of capturing these effects, whereas the high dimensionality in purely kinetic models render them practically intractable for most relevant purposes.

        We consider a $\delta\!f$ decomposition model, with a macroscopic fluid background and microscopic kinetic correction, both fully coupled to each other. A similar manner of discretization is proposed to that used in the recent \texttt{STRUPHY} code \cite{Holderied_Possanner_Wang_2021, Holderied_2022, Li_et_al_2023} with a finite-element model for the background and a pseudo-particle/PiC model for the correction.

        The fluid background satisfies the full, non-linear, resistive, compressible, Hall MHD equations. \cite{Laakmann_Hu_Farrell_2022} introduces finite-element(-in-space) implicit timesteppers for the incompressible analogue to this system with structure-preserving (SP) properties in the ideal case, alongside parameter-robust preconditioners. We show that these timesteppers can derive from a finite-element-in-time (FET) (and finite-element-in-space) interpretation. The benefits of this reformulation are discussed, including the derivation of timesteppers that are higher order in time, and the quantifiable dissipative SP properties in the non-ideal, resistive case.
        
        We discuss possible options for extending this FET approach to timesteppers for the compressible case.

        The kinetic corrections satisfy linearized Boltzmann equations. Using a Lénard--Bernstein collision operator, these take Fokker--Planck-like forms \cite{Fokker_1914, Planck_1917} wherein pseudo-particles in the numerical model obey the neoclassical transport equations, with particle-independent Brownian drift terms. This offers a rigorous methodology for incorporating collisions into the particle transport model, without coupling the equations of motions for each particle.
        
        Works by Chen, Chacón et al. \cite{Chen_Chacón_Barnes_2011, Chacón_Chen_Barnes_2013, Chen_Chacón_2014, Chen_Chacón_2015} have developed structure-preserving particle pushers for neoclassical transport in the Vlasov equations, derived from Crank--Nicolson integrators. We show these too can can derive from a FET interpretation, similarly offering potential extensions to higher-order-in-time particle pushers. The FET formulation is used also to consider how the stochastic drift terms can be incorporated into the pushers. Stochastic gyrokinetic expansions are also discussed.

        Different options for the numerical implementation of these schemes are considered.

        Due to the efficacy of FET in the development of SP timesteppers for both the fluid and kinetic component, we hope this approach will prove effective in the future for developing SP timesteppers for the full hybrid model. We hope this will give us the opportunity to incorporate previously inaccessible kinetic effects into the highly effective, modern, finite-element MHD models.
    \end{abstract}
    
    
    \newpage
    \tableofcontents
    
    
    \newpage
    \pagenumbering{arabic}
    %\linenumbers\renewcommand\thelinenumber{\color{black!50}\arabic{linenumber}}
            \input{0 - introduction/main.tex}
        \part{Research}
            \input{1 - low-noise PiC models/main.tex}
            \input{2 - kinetic component/main.tex}
            \input{3 - fluid component/main.tex}
            \input{4 - numerical implementation/main.tex}
        \part{Project Overview}
            \input{5 - research plan/main.tex}
            \input{6 - summary/main.tex}
    
    
    %\section{}
    \newpage
    \pagenumbering{gobble}
        \printbibliography


    \newpage
    \pagenumbering{roman}
    \appendix
        \part{Appendices}
            \input{8 - Hilbert complexes/main.tex}
            \input{9 - weak conservation proofs/main.tex}
\end{document}

        \part{Research}
            \documentclass[12pt, a4paper]{report}

\input{template/main.tex}

\title{\BA{Title in Progress...}}
\author{Boris Andrews}
\affil{Mathematical Institute, University of Oxford}
\date{\today}


\begin{document}
    \pagenumbering{gobble}
    \maketitle
    
    
    \begin{abstract}
        Magnetic confinement reactors---in particular tokamaks---offer one of the most promising options for achieving practical nuclear fusion, with the potential to provide virtually limitless, clean energy. The theoretical and numerical modeling of tokamak plasmas is simultaneously an essential component of effective reactor design, and a great research barrier. Tokamak operational conditions exhibit comparatively low Knudsen numbers. Kinetic effects, including kinetic waves and instabilities, Landau damping, bump-on-tail instabilities and more, are therefore highly influential in tokamak plasma dynamics. Purely fluid models are inherently incapable of capturing these effects, whereas the high dimensionality in purely kinetic models render them practically intractable for most relevant purposes.

        We consider a $\delta\!f$ decomposition model, with a macroscopic fluid background and microscopic kinetic correction, both fully coupled to each other. A similar manner of discretization is proposed to that used in the recent \texttt{STRUPHY} code \cite{Holderied_Possanner_Wang_2021, Holderied_2022, Li_et_al_2023} with a finite-element model for the background and a pseudo-particle/PiC model for the correction.

        The fluid background satisfies the full, non-linear, resistive, compressible, Hall MHD equations. \cite{Laakmann_Hu_Farrell_2022} introduces finite-element(-in-space) implicit timesteppers for the incompressible analogue to this system with structure-preserving (SP) properties in the ideal case, alongside parameter-robust preconditioners. We show that these timesteppers can derive from a finite-element-in-time (FET) (and finite-element-in-space) interpretation. The benefits of this reformulation are discussed, including the derivation of timesteppers that are higher order in time, and the quantifiable dissipative SP properties in the non-ideal, resistive case.
        
        We discuss possible options for extending this FET approach to timesteppers for the compressible case.

        The kinetic corrections satisfy linearized Boltzmann equations. Using a Lénard--Bernstein collision operator, these take Fokker--Planck-like forms \cite{Fokker_1914, Planck_1917} wherein pseudo-particles in the numerical model obey the neoclassical transport equations, with particle-independent Brownian drift terms. This offers a rigorous methodology for incorporating collisions into the particle transport model, without coupling the equations of motions for each particle.
        
        Works by Chen, Chacón et al. \cite{Chen_Chacón_Barnes_2011, Chacón_Chen_Barnes_2013, Chen_Chacón_2014, Chen_Chacón_2015} have developed structure-preserving particle pushers for neoclassical transport in the Vlasov equations, derived from Crank--Nicolson integrators. We show these too can can derive from a FET interpretation, similarly offering potential extensions to higher-order-in-time particle pushers. The FET formulation is used also to consider how the stochastic drift terms can be incorporated into the pushers. Stochastic gyrokinetic expansions are also discussed.

        Different options for the numerical implementation of these schemes are considered.

        Due to the efficacy of FET in the development of SP timesteppers for both the fluid and kinetic component, we hope this approach will prove effective in the future for developing SP timesteppers for the full hybrid model. We hope this will give us the opportunity to incorporate previously inaccessible kinetic effects into the highly effective, modern, finite-element MHD models.
    \end{abstract}
    
    
    \newpage
    \tableofcontents
    
    
    \newpage
    \pagenumbering{arabic}
    %\linenumbers\renewcommand\thelinenumber{\color{black!50}\arabic{linenumber}}
            \input{0 - introduction/main.tex}
        \part{Research}
            \input{1 - low-noise PiC models/main.tex}
            \input{2 - kinetic component/main.tex}
            \input{3 - fluid component/main.tex}
            \input{4 - numerical implementation/main.tex}
        \part{Project Overview}
            \input{5 - research plan/main.tex}
            \input{6 - summary/main.tex}
    
    
    %\section{}
    \newpage
    \pagenumbering{gobble}
        \printbibliography


    \newpage
    \pagenumbering{roman}
    \appendix
        \part{Appendices}
            \input{8 - Hilbert complexes/main.tex}
            \input{9 - weak conservation proofs/main.tex}
\end{document}

            \documentclass[12pt, a4paper]{report}

\input{template/main.tex}

\title{\BA{Title in Progress...}}
\author{Boris Andrews}
\affil{Mathematical Institute, University of Oxford}
\date{\today}


\begin{document}
    \pagenumbering{gobble}
    \maketitle
    
    
    \begin{abstract}
        Magnetic confinement reactors---in particular tokamaks---offer one of the most promising options for achieving practical nuclear fusion, with the potential to provide virtually limitless, clean energy. The theoretical and numerical modeling of tokamak plasmas is simultaneously an essential component of effective reactor design, and a great research barrier. Tokamak operational conditions exhibit comparatively low Knudsen numbers. Kinetic effects, including kinetic waves and instabilities, Landau damping, bump-on-tail instabilities and more, are therefore highly influential in tokamak plasma dynamics. Purely fluid models are inherently incapable of capturing these effects, whereas the high dimensionality in purely kinetic models render them practically intractable for most relevant purposes.

        We consider a $\delta\!f$ decomposition model, with a macroscopic fluid background and microscopic kinetic correction, both fully coupled to each other. A similar manner of discretization is proposed to that used in the recent \texttt{STRUPHY} code \cite{Holderied_Possanner_Wang_2021, Holderied_2022, Li_et_al_2023} with a finite-element model for the background and a pseudo-particle/PiC model for the correction.

        The fluid background satisfies the full, non-linear, resistive, compressible, Hall MHD equations. \cite{Laakmann_Hu_Farrell_2022} introduces finite-element(-in-space) implicit timesteppers for the incompressible analogue to this system with structure-preserving (SP) properties in the ideal case, alongside parameter-robust preconditioners. We show that these timesteppers can derive from a finite-element-in-time (FET) (and finite-element-in-space) interpretation. The benefits of this reformulation are discussed, including the derivation of timesteppers that are higher order in time, and the quantifiable dissipative SP properties in the non-ideal, resistive case.
        
        We discuss possible options for extending this FET approach to timesteppers for the compressible case.

        The kinetic corrections satisfy linearized Boltzmann equations. Using a Lénard--Bernstein collision operator, these take Fokker--Planck-like forms \cite{Fokker_1914, Planck_1917} wherein pseudo-particles in the numerical model obey the neoclassical transport equations, with particle-independent Brownian drift terms. This offers a rigorous methodology for incorporating collisions into the particle transport model, without coupling the equations of motions for each particle.
        
        Works by Chen, Chacón et al. \cite{Chen_Chacón_Barnes_2011, Chacón_Chen_Barnes_2013, Chen_Chacón_2014, Chen_Chacón_2015} have developed structure-preserving particle pushers for neoclassical transport in the Vlasov equations, derived from Crank--Nicolson integrators. We show these too can can derive from a FET interpretation, similarly offering potential extensions to higher-order-in-time particle pushers. The FET formulation is used also to consider how the stochastic drift terms can be incorporated into the pushers. Stochastic gyrokinetic expansions are also discussed.

        Different options for the numerical implementation of these schemes are considered.

        Due to the efficacy of FET in the development of SP timesteppers for both the fluid and kinetic component, we hope this approach will prove effective in the future for developing SP timesteppers for the full hybrid model. We hope this will give us the opportunity to incorporate previously inaccessible kinetic effects into the highly effective, modern, finite-element MHD models.
    \end{abstract}
    
    
    \newpage
    \tableofcontents
    
    
    \newpage
    \pagenumbering{arabic}
    %\linenumbers\renewcommand\thelinenumber{\color{black!50}\arabic{linenumber}}
            \input{0 - introduction/main.tex}
        \part{Research}
            \input{1 - low-noise PiC models/main.tex}
            \input{2 - kinetic component/main.tex}
            \input{3 - fluid component/main.tex}
            \input{4 - numerical implementation/main.tex}
        \part{Project Overview}
            \input{5 - research plan/main.tex}
            \input{6 - summary/main.tex}
    
    
    %\section{}
    \newpage
    \pagenumbering{gobble}
        \printbibliography


    \newpage
    \pagenumbering{roman}
    \appendix
        \part{Appendices}
            \input{8 - Hilbert complexes/main.tex}
            \input{9 - weak conservation proofs/main.tex}
\end{document}

            \documentclass[12pt, a4paper]{report}

\input{template/main.tex}

\title{\BA{Title in Progress...}}
\author{Boris Andrews}
\affil{Mathematical Institute, University of Oxford}
\date{\today}


\begin{document}
    \pagenumbering{gobble}
    \maketitle
    
    
    \begin{abstract}
        Magnetic confinement reactors---in particular tokamaks---offer one of the most promising options for achieving practical nuclear fusion, with the potential to provide virtually limitless, clean energy. The theoretical and numerical modeling of tokamak plasmas is simultaneously an essential component of effective reactor design, and a great research barrier. Tokamak operational conditions exhibit comparatively low Knudsen numbers. Kinetic effects, including kinetic waves and instabilities, Landau damping, bump-on-tail instabilities and more, are therefore highly influential in tokamak plasma dynamics. Purely fluid models are inherently incapable of capturing these effects, whereas the high dimensionality in purely kinetic models render them practically intractable for most relevant purposes.

        We consider a $\delta\!f$ decomposition model, with a macroscopic fluid background and microscopic kinetic correction, both fully coupled to each other. A similar manner of discretization is proposed to that used in the recent \texttt{STRUPHY} code \cite{Holderied_Possanner_Wang_2021, Holderied_2022, Li_et_al_2023} with a finite-element model for the background and a pseudo-particle/PiC model for the correction.

        The fluid background satisfies the full, non-linear, resistive, compressible, Hall MHD equations. \cite{Laakmann_Hu_Farrell_2022} introduces finite-element(-in-space) implicit timesteppers for the incompressible analogue to this system with structure-preserving (SP) properties in the ideal case, alongside parameter-robust preconditioners. We show that these timesteppers can derive from a finite-element-in-time (FET) (and finite-element-in-space) interpretation. The benefits of this reformulation are discussed, including the derivation of timesteppers that are higher order in time, and the quantifiable dissipative SP properties in the non-ideal, resistive case.
        
        We discuss possible options for extending this FET approach to timesteppers for the compressible case.

        The kinetic corrections satisfy linearized Boltzmann equations. Using a Lénard--Bernstein collision operator, these take Fokker--Planck-like forms \cite{Fokker_1914, Planck_1917} wherein pseudo-particles in the numerical model obey the neoclassical transport equations, with particle-independent Brownian drift terms. This offers a rigorous methodology for incorporating collisions into the particle transport model, without coupling the equations of motions for each particle.
        
        Works by Chen, Chacón et al. \cite{Chen_Chacón_Barnes_2011, Chacón_Chen_Barnes_2013, Chen_Chacón_2014, Chen_Chacón_2015} have developed structure-preserving particle pushers for neoclassical transport in the Vlasov equations, derived from Crank--Nicolson integrators. We show these too can can derive from a FET interpretation, similarly offering potential extensions to higher-order-in-time particle pushers. The FET formulation is used also to consider how the stochastic drift terms can be incorporated into the pushers. Stochastic gyrokinetic expansions are also discussed.

        Different options for the numerical implementation of these schemes are considered.

        Due to the efficacy of FET in the development of SP timesteppers for both the fluid and kinetic component, we hope this approach will prove effective in the future for developing SP timesteppers for the full hybrid model. We hope this will give us the opportunity to incorporate previously inaccessible kinetic effects into the highly effective, modern, finite-element MHD models.
    \end{abstract}
    
    
    \newpage
    \tableofcontents
    
    
    \newpage
    \pagenumbering{arabic}
    %\linenumbers\renewcommand\thelinenumber{\color{black!50}\arabic{linenumber}}
            \input{0 - introduction/main.tex}
        \part{Research}
            \input{1 - low-noise PiC models/main.tex}
            \input{2 - kinetic component/main.tex}
            \input{3 - fluid component/main.tex}
            \input{4 - numerical implementation/main.tex}
        \part{Project Overview}
            \input{5 - research plan/main.tex}
            \input{6 - summary/main.tex}
    
    
    %\section{}
    \newpage
    \pagenumbering{gobble}
        \printbibliography


    \newpage
    \pagenumbering{roman}
    \appendix
        \part{Appendices}
            \input{8 - Hilbert complexes/main.tex}
            \input{9 - weak conservation proofs/main.tex}
\end{document}

            \documentclass[12pt, a4paper]{report}

\input{template/main.tex}

\title{\BA{Title in Progress...}}
\author{Boris Andrews}
\affil{Mathematical Institute, University of Oxford}
\date{\today}


\begin{document}
    \pagenumbering{gobble}
    \maketitle
    
    
    \begin{abstract}
        Magnetic confinement reactors---in particular tokamaks---offer one of the most promising options for achieving practical nuclear fusion, with the potential to provide virtually limitless, clean energy. The theoretical and numerical modeling of tokamak plasmas is simultaneously an essential component of effective reactor design, and a great research barrier. Tokamak operational conditions exhibit comparatively low Knudsen numbers. Kinetic effects, including kinetic waves and instabilities, Landau damping, bump-on-tail instabilities and more, are therefore highly influential in tokamak plasma dynamics. Purely fluid models are inherently incapable of capturing these effects, whereas the high dimensionality in purely kinetic models render them practically intractable for most relevant purposes.

        We consider a $\delta\!f$ decomposition model, with a macroscopic fluid background and microscopic kinetic correction, both fully coupled to each other. A similar manner of discretization is proposed to that used in the recent \texttt{STRUPHY} code \cite{Holderied_Possanner_Wang_2021, Holderied_2022, Li_et_al_2023} with a finite-element model for the background and a pseudo-particle/PiC model for the correction.

        The fluid background satisfies the full, non-linear, resistive, compressible, Hall MHD equations. \cite{Laakmann_Hu_Farrell_2022} introduces finite-element(-in-space) implicit timesteppers for the incompressible analogue to this system with structure-preserving (SP) properties in the ideal case, alongside parameter-robust preconditioners. We show that these timesteppers can derive from a finite-element-in-time (FET) (and finite-element-in-space) interpretation. The benefits of this reformulation are discussed, including the derivation of timesteppers that are higher order in time, and the quantifiable dissipative SP properties in the non-ideal, resistive case.
        
        We discuss possible options for extending this FET approach to timesteppers for the compressible case.

        The kinetic corrections satisfy linearized Boltzmann equations. Using a Lénard--Bernstein collision operator, these take Fokker--Planck-like forms \cite{Fokker_1914, Planck_1917} wherein pseudo-particles in the numerical model obey the neoclassical transport equations, with particle-independent Brownian drift terms. This offers a rigorous methodology for incorporating collisions into the particle transport model, without coupling the equations of motions for each particle.
        
        Works by Chen, Chacón et al. \cite{Chen_Chacón_Barnes_2011, Chacón_Chen_Barnes_2013, Chen_Chacón_2014, Chen_Chacón_2015} have developed structure-preserving particle pushers for neoclassical transport in the Vlasov equations, derived from Crank--Nicolson integrators. We show these too can can derive from a FET interpretation, similarly offering potential extensions to higher-order-in-time particle pushers. The FET formulation is used also to consider how the stochastic drift terms can be incorporated into the pushers. Stochastic gyrokinetic expansions are also discussed.

        Different options for the numerical implementation of these schemes are considered.

        Due to the efficacy of FET in the development of SP timesteppers for both the fluid and kinetic component, we hope this approach will prove effective in the future for developing SP timesteppers for the full hybrid model. We hope this will give us the opportunity to incorporate previously inaccessible kinetic effects into the highly effective, modern, finite-element MHD models.
    \end{abstract}
    
    
    \newpage
    \tableofcontents
    
    
    \newpage
    \pagenumbering{arabic}
    %\linenumbers\renewcommand\thelinenumber{\color{black!50}\arabic{linenumber}}
            \input{0 - introduction/main.tex}
        \part{Research}
            \input{1 - low-noise PiC models/main.tex}
            \input{2 - kinetic component/main.tex}
            \input{3 - fluid component/main.tex}
            \input{4 - numerical implementation/main.tex}
        \part{Project Overview}
            \input{5 - research plan/main.tex}
            \input{6 - summary/main.tex}
    
    
    %\section{}
    \newpage
    \pagenumbering{gobble}
        \printbibliography


    \newpage
    \pagenumbering{roman}
    \appendix
        \part{Appendices}
            \input{8 - Hilbert complexes/main.tex}
            \input{9 - weak conservation proofs/main.tex}
\end{document}

        \part{Project Overview}
            \documentclass[12pt, a4paper]{report}

\input{template/main.tex}

\title{\BA{Title in Progress...}}
\author{Boris Andrews}
\affil{Mathematical Institute, University of Oxford}
\date{\today}


\begin{document}
    \pagenumbering{gobble}
    \maketitle
    
    
    \begin{abstract}
        Magnetic confinement reactors---in particular tokamaks---offer one of the most promising options for achieving practical nuclear fusion, with the potential to provide virtually limitless, clean energy. The theoretical and numerical modeling of tokamak plasmas is simultaneously an essential component of effective reactor design, and a great research barrier. Tokamak operational conditions exhibit comparatively low Knudsen numbers. Kinetic effects, including kinetic waves and instabilities, Landau damping, bump-on-tail instabilities and more, are therefore highly influential in tokamak plasma dynamics. Purely fluid models are inherently incapable of capturing these effects, whereas the high dimensionality in purely kinetic models render them practically intractable for most relevant purposes.

        We consider a $\delta\!f$ decomposition model, with a macroscopic fluid background and microscopic kinetic correction, both fully coupled to each other. A similar manner of discretization is proposed to that used in the recent \texttt{STRUPHY} code \cite{Holderied_Possanner_Wang_2021, Holderied_2022, Li_et_al_2023} with a finite-element model for the background and a pseudo-particle/PiC model for the correction.

        The fluid background satisfies the full, non-linear, resistive, compressible, Hall MHD equations. \cite{Laakmann_Hu_Farrell_2022} introduces finite-element(-in-space) implicit timesteppers for the incompressible analogue to this system with structure-preserving (SP) properties in the ideal case, alongside parameter-robust preconditioners. We show that these timesteppers can derive from a finite-element-in-time (FET) (and finite-element-in-space) interpretation. The benefits of this reformulation are discussed, including the derivation of timesteppers that are higher order in time, and the quantifiable dissipative SP properties in the non-ideal, resistive case.
        
        We discuss possible options for extending this FET approach to timesteppers for the compressible case.

        The kinetic corrections satisfy linearized Boltzmann equations. Using a Lénard--Bernstein collision operator, these take Fokker--Planck-like forms \cite{Fokker_1914, Planck_1917} wherein pseudo-particles in the numerical model obey the neoclassical transport equations, with particle-independent Brownian drift terms. This offers a rigorous methodology for incorporating collisions into the particle transport model, without coupling the equations of motions for each particle.
        
        Works by Chen, Chacón et al. \cite{Chen_Chacón_Barnes_2011, Chacón_Chen_Barnes_2013, Chen_Chacón_2014, Chen_Chacón_2015} have developed structure-preserving particle pushers for neoclassical transport in the Vlasov equations, derived from Crank--Nicolson integrators. We show these too can can derive from a FET interpretation, similarly offering potential extensions to higher-order-in-time particle pushers. The FET formulation is used also to consider how the stochastic drift terms can be incorporated into the pushers. Stochastic gyrokinetic expansions are also discussed.

        Different options for the numerical implementation of these schemes are considered.

        Due to the efficacy of FET in the development of SP timesteppers for both the fluid and kinetic component, we hope this approach will prove effective in the future for developing SP timesteppers for the full hybrid model. We hope this will give us the opportunity to incorporate previously inaccessible kinetic effects into the highly effective, modern, finite-element MHD models.
    \end{abstract}
    
    
    \newpage
    \tableofcontents
    
    
    \newpage
    \pagenumbering{arabic}
    %\linenumbers\renewcommand\thelinenumber{\color{black!50}\arabic{linenumber}}
            \input{0 - introduction/main.tex}
        \part{Research}
            \input{1 - low-noise PiC models/main.tex}
            \input{2 - kinetic component/main.tex}
            \input{3 - fluid component/main.tex}
            \input{4 - numerical implementation/main.tex}
        \part{Project Overview}
            \input{5 - research plan/main.tex}
            \input{6 - summary/main.tex}
    
    
    %\section{}
    \newpage
    \pagenumbering{gobble}
        \printbibliography


    \newpage
    \pagenumbering{roman}
    \appendix
        \part{Appendices}
            \input{8 - Hilbert complexes/main.tex}
            \input{9 - weak conservation proofs/main.tex}
\end{document}

            \documentclass[12pt, a4paper]{report}

\input{template/main.tex}

\title{\BA{Title in Progress...}}
\author{Boris Andrews}
\affil{Mathematical Institute, University of Oxford}
\date{\today}


\begin{document}
    \pagenumbering{gobble}
    \maketitle
    
    
    \begin{abstract}
        Magnetic confinement reactors---in particular tokamaks---offer one of the most promising options for achieving practical nuclear fusion, with the potential to provide virtually limitless, clean energy. The theoretical and numerical modeling of tokamak plasmas is simultaneously an essential component of effective reactor design, and a great research barrier. Tokamak operational conditions exhibit comparatively low Knudsen numbers. Kinetic effects, including kinetic waves and instabilities, Landau damping, bump-on-tail instabilities and more, are therefore highly influential in tokamak plasma dynamics. Purely fluid models are inherently incapable of capturing these effects, whereas the high dimensionality in purely kinetic models render them practically intractable for most relevant purposes.

        We consider a $\delta\!f$ decomposition model, with a macroscopic fluid background and microscopic kinetic correction, both fully coupled to each other. A similar manner of discretization is proposed to that used in the recent \texttt{STRUPHY} code \cite{Holderied_Possanner_Wang_2021, Holderied_2022, Li_et_al_2023} with a finite-element model for the background and a pseudo-particle/PiC model for the correction.

        The fluid background satisfies the full, non-linear, resistive, compressible, Hall MHD equations. \cite{Laakmann_Hu_Farrell_2022} introduces finite-element(-in-space) implicit timesteppers for the incompressible analogue to this system with structure-preserving (SP) properties in the ideal case, alongside parameter-robust preconditioners. We show that these timesteppers can derive from a finite-element-in-time (FET) (and finite-element-in-space) interpretation. The benefits of this reformulation are discussed, including the derivation of timesteppers that are higher order in time, and the quantifiable dissipative SP properties in the non-ideal, resistive case.
        
        We discuss possible options for extending this FET approach to timesteppers for the compressible case.

        The kinetic corrections satisfy linearized Boltzmann equations. Using a Lénard--Bernstein collision operator, these take Fokker--Planck-like forms \cite{Fokker_1914, Planck_1917} wherein pseudo-particles in the numerical model obey the neoclassical transport equations, with particle-independent Brownian drift terms. This offers a rigorous methodology for incorporating collisions into the particle transport model, without coupling the equations of motions for each particle.
        
        Works by Chen, Chacón et al. \cite{Chen_Chacón_Barnes_2011, Chacón_Chen_Barnes_2013, Chen_Chacón_2014, Chen_Chacón_2015} have developed structure-preserving particle pushers for neoclassical transport in the Vlasov equations, derived from Crank--Nicolson integrators. We show these too can can derive from a FET interpretation, similarly offering potential extensions to higher-order-in-time particle pushers. The FET formulation is used also to consider how the stochastic drift terms can be incorporated into the pushers. Stochastic gyrokinetic expansions are also discussed.

        Different options for the numerical implementation of these schemes are considered.

        Due to the efficacy of FET in the development of SP timesteppers for both the fluid and kinetic component, we hope this approach will prove effective in the future for developing SP timesteppers for the full hybrid model. We hope this will give us the opportunity to incorporate previously inaccessible kinetic effects into the highly effective, modern, finite-element MHD models.
    \end{abstract}
    
    
    \newpage
    \tableofcontents
    
    
    \newpage
    \pagenumbering{arabic}
    %\linenumbers\renewcommand\thelinenumber{\color{black!50}\arabic{linenumber}}
            \input{0 - introduction/main.tex}
        \part{Research}
            \input{1 - low-noise PiC models/main.tex}
            \input{2 - kinetic component/main.tex}
            \input{3 - fluid component/main.tex}
            \input{4 - numerical implementation/main.tex}
        \part{Project Overview}
            \input{5 - research plan/main.tex}
            \input{6 - summary/main.tex}
    
    
    %\section{}
    \newpage
    \pagenumbering{gobble}
        \printbibliography


    \newpage
    \pagenumbering{roman}
    \appendix
        \part{Appendices}
            \input{8 - Hilbert complexes/main.tex}
            \input{9 - weak conservation proofs/main.tex}
\end{document}

    
    
    %\section{}
    \newpage
    \pagenumbering{gobble}
        \printbibliography


    \newpage
    \pagenumbering{roman}
    \appendix
        \part{Appendices}
            \documentclass[12pt, a4paper]{report}

\input{template/main.tex}

\title{\BA{Title in Progress...}}
\author{Boris Andrews}
\affil{Mathematical Institute, University of Oxford}
\date{\today}


\begin{document}
    \pagenumbering{gobble}
    \maketitle
    
    
    \begin{abstract}
        Magnetic confinement reactors---in particular tokamaks---offer one of the most promising options for achieving practical nuclear fusion, with the potential to provide virtually limitless, clean energy. The theoretical and numerical modeling of tokamak plasmas is simultaneously an essential component of effective reactor design, and a great research barrier. Tokamak operational conditions exhibit comparatively low Knudsen numbers. Kinetic effects, including kinetic waves and instabilities, Landau damping, bump-on-tail instabilities and more, are therefore highly influential in tokamak plasma dynamics. Purely fluid models are inherently incapable of capturing these effects, whereas the high dimensionality in purely kinetic models render them practically intractable for most relevant purposes.

        We consider a $\delta\!f$ decomposition model, with a macroscopic fluid background and microscopic kinetic correction, both fully coupled to each other. A similar manner of discretization is proposed to that used in the recent \texttt{STRUPHY} code \cite{Holderied_Possanner_Wang_2021, Holderied_2022, Li_et_al_2023} with a finite-element model for the background and a pseudo-particle/PiC model for the correction.

        The fluid background satisfies the full, non-linear, resistive, compressible, Hall MHD equations. \cite{Laakmann_Hu_Farrell_2022} introduces finite-element(-in-space) implicit timesteppers for the incompressible analogue to this system with structure-preserving (SP) properties in the ideal case, alongside parameter-robust preconditioners. We show that these timesteppers can derive from a finite-element-in-time (FET) (and finite-element-in-space) interpretation. The benefits of this reformulation are discussed, including the derivation of timesteppers that are higher order in time, and the quantifiable dissipative SP properties in the non-ideal, resistive case.
        
        We discuss possible options for extending this FET approach to timesteppers for the compressible case.

        The kinetic corrections satisfy linearized Boltzmann equations. Using a Lénard--Bernstein collision operator, these take Fokker--Planck-like forms \cite{Fokker_1914, Planck_1917} wherein pseudo-particles in the numerical model obey the neoclassical transport equations, with particle-independent Brownian drift terms. This offers a rigorous methodology for incorporating collisions into the particle transport model, without coupling the equations of motions for each particle.
        
        Works by Chen, Chacón et al. \cite{Chen_Chacón_Barnes_2011, Chacón_Chen_Barnes_2013, Chen_Chacón_2014, Chen_Chacón_2015} have developed structure-preserving particle pushers for neoclassical transport in the Vlasov equations, derived from Crank--Nicolson integrators. We show these too can can derive from a FET interpretation, similarly offering potential extensions to higher-order-in-time particle pushers. The FET formulation is used also to consider how the stochastic drift terms can be incorporated into the pushers. Stochastic gyrokinetic expansions are also discussed.

        Different options for the numerical implementation of these schemes are considered.

        Due to the efficacy of FET in the development of SP timesteppers for both the fluid and kinetic component, we hope this approach will prove effective in the future for developing SP timesteppers for the full hybrid model. We hope this will give us the opportunity to incorporate previously inaccessible kinetic effects into the highly effective, modern, finite-element MHD models.
    \end{abstract}
    
    
    \newpage
    \tableofcontents
    
    
    \newpage
    \pagenumbering{arabic}
    %\linenumbers\renewcommand\thelinenumber{\color{black!50}\arabic{linenumber}}
            \input{0 - introduction/main.tex}
        \part{Research}
            \input{1 - low-noise PiC models/main.tex}
            \input{2 - kinetic component/main.tex}
            \input{3 - fluid component/main.tex}
            \input{4 - numerical implementation/main.tex}
        \part{Project Overview}
            \input{5 - research plan/main.tex}
            \input{6 - summary/main.tex}
    
    
    %\section{}
    \newpage
    \pagenumbering{gobble}
        \printbibliography


    \newpage
    \pagenumbering{roman}
    \appendix
        \part{Appendices}
            \input{8 - Hilbert complexes/main.tex}
            \input{9 - weak conservation proofs/main.tex}
\end{document}

            \documentclass[12pt, a4paper]{report}

\input{template/main.tex}

\title{\BA{Title in Progress...}}
\author{Boris Andrews}
\affil{Mathematical Institute, University of Oxford}
\date{\today}


\begin{document}
    \pagenumbering{gobble}
    \maketitle
    
    
    \begin{abstract}
        Magnetic confinement reactors---in particular tokamaks---offer one of the most promising options for achieving practical nuclear fusion, with the potential to provide virtually limitless, clean energy. The theoretical and numerical modeling of tokamak plasmas is simultaneously an essential component of effective reactor design, and a great research barrier. Tokamak operational conditions exhibit comparatively low Knudsen numbers. Kinetic effects, including kinetic waves and instabilities, Landau damping, bump-on-tail instabilities and more, are therefore highly influential in tokamak plasma dynamics. Purely fluid models are inherently incapable of capturing these effects, whereas the high dimensionality in purely kinetic models render them practically intractable for most relevant purposes.

        We consider a $\delta\!f$ decomposition model, with a macroscopic fluid background and microscopic kinetic correction, both fully coupled to each other. A similar manner of discretization is proposed to that used in the recent \texttt{STRUPHY} code \cite{Holderied_Possanner_Wang_2021, Holderied_2022, Li_et_al_2023} with a finite-element model for the background and a pseudo-particle/PiC model for the correction.

        The fluid background satisfies the full, non-linear, resistive, compressible, Hall MHD equations. \cite{Laakmann_Hu_Farrell_2022} introduces finite-element(-in-space) implicit timesteppers for the incompressible analogue to this system with structure-preserving (SP) properties in the ideal case, alongside parameter-robust preconditioners. We show that these timesteppers can derive from a finite-element-in-time (FET) (and finite-element-in-space) interpretation. The benefits of this reformulation are discussed, including the derivation of timesteppers that are higher order in time, and the quantifiable dissipative SP properties in the non-ideal, resistive case.
        
        We discuss possible options for extending this FET approach to timesteppers for the compressible case.

        The kinetic corrections satisfy linearized Boltzmann equations. Using a Lénard--Bernstein collision operator, these take Fokker--Planck-like forms \cite{Fokker_1914, Planck_1917} wherein pseudo-particles in the numerical model obey the neoclassical transport equations, with particle-independent Brownian drift terms. This offers a rigorous methodology for incorporating collisions into the particle transport model, without coupling the equations of motions for each particle.
        
        Works by Chen, Chacón et al. \cite{Chen_Chacón_Barnes_2011, Chacón_Chen_Barnes_2013, Chen_Chacón_2014, Chen_Chacón_2015} have developed structure-preserving particle pushers for neoclassical transport in the Vlasov equations, derived from Crank--Nicolson integrators. We show these too can can derive from a FET interpretation, similarly offering potential extensions to higher-order-in-time particle pushers. The FET formulation is used also to consider how the stochastic drift terms can be incorporated into the pushers. Stochastic gyrokinetic expansions are also discussed.

        Different options for the numerical implementation of these schemes are considered.

        Due to the efficacy of FET in the development of SP timesteppers for both the fluid and kinetic component, we hope this approach will prove effective in the future for developing SP timesteppers for the full hybrid model. We hope this will give us the opportunity to incorporate previously inaccessible kinetic effects into the highly effective, modern, finite-element MHD models.
    \end{abstract}
    
    
    \newpage
    \tableofcontents
    
    
    \newpage
    \pagenumbering{arabic}
    %\linenumbers\renewcommand\thelinenumber{\color{black!50}\arabic{linenumber}}
            \input{0 - introduction/main.tex}
        \part{Research}
            \input{1 - low-noise PiC models/main.tex}
            \input{2 - kinetic component/main.tex}
            \input{3 - fluid component/main.tex}
            \input{4 - numerical implementation/main.tex}
        \part{Project Overview}
            \input{5 - research plan/main.tex}
            \input{6 - summary/main.tex}
    
    
    %\section{}
    \newpage
    \pagenumbering{gobble}
        \printbibliography


    \newpage
    \pagenumbering{roman}
    \appendix
        \part{Appendices}
            \input{8 - Hilbert complexes/main.tex}
            \input{9 - weak conservation proofs/main.tex}
\end{document}

\end{document}


\title{\BA{Title in Progress...}}
\author{Boris Andrews}
\affil{Mathematical Institute, University of Oxford}
\date{\today}


\begin{document}
    \pagenumbering{gobble}
    \maketitle
    
    
    \begin{abstract}
        Magnetic confinement reactors---in particular tokamaks---offer one of the most promising options for achieving practical nuclear fusion, with the potential to provide virtually limitless, clean energy. The theoretical and numerical modeling of tokamak plasmas is simultaneously an essential component of effective reactor design, and a great research barrier. Tokamak operational conditions exhibit comparatively low Knudsen numbers. Kinetic effects, including kinetic waves and instabilities, Landau damping, bump-on-tail instabilities and more, are therefore highly influential in tokamak plasma dynamics. Purely fluid models are inherently incapable of capturing these effects, whereas the high dimensionality in purely kinetic models render them practically intractable for most relevant purposes.

        We consider a $\delta\!f$ decomposition model, with a macroscopic fluid background and microscopic kinetic correction, both fully coupled to each other. A similar manner of discretization is proposed to that used in the recent \texttt{STRUPHY} code \cite{Holderied_Possanner_Wang_2021, Holderied_2022, Li_et_al_2023} with a finite-element model for the background and a pseudo-particle/PiC model for the correction.

        The fluid background satisfies the full, non-linear, resistive, compressible, Hall MHD equations. \cite{Laakmann_Hu_Farrell_2022} introduces finite-element(-in-space) implicit timesteppers for the incompressible analogue to this system with structure-preserving (SP) properties in the ideal case, alongside parameter-robust preconditioners. We show that these timesteppers can derive from a finite-element-in-time (FET) (and finite-element-in-space) interpretation. The benefits of this reformulation are discussed, including the derivation of timesteppers that are higher order in time, and the quantifiable dissipative SP properties in the non-ideal, resistive case.
        
        We discuss possible options for extending this FET approach to timesteppers for the compressible case.

        The kinetic corrections satisfy linearized Boltzmann equations. Using a Lénard--Bernstein collision operator, these take Fokker--Planck-like forms \cite{Fokker_1914, Planck_1917} wherein pseudo-particles in the numerical model obey the neoclassical transport equations, with particle-independent Brownian drift terms. This offers a rigorous methodology for incorporating collisions into the particle transport model, without coupling the equations of motions for each particle.
        
        Works by Chen, Chacón et al. \cite{Chen_Chacón_Barnes_2011, Chacón_Chen_Barnes_2013, Chen_Chacón_2014, Chen_Chacón_2015} have developed structure-preserving particle pushers for neoclassical transport in the Vlasov equations, derived from Crank--Nicolson integrators. We show these too can can derive from a FET interpretation, similarly offering potential extensions to higher-order-in-time particle pushers. The FET formulation is used also to consider how the stochastic drift terms can be incorporated into the pushers. Stochastic gyrokinetic expansions are also discussed.

        Different options for the numerical implementation of these schemes are considered.

        Due to the efficacy of FET in the development of SP timesteppers for both the fluid and kinetic component, we hope this approach will prove effective in the future for developing SP timesteppers for the full hybrid model. We hope this will give us the opportunity to incorporate previously inaccessible kinetic effects into the highly effective, modern, finite-element MHD models.
    \end{abstract}
    
    
    \newpage
    \tableofcontents
    
    
    \newpage
    \pagenumbering{arabic}
    %\linenumbers\renewcommand\thelinenumber{\color{black!50}\arabic{linenumber}}
            \documentclass[12pt, a4paper]{report}

\documentclass[12pt, a4paper]{report}

\input{template/main.tex}

\title{\BA{Title in Progress...}}
\author{Boris Andrews}
\affil{Mathematical Institute, University of Oxford}
\date{\today}


\begin{document}
    \pagenumbering{gobble}
    \maketitle
    
    
    \begin{abstract}
        Magnetic confinement reactors---in particular tokamaks---offer one of the most promising options for achieving practical nuclear fusion, with the potential to provide virtually limitless, clean energy. The theoretical and numerical modeling of tokamak plasmas is simultaneously an essential component of effective reactor design, and a great research barrier. Tokamak operational conditions exhibit comparatively low Knudsen numbers. Kinetic effects, including kinetic waves and instabilities, Landau damping, bump-on-tail instabilities and more, are therefore highly influential in tokamak plasma dynamics. Purely fluid models are inherently incapable of capturing these effects, whereas the high dimensionality in purely kinetic models render them practically intractable for most relevant purposes.

        We consider a $\delta\!f$ decomposition model, with a macroscopic fluid background and microscopic kinetic correction, both fully coupled to each other. A similar manner of discretization is proposed to that used in the recent \texttt{STRUPHY} code \cite{Holderied_Possanner_Wang_2021, Holderied_2022, Li_et_al_2023} with a finite-element model for the background and a pseudo-particle/PiC model for the correction.

        The fluid background satisfies the full, non-linear, resistive, compressible, Hall MHD equations. \cite{Laakmann_Hu_Farrell_2022} introduces finite-element(-in-space) implicit timesteppers for the incompressible analogue to this system with structure-preserving (SP) properties in the ideal case, alongside parameter-robust preconditioners. We show that these timesteppers can derive from a finite-element-in-time (FET) (and finite-element-in-space) interpretation. The benefits of this reformulation are discussed, including the derivation of timesteppers that are higher order in time, and the quantifiable dissipative SP properties in the non-ideal, resistive case.
        
        We discuss possible options for extending this FET approach to timesteppers for the compressible case.

        The kinetic corrections satisfy linearized Boltzmann equations. Using a Lénard--Bernstein collision operator, these take Fokker--Planck-like forms \cite{Fokker_1914, Planck_1917} wherein pseudo-particles in the numerical model obey the neoclassical transport equations, with particle-independent Brownian drift terms. This offers a rigorous methodology for incorporating collisions into the particle transport model, without coupling the equations of motions for each particle.
        
        Works by Chen, Chacón et al. \cite{Chen_Chacón_Barnes_2011, Chacón_Chen_Barnes_2013, Chen_Chacón_2014, Chen_Chacón_2015} have developed structure-preserving particle pushers for neoclassical transport in the Vlasov equations, derived from Crank--Nicolson integrators. We show these too can can derive from a FET interpretation, similarly offering potential extensions to higher-order-in-time particle pushers. The FET formulation is used also to consider how the stochastic drift terms can be incorporated into the pushers. Stochastic gyrokinetic expansions are also discussed.

        Different options for the numerical implementation of these schemes are considered.

        Due to the efficacy of FET in the development of SP timesteppers for both the fluid and kinetic component, we hope this approach will prove effective in the future for developing SP timesteppers for the full hybrid model. We hope this will give us the opportunity to incorporate previously inaccessible kinetic effects into the highly effective, modern, finite-element MHD models.
    \end{abstract}
    
    
    \newpage
    \tableofcontents
    
    
    \newpage
    \pagenumbering{arabic}
    %\linenumbers\renewcommand\thelinenumber{\color{black!50}\arabic{linenumber}}
            \input{0 - introduction/main.tex}
        \part{Research}
            \input{1 - low-noise PiC models/main.tex}
            \input{2 - kinetic component/main.tex}
            \input{3 - fluid component/main.tex}
            \input{4 - numerical implementation/main.tex}
        \part{Project Overview}
            \input{5 - research plan/main.tex}
            \input{6 - summary/main.tex}
    
    
    %\section{}
    \newpage
    \pagenumbering{gobble}
        \printbibliography


    \newpage
    \pagenumbering{roman}
    \appendix
        \part{Appendices}
            \input{8 - Hilbert complexes/main.tex}
            \input{9 - weak conservation proofs/main.tex}
\end{document}


\title{\BA{Title in Progress...}}
\author{Boris Andrews}
\affil{Mathematical Institute, University of Oxford}
\date{\today}


\begin{document}
    \pagenumbering{gobble}
    \maketitle
    
    
    \begin{abstract}
        Magnetic confinement reactors---in particular tokamaks---offer one of the most promising options for achieving practical nuclear fusion, with the potential to provide virtually limitless, clean energy. The theoretical and numerical modeling of tokamak plasmas is simultaneously an essential component of effective reactor design, and a great research barrier. Tokamak operational conditions exhibit comparatively low Knudsen numbers. Kinetic effects, including kinetic waves and instabilities, Landau damping, bump-on-tail instabilities and more, are therefore highly influential in tokamak plasma dynamics. Purely fluid models are inherently incapable of capturing these effects, whereas the high dimensionality in purely kinetic models render them practically intractable for most relevant purposes.

        We consider a $\delta\!f$ decomposition model, with a macroscopic fluid background and microscopic kinetic correction, both fully coupled to each other. A similar manner of discretization is proposed to that used in the recent \texttt{STRUPHY} code \cite{Holderied_Possanner_Wang_2021, Holderied_2022, Li_et_al_2023} with a finite-element model for the background and a pseudo-particle/PiC model for the correction.

        The fluid background satisfies the full, non-linear, resistive, compressible, Hall MHD equations. \cite{Laakmann_Hu_Farrell_2022} introduces finite-element(-in-space) implicit timesteppers for the incompressible analogue to this system with structure-preserving (SP) properties in the ideal case, alongside parameter-robust preconditioners. We show that these timesteppers can derive from a finite-element-in-time (FET) (and finite-element-in-space) interpretation. The benefits of this reformulation are discussed, including the derivation of timesteppers that are higher order in time, and the quantifiable dissipative SP properties in the non-ideal, resistive case.
        
        We discuss possible options for extending this FET approach to timesteppers for the compressible case.

        The kinetic corrections satisfy linearized Boltzmann equations. Using a Lénard--Bernstein collision operator, these take Fokker--Planck-like forms \cite{Fokker_1914, Planck_1917} wherein pseudo-particles in the numerical model obey the neoclassical transport equations, with particle-independent Brownian drift terms. This offers a rigorous methodology for incorporating collisions into the particle transport model, without coupling the equations of motions for each particle.
        
        Works by Chen, Chacón et al. \cite{Chen_Chacón_Barnes_2011, Chacón_Chen_Barnes_2013, Chen_Chacón_2014, Chen_Chacón_2015} have developed structure-preserving particle pushers for neoclassical transport in the Vlasov equations, derived from Crank--Nicolson integrators. We show these too can can derive from a FET interpretation, similarly offering potential extensions to higher-order-in-time particle pushers. The FET formulation is used also to consider how the stochastic drift terms can be incorporated into the pushers. Stochastic gyrokinetic expansions are also discussed.

        Different options for the numerical implementation of these schemes are considered.

        Due to the efficacy of FET in the development of SP timesteppers for both the fluid and kinetic component, we hope this approach will prove effective in the future for developing SP timesteppers for the full hybrid model. We hope this will give us the opportunity to incorporate previously inaccessible kinetic effects into the highly effective, modern, finite-element MHD models.
    \end{abstract}
    
    
    \newpage
    \tableofcontents
    
    
    \newpage
    \pagenumbering{arabic}
    %\linenumbers\renewcommand\thelinenumber{\color{black!50}\arabic{linenumber}}
            \documentclass[12pt, a4paper]{report}

\input{template/main.tex}

\title{\BA{Title in Progress...}}
\author{Boris Andrews}
\affil{Mathematical Institute, University of Oxford}
\date{\today}


\begin{document}
    \pagenumbering{gobble}
    \maketitle
    
    
    \begin{abstract}
        Magnetic confinement reactors---in particular tokamaks---offer one of the most promising options for achieving practical nuclear fusion, with the potential to provide virtually limitless, clean energy. The theoretical and numerical modeling of tokamak plasmas is simultaneously an essential component of effective reactor design, and a great research barrier. Tokamak operational conditions exhibit comparatively low Knudsen numbers. Kinetic effects, including kinetic waves and instabilities, Landau damping, bump-on-tail instabilities and more, are therefore highly influential in tokamak plasma dynamics. Purely fluid models are inherently incapable of capturing these effects, whereas the high dimensionality in purely kinetic models render them practically intractable for most relevant purposes.

        We consider a $\delta\!f$ decomposition model, with a macroscopic fluid background and microscopic kinetic correction, both fully coupled to each other. A similar manner of discretization is proposed to that used in the recent \texttt{STRUPHY} code \cite{Holderied_Possanner_Wang_2021, Holderied_2022, Li_et_al_2023} with a finite-element model for the background and a pseudo-particle/PiC model for the correction.

        The fluid background satisfies the full, non-linear, resistive, compressible, Hall MHD equations. \cite{Laakmann_Hu_Farrell_2022} introduces finite-element(-in-space) implicit timesteppers for the incompressible analogue to this system with structure-preserving (SP) properties in the ideal case, alongside parameter-robust preconditioners. We show that these timesteppers can derive from a finite-element-in-time (FET) (and finite-element-in-space) interpretation. The benefits of this reformulation are discussed, including the derivation of timesteppers that are higher order in time, and the quantifiable dissipative SP properties in the non-ideal, resistive case.
        
        We discuss possible options for extending this FET approach to timesteppers for the compressible case.

        The kinetic corrections satisfy linearized Boltzmann equations. Using a Lénard--Bernstein collision operator, these take Fokker--Planck-like forms \cite{Fokker_1914, Planck_1917} wherein pseudo-particles in the numerical model obey the neoclassical transport equations, with particle-independent Brownian drift terms. This offers a rigorous methodology for incorporating collisions into the particle transport model, without coupling the equations of motions for each particle.
        
        Works by Chen, Chacón et al. \cite{Chen_Chacón_Barnes_2011, Chacón_Chen_Barnes_2013, Chen_Chacón_2014, Chen_Chacón_2015} have developed structure-preserving particle pushers for neoclassical transport in the Vlasov equations, derived from Crank--Nicolson integrators. We show these too can can derive from a FET interpretation, similarly offering potential extensions to higher-order-in-time particle pushers. The FET formulation is used also to consider how the stochastic drift terms can be incorporated into the pushers. Stochastic gyrokinetic expansions are also discussed.

        Different options for the numerical implementation of these schemes are considered.

        Due to the efficacy of FET in the development of SP timesteppers for both the fluid and kinetic component, we hope this approach will prove effective in the future for developing SP timesteppers for the full hybrid model. We hope this will give us the opportunity to incorporate previously inaccessible kinetic effects into the highly effective, modern, finite-element MHD models.
    \end{abstract}
    
    
    \newpage
    \tableofcontents
    
    
    \newpage
    \pagenumbering{arabic}
    %\linenumbers\renewcommand\thelinenumber{\color{black!50}\arabic{linenumber}}
            \input{0 - introduction/main.tex}
        \part{Research}
            \input{1 - low-noise PiC models/main.tex}
            \input{2 - kinetic component/main.tex}
            \input{3 - fluid component/main.tex}
            \input{4 - numerical implementation/main.tex}
        \part{Project Overview}
            \input{5 - research plan/main.tex}
            \input{6 - summary/main.tex}
    
    
    %\section{}
    \newpage
    \pagenumbering{gobble}
        \printbibliography


    \newpage
    \pagenumbering{roman}
    \appendix
        \part{Appendices}
            \input{8 - Hilbert complexes/main.tex}
            \input{9 - weak conservation proofs/main.tex}
\end{document}

        \part{Research}
            \documentclass[12pt, a4paper]{report}

\input{template/main.tex}

\title{\BA{Title in Progress...}}
\author{Boris Andrews}
\affil{Mathematical Institute, University of Oxford}
\date{\today}


\begin{document}
    \pagenumbering{gobble}
    \maketitle
    
    
    \begin{abstract}
        Magnetic confinement reactors---in particular tokamaks---offer one of the most promising options for achieving practical nuclear fusion, with the potential to provide virtually limitless, clean energy. The theoretical and numerical modeling of tokamak plasmas is simultaneously an essential component of effective reactor design, and a great research barrier. Tokamak operational conditions exhibit comparatively low Knudsen numbers. Kinetic effects, including kinetic waves and instabilities, Landau damping, bump-on-tail instabilities and more, are therefore highly influential in tokamak plasma dynamics. Purely fluid models are inherently incapable of capturing these effects, whereas the high dimensionality in purely kinetic models render them practically intractable for most relevant purposes.

        We consider a $\delta\!f$ decomposition model, with a macroscopic fluid background and microscopic kinetic correction, both fully coupled to each other. A similar manner of discretization is proposed to that used in the recent \texttt{STRUPHY} code \cite{Holderied_Possanner_Wang_2021, Holderied_2022, Li_et_al_2023} with a finite-element model for the background and a pseudo-particle/PiC model for the correction.

        The fluid background satisfies the full, non-linear, resistive, compressible, Hall MHD equations. \cite{Laakmann_Hu_Farrell_2022} introduces finite-element(-in-space) implicit timesteppers for the incompressible analogue to this system with structure-preserving (SP) properties in the ideal case, alongside parameter-robust preconditioners. We show that these timesteppers can derive from a finite-element-in-time (FET) (and finite-element-in-space) interpretation. The benefits of this reformulation are discussed, including the derivation of timesteppers that are higher order in time, and the quantifiable dissipative SP properties in the non-ideal, resistive case.
        
        We discuss possible options for extending this FET approach to timesteppers for the compressible case.

        The kinetic corrections satisfy linearized Boltzmann equations. Using a Lénard--Bernstein collision operator, these take Fokker--Planck-like forms \cite{Fokker_1914, Planck_1917} wherein pseudo-particles in the numerical model obey the neoclassical transport equations, with particle-independent Brownian drift terms. This offers a rigorous methodology for incorporating collisions into the particle transport model, without coupling the equations of motions for each particle.
        
        Works by Chen, Chacón et al. \cite{Chen_Chacón_Barnes_2011, Chacón_Chen_Barnes_2013, Chen_Chacón_2014, Chen_Chacón_2015} have developed structure-preserving particle pushers for neoclassical transport in the Vlasov equations, derived from Crank--Nicolson integrators. We show these too can can derive from a FET interpretation, similarly offering potential extensions to higher-order-in-time particle pushers. The FET formulation is used also to consider how the stochastic drift terms can be incorporated into the pushers. Stochastic gyrokinetic expansions are also discussed.

        Different options for the numerical implementation of these schemes are considered.

        Due to the efficacy of FET in the development of SP timesteppers for both the fluid and kinetic component, we hope this approach will prove effective in the future for developing SP timesteppers for the full hybrid model. We hope this will give us the opportunity to incorporate previously inaccessible kinetic effects into the highly effective, modern, finite-element MHD models.
    \end{abstract}
    
    
    \newpage
    \tableofcontents
    
    
    \newpage
    \pagenumbering{arabic}
    %\linenumbers\renewcommand\thelinenumber{\color{black!50}\arabic{linenumber}}
            \input{0 - introduction/main.tex}
        \part{Research}
            \input{1 - low-noise PiC models/main.tex}
            \input{2 - kinetic component/main.tex}
            \input{3 - fluid component/main.tex}
            \input{4 - numerical implementation/main.tex}
        \part{Project Overview}
            \input{5 - research plan/main.tex}
            \input{6 - summary/main.tex}
    
    
    %\section{}
    \newpage
    \pagenumbering{gobble}
        \printbibliography


    \newpage
    \pagenumbering{roman}
    \appendix
        \part{Appendices}
            \input{8 - Hilbert complexes/main.tex}
            \input{9 - weak conservation proofs/main.tex}
\end{document}

            \documentclass[12pt, a4paper]{report}

\input{template/main.tex}

\title{\BA{Title in Progress...}}
\author{Boris Andrews}
\affil{Mathematical Institute, University of Oxford}
\date{\today}


\begin{document}
    \pagenumbering{gobble}
    \maketitle
    
    
    \begin{abstract}
        Magnetic confinement reactors---in particular tokamaks---offer one of the most promising options for achieving practical nuclear fusion, with the potential to provide virtually limitless, clean energy. The theoretical and numerical modeling of tokamak plasmas is simultaneously an essential component of effective reactor design, and a great research barrier. Tokamak operational conditions exhibit comparatively low Knudsen numbers. Kinetic effects, including kinetic waves and instabilities, Landau damping, bump-on-tail instabilities and more, are therefore highly influential in tokamak plasma dynamics. Purely fluid models are inherently incapable of capturing these effects, whereas the high dimensionality in purely kinetic models render them practically intractable for most relevant purposes.

        We consider a $\delta\!f$ decomposition model, with a macroscopic fluid background and microscopic kinetic correction, both fully coupled to each other. A similar manner of discretization is proposed to that used in the recent \texttt{STRUPHY} code \cite{Holderied_Possanner_Wang_2021, Holderied_2022, Li_et_al_2023} with a finite-element model for the background and a pseudo-particle/PiC model for the correction.

        The fluid background satisfies the full, non-linear, resistive, compressible, Hall MHD equations. \cite{Laakmann_Hu_Farrell_2022} introduces finite-element(-in-space) implicit timesteppers for the incompressible analogue to this system with structure-preserving (SP) properties in the ideal case, alongside parameter-robust preconditioners. We show that these timesteppers can derive from a finite-element-in-time (FET) (and finite-element-in-space) interpretation. The benefits of this reformulation are discussed, including the derivation of timesteppers that are higher order in time, and the quantifiable dissipative SP properties in the non-ideal, resistive case.
        
        We discuss possible options for extending this FET approach to timesteppers for the compressible case.

        The kinetic corrections satisfy linearized Boltzmann equations. Using a Lénard--Bernstein collision operator, these take Fokker--Planck-like forms \cite{Fokker_1914, Planck_1917} wherein pseudo-particles in the numerical model obey the neoclassical transport equations, with particle-independent Brownian drift terms. This offers a rigorous methodology for incorporating collisions into the particle transport model, without coupling the equations of motions for each particle.
        
        Works by Chen, Chacón et al. \cite{Chen_Chacón_Barnes_2011, Chacón_Chen_Barnes_2013, Chen_Chacón_2014, Chen_Chacón_2015} have developed structure-preserving particle pushers for neoclassical transport in the Vlasov equations, derived from Crank--Nicolson integrators. We show these too can can derive from a FET interpretation, similarly offering potential extensions to higher-order-in-time particle pushers. The FET formulation is used also to consider how the stochastic drift terms can be incorporated into the pushers. Stochastic gyrokinetic expansions are also discussed.

        Different options for the numerical implementation of these schemes are considered.

        Due to the efficacy of FET in the development of SP timesteppers for both the fluid and kinetic component, we hope this approach will prove effective in the future for developing SP timesteppers for the full hybrid model. We hope this will give us the opportunity to incorporate previously inaccessible kinetic effects into the highly effective, modern, finite-element MHD models.
    \end{abstract}
    
    
    \newpage
    \tableofcontents
    
    
    \newpage
    \pagenumbering{arabic}
    %\linenumbers\renewcommand\thelinenumber{\color{black!50}\arabic{linenumber}}
            \input{0 - introduction/main.tex}
        \part{Research}
            \input{1 - low-noise PiC models/main.tex}
            \input{2 - kinetic component/main.tex}
            \input{3 - fluid component/main.tex}
            \input{4 - numerical implementation/main.tex}
        \part{Project Overview}
            \input{5 - research plan/main.tex}
            \input{6 - summary/main.tex}
    
    
    %\section{}
    \newpage
    \pagenumbering{gobble}
        \printbibliography


    \newpage
    \pagenumbering{roman}
    \appendix
        \part{Appendices}
            \input{8 - Hilbert complexes/main.tex}
            \input{9 - weak conservation proofs/main.tex}
\end{document}

            \documentclass[12pt, a4paper]{report}

\input{template/main.tex}

\title{\BA{Title in Progress...}}
\author{Boris Andrews}
\affil{Mathematical Institute, University of Oxford}
\date{\today}


\begin{document}
    \pagenumbering{gobble}
    \maketitle
    
    
    \begin{abstract}
        Magnetic confinement reactors---in particular tokamaks---offer one of the most promising options for achieving practical nuclear fusion, with the potential to provide virtually limitless, clean energy. The theoretical and numerical modeling of tokamak plasmas is simultaneously an essential component of effective reactor design, and a great research barrier. Tokamak operational conditions exhibit comparatively low Knudsen numbers. Kinetic effects, including kinetic waves and instabilities, Landau damping, bump-on-tail instabilities and more, are therefore highly influential in tokamak plasma dynamics. Purely fluid models are inherently incapable of capturing these effects, whereas the high dimensionality in purely kinetic models render them practically intractable for most relevant purposes.

        We consider a $\delta\!f$ decomposition model, with a macroscopic fluid background and microscopic kinetic correction, both fully coupled to each other. A similar manner of discretization is proposed to that used in the recent \texttt{STRUPHY} code \cite{Holderied_Possanner_Wang_2021, Holderied_2022, Li_et_al_2023} with a finite-element model for the background and a pseudo-particle/PiC model for the correction.

        The fluid background satisfies the full, non-linear, resistive, compressible, Hall MHD equations. \cite{Laakmann_Hu_Farrell_2022} introduces finite-element(-in-space) implicit timesteppers for the incompressible analogue to this system with structure-preserving (SP) properties in the ideal case, alongside parameter-robust preconditioners. We show that these timesteppers can derive from a finite-element-in-time (FET) (and finite-element-in-space) interpretation. The benefits of this reformulation are discussed, including the derivation of timesteppers that are higher order in time, and the quantifiable dissipative SP properties in the non-ideal, resistive case.
        
        We discuss possible options for extending this FET approach to timesteppers for the compressible case.

        The kinetic corrections satisfy linearized Boltzmann equations. Using a Lénard--Bernstein collision operator, these take Fokker--Planck-like forms \cite{Fokker_1914, Planck_1917} wherein pseudo-particles in the numerical model obey the neoclassical transport equations, with particle-independent Brownian drift terms. This offers a rigorous methodology for incorporating collisions into the particle transport model, without coupling the equations of motions for each particle.
        
        Works by Chen, Chacón et al. \cite{Chen_Chacón_Barnes_2011, Chacón_Chen_Barnes_2013, Chen_Chacón_2014, Chen_Chacón_2015} have developed structure-preserving particle pushers for neoclassical transport in the Vlasov equations, derived from Crank--Nicolson integrators. We show these too can can derive from a FET interpretation, similarly offering potential extensions to higher-order-in-time particle pushers. The FET formulation is used also to consider how the stochastic drift terms can be incorporated into the pushers. Stochastic gyrokinetic expansions are also discussed.

        Different options for the numerical implementation of these schemes are considered.

        Due to the efficacy of FET in the development of SP timesteppers for both the fluid and kinetic component, we hope this approach will prove effective in the future for developing SP timesteppers for the full hybrid model. We hope this will give us the opportunity to incorporate previously inaccessible kinetic effects into the highly effective, modern, finite-element MHD models.
    \end{abstract}
    
    
    \newpage
    \tableofcontents
    
    
    \newpage
    \pagenumbering{arabic}
    %\linenumbers\renewcommand\thelinenumber{\color{black!50}\arabic{linenumber}}
            \input{0 - introduction/main.tex}
        \part{Research}
            \input{1 - low-noise PiC models/main.tex}
            \input{2 - kinetic component/main.tex}
            \input{3 - fluid component/main.tex}
            \input{4 - numerical implementation/main.tex}
        \part{Project Overview}
            \input{5 - research plan/main.tex}
            \input{6 - summary/main.tex}
    
    
    %\section{}
    \newpage
    \pagenumbering{gobble}
        \printbibliography


    \newpage
    \pagenumbering{roman}
    \appendix
        \part{Appendices}
            \input{8 - Hilbert complexes/main.tex}
            \input{9 - weak conservation proofs/main.tex}
\end{document}

            \documentclass[12pt, a4paper]{report}

\input{template/main.tex}

\title{\BA{Title in Progress...}}
\author{Boris Andrews}
\affil{Mathematical Institute, University of Oxford}
\date{\today}


\begin{document}
    \pagenumbering{gobble}
    \maketitle
    
    
    \begin{abstract}
        Magnetic confinement reactors---in particular tokamaks---offer one of the most promising options for achieving practical nuclear fusion, with the potential to provide virtually limitless, clean energy. The theoretical and numerical modeling of tokamak plasmas is simultaneously an essential component of effective reactor design, and a great research barrier. Tokamak operational conditions exhibit comparatively low Knudsen numbers. Kinetic effects, including kinetic waves and instabilities, Landau damping, bump-on-tail instabilities and more, are therefore highly influential in tokamak plasma dynamics. Purely fluid models are inherently incapable of capturing these effects, whereas the high dimensionality in purely kinetic models render them practically intractable for most relevant purposes.

        We consider a $\delta\!f$ decomposition model, with a macroscopic fluid background and microscopic kinetic correction, both fully coupled to each other. A similar manner of discretization is proposed to that used in the recent \texttt{STRUPHY} code \cite{Holderied_Possanner_Wang_2021, Holderied_2022, Li_et_al_2023} with a finite-element model for the background and a pseudo-particle/PiC model for the correction.

        The fluid background satisfies the full, non-linear, resistive, compressible, Hall MHD equations. \cite{Laakmann_Hu_Farrell_2022} introduces finite-element(-in-space) implicit timesteppers for the incompressible analogue to this system with structure-preserving (SP) properties in the ideal case, alongside parameter-robust preconditioners. We show that these timesteppers can derive from a finite-element-in-time (FET) (and finite-element-in-space) interpretation. The benefits of this reformulation are discussed, including the derivation of timesteppers that are higher order in time, and the quantifiable dissipative SP properties in the non-ideal, resistive case.
        
        We discuss possible options for extending this FET approach to timesteppers for the compressible case.

        The kinetic corrections satisfy linearized Boltzmann equations. Using a Lénard--Bernstein collision operator, these take Fokker--Planck-like forms \cite{Fokker_1914, Planck_1917} wherein pseudo-particles in the numerical model obey the neoclassical transport equations, with particle-independent Brownian drift terms. This offers a rigorous methodology for incorporating collisions into the particle transport model, without coupling the equations of motions for each particle.
        
        Works by Chen, Chacón et al. \cite{Chen_Chacón_Barnes_2011, Chacón_Chen_Barnes_2013, Chen_Chacón_2014, Chen_Chacón_2015} have developed structure-preserving particle pushers for neoclassical transport in the Vlasov equations, derived from Crank--Nicolson integrators. We show these too can can derive from a FET interpretation, similarly offering potential extensions to higher-order-in-time particle pushers. The FET formulation is used also to consider how the stochastic drift terms can be incorporated into the pushers. Stochastic gyrokinetic expansions are also discussed.

        Different options for the numerical implementation of these schemes are considered.

        Due to the efficacy of FET in the development of SP timesteppers for both the fluid and kinetic component, we hope this approach will prove effective in the future for developing SP timesteppers for the full hybrid model. We hope this will give us the opportunity to incorporate previously inaccessible kinetic effects into the highly effective, modern, finite-element MHD models.
    \end{abstract}
    
    
    \newpage
    \tableofcontents
    
    
    \newpage
    \pagenumbering{arabic}
    %\linenumbers\renewcommand\thelinenumber{\color{black!50}\arabic{linenumber}}
            \input{0 - introduction/main.tex}
        \part{Research}
            \input{1 - low-noise PiC models/main.tex}
            \input{2 - kinetic component/main.tex}
            \input{3 - fluid component/main.tex}
            \input{4 - numerical implementation/main.tex}
        \part{Project Overview}
            \input{5 - research plan/main.tex}
            \input{6 - summary/main.tex}
    
    
    %\section{}
    \newpage
    \pagenumbering{gobble}
        \printbibliography


    \newpage
    \pagenumbering{roman}
    \appendix
        \part{Appendices}
            \input{8 - Hilbert complexes/main.tex}
            \input{9 - weak conservation proofs/main.tex}
\end{document}

        \part{Project Overview}
            \documentclass[12pt, a4paper]{report}

\input{template/main.tex}

\title{\BA{Title in Progress...}}
\author{Boris Andrews}
\affil{Mathematical Institute, University of Oxford}
\date{\today}


\begin{document}
    \pagenumbering{gobble}
    \maketitle
    
    
    \begin{abstract}
        Magnetic confinement reactors---in particular tokamaks---offer one of the most promising options for achieving practical nuclear fusion, with the potential to provide virtually limitless, clean energy. The theoretical and numerical modeling of tokamak plasmas is simultaneously an essential component of effective reactor design, and a great research barrier. Tokamak operational conditions exhibit comparatively low Knudsen numbers. Kinetic effects, including kinetic waves and instabilities, Landau damping, bump-on-tail instabilities and more, are therefore highly influential in tokamak plasma dynamics. Purely fluid models are inherently incapable of capturing these effects, whereas the high dimensionality in purely kinetic models render them practically intractable for most relevant purposes.

        We consider a $\delta\!f$ decomposition model, with a macroscopic fluid background and microscopic kinetic correction, both fully coupled to each other. A similar manner of discretization is proposed to that used in the recent \texttt{STRUPHY} code \cite{Holderied_Possanner_Wang_2021, Holderied_2022, Li_et_al_2023} with a finite-element model for the background and a pseudo-particle/PiC model for the correction.

        The fluid background satisfies the full, non-linear, resistive, compressible, Hall MHD equations. \cite{Laakmann_Hu_Farrell_2022} introduces finite-element(-in-space) implicit timesteppers for the incompressible analogue to this system with structure-preserving (SP) properties in the ideal case, alongside parameter-robust preconditioners. We show that these timesteppers can derive from a finite-element-in-time (FET) (and finite-element-in-space) interpretation. The benefits of this reformulation are discussed, including the derivation of timesteppers that are higher order in time, and the quantifiable dissipative SP properties in the non-ideal, resistive case.
        
        We discuss possible options for extending this FET approach to timesteppers for the compressible case.

        The kinetic corrections satisfy linearized Boltzmann equations. Using a Lénard--Bernstein collision operator, these take Fokker--Planck-like forms \cite{Fokker_1914, Planck_1917} wherein pseudo-particles in the numerical model obey the neoclassical transport equations, with particle-independent Brownian drift terms. This offers a rigorous methodology for incorporating collisions into the particle transport model, without coupling the equations of motions for each particle.
        
        Works by Chen, Chacón et al. \cite{Chen_Chacón_Barnes_2011, Chacón_Chen_Barnes_2013, Chen_Chacón_2014, Chen_Chacón_2015} have developed structure-preserving particle pushers for neoclassical transport in the Vlasov equations, derived from Crank--Nicolson integrators. We show these too can can derive from a FET interpretation, similarly offering potential extensions to higher-order-in-time particle pushers. The FET formulation is used also to consider how the stochastic drift terms can be incorporated into the pushers. Stochastic gyrokinetic expansions are also discussed.

        Different options for the numerical implementation of these schemes are considered.

        Due to the efficacy of FET in the development of SP timesteppers for both the fluid and kinetic component, we hope this approach will prove effective in the future for developing SP timesteppers for the full hybrid model. We hope this will give us the opportunity to incorporate previously inaccessible kinetic effects into the highly effective, modern, finite-element MHD models.
    \end{abstract}
    
    
    \newpage
    \tableofcontents
    
    
    \newpage
    \pagenumbering{arabic}
    %\linenumbers\renewcommand\thelinenumber{\color{black!50}\arabic{linenumber}}
            \input{0 - introduction/main.tex}
        \part{Research}
            \input{1 - low-noise PiC models/main.tex}
            \input{2 - kinetic component/main.tex}
            \input{3 - fluid component/main.tex}
            \input{4 - numerical implementation/main.tex}
        \part{Project Overview}
            \input{5 - research plan/main.tex}
            \input{6 - summary/main.tex}
    
    
    %\section{}
    \newpage
    \pagenumbering{gobble}
        \printbibliography


    \newpage
    \pagenumbering{roman}
    \appendix
        \part{Appendices}
            \input{8 - Hilbert complexes/main.tex}
            \input{9 - weak conservation proofs/main.tex}
\end{document}

            \documentclass[12pt, a4paper]{report}

\input{template/main.tex}

\title{\BA{Title in Progress...}}
\author{Boris Andrews}
\affil{Mathematical Institute, University of Oxford}
\date{\today}


\begin{document}
    \pagenumbering{gobble}
    \maketitle
    
    
    \begin{abstract}
        Magnetic confinement reactors---in particular tokamaks---offer one of the most promising options for achieving practical nuclear fusion, with the potential to provide virtually limitless, clean energy. The theoretical and numerical modeling of tokamak plasmas is simultaneously an essential component of effective reactor design, and a great research barrier. Tokamak operational conditions exhibit comparatively low Knudsen numbers. Kinetic effects, including kinetic waves and instabilities, Landau damping, bump-on-tail instabilities and more, are therefore highly influential in tokamak plasma dynamics. Purely fluid models are inherently incapable of capturing these effects, whereas the high dimensionality in purely kinetic models render them practically intractable for most relevant purposes.

        We consider a $\delta\!f$ decomposition model, with a macroscopic fluid background and microscopic kinetic correction, both fully coupled to each other. A similar manner of discretization is proposed to that used in the recent \texttt{STRUPHY} code \cite{Holderied_Possanner_Wang_2021, Holderied_2022, Li_et_al_2023} with a finite-element model for the background and a pseudo-particle/PiC model for the correction.

        The fluid background satisfies the full, non-linear, resistive, compressible, Hall MHD equations. \cite{Laakmann_Hu_Farrell_2022} introduces finite-element(-in-space) implicit timesteppers for the incompressible analogue to this system with structure-preserving (SP) properties in the ideal case, alongside parameter-robust preconditioners. We show that these timesteppers can derive from a finite-element-in-time (FET) (and finite-element-in-space) interpretation. The benefits of this reformulation are discussed, including the derivation of timesteppers that are higher order in time, and the quantifiable dissipative SP properties in the non-ideal, resistive case.
        
        We discuss possible options for extending this FET approach to timesteppers for the compressible case.

        The kinetic corrections satisfy linearized Boltzmann equations. Using a Lénard--Bernstein collision operator, these take Fokker--Planck-like forms \cite{Fokker_1914, Planck_1917} wherein pseudo-particles in the numerical model obey the neoclassical transport equations, with particle-independent Brownian drift terms. This offers a rigorous methodology for incorporating collisions into the particle transport model, without coupling the equations of motions for each particle.
        
        Works by Chen, Chacón et al. \cite{Chen_Chacón_Barnes_2011, Chacón_Chen_Barnes_2013, Chen_Chacón_2014, Chen_Chacón_2015} have developed structure-preserving particle pushers for neoclassical transport in the Vlasov equations, derived from Crank--Nicolson integrators. We show these too can can derive from a FET interpretation, similarly offering potential extensions to higher-order-in-time particle pushers. The FET formulation is used also to consider how the stochastic drift terms can be incorporated into the pushers. Stochastic gyrokinetic expansions are also discussed.

        Different options for the numerical implementation of these schemes are considered.

        Due to the efficacy of FET in the development of SP timesteppers for both the fluid and kinetic component, we hope this approach will prove effective in the future for developing SP timesteppers for the full hybrid model. We hope this will give us the opportunity to incorporate previously inaccessible kinetic effects into the highly effective, modern, finite-element MHD models.
    \end{abstract}
    
    
    \newpage
    \tableofcontents
    
    
    \newpage
    \pagenumbering{arabic}
    %\linenumbers\renewcommand\thelinenumber{\color{black!50}\arabic{linenumber}}
            \input{0 - introduction/main.tex}
        \part{Research}
            \input{1 - low-noise PiC models/main.tex}
            \input{2 - kinetic component/main.tex}
            \input{3 - fluid component/main.tex}
            \input{4 - numerical implementation/main.tex}
        \part{Project Overview}
            \input{5 - research plan/main.tex}
            \input{6 - summary/main.tex}
    
    
    %\section{}
    \newpage
    \pagenumbering{gobble}
        \printbibliography


    \newpage
    \pagenumbering{roman}
    \appendix
        \part{Appendices}
            \input{8 - Hilbert complexes/main.tex}
            \input{9 - weak conservation proofs/main.tex}
\end{document}

    
    
    %\section{}
    \newpage
    \pagenumbering{gobble}
        \printbibliography


    \newpage
    \pagenumbering{roman}
    \appendix
        \part{Appendices}
            \documentclass[12pt, a4paper]{report}

\input{template/main.tex}

\title{\BA{Title in Progress...}}
\author{Boris Andrews}
\affil{Mathematical Institute, University of Oxford}
\date{\today}


\begin{document}
    \pagenumbering{gobble}
    \maketitle
    
    
    \begin{abstract}
        Magnetic confinement reactors---in particular tokamaks---offer one of the most promising options for achieving practical nuclear fusion, with the potential to provide virtually limitless, clean energy. The theoretical and numerical modeling of tokamak plasmas is simultaneously an essential component of effective reactor design, and a great research barrier. Tokamak operational conditions exhibit comparatively low Knudsen numbers. Kinetic effects, including kinetic waves and instabilities, Landau damping, bump-on-tail instabilities and more, are therefore highly influential in tokamak plasma dynamics. Purely fluid models are inherently incapable of capturing these effects, whereas the high dimensionality in purely kinetic models render them practically intractable for most relevant purposes.

        We consider a $\delta\!f$ decomposition model, with a macroscopic fluid background and microscopic kinetic correction, both fully coupled to each other. A similar manner of discretization is proposed to that used in the recent \texttt{STRUPHY} code \cite{Holderied_Possanner_Wang_2021, Holderied_2022, Li_et_al_2023} with a finite-element model for the background and a pseudo-particle/PiC model for the correction.

        The fluid background satisfies the full, non-linear, resistive, compressible, Hall MHD equations. \cite{Laakmann_Hu_Farrell_2022} introduces finite-element(-in-space) implicit timesteppers for the incompressible analogue to this system with structure-preserving (SP) properties in the ideal case, alongside parameter-robust preconditioners. We show that these timesteppers can derive from a finite-element-in-time (FET) (and finite-element-in-space) interpretation. The benefits of this reformulation are discussed, including the derivation of timesteppers that are higher order in time, and the quantifiable dissipative SP properties in the non-ideal, resistive case.
        
        We discuss possible options for extending this FET approach to timesteppers for the compressible case.

        The kinetic corrections satisfy linearized Boltzmann equations. Using a Lénard--Bernstein collision operator, these take Fokker--Planck-like forms \cite{Fokker_1914, Planck_1917} wherein pseudo-particles in the numerical model obey the neoclassical transport equations, with particle-independent Brownian drift terms. This offers a rigorous methodology for incorporating collisions into the particle transport model, without coupling the equations of motions for each particle.
        
        Works by Chen, Chacón et al. \cite{Chen_Chacón_Barnes_2011, Chacón_Chen_Barnes_2013, Chen_Chacón_2014, Chen_Chacón_2015} have developed structure-preserving particle pushers for neoclassical transport in the Vlasov equations, derived from Crank--Nicolson integrators. We show these too can can derive from a FET interpretation, similarly offering potential extensions to higher-order-in-time particle pushers. The FET formulation is used also to consider how the stochastic drift terms can be incorporated into the pushers. Stochastic gyrokinetic expansions are also discussed.

        Different options for the numerical implementation of these schemes are considered.

        Due to the efficacy of FET in the development of SP timesteppers for both the fluid and kinetic component, we hope this approach will prove effective in the future for developing SP timesteppers for the full hybrid model. We hope this will give us the opportunity to incorporate previously inaccessible kinetic effects into the highly effective, modern, finite-element MHD models.
    \end{abstract}
    
    
    \newpage
    \tableofcontents
    
    
    \newpage
    \pagenumbering{arabic}
    %\linenumbers\renewcommand\thelinenumber{\color{black!50}\arabic{linenumber}}
            \input{0 - introduction/main.tex}
        \part{Research}
            \input{1 - low-noise PiC models/main.tex}
            \input{2 - kinetic component/main.tex}
            \input{3 - fluid component/main.tex}
            \input{4 - numerical implementation/main.tex}
        \part{Project Overview}
            \input{5 - research plan/main.tex}
            \input{6 - summary/main.tex}
    
    
    %\section{}
    \newpage
    \pagenumbering{gobble}
        \printbibliography


    \newpage
    \pagenumbering{roman}
    \appendix
        \part{Appendices}
            \input{8 - Hilbert complexes/main.tex}
            \input{9 - weak conservation proofs/main.tex}
\end{document}

            \documentclass[12pt, a4paper]{report}

\input{template/main.tex}

\title{\BA{Title in Progress...}}
\author{Boris Andrews}
\affil{Mathematical Institute, University of Oxford}
\date{\today}


\begin{document}
    \pagenumbering{gobble}
    \maketitle
    
    
    \begin{abstract}
        Magnetic confinement reactors---in particular tokamaks---offer one of the most promising options for achieving practical nuclear fusion, with the potential to provide virtually limitless, clean energy. The theoretical and numerical modeling of tokamak plasmas is simultaneously an essential component of effective reactor design, and a great research barrier. Tokamak operational conditions exhibit comparatively low Knudsen numbers. Kinetic effects, including kinetic waves and instabilities, Landau damping, bump-on-tail instabilities and more, are therefore highly influential in tokamak plasma dynamics. Purely fluid models are inherently incapable of capturing these effects, whereas the high dimensionality in purely kinetic models render them practically intractable for most relevant purposes.

        We consider a $\delta\!f$ decomposition model, with a macroscopic fluid background and microscopic kinetic correction, both fully coupled to each other. A similar manner of discretization is proposed to that used in the recent \texttt{STRUPHY} code \cite{Holderied_Possanner_Wang_2021, Holderied_2022, Li_et_al_2023} with a finite-element model for the background and a pseudo-particle/PiC model for the correction.

        The fluid background satisfies the full, non-linear, resistive, compressible, Hall MHD equations. \cite{Laakmann_Hu_Farrell_2022} introduces finite-element(-in-space) implicit timesteppers for the incompressible analogue to this system with structure-preserving (SP) properties in the ideal case, alongside parameter-robust preconditioners. We show that these timesteppers can derive from a finite-element-in-time (FET) (and finite-element-in-space) interpretation. The benefits of this reformulation are discussed, including the derivation of timesteppers that are higher order in time, and the quantifiable dissipative SP properties in the non-ideal, resistive case.
        
        We discuss possible options for extending this FET approach to timesteppers for the compressible case.

        The kinetic corrections satisfy linearized Boltzmann equations. Using a Lénard--Bernstein collision operator, these take Fokker--Planck-like forms \cite{Fokker_1914, Planck_1917} wherein pseudo-particles in the numerical model obey the neoclassical transport equations, with particle-independent Brownian drift terms. This offers a rigorous methodology for incorporating collisions into the particle transport model, without coupling the equations of motions for each particle.
        
        Works by Chen, Chacón et al. \cite{Chen_Chacón_Barnes_2011, Chacón_Chen_Barnes_2013, Chen_Chacón_2014, Chen_Chacón_2015} have developed structure-preserving particle pushers for neoclassical transport in the Vlasov equations, derived from Crank--Nicolson integrators. We show these too can can derive from a FET interpretation, similarly offering potential extensions to higher-order-in-time particle pushers. The FET formulation is used also to consider how the stochastic drift terms can be incorporated into the pushers. Stochastic gyrokinetic expansions are also discussed.

        Different options for the numerical implementation of these schemes are considered.

        Due to the efficacy of FET in the development of SP timesteppers for both the fluid and kinetic component, we hope this approach will prove effective in the future for developing SP timesteppers for the full hybrid model. We hope this will give us the opportunity to incorporate previously inaccessible kinetic effects into the highly effective, modern, finite-element MHD models.
    \end{abstract}
    
    
    \newpage
    \tableofcontents
    
    
    \newpage
    \pagenumbering{arabic}
    %\linenumbers\renewcommand\thelinenumber{\color{black!50}\arabic{linenumber}}
            \input{0 - introduction/main.tex}
        \part{Research}
            \input{1 - low-noise PiC models/main.tex}
            \input{2 - kinetic component/main.tex}
            \input{3 - fluid component/main.tex}
            \input{4 - numerical implementation/main.tex}
        \part{Project Overview}
            \input{5 - research plan/main.tex}
            \input{6 - summary/main.tex}
    
    
    %\section{}
    \newpage
    \pagenumbering{gobble}
        \printbibliography


    \newpage
    \pagenumbering{roman}
    \appendix
        \part{Appendices}
            \input{8 - Hilbert complexes/main.tex}
            \input{9 - weak conservation proofs/main.tex}
\end{document}

\end{document}

        \part{Research}
            \documentclass[12pt, a4paper]{report}

\documentclass[12pt, a4paper]{report}

\input{template/main.tex}

\title{\BA{Title in Progress...}}
\author{Boris Andrews}
\affil{Mathematical Institute, University of Oxford}
\date{\today}


\begin{document}
    \pagenumbering{gobble}
    \maketitle
    
    
    \begin{abstract}
        Magnetic confinement reactors---in particular tokamaks---offer one of the most promising options for achieving practical nuclear fusion, with the potential to provide virtually limitless, clean energy. The theoretical and numerical modeling of tokamak plasmas is simultaneously an essential component of effective reactor design, and a great research barrier. Tokamak operational conditions exhibit comparatively low Knudsen numbers. Kinetic effects, including kinetic waves and instabilities, Landau damping, bump-on-tail instabilities and more, are therefore highly influential in tokamak plasma dynamics. Purely fluid models are inherently incapable of capturing these effects, whereas the high dimensionality in purely kinetic models render them practically intractable for most relevant purposes.

        We consider a $\delta\!f$ decomposition model, with a macroscopic fluid background and microscopic kinetic correction, both fully coupled to each other. A similar manner of discretization is proposed to that used in the recent \texttt{STRUPHY} code \cite{Holderied_Possanner_Wang_2021, Holderied_2022, Li_et_al_2023} with a finite-element model for the background and a pseudo-particle/PiC model for the correction.

        The fluid background satisfies the full, non-linear, resistive, compressible, Hall MHD equations. \cite{Laakmann_Hu_Farrell_2022} introduces finite-element(-in-space) implicit timesteppers for the incompressible analogue to this system with structure-preserving (SP) properties in the ideal case, alongside parameter-robust preconditioners. We show that these timesteppers can derive from a finite-element-in-time (FET) (and finite-element-in-space) interpretation. The benefits of this reformulation are discussed, including the derivation of timesteppers that are higher order in time, and the quantifiable dissipative SP properties in the non-ideal, resistive case.
        
        We discuss possible options for extending this FET approach to timesteppers for the compressible case.

        The kinetic corrections satisfy linearized Boltzmann equations. Using a Lénard--Bernstein collision operator, these take Fokker--Planck-like forms \cite{Fokker_1914, Planck_1917} wherein pseudo-particles in the numerical model obey the neoclassical transport equations, with particle-independent Brownian drift terms. This offers a rigorous methodology for incorporating collisions into the particle transport model, without coupling the equations of motions for each particle.
        
        Works by Chen, Chacón et al. \cite{Chen_Chacón_Barnes_2011, Chacón_Chen_Barnes_2013, Chen_Chacón_2014, Chen_Chacón_2015} have developed structure-preserving particle pushers for neoclassical transport in the Vlasov equations, derived from Crank--Nicolson integrators. We show these too can can derive from a FET interpretation, similarly offering potential extensions to higher-order-in-time particle pushers. The FET formulation is used also to consider how the stochastic drift terms can be incorporated into the pushers. Stochastic gyrokinetic expansions are also discussed.

        Different options for the numerical implementation of these schemes are considered.

        Due to the efficacy of FET in the development of SP timesteppers for both the fluid and kinetic component, we hope this approach will prove effective in the future for developing SP timesteppers for the full hybrid model. We hope this will give us the opportunity to incorporate previously inaccessible kinetic effects into the highly effective, modern, finite-element MHD models.
    \end{abstract}
    
    
    \newpage
    \tableofcontents
    
    
    \newpage
    \pagenumbering{arabic}
    %\linenumbers\renewcommand\thelinenumber{\color{black!50}\arabic{linenumber}}
            \input{0 - introduction/main.tex}
        \part{Research}
            \input{1 - low-noise PiC models/main.tex}
            \input{2 - kinetic component/main.tex}
            \input{3 - fluid component/main.tex}
            \input{4 - numerical implementation/main.tex}
        \part{Project Overview}
            \input{5 - research plan/main.tex}
            \input{6 - summary/main.tex}
    
    
    %\section{}
    \newpage
    \pagenumbering{gobble}
        \printbibliography


    \newpage
    \pagenumbering{roman}
    \appendix
        \part{Appendices}
            \input{8 - Hilbert complexes/main.tex}
            \input{9 - weak conservation proofs/main.tex}
\end{document}


\title{\BA{Title in Progress...}}
\author{Boris Andrews}
\affil{Mathematical Institute, University of Oxford}
\date{\today}


\begin{document}
    \pagenumbering{gobble}
    \maketitle
    
    
    \begin{abstract}
        Magnetic confinement reactors---in particular tokamaks---offer one of the most promising options for achieving practical nuclear fusion, with the potential to provide virtually limitless, clean energy. The theoretical and numerical modeling of tokamak plasmas is simultaneously an essential component of effective reactor design, and a great research barrier. Tokamak operational conditions exhibit comparatively low Knudsen numbers. Kinetic effects, including kinetic waves and instabilities, Landau damping, bump-on-tail instabilities and more, are therefore highly influential in tokamak plasma dynamics. Purely fluid models are inherently incapable of capturing these effects, whereas the high dimensionality in purely kinetic models render them practically intractable for most relevant purposes.

        We consider a $\delta\!f$ decomposition model, with a macroscopic fluid background and microscopic kinetic correction, both fully coupled to each other. A similar manner of discretization is proposed to that used in the recent \texttt{STRUPHY} code \cite{Holderied_Possanner_Wang_2021, Holderied_2022, Li_et_al_2023} with a finite-element model for the background and a pseudo-particle/PiC model for the correction.

        The fluid background satisfies the full, non-linear, resistive, compressible, Hall MHD equations. \cite{Laakmann_Hu_Farrell_2022} introduces finite-element(-in-space) implicit timesteppers for the incompressible analogue to this system with structure-preserving (SP) properties in the ideal case, alongside parameter-robust preconditioners. We show that these timesteppers can derive from a finite-element-in-time (FET) (and finite-element-in-space) interpretation. The benefits of this reformulation are discussed, including the derivation of timesteppers that are higher order in time, and the quantifiable dissipative SP properties in the non-ideal, resistive case.
        
        We discuss possible options for extending this FET approach to timesteppers for the compressible case.

        The kinetic corrections satisfy linearized Boltzmann equations. Using a Lénard--Bernstein collision operator, these take Fokker--Planck-like forms \cite{Fokker_1914, Planck_1917} wherein pseudo-particles in the numerical model obey the neoclassical transport equations, with particle-independent Brownian drift terms. This offers a rigorous methodology for incorporating collisions into the particle transport model, without coupling the equations of motions for each particle.
        
        Works by Chen, Chacón et al. \cite{Chen_Chacón_Barnes_2011, Chacón_Chen_Barnes_2013, Chen_Chacón_2014, Chen_Chacón_2015} have developed structure-preserving particle pushers for neoclassical transport in the Vlasov equations, derived from Crank--Nicolson integrators. We show these too can can derive from a FET interpretation, similarly offering potential extensions to higher-order-in-time particle pushers. The FET formulation is used also to consider how the stochastic drift terms can be incorporated into the pushers. Stochastic gyrokinetic expansions are also discussed.

        Different options for the numerical implementation of these schemes are considered.

        Due to the efficacy of FET in the development of SP timesteppers for both the fluid and kinetic component, we hope this approach will prove effective in the future for developing SP timesteppers for the full hybrid model. We hope this will give us the opportunity to incorporate previously inaccessible kinetic effects into the highly effective, modern, finite-element MHD models.
    \end{abstract}
    
    
    \newpage
    \tableofcontents
    
    
    \newpage
    \pagenumbering{arabic}
    %\linenumbers\renewcommand\thelinenumber{\color{black!50}\arabic{linenumber}}
            \documentclass[12pt, a4paper]{report}

\input{template/main.tex}

\title{\BA{Title in Progress...}}
\author{Boris Andrews}
\affil{Mathematical Institute, University of Oxford}
\date{\today}


\begin{document}
    \pagenumbering{gobble}
    \maketitle
    
    
    \begin{abstract}
        Magnetic confinement reactors---in particular tokamaks---offer one of the most promising options for achieving practical nuclear fusion, with the potential to provide virtually limitless, clean energy. The theoretical and numerical modeling of tokamak plasmas is simultaneously an essential component of effective reactor design, and a great research barrier. Tokamak operational conditions exhibit comparatively low Knudsen numbers. Kinetic effects, including kinetic waves and instabilities, Landau damping, bump-on-tail instabilities and more, are therefore highly influential in tokamak plasma dynamics. Purely fluid models are inherently incapable of capturing these effects, whereas the high dimensionality in purely kinetic models render them practically intractable for most relevant purposes.

        We consider a $\delta\!f$ decomposition model, with a macroscopic fluid background and microscopic kinetic correction, both fully coupled to each other. A similar manner of discretization is proposed to that used in the recent \texttt{STRUPHY} code \cite{Holderied_Possanner_Wang_2021, Holderied_2022, Li_et_al_2023} with a finite-element model for the background and a pseudo-particle/PiC model for the correction.

        The fluid background satisfies the full, non-linear, resistive, compressible, Hall MHD equations. \cite{Laakmann_Hu_Farrell_2022} introduces finite-element(-in-space) implicit timesteppers for the incompressible analogue to this system with structure-preserving (SP) properties in the ideal case, alongside parameter-robust preconditioners. We show that these timesteppers can derive from a finite-element-in-time (FET) (and finite-element-in-space) interpretation. The benefits of this reformulation are discussed, including the derivation of timesteppers that are higher order in time, and the quantifiable dissipative SP properties in the non-ideal, resistive case.
        
        We discuss possible options for extending this FET approach to timesteppers for the compressible case.

        The kinetic corrections satisfy linearized Boltzmann equations. Using a Lénard--Bernstein collision operator, these take Fokker--Planck-like forms \cite{Fokker_1914, Planck_1917} wherein pseudo-particles in the numerical model obey the neoclassical transport equations, with particle-independent Brownian drift terms. This offers a rigorous methodology for incorporating collisions into the particle transport model, without coupling the equations of motions for each particle.
        
        Works by Chen, Chacón et al. \cite{Chen_Chacón_Barnes_2011, Chacón_Chen_Barnes_2013, Chen_Chacón_2014, Chen_Chacón_2015} have developed structure-preserving particle pushers for neoclassical transport in the Vlasov equations, derived from Crank--Nicolson integrators. We show these too can can derive from a FET interpretation, similarly offering potential extensions to higher-order-in-time particle pushers. The FET formulation is used also to consider how the stochastic drift terms can be incorporated into the pushers. Stochastic gyrokinetic expansions are also discussed.

        Different options for the numerical implementation of these schemes are considered.

        Due to the efficacy of FET in the development of SP timesteppers for both the fluid and kinetic component, we hope this approach will prove effective in the future for developing SP timesteppers for the full hybrid model. We hope this will give us the opportunity to incorporate previously inaccessible kinetic effects into the highly effective, modern, finite-element MHD models.
    \end{abstract}
    
    
    \newpage
    \tableofcontents
    
    
    \newpage
    \pagenumbering{arabic}
    %\linenumbers\renewcommand\thelinenumber{\color{black!50}\arabic{linenumber}}
            \input{0 - introduction/main.tex}
        \part{Research}
            \input{1 - low-noise PiC models/main.tex}
            \input{2 - kinetic component/main.tex}
            \input{3 - fluid component/main.tex}
            \input{4 - numerical implementation/main.tex}
        \part{Project Overview}
            \input{5 - research plan/main.tex}
            \input{6 - summary/main.tex}
    
    
    %\section{}
    \newpage
    \pagenumbering{gobble}
        \printbibliography


    \newpage
    \pagenumbering{roman}
    \appendix
        \part{Appendices}
            \input{8 - Hilbert complexes/main.tex}
            \input{9 - weak conservation proofs/main.tex}
\end{document}

        \part{Research}
            \documentclass[12pt, a4paper]{report}

\input{template/main.tex}

\title{\BA{Title in Progress...}}
\author{Boris Andrews}
\affil{Mathematical Institute, University of Oxford}
\date{\today}


\begin{document}
    \pagenumbering{gobble}
    \maketitle
    
    
    \begin{abstract}
        Magnetic confinement reactors---in particular tokamaks---offer one of the most promising options for achieving practical nuclear fusion, with the potential to provide virtually limitless, clean energy. The theoretical and numerical modeling of tokamak plasmas is simultaneously an essential component of effective reactor design, and a great research barrier. Tokamak operational conditions exhibit comparatively low Knudsen numbers. Kinetic effects, including kinetic waves and instabilities, Landau damping, bump-on-tail instabilities and more, are therefore highly influential in tokamak plasma dynamics. Purely fluid models are inherently incapable of capturing these effects, whereas the high dimensionality in purely kinetic models render them practically intractable for most relevant purposes.

        We consider a $\delta\!f$ decomposition model, with a macroscopic fluid background and microscopic kinetic correction, both fully coupled to each other. A similar manner of discretization is proposed to that used in the recent \texttt{STRUPHY} code \cite{Holderied_Possanner_Wang_2021, Holderied_2022, Li_et_al_2023} with a finite-element model for the background and a pseudo-particle/PiC model for the correction.

        The fluid background satisfies the full, non-linear, resistive, compressible, Hall MHD equations. \cite{Laakmann_Hu_Farrell_2022} introduces finite-element(-in-space) implicit timesteppers for the incompressible analogue to this system with structure-preserving (SP) properties in the ideal case, alongside parameter-robust preconditioners. We show that these timesteppers can derive from a finite-element-in-time (FET) (and finite-element-in-space) interpretation. The benefits of this reformulation are discussed, including the derivation of timesteppers that are higher order in time, and the quantifiable dissipative SP properties in the non-ideal, resistive case.
        
        We discuss possible options for extending this FET approach to timesteppers for the compressible case.

        The kinetic corrections satisfy linearized Boltzmann equations. Using a Lénard--Bernstein collision operator, these take Fokker--Planck-like forms \cite{Fokker_1914, Planck_1917} wherein pseudo-particles in the numerical model obey the neoclassical transport equations, with particle-independent Brownian drift terms. This offers a rigorous methodology for incorporating collisions into the particle transport model, without coupling the equations of motions for each particle.
        
        Works by Chen, Chacón et al. \cite{Chen_Chacón_Barnes_2011, Chacón_Chen_Barnes_2013, Chen_Chacón_2014, Chen_Chacón_2015} have developed structure-preserving particle pushers for neoclassical transport in the Vlasov equations, derived from Crank--Nicolson integrators. We show these too can can derive from a FET interpretation, similarly offering potential extensions to higher-order-in-time particle pushers. The FET formulation is used also to consider how the stochastic drift terms can be incorporated into the pushers. Stochastic gyrokinetic expansions are also discussed.

        Different options for the numerical implementation of these schemes are considered.

        Due to the efficacy of FET in the development of SP timesteppers for both the fluid and kinetic component, we hope this approach will prove effective in the future for developing SP timesteppers for the full hybrid model. We hope this will give us the opportunity to incorporate previously inaccessible kinetic effects into the highly effective, modern, finite-element MHD models.
    \end{abstract}
    
    
    \newpage
    \tableofcontents
    
    
    \newpage
    \pagenumbering{arabic}
    %\linenumbers\renewcommand\thelinenumber{\color{black!50}\arabic{linenumber}}
            \input{0 - introduction/main.tex}
        \part{Research}
            \input{1 - low-noise PiC models/main.tex}
            \input{2 - kinetic component/main.tex}
            \input{3 - fluid component/main.tex}
            \input{4 - numerical implementation/main.tex}
        \part{Project Overview}
            \input{5 - research plan/main.tex}
            \input{6 - summary/main.tex}
    
    
    %\section{}
    \newpage
    \pagenumbering{gobble}
        \printbibliography


    \newpage
    \pagenumbering{roman}
    \appendix
        \part{Appendices}
            \input{8 - Hilbert complexes/main.tex}
            \input{9 - weak conservation proofs/main.tex}
\end{document}

            \documentclass[12pt, a4paper]{report}

\input{template/main.tex}

\title{\BA{Title in Progress...}}
\author{Boris Andrews}
\affil{Mathematical Institute, University of Oxford}
\date{\today}


\begin{document}
    \pagenumbering{gobble}
    \maketitle
    
    
    \begin{abstract}
        Magnetic confinement reactors---in particular tokamaks---offer one of the most promising options for achieving practical nuclear fusion, with the potential to provide virtually limitless, clean energy. The theoretical and numerical modeling of tokamak plasmas is simultaneously an essential component of effective reactor design, and a great research barrier. Tokamak operational conditions exhibit comparatively low Knudsen numbers. Kinetic effects, including kinetic waves and instabilities, Landau damping, bump-on-tail instabilities and more, are therefore highly influential in tokamak plasma dynamics. Purely fluid models are inherently incapable of capturing these effects, whereas the high dimensionality in purely kinetic models render them practically intractable for most relevant purposes.

        We consider a $\delta\!f$ decomposition model, with a macroscopic fluid background and microscopic kinetic correction, both fully coupled to each other. A similar manner of discretization is proposed to that used in the recent \texttt{STRUPHY} code \cite{Holderied_Possanner_Wang_2021, Holderied_2022, Li_et_al_2023} with a finite-element model for the background and a pseudo-particle/PiC model for the correction.

        The fluid background satisfies the full, non-linear, resistive, compressible, Hall MHD equations. \cite{Laakmann_Hu_Farrell_2022} introduces finite-element(-in-space) implicit timesteppers for the incompressible analogue to this system with structure-preserving (SP) properties in the ideal case, alongside parameter-robust preconditioners. We show that these timesteppers can derive from a finite-element-in-time (FET) (and finite-element-in-space) interpretation. The benefits of this reformulation are discussed, including the derivation of timesteppers that are higher order in time, and the quantifiable dissipative SP properties in the non-ideal, resistive case.
        
        We discuss possible options for extending this FET approach to timesteppers for the compressible case.

        The kinetic corrections satisfy linearized Boltzmann equations. Using a Lénard--Bernstein collision operator, these take Fokker--Planck-like forms \cite{Fokker_1914, Planck_1917} wherein pseudo-particles in the numerical model obey the neoclassical transport equations, with particle-independent Brownian drift terms. This offers a rigorous methodology for incorporating collisions into the particle transport model, without coupling the equations of motions for each particle.
        
        Works by Chen, Chacón et al. \cite{Chen_Chacón_Barnes_2011, Chacón_Chen_Barnes_2013, Chen_Chacón_2014, Chen_Chacón_2015} have developed structure-preserving particle pushers for neoclassical transport in the Vlasov equations, derived from Crank--Nicolson integrators. We show these too can can derive from a FET interpretation, similarly offering potential extensions to higher-order-in-time particle pushers. The FET formulation is used also to consider how the stochastic drift terms can be incorporated into the pushers. Stochastic gyrokinetic expansions are also discussed.

        Different options for the numerical implementation of these schemes are considered.

        Due to the efficacy of FET in the development of SP timesteppers for both the fluid and kinetic component, we hope this approach will prove effective in the future for developing SP timesteppers for the full hybrid model. We hope this will give us the opportunity to incorporate previously inaccessible kinetic effects into the highly effective, modern, finite-element MHD models.
    \end{abstract}
    
    
    \newpage
    \tableofcontents
    
    
    \newpage
    \pagenumbering{arabic}
    %\linenumbers\renewcommand\thelinenumber{\color{black!50}\arabic{linenumber}}
            \input{0 - introduction/main.tex}
        \part{Research}
            \input{1 - low-noise PiC models/main.tex}
            \input{2 - kinetic component/main.tex}
            \input{3 - fluid component/main.tex}
            \input{4 - numerical implementation/main.tex}
        \part{Project Overview}
            \input{5 - research plan/main.tex}
            \input{6 - summary/main.tex}
    
    
    %\section{}
    \newpage
    \pagenumbering{gobble}
        \printbibliography


    \newpage
    \pagenumbering{roman}
    \appendix
        \part{Appendices}
            \input{8 - Hilbert complexes/main.tex}
            \input{9 - weak conservation proofs/main.tex}
\end{document}

            \documentclass[12pt, a4paper]{report}

\input{template/main.tex}

\title{\BA{Title in Progress...}}
\author{Boris Andrews}
\affil{Mathematical Institute, University of Oxford}
\date{\today}


\begin{document}
    \pagenumbering{gobble}
    \maketitle
    
    
    \begin{abstract}
        Magnetic confinement reactors---in particular tokamaks---offer one of the most promising options for achieving practical nuclear fusion, with the potential to provide virtually limitless, clean energy. The theoretical and numerical modeling of tokamak plasmas is simultaneously an essential component of effective reactor design, and a great research barrier. Tokamak operational conditions exhibit comparatively low Knudsen numbers. Kinetic effects, including kinetic waves and instabilities, Landau damping, bump-on-tail instabilities and more, are therefore highly influential in tokamak plasma dynamics. Purely fluid models are inherently incapable of capturing these effects, whereas the high dimensionality in purely kinetic models render them practically intractable for most relevant purposes.

        We consider a $\delta\!f$ decomposition model, with a macroscopic fluid background and microscopic kinetic correction, both fully coupled to each other. A similar manner of discretization is proposed to that used in the recent \texttt{STRUPHY} code \cite{Holderied_Possanner_Wang_2021, Holderied_2022, Li_et_al_2023} with a finite-element model for the background and a pseudo-particle/PiC model for the correction.

        The fluid background satisfies the full, non-linear, resistive, compressible, Hall MHD equations. \cite{Laakmann_Hu_Farrell_2022} introduces finite-element(-in-space) implicit timesteppers for the incompressible analogue to this system with structure-preserving (SP) properties in the ideal case, alongside parameter-robust preconditioners. We show that these timesteppers can derive from a finite-element-in-time (FET) (and finite-element-in-space) interpretation. The benefits of this reformulation are discussed, including the derivation of timesteppers that are higher order in time, and the quantifiable dissipative SP properties in the non-ideal, resistive case.
        
        We discuss possible options for extending this FET approach to timesteppers for the compressible case.

        The kinetic corrections satisfy linearized Boltzmann equations. Using a Lénard--Bernstein collision operator, these take Fokker--Planck-like forms \cite{Fokker_1914, Planck_1917} wherein pseudo-particles in the numerical model obey the neoclassical transport equations, with particle-independent Brownian drift terms. This offers a rigorous methodology for incorporating collisions into the particle transport model, without coupling the equations of motions for each particle.
        
        Works by Chen, Chacón et al. \cite{Chen_Chacón_Barnes_2011, Chacón_Chen_Barnes_2013, Chen_Chacón_2014, Chen_Chacón_2015} have developed structure-preserving particle pushers for neoclassical transport in the Vlasov equations, derived from Crank--Nicolson integrators. We show these too can can derive from a FET interpretation, similarly offering potential extensions to higher-order-in-time particle pushers. The FET formulation is used also to consider how the stochastic drift terms can be incorporated into the pushers. Stochastic gyrokinetic expansions are also discussed.

        Different options for the numerical implementation of these schemes are considered.

        Due to the efficacy of FET in the development of SP timesteppers for both the fluid and kinetic component, we hope this approach will prove effective in the future for developing SP timesteppers for the full hybrid model. We hope this will give us the opportunity to incorporate previously inaccessible kinetic effects into the highly effective, modern, finite-element MHD models.
    \end{abstract}
    
    
    \newpage
    \tableofcontents
    
    
    \newpage
    \pagenumbering{arabic}
    %\linenumbers\renewcommand\thelinenumber{\color{black!50}\arabic{linenumber}}
            \input{0 - introduction/main.tex}
        \part{Research}
            \input{1 - low-noise PiC models/main.tex}
            \input{2 - kinetic component/main.tex}
            \input{3 - fluid component/main.tex}
            \input{4 - numerical implementation/main.tex}
        \part{Project Overview}
            \input{5 - research plan/main.tex}
            \input{6 - summary/main.tex}
    
    
    %\section{}
    \newpage
    \pagenumbering{gobble}
        \printbibliography


    \newpage
    \pagenumbering{roman}
    \appendix
        \part{Appendices}
            \input{8 - Hilbert complexes/main.tex}
            \input{9 - weak conservation proofs/main.tex}
\end{document}

            \documentclass[12pt, a4paper]{report}

\input{template/main.tex}

\title{\BA{Title in Progress...}}
\author{Boris Andrews}
\affil{Mathematical Institute, University of Oxford}
\date{\today}


\begin{document}
    \pagenumbering{gobble}
    \maketitle
    
    
    \begin{abstract}
        Magnetic confinement reactors---in particular tokamaks---offer one of the most promising options for achieving practical nuclear fusion, with the potential to provide virtually limitless, clean energy. The theoretical and numerical modeling of tokamak plasmas is simultaneously an essential component of effective reactor design, and a great research barrier. Tokamak operational conditions exhibit comparatively low Knudsen numbers. Kinetic effects, including kinetic waves and instabilities, Landau damping, bump-on-tail instabilities and more, are therefore highly influential in tokamak plasma dynamics. Purely fluid models are inherently incapable of capturing these effects, whereas the high dimensionality in purely kinetic models render them practically intractable for most relevant purposes.

        We consider a $\delta\!f$ decomposition model, with a macroscopic fluid background and microscopic kinetic correction, both fully coupled to each other. A similar manner of discretization is proposed to that used in the recent \texttt{STRUPHY} code \cite{Holderied_Possanner_Wang_2021, Holderied_2022, Li_et_al_2023} with a finite-element model for the background and a pseudo-particle/PiC model for the correction.

        The fluid background satisfies the full, non-linear, resistive, compressible, Hall MHD equations. \cite{Laakmann_Hu_Farrell_2022} introduces finite-element(-in-space) implicit timesteppers for the incompressible analogue to this system with structure-preserving (SP) properties in the ideal case, alongside parameter-robust preconditioners. We show that these timesteppers can derive from a finite-element-in-time (FET) (and finite-element-in-space) interpretation. The benefits of this reformulation are discussed, including the derivation of timesteppers that are higher order in time, and the quantifiable dissipative SP properties in the non-ideal, resistive case.
        
        We discuss possible options for extending this FET approach to timesteppers for the compressible case.

        The kinetic corrections satisfy linearized Boltzmann equations. Using a Lénard--Bernstein collision operator, these take Fokker--Planck-like forms \cite{Fokker_1914, Planck_1917} wherein pseudo-particles in the numerical model obey the neoclassical transport equations, with particle-independent Brownian drift terms. This offers a rigorous methodology for incorporating collisions into the particle transport model, without coupling the equations of motions for each particle.
        
        Works by Chen, Chacón et al. \cite{Chen_Chacón_Barnes_2011, Chacón_Chen_Barnes_2013, Chen_Chacón_2014, Chen_Chacón_2015} have developed structure-preserving particle pushers for neoclassical transport in the Vlasov equations, derived from Crank--Nicolson integrators. We show these too can can derive from a FET interpretation, similarly offering potential extensions to higher-order-in-time particle pushers. The FET formulation is used also to consider how the stochastic drift terms can be incorporated into the pushers. Stochastic gyrokinetic expansions are also discussed.

        Different options for the numerical implementation of these schemes are considered.

        Due to the efficacy of FET in the development of SP timesteppers for both the fluid and kinetic component, we hope this approach will prove effective in the future for developing SP timesteppers for the full hybrid model. We hope this will give us the opportunity to incorporate previously inaccessible kinetic effects into the highly effective, modern, finite-element MHD models.
    \end{abstract}
    
    
    \newpage
    \tableofcontents
    
    
    \newpage
    \pagenumbering{arabic}
    %\linenumbers\renewcommand\thelinenumber{\color{black!50}\arabic{linenumber}}
            \input{0 - introduction/main.tex}
        \part{Research}
            \input{1 - low-noise PiC models/main.tex}
            \input{2 - kinetic component/main.tex}
            \input{3 - fluid component/main.tex}
            \input{4 - numerical implementation/main.tex}
        \part{Project Overview}
            \input{5 - research plan/main.tex}
            \input{6 - summary/main.tex}
    
    
    %\section{}
    \newpage
    \pagenumbering{gobble}
        \printbibliography


    \newpage
    \pagenumbering{roman}
    \appendix
        \part{Appendices}
            \input{8 - Hilbert complexes/main.tex}
            \input{9 - weak conservation proofs/main.tex}
\end{document}

        \part{Project Overview}
            \documentclass[12pt, a4paper]{report}

\input{template/main.tex}

\title{\BA{Title in Progress...}}
\author{Boris Andrews}
\affil{Mathematical Institute, University of Oxford}
\date{\today}


\begin{document}
    \pagenumbering{gobble}
    \maketitle
    
    
    \begin{abstract}
        Magnetic confinement reactors---in particular tokamaks---offer one of the most promising options for achieving practical nuclear fusion, with the potential to provide virtually limitless, clean energy. The theoretical and numerical modeling of tokamak plasmas is simultaneously an essential component of effective reactor design, and a great research barrier. Tokamak operational conditions exhibit comparatively low Knudsen numbers. Kinetic effects, including kinetic waves and instabilities, Landau damping, bump-on-tail instabilities and more, are therefore highly influential in tokamak plasma dynamics. Purely fluid models are inherently incapable of capturing these effects, whereas the high dimensionality in purely kinetic models render them practically intractable for most relevant purposes.

        We consider a $\delta\!f$ decomposition model, with a macroscopic fluid background and microscopic kinetic correction, both fully coupled to each other. A similar manner of discretization is proposed to that used in the recent \texttt{STRUPHY} code \cite{Holderied_Possanner_Wang_2021, Holderied_2022, Li_et_al_2023} with a finite-element model for the background and a pseudo-particle/PiC model for the correction.

        The fluid background satisfies the full, non-linear, resistive, compressible, Hall MHD equations. \cite{Laakmann_Hu_Farrell_2022} introduces finite-element(-in-space) implicit timesteppers for the incompressible analogue to this system with structure-preserving (SP) properties in the ideal case, alongside parameter-robust preconditioners. We show that these timesteppers can derive from a finite-element-in-time (FET) (and finite-element-in-space) interpretation. The benefits of this reformulation are discussed, including the derivation of timesteppers that are higher order in time, and the quantifiable dissipative SP properties in the non-ideal, resistive case.
        
        We discuss possible options for extending this FET approach to timesteppers for the compressible case.

        The kinetic corrections satisfy linearized Boltzmann equations. Using a Lénard--Bernstein collision operator, these take Fokker--Planck-like forms \cite{Fokker_1914, Planck_1917} wherein pseudo-particles in the numerical model obey the neoclassical transport equations, with particle-independent Brownian drift terms. This offers a rigorous methodology for incorporating collisions into the particle transport model, without coupling the equations of motions for each particle.
        
        Works by Chen, Chacón et al. \cite{Chen_Chacón_Barnes_2011, Chacón_Chen_Barnes_2013, Chen_Chacón_2014, Chen_Chacón_2015} have developed structure-preserving particle pushers for neoclassical transport in the Vlasov equations, derived from Crank--Nicolson integrators. We show these too can can derive from a FET interpretation, similarly offering potential extensions to higher-order-in-time particle pushers. The FET formulation is used also to consider how the stochastic drift terms can be incorporated into the pushers. Stochastic gyrokinetic expansions are also discussed.

        Different options for the numerical implementation of these schemes are considered.

        Due to the efficacy of FET in the development of SP timesteppers for both the fluid and kinetic component, we hope this approach will prove effective in the future for developing SP timesteppers for the full hybrid model. We hope this will give us the opportunity to incorporate previously inaccessible kinetic effects into the highly effective, modern, finite-element MHD models.
    \end{abstract}
    
    
    \newpage
    \tableofcontents
    
    
    \newpage
    \pagenumbering{arabic}
    %\linenumbers\renewcommand\thelinenumber{\color{black!50}\arabic{linenumber}}
            \input{0 - introduction/main.tex}
        \part{Research}
            \input{1 - low-noise PiC models/main.tex}
            \input{2 - kinetic component/main.tex}
            \input{3 - fluid component/main.tex}
            \input{4 - numerical implementation/main.tex}
        \part{Project Overview}
            \input{5 - research plan/main.tex}
            \input{6 - summary/main.tex}
    
    
    %\section{}
    \newpage
    \pagenumbering{gobble}
        \printbibliography


    \newpage
    \pagenumbering{roman}
    \appendix
        \part{Appendices}
            \input{8 - Hilbert complexes/main.tex}
            \input{9 - weak conservation proofs/main.tex}
\end{document}

            \documentclass[12pt, a4paper]{report}

\input{template/main.tex}

\title{\BA{Title in Progress...}}
\author{Boris Andrews}
\affil{Mathematical Institute, University of Oxford}
\date{\today}


\begin{document}
    \pagenumbering{gobble}
    \maketitle
    
    
    \begin{abstract}
        Magnetic confinement reactors---in particular tokamaks---offer one of the most promising options for achieving practical nuclear fusion, with the potential to provide virtually limitless, clean energy. The theoretical and numerical modeling of tokamak plasmas is simultaneously an essential component of effective reactor design, and a great research barrier. Tokamak operational conditions exhibit comparatively low Knudsen numbers. Kinetic effects, including kinetic waves and instabilities, Landau damping, bump-on-tail instabilities and more, are therefore highly influential in tokamak plasma dynamics. Purely fluid models are inherently incapable of capturing these effects, whereas the high dimensionality in purely kinetic models render them practically intractable for most relevant purposes.

        We consider a $\delta\!f$ decomposition model, with a macroscopic fluid background and microscopic kinetic correction, both fully coupled to each other. A similar manner of discretization is proposed to that used in the recent \texttt{STRUPHY} code \cite{Holderied_Possanner_Wang_2021, Holderied_2022, Li_et_al_2023} with a finite-element model for the background and a pseudo-particle/PiC model for the correction.

        The fluid background satisfies the full, non-linear, resistive, compressible, Hall MHD equations. \cite{Laakmann_Hu_Farrell_2022} introduces finite-element(-in-space) implicit timesteppers for the incompressible analogue to this system with structure-preserving (SP) properties in the ideal case, alongside parameter-robust preconditioners. We show that these timesteppers can derive from a finite-element-in-time (FET) (and finite-element-in-space) interpretation. The benefits of this reformulation are discussed, including the derivation of timesteppers that are higher order in time, and the quantifiable dissipative SP properties in the non-ideal, resistive case.
        
        We discuss possible options for extending this FET approach to timesteppers for the compressible case.

        The kinetic corrections satisfy linearized Boltzmann equations. Using a Lénard--Bernstein collision operator, these take Fokker--Planck-like forms \cite{Fokker_1914, Planck_1917} wherein pseudo-particles in the numerical model obey the neoclassical transport equations, with particle-independent Brownian drift terms. This offers a rigorous methodology for incorporating collisions into the particle transport model, without coupling the equations of motions for each particle.
        
        Works by Chen, Chacón et al. \cite{Chen_Chacón_Barnes_2011, Chacón_Chen_Barnes_2013, Chen_Chacón_2014, Chen_Chacón_2015} have developed structure-preserving particle pushers for neoclassical transport in the Vlasov equations, derived from Crank--Nicolson integrators. We show these too can can derive from a FET interpretation, similarly offering potential extensions to higher-order-in-time particle pushers. The FET formulation is used also to consider how the stochastic drift terms can be incorporated into the pushers. Stochastic gyrokinetic expansions are also discussed.

        Different options for the numerical implementation of these schemes are considered.

        Due to the efficacy of FET in the development of SP timesteppers for both the fluid and kinetic component, we hope this approach will prove effective in the future for developing SP timesteppers for the full hybrid model. We hope this will give us the opportunity to incorporate previously inaccessible kinetic effects into the highly effective, modern, finite-element MHD models.
    \end{abstract}
    
    
    \newpage
    \tableofcontents
    
    
    \newpage
    \pagenumbering{arabic}
    %\linenumbers\renewcommand\thelinenumber{\color{black!50}\arabic{linenumber}}
            \input{0 - introduction/main.tex}
        \part{Research}
            \input{1 - low-noise PiC models/main.tex}
            \input{2 - kinetic component/main.tex}
            \input{3 - fluid component/main.tex}
            \input{4 - numerical implementation/main.tex}
        \part{Project Overview}
            \input{5 - research plan/main.tex}
            \input{6 - summary/main.tex}
    
    
    %\section{}
    \newpage
    \pagenumbering{gobble}
        \printbibliography


    \newpage
    \pagenumbering{roman}
    \appendix
        \part{Appendices}
            \input{8 - Hilbert complexes/main.tex}
            \input{9 - weak conservation proofs/main.tex}
\end{document}

    
    
    %\section{}
    \newpage
    \pagenumbering{gobble}
        \printbibliography


    \newpage
    \pagenumbering{roman}
    \appendix
        \part{Appendices}
            \documentclass[12pt, a4paper]{report}

\input{template/main.tex}

\title{\BA{Title in Progress...}}
\author{Boris Andrews}
\affil{Mathematical Institute, University of Oxford}
\date{\today}


\begin{document}
    \pagenumbering{gobble}
    \maketitle
    
    
    \begin{abstract}
        Magnetic confinement reactors---in particular tokamaks---offer one of the most promising options for achieving practical nuclear fusion, with the potential to provide virtually limitless, clean energy. The theoretical and numerical modeling of tokamak plasmas is simultaneously an essential component of effective reactor design, and a great research barrier. Tokamak operational conditions exhibit comparatively low Knudsen numbers. Kinetic effects, including kinetic waves and instabilities, Landau damping, bump-on-tail instabilities and more, are therefore highly influential in tokamak plasma dynamics. Purely fluid models are inherently incapable of capturing these effects, whereas the high dimensionality in purely kinetic models render them practically intractable for most relevant purposes.

        We consider a $\delta\!f$ decomposition model, with a macroscopic fluid background and microscopic kinetic correction, both fully coupled to each other. A similar manner of discretization is proposed to that used in the recent \texttt{STRUPHY} code \cite{Holderied_Possanner_Wang_2021, Holderied_2022, Li_et_al_2023} with a finite-element model for the background and a pseudo-particle/PiC model for the correction.

        The fluid background satisfies the full, non-linear, resistive, compressible, Hall MHD equations. \cite{Laakmann_Hu_Farrell_2022} introduces finite-element(-in-space) implicit timesteppers for the incompressible analogue to this system with structure-preserving (SP) properties in the ideal case, alongside parameter-robust preconditioners. We show that these timesteppers can derive from a finite-element-in-time (FET) (and finite-element-in-space) interpretation. The benefits of this reformulation are discussed, including the derivation of timesteppers that are higher order in time, and the quantifiable dissipative SP properties in the non-ideal, resistive case.
        
        We discuss possible options for extending this FET approach to timesteppers for the compressible case.

        The kinetic corrections satisfy linearized Boltzmann equations. Using a Lénard--Bernstein collision operator, these take Fokker--Planck-like forms \cite{Fokker_1914, Planck_1917} wherein pseudo-particles in the numerical model obey the neoclassical transport equations, with particle-independent Brownian drift terms. This offers a rigorous methodology for incorporating collisions into the particle transport model, without coupling the equations of motions for each particle.
        
        Works by Chen, Chacón et al. \cite{Chen_Chacón_Barnes_2011, Chacón_Chen_Barnes_2013, Chen_Chacón_2014, Chen_Chacón_2015} have developed structure-preserving particle pushers for neoclassical transport in the Vlasov equations, derived from Crank--Nicolson integrators. We show these too can can derive from a FET interpretation, similarly offering potential extensions to higher-order-in-time particle pushers. The FET formulation is used also to consider how the stochastic drift terms can be incorporated into the pushers. Stochastic gyrokinetic expansions are also discussed.

        Different options for the numerical implementation of these schemes are considered.

        Due to the efficacy of FET in the development of SP timesteppers for both the fluid and kinetic component, we hope this approach will prove effective in the future for developing SP timesteppers for the full hybrid model. We hope this will give us the opportunity to incorporate previously inaccessible kinetic effects into the highly effective, modern, finite-element MHD models.
    \end{abstract}
    
    
    \newpage
    \tableofcontents
    
    
    \newpage
    \pagenumbering{arabic}
    %\linenumbers\renewcommand\thelinenumber{\color{black!50}\arabic{linenumber}}
            \input{0 - introduction/main.tex}
        \part{Research}
            \input{1 - low-noise PiC models/main.tex}
            \input{2 - kinetic component/main.tex}
            \input{3 - fluid component/main.tex}
            \input{4 - numerical implementation/main.tex}
        \part{Project Overview}
            \input{5 - research plan/main.tex}
            \input{6 - summary/main.tex}
    
    
    %\section{}
    \newpage
    \pagenumbering{gobble}
        \printbibliography


    \newpage
    \pagenumbering{roman}
    \appendix
        \part{Appendices}
            \input{8 - Hilbert complexes/main.tex}
            \input{9 - weak conservation proofs/main.tex}
\end{document}

            \documentclass[12pt, a4paper]{report}

\input{template/main.tex}

\title{\BA{Title in Progress...}}
\author{Boris Andrews}
\affil{Mathematical Institute, University of Oxford}
\date{\today}


\begin{document}
    \pagenumbering{gobble}
    \maketitle
    
    
    \begin{abstract}
        Magnetic confinement reactors---in particular tokamaks---offer one of the most promising options for achieving practical nuclear fusion, with the potential to provide virtually limitless, clean energy. The theoretical and numerical modeling of tokamak plasmas is simultaneously an essential component of effective reactor design, and a great research barrier. Tokamak operational conditions exhibit comparatively low Knudsen numbers. Kinetic effects, including kinetic waves and instabilities, Landau damping, bump-on-tail instabilities and more, are therefore highly influential in tokamak plasma dynamics. Purely fluid models are inherently incapable of capturing these effects, whereas the high dimensionality in purely kinetic models render them practically intractable for most relevant purposes.

        We consider a $\delta\!f$ decomposition model, with a macroscopic fluid background and microscopic kinetic correction, both fully coupled to each other. A similar manner of discretization is proposed to that used in the recent \texttt{STRUPHY} code \cite{Holderied_Possanner_Wang_2021, Holderied_2022, Li_et_al_2023} with a finite-element model for the background and a pseudo-particle/PiC model for the correction.

        The fluid background satisfies the full, non-linear, resistive, compressible, Hall MHD equations. \cite{Laakmann_Hu_Farrell_2022} introduces finite-element(-in-space) implicit timesteppers for the incompressible analogue to this system with structure-preserving (SP) properties in the ideal case, alongside parameter-robust preconditioners. We show that these timesteppers can derive from a finite-element-in-time (FET) (and finite-element-in-space) interpretation. The benefits of this reformulation are discussed, including the derivation of timesteppers that are higher order in time, and the quantifiable dissipative SP properties in the non-ideal, resistive case.
        
        We discuss possible options for extending this FET approach to timesteppers for the compressible case.

        The kinetic corrections satisfy linearized Boltzmann equations. Using a Lénard--Bernstein collision operator, these take Fokker--Planck-like forms \cite{Fokker_1914, Planck_1917} wherein pseudo-particles in the numerical model obey the neoclassical transport equations, with particle-independent Brownian drift terms. This offers a rigorous methodology for incorporating collisions into the particle transport model, without coupling the equations of motions for each particle.
        
        Works by Chen, Chacón et al. \cite{Chen_Chacón_Barnes_2011, Chacón_Chen_Barnes_2013, Chen_Chacón_2014, Chen_Chacón_2015} have developed structure-preserving particle pushers for neoclassical transport in the Vlasov equations, derived from Crank--Nicolson integrators. We show these too can can derive from a FET interpretation, similarly offering potential extensions to higher-order-in-time particle pushers. The FET formulation is used also to consider how the stochastic drift terms can be incorporated into the pushers. Stochastic gyrokinetic expansions are also discussed.

        Different options for the numerical implementation of these schemes are considered.

        Due to the efficacy of FET in the development of SP timesteppers for both the fluid and kinetic component, we hope this approach will prove effective in the future for developing SP timesteppers for the full hybrid model. We hope this will give us the opportunity to incorporate previously inaccessible kinetic effects into the highly effective, modern, finite-element MHD models.
    \end{abstract}
    
    
    \newpage
    \tableofcontents
    
    
    \newpage
    \pagenumbering{arabic}
    %\linenumbers\renewcommand\thelinenumber{\color{black!50}\arabic{linenumber}}
            \input{0 - introduction/main.tex}
        \part{Research}
            \input{1 - low-noise PiC models/main.tex}
            \input{2 - kinetic component/main.tex}
            \input{3 - fluid component/main.tex}
            \input{4 - numerical implementation/main.tex}
        \part{Project Overview}
            \input{5 - research plan/main.tex}
            \input{6 - summary/main.tex}
    
    
    %\section{}
    \newpage
    \pagenumbering{gobble}
        \printbibliography


    \newpage
    \pagenumbering{roman}
    \appendix
        \part{Appendices}
            \input{8 - Hilbert complexes/main.tex}
            \input{9 - weak conservation proofs/main.tex}
\end{document}

\end{document}

            \documentclass[12pt, a4paper]{report}

\documentclass[12pt, a4paper]{report}

\input{template/main.tex}

\title{\BA{Title in Progress...}}
\author{Boris Andrews}
\affil{Mathematical Institute, University of Oxford}
\date{\today}


\begin{document}
    \pagenumbering{gobble}
    \maketitle
    
    
    \begin{abstract}
        Magnetic confinement reactors---in particular tokamaks---offer one of the most promising options for achieving practical nuclear fusion, with the potential to provide virtually limitless, clean energy. The theoretical and numerical modeling of tokamak plasmas is simultaneously an essential component of effective reactor design, and a great research barrier. Tokamak operational conditions exhibit comparatively low Knudsen numbers. Kinetic effects, including kinetic waves and instabilities, Landau damping, bump-on-tail instabilities and more, are therefore highly influential in tokamak plasma dynamics. Purely fluid models are inherently incapable of capturing these effects, whereas the high dimensionality in purely kinetic models render them practically intractable for most relevant purposes.

        We consider a $\delta\!f$ decomposition model, with a macroscopic fluid background and microscopic kinetic correction, both fully coupled to each other. A similar manner of discretization is proposed to that used in the recent \texttt{STRUPHY} code \cite{Holderied_Possanner_Wang_2021, Holderied_2022, Li_et_al_2023} with a finite-element model for the background and a pseudo-particle/PiC model for the correction.

        The fluid background satisfies the full, non-linear, resistive, compressible, Hall MHD equations. \cite{Laakmann_Hu_Farrell_2022} introduces finite-element(-in-space) implicit timesteppers for the incompressible analogue to this system with structure-preserving (SP) properties in the ideal case, alongside parameter-robust preconditioners. We show that these timesteppers can derive from a finite-element-in-time (FET) (and finite-element-in-space) interpretation. The benefits of this reformulation are discussed, including the derivation of timesteppers that are higher order in time, and the quantifiable dissipative SP properties in the non-ideal, resistive case.
        
        We discuss possible options for extending this FET approach to timesteppers for the compressible case.

        The kinetic corrections satisfy linearized Boltzmann equations. Using a Lénard--Bernstein collision operator, these take Fokker--Planck-like forms \cite{Fokker_1914, Planck_1917} wherein pseudo-particles in the numerical model obey the neoclassical transport equations, with particle-independent Brownian drift terms. This offers a rigorous methodology for incorporating collisions into the particle transport model, without coupling the equations of motions for each particle.
        
        Works by Chen, Chacón et al. \cite{Chen_Chacón_Barnes_2011, Chacón_Chen_Barnes_2013, Chen_Chacón_2014, Chen_Chacón_2015} have developed structure-preserving particle pushers for neoclassical transport in the Vlasov equations, derived from Crank--Nicolson integrators. We show these too can can derive from a FET interpretation, similarly offering potential extensions to higher-order-in-time particle pushers. The FET formulation is used also to consider how the stochastic drift terms can be incorporated into the pushers. Stochastic gyrokinetic expansions are also discussed.

        Different options for the numerical implementation of these schemes are considered.

        Due to the efficacy of FET in the development of SP timesteppers for both the fluid and kinetic component, we hope this approach will prove effective in the future for developing SP timesteppers for the full hybrid model. We hope this will give us the opportunity to incorporate previously inaccessible kinetic effects into the highly effective, modern, finite-element MHD models.
    \end{abstract}
    
    
    \newpage
    \tableofcontents
    
    
    \newpage
    \pagenumbering{arabic}
    %\linenumbers\renewcommand\thelinenumber{\color{black!50}\arabic{linenumber}}
            \input{0 - introduction/main.tex}
        \part{Research}
            \input{1 - low-noise PiC models/main.tex}
            \input{2 - kinetic component/main.tex}
            \input{3 - fluid component/main.tex}
            \input{4 - numerical implementation/main.tex}
        \part{Project Overview}
            \input{5 - research plan/main.tex}
            \input{6 - summary/main.tex}
    
    
    %\section{}
    \newpage
    \pagenumbering{gobble}
        \printbibliography


    \newpage
    \pagenumbering{roman}
    \appendix
        \part{Appendices}
            \input{8 - Hilbert complexes/main.tex}
            \input{9 - weak conservation proofs/main.tex}
\end{document}


\title{\BA{Title in Progress...}}
\author{Boris Andrews}
\affil{Mathematical Institute, University of Oxford}
\date{\today}


\begin{document}
    \pagenumbering{gobble}
    \maketitle
    
    
    \begin{abstract}
        Magnetic confinement reactors---in particular tokamaks---offer one of the most promising options for achieving practical nuclear fusion, with the potential to provide virtually limitless, clean energy. The theoretical and numerical modeling of tokamak plasmas is simultaneously an essential component of effective reactor design, and a great research barrier. Tokamak operational conditions exhibit comparatively low Knudsen numbers. Kinetic effects, including kinetic waves and instabilities, Landau damping, bump-on-tail instabilities and more, are therefore highly influential in tokamak plasma dynamics. Purely fluid models are inherently incapable of capturing these effects, whereas the high dimensionality in purely kinetic models render them practically intractable for most relevant purposes.

        We consider a $\delta\!f$ decomposition model, with a macroscopic fluid background and microscopic kinetic correction, both fully coupled to each other. A similar manner of discretization is proposed to that used in the recent \texttt{STRUPHY} code \cite{Holderied_Possanner_Wang_2021, Holderied_2022, Li_et_al_2023} with a finite-element model for the background and a pseudo-particle/PiC model for the correction.

        The fluid background satisfies the full, non-linear, resistive, compressible, Hall MHD equations. \cite{Laakmann_Hu_Farrell_2022} introduces finite-element(-in-space) implicit timesteppers for the incompressible analogue to this system with structure-preserving (SP) properties in the ideal case, alongside parameter-robust preconditioners. We show that these timesteppers can derive from a finite-element-in-time (FET) (and finite-element-in-space) interpretation. The benefits of this reformulation are discussed, including the derivation of timesteppers that are higher order in time, and the quantifiable dissipative SP properties in the non-ideal, resistive case.
        
        We discuss possible options for extending this FET approach to timesteppers for the compressible case.

        The kinetic corrections satisfy linearized Boltzmann equations. Using a Lénard--Bernstein collision operator, these take Fokker--Planck-like forms \cite{Fokker_1914, Planck_1917} wherein pseudo-particles in the numerical model obey the neoclassical transport equations, with particle-independent Brownian drift terms. This offers a rigorous methodology for incorporating collisions into the particle transport model, without coupling the equations of motions for each particle.
        
        Works by Chen, Chacón et al. \cite{Chen_Chacón_Barnes_2011, Chacón_Chen_Barnes_2013, Chen_Chacón_2014, Chen_Chacón_2015} have developed structure-preserving particle pushers for neoclassical transport in the Vlasov equations, derived from Crank--Nicolson integrators. We show these too can can derive from a FET interpretation, similarly offering potential extensions to higher-order-in-time particle pushers. The FET formulation is used also to consider how the stochastic drift terms can be incorporated into the pushers. Stochastic gyrokinetic expansions are also discussed.

        Different options for the numerical implementation of these schemes are considered.

        Due to the efficacy of FET in the development of SP timesteppers for both the fluid and kinetic component, we hope this approach will prove effective in the future for developing SP timesteppers for the full hybrid model. We hope this will give us the opportunity to incorporate previously inaccessible kinetic effects into the highly effective, modern, finite-element MHD models.
    \end{abstract}
    
    
    \newpage
    \tableofcontents
    
    
    \newpage
    \pagenumbering{arabic}
    %\linenumbers\renewcommand\thelinenumber{\color{black!50}\arabic{linenumber}}
            \documentclass[12pt, a4paper]{report}

\input{template/main.tex}

\title{\BA{Title in Progress...}}
\author{Boris Andrews}
\affil{Mathematical Institute, University of Oxford}
\date{\today}


\begin{document}
    \pagenumbering{gobble}
    \maketitle
    
    
    \begin{abstract}
        Magnetic confinement reactors---in particular tokamaks---offer one of the most promising options for achieving practical nuclear fusion, with the potential to provide virtually limitless, clean energy. The theoretical and numerical modeling of tokamak plasmas is simultaneously an essential component of effective reactor design, and a great research barrier. Tokamak operational conditions exhibit comparatively low Knudsen numbers. Kinetic effects, including kinetic waves and instabilities, Landau damping, bump-on-tail instabilities and more, are therefore highly influential in tokamak plasma dynamics. Purely fluid models are inherently incapable of capturing these effects, whereas the high dimensionality in purely kinetic models render them practically intractable for most relevant purposes.

        We consider a $\delta\!f$ decomposition model, with a macroscopic fluid background and microscopic kinetic correction, both fully coupled to each other. A similar manner of discretization is proposed to that used in the recent \texttt{STRUPHY} code \cite{Holderied_Possanner_Wang_2021, Holderied_2022, Li_et_al_2023} with a finite-element model for the background and a pseudo-particle/PiC model for the correction.

        The fluid background satisfies the full, non-linear, resistive, compressible, Hall MHD equations. \cite{Laakmann_Hu_Farrell_2022} introduces finite-element(-in-space) implicit timesteppers for the incompressible analogue to this system with structure-preserving (SP) properties in the ideal case, alongside parameter-robust preconditioners. We show that these timesteppers can derive from a finite-element-in-time (FET) (and finite-element-in-space) interpretation. The benefits of this reformulation are discussed, including the derivation of timesteppers that are higher order in time, and the quantifiable dissipative SP properties in the non-ideal, resistive case.
        
        We discuss possible options for extending this FET approach to timesteppers for the compressible case.

        The kinetic corrections satisfy linearized Boltzmann equations. Using a Lénard--Bernstein collision operator, these take Fokker--Planck-like forms \cite{Fokker_1914, Planck_1917} wherein pseudo-particles in the numerical model obey the neoclassical transport equations, with particle-independent Brownian drift terms. This offers a rigorous methodology for incorporating collisions into the particle transport model, without coupling the equations of motions for each particle.
        
        Works by Chen, Chacón et al. \cite{Chen_Chacón_Barnes_2011, Chacón_Chen_Barnes_2013, Chen_Chacón_2014, Chen_Chacón_2015} have developed structure-preserving particle pushers for neoclassical transport in the Vlasov equations, derived from Crank--Nicolson integrators. We show these too can can derive from a FET interpretation, similarly offering potential extensions to higher-order-in-time particle pushers. The FET formulation is used also to consider how the stochastic drift terms can be incorporated into the pushers. Stochastic gyrokinetic expansions are also discussed.

        Different options for the numerical implementation of these schemes are considered.

        Due to the efficacy of FET in the development of SP timesteppers for both the fluid and kinetic component, we hope this approach will prove effective in the future for developing SP timesteppers for the full hybrid model. We hope this will give us the opportunity to incorporate previously inaccessible kinetic effects into the highly effective, modern, finite-element MHD models.
    \end{abstract}
    
    
    \newpage
    \tableofcontents
    
    
    \newpage
    \pagenumbering{arabic}
    %\linenumbers\renewcommand\thelinenumber{\color{black!50}\arabic{linenumber}}
            \input{0 - introduction/main.tex}
        \part{Research}
            \input{1 - low-noise PiC models/main.tex}
            \input{2 - kinetic component/main.tex}
            \input{3 - fluid component/main.tex}
            \input{4 - numerical implementation/main.tex}
        \part{Project Overview}
            \input{5 - research plan/main.tex}
            \input{6 - summary/main.tex}
    
    
    %\section{}
    \newpage
    \pagenumbering{gobble}
        \printbibliography


    \newpage
    \pagenumbering{roman}
    \appendix
        \part{Appendices}
            \input{8 - Hilbert complexes/main.tex}
            \input{9 - weak conservation proofs/main.tex}
\end{document}

        \part{Research}
            \documentclass[12pt, a4paper]{report}

\input{template/main.tex}

\title{\BA{Title in Progress...}}
\author{Boris Andrews}
\affil{Mathematical Institute, University of Oxford}
\date{\today}


\begin{document}
    \pagenumbering{gobble}
    \maketitle
    
    
    \begin{abstract}
        Magnetic confinement reactors---in particular tokamaks---offer one of the most promising options for achieving practical nuclear fusion, with the potential to provide virtually limitless, clean energy. The theoretical and numerical modeling of tokamak plasmas is simultaneously an essential component of effective reactor design, and a great research barrier. Tokamak operational conditions exhibit comparatively low Knudsen numbers. Kinetic effects, including kinetic waves and instabilities, Landau damping, bump-on-tail instabilities and more, are therefore highly influential in tokamak plasma dynamics. Purely fluid models are inherently incapable of capturing these effects, whereas the high dimensionality in purely kinetic models render them practically intractable for most relevant purposes.

        We consider a $\delta\!f$ decomposition model, with a macroscopic fluid background and microscopic kinetic correction, both fully coupled to each other. A similar manner of discretization is proposed to that used in the recent \texttt{STRUPHY} code \cite{Holderied_Possanner_Wang_2021, Holderied_2022, Li_et_al_2023} with a finite-element model for the background and a pseudo-particle/PiC model for the correction.

        The fluid background satisfies the full, non-linear, resistive, compressible, Hall MHD equations. \cite{Laakmann_Hu_Farrell_2022} introduces finite-element(-in-space) implicit timesteppers for the incompressible analogue to this system with structure-preserving (SP) properties in the ideal case, alongside parameter-robust preconditioners. We show that these timesteppers can derive from a finite-element-in-time (FET) (and finite-element-in-space) interpretation. The benefits of this reformulation are discussed, including the derivation of timesteppers that are higher order in time, and the quantifiable dissipative SP properties in the non-ideal, resistive case.
        
        We discuss possible options for extending this FET approach to timesteppers for the compressible case.

        The kinetic corrections satisfy linearized Boltzmann equations. Using a Lénard--Bernstein collision operator, these take Fokker--Planck-like forms \cite{Fokker_1914, Planck_1917} wherein pseudo-particles in the numerical model obey the neoclassical transport equations, with particle-independent Brownian drift terms. This offers a rigorous methodology for incorporating collisions into the particle transport model, without coupling the equations of motions for each particle.
        
        Works by Chen, Chacón et al. \cite{Chen_Chacón_Barnes_2011, Chacón_Chen_Barnes_2013, Chen_Chacón_2014, Chen_Chacón_2015} have developed structure-preserving particle pushers for neoclassical transport in the Vlasov equations, derived from Crank--Nicolson integrators. We show these too can can derive from a FET interpretation, similarly offering potential extensions to higher-order-in-time particle pushers. The FET formulation is used also to consider how the stochastic drift terms can be incorporated into the pushers. Stochastic gyrokinetic expansions are also discussed.

        Different options for the numerical implementation of these schemes are considered.

        Due to the efficacy of FET in the development of SP timesteppers for both the fluid and kinetic component, we hope this approach will prove effective in the future for developing SP timesteppers for the full hybrid model. We hope this will give us the opportunity to incorporate previously inaccessible kinetic effects into the highly effective, modern, finite-element MHD models.
    \end{abstract}
    
    
    \newpage
    \tableofcontents
    
    
    \newpage
    \pagenumbering{arabic}
    %\linenumbers\renewcommand\thelinenumber{\color{black!50}\arabic{linenumber}}
            \input{0 - introduction/main.tex}
        \part{Research}
            \input{1 - low-noise PiC models/main.tex}
            \input{2 - kinetic component/main.tex}
            \input{3 - fluid component/main.tex}
            \input{4 - numerical implementation/main.tex}
        \part{Project Overview}
            \input{5 - research plan/main.tex}
            \input{6 - summary/main.tex}
    
    
    %\section{}
    \newpage
    \pagenumbering{gobble}
        \printbibliography


    \newpage
    \pagenumbering{roman}
    \appendix
        \part{Appendices}
            \input{8 - Hilbert complexes/main.tex}
            \input{9 - weak conservation proofs/main.tex}
\end{document}

            \documentclass[12pt, a4paper]{report}

\input{template/main.tex}

\title{\BA{Title in Progress...}}
\author{Boris Andrews}
\affil{Mathematical Institute, University of Oxford}
\date{\today}


\begin{document}
    \pagenumbering{gobble}
    \maketitle
    
    
    \begin{abstract}
        Magnetic confinement reactors---in particular tokamaks---offer one of the most promising options for achieving practical nuclear fusion, with the potential to provide virtually limitless, clean energy. The theoretical and numerical modeling of tokamak plasmas is simultaneously an essential component of effective reactor design, and a great research barrier. Tokamak operational conditions exhibit comparatively low Knudsen numbers. Kinetic effects, including kinetic waves and instabilities, Landau damping, bump-on-tail instabilities and more, are therefore highly influential in tokamak plasma dynamics. Purely fluid models are inherently incapable of capturing these effects, whereas the high dimensionality in purely kinetic models render them practically intractable for most relevant purposes.

        We consider a $\delta\!f$ decomposition model, with a macroscopic fluid background and microscopic kinetic correction, both fully coupled to each other. A similar manner of discretization is proposed to that used in the recent \texttt{STRUPHY} code \cite{Holderied_Possanner_Wang_2021, Holderied_2022, Li_et_al_2023} with a finite-element model for the background and a pseudo-particle/PiC model for the correction.

        The fluid background satisfies the full, non-linear, resistive, compressible, Hall MHD equations. \cite{Laakmann_Hu_Farrell_2022} introduces finite-element(-in-space) implicit timesteppers for the incompressible analogue to this system with structure-preserving (SP) properties in the ideal case, alongside parameter-robust preconditioners. We show that these timesteppers can derive from a finite-element-in-time (FET) (and finite-element-in-space) interpretation. The benefits of this reformulation are discussed, including the derivation of timesteppers that are higher order in time, and the quantifiable dissipative SP properties in the non-ideal, resistive case.
        
        We discuss possible options for extending this FET approach to timesteppers for the compressible case.

        The kinetic corrections satisfy linearized Boltzmann equations. Using a Lénard--Bernstein collision operator, these take Fokker--Planck-like forms \cite{Fokker_1914, Planck_1917} wherein pseudo-particles in the numerical model obey the neoclassical transport equations, with particle-independent Brownian drift terms. This offers a rigorous methodology for incorporating collisions into the particle transport model, without coupling the equations of motions for each particle.
        
        Works by Chen, Chacón et al. \cite{Chen_Chacón_Barnes_2011, Chacón_Chen_Barnes_2013, Chen_Chacón_2014, Chen_Chacón_2015} have developed structure-preserving particle pushers for neoclassical transport in the Vlasov equations, derived from Crank--Nicolson integrators. We show these too can can derive from a FET interpretation, similarly offering potential extensions to higher-order-in-time particle pushers. The FET formulation is used also to consider how the stochastic drift terms can be incorporated into the pushers. Stochastic gyrokinetic expansions are also discussed.

        Different options for the numerical implementation of these schemes are considered.

        Due to the efficacy of FET in the development of SP timesteppers for both the fluid and kinetic component, we hope this approach will prove effective in the future for developing SP timesteppers for the full hybrid model. We hope this will give us the opportunity to incorporate previously inaccessible kinetic effects into the highly effective, modern, finite-element MHD models.
    \end{abstract}
    
    
    \newpage
    \tableofcontents
    
    
    \newpage
    \pagenumbering{arabic}
    %\linenumbers\renewcommand\thelinenumber{\color{black!50}\arabic{linenumber}}
            \input{0 - introduction/main.tex}
        \part{Research}
            \input{1 - low-noise PiC models/main.tex}
            \input{2 - kinetic component/main.tex}
            \input{3 - fluid component/main.tex}
            \input{4 - numerical implementation/main.tex}
        \part{Project Overview}
            \input{5 - research plan/main.tex}
            \input{6 - summary/main.tex}
    
    
    %\section{}
    \newpage
    \pagenumbering{gobble}
        \printbibliography


    \newpage
    \pagenumbering{roman}
    \appendix
        \part{Appendices}
            \input{8 - Hilbert complexes/main.tex}
            \input{9 - weak conservation proofs/main.tex}
\end{document}

            \documentclass[12pt, a4paper]{report}

\input{template/main.tex}

\title{\BA{Title in Progress...}}
\author{Boris Andrews}
\affil{Mathematical Institute, University of Oxford}
\date{\today}


\begin{document}
    \pagenumbering{gobble}
    \maketitle
    
    
    \begin{abstract}
        Magnetic confinement reactors---in particular tokamaks---offer one of the most promising options for achieving practical nuclear fusion, with the potential to provide virtually limitless, clean energy. The theoretical and numerical modeling of tokamak plasmas is simultaneously an essential component of effective reactor design, and a great research barrier. Tokamak operational conditions exhibit comparatively low Knudsen numbers. Kinetic effects, including kinetic waves and instabilities, Landau damping, bump-on-tail instabilities and more, are therefore highly influential in tokamak plasma dynamics. Purely fluid models are inherently incapable of capturing these effects, whereas the high dimensionality in purely kinetic models render them practically intractable for most relevant purposes.

        We consider a $\delta\!f$ decomposition model, with a macroscopic fluid background and microscopic kinetic correction, both fully coupled to each other. A similar manner of discretization is proposed to that used in the recent \texttt{STRUPHY} code \cite{Holderied_Possanner_Wang_2021, Holderied_2022, Li_et_al_2023} with a finite-element model for the background and a pseudo-particle/PiC model for the correction.

        The fluid background satisfies the full, non-linear, resistive, compressible, Hall MHD equations. \cite{Laakmann_Hu_Farrell_2022} introduces finite-element(-in-space) implicit timesteppers for the incompressible analogue to this system with structure-preserving (SP) properties in the ideal case, alongside parameter-robust preconditioners. We show that these timesteppers can derive from a finite-element-in-time (FET) (and finite-element-in-space) interpretation. The benefits of this reformulation are discussed, including the derivation of timesteppers that are higher order in time, and the quantifiable dissipative SP properties in the non-ideal, resistive case.
        
        We discuss possible options for extending this FET approach to timesteppers for the compressible case.

        The kinetic corrections satisfy linearized Boltzmann equations. Using a Lénard--Bernstein collision operator, these take Fokker--Planck-like forms \cite{Fokker_1914, Planck_1917} wherein pseudo-particles in the numerical model obey the neoclassical transport equations, with particle-independent Brownian drift terms. This offers a rigorous methodology for incorporating collisions into the particle transport model, without coupling the equations of motions for each particle.
        
        Works by Chen, Chacón et al. \cite{Chen_Chacón_Barnes_2011, Chacón_Chen_Barnes_2013, Chen_Chacón_2014, Chen_Chacón_2015} have developed structure-preserving particle pushers for neoclassical transport in the Vlasov equations, derived from Crank--Nicolson integrators. We show these too can can derive from a FET interpretation, similarly offering potential extensions to higher-order-in-time particle pushers. The FET formulation is used also to consider how the stochastic drift terms can be incorporated into the pushers. Stochastic gyrokinetic expansions are also discussed.

        Different options for the numerical implementation of these schemes are considered.

        Due to the efficacy of FET in the development of SP timesteppers for both the fluid and kinetic component, we hope this approach will prove effective in the future for developing SP timesteppers for the full hybrid model. We hope this will give us the opportunity to incorporate previously inaccessible kinetic effects into the highly effective, modern, finite-element MHD models.
    \end{abstract}
    
    
    \newpage
    \tableofcontents
    
    
    \newpage
    \pagenumbering{arabic}
    %\linenumbers\renewcommand\thelinenumber{\color{black!50}\arabic{linenumber}}
            \input{0 - introduction/main.tex}
        \part{Research}
            \input{1 - low-noise PiC models/main.tex}
            \input{2 - kinetic component/main.tex}
            \input{3 - fluid component/main.tex}
            \input{4 - numerical implementation/main.tex}
        \part{Project Overview}
            \input{5 - research plan/main.tex}
            \input{6 - summary/main.tex}
    
    
    %\section{}
    \newpage
    \pagenumbering{gobble}
        \printbibliography


    \newpage
    \pagenumbering{roman}
    \appendix
        \part{Appendices}
            \input{8 - Hilbert complexes/main.tex}
            \input{9 - weak conservation proofs/main.tex}
\end{document}

            \documentclass[12pt, a4paper]{report}

\input{template/main.tex}

\title{\BA{Title in Progress...}}
\author{Boris Andrews}
\affil{Mathematical Institute, University of Oxford}
\date{\today}


\begin{document}
    \pagenumbering{gobble}
    \maketitle
    
    
    \begin{abstract}
        Magnetic confinement reactors---in particular tokamaks---offer one of the most promising options for achieving practical nuclear fusion, with the potential to provide virtually limitless, clean energy. The theoretical and numerical modeling of tokamak plasmas is simultaneously an essential component of effective reactor design, and a great research barrier. Tokamak operational conditions exhibit comparatively low Knudsen numbers. Kinetic effects, including kinetic waves and instabilities, Landau damping, bump-on-tail instabilities and more, are therefore highly influential in tokamak plasma dynamics. Purely fluid models are inherently incapable of capturing these effects, whereas the high dimensionality in purely kinetic models render them practically intractable for most relevant purposes.

        We consider a $\delta\!f$ decomposition model, with a macroscopic fluid background and microscopic kinetic correction, both fully coupled to each other. A similar manner of discretization is proposed to that used in the recent \texttt{STRUPHY} code \cite{Holderied_Possanner_Wang_2021, Holderied_2022, Li_et_al_2023} with a finite-element model for the background and a pseudo-particle/PiC model for the correction.

        The fluid background satisfies the full, non-linear, resistive, compressible, Hall MHD equations. \cite{Laakmann_Hu_Farrell_2022} introduces finite-element(-in-space) implicit timesteppers for the incompressible analogue to this system with structure-preserving (SP) properties in the ideal case, alongside parameter-robust preconditioners. We show that these timesteppers can derive from a finite-element-in-time (FET) (and finite-element-in-space) interpretation. The benefits of this reformulation are discussed, including the derivation of timesteppers that are higher order in time, and the quantifiable dissipative SP properties in the non-ideal, resistive case.
        
        We discuss possible options for extending this FET approach to timesteppers for the compressible case.

        The kinetic corrections satisfy linearized Boltzmann equations. Using a Lénard--Bernstein collision operator, these take Fokker--Planck-like forms \cite{Fokker_1914, Planck_1917} wherein pseudo-particles in the numerical model obey the neoclassical transport equations, with particle-independent Brownian drift terms. This offers a rigorous methodology for incorporating collisions into the particle transport model, without coupling the equations of motions for each particle.
        
        Works by Chen, Chacón et al. \cite{Chen_Chacón_Barnes_2011, Chacón_Chen_Barnes_2013, Chen_Chacón_2014, Chen_Chacón_2015} have developed structure-preserving particle pushers for neoclassical transport in the Vlasov equations, derived from Crank--Nicolson integrators. We show these too can can derive from a FET interpretation, similarly offering potential extensions to higher-order-in-time particle pushers. The FET formulation is used also to consider how the stochastic drift terms can be incorporated into the pushers. Stochastic gyrokinetic expansions are also discussed.

        Different options for the numerical implementation of these schemes are considered.

        Due to the efficacy of FET in the development of SP timesteppers for both the fluid and kinetic component, we hope this approach will prove effective in the future for developing SP timesteppers for the full hybrid model. We hope this will give us the opportunity to incorporate previously inaccessible kinetic effects into the highly effective, modern, finite-element MHD models.
    \end{abstract}
    
    
    \newpage
    \tableofcontents
    
    
    \newpage
    \pagenumbering{arabic}
    %\linenumbers\renewcommand\thelinenumber{\color{black!50}\arabic{linenumber}}
            \input{0 - introduction/main.tex}
        \part{Research}
            \input{1 - low-noise PiC models/main.tex}
            \input{2 - kinetic component/main.tex}
            \input{3 - fluid component/main.tex}
            \input{4 - numerical implementation/main.tex}
        \part{Project Overview}
            \input{5 - research plan/main.tex}
            \input{6 - summary/main.tex}
    
    
    %\section{}
    \newpage
    \pagenumbering{gobble}
        \printbibliography


    \newpage
    \pagenumbering{roman}
    \appendix
        \part{Appendices}
            \input{8 - Hilbert complexes/main.tex}
            \input{9 - weak conservation proofs/main.tex}
\end{document}

        \part{Project Overview}
            \documentclass[12pt, a4paper]{report}

\input{template/main.tex}

\title{\BA{Title in Progress...}}
\author{Boris Andrews}
\affil{Mathematical Institute, University of Oxford}
\date{\today}


\begin{document}
    \pagenumbering{gobble}
    \maketitle
    
    
    \begin{abstract}
        Magnetic confinement reactors---in particular tokamaks---offer one of the most promising options for achieving practical nuclear fusion, with the potential to provide virtually limitless, clean energy. The theoretical and numerical modeling of tokamak plasmas is simultaneously an essential component of effective reactor design, and a great research barrier. Tokamak operational conditions exhibit comparatively low Knudsen numbers. Kinetic effects, including kinetic waves and instabilities, Landau damping, bump-on-tail instabilities and more, are therefore highly influential in tokamak plasma dynamics. Purely fluid models are inherently incapable of capturing these effects, whereas the high dimensionality in purely kinetic models render them practically intractable for most relevant purposes.

        We consider a $\delta\!f$ decomposition model, with a macroscopic fluid background and microscopic kinetic correction, both fully coupled to each other. A similar manner of discretization is proposed to that used in the recent \texttt{STRUPHY} code \cite{Holderied_Possanner_Wang_2021, Holderied_2022, Li_et_al_2023} with a finite-element model for the background and a pseudo-particle/PiC model for the correction.

        The fluid background satisfies the full, non-linear, resistive, compressible, Hall MHD equations. \cite{Laakmann_Hu_Farrell_2022} introduces finite-element(-in-space) implicit timesteppers for the incompressible analogue to this system with structure-preserving (SP) properties in the ideal case, alongside parameter-robust preconditioners. We show that these timesteppers can derive from a finite-element-in-time (FET) (and finite-element-in-space) interpretation. The benefits of this reformulation are discussed, including the derivation of timesteppers that are higher order in time, and the quantifiable dissipative SP properties in the non-ideal, resistive case.
        
        We discuss possible options for extending this FET approach to timesteppers for the compressible case.

        The kinetic corrections satisfy linearized Boltzmann equations. Using a Lénard--Bernstein collision operator, these take Fokker--Planck-like forms \cite{Fokker_1914, Planck_1917} wherein pseudo-particles in the numerical model obey the neoclassical transport equations, with particle-independent Brownian drift terms. This offers a rigorous methodology for incorporating collisions into the particle transport model, without coupling the equations of motions for each particle.
        
        Works by Chen, Chacón et al. \cite{Chen_Chacón_Barnes_2011, Chacón_Chen_Barnes_2013, Chen_Chacón_2014, Chen_Chacón_2015} have developed structure-preserving particle pushers for neoclassical transport in the Vlasov equations, derived from Crank--Nicolson integrators. We show these too can can derive from a FET interpretation, similarly offering potential extensions to higher-order-in-time particle pushers. The FET formulation is used also to consider how the stochastic drift terms can be incorporated into the pushers. Stochastic gyrokinetic expansions are also discussed.

        Different options for the numerical implementation of these schemes are considered.

        Due to the efficacy of FET in the development of SP timesteppers for both the fluid and kinetic component, we hope this approach will prove effective in the future for developing SP timesteppers for the full hybrid model. We hope this will give us the opportunity to incorporate previously inaccessible kinetic effects into the highly effective, modern, finite-element MHD models.
    \end{abstract}
    
    
    \newpage
    \tableofcontents
    
    
    \newpage
    \pagenumbering{arabic}
    %\linenumbers\renewcommand\thelinenumber{\color{black!50}\arabic{linenumber}}
            \input{0 - introduction/main.tex}
        \part{Research}
            \input{1 - low-noise PiC models/main.tex}
            \input{2 - kinetic component/main.tex}
            \input{3 - fluid component/main.tex}
            \input{4 - numerical implementation/main.tex}
        \part{Project Overview}
            \input{5 - research plan/main.tex}
            \input{6 - summary/main.tex}
    
    
    %\section{}
    \newpage
    \pagenumbering{gobble}
        \printbibliography


    \newpage
    \pagenumbering{roman}
    \appendix
        \part{Appendices}
            \input{8 - Hilbert complexes/main.tex}
            \input{9 - weak conservation proofs/main.tex}
\end{document}

            \documentclass[12pt, a4paper]{report}

\input{template/main.tex}

\title{\BA{Title in Progress...}}
\author{Boris Andrews}
\affil{Mathematical Institute, University of Oxford}
\date{\today}


\begin{document}
    \pagenumbering{gobble}
    \maketitle
    
    
    \begin{abstract}
        Magnetic confinement reactors---in particular tokamaks---offer one of the most promising options for achieving practical nuclear fusion, with the potential to provide virtually limitless, clean energy. The theoretical and numerical modeling of tokamak plasmas is simultaneously an essential component of effective reactor design, and a great research barrier. Tokamak operational conditions exhibit comparatively low Knudsen numbers. Kinetic effects, including kinetic waves and instabilities, Landau damping, bump-on-tail instabilities and more, are therefore highly influential in tokamak plasma dynamics. Purely fluid models are inherently incapable of capturing these effects, whereas the high dimensionality in purely kinetic models render them practically intractable for most relevant purposes.

        We consider a $\delta\!f$ decomposition model, with a macroscopic fluid background and microscopic kinetic correction, both fully coupled to each other. A similar manner of discretization is proposed to that used in the recent \texttt{STRUPHY} code \cite{Holderied_Possanner_Wang_2021, Holderied_2022, Li_et_al_2023} with a finite-element model for the background and a pseudo-particle/PiC model for the correction.

        The fluid background satisfies the full, non-linear, resistive, compressible, Hall MHD equations. \cite{Laakmann_Hu_Farrell_2022} introduces finite-element(-in-space) implicit timesteppers for the incompressible analogue to this system with structure-preserving (SP) properties in the ideal case, alongside parameter-robust preconditioners. We show that these timesteppers can derive from a finite-element-in-time (FET) (and finite-element-in-space) interpretation. The benefits of this reformulation are discussed, including the derivation of timesteppers that are higher order in time, and the quantifiable dissipative SP properties in the non-ideal, resistive case.
        
        We discuss possible options for extending this FET approach to timesteppers for the compressible case.

        The kinetic corrections satisfy linearized Boltzmann equations. Using a Lénard--Bernstein collision operator, these take Fokker--Planck-like forms \cite{Fokker_1914, Planck_1917} wherein pseudo-particles in the numerical model obey the neoclassical transport equations, with particle-independent Brownian drift terms. This offers a rigorous methodology for incorporating collisions into the particle transport model, without coupling the equations of motions for each particle.
        
        Works by Chen, Chacón et al. \cite{Chen_Chacón_Barnes_2011, Chacón_Chen_Barnes_2013, Chen_Chacón_2014, Chen_Chacón_2015} have developed structure-preserving particle pushers for neoclassical transport in the Vlasov equations, derived from Crank--Nicolson integrators. We show these too can can derive from a FET interpretation, similarly offering potential extensions to higher-order-in-time particle pushers. The FET formulation is used also to consider how the stochastic drift terms can be incorporated into the pushers. Stochastic gyrokinetic expansions are also discussed.

        Different options for the numerical implementation of these schemes are considered.

        Due to the efficacy of FET in the development of SP timesteppers for both the fluid and kinetic component, we hope this approach will prove effective in the future for developing SP timesteppers for the full hybrid model. We hope this will give us the opportunity to incorporate previously inaccessible kinetic effects into the highly effective, modern, finite-element MHD models.
    \end{abstract}
    
    
    \newpage
    \tableofcontents
    
    
    \newpage
    \pagenumbering{arabic}
    %\linenumbers\renewcommand\thelinenumber{\color{black!50}\arabic{linenumber}}
            \input{0 - introduction/main.tex}
        \part{Research}
            \input{1 - low-noise PiC models/main.tex}
            \input{2 - kinetic component/main.tex}
            \input{3 - fluid component/main.tex}
            \input{4 - numerical implementation/main.tex}
        \part{Project Overview}
            \input{5 - research plan/main.tex}
            \input{6 - summary/main.tex}
    
    
    %\section{}
    \newpage
    \pagenumbering{gobble}
        \printbibliography


    \newpage
    \pagenumbering{roman}
    \appendix
        \part{Appendices}
            \input{8 - Hilbert complexes/main.tex}
            \input{9 - weak conservation proofs/main.tex}
\end{document}

    
    
    %\section{}
    \newpage
    \pagenumbering{gobble}
        \printbibliography


    \newpage
    \pagenumbering{roman}
    \appendix
        \part{Appendices}
            \documentclass[12pt, a4paper]{report}

\input{template/main.tex}

\title{\BA{Title in Progress...}}
\author{Boris Andrews}
\affil{Mathematical Institute, University of Oxford}
\date{\today}


\begin{document}
    \pagenumbering{gobble}
    \maketitle
    
    
    \begin{abstract}
        Magnetic confinement reactors---in particular tokamaks---offer one of the most promising options for achieving practical nuclear fusion, with the potential to provide virtually limitless, clean energy. The theoretical and numerical modeling of tokamak plasmas is simultaneously an essential component of effective reactor design, and a great research barrier. Tokamak operational conditions exhibit comparatively low Knudsen numbers. Kinetic effects, including kinetic waves and instabilities, Landau damping, bump-on-tail instabilities and more, are therefore highly influential in tokamak plasma dynamics. Purely fluid models are inherently incapable of capturing these effects, whereas the high dimensionality in purely kinetic models render them practically intractable for most relevant purposes.

        We consider a $\delta\!f$ decomposition model, with a macroscopic fluid background and microscopic kinetic correction, both fully coupled to each other. A similar manner of discretization is proposed to that used in the recent \texttt{STRUPHY} code \cite{Holderied_Possanner_Wang_2021, Holderied_2022, Li_et_al_2023} with a finite-element model for the background and a pseudo-particle/PiC model for the correction.

        The fluid background satisfies the full, non-linear, resistive, compressible, Hall MHD equations. \cite{Laakmann_Hu_Farrell_2022} introduces finite-element(-in-space) implicit timesteppers for the incompressible analogue to this system with structure-preserving (SP) properties in the ideal case, alongside parameter-robust preconditioners. We show that these timesteppers can derive from a finite-element-in-time (FET) (and finite-element-in-space) interpretation. The benefits of this reformulation are discussed, including the derivation of timesteppers that are higher order in time, and the quantifiable dissipative SP properties in the non-ideal, resistive case.
        
        We discuss possible options for extending this FET approach to timesteppers for the compressible case.

        The kinetic corrections satisfy linearized Boltzmann equations. Using a Lénard--Bernstein collision operator, these take Fokker--Planck-like forms \cite{Fokker_1914, Planck_1917} wherein pseudo-particles in the numerical model obey the neoclassical transport equations, with particle-independent Brownian drift terms. This offers a rigorous methodology for incorporating collisions into the particle transport model, without coupling the equations of motions for each particle.
        
        Works by Chen, Chacón et al. \cite{Chen_Chacón_Barnes_2011, Chacón_Chen_Barnes_2013, Chen_Chacón_2014, Chen_Chacón_2015} have developed structure-preserving particle pushers for neoclassical transport in the Vlasov equations, derived from Crank--Nicolson integrators. We show these too can can derive from a FET interpretation, similarly offering potential extensions to higher-order-in-time particle pushers. The FET formulation is used also to consider how the stochastic drift terms can be incorporated into the pushers. Stochastic gyrokinetic expansions are also discussed.

        Different options for the numerical implementation of these schemes are considered.

        Due to the efficacy of FET in the development of SP timesteppers for both the fluid and kinetic component, we hope this approach will prove effective in the future for developing SP timesteppers for the full hybrid model. We hope this will give us the opportunity to incorporate previously inaccessible kinetic effects into the highly effective, modern, finite-element MHD models.
    \end{abstract}
    
    
    \newpage
    \tableofcontents
    
    
    \newpage
    \pagenumbering{arabic}
    %\linenumbers\renewcommand\thelinenumber{\color{black!50}\arabic{linenumber}}
            \input{0 - introduction/main.tex}
        \part{Research}
            \input{1 - low-noise PiC models/main.tex}
            \input{2 - kinetic component/main.tex}
            \input{3 - fluid component/main.tex}
            \input{4 - numerical implementation/main.tex}
        \part{Project Overview}
            \input{5 - research plan/main.tex}
            \input{6 - summary/main.tex}
    
    
    %\section{}
    \newpage
    \pagenumbering{gobble}
        \printbibliography


    \newpage
    \pagenumbering{roman}
    \appendix
        \part{Appendices}
            \input{8 - Hilbert complexes/main.tex}
            \input{9 - weak conservation proofs/main.tex}
\end{document}

            \documentclass[12pt, a4paper]{report}

\input{template/main.tex}

\title{\BA{Title in Progress...}}
\author{Boris Andrews}
\affil{Mathematical Institute, University of Oxford}
\date{\today}


\begin{document}
    \pagenumbering{gobble}
    \maketitle
    
    
    \begin{abstract}
        Magnetic confinement reactors---in particular tokamaks---offer one of the most promising options for achieving practical nuclear fusion, with the potential to provide virtually limitless, clean energy. The theoretical and numerical modeling of tokamak plasmas is simultaneously an essential component of effective reactor design, and a great research barrier. Tokamak operational conditions exhibit comparatively low Knudsen numbers. Kinetic effects, including kinetic waves and instabilities, Landau damping, bump-on-tail instabilities and more, are therefore highly influential in tokamak plasma dynamics. Purely fluid models are inherently incapable of capturing these effects, whereas the high dimensionality in purely kinetic models render them practically intractable for most relevant purposes.

        We consider a $\delta\!f$ decomposition model, with a macroscopic fluid background and microscopic kinetic correction, both fully coupled to each other. A similar manner of discretization is proposed to that used in the recent \texttt{STRUPHY} code \cite{Holderied_Possanner_Wang_2021, Holderied_2022, Li_et_al_2023} with a finite-element model for the background and a pseudo-particle/PiC model for the correction.

        The fluid background satisfies the full, non-linear, resistive, compressible, Hall MHD equations. \cite{Laakmann_Hu_Farrell_2022} introduces finite-element(-in-space) implicit timesteppers for the incompressible analogue to this system with structure-preserving (SP) properties in the ideal case, alongside parameter-robust preconditioners. We show that these timesteppers can derive from a finite-element-in-time (FET) (and finite-element-in-space) interpretation. The benefits of this reformulation are discussed, including the derivation of timesteppers that are higher order in time, and the quantifiable dissipative SP properties in the non-ideal, resistive case.
        
        We discuss possible options for extending this FET approach to timesteppers for the compressible case.

        The kinetic corrections satisfy linearized Boltzmann equations. Using a Lénard--Bernstein collision operator, these take Fokker--Planck-like forms \cite{Fokker_1914, Planck_1917} wherein pseudo-particles in the numerical model obey the neoclassical transport equations, with particle-independent Brownian drift terms. This offers a rigorous methodology for incorporating collisions into the particle transport model, without coupling the equations of motions for each particle.
        
        Works by Chen, Chacón et al. \cite{Chen_Chacón_Barnes_2011, Chacón_Chen_Barnes_2013, Chen_Chacón_2014, Chen_Chacón_2015} have developed structure-preserving particle pushers for neoclassical transport in the Vlasov equations, derived from Crank--Nicolson integrators. We show these too can can derive from a FET interpretation, similarly offering potential extensions to higher-order-in-time particle pushers. The FET formulation is used also to consider how the stochastic drift terms can be incorporated into the pushers. Stochastic gyrokinetic expansions are also discussed.

        Different options for the numerical implementation of these schemes are considered.

        Due to the efficacy of FET in the development of SP timesteppers for both the fluid and kinetic component, we hope this approach will prove effective in the future for developing SP timesteppers for the full hybrid model. We hope this will give us the opportunity to incorporate previously inaccessible kinetic effects into the highly effective, modern, finite-element MHD models.
    \end{abstract}
    
    
    \newpage
    \tableofcontents
    
    
    \newpage
    \pagenumbering{arabic}
    %\linenumbers\renewcommand\thelinenumber{\color{black!50}\arabic{linenumber}}
            \input{0 - introduction/main.tex}
        \part{Research}
            \input{1 - low-noise PiC models/main.tex}
            \input{2 - kinetic component/main.tex}
            \input{3 - fluid component/main.tex}
            \input{4 - numerical implementation/main.tex}
        \part{Project Overview}
            \input{5 - research plan/main.tex}
            \input{6 - summary/main.tex}
    
    
    %\section{}
    \newpage
    \pagenumbering{gobble}
        \printbibliography


    \newpage
    \pagenumbering{roman}
    \appendix
        \part{Appendices}
            \input{8 - Hilbert complexes/main.tex}
            \input{9 - weak conservation proofs/main.tex}
\end{document}

\end{document}

            \documentclass[12pt, a4paper]{report}

\documentclass[12pt, a4paper]{report}

\input{template/main.tex}

\title{\BA{Title in Progress...}}
\author{Boris Andrews}
\affil{Mathematical Institute, University of Oxford}
\date{\today}


\begin{document}
    \pagenumbering{gobble}
    \maketitle
    
    
    \begin{abstract}
        Magnetic confinement reactors---in particular tokamaks---offer one of the most promising options for achieving practical nuclear fusion, with the potential to provide virtually limitless, clean energy. The theoretical and numerical modeling of tokamak plasmas is simultaneously an essential component of effective reactor design, and a great research barrier. Tokamak operational conditions exhibit comparatively low Knudsen numbers. Kinetic effects, including kinetic waves and instabilities, Landau damping, bump-on-tail instabilities and more, are therefore highly influential in tokamak plasma dynamics. Purely fluid models are inherently incapable of capturing these effects, whereas the high dimensionality in purely kinetic models render them practically intractable for most relevant purposes.

        We consider a $\delta\!f$ decomposition model, with a macroscopic fluid background and microscopic kinetic correction, both fully coupled to each other. A similar manner of discretization is proposed to that used in the recent \texttt{STRUPHY} code \cite{Holderied_Possanner_Wang_2021, Holderied_2022, Li_et_al_2023} with a finite-element model for the background and a pseudo-particle/PiC model for the correction.

        The fluid background satisfies the full, non-linear, resistive, compressible, Hall MHD equations. \cite{Laakmann_Hu_Farrell_2022} introduces finite-element(-in-space) implicit timesteppers for the incompressible analogue to this system with structure-preserving (SP) properties in the ideal case, alongside parameter-robust preconditioners. We show that these timesteppers can derive from a finite-element-in-time (FET) (and finite-element-in-space) interpretation. The benefits of this reformulation are discussed, including the derivation of timesteppers that are higher order in time, and the quantifiable dissipative SP properties in the non-ideal, resistive case.
        
        We discuss possible options for extending this FET approach to timesteppers for the compressible case.

        The kinetic corrections satisfy linearized Boltzmann equations. Using a Lénard--Bernstein collision operator, these take Fokker--Planck-like forms \cite{Fokker_1914, Planck_1917} wherein pseudo-particles in the numerical model obey the neoclassical transport equations, with particle-independent Brownian drift terms. This offers a rigorous methodology for incorporating collisions into the particle transport model, without coupling the equations of motions for each particle.
        
        Works by Chen, Chacón et al. \cite{Chen_Chacón_Barnes_2011, Chacón_Chen_Barnes_2013, Chen_Chacón_2014, Chen_Chacón_2015} have developed structure-preserving particle pushers for neoclassical transport in the Vlasov equations, derived from Crank--Nicolson integrators. We show these too can can derive from a FET interpretation, similarly offering potential extensions to higher-order-in-time particle pushers. The FET formulation is used also to consider how the stochastic drift terms can be incorporated into the pushers. Stochastic gyrokinetic expansions are also discussed.

        Different options for the numerical implementation of these schemes are considered.

        Due to the efficacy of FET in the development of SP timesteppers for both the fluid and kinetic component, we hope this approach will prove effective in the future for developing SP timesteppers for the full hybrid model. We hope this will give us the opportunity to incorporate previously inaccessible kinetic effects into the highly effective, modern, finite-element MHD models.
    \end{abstract}
    
    
    \newpage
    \tableofcontents
    
    
    \newpage
    \pagenumbering{arabic}
    %\linenumbers\renewcommand\thelinenumber{\color{black!50}\arabic{linenumber}}
            \input{0 - introduction/main.tex}
        \part{Research}
            \input{1 - low-noise PiC models/main.tex}
            \input{2 - kinetic component/main.tex}
            \input{3 - fluid component/main.tex}
            \input{4 - numerical implementation/main.tex}
        \part{Project Overview}
            \input{5 - research plan/main.tex}
            \input{6 - summary/main.tex}
    
    
    %\section{}
    \newpage
    \pagenumbering{gobble}
        \printbibliography


    \newpage
    \pagenumbering{roman}
    \appendix
        \part{Appendices}
            \input{8 - Hilbert complexes/main.tex}
            \input{9 - weak conservation proofs/main.tex}
\end{document}


\title{\BA{Title in Progress...}}
\author{Boris Andrews}
\affil{Mathematical Institute, University of Oxford}
\date{\today}


\begin{document}
    \pagenumbering{gobble}
    \maketitle
    
    
    \begin{abstract}
        Magnetic confinement reactors---in particular tokamaks---offer one of the most promising options for achieving practical nuclear fusion, with the potential to provide virtually limitless, clean energy. The theoretical and numerical modeling of tokamak plasmas is simultaneously an essential component of effective reactor design, and a great research barrier. Tokamak operational conditions exhibit comparatively low Knudsen numbers. Kinetic effects, including kinetic waves and instabilities, Landau damping, bump-on-tail instabilities and more, are therefore highly influential in tokamak plasma dynamics. Purely fluid models are inherently incapable of capturing these effects, whereas the high dimensionality in purely kinetic models render them practically intractable for most relevant purposes.

        We consider a $\delta\!f$ decomposition model, with a macroscopic fluid background and microscopic kinetic correction, both fully coupled to each other. A similar manner of discretization is proposed to that used in the recent \texttt{STRUPHY} code \cite{Holderied_Possanner_Wang_2021, Holderied_2022, Li_et_al_2023} with a finite-element model for the background and a pseudo-particle/PiC model for the correction.

        The fluid background satisfies the full, non-linear, resistive, compressible, Hall MHD equations. \cite{Laakmann_Hu_Farrell_2022} introduces finite-element(-in-space) implicit timesteppers for the incompressible analogue to this system with structure-preserving (SP) properties in the ideal case, alongside parameter-robust preconditioners. We show that these timesteppers can derive from a finite-element-in-time (FET) (and finite-element-in-space) interpretation. The benefits of this reformulation are discussed, including the derivation of timesteppers that are higher order in time, and the quantifiable dissipative SP properties in the non-ideal, resistive case.
        
        We discuss possible options for extending this FET approach to timesteppers for the compressible case.

        The kinetic corrections satisfy linearized Boltzmann equations. Using a Lénard--Bernstein collision operator, these take Fokker--Planck-like forms \cite{Fokker_1914, Planck_1917} wherein pseudo-particles in the numerical model obey the neoclassical transport equations, with particle-independent Brownian drift terms. This offers a rigorous methodology for incorporating collisions into the particle transport model, without coupling the equations of motions for each particle.
        
        Works by Chen, Chacón et al. \cite{Chen_Chacón_Barnes_2011, Chacón_Chen_Barnes_2013, Chen_Chacón_2014, Chen_Chacón_2015} have developed structure-preserving particle pushers for neoclassical transport in the Vlasov equations, derived from Crank--Nicolson integrators. We show these too can can derive from a FET interpretation, similarly offering potential extensions to higher-order-in-time particle pushers. The FET formulation is used also to consider how the stochastic drift terms can be incorporated into the pushers. Stochastic gyrokinetic expansions are also discussed.

        Different options for the numerical implementation of these schemes are considered.

        Due to the efficacy of FET in the development of SP timesteppers for both the fluid and kinetic component, we hope this approach will prove effective in the future for developing SP timesteppers for the full hybrid model. We hope this will give us the opportunity to incorporate previously inaccessible kinetic effects into the highly effective, modern, finite-element MHD models.
    \end{abstract}
    
    
    \newpage
    \tableofcontents
    
    
    \newpage
    \pagenumbering{arabic}
    %\linenumbers\renewcommand\thelinenumber{\color{black!50}\arabic{linenumber}}
            \documentclass[12pt, a4paper]{report}

\input{template/main.tex}

\title{\BA{Title in Progress...}}
\author{Boris Andrews}
\affil{Mathematical Institute, University of Oxford}
\date{\today}


\begin{document}
    \pagenumbering{gobble}
    \maketitle
    
    
    \begin{abstract}
        Magnetic confinement reactors---in particular tokamaks---offer one of the most promising options for achieving practical nuclear fusion, with the potential to provide virtually limitless, clean energy. The theoretical and numerical modeling of tokamak plasmas is simultaneously an essential component of effective reactor design, and a great research barrier. Tokamak operational conditions exhibit comparatively low Knudsen numbers. Kinetic effects, including kinetic waves and instabilities, Landau damping, bump-on-tail instabilities and more, are therefore highly influential in tokamak plasma dynamics. Purely fluid models are inherently incapable of capturing these effects, whereas the high dimensionality in purely kinetic models render them practically intractable for most relevant purposes.

        We consider a $\delta\!f$ decomposition model, with a macroscopic fluid background and microscopic kinetic correction, both fully coupled to each other. A similar manner of discretization is proposed to that used in the recent \texttt{STRUPHY} code \cite{Holderied_Possanner_Wang_2021, Holderied_2022, Li_et_al_2023} with a finite-element model for the background and a pseudo-particle/PiC model for the correction.

        The fluid background satisfies the full, non-linear, resistive, compressible, Hall MHD equations. \cite{Laakmann_Hu_Farrell_2022} introduces finite-element(-in-space) implicit timesteppers for the incompressible analogue to this system with structure-preserving (SP) properties in the ideal case, alongside parameter-robust preconditioners. We show that these timesteppers can derive from a finite-element-in-time (FET) (and finite-element-in-space) interpretation. The benefits of this reformulation are discussed, including the derivation of timesteppers that are higher order in time, and the quantifiable dissipative SP properties in the non-ideal, resistive case.
        
        We discuss possible options for extending this FET approach to timesteppers for the compressible case.

        The kinetic corrections satisfy linearized Boltzmann equations. Using a Lénard--Bernstein collision operator, these take Fokker--Planck-like forms \cite{Fokker_1914, Planck_1917} wherein pseudo-particles in the numerical model obey the neoclassical transport equations, with particle-independent Brownian drift terms. This offers a rigorous methodology for incorporating collisions into the particle transport model, without coupling the equations of motions for each particle.
        
        Works by Chen, Chacón et al. \cite{Chen_Chacón_Barnes_2011, Chacón_Chen_Barnes_2013, Chen_Chacón_2014, Chen_Chacón_2015} have developed structure-preserving particle pushers for neoclassical transport in the Vlasov equations, derived from Crank--Nicolson integrators. We show these too can can derive from a FET interpretation, similarly offering potential extensions to higher-order-in-time particle pushers. The FET formulation is used also to consider how the stochastic drift terms can be incorporated into the pushers. Stochastic gyrokinetic expansions are also discussed.

        Different options for the numerical implementation of these schemes are considered.

        Due to the efficacy of FET in the development of SP timesteppers for both the fluid and kinetic component, we hope this approach will prove effective in the future for developing SP timesteppers for the full hybrid model. We hope this will give us the opportunity to incorporate previously inaccessible kinetic effects into the highly effective, modern, finite-element MHD models.
    \end{abstract}
    
    
    \newpage
    \tableofcontents
    
    
    \newpage
    \pagenumbering{arabic}
    %\linenumbers\renewcommand\thelinenumber{\color{black!50}\arabic{linenumber}}
            \input{0 - introduction/main.tex}
        \part{Research}
            \input{1 - low-noise PiC models/main.tex}
            \input{2 - kinetic component/main.tex}
            \input{3 - fluid component/main.tex}
            \input{4 - numerical implementation/main.tex}
        \part{Project Overview}
            \input{5 - research plan/main.tex}
            \input{6 - summary/main.tex}
    
    
    %\section{}
    \newpage
    \pagenumbering{gobble}
        \printbibliography


    \newpage
    \pagenumbering{roman}
    \appendix
        \part{Appendices}
            \input{8 - Hilbert complexes/main.tex}
            \input{9 - weak conservation proofs/main.tex}
\end{document}

        \part{Research}
            \documentclass[12pt, a4paper]{report}

\input{template/main.tex}

\title{\BA{Title in Progress...}}
\author{Boris Andrews}
\affil{Mathematical Institute, University of Oxford}
\date{\today}


\begin{document}
    \pagenumbering{gobble}
    \maketitle
    
    
    \begin{abstract}
        Magnetic confinement reactors---in particular tokamaks---offer one of the most promising options for achieving practical nuclear fusion, with the potential to provide virtually limitless, clean energy. The theoretical and numerical modeling of tokamak plasmas is simultaneously an essential component of effective reactor design, and a great research barrier. Tokamak operational conditions exhibit comparatively low Knudsen numbers. Kinetic effects, including kinetic waves and instabilities, Landau damping, bump-on-tail instabilities and more, are therefore highly influential in tokamak plasma dynamics. Purely fluid models are inherently incapable of capturing these effects, whereas the high dimensionality in purely kinetic models render them practically intractable for most relevant purposes.

        We consider a $\delta\!f$ decomposition model, with a macroscopic fluid background and microscopic kinetic correction, both fully coupled to each other. A similar manner of discretization is proposed to that used in the recent \texttt{STRUPHY} code \cite{Holderied_Possanner_Wang_2021, Holderied_2022, Li_et_al_2023} with a finite-element model for the background and a pseudo-particle/PiC model for the correction.

        The fluid background satisfies the full, non-linear, resistive, compressible, Hall MHD equations. \cite{Laakmann_Hu_Farrell_2022} introduces finite-element(-in-space) implicit timesteppers for the incompressible analogue to this system with structure-preserving (SP) properties in the ideal case, alongside parameter-robust preconditioners. We show that these timesteppers can derive from a finite-element-in-time (FET) (and finite-element-in-space) interpretation. The benefits of this reformulation are discussed, including the derivation of timesteppers that are higher order in time, and the quantifiable dissipative SP properties in the non-ideal, resistive case.
        
        We discuss possible options for extending this FET approach to timesteppers for the compressible case.

        The kinetic corrections satisfy linearized Boltzmann equations. Using a Lénard--Bernstein collision operator, these take Fokker--Planck-like forms \cite{Fokker_1914, Planck_1917} wherein pseudo-particles in the numerical model obey the neoclassical transport equations, with particle-independent Brownian drift terms. This offers a rigorous methodology for incorporating collisions into the particle transport model, without coupling the equations of motions for each particle.
        
        Works by Chen, Chacón et al. \cite{Chen_Chacón_Barnes_2011, Chacón_Chen_Barnes_2013, Chen_Chacón_2014, Chen_Chacón_2015} have developed structure-preserving particle pushers for neoclassical transport in the Vlasov equations, derived from Crank--Nicolson integrators. We show these too can can derive from a FET interpretation, similarly offering potential extensions to higher-order-in-time particle pushers. The FET formulation is used also to consider how the stochastic drift terms can be incorporated into the pushers. Stochastic gyrokinetic expansions are also discussed.

        Different options for the numerical implementation of these schemes are considered.

        Due to the efficacy of FET in the development of SP timesteppers for both the fluid and kinetic component, we hope this approach will prove effective in the future for developing SP timesteppers for the full hybrid model. We hope this will give us the opportunity to incorporate previously inaccessible kinetic effects into the highly effective, modern, finite-element MHD models.
    \end{abstract}
    
    
    \newpage
    \tableofcontents
    
    
    \newpage
    \pagenumbering{arabic}
    %\linenumbers\renewcommand\thelinenumber{\color{black!50}\arabic{linenumber}}
            \input{0 - introduction/main.tex}
        \part{Research}
            \input{1 - low-noise PiC models/main.tex}
            \input{2 - kinetic component/main.tex}
            \input{3 - fluid component/main.tex}
            \input{4 - numerical implementation/main.tex}
        \part{Project Overview}
            \input{5 - research plan/main.tex}
            \input{6 - summary/main.tex}
    
    
    %\section{}
    \newpage
    \pagenumbering{gobble}
        \printbibliography


    \newpage
    \pagenumbering{roman}
    \appendix
        \part{Appendices}
            \input{8 - Hilbert complexes/main.tex}
            \input{9 - weak conservation proofs/main.tex}
\end{document}

            \documentclass[12pt, a4paper]{report}

\input{template/main.tex}

\title{\BA{Title in Progress...}}
\author{Boris Andrews}
\affil{Mathematical Institute, University of Oxford}
\date{\today}


\begin{document}
    \pagenumbering{gobble}
    \maketitle
    
    
    \begin{abstract}
        Magnetic confinement reactors---in particular tokamaks---offer one of the most promising options for achieving practical nuclear fusion, with the potential to provide virtually limitless, clean energy. The theoretical and numerical modeling of tokamak plasmas is simultaneously an essential component of effective reactor design, and a great research barrier. Tokamak operational conditions exhibit comparatively low Knudsen numbers. Kinetic effects, including kinetic waves and instabilities, Landau damping, bump-on-tail instabilities and more, are therefore highly influential in tokamak plasma dynamics. Purely fluid models are inherently incapable of capturing these effects, whereas the high dimensionality in purely kinetic models render them practically intractable for most relevant purposes.

        We consider a $\delta\!f$ decomposition model, with a macroscopic fluid background and microscopic kinetic correction, both fully coupled to each other. A similar manner of discretization is proposed to that used in the recent \texttt{STRUPHY} code \cite{Holderied_Possanner_Wang_2021, Holderied_2022, Li_et_al_2023} with a finite-element model for the background and a pseudo-particle/PiC model for the correction.

        The fluid background satisfies the full, non-linear, resistive, compressible, Hall MHD equations. \cite{Laakmann_Hu_Farrell_2022} introduces finite-element(-in-space) implicit timesteppers for the incompressible analogue to this system with structure-preserving (SP) properties in the ideal case, alongside parameter-robust preconditioners. We show that these timesteppers can derive from a finite-element-in-time (FET) (and finite-element-in-space) interpretation. The benefits of this reformulation are discussed, including the derivation of timesteppers that are higher order in time, and the quantifiable dissipative SP properties in the non-ideal, resistive case.
        
        We discuss possible options for extending this FET approach to timesteppers for the compressible case.

        The kinetic corrections satisfy linearized Boltzmann equations. Using a Lénard--Bernstein collision operator, these take Fokker--Planck-like forms \cite{Fokker_1914, Planck_1917} wherein pseudo-particles in the numerical model obey the neoclassical transport equations, with particle-independent Brownian drift terms. This offers a rigorous methodology for incorporating collisions into the particle transport model, without coupling the equations of motions for each particle.
        
        Works by Chen, Chacón et al. \cite{Chen_Chacón_Barnes_2011, Chacón_Chen_Barnes_2013, Chen_Chacón_2014, Chen_Chacón_2015} have developed structure-preserving particle pushers for neoclassical transport in the Vlasov equations, derived from Crank--Nicolson integrators. We show these too can can derive from a FET interpretation, similarly offering potential extensions to higher-order-in-time particle pushers. The FET formulation is used also to consider how the stochastic drift terms can be incorporated into the pushers. Stochastic gyrokinetic expansions are also discussed.

        Different options for the numerical implementation of these schemes are considered.

        Due to the efficacy of FET in the development of SP timesteppers for both the fluid and kinetic component, we hope this approach will prove effective in the future for developing SP timesteppers for the full hybrid model. We hope this will give us the opportunity to incorporate previously inaccessible kinetic effects into the highly effective, modern, finite-element MHD models.
    \end{abstract}
    
    
    \newpage
    \tableofcontents
    
    
    \newpage
    \pagenumbering{arabic}
    %\linenumbers\renewcommand\thelinenumber{\color{black!50}\arabic{linenumber}}
            \input{0 - introduction/main.tex}
        \part{Research}
            \input{1 - low-noise PiC models/main.tex}
            \input{2 - kinetic component/main.tex}
            \input{3 - fluid component/main.tex}
            \input{4 - numerical implementation/main.tex}
        \part{Project Overview}
            \input{5 - research plan/main.tex}
            \input{6 - summary/main.tex}
    
    
    %\section{}
    \newpage
    \pagenumbering{gobble}
        \printbibliography


    \newpage
    \pagenumbering{roman}
    \appendix
        \part{Appendices}
            \input{8 - Hilbert complexes/main.tex}
            \input{9 - weak conservation proofs/main.tex}
\end{document}

            \documentclass[12pt, a4paper]{report}

\input{template/main.tex}

\title{\BA{Title in Progress...}}
\author{Boris Andrews}
\affil{Mathematical Institute, University of Oxford}
\date{\today}


\begin{document}
    \pagenumbering{gobble}
    \maketitle
    
    
    \begin{abstract}
        Magnetic confinement reactors---in particular tokamaks---offer one of the most promising options for achieving practical nuclear fusion, with the potential to provide virtually limitless, clean energy. The theoretical and numerical modeling of tokamak plasmas is simultaneously an essential component of effective reactor design, and a great research barrier. Tokamak operational conditions exhibit comparatively low Knudsen numbers. Kinetic effects, including kinetic waves and instabilities, Landau damping, bump-on-tail instabilities and more, are therefore highly influential in tokamak plasma dynamics. Purely fluid models are inherently incapable of capturing these effects, whereas the high dimensionality in purely kinetic models render them practically intractable for most relevant purposes.

        We consider a $\delta\!f$ decomposition model, with a macroscopic fluid background and microscopic kinetic correction, both fully coupled to each other. A similar manner of discretization is proposed to that used in the recent \texttt{STRUPHY} code \cite{Holderied_Possanner_Wang_2021, Holderied_2022, Li_et_al_2023} with a finite-element model for the background and a pseudo-particle/PiC model for the correction.

        The fluid background satisfies the full, non-linear, resistive, compressible, Hall MHD equations. \cite{Laakmann_Hu_Farrell_2022} introduces finite-element(-in-space) implicit timesteppers for the incompressible analogue to this system with structure-preserving (SP) properties in the ideal case, alongside parameter-robust preconditioners. We show that these timesteppers can derive from a finite-element-in-time (FET) (and finite-element-in-space) interpretation. The benefits of this reformulation are discussed, including the derivation of timesteppers that are higher order in time, and the quantifiable dissipative SP properties in the non-ideal, resistive case.
        
        We discuss possible options for extending this FET approach to timesteppers for the compressible case.

        The kinetic corrections satisfy linearized Boltzmann equations. Using a Lénard--Bernstein collision operator, these take Fokker--Planck-like forms \cite{Fokker_1914, Planck_1917} wherein pseudo-particles in the numerical model obey the neoclassical transport equations, with particle-independent Brownian drift terms. This offers a rigorous methodology for incorporating collisions into the particle transport model, without coupling the equations of motions for each particle.
        
        Works by Chen, Chacón et al. \cite{Chen_Chacón_Barnes_2011, Chacón_Chen_Barnes_2013, Chen_Chacón_2014, Chen_Chacón_2015} have developed structure-preserving particle pushers for neoclassical transport in the Vlasov equations, derived from Crank--Nicolson integrators. We show these too can can derive from a FET interpretation, similarly offering potential extensions to higher-order-in-time particle pushers. The FET formulation is used also to consider how the stochastic drift terms can be incorporated into the pushers. Stochastic gyrokinetic expansions are also discussed.

        Different options for the numerical implementation of these schemes are considered.

        Due to the efficacy of FET in the development of SP timesteppers for both the fluid and kinetic component, we hope this approach will prove effective in the future for developing SP timesteppers for the full hybrid model. We hope this will give us the opportunity to incorporate previously inaccessible kinetic effects into the highly effective, modern, finite-element MHD models.
    \end{abstract}
    
    
    \newpage
    \tableofcontents
    
    
    \newpage
    \pagenumbering{arabic}
    %\linenumbers\renewcommand\thelinenumber{\color{black!50}\arabic{linenumber}}
            \input{0 - introduction/main.tex}
        \part{Research}
            \input{1 - low-noise PiC models/main.tex}
            \input{2 - kinetic component/main.tex}
            \input{3 - fluid component/main.tex}
            \input{4 - numerical implementation/main.tex}
        \part{Project Overview}
            \input{5 - research plan/main.tex}
            \input{6 - summary/main.tex}
    
    
    %\section{}
    \newpage
    \pagenumbering{gobble}
        \printbibliography


    \newpage
    \pagenumbering{roman}
    \appendix
        \part{Appendices}
            \input{8 - Hilbert complexes/main.tex}
            \input{9 - weak conservation proofs/main.tex}
\end{document}

            \documentclass[12pt, a4paper]{report}

\input{template/main.tex}

\title{\BA{Title in Progress...}}
\author{Boris Andrews}
\affil{Mathematical Institute, University of Oxford}
\date{\today}


\begin{document}
    \pagenumbering{gobble}
    \maketitle
    
    
    \begin{abstract}
        Magnetic confinement reactors---in particular tokamaks---offer one of the most promising options for achieving practical nuclear fusion, with the potential to provide virtually limitless, clean energy. The theoretical and numerical modeling of tokamak plasmas is simultaneously an essential component of effective reactor design, and a great research barrier. Tokamak operational conditions exhibit comparatively low Knudsen numbers. Kinetic effects, including kinetic waves and instabilities, Landau damping, bump-on-tail instabilities and more, are therefore highly influential in tokamak plasma dynamics. Purely fluid models are inherently incapable of capturing these effects, whereas the high dimensionality in purely kinetic models render them practically intractable for most relevant purposes.

        We consider a $\delta\!f$ decomposition model, with a macroscopic fluid background and microscopic kinetic correction, both fully coupled to each other. A similar manner of discretization is proposed to that used in the recent \texttt{STRUPHY} code \cite{Holderied_Possanner_Wang_2021, Holderied_2022, Li_et_al_2023} with a finite-element model for the background and a pseudo-particle/PiC model for the correction.

        The fluid background satisfies the full, non-linear, resistive, compressible, Hall MHD equations. \cite{Laakmann_Hu_Farrell_2022} introduces finite-element(-in-space) implicit timesteppers for the incompressible analogue to this system with structure-preserving (SP) properties in the ideal case, alongside parameter-robust preconditioners. We show that these timesteppers can derive from a finite-element-in-time (FET) (and finite-element-in-space) interpretation. The benefits of this reformulation are discussed, including the derivation of timesteppers that are higher order in time, and the quantifiable dissipative SP properties in the non-ideal, resistive case.
        
        We discuss possible options for extending this FET approach to timesteppers for the compressible case.

        The kinetic corrections satisfy linearized Boltzmann equations. Using a Lénard--Bernstein collision operator, these take Fokker--Planck-like forms \cite{Fokker_1914, Planck_1917} wherein pseudo-particles in the numerical model obey the neoclassical transport equations, with particle-independent Brownian drift terms. This offers a rigorous methodology for incorporating collisions into the particle transport model, without coupling the equations of motions for each particle.
        
        Works by Chen, Chacón et al. \cite{Chen_Chacón_Barnes_2011, Chacón_Chen_Barnes_2013, Chen_Chacón_2014, Chen_Chacón_2015} have developed structure-preserving particle pushers for neoclassical transport in the Vlasov equations, derived from Crank--Nicolson integrators. We show these too can can derive from a FET interpretation, similarly offering potential extensions to higher-order-in-time particle pushers. The FET formulation is used also to consider how the stochastic drift terms can be incorporated into the pushers. Stochastic gyrokinetic expansions are also discussed.

        Different options for the numerical implementation of these schemes are considered.

        Due to the efficacy of FET in the development of SP timesteppers for both the fluid and kinetic component, we hope this approach will prove effective in the future for developing SP timesteppers for the full hybrid model. We hope this will give us the opportunity to incorporate previously inaccessible kinetic effects into the highly effective, modern, finite-element MHD models.
    \end{abstract}
    
    
    \newpage
    \tableofcontents
    
    
    \newpage
    \pagenumbering{arabic}
    %\linenumbers\renewcommand\thelinenumber{\color{black!50}\arabic{linenumber}}
            \input{0 - introduction/main.tex}
        \part{Research}
            \input{1 - low-noise PiC models/main.tex}
            \input{2 - kinetic component/main.tex}
            \input{3 - fluid component/main.tex}
            \input{4 - numerical implementation/main.tex}
        \part{Project Overview}
            \input{5 - research plan/main.tex}
            \input{6 - summary/main.tex}
    
    
    %\section{}
    \newpage
    \pagenumbering{gobble}
        \printbibliography


    \newpage
    \pagenumbering{roman}
    \appendix
        \part{Appendices}
            \input{8 - Hilbert complexes/main.tex}
            \input{9 - weak conservation proofs/main.tex}
\end{document}

        \part{Project Overview}
            \documentclass[12pt, a4paper]{report}

\input{template/main.tex}

\title{\BA{Title in Progress...}}
\author{Boris Andrews}
\affil{Mathematical Institute, University of Oxford}
\date{\today}


\begin{document}
    \pagenumbering{gobble}
    \maketitle
    
    
    \begin{abstract}
        Magnetic confinement reactors---in particular tokamaks---offer one of the most promising options for achieving practical nuclear fusion, with the potential to provide virtually limitless, clean energy. The theoretical and numerical modeling of tokamak plasmas is simultaneously an essential component of effective reactor design, and a great research barrier. Tokamak operational conditions exhibit comparatively low Knudsen numbers. Kinetic effects, including kinetic waves and instabilities, Landau damping, bump-on-tail instabilities and more, are therefore highly influential in tokamak plasma dynamics. Purely fluid models are inherently incapable of capturing these effects, whereas the high dimensionality in purely kinetic models render them practically intractable for most relevant purposes.

        We consider a $\delta\!f$ decomposition model, with a macroscopic fluid background and microscopic kinetic correction, both fully coupled to each other. A similar manner of discretization is proposed to that used in the recent \texttt{STRUPHY} code \cite{Holderied_Possanner_Wang_2021, Holderied_2022, Li_et_al_2023} with a finite-element model for the background and a pseudo-particle/PiC model for the correction.

        The fluid background satisfies the full, non-linear, resistive, compressible, Hall MHD equations. \cite{Laakmann_Hu_Farrell_2022} introduces finite-element(-in-space) implicit timesteppers for the incompressible analogue to this system with structure-preserving (SP) properties in the ideal case, alongside parameter-robust preconditioners. We show that these timesteppers can derive from a finite-element-in-time (FET) (and finite-element-in-space) interpretation. The benefits of this reformulation are discussed, including the derivation of timesteppers that are higher order in time, and the quantifiable dissipative SP properties in the non-ideal, resistive case.
        
        We discuss possible options for extending this FET approach to timesteppers for the compressible case.

        The kinetic corrections satisfy linearized Boltzmann equations. Using a Lénard--Bernstein collision operator, these take Fokker--Planck-like forms \cite{Fokker_1914, Planck_1917} wherein pseudo-particles in the numerical model obey the neoclassical transport equations, with particle-independent Brownian drift terms. This offers a rigorous methodology for incorporating collisions into the particle transport model, without coupling the equations of motions for each particle.
        
        Works by Chen, Chacón et al. \cite{Chen_Chacón_Barnes_2011, Chacón_Chen_Barnes_2013, Chen_Chacón_2014, Chen_Chacón_2015} have developed structure-preserving particle pushers for neoclassical transport in the Vlasov equations, derived from Crank--Nicolson integrators. We show these too can can derive from a FET interpretation, similarly offering potential extensions to higher-order-in-time particle pushers. The FET formulation is used also to consider how the stochastic drift terms can be incorporated into the pushers. Stochastic gyrokinetic expansions are also discussed.

        Different options for the numerical implementation of these schemes are considered.

        Due to the efficacy of FET in the development of SP timesteppers for both the fluid and kinetic component, we hope this approach will prove effective in the future for developing SP timesteppers for the full hybrid model. We hope this will give us the opportunity to incorporate previously inaccessible kinetic effects into the highly effective, modern, finite-element MHD models.
    \end{abstract}
    
    
    \newpage
    \tableofcontents
    
    
    \newpage
    \pagenumbering{arabic}
    %\linenumbers\renewcommand\thelinenumber{\color{black!50}\arabic{linenumber}}
            \input{0 - introduction/main.tex}
        \part{Research}
            \input{1 - low-noise PiC models/main.tex}
            \input{2 - kinetic component/main.tex}
            \input{3 - fluid component/main.tex}
            \input{4 - numerical implementation/main.tex}
        \part{Project Overview}
            \input{5 - research plan/main.tex}
            \input{6 - summary/main.tex}
    
    
    %\section{}
    \newpage
    \pagenumbering{gobble}
        \printbibliography


    \newpage
    \pagenumbering{roman}
    \appendix
        \part{Appendices}
            \input{8 - Hilbert complexes/main.tex}
            \input{9 - weak conservation proofs/main.tex}
\end{document}

            \documentclass[12pt, a4paper]{report}

\input{template/main.tex}

\title{\BA{Title in Progress...}}
\author{Boris Andrews}
\affil{Mathematical Institute, University of Oxford}
\date{\today}


\begin{document}
    \pagenumbering{gobble}
    \maketitle
    
    
    \begin{abstract}
        Magnetic confinement reactors---in particular tokamaks---offer one of the most promising options for achieving practical nuclear fusion, with the potential to provide virtually limitless, clean energy. The theoretical and numerical modeling of tokamak plasmas is simultaneously an essential component of effective reactor design, and a great research barrier. Tokamak operational conditions exhibit comparatively low Knudsen numbers. Kinetic effects, including kinetic waves and instabilities, Landau damping, bump-on-tail instabilities and more, are therefore highly influential in tokamak plasma dynamics. Purely fluid models are inherently incapable of capturing these effects, whereas the high dimensionality in purely kinetic models render them practically intractable for most relevant purposes.

        We consider a $\delta\!f$ decomposition model, with a macroscopic fluid background and microscopic kinetic correction, both fully coupled to each other. A similar manner of discretization is proposed to that used in the recent \texttt{STRUPHY} code \cite{Holderied_Possanner_Wang_2021, Holderied_2022, Li_et_al_2023} with a finite-element model for the background and a pseudo-particle/PiC model for the correction.

        The fluid background satisfies the full, non-linear, resistive, compressible, Hall MHD equations. \cite{Laakmann_Hu_Farrell_2022} introduces finite-element(-in-space) implicit timesteppers for the incompressible analogue to this system with structure-preserving (SP) properties in the ideal case, alongside parameter-robust preconditioners. We show that these timesteppers can derive from a finite-element-in-time (FET) (and finite-element-in-space) interpretation. The benefits of this reformulation are discussed, including the derivation of timesteppers that are higher order in time, and the quantifiable dissipative SP properties in the non-ideal, resistive case.
        
        We discuss possible options for extending this FET approach to timesteppers for the compressible case.

        The kinetic corrections satisfy linearized Boltzmann equations. Using a Lénard--Bernstein collision operator, these take Fokker--Planck-like forms \cite{Fokker_1914, Planck_1917} wherein pseudo-particles in the numerical model obey the neoclassical transport equations, with particle-independent Brownian drift terms. This offers a rigorous methodology for incorporating collisions into the particle transport model, without coupling the equations of motions for each particle.
        
        Works by Chen, Chacón et al. \cite{Chen_Chacón_Barnes_2011, Chacón_Chen_Barnes_2013, Chen_Chacón_2014, Chen_Chacón_2015} have developed structure-preserving particle pushers for neoclassical transport in the Vlasov equations, derived from Crank--Nicolson integrators. We show these too can can derive from a FET interpretation, similarly offering potential extensions to higher-order-in-time particle pushers. The FET formulation is used also to consider how the stochastic drift terms can be incorporated into the pushers. Stochastic gyrokinetic expansions are also discussed.

        Different options for the numerical implementation of these schemes are considered.

        Due to the efficacy of FET in the development of SP timesteppers for both the fluid and kinetic component, we hope this approach will prove effective in the future for developing SP timesteppers for the full hybrid model. We hope this will give us the opportunity to incorporate previously inaccessible kinetic effects into the highly effective, modern, finite-element MHD models.
    \end{abstract}
    
    
    \newpage
    \tableofcontents
    
    
    \newpage
    \pagenumbering{arabic}
    %\linenumbers\renewcommand\thelinenumber{\color{black!50}\arabic{linenumber}}
            \input{0 - introduction/main.tex}
        \part{Research}
            \input{1 - low-noise PiC models/main.tex}
            \input{2 - kinetic component/main.tex}
            \input{3 - fluid component/main.tex}
            \input{4 - numerical implementation/main.tex}
        \part{Project Overview}
            \input{5 - research plan/main.tex}
            \input{6 - summary/main.tex}
    
    
    %\section{}
    \newpage
    \pagenumbering{gobble}
        \printbibliography


    \newpage
    \pagenumbering{roman}
    \appendix
        \part{Appendices}
            \input{8 - Hilbert complexes/main.tex}
            \input{9 - weak conservation proofs/main.tex}
\end{document}

    
    
    %\section{}
    \newpage
    \pagenumbering{gobble}
        \printbibliography


    \newpage
    \pagenumbering{roman}
    \appendix
        \part{Appendices}
            \documentclass[12pt, a4paper]{report}

\input{template/main.tex}

\title{\BA{Title in Progress...}}
\author{Boris Andrews}
\affil{Mathematical Institute, University of Oxford}
\date{\today}


\begin{document}
    \pagenumbering{gobble}
    \maketitle
    
    
    \begin{abstract}
        Magnetic confinement reactors---in particular tokamaks---offer one of the most promising options for achieving practical nuclear fusion, with the potential to provide virtually limitless, clean energy. The theoretical and numerical modeling of tokamak plasmas is simultaneously an essential component of effective reactor design, and a great research barrier. Tokamak operational conditions exhibit comparatively low Knudsen numbers. Kinetic effects, including kinetic waves and instabilities, Landau damping, bump-on-tail instabilities and more, are therefore highly influential in tokamak plasma dynamics. Purely fluid models are inherently incapable of capturing these effects, whereas the high dimensionality in purely kinetic models render them practically intractable for most relevant purposes.

        We consider a $\delta\!f$ decomposition model, with a macroscopic fluid background and microscopic kinetic correction, both fully coupled to each other. A similar manner of discretization is proposed to that used in the recent \texttt{STRUPHY} code \cite{Holderied_Possanner_Wang_2021, Holderied_2022, Li_et_al_2023} with a finite-element model for the background and a pseudo-particle/PiC model for the correction.

        The fluid background satisfies the full, non-linear, resistive, compressible, Hall MHD equations. \cite{Laakmann_Hu_Farrell_2022} introduces finite-element(-in-space) implicit timesteppers for the incompressible analogue to this system with structure-preserving (SP) properties in the ideal case, alongside parameter-robust preconditioners. We show that these timesteppers can derive from a finite-element-in-time (FET) (and finite-element-in-space) interpretation. The benefits of this reformulation are discussed, including the derivation of timesteppers that are higher order in time, and the quantifiable dissipative SP properties in the non-ideal, resistive case.
        
        We discuss possible options for extending this FET approach to timesteppers for the compressible case.

        The kinetic corrections satisfy linearized Boltzmann equations. Using a Lénard--Bernstein collision operator, these take Fokker--Planck-like forms \cite{Fokker_1914, Planck_1917} wherein pseudo-particles in the numerical model obey the neoclassical transport equations, with particle-independent Brownian drift terms. This offers a rigorous methodology for incorporating collisions into the particle transport model, without coupling the equations of motions for each particle.
        
        Works by Chen, Chacón et al. \cite{Chen_Chacón_Barnes_2011, Chacón_Chen_Barnes_2013, Chen_Chacón_2014, Chen_Chacón_2015} have developed structure-preserving particle pushers for neoclassical transport in the Vlasov equations, derived from Crank--Nicolson integrators. We show these too can can derive from a FET interpretation, similarly offering potential extensions to higher-order-in-time particle pushers. The FET formulation is used also to consider how the stochastic drift terms can be incorporated into the pushers. Stochastic gyrokinetic expansions are also discussed.

        Different options for the numerical implementation of these schemes are considered.

        Due to the efficacy of FET in the development of SP timesteppers for both the fluid and kinetic component, we hope this approach will prove effective in the future for developing SP timesteppers for the full hybrid model. We hope this will give us the opportunity to incorporate previously inaccessible kinetic effects into the highly effective, modern, finite-element MHD models.
    \end{abstract}
    
    
    \newpage
    \tableofcontents
    
    
    \newpage
    \pagenumbering{arabic}
    %\linenumbers\renewcommand\thelinenumber{\color{black!50}\arabic{linenumber}}
            \input{0 - introduction/main.tex}
        \part{Research}
            \input{1 - low-noise PiC models/main.tex}
            \input{2 - kinetic component/main.tex}
            \input{3 - fluid component/main.tex}
            \input{4 - numerical implementation/main.tex}
        \part{Project Overview}
            \input{5 - research plan/main.tex}
            \input{6 - summary/main.tex}
    
    
    %\section{}
    \newpage
    \pagenumbering{gobble}
        \printbibliography


    \newpage
    \pagenumbering{roman}
    \appendix
        \part{Appendices}
            \input{8 - Hilbert complexes/main.tex}
            \input{9 - weak conservation proofs/main.tex}
\end{document}

            \documentclass[12pt, a4paper]{report}

\input{template/main.tex}

\title{\BA{Title in Progress...}}
\author{Boris Andrews}
\affil{Mathematical Institute, University of Oxford}
\date{\today}


\begin{document}
    \pagenumbering{gobble}
    \maketitle
    
    
    \begin{abstract}
        Magnetic confinement reactors---in particular tokamaks---offer one of the most promising options for achieving practical nuclear fusion, with the potential to provide virtually limitless, clean energy. The theoretical and numerical modeling of tokamak plasmas is simultaneously an essential component of effective reactor design, and a great research barrier. Tokamak operational conditions exhibit comparatively low Knudsen numbers. Kinetic effects, including kinetic waves and instabilities, Landau damping, bump-on-tail instabilities and more, are therefore highly influential in tokamak plasma dynamics. Purely fluid models are inherently incapable of capturing these effects, whereas the high dimensionality in purely kinetic models render them practically intractable for most relevant purposes.

        We consider a $\delta\!f$ decomposition model, with a macroscopic fluid background and microscopic kinetic correction, both fully coupled to each other. A similar manner of discretization is proposed to that used in the recent \texttt{STRUPHY} code \cite{Holderied_Possanner_Wang_2021, Holderied_2022, Li_et_al_2023} with a finite-element model for the background and a pseudo-particle/PiC model for the correction.

        The fluid background satisfies the full, non-linear, resistive, compressible, Hall MHD equations. \cite{Laakmann_Hu_Farrell_2022} introduces finite-element(-in-space) implicit timesteppers for the incompressible analogue to this system with structure-preserving (SP) properties in the ideal case, alongside parameter-robust preconditioners. We show that these timesteppers can derive from a finite-element-in-time (FET) (and finite-element-in-space) interpretation. The benefits of this reformulation are discussed, including the derivation of timesteppers that are higher order in time, and the quantifiable dissipative SP properties in the non-ideal, resistive case.
        
        We discuss possible options for extending this FET approach to timesteppers for the compressible case.

        The kinetic corrections satisfy linearized Boltzmann equations. Using a Lénard--Bernstein collision operator, these take Fokker--Planck-like forms \cite{Fokker_1914, Planck_1917} wherein pseudo-particles in the numerical model obey the neoclassical transport equations, with particle-independent Brownian drift terms. This offers a rigorous methodology for incorporating collisions into the particle transport model, without coupling the equations of motions for each particle.
        
        Works by Chen, Chacón et al. \cite{Chen_Chacón_Barnes_2011, Chacón_Chen_Barnes_2013, Chen_Chacón_2014, Chen_Chacón_2015} have developed structure-preserving particle pushers for neoclassical transport in the Vlasov equations, derived from Crank--Nicolson integrators. We show these too can can derive from a FET interpretation, similarly offering potential extensions to higher-order-in-time particle pushers. The FET formulation is used also to consider how the stochastic drift terms can be incorporated into the pushers. Stochastic gyrokinetic expansions are also discussed.

        Different options for the numerical implementation of these schemes are considered.

        Due to the efficacy of FET in the development of SP timesteppers for both the fluid and kinetic component, we hope this approach will prove effective in the future for developing SP timesteppers for the full hybrid model. We hope this will give us the opportunity to incorporate previously inaccessible kinetic effects into the highly effective, modern, finite-element MHD models.
    \end{abstract}
    
    
    \newpage
    \tableofcontents
    
    
    \newpage
    \pagenumbering{arabic}
    %\linenumbers\renewcommand\thelinenumber{\color{black!50}\arabic{linenumber}}
            \input{0 - introduction/main.tex}
        \part{Research}
            \input{1 - low-noise PiC models/main.tex}
            \input{2 - kinetic component/main.tex}
            \input{3 - fluid component/main.tex}
            \input{4 - numerical implementation/main.tex}
        \part{Project Overview}
            \input{5 - research plan/main.tex}
            \input{6 - summary/main.tex}
    
    
    %\section{}
    \newpage
    \pagenumbering{gobble}
        \printbibliography


    \newpage
    \pagenumbering{roman}
    \appendix
        \part{Appendices}
            \input{8 - Hilbert complexes/main.tex}
            \input{9 - weak conservation proofs/main.tex}
\end{document}

\end{document}

            \documentclass[12pt, a4paper]{report}

\documentclass[12pt, a4paper]{report}

\input{template/main.tex}

\title{\BA{Title in Progress...}}
\author{Boris Andrews}
\affil{Mathematical Institute, University of Oxford}
\date{\today}


\begin{document}
    \pagenumbering{gobble}
    \maketitle
    
    
    \begin{abstract}
        Magnetic confinement reactors---in particular tokamaks---offer one of the most promising options for achieving practical nuclear fusion, with the potential to provide virtually limitless, clean energy. The theoretical and numerical modeling of tokamak plasmas is simultaneously an essential component of effective reactor design, and a great research barrier. Tokamak operational conditions exhibit comparatively low Knudsen numbers. Kinetic effects, including kinetic waves and instabilities, Landau damping, bump-on-tail instabilities and more, are therefore highly influential in tokamak plasma dynamics. Purely fluid models are inherently incapable of capturing these effects, whereas the high dimensionality in purely kinetic models render them practically intractable for most relevant purposes.

        We consider a $\delta\!f$ decomposition model, with a macroscopic fluid background and microscopic kinetic correction, both fully coupled to each other. A similar manner of discretization is proposed to that used in the recent \texttt{STRUPHY} code \cite{Holderied_Possanner_Wang_2021, Holderied_2022, Li_et_al_2023} with a finite-element model for the background and a pseudo-particle/PiC model for the correction.

        The fluid background satisfies the full, non-linear, resistive, compressible, Hall MHD equations. \cite{Laakmann_Hu_Farrell_2022} introduces finite-element(-in-space) implicit timesteppers for the incompressible analogue to this system with structure-preserving (SP) properties in the ideal case, alongside parameter-robust preconditioners. We show that these timesteppers can derive from a finite-element-in-time (FET) (and finite-element-in-space) interpretation. The benefits of this reformulation are discussed, including the derivation of timesteppers that are higher order in time, and the quantifiable dissipative SP properties in the non-ideal, resistive case.
        
        We discuss possible options for extending this FET approach to timesteppers for the compressible case.

        The kinetic corrections satisfy linearized Boltzmann equations. Using a Lénard--Bernstein collision operator, these take Fokker--Planck-like forms \cite{Fokker_1914, Planck_1917} wherein pseudo-particles in the numerical model obey the neoclassical transport equations, with particle-independent Brownian drift terms. This offers a rigorous methodology for incorporating collisions into the particle transport model, without coupling the equations of motions for each particle.
        
        Works by Chen, Chacón et al. \cite{Chen_Chacón_Barnes_2011, Chacón_Chen_Barnes_2013, Chen_Chacón_2014, Chen_Chacón_2015} have developed structure-preserving particle pushers for neoclassical transport in the Vlasov equations, derived from Crank--Nicolson integrators. We show these too can can derive from a FET interpretation, similarly offering potential extensions to higher-order-in-time particle pushers. The FET formulation is used also to consider how the stochastic drift terms can be incorporated into the pushers. Stochastic gyrokinetic expansions are also discussed.

        Different options for the numerical implementation of these schemes are considered.

        Due to the efficacy of FET in the development of SP timesteppers for both the fluid and kinetic component, we hope this approach will prove effective in the future for developing SP timesteppers for the full hybrid model. We hope this will give us the opportunity to incorporate previously inaccessible kinetic effects into the highly effective, modern, finite-element MHD models.
    \end{abstract}
    
    
    \newpage
    \tableofcontents
    
    
    \newpage
    \pagenumbering{arabic}
    %\linenumbers\renewcommand\thelinenumber{\color{black!50}\arabic{linenumber}}
            \input{0 - introduction/main.tex}
        \part{Research}
            \input{1 - low-noise PiC models/main.tex}
            \input{2 - kinetic component/main.tex}
            \input{3 - fluid component/main.tex}
            \input{4 - numerical implementation/main.tex}
        \part{Project Overview}
            \input{5 - research plan/main.tex}
            \input{6 - summary/main.tex}
    
    
    %\section{}
    \newpage
    \pagenumbering{gobble}
        \printbibliography


    \newpage
    \pagenumbering{roman}
    \appendix
        \part{Appendices}
            \input{8 - Hilbert complexes/main.tex}
            \input{9 - weak conservation proofs/main.tex}
\end{document}


\title{\BA{Title in Progress...}}
\author{Boris Andrews}
\affil{Mathematical Institute, University of Oxford}
\date{\today}


\begin{document}
    \pagenumbering{gobble}
    \maketitle
    
    
    \begin{abstract}
        Magnetic confinement reactors---in particular tokamaks---offer one of the most promising options for achieving practical nuclear fusion, with the potential to provide virtually limitless, clean energy. The theoretical and numerical modeling of tokamak plasmas is simultaneously an essential component of effective reactor design, and a great research barrier. Tokamak operational conditions exhibit comparatively low Knudsen numbers. Kinetic effects, including kinetic waves and instabilities, Landau damping, bump-on-tail instabilities and more, are therefore highly influential in tokamak plasma dynamics. Purely fluid models are inherently incapable of capturing these effects, whereas the high dimensionality in purely kinetic models render them practically intractable for most relevant purposes.

        We consider a $\delta\!f$ decomposition model, with a macroscopic fluid background and microscopic kinetic correction, both fully coupled to each other. A similar manner of discretization is proposed to that used in the recent \texttt{STRUPHY} code \cite{Holderied_Possanner_Wang_2021, Holderied_2022, Li_et_al_2023} with a finite-element model for the background and a pseudo-particle/PiC model for the correction.

        The fluid background satisfies the full, non-linear, resistive, compressible, Hall MHD equations. \cite{Laakmann_Hu_Farrell_2022} introduces finite-element(-in-space) implicit timesteppers for the incompressible analogue to this system with structure-preserving (SP) properties in the ideal case, alongside parameter-robust preconditioners. We show that these timesteppers can derive from a finite-element-in-time (FET) (and finite-element-in-space) interpretation. The benefits of this reformulation are discussed, including the derivation of timesteppers that are higher order in time, and the quantifiable dissipative SP properties in the non-ideal, resistive case.
        
        We discuss possible options for extending this FET approach to timesteppers for the compressible case.

        The kinetic corrections satisfy linearized Boltzmann equations. Using a Lénard--Bernstein collision operator, these take Fokker--Planck-like forms \cite{Fokker_1914, Planck_1917} wherein pseudo-particles in the numerical model obey the neoclassical transport equations, with particle-independent Brownian drift terms. This offers a rigorous methodology for incorporating collisions into the particle transport model, without coupling the equations of motions for each particle.
        
        Works by Chen, Chacón et al. \cite{Chen_Chacón_Barnes_2011, Chacón_Chen_Barnes_2013, Chen_Chacón_2014, Chen_Chacón_2015} have developed structure-preserving particle pushers for neoclassical transport in the Vlasov equations, derived from Crank--Nicolson integrators. We show these too can can derive from a FET interpretation, similarly offering potential extensions to higher-order-in-time particle pushers. The FET formulation is used also to consider how the stochastic drift terms can be incorporated into the pushers. Stochastic gyrokinetic expansions are also discussed.

        Different options for the numerical implementation of these schemes are considered.

        Due to the efficacy of FET in the development of SP timesteppers for both the fluid and kinetic component, we hope this approach will prove effective in the future for developing SP timesteppers for the full hybrid model. We hope this will give us the opportunity to incorporate previously inaccessible kinetic effects into the highly effective, modern, finite-element MHD models.
    \end{abstract}
    
    
    \newpage
    \tableofcontents
    
    
    \newpage
    \pagenumbering{arabic}
    %\linenumbers\renewcommand\thelinenumber{\color{black!50}\arabic{linenumber}}
            \documentclass[12pt, a4paper]{report}

\input{template/main.tex}

\title{\BA{Title in Progress...}}
\author{Boris Andrews}
\affil{Mathematical Institute, University of Oxford}
\date{\today}


\begin{document}
    \pagenumbering{gobble}
    \maketitle
    
    
    \begin{abstract}
        Magnetic confinement reactors---in particular tokamaks---offer one of the most promising options for achieving practical nuclear fusion, with the potential to provide virtually limitless, clean energy. The theoretical and numerical modeling of tokamak plasmas is simultaneously an essential component of effective reactor design, and a great research barrier. Tokamak operational conditions exhibit comparatively low Knudsen numbers. Kinetic effects, including kinetic waves and instabilities, Landau damping, bump-on-tail instabilities and more, are therefore highly influential in tokamak plasma dynamics. Purely fluid models are inherently incapable of capturing these effects, whereas the high dimensionality in purely kinetic models render them practically intractable for most relevant purposes.

        We consider a $\delta\!f$ decomposition model, with a macroscopic fluid background and microscopic kinetic correction, both fully coupled to each other. A similar manner of discretization is proposed to that used in the recent \texttt{STRUPHY} code \cite{Holderied_Possanner_Wang_2021, Holderied_2022, Li_et_al_2023} with a finite-element model for the background and a pseudo-particle/PiC model for the correction.

        The fluid background satisfies the full, non-linear, resistive, compressible, Hall MHD equations. \cite{Laakmann_Hu_Farrell_2022} introduces finite-element(-in-space) implicit timesteppers for the incompressible analogue to this system with structure-preserving (SP) properties in the ideal case, alongside parameter-robust preconditioners. We show that these timesteppers can derive from a finite-element-in-time (FET) (and finite-element-in-space) interpretation. The benefits of this reformulation are discussed, including the derivation of timesteppers that are higher order in time, and the quantifiable dissipative SP properties in the non-ideal, resistive case.
        
        We discuss possible options for extending this FET approach to timesteppers for the compressible case.

        The kinetic corrections satisfy linearized Boltzmann equations. Using a Lénard--Bernstein collision operator, these take Fokker--Planck-like forms \cite{Fokker_1914, Planck_1917} wherein pseudo-particles in the numerical model obey the neoclassical transport equations, with particle-independent Brownian drift terms. This offers a rigorous methodology for incorporating collisions into the particle transport model, without coupling the equations of motions for each particle.
        
        Works by Chen, Chacón et al. \cite{Chen_Chacón_Barnes_2011, Chacón_Chen_Barnes_2013, Chen_Chacón_2014, Chen_Chacón_2015} have developed structure-preserving particle pushers for neoclassical transport in the Vlasov equations, derived from Crank--Nicolson integrators. We show these too can can derive from a FET interpretation, similarly offering potential extensions to higher-order-in-time particle pushers. The FET formulation is used also to consider how the stochastic drift terms can be incorporated into the pushers. Stochastic gyrokinetic expansions are also discussed.

        Different options for the numerical implementation of these schemes are considered.

        Due to the efficacy of FET in the development of SP timesteppers for both the fluid and kinetic component, we hope this approach will prove effective in the future for developing SP timesteppers for the full hybrid model. We hope this will give us the opportunity to incorporate previously inaccessible kinetic effects into the highly effective, modern, finite-element MHD models.
    \end{abstract}
    
    
    \newpage
    \tableofcontents
    
    
    \newpage
    \pagenumbering{arabic}
    %\linenumbers\renewcommand\thelinenumber{\color{black!50}\arabic{linenumber}}
            \input{0 - introduction/main.tex}
        \part{Research}
            \input{1 - low-noise PiC models/main.tex}
            \input{2 - kinetic component/main.tex}
            \input{3 - fluid component/main.tex}
            \input{4 - numerical implementation/main.tex}
        \part{Project Overview}
            \input{5 - research plan/main.tex}
            \input{6 - summary/main.tex}
    
    
    %\section{}
    \newpage
    \pagenumbering{gobble}
        \printbibliography


    \newpage
    \pagenumbering{roman}
    \appendix
        \part{Appendices}
            \input{8 - Hilbert complexes/main.tex}
            \input{9 - weak conservation proofs/main.tex}
\end{document}

        \part{Research}
            \documentclass[12pt, a4paper]{report}

\input{template/main.tex}

\title{\BA{Title in Progress...}}
\author{Boris Andrews}
\affil{Mathematical Institute, University of Oxford}
\date{\today}


\begin{document}
    \pagenumbering{gobble}
    \maketitle
    
    
    \begin{abstract}
        Magnetic confinement reactors---in particular tokamaks---offer one of the most promising options for achieving practical nuclear fusion, with the potential to provide virtually limitless, clean energy. The theoretical and numerical modeling of tokamak plasmas is simultaneously an essential component of effective reactor design, and a great research barrier. Tokamak operational conditions exhibit comparatively low Knudsen numbers. Kinetic effects, including kinetic waves and instabilities, Landau damping, bump-on-tail instabilities and more, are therefore highly influential in tokamak plasma dynamics. Purely fluid models are inherently incapable of capturing these effects, whereas the high dimensionality in purely kinetic models render them practically intractable for most relevant purposes.

        We consider a $\delta\!f$ decomposition model, with a macroscopic fluid background and microscopic kinetic correction, both fully coupled to each other. A similar manner of discretization is proposed to that used in the recent \texttt{STRUPHY} code \cite{Holderied_Possanner_Wang_2021, Holderied_2022, Li_et_al_2023} with a finite-element model for the background and a pseudo-particle/PiC model for the correction.

        The fluid background satisfies the full, non-linear, resistive, compressible, Hall MHD equations. \cite{Laakmann_Hu_Farrell_2022} introduces finite-element(-in-space) implicit timesteppers for the incompressible analogue to this system with structure-preserving (SP) properties in the ideal case, alongside parameter-robust preconditioners. We show that these timesteppers can derive from a finite-element-in-time (FET) (and finite-element-in-space) interpretation. The benefits of this reformulation are discussed, including the derivation of timesteppers that are higher order in time, and the quantifiable dissipative SP properties in the non-ideal, resistive case.
        
        We discuss possible options for extending this FET approach to timesteppers for the compressible case.

        The kinetic corrections satisfy linearized Boltzmann equations. Using a Lénard--Bernstein collision operator, these take Fokker--Planck-like forms \cite{Fokker_1914, Planck_1917} wherein pseudo-particles in the numerical model obey the neoclassical transport equations, with particle-independent Brownian drift terms. This offers a rigorous methodology for incorporating collisions into the particle transport model, without coupling the equations of motions for each particle.
        
        Works by Chen, Chacón et al. \cite{Chen_Chacón_Barnes_2011, Chacón_Chen_Barnes_2013, Chen_Chacón_2014, Chen_Chacón_2015} have developed structure-preserving particle pushers for neoclassical transport in the Vlasov equations, derived from Crank--Nicolson integrators. We show these too can can derive from a FET interpretation, similarly offering potential extensions to higher-order-in-time particle pushers. The FET formulation is used also to consider how the stochastic drift terms can be incorporated into the pushers. Stochastic gyrokinetic expansions are also discussed.

        Different options for the numerical implementation of these schemes are considered.

        Due to the efficacy of FET in the development of SP timesteppers for both the fluid and kinetic component, we hope this approach will prove effective in the future for developing SP timesteppers for the full hybrid model. We hope this will give us the opportunity to incorporate previously inaccessible kinetic effects into the highly effective, modern, finite-element MHD models.
    \end{abstract}
    
    
    \newpage
    \tableofcontents
    
    
    \newpage
    \pagenumbering{arabic}
    %\linenumbers\renewcommand\thelinenumber{\color{black!50}\arabic{linenumber}}
            \input{0 - introduction/main.tex}
        \part{Research}
            \input{1 - low-noise PiC models/main.tex}
            \input{2 - kinetic component/main.tex}
            \input{3 - fluid component/main.tex}
            \input{4 - numerical implementation/main.tex}
        \part{Project Overview}
            \input{5 - research plan/main.tex}
            \input{6 - summary/main.tex}
    
    
    %\section{}
    \newpage
    \pagenumbering{gobble}
        \printbibliography


    \newpage
    \pagenumbering{roman}
    \appendix
        \part{Appendices}
            \input{8 - Hilbert complexes/main.tex}
            \input{9 - weak conservation proofs/main.tex}
\end{document}

            \documentclass[12pt, a4paper]{report}

\input{template/main.tex}

\title{\BA{Title in Progress...}}
\author{Boris Andrews}
\affil{Mathematical Institute, University of Oxford}
\date{\today}


\begin{document}
    \pagenumbering{gobble}
    \maketitle
    
    
    \begin{abstract}
        Magnetic confinement reactors---in particular tokamaks---offer one of the most promising options for achieving practical nuclear fusion, with the potential to provide virtually limitless, clean energy. The theoretical and numerical modeling of tokamak plasmas is simultaneously an essential component of effective reactor design, and a great research barrier. Tokamak operational conditions exhibit comparatively low Knudsen numbers. Kinetic effects, including kinetic waves and instabilities, Landau damping, bump-on-tail instabilities and more, are therefore highly influential in tokamak plasma dynamics. Purely fluid models are inherently incapable of capturing these effects, whereas the high dimensionality in purely kinetic models render them practically intractable for most relevant purposes.

        We consider a $\delta\!f$ decomposition model, with a macroscopic fluid background and microscopic kinetic correction, both fully coupled to each other. A similar manner of discretization is proposed to that used in the recent \texttt{STRUPHY} code \cite{Holderied_Possanner_Wang_2021, Holderied_2022, Li_et_al_2023} with a finite-element model for the background and a pseudo-particle/PiC model for the correction.

        The fluid background satisfies the full, non-linear, resistive, compressible, Hall MHD equations. \cite{Laakmann_Hu_Farrell_2022} introduces finite-element(-in-space) implicit timesteppers for the incompressible analogue to this system with structure-preserving (SP) properties in the ideal case, alongside parameter-robust preconditioners. We show that these timesteppers can derive from a finite-element-in-time (FET) (and finite-element-in-space) interpretation. The benefits of this reformulation are discussed, including the derivation of timesteppers that are higher order in time, and the quantifiable dissipative SP properties in the non-ideal, resistive case.
        
        We discuss possible options for extending this FET approach to timesteppers for the compressible case.

        The kinetic corrections satisfy linearized Boltzmann equations. Using a Lénard--Bernstein collision operator, these take Fokker--Planck-like forms \cite{Fokker_1914, Planck_1917} wherein pseudo-particles in the numerical model obey the neoclassical transport equations, with particle-independent Brownian drift terms. This offers a rigorous methodology for incorporating collisions into the particle transport model, without coupling the equations of motions for each particle.
        
        Works by Chen, Chacón et al. \cite{Chen_Chacón_Barnes_2011, Chacón_Chen_Barnes_2013, Chen_Chacón_2014, Chen_Chacón_2015} have developed structure-preserving particle pushers for neoclassical transport in the Vlasov equations, derived from Crank--Nicolson integrators. We show these too can can derive from a FET interpretation, similarly offering potential extensions to higher-order-in-time particle pushers. The FET formulation is used also to consider how the stochastic drift terms can be incorporated into the pushers. Stochastic gyrokinetic expansions are also discussed.

        Different options for the numerical implementation of these schemes are considered.

        Due to the efficacy of FET in the development of SP timesteppers for both the fluid and kinetic component, we hope this approach will prove effective in the future for developing SP timesteppers for the full hybrid model. We hope this will give us the opportunity to incorporate previously inaccessible kinetic effects into the highly effective, modern, finite-element MHD models.
    \end{abstract}
    
    
    \newpage
    \tableofcontents
    
    
    \newpage
    \pagenumbering{arabic}
    %\linenumbers\renewcommand\thelinenumber{\color{black!50}\arabic{linenumber}}
            \input{0 - introduction/main.tex}
        \part{Research}
            \input{1 - low-noise PiC models/main.tex}
            \input{2 - kinetic component/main.tex}
            \input{3 - fluid component/main.tex}
            \input{4 - numerical implementation/main.tex}
        \part{Project Overview}
            \input{5 - research plan/main.tex}
            \input{6 - summary/main.tex}
    
    
    %\section{}
    \newpage
    \pagenumbering{gobble}
        \printbibliography


    \newpage
    \pagenumbering{roman}
    \appendix
        \part{Appendices}
            \input{8 - Hilbert complexes/main.tex}
            \input{9 - weak conservation proofs/main.tex}
\end{document}

            \documentclass[12pt, a4paper]{report}

\input{template/main.tex}

\title{\BA{Title in Progress...}}
\author{Boris Andrews}
\affil{Mathematical Institute, University of Oxford}
\date{\today}


\begin{document}
    \pagenumbering{gobble}
    \maketitle
    
    
    \begin{abstract}
        Magnetic confinement reactors---in particular tokamaks---offer one of the most promising options for achieving practical nuclear fusion, with the potential to provide virtually limitless, clean energy. The theoretical and numerical modeling of tokamak plasmas is simultaneously an essential component of effective reactor design, and a great research barrier. Tokamak operational conditions exhibit comparatively low Knudsen numbers. Kinetic effects, including kinetic waves and instabilities, Landau damping, bump-on-tail instabilities and more, are therefore highly influential in tokamak plasma dynamics. Purely fluid models are inherently incapable of capturing these effects, whereas the high dimensionality in purely kinetic models render them practically intractable for most relevant purposes.

        We consider a $\delta\!f$ decomposition model, with a macroscopic fluid background and microscopic kinetic correction, both fully coupled to each other. A similar manner of discretization is proposed to that used in the recent \texttt{STRUPHY} code \cite{Holderied_Possanner_Wang_2021, Holderied_2022, Li_et_al_2023} with a finite-element model for the background and a pseudo-particle/PiC model for the correction.

        The fluid background satisfies the full, non-linear, resistive, compressible, Hall MHD equations. \cite{Laakmann_Hu_Farrell_2022} introduces finite-element(-in-space) implicit timesteppers for the incompressible analogue to this system with structure-preserving (SP) properties in the ideal case, alongside parameter-robust preconditioners. We show that these timesteppers can derive from a finite-element-in-time (FET) (and finite-element-in-space) interpretation. The benefits of this reformulation are discussed, including the derivation of timesteppers that are higher order in time, and the quantifiable dissipative SP properties in the non-ideal, resistive case.
        
        We discuss possible options for extending this FET approach to timesteppers for the compressible case.

        The kinetic corrections satisfy linearized Boltzmann equations. Using a Lénard--Bernstein collision operator, these take Fokker--Planck-like forms \cite{Fokker_1914, Planck_1917} wherein pseudo-particles in the numerical model obey the neoclassical transport equations, with particle-independent Brownian drift terms. This offers a rigorous methodology for incorporating collisions into the particle transport model, without coupling the equations of motions for each particle.
        
        Works by Chen, Chacón et al. \cite{Chen_Chacón_Barnes_2011, Chacón_Chen_Barnes_2013, Chen_Chacón_2014, Chen_Chacón_2015} have developed structure-preserving particle pushers for neoclassical transport in the Vlasov equations, derived from Crank--Nicolson integrators. We show these too can can derive from a FET interpretation, similarly offering potential extensions to higher-order-in-time particle pushers. The FET formulation is used also to consider how the stochastic drift terms can be incorporated into the pushers. Stochastic gyrokinetic expansions are also discussed.

        Different options for the numerical implementation of these schemes are considered.

        Due to the efficacy of FET in the development of SP timesteppers for both the fluid and kinetic component, we hope this approach will prove effective in the future for developing SP timesteppers for the full hybrid model. We hope this will give us the opportunity to incorporate previously inaccessible kinetic effects into the highly effective, modern, finite-element MHD models.
    \end{abstract}
    
    
    \newpage
    \tableofcontents
    
    
    \newpage
    \pagenumbering{arabic}
    %\linenumbers\renewcommand\thelinenumber{\color{black!50}\arabic{linenumber}}
            \input{0 - introduction/main.tex}
        \part{Research}
            \input{1 - low-noise PiC models/main.tex}
            \input{2 - kinetic component/main.tex}
            \input{3 - fluid component/main.tex}
            \input{4 - numerical implementation/main.tex}
        \part{Project Overview}
            \input{5 - research plan/main.tex}
            \input{6 - summary/main.tex}
    
    
    %\section{}
    \newpage
    \pagenumbering{gobble}
        \printbibliography


    \newpage
    \pagenumbering{roman}
    \appendix
        \part{Appendices}
            \input{8 - Hilbert complexes/main.tex}
            \input{9 - weak conservation proofs/main.tex}
\end{document}

            \documentclass[12pt, a4paper]{report}

\input{template/main.tex}

\title{\BA{Title in Progress...}}
\author{Boris Andrews}
\affil{Mathematical Institute, University of Oxford}
\date{\today}


\begin{document}
    \pagenumbering{gobble}
    \maketitle
    
    
    \begin{abstract}
        Magnetic confinement reactors---in particular tokamaks---offer one of the most promising options for achieving practical nuclear fusion, with the potential to provide virtually limitless, clean energy. The theoretical and numerical modeling of tokamak plasmas is simultaneously an essential component of effective reactor design, and a great research barrier. Tokamak operational conditions exhibit comparatively low Knudsen numbers. Kinetic effects, including kinetic waves and instabilities, Landau damping, bump-on-tail instabilities and more, are therefore highly influential in tokamak plasma dynamics. Purely fluid models are inherently incapable of capturing these effects, whereas the high dimensionality in purely kinetic models render them practically intractable for most relevant purposes.

        We consider a $\delta\!f$ decomposition model, with a macroscopic fluid background and microscopic kinetic correction, both fully coupled to each other. A similar manner of discretization is proposed to that used in the recent \texttt{STRUPHY} code \cite{Holderied_Possanner_Wang_2021, Holderied_2022, Li_et_al_2023} with a finite-element model for the background and a pseudo-particle/PiC model for the correction.

        The fluid background satisfies the full, non-linear, resistive, compressible, Hall MHD equations. \cite{Laakmann_Hu_Farrell_2022} introduces finite-element(-in-space) implicit timesteppers for the incompressible analogue to this system with structure-preserving (SP) properties in the ideal case, alongside parameter-robust preconditioners. We show that these timesteppers can derive from a finite-element-in-time (FET) (and finite-element-in-space) interpretation. The benefits of this reformulation are discussed, including the derivation of timesteppers that are higher order in time, and the quantifiable dissipative SP properties in the non-ideal, resistive case.
        
        We discuss possible options for extending this FET approach to timesteppers for the compressible case.

        The kinetic corrections satisfy linearized Boltzmann equations. Using a Lénard--Bernstein collision operator, these take Fokker--Planck-like forms \cite{Fokker_1914, Planck_1917} wherein pseudo-particles in the numerical model obey the neoclassical transport equations, with particle-independent Brownian drift terms. This offers a rigorous methodology for incorporating collisions into the particle transport model, without coupling the equations of motions for each particle.
        
        Works by Chen, Chacón et al. \cite{Chen_Chacón_Barnes_2011, Chacón_Chen_Barnes_2013, Chen_Chacón_2014, Chen_Chacón_2015} have developed structure-preserving particle pushers for neoclassical transport in the Vlasov equations, derived from Crank--Nicolson integrators. We show these too can can derive from a FET interpretation, similarly offering potential extensions to higher-order-in-time particle pushers. The FET formulation is used also to consider how the stochastic drift terms can be incorporated into the pushers. Stochastic gyrokinetic expansions are also discussed.

        Different options for the numerical implementation of these schemes are considered.

        Due to the efficacy of FET in the development of SP timesteppers for both the fluid and kinetic component, we hope this approach will prove effective in the future for developing SP timesteppers for the full hybrid model. We hope this will give us the opportunity to incorporate previously inaccessible kinetic effects into the highly effective, modern, finite-element MHD models.
    \end{abstract}
    
    
    \newpage
    \tableofcontents
    
    
    \newpage
    \pagenumbering{arabic}
    %\linenumbers\renewcommand\thelinenumber{\color{black!50}\arabic{linenumber}}
            \input{0 - introduction/main.tex}
        \part{Research}
            \input{1 - low-noise PiC models/main.tex}
            \input{2 - kinetic component/main.tex}
            \input{3 - fluid component/main.tex}
            \input{4 - numerical implementation/main.tex}
        \part{Project Overview}
            \input{5 - research plan/main.tex}
            \input{6 - summary/main.tex}
    
    
    %\section{}
    \newpage
    \pagenumbering{gobble}
        \printbibliography


    \newpage
    \pagenumbering{roman}
    \appendix
        \part{Appendices}
            \input{8 - Hilbert complexes/main.tex}
            \input{9 - weak conservation proofs/main.tex}
\end{document}

        \part{Project Overview}
            \documentclass[12pt, a4paper]{report}

\input{template/main.tex}

\title{\BA{Title in Progress...}}
\author{Boris Andrews}
\affil{Mathematical Institute, University of Oxford}
\date{\today}


\begin{document}
    \pagenumbering{gobble}
    \maketitle
    
    
    \begin{abstract}
        Magnetic confinement reactors---in particular tokamaks---offer one of the most promising options for achieving practical nuclear fusion, with the potential to provide virtually limitless, clean energy. The theoretical and numerical modeling of tokamak plasmas is simultaneously an essential component of effective reactor design, and a great research barrier. Tokamak operational conditions exhibit comparatively low Knudsen numbers. Kinetic effects, including kinetic waves and instabilities, Landau damping, bump-on-tail instabilities and more, are therefore highly influential in tokamak plasma dynamics. Purely fluid models are inherently incapable of capturing these effects, whereas the high dimensionality in purely kinetic models render them practically intractable for most relevant purposes.

        We consider a $\delta\!f$ decomposition model, with a macroscopic fluid background and microscopic kinetic correction, both fully coupled to each other. A similar manner of discretization is proposed to that used in the recent \texttt{STRUPHY} code \cite{Holderied_Possanner_Wang_2021, Holderied_2022, Li_et_al_2023} with a finite-element model for the background and a pseudo-particle/PiC model for the correction.

        The fluid background satisfies the full, non-linear, resistive, compressible, Hall MHD equations. \cite{Laakmann_Hu_Farrell_2022} introduces finite-element(-in-space) implicit timesteppers for the incompressible analogue to this system with structure-preserving (SP) properties in the ideal case, alongside parameter-robust preconditioners. We show that these timesteppers can derive from a finite-element-in-time (FET) (and finite-element-in-space) interpretation. The benefits of this reformulation are discussed, including the derivation of timesteppers that are higher order in time, and the quantifiable dissipative SP properties in the non-ideal, resistive case.
        
        We discuss possible options for extending this FET approach to timesteppers for the compressible case.

        The kinetic corrections satisfy linearized Boltzmann equations. Using a Lénard--Bernstein collision operator, these take Fokker--Planck-like forms \cite{Fokker_1914, Planck_1917} wherein pseudo-particles in the numerical model obey the neoclassical transport equations, with particle-independent Brownian drift terms. This offers a rigorous methodology for incorporating collisions into the particle transport model, without coupling the equations of motions for each particle.
        
        Works by Chen, Chacón et al. \cite{Chen_Chacón_Barnes_2011, Chacón_Chen_Barnes_2013, Chen_Chacón_2014, Chen_Chacón_2015} have developed structure-preserving particle pushers for neoclassical transport in the Vlasov equations, derived from Crank--Nicolson integrators. We show these too can can derive from a FET interpretation, similarly offering potential extensions to higher-order-in-time particle pushers. The FET formulation is used also to consider how the stochastic drift terms can be incorporated into the pushers. Stochastic gyrokinetic expansions are also discussed.

        Different options for the numerical implementation of these schemes are considered.

        Due to the efficacy of FET in the development of SP timesteppers for both the fluid and kinetic component, we hope this approach will prove effective in the future for developing SP timesteppers for the full hybrid model. We hope this will give us the opportunity to incorporate previously inaccessible kinetic effects into the highly effective, modern, finite-element MHD models.
    \end{abstract}
    
    
    \newpage
    \tableofcontents
    
    
    \newpage
    \pagenumbering{arabic}
    %\linenumbers\renewcommand\thelinenumber{\color{black!50}\arabic{linenumber}}
            \input{0 - introduction/main.tex}
        \part{Research}
            \input{1 - low-noise PiC models/main.tex}
            \input{2 - kinetic component/main.tex}
            \input{3 - fluid component/main.tex}
            \input{4 - numerical implementation/main.tex}
        \part{Project Overview}
            \input{5 - research plan/main.tex}
            \input{6 - summary/main.tex}
    
    
    %\section{}
    \newpage
    \pagenumbering{gobble}
        \printbibliography


    \newpage
    \pagenumbering{roman}
    \appendix
        \part{Appendices}
            \input{8 - Hilbert complexes/main.tex}
            \input{9 - weak conservation proofs/main.tex}
\end{document}

            \documentclass[12pt, a4paper]{report}

\input{template/main.tex}

\title{\BA{Title in Progress...}}
\author{Boris Andrews}
\affil{Mathematical Institute, University of Oxford}
\date{\today}


\begin{document}
    \pagenumbering{gobble}
    \maketitle
    
    
    \begin{abstract}
        Magnetic confinement reactors---in particular tokamaks---offer one of the most promising options for achieving practical nuclear fusion, with the potential to provide virtually limitless, clean energy. The theoretical and numerical modeling of tokamak plasmas is simultaneously an essential component of effective reactor design, and a great research barrier. Tokamak operational conditions exhibit comparatively low Knudsen numbers. Kinetic effects, including kinetic waves and instabilities, Landau damping, bump-on-tail instabilities and more, are therefore highly influential in tokamak plasma dynamics. Purely fluid models are inherently incapable of capturing these effects, whereas the high dimensionality in purely kinetic models render them practically intractable for most relevant purposes.

        We consider a $\delta\!f$ decomposition model, with a macroscopic fluid background and microscopic kinetic correction, both fully coupled to each other. A similar manner of discretization is proposed to that used in the recent \texttt{STRUPHY} code \cite{Holderied_Possanner_Wang_2021, Holderied_2022, Li_et_al_2023} with a finite-element model for the background and a pseudo-particle/PiC model for the correction.

        The fluid background satisfies the full, non-linear, resistive, compressible, Hall MHD equations. \cite{Laakmann_Hu_Farrell_2022} introduces finite-element(-in-space) implicit timesteppers for the incompressible analogue to this system with structure-preserving (SP) properties in the ideal case, alongside parameter-robust preconditioners. We show that these timesteppers can derive from a finite-element-in-time (FET) (and finite-element-in-space) interpretation. The benefits of this reformulation are discussed, including the derivation of timesteppers that are higher order in time, and the quantifiable dissipative SP properties in the non-ideal, resistive case.
        
        We discuss possible options for extending this FET approach to timesteppers for the compressible case.

        The kinetic corrections satisfy linearized Boltzmann equations. Using a Lénard--Bernstein collision operator, these take Fokker--Planck-like forms \cite{Fokker_1914, Planck_1917} wherein pseudo-particles in the numerical model obey the neoclassical transport equations, with particle-independent Brownian drift terms. This offers a rigorous methodology for incorporating collisions into the particle transport model, without coupling the equations of motions for each particle.
        
        Works by Chen, Chacón et al. \cite{Chen_Chacón_Barnes_2011, Chacón_Chen_Barnes_2013, Chen_Chacón_2014, Chen_Chacón_2015} have developed structure-preserving particle pushers for neoclassical transport in the Vlasov equations, derived from Crank--Nicolson integrators. We show these too can can derive from a FET interpretation, similarly offering potential extensions to higher-order-in-time particle pushers. The FET formulation is used also to consider how the stochastic drift terms can be incorporated into the pushers. Stochastic gyrokinetic expansions are also discussed.

        Different options for the numerical implementation of these schemes are considered.

        Due to the efficacy of FET in the development of SP timesteppers for both the fluid and kinetic component, we hope this approach will prove effective in the future for developing SP timesteppers for the full hybrid model. We hope this will give us the opportunity to incorporate previously inaccessible kinetic effects into the highly effective, modern, finite-element MHD models.
    \end{abstract}
    
    
    \newpage
    \tableofcontents
    
    
    \newpage
    \pagenumbering{arabic}
    %\linenumbers\renewcommand\thelinenumber{\color{black!50}\arabic{linenumber}}
            \input{0 - introduction/main.tex}
        \part{Research}
            \input{1 - low-noise PiC models/main.tex}
            \input{2 - kinetic component/main.tex}
            \input{3 - fluid component/main.tex}
            \input{4 - numerical implementation/main.tex}
        \part{Project Overview}
            \input{5 - research plan/main.tex}
            \input{6 - summary/main.tex}
    
    
    %\section{}
    \newpage
    \pagenumbering{gobble}
        \printbibliography


    \newpage
    \pagenumbering{roman}
    \appendix
        \part{Appendices}
            \input{8 - Hilbert complexes/main.tex}
            \input{9 - weak conservation proofs/main.tex}
\end{document}

    
    
    %\section{}
    \newpage
    \pagenumbering{gobble}
        \printbibliography


    \newpage
    \pagenumbering{roman}
    \appendix
        \part{Appendices}
            \documentclass[12pt, a4paper]{report}

\input{template/main.tex}

\title{\BA{Title in Progress...}}
\author{Boris Andrews}
\affil{Mathematical Institute, University of Oxford}
\date{\today}


\begin{document}
    \pagenumbering{gobble}
    \maketitle
    
    
    \begin{abstract}
        Magnetic confinement reactors---in particular tokamaks---offer one of the most promising options for achieving practical nuclear fusion, with the potential to provide virtually limitless, clean energy. The theoretical and numerical modeling of tokamak plasmas is simultaneously an essential component of effective reactor design, and a great research barrier. Tokamak operational conditions exhibit comparatively low Knudsen numbers. Kinetic effects, including kinetic waves and instabilities, Landau damping, bump-on-tail instabilities and more, are therefore highly influential in tokamak plasma dynamics. Purely fluid models are inherently incapable of capturing these effects, whereas the high dimensionality in purely kinetic models render them practically intractable for most relevant purposes.

        We consider a $\delta\!f$ decomposition model, with a macroscopic fluid background and microscopic kinetic correction, both fully coupled to each other. A similar manner of discretization is proposed to that used in the recent \texttt{STRUPHY} code \cite{Holderied_Possanner_Wang_2021, Holderied_2022, Li_et_al_2023} with a finite-element model for the background and a pseudo-particle/PiC model for the correction.

        The fluid background satisfies the full, non-linear, resistive, compressible, Hall MHD equations. \cite{Laakmann_Hu_Farrell_2022} introduces finite-element(-in-space) implicit timesteppers for the incompressible analogue to this system with structure-preserving (SP) properties in the ideal case, alongside parameter-robust preconditioners. We show that these timesteppers can derive from a finite-element-in-time (FET) (and finite-element-in-space) interpretation. The benefits of this reformulation are discussed, including the derivation of timesteppers that are higher order in time, and the quantifiable dissipative SP properties in the non-ideal, resistive case.
        
        We discuss possible options for extending this FET approach to timesteppers for the compressible case.

        The kinetic corrections satisfy linearized Boltzmann equations. Using a Lénard--Bernstein collision operator, these take Fokker--Planck-like forms \cite{Fokker_1914, Planck_1917} wherein pseudo-particles in the numerical model obey the neoclassical transport equations, with particle-independent Brownian drift terms. This offers a rigorous methodology for incorporating collisions into the particle transport model, without coupling the equations of motions for each particle.
        
        Works by Chen, Chacón et al. \cite{Chen_Chacón_Barnes_2011, Chacón_Chen_Barnes_2013, Chen_Chacón_2014, Chen_Chacón_2015} have developed structure-preserving particle pushers for neoclassical transport in the Vlasov equations, derived from Crank--Nicolson integrators. We show these too can can derive from a FET interpretation, similarly offering potential extensions to higher-order-in-time particle pushers. The FET formulation is used also to consider how the stochastic drift terms can be incorporated into the pushers. Stochastic gyrokinetic expansions are also discussed.

        Different options for the numerical implementation of these schemes are considered.

        Due to the efficacy of FET in the development of SP timesteppers for both the fluid and kinetic component, we hope this approach will prove effective in the future for developing SP timesteppers for the full hybrid model. We hope this will give us the opportunity to incorporate previously inaccessible kinetic effects into the highly effective, modern, finite-element MHD models.
    \end{abstract}
    
    
    \newpage
    \tableofcontents
    
    
    \newpage
    \pagenumbering{arabic}
    %\linenumbers\renewcommand\thelinenumber{\color{black!50}\arabic{linenumber}}
            \input{0 - introduction/main.tex}
        \part{Research}
            \input{1 - low-noise PiC models/main.tex}
            \input{2 - kinetic component/main.tex}
            \input{3 - fluid component/main.tex}
            \input{4 - numerical implementation/main.tex}
        \part{Project Overview}
            \input{5 - research plan/main.tex}
            \input{6 - summary/main.tex}
    
    
    %\section{}
    \newpage
    \pagenumbering{gobble}
        \printbibliography


    \newpage
    \pagenumbering{roman}
    \appendix
        \part{Appendices}
            \input{8 - Hilbert complexes/main.tex}
            \input{9 - weak conservation proofs/main.tex}
\end{document}

            \documentclass[12pt, a4paper]{report}

\input{template/main.tex}

\title{\BA{Title in Progress...}}
\author{Boris Andrews}
\affil{Mathematical Institute, University of Oxford}
\date{\today}


\begin{document}
    \pagenumbering{gobble}
    \maketitle
    
    
    \begin{abstract}
        Magnetic confinement reactors---in particular tokamaks---offer one of the most promising options for achieving practical nuclear fusion, with the potential to provide virtually limitless, clean energy. The theoretical and numerical modeling of tokamak plasmas is simultaneously an essential component of effective reactor design, and a great research barrier. Tokamak operational conditions exhibit comparatively low Knudsen numbers. Kinetic effects, including kinetic waves and instabilities, Landau damping, bump-on-tail instabilities and more, are therefore highly influential in tokamak plasma dynamics. Purely fluid models are inherently incapable of capturing these effects, whereas the high dimensionality in purely kinetic models render them practically intractable for most relevant purposes.

        We consider a $\delta\!f$ decomposition model, with a macroscopic fluid background and microscopic kinetic correction, both fully coupled to each other. A similar manner of discretization is proposed to that used in the recent \texttt{STRUPHY} code \cite{Holderied_Possanner_Wang_2021, Holderied_2022, Li_et_al_2023} with a finite-element model for the background and a pseudo-particle/PiC model for the correction.

        The fluid background satisfies the full, non-linear, resistive, compressible, Hall MHD equations. \cite{Laakmann_Hu_Farrell_2022} introduces finite-element(-in-space) implicit timesteppers for the incompressible analogue to this system with structure-preserving (SP) properties in the ideal case, alongside parameter-robust preconditioners. We show that these timesteppers can derive from a finite-element-in-time (FET) (and finite-element-in-space) interpretation. The benefits of this reformulation are discussed, including the derivation of timesteppers that are higher order in time, and the quantifiable dissipative SP properties in the non-ideal, resistive case.
        
        We discuss possible options for extending this FET approach to timesteppers for the compressible case.

        The kinetic corrections satisfy linearized Boltzmann equations. Using a Lénard--Bernstein collision operator, these take Fokker--Planck-like forms \cite{Fokker_1914, Planck_1917} wherein pseudo-particles in the numerical model obey the neoclassical transport equations, with particle-independent Brownian drift terms. This offers a rigorous methodology for incorporating collisions into the particle transport model, without coupling the equations of motions for each particle.
        
        Works by Chen, Chacón et al. \cite{Chen_Chacón_Barnes_2011, Chacón_Chen_Barnes_2013, Chen_Chacón_2014, Chen_Chacón_2015} have developed structure-preserving particle pushers for neoclassical transport in the Vlasov equations, derived from Crank--Nicolson integrators. We show these too can can derive from a FET interpretation, similarly offering potential extensions to higher-order-in-time particle pushers. The FET formulation is used also to consider how the stochastic drift terms can be incorporated into the pushers. Stochastic gyrokinetic expansions are also discussed.

        Different options for the numerical implementation of these schemes are considered.

        Due to the efficacy of FET in the development of SP timesteppers for both the fluid and kinetic component, we hope this approach will prove effective in the future for developing SP timesteppers for the full hybrid model. We hope this will give us the opportunity to incorporate previously inaccessible kinetic effects into the highly effective, modern, finite-element MHD models.
    \end{abstract}
    
    
    \newpage
    \tableofcontents
    
    
    \newpage
    \pagenumbering{arabic}
    %\linenumbers\renewcommand\thelinenumber{\color{black!50}\arabic{linenumber}}
            \input{0 - introduction/main.tex}
        \part{Research}
            \input{1 - low-noise PiC models/main.tex}
            \input{2 - kinetic component/main.tex}
            \input{3 - fluid component/main.tex}
            \input{4 - numerical implementation/main.tex}
        \part{Project Overview}
            \input{5 - research plan/main.tex}
            \input{6 - summary/main.tex}
    
    
    %\section{}
    \newpage
    \pagenumbering{gobble}
        \printbibliography


    \newpage
    \pagenumbering{roman}
    \appendix
        \part{Appendices}
            \input{8 - Hilbert complexes/main.tex}
            \input{9 - weak conservation proofs/main.tex}
\end{document}

\end{document}

        \part{Project Overview}
            \documentclass[12pt, a4paper]{report}

\documentclass[12pt, a4paper]{report}

\input{template/main.tex}

\title{\BA{Title in Progress...}}
\author{Boris Andrews}
\affil{Mathematical Institute, University of Oxford}
\date{\today}


\begin{document}
    \pagenumbering{gobble}
    \maketitle
    
    
    \begin{abstract}
        Magnetic confinement reactors---in particular tokamaks---offer one of the most promising options for achieving practical nuclear fusion, with the potential to provide virtually limitless, clean energy. The theoretical and numerical modeling of tokamak plasmas is simultaneously an essential component of effective reactor design, and a great research barrier. Tokamak operational conditions exhibit comparatively low Knudsen numbers. Kinetic effects, including kinetic waves and instabilities, Landau damping, bump-on-tail instabilities and more, are therefore highly influential in tokamak plasma dynamics. Purely fluid models are inherently incapable of capturing these effects, whereas the high dimensionality in purely kinetic models render them practically intractable for most relevant purposes.

        We consider a $\delta\!f$ decomposition model, with a macroscopic fluid background and microscopic kinetic correction, both fully coupled to each other. A similar manner of discretization is proposed to that used in the recent \texttt{STRUPHY} code \cite{Holderied_Possanner_Wang_2021, Holderied_2022, Li_et_al_2023} with a finite-element model for the background and a pseudo-particle/PiC model for the correction.

        The fluid background satisfies the full, non-linear, resistive, compressible, Hall MHD equations. \cite{Laakmann_Hu_Farrell_2022} introduces finite-element(-in-space) implicit timesteppers for the incompressible analogue to this system with structure-preserving (SP) properties in the ideal case, alongside parameter-robust preconditioners. We show that these timesteppers can derive from a finite-element-in-time (FET) (and finite-element-in-space) interpretation. The benefits of this reformulation are discussed, including the derivation of timesteppers that are higher order in time, and the quantifiable dissipative SP properties in the non-ideal, resistive case.
        
        We discuss possible options for extending this FET approach to timesteppers for the compressible case.

        The kinetic corrections satisfy linearized Boltzmann equations. Using a Lénard--Bernstein collision operator, these take Fokker--Planck-like forms \cite{Fokker_1914, Planck_1917} wherein pseudo-particles in the numerical model obey the neoclassical transport equations, with particle-independent Brownian drift terms. This offers a rigorous methodology for incorporating collisions into the particle transport model, without coupling the equations of motions for each particle.
        
        Works by Chen, Chacón et al. \cite{Chen_Chacón_Barnes_2011, Chacón_Chen_Barnes_2013, Chen_Chacón_2014, Chen_Chacón_2015} have developed structure-preserving particle pushers for neoclassical transport in the Vlasov equations, derived from Crank--Nicolson integrators. We show these too can can derive from a FET interpretation, similarly offering potential extensions to higher-order-in-time particle pushers. The FET formulation is used also to consider how the stochastic drift terms can be incorporated into the pushers. Stochastic gyrokinetic expansions are also discussed.

        Different options for the numerical implementation of these schemes are considered.

        Due to the efficacy of FET in the development of SP timesteppers for both the fluid and kinetic component, we hope this approach will prove effective in the future for developing SP timesteppers for the full hybrid model. We hope this will give us the opportunity to incorporate previously inaccessible kinetic effects into the highly effective, modern, finite-element MHD models.
    \end{abstract}
    
    
    \newpage
    \tableofcontents
    
    
    \newpage
    \pagenumbering{arabic}
    %\linenumbers\renewcommand\thelinenumber{\color{black!50}\arabic{linenumber}}
            \input{0 - introduction/main.tex}
        \part{Research}
            \input{1 - low-noise PiC models/main.tex}
            \input{2 - kinetic component/main.tex}
            \input{3 - fluid component/main.tex}
            \input{4 - numerical implementation/main.tex}
        \part{Project Overview}
            \input{5 - research plan/main.tex}
            \input{6 - summary/main.tex}
    
    
    %\section{}
    \newpage
    \pagenumbering{gobble}
        \printbibliography


    \newpage
    \pagenumbering{roman}
    \appendix
        \part{Appendices}
            \input{8 - Hilbert complexes/main.tex}
            \input{9 - weak conservation proofs/main.tex}
\end{document}


\title{\BA{Title in Progress...}}
\author{Boris Andrews}
\affil{Mathematical Institute, University of Oxford}
\date{\today}


\begin{document}
    \pagenumbering{gobble}
    \maketitle
    
    
    \begin{abstract}
        Magnetic confinement reactors---in particular tokamaks---offer one of the most promising options for achieving practical nuclear fusion, with the potential to provide virtually limitless, clean energy. The theoretical and numerical modeling of tokamak plasmas is simultaneously an essential component of effective reactor design, and a great research barrier. Tokamak operational conditions exhibit comparatively low Knudsen numbers. Kinetic effects, including kinetic waves and instabilities, Landau damping, bump-on-tail instabilities and more, are therefore highly influential in tokamak plasma dynamics. Purely fluid models are inherently incapable of capturing these effects, whereas the high dimensionality in purely kinetic models render them practically intractable for most relevant purposes.

        We consider a $\delta\!f$ decomposition model, with a macroscopic fluid background and microscopic kinetic correction, both fully coupled to each other. A similar manner of discretization is proposed to that used in the recent \texttt{STRUPHY} code \cite{Holderied_Possanner_Wang_2021, Holderied_2022, Li_et_al_2023} with a finite-element model for the background and a pseudo-particle/PiC model for the correction.

        The fluid background satisfies the full, non-linear, resistive, compressible, Hall MHD equations. \cite{Laakmann_Hu_Farrell_2022} introduces finite-element(-in-space) implicit timesteppers for the incompressible analogue to this system with structure-preserving (SP) properties in the ideal case, alongside parameter-robust preconditioners. We show that these timesteppers can derive from a finite-element-in-time (FET) (and finite-element-in-space) interpretation. The benefits of this reformulation are discussed, including the derivation of timesteppers that are higher order in time, and the quantifiable dissipative SP properties in the non-ideal, resistive case.
        
        We discuss possible options for extending this FET approach to timesteppers for the compressible case.

        The kinetic corrections satisfy linearized Boltzmann equations. Using a Lénard--Bernstein collision operator, these take Fokker--Planck-like forms \cite{Fokker_1914, Planck_1917} wherein pseudo-particles in the numerical model obey the neoclassical transport equations, with particle-independent Brownian drift terms. This offers a rigorous methodology for incorporating collisions into the particle transport model, without coupling the equations of motions for each particle.
        
        Works by Chen, Chacón et al. \cite{Chen_Chacón_Barnes_2011, Chacón_Chen_Barnes_2013, Chen_Chacón_2014, Chen_Chacón_2015} have developed structure-preserving particle pushers for neoclassical transport in the Vlasov equations, derived from Crank--Nicolson integrators. We show these too can can derive from a FET interpretation, similarly offering potential extensions to higher-order-in-time particle pushers. The FET formulation is used also to consider how the stochastic drift terms can be incorporated into the pushers. Stochastic gyrokinetic expansions are also discussed.

        Different options for the numerical implementation of these schemes are considered.

        Due to the efficacy of FET in the development of SP timesteppers for both the fluid and kinetic component, we hope this approach will prove effective in the future for developing SP timesteppers for the full hybrid model. We hope this will give us the opportunity to incorporate previously inaccessible kinetic effects into the highly effective, modern, finite-element MHD models.
    \end{abstract}
    
    
    \newpage
    \tableofcontents
    
    
    \newpage
    \pagenumbering{arabic}
    %\linenumbers\renewcommand\thelinenumber{\color{black!50}\arabic{linenumber}}
            \documentclass[12pt, a4paper]{report}

\input{template/main.tex}

\title{\BA{Title in Progress...}}
\author{Boris Andrews}
\affil{Mathematical Institute, University of Oxford}
\date{\today}


\begin{document}
    \pagenumbering{gobble}
    \maketitle
    
    
    \begin{abstract}
        Magnetic confinement reactors---in particular tokamaks---offer one of the most promising options for achieving practical nuclear fusion, with the potential to provide virtually limitless, clean energy. The theoretical and numerical modeling of tokamak plasmas is simultaneously an essential component of effective reactor design, and a great research barrier. Tokamak operational conditions exhibit comparatively low Knudsen numbers. Kinetic effects, including kinetic waves and instabilities, Landau damping, bump-on-tail instabilities and more, are therefore highly influential in tokamak plasma dynamics. Purely fluid models are inherently incapable of capturing these effects, whereas the high dimensionality in purely kinetic models render them practically intractable for most relevant purposes.

        We consider a $\delta\!f$ decomposition model, with a macroscopic fluid background and microscopic kinetic correction, both fully coupled to each other. A similar manner of discretization is proposed to that used in the recent \texttt{STRUPHY} code \cite{Holderied_Possanner_Wang_2021, Holderied_2022, Li_et_al_2023} with a finite-element model for the background and a pseudo-particle/PiC model for the correction.

        The fluid background satisfies the full, non-linear, resistive, compressible, Hall MHD equations. \cite{Laakmann_Hu_Farrell_2022} introduces finite-element(-in-space) implicit timesteppers for the incompressible analogue to this system with structure-preserving (SP) properties in the ideal case, alongside parameter-robust preconditioners. We show that these timesteppers can derive from a finite-element-in-time (FET) (and finite-element-in-space) interpretation. The benefits of this reformulation are discussed, including the derivation of timesteppers that are higher order in time, and the quantifiable dissipative SP properties in the non-ideal, resistive case.
        
        We discuss possible options for extending this FET approach to timesteppers for the compressible case.

        The kinetic corrections satisfy linearized Boltzmann equations. Using a Lénard--Bernstein collision operator, these take Fokker--Planck-like forms \cite{Fokker_1914, Planck_1917} wherein pseudo-particles in the numerical model obey the neoclassical transport equations, with particle-independent Brownian drift terms. This offers a rigorous methodology for incorporating collisions into the particle transport model, without coupling the equations of motions for each particle.
        
        Works by Chen, Chacón et al. \cite{Chen_Chacón_Barnes_2011, Chacón_Chen_Barnes_2013, Chen_Chacón_2014, Chen_Chacón_2015} have developed structure-preserving particle pushers for neoclassical transport in the Vlasov equations, derived from Crank--Nicolson integrators. We show these too can can derive from a FET interpretation, similarly offering potential extensions to higher-order-in-time particle pushers. The FET formulation is used also to consider how the stochastic drift terms can be incorporated into the pushers. Stochastic gyrokinetic expansions are also discussed.

        Different options for the numerical implementation of these schemes are considered.

        Due to the efficacy of FET in the development of SP timesteppers for both the fluid and kinetic component, we hope this approach will prove effective in the future for developing SP timesteppers for the full hybrid model. We hope this will give us the opportunity to incorporate previously inaccessible kinetic effects into the highly effective, modern, finite-element MHD models.
    \end{abstract}
    
    
    \newpage
    \tableofcontents
    
    
    \newpage
    \pagenumbering{arabic}
    %\linenumbers\renewcommand\thelinenumber{\color{black!50}\arabic{linenumber}}
            \input{0 - introduction/main.tex}
        \part{Research}
            \input{1 - low-noise PiC models/main.tex}
            \input{2 - kinetic component/main.tex}
            \input{3 - fluid component/main.tex}
            \input{4 - numerical implementation/main.tex}
        \part{Project Overview}
            \input{5 - research plan/main.tex}
            \input{6 - summary/main.tex}
    
    
    %\section{}
    \newpage
    \pagenumbering{gobble}
        \printbibliography


    \newpage
    \pagenumbering{roman}
    \appendix
        \part{Appendices}
            \input{8 - Hilbert complexes/main.tex}
            \input{9 - weak conservation proofs/main.tex}
\end{document}

        \part{Research}
            \documentclass[12pt, a4paper]{report}

\input{template/main.tex}

\title{\BA{Title in Progress...}}
\author{Boris Andrews}
\affil{Mathematical Institute, University of Oxford}
\date{\today}


\begin{document}
    \pagenumbering{gobble}
    \maketitle
    
    
    \begin{abstract}
        Magnetic confinement reactors---in particular tokamaks---offer one of the most promising options for achieving practical nuclear fusion, with the potential to provide virtually limitless, clean energy. The theoretical and numerical modeling of tokamak plasmas is simultaneously an essential component of effective reactor design, and a great research barrier. Tokamak operational conditions exhibit comparatively low Knudsen numbers. Kinetic effects, including kinetic waves and instabilities, Landau damping, bump-on-tail instabilities and more, are therefore highly influential in tokamak plasma dynamics. Purely fluid models are inherently incapable of capturing these effects, whereas the high dimensionality in purely kinetic models render them practically intractable for most relevant purposes.

        We consider a $\delta\!f$ decomposition model, with a macroscopic fluid background and microscopic kinetic correction, both fully coupled to each other. A similar manner of discretization is proposed to that used in the recent \texttt{STRUPHY} code \cite{Holderied_Possanner_Wang_2021, Holderied_2022, Li_et_al_2023} with a finite-element model for the background and a pseudo-particle/PiC model for the correction.

        The fluid background satisfies the full, non-linear, resistive, compressible, Hall MHD equations. \cite{Laakmann_Hu_Farrell_2022} introduces finite-element(-in-space) implicit timesteppers for the incompressible analogue to this system with structure-preserving (SP) properties in the ideal case, alongside parameter-robust preconditioners. We show that these timesteppers can derive from a finite-element-in-time (FET) (and finite-element-in-space) interpretation. The benefits of this reformulation are discussed, including the derivation of timesteppers that are higher order in time, and the quantifiable dissipative SP properties in the non-ideal, resistive case.
        
        We discuss possible options for extending this FET approach to timesteppers for the compressible case.

        The kinetic corrections satisfy linearized Boltzmann equations. Using a Lénard--Bernstein collision operator, these take Fokker--Planck-like forms \cite{Fokker_1914, Planck_1917} wherein pseudo-particles in the numerical model obey the neoclassical transport equations, with particle-independent Brownian drift terms. This offers a rigorous methodology for incorporating collisions into the particle transport model, without coupling the equations of motions for each particle.
        
        Works by Chen, Chacón et al. \cite{Chen_Chacón_Barnes_2011, Chacón_Chen_Barnes_2013, Chen_Chacón_2014, Chen_Chacón_2015} have developed structure-preserving particle pushers for neoclassical transport in the Vlasov equations, derived from Crank--Nicolson integrators. We show these too can can derive from a FET interpretation, similarly offering potential extensions to higher-order-in-time particle pushers. The FET formulation is used also to consider how the stochastic drift terms can be incorporated into the pushers. Stochastic gyrokinetic expansions are also discussed.

        Different options for the numerical implementation of these schemes are considered.

        Due to the efficacy of FET in the development of SP timesteppers for both the fluid and kinetic component, we hope this approach will prove effective in the future for developing SP timesteppers for the full hybrid model. We hope this will give us the opportunity to incorporate previously inaccessible kinetic effects into the highly effective, modern, finite-element MHD models.
    \end{abstract}
    
    
    \newpage
    \tableofcontents
    
    
    \newpage
    \pagenumbering{arabic}
    %\linenumbers\renewcommand\thelinenumber{\color{black!50}\arabic{linenumber}}
            \input{0 - introduction/main.tex}
        \part{Research}
            \input{1 - low-noise PiC models/main.tex}
            \input{2 - kinetic component/main.tex}
            \input{3 - fluid component/main.tex}
            \input{4 - numerical implementation/main.tex}
        \part{Project Overview}
            \input{5 - research plan/main.tex}
            \input{6 - summary/main.tex}
    
    
    %\section{}
    \newpage
    \pagenumbering{gobble}
        \printbibliography


    \newpage
    \pagenumbering{roman}
    \appendix
        \part{Appendices}
            \input{8 - Hilbert complexes/main.tex}
            \input{9 - weak conservation proofs/main.tex}
\end{document}

            \documentclass[12pt, a4paper]{report}

\input{template/main.tex}

\title{\BA{Title in Progress...}}
\author{Boris Andrews}
\affil{Mathematical Institute, University of Oxford}
\date{\today}


\begin{document}
    \pagenumbering{gobble}
    \maketitle
    
    
    \begin{abstract}
        Magnetic confinement reactors---in particular tokamaks---offer one of the most promising options for achieving practical nuclear fusion, with the potential to provide virtually limitless, clean energy. The theoretical and numerical modeling of tokamak plasmas is simultaneously an essential component of effective reactor design, and a great research barrier. Tokamak operational conditions exhibit comparatively low Knudsen numbers. Kinetic effects, including kinetic waves and instabilities, Landau damping, bump-on-tail instabilities and more, are therefore highly influential in tokamak plasma dynamics. Purely fluid models are inherently incapable of capturing these effects, whereas the high dimensionality in purely kinetic models render them practically intractable for most relevant purposes.

        We consider a $\delta\!f$ decomposition model, with a macroscopic fluid background and microscopic kinetic correction, both fully coupled to each other. A similar manner of discretization is proposed to that used in the recent \texttt{STRUPHY} code \cite{Holderied_Possanner_Wang_2021, Holderied_2022, Li_et_al_2023} with a finite-element model for the background and a pseudo-particle/PiC model for the correction.

        The fluid background satisfies the full, non-linear, resistive, compressible, Hall MHD equations. \cite{Laakmann_Hu_Farrell_2022} introduces finite-element(-in-space) implicit timesteppers for the incompressible analogue to this system with structure-preserving (SP) properties in the ideal case, alongside parameter-robust preconditioners. We show that these timesteppers can derive from a finite-element-in-time (FET) (and finite-element-in-space) interpretation. The benefits of this reformulation are discussed, including the derivation of timesteppers that are higher order in time, and the quantifiable dissipative SP properties in the non-ideal, resistive case.
        
        We discuss possible options for extending this FET approach to timesteppers for the compressible case.

        The kinetic corrections satisfy linearized Boltzmann equations. Using a Lénard--Bernstein collision operator, these take Fokker--Planck-like forms \cite{Fokker_1914, Planck_1917} wherein pseudo-particles in the numerical model obey the neoclassical transport equations, with particle-independent Brownian drift terms. This offers a rigorous methodology for incorporating collisions into the particle transport model, without coupling the equations of motions for each particle.
        
        Works by Chen, Chacón et al. \cite{Chen_Chacón_Barnes_2011, Chacón_Chen_Barnes_2013, Chen_Chacón_2014, Chen_Chacón_2015} have developed structure-preserving particle pushers for neoclassical transport in the Vlasov equations, derived from Crank--Nicolson integrators. We show these too can can derive from a FET interpretation, similarly offering potential extensions to higher-order-in-time particle pushers. The FET formulation is used also to consider how the stochastic drift terms can be incorporated into the pushers. Stochastic gyrokinetic expansions are also discussed.

        Different options for the numerical implementation of these schemes are considered.

        Due to the efficacy of FET in the development of SP timesteppers for both the fluid and kinetic component, we hope this approach will prove effective in the future for developing SP timesteppers for the full hybrid model. We hope this will give us the opportunity to incorporate previously inaccessible kinetic effects into the highly effective, modern, finite-element MHD models.
    \end{abstract}
    
    
    \newpage
    \tableofcontents
    
    
    \newpage
    \pagenumbering{arabic}
    %\linenumbers\renewcommand\thelinenumber{\color{black!50}\arabic{linenumber}}
            \input{0 - introduction/main.tex}
        \part{Research}
            \input{1 - low-noise PiC models/main.tex}
            \input{2 - kinetic component/main.tex}
            \input{3 - fluid component/main.tex}
            \input{4 - numerical implementation/main.tex}
        \part{Project Overview}
            \input{5 - research plan/main.tex}
            \input{6 - summary/main.tex}
    
    
    %\section{}
    \newpage
    \pagenumbering{gobble}
        \printbibliography


    \newpage
    \pagenumbering{roman}
    \appendix
        \part{Appendices}
            \input{8 - Hilbert complexes/main.tex}
            \input{9 - weak conservation proofs/main.tex}
\end{document}

            \documentclass[12pt, a4paper]{report}

\input{template/main.tex}

\title{\BA{Title in Progress...}}
\author{Boris Andrews}
\affil{Mathematical Institute, University of Oxford}
\date{\today}


\begin{document}
    \pagenumbering{gobble}
    \maketitle
    
    
    \begin{abstract}
        Magnetic confinement reactors---in particular tokamaks---offer one of the most promising options for achieving practical nuclear fusion, with the potential to provide virtually limitless, clean energy. The theoretical and numerical modeling of tokamak plasmas is simultaneously an essential component of effective reactor design, and a great research barrier. Tokamak operational conditions exhibit comparatively low Knudsen numbers. Kinetic effects, including kinetic waves and instabilities, Landau damping, bump-on-tail instabilities and more, are therefore highly influential in tokamak plasma dynamics. Purely fluid models are inherently incapable of capturing these effects, whereas the high dimensionality in purely kinetic models render them practically intractable for most relevant purposes.

        We consider a $\delta\!f$ decomposition model, with a macroscopic fluid background and microscopic kinetic correction, both fully coupled to each other. A similar manner of discretization is proposed to that used in the recent \texttt{STRUPHY} code \cite{Holderied_Possanner_Wang_2021, Holderied_2022, Li_et_al_2023} with a finite-element model for the background and a pseudo-particle/PiC model for the correction.

        The fluid background satisfies the full, non-linear, resistive, compressible, Hall MHD equations. \cite{Laakmann_Hu_Farrell_2022} introduces finite-element(-in-space) implicit timesteppers for the incompressible analogue to this system with structure-preserving (SP) properties in the ideal case, alongside parameter-robust preconditioners. We show that these timesteppers can derive from a finite-element-in-time (FET) (and finite-element-in-space) interpretation. The benefits of this reformulation are discussed, including the derivation of timesteppers that are higher order in time, and the quantifiable dissipative SP properties in the non-ideal, resistive case.
        
        We discuss possible options for extending this FET approach to timesteppers for the compressible case.

        The kinetic corrections satisfy linearized Boltzmann equations. Using a Lénard--Bernstein collision operator, these take Fokker--Planck-like forms \cite{Fokker_1914, Planck_1917} wherein pseudo-particles in the numerical model obey the neoclassical transport equations, with particle-independent Brownian drift terms. This offers a rigorous methodology for incorporating collisions into the particle transport model, without coupling the equations of motions for each particle.
        
        Works by Chen, Chacón et al. \cite{Chen_Chacón_Barnes_2011, Chacón_Chen_Barnes_2013, Chen_Chacón_2014, Chen_Chacón_2015} have developed structure-preserving particle pushers for neoclassical transport in the Vlasov equations, derived from Crank--Nicolson integrators. We show these too can can derive from a FET interpretation, similarly offering potential extensions to higher-order-in-time particle pushers. The FET formulation is used also to consider how the stochastic drift terms can be incorporated into the pushers. Stochastic gyrokinetic expansions are also discussed.

        Different options for the numerical implementation of these schemes are considered.

        Due to the efficacy of FET in the development of SP timesteppers for both the fluid and kinetic component, we hope this approach will prove effective in the future for developing SP timesteppers for the full hybrid model. We hope this will give us the opportunity to incorporate previously inaccessible kinetic effects into the highly effective, modern, finite-element MHD models.
    \end{abstract}
    
    
    \newpage
    \tableofcontents
    
    
    \newpage
    \pagenumbering{arabic}
    %\linenumbers\renewcommand\thelinenumber{\color{black!50}\arabic{linenumber}}
            \input{0 - introduction/main.tex}
        \part{Research}
            \input{1 - low-noise PiC models/main.tex}
            \input{2 - kinetic component/main.tex}
            \input{3 - fluid component/main.tex}
            \input{4 - numerical implementation/main.tex}
        \part{Project Overview}
            \input{5 - research plan/main.tex}
            \input{6 - summary/main.tex}
    
    
    %\section{}
    \newpage
    \pagenumbering{gobble}
        \printbibliography


    \newpage
    \pagenumbering{roman}
    \appendix
        \part{Appendices}
            \input{8 - Hilbert complexes/main.tex}
            \input{9 - weak conservation proofs/main.tex}
\end{document}

            \documentclass[12pt, a4paper]{report}

\input{template/main.tex}

\title{\BA{Title in Progress...}}
\author{Boris Andrews}
\affil{Mathematical Institute, University of Oxford}
\date{\today}


\begin{document}
    \pagenumbering{gobble}
    \maketitle
    
    
    \begin{abstract}
        Magnetic confinement reactors---in particular tokamaks---offer one of the most promising options for achieving practical nuclear fusion, with the potential to provide virtually limitless, clean energy. The theoretical and numerical modeling of tokamak plasmas is simultaneously an essential component of effective reactor design, and a great research barrier. Tokamak operational conditions exhibit comparatively low Knudsen numbers. Kinetic effects, including kinetic waves and instabilities, Landau damping, bump-on-tail instabilities and more, are therefore highly influential in tokamak plasma dynamics. Purely fluid models are inherently incapable of capturing these effects, whereas the high dimensionality in purely kinetic models render them practically intractable for most relevant purposes.

        We consider a $\delta\!f$ decomposition model, with a macroscopic fluid background and microscopic kinetic correction, both fully coupled to each other. A similar manner of discretization is proposed to that used in the recent \texttt{STRUPHY} code \cite{Holderied_Possanner_Wang_2021, Holderied_2022, Li_et_al_2023} with a finite-element model for the background and a pseudo-particle/PiC model for the correction.

        The fluid background satisfies the full, non-linear, resistive, compressible, Hall MHD equations. \cite{Laakmann_Hu_Farrell_2022} introduces finite-element(-in-space) implicit timesteppers for the incompressible analogue to this system with structure-preserving (SP) properties in the ideal case, alongside parameter-robust preconditioners. We show that these timesteppers can derive from a finite-element-in-time (FET) (and finite-element-in-space) interpretation. The benefits of this reformulation are discussed, including the derivation of timesteppers that are higher order in time, and the quantifiable dissipative SP properties in the non-ideal, resistive case.
        
        We discuss possible options for extending this FET approach to timesteppers for the compressible case.

        The kinetic corrections satisfy linearized Boltzmann equations. Using a Lénard--Bernstein collision operator, these take Fokker--Planck-like forms \cite{Fokker_1914, Planck_1917} wherein pseudo-particles in the numerical model obey the neoclassical transport equations, with particle-independent Brownian drift terms. This offers a rigorous methodology for incorporating collisions into the particle transport model, without coupling the equations of motions for each particle.
        
        Works by Chen, Chacón et al. \cite{Chen_Chacón_Barnes_2011, Chacón_Chen_Barnes_2013, Chen_Chacón_2014, Chen_Chacón_2015} have developed structure-preserving particle pushers for neoclassical transport in the Vlasov equations, derived from Crank--Nicolson integrators. We show these too can can derive from a FET interpretation, similarly offering potential extensions to higher-order-in-time particle pushers. The FET formulation is used also to consider how the stochastic drift terms can be incorporated into the pushers. Stochastic gyrokinetic expansions are also discussed.

        Different options for the numerical implementation of these schemes are considered.

        Due to the efficacy of FET in the development of SP timesteppers for both the fluid and kinetic component, we hope this approach will prove effective in the future for developing SP timesteppers for the full hybrid model. We hope this will give us the opportunity to incorporate previously inaccessible kinetic effects into the highly effective, modern, finite-element MHD models.
    \end{abstract}
    
    
    \newpage
    \tableofcontents
    
    
    \newpage
    \pagenumbering{arabic}
    %\linenumbers\renewcommand\thelinenumber{\color{black!50}\arabic{linenumber}}
            \input{0 - introduction/main.tex}
        \part{Research}
            \input{1 - low-noise PiC models/main.tex}
            \input{2 - kinetic component/main.tex}
            \input{3 - fluid component/main.tex}
            \input{4 - numerical implementation/main.tex}
        \part{Project Overview}
            \input{5 - research plan/main.tex}
            \input{6 - summary/main.tex}
    
    
    %\section{}
    \newpage
    \pagenumbering{gobble}
        \printbibliography


    \newpage
    \pagenumbering{roman}
    \appendix
        \part{Appendices}
            \input{8 - Hilbert complexes/main.tex}
            \input{9 - weak conservation proofs/main.tex}
\end{document}

        \part{Project Overview}
            \documentclass[12pt, a4paper]{report}

\input{template/main.tex}

\title{\BA{Title in Progress...}}
\author{Boris Andrews}
\affil{Mathematical Institute, University of Oxford}
\date{\today}


\begin{document}
    \pagenumbering{gobble}
    \maketitle
    
    
    \begin{abstract}
        Magnetic confinement reactors---in particular tokamaks---offer one of the most promising options for achieving practical nuclear fusion, with the potential to provide virtually limitless, clean energy. The theoretical and numerical modeling of tokamak plasmas is simultaneously an essential component of effective reactor design, and a great research barrier. Tokamak operational conditions exhibit comparatively low Knudsen numbers. Kinetic effects, including kinetic waves and instabilities, Landau damping, bump-on-tail instabilities and more, are therefore highly influential in tokamak plasma dynamics. Purely fluid models are inherently incapable of capturing these effects, whereas the high dimensionality in purely kinetic models render them practically intractable for most relevant purposes.

        We consider a $\delta\!f$ decomposition model, with a macroscopic fluid background and microscopic kinetic correction, both fully coupled to each other. A similar manner of discretization is proposed to that used in the recent \texttt{STRUPHY} code \cite{Holderied_Possanner_Wang_2021, Holderied_2022, Li_et_al_2023} with a finite-element model for the background and a pseudo-particle/PiC model for the correction.

        The fluid background satisfies the full, non-linear, resistive, compressible, Hall MHD equations. \cite{Laakmann_Hu_Farrell_2022} introduces finite-element(-in-space) implicit timesteppers for the incompressible analogue to this system with structure-preserving (SP) properties in the ideal case, alongside parameter-robust preconditioners. We show that these timesteppers can derive from a finite-element-in-time (FET) (and finite-element-in-space) interpretation. The benefits of this reformulation are discussed, including the derivation of timesteppers that are higher order in time, and the quantifiable dissipative SP properties in the non-ideal, resistive case.
        
        We discuss possible options for extending this FET approach to timesteppers for the compressible case.

        The kinetic corrections satisfy linearized Boltzmann equations. Using a Lénard--Bernstein collision operator, these take Fokker--Planck-like forms \cite{Fokker_1914, Planck_1917} wherein pseudo-particles in the numerical model obey the neoclassical transport equations, with particle-independent Brownian drift terms. This offers a rigorous methodology for incorporating collisions into the particle transport model, without coupling the equations of motions for each particle.
        
        Works by Chen, Chacón et al. \cite{Chen_Chacón_Barnes_2011, Chacón_Chen_Barnes_2013, Chen_Chacón_2014, Chen_Chacón_2015} have developed structure-preserving particle pushers for neoclassical transport in the Vlasov equations, derived from Crank--Nicolson integrators. We show these too can can derive from a FET interpretation, similarly offering potential extensions to higher-order-in-time particle pushers. The FET formulation is used also to consider how the stochastic drift terms can be incorporated into the pushers. Stochastic gyrokinetic expansions are also discussed.

        Different options for the numerical implementation of these schemes are considered.

        Due to the efficacy of FET in the development of SP timesteppers for both the fluid and kinetic component, we hope this approach will prove effective in the future for developing SP timesteppers for the full hybrid model. We hope this will give us the opportunity to incorporate previously inaccessible kinetic effects into the highly effective, modern, finite-element MHD models.
    \end{abstract}
    
    
    \newpage
    \tableofcontents
    
    
    \newpage
    \pagenumbering{arabic}
    %\linenumbers\renewcommand\thelinenumber{\color{black!50}\arabic{linenumber}}
            \input{0 - introduction/main.tex}
        \part{Research}
            \input{1 - low-noise PiC models/main.tex}
            \input{2 - kinetic component/main.tex}
            \input{3 - fluid component/main.tex}
            \input{4 - numerical implementation/main.tex}
        \part{Project Overview}
            \input{5 - research plan/main.tex}
            \input{6 - summary/main.tex}
    
    
    %\section{}
    \newpage
    \pagenumbering{gobble}
        \printbibliography


    \newpage
    \pagenumbering{roman}
    \appendix
        \part{Appendices}
            \input{8 - Hilbert complexes/main.tex}
            \input{9 - weak conservation proofs/main.tex}
\end{document}

            \documentclass[12pt, a4paper]{report}

\input{template/main.tex}

\title{\BA{Title in Progress...}}
\author{Boris Andrews}
\affil{Mathematical Institute, University of Oxford}
\date{\today}


\begin{document}
    \pagenumbering{gobble}
    \maketitle
    
    
    \begin{abstract}
        Magnetic confinement reactors---in particular tokamaks---offer one of the most promising options for achieving practical nuclear fusion, with the potential to provide virtually limitless, clean energy. The theoretical and numerical modeling of tokamak plasmas is simultaneously an essential component of effective reactor design, and a great research barrier. Tokamak operational conditions exhibit comparatively low Knudsen numbers. Kinetic effects, including kinetic waves and instabilities, Landau damping, bump-on-tail instabilities and more, are therefore highly influential in tokamak plasma dynamics. Purely fluid models are inherently incapable of capturing these effects, whereas the high dimensionality in purely kinetic models render them practically intractable for most relevant purposes.

        We consider a $\delta\!f$ decomposition model, with a macroscopic fluid background and microscopic kinetic correction, both fully coupled to each other. A similar manner of discretization is proposed to that used in the recent \texttt{STRUPHY} code \cite{Holderied_Possanner_Wang_2021, Holderied_2022, Li_et_al_2023} with a finite-element model for the background and a pseudo-particle/PiC model for the correction.

        The fluid background satisfies the full, non-linear, resistive, compressible, Hall MHD equations. \cite{Laakmann_Hu_Farrell_2022} introduces finite-element(-in-space) implicit timesteppers for the incompressible analogue to this system with structure-preserving (SP) properties in the ideal case, alongside parameter-robust preconditioners. We show that these timesteppers can derive from a finite-element-in-time (FET) (and finite-element-in-space) interpretation. The benefits of this reformulation are discussed, including the derivation of timesteppers that are higher order in time, and the quantifiable dissipative SP properties in the non-ideal, resistive case.
        
        We discuss possible options for extending this FET approach to timesteppers for the compressible case.

        The kinetic corrections satisfy linearized Boltzmann equations. Using a Lénard--Bernstein collision operator, these take Fokker--Planck-like forms \cite{Fokker_1914, Planck_1917} wherein pseudo-particles in the numerical model obey the neoclassical transport equations, with particle-independent Brownian drift terms. This offers a rigorous methodology for incorporating collisions into the particle transport model, without coupling the equations of motions for each particle.
        
        Works by Chen, Chacón et al. \cite{Chen_Chacón_Barnes_2011, Chacón_Chen_Barnes_2013, Chen_Chacón_2014, Chen_Chacón_2015} have developed structure-preserving particle pushers for neoclassical transport in the Vlasov equations, derived from Crank--Nicolson integrators. We show these too can can derive from a FET interpretation, similarly offering potential extensions to higher-order-in-time particle pushers. The FET formulation is used also to consider how the stochastic drift terms can be incorporated into the pushers. Stochastic gyrokinetic expansions are also discussed.

        Different options for the numerical implementation of these schemes are considered.

        Due to the efficacy of FET in the development of SP timesteppers for both the fluid and kinetic component, we hope this approach will prove effective in the future for developing SP timesteppers for the full hybrid model. We hope this will give us the opportunity to incorporate previously inaccessible kinetic effects into the highly effective, modern, finite-element MHD models.
    \end{abstract}
    
    
    \newpage
    \tableofcontents
    
    
    \newpage
    \pagenumbering{arabic}
    %\linenumbers\renewcommand\thelinenumber{\color{black!50}\arabic{linenumber}}
            \input{0 - introduction/main.tex}
        \part{Research}
            \input{1 - low-noise PiC models/main.tex}
            \input{2 - kinetic component/main.tex}
            \input{3 - fluid component/main.tex}
            \input{4 - numerical implementation/main.tex}
        \part{Project Overview}
            \input{5 - research plan/main.tex}
            \input{6 - summary/main.tex}
    
    
    %\section{}
    \newpage
    \pagenumbering{gobble}
        \printbibliography


    \newpage
    \pagenumbering{roman}
    \appendix
        \part{Appendices}
            \input{8 - Hilbert complexes/main.tex}
            \input{9 - weak conservation proofs/main.tex}
\end{document}

    
    
    %\section{}
    \newpage
    \pagenumbering{gobble}
        \printbibliography


    \newpage
    \pagenumbering{roman}
    \appendix
        \part{Appendices}
            \documentclass[12pt, a4paper]{report}

\input{template/main.tex}

\title{\BA{Title in Progress...}}
\author{Boris Andrews}
\affil{Mathematical Institute, University of Oxford}
\date{\today}


\begin{document}
    \pagenumbering{gobble}
    \maketitle
    
    
    \begin{abstract}
        Magnetic confinement reactors---in particular tokamaks---offer one of the most promising options for achieving practical nuclear fusion, with the potential to provide virtually limitless, clean energy. The theoretical and numerical modeling of tokamak plasmas is simultaneously an essential component of effective reactor design, and a great research barrier. Tokamak operational conditions exhibit comparatively low Knudsen numbers. Kinetic effects, including kinetic waves and instabilities, Landau damping, bump-on-tail instabilities and more, are therefore highly influential in tokamak plasma dynamics. Purely fluid models are inherently incapable of capturing these effects, whereas the high dimensionality in purely kinetic models render them practically intractable for most relevant purposes.

        We consider a $\delta\!f$ decomposition model, with a macroscopic fluid background and microscopic kinetic correction, both fully coupled to each other. A similar manner of discretization is proposed to that used in the recent \texttt{STRUPHY} code \cite{Holderied_Possanner_Wang_2021, Holderied_2022, Li_et_al_2023} with a finite-element model for the background and a pseudo-particle/PiC model for the correction.

        The fluid background satisfies the full, non-linear, resistive, compressible, Hall MHD equations. \cite{Laakmann_Hu_Farrell_2022} introduces finite-element(-in-space) implicit timesteppers for the incompressible analogue to this system with structure-preserving (SP) properties in the ideal case, alongside parameter-robust preconditioners. We show that these timesteppers can derive from a finite-element-in-time (FET) (and finite-element-in-space) interpretation. The benefits of this reformulation are discussed, including the derivation of timesteppers that are higher order in time, and the quantifiable dissipative SP properties in the non-ideal, resistive case.
        
        We discuss possible options for extending this FET approach to timesteppers for the compressible case.

        The kinetic corrections satisfy linearized Boltzmann equations. Using a Lénard--Bernstein collision operator, these take Fokker--Planck-like forms \cite{Fokker_1914, Planck_1917} wherein pseudo-particles in the numerical model obey the neoclassical transport equations, with particle-independent Brownian drift terms. This offers a rigorous methodology for incorporating collisions into the particle transport model, without coupling the equations of motions for each particle.
        
        Works by Chen, Chacón et al. \cite{Chen_Chacón_Barnes_2011, Chacón_Chen_Barnes_2013, Chen_Chacón_2014, Chen_Chacón_2015} have developed structure-preserving particle pushers for neoclassical transport in the Vlasov equations, derived from Crank--Nicolson integrators. We show these too can can derive from a FET interpretation, similarly offering potential extensions to higher-order-in-time particle pushers. The FET formulation is used also to consider how the stochastic drift terms can be incorporated into the pushers. Stochastic gyrokinetic expansions are also discussed.

        Different options for the numerical implementation of these schemes are considered.

        Due to the efficacy of FET in the development of SP timesteppers for both the fluid and kinetic component, we hope this approach will prove effective in the future for developing SP timesteppers for the full hybrid model. We hope this will give us the opportunity to incorporate previously inaccessible kinetic effects into the highly effective, modern, finite-element MHD models.
    \end{abstract}
    
    
    \newpage
    \tableofcontents
    
    
    \newpage
    \pagenumbering{arabic}
    %\linenumbers\renewcommand\thelinenumber{\color{black!50}\arabic{linenumber}}
            \input{0 - introduction/main.tex}
        \part{Research}
            \input{1 - low-noise PiC models/main.tex}
            \input{2 - kinetic component/main.tex}
            \input{3 - fluid component/main.tex}
            \input{4 - numerical implementation/main.tex}
        \part{Project Overview}
            \input{5 - research plan/main.tex}
            \input{6 - summary/main.tex}
    
    
    %\section{}
    \newpage
    \pagenumbering{gobble}
        \printbibliography


    \newpage
    \pagenumbering{roman}
    \appendix
        \part{Appendices}
            \input{8 - Hilbert complexes/main.tex}
            \input{9 - weak conservation proofs/main.tex}
\end{document}

            \documentclass[12pt, a4paper]{report}

\input{template/main.tex}

\title{\BA{Title in Progress...}}
\author{Boris Andrews}
\affil{Mathematical Institute, University of Oxford}
\date{\today}


\begin{document}
    \pagenumbering{gobble}
    \maketitle
    
    
    \begin{abstract}
        Magnetic confinement reactors---in particular tokamaks---offer one of the most promising options for achieving practical nuclear fusion, with the potential to provide virtually limitless, clean energy. The theoretical and numerical modeling of tokamak plasmas is simultaneously an essential component of effective reactor design, and a great research barrier. Tokamak operational conditions exhibit comparatively low Knudsen numbers. Kinetic effects, including kinetic waves and instabilities, Landau damping, bump-on-tail instabilities and more, are therefore highly influential in tokamak plasma dynamics. Purely fluid models are inherently incapable of capturing these effects, whereas the high dimensionality in purely kinetic models render them practically intractable for most relevant purposes.

        We consider a $\delta\!f$ decomposition model, with a macroscopic fluid background and microscopic kinetic correction, both fully coupled to each other. A similar manner of discretization is proposed to that used in the recent \texttt{STRUPHY} code \cite{Holderied_Possanner_Wang_2021, Holderied_2022, Li_et_al_2023} with a finite-element model for the background and a pseudo-particle/PiC model for the correction.

        The fluid background satisfies the full, non-linear, resistive, compressible, Hall MHD equations. \cite{Laakmann_Hu_Farrell_2022} introduces finite-element(-in-space) implicit timesteppers for the incompressible analogue to this system with structure-preserving (SP) properties in the ideal case, alongside parameter-robust preconditioners. We show that these timesteppers can derive from a finite-element-in-time (FET) (and finite-element-in-space) interpretation. The benefits of this reformulation are discussed, including the derivation of timesteppers that are higher order in time, and the quantifiable dissipative SP properties in the non-ideal, resistive case.
        
        We discuss possible options for extending this FET approach to timesteppers for the compressible case.

        The kinetic corrections satisfy linearized Boltzmann equations. Using a Lénard--Bernstein collision operator, these take Fokker--Planck-like forms \cite{Fokker_1914, Planck_1917} wherein pseudo-particles in the numerical model obey the neoclassical transport equations, with particle-independent Brownian drift terms. This offers a rigorous methodology for incorporating collisions into the particle transport model, without coupling the equations of motions for each particle.
        
        Works by Chen, Chacón et al. \cite{Chen_Chacón_Barnes_2011, Chacón_Chen_Barnes_2013, Chen_Chacón_2014, Chen_Chacón_2015} have developed structure-preserving particle pushers for neoclassical transport in the Vlasov equations, derived from Crank--Nicolson integrators. We show these too can can derive from a FET interpretation, similarly offering potential extensions to higher-order-in-time particle pushers. The FET formulation is used also to consider how the stochastic drift terms can be incorporated into the pushers. Stochastic gyrokinetic expansions are also discussed.

        Different options for the numerical implementation of these schemes are considered.

        Due to the efficacy of FET in the development of SP timesteppers for both the fluid and kinetic component, we hope this approach will prove effective in the future for developing SP timesteppers for the full hybrid model. We hope this will give us the opportunity to incorporate previously inaccessible kinetic effects into the highly effective, modern, finite-element MHD models.
    \end{abstract}
    
    
    \newpage
    \tableofcontents
    
    
    \newpage
    \pagenumbering{arabic}
    %\linenumbers\renewcommand\thelinenumber{\color{black!50}\arabic{linenumber}}
            \input{0 - introduction/main.tex}
        \part{Research}
            \input{1 - low-noise PiC models/main.tex}
            \input{2 - kinetic component/main.tex}
            \input{3 - fluid component/main.tex}
            \input{4 - numerical implementation/main.tex}
        \part{Project Overview}
            \input{5 - research plan/main.tex}
            \input{6 - summary/main.tex}
    
    
    %\section{}
    \newpage
    \pagenumbering{gobble}
        \printbibliography


    \newpage
    \pagenumbering{roman}
    \appendix
        \part{Appendices}
            \input{8 - Hilbert complexes/main.tex}
            \input{9 - weak conservation proofs/main.tex}
\end{document}

\end{document}

            \documentclass[12pt, a4paper]{report}

\documentclass[12pt, a4paper]{report}

\input{template/main.tex}

\title{\BA{Title in Progress...}}
\author{Boris Andrews}
\affil{Mathematical Institute, University of Oxford}
\date{\today}


\begin{document}
    \pagenumbering{gobble}
    \maketitle
    
    
    \begin{abstract}
        Magnetic confinement reactors---in particular tokamaks---offer one of the most promising options for achieving practical nuclear fusion, with the potential to provide virtually limitless, clean energy. The theoretical and numerical modeling of tokamak plasmas is simultaneously an essential component of effective reactor design, and a great research barrier. Tokamak operational conditions exhibit comparatively low Knudsen numbers. Kinetic effects, including kinetic waves and instabilities, Landau damping, bump-on-tail instabilities and more, are therefore highly influential in tokamak plasma dynamics. Purely fluid models are inherently incapable of capturing these effects, whereas the high dimensionality in purely kinetic models render them practically intractable for most relevant purposes.

        We consider a $\delta\!f$ decomposition model, with a macroscopic fluid background and microscopic kinetic correction, both fully coupled to each other. A similar manner of discretization is proposed to that used in the recent \texttt{STRUPHY} code \cite{Holderied_Possanner_Wang_2021, Holderied_2022, Li_et_al_2023} with a finite-element model for the background and a pseudo-particle/PiC model for the correction.

        The fluid background satisfies the full, non-linear, resistive, compressible, Hall MHD equations. \cite{Laakmann_Hu_Farrell_2022} introduces finite-element(-in-space) implicit timesteppers for the incompressible analogue to this system with structure-preserving (SP) properties in the ideal case, alongside parameter-robust preconditioners. We show that these timesteppers can derive from a finite-element-in-time (FET) (and finite-element-in-space) interpretation. The benefits of this reformulation are discussed, including the derivation of timesteppers that are higher order in time, and the quantifiable dissipative SP properties in the non-ideal, resistive case.
        
        We discuss possible options for extending this FET approach to timesteppers for the compressible case.

        The kinetic corrections satisfy linearized Boltzmann equations. Using a Lénard--Bernstein collision operator, these take Fokker--Planck-like forms \cite{Fokker_1914, Planck_1917} wherein pseudo-particles in the numerical model obey the neoclassical transport equations, with particle-independent Brownian drift terms. This offers a rigorous methodology for incorporating collisions into the particle transport model, without coupling the equations of motions for each particle.
        
        Works by Chen, Chacón et al. \cite{Chen_Chacón_Barnes_2011, Chacón_Chen_Barnes_2013, Chen_Chacón_2014, Chen_Chacón_2015} have developed structure-preserving particle pushers for neoclassical transport in the Vlasov equations, derived from Crank--Nicolson integrators. We show these too can can derive from a FET interpretation, similarly offering potential extensions to higher-order-in-time particle pushers. The FET formulation is used also to consider how the stochastic drift terms can be incorporated into the pushers. Stochastic gyrokinetic expansions are also discussed.

        Different options for the numerical implementation of these schemes are considered.

        Due to the efficacy of FET in the development of SP timesteppers for both the fluid and kinetic component, we hope this approach will prove effective in the future for developing SP timesteppers for the full hybrid model. We hope this will give us the opportunity to incorporate previously inaccessible kinetic effects into the highly effective, modern, finite-element MHD models.
    \end{abstract}
    
    
    \newpage
    \tableofcontents
    
    
    \newpage
    \pagenumbering{arabic}
    %\linenumbers\renewcommand\thelinenumber{\color{black!50}\arabic{linenumber}}
            \input{0 - introduction/main.tex}
        \part{Research}
            \input{1 - low-noise PiC models/main.tex}
            \input{2 - kinetic component/main.tex}
            \input{3 - fluid component/main.tex}
            \input{4 - numerical implementation/main.tex}
        \part{Project Overview}
            \input{5 - research plan/main.tex}
            \input{6 - summary/main.tex}
    
    
    %\section{}
    \newpage
    \pagenumbering{gobble}
        \printbibliography


    \newpage
    \pagenumbering{roman}
    \appendix
        \part{Appendices}
            \input{8 - Hilbert complexes/main.tex}
            \input{9 - weak conservation proofs/main.tex}
\end{document}


\title{\BA{Title in Progress...}}
\author{Boris Andrews}
\affil{Mathematical Institute, University of Oxford}
\date{\today}


\begin{document}
    \pagenumbering{gobble}
    \maketitle
    
    
    \begin{abstract}
        Magnetic confinement reactors---in particular tokamaks---offer one of the most promising options for achieving practical nuclear fusion, with the potential to provide virtually limitless, clean energy. The theoretical and numerical modeling of tokamak plasmas is simultaneously an essential component of effective reactor design, and a great research barrier. Tokamak operational conditions exhibit comparatively low Knudsen numbers. Kinetic effects, including kinetic waves and instabilities, Landau damping, bump-on-tail instabilities and more, are therefore highly influential in tokamak plasma dynamics. Purely fluid models are inherently incapable of capturing these effects, whereas the high dimensionality in purely kinetic models render them practically intractable for most relevant purposes.

        We consider a $\delta\!f$ decomposition model, with a macroscopic fluid background and microscopic kinetic correction, both fully coupled to each other. A similar manner of discretization is proposed to that used in the recent \texttt{STRUPHY} code \cite{Holderied_Possanner_Wang_2021, Holderied_2022, Li_et_al_2023} with a finite-element model for the background and a pseudo-particle/PiC model for the correction.

        The fluid background satisfies the full, non-linear, resistive, compressible, Hall MHD equations. \cite{Laakmann_Hu_Farrell_2022} introduces finite-element(-in-space) implicit timesteppers for the incompressible analogue to this system with structure-preserving (SP) properties in the ideal case, alongside parameter-robust preconditioners. We show that these timesteppers can derive from a finite-element-in-time (FET) (and finite-element-in-space) interpretation. The benefits of this reformulation are discussed, including the derivation of timesteppers that are higher order in time, and the quantifiable dissipative SP properties in the non-ideal, resistive case.
        
        We discuss possible options for extending this FET approach to timesteppers for the compressible case.

        The kinetic corrections satisfy linearized Boltzmann equations. Using a Lénard--Bernstein collision operator, these take Fokker--Planck-like forms \cite{Fokker_1914, Planck_1917} wherein pseudo-particles in the numerical model obey the neoclassical transport equations, with particle-independent Brownian drift terms. This offers a rigorous methodology for incorporating collisions into the particle transport model, without coupling the equations of motions for each particle.
        
        Works by Chen, Chacón et al. \cite{Chen_Chacón_Barnes_2011, Chacón_Chen_Barnes_2013, Chen_Chacón_2014, Chen_Chacón_2015} have developed structure-preserving particle pushers for neoclassical transport in the Vlasov equations, derived from Crank--Nicolson integrators. We show these too can can derive from a FET interpretation, similarly offering potential extensions to higher-order-in-time particle pushers. The FET formulation is used also to consider how the stochastic drift terms can be incorporated into the pushers. Stochastic gyrokinetic expansions are also discussed.

        Different options for the numerical implementation of these schemes are considered.

        Due to the efficacy of FET in the development of SP timesteppers for both the fluid and kinetic component, we hope this approach will prove effective in the future for developing SP timesteppers for the full hybrid model. We hope this will give us the opportunity to incorporate previously inaccessible kinetic effects into the highly effective, modern, finite-element MHD models.
    \end{abstract}
    
    
    \newpage
    \tableofcontents
    
    
    \newpage
    \pagenumbering{arabic}
    %\linenumbers\renewcommand\thelinenumber{\color{black!50}\arabic{linenumber}}
            \documentclass[12pt, a4paper]{report}

\input{template/main.tex}

\title{\BA{Title in Progress...}}
\author{Boris Andrews}
\affil{Mathematical Institute, University of Oxford}
\date{\today}


\begin{document}
    \pagenumbering{gobble}
    \maketitle
    
    
    \begin{abstract}
        Magnetic confinement reactors---in particular tokamaks---offer one of the most promising options for achieving practical nuclear fusion, with the potential to provide virtually limitless, clean energy. The theoretical and numerical modeling of tokamak plasmas is simultaneously an essential component of effective reactor design, and a great research barrier. Tokamak operational conditions exhibit comparatively low Knudsen numbers. Kinetic effects, including kinetic waves and instabilities, Landau damping, bump-on-tail instabilities and more, are therefore highly influential in tokamak plasma dynamics. Purely fluid models are inherently incapable of capturing these effects, whereas the high dimensionality in purely kinetic models render them practically intractable for most relevant purposes.

        We consider a $\delta\!f$ decomposition model, with a macroscopic fluid background and microscopic kinetic correction, both fully coupled to each other. A similar manner of discretization is proposed to that used in the recent \texttt{STRUPHY} code \cite{Holderied_Possanner_Wang_2021, Holderied_2022, Li_et_al_2023} with a finite-element model for the background and a pseudo-particle/PiC model for the correction.

        The fluid background satisfies the full, non-linear, resistive, compressible, Hall MHD equations. \cite{Laakmann_Hu_Farrell_2022} introduces finite-element(-in-space) implicit timesteppers for the incompressible analogue to this system with structure-preserving (SP) properties in the ideal case, alongside parameter-robust preconditioners. We show that these timesteppers can derive from a finite-element-in-time (FET) (and finite-element-in-space) interpretation. The benefits of this reformulation are discussed, including the derivation of timesteppers that are higher order in time, and the quantifiable dissipative SP properties in the non-ideal, resistive case.
        
        We discuss possible options for extending this FET approach to timesteppers for the compressible case.

        The kinetic corrections satisfy linearized Boltzmann equations. Using a Lénard--Bernstein collision operator, these take Fokker--Planck-like forms \cite{Fokker_1914, Planck_1917} wherein pseudo-particles in the numerical model obey the neoclassical transport equations, with particle-independent Brownian drift terms. This offers a rigorous methodology for incorporating collisions into the particle transport model, without coupling the equations of motions for each particle.
        
        Works by Chen, Chacón et al. \cite{Chen_Chacón_Barnes_2011, Chacón_Chen_Barnes_2013, Chen_Chacón_2014, Chen_Chacón_2015} have developed structure-preserving particle pushers for neoclassical transport in the Vlasov equations, derived from Crank--Nicolson integrators. We show these too can can derive from a FET interpretation, similarly offering potential extensions to higher-order-in-time particle pushers. The FET formulation is used also to consider how the stochastic drift terms can be incorporated into the pushers. Stochastic gyrokinetic expansions are also discussed.

        Different options for the numerical implementation of these schemes are considered.

        Due to the efficacy of FET in the development of SP timesteppers for both the fluid and kinetic component, we hope this approach will prove effective in the future for developing SP timesteppers for the full hybrid model. We hope this will give us the opportunity to incorporate previously inaccessible kinetic effects into the highly effective, modern, finite-element MHD models.
    \end{abstract}
    
    
    \newpage
    \tableofcontents
    
    
    \newpage
    \pagenumbering{arabic}
    %\linenumbers\renewcommand\thelinenumber{\color{black!50}\arabic{linenumber}}
            \input{0 - introduction/main.tex}
        \part{Research}
            \input{1 - low-noise PiC models/main.tex}
            \input{2 - kinetic component/main.tex}
            \input{3 - fluid component/main.tex}
            \input{4 - numerical implementation/main.tex}
        \part{Project Overview}
            \input{5 - research plan/main.tex}
            \input{6 - summary/main.tex}
    
    
    %\section{}
    \newpage
    \pagenumbering{gobble}
        \printbibliography


    \newpage
    \pagenumbering{roman}
    \appendix
        \part{Appendices}
            \input{8 - Hilbert complexes/main.tex}
            \input{9 - weak conservation proofs/main.tex}
\end{document}

        \part{Research}
            \documentclass[12pt, a4paper]{report}

\input{template/main.tex}

\title{\BA{Title in Progress...}}
\author{Boris Andrews}
\affil{Mathematical Institute, University of Oxford}
\date{\today}


\begin{document}
    \pagenumbering{gobble}
    \maketitle
    
    
    \begin{abstract}
        Magnetic confinement reactors---in particular tokamaks---offer one of the most promising options for achieving practical nuclear fusion, with the potential to provide virtually limitless, clean energy. The theoretical and numerical modeling of tokamak plasmas is simultaneously an essential component of effective reactor design, and a great research barrier. Tokamak operational conditions exhibit comparatively low Knudsen numbers. Kinetic effects, including kinetic waves and instabilities, Landau damping, bump-on-tail instabilities and more, are therefore highly influential in tokamak plasma dynamics. Purely fluid models are inherently incapable of capturing these effects, whereas the high dimensionality in purely kinetic models render them practically intractable for most relevant purposes.

        We consider a $\delta\!f$ decomposition model, with a macroscopic fluid background and microscopic kinetic correction, both fully coupled to each other. A similar manner of discretization is proposed to that used in the recent \texttt{STRUPHY} code \cite{Holderied_Possanner_Wang_2021, Holderied_2022, Li_et_al_2023} with a finite-element model for the background and a pseudo-particle/PiC model for the correction.

        The fluid background satisfies the full, non-linear, resistive, compressible, Hall MHD equations. \cite{Laakmann_Hu_Farrell_2022} introduces finite-element(-in-space) implicit timesteppers for the incompressible analogue to this system with structure-preserving (SP) properties in the ideal case, alongside parameter-robust preconditioners. We show that these timesteppers can derive from a finite-element-in-time (FET) (and finite-element-in-space) interpretation. The benefits of this reformulation are discussed, including the derivation of timesteppers that are higher order in time, and the quantifiable dissipative SP properties in the non-ideal, resistive case.
        
        We discuss possible options for extending this FET approach to timesteppers for the compressible case.

        The kinetic corrections satisfy linearized Boltzmann equations. Using a Lénard--Bernstein collision operator, these take Fokker--Planck-like forms \cite{Fokker_1914, Planck_1917} wherein pseudo-particles in the numerical model obey the neoclassical transport equations, with particle-independent Brownian drift terms. This offers a rigorous methodology for incorporating collisions into the particle transport model, without coupling the equations of motions for each particle.
        
        Works by Chen, Chacón et al. \cite{Chen_Chacón_Barnes_2011, Chacón_Chen_Barnes_2013, Chen_Chacón_2014, Chen_Chacón_2015} have developed structure-preserving particle pushers for neoclassical transport in the Vlasov equations, derived from Crank--Nicolson integrators. We show these too can can derive from a FET interpretation, similarly offering potential extensions to higher-order-in-time particle pushers. The FET formulation is used also to consider how the stochastic drift terms can be incorporated into the pushers. Stochastic gyrokinetic expansions are also discussed.

        Different options for the numerical implementation of these schemes are considered.

        Due to the efficacy of FET in the development of SP timesteppers for both the fluid and kinetic component, we hope this approach will prove effective in the future for developing SP timesteppers for the full hybrid model. We hope this will give us the opportunity to incorporate previously inaccessible kinetic effects into the highly effective, modern, finite-element MHD models.
    \end{abstract}
    
    
    \newpage
    \tableofcontents
    
    
    \newpage
    \pagenumbering{arabic}
    %\linenumbers\renewcommand\thelinenumber{\color{black!50}\arabic{linenumber}}
            \input{0 - introduction/main.tex}
        \part{Research}
            \input{1 - low-noise PiC models/main.tex}
            \input{2 - kinetic component/main.tex}
            \input{3 - fluid component/main.tex}
            \input{4 - numerical implementation/main.tex}
        \part{Project Overview}
            \input{5 - research plan/main.tex}
            \input{6 - summary/main.tex}
    
    
    %\section{}
    \newpage
    \pagenumbering{gobble}
        \printbibliography


    \newpage
    \pagenumbering{roman}
    \appendix
        \part{Appendices}
            \input{8 - Hilbert complexes/main.tex}
            \input{9 - weak conservation proofs/main.tex}
\end{document}

            \documentclass[12pt, a4paper]{report}

\input{template/main.tex}

\title{\BA{Title in Progress...}}
\author{Boris Andrews}
\affil{Mathematical Institute, University of Oxford}
\date{\today}


\begin{document}
    \pagenumbering{gobble}
    \maketitle
    
    
    \begin{abstract}
        Magnetic confinement reactors---in particular tokamaks---offer one of the most promising options for achieving practical nuclear fusion, with the potential to provide virtually limitless, clean energy. The theoretical and numerical modeling of tokamak plasmas is simultaneously an essential component of effective reactor design, and a great research barrier. Tokamak operational conditions exhibit comparatively low Knudsen numbers. Kinetic effects, including kinetic waves and instabilities, Landau damping, bump-on-tail instabilities and more, are therefore highly influential in tokamak plasma dynamics. Purely fluid models are inherently incapable of capturing these effects, whereas the high dimensionality in purely kinetic models render them practically intractable for most relevant purposes.

        We consider a $\delta\!f$ decomposition model, with a macroscopic fluid background and microscopic kinetic correction, both fully coupled to each other. A similar manner of discretization is proposed to that used in the recent \texttt{STRUPHY} code \cite{Holderied_Possanner_Wang_2021, Holderied_2022, Li_et_al_2023} with a finite-element model for the background and a pseudo-particle/PiC model for the correction.

        The fluid background satisfies the full, non-linear, resistive, compressible, Hall MHD equations. \cite{Laakmann_Hu_Farrell_2022} introduces finite-element(-in-space) implicit timesteppers for the incompressible analogue to this system with structure-preserving (SP) properties in the ideal case, alongside parameter-robust preconditioners. We show that these timesteppers can derive from a finite-element-in-time (FET) (and finite-element-in-space) interpretation. The benefits of this reformulation are discussed, including the derivation of timesteppers that are higher order in time, and the quantifiable dissipative SP properties in the non-ideal, resistive case.
        
        We discuss possible options for extending this FET approach to timesteppers for the compressible case.

        The kinetic corrections satisfy linearized Boltzmann equations. Using a Lénard--Bernstein collision operator, these take Fokker--Planck-like forms \cite{Fokker_1914, Planck_1917} wherein pseudo-particles in the numerical model obey the neoclassical transport equations, with particle-independent Brownian drift terms. This offers a rigorous methodology for incorporating collisions into the particle transport model, without coupling the equations of motions for each particle.
        
        Works by Chen, Chacón et al. \cite{Chen_Chacón_Barnes_2011, Chacón_Chen_Barnes_2013, Chen_Chacón_2014, Chen_Chacón_2015} have developed structure-preserving particle pushers for neoclassical transport in the Vlasov equations, derived from Crank--Nicolson integrators. We show these too can can derive from a FET interpretation, similarly offering potential extensions to higher-order-in-time particle pushers. The FET formulation is used also to consider how the stochastic drift terms can be incorporated into the pushers. Stochastic gyrokinetic expansions are also discussed.

        Different options for the numerical implementation of these schemes are considered.

        Due to the efficacy of FET in the development of SP timesteppers for both the fluid and kinetic component, we hope this approach will prove effective in the future for developing SP timesteppers for the full hybrid model. We hope this will give us the opportunity to incorporate previously inaccessible kinetic effects into the highly effective, modern, finite-element MHD models.
    \end{abstract}
    
    
    \newpage
    \tableofcontents
    
    
    \newpage
    \pagenumbering{arabic}
    %\linenumbers\renewcommand\thelinenumber{\color{black!50}\arabic{linenumber}}
            \input{0 - introduction/main.tex}
        \part{Research}
            \input{1 - low-noise PiC models/main.tex}
            \input{2 - kinetic component/main.tex}
            \input{3 - fluid component/main.tex}
            \input{4 - numerical implementation/main.tex}
        \part{Project Overview}
            \input{5 - research plan/main.tex}
            \input{6 - summary/main.tex}
    
    
    %\section{}
    \newpage
    \pagenumbering{gobble}
        \printbibliography


    \newpage
    \pagenumbering{roman}
    \appendix
        \part{Appendices}
            \input{8 - Hilbert complexes/main.tex}
            \input{9 - weak conservation proofs/main.tex}
\end{document}

            \documentclass[12pt, a4paper]{report}

\input{template/main.tex}

\title{\BA{Title in Progress...}}
\author{Boris Andrews}
\affil{Mathematical Institute, University of Oxford}
\date{\today}


\begin{document}
    \pagenumbering{gobble}
    \maketitle
    
    
    \begin{abstract}
        Magnetic confinement reactors---in particular tokamaks---offer one of the most promising options for achieving practical nuclear fusion, with the potential to provide virtually limitless, clean energy. The theoretical and numerical modeling of tokamak plasmas is simultaneously an essential component of effective reactor design, and a great research barrier. Tokamak operational conditions exhibit comparatively low Knudsen numbers. Kinetic effects, including kinetic waves and instabilities, Landau damping, bump-on-tail instabilities and more, are therefore highly influential in tokamak plasma dynamics. Purely fluid models are inherently incapable of capturing these effects, whereas the high dimensionality in purely kinetic models render them practically intractable for most relevant purposes.

        We consider a $\delta\!f$ decomposition model, with a macroscopic fluid background and microscopic kinetic correction, both fully coupled to each other. A similar manner of discretization is proposed to that used in the recent \texttt{STRUPHY} code \cite{Holderied_Possanner_Wang_2021, Holderied_2022, Li_et_al_2023} with a finite-element model for the background and a pseudo-particle/PiC model for the correction.

        The fluid background satisfies the full, non-linear, resistive, compressible, Hall MHD equations. \cite{Laakmann_Hu_Farrell_2022} introduces finite-element(-in-space) implicit timesteppers for the incompressible analogue to this system with structure-preserving (SP) properties in the ideal case, alongside parameter-robust preconditioners. We show that these timesteppers can derive from a finite-element-in-time (FET) (and finite-element-in-space) interpretation. The benefits of this reformulation are discussed, including the derivation of timesteppers that are higher order in time, and the quantifiable dissipative SP properties in the non-ideal, resistive case.
        
        We discuss possible options for extending this FET approach to timesteppers for the compressible case.

        The kinetic corrections satisfy linearized Boltzmann equations. Using a Lénard--Bernstein collision operator, these take Fokker--Planck-like forms \cite{Fokker_1914, Planck_1917} wherein pseudo-particles in the numerical model obey the neoclassical transport equations, with particle-independent Brownian drift terms. This offers a rigorous methodology for incorporating collisions into the particle transport model, without coupling the equations of motions for each particle.
        
        Works by Chen, Chacón et al. \cite{Chen_Chacón_Barnes_2011, Chacón_Chen_Barnes_2013, Chen_Chacón_2014, Chen_Chacón_2015} have developed structure-preserving particle pushers for neoclassical transport in the Vlasov equations, derived from Crank--Nicolson integrators. We show these too can can derive from a FET interpretation, similarly offering potential extensions to higher-order-in-time particle pushers. The FET formulation is used also to consider how the stochastic drift terms can be incorporated into the pushers. Stochastic gyrokinetic expansions are also discussed.

        Different options for the numerical implementation of these schemes are considered.

        Due to the efficacy of FET in the development of SP timesteppers for both the fluid and kinetic component, we hope this approach will prove effective in the future for developing SP timesteppers for the full hybrid model. We hope this will give us the opportunity to incorporate previously inaccessible kinetic effects into the highly effective, modern, finite-element MHD models.
    \end{abstract}
    
    
    \newpage
    \tableofcontents
    
    
    \newpage
    \pagenumbering{arabic}
    %\linenumbers\renewcommand\thelinenumber{\color{black!50}\arabic{linenumber}}
            \input{0 - introduction/main.tex}
        \part{Research}
            \input{1 - low-noise PiC models/main.tex}
            \input{2 - kinetic component/main.tex}
            \input{3 - fluid component/main.tex}
            \input{4 - numerical implementation/main.tex}
        \part{Project Overview}
            \input{5 - research plan/main.tex}
            \input{6 - summary/main.tex}
    
    
    %\section{}
    \newpage
    \pagenumbering{gobble}
        \printbibliography


    \newpage
    \pagenumbering{roman}
    \appendix
        \part{Appendices}
            \input{8 - Hilbert complexes/main.tex}
            \input{9 - weak conservation proofs/main.tex}
\end{document}

            \documentclass[12pt, a4paper]{report}

\input{template/main.tex}

\title{\BA{Title in Progress...}}
\author{Boris Andrews}
\affil{Mathematical Institute, University of Oxford}
\date{\today}


\begin{document}
    \pagenumbering{gobble}
    \maketitle
    
    
    \begin{abstract}
        Magnetic confinement reactors---in particular tokamaks---offer one of the most promising options for achieving practical nuclear fusion, with the potential to provide virtually limitless, clean energy. The theoretical and numerical modeling of tokamak plasmas is simultaneously an essential component of effective reactor design, and a great research barrier. Tokamak operational conditions exhibit comparatively low Knudsen numbers. Kinetic effects, including kinetic waves and instabilities, Landau damping, bump-on-tail instabilities and more, are therefore highly influential in tokamak plasma dynamics. Purely fluid models are inherently incapable of capturing these effects, whereas the high dimensionality in purely kinetic models render them practically intractable for most relevant purposes.

        We consider a $\delta\!f$ decomposition model, with a macroscopic fluid background and microscopic kinetic correction, both fully coupled to each other. A similar manner of discretization is proposed to that used in the recent \texttt{STRUPHY} code \cite{Holderied_Possanner_Wang_2021, Holderied_2022, Li_et_al_2023} with a finite-element model for the background and a pseudo-particle/PiC model for the correction.

        The fluid background satisfies the full, non-linear, resistive, compressible, Hall MHD equations. \cite{Laakmann_Hu_Farrell_2022} introduces finite-element(-in-space) implicit timesteppers for the incompressible analogue to this system with structure-preserving (SP) properties in the ideal case, alongside parameter-robust preconditioners. We show that these timesteppers can derive from a finite-element-in-time (FET) (and finite-element-in-space) interpretation. The benefits of this reformulation are discussed, including the derivation of timesteppers that are higher order in time, and the quantifiable dissipative SP properties in the non-ideal, resistive case.
        
        We discuss possible options for extending this FET approach to timesteppers for the compressible case.

        The kinetic corrections satisfy linearized Boltzmann equations. Using a Lénard--Bernstein collision operator, these take Fokker--Planck-like forms \cite{Fokker_1914, Planck_1917} wherein pseudo-particles in the numerical model obey the neoclassical transport equations, with particle-independent Brownian drift terms. This offers a rigorous methodology for incorporating collisions into the particle transport model, without coupling the equations of motions for each particle.
        
        Works by Chen, Chacón et al. \cite{Chen_Chacón_Barnes_2011, Chacón_Chen_Barnes_2013, Chen_Chacón_2014, Chen_Chacón_2015} have developed structure-preserving particle pushers for neoclassical transport in the Vlasov equations, derived from Crank--Nicolson integrators. We show these too can can derive from a FET interpretation, similarly offering potential extensions to higher-order-in-time particle pushers. The FET formulation is used also to consider how the stochastic drift terms can be incorporated into the pushers. Stochastic gyrokinetic expansions are also discussed.

        Different options for the numerical implementation of these schemes are considered.

        Due to the efficacy of FET in the development of SP timesteppers for both the fluid and kinetic component, we hope this approach will prove effective in the future for developing SP timesteppers for the full hybrid model. We hope this will give us the opportunity to incorporate previously inaccessible kinetic effects into the highly effective, modern, finite-element MHD models.
    \end{abstract}
    
    
    \newpage
    \tableofcontents
    
    
    \newpage
    \pagenumbering{arabic}
    %\linenumbers\renewcommand\thelinenumber{\color{black!50}\arabic{linenumber}}
            \input{0 - introduction/main.tex}
        \part{Research}
            \input{1 - low-noise PiC models/main.tex}
            \input{2 - kinetic component/main.tex}
            \input{3 - fluid component/main.tex}
            \input{4 - numerical implementation/main.tex}
        \part{Project Overview}
            \input{5 - research plan/main.tex}
            \input{6 - summary/main.tex}
    
    
    %\section{}
    \newpage
    \pagenumbering{gobble}
        \printbibliography


    \newpage
    \pagenumbering{roman}
    \appendix
        \part{Appendices}
            \input{8 - Hilbert complexes/main.tex}
            \input{9 - weak conservation proofs/main.tex}
\end{document}

        \part{Project Overview}
            \documentclass[12pt, a4paper]{report}

\input{template/main.tex}

\title{\BA{Title in Progress...}}
\author{Boris Andrews}
\affil{Mathematical Institute, University of Oxford}
\date{\today}


\begin{document}
    \pagenumbering{gobble}
    \maketitle
    
    
    \begin{abstract}
        Magnetic confinement reactors---in particular tokamaks---offer one of the most promising options for achieving practical nuclear fusion, with the potential to provide virtually limitless, clean energy. The theoretical and numerical modeling of tokamak plasmas is simultaneously an essential component of effective reactor design, and a great research barrier. Tokamak operational conditions exhibit comparatively low Knudsen numbers. Kinetic effects, including kinetic waves and instabilities, Landau damping, bump-on-tail instabilities and more, are therefore highly influential in tokamak plasma dynamics. Purely fluid models are inherently incapable of capturing these effects, whereas the high dimensionality in purely kinetic models render them practically intractable for most relevant purposes.

        We consider a $\delta\!f$ decomposition model, with a macroscopic fluid background and microscopic kinetic correction, both fully coupled to each other. A similar manner of discretization is proposed to that used in the recent \texttt{STRUPHY} code \cite{Holderied_Possanner_Wang_2021, Holderied_2022, Li_et_al_2023} with a finite-element model for the background and a pseudo-particle/PiC model for the correction.

        The fluid background satisfies the full, non-linear, resistive, compressible, Hall MHD equations. \cite{Laakmann_Hu_Farrell_2022} introduces finite-element(-in-space) implicit timesteppers for the incompressible analogue to this system with structure-preserving (SP) properties in the ideal case, alongside parameter-robust preconditioners. We show that these timesteppers can derive from a finite-element-in-time (FET) (and finite-element-in-space) interpretation. The benefits of this reformulation are discussed, including the derivation of timesteppers that are higher order in time, and the quantifiable dissipative SP properties in the non-ideal, resistive case.
        
        We discuss possible options for extending this FET approach to timesteppers for the compressible case.

        The kinetic corrections satisfy linearized Boltzmann equations. Using a Lénard--Bernstein collision operator, these take Fokker--Planck-like forms \cite{Fokker_1914, Planck_1917} wherein pseudo-particles in the numerical model obey the neoclassical transport equations, with particle-independent Brownian drift terms. This offers a rigorous methodology for incorporating collisions into the particle transport model, without coupling the equations of motions for each particle.
        
        Works by Chen, Chacón et al. \cite{Chen_Chacón_Barnes_2011, Chacón_Chen_Barnes_2013, Chen_Chacón_2014, Chen_Chacón_2015} have developed structure-preserving particle pushers for neoclassical transport in the Vlasov equations, derived from Crank--Nicolson integrators. We show these too can can derive from a FET interpretation, similarly offering potential extensions to higher-order-in-time particle pushers. The FET formulation is used also to consider how the stochastic drift terms can be incorporated into the pushers. Stochastic gyrokinetic expansions are also discussed.

        Different options for the numerical implementation of these schemes are considered.

        Due to the efficacy of FET in the development of SP timesteppers for both the fluid and kinetic component, we hope this approach will prove effective in the future for developing SP timesteppers for the full hybrid model. We hope this will give us the opportunity to incorporate previously inaccessible kinetic effects into the highly effective, modern, finite-element MHD models.
    \end{abstract}
    
    
    \newpage
    \tableofcontents
    
    
    \newpage
    \pagenumbering{arabic}
    %\linenumbers\renewcommand\thelinenumber{\color{black!50}\arabic{linenumber}}
            \input{0 - introduction/main.tex}
        \part{Research}
            \input{1 - low-noise PiC models/main.tex}
            \input{2 - kinetic component/main.tex}
            \input{3 - fluid component/main.tex}
            \input{4 - numerical implementation/main.tex}
        \part{Project Overview}
            \input{5 - research plan/main.tex}
            \input{6 - summary/main.tex}
    
    
    %\section{}
    \newpage
    \pagenumbering{gobble}
        \printbibliography


    \newpage
    \pagenumbering{roman}
    \appendix
        \part{Appendices}
            \input{8 - Hilbert complexes/main.tex}
            \input{9 - weak conservation proofs/main.tex}
\end{document}

            \documentclass[12pt, a4paper]{report}

\input{template/main.tex}

\title{\BA{Title in Progress...}}
\author{Boris Andrews}
\affil{Mathematical Institute, University of Oxford}
\date{\today}


\begin{document}
    \pagenumbering{gobble}
    \maketitle
    
    
    \begin{abstract}
        Magnetic confinement reactors---in particular tokamaks---offer one of the most promising options for achieving practical nuclear fusion, with the potential to provide virtually limitless, clean energy. The theoretical and numerical modeling of tokamak plasmas is simultaneously an essential component of effective reactor design, and a great research barrier. Tokamak operational conditions exhibit comparatively low Knudsen numbers. Kinetic effects, including kinetic waves and instabilities, Landau damping, bump-on-tail instabilities and more, are therefore highly influential in tokamak plasma dynamics. Purely fluid models are inherently incapable of capturing these effects, whereas the high dimensionality in purely kinetic models render them practically intractable for most relevant purposes.

        We consider a $\delta\!f$ decomposition model, with a macroscopic fluid background and microscopic kinetic correction, both fully coupled to each other. A similar manner of discretization is proposed to that used in the recent \texttt{STRUPHY} code \cite{Holderied_Possanner_Wang_2021, Holderied_2022, Li_et_al_2023} with a finite-element model for the background and a pseudo-particle/PiC model for the correction.

        The fluid background satisfies the full, non-linear, resistive, compressible, Hall MHD equations. \cite{Laakmann_Hu_Farrell_2022} introduces finite-element(-in-space) implicit timesteppers for the incompressible analogue to this system with structure-preserving (SP) properties in the ideal case, alongside parameter-robust preconditioners. We show that these timesteppers can derive from a finite-element-in-time (FET) (and finite-element-in-space) interpretation. The benefits of this reformulation are discussed, including the derivation of timesteppers that are higher order in time, and the quantifiable dissipative SP properties in the non-ideal, resistive case.
        
        We discuss possible options for extending this FET approach to timesteppers for the compressible case.

        The kinetic corrections satisfy linearized Boltzmann equations. Using a Lénard--Bernstein collision operator, these take Fokker--Planck-like forms \cite{Fokker_1914, Planck_1917} wherein pseudo-particles in the numerical model obey the neoclassical transport equations, with particle-independent Brownian drift terms. This offers a rigorous methodology for incorporating collisions into the particle transport model, without coupling the equations of motions for each particle.
        
        Works by Chen, Chacón et al. \cite{Chen_Chacón_Barnes_2011, Chacón_Chen_Barnes_2013, Chen_Chacón_2014, Chen_Chacón_2015} have developed structure-preserving particle pushers for neoclassical transport in the Vlasov equations, derived from Crank--Nicolson integrators. We show these too can can derive from a FET interpretation, similarly offering potential extensions to higher-order-in-time particle pushers. The FET formulation is used also to consider how the stochastic drift terms can be incorporated into the pushers. Stochastic gyrokinetic expansions are also discussed.

        Different options for the numerical implementation of these schemes are considered.

        Due to the efficacy of FET in the development of SP timesteppers for both the fluid and kinetic component, we hope this approach will prove effective in the future for developing SP timesteppers for the full hybrid model. We hope this will give us the opportunity to incorporate previously inaccessible kinetic effects into the highly effective, modern, finite-element MHD models.
    \end{abstract}
    
    
    \newpage
    \tableofcontents
    
    
    \newpage
    \pagenumbering{arabic}
    %\linenumbers\renewcommand\thelinenumber{\color{black!50}\arabic{linenumber}}
            \input{0 - introduction/main.tex}
        \part{Research}
            \input{1 - low-noise PiC models/main.tex}
            \input{2 - kinetic component/main.tex}
            \input{3 - fluid component/main.tex}
            \input{4 - numerical implementation/main.tex}
        \part{Project Overview}
            \input{5 - research plan/main.tex}
            \input{6 - summary/main.tex}
    
    
    %\section{}
    \newpage
    \pagenumbering{gobble}
        \printbibliography


    \newpage
    \pagenumbering{roman}
    \appendix
        \part{Appendices}
            \input{8 - Hilbert complexes/main.tex}
            \input{9 - weak conservation proofs/main.tex}
\end{document}

    
    
    %\section{}
    \newpage
    \pagenumbering{gobble}
        \printbibliography


    \newpage
    \pagenumbering{roman}
    \appendix
        \part{Appendices}
            \documentclass[12pt, a4paper]{report}

\input{template/main.tex}

\title{\BA{Title in Progress...}}
\author{Boris Andrews}
\affil{Mathematical Institute, University of Oxford}
\date{\today}


\begin{document}
    \pagenumbering{gobble}
    \maketitle
    
    
    \begin{abstract}
        Magnetic confinement reactors---in particular tokamaks---offer one of the most promising options for achieving practical nuclear fusion, with the potential to provide virtually limitless, clean energy. The theoretical and numerical modeling of tokamak plasmas is simultaneously an essential component of effective reactor design, and a great research barrier. Tokamak operational conditions exhibit comparatively low Knudsen numbers. Kinetic effects, including kinetic waves and instabilities, Landau damping, bump-on-tail instabilities and more, are therefore highly influential in tokamak plasma dynamics. Purely fluid models are inherently incapable of capturing these effects, whereas the high dimensionality in purely kinetic models render them practically intractable for most relevant purposes.

        We consider a $\delta\!f$ decomposition model, with a macroscopic fluid background and microscopic kinetic correction, both fully coupled to each other. A similar manner of discretization is proposed to that used in the recent \texttt{STRUPHY} code \cite{Holderied_Possanner_Wang_2021, Holderied_2022, Li_et_al_2023} with a finite-element model for the background and a pseudo-particle/PiC model for the correction.

        The fluid background satisfies the full, non-linear, resistive, compressible, Hall MHD equations. \cite{Laakmann_Hu_Farrell_2022} introduces finite-element(-in-space) implicit timesteppers for the incompressible analogue to this system with structure-preserving (SP) properties in the ideal case, alongside parameter-robust preconditioners. We show that these timesteppers can derive from a finite-element-in-time (FET) (and finite-element-in-space) interpretation. The benefits of this reformulation are discussed, including the derivation of timesteppers that are higher order in time, and the quantifiable dissipative SP properties in the non-ideal, resistive case.
        
        We discuss possible options for extending this FET approach to timesteppers for the compressible case.

        The kinetic corrections satisfy linearized Boltzmann equations. Using a Lénard--Bernstein collision operator, these take Fokker--Planck-like forms \cite{Fokker_1914, Planck_1917} wherein pseudo-particles in the numerical model obey the neoclassical transport equations, with particle-independent Brownian drift terms. This offers a rigorous methodology for incorporating collisions into the particle transport model, without coupling the equations of motions for each particle.
        
        Works by Chen, Chacón et al. \cite{Chen_Chacón_Barnes_2011, Chacón_Chen_Barnes_2013, Chen_Chacón_2014, Chen_Chacón_2015} have developed structure-preserving particle pushers for neoclassical transport in the Vlasov equations, derived from Crank--Nicolson integrators. We show these too can can derive from a FET interpretation, similarly offering potential extensions to higher-order-in-time particle pushers. The FET formulation is used also to consider how the stochastic drift terms can be incorporated into the pushers. Stochastic gyrokinetic expansions are also discussed.

        Different options for the numerical implementation of these schemes are considered.

        Due to the efficacy of FET in the development of SP timesteppers for both the fluid and kinetic component, we hope this approach will prove effective in the future for developing SP timesteppers for the full hybrid model. We hope this will give us the opportunity to incorporate previously inaccessible kinetic effects into the highly effective, modern, finite-element MHD models.
    \end{abstract}
    
    
    \newpage
    \tableofcontents
    
    
    \newpage
    \pagenumbering{arabic}
    %\linenumbers\renewcommand\thelinenumber{\color{black!50}\arabic{linenumber}}
            \input{0 - introduction/main.tex}
        \part{Research}
            \input{1 - low-noise PiC models/main.tex}
            \input{2 - kinetic component/main.tex}
            \input{3 - fluid component/main.tex}
            \input{4 - numerical implementation/main.tex}
        \part{Project Overview}
            \input{5 - research plan/main.tex}
            \input{6 - summary/main.tex}
    
    
    %\section{}
    \newpage
    \pagenumbering{gobble}
        \printbibliography


    \newpage
    \pagenumbering{roman}
    \appendix
        \part{Appendices}
            \input{8 - Hilbert complexes/main.tex}
            \input{9 - weak conservation proofs/main.tex}
\end{document}

            \documentclass[12pt, a4paper]{report}

\input{template/main.tex}

\title{\BA{Title in Progress...}}
\author{Boris Andrews}
\affil{Mathematical Institute, University of Oxford}
\date{\today}


\begin{document}
    \pagenumbering{gobble}
    \maketitle
    
    
    \begin{abstract}
        Magnetic confinement reactors---in particular tokamaks---offer one of the most promising options for achieving practical nuclear fusion, with the potential to provide virtually limitless, clean energy. The theoretical and numerical modeling of tokamak plasmas is simultaneously an essential component of effective reactor design, and a great research barrier. Tokamak operational conditions exhibit comparatively low Knudsen numbers. Kinetic effects, including kinetic waves and instabilities, Landau damping, bump-on-tail instabilities and more, are therefore highly influential in tokamak plasma dynamics. Purely fluid models are inherently incapable of capturing these effects, whereas the high dimensionality in purely kinetic models render them practically intractable for most relevant purposes.

        We consider a $\delta\!f$ decomposition model, with a macroscopic fluid background and microscopic kinetic correction, both fully coupled to each other. A similar manner of discretization is proposed to that used in the recent \texttt{STRUPHY} code \cite{Holderied_Possanner_Wang_2021, Holderied_2022, Li_et_al_2023} with a finite-element model for the background and a pseudo-particle/PiC model for the correction.

        The fluid background satisfies the full, non-linear, resistive, compressible, Hall MHD equations. \cite{Laakmann_Hu_Farrell_2022} introduces finite-element(-in-space) implicit timesteppers for the incompressible analogue to this system with structure-preserving (SP) properties in the ideal case, alongside parameter-robust preconditioners. We show that these timesteppers can derive from a finite-element-in-time (FET) (and finite-element-in-space) interpretation. The benefits of this reformulation are discussed, including the derivation of timesteppers that are higher order in time, and the quantifiable dissipative SP properties in the non-ideal, resistive case.
        
        We discuss possible options for extending this FET approach to timesteppers for the compressible case.

        The kinetic corrections satisfy linearized Boltzmann equations. Using a Lénard--Bernstein collision operator, these take Fokker--Planck-like forms \cite{Fokker_1914, Planck_1917} wherein pseudo-particles in the numerical model obey the neoclassical transport equations, with particle-independent Brownian drift terms. This offers a rigorous methodology for incorporating collisions into the particle transport model, without coupling the equations of motions for each particle.
        
        Works by Chen, Chacón et al. \cite{Chen_Chacón_Barnes_2011, Chacón_Chen_Barnes_2013, Chen_Chacón_2014, Chen_Chacón_2015} have developed structure-preserving particle pushers for neoclassical transport in the Vlasov equations, derived from Crank--Nicolson integrators. We show these too can can derive from a FET interpretation, similarly offering potential extensions to higher-order-in-time particle pushers. The FET formulation is used also to consider how the stochastic drift terms can be incorporated into the pushers. Stochastic gyrokinetic expansions are also discussed.

        Different options for the numerical implementation of these schemes are considered.

        Due to the efficacy of FET in the development of SP timesteppers for both the fluid and kinetic component, we hope this approach will prove effective in the future for developing SP timesteppers for the full hybrid model. We hope this will give us the opportunity to incorporate previously inaccessible kinetic effects into the highly effective, modern, finite-element MHD models.
    \end{abstract}
    
    
    \newpage
    \tableofcontents
    
    
    \newpage
    \pagenumbering{arabic}
    %\linenumbers\renewcommand\thelinenumber{\color{black!50}\arabic{linenumber}}
            \input{0 - introduction/main.tex}
        \part{Research}
            \input{1 - low-noise PiC models/main.tex}
            \input{2 - kinetic component/main.tex}
            \input{3 - fluid component/main.tex}
            \input{4 - numerical implementation/main.tex}
        \part{Project Overview}
            \input{5 - research plan/main.tex}
            \input{6 - summary/main.tex}
    
    
    %\section{}
    \newpage
    \pagenumbering{gobble}
        \printbibliography


    \newpage
    \pagenumbering{roman}
    \appendix
        \part{Appendices}
            \input{8 - Hilbert complexes/main.tex}
            \input{9 - weak conservation proofs/main.tex}
\end{document}

\end{document}

    
    
    %\section{}
    \newpage
    \pagenumbering{gobble}
        \printbibliography


    \newpage
    \pagenumbering{roman}
    \appendix
        \part{Appendices}
            \documentclass[12pt, a4paper]{report}

\documentclass[12pt, a4paper]{report}

\input{template/main.tex}

\title{\BA{Title in Progress...}}
\author{Boris Andrews}
\affil{Mathematical Institute, University of Oxford}
\date{\today}


\begin{document}
    \pagenumbering{gobble}
    \maketitle
    
    
    \begin{abstract}
        Magnetic confinement reactors---in particular tokamaks---offer one of the most promising options for achieving practical nuclear fusion, with the potential to provide virtually limitless, clean energy. The theoretical and numerical modeling of tokamak plasmas is simultaneously an essential component of effective reactor design, and a great research barrier. Tokamak operational conditions exhibit comparatively low Knudsen numbers. Kinetic effects, including kinetic waves and instabilities, Landau damping, bump-on-tail instabilities and more, are therefore highly influential in tokamak plasma dynamics. Purely fluid models are inherently incapable of capturing these effects, whereas the high dimensionality in purely kinetic models render them practically intractable for most relevant purposes.

        We consider a $\delta\!f$ decomposition model, with a macroscopic fluid background and microscopic kinetic correction, both fully coupled to each other. A similar manner of discretization is proposed to that used in the recent \texttt{STRUPHY} code \cite{Holderied_Possanner_Wang_2021, Holderied_2022, Li_et_al_2023} with a finite-element model for the background and a pseudo-particle/PiC model for the correction.

        The fluid background satisfies the full, non-linear, resistive, compressible, Hall MHD equations. \cite{Laakmann_Hu_Farrell_2022} introduces finite-element(-in-space) implicit timesteppers for the incompressible analogue to this system with structure-preserving (SP) properties in the ideal case, alongside parameter-robust preconditioners. We show that these timesteppers can derive from a finite-element-in-time (FET) (and finite-element-in-space) interpretation. The benefits of this reformulation are discussed, including the derivation of timesteppers that are higher order in time, and the quantifiable dissipative SP properties in the non-ideal, resistive case.
        
        We discuss possible options for extending this FET approach to timesteppers for the compressible case.

        The kinetic corrections satisfy linearized Boltzmann equations. Using a Lénard--Bernstein collision operator, these take Fokker--Planck-like forms \cite{Fokker_1914, Planck_1917} wherein pseudo-particles in the numerical model obey the neoclassical transport equations, with particle-independent Brownian drift terms. This offers a rigorous methodology for incorporating collisions into the particle transport model, without coupling the equations of motions for each particle.
        
        Works by Chen, Chacón et al. \cite{Chen_Chacón_Barnes_2011, Chacón_Chen_Barnes_2013, Chen_Chacón_2014, Chen_Chacón_2015} have developed structure-preserving particle pushers for neoclassical transport in the Vlasov equations, derived from Crank--Nicolson integrators. We show these too can can derive from a FET interpretation, similarly offering potential extensions to higher-order-in-time particle pushers. The FET formulation is used also to consider how the stochastic drift terms can be incorporated into the pushers. Stochastic gyrokinetic expansions are also discussed.

        Different options for the numerical implementation of these schemes are considered.

        Due to the efficacy of FET in the development of SP timesteppers for both the fluid and kinetic component, we hope this approach will prove effective in the future for developing SP timesteppers for the full hybrid model. We hope this will give us the opportunity to incorporate previously inaccessible kinetic effects into the highly effective, modern, finite-element MHD models.
    \end{abstract}
    
    
    \newpage
    \tableofcontents
    
    
    \newpage
    \pagenumbering{arabic}
    %\linenumbers\renewcommand\thelinenumber{\color{black!50}\arabic{linenumber}}
            \input{0 - introduction/main.tex}
        \part{Research}
            \input{1 - low-noise PiC models/main.tex}
            \input{2 - kinetic component/main.tex}
            \input{3 - fluid component/main.tex}
            \input{4 - numerical implementation/main.tex}
        \part{Project Overview}
            \input{5 - research plan/main.tex}
            \input{6 - summary/main.tex}
    
    
    %\section{}
    \newpage
    \pagenumbering{gobble}
        \printbibliography


    \newpage
    \pagenumbering{roman}
    \appendix
        \part{Appendices}
            \input{8 - Hilbert complexes/main.tex}
            \input{9 - weak conservation proofs/main.tex}
\end{document}


\title{\BA{Title in Progress...}}
\author{Boris Andrews}
\affil{Mathematical Institute, University of Oxford}
\date{\today}


\begin{document}
    \pagenumbering{gobble}
    \maketitle
    
    
    \begin{abstract}
        Magnetic confinement reactors---in particular tokamaks---offer one of the most promising options for achieving practical nuclear fusion, with the potential to provide virtually limitless, clean energy. The theoretical and numerical modeling of tokamak plasmas is simultaneously an essential component of effective reactor design, and a great research barrier. Tokamak operational conditions exhibit comparatively low Knudsen numbers. Kinetic effects, including kinetic waves and instabilities, Landau damping, bump-on-tail instabilities and more, are therefore highly influential in tokamak plasma dynamics. Purely fluid models are inherently incapable of capturing these effects, whereas the high dimensionality in purely kinetic models render them practically intractable for most relevant purposes.

        We consider a $\delta\!f$ decomposition model, with a macroscopic fluid background and microscopic kinetic correction, both fully coupled to each other. A similar manner of discretization is proposed to that used in the recent \texttt{STRUPHY} code \cite{Holderied_Possanner_Wang_2021, Holderied_2022, Li_et_al_2023} with a finite-element model for the background and a pseudo-particle/PiC model for the correction.

        The fluid background satisfies the full, non-linear, resistive, compressible, Hall MHD equations. \cite{Laakmann_Hu_Farrell_2022} introduces finite-element(-in-space) implicit timesteppers for the incompressible analogue to this system with structure-preserving (SP) properties in the ideal case, alongside parameter-robust preconditioners. We show that these timesteppers can derive from a finite-element-in-time (FET) (and finite-element-in-space) interpretation. The benefits of this reformulation are discussed, including the derivation of timesteppers that are higher order in time, and the quantifiable dissipative SP properties in the non-ideal, resistive case.
        
        We discuss possible options for extending this FET approach to timesteppers for the compressible case.

        The kinetic corrections satisfy linearized Boltzmann equations. Using a Lénard--Bernstein collision operator, these take Fokker--Planck-like forms \cite{Fokker_1914, Planck_1917} wherein pseudo-particles in the numerical model obey the neoclassical transport equations, with particle-independent Brownian drift terms. This offers a rigorous methodology for incorporating collisions into the particle transport model, without coupling the equations of motions for each particle.
        
        Works by Chen, Chacón et al. \cite{Chen_Chacón_Barnes_2011, Chacón_Chen_Barnes_2013, Chen_Chacón_2014, Chen_Chacón_2015} have developed structure-preserving particle pushers for neoclassical transport in the Vlasov equations, derived from Crank--Nicolson integrators. We show these too can can derive from a FET interpretation, similarly offering potential extensions to higher-order-in-time particle pushers. The FET formulation is used also to consider how the stochastic drift terms can be incorporated into the pushers. Stochastic gyrokinetic expansions are also discussed.

        Different options for the numerical implementation of these schemes are considered.

        Due to the efficacy of FET in the development of SP timesteppers for both the fluid and kinetic component, we hope this approach will prove effective in the future for developing SP timesteppers for the full hybrid model. We hope this will give us the opportunity to incorporate previously inaccessible kinetic effects into the highly effective, modern, finite-element MHD models.
    \end{abstract}
    
    
    \newpage
    \tableofcontents
    
    
    \newpage
    \pagenumbering{arabic}
    %\linenumbers\renewcommand\thelinenumber{\color{black!50}\arabic{linenumber}}
            \documentclass[12pt, a4paper]{report}

\input{template/main.tex}

\title{\BA{Title in Progress...}}
\author{Boris Andrews}
\affil{Mathematical Institute, University of Oxford}
\date{\today}


\begin{document}
    \pagenumbering{gobble}
    \maketitle
    
    
    \begin{abstract}
        Magnetic confinement reactors---in particular tokamaks---offer one of the most promising options for achieving practical nuclear fusion, with the potential to provide virtually limitless, clean energy. The theoretical and numerical modeling of tokamak plasmas is simultaneously an essential component of effective reactor design, and a great research barrier. Tokamak operational conditions exhibit comparatively low Knudsen numbers. Kinetic effects, including kinetic waves and instabilities, Landau damping, bump-on-tail instabilities and more, are therefore highly influential in tokamak plasma dynamics. Purely fluid models are inherently incapable of capturing these effects, whereas the high dimensionality in purely kinetic models render them practically intractable for most relevant purposes.

        We consider a $\delta\!f$ decomposition model, with a macroscopic fluid background and microscopic kinetic correction, both fully coupled to each other. A similar manner of discretization is proposed to that used in the recent \texttt{STRUPHY} code \cite{Holderied_Possanner_Wang_2021, Holderied_2022, Li_et_al_2023} with a finite-element model for the background and a pseudo-particle/PiC model for the correction.

        The fluid background satisfies the full, non-linear, resistive, compressible, Hall MHD equations. \cite{Laakmann_Hu_Farrell_2022} introduces finite-element(-in-space) implicit timesteppers for the incompressible analogue to this system with structure-preserving (SP) properties in the ideal case, alongside parameter-robust preconditioners. We show that these timesteppers can derive from a finite-element-in-time (FET) (and finite-element-in-space) interpretation. The benefits of this reformulation are discussed, including the derivation of timesteppers that are higher order in time, and the quantifiable dissipative SP properties in the non-ideal, resistive case.
        
        We discuss possible options for extending this FET approach to timesteppers for the compressible case.

        The kinetic corrections satisfy linearized Boltzmann equations. Using a Lénard--Bernstein collision operator, these take Fokker--Planck-like forms \cite{Fokker_1914, Planck_1917} wherein pseudo-particles in the numerical model obey the neoclassical transport equations, with particle-independent Brownian drift terms. This offers a rigorous methodology for incorporating collisions into the particle transport model, without coupling the equations of motions for each particle.
        
        Works by Chen, Chacón et al. \cite{Chen_Chacón_Barnes_2011, Chacón_Chen_Barnes_2013, Chen_Chacón_2014, Chen_Chacón_2015} have developed structure-preserving particle pushers for neoclassical transport in the Vlasov equations, derived from Crank--Nicolson integrators. We show these too can can derive from a FET interpretation, similarly offering potential extensions to higher-order-in-time particle pushers. The FET formulation is used also to consider how the stochastic drift terms can be incorporated into the pushers. Stochastic gyrokinetic expansions are also discussed.

        Different options for the numerical implementation of these schemes are considered.

        Due to the efficacy of FET in the development of SP timesteppers for both the fluid and kinetic component, we hope this approach will prove effective in the future for developing SP timesteppers for the full hybrid model. We hope this will give us the opportunity to incorporate previously inaccessible kinetic effects into the highly effective, modern, finite-element MHD models.
    \end{abstract}
    
    
    \newpage
    \tableofcontents
    
    
    \newpage
    \pagenumbering{arabic}
    %\linenumbers\renewcommand\thelinenumber{\color{black!50}\arabic{linenumber}}
            \input{0 - introduction/main.tex}
        \part{Research}
            \input{1 - low-noise PiC models/main.tex}
            \input{2 - kinetic component/main.tex}
            \input{3 - fluid component/main.tex}
            \input{4 - numerical implementation/main.tex}
        \part{Project Overview}
            \input{5 - research plan/main.tex}
            \input{6 - summary/main.tex}
    
    
    %\section{}
    \newpage
    \pagenumbering{gobble}
        \printbibliography


    \newpage
    \pagenumbering{roman}
    \appendix
        \part{Appendices}
            \input{8 - Hilbert complexes/main.tex}
            \input{9 - weak conservation proofs/main.tex}
\end{document}

        \part{Research}
            \documentclass[12pt, a4paper]{report}

\input{template/main.tex}

\title{\BA{Title in Progress...}}
\author{Boris Andrews}
\affil{Mathematical Institute, University of Oxford}
\date{\today}


\begin{document}
    \pagenumbering{gobble}
    \maketitle
    
    
    \begin{abstract}
        Magnetic confinement reactors---in particular tokamaks---offer one of the most promising options for achieving practical nuclear fusion, with the potential to provide virtually limitless, clean energy. The theoretical and numerical modeling of tokamak plasmas is simultaneously an essential component of effective reactor design, and a great research barrier. Tokamak operational conditions exhibit comparatively low Knudsen numbers. Kinetic effects, including kinetic waves and instabilities, Landau damping, bump-on-tail instabilities and more, are therefore highly influential in tokamak plasma dynamics. Purely fluid models are inherently incapable of capturing these effects, whereas the high dimensionality in purely kinetic models render them practically intractable for most relevant purposes.

        We consider a $\delta\!f$ decomposition model, with a macroscopic fluid background and microscopic kinetic correction, both fully coupled to each other. A similar manner of discretization is proposed to that used in the recent \texttt{STRUPHY} code \cite{Holderied_Possanner_Wang_2021, Holderied_2022, Li_et_al_2023} with a finite-element model for the background and a pseudo-particle/PiC model for the correction.

        The fluid background satisfies the full, non-linear, resistive, compressible, Hall MHD equations. \cite{Laakmann_Hu_Farrell_2022} introduces finite-element(-in-space) implicit timesteppers for the incompressible analogue to this system with structure-preserving (SP) properties in the ideal case, alongside parameter-robust preconditioners. We show that these timesteppers can derive from a finite-element-in-time (FET) (and finite-element-in-space) interpretation. The benefits of this reformulation are discussed, including the derivation of timesteppers that are higher order in time, and the quantifiable dissipative SP properties in the non-ideal, resistive case.
        
        We discuss possible options for extending this FET approach to timesteppers for the compressible case.

        The kinetic corrections satisfy linearized Boltzmann equations. Using a Lénard--Bernstein collision operator, these take Fokker--Planck-like forms \cite{Fokker_1914, Planck_1917} wherein pseudo-particles in the numerical model obey the neoclassical transport equations, with particle-independent Brownian drift terms. This offers a rigorous methodology for incorporating collisions into the particle transport model, without coupling the equations of motions for each particle.
        
        Works by Chen, Chacón et al. \cite{Chen_Chacón_Barnes_2011, Chacón_Chen_Barnes_2013, Chen_Chacón_2014, Chen_Chacón_2015} have developed structure-preserving particle pushers for neoclassical transport in the Vlasov equations, derived from Crank--Nicolson integrators. We show these too can can derive from a FET interpretation, similarly offering potential extensions to higher-order-in-time particle pushers. The FET formulation is used also to consider how the stochastic drift terms can be incorporated into the pushers. Stochastic gyrokinetic expansions are also discussed.

        Different options for the numerical implementation of these schemes are considered.

        Due to the efficacy of FET in the development of SP timesteppers for both the fluid and kinetic component, we hope this approach will prove effective in the future for developing SP timesteppers for the full hybrid model. We hope this will give us the opportunity to incorporate previously inaccessible kinetic effects into the highly effective, modern, finite-element MHD models.
    \end{abstract}
    
    
    \newpage
    \tableofcontents
    
    
    \newpage
    \pagenumbering{arabic}
    %\linenumbers\renewcommand\thelinenumber{\color{black!50}\arabic{linenumber}}
            \input{0 - introduction/main.tex}
        \part{Research}
            \input{1 - low-noise PiC models/main.tex}
            \input{2 - kinetic component/main.tex}
            \input{3 - fluid component/main.tex}
            \input{4 - numerical implementation/main.tex}
        \part{Project Overview}
            \input{5 - research plan/main.tex}
            \input{6 - summary/main.tex}
    
    
    %\section{}
    \newpage
    \pagenumbering{gobble}
        \printbibliography


    \newpage
    \pagenumbering{roman}
    \appendix
        \part{Appendices}
            \input{8 - Hilbert complexes/main.tex}
            \input{9 - weak conservation proofs/main.tex}
\end{document}

            \documentclass[12pt, a4paper]{report}

\input{template/main.tex}

\title{\BA{Title in Progress...}}
\author{Boris Andrews}
\affil{Mathematical Institute, University of Oxford}
\date{\today}


\begin{document}
    \pagenumbering{gobble}
    \maketitle
    
    
    \begin{abstract}
        Magnetic confinement reactors---in particular tokamaks---offer one of the most promising options for achieving practical nuclear fusion, with the potential to provide virtually limitless, clean energy. The theoretical and numerical modeling of tokamak plasmas is simultaneously an essential component of effective reactor design, and a great research barrier. Tokamak operational conditions exhibit comparatively low Knudsen numbers. Kinetic effects, including kinetic waves and instabilities, Landau damping, bump-on-tail instabilities and more, are therefore highly influential in tokamak plasma dynamics. Purely fluid models are inherently incapable of capturing these effects, whereas the high dimensionality in purely kinetic models render them practically intractable for most relevant purposes.

        We consider a $\delta\!f$ decomposition model, with a macroscopic fluid background and microscopic kinetic correction, both fully coupled to each other. A similar manner of discretization is proposed to that used in the recent \texttt{STRUPHY} code \cite{Holderied_Possanner_Wang_2021, Holderied_2022, Li_et_al_2023} with a finite-element model for the background and a pseudo-particle/PiC model for the correction.

        The fluid background satisfies the full, non-linear, resistive, compressible, Hall MHD equations. \cite{Laakmann_Hu_Farrell_2022} introduces finite-element(-in-space) implicit timesteppers for the incompressible analogue to this system with structure-preserving (SP) properties in the ideal case, alongside parameter-robust preconditioners. We show that these timesteppers can derive from a finite-element-in-time (FET) (and finite-element-in-space) interpretation. The benefits of this reformulation are discussed, including the derivation of timesteppers that are higher order in time, and the quantifiable dissipative SP properties in the non-ideal, resistive case.
        
        We discuss possible options for extending this FET approach to timesteppers for the compressible case.

        The kinetic corrections satisfy linearized Boltzmann equations. Using a Lénard--Bernstein collision operator, these take Fokker--Planck-like forms \cite{Fokker_1914, Planck_1917} wherein pseudo-particles in the numerical model obey the neoclassical transport equations, with particle-independent Brownian drift terms. This offers a rigorous methodology for incorporating collisions into the particle transport model, without coupling the equations of motions for each particle.
        
        Works by Chen, Chacón et al. \cite{Chen_Chacón_Barnes_2011, Chacón_Chen_Barnes_2013, Chen_Chacón_2014, Chen_Chacón_2015} have developed structure-preserving particle pushers for neoclassical transport in the Vlasov equations, derived from Crank--Nicolson integrators. We show these too can can derive from a FET interpretation, similarly offering potential extensions to higher-order-in-time particle pushers. The FET formulation is used also to consider how the stochastic drift terms can be incorporated into the pushers. Stochastic gyrokinetic expansions are also discussed.

        Different options for the numerical implementation of these schemes are considered.

        Due to the efficacy of FET in the development of SP timesteppers for both the fluid and kinetic component, we hope this approach will prove effective in the future for developing SP timesteppers for the full hybrid model. We hope this will give us the opportunity to incorporate previously inaccessible kinetic effects into the highly effective, modern, finite-element MHD models.
    \end{abstract}
    
    
    \newpage
    \tableofcontents
    
    
    \newpage
    \pagenumbering{arabic}
    %\linenumbers\renewcommand\thelinenumber{\color{black!50}\arabic{linenumber}}
            \input{0 - introduction/main.tex}
        \part{Research}
            \input{1 - low-noise PiC models/main.tex}
            \input{2 - kinetic component/main.tex}
            \input{3 - fluid component/main.tex}
            \input{4 - numerical implementation/main.tex}
        \part{Project Overview}
            \input{5 - research plan/main.tex}
            \input{6 - summary/main.tex}
    
    
    %\section{}
    \newpage
    \pagenumbering{gobble}
        \printbibliography


    \newpage
    \pagenumbering{roman}
    \appendix
        \part{Appendices}
            \input{8 - Hilbert complexes/main.tex}
            \input{9 - weak conservation proofs/main.tex}
\end{document}

            \documentclass[12pt, a4paper]{report}

\input{template/main.tex}

\title{\BA{Title in Progress...}}
\author{Boris Andrews}
\affil{Mathematical Institute, University of Oxford}
\date{\today}


\begin{document}
    \pagenumbering{gobble}
    \maketitle
    
    
    \begin{abstract}
        Magnetic confinement reactors---in particular tokamaks---offer one of the most promising options for achieving practical nuclear fusion, with the potential to provide virtually limitless, clean energy. The theoretical and numerical modeling of tokamak plasmas is simultaneously an essential component of effective reactor design, and a great research barrier. Tokamak operational conditions exhibit comparatively low Knudsen numbers. Kinetic effects, including kinetic waves and instabilities, Landau damping, bump-on-tail instabilities and more, are therefore highly influential in tokamak plasma dynamics. Purely fluid models are inherently incapable of capturing these effects, whereas the high dimensionality in purely kinetic models render them practically intractable for most relevant purposes.

        We consider a $\delta\!f$ decomposition model, with a macroscopic fluid background and microscopic kinetic correction, both fully coupled to each other. A similar manner of discretization is proposed to that used in the recent \texttt{STRUPHY} code \cite{Holderied_Possanner_Wang_2021, Holderied_2022, Li_et_al_2023} with a finite-element model for the background and a pseudo-particle/PiC model for the correction.

        The fluid background satisfies the full, non-linear, resistive, compressible, Hall MHD equations. \cite{Laakmann_Hu_Farrell_2022} introduces finite-element(-in-space) implicit timesteppers for the incompressible analogue to this system with structure-preserving (SP) properties in the ideal case, alongside parameter-robust preconditioners. We show that these timesteppers can derive from a finite-element-in-time (FET) (and finite-element-in-space) interpretation. The benefits of this reformulation are discussed, including the derivation of timesteppers that are higher order in time, and the quantifiable dissipative SP properties in the non-ideal, resistive case.
        
        We discuss possible options for extending this FET approach to timesteppers for the compressible case.

        The kinetic corrections satisfy linearized Boltzmann equations. Using a Lénard--Bernstein collision operator, these take Fokker--Planck-like forms \cite{Fokker_1914, Planck_1917} wherein pseudo-particles in the numerical model obey the neoclassical transport equations, with particle-independent Brownian drift terms. This offers a rigorous methodology for incorporating collisions into the particle transport model, without coupling the equations of motions for each particle.
        
        Works by Chen, Chacón et al. \cite{Chen_Chacón_Barnes_2011, Chacón_Chen_Barnes_2013, Chen_Chacón_2014, Chen_Chacón_2015} have developed structure-preserving particle pushers for neoclassical transport in the Vlasov equations, derived from Crank--Nicolson integrators. We show these too can can derive from a FET interpretation, similarly offering potential extensions to higher-order-in-time particle pushers. The FET formulation is used also to consider how the stochastic drift terms can be incorporated into the pushers. Stochastic gyrokinetic expansions are also discussed.

        Different options for the numerical implementation of these schemes are considered.

        Due to the efficacy of FET in the development of SP timesteppers for both the fluid and kinetic component, we hope this approach will prove effective in the future for developing SP timesteppers for the full hybrid model. We hope this will give us the opportunity to incorporate previously inaccessible kinetic effects into the highly effective, modern, finite-element MHD models.
    \end{abstract}
    
    
    \newpage
    \tableofcontents
    
    
    \newpage
    \pagenumbering{arabic}
    %\linenumbers\renewcommand\thelinenumber{\color{black!50}\arabic{linenumber}}
            \input{0 - introduction/main.tex}
        \part{Research}
            \input{1 - low-noise PiC models/main.tex}
            \input{2 - kinetic component/main.tex}
            \input{3 - fluid component/main.tex}
            \input{4 - numerical implementation/main.tex}
        \part{Project Overview}
            \input{5 - research plan/main.tex}
            \input{6 - summary/main.tex}
    
    
    %\section{}
    \newpage
    \pagenumbering{gobble}
        \printbibliography


    \newpage
    \pagenumbering{roman}
    \appendix
        \part{Appendices}
            \input{8 - Hilbert complexes/main.tex}
            \input{9 - weak conservation proofs/main.tex}
\end{document}

            \documentclass[12pt, a4paper]{report}

\input{template/main.tex}

\title{\BA{Title in Progress...}}
\author{Boris Andrews}
\affil{Mathematical Institute, University of Oxford}
\date{\today}


\begin{document}
    \pagenumbering{gobble}
    \maketitle
    
    
    \begin{abstract}
        Magnetic confinement reactors---in particular tokamaks---offer one of the most promising options for achieving practical nuclear fusion, with the potential to provide virtually limitless, clean energy. The theoretical and numerical modeling of tokamak plasmas is simultaneously an essential component of effective reactor design, and a great research barrier. Tokamak operational conditions exhibit comparatively low Knudsen numbers. Kinetic effects, including kinetic waves and instabilities, Landau damping, bump-on-tail instabilities and more, are therefore highly influential in tokamak plasma dynamics. Purely fluid models are inherently incapable of capturing these effects, whereas the high dimensionality in purely kinetic models render them practically intractable for most relevant purposes.

        We consider a $\delta\!f$ decomposition model, with a macroscopic fluid background and microscopic kinetic correction, both fully coupled to each other. A similar manner of discretization is proposed to that used in the recent \texttt{STRUPHY} code \cite{Holderied_Possanner_Wang_2021, Holderied_2022, Li_et_al_2023} with a finite-element model for the background and a pseudo-particle/PiC model for the correction.

        The fluid background satisfies the full, non-linear, resistive, compressible, Hall MHD equations. \cite{Laakmann_Hu_Farrell_2022} introduces finite-element(-in-space) implicit timesteppers for the incompressible analogue to this system with structure-preserving (SP) properties in the ideal case, alongside parameter-robust preconditioners. We show that these timesteppers can derive from a finite-element-in-time (FET) (and finite-element-in-space) interpretation. The benefits of this reformulation are discussed, including the derivation of timesteppers that are higher order in time, and the quantifiable dissipative SP properties in the non-ideal, resistive case.
        
        We discuss possible options for extending this FET approach to timesteppers for the compressible case.

        The kinetic corrections satisfy linearized Boltzmann equations. Using a Lénard--Bernstein collision operator, these take Fokker--Planck-like forms \cite{Fokker_1914, Planck_1917} wherein pseudo-particles in the numerical model obey the neoclassical transport equations, with particle-independent Brownian drift terms. This offers a rigorous methodology for incorporating collisions into the particle transport model, without coupling the equations of motions for each particle.
        
        Works by Chen, Chacón et al. \cite{Chen_Chacón_Barnes_2011, Chacón_Chen_Barnes_2013, Chen_Chacón_2014, Chen_Chacón_2015} have developed structure-preserving particle pushers for neoclassical transport in the Vlasov equations, derived from Crank--Nicolson integrators. We show these too can can derive from a FET interpretation, similarly offering potential extensions to higher-order-in-time particle pushers. The FET formulation is used also to consider how the stochastic drift terms can be incorporated into the pushers. Stochastic gyrokinetic expansions are also discussed.

        Different options for the numerical implementation of these schemes are considered.

        Due to the efficacy of FET in the development of SP timesteppers for both the fluid and kinetic component, we hope this approach will prove effective in the future for developing SP timesteppers for the full hybrid model. We hope this will give us the opportunity to incorporate previously inaccessible kinetic effects into the highly effective, modern, finite-element MHD models.
    \end{abstract}
    
    
    \newpage
    \tableofcontents
    
    
    \newpage
    \pagenumbering{arabic}
    %\linenumbers\renewcommand\thelinenumber{\color{black!50}\arabic{linenumber}}
            \input{0 - introduction/main.tex}
        \part{Research}
            \input{1 - low-noise PiC models/main.tex}
            \input{2 - kinetic component/main.tex}
            \input{3 - fluid component/main.tex}
            \input{4 - numerical implementation/main.tex}
        \part{Project Overview}
            \input{5 - research plan/main.tex}
            \input{6 - summary/main.tex}
    
    
    %\section{}
    \newpage
    \pagenumbering{gobble}
        \printbibliography


    \newpage
    \pagenumbering{roman}
    \appendix
        \part{Appendices}
            \input{8 - Hilbert complexes/main.tex}
            \input{9 - weak conservation proofs/main.tex}
\end{document}

        \part{Project Overview}
            \documentclass[12pt, a4paper]{report}

\input{template/main.tex}

\title{\BA{Title in Progress...}}
\author{Boris Andrews}
\affil{Mathematical Institute, University of Oxford}
\date{\today}


\begin{document}
    \pagenumbering{gobble}
    \maketitle
    
    
    \begin{abstract}
        Magnetic confinement reactors---in particular tokamaks---offer one of the most promising options for achieving practical nuclear fusion, with the potential to provide virtually limitless, clean energy. The theoretical and numerical modeling of tokamak plasmas is simultaneously an essential component of effective reactor design, and a great research barrier. Tokamak operational conditions exhibit comparatively low Knudsen numbers. Kinetic effects, including kinetic waves and instabilities, Landau damping, bump-on-tail instabilities and more, are therefore highly influential in tokamak plasma dynamics. Purely fluid models are inherently incapable of capturing these effects, whereas the high dimensionality in purely kinetic models render them practically intractable for most relevant purposes.

        We consider a $\delta\!f$ decomposition model, with a macroscopic fluid background and microscopic kinetic correction, both fully coupled to each other. A similar manner of discretization is proposed to that used in the recent \texttt{STRUPHY} code \cite{Holderied_Possanner_Wang_2021, Holderied_2022, Li_et_al_2023} with a finite-element model for the background and a pseudo-particle/PiC model for the correction.

        The fluid background satisfies the full, non-linear, resistive, compressible, Hall MHD equations. \cite{Laakmann_Hu_Farrell_2022} introduces finite-element(-in-space) implicit timesteppers for the incompressible analogue to this system with structure-preserving (SP) properties in the ideal case, alongside parameter-robust preconditioners. We show that these timesteppers can derive from a finite-element-in-time (FET) (and finite-element-in-space) interpretation. The benefits of this reformulation are discussed, including the derivation of timesteppers that are higher order in time, and the quantifiable dissipative SP properties in the non-ideal, resistive case.
        
        We discuss possible options for extending this FET approach to timesteppers for the compressible case.

        The kinetic corrections satisfy linearized Boltzmann equations. Using a Lénard--Bernstein collision operator, these take Fokker--Planck-like forms \cite{Fokker_1914, Planck_1917} wherein pseudo-particles in the numerical model obey the neoclassical transport equations, with particle-independent Brownian drift terms. This offers a rigorous methodology for incorporating collisions into the particle transport model, without coupling the equations of motions for each particle.
        
        Works by Chen, Chacón et al. \cite{Chen_Chacón_Barnes_2011, Chacón_Chen_Barnes_2013, Chen_Chacón_2014, Chen_Chacón_2015} have developed structure-preserving particle pushers for neoclassical transport in the Vlasov equations, derived from Crank--Nicolson integrators. We show these too can can derive from a FET interpretation, similarly offering potential extensions to higher-order-in-time particle pushers. The FET formulation is used also to consider how the stochastic drift terms can be incorporated into the pushers. Stochastic gyrokinetic expansions are also discussed.

        Different options for the numerical implementation of these schemes are considered.

        Due to the efficacy of FET in the development of SP timesteppers for both the fluid and kinetic component, we hope this approach will prove effective in the future for developing SP timesteppers for the full hybrid model. We hope this will give us the opportunity to incorporate previously inaccessible kinetic effects into the highly effective, modern, finite-element MHD models.
    \end{abstract}
    
    
    \newpage
    \tableofcontents
    
    
    \newpage
    \pagenumbering{arabic}
    %\linenumbers\renewcommand\thelinenumber{\color{black!50}\arabic{linenumber}}
            \input{0 - introduction/main.tex}
        \part{Research}
            \input{1 - low-noise PiC models/main.tex}
            \input{2 - kinetic component/main.tex}
            \input{3 - fluid component/main.tex}
            \input{4 - numerical implementation/main.tex}
        \part{Project Overview}
            \input{5 - research plan/main.tex}
            \input{6 - summary/main.tex}
    
    
    %\section{}
    \newpage
    \pagenumbering{gobble}
        \printbibliography


    \newpage
    \pagenumbering{roman}
    \appendix
        \part{Appendices}
            \input{8 - Hilbert complexes/main.tex}
            \input{9 - weak conservation proofs/main.tex}
\end{document}

            \documentclass[12pt, a4paper]{report}

\input{template/main.tex}

\title{\BA{Title in Progress...}}
\author{Boris Andrews}
\affil{Mathematical Institute, University of Oxford}
\date{\today}


\begin{document}
    \pagenumbering{gobble}
    \maketitle
    
    
    \begin{abstract}
        Magnetic confinement reactors---in particular tokamaks---offer one of the most promising options for achieving practical nuclear fusion, with the potential to provide virtually limitless, clean energy. The theoretical and numerical modeling of tokamak plasmas is simultaneously an essential component of effective reactor design, and a great research barrier. Tokamak operational conditions exhibit comparatively low Knudsen numbers. Kinetic effects, including kinetic waves and instabilities, Landau damping, bump-on-tail instabilities and more, are therefore highly influential in tokamak plasma dynamics. Purely fluid models are inherently incapable of capturing these effects, whereas the high dimensionality in purely kinetic models render them practically intractable for most relevant purposes.

        We consider a $\delta\!f$ decomposition model, with a macroscopic fluid background and microscopic kinetic correction, both fully coupled to each other. A similar manner of discretization is proposed to that used in the recent \texttt{STRUPHY} code \cite{Holderied_Possanner_Wang_2021, Holderied_2022, Li_et_al_2023} with a finite-element model for the background and a pseudo-particle/PiC model for the correction.

        The fluid background satisfies the full, non-linear, resistive, compressible, Hall MHD equations. \cite{Laakmann_Hu_Farrell_2022} introduces finite-element(-in-space) implicit timesteppers for the incompressible analogue to this system with structure-preserving (SP) properties in the ideal case, alongside parameter-robust preconditioners. We show that these timesteppers can derive from a finite-element-in-time (FET) (and finite-element-in-space) interpretation. The benefits of this reformulation are discussed, including the derivation of timesteppers that are higher order in time, and the quantifiable dissipative SP properties in the non-ideal, resistive case.
        
        We discuss possible options for extending this FET approach to timesteppers for the compressible case.

        The kinetic corrections satisfy linearized Boltzmann equations. Using a Lénard--Bernstein collision operator, these take Fokker--Planck-like forms \cite{Fokker_1914, Planck_1917} wherein pseudo-particles in the numerical model obey the neoclassical transport equations, with particle-independent Brownian drift terms. This offers a rigorous methodology for incorporating collisions into the particle transport model, without coupling the equations of motions for each particle.
        
        Works by Chen, Chacón et al. \cite{Chen_Chacón_Barnes_2011, Chacón_Chen_Barnes_2013, Chen_Chacón_2014, Chen_Chacón_2015} have developed structure-preserving particle pushers for neoclassical transport in the Vlasov equations, derived from Crank--Nicolson integrators. We show these too can can derive from a FET interpretation, similarly offering potential extensions to higher-order-in-time particle pushers. The FET formulation is used also to consider how the stochastic drift terms can be incorporated into the pushers. Stochastic gyrokinetic expansions are also discussed.

        Different options for the numerical implementation of these schemes are considered.

        Due to the efficacy of FET in the development of SP timesteppers for both the fluid and kinetic component, we hope this approach will prove effective in the future for developing SP timesteppers for the full hybrid model. We hope this will give us the opportunity to incorporate previously inaccessible kinetic effects into the highly effective, modern, finite-element MHD models.
    \end{abstract}
    
    
    \newpage
    \tableofcontents
    
    
    \newpage
    \pagenumbering{arabic}
    %\linenumbers\renewcommand\thelinenumber{\color{black!50}\arabic{linenumber}}
            \input{0 - introduction/main.tex}
        \part{Research}
            \input{1 - low-noise PiC models/main.tex}
            \input{2 - kinetic component/main.tex}
            \input{3 - fluid component/main.tex}
            \input{4 - numerical implementation/main.tex}
        \part{Project Overview}
            \input{5 - research plan/main.tex}
            \input{6 - summary/main.tex}
    
    
    %\section{}
    \newpage
    \pagenumbering{gobble}
        \printbibliography


    \newpage
    \pagenumbering{roman}
    \appendix
        \part{Appendices}
            \input{8 - Hilbert complexes/main.tex}
            \input{9 - weak conservation proofs/main.tex}
\end{document}

    
    
    %\section{}
    \newpage
    \pagenumbering{gobble}
        \printbibliography


    \newpage
    \pagenumbering{roman}
    \appendix
        \part{Appendices}
            \documentclass[12pt, a4paper]{report}

\input{template/main.tex}

\title{\BA{Title in Progress...}}
\author{Boris Andrews}
\affil{Mathematical Institute, University of Oxford}
\date{\today}


\begin{document}
    \pagenumbering{gobble}
    \maketitle
    
    
    \begin{abstract}
        Magnetic confinement reactors---in particular tokamaks---offer one of the most promising options for achieving practical nuclear fusion, with the potential to provide virtually limitless, clean energy. The theoretical and numerical modeling of tokamak plasmas is simultaneously an essential component of effective reactor design, and a great research barrier. Tokamak operational conditions exhibit comparatively low Knudsen numbers. Kinetic effects, including kinetic waves and instabilities, Landau damping, bump-on-tail instabilities and more, are therefore highly influential in tokamak plasma dynamics. Purely fluid models are inherently incapable of capturing these effects, whereas the high dimensionality in purely kinetic models render them practically intractable for most relevant purposes.

        We consider a $\delta\!f$ decomposition model, with a macroscopic fluid background and microscopic kinetic correction, both fully coupled to each other. A similar manner of discretization is proposed to that used in the recent \texttt{STRUPHY} code \cite{Holderied_Possanner_Wang_2021, Holderied_2022, Li_et_al_2023} with a finite-element model for the background and a pseudo-particle/PiC model for the correction.

        The fluid background satisfies the full, non-linear, resistive, compressible, Hall MHD equations. \cite{Laakmann_Hu_Farrell_2022} introduces finite-element(-in-space) implicit timesteppers for the incompressible analogue to this system with structure-preserving (SP) properties in the ideal case, alongside parameter-robust preconditioners. We show that these timesteppers can derive from a finite-element-in-time (FET) (and finite-element-in-space) interpretation. The benefits of this reformulation are discussed, including the derivation of timesteppers that are higher order in time, and the quantifiable dissipative SP properties in the non-ideal, resistive case.
        
        We discuss possible options for extending this FET approach to timesteppers for the compressible case.

        The kinetic corrections satisfy linearized Boltzmann equations. Using a Lénard--Bernstein collision operator, these take Fokker--Planck-like forms \cite{Fokker_1914, Planck_1917} wherein pseudo-particles in the numerical model obey the neoclassical transport equations, with particle-independent Brownian drift terms. This offers a rigorous methodology for incorporating collisions into the particle transport model, without coupling the equations of motions for each particle.
        
        Works by Chen, Chacón et al. \cite{Chen_Chacón_Barnes_2011, Chacón_Chen_Barnes_2013, Chen_Chacón_2014, Chen_Chacón_2015} have developed structure-preserving particle pushers for neoclassical transport in the Vlasov equations, derived from Crank--Nicolson integrators. We show these too can can derive from a FET interpretation, similarly offering potential extensions to higher-order-in-time particle pushers. The FET formulation is used also to consider how the stochastic drift terms can be incorporated into the pushers. Stochastic gyrokinetic expansions are also discussed.

        Different options for the numerical implementation of these schemes are considered.

        Due to the efficacy of FET in the development of SP timesteppers for both the fluid and kinetic component, we hope this approach will prove effective in the future for developing SP timesteppers for the full hybrid model. We hope this will give us the opportunity to incorporate previously inaccessible kinetic effects into the highly effective, modern, finite-element MHD models.
    \end{abstract}
    
    
    \newpage
    \tableofcontents
    
    
    \newpage
    \pagenumbering{arabic}
    %\linenumbers\renewcommand\thelinenumber{\color{black!50}\arabic{linenumber}}
            \input{0 - introduction/main.tex}
        \part{Research}
            \input{1 - low-noise PiC models/main.tex}
            \input{2 - kinetic component/main.tex}
            \input{3 - fluid component/main.tex}
            \input{4 - numerical implementation/main.tex}
        \part{Project Overview}
            \input{5 - research plan/main.tex}
            \input{6 - summary/main.tex}
    
    
    %\section{}
    \newpage
    \pagenumbering{gobble}
        \printbibliography


    \newpage
    \pagenumbering{roman}
    \appendix
        \part{Appendices}
            \input{8 - Hilbert complexes/main.tex}
            \input{9 - weak conservation proofs/main.tex}
\end{document}

            \documentclass[12pt, a4paper]{report}

\input{template/main.tex}

\title{\BA{Title in Progress...}}
\author{Boris Andrews}
\affil{Mathematical Institute, University of Oxford}
\date{\today}


\begin{document}
    \pagenumbering{gobble}
    \maketitle
    
    
    \begin{abstract}
        Magnetic confinement reactors---in particular tokamaks---offer one of the most promising options for achieving practical nuclear fusion, with the potential to provide virtually limitless, clean energy. The theoretical and numerical modeling of tokamak plasmas is simultaneously an essential component of effective reactor design, and a great research barrier. Tokamak operational conditions exhibit comparatively low Knudsen numbers. Kinetic effects, including kinetic waves and instabilities, Landau damping, bump-on-tail instabilities and more, are therefore highly influential in tokamak plasma dynamics. Purely fluid models are inherently incapable of capturing these effects, whereas the high dimensionality in purely kinetic models render them practically intractable for most relevant purposes.

        We consider a $\delta\!f$ decomposition model, with a macroscopic fluid background and microscopic kinetic correction, both fully coupled to each other. A similar manner of discretization is proposed to that used in the recent \texttt{STRUPHY} code \cite{Holderied_Possanner_Wang_2021, Holderied_2022, Li_et_al_2023} with a finite-element model for the background and a pseudo-particle/PiC model for the correction.

        The fluid background satisfies the full, non-linear, resistive, compressible, Hall MHD equations. \cite{Laakmann_Hu_Farrell_2022} introduces finite-element(-in-space) implicit timesteppers for the incompressible analogue to this system with structure-preserving (SP) properties in the ideal case, alongside parameter-robust preconditioners. We show that these timesteppers can derive from a finite-element-in-time (FET) (and finite-element-in-space) interpretation. The benefits of this reformulation are discussed, including the derivation of timesteppers that are higher order in time, and the quantifiable dissipative SP properties in the non-ideal, resistive case.
        
        We discuss possible options for extending this FET approach to timesteppers for the compressible case.

        The kinetic corrections satisfy linearized Boltzmann equations. Using a Lénard--Bernstein collision operator, these take Fokker--Planck-like forms \cite{Fokker_1914, Planck_1917} wherein pseudo-particles in the numerical model obey the neoclassical transport equations, with particle-independent Brownian drift terms. This offers a rigorous methodology for incorporating collisions into the particle transport model, without coupling the equations of motions for each particle.
        
        Works by Chen, Chacón et al. \cite{Chen_Chacón_Barnes_2011, Chacón_Chen_Barnes_2013, Chen_Chacón_2014, Chen_Chacón_2015} have developed structure-preserving particle pushers for neoclassical transport in the Vlasov equations, derived from Crank--Nicolson integrators. We show these too can can derive from a FET interpretation, similarly offering potential extensions to higher-order-in-time particle pushers. The FET formulation is used also to consider how the stochastic drift terms can be incorporated into the pushers. Stochastic gyrokinetic expansions are also discussed.

        Different options for the numerical implementation of these schemes are considered.

        Due to the efficacy of FET in the development of SP timesteppers for both the fluid and kinetic component, we hope this approach will prove effective in the future for developing SP timesteppers for the full hybrid model. We hope this will give us the opportunity to incorporate previously inaccessible kinetic effects into the highly effective, modern, finite-element MHD models.
    \end{abstract}
    
    
    \newpage
    \tableofcontents
    
    
    \newpage
    \pagenumbering{arabic}
    %\linenumbers\renewcommand\thelinenumber{\color{black!50}\arabic{linenumber}}
            \input{0 - introduction/main.tex}
        \part{Research}
            \input{1 - low-noise PiC models/main.tex}
            \input{2 - kinetic component/main.tex}
            \input{3 - fluid component/main.tex}
            \input{4 - numerical implementation/main.tex}
        \part{Project Overview}
            \input{5 - research plan/main.tex}
            \input{6 - summary/main.tex}
    
    
    %\section{}
    \newpage
    \pagenumbering{gobble}
        \printbibliography


    \newpage
    \pagenumbering{roman}
    \appendix
        \part{Appendices}
            \input{8 - Hilbert complexes/main.tex}
            \input{9 - weak conservation proofs/main.tex}
\end{document}

\end{document}

            \documentclass[12pt, a4paper]{report}

\documentclass[12pt, a4paper]{report}

\input{template/main.tex}

\title{\BA{Title in Progress...}}
\author{Boris Andrews}
\affil{Mathematical Institute, University of Oxford}
\date{\today}


\begin{document}
    \pagenumbering{gobble}
    \maketitle
    
    
    \begin{abstract}
        Magnetic confinement reactors---in particular tokamaks---offer one of the most promising options for achieving practical nuclear fusion, with the potential to provide virtually limitless, clean energy. The theoretical and numerical modeling of tokamak plasmas is simultaneously an essential component of effective reactor design, and a great research barrier. Tokamak operational conditions exhibit comparatively low Knudsen numbers. Kinetic effects, including kinetic waves and instabilities, Landau damping, bump-on-tail instabilities and more, are therefore highly influential in tokamak plasma dynamics. Purely fluid models are inherently incapable of capturing these effects, whereas the high dimensionality in purely kinetic models render them practically intractable for most relevant purposes.

        We consider a $\delta\!f$ decomposition model, with a macroscopic fluid background and microscopic kinetic correction, both fully coupled to each other. A similar manner of discretization is proposed to that used in the recent \texttt{STRUPHY} code \cite{Holderied_Possanner_Wang_2021, Holderied_2022, Li_et_al_2023} with a finite-element model for the background and a pseudo-particle/PiC model for the correction.

        The fluid background satisfies the full, non-linear, resistive, compressible, Hall MHD equations. \cite{Laakmann_Hu_Farrell_2022} introduces finite-element(-in-space) implicit timesteppers for the incompressible analogue to this system with structure-preserving (SP) properties in the ideal case, alongside parameter-robust preconditioners. We show that these timesteppers can derive from a finite-element-in-time (FET) (and finite-element-in-space) interpretation. The benefits of this reformulation are discussed, including the derivation of timesteppers that are higher order in time, and the quantifiable dissipative SP properties in the non-ideal, resistive case.
        
        We discuss possible options for extending this FET approach to timesteppers for the compressible case.

        The kinetic corrections satisfy linearized Boltzmann equations. Using a Lénard--Bernstein collision operator, these take Fokker--Planck-like forms \cite{Fokker_1914, Planck_1917} wherein pseudo-particles in the numerical model obey the neoclassical transport equations, with particle-independent Brownian drift terms. This offers a rigorous methodology for incorporating collisions into the particle transport model, without coupling the equations of motions for each particle.
        
        Works by Chen, Chacón et al. \cite{Chen_Chacón_Barnes_2011, Chacón_Chen_Barnes_2013, Chen_Chacón_2014, Chen_Chacón_2015} have developed structure-preserving particle pushers for neoclassical transport in the Vlasov equations, derived from Crank--Nicolson integrators. We show these too can can derive from a FET interpretation, similarly offering potential extensions to higher-order-in-time particle pushers. The FET formulation is used also to consider how the stochastic drift terms can be incorporated into the pushers. Stochastic gyrokinetic expansions are also discussed.

        Different options for the numerical implementation of these schemes are considered.

        Due to the efficacy of FET in the development of SP timesteppers for both the fluid and kinetic component, we hope this approach will prove effective in the future for developing SP timesteppers for the full hybrid model. We hope this will give us the opportunity to incorporate previously inaccessible kinetic effects into the highly effective, modern, finite-element MHD models.
    \end{abstract}
    
    
    \newpage
    \tableofcontents
    
    
    \newpage
    \pagenumbering{arabic}
    %\linenumbers\renewcommand\thelinenumber{\color{black!50}\arabic{linenumber}}
            \input{0 - introduction/main.tex}
        \part{Research}
            \input{1 - low-noise PiC models/main.tex}
            \input{2 - kinetic component/main.tex}
            \input{3 - fluid component/main.tex}
            \input{4 - numerical implementation/main.tex}
        \part{Project Overview}
            \input{5 - research plan/main.tex}
            \input{6 - summary/main.tex}
    
    
    %\section{}
    \newpage
    \pagenumbering{gobble}
        \printbibliography


    \newpage
    \pagenumbering{roman}
    \appendix
        \part{Appendices}
            \input{8 - Hilbert complexes/main.tex}
            \input{9 - weak conservation proofs/main.tex}
\end{document}


\title{\BA{Title in Progress...}}
\author{Boris Andrews}
\affil{Mathematical Institute, University of Oxford}
\date{\today}


\begin{document}
    \pagenumbering{gobble}
    \maketitle
    
    
    \begin{abstract}
        Magnetic confinement reactors---in particular tokamaks---offer one of the most promising options for achieving practical nuclear fusion, with the potential to provide virtually limitless, clean energy. The theoretical and numerical modeling of tokamak plasmas is simultaneously an essential component of effective reactor design, and a great research barrier. Tokamak operational conditions exhibit comparatively low Knudsen numbers. Kinetic effects, including kinetic waves and instabilities, Landau damping, bump-on-tail instabilities and more, are therefore highly influential in tokamak plasma dynamics. Purely fluid models are inherently incapable of capturing these effects, whereas the high dimensionality in purely kinetic models render them practically intractable for most relevant purposes.

        We consider a $\delta\!f$ decomposition model, with a macroscopic fluid background and microscopic kinetic correction, both fully coupled to each other. A similar manner of discretization is proposed to that used in the recent \texttt{STRUPHY} code \cite{Holderied_Possanner_Wang_2021, Holderied_2022, Li_et_al_2023} with a finite-element model for the background and a pseudo-particle/PiC model for the correction.

        The fluid background satisfies the full, non-linear, resistive, compressible, Hall MHD equations. \cite{Laakmann_Hu_Farrell_2022} introduces finite-element(-in-space) implicit timesteppers for the incompressible analogue to this system with structure-preserving (SP) properties in the ideal case, alongside parameter-robust preconditioners. We show that these timesteppers can derive from a finite-element-in-time (FET) (and finite-element-in-space) interpretation. The benefits of this reformulation are discussed, including the derivation of timesteppers that are higher order in time, and the quantifiable dissipative SP properties in the non-ideal, resistive case.
        
        We discuss possible options for extending this FET approach to timesteppers for the compressible case.

        The kinetic corrections satisfy linearized Boltzmann equations. Using a Lénard--Bernstein collision operator, these take Fokker--Planck-like forms \cite{Fokker_1914, Planck_1917} wherein pseudo-particles in the numerical model obey the neoclassical transport equations, with particle-independent Brownian drift terms. This offers a rigorous methodology for incorporating collisions into the particle transport model, without coupling the equations of motions for each particle.
        
        Works by Chen, Chacón et al. \cite{Chen_Chacón_Barnes_2011, Chacón_Chen_Barnes_2013, Chen_Chacón_2014, Chen_Chacón_2015} have developed structure-preserving particle pushers for neoclassical transport in the Vlasov equations, derived from Crank--Nicolson integrators. We show these too can can derive from a FET interpretation, similarly offering potential extensions to higher-order-in-time particle pushers. The FET formulation is used also to consider how the stochastic drift terms can be incorporated into the pushers. Stochastic gyrokinetic expansions are also discussed.

        Different options for the numerical implementation of these schemes are considered.

        Due to the efficacy of FET in the development of SP timesteppers for both the fluid and kinetic component, we hope this approach will prove effective in the future for developing SP timesteppers for the full hybrid model. We hope this will give us the opportunity to incorporate previously inaccessible kinetic effects into the highly effective, modern, finite-element MHD models.
    \end{abstract}
    
    
    \newpage
    \tableofcontents
    
    
    \newpage
    \pagenumbering{arabic}
    %\linenumbers\renewcommand\thelinenumber{\color{black!50}\arabic{linenumber}}
            \documentclass[12pt, a4paper]{report}

\input{template/main.tex}

\title{\BA{Title in Progress...}}
\author{Boris Andrews}
\affil{Mathematical Institute, University of Oxford}
\date{\today}


\begin{document}
    \pagenumbering{gobble}
    \maketitle
    
    
    \begin{abstract}
        Magnetic confinement reactors---in particular tokamaks---offer one of the most promising options for achieving practical nuclear fusion, with the potential to provide virtually limitless, clean energy. The theoretical and numerical modeling of tokamak plasmas is simultaneously an essential component of effective reactor design, and a great research barrier. Tokamak operational conditions exhibit comparatively low Knudsen numbers. Kinetic effects, including kinetic waves and instabilities, Landau damping, bump-on-tail instabilities and more, are therefore highly influential in tokamak plasma dynamics. Purely fluid models are inherently incapable of capturing these effects, whereas the high dimensionality in purely kinetic models render them practically intractable for most relevant purposes.

        We consider a $\delta\!f$ decomposition model, with a macroscopic fluid background and microscopic kinetic correction, both fully coupled to each other. A similar manner of discretization is proposed to that used in the recent \texttt{STRUPHY} code \cite{Holderied_Possanner_Wang_2021, Holderied_2022, Li_et_al_2023} with a finite-element model for the background and a pseudo-particle/PiC model for the correction.

        The fluid background satisfies the full, non-linear, resistive, compressible, Hall MHD equations. \cite{Laakmann_Hu_Farrell_2022} introduces finite-element(-in-space) implicit timesteppers for the incompressible analogue to this system with structure-preserving (SP) properties in the ideal case, alongside parameter-robust preconditioners. We show that these timesteppers can derive from a finite-element-in-time (FET) (and finite-element-in-space) interpretation. The benefits of this reformulation are discussed, including the derivation of timesteppers that are higher order in time, and the quantifiable dissipative SP properties in the non-ideal, resistive case.
        
        We discuss possible options for extending this FET approach to timesteppers for the compressible case.

        The kinetic corrections satisfy linearized Boltzmann equations. Using a Lénard--Bernstein collision operator, these take Fokker--Planck-like forms \cite{Fokker_1914, Planck_1917} wherein pseudo-particles in the numerical model obey the neoclassical transport equations, with particle-independent Brownian drift terms. This offers a rigorous methodology for incorporating collisions into the particle transport model, without coupling the equations of motions for each particle.
        
        Works by Chen, Chacón et al. \cite{Chen_Chacón_Barnes_2011, Chacón_Chen_Barnes_2013, Chen_Chacón_2014, Chen_Chacón_2015} have developed structure-preserving particle pushers for neoclassical transport in the Vlasov equations, derived from Crank--Nicolson integrators. We show these too can can derive from a FET interpretation, similarly offering potential extensions to higher-order-in-time particle pushers. The FET formulation is used also to consider how the stochastic drift terms can be incorporated into the pushers. Stochastic gyrokinetic expansions are also discussed.

        Different options for the numerical implementation of these schemes are considered.

        Due to the efficacy of FET in the development of SP timesteppers for both the fluid and kinetic component, we hope this approach will prove effective in the future for developing SP timesteppers for the full hybrid model. We hope this will give us the opportunity to incorporate previously inaccessible kinetic effects into the highly effective, modern, finite-element MHD models.
    \end{abstract}
    
    
    \newpage
    \tableofcontents
    
    
    \newpage
    \pagenumbering{arabic}
    %\linenumbers\renewcommand\thelinenumber{\color{black!50}\arabic{linenumber}}
            \input{0 - introduction/main.tex}
        \part{Research}
            \input{1 - low-noise PiC models/main.tex}
            \input{2 - kinetic component/main.tex}
            \input{3 - fluid component/main.tex}
            \input{4 - numerical implementation/main.tex}
        \part{Project Overview}
            \input{5 - research plan/main.tex}
            \input{6 - summary/main.tex}
    
    
    %\section{}
    \newpage
    \pagenumbering{gobble}
        \printbibliography


    \newpage
    \pagenumbering{roman}
    \appendix
        \part{Appendices}
            \input{8 - Hilbert complexes/main.tex}
            \input{9 - weak conservation proofs/main.tex}
\end{document}

        \part{Research}
            \documentclass[12pt, a4paper]{report}

\input{template/main.tex}

\title{\BA{Title in Progress...}}
\author{Boris Andrews}
\affil{Mathematical Institute, University of Oxford}
\date{\today}


\begin{document}
    \pagenumbering{gobble}
    \maketitle
    
    
    \begin{abstract}
        Magnetic confinement reactors---in particular tokamaks---offer one of the most promising options for achieving practical nuclear fusion, with the potential to provide virtually limitless, clean energy. The theoretical and numerical modeling of tokamak plasmas is simultaneously an essential component of effective reactor design, and a great research barrier. Tokamak operational conditions exhibit comparatively low Knudsen numbers. Kinetic effects, including kinetic waves and instabilities, Landau damping, bump-on-tail instabilities and more, are therefore highly influential in tokamak plasma dynamics. Purely fluid models are inherently incapable of capturing these effects, whereas the high dimensionality in purely kinetic models render them practically intractable for most relevant purposes.

        We consider a $\delta\!f$ decomposition model, with a macroscopic fluid background and microscopic kinetic correction, both fully coupled to each other. A similar manner of discretization is proposed to that used in the recent \texttt{STRUPHY} code \cite{Holderied_Possanner_Wang_2021, Holderied_2022, Li_et_al_2023} with a finite-element model for the background and a pseudo-particle/PiC model for the correction.

        The fluid background satisfies the full, non-linear, resistive, compressible, Hall MHD equations. \cite{Laakmann_Hu_Farrell_2022} introduces finite-element(-in-space) implicit timesteppers for the incompressible analogue to this system with structure-preserving (SP) properties in the ideal case, alongside parameter-robust preconditioners. We show that these timesteppers can derive from a finite-element-in-time (FET) (and finite-element-in-space) interpretation. The benefits of this reformulation are discussed, including the derivation of timesteppers that are higher order in time, and the quantifiable dissipative SP properties in the non-ideal, resistive case.
        
        We discuss possible options for extending this FET approach to timesteppers for the compressible case.

        The kinetic corrections satisfy linearized Boltzmann equations. Using a Lénard--Bernstein collision operator, these take Fokker--Planck-like forms \cite{Fokker_1914, Planck_1917} wherein pseudo-particles in the numerical model obey the neoclassical transport equations, with particle-independent Brownian drift terms. This offers a rigorous methodology for incorporating collisions into the particle transport model, without coupling the equations of motions for each particle.
        
        Works by Chen, Chacón et al. \cite{Chen_Chacón_Barnes_2011, Chacón_Chen_Barnes_2013, Chen_Chacón_2014, Chen_Chacón_2015} have developed structure-preserving particle pushers for neoclassical transport in the Vlasov equations, derived from Crank--Nicolson integrators. We show these too can can derive from a FET interpretation, similarly offering potential extensions to higher-order-in-time particle pushers. The FET formulation is used also to consider how the stochastic drift terms can be incorporated into the pushers. Stochastic gyrokinetic expansions are also discussed.

        Different options for the numerical implementation of these schemes are considered.

        Due to the efficacy of FET in the development of SP timesteppers for both the fluid and kinetic component, we hope this approach will prove effective in the future for developing SP timesteppers for the full hybrid model. We hope this will give us the opportunity to incorporate previously inaccessible kinetic effects into the highly effective, modern, finite-element MHD models.
    \end{abstract}
    
    
    \newpage
    \tableofcontents
    
    
    \newpage
    \pagenumbering{arabic}
    %\linenumbers\renewcommand\thelinenumber{\color{black!50}\arabic{linenumber}}
            \input{0 - introduction/main.tex}
        \part{Research}
            \input{1 - low-noise PiC models/main.tex}
            \input{2 - kinetic component/main.tex}
            \input{3 - fluid component/main.tex}
            \input{4 - numerical implementation/main.tex}
        \part{Project Overview}
            \input{5 - research plan/main.tex}
            \input{6 - summary/main.tex}
    
    
    %\section{}
    \newpage
    \pagenumbering{gobble}
        \printbibliography


    \newpage
    \pagenumbering{roman}
    \appendix
        \part{Appendices}
            \input{8 - Hilbert complexes/main.tex}
            \input{9 - weak conservation proofs/main.tex}
\end{document}

            \documentclass[12pt, a4paper]{report}

\input{template/main.tex}

\title{\BA{Title in Progress...}}
\author{Boris Andrews}
\affil{Mathematical Institute, University of Oxford}
\date{\today}


\begin{document}
    \pagenumbering{gobble}
    \maketitle
    
    
    \begin{abstract}
        Magnetic confinement reactors---in particular tokamaks---offer one of the most promising options for achieving practical nuclear fusion, with the potential to provide virtually limitless, clean energy. The theoretical and numerical modeling of tokamak plasmas is simultaneously an essential component of effective reactor design, and a great research barrier. Tokamak operational conditions exhibit comparatively low Knudsen numbers. Kinetic effects, including kinetic waves and instabilities, Landau damping, bump-on-tail instabilities and more, are therefore highly influential in tokamak plasma dynamics. Purely fluid models are inherently incapable of capturing these effects, whereas the high dimensionality in purely kinetic models render them practically intractable for most relevant purposes.

        We consider a $\delta\!f$ decomposition model, with a macroscopic fluid background and microscopic kinetic correction, both fully coupled to each other. A similar manner of discretization is proposed to that used in the recent \texttt{STRUPHY} code \cite{Holderied_Possanner_Wang_2021, Holderied_2022, Li_et_al_2023} with a finite-element model for the background and a pseudo-particle/PiC model for the correction.

        The fluid background satisfies the full, non-linear, resistive, compressible, Hall MHD equations. \cite{Laakmann_Hu_Farrell_2022} introduces finite-element(-in-space) implicit timesteppers for the incompressible analogue to this system with structure-preserving (SP) properties in the ideal case, alongside parameter-robust preconditioners. We show that these timesteppers can derive from a finite-element-in-time (FET) (and finite-element-in-space) interpretation. The benefits of this reformulation are discussed, including the derivation of timesteppers that are higher order in time, and the quantifiable dissipative SP properties in the non-ideal, resistive case.
        
        We discuss possible options for extending this FET approach to timesteppers for the compressible case.

        The kinetic corrections satisfy linearized Boltzmann equations. Using a Lénard--Bernstein collision operator, these take Fokker--Planck-like forms \cite{Fokker_1914, Planck_1917} wherein pseudo-particles in the numerical model obey the neoclassical transport equations, with particle-independent Brownian drift terms. This offers a rigorous methodology for incorporating collisions into the particle transport model, without coupling the equations of motions for each particle.
        
        Works by Chen, Chacón et al. \cite{Chen_Chacón_Barnes_2011, Chacón_Chen_Barnes_2013, Chen_Chacón_2014, Chen_Chacón_2015} have developed structure-preserving particle pushers for neoclassical transport in the Vlasov equations, derived from Crank--Nicolson integrators. We show these too can can derive from a FET interpretation, similarly offering potential extensions to higher-order-in-time particle pushers. The FET formulation is used also to consider how the stochastic drift terms can be incorporated into the pushers. Stochastic gyrokinetic expansions are also discussed.

        Different options for the numerical implementation of these schemes are considered.

        Due to the efficacy of FET in the development of SP timesteppers for both the fluid and kinetic component, we hope this approach will prove effective in the future for developing SP timesteppers for the full hybrid model. We hope this will give us the opportunity to incorporate previously inaccessible kinetic effects into the highly effective, modern, finite-element MHD models.
    \end{abstract}
    
    
    \newpage
    \tableofcontents
    
    
    \newpage
    \pagenumbering{arabic}
    %\linenumbers\renewcommand\thelinenumber{\color{black!50}\arabic{linenumber}}
            \input{0 - introduction/main.tex}
        \part{Research}
            \input{1 - low-noise PiC models/main.tex}
            \input{2 - kinetic component/main.tex}
            \input{3 - fluid component/main.tex}
            \input{4 - numerical implementation/main.tex}
        \part{Project Overview}
            \input{5 - research plan/main.tex}
            \input{6 - summary/main.tex}
    
    
    %\section{}
    \newpage
    \pagenumbering{gobble}
        \printbibliography


    \newpage
    \pagenumbering{roman}
    \appendix
        \part{Appendices}
            \input{8 - Hilbert complexes/main.tex}
            \input{9 - weak conservation proofs/main.tex}
\end{document}

            \documentclass[12pt, a4paper]{report}

\input{template/main.tex}

\title{\BA{Title in Progress...}}
\author{Boris Andrews}
\affil{Mathematical Institute, University of Oxford}
\date{\today}


\begin{document}
    \pagenumbering{gobble}
    \maketitle
    
    
    \begin{abstract}
        Magnetic confinement reactors---in particular tokamaks---offer one of the most promising options for achieving practical nuclear fusion, with the potential to provide virtually limitless, clean energy. The theoretical and numerical modeling of tokamak plasmas is simultaneously an essential component of effective reactor design, and a great research barrier. Tokamak operational conditions exhibit comparatively low Knudsen numbers. Kinetic effects, including kinetic waves and instabilities, Landau damping, bump-on-tail instabilities and more, are therefore highly influential in tokamak plasma dynamics. Purely fluid models are inherently incapable of capturing these effects, whereas the high dimensionality in purely kinetic models render them practically intractable for most relevant purposes.

        We consider a $\delta\!f$ decomposition model, with a macroscopic fluid background and microscopic kinetic correction, both fully coupled to each other. A similar manner of discretization is proposed to that used in the recent \texttt{STRUPHY} code \cite{Holderied_Possanner_Wang_2021, Holderied_2022, Li_et_al_2023} with a finite-element model for the background and a pseudo-particle/PiC model for the correction.

        The fluid background satisfies the full, non-linear, resistive, compressible, Hall MHD equations. \cite{Laakmann_Hu_Farrell_2022} introduces finite-element(-in-space) implicit timesteppers for the incompressible analogue to this system with structure-preserving (SP) properties in the ideal case, alongside parameter-robust preconditioners. We show that these timesteppers can derive from a finite-element-in-time (FET) (and finite-element-in-space) interpretation. The benefits of this reformulation are discussed, including the derivation of timesteppers that are higher order in time, and the quantifiable dissipative SP properties in the non-ideal, resistive case.
        
        We discuss possible options for extending this FET approach to timesteppers for the compressible case.

        The kinetic corrections satisfy linearized Boltzmann equations. Using a Lénard--Bernstein collision operator, these take Fokker--Planck-like forms \cite{Fokker_1914, Planck_1917} wherein pseudo-particles in the numerical model obey the neoclassical transport equations, with particle-independent Brownian drift terms. This offers a rigorous methodology for incorporating collisions into the particle transport model, without coupling the equations of motions for each particle.
        
        Works by Chen, Chacón et al. \cite{Chen_Chacón_Barnes_2011, Chacón_Chen_Barnes_2013, Chen_Chacón_2014, Chen_Chacón_2015} have developed structure-preserving particle pushers for neoclassical transport in the Vlasov equations, derived from Crank--Nicolson integrators. We show these too can can derive from a FET interpretation, similarly offering potential extensions to higher-order-in-time particle pushers. The FET formulation is used also to consider how the stochastic drift terms can be incorporated into the pushers. Stochastic gyrokinetic expansions are also discussed.

        Different options for the numerical implementation of these schemes are considered.

        Due to the efficacy of FET in the development of SP timesteppers for both the fluid and kinetic component, we hope this approach will prove effective in the future for developing SP timesteppers for the full hybrid model. We hope this will give us the opportunity to incorporate previously inaccessible kinetic effects into the highly effective, modern, finite-element MHD models.
    \end{abstract}
    
    
    \newpage
    \tableofcontents
    
    
    \newpage
    \pagenumbering{arabic}
    %\linenumbers\renewcommand\thelinenumber{\color{black!50}\arabic{linenumber}}
            \input{0 - introduction/main.tex}
        \part{Research}
            \input{1 - low-noise PiC models/main.tex}
            \input{2 - kinetic component/main.tex}
            \input{3 - fluid component/main.tex}
            \input{4 - numerical implementation/main.tex}
        \part{Project Overview}
            \input{5 - research plan/main.tex}
            \input{6 - summary/main.tex}
    
    
    %\section{}
    \newpage
    \pagenumbering{gobble}
        \printbibliography


    \newpage
    \pagenumbering{roman}
    \appendix
        \part{Appendices}
            \input{8 - Hilbert complexes/main.tex}
            \input{9 - weak conservation proofs/main.tex}
\end{document}

            \documentclass[12pt, a4paper]{report}

\input{template/main.tex}

\title{\BA{Title in Progress...}}
\author{Boris Andrews}
\affil{Mathematical Institute, University of Oxford}
\date{\today}


\begin{document}
    \pagenumbering{gobble}
    \maketitle
    
    
    \begin{abstract}
        Magnetic confinement reactors---in particular tokamaks---offer one of the most promising options for achieving practical nuclear fusion, with the potential to provide virtually limitless, clean energy. The theoretical and numerical modeling of tokamak plasmas is simultaneously an essential component of effective reactor design, and a great research barrier. Tokamak operational conditions exhibit comparatively low Knudsen numbers. Kinetic effects, including kinetic waves and instabilities, Landau damping, bump-on-tail instabilities and more, are therefore highly influential in tokamak plasma dynamics. Purely fluid models are inherently incapable of capturing these effects, whereas the high dimensionality in purely kinetic models render them practically intractable for most relevant purposes.

        We consider a $\delta\!f$ decomposition model, with a macroscopic fluid background and microscopic kinetic correction, both fully coupled to each other. A similar manner of discretization is proposed to that used in the recent \texttt{STRUPHY} code \cite{Holderied_Possanner_Wang_2021, Holderied_2022, Li_et_al_2023} with a finite-element model for the background and a pseudo-particle/PiC model for the correction.

        The fluid background satisfies the full, non-linear, resistive, compressible, Hall MHD equations. \cite{Laakmann_Hu_Farrell_2022} introduces finite-element(-in-space) implicit timesteppers for the incompressible analogue to this system with structure-preserving (SP) properties in the ideal case, alongside parameter-robust preconditioners. We show that these timesteppers can derive from a finite-element-in-time (FET) (and finite-element-in-space) interpretation. The benefits of this reformulation are discussed, including the derivation of timesteppers that are higher order in time, and the quantifiable dissipative SP properties in the non-ideal, resistive case.
        
        We discuss possible options for extending this FET approach to timesteppers for the compressible case.

        The kinetic corrections satisfy linearized Boltzmann equations. Using a Lénard--Bernstein collision operator, these take Fokker--Planck-like forms \cite{Fokker_1914, Planck_1917} wherein pseudo-particles in the numerical model obey the neoclassical transport equations, with particle-independent Brownian drift terms. This offers a rigorous methodology for incorporating collisions into the particle transport model, without coupling the equations of motions for each particle.
        
        Works by Chen, Chacón et al. \cite{Chen_Chacón_Barnes_2011, Chacón_Chen_Barnes_2013, Chen_Chacón_2014, Chen_Chacón_2015} have developed structure-preserving particle pushers for neoclassical transport in the Vlasov equations, derived from Crank--Nicolson integrators. We show these too can can derive from a FET interpretation, similarly offering potential extensions to higher-order-in-time particle pushers. The FET formulation is used also to consider how the stochastic drift terms can be incorporated into the pushers. Stochastic gyrokinetic expansions are also discussed.

        Different options for the numerical implementation of these schemes are considered.

        Due to the efficacy of FET in the development of SP timesteppers for both the fluid and kinetic component, we hope this approach will prove effective in the future for developing SP timesteppers for the full hybrid model. We hope this will give us the opportunity to incorporate previously inaccessible kinetic effects into the highly effective, modern, finite-element MHD models.
    \end{abstract}
    
    
    \newpage
    \tableofcontents
    
    
    \newpage
    \pagenumbering{arabic}
    %\linenumbers\renewcommand\thelinenumber{\color{black!50}\arabic{linenumber}}
            \input{0 - introduction/main.tex}
        \part{Research}
            \input{1 - low-noise PiC models/main.tex}
            \input{2 - kinetic component/main.tex}
            \input{3 - fluid component/main.tex}
            \input{4 - numerical implementation/main.tex}
        \part{Project Overview}
            \input{5 - research plan/main.tex}
            \input{6 - summary/main.tex}
    
    
    %\section{}
    \newpage
    \pagenumbering{gobble}
        \printbibliography


    \newpage
    \pagenumbering{roman}
    \appendix
        \part{Appendices}
            \input{8 - Hilbert complexes/main.tex}
            \input{9 - weak conservation proofs/main.tex}
\end{document}

        \part{Project Overview}
            \documentclass[12pt, a4paper]{report}

\input{template/main.tex}

\title{\BA{Title in Progress...}}
\author{Boris Andrews}
\affil{Mathematical Institute, University of Oxford}
\date{\today}


\begin{document}
    \pagenumbering{gobble}
    \maketitle
    
    
    \begin{abstract}
        Magnetic confinement reactors---in particular tokamaks---offer one of the most promising options for achieving practical nuclear fusion, with the potential to provide virtually limitless, clean energy. The theoretical and numerical modeling of tokamak plasmas is simultaneously an essential component of effective reactor design, and a great research barrier. Tokamak operational conditions exhibit comparatively low Knudsen numbers. Kinetic effects, including kinetic waves and instabilities, Landau damping, bump-on-tail instabilities and more, are therefore highly influential in tokamak plasma dynamics. Purely fluid models are inherently incapable of capturing these effects, whereas the high dimensionality in purely kinetic models render them practically intractable for most relevant purposes.

        We consider a $\delta\!f$ decomposition model, with a macroscopic fluid background and microscopic kinetic correction, both fully coupled to each other. A similar manner of discretization is proposed to that used in the recent \texttt{STRUPHY} code \cite{Holderied_Possanner_Wang_2021, Holderied_2022, Li_et_al_2023} with a finite-element model for the background and a pseudo-particle/PiC model for the correction.

        The fluid background satisfies the full, non-linear, resistive, compressible, Hall MHD equations. \cite{Laakmann_Hu_Farrell_2022} introduces finite-element(-in-space) implicit timesteppers for the incompressible analogue to this system with structure-preserving (SP) properties in the ideal case, alongside parameter-robust preconditioners. We show that these timesteppers can derive from a finite-element-in-time (FET) (and finite-element-in-space) interpretation. The benefits of this reformulation are discussed, including the derivation of timesteppers that are higher order in time, and the quantifiable dissipative SP properties in the non-ideal, resistive case.
        
        We discuss possible options for extending this FET approach to timesteppers for the compressible case.

        The kinetic corrections satisfy linearized Boltzmann equations. Using a Lénard--Bernstein collision operator, these take Fokker--Planck-like forms \cite{Fokker_1914, Planck_1917} wherein pseudo-particles in the numerical model obey the neoclassical transport equations, with particle-independent Brownian drift terms. This offers a rigorous methodology for incorporating collisions into the particle transport model, without coupling the equations of motions for each particle.
        
        Works by Chen, Chacón et al. \cite{Chen_Chacón_Barnes_2011, Chacón_Chen_Barnes_2013, Chen_Chacón_2014, Chen_Chacón_2015} have developed structure-preserving particle pushers for neoclassical transport in the Vlasov equations, derived from Crank--Nicolson integrators. We show these too can can derive from a FET interpretation, similarly offering potential extensions to higher-order-in-time particle pushers. The FET formulation is used also to consider how the stochastic drift terms can be incorporated into the pushers. Stochastic gyrokinetic expansions are also discussed.

        Different options for the numerical implementation of these schemes are considered.

        Due to the efficacy of FET in the development of SP timesteppers for both the fluid and kinetic component, we hope this approach will prove effective in the future for developing SP timesteppers for the full hybrid model. We hope this will give us the opportunity to incorporate previously inaccessible kinetic effects into the highly effective, modern, finite-element MHD models.
    \end{abstract}
    
    
    \newpage
    \tableofcontents
    
    
    \newpage
    \pagenumbering{arabic}
    %\linenumbers\renewcommand\thelinenumber{\color{black!50}\arabic{linenumber}}
            \input{0 - introduction/main.tex}
        \part{Research}
            \input{1 - low-noise PiC models/main.tex}
            \input{2 - kinetic component/main.tex}
            \input{3 - fluid component/main.tex}
            \input{4 - numerical implementation/main.tex}
        \part{Project Overview}
            \input{5 - research plan/main.tex}
            \input{6 - summary/main.tex}
    
    
    %\section{}
    \newpage
    \pagenumbering{gobble}
        \printbibliography


    \newpage
    \pagenumbering{roman}
    \appendix
        \part{Appendices}
            \input{8 - Hilbert complexes/main.tex}
            \input{9 - weak conservation proofs/main.tex}
\end{document}

            \documentclass[12pt, a4paper]{report}

\input{template/main.tex}

\title{\BA{Title in Progress...}}
\author{Boris Andrews}
\affil{Mathematical Institute, University of Oxford}
\date{\today}


\begin{document}
    \pagenumbering{gobble}
    \maketitle
    
    
    \begin{abstract}
        Magnetic confinement reactors---in particular tokamaks---offer one of the most promising options for achieving practical nuclear fusion, with the potential to provide virtually limitless, clean energy. The theoretical and numerical modeling of tokamak plasmas is simultaneously an essential component of effective reactor design, and a great research barrier. Tokamak operational conditions exhibit comparatively low Knudsen numbers. Kinetic effects, including kinetic waves and instabilities, Landau damping, bump-on-tail instabilities and more, are therefore highly influential in tokamak plasma dynamics. Purely fluid models are inherently incapable of capturing these effects, whereas the high dimensionality in purely kinetic models render them practically intractable for most relevant purposes.

        We consider a $\delta\!f$ decomposition model, with a macroscopic fluid background and microscopic kinetic correction, both fully coupled to each other. A similar manner of discretization is proposed to that used in the recent \texttt{STRUPHY} code \cite{Holderied_Possanner_Wang_2021, Holderied_2022, Li_et_al_2023} with a finite-element model for the background and a pseudo-particle/PiC model for the correction.

        The fluid background satisfies the full, non-linear, resistive, compressible, Hall MHD equations. \cite{Laakmann_Hu_Farrell_2022} introduces finite-element(-in-space) implicit timesteppers for the incompressible analogue to this system with structure-preserving (SP) properties in the ideal case, alongside parameter-robust preconditioners. We show that these timesteppers can derive from a finite-element-in-time (FET) (and finite-element-in-space) interpretation. The benefits of this reformulation are discussed, including the derivation of timesteppers that are higher order in time, and the quantifiable dissipative SP properties in the non-ideal, resistive case.
        
        We discuss possible options for extending this FET approach to timesteppers for the compressible case.

        The kinetic corrections satisfy linearized Boltzmann equations. Using a Lénard--Bernstein collision operator, these take Fokker--Planck-like forms \cite{Fokker_1914, Planck_1917} wherein pseudo-particles in the numerical model obey the neoclassical transport equations, with particle-independent Brownian drift terms. This offers a rigorous methodology for incorporating collisions into the particle transport model, without coupling the equations of motions for each particle.
        
        Works by Chen, Chacón et al. \cite{Chen_Chacón_Barnes_2011, Chacón_Chen_Barnes_2013, Chen_Chacón_2014, Chen_Chacón_2015} have developed structure-preserving particle pushers for neoclassical transport in the Vlasov equations, derived from Crank--Nicolson integrators. We show these too can can derive from a FET interpretation, similarly offering potential extensions to higher-order-in-time particle pushers. The FET formulation is used also to consider how the stochastic drift terms can be incorporated into the pushers. Stochastic gyrokinetic expansions are also discussed.

        Different options for the numerical implementation of these schemes are considered.

        Due to the efficacy of FET in the development of SP timesteppers for both the fluid and kinetic component, we hope this approach will prove effective in the future for developing SP timesteppers for the full hybrid model. We hope this will give us the opportunity to incorporate previously inaccessible kinetic effects into the highly effective, modern, finite-element MHD models.
    \end{abstract}
    
    
    \newpage
    \tableofcontents
    
    
    \newpage
    \pagenumbering{arabic}
    %\linenumbers\renewcommand\thelinenumber{\color{black!50}\arabic{linenumber}}
            \input{0 - introduction/main.tex}
        \part{Research}
            \input{1 - low-noise PiC models/main.tex}
            \input{2 - kinetic component/main.tex}
            \input{3 - fluid component/main.tex}
            \input{4 - numerical implementation/main.tex}
        \part{Project Overview}
            \input{5 - research plan/main.tex}
            \input{6 - summary/main.tex}
    
    
    %\section{}
    \newpage
    \pagenumbering{gobble}
        \printbibliography


    \newpage
    \pagenumbering{roman}
    \appendix
        \part{Appendices}
            \input{8 - Hilbert complexes/main.tex}
            \input{9 - weak conservation proofs/main.tex}
\end{document}

    
    
    %\section{}
    \newpage
    \pagenumbering{gobble}
        \printbibliography


    \newpage
    \pagenumbering{roman}
    \appendix
        \part{Appendices}
            \documentclass[12pt, a4paper]{report}

\input{template/main.tex}

\title{\BA{Title in Progress...}}
\author{Boris Andrews}
\affil{Mathematical Institute, University of Oxford}
\date{\today}


\begin{document}
    \pagenumbering{gobble}
    \maketitle
    
    
    \begin{abstract}
        Magnetic confinement reactors---in particular tokamaks---offer one of the most promising options for achieving practical nuclear fusion, with the potential to provide virtually limitless, clean energy. The theoretical and numerical modeling of tokamak plasmas is simultaneously an essential component of effective reactor design, and a great research barrier. Tokamak operational conditions exhibit comparatively low Knudsen numbers. Kinetic effects, including kinetic waves and instabilities, Landau damping, bump-on-tail instabilities and more, are therefore highly influential in tokamak plasma dynamics. Purely fluid models are inherently incapable of capturing these effects, whereas the high dimensionality in purely kinetic models render them practically intractable for most relevant purposes.

        We consider a $\delta\!f$ decomposition model, with a macroscopic fluid background and microscopic kinetic correction, both fully coupled to each other. A similar manner of discretization is proposed to that used in the recent \texttt{STRUPHY} code \cite{Holderied_Possanner_Wang_2021, Holderied_2022, Li_et_al_2023} with a finite-element model for the background and a pseudo-particle/PiC model for the correction.

        The fluid background satisfies the full, non-linear, resistive, compressible, Hall MHD equations. \cite{Laakmann_Hu_Farrell_2022} introduces finite-element(-in-space) implicit timesteppers for the incompressible analogue to this system with structure-preserving (SP) properties in the ideal case, alongside parameter-robust preconditioners. We show that these timesteppers can derive from a finite-element-in-time (FET) (and finite-element-in-space) interpretation. The benefits of this reformulation are discussed, including the derivation of timesteppers that are higher order in time, and the quantifiable dissipative SP properties in the non-ideal, resistive case.
        
        We discuss possible options for extending this FET approach to timesteppers for the compressible case.

        The kinetic corrections satisfy linearized Boltzmann equations. Using a Lénard--Bernstein collision operator, these take Fokker--Planck-like forms \cite{Fokker_1914, Planck_1917} wherein pseudo-particles in the numerical model obey the neoclassical transport equations, with particle-independent Brownian drift terms. This offers a rigorous methodology for incorporating collisions into the particle transport model, without coupling the equations of motions for each particle.
        
        Works by Chen, Chacón et al. \cite{Chen_Chacón_Barnes_2011, Chacón_Chen_Barnes_2013, Chen_Chacón_2014, Chen_Chacón_2015} have developed structure-preserving particle pushers for neoclassical transport in the Vlasov equations, derived from Crank--Nicolson integrators. We show these too can can derive from a FET interpretation, similarly offering potential extensions to higher-order-in-time particle pushers. The FET formulation is used also to consider how the stochastic drift terms can be incorporated into the pushers. Stochastic gyrokinetic expansions are also discussed.

        Different options for the numerical implementation of these schemes are considered.

        Due to the efficacy of FET in the development of SP timesteppers for both the fluid and kinetic component, we hope this approach will prove effective in the future for developing SP timesteppers for the full hybrid model. We hope this will give us the opportunity to incorporate previously inaccessible kinetic effects into the highly effective, modern, finite-element MHD models.
    \end{abstract}
    
    
    \newpage
    \tableofcontents
    
    
    \newpage
    \pagenumbering{arabic}
    %\linenumbers\renewcommand\thelinenumber{\color{black!50}\arabic{linenumber}}
            \input{0 - introduction/main.tex}
        \part{Research}
            \input{1 - low-noise PiC models/main.tex}
            \input{2 - kinetic component/main.tex}
            \input{3 - fluid component/main.tex}
            \input{4 - numerical implementation/main.tex}
        \part{Project Overview}
            \input{5 - research plan/main.tex}
            \input{6 - summary/main.tex}
    
    
    %\section{}
    \newpage
    \pagenumbering{gobble}
        \printbibliography


    \newpage
    \pagenumbering{roman}
    \appendix
        \part{Appendices}
            \input{8 - Hilbert complexes/main.tex}
            \input{9 - weak conservation proofs/main.tex}
\end{document}

            \documentclass[12pt, a4paper]{report}

\input{template/main.tex}

\title{\BA{Title in Progress...}}
\author{Boris Andrews}
\affil{Mathematical Institute, University of Oxford}
\date{\today}


\begin{document}
    \pagenumbering{gobble}
    \maketitle
    
    
    \begin{abstract}
        Magnetic confinement reactors---in particular tokamaks---offer one of the most promising options for achieving practical nuclear fusion, with the potential to provide virtually limitless, clean energy. The theoretical and numerical modeling of tokamak plasmas is simultaneously an essential component of effective reactor design, and a great research barrier. Tokamak operational conditions exhibit comparatively low Knudsen numbers. Kinetic effects, including kinetic waves and instabilities, Landau damping, bump-on-tail instabilities and more, are therefore highly influential in tokamak plasma dynamics. Purely fluid models are inherently incapable of capturing these effects, whereas the high dimensionality in purely kinetic models render them practically intractable for most relevant purposes.

        We consider a $\delta\!f$ decomposition model, with a macroscopic fluid background and microscopic kinetic correction, both fully coupled to each other. A similar manner of discretization is proposed to that used in the recent \texttt{STRUPHY} code \cite{Holderied_Possanner_Wang_2021, Holderied_2022, Li_et_al_2023} with a finite-element model for the background and a pseudo-particle/PiC model for the correction.

        The fluid background satisfies the full, non-linear, resistive, compressible, Hall MHD equations. \cite{Laakmann_Hu_Farrell_2022} introduces finite-element(-in-space) implicit timesteppers for the incompressible analogue to this system with structure-preserving (SP) properties in the ideal case, alongside parameter-robust preconditioners. We show that these timesteppers can derive from a finite-element-in-time (FET) (and finite-element-in-space) interpretation. The benefits of this reformulation are discussed, including the derivation of timesteppers that are higher order in time, and the quantifiable dissipative SP properties in the non-ideal, resistive case.
        
        We discuss possible options for extending this FET approach to timesteppers for the compressible case.

        The kinetic corrections satisfy linearized Boltzmann equations. Using a Lénard--Bernstein collision operator, these take Fokker--Planck-like forms \cite{Fokker_1914, Planck_1917} wherein pseudo-particles in the numerical model obey the neoclassical transport equations, with particle-independent Brownian drift terms. This offers a rigorous methodology for incorporating collisions into the particle transport model, without coupling the equations of motions for each particle.
        
        Works by Chen, Chacón et al. \cite{Chen_Chacón_Barnes_2011, Chacón_Chen_Barnes_2013, Chen_Chacón_2014, Chen_Chacón_2015} have developed structure-preserving particle pushers for neoclassical transport in the Vlasov equations, derived from Crank--Nicolson integrators. We show these too can can derive from a FET interpretation, similarly offering potential extensions to higher-order-in-time particle pushers. The FET formulation is used also to consider how the stochastic drift terms can be incorporated into the pushers. Stochastic gyrokinetic expansions are also discussed.

        Different options for the numerical implementation of these schemes are considered.

        Due to the efficacy of FET in the development of SP timesteppers for both the fluid and kinetic component, we hope this approach will prove effective in the future for developing SP timesteppers for the full hybrid model. We hope this will give us the opportunity to incorporate previously inaccessible kinetic effects into the highly effective, modern, finite-element MHD models.
    \end{abstract}
    
    
    \newpage
    \tableofcontents
    
    
    \newpage
    \pagenumbering{arabic}
    %\linenumbers\renewcommand\thelinenumber{\color{black!50}\arabic{linenumber}}
            \input{0 - introduction/main.tex}
        \part{Research}
            \input{1 - low-noise PiC models/main.tex}
            \input{2 - kinetic component/main.tex}
            \input{3 - fluid component/main.tex}
            \input{4 - numerical implementation/main.tex}
        \part{Project Overview}
            \input{5 - research plan/main.tex}
            \input{6 - summary/main.tex}
    
    
    %\section{}
    \newpage
    \pagenumbering{gobble}
        \printbibliography


    \newpage
    \pagenumbering{roman}
    \appendix
        \part{Appendices}
            \input{8 - Hilbert complexes/main.tex}
            \input{9 - weak conservation proofs/main.tex}
\end{document}

\end{document}

\end{document}

    \documentclass[12pt, a4paper]{report}

\documentclass[12pt, a4paper]{report}

\documentclass[12pt, a4paper]{report}

\input{template/main.tex}

\title{\BA{Title in Progress...}}
\author{Boris Andrews}
\affil{Mathematical Institute, University of Oxford}
\date{\today}


\begin{document}
    \pagenumbering{gobble}
    \maketitle
    
    
    \begin{abstract}
        Magnetic confinement reactors---in particular tokamaks---offer one of the most promising options for achieving practical nuclear fusion, with the potential to provide virtually limitless, clean energy. The theoretical and numerical modeling of tokamak plasmas is simultaneously an essential component of effective reactor design, and a great research barrier. Tokamak operational conditions exhibit comparatively low Knudsen numbers. Kinetic effects, including kinetic waves and instabilities, Landau damping, bump-on-tail instabilities and more, are therefore highly influential in tokamak plasma dynamics. Purely fluid models are inherently incapable of capturing these effects, whereas the high dimensionality in purely kinetic models render them practically intractable for most relevant purposes.

        We consider a $\delta\!f$ decomposition model, with a macroscopic fluid background and microscopic kinetic correction, both fully coupled to each other. A similar manner of discretization is proposed to that used in the recent \texttt{STRUPHY} code \cite{Holderied_Possanner_Wang_2021, Holderied_2022, Li_et_al_2023} with a finite-element model for the background and a pseudo-particle/PiC model for the correction.

        The fluid background satisfies the full, non-linear, resistive, compressible, Hall MHD equations. \cite{Laakmann_Hu_Farrell_2022} introduces finite-element(-in-space) implicit timesteppers for the incompressible analogue to this system with structure-preserving (SP) properties in the ideal case, alongside parameter-robust preconditioners. We show that these timesteppers can derive from a finite-element-in-time (FET) (and finite-element-in-space) interpretation. The benefits of this reformulation are discussed, including the derivation of timesteppers that are higher order in time, and the quantifiable dissipative SP properties in the non-ideal, resistive case.
        
        We discuss possible options for extending this FET approach to timesteppers for the compressible case.

        The kinetic corrections satisfy linearized Boltzmann equations. Using a Lénard--Bernstein collision operator, these take Fokker--Planck-like forms \cite{Fokker_1914, Planck_1917} wherein pseudo-particles in the numerical model obey the neoclassical transport equations, with particle-independent Brownian drift terms. This offers a rigorous methodology for incorporating collisions into the particle transport model, without coupling the equations of motions for each particle.
        
        Works by Chen, Chacón et al. \cite{Chen_Chacón_Barnes_2011, Chacón_Chen_Barnes_2013, Chen_Chacón_2014, Chen_Chacón_2015} have developed structure-preserving particle pushers for neoclassical transport in the Vlasov equations, derived from Crank--Nicolson integrators. We show these too can can derive from a FET interpretation, similarly offering potential extensions to higher-order-in-time particle pushers. The FET formulation is used also to consider how the stochastic drift terms can be incorporated into the pushers. Stochastic gyrokinetic expansions are also discussed.

        Different options for the numerical implementation of these schemes are considered.

        Due to the efficacy of FET in the development of SP timesteppers for both the fluid and kinetic component, we hope this approach will prove effective in the future for developing SP timesteppers for the full hybrid model. We hope this will give us the opportunity to incorporate previously inaccessible kinetic effects into the highly effective, modern, finite-element MHD models.
    \end{abstract}
    
    
    \newpage
    \tableofcontents
    
    
    \newpage
    \pagenumbering{arabic}
    %\linenumbers\renewcommand\thelinenumber{\color{black!50}\arabic{linenumber}}
            \input{0 - introduction/main.tex}
        \part{Research}
            \input{1 - low-noise PiC models/main.tex}
            \input{2 - kinetic component/main.tex}
            \input{3 - fluid component/main.tex}
            \input{4 - numerical implementation/main.tex}
        \part{Project Overview}
            \input{5 - research plan/main.tex}
            \input{6 - summary/main.tex}
    
    
    %\section{}
    \newpage
    \pagenumbering{gobble}
        \printbibliography


    \newpage
    \pagenumbering{roman}
    \appendix
        \part{Appendices}
            \input{8 - Hilbert complexes/main.tex}
            \input{9 - weak conservation proofs/main.tex}
\end{document}


\title{\BA{Title in Progress...}}
\author{Boris Andrews}
\affil{Mathematical Institute, University of Oxford}
\date{\today}


\begin{document}
    \pagenumbering{gobble}
    \maketitle
    
    
    \begin{abstract}
        Magnetic confinement reactors---in particular tokamaks---offer one of the most promising options for achieving practical nuclear fusion, with the potential to provide virtually limitless, clean energy. The theoretical and numerical modeling of tokamak plasmas is simultaneously an essential component of effective reactor design, and a great research barrier. Tokamak operational conditions exhibit comparatively low Knudsen numbers. Kinetic effects, including kinetic waves and instabilities, Landau damping, bump-on-tail instabilities and more, are therefore highly influential in tokamak plasma dynamics. Purely fluid models are inherently incapable of capturing these effects, whereas the high dimensionality in purely kinetic models render them practically intractable for most relevant purposes.

        We consider a $\delta\!f$ decomposition model, with a macroscopic fluid background and microscopic kinetic correction, both fully coupled to each other. A similar manner of discretization is proposed to that used in the recent \texttt{STRUPHY} code \cite{Holderied_Possanner_Wang_2021, Holderied_2022, Li_et_al_2023} with a finite-element model for the background and a pseudo-particle/PiC model for the correction.

        The fluid background satisfies the full, non-linear, resistive, compressible, Hall MHD equations. \cite{Laakmann_Hu_Farrell_2022} introduces finite-element(-in-space) implicit timesteppers for the incompressible analogue to this system with structure-preserving (SP) properties in the ideal case, alongside parameter-robust preconditioners. We show that these timesteppers can derive from a finite-element-in-time (FET) (and finite-element-in-space) interpretation. The benefits of this reformulation are discussed, including the derivation of timesteppers that are higher order in time, and the quantifiable dissipative SP properties in the non-ideal, resistive case.
        
        We discuss possible options for extending this FET approach to timesteppers for the compressible case.

        The kinetic corrections satisfy linearized Boltzmann equations. Using a Lénard--Bernstein collision operator, these take Fokker--Planck-like forms \cite{Fokker_1914, Planck_1917} wherein pseudo-particles in the numerical model obey the neoclassical transport equations, with particle-independent Brownian drift terms. This offers a rigorous methodology for incorporating collisions into the particle transport model, without coupling the equations of motions for each particle.
        
        Works by Chen, Chacón et al. \cite{Chen_Chacón_Barnes_2011, Chacón_Chen_Barnes_2013, Chen_Chacón_2014, Chen_Chacón_2015} have developed structure-preserving particle pushers for neoclassical transport in the Vlasov equations, derived from Crank--Nicolson integrators. We show these too can can derive from a FET interpretation, similarly offering potential extensions to higher-order-in-time particle pushers. The FET formulation is used also to consider how the stochastic drift terms can be incorporated into the pushers. Stochastic gyrokinetic expansions are also discussed.

        Different options for the numerical implementation of these schemes are considered.

        Due to the efficacy of FET in the development of SP timesteppers for both the fluid and kinetic component, we hope this approach will prove effective in the future for developing SP timesteppers for the full hybrid model. We hope this will give us the opportunity to incorporate previously inaccessible kinetic effects into the highly effective, modern, finite-element MHD models.
    \end{abstract}
    
    
    \newpage
    \tableofcontents
    
    
    \newpage
    \pagenumbering{arabic}
    %\linenumbers\renewcommand\thelinenumber{\color{black!50}\arabic{linenumber}}
            \documentclass[12pt, a4paper]{report}

\input{template/main.tex}

\title{\BA{Title in Progress...}}
\author{Boris Andrews}
\affil{Mathematical Institute, University of Oxford}
\date{\today}


\begin{document}
    \pagenumbering{gobble}
    \maketitle
    
    
    \begin{abstract}
        Magnetic confinement reactors---in particular tokamaks---offer one of the most promising options for achieving practical nuclear fusion, with the potential to provide virtually limitless, clean energy. The theoretical and numerical modeling of tokamak plasmas is simultaneously an essential component of effective reactor design, and a great research barrier. Tokamak operational conditions exhibit comparatively low Knudsen numbers. Kinetic effects, including kinetic waves and instabilities, Landau damping, bump-on-tail instabilities and more, are therefore highly influential in tokamak plasma dynamics. Purely fluid models are inherently incapable of capturing these effects, whereas the high dimensionality in purely kinetic models render them practically intractable for most relevant purposes.

        We consider a $\delta\!f$ decomposition model, with a macroscopic fluid background and microscopic kinetic correction, both fully coupled to each other. A similar manner of discretization is proposed to that used in the recent \texttt{STRUPHY} code \cite{Holderied_Possanner_Wang_2021, Holderied_2022, Li_et_al_2023} with a finite-element model for the background and a pseudo-particle/PiC model for the correction.

        The fluid background satisfies the full, non-linear, resistive, compressible, Hall MHD equations. \cite{Laakmann_Hu_Farrell_2022} introduces finite-element(-in-space) implicit timesteppers for the incompressible analogue to this system with structure-preserving (SP) properties in the ideal case, alongside parameter-robust preconditioners. We show that these timesteppers can derive from a finite-element-in-time (FET) (and finite-element-in-space) interpretation. The benefits of this reformulation are discussed, including the derivation of timesteppers that are higher order in time, and the quantifiable dissipative SP properties in the non-ideal, resistive case.
        
        We discuss possible options for extending this FET approach to timesteppers for the compressible case.

        The kinetic corrections satisfy linearized Boltzmann equations. Using a Lénard--Bernstein collision operator, these take Fokker--Planck-like forms \cite{Fokker_1914, Planck_1917} wherein pseudo-particles in the numerical model obey the neoclassical transport equations, with particle-independent Brownian drift terms. This offers a rigorous methodology for incorporating collisions into the particle transport model, without coupling the equations of motions for each particle.
        
        Works by Chen, Chacón et al. \cite{Chen_Chacón_Barnes_2011, Chacón_Chen_Barnes_2013, Chen_Chacón_2014, Chen_Chacón_2015} have developed structure-preserving particle pushers for neoclassical transport in the Vlasov equations, derived from Crank--Nicolson integrators. We show these too can can derive from a FET interpretation, similarly offering potential extensions to higher-order-in-time particle pushers. The FET formulation is used also to consider how the stochastic drift terms can be incorporated into the pushers. Stochastic gyrokinetic expansions are also discussed.

        Different options for the numerical implementation of these schemes are considered.

        Due to the efficacy of FET in the development of SP timesteppers for both the fluid and kinetic component, we hope this approach will prove effective in the future for developing SP timesteppers for the full hybrid model. We hope this will give us the opportunity to incorporate previously inaccessible kinetic effects into the highly effective, modern, finite-element MHD models.
    \end{abstract}
    
    
    \newpage
    \tableofcontents
    
    
    \newpage
    \pagenumbering{arabic}
    %\linenumbers\renewcommand\thelinenumber{\color{black!50}\arabic{linenumber}}
            \input{0 - introduction/main.tex}
        \part{Research}
            \input{1 - low-noise PiC models/main.tex}
            \input{2 - kinetic component/main.tex}
            \input{3 - fluid component/main.tex}
            \input{4 - numerical implementation/main.tex}
        \part{Project Overview}
            \input{5 - research plan/main.tex}
            \input{6 - summary/main.tex}
    
    
    %\section{}
    \newpage
    \pagenumbering{gobble}
        \printbibliography


    \newpage
    \pagenumbering{roman}
    \appendix
        \part{Appendices}
            \input{8 - Hilbert complexes/main.tex}
            \input{9 - weak conservation proofs/main.tex}
\end{document}

        \part{Research}
            \documentclass[12pt, a4paper]{report}

\input{template/main.tex}

\title{\BA{Title in Progress...}}
\author{Boris Andrews}
\affil{Mathematical Institute, University of Oxford}
\date{\today}


\begin{document}
    \pagenumbering{gobble}
    \maketitle
    
    
    \begin{abstract}
        Magnetic confinement reactors---in particular tokamaks---offer one of the most promising options for achieving practical nuclear fusion, with the potential to provide virtually limitless, clean energy. The theoretical and numerical modeling of tokamak plasmas is simultaneously an essential component of effective reactor design, and a great research barrier. Tokamak operational conditions exhibit comparatively low Knudsen numbers. Kinetic effects, including kinetic waves and instabilities, Landau damping, bump-on-tail instabilities and more, are therefore highly influential in tokamak plasma dynamics. Purely fluid models are inherently incapable of capturing these effects, whereas the high dimensionality in purely kinetic models render them practically intractable for most relevant purposes.

        We consider a $\delta\!f$ decomposition model, with a macroscopic fluid background and microscopic kinetic correction, both fully coupled to each other. A similar manner of discretization is proposed to that used in the recent \texttt{STRUPHY} code \cite{Holderied_Possanner_Wang_2021, Holderied_2022, Li_et_al_2023} with a finite-element model for the background and a pseudo-particle/PiC model for the correction.

        The fluid background satisfies the full, non-linear, resistive, compressible, Hall MHD equations. \cite{Laakmann_Hu_Farrell_2022} introduces finite-element(-in-space) implicit timesteppers for the incompressible analogue to this system with structure-preserving (SP) properties in the ideal case, alongside parameter-robust preconditioners. We show that these timesteppers can derive from a finite-element-in-time (FET) (and finite-element-in-space) interpretation. The benefits of this reformulation are discussed, including the derivation of timesteppers that are higher order in time, and the quantifiable dissipative SP properties in the non-ideal, resistive case.
        
        We discuss possible options for extending this FET approach to timesteppers for the compressible case.

        The kinetic corrections satisfy linearized Boltzmann equations. Using a Lénard--Bernstein collision operator, these take Fokker--Planck-like forms \cite{Fokker_1914, Planck_1917} wherein pseudo-particles in the numerical model obey the neoclassical transport equations, with particle-independent Brownian drift terms. This offers a rigorous methodology for incorporating collisions into the particle transport model, without coupling the equations of motions for each particle.
        
        Works by Chen, Chacón et al. \cite{Chen_Chacón_Barnes_2011, Chacón_Chen_Barnes_2013, Chen_Chacón_2014, Chen_Chacón_2015} have developed structure-preserving particle pushers for neoclassical transport in the Vlasov equations, derived from Crank--Nicolson integrators. We show these too can can derive from a FET interpretation, similarly offering potential extensions to higher-order-in-time particle pushers. The FET formulation is used also to consider how the stochastic drift terms can be incorporated into the pushers. Stochastic gyrokinetic expansions are also discussed.

        Different options for the numerical implementation of these schemes are considered.

        Due to the efficacy of FET in the development of SP timesteppers for both the fluid and kinetic component, we hope this approach will prove effective in the future for developing SP timesteppers for the full hybrid model. We hope this will give us the opportunity to incorporate previously inaccessible kinetic effects into the highly effective, modern, finite-element MHD models.
    \end{abstract}
    
    
    \newpage
    \tableofcontents
    
    
    \newpage
    \pagenumbering{arabic}
    %\linenumbers\renewcommand\thelinenumber{\color{black!50}\arabic{linenumber}}
            \input{0 - introduction/main.tex}
        \part{Research}
            \input{1 - low-noise PiC models/main.tex}
            \input{2 - kinetic component/main.tex}
            \input{3 - fluid component/main.tex}
            \input{4 - numerical implementation/main.tex}
        \part{Project Overview}
            \input{5 - research plan/main.tex}
            \input{6 - summary/main.tex}
    
    
    %\section{}
    \newpage
    \pagenumbering{gobble}
        \printbibliography


    \newpage
    \pagenumbering{roman}
    \appendix
        \part{Appendices}
            \input{8 - Hilbert complexes/main.tex}
            \input{9 - weak conservation proofs/main.tex}
\end{document}

            \documentclass[12pt, a4paper]{report}

\input{template/main.tex}

\title{\BA{Title in Progress...}}
\author{Boris Andrews}
\affil{Mathematical Institute, University of Oxford}
\date{\today}


\begin{document}
    \pagenumbering{gobble}
    \maketitle
    
    
    \begin{abstract}
        Magnetic confinement reactors---in particular tokamaks---offer one of the most promising options for achieving practical nuclear fusion, with the potential to provide virtually limitless, clean energy. The theoretical and numerical modeling of tokamak plasmas is simultaneously an essential component of effective reactor design, and a great research barrier. Tokamak operational conditions exhibit comparatively low Knudsen numbers. Kinetic effects, including kinetic waves and instabilities, Landau damping, bump-on-tail instabilities and more, are therefore highly influential in tokamak plasma dynamics. Purely fluid models are inherently incapable of capturing these effects, whereas the high dimensionality in purely kinetic models render them practically intractable for most relevant purposes.

        We consider a $\delta\!f$ decomposition model, with a macroscopic fluid background and microscopic kinetic correction, both fully coupled to each other. A similar manner of discretization is proposed to that used in the recent \texttt{STRUPHY} code \cite{Holderied_Possanner_Wang_2021, Holderied_2022, Li_et_al_2023} with a finite-element model for the background and a pseudo-particle/PiC model for the correction.

        The fluid background satisfies the full, non-linear, resistive, compressible, Hall MHD equations. \cite{Laakmann_Hu_Farrell_2022} introduces finite-element(-in-space) implicit timesteppers for the incompressible analogue to this system with structure-preserving (SP) properties in the ideal case, alongside parameter-robust preconditioners. We show that these timesteppers can derive from a finite-element-in-time (FET) (and finite-element-in-space) interpretation. The benefits of this reformulation are discussed, including the derivation of timesteppers that are higher order in time, and the quantifiable dissipative SP properties in the non-ideal, resistive case.
        
        We discuss possible options for extending this FET approach to timesteppers for the compressible case.

        The kinetic corrections satisfy linearized Boltzmann equations. Using a Lénard--Bernstein collision operator, these take Fokker--Planck-like forms \cite{Fokker_1914, Planck_1917} wherein pseudo-particles in the numerical model obey the neoclassical transport equations, with particle-independent Brownian drift terms. This offers a rigorous methodology for incorporating collisions into the particle transport model, without coupling the equations of motions for each particle.
        
        Works by Chen, Chacón et al. \cite{Chen_Chacón_Barnes_2011, Chacón_Chen_Barnes_2013, Chen_Chacón_2014, Chen_Chacón_2015} have developed structure-preserving particle pushers for neoclassical transport in the Vlasov equations, derived from Crank--Nicolson integrators. We show these too can can derive from a FET interpretation, similarly offering potential extensions to higher-order-in-time particle pushers. The FET formulation is used also to consider how the stochastic drift terms can be incorporated into the pushers. Stochastic gyrokinetic expansions are also discussed.

        Different options for the numerical implementation of these schemes are considered.

        Due to the efficacy of FET in the development of SP timesteppers for both the fluid and kinetic component, we hope this approach will prove effective in the future for developing SP timesteppers for the full hybrid model. We hope this will give us the opportunity to incorporate previously inaccessible kinetic effects into the highly effective, modern, finite-element MHD models.
    \end{abstract}
    
    
    \newpage
    \tableofcontents
    
    
    \newpage
    \pagenumbering{arabic}
    %\linenumbers\renewcommand\thelinenumber{\color{black!50}\arabic{linenumber}}
            \input{0 - introduction/main.tex}
        \part{Research}
            \input{1 - low-noise PiC models/main.tex}
            \input{2 - kinetic component/main.tex}
            \input{3 - fluid component/main.tex}
            \input{4 - numerical implementation/main.tex}
        \part{Project Overview}
            \input{5 - research plan/main.tex}
            \input{6 - summary/main.tex}
    
    
    %\section{}
    \newpage
    \pagenumbering{gobble}
        \printbibliography


    \newpage
    \pagenumbering{roman}
    \appendix
        \part{Appendices}
            \input{8 - Hilbert complexes/main.tex}
            \input{9 - weak conservation proofs/main.tex}
\end{document}

            \documentclass[12pt, a4paper]{report}

\input{template/main.tex}

\title{\BA{Title in Progress...}}
\author{Boris Andrews}
\affil{Mathematical Institute, University of Oxford}
\date{\today}


\begin{document}
    \pagenumbering{gobble}
    \maketitle
    
    
    \begin{abstract}
        Magnetic confinement reactors---in particular tokamaks---offer one of the most promising options for achieving practical nuclear fusion, with the potential to provide virtually limitless, clean energy. The theoretical and numerical modeling of tokamak plasmas is simultaneously an essential component of effective reactor design, and a great research barrier. Tokamak operational conditions exhibit comparatively low Knudsen numbers. Kinetic effects, including kinetic waves and instabilities, Landau damping, bump-on-tail instabilities and more, are therefore highly influential in tokamak plasma dynamics. Purely fluid models are inherently incapable of capturing these effects, whereas the high dimensionality in purely kinetic models render them practically intractable for most relevant purposes.

        We consider a $\delta\!f$ decomposition model, with a macroscopic fluid background and microscopic kinetic correction, both fully coupled to each other. A similar manner of discretization is proposed to that used in the recent \texttt{STRUPHY} code \cite{Holderied_Possanner_Wang_2021, Holderied_2022, Li_et_al_2023} with a finite-element model for the background and a pseudo-particle/PiC model for the correction.

        The fluid background satisfies the full, non-linear, resistive, compressible, Hall MHD equations. \cite{Laakmann_Hu_Farrell_2022} introduces finite-element(-in-space) implicit timesteppers for the incompressible analogue to this system with structure-preserving (SP) properties in the ideal case, alongside parameter-robust preconditioners. We show that these timesteppers can derive from a finite-element-in-time (FET) (and finite-element-in-space) interpretation. The benefits of this reformulation are discussed, including the derivation of timesteppers that are higher order in time, and the quantifiable dissipative SP properties in the non-ideal, resistive case.
        
        We discuss possible options for extending this FET approach to timesteppers for the compressible case.

        The kinetic corrections satisfy linearized Boltzmann equations. Using a Lénard--Bernstein collision operator, these take Fokker--Planck-like forms \cite{Fokker_1914, Planck_1917} wherein pseudo-particles in the numerical model obey the neoclassical transport equations, with particle-independent Brownian drift terms. This offers a rigorous methodology for incorporating collisions into the particle transport model, without coupling the equations of motions for each particle.
        
        Works by Chen, Chacón et al. \cite{Chen_Chacón_Barnes_2011, Chacón_Chen_Barnes_2013, Chen_Chacón_2014, Chen_Chacón_2015} have developed structure-preserving particle pushers for neoclassical transport in the Vlasov equations, derived from Crank--Nicolson integrators. We show these too can can derive from a FET interpretation, similarly offering potential extensions to higher-order-in-time particle pushers. The FET formulation is used also to consider how the stochastic drift terms can be incorporated into the pushers. Stochastic gyrokinetic expansions are also discussed.

        Different options for the numerical implementation of these schemes are considered.

        Due to the efficacy of FET in the development of SP timesteppers for both the fluid and kinetic component, we hope this approach will prove effective in the future for developing SP timesteppers for the full hybrid model. We hope this will give us the opportunity to incorporate previously inaccessible kinetic effects into the highly effective, modern, finite-element MHD models.
    \end{abstract}
    
    
    \newpage
    \tableofcontents
    
    
    \newpage
    \pagenumbering{arabic}
    %\linenumbers\renewcommand\thelinenumber{\color{black!50}\arabic{linenumber}}
            \input{0 - introduction/main.tex}
        \part{Research}
            \input{1 - low-noise PiC models/main.tex}
            \input{2 - kinetic component/main.tex}
            \input{3 - fluid component/main.tex}
            \input{4 - numerical implementation/main.tex}
        \part{Project Overview}
            \input{5 - research plan/main.tex}
            \input{6 - summary/main.tex}
    
    
    %\section{}
    \newpage
    \pagenumbering{gobble}
        \printbibliography


    \newpage
    \pagenumbering{roman}
    \appendix
        \part{Appendices}
            \input{8 - Hilbert complexes/main.tex}
            \input{9 - weak conservation proofs/main.tex}
\end{document}

            \documentclass[12pt, a4paper]{report}

\input{template/main.tex}

\title{\BA{Title in Progress...}}
\author{Boris Andrews}
\affil{Mathematical Institute, University of Oxford}
\date{\today}


\begin{document}
    \pagenumbering{gobble}
    \maketitle
    
    
    \begin{abstract}
        Magnetic confinement reactors---in particular tokamaks---offer one of the most promising options for achieving practical nuclear fusion, with the potential to provide virtually limitless, clean energy. The theoretical and numerical modeling of tokamak plasmas is simultaneously an essential component of effective reactor design, and a great research barrier. Tokamak operational conditions exhibit comparatively low Knudsen numbers. Kinetic effects, including kinetic waves and instabilities, Landau damping, bump-on-tail instabilities and more, are therefore highly influential in tokamak plasma dynamics. Purely fluid models are inherently incapable of capturing these effects, whereas the high dimensionality in purely kinetic models render them practically intractable for most relevant purposes.

        We consider a $\delta\!f$ decomposition model, with a macroscopic fluid background and microscopic kinetic correction, both fully coupled to each other. A similar manner of discretization is proposed to that used in the recent \texttt{STRUPHY} code \cite{Holderied_Possanner_Wang_2021, Holderied_2022, Li_et_al_2023} with a finite-element model for the background and a pseudo-particle/PiC model for the correction.

        The fluid background satisfies the full, non-linear, resistive, compressible, Hall MHD equations. \cite{Laakmann_Hu_Farrell_2022} introduces finite-element(-in-space) implicit timesteppers for the incompressible analogue to this system with structure-preserving (SP) properties in the ideal case, alongside parameter-robust preconditioners. We show that these timesteppers can derive from a finite-element-in-time (FET) (and finite-element-in-space) interpretation. The benefits of this reformulation are discussed, including the derivation of timesteppers that are higher order in time, and the quantifiable dissipative SP properties in the non-ideal, resistive case.
        
        We discuss possible options for extending this FET approach to timesteppers for the compressible case.

        The kinetic corrections satisfy linearized Boltzmann equations. Using a Lénard--Bernstein collision operator, these take Fokker--Planck-like forms \cite{Fokker_1914, Planck_1917} wherein pseudo-particles in the numerical model obey the neoclassical transport equations, with particle-independent Brownian drift terms. This offers a rigorous methodology for incorporating collisions into the particle transport model, without coupling the equations of motions for each particle.
        
        Works by Chen, Chacón et al. \cite{Chen_Chacón_Barnes_2011, Chacón_Chen_Barnes_2013, Chen_Chacón_2014, Chen_Chacón_2015} have developed structure-preserving particle pushers for neoclassical transport in the Vlasov equations, derived from Crank--Nicolson integrators. We show these too can can derive from a FET interpretation, similarly offering potential extensions to higher-order-in-time particle pushers. The FET formulation is used also to consider how the stochastic drift terms can be incorporated into the pushers. Stochastic gyrokinetic expansions are also discussed.

        Different options for the numerical implementation of these schemes are considered.

        Due to the efficacy of FET in the development of SP timesteppers for both the fluid and kinetic component, we hope this approach will prove effective in the future for developing SP timesteppers for the full hybrid model. We hope this will give us the opportunity to incorporate previously inaccessible kinetic effects into the highly effective, modern, finite-element MHD models.
    \end{abstract}
    
    
    \newpage
    \tableofcontents
    
    
    \newpage
    \pagenumbering{arabic}
    %\linenumbers\renewcommand\thelinenumber{\color{black!50}\arabic{linenumber}}
            \input{0 - introduction/main.tex}
        \part{Research}
            \input{1 - low-noise PiC models/main.tex}
            \input{2 - kinetic component/main.tex}
            \input{3 - fluid component/main.tex}
            \input{4 - numerical implementation/main.tex}
        \part{Project Overview}
            \input{5 - research plan/main.tex}
            \input{6 - summary/main.tex}
    
    
    %\section{}
    \newpage
    \pagenumbering{gobble}
        \printbibliography


    \newpage
    \pagenumbering{roman}
    \appendix
        \part{Appendices}
            \input{8 - Hilbert complexes/main.tex}
            \input{9 - weak conservation proofs/main.tex}
\end{document}

        \part{Project Overview}
            \documentclass[12pt, a4paper]{report}

\input{template/main.tex}

\title{\BA{Title in Progress...}}
\author{Boris Andrews}
\affil{Mathematical Institute, University of Oxford}
\date{\today}


\begin{document}
    \pagenumbering{gobble}
    \maketitle
    
    
    \begin{abstract}
        Magnetic confinement reactors---in particular tokamaks---offer one of the most promising options for achieving practical nuclear fusion, with the potential to provide virtually limitless, clean energy. The theoretical and numerical modeling of tokamak plasmas is simultaneously an essential component of effective reactor design, and a great research barrier. Tokamak operational conditions exhibit comparatively low Knudsen numbers. Kinetic effects, including kinetic waves and instabilities, Landau damping, bump-on-tail instabilities and more, are therefore highly influential in tokamak plasma dynamics. Purely fluid models are inherently incapable of capturing these effects, whereas the high dimensionality in purely kinetic models render them practically intractable for most relevant purposes.

        We consider a $\delta\!f$ decomposition model, with a macroscopic fluid background and microscopic kinetic correction, both fully coupled to each other. A similar manner of discretization is proposed to that used in the recent \texttt{STRUPHY} code \cite{Holderied_Possanner_Wang_2021, Holderied_2022, Li_et_al_2023} with a finite-element model for the background and a pseudo-particle/PiC model for the correction.

        The fluid background satisfies the full, non-linear, resistive, compressible, Hall MHD equations. \cite{Laakmann_Hu_Farrell_2022} introduces finite-element(-in-space) implicit timesteppers for the incompressible analogue to this system with structure-preserving (SP) properties in the ideal case, alongside parameter-robust preconditioners. We show that these timesteppers can derive from a finite-element-in-time (FET) (and finite-element-in-space) interpretation. The benefits of this reformulation are discussed, including the derivation of timesteppers that are higher order in time, and the quantifiable dissipative SP properties in the non-ideal, resistive case.
        
        We discuss possible options for extending this FET approach to timesteppers for the compressible case.

        The kinetic corrections satisfy linearized Boltzmann equations. Using a Lénard--Bernstein collision operator, these take Fokker--Planck-like forms \cite{Fokker_1914, Planck_1917} wherein pseudo-particles in the numerical model obey the neoclassical transport equations, with particle-independent Brownian drift terms. This offers a rigorous methodology for incorporating collisions into the particle transport model, without coupling the equations of motions for each particle.
        
        Works by Chen, Chacón et al. \cite{Chen_Chacón_Barnes_2011, Chacón_Chen_Barnes_2013, Chen_Chacón_2014, Chen_Chacón_2015} have developed structure-preserving particle pushers for neoclassical transport in the Vlasov equations, derived from Crank--Nicolson integrators. We show these too can can derive from a FET interpretation, similarly offering potential extensions to higher-order-in-time particle pushers. The FET formulation is used also to consider how the stochastic drift terms can be incorporated into the pushers. Stochastic gyrokinetic expansions are also discussed.

        Different options for the numerical implementation of these schemes are considered.

        Due to the efficacy of FET in the development of SP timesteppers for both the fluid and kinetic component, we hope this approach will prove effective in the future for developing SP timesteppers for the full hybrid model. We hope this will give us the opportunity to incorporate previously inaccessible kinetic effects into the highly effective, modern, finite-element MHD models.
    \end{abstract}
    
    
    \newpage
    \tableofcontents
    
    
    \newpage
    \pagenumbering{arabic}
    %\linenumbers\renewcommand\thelinenumber{\color{black!50}\arabic{linenumber}}
            \input{0 - introduction/main.tex}
        \part{Research}
            \input{1 - low-noise PiC models/main.tex}
            \input{2 - kinetic component/main.tex}
            \input{3 - fluid component/main.tex}
            \input{4 - numerical implementation/main.tex}
        \part{Project Overview}
            \input{5 - research plan/main.tex}
            \input{6 - summary/main.tex}
    
    
    %\section{}
    \newpage
    \pagenumbering{gobble}
        \printbibliography


    \newpage
    \pagenumbering{roman}
    \appendix
        \part{Appendices}
            \input{8 - Hilbert complexes/main.tex}
            \input{9 - weak conservation proofs/main.tex}
\end{document}

            \documentclass[12pt, a4paper]{report}

\input{template/main.tex}

\title{\BA{Title in Progress...}}
\author{Boris Andrews}
\affil{Mathematical Institute, University of Oxford}
\date{\today}


\begin{document}
    \pagenumbering{gobble}
    \maketitle
    
    
    \begin{abstract}
        Magnetic confinement reactors---in particular tokamaks---offer one of the most promising options for achieving practical nuclear fusion, with the potential to provide virtually limitless, clean energy. The theoretical and numerical modeling of tokamak plasmas is simultaneously an essential component of effective reactor design, and a great research barrier. Tokamak operational conditions exhibit comparatively low Knudsen numbers. Kinetic effects, including kinetic waves and instabilities, Landau damping, bump-on-tail instabilities and more, are therefore highly influential in tokamak plasma dynamics. Purely fluid models are inherently incapable of capturing these effects, whereas the high dimensionality in purely kinetic models render them practically intractable for most relevant purposes.

        We consider a $\delta\!f$ decomposition model, with a macroscopic fluid background and microscopic kinetic correction, both fully coupled to each other. A similar manner of discretization is proposed to that used in the recent \texttt{STRUPHY} code \cite{Holderied_Possanner_Wang_2021, Holderied_2022, Li_et_al_2023} with a finite-element model for the background and a pseudo-particle/PiC model for the correction.

        The fluid background satisfies the full, non-linear, resistive, compressible, Hall MHD equations. \cite{Laakmann_Hu_Farrell_2022} introduces finite-element(-in-space) implicit timesteppers for the incompressible analogue to this system with structure-preserving (SP) properties in the ideal case, alongside parameter-robust preconditioners. We show that these timesteppers can derive from a finite-element-in-time (FET) (and finite-element-in-space) interpretation. The benefits of this reformulation are discussed, including the derivation of timesteppers that are higher order in time, and the quantifiable dissipative SP properties in the non-ideal, resistive case.
        
        We discuss possible options for extending this FET approach to timesteppers for the compressible case.

        The kinetic corrections satisfy linearized Boltzmann equations. Using a Lénard--Bernstein collision operator, these take Fokker--Planck-like forms \cite{Fokker_1914, Planck_1917} wherein pseudo-particles in the numerical model obey the neoclassical transport equations, with particle-independent Brownian drift terms. This offers a rigorous methodology for incorporating collisions into the particle transport model, without coupling the equations of motions for each particle.
        
        Works by Chen, Chacón et al. \cite{Chen_Chacón_Barnes_2011, Chacón_Chen_Barnes_2013, Chen_Chacón_2014, Chen_Chacón_2015} have developed structure-preserving particle pushers for neoclassical transport in the Vlasov equations, derived from Crank--Nicolson integrators. We show these too can can derive from a FET interpretation, similarly offering potential extensions to higher-order-in-time particle pushers. The FET formulation is used also to consider how the stochastic drift terms can be incorporated into the pushers. Stochastic gyrokinetic expansions are also discussed.

        Different options for the numerical implementation of these schemes are considered.

        Due to the efficacy of FET in the development of SP timesteppers for both the fluid and kinetic component, we hope this approach will prove effective in the future for developing SP timesteppers for the full hybrid model. We hope this will give us the opportunity to incorporate previously inaccessible kinetic effects into the highly effective, modern, finite-element MHD models.
    \end{abstract}
    
    
    \newpage
    \tableofcontents
    
    
    \newpage
    \pagenumbering{arabic}
    %\linenumbers\renewcommand\thelinenumber{\color{black!50}\arabic{linenumber}}
            \input{0 - introduction/main.tex}
        \part{Research}
            \input{1 - low-noise PiC models/main.tex}
            \input{2 - kinetic component/main.tex}
            \input{3 - fluid component/main.tex}
            \input{4 - numerical implementation/main.tex}
        \part{Project Overview}
            \input{5 - research plan/main.tex}
            \input{6 - summary/main.tex}
    
    
    %\section{}
    \newpage
    \pagenumbering{gobble}
        \printbibliography


    \newpage
    \pagenumbering{roman}
    \appendix
        \part{Appendices}
            \input{8 - Hilbert complexes/main.tex}
            \input{9 - weak conservation proofs/main.tex}
\end{document}

    
    
    %\section{}
    \newpage
    \pagenumbering{gobble}
        \printbibliography


    \newpage
    \pagenumbering{roman}
    \appendix
        \part{Appendices}
            \documentclass[12pt, a4paper]{report}

\input{template/main.tex}

\title{\BA{Title in Progress...}}
\author{Boris Andrews}
\affil{Mathematical Institute, University of Oxford}
\date{\today}


\begin{document}
    \pagenumbering{gobble}
    \maketitle
    
    
    \begin{abstract}
        Magnetic confinement reactors---in particular tokamaks---offer one of the most promising options for achieving practical nuclear fusion, with the potential to provide virtually limitless, clean energy. The theoretical and numerical modeling of tokamak plasmas is simultaneously an essential component of effective reactor design, and a great research barrier. Tokamak operational conditions exhibit comparatively low Knudsen numbers. Kinetic effects, including kinetic waves and instabilities, Landau damping, bump-on-tail instabilities and more, are therefore highly influential in tokamak plasma dynamics. Purely fluid models are inherently incapable of capturing these effects, whereas the high dimensionality in purely kinetic models render them practically intractable for most relevant purposes.

        We consider a $\delta\!f$ decomposition model, with a macroscopic fluid background and microscopic kinetic correction, both fully coupled to each other. A similar manner of discretization is proposed to that used in the recent \texttt{STRUPHY} code \cite{Holderied_Possanner_Wang_2021, Holderied_2022, Li_et_al_2023} with a finite-element model for the background and a pseudo-particle/PiC model for the correction.

        The fluid background satisfies the full, non-linear, resistive, compressible, Hall MHD equations. \cite{Laakmann_Hu_Farrell_2022} introduces finite-element(-in-space) implicit timesteppers for the incompressible analogue to this system with structure-preserving (SP) properties in the ideal case, alongside parameter-robust preconditioners. We show that these timesteppers can derive from a finite-element-in-time (FET) (and finite-element-in-space) interpretation. The benefits of this reformulation are discussed, including the derivation of timesteppers that are higher order in time, and the quantifiable dissipative SP properties in the non-ideal, resistive case.
        
        We discuss possible options for extending this FET approach to timesteppers for the compressible case.

        The kinetic corrections satisfy linearized Boltzmann equations. Using a Lénard--Bernstein collision operator, these take Fokker--Planck-like forms \cite{Fokker_1914, Planck_1917} wherein pseudo-particles in the numerical model obey the neoclassical transport equations, with particle-independent Brownian drift terms. This offers a rigorous methodology for incorporating collisions into the particle transport model, without coupling the equations of motions for each particle.
        
        Works by Chen, Chacón et al. \cite{Chen_Chacón_Barnes_2011, Chacón_Chen_Barnes_2013, Chen_Chacón_2014, Chen_Chacón_2015} have developed structure-preserving particle pushers for neoclassical transport in the Vlasov equations, derived from Crank--Nicolson integrators. We show these too can can derive from a FET interpretation, similarly offering potential extensions to higher-order-in-time particle pushers. The FET formulation is used also to consider how the stochastic drift terms can be incorporated into the pushers. Stochastic gyrokinetic expansions are also discussed.

        Different options for the numerical implementation of these schemes are considered.

        Due to the efficacy of FET in the development of SP timesteppers for both the fluid and kinetic component, we hope this approach will prove effective in the future for developing SP timesteppers for the full hybrid model. We hope this will give us the opportunity to incorporate previously inaccessible kinetic effects into the highly effective, modern, finite-element MHD models.
    \end{abstract}
    
    
    \newpage
    \tableofcontents
    
    
    \newpage
    \pagenumbering{arabic}
    %\linenumbers\renewcommand\thelinenumber{\color{black!50}\arabic{linenumber}}
            \input{0 - introduction/main.tex}
        \part{Research}
            \input{1 - low-noise PiC models/main.tex}
            \input{2 - kinetic component/main.tex}
            \input{3 - fluid component/main.tex}
            \input{4 - numerical implementation/main.tex}
        \part{Project Overview}
            \input{5 - research plan/main.tex}
            \input{6 - summary/main.tex}
    
    
    %\section{}
    \newpage
    \pagenumbering{gobble}
        \printbibliography


    \newpage
    \pagenumbering{roman}
    \appendix
        \part{Appendices}
            \input{8 - Hilbert complexes/main.tex}
            \input{9 - weak conservation proofs/main.tex}
\end{document}

            \documentclass[12pt, a4paper]{report}

\input{template/main.tex}

\title{\BA{Title in Progress...}}
\author{Boris Andrews}
\affil{Mathematical Institute, University of Oxford}
\date{\today}


\begin{document}
    \pagenumbering{gobble}
    \maketitle
    
    
    \begin{abstract}
        Magnetic confinement reactors---in particular tokamaks---offer one of the most promising options for achieving practical nuclear fusion, with the potential to provide virtually limitless, clean energy. The theoretical and numerical modeling of tokamak plasmas is simultaneously an essential component of effective reactor design, and a great research barrier. Tokamak operational conditions exhibit comparatively low Knudsen numbers. Kinetic effects, including kinetic waves and instabilities, Landau damping, bump-on-tail instabilities and more, are therefore highly influential in tokamak plasma dynamics. Purely fluid models are inherently incapable of capturing these effects, whereas the high dimensionality in purely kinetic models render them practically intractable for most relevant purposes.

        We consider a $\delta\!f$ decomposition model, with a macroscopic fluid background and microscopic kinetic correction, both fully coupled to each other. A similar manner of discretization is proposed to that used in the recent \texttt{STRUPHY} code \cite{Holderied_Possanner_Wang_2021, Holderied_2022, Li_et_al_2023} with a finite-element model for the background and a pseudo-particle/PiC model for the correction.

        The fluid background satisfies the full, non-linear, resistive, compressible, Hall MHD equations. \cite{Laakmann_Hu_Farrell_2022} introduces finite-element(-in-space) implicit timesteppers for the incompressible analogue to this system with structure-preserving (SP) properties in the ideal case, alongside parameter-robust preconditioners. We show that these timesteppers can derive from a finite-element-in-time (FET) (and finite-element-in-space) interpretation. The benefits of this reformulation are discussed, including the derivation of timesteppers that are higher order in time, and the quantifiable dissipative SP properties in the non-ideal, resistive case.
        
        We discuss possible options for extending this FET approach to timesteppers for the compressible case.

        The kinetic corrections satisfy linearized Boltzmann equations. Using a Lénard--Bernstein collision operator, these take Fokker--Planck-like forms \cite{Fokker_1914, Planck_1917} wherein pseudo-particles in the numerical model obey the neoclassical transport equations, with particle-independent Brownian drift terms. This offers a rigorous methodology for incorporating collisions into the particle transport model, without coupling the equations of motions for each particle.
        
        Works by Chen, Chacón et al. \cite{Chen_Chacón_Barnes_2011, Chacón_Chen_Barnes_2013, Chen_Chacón_2014, Chen_Chacón_2015} have developed structure-preserving particle pushers for neoclassical transport in the Vlasov equations, derived from Crank--Nicolson integrators. We show these too can can derive from a FET interpretation, similarly offering potential extensions to higher-order-in-time particle pushers. The FET formulation is used also to consider how the stochastic drift terms can be incorporated into the pushers. Stochastic gyrokinetic expansions are also discussed.

        Different options for the numerical implementation of these schemes are considered.

        Due to the efficacy of FET in the development of SP timesteppers for both the fluid and kinetic component, we hope this approach will prove effective in the future for developing SP timesteppers for the full hybrid model. We hope this will give us the opportunity to incorporate previously inaccessible kinetic effects into the highly effective, modern, finite-element MHD models.
    \end{abstract}
    
    
    \newpage
    \tableofcontents
    
    
    \newpage
    \pagenumbering{arabic}
    %\linenumbers\renewcommand\thelinenumber{\color{black!50}\arabic{linenumber}}
            \input{0 - introduction/main.tex}
        \part{Research}
            \input{1 - low-noise PiC models/main.tex}
            \input{2 - kinetic component/main.tex}
            \input{3 - fluid component/main.tex}
            \input{4 - numerical implementation/main.tex}
        \part{Project Overview}
            \input{5 - research plan/main.tex}
            \input{6 - summary/main.tex}
    
    
    %\section{}
    \newpage
    \pagenumbering{gobble}
        \printbibliography


    \newpage
    \pagenumbering{roman}
    \appendix
        \part{Appendices}
            \input{8 - Hilbert complexes/main.tex}
            \input{9 - weak conservation proofs/main.tex}
\end{document}

\end{document}


\title{\BA{Title in Progress...}}
\author{Boris Andrews}
\affil{Mathematical Institute, University of Oxford}
\date{\today}


\begin{document}
    \pagenumbering{gobble}
    \maketitle
    
    
    \begin{abstract}
        Magnetic confinement reactors---in particular tokamaks---offer one of the most promising options for achieving practical nuclear fusion, with the potential to provide virtually limitless, clean energy. The theoretical and numerical modeling of tokamak plasmas is simultaneously an essential component of effective reactor design, and a great research barrier. Tokamak operational conditions exhibit comparatively low Knudsen numbers. Kinetic effects, including kinetic waves and instabilities, Landau damping, bump-on-tail instabilities and more, are therefore highly influential in tokamak plasma dynamics. Purely fluid models are inherently incapable of capturing these effects, whereas the high dimensionality in purely kinetic models render them practically intractable for most relevant purposes.

        We consider a $\delta\!f$ decomposition model, with a macroscopic fluid background and microscopic kinetic correction, both fully coupled to each other. A similar manner of discretization is proposed to that used in the recent \texttt{STRUPHY} code \cite{Holderied_Possanner_Wang_2021, Holderied_2022, Li_et_al_2023} with a finite-element model for the background and a pseudo-particle/PiC model for the correction.

        The fluid background satisfies the full, non-linear, resistive, compressible, Hall MHD equations. \cite{Laakmann_Hu_Farrell_2022} introduces finite-element(-in-space) implicit timesteppers for the incompressible analogue to this system with structure-preserving (SP) properties in the ideal case, alongside parameter-robust preconditioners. We show that these timesteppers can derive from a finite-element-in-time (FET) (and finite-element-in-space) interpretation. The benefits of this reformulation are discussed, including the derivation of timesteppers that are higher order in time, and the quantifiable dissipative SP properties in the non-ideal, resistive case.
        
        We discuss possible options for extending this FET approach to timesteppers for the compressible case.

        The kinetic corrections satisfy linearized Boltzmann equations. Using a Lénard--Bernstein collision operator, these take Fokker--Planck-like forms \cite{Fokker_1914, Planck_1917} wherein pseudo-particles in the numerical model obey the neoclassical transport equations, with particle-independent Brownian drift terms. This offers a rigorous methodology for incorporating collisions into the particle transport model, without coupling the equations of motions for each particle.
        
        Works by Chen, Chacón et al. \cite{Chen_Chacón_Barnes_2011, Chacón_Chen_Barnes_2013, Chen_Chacón_2014, Chen_Chacón_2015} have developed structure-preserving particle pushers for neoclassical transport in the Vlasov equations, derived from Crank--Nicolson integrators. We show these too can can derive from a FET interpretation, similarly offering potential extensions to higher-order-in-time particle pushers. The FET formulation is used also to consider how the stochastic drift terms can be incorporated into the pushers. Stochastic gyrokinetic expansions are also discussed.

        Different options for the numerical implementation of these schemes are considered.

        Due to the efficacy of FET in the development of SP timesteppers for both the fluid and kinetic component, we hope this approach will prove effective in the future for developing SP timesteppers for the full hybrid model. We hope this will give us the opportunity to incorporate previously inaccessible kinetic effects into the highly effective, modern, finite-element MHD models.
    \end{abstract}
    
    
    \newpage
    \tableofcontents
    
    
    \newpage
    \pagenumbering{arabic}
    %\linenumbers\renewcommand\thelinenumber{\color{black!50}\arabic{linenumber}}
            \documentclass[12pt, a4paper]{report}

\documentclass[12pt, a4paper]{report}

\input{template/main.tex}

\title{\BA{Title in Progress...}}
\author{Boris Andrews}
\affil{Mathematical Institute, University of Oxford}
\date{\today}


\begin{document}
    \pagenumbering{gobble}
    \maketitle
    
    
    \begin{abstract}
        Magnetic confinement reactors---in particular tokamaks---offer one of the most promising options for achieving practical nuclear fusion, with the potential to provide virtually limitless, clean energy. The theoretical and numerical modeling of tokamak plasmas is simultaneously an essential component of effective reactor design, and a great research barrier. Tokamak operational conditions exhibit comparatively low Knudsen numbers. Kinetic effects, including kinetic waves and instabilities, Landau damping, bump-on-tail instabilities and more, are therefore highly influential in tokamak plasma dynamics. Purely fluid models are inherently incapable of capturing these effects, whereas the high dimensionality in purely kinetic models render them practically intractable for most relevant purposes.

        We consider a $\delta\!f$ decomposition model, with a macroscopic fluid background and microscopic kinetic correction, both fully coupled to each other. A similar manner of discretization is proposed to that used in the recent \texttt{STRUPHY} code \cite{Holderied_Possanner_Wang_2021, Holderied_2022, Li_et_al_2023} with a finite-element model for the background and a pseudo-particle/PiC model for the correction.

        The fluid background satisfies the full, non-linear, resistive, compressible, Hall MHD equations. \cite{Laakmann_Hu_Farrell_2022} introduces finite-element(-in-space) implicit timesteppers for the incompressible analogue to this system with structure-preserving (SP) properties in the ideal case, alongside parameter-robust preconditioners. We show that these timesteppers can derive from a finite-element-in-time (FET) (and finite-element-in-space) interpretation. The benefits of this reformulation are discussed, including the derivation of timesteppers that are higher order in time, and the quantifiable dissipative SP properties in the non-ideal, resistive case.
        
        We discuss possible options for extending this FET approach to timesteppers for the compressible case.

        The kinetic corrections satisfy linearized Boltzmann equations. Using a Lénard--Bernstein collision operator, these take Fokker--Planck-like forms \cite{Fokker_1914, Planck_1917} wherein pseudo-particles in the numerical model obey the neoclassical transport equations, with particle-independent Brownian drift terms. This offers a rigorous methodology for incorporating collisions into the particle transport model, without coupling the equations of motions for each particle.
        
        Works by Chen, Chacón et al. \cite{Chen_Chacón_Barnes_2011, Chacón_Chen_Barnes_2013, Chen_Chacón_2014, Chen_Chacón_2015} have developed structure-preserving particle pushers for neoclassical transport in the Vlasov equations, derived from Crank--Nicolson integrators. We show these too can can derive from a FET interpretation, similarly offering potential extensions to higher-order-in-time particle pushers. The FET formulation is used also to consider how the stochastic drift terms can be incorporated into the pushers. Stochastic gyrokinetic expansions are also discussed.

        Different options for the numerical implementation of these schemes are considered.

        Due to the efficacy of FET in the development of SP timesteppers for both the fluid and kinetic component, we hope this approach will prove effective in the future for developing SP timesteppers for the full hybrid model. We hope this will give us the opportunity to incorporate previously inaccessible kinetic effects into the highly effective, modern, finite-element MHD models.
    \end{abstract}
    
    
    \newpage
    \tableofcontents
    
    
    \newpage
    \pagenumbering{arabic}
    %\linenumbers\renewcommand\thelinenumber{\color{black!50}\arabic{linenumber}}
            \input{0 - introduction/main.tex}
        \part{Research}
            \input{1 - low-noise PiC models/main.tex}
            \input{2 - kinetic component/main.tex}
            \input{3 - fluid component/main.tex}
            \input{4 - numerical implementation/main.tex}
        \part{Project Overview}
            \input{5 - research plan/main.tex}
            \input{6 - summary/main.tex}
    
    
    %\section{}
    \newpage
    \pagenumbering{gobble}
        \printbibliography


    \newpage
    \pagenumbering{roman}
    \appendix
        \part{Appendices}
            \input{8 - Hilbert complexes/main.tex}
            \input{9 - weak conservation proofs/main.tex}
\end{document}


\title{\BA{Title in Progress...}}
\author{Boris Andrews}
\affil{Mathematical Institute, University of Oxford}
\date{\today}


\begin{document}
    \pagenumbering{gobble}
    \maketitle
    
    
    \begin{abstract}
        Magnetic confinement reactors---in particular tokamaks---offer one of the most promising options for achieving practical nuclear fusion, with the potential to provide virtually limitless, clean energy. The theoretical and numerical modeling of tokamak plasmas is simultaneously an essential component of effective reactor design, and a great research barrier. Tokamak operational conditions exhibit comparatively low Knudsen numbers. Kinetic effects, including kinetic waves and instabilities, Landau damping, bump-on-tail instabilities and more, are therefore highly influential in tokamak plasma dynamics. Purely fluid models are inherently incapable of capturing these effects, whereas the high dimensionality in purely kinetic models render them practically intractable for most relevant purposes.

        We consider a $\delta\!f$ decomposition model, with a macroscopic fluid background and microscopic kinetic correction, both fully coupled to each other. A similar manner of discretization is proposed to that used in the recent \texttt{STRUPHY} code \cite{Holderied_Possanner_Wang_2021, Holderied_2022, Li_et_al_2023} with a finite-element model for the background and a pseudo-particle/PiC model for the correction.

        The fluid background satisfies the full, non-linear, resistive, compressible, Hall MHD equations. \cite{Laakmann_Hu_Farrell_2022} introduces finite-element(-in-space) implicit timesteppers for the incompressible analogue to this system with structure-preserving (SP) properties in the ideal case, alongside parameter-robust preconditioners. We show that these timesteppers can derive from a finite-element-in-time (FET) (and finite-element-in-space) interpretation. The benefits of this reformulation are discussed, including the derivation of timesteppers that are higher order in time, and the quantifiable dissipative SP properties in the non-ideal, resistive case.
        
        We discuss possible options for extending this FET approach to timesteppers for the compressible case.

        The kinetic corrections satisfy linearized Boltzmann equations. Using a Lénard--Bernstein collision operator, these take Fokker--Planck-like forms \cite{Fokker_1914, Planck_1917} wherein pseudo-particles in the numerical model obey the neoclassical transport equations, with particle-independent Brownian drift terms. This offers a rigorous methodology for incorporating collisions into the particle transport model, without coupling the equations of motions for each particle.
        
        Works by Chen, Chacón et al. \cite{Chen_Chacón_Barnes_2011, Chacón_Chen_Barnes_2013, Chen_Chacón_2014, Chen_Chacón_2015} have developed structure-preserving particle pushers for neoclassical transport in the Vlasov equations, derived from Crank--Nicolson integrators. We show these too can can derive from a FET interpretation, similarly offering potential extensions to higher-order-in-time particle pushers. The FET formulation is used also to consider how the stochastic drift terms can be incorporated into the pushers. Stochastic gyrokinetic expansions are also discussed.

        Different options for the numerical implementation of these schemes are considered.

        Due to the efficacy of FET in the development of SP timesteppers for both the fluid and kinetic component, we hope this approach will prove effective in the future for developing SP timesteppers for the full hybrid model. We hope this will give us the opportunity to incorporate previously inaccessible kinetic effects into the highly effective, modern, finite-element MHD models.
    \end{abstract}
    
    
    \newpage
    \tableofcontents
    
    
    \newpage
    \pagenumbering{arabic}
    %\linenumbers\renewcommand\thelinenumber{\color{black!50}\arabic{linenumber}}
            \documentclass[12pt, a4paper]{report}

\input{template/main.tex}

\title{\BA{Title in Progress...}}
\author{Boris Andrews}
\affil{Mathematical Institute, University of Oxford}
\date{\today}


\begin{document}
    \pagenumbering{gobble}
    \maketitle
    
    
    \begin{abstract}
        Magnetic confinement reactors---in particular tokamaks---offer one of the most promising options for achieving practical nuclear fusion, with the potential to provide virtually limitless, clean energy. The theoretical and numerical modeling of tokamak plasmas is simultaneously an essential component of effective reactor design, and a great research barrier. Tokamak operational conditions exhibit comparatively low Knudsen numbers. Kinetic effects, including kinetic waves and instabilities, Landau damping, bump-on-tail instabilities and more, are therefore highly influential in tokamak plasma dynamics. Purely fluid models are inherently incapable of capturing these effects, whereas the high dimensionality in purely kinetic models render them practically intractable for most relevant purposes.

        We consider a $\delta\!f$ decomposition model, with a macroscopic fluid background and microscopic kinetic correction, both fully coupled to each other. A similar manner of discretization is proposed to that used in the recent \texttt{STRUPHY} code \cite{Holderied_Possanner_Wang_2021, Holderied_2022, Li_et_al_2023} with a finite-element model for the background and a pseudo-particle/PiC model for the correction.

        The fluid background satisfies the full, non-linear, resistive, compressible, Hall MHD equations. \cite{Laakmann_Hu_Farrell_2022} introduces finite-element(-in-space) implicit timesteppers for the incompressible analogue to this system with structure-preserving (SP) properties in the ideal case, alongside parameter-robust preconditioners. We show that these timesteppers can derive from a finite-element-in-time (FET) (and finite-element-in-space) interpretation. The benefits of this reformulation are discussed, including the derivation of timesteppers that are higher order in time, and the quantifiable dissipative SP properties in the non-ideal, resistive case.
        
        We discuss possible options for extending this FET approach to timesteppers for the compressible case.

        The kinetic corrections satisfy linearized Boltzmann equations. Using a Lénard--Bernstein collision operator, these take Fokker--Planck-like forms \cite{Fokker_1914, Planck_1917} wherein pseudo-particles in the numerical model obey the neoclassical transport equations, with particle-independent Brownian drift terms. This offers a rigorous methodology for incorporating collisions into the particle transport model, without coupling the equations of motions for each particle.
        
        Works by Chen, Chacón et al. \cite{Chen_Chacón_Barnes_2011, Chacón_Chen_Barnes_2013, Chen_Chacón_2014, Chen_Chacón_2015} have developed structure-preserving particle pushers for neoclassical transport in the Vlasov equations, derived from Crank--Nicolson integrators. We show these too can can derive from a FET interpretation, similarly offering potential extensions to higher-order-in-time particle pushers. The FET formulation is used also to consider how the stochastic drift terms can be incorporated into the pushers. Stochastic gyrokinetic expansions are also discussed.

        Different options for the numerical implementation of these schemes are considered.

        Due to the efficacy of FET in the development of SP timesteppers for both the fluid and kinetic component, we hope this approach will prove effective in the future for developing SP timesteppers for the full hybrid model. We hope this will give us the opportunity to incorporate previously inaccessible kinetic effects into the highly effective, modern, finite-element MHD models.
    \end{abstract}
    
    
    \newpage
    \tableofcontents
    
    
    \newpage
    \pagenumbering{arabic}
    %\linenumbers\renewcommand\thelinenumber{\color{black!50}\arabic{linenumber}}
            \input{0 - introduction/main.tex}
        \part{Research}
            \input{1 - low-noise PiC models/main.tex}
            \input{2 - kinetic component/main.tex}
            \input{3 - fluid component/main.tex}
            \input{4 - numerical implementation/main.tex}
        \part{Project Overview}
            \input{5 - research plan/main.tex}
            \input{6 - summary/main.tex}
    
    
    %\section{}
    \newpage
    \pagenumbering{gobble}
        \printbibliography


    \newpage
    \pagenumbering{roman}
    \appendix
        \part{Appendices}
            \input{8 - Hilbert complexes/main.tex}
            \input{9 - weak conservation proofs/main.tex}
\end{document}

        \part{Research}
            \documentclass[12pt, a4paper]{report}

\input{template/main.tex}

\title{\BA{Title in Progress...}}
\author{Boris Andrews}
\affil{Mathematical Institute, University of Oxford}
\date{\today}


\begin{document}
    \pagenumbering{gobble}
    \maketitle
    
    
    \begin{abstract}
        Magnetic confinement reactors---in particular tokamaks---offer one of the most promising options for achieving practical nuclear fusion, with the potential to provide virtually limitless, clean energy. The theoretical and numerical modeling of tokamak plasmas is simultaneously an essential component of effective reactor design, and a great research barrier. Tokamak operational conditions exhibit comparatively low Knudsen numbers. Kinetic effects, including kinetic waves and instabilities, Landau damping, bump-on-tail instabilities and more, are therefore highly influential in tokamak plasma dynamics. Purely fluid models are inherently incapable of capturing these effects, whereas the high dimensionality in purely kinetic models render them practically intractable for most relevant purposes.

        We consider a $\delta\!f$ decomposition model, with a macroscopic fluid background and microscopic kinetic correction, both fully coupled to each other. A similar manner of discretization is proposed to that used in the recent \texttt{STRUPHY} code \cite{Holderied_Possanner_Wang_2021, Holderied_2022, Li_et_al_2023} with a finite-element model for the background and a pseudo-particle/PiC model for the correction.

        The fluid background satisfies the full, non-linear, resistive, compressible, Hall MHD equations. \cite{Laakmann_Hu_Farrell_2022} introduces finite-element(-in-space) implicit timesteppers for the incompressible analogue to this system with structure-preserving (SP) properties in the ideal case, alongside parameter-robust preconditioners. We show that these timesteppers can derive from a finite-element-in-time (FET) (and finite-element-in-space) interpretation. The benefits of this reformulation are discussed, including the derivation of timesteppers that are higher order in time, and the quantifiable dissipative SP properties in the non-ideal, resistive case.
        
        We discuss possible options for extending this FET approach to timesteppers for the compressible case.

        The kinetic corrections satisfy linearized Boltzmann equations. Using a Lénard--Bernstein collision operator, these take Fokker--Planck-like forms \cite{Fokker_1914, Planck_1917} wherein pseudo-particles in the numerical model obey the neoclassical transport equations, with particle-independent Brownian drift terms. This offers a rigorous methodology for incorporating collisions into the particle transport model, without coupling the equations of motions for each particle.
        
        Works by Chen, Chacón et al. \cite{Chen_Chacón_Barnes_2011, Chacón_Chen_Barnes_2013, Chen_Chacón_2014, Chen_Chacón_2015} have developed structure-preserving particle pushers for neoclassical transport in the Vlasov equations, derived from Crank--Nicolson integrators. We show these too can can derive from a FET interpretation, similarly offering potential extensions to higher-order-in-time particle pushers. The FET formulation is used also to consider how the stochastic drift terms can be incorporated into the pushers. Stochastic gyrokinetic expansions are also discussed.

        Different options for the numerical implementation of these schemes are considered.

        Due to the efficacy of FET in the development of SP timesteppers for both the fluid and kinetic component, we hope this approach will prove effective in the future for developing SP timesteppers for the full hybrid model. We hope this will give us the opportunity to incorporate previously inaccessible kinetic effects into the highly effective, modern, finite-element MHD models.
    \end{abstract}
    
    
    \newpage
    \tableofcontents
    
    
    \newpage
    \pagenumbering{arabic}
    %\linenumbers\renewcommand\thelinenumber{\color{black!50}\arabic{linenumber}}
            \input{0 - introduction/main.tex}
        \part{Research}
            \input{1 - low-noise PiC models/main.tex}
            \input{2 - kinetic component/main.tex}
            \input{3 - fluid component/main.tex}
            \input{4 - numerical implementation/main.tex}
        \part{Project Overview}
            \input{5 - research plan/main.tex}
            \input{6 - summary/main.tex}
    
    
    %\section{}
    \newpage
    \pagenumbering{gobble}
        \printbibliography


    \newpage
    \pagenumbering{roman}
    \appendix
        \part{Appendices}
            \input{8 - Hilbert complexes/main.tex}
            \input{9 - weak conservation proofs/main.tex}
\end{document}

            \documentclass[12pt, a4paper]{report}

\input{template/main.tex}

\title{\BA{Title in Progress...}}
\author{Boris Andrews}
\affil{Mathematical Institute, University of Oxford}
\date{\today}


\begin{document}
    \pagenumbering{gobble}
    \maketitle
    
    
    \begin{abstract}
        Magnetic confinement reactors---in particular tokamaks---offer one of the most promising options for achieving practical nuclear fusion, with the potential to provide virtually limitless, clean energy. The theoretical and numerical modeling of tokamak plasmas is simultaneously an essential component of effective reactor design, and a great research barrier. Tokamak operational conditions exhibit comparatively low Knudsen numbers. Kinetic effects, including kinetic waves and instabilities, Landau damping, bump-on-tail instabilities and more, are therefore highly influential in tokamak plasma dynamics. Purely fluid models are inherently incapable of capturing these effects, whereas the high dimensionality in purely kinetic models render them practically intractable for most relevant purposes.

        We consider a $\delta\!f$ decomposition model, with a macroscopic fluid background and microscopic kinetic correction, both fully coupled to each other. A similar manner of discretization is proposed to that used in the recent \texttt{STRUPHY} code \cite{Holderied_Possanner_Wang_2021, Holderied_2022, Li_et_al_2023} with a finite-element model for the background and a pseudo-particle/PiC model for the correction.

        The fluid background satisfies the full, non-linear, resistive, compressible, Hall MHD equations. \cite{Laakmann_Hu_Farrell_2022} introduces finite-element(-in-space) implicit timesteppers for the incompressible analogue to this system with structure-preserving (SP) properties in the ideal case, alongside parameter-robust preconditioners. We show that these timesteppers can derive from a finite-element-in-time (FET) (and finite-element-in-space) interpretation. The benefits of this reformulation are discussed, including the derivation of timesteppers that are higher order in time, and the quantifiable dissipative SP properties in the non-ideal, resistive case.
        
        We discuss possible options for extending this FET approach to timesteppers for the compressible case.

        The kinetic corrections satisfy linearized Boltzmann equations. Using a Lénard--Bernstein collision operator, these take Fokker--Planck-like forms \cite{Fokker_1914, Planck_1917} wherein pseudo-particles in the numerical model obey the neoclassical transport equations, with particle-independent Brownian drift terms. This offers a rigorous methodology for incorporating collisions into the particle transport model, without coupling the equations of motions for each particle.
        
        Works by Chen, Chacón et al. \cite{Chen_Chacón_Barnes_2011, Chacón_Chen_Barnes_2013, Chen_Chacón_2014, Chen_Chacón_2015} have developed structure-preserving particle pushers for neoclassical transport in the Vlasov equations, derived from Crank--Nicolson integrators. We show these too can can derive from a FET interpretation, similarly offering potential extensions to higher-order-in-time particle pushers. The FET formulation is used also to consider how the stochastic drift terms can be incorporated into the pushers. Stochastic gyrokinetic expansions are also discussed.

        Different options for the numerical implementation of these schemes are considered.

        Due to the efficacy of FET in the development of SP timesteppers for both the fluid and kinetic component, we hope this approach will prove effective in the future for developing SP timesteppers for the full hybrid model. We hope this will give us the opportunity to incorporate previously inaccessible kinetic effects into the highly effective, modern, finite-element MHD models.
    \end{abstract}
    
    
    \newpage
    \tableofcontents
    
    
    \newpage
    \pagenumbering{arabic}
    %\linenumbers\renewcommand\thelinenumber{\color{black!50}\arabic{linenumber}}
            \input{0 - introduction/main.tex}
        \part{Research}
            \input{1 - low-noise PiC models/main.tex}
            \input{2 - kinetic component/main.tex}
            \input{3 - fluid component/main.tex}
            \input{4 - numerical implementation/main.tex}
        \part{Project Overview}
            \input{5 - research plan/main.tex}
            \input{6 - summary/main.tex}
    
    
    %\section{}
    \newpage
    \pagenumbering{gobble}
        \printbibliography


    \newpage
    \pagenumbering{roman}
    \appendix
        \part{Appendices}
            \input{8 - Hilbert complexes/main.tex}
            \input{9 - weak conservation proofs/main.tex}
\end{document}

            \documentclass[12pt, a4paper]{report}

\input{template/main.tex}

\title{\BA{Title in Progress...}}
\author{Boris Andrews}
\affil{Mathematical Institute, University of Oxford}
\date{\today}


\begin{document}
    \pagenumbering{gobble}
    \maketitle
    
    
    \begin{abstract}
        Magnetic confinement reactors---in particular tokamaks---offer one of the most promising options for achieving practical nuclear fusion, with the potential to provide virtually limitless, clean energy. The theoretical and numerical modeling of tokamak plasmas is simultaneously an essential component of effective reactor design, and a great research barrier. Tokamak operational conditions exhibit comparatively low Knudsen numbers. Kinetic effects, including kinetic waves and instabilities, Landau damping, bump-on-tail instabilities and more, are therefore highly influential in tokamak plasma dynamics. Purely fluid models are inherently incapable of capturing these effects, whereas the high dimensionality in purely kinetic models render them practically intractable for most relevant purposes.

        We consider a $\delta\!f$ decomposition model, with a macroscopic fluid background and microscopic kinetic correction, both fully coupled to each other. A similar manner of discretization is proposed to that used in the recent \texttt{STRUPHY} code \cite{Holderied_Possanner_Wang_2021, Holderied_2022, Li_et_al_2023} with a finite-element model for the background and a pseudo-particle/PiC model for the correction.

        The fluid background satisfies the full, non-linear, resistive, compressible, Hall MHD equations. \cite{Laakmann_Hu_Farrell_2022} introduces finite-element(-in-space) implicit timesteppers for the incompressible analogue to this system with structure-preserving (SP) properties in the ideal case, alongside parameter-robust preconditioners. We show that these timesteppers can derive from a finite-element-in-time (FET) (and finite-element-in-space) interpretation. The benefits of this reformulation are discussed, including the derivation of timesteppers that are higher order in time, and the quantifiable dissipative SP properties in the non-ideal, resistive case.
        
        We discuss possible options for extending this FET approach to timesteppers for the compressible case.

        The kinetic corrections satisfy linearized Boltzmann equations. Using a Lénard--Bernstein collision operator, these take Fokker--Planck-like forms \cite{Fokker_1914, Planck_1917} wherein pseudo-particles in the numerical model obey the neoclassical transport equations, with particle-independent Brownian drift terms. This offers a rigorous methodology for incorporating collisions into the particle transport model, without coupling the equations of motions for each particle.
        
        Works by Chen, Chacón et al. \cite{Chen_Chacón_Barnes_2011, Chacón_Chen_Barnes_2013, Chen_Chacón_2014, Chen_Chacón_2015} have developed structure-preserving particle pushers for neoclassical transport in the Vlasov equations, derived from Crank--Nicolson integrators. We show these too can can derive from a FET interpretation, similarly offering potential extensions to higher-order-in-time particle pushers. The FET formulation is used also to consider how the stochastic drift terms can be incorporated into the pushers. Stochastic gyrokinetic expansions are also discussed.

        Different options for the numerical implementation of these schemes are considered.

        Due to the efficacy of FET in the development of SP timesteppers for both the fluid and kinetic component, we hope this approach will prove effective in the future for developing SP timesteppers for the full hybrid model. We hope this will give us the opportunity to incorporate previously inaccessible kinetic effects into the highly effective, modern, finite-element MHD models.
    \end{abstract}
    
    
    \newpage
    \tableofcontents
    
    
    \newpage
    \pagenumbering{arabic}
    %\linenumbers\renewcommand\thelinenumber{\color{black!50}\arabic{linenumber}}
            \input{0 - introduction/main.tex}
        \part{Research}
            \input{1 - low-noise PiC models/main.tex}
            \input{2 - kinetic component/main.tex}
            \input{3 - fluid component/main.tex}
            \input{4 - numerical implementation/main.tex}
        \part{Project Overview}
            \input{5 - research plan/main.tex}
            \input{6 - summary/main.tex}
    
    
    %\section{}
    \newpage
    \pagenumbering{gobble}
        \printbibliography


    \newpage
    \pagenumbering{roman}
    \appendix
        \part{Appendices}
            \input{8 - Hilbert complexes/main.tex}
            \input{9 - weak conservation proofs/main.tex}
\end{document}

            \documentclass[12pt, a4paper]{report}

\input{template/main.tex}

\title{\BA{Title in Progress...}}
\author{Boris Andrews}
\affil{Mathematical Institute, University of Oxford}
\date{\today}


\begin{document}
    \pagenumbering{gobble}
    \maketitle
    
    
    \begin{abstract}
        Magnetic confinement reactors---in particular tokamaks---offer one of the most promising options for achieving practical nuclear fusion, with the potential to provide virtually limitless, clean energy. The theoretical and numerical modeling of tokamak plasmas is simultaneously an essential component of effective reactor design, and a great research barrier. Tokamak operational conditions exhibit comparatively low Knudsen numbers. Kinetic effects, including kinetic waves and instabilities, Landau damping, bump-on-tail instabilities and more, are therefore highly influential in tokamak plasma dynamics. Purely fluid models are inherently incapable of capturing these effects, whereas the high dimensionality in purely kinetic models render them practically intractable for most relevant purposes.

        We consider a $\delta\!f$ decomposition model, with a macroscopic fluid background and microscopic kinetic correction, both fully coupled to each other. A similar manner of discretization is proposed to that used in the recent \texttt{STRUPHY} code \cite{Holderied_Possanner_Wang_2021, Holderied_2022, Li_et_al_2023} with a finite-element model for the background and a pseudo-particle/PiC model for the correction.

        The fluid background satisfies the full, non-linear, resistive, compressible, Hall MHD equations. \cite{Laakmann_Hu_Farrell_2022} introduces finite-element(-in-space) implicit timesteppers for the incompressible analogue to this system with structure-preserving (SP) properties in the ideal case, alongside parameter-robust preconditioners. We show that these timesteppers can derive from a finite-element-in-time (FET) (and finite-element-in-space) interpretation. The benefits of this reformulation are discussed, including the derivation of timesteppers that are higher order in time, and the quantifiable dissipative SP properties in the non-ideal, resistive case.
        
        We discuss possible options for extending this FET approach to timesteppers for the compressible case.

        The kinetic corrections satisfy linearized Boltzmann equations. Using a Lénard--Bernstein collision operator, these take Fokker--Planck-like forms \cite{Fokker_1914, Planck_1917} wherein pseudo-particles in the numerical model obey the neoclassical transport equations, with particle-independent Brownian drift terms. This offers a rigorous methodology for incorporating collisions into the particle transport model, without coupling the equations of motions for each particle.
        
        Works by Chen, Chacón et al. \cite{Chen_Chacón_Barnes_2011, Chacón_Chen_Barnes_2013, Chen_Chacón_2014, Chen_Chacón_2015} have developed structure-preserving particle pushers for neoclassical transport in the Vlasov equations, derived from Crank--Nicolson integrators. We show these too can can derive from a FET interpretation, similarly offering potential extensions to higher-order-in-time particle pushers. The FET formulation is used also to consider how the stochastic drift terms can be incorporated into the pushers. Stochastic gyrokinetic expansions are also discussed.

        Different options for the numerical implementation of these schemes are considered.

        Due to the efficacy of FET in the development of SP timesteppers for both the fluid and kinetic component, we hope this approach will prove effective in the future for developing SP timesteppers for the full hybrid model. We hope this will give us the opportunity to incorporate previously inaccessible kinetic effects into the highly effective, modern, finite-element MHD models.
    \end{abstract}
    
    
    \newpage
    \tableofcontents
    
    
    \newpage
    \pagenumbering{arabic}
    %\linenumbers\renewcommand\thelinenumber{\color{black!50}\arabic{linenumber}}
            \input{0 - introduction/main.tex}
        \part{Research}
            \input{1 - low-noise PiC models/main.tex}
            \input{2 - kinetic component/main.tex}
            \input{3 - fluid component/main.tex}
            \input{4 - numerical implementation/main.tex}
        \part{Project Overview}
            \input{5 - research plan/main.tex}
            \input{6 - summary/main.tex}
    
    
    %\section{}
    \newpage
    \pagenumbering{gobble}
        \printbibliography


    \newpage
    \pagenumbering{roman}
    \appendix
        \part{Appendices}
            \input{8 - Hilbert complexes/main.tex}
            \input{9 - weak conservation proofs/main.tex}
\end{document}

        \part{Project Overview}
            \documentclass[12pt, a4paper]{report}

\input{template/main.tex}

\title{\BA{Title in Progress...}}
\author{Boris Andrews}
\affil{Mathematical Institute, University of Oxford}
\date{\today}


\begin{document}
    \pagenumbering{gobble}
    \maketitle
    
    
    \begin{abstract}
        Magnetic confinement reactors---in particular tokamaks---offer one of the most promising options for achieving practical nuclear fusion, with the potential to provide virtually limitless, clean energy. The theoretical and numerical modeling of tokamak plasmas is simultaneously an essential component of effective reactor design, and a great research barrier. Tokamak operational conditions exhibit comparatively low Knudsen numbers. Kinetic effects, including kinetic waves and instabilities, Landau damping, bump-on-tail instabilities and more, are therefore highly influential in tokamak plasma dynamics. Purely fluid models are inherently incapable of capturing these effects, whereas the high dimensionality in purely kinetic models render them practically intractable for most relevant purposes.

        We consider a $\delta\!f$ decomposition model, with a macroscopic fluid background and microscopic kinetic correction, both fully coupled to each other. A similar manner of discretization is proposed to that used in the recent \texttt{STRUPHY} code \cite{Holderied_Possanner_Wang_2021, Holderied_2022, Li_et_al_2023} with a finite-element model for the background and a pseudo-particle/PiC model for the correction.

        The fluid background satisfies the full, non-linear, resistive, compressible, Hall MHD equations. \cite{Laakmann_Hu_Farrell_2022} introduces finite-element(-in-space) implicit timesteppers for the incompressible analogue to this system with structure-preserving (SP) properties in the ideal case, alongside parameter-robust preconditioners. We show that these timesteppers can derive from a finite-element-in-time (FET) (and finite-element-in-space) interpretation. The benefits of this reformulation are discussed, including the derivation of timesteppers that are higher order in time, and the quantifiable dissipative SP properties in the non-ideal, resistive case.
        
        We discuss possible options for extending this FET approach to timesteppers for the compressible case.

        The kinetic corrections satisfy linearized Boltzmann equations. Using a Lénard--Bernstein collision operator, these take Fokker--Planck-like forms \cite{Fokker_1914, Planck_1917} wherein pseudo-particles in the numerical model obey the neoclassical transport equations, with particle-independent Brownian drift terms. This offers a rigorous methodology for incorporating collisions into the particle transport model, without coupling the equations of motions for each particle.
        
        Works by Chen, Chacón et al. \cite{Chen_Chacón_Barnes_2011, Chacón_Chen_Barnes_2013, Chen_Chacón_2014, Chen_Chacón_2015} have developed structure-preserving particle pushers for neoclassical transport in the Vlasov equations, derived from Crank--Nicolson integrators. We show these too can can derive from a FET interpretation, similarly offering potential extensions to higher-order-in-time particle pushers. The FET formulation is used also to consider how the stochastic drift terms can be incorporated into the pushers. Stochastic gyrokinetic expansions are also discussed.

        Different options for the numerical implementation of these schemes are considered.

        Due to the efficacy of FET in the development of SP timesteppers for both the fluid and kinetic component, we hope this approach will prove effective in the future for developing SP timesteppers for the full hybrid model. We hope this will give us the opportunity to incorporate previously inaccessible kinetic effects into the highly effective, modern, finite-element MHD models.
    \end{abstract}
    
    
    \newpage
    \tableofcontents
    
    
    \newpage
    \pagenumbering{arabic}
    %\linenumbers\renewcommand\thelinenumber{\color{black!50}\arabic{linenumber}}
            \input{0 - introduction/main.tex}
        \part{Research}
            \input{1 - low-noise PiC models/main.tex}
            \input{2 - kinetic component/main.tex}
            \input{3 - fluid component/main.tex}
            \input{4 - numerical implementation/main.tex}
        \part{Project Overview}
            \input{5 - research plan/main.tex}
            \input{6 - summary/main.tex}
    
    
    %\section{}
    \newpage
    \pagenumbering{gobble}
        \printbibliography


    \newpage
    \pagenumbering{roman}
    \appendix
        \part{Appendices}
            \input{8 - Hilbert complexes/main.tex}
            \input{9 - weak conservation proofs/main.tex}
\end{document}

            \documentclass[12pt, a4paper]{report}

\input{template/main.tex}

\title{\BA{Title in Progress...}}
\author{Boris Andrews}
\affil{Mathematical Institute, University of Oxford}
\date{\today}


\begin{document}
    \pagenumbering{gobble}
    \maketitle
    
    
    \begin{abstract}
        Magnetic confinement reactors---in particular tokamaks---offer one of the most promising options for achieving practical nuclear fusion, with the potential to provide virtually limitless, clean energy. The theoretical and numerical modeling of tokamak plasmas is simultaneously an essential component of effective reactor design, and a great research barrier. Tokamak operational conditions exhibit comparatively low Knudsen numbers. Kinetic effects, including kinetic waves and instabilities, Landau damping, bump-on-tail instabilities and more, are therefore highly influential in tokamak plasma dynamics. Purely fluid models are inherently incapable of capturing these effects, whereas the high dimensionality in purely kinetic models render them practically intractable for most relevant purposes.

        We consider a $\delta\!f$ decomposition model, with a macroscopic fluid background and microscopic kinetic correction, both fully coupled to each other. A similar manner of discretization is proposed to that used in the recent \texttt{STRUPHY} code \cite{Holderied_Possanner_Wang_2021, Holderied_2022, Li_et_al_2023} with a finite-element model for the background and a pseudo-particle/PiC model for the correction.

        The fluid background satisfies the full, non-linear, resistive, compressible, Hall MHD equations. \cite{Laakmann_Hu_Farrell_2022} introduces finite-element(-in-space) implicit timesteppers for the incompressible analogue to this system with structure-preserving (SP) properties in the ideal case, alongside parameter-robust preconditioners. We show that these timesteppers can derive from a finite-element-in-time (FET) (and finite-element-in-space) interpretation. The benefits of this reformulation are discussed, including the derivation of timesteppers that are higher order in time, and the quantifiable dissipative SP properties in the non-ideal, resistive case.
        
        We discuss possible options for extending this FET approach to timesteppers for the compressible case.

        The kinetic corrections satisfy linearized Boltzmann equations. Using a Lénard--Bernstein collision operator, these take Fokker--Planck-like forms \cite{Fokker_1914, Planck_1917} wherein pseudo-particles in the numerical model obey the neoclassical transport equations, with particle-independent Brownian drift terms. This offers a rigorous methodology for incorporating collisions into the particle transport model, without coupling the equations of motions for each particle.
        
        Works by Chen, Chacón et al. \cite{Chen_Chacón_Barnes_2011, Chacón_Chen_Barnes_2013, Chen_Chacón_2014, Chen_Chacón_2015} have developed structure-preserving particle pushers for neoclassical transport in the Vlasov equations, derived from Crank--Nicolson integrators. We show these too can can derive from a FET interpretation, similarly offering potential extensions to higher-order-in-time particle pushers. The FET formulation is used also to consider how the stochastic drift terms can be incorporated into the pushers. Stochastic gyrokinetic expansions are also discussed.

        Different options for the numerical implementation of these schemes are considered.

        Due to the efficacy of FET in the development of SP timesteppers for both the fluid and kinetic component, we hope this approach will prove effective in the future for developing SP timesteppers for the full hybrid model. We hope this will give us the opportunity to incorporate previously inaccessible kinetic effects into the highly effective, modern, finite-element MHD models.
    \end{abstract}
    
    
    \newpage
    \tableofcontents
    
    
    \newpage
    \pagenumbering{arabic}
    %\linenumbers\renewcommand\thelinenumber{\color{black!50}\arabic{linenumber}}
            \input{0 - introduction/main.tex}
        \part{Research}
            \input{1 - low-noise PiC models/main.tex}
            \input{2 - kinetic component/main.tex}
            \input{3 - fluid component/main.tex}
            \input{4 - numerical implementation/main.tex}
        \part{Project Overview}
            \input{5 - research plan/main.tex}
            \input{6 - summary/main.tex}
    
    
    %\section{}
    \newpage
    \pagenumbering{gobble}
        \printbibliography


    \newpage
    \pagenumbering{roman}
    \appendix
        \part{Appendices}
            \input{8 - Hilbert complexes/main.tex}
            \input{9 - weak conservation proofs/main.tex}
\end{document}

    
    
    %\section{}
    \newpage
    \pagenumbering{gobble}
        \printbibliography


    \newpage
    \pagenumbering{roman}
    \appendix
        \part{Appendices}
            \documentclass[12pt, a4paper]{report}

\input{template/main.tex}

\title{\BA{Title in Progress...}}
\author{Boris Andrews}
\affil{Mathematical Institute, University of Oxford}
\date{\today}


\begin{document}
    \pagenumbering{gobble}
    \maketitle
    
    
    \begin{abstract}
        Magnetic confinement reactors---in particular tokamaks---offer one of the most promising options for achieving practical nuclear fusion, with the potential to provide virtually limitless, clean energy. The theoretical and numerical modeling of tokamak plasmas is simultaneously an essential component of effective reactor design, and a great research barrier. Tokamak operational conditions exhibit comparatively low Knudsen numbers. Kinetic effects, including kinetic waves and instabilities, Landau damping, bump-on-tail instabilities and more, are therefore highly influential in tokamak plasma dynamics. Purely fluid models are inherently incapable of capturing these effects, whereas the high dimensionality in purely kinetic models render them practically intractable for most relevant purposes.

        We consider a $\delta\!f$ decomposition model, with a macroscopic fluid background and microscopic kinetic correction, both fully coupled to each other. A similar manner of discretization is proposed to that used in the recent \texttt{STRUPHY} code \cite{Holderied_Possanner_Wang_2021, Holderied_2022, Li_et_al_2023} with a finite-element model for the background and a pseudo-particle/PiC model for the correction.

        The fluid background satisfies the full, non-linear, resistive, compressible, Hall MHD equations. \cite{Laakmann_Hu_Farrell_2022} introduces finite-element(-in-space) implicit timesteppers for the incompressible analogue to this system with structure-preserving (SP) properties in the ideal case, alongside parameter-robust preconditioners. We show that these timesteppers can derive from a finite-element-in-time (FET) (and finite-element-in-space) interpretation. The benefits of this reformulation are discussed, including the derivation of timesteppers that are higher order in time, and the quantifiable dissipative SP properties in the non-ideal, resistive case.
        
        We discuss possible options for extending this FET approach to timesteppers for the compressible case.

        The kinetic corrections satisfy linearized Boltzmann equations. Using a Lénard--Bernstein collision operator, these take Fokker--Planck-like forms \cite{Fokker_1914, Planck_1917} wherein pseudo-particles in the numerical model obey the neoclassical transport equations, with particle-independent Brownian drift terms. This offers a rigorous methodology for incorporating collisions into the particle transport model, without coupling the equations of motions for each particle.
        
        Works by Chen, Chacón et al. \cite{Chen_Chacón_Barnes_2011, Chacón_Chen_Barnes_2013, Chen_Chacón_2014, Chen_Chacón_2015} have developed structure-preserving particle pushers for neoclassical transport in the Vlasov equations, derived from Crank--Nicolson integrators. We show these too can can derive from a FET interpretation, similarly offering potential extensions to higher-order-in-time particle pushers. The FET formulation is used also to consider how the stochastic drift terms can be incorporated into the pushers. Stochastic gyrokinetic expansions are also discussed.

        Different options for the numerical implementation of these schemes are considered.

        Due to the efficacy of FET in the development of SP timesteppers for both the fluid and kinetic component, we hope this approach will prove effective in the future for developing SP timesteppers for the full hybrid model. We hope this will give us the opportunity to incorporate previously inaccessible kinetic effects into the highly effective, modern, finite-element MHD models.
    \end{abstract}
    
    
    \newpage
    \tableofcontents
    
    
    \newpage
    \pagenumbering{arabic}
    %\linenumbers\renewcommand\thelinenumber{\color{black!50}\arabic{linenumber}}
            \input{0 - introduction/main.tex}
        \part{Research}
            \input{1 - low-noise PiC models/main.tex}
            \input{2 - kinetic component/main.tex}
            \input{3 - fluid component/main.tex}
            \input{4 - numerical implementation/main.tex}
        \part{Project Overview}
            \input{5 - research plan/main.tex}
            \input{6 - summary/main.tex}
    
    
    %\section{}
    \newpage
    \pagenumbering{gobble}
        \printbibliography


    \newpage
    \pagenumbering{roman}
    \appendix
        \part{Appendices}
            \input{8 - Hilbert complexes/main.tex}
            \input{9 - weak conservation proofs/main.tex}
\end{document}

            \documentclass[12pt, a4paper]{report}

\input{template/main.tex}

\title{\BA{Title in Progress...}}
\author{Boris Andrews}
\affil{Mathematical Institute, University of Oxford}
\date{\today}


\begin{document}
    \pagenumbering{gobble}
    \maketitle
    
    
    \begin{abstract}
        Magnetic confinement reactors---in particular tokamaks---offer one of the most promising options for achieving practical nuclear fusion, with the potential to provide virtually limitless, clean energy. The theoretical and numerical modeling of tokamak plasmas is simultaneously an essential component of effective reactor design, and a great research barrier. Tokamak operational conditions exhibit comparatively low Knudsen numbers. Kinetic effects, including kinetic waves and instabilities, Landau damping, bump-on-tail instabilities and more, are therefore highly influential in tokamak plasma dynamics. Purely fluid models are inherently incapable of capturing these effects, whereas the high dimensionality in purely kinetic models render them practically intractable for most relevant purposes.

        We consider a $\delta\!f$ decomposition model, with a macroscopic fluid background and microscopic kinetic correction, both fully coupled to each other. A similar manner of discretization is proposed to that used in the recent \texttt{STRUPHY} code \cite{Holderied_Possanner_Wang_2021, Holderied_2022, Li_et_al_2023} with a finite-element model for the background and a pseudo-particle/PiC model for the correction.

        The fluid background satisfies the full, non-linear, resistive, compressible, Hall MHD equations. \cite{Laakmann_Hu_Farrell_2022} introduces finite-element(-in-space) implicit timesteppers for the incompressible analogue to this system with structure-preserving (SP) properties in the ideal case, alongside parameter-robust preconditioners. We show that these timesteppers can derive from a finite-element-in-time (FET) (and finite-element-in-space) interpretation. The benefits of this reformulation are discussed, including the derivation of timesteppers that are higher order in time, and the quantifiable dissipative SP properties in the non-ideal, resistive case.
        
        We discuss possible options for extending this FET approach to timesteppers for the compressible case.

        The kinetic corrections satisfy linearized Boltzmann equations. Using a Lénard--Bernstein collision operator, these take Fokker--Planck-like forms \cite{Fokker_1914, Planck_1917} wherein pseudo-particles in the numerical model obey the neoclassical transport equations, with particle-independent Brownian drift terms. This offers a rigorous methodology for incorporating collisions into the particle transport model, without coupling the equations of motions for each particle.
        
        Works by Chen, Chacón et al. \cite{Chen_Chacón_Barnes_2011, Chacón_Chen_Barnes_2013, Chen_Chacón_2014, Chen_Chacón_2015} have developed structure-preserving particle pushers for neoclassical transport in the Vlasov equations, derived from Crank--Nicolson integrators. We show these too can can derive from a FET interpretation, similarly offering potential extensions to higher-order-in-time particle pushers. The FET formulation is used also to consider how the stochastic drift terms can be incorporated into the pushers. Stochastic gyrokinetic expansions are also discussed.

        Different options for the numerical implementation of these schemes are considered.

        Due to the efficacy of FET in the development of SP timesteppers for both the fluid and kinetic component, we hope this approach will prove effective in the future for developing SP timesteppers for the full hybrid model. We hope this will give us the opportunity to incorporate previously inaccessible kinetic effects into the highly effective, modern, finite-element MHD models.
    \end{abstract}
    
    
    \newpage
    \tableofcontents
    
    
    \newpage
    \pagenumbering{arabic}
    %\linenumbers\renewcommand\thelinenumber{\color{black!50}\arabic{linenumber}}
            \input{0 - introduction/main.tex}
        \part{Research}
            \input{1 - low-noise PiC models/main.tex}
            \input{2 - kinetic component/main.tex}
            \input{3 - fluid component/main.tex}
            \input{4 - numerical implementation/main.tex}
        \part{Project Overview}
            \input{5 - research plan/main.tex}
            \input{6 - summary/main.tex}
    
    
    %\section{}
    \newpage
    \pagenumbering{gobble}
        \printbibliography


    \newpage
    \pagenumbering{roman}
    \appendix
        \part{Appendices}
            \input{8 - Hilbert complexes/main.tex}
            \input{9 - weak conservation proofs/main.tex}
\end{document}

\end{document}

        \part{Research}
            \documentclass[12pt, a4paper]{report}

\documentclass[12pt, a4paper]{report}

\input{template/main.tex}

\title{\BA{Title in Progress...}}
\author{Boris Andrews}
\affil{Mathematical Institute, University of Oxford}
\date{\today}


\begin{document}
    \pagenumbering{gobble}
    \maketitle
    
    
    \begin{abstract}
        Magnetic confinement reactors---in particular tokamaks---offer one of the most promising options for achieving practical nuclear fusion, with the potential to provide virtually limitless, clean energy. The theoretical and numerical modeling of tokamak plasmas is simultaneously an essential component of effective reactor design, and a great research barrier. Tokamak operational conditions exhibit comparatively low Knudsen numbers. Kinetic effects, including kinetic waves and instabilities, Landau damping, bump-on-tail instabilities and more, are therefore highly influential in tokamak plasma dynamics. Purely fluid models are inherently incapable of capturing these effects, whereas the high dimensionality in purely kinetic models render them practically intractable for most relevant purposes.

        We consider a $\delta\!f$ decomposition model, with a macroscopic fluid background and microscopic kinetic correction, both fully coupled to each other. A similar manner of discretization is proposed to that used in the recent \texttt{STRUPHY} code \cite{Holderied_Possanner_Wang_2021, Holderied_2022, Li_et_al_2023} with a finite-element model for the background and a pseudo-particle/PiC model for the correction.

        The fluid background satisfies the full, non-linear, resistive, compressible, Hall MHD equations. \cite{Laakmann_Hu_Farrell_2022} introduces finite-element(-in-space) implicit timesteppers for the incompressible analogue to this system with structure-preserving (SP) properties in the ideal case, alongside parameter-robust preconditioners. We show that these timesteppers can derive from a finite-element-in-time (FET) (and finite-element-in-space) interpretation. The benefits of this reformulation are discussed, including the derivation of timesteppers that are higher order in time, and the quantifiable dissipative SP properties in the non-ideal, resistive case.
        
        We discuss possible options for extending this FET approach to timesteppers for the compressible case.

        The kinetic corrections satisfy linearized Boltzmann equations. Using a Lénard--Bernstein collision operator, these take Fokker--Planck-like forms \cite{Fokker_1914, Planck_1917} wherein pseudo-particles in the numerical model obey the neoclassical transport equations, with particle-independent Brownian drift terms. This offers a rigorous methodology for incorporating collisions into the particle transport model, without coupling the equations of motions for each particle.
        
        Works by Chen, Chacón et al. \cite{Chen_Chacón_Barnes_2011, Chacón_Chen_Barnes_2013, Chen_Chacón_2014, Chen_Chacón_2015} have developed structure-preserving particle pushers for neoclassical transport in the Vlasov equations, derived from Crank--Nicolson integrators. We show these too can can derive from a FET interpretation, similarly offering potential extensions to higher-order-in-time particle pushers. The FET formulation is used also to consider how the stochastic drift terms can be incorporated into the pushers. Stochastic gyrokinetic expansions are also discussed.

        Different options for the numerical implementation of these schemes are considered.

        Due to the efficacy of FET in the development of SP timesteppers for both the fluid and kinetic component, we hope this approach will prove effective in the future for developing SP timesteppers for the full hybrid model. We hope this will give us the opportunity to incorporate previously inaccessible kinetic effects into the highly effective, modern, finite-element MHD models.
    \end{abstract}
    
    
    \newpage
    \tableofcontents
    
    
    \newpage
    \pagenumbering{arabic}
    %\linenumbers\renewcommand\thelinenumber{\color{black!50}\arabic{linenumber}}
            \input{0 - introduction/main.tex}
        \part{Research}
            \input{1 - low-noise PiC models/main.tex}
            \input{2 - kinetic component/main.tex}
            \input{3 - fluid component/main.tex}
            \input{4 - numerical implementation/main.tex}
        \part{Project Overview}
            \input{5 - research plan/main.tex}
            \input{6 - summary/main.tex}
    
    
    %\section{}
    \newpage
    \pagenumbering{gobble}
        \printbibliography


    \newpage
    \pagenumbering{roman}
    \appendix
        \part{Appendices}
            \input{8 - Hilbert complexes/main.tex}
            \input{9 - weak conservation proofs/main.tex}
\end{document}


\title{\BA{Title in Progress...}}
\author{Boris Andrews}
\affil{Mathematical Institute, University of Oxford}
\date{\today}


\begin{document}
    \pagenumbering{gobble}
    \maketitle
    
    
    \begin{abstract}
        Magnetic confinement reactors---in particular tokamaks---offer one of the most promising options for achieving practical nuclear fusion, with the potential to provide virtually limitless, clean energy. The theoretical and numerical modeling of tokamak plasmas is simultaneously an essential component of effective reactor design, and a great research barrier. Tokamak operational conditions exhibit comparatively low Knudsen numbers. Kinetic effects, including kinetic waves and instabilities, Landau damping, bump-on-tail instabilities and more, are therefore highly influential in tokamak plasma dynamics. Purely fluid models are inherently incapable of capturing these effects, whereas the high dimensionality in purely kinetic models render them practically intractable for most relevant purposes.

        We consider a $\delta\!f$ decomposition model, with a macroscopic fluid background and microscopic kinetic correction, both fully coupled to each other. A similar manner of discretization is proposed to that used in the recent \texttt{STRUPHY} code \cite{Holderied_Possanner_Wang_2021, Holderied_2022, Li_et_al_2023} with a finite-element model for the background and a pseudo-particle/PiC model for the correction.

        The fluid background satisfies the full, non-linear, resistive, compressible, Hall MHD equations. \cite{Laakmann_Hu_Farrell_2022} introduces finite-element(-in-space) implicit timesteppers for the incompressible analogue to this system with structure-preserving (SP) properties in the ideal case, alongside parameter-robust preconditioners. We show that these timesteppers can derive from a finite-element-in-time (FET) (and finite-element-in-space) interpretation. The benefits of this reformulation are discussed, including the derivation of timesteppers that are higher order in time, and the quantifiable dissipative SP properties in the non-ideal, resistive case.
        
        We discuss possible options for extending this FET approach to timesteppers for the compressible case.

        The kinetic corrections satisfy linearized Boltzmann equations. Using a Lénard--Bernstein collision operator, these take Fokker--Planck-like forms \cite{Fokker_1914, Planck_1917} wherein pseudo-particles in the numerical model obey the neoclassical transport equations, with particle-independent Brownian drift terms. This offers a rigorous methodology for incorporating collisions into the particle transport model, without coupling the equations of motions for each particle.
        
        Works by Chen, Chacón et al. \cite{Chen_Chacón_Barnes_2011, Chacón_Chen_Barnes_2013, Chen_Chacón_2014, Chen_Chacón_2015} have developed structure-preserving particle pushers for neoclassical transport in the Vlasov equations, derived from Crank--Nicolson integrators. We show these too can can derive from a FET interpretation, similarly offering potential extensions to higher-order-in-time particle pushers. The FET formulation is used also to consider how the stochastic drift terms can be incorporated into the pushers. Stochastic gyrokinetic expansions are also discussed.

        Different options for the numerical implementation of these schemes are considered.

        Due to the efficacy of FET in the development of SP timesteppers for both the fluid and kinetic component, we hope this approach will prove effective in the future for developing SP timesteppers for the full hybrid model. We hope this will give us the opportunity to incorporate previously inaccessible kinetic effects into the highly effective, modern, finite-element MHD models.
    \end{abstract}
    
    
    \newpage
    \tableofcontents
    
    
    \newpage
    \pagenumbering{arabic}
    %\linenumbers\renewcommand\thelinenumber{\color{black!50}\arabic{linenumber}}
            \documentclass[12pt, a4paper]{report}

\input{template/main.tex}

\title{\BA{Title in Progress...}}
\author{Boris Andrews}
\affil{Mathematical Institute, University of Oxford}
\date{\today}


\begin{document}
    \pagenumbering{gobble}
    \maketitle
    
    
    \begin{abstract}
        Magnetic confinement reactors---in particular tokamaks---offer one of the most promising options for achieving practical nuclear fusion, with the potential to provide virtually limitless, clean energy. The theoretical and numerical modeling of tokamak plasmas is simultaneously an essential component of effective reactor design, and a great research barrier. Tokamak operational conditions exhibit comparatively low Knudsen numbers. Kinetic effects, including kinetic waves and instabilities, Landau damping, bump-on-tail instabilities and more, are therefore highly influential in tokamak plasma dynamics. Purely fluid models are inherently incapable of capturing these effects, whereas the high dimensionality in purely kinetic models render them practically intractable for most relevant purposes.

        We consider a $\delta\!f$ decomposition model, with a macroscopic fluid background and microscopic kinetic correction, both fully coupled to each other. A similar manner of discretization is proposed to that used in the recent \texttt{STRUPHY} code \cite{Holderied_Possanner_Wang_2021, Holderied_2022, Li_et_al_2023} with a finite-element model for the background and a pseudo-particle/PiC model for the correction.

        The fluid background satisfies the full, non-linear, resistive, compressible, Hall MHD equations. \cite{Laakmann_Hu_Farrell_2022} introduces finite-element(-in-space) implicit timesteppers for the incompressible analogue to this system with structure-preserving (SP) properties in the ideal case, alongside parameter-robust preconditioners. We show that these timesteppers can derive from a finite-element-in-time (FET) (and finite-element-in-space) interpretation. The benefits of this reformulation are discussed, including the derivation of timesteppers that are higher order in time, and the quantifiable dissipative SP properties in the non-ideal, resistive case.
        
        We discuss possible options for extending this FET approach to timesteppers for the compressible case.

        The kinetic corrections satisfy linearized Boltzmann equations. Using a Lénard--Bernstein collision operator, these take Fokker--Planck-like forms \cite{Fokker_1914, Planck_1917} wherein pseudo-particles in the numerical model obey the neoclassical transport equations, with particle-independent Brownian drift terms. This offers a rigorous methodology for incorporating collisions into the particle transport model, without coupling the equations of motions for each particle.
        
        Works by Chen, Chacón et al. \cite{Chen_Chacón_Barnes_2011, Chacón_Chen_Barnes_2013, Chen_Chacón_2014, Chen_Chacón_2015} have developed structure-preserving particle pushers for neoclassical transport in the Vlasov equations, derived from Crank--Nicolson integrators. We show these too can can derive from a FET interpretation, similarly offering potential extensions to higher-order-in-time particle pushers. The FET formulation is used also to consider how the stochastic drift terms can be incorporated into the pushers. Stochastic gyrokinetic expansions are also discussed.

        Different options for the numerical implementation of these schemes are considered.

        Due to the efficacy of FET in the development of SP timesteppers for both the fluid and kinetic component, we hope this approach will prove effective in the future for developing SP timesteppers for the full hybrid model. We hope this will give us the opportunity to incorporate previously inaccessible kinetic effects into the highly effective, modern, finite-element MHD models.
    \end{abstract}
    
    
    \newpage
    \tableofcontents
    
    
    \newpage
    \pagenumbering{arabic}
    %\linenumbers\renewcommand\thelinenumber{\color{black!50}\arabic{linenumber}}
            \input{0 - introduction/main.tex}
        \part{Research}
            \input{1 - low-noise PiC models/main.tex}
            \input{2 - kinetic component/main.tex}
            \input{3 - fluid component/main.tex}
            \input{4 - numerical implementation/main.tex}
        \part{Project Overview}
            \input{5 - research plan/main.tex}
            \input{6 - summary/main.tex}
    
    
    %\section{}
    \newpage
    \pagenumbering{gobble}
        \printbibliography


    \newpage
    \pagenumbering{roman}
    \appendix
        \part{Appendices}
            \input{8 - Hilbert complexes/main.tex}
            \input{9 - weak conservation proofs/main.tex}
\end{document}

        \part{Research}
            \documentclass[12pt, a4paper]{report}

\input{template/main.tex}

\title{\BA{Title in Progress...}}
\author{Boris Andrews}
\affil{Mathematical Institute, University of Oxford}
\date{\today}


\begin{document}
    \pagenumbering{gobble}
    \maketitle
    
    
    \begin{abstract}
        Magnetic confinement reactors---in particular tokamaks---offer one of the most promising options for achieving practical nuclear fusion, with the potential to provide virtually limitless, clean energy. The theoretical and numerical modeling of tokamak plasmas is simultaneously an essential component of effective reactor design, and a great research barrier. Tokamak operational conditions exhibit comparatively low Knudsen numbers. Kinetic effects, including kinetic waves and instabilities, Landau damping, bump-on-tail instabilities and more, are therefore highly influential in tokamak plasma dynamics. Purely fluid models are inherently incapable of capturing these effects, whereas the high dimensionality in purely kinetic models render them practically intractable for most relevant purposes.

        We consider a $\delta\!f$ decomposition model, with a macroscopic fluid background and microscopic kinetic correction, both fully coupled to each other. A similar manner of discretization is proposed to that used in the recent \texttt{STRUPHY} code \cite{Holderied_Possanner_Wang_2021, Holderied_2022, Li_et_al_2023} with a finite-element model for the background and a pseudo-particle/PiC model for the correction.

        The fluid background satisfies the full, non-linear, resistive, compressible, Hall MHD equations. \cite{Laakmann_Hu_Farrell_2022} introduces finite-element(-in-space) implicit timesteppers for the incompressible analogue to this system with structure-preserving (SP) properties in the ideal case, alongside parameter-robust preconditioners. We show that these timesteppers can derive from a finite-element-in-time (FET) (and finite-element-in-space) interpretation. The benefits of this reformulation are discussed, including the derivation of timesteppers that are higher order in time, and the quantifiable dissipative SP properties in the non-ideal, resistive case.
        
        We discuss possible options for extending this FET approach to timesteppers for the compressible case.

        The kinetic corrections satisfy linearized Boltzmann equations. Using a Lénard--Bernstein collision operator, these take Fokker--Planck-like forms \cite{Fokker_1914, Planck_1917} wherein pseudo-particles in the numerical model obey the neoclassical transport equations, with particle-independent Brownian drift terms. This offers a rigorous methodology for incorporating collisions into the particle transport model, without coupling the equations of motions for each particle.
        
        Works by Chen, Chacón et al. \cite{Chen_Chacón_Barnes_2011, Chacón_Chen_Barnes_2013, Chen_Chacón_2014, Chen_Chacón_2015} have developed structure-preserving particle pushers for neoclassical transport in the Vlasov equations, derived from Crank--Nicolson integrators. We show these too can can derive from a FET interpretation, similarly offering potential extensions to higher-order-in-time particle pushers. The FET formulation is used also to consider how the stochastic drift terms can be incorporated into the pushers. Stochastic gyrokinetic expansions are also discussed.

        Different options for the numerical implementation of these schemes are considered.

        Due to the efficacy of FET in the development of SP timesteppers for both the fluid and kinetic component, we hope this approach will prove effective in the future for developing SP timesteppers for the full hybrid model. We hope this will give us the opportunity to incorporate previously inaccessible kinetic effects into the highly effective, modern, finite-element MHD models.
    \end{abstract}
    
    
    \newpage
    \tableofcontents
    
    
    \newpage
    \pagenumbering{arabic}
    %\linenumbers\renewcommand\thelinenumber{\color{black!50}\arabic{linenumber}}
            \input{0 - introduction/main.tex}
        \part{Research}
            \input{1 - low-noise PiC models/main.tex}
            \input{2 - kinetic component/main.tex}
            \input{3 - fluid component/main.tex}
            \input{4 - numerical implementation/main.tex}
        \part{Project Overview}
            \input{5 - research plan/main.tex}
            \input{6 - summary/main.tex}
    
    
    %\section{}
    \newpage
    \pagenumbering{gobble}
        \printbibliography


    \newpage
    \pagenumbering{roman}
    \appendix
        \part{Appendices}
            \input{8 - Hilbert complexes/main.tex}
            \input{9 - weak conservation proofs/main.tex}
\end{document}

            \documentclass[12pt, a4paper]{report}

\input{template/main.tex}

\title{\BA{Title in Progress...}}
\author{Boris Andrews}
\affil{Mathematical Institute, University of Oxford}
\date{\today}


\begin{document}
    \pagenumbering{gobble}
    \maketitle
    
    
    \begin{abstract}
        Magnetic confinement reactors---in particular tokamaks---offer one of the most promising options for achieving practical nuclear fusion, with the potential to provide virtually limitless, clean energy. The theoretical and numerical modeling of tokamak plasmas is simultaneously an essential component of effective reactor design, and a great research barrier. Tokamak operational conditions exhibit comparatively low Knudsen numbers. Kinetic effects, including kinetic waves and instabilities, Landau damping, bump-on-tail instabilities and more, are therefore highly influential in tokamak plasma dynamics. Purely fluid models are inherently incapable of capturing these effects, whereas the high dimensionality in purely kinetic models render them practically intractable for most relevant purposes.

        We consider a $\delta\!f$ decomposition model, with a macroscopic fluid background and microscopic kinetic correction, both fully coupled to each other. A similar manner of discretization is proposed to that used in the recent \texttt{STRUPHY} code \cite{Holderied_Possanner_Wang_2021, Holderied_2022, Li_et_al_2023} with a finite-element model for the background and a pseudo-particle/PiC model for the correction.

        The fluid background satisfies the full, non-linear, resistive, compressible, Hall MHD equations. \cite{Laakmann_Hu_Farrell_2022} introduces finite-element(-in-space) implicit timesteppers for the incompressible analogue to this system with structure-preserving (SP) properties in the ideal case, alongside parameter-robust preconditioners. We show that these timesteppers can derive from a finite-element-in-time (FET) (and finite-element-in-space) interpretation. The benefits of this reformulation are discussed, including the derivation of timesteppers that are higher order in time, and the quantifiable dissipative SP properties in the non-ideal, resistive case.
        
        We discuss possible options for extending this FET approach to timesteppers for the compressible case.

        The kinetic corrections satisfy linearized Boltzmann equations. Using a Lénard--Bernstein collision operator, these take Fokker--Planck-like forms \cite{Fokker_1914, Planck_1917} wherein pseudo-particles in the numerical model obey the neoclassical transport equations, with particle-independent Brownian drift terms. This offers a rigorous methodology for incorporating collisions into the particle transport model, without coupling the equations of motions for each particle.
        
        Works by Chen, Chacón et al. \cite{Chen_Chacón_Barnes_2011, Chacón_Chen_Barnes_2013, Chen_Chacón_2014, Chen_Chacón_2015} have developed structure-preserving particle pushers for neoclassical transport in the Vlasov equations, derived from Crank--Nicolson integrators. We show these too can can derive from a FET interpretation, similarly offering potential extensions to higher-order-in-time particle pushers. The FET formulation is used also to consider how the stochastic drift terms can be incorporated into the pushers. Stochastic gyrokinetic expansions are also discussed.

        Different options for the numerical implementation of these schemes are considered.

        Due to the efficacy of FET in the development of SP timesteppers for both the fluid and kinetic component, we hope this approach will prove effective in the future for developing SP timesteppers for the full hybrid model. We hope this will give us the opportunity to incorporate previously inaccessible kinetic effects into the highly effective, modern, finite-element MHD models.
    \end{abstract}
    
    
    \newpage
    \tableofcontents
    
    
    \newpage
    \pagenumbering{arabic}
    %\linenumbers\renewcommand\thelinenumber{\color{black!50}\arabic{linenumber}}
            \input{0 - introduction/main.tex}
        \part{Research}
            \input{1 - low-noise PiC models/main.tex}
            \input{2 - kinetic component/main.tex}
            \input{3 - fluid component/main.tex}
            \input{4 - numerical implementation/main.tex}
        \part{Project Overview}
            \input{5 - research plan/main.tex}
            \input{6 - summary/main.tex}
    
    
    %\section{}
    \newpage
    \pagenumbering{gobble}
        \printbibliography


    \newpage
    \pagenumbering{roman}
    \appendix
        \part{Appendices}
            \input{8 - Hilbert complexes/main.tex}
            \input{9 - weak conservation proofs/main.tex}
\end{document}

            \documentclass[12pt, a4paper]{report}

\input{template/main.tex}

\title{\BA{Title in Progress...}}
\author{Boris Andrews}
\affil{Mathematical Institute, University of Oxford}
\date{\today}


\begin{document}
    \pagenumbering{gobble}
    \maketitle
    
    
    \begin{abstract}
        Magnetic confinement reactors---in particular tokamaks---offer one of the most promising options for achieving practical nuclear fusion, with the potential to provide virtually limitless, clean energy. The theoretical and numerical modeling of tokamak plasmas is simultaneously an essential component of effective reactor design, and a great research barrier. Tokamak operational conditions exhibit comparatively low Knudsen numbers. Kinetic effects, including kinetic waves and instabilities, Landau damping, bump-on-tail instabilities and more, are therefore highly influential in tokamak plasma dynamics. Purely fluid models are inherently incapable of capturing these effects, whereas the high dimensionality in purely kinetic models render them practically intractable for most relevant purposes.

        We consider a $\delta\!f$ decomposition model, with a macroscopic fluid background and microscopic kinetic correction, both fully coupled to each other. A similar manner of discretization is proposed to that used in the recent \texttt{STRUPHY} code \cite{Holderied_Possanner_Wang_2021, Holderied_2022, Li_et_al_2023} with a finite-element model for the background and a pseudo-particle/PiC model for the correction.

        The fluid background satisfies the full, non-linear, resistive, compressible, Hall MHD equations. \cite{Laakmann_Hu_Farrell_2022} introduces finite-element(-in-space) implicit timesteppers for the incompressible analogue to this system with structure-preserving (SP) properties in the ideal case, alongside parameter-robust preconditioners. We show that these timesteppers can derive from a finite-element-in-time (FET) (and finite-element-in-space) interpretation. The benefits of this reformulation are discussed, including the derivation of timesteppers that are higher order in time, and the quantifiable dissipative SP properties in the non-ideal, resistive case.
        
        We discuss possible options for extending this FET approach to timesteppers for the compressible case.

        The kinetic corrections satisfy linearized Boltzmann equations. Using a Lénard--Bernstein collision operator, these take Fokker--Planck-like forms \cite{Fokker_1914, Planck_1917} wherein pseudo-particles in the numerical model obey the neoclassical transport equations, with particle-independent Brownian drift terms. This offers a rigorous methodology for incorporating collisions into the particle transport model, without coupling the equations of motions for each particle.
        
        Works by Chen, Chacón et al. \cite{Chen_Chacón_Barnes_2011, Chacón_Chen_Barnes_2013, Chen_Chacón_2014, Chen_Chacón_2015} have developed structure-preserving particle pushers for neoclassical transport in the Vlasov equations, derived from Crank--Nicolson integrators. We show these too can can derive from a FET interpretation, similarly offering potential extensions to higher-order-in-time particle pushers. The FET formulation is used also to consider how the stochastic drift terms can be incorporated into the pushers. Stochastic gyrokinetic expansions are also discussed.

        Different options for the numerical implementation of these schemes are considered.

        Due to the efficacy of FET in the development of SP timesteppers for both the fluid and kinetic component, we hope this approach will prove effective in the future for developing SP timesteppers for the full hybrid model. We hope this will give us the opportunity to incorporate previously inaccessible kinetic effects into the highly effective, modern, finite-element MHD models.
    \end{abstract}
    
    
    \newpage
    \tableofcontents
    
    
    \newpage
    \pagenumbering{arabic}
    %\linenumbers\renewcommand\thelinenumber{\color{black!50}\arabic{linenumber}}
            \input{0 - introduction/main.tex}
        \part{Research}
            \input{1 - low-noise PiC models/main.tex}
            \input{2 - kinetic component/main.tex}
            \input{3 - fluid component/main.tex}
            \input{4 - numerical implementation/main.tex}
        \part{Project Overview}
            \input{5 - research plan/main.tex}
            \input{6 - summary/main.tex}
    
    
    %\section{}
    \newpage
    \pagenumbering{gobble}
        \printbibliography


    \newpage
    \pagenumbering{roman}
    \appendix
        \part{Appendices}
            \input{8 - Hilbert complexes/main.tex}
            \input{9 - weak conservation proofs/main.tex}
\end{document}

            \documentclass[12pt, a4paper]{report}

\input{template/main.tex}

\title{\BA{Title in Progress...}}
\author{Boris Andrews}
\affil{Mathematical Institute, University of Oxford}
\date{\today}


\begin{document}
    \pagenumbering{gobble}
    \maketitle
    
    
    \begin{abstract}
        Magnetic confinement reactors---in particular tokamaks---offer one of the most promising options for achieving practical nuclear fusion, with the potential to provide virtually limitless, clean energy. The theoretical and numerical modeling of tokamak plasmas is simultaneously an essential component of effective reactor design, and a great research barrier. Tokamak operational conditions exhibit comparatively low Knudsen numbers. Kinetic effects, including kinetic waves and instabilities, Landau damping, bump-on-tail instabilities and more, are therefore highly influential in tokamak plasma dynamics. Purely fluid models are inherently incapable of capturing these effects, whereas the high dimensionality in purely kinetic models render them practically intractable for most relevant purposes.

        We consider a $\delta\!f$ decomposition model, with a macroscopic fluid background and microscopic kinetic correction, both fully coupled to each other. A similar manner of discretization is proposed to that used in the recent \texttt{STRUPHY} code \cite{Holderied_Possanner_Wang_2021, Holderied_2022, Li_et_al_2023} with a finite-element model for the background and a pseudo-particle/PiC model for the correction.

        The fluid background satisfies the full, non-linear, resistive, compressible, Hall MHD equations. \cite{Laakmann_Hu_Farrell_2022} introduces finite-element(-in-space) implicit timesteppers for the incompressible analogue to this system with structure-preserving (SP) properties in the ideal case, alongside parameter-robust preconditioners. We show that these timesteppers can derive from a finite-element-in-time (FET) (and finite-element-in-space) interpretation. The benefits of this reformulation are discussed, including the derivation of timesteppers that are higher order in time, and the quantifiable dissipative SP properties in the non-ideal, resistive case.
        
        We discuss possible options for extending this FET approach to timesteppers for the compressible case.

        The kinetic corrections satisfy linearized Boltzmann equations. Using a Lénard--Bernstein collision operator, these take Fokker--Planck-like forms \cite{Fokker_1914, Planck_1917} wherein pseudo-particles in the numerical model obey the neoclassical transport equations, with particle-independent Brownian drift terms. This offers a rigorous methodology for incorporating collisions into the particle transport model, without coupling the equations of motions for each particle.
        
        Works by Chen, Chacón et al. \cite{Chen_Chacón_Barnes_2011, Chacón_Chen_Barnes_2013, Chen_Chacón_2014, Chen_Chacón_2015} have developed structure-preserving particle pushers for neoclassical transport in the Vlasov equations, derived from Crank--Nicolson integrators. We show these too can can derive from a FET interpretation, similarly offering potential extensions to higher-order-in-time particle pushers. The FET formulation is used also to consider how the stochastic drift terms can be incorporated into the pushers. Stochastic gyrokinetic expansions are also discussed.

        Different options for the numerical implementation of these schemes are considered.

        Due to the efficacy of FET in the development of SP timesteppers for both the fluid and kinetic component, we hope this approach will prove effective in the future for developing SP timesteppers for the full hybrid model. We hope this will give us the opportunity to incorporate previously inaccessible kinetic effects into the highly effective, modern, finite-element MHD models.
    \end{abstract}
    
    
    \newpage
    \tableofcontents
    
    
    \newpage
    \pagenumbering{arabic}
    %\linenumbers\renewcommand\thelinenumber{\color{black!50}\arabic{linenumber}}
            \input{0 - introduction/main.tex}
        \part{Research}
            \input{1 - low-noise PiC models/main.tex}
            \input{2 - kinetic component/main.tex}
            \input{3 - fluid component/main.tex}
            \input{4 - numerical implementation/main.tex}
        \part{Project Overview}
            \input{5 - research plan/main.tex}
            \input{6 - summary/main.tex}
    
    
    %\section{}
    \newpage
    \pagenumbering{gobble}
        \printbibliography


    \newpage
    \pagenumbering{roman}
    \appendix
        \part{Appendices}
            \input{8 - Hilbert complexes/main.tex}
            \input{9 - weak conservation proofs/main.tex}
\end{document}

        \part{Project Overview}
            \documentclass[12pt, a4paper]{report}

\input{template/main.tex}

\title{\BA{Title in Progress...}}
\author{Boris Andrews}
\affil{Mathematical Institute, University of Oxford}
\date{\today}


\begin{document}
    \pagenumbering{gobble}
    \maketitle
    
    
    \begin{abstract}
        Magnetic confinement reactors---in particular tokamaks---offer one of the most promising options for achieving practical nuclear fusion, with the potential to provide virtually limitless, clean energy. The theoretical and numerical modeling of tokamak plasmas is simultaneously an essential component of effective reactor design, and a great research barrier. Tokamak operational conditions exhibit comparatively low Knudsen numbers. Kinetic effects, including kinetic waves and instabilities, Landau damping, bump-on-tail instabilities and more, are therefore highly influential in tokamak plasma dynamics. Purely fluid models are inherently incapable of capturing these effects, whereas the high dimensionality in purely kinetic models render them practically intractable for most relevant purposes.

        We consider a $\delta\!f$ decomposition model, with a macroscopic fluid background and microscopic kinetic correction, both fully coupled to each other. A similar manner of discretization is proposed to that used in the recent \texttt{STRUPHY} code \cite{Holderied_Possanner_Wang_2021, Holderied_2022, Li_et_al_2023} with a finite-element model for the background and a pseudo-particle/PiC model for the correction.

        The fluid background satisfies the full, non-linear, resistive, compressible, Hall MHD equations. \cite{Laakmann_Hu_Farrell_2022} introduces finite-element(-in-space) implicit timesteppers for the incompressible analogue to this system with structure-preserving (SP) properties in the ideal case, alongside parameter-robust preconditioners. We show that these timesteppers can derive from a finite-element-in-time (FET) (and finite-element-in-space) interpretation. The benefits of this reformulation are discussed, including the derivation of timesteppers that are higher order in time, and the quantifiable dissipative SP properties in the non-ideal, resistive case.
        
        We discuss possible options for extending this FET approach to timesteppers for the compressible case.

        The kinetic corrections satisfy linearized Boltzmann equations. Using a Lénard--Bernstein collision operator, these take Fokker--Planck-like forms \cite{Fokker_1914, Planck_1917} wherein pseudo-particles in the numerical model obey the neoclassical transport equations, with particle-independent Brownian drift terms. This offers a rigorous methodology for incorporating collisions into the particle transport model, without coupling the equations of motions for each particle.
        
        Works by Chen, Chacón et al. \cite{Chen_Chacón_Barnes_2011, Chacón_Chen_Barnes_2013, Chen_Chacón_2014, Chen_Chacón_2015} have developed structure-preserving particle pushers for neoclassical transport in the Vlasov equations, derived from Crank--Nicolson integrators. We show these too can can derive from a FET interpretation, similarly offering potential extensions to higher-order-in-time particle pushers. The FET formulation is used also to consider how the stochastic drift terms can be incorporated into the pushers. Stochastic gyrokinetic expansions are also discussed.

        Different options for the numerical implementation of these schemes are considered.

        Due to the efficacy of FET in the development of SP timesteppers for both the fluid and kinetic component, we hope this approach will prove effective in the future for developing SP timesteppers for the full hybrid model. We hope this will give us the opportunity to incorporate previously inaccessible kinetic effects into the highly effective, modern, finite-element MHD models.
    \end{abstract}
    
    
    \newpage
    \tableofcontents
    
    
    \newpage
    \pagenumbering{arabic}
    %\linenumbers\renewcommand\thelinenumber{\color{black!50}\arabic{linenumber}}
            \input{0 - introduction/main.tex}
        \part{Research}
            \input{1 - low-noise PiC models/main.tex}
            \input{2 - kinetic component/main.tex}
            \input{3 - fluid component/main.tex}
            \input{4 - numerical implementation/main.tex}
        \part{Project Overview}
            \input{5 - research plan/main.tex}
            \input{6 - summary/main.tex}
    
    
    %\section{}
    \newpage
    \pagenumbering{gobble}
        \printbibliography


    \newpage
    \pagenumbering{roman}
    \appendix
        \part{Appendices}
            \input{8 - Hilbert complexes/main.tex}
            \input{9 - weak conservation proofs/main.tex}
\end{document}

            \documentclass[12pt, a4paper]{report}

\input{template/main.tex}

\title{\BA{Title in Progress...}}
\author{Boris Andrews}
\affil{Mathematical Institute, University of Oxford}
\date{\today}


\begin{document}
    \pagenumbering{gobble}
    \maketitle
    
    
    \begin{abstract}
        Magnetic confinement reactors---in particular tokamaks---offer one of the most promising options for achieving practical nuclear fusion, with the potential to provide virtually limitless, clean energy. The theoretical and numerical modeling of tokamak plasmas is simultaneously an essential component of effective reactor design, and a great research barrier. Tokamak operational conditions exhibit comparatively low Knudsen numbers. Kinetic effects, including kinetic waves and instabilities, Landau damping, bump-on-tail instabilities and more, are therefore highly influential in tokamak plasma dynamics. Purely fluid models are inherently incapable of capturing these effects, whereas the high dimensionality in purely kinetic models render them practically intractable for most relevant purposes.

        We consider a $\delta\!f$ decomposition model, with a macroscopic fluid background and microscopic kinetic correction, both fully coupled to each other. A similar manner of discretization is proposed to that used in the recent \texttt{STRUPHY} code \cite{Holderied_Possanner_Wang_2021, Holderied_2022, Li_et_al_2023} with a finite-element model for the background and a pseudo-particle/PiC model for the correction.

        The fluid background satisfies the full, non-linear, resistive, compressible, Hall MHD equations. \cite{Laakmann_Hu_Farrell_2022} introduces finite-element(-in-space) implicit timesteppers for the incompressible analogue to this system with structure-preserving (SP) properties in the ideal case, alongside parameter-robust preconditioners. We show that these timesteppers can derive from a finite-element-in-time (FET) (and finite-element-in-space) interpretation. The benefits of this reformulation are discussed, including the derivation of timesteppers that are higher order in time, and the quantifiable dissipative SP properties in the non-ideal, resistive case.
        
        We discuss possible options for extending this FET approach to timesteppers for the compressible case.

        The kinetic corrections satisfy linearized Boltzmann equations. Using a Lénard--Bernstein collision operator, these take Fokker--Planck-like forms \cite{Fokker_1914, Planck_1917} wherein pseudo-particles in the numerical model obey the neoclassical transport equations, with particle-independent Brownian drift terms. This offers a rigorous methodology for incorporating collisions into the particle transport model, without coupling the equations of motions for each particle.
        
        Works by Chen, Chacón et al. \cite{Chen_Chacón_Barnes_2011, Chacón_Chen_Barnes_2013, Chen_Chacón_2014, Chen_Chacón_2015} have developed structure-preserving particle pushers for neoclassical transport in the Vlasov equations, derived from Crank--Nicolson integrators. We show these too can can derive from a FET interpretation, similarly offering potential extensions to higher-order-in-time particle pushers. The FET formulation is used also to consider how the stochastic drift terms can be incorporated into the pushers. Stochastic gyrokinetic expansions are also discussed.

        Different options for the numerical implementation of these schemes are considered.

        Due to the efficacy of FET in the development of SP timesteppers for both the fluid and kinetic component, we hope this approach will prove effective in the future for developing SP timesteppers for the full hybrid model. We hope this will give us the opportunity to incorporate previously inaccessible kinetic effects into the highly effective, modern, finite-element MHD models.
    \end{abstract}
    
    
    \newpage
    \tableofcontents
    
    
    \newpage
    \pagenumbering{arabic}
    %\linenumbers\renewcommand\thelinenumber{\color{black!50}\arabic{linenumber}}
            \input{0 - introduction/main.tex}
        \part{Research}
            \input{1 - low-noise PiC models/main.tex}
            \input{2 - kinetic component/main.tex}
            \input{3 - fluid component/main.tex}
            \input{4 - numerical implementation/main.tex}
        \part{Project Overview}
            \input{5 - research plan/main.tex}
            \input{6 - summary/main.tex}
    
    
    %\section{}
    \newpage
    \pagenumbering{gobble}
        \printbibliography


    \newpage
    \pagenumbering{roman}
    \appendix
        \part{Appendices}
            \input{8 - Hilbert complexes/main.tex}
            \input{9 - weak conservation proofs/main.tex}
\end{document}

    
    
    %\section{}
    \newpage
    \pagenumbering{gobble}
        \printbibliography


    \newpage
    \pagenumbering{roman}
    \appendix
        \part{Appendices}
            \documentclass[12pt, a4paper]{report}

\input{template/main.tex}

\title{\BA{Title in Progress...}}
\author{Boris Andrews}
\affil{Mathematical Institute, University of Oxford}
\date{\today}


\begin{document}
    \pagenumbering{gobble}
    \maketitle
    
    
    \begin{abstract}
        Magnetic confinement reactors---in particular tokamaks---offer one of the most promising options for achieving practical nuclear fusion, with the potential to provide virtually limitless, clean energy. The theoretical and numerical modeling of tokamak plasmas is simultaneously an essential component of effective reactor design, and a great research barrier. Tokamak operational conditions exhibit comparatively low Knudsen numbers. Kinetic effects, including kinetic waves and instabilities, Landau damping, bump-on-tail instabilities and more, are therefore highly influential in tokamak plasma dynamics. Purely fluid models are inherently incapable of capturing these effects, whereas the high dimensionality in purely kinetic models render them practically intractable for most relevant purposes.

        We consider a $\delta\!f$ decomposition model, with a macroscopic fluid background and microscopic kinetic correction, both fully coupled to each other. A similar manner of discretization is proposed to that used in the recent \texttt{STRUPHY} code \cite{Holderied_Possanner_Wang_2021, Holderied_2022, Li_et_al_2023} with a finite-element model for the background and a pseudo-particle/PiC model for the correction.

        The fluid background satisfies the full, non-linear, resistive, compressible, Hall MHD equations. \cite{Laakmann_Hu_Farrell_2022} introduces finite-element(-in-space) implicit timesteppers for the incompressible analogue to this system with structure-preserving (SP) properties in the ideal case, alongside parameter-robust preconditioners. We show that these timesteppers can derive from a finite-element-in-time (FET) (and finite-element-in-space) interpretation. The benefits of this reformulation are discussed, including the derivation of timesteppers that are higher order in time, and the quantifiable dissipative SP properties in the non-ideal, resistive case.
        
        We discuss possible options for extending this FET approach to timesteppers for the compressible case.

        The kinetic corrections satisfy linearized Boltzmann equations. Using a Lénard--Bernstein collision operator, these take Fokker--Planck-like forms \cite{Fokker_1914, Planck_1917} wherein pseudo-particles in the numerical model obey the neoclassical transport equations, with particle-independent Brownian drift terms. This offers a rigorous methodology for incorporating collisions into the particle transport model, without coupling the equations of motions for each particle.
        
        Works by Chen, Chacón et al. \cite{Chen_Chacón_Barnes_2011, Chacón_Chen_Barnes_2013, Chen_Chacón_2014, Chen_Chacón_2015} have developed structure-preserving particle pushers for neoclassical transport in the Vlasov equations, derived from Crank--Nicolson integrators. We show these too can can derive from a FET interpretation, similarly offering potential extensions to higher-order-in-time particle pushers. The FET formulation is used also to consider how the stochastic drift terms can be incorporated into the pushers. Stochastic gyrokinetic expansions are also discussed.

        Different options for the numerical implementation of these schemes are considered.

        Due to the efficacy of FET in the development of SP timesteppers for both the fluid and kinetic component, we hope this approach will prove effective in the future for developing SP timesteppers for the full hybrid model. We hope this will give us the opportunity to incorporate previously inaccessible kinetic effects into the highly effective, modern, finite-element MHD models.
    \end{abstract}
    
    
    \newpage
    \tableofcontents
    
    
    \newpage
    \pagenumbering{arabic}
    %\linenumbers\renewcommand\thelinenumber{\color{black!50}\arabic{linenumber}}
            \input{0 - introduction/main.tex}
        \part{Research}
            \input{1 - low-noise PiC models/main.tex}
            \input{2 - kinetic component/main.tex}
            \input{3 - fluid component/main.tex}
            \input{4 - numerical implementation/main.tex}
        \part{Project Overview}
            \input{5 - research plan/main.tex}
            \input{6 - summary/main.tex}
    
    
    %\section{}
    \newpage
    \pagenumbering{gobble}
        \printbibliography


    \newpage
    \pagenumbering{roman}
    \appendix
        \part{Appendices}
            \input{8 - Hilbert complexes/main.tex}
            \input{9 - weak conservation proofs/main.tex}
\end{document}

            \documentclass[12pt, a4paper]{report}

\input{template/main.tex}

\title{\BA{Title in Progress...}}
\author{Boris Andrews}
\affil{Mathematical Institute, University of Oxford}
\date{\today}


\begin{document}
    \pagenumbering{gobble}
    \maketitle
    
    
    \begin{abstract}
        Magnetic confinement reactors---in particular tokamaks---offer one of the most promising options for achieving practical nuclear fusion, with the potential to provide virtually limitless, clean energy. The theoretical and numerical modeling of tokamak plasmas is simultaneously an essential component of effective reactor design, and a great research barrier. Tokamak operational conditions exhibit comparatively low Knudsen numbers. Kinetic effects, including kinetic waves and instabilities, Landau damping, bump-on-tail instabilities and more, are therefore highly influential in tokamak plasma dynamics. Purely fluid models are inherently incapable of capturing these effects, whereas the high dimensionality in purely kinetic models render them practically intractable for most relevant purposes.

        We consider a $\delta\!f$ decomposition model, with a macroscopic fluid background and microscopic kinetic correction, both fully coupled to each other. A similar manner of discretization is proposed to that used in the recent \texttt{STRUPHY} code \cite{Holderied_Possanner_Wang_2021, Holderied_2022, Li_et_al_2023} with a finite-element model for the background and a pseudo-particle/PiC model for the correction.

        The fluid background satisfies the full, non-linear, resistive, compressible, Hall MHD equations. \cite{Laakmann_Hu_Farrell_2022} introduces finite-element(-in-space) implicit timesteppers for the incompressible analogue to this system with structure-preserving (SP) properties in the ideal case, alongside parameter-robust preconditioners. We show that these timesteppers can derive from a finite-element-in-time (FET) (and finite-element-in-space) interpretation. The benefits of this reformulation are discussed, including the derivation of timesteppers that are higher order in time, and the quantifiable dissipative SP properties in the non-ideal, resistive case.
        
        We discuss possible options for extending this FET approach to timesteppers for the compressible case.

        The kinetic corrections satisfy linearized Boltzmann equations. Using a Lénard--Bernstein collision operator, these take Fokker--Planck-like forms \cite{Fokker_1914, Planck_1917} wherein pseudo-particles in the numerical model obey the neoclassical transport equations, with particle-independent Brownian drift terms. This offers a rigorous methodology for incorporating collisions into the particle transport model, without coupling the equations of motions for each particle.
        
        Works by Chen, Chacón et al. \cite{Chen_Chacón_Barnes_2011, Chacón_Chen_Barnes_2013, Chen_Chacón_2014, Chen_Chacón_2015} have developed structure-preserving particle pushers for neoclassical transport in the Vlasov equations, derived from Crank--Nicolson integrators. We show these too can can derive from a FET interpretation, similarly offering potential extensions to higher-order-in-time particle pushers. The FET formulation is used also to consider how the stochastic drift terms can be incorporated into the pushers. Stochastic gyrokinetic expansions are also discussed.

        Different options for the numerical implementation of these schemes are considered.

        Due to the efficacy of FET in the development of SP timesteppers for both the fluid and kinetic component, we hope this approach will prove effective in the future for developing SP timesteppers for the full hybrid model. We hope this will give us the opportunity to incorporate previously inaccessible kinetic effects into the highly effective, modern, finite-element MHD models.
    \end{abstract}
    
    
    \newpage
    \tableofcontents
    
    
    \newpage
    \pagenumbering{arabic}
    %\linenumbers\renewcommand\thelinenumber{\color{black!50}\arabic{linenumber}}
            \input{0 - introduction/main.tex}
        \part{Research}
            \input{1 - low-noise PiC models/main.tex}
            \input{2 - kinetic component/main.tex}
            \input{3 - fluid component/main.tex}
            \input{4 - numerical implementation/main.tex}
        \part{Project Overview}
            \input{5 - research plan/main.tex}
            \input{6 - summary/main.tex}
    
    
    %\section{}
    \newpage
    \pagenumbering{gobble}
        \printbibliography


    \newpage
    \pagenumbering{roman}
    \appendix
        \part{Appendices}
            \input{8 - Hilbert complexes/main.tex}
            \input{9 - weak conservation proofs/main.tex}
\end{document}

\end{document}

            \documentclass[12pt, a4paper]{report}

\documentclass[12pt, a4paper]{report}

\input{template/main.tex}

\title{\BA{Title in Progress...}}
\author{Boris Andrews}
\affil{Mathematical Institute, University of Oxford}
\date{\today}


\begin{document}
    \pagenumbering{gobble}
    \maketitle
    
    
    \begin{abstract}
        Magnetic confinement reactors---in particular tokamaks---offer one of the most promising options for achieving practical nuclear fusion, with the potential to provide virtually limitless, clean energy. The theoretical and numerical modeling of tokamak plasmas is simultaneously an essential component of effective reactor design, and a great research barrier. Tokamak operational conditions exhibit comparatively low Knudsen numbers. Kinetic effects, including kinetic waves and instabilities, Landau damping, bump-on-tail instabilities and more, are therefore highly influential in tokamak plasma dynamics. Purely fluid models are inherently incapable of capturing these effects, whereas the high dimensionality in purely kinetic models render them practically intractable for most relevant purposes.

        We consider a $\delta\!f$ decomposition model, with a macroscopic fluid background and microscopic kinetic correction, both fully coupled to each other. A similar manner of discretization is proposed to that used in the recent \texttt{STRUPHY} code \cite{Holderied_Possanner_Wang_2021, Holderied_2022, Li_et_al_2023} with a finite-element model for the background and a pseudo-particle/PiC model for the correction.

        The fluid background satisfies the full, non-linear, resistive, compressible, Hall MHD equations. \cite{Laakmann_Hu_Farrell_2022} introduces finite-element(-in-space) implicit timesteppers for the incompressible analogue to this system with structure-preserving (SP) properties in the ideal case, alongside parameter-robust preconditioners. We show that these timesteppers can derive from a finite-element-in-time (FET) (and finite-element-in-space) interpretation. The benefits of this reformulation are discussed, including the derivation of timesteppers that are higher order in time, and the quantifiable dissipative SP properties in the non-ideal, resistive case.
        
        We discuss possible options for extending this FET approach to timesteppers for the compressible case.

        The kinetic corrections satisfy linearized Boltzmann equations. Using a Lénard--Bernstein collision operator, these take Fokker--Planck-like forms \cite{Fokker_1914, Planck_1917} wherein pseudo-particles in the numerical model obey the neoclassical transport equations, with particle-independent Brownian drift terms. This offers a rigorous methodology for incorporating collisions into the particle transport model, without coupling the equations of motions for each particle.
        
        Works by Chen, Chacón et al. \cite{Chen_Chacón_Barnes_2011, Chacón_Chen_Barnes_2013, Chen_Chacón_2014, Chen_Chacón_2015} have developed structure-preserving particle pushers for neoclassical transport in the Vlasov equations, derived from Crank--Nicolson integrators. We show these too can can derive from a FET interpretation, similarly offering potential extensions to higher-order-in-time particle pushers. The FET formulation is used also to consider how the stochastic drift terms can be incorporated into the pushers. Stochastic gyrokinetic expansions are also discussed.

        Different options for the numerical implementation of these schemes are considered.

        Due to the efficacy of FET in the development of SP timesteppers for both the fluid and kinetic component, we hope this approach will prove effective in the future for developing SP timesteppers for the full hybrid model. We hope this will give us the opportunity to incorporate previously inaccessible kinetic effects into the highly effective, modern, finite-element MHD models.
    \end{abstract}
    
    
    \newpage
    \tableofcontents
    
    
    \newpage
    \pagenumbering{arabic}
    %\linenumbers\renewcommand\thelinenumber{\color{black!50}\arabic{linenumber}}
            \input{0 - introduction/main.tex}
        \part{Research}
            \input{1 - low-noise PiC models/main.tex}
            \input{2 - kinetic component/main.tex}
            \input{3 - fluid component/main.tex}
            \input{4 - numerical implementation/main.tex}
        \part{Project Overview}
            \input{5 - research plan/main.tex}
            \input{6 - summary/main.tex}
    
    
    %\section{}
    \newpage
    \pagenumbering{gobble}
        \printbibliography


    \newpage
    \pagenumbering{roman}
    \appendix
        \part{Appendices}
            \input{8 - Hilbert complexes/main.tex}
            \input{9 - weak conservation proofs/main.tex}
\end{document}


\title{\BA{Title in Progress...}}
\author{Boris Andrews}
\affil{Mathematical Institute, University of Oxford}
\date{\today}


\begin{document}
    \pagenumbering{gobble}
    \maketitle
    
    
    \begin{abstract}
        Magnetic confinement reactors---in particular tokamaks---offer one of the most promising options for achieving practical nuclear fusion, with the potential to provide virtually limitless, clean energy. The theoretical and numerical modeling of tokamak plasmas is simultaneously an essential component of effective reactor design, and a great research barrier. Tokamak operational conditions exhibit comparatively low Knudsen numbers. Kinetic effects, including kinetic waves and instabilities, Landau damping, bump-on-tail instabilities and more, are therefore highly influential in tokamak plasma dynamics. Purely fluid models are inherently incapable of capturing these effects, whereas the high dimensionality in purely kinetic models render them practically intractable for most relevant purposes.

        We consider a $\delta\!f$ decomposition model, with a macroscopic fluid background and microscopic kinetic correction, both fully coupled to each other. A similar manner of discretization is proposed to that used in the recent \texttt{STRUPHY} code \cite{Holderied_Possanner_Wang_2021, Holderied_2022, Li_et_al_2023} with a finite-element model for the background and a pseudo-particle/PiC model for the correction.

        The fluid background satisfies the full, non-linear, resistive, compressible, Hall MHD equations. \cite{Laakmann_Hu_Farrell_2022} introduces finite-element(-in-space) implicit timesteppers for the incompressible analogue to this system with structure-preserving (SP) properties in the ideal case, alongside parameter-robust preconditioners. We show that these timesteppers can derive from a finite-element-in-time (FET) (and finite-element-in-space) interpretation. The benefits of this reformulation are discussed, including the derivation of timesteppers that are higher order in time, and the quantifiable dissipative SP properties in the non-ideal, resistive case.
        
        We discuss possible options for extending this FET approach to timesteppers for the compressible case.

        The kinetic corrections satisfy linearized Boltzmann equations. Using a Lénard--Bernstein collision operator, these take Fokker--Planck-like forms \cite{Fokker_1914, Planck_1917} wherein pseudo-particles in the numerical model obey the neoclassical transport equations, with particle-independent Brownian drift terms. This offers a rigorous methodology for incorporating collisions into the particle transport model, without coupling the equations of motions for each particle.
        
        Works by Chen, Chacón et al. \cite{Chen_Chacón_Barnes_2011, Chacón_Chen_Barnes_2013, Chen_Chacón_2014, Chen_Chacón_2015} have developed structure-preserving particle pushers for neoclassical transport in the Vlasov equations, derived from Crank--Nicolson integrators. We show these too can can derive from a FET interpretation, similarly offering potential extensions to higher-order-in-time particle pushers. The FET formulation is used also to consider how the stochastic drift terms can be incorporated into the pushers. Stochastic gyrokinetic expansions are also discussed.

        Different options for the numerical implementation of these schemes are considered.

        Due to the efficacy of FET in the development of SP timesteppers for both the fluid and kinetic component, we hope this approach will prove effective in the future for developing SP timesteppers for the full hybrid model. We hope this will give us the opportunity to incorporate previously inaccessible kinetic effects into the highly effective, modern, finite-element MHD models.
    \end{abstract}
    
    
    \newpage
    \tableofcontents
    
    
    \newpage
    \pagenumbering{arabic}
    %\linenumbers\renewcommand\thelinenumber{\color{black!50}\arabic{linenumber}}
            \documentclass[12pt, a4paper]{report}

\input{template/main.tex}

\title{\BA{Title in Progress...}}
\author{Boris Andrews}
\affil{Mathematical Institute, University of Oxford}
\date{\today}


\begin{document}
    \pagenumbering{gobble}
    \maketitle
    
    
    \begin{abstract}
        Magnetic confinement reactors---in particular tokamaks---offer one of the most promising options for achieving practical nuclear fusion, with the potential to provide virtually limitless, clean energy. The theoretical and numerical modeling of tokamak plasmas is simultaneously an essential component of effective reactor design, and a great research barrier. Tokamak operational conditions exhibit comparatively low Knudsen numbers. Kinetic effects, including kinetic waves and instabilities, Landau damping, bump-on-tail instabilities and more, are therefore highly influential in tokamak plasma dynamics. Purely fluid models are inherently incapable of capturing these effects, whereas the high dimensionality in purely kinetic models render them practically intractable for most relevant purposes.

        We consider a $\delta\!f$ decomposition model, with a macroscopic fluid background and microscopic kinetic correction, both fully coupled to each other. A similar manner of discretization is proposed to that used in the recent \texttt{STRUPHY} code \cite{Holderied_Possanner_Wang_2021, Holderied_2022, Li_et_al_2023} with a finite-element model for the background and a pseudo-particle/PiC model for the correction.

        The fluid background satisfies the full, non-linear, resistive, compressible, Hall MHD equations. \cite{Laakmann_Hu_Farrell_2022} introduces finite-element(-in-space) implicit timesteppers for the incompressible analogue to this system with structure-preserving (SP) properties in the ideal case, alongside parameter-robust preconditioners. We show that these timesteppers can derive from a finite-element-in-time (FET) (and finite-element-in-space) interpretation. The benefits of this reformulation are discussed, including the derivation of timesteppers that are higher order in time, and the quantifiable dissipative SP properties in the non-ideal, resistive case.
        
        We discuss possible options for extending this FET approach to timesteppers for the compressible case.

        The kinetic corrections satisfy linearized Boltzmann equations. Using a Lénard--Bernstein collision operator, these take Fokker--Planck-like forms \cite{Fokker_1914, Planck_1917} wherein pseudo-particles in the numerical model obey the neoclassical transport equations, with particle-independent Brownian drift terms. This offers a rigorous methodology for incorporating collisions into the particle transport model, without coupling the equations of motions for each particle.
        
        Works by Chen, Chacón et al. \cite{Chen_Chacón_Barnes_2011, Chacón_Chen_Barnes_2013, Chen_Chacón_2014, Chen_Chacón_2015} have developed structure-preserving particle pushers for neoclassical transport in the Vlasov equations, derived from Crank--Nicolson integrators. We show these too can can derive from a FET interpretation, similarly offering potential extensions to higher-order-in-time particle pushers. The FET formulation is used also to consider how the stochastic drift terms can be incorporated into the pushers. Stochastic gyrokinetic expansions are also discussed.

        Different options for the numerical implementation of these schemes are considered.

        Due to the efficacy of FET in the development of SP timesteppers for both the fluid and kinetic component, we hope this approach will prove effective in the future for developing SP timesteppers for the full hybrid model. We hope this will give us the opportunity to incorporate previously inaccessible kinetic effects into the highly effective, modern, finite-element MHD models.
    \end{abstract}
    
    
    \newpage
    \tableofcontents
    
    
    \newpage
    \pagenumbering{arabic}
    %\linenumbers\renewcommand\thelinenumber{\color{black!50}\arabic{linenumber}}
            \input{0 - introduction/main.tex}
        \part{Research}
            \input{1 - low-noise PiC models/main.tex}
            \input{2 - kinetic component/main.tex}
            \input{3 - fluid component/main.tex}
            \input{4 - numerical implementation/main.tex}
        \part{Project Overview}
            \input{5 - research plan/main.tex}
            \input{6 - summary/main.tex}
    
    
    %\section{}
    \newpage
    \pagenumbering{gobble}
        \printbibliography


    \newpage
    \pagenumbering{roman}
    \appendix
        \part{Appendices}
            \input{8 - Hilbert complexes/main.tex}
            \input{9 - weak conservation proofs/main.tex}
\end{document}

        \part{Research}
            \documentclass[12pt, a4paper]{report}

\input{template/main.tex}

\title{\BA{Title in Progress...}}
\author{Boris Andrews}
\affil{Mathematical Institute, University of Oxford}
\date{\today}


\begin{document}
    \pagenumbering{gobble}
    \maketitle
    
    
    \begin{abstract}
        Magnetic confinement reactors---in particular tokamaks---offer one of the most promising options for achieving practical nuclear fusion, with the potential to provide virtually limitless, clean energy. The theoretical and numerical modeling of tokamak plasmas is simultaneously an essential component of effective reactor design, and a great research barrier. Tokamak operational conditions exhibit comparatively low Knudsen numbers. Kinetic effects, including kinetic waves and instabilities, Landau damping, bump-on-tail instabilities and more, are therefore highly influential in tokamak plasma dynamics. Purely fluid models are inherently incapable of capturing these effects, whereas the high dimensionality in purely kinetic models render them practically intractable for most relevant purposes.

        We consider a $\delta\!f$ decomposition model, with a macroscopic fluid background and microscopic kinetic correction, both fully coupled to each other. A similar manner of discretization is proposed to that used in the recent \texttt{STRUPHY} code \cite{Holderied_Possanner_Wang_2021, Holderied_2022, Li_et_al_2023} with a finite-element model for the background and a pseudo-particle/PiC model for the correction.

        The fluid background satisfies the full, non-linear, resistive, compressible, Hall MHD equations. \cite{Laakmann_Hu_Farrell_2022} introduces finite-element(-in-space) implicit timesteppers for the incompressible analogue to this system with structure-preserving (SP) properties in the ideal case, alongside parameter-robust preconditioners. We show that these timesteppers can derive from a finite-element-in-time (FET) (and finite-element-in-space) interpretation. The benefits of this reformulation are discussed, including the derivation of timesteppers that are higher order in time, and the quantifiable dissipative SP properties in the non-ideal, resistive case.
        
        We discuss possible options for extending this FET approach to timesteppers for the compressible case.

        The kinetic corrections satisfy linearized Boltzmann equations. Using a Lénard--Bernstein collision operator, these take Fokker--Planck-like forms \cite{Fokker_1914, Planck_1917} wherein pseudo-particles in the numerical model obey the neoclassical transport equations, with particle-independent Brownian drift terms. This offers a rigorous methodology for incorporating collisions into the particle transport model, without coupling the equations of motions for each particle.
        
        Works by Chen, Chacón et al. \cite{Chen_Chacón_Barnes_2011, Chacón_Chen_Barnes_2013, Chen_Chacón_2014, Chen_Chacón_2015} have developed structure-preserving particle pushers for neoclassical transport in the Vlasov equations, derived from Crank--Nicolson integrators. We show these too can can derive from a FET interpretation, similarly offering potential extensions to higher-order-in-time particle pushers. The FET formulation is used also to consider how the stochastic drift terms can be incorporated into the pushers. Stochastic gyrokinetic expansions are also discussed.

        Different options for the numerical implementation of these schemes are considered.

        Due to the efficacy of FET in the development of SP timesteppers for both the fluid and kinetic component, we hope this approach will prove effective in the future for developing SP timesteppers for the full hybrid model. We hope this will give us the opportunity to incorporate previously inaccessible kinetic effects into the highly effective, modern, finite-element MHD models.
    \end{abstract}
    
    
    \newpage
    \tableofcontents
    
    
    \newpage
    \pagenumbering{arabic}
    %\linenumbers\renewcommand\thelinenumber{\color{black!50}\arabic{linenumber}}
            \input{0 - introduction/main.tex}
        \part{Research}
            \input{1 - low-noise PiC models/main.tex}
            \input{2 - kinetic component/main.tex}
            \input{3 - fluid component/main.tex}
            \input{4 - numerical implementation/main.tex}
        \part{Project Overview}
            \input{5 - research plan/main.tex}
            \input{6 - summary/main.tex}
    
    
    %\section{}
    \newpage
    \pagenumbering{gobble}
        \printbibliography


    \newpage
    \pagenumbering{roman}
    \appendix
        \part{Appendices}
            \input{8 - Hilbert complexes/main.tex}
            \input{9 - weak conservation proofs/main.tex}
\end{document}

            \documentclass[12pt, a4paper]{report}

\input{template/main.tex}

\title{\BA{Title in Progress...}}
\author{Boris Andrews}
\affil{Mathematical Institute, University of Oxford}
\date{\today}


\begin{document}
    \pagenumbering{gobble}
    \maketitle
    
    
    \begin{abstract}
        Magnetic confinement reactors---in particular tokamaks---offer one of the most promising options for achieving practical nuclear fusion, with the potential to provide virtually limitless, clean energy. The theoretical and numerical modeling of tokamak plasmas is simultaneously an essential component of effective reactor design, and a great research barrier. Tokamak operational conditions exhibit comparatively low Knudsen numbers. Kinetic effects, including kinetic waves and instabilities, Landau damping, bump-on-tail instabilities and more, are therefore highly influential in tokamak plasma dynamics. Purely fluid models are inherently incapable of capturing these effects, whereas the high dimensionality in purely kinetic models render them practically intractable for most relevant purposes.

        We consider a $\delta\!f$ decomposition model, with a macroscopic fluid background and microscopic kinetic correction, both fully coupled to each other. A similar manner of discretization is proposed to that used in the recent \texttt{STRUPHY} code \cite{Holderied_Possanner_Wang_2021, Holderied_2022, Li_et_al_2023} with a finite-element model for the background and a pseudo-particle/PiC model for the correction.

        The fluid background satisfies the full, non-linear, resistive, compressible, Hall MHD equations. \cite{Laakmann_Hu_Farrell_2022} introduces finite-element(-in-space) implicit timesteppers for the incompressible analogue to this system with structure-preserving (SP) properties in the ideal case, alongside parameter-robust preconditioners. We show that these timesteppers can derive from a finite-element-in-time (FET) (and finite-element-in-space) interpretation. The benefits of this reformulation are discussed, including the derivation of timesteppers that are higher order in time, and the quantifiable dissipative SP properties in the non-ideal, resistive case.
        
        We discuss possible options for extending this FET approach to timesteppers for the compressible case.

        The kinetic corrections satisfy linearized Boltzmann equations. Using a Lénard--Bernstein collision operator, these take Fokker--Planck-like forms \cite{Fokker_1914, Planck_1917} wherein pseudo-particles in the numerical model obey the neoclassical transport equations, with particle-independent Brownian drift terms. This offers a rigorous methodology for incorporating collisions into the particle transport model, without coupling the equations of motions for each particle.
        
        Works by Chen, Chacón et al. \cite{Chen_Chacón_Barnes_2011, Chacón_Chen_Barnes_2013, Chen_Chacón_2014, Chen_Chacón_2015} have developed structure-preserving particle pushers for neoclassical transport in the Vlasov equations, derived from Crank--Nicolson integrators. We show these too can can derive from a FET interpretation, similarly offering potential extensions to higher-order-in-time particle pushers. The FET formulation is used also to consider how the stochastic drift terms can be incorporated into the pushers. Stochastic gyrokinetic expansions are also discussed.

        Different options for the numerical implementation of these schemes are considered.

        Due to the efficacy of FET in the development of SP timesteppers for both the fluid and kinetic component, we hope this approach will prove effective in the future for developing SP timesteppers for the full hybrid model. We hope this will give us the opportunity to incorporate previously inaccessible kinetic effects into the highly effective, modern, finite-element MHD models.
    \end{abstract}
    
    
    \newpage
    \tableofcontents
    
    
    \newpage
    \pagenumbering{arabic}
    %\linenumbers\renewcommand\thelinenumber{\color{black!50}\arabic{linenumber}}
            \input{0 - introduction/main.tex}
        \part{Research}
            \input{1 - low-noise PiC models/main.tex}
            \input{2 - kinetic component/main.tex}
            \input{3 - fluid component/main.tex}
            \input{4 - numerical implementation/main.tex}
        \part{Project Overview}
            \input{5 - research plan/main.tex}
            \input{6 - summary/main.tex}
    
    
    %\section{}
    \newpage
    \pagenumbering{gobble}
        \printbibliography


    \newpage
    \pagenumbering{roman}
    \appendix
        \part{Appendices}
            \input{8 - Hilbert complexes/main.tex}
            \input{9 - weak conservation proofs/main.tex}
\end{document}

            \documentclass[12pt, a4paper]{report}

\input{template/main.tex}

\title{\BA{Title in Progress...}}
\author{Boris Andrews}
\affil{Mathematical Institute, University of Oxford}
\date{\today}


\begin{document}
    \pagenumbering{gobble}
    \maketitle
    
    
    \begin{abstract}
        Magnetic confinement reactors---in particular tokamaks---offer one of the most promising options for achieving practical nuclear fusion, with the potential to provide virtually limitless, clean energy. The theoretical and numerical modeling of tokamak plasmas is simultaneously an essential component of effective reactor design, and a great research barrier. Tokamak operational conditions exhibit comparatively low Knudsen numbers. Kinetic effects, including kinetic waves and instabilities, Landau damping, bump-on-tail instabilities and more, are therefore highly influential in tokamak plasma dynamics. Purely fluid models are inherently incapable of capturing these effects, whereas the high dimensionality in purely kinetic models render them practically intractable for most relevant purposes.

        We consider a $\delta\!f$ decomposition model, with a macroscopic fluid background and microscopic kinetic correction, both fully coupled to each other. A similar manner of discretization is proposed to that used in the recent \texttt{STRUPHY} code \cite{Holderied_Possanner_Wang_2021, Holderied_2022, Li_et_al_2023} with a finite-element model for the background and a pseudo-particle/PiC model for the correction.

        The fluid background satisfies the full, non-linear, resistive, compressible, Hall MHD equations. \cite{Laakmann_Hu_Farrell_2022} introduces finite-element(-in-space) implicit timesteppers for the incompressible analogue to this system with structure-preserving (SP) properties in the ideal case, alongside parameter-robust preconditioners. We show that these timesteppers can derive from a finite-element-in-time (FET) (and finite-element-in-space) interpretation. The benefits of this reformulation are discussed, including the derivation of timesteppers that are higher order in time, and the quantifiable dissipative SP properties in the non-ideal, resistive case.
        
        We discuss possible options for extending this FET approach to timesteppers for the compressible case.

        The kinetic corrections satisfy linearized Boltzmann equations. Using a Lénard--Bernstein collision operator, these take Fokker--Planck-like forms \cite{Fokker_1914, Planck_1917} wherein pseudo-particles in the numerical model obey the neoclassical transport equations, with particle-independent Brownian drift terms. This offers a rigorous methodology for incorporating collisions into the particle transport model, without coupling the equations of motions for each particle.
        
        Works by Chen, Chacón et al. \cite{Chen_Chacón_Barnes_2011, Chacón_Chen_Barnes_2013, Chen_Chacón_2014, Chen_Chacón_2015} have developed structure-preserving particle pushers for neoclassical transport in the Vlasov equations, derived from Crank--Nicolson integrators. We show these too can can derive from a FET interpretation, similarly offering potential extensions to higher-order-in-time particle pushers. The FET formulation is used also to consider how the stochastic drift terms can be incorporated into the pushers. Stochastic gyrokinetic expansions are also discussed.

        Different options for the numerical implementation of these schemes are considered.

        Due to the efficacy of FET in the development of SP timesteppers for both the fluid and kinetic component, we hope this approach will prove effective in the future for developing SP timesteppers for the full hybrid model. We hope this will give us the opportunity to incorporate previously inaccessible kinetic effects into the highly effective, modern, finite-element MHD models.
    \end{abstract}
    
    
    \newpage
    \tableofcontents
    
    
    \newpage
    \pagenumbering{arabic}
    %\linenumbers\renewcommand\thelinenumber{\color{black!50}\arabic{linenumber}}
            \input{0 - introduction/main.tex}
        \part{Research}
            \input{1 - low-noise PiC models/main.tex}
            \input{2 - kinetic component/main.tex}
            \input{3 - fluid component/main.tex}
            \input{4 - numerical implementation/main.tex}
        \part{Project Overview}
            \input{5 - research plan/main.tex}
            \input{6 - summary/main.tex}
    
    
    %\section{}
    \newpage
    \pagenumbering{gobble}
        \printbibliography


    \newpage
    \pagenumbering{roman}
    \appendix
        \part{Appendices}
            \input{8 - Hilbert complexes/main.tex}
            \input{9 - weak conservation proofs/main.tex}
\end{document}

            \documentclass[12pt, a4paper]{report}

\input{template/main.tex}

\title{\BA{Title in Progress...}}
\author{Boris Andrews}
\affil{Mathematical Institute, University of Oxford}
\date{\today}


\begin{document}
    \pagenumbering{gobble}
    \maketitle
    
    
    \begin{abstract}
        Magnetic confinement reactors---in particular tokamaks---offer one of the most promising options for achieving practical nuclear fusion, with the potential to provide virtually limitless, clean energy. The theoretical and numerical modeling of tokamak plasmas is simultaneously an essential component of effective reactor design, and a great research barrier. Tokamak operational conditions exhibit comparatively low Knudsen numbers. Kinetic effects, including kinetic waves and instabilities, Landau damping, bump-on-tail instabilities and more, are therefore highly influential in tokamak plasma dynamics. Purely fluid models are inherently incapable of capturing these effects, whereas the high dimensionality in purely kinetic models render them practically intractable for most relevant purposes.

        We consider a $\delta\!f$ decomposition model, with a macroscopic fluid background and microscopic kinetic correction, both fully coupled to each other. A similar manner of discretization is proposed to that used in the recent \texttt{STRUPHY} code \cite{Holderied_Possanner_Wang_2021, Holderied_2022, Li_et_al_2023} with a finite-element model for the background and a pseudo-particle/PiC model for the correction.

        The fluid background satisfies the full, non-linear, resistive, compressible, Hall MHD equations. \cite{Laakmann_Hu_Farrell_2022} introduces finite-element(-in-space) implicit timesteppers for the incompressible analogue to this system with structure-preserving (SP) properties in the ideal case, alongside parameter-robust preconditioners. We show that these timesteppers can derive from a finite-element-in-time (FET) (and finite-element-in-space) interpretation. The benefits of this reformulation are discussed, including the derivation of timesteppers that are higher order in time, and the quantifiable dissipative SP properties in the non-ideal, resistive case.
        
        We discuss possible options for extending this FET approach to timesteppers for the compressible case.

        The kinetic corrections satisfy linearized Boltzmann equations. Using a Lénard--Bernstein collision operator, these take Fokker--Planck-like forms \cite{Fokker_1914, Planck_1917} wherein pseudo-particles in the numerical model obey the neoclassical transport equations, with particle-independent Brownian drift terms. This offers a rigorous methodology for incorporating collisions into the particle transport model, without coupling the equations of motions for each particle.
        
        Works by Chen, Chacón et al. \cite{Chen_Chacón_Barnes_2011, Chacón_Chen_Barnes_2013, Chen_Chacón_2014, Chen_Chacón_2015} have developed structure-preserving particle pushers for neoclassical transport in the Vlasov equations, derived from Crank--Nicolson integrators. We show these too can can derive from a FET interpretation, similarly offering potential extensions to higher-order-in-time particle pushers. The FET formulation is used also to consider how the stochastic drift terms can be incorporated into the pushers. Stochastic gyrokinetic expansions are also discussed.

        Different options for the numerical implementation of these schemes are considered.

        Due to the efficacy of FET in the development of SP timesteppers for both the fluid and kinetic component, we hope this approach will prove effective in the future for developing SP timesteppers for the full hybrid model. We hope this will give us the opportunity to incorporate previously inaccessible kinetic effects into the highly effective, modern, finite-element MHD models.
    \end{abstract}
    
    
    \newpage
    \tableofcontents
    
    
    \newpage
    \pagenumbering{arabic}
    %\linenumbers\renewcommand\thelinenumber{\color{black!50}\arabic{linenumber}}
            \input{0 - introduction/main.tex}
        \part{Research}
            \input{1 - low-noise PiC models/main.tex}
            \input{2 - kinetic component/main.tex}
            \input{3 - fluid component/main.tex}
            \input{4 - numerical implementation/main.tex}
        \part{Project Overview}
            \input{5 - research plan/main.tex}
            \input{6 - summary/main.tex}
    
    
    %\section{}
    \newpage
    \pagenumbering{gobble}
        \printbibliography


    \newpage
    \pagenumbering{roman}
    \appendix
        \part{Appendices}
            \input{8 - Hilbert complexes/main.tex}
            \input{9 - weak conservation proofs/main.tex}
\end{document}

        \part{Project Overview}
            \documentclass[12pt, a4paper]{report}

\input{template/main.tex}

\title{\BA{Title in Progress...}}
\author{Boris Andrews}
\affil{Mathematical Institute, University of Oxford}
\date{\today}


\begin{document}
    \pagenumbering{gobble}
    \maketitle
    
    
    \begin{abstract}
        Magnetic confinement reactors---in particular tokamaks---offer one of the most promising options for achieving practical nuclear fusion, with the potential to provide virtually limitless, clean energy. The theoretical and numerical modeling of tokamak plasmas is simultaneously an essential component of effective reactor design, and a great research barrier. Tokamak operational conditions exhibit comparatively low Knudsen numbers. Kinetic effects, including kinetic waves and instabilities, Landau damping, bump-on-tail instabilities and more, are therefore highly influential in tokamak plasma dynamics. Purely fluid models are inherently incapable of capturing these effects, whereas the high dimensionality in purely kinetic models render them practically intractable for most relevant purposes.

        We consider a $\delta\!f$ decomposition model, with a macroscopic fluid background and microscopic kinetic correction, both fully coupled to each other. A similar manner of discretization is proposed to that used in the recent \texttt{STRUPHY} code \cite{Holderied_Possanner_Wang_2021, Holderied_2022, Li_et_al_2023} with a finite-element model for the background and a pseudo-particle/PiC model for the correction.

        The fluid background satisfies the full, non-linear, resistive, compressible, Hall MHD equations. \cite{Laakmann_Hu_Farrell_2022} introduces finite-element(-in-space) implicit timesteppers for the incompressible analogue to this system with structure-preserving (SP) properties in the ideal case, alongside parameter-robust preconditioners. We show that these timesteppers can derive from a finite-element-in-time (FET) (and finite-element-in-space) interpretation. The benefits of this reformulation are discussed, including the derivation of timesteppers that are higher order in time, and the quantifiable dissipative SP properties in the non-ideal, resistive case.
        
        We discuss possible options for extending this FET approach to timesteppers for the compressible case.

        The kinetic corrections satisfy linearized Boltzmann equations. Using a Lénard--Bernstein collision operator, these take Fokker--Planck-like forms \cite{Fokker_1914, Planck_1917} wherein pseudo-particles in the numerical model obey the neoclassical transport equations, with particle-independent Brownian drift terms. This offers a rigorous methodology for incorporating collisions into the particle transport model, without coupling the equations of motions for each particle.
        
        Works by Chen, Chacón et al. \cite{Chen_Chacón_Barnes_2011, Chacón_Chen_Barnes_2013, Chen_Chacón_2014, Chen_Chacón_2015} have developed structure-preserving particle pushers for neoclassical transport in the Vlasov equations, derived from Crank--Nicolson integrators. We show these too can can derive from a FET interpretation, similarly offering potential extensions to higher-order-in-time particle pushers. The FET formulation is used also to consider how the stochastic drift terms can be incorporated into the pushers. Stochastic gyrokinetic expansions are also discussed.

        Different options for the numerical implementation of these schemes are considered.

        Due to the efficacy of FET in the development of SP timesteppers for both the fluid and kinetic component, we hope this approach will prove effective in the future for developing SP timesteppers for the full hybrid model. We hope this will give us the opportunity to incorporate previously inaccessible kinetic effects into the highly effective, modern, finite-element MHD models.
    \end{abstract}
    
    
    \newpage
    \tableofcontents
    
    
    \newpage
    \pagenumbering{arabic}
    %\linenumbers\renewcommand\thelinenumber{\color{black!50}\arabic{linenumber}}
            \input{0 - introduction/main.tex}
        \part{Research}
            \input{1 - low-noise PiC models/main.tex}
            \input{2 - kinetic component/main.tex}
            \input{3 - fluid component/main.tex}
            \input{4 - numerical implementation/main.tex}
        \part{Project Overview}
            \input{5 - research plan/main.tex}
            \input{6 - summary/main.tex}
    
    
    %\section{}
    \newpage
    \pagenumbering{gobble}
        \printbibliography


    \newpage
    \pagenumbering{roman}
    \appendix
        \part{Appendices}
            \input{8 - Hilbert complexes/main.tex}
            \input{9 - weak conservation proofs/main.tex}
\end{document}

            \documentclass[12pt, a4paper]{report}

\input{template/main.tex}

\title{\BA{Title in Progress...}}
\author{Boris Andrews}
\affil{Mathematical Institute, University of Oxford}
\date{\today}


\begin{document}
    \pagenumbering{gobble}
    \maketitle
    
    
    \begin{abstract}
        Magnetic confinement reactors---in particular tokamaks---offer one of the most promising options for achieving practical nuclear fusion, with the potential to provide virtually limitless, clean energy. The theoretical and numerical modeling of tokamak plasmas is simultaneously an essential component of effective reactor design, and a great research barrier. Tokamak operational conditions exhibit comparatively low Knudsen numbers. Kinetic effects, including kinetic waves and instabilities, Landau damping, bump-on-tail instabilities and more, are therefore highly influential in tokamak plasma dynamics. Purely fluid models are inherently incapable of capturing these effects, whereas the high dimensionality in purely kinetic models render them practically intractable for most relevant purposes.

        We consider a $\delta\!f$ decomposition model, with a macroscopic fluid background and microscopic kinetic correction, both fully coupled to each other. A similar manner of discretization is proposed to that used in the recent \texttt{STRUPHY} code \cite{Holderied_Possanner_Wang_2021, Holderied_2022, Li_et_al_2023} with a finite-element model for the background and a pseudo-particle/PiC model for the correction.

        The fluid background satisfies the full, non-linear, resistive, compressible, Hall MHD equations. \cite{Laakmann_Hu_Farrell_2022} introduces finite-element(-in-space) implicit timesteppers for the incompressible analogue to this system with structure-preserving (SP) properties in the ideal case, alongside parameter-robust preconditioners. We show that these timesteppers can derive from a finite-element-in-time (FET) (and finite-element-in-space) interpretation. The benefits of this reformulation are discussed, including the derivation of timesteppers that are higher order in time, and the quantifiable dissipative SP properties in the non-ideal, resistive case.
        
        We discuss possible options for extending this FET approach to timesteppers for the compressible case.

        The kinetic corrections satisfy linearized Boltzmann equations. Using a Lénard--Bernstein collision operator, these take Fokker--Planck-like forms \cite{Fokker_1914, Planck_1917} wherein pseudo-particles in the numerical model obey the neoclassical transport equations, with particle-independent Brownian drift terms. This offers a rigorous methodology for incorporating collisions into the particle transport model, without coupling the equations of motions for each particle.
        
        Works by Chen, Chacón et al. \cite{Chen_Chacón_Barnes_2011, Chacón_Chen_Barnes_2013, Chen_Chacón_2014, Chen_Chacón_2015} have developed structure-preserving particle pushers for neoclassical transport in the Vlasov equations, derived from Crank--Nicolson integrators. We show these too can can derive from a FET interpretation, similarly offering potential extensions to higher-order-in-time particle pushers. The FET formulation is used also to consider how the stochastic drift terms can be incorporated into the pushers. Stochastic gyrokinetic expansions are also discussed.

        Different options for the numerical implementation of these schemes are considered.

        Due to the efficacy of FET in the development of SP timesteppers for both the fluid and kinetic component, we hope this approach will prove effective in the future for developing SP timesteppers for the full hybrid model. We hope this will give us the opportunity to incorporate previously inaccessible kinetic effects into the highly effective, modern, finite-element MHD models.
    \end{abstract}
    
    
    \newpage
    \tableofcontents
    
    
    \newpage
    \pagenumbering{arabic}
    %\linenumbers\renewcommand\thelinenumber{\color{black!50}\arabic{linenumber}}
            \input{0 - introduction/main.tex}
        \part{Research}
            \input{1 - low-noise PiC models/main.tex}
            \input{2 - kinetic component/main.tex}
            \input{3 - fluid component/main.tex}
            \input{4 - numerical implementation/main.tex}
        \part{Project Overview}
            \input{5 - research plan/main.tex}
            \input{6 - summary/main.tex}
    
    
    %\section{}
    \newpage
    \pagenumbering{gobble}
        \printbibliography


    \newpage
    \pagenumbering{roman}
    \appendix
        \part{Appendices}
            \input{8 - Hilbert complexes/main.tex}
            \input{9 - weak conservation proofs/main.tex}
\end{document}

    
    
    %\section{}
    \newpage
    \pagenumbering{gobble}
        \printbibliography


    \newpage
    \pagenumbering{roman}
    \appendix
        \part{Appendices}
            \documentclass[12pt, a4paper]{report}

\input{template/main.tex}

\title{\BA{Title in Progress...}}
\author{Boris Andrews}
\affil{Mathematical Institute, University of Oxford}
\date{\today}


\begin{document}
    \pagenumbering{gobble}
    \maketitle
    
    
    \begin{abstract}
        Magnetic confinement reactors---in particular tokamaks---offer one of the most promising options for achieving practical nuclear fusion, with the potential to provide virtually limitless, clean energy. The theoretical and numerical modeling of tokamak plasmas is simultaneously an essential component of effective reactor design, and a great research barrier. Tokamak operational conditions exhibit comparatively low Knudsen numbers. Kinetic effects, including kinetic waves and instabilities, Landau damping, bump-on-tail instabilities and more, are therefore highly influential in tokamak plasma dynamics. Purely fluid models are inherently incapable of capturing these effects, whereas the high dimensionality in purely kinetic models render them practically intractable for most relevant purposes.

        We consider a $\delta\!f$ decomposition model, with a macroscopic fluid background and microscopic kinetic correction, both fully coupled to each other. A similar manner of discretization is proposed to that used in the recent \texttt{STRUPHY} code \cite{Holderied_Possanner_Wang_2021, Holderied_2022, Li_et_al_2023} with a finite-element model for the background and a pseudo-particle/PiC model for the correction.

        The fluid background satisfies the full, non-linear, resistive, compressible, Hall MHD equations. \cite{Laakmann_Hu_Farrell_2022} introduces finite-element(-in-space) implicit timesteppers for the incompressible analogue to this system with structure-preserving (SP) properties in the ideal case, alongside parameter-robust preconditioners. We show that these timesteppers can derive from a finite-element-in-time (FET) (and finite-element-in-space) interpretation. The benefits of this reformulation are discussed, including the derivation of timesteppers that are higher order in time, and the quantifiable dissipative SP properties in the non-ideal, resistive case.
        
        We discuss possible options for extending this FET approach to timesteppers for the compressible case.

        The kinetic corrections satisfy linearized Boltzmann equations. Using a Lénard--Bernstein collision operator, these take Fokker--Planck-like forms \cite{Fokker_1914, Planck_1917} wherein pseudo-particles in the numerical model obey the neoclassical transport equations, with particle-independent Brownian drift terms. This offers a rigorous methodology for incorporating collisions into the particle transport model, without coupling the equations of motions for each particle.
        
        Works by Chen, Chacón et al. \cite{Chen_Chacón_Barnes_2011, Chacón_Chen_Barnes_2013, Chen_Chacón_2014, Chen_Chacón_2015} have developed structure-preserving particle pushers for neoclassical transport in the Vlasov equations, derived from Crank--Nicolson integrators. We show these too can can derive from a FET interpretation, similarly offering potential extensions to higher-order-in-time particle pushers. The FET formulation is used also to consider how the stochastic drift terms can be incorporated into the pushers. Stochastic gyrokinetic expansions are also discussed.

        Different options for the numerical implementation of these schemes are considered.

        Due to the efficacy of FET in the development of SP timesteppers for both the fluid and kinetic component, we hope this approach will prove effective in the future for developing SP timesteppers for the full hybrid model. We hope this will give us the opportunity to incorporate previously inaccessible kinetic effects into the highly effective, modern, finite-element MHD models.
    \end{abstract}
    
    
    \newpage
    \tableofcontents
    
    
    \newpage
    \pagenumbering{arabic}
    %\linenumbers\renewcommand\thelinenumber{\color{black!50}\arabic{linenumber}}
            \input{0 - introduction/main.tex}
        \part{Research}
            \input{1 - low-noise PiC models/main.tex}
            \input{2 - kinetic component/main.tex}
            \input{3 - fluid component/main.tex}
            \input{4 - numerical implementation/main.tex}
        \part{Project Overview}
            \input{5 - research plan/main.tex}
            \input{6 - summary/main.tex}
    
    
    %\section{}
    \newpage
    \pagenumbering{gobble}
        \printbibliography


    \newpage
    \pagenumbering{roman}
    \appendix
        \part{Appendices}
            \input{8 - Hilbert complexes/main.tex}
            \input{9 - weak conservation proofs/main.tex}
\end{document}

            \documentclass[12pt, a4paper]{report}

\input{template/main.tex}

\title{\BA{Title in Progress...}}
\author{Boris Andrews}
\affil{Mathematical Institute, University of Oxford}
\date{\today}


\begin{document}
    \pagenumbering{gobble}
    \maketitle
    
    
    \begin{abstract}
        Magnetic confinement reactors---in particular tokamaks---offer one of the most promising options for achieving practical nuclear fusion, with the potential to provide virtually limitless, clean energy. The theoretical and numerical modeling of tokamak plasmas is simultaneously an essential component of effective reactor design, and a great research barrier. Tokamak operational conditions exhibit comparatively low Knudsen numbers. Kinetic effects, including kinetic waves and instabilities, Landau damping, bump-on-tail instabilities and more, are therefore highly influential in tokamak plasma dynamics. Purely fluid models are inherently incapable of capturing these effects, whereas the high dimensionality in purely kinetic models render them practically intractable for most relevant purposes.

        We consider a $\delta\!f$ decomposition model, with a macroscopic fluid background and microscopic kinetic correction, both fully coupled to each other. A similar manner of discretization is proposed to that used in the recent \texttt{STRUPHY} code \cite{Holderied_Possanner_Wang_2021, Holderied_2022, Li_et_al_2023} with a finite-element model for the background and a pseudo-particle/PiC model for the correction.

        The fluid background satisfies the full, non-linear, resistive, compressible, Hall MHD equations. \cite{Laakmann_Hu_Farrell_2022} introduces finite-element(-in-space) implicit timesteppers for the incompressible analogue to this system with structure-preserving (SP) properties in the ideal case, alongside parameter-robust preconditioners. We show that these timesteppers can derive from a finite-element-in-time (FET) (and finite-element-in-space) interpretation. The benefits of this reformulation are discussed, including the derivation of timesteppers that are higher order in time, and the quantifiable dissipative SP properties in the non-ideal, resistive case.
        
        We discuss possible options for extending this FET approach to timesteppers for the compressible case.

        The kinetic corrections satisfy linearized Boltzmann equations. Using a Lénard--Bernstein collision operator, these take Fokker--Planck-like forms \cite{Fokker_1914, Planck_1917} wherein pseudo-particles in the numerical model obey the neoclassical transport equations, with particle-independent Brownian drift terms. This offers a rigorous methodology for incorporating collisions into the particle transport model, without coupling the equations of motions for each particle.
        
        Works by Chen, Chacón et al. \cite{Chen_Chacón_Barnes_2011, Chacón_Chen_Barnes_2013, Chen_Chacón_2014, Chen_Chacón_2015} have developed structure-preserving particle pushers for neoclassical transport in the Vlasov equations, derived from Crank--Nicolson integrators. We show these too can can derive from a FET interpretation, similarly offering potential extensions to higher-order-in-time particle pushers. The FET formulation is used also to consider how the stochastic drift terms can be incorporated into the pushers. Stochastic gyrokinetic expansions are also discussed.

        Different options for the numerical implementation of these schemes are considered.

        Due to the efficacy of FET in the development of SP timesteppers for both the fluid and kinetic component, we hope this approach will prove effective in the future for developing SP timesteppers for the full hybrid model. We hope this will give us the opportunity to incorporate previously inaccessible kinetic effects into the highly effective, modern, finite-element MHD models.
    \end{abstract}
    
    
    \newpage
    \tableofcontents
    
    
    \newpage
    \pagenumbering{arabic}
    %\linenumbers\renewcommand\thelinenumber{\color{black!50}\arabic{linenumber}}
            \input{0 - introduction/main.tex}
        \part{Research}
            \input{1 - low-noise PiC models/main.tex}
            \input{2 - kinetic component/main.tex}
            \input{3 - fluid component/main.tex}
            \input{4 - numerical implementation/main.tex}
        \part{Project Overview}
            \input{5 - research plan/main.tex}
            \input{6 - summary/main.tex}
    
    
    %\section{}
    \newpage
    \pagenumbering{gobble}
        \printbibliography


    \newpage
    \pagenumbering{roman}
    \appendix
        \part{Appendices}
            \input{8 - Hilbert complexes/main.tex}
            \input{9 - weak conservation proofs/main.tex}
\end{document}

\end{document}

            \documentclass[12pt, a4paper]{report}

\documentclass[12pt, a4paper]{report}

\input{template/main.tex}

\title{\BA{Title in Progress...}}
\author{Boris Andrews}
\affil{Mathematical Institute, University of Oxford}
\date{\today}


\begin{document}
    \pagenumbering{gobble}
    \maketitle
    
    
    \begin{abstract}
        Magnetic confinement reactors---in particular tokamaks---offer one of the most promising options for achieving practical nuclear fusion, with the potential to provide virtually limitless, clean energy. The theoretical and numerical modeling of tokamak plasmas is simultaneously an essential component of effective reactor design, and a great research barrier. Tokamak operational conditions exhibit comparatively low Knudsen numbers. Kinetic effects, including kinetic waves and instabilities, Landau damping, bump-on-tail instabilities and more, are therefore highly influential in tokamak plasma dynamics. Purely fluid models are inherently incapable of capturing these effects, whereas the high dimensionality in purely kinetic models render them practically intractable for most relevant purposes.

        We consider a $\delta\!f$ decomposition model, with a macroscopic fluid background and microscopic kinetic correction, both fully coupled to each other. A similar manner of discretization is proposed to that used in the recent \texttt{STRUPHY} code \cite{Holderied_Possanner_Wang_2021, Holderied_2022, Li_et_al_2023} with a finite-element model for the background and a pseudo-particle/PiC model for the correction.

        The fluid background satisfies the full, non-linear, resistive, compressible, Hall MHD equations. \cite{Laakmann_Hu_Farrell_2022} introduces finite-element(-in-space) implicit timesteppers for the incompressible analogue to this system with structure-preserving (SP) properties in the ideal case, alongside parameter-robust preconditioners. We show that these timesteppers can derive from a finite-element-in-time (FET) (and finite-element-in-space) interpretation. The benefits of this reformulation are discussed, including the derivation of timesteppers that are higher order in time, and the quantifiable dissipative SP properties in the non-ideal, resistive case.
        
        We discuss possible options for extending this FET approach to timesteppers for the compressible case.

        The kinetic corrections satisfy linearized Boltzmann equations. Using a Lénard--Bernstein collision operator, these take Fokker--Planck-like forms \cite{Fokker_1914, Planck_1917} wherein pseudo-particles in the numerical model obey the neoclassical transport equations, with particle-independent Brownian drift terms. This offers a rigorous methodology for incorporating collisions into the particle transport model, without coupling the equations of motions for each particle.
        
        Works by Chen, Chacón et al. \cite{Chen_Chacón_Barnes_2011, Chacón_Chen_Barnes_2013, Chen_Chacón_2014, Chen_Chacón_2015} have developed structure-preserving particle pushers for neoclassical transport in the Vlasov equations, derived from Crank--Nicolson integrators. We show these too can can derive from a FET interpretation, similarly offering potential extensions to higher-order-in-time particle pushers. The FET formulation is used also to consider how the stochastic drift terms can be incorporated into the pushers. Stochastic gyrokinetic expansions are also discussed.

        Different options for the numerical implementation of these schemes are considered.

        Due to the efficacy of FET in the development of SP timesteppers for both the fluid and kinetic component, we hope this approach will prove effective in the future for developing SP timesteppers for the full hybrid model. We hope this will give us the opportunity to incorporate previously inaccessible kinetic effects into the highly effective, modern, finite-element MHD models.
    \end{abstract}
    
    
    \newpage
    \tableofcontents
    
    
    \newpage
    \pagenumbering{arabic}
    %\linenumbers\renewcommand\thelinenumber{\color{black!50}\arabic{linenumber}}
            \input{0 - introduction/main.tex}
        \part{Research}
            \input{1 - low-noise PiC models/main.tex}
            \input{2 - kinetic component/main.tex}
            \input{3 - fluid component/main.tex}
            \input{4 - numerical implementation/main.tex}
        \part{Project Overview}
            \input{5 - research plan/main.tex}
            \input{6 - summary/main.tex}
    
    
    %\section{}
    \newpage
    \pagenumbering{gobble}
        \printbibliography


    \newpage
    \pagenumbering{roman}
    \appendix
        \part{Appendices}
            \input{8 - Hilbert complexes/main.tex}
            \input{9 - weak conservation proofs/main.tex}
\end{document}


\title{\BA{Title in Progress...}}
\author{Boris Andrews}
\affil{Mathematical Institute, University of Oxford}
\date{\today}


\begin{document}
    \pagenumbering{gobble}
    \maketitle
    
    
    \begin{abstract}
        Magnetic confinement reactors---in particular tokamaks---offer one of the most promising options for achieving practical nuclear fusion, with the potential to provide virtually limitless, clean energy. The theoretical and numerical modeling of tokamak plasmas is simultaneously an essential component of effective reactor design, and a great research barrier. Tokamak operational conditions exhibit comparatively low Knudsen numbers. Kinetic effects, including kinetic waves and instabilities, Landau damping, bump-on-tail instabilities and more, are therefore highly influential in tokamak plasma dynamics. Purely fluid models are inherently incapable of capturing these effects, whereas the high dimensionality in purely kinetic models render them practically intractable for most relevant purposes.

        We consider a $\delta\!f$ decomposition model, with a macroscopic fluid background and microscopic kinetic correction, both fully coupled to each other. A similar manner of discretization is proposed to that used in the recent \texttt{STRUPHY} code \cite{Holderied_Possanner_Wang_2021, Holderied_2022, Li_et_al_2023} with a finite-element model for the background and a pseudo-particle/PiC model for the correction.

        The fluid background satisfies the full, non-linear, resistive, compressible, Hall MHD equations. \cite{Laakmann_Hu_Farrell_2022} introduces finite-element(-in-space) implicit timesteppers for the incompressible analogue to this system with structure-preserving (SP) properties in the ideal case, alongside parameter-robust preconditioners. We show that these timesteppers can derive from a finite-element-in-time (FET) (and finite-element-in-space) interpretation. The benefits of this reformulation are discussed, including the derivation of timesteppers that are higher order in time, and the quantifiable dissipative SP properties in the non-ideal, resistive case.
        
        We discuss possible options for extending this FET approach to timesteppers for the compressible case.

        The kinetic corrections satisfy linearized Boltzmann equations. Using a Lénard--Bernstein collision operator, these take Fokker--Planck-like forms \cite{Fokker_1914, Planck_1917} wherein pseudo-particles in the numerical model obey the neoclassical transport equations, with particle-independent Brownian drift terms. This offers a rigorous methodology for incorporating collisions into the particle transport model, without coupling the equations of motions for each particle.
        
        Works by Chen, Chacón et al. \cite{Chen_Chacón_Barnes_2011, Chacón_Chen_Barnes_2013, Chen_Chacón_2014, Chen_Chacón_2015} have developed structure-preserving particle pushers for neoclassical transport in the Vlasov equations, derived from Crank--Nicolson integrators. We show these too can can derive from a FET interpretation, similarly offering potential extensions to higher-order-in-time particle pushers. The FET formulation is used also to consider how the stochastic drift terms can be incorporated into the pushers. Stochastic gyrokinetic expansions are also discussed.

        Different options for the numerical implementation of these schemes are considered.

        Due to the efficacy of FET in the development of SP timesteppers for both the fluid and kinetic component, we hope this approach will prove effective in the future for developing SP timesteppers for the full hybrid model. We hope this will give us the opportunity to incorporate previously inaccessible kinetic effects into the highly effective, modern, finite-element MHD models.
    \end{abstract}
    
    
    \newpage
    \tableofcontents
    
    
    \newpage
    \pagenumbering{arabic}
    %\linenumbers\renewcommand\thelinenumber{\color{black!50}\arabic{linenumber}}
            \documentclass[12pt, a4paper]{report}

\input{template/main.tex}

\title{\BA{Title in Progress...}}
\author{Boris Andrews}
\affil{Mathematical Institute, University of Oxford}
\date{\today}


\begin{document}
    \pagenumbering{gobble}
    \maketitle
    
    
    \begin{abstract}
        Magnetic confinement reactors---in particular tokamaks---offer one of the most promising options for achieving practical nuclear fusion, with the potential to provide virtually limitless, clean energy. The theoretical and numerical modeling of tokamak plasmas is simultaneously an essential component of effective reactor design, and a great research barrier. Tokamak operational conditions exhibit comparatively low Knudsen numbers. Kinetic effects, including kinetic waves and instabilities, Landau damping, bump-on-tail instabilities and more, are therefore highly influential in tokamak plasma dynamics. Purely fluid models are inherently incapable of capturing these effects, whereas the high dimensionality in purely kinetic models render them practically intractable for most relevant purposes.

        We consider a $\delta\!f$ decomposition model, with a macroscopic fluid background and microscopic kinetic correction, both fully coupled to each other. A similar manner of discretization is proposed to that used in the recent \texttt{STRUPHY} code \cite{Holderied_Possanner_Wang_2021, Holderied_2022, Li_et_al_2023} with a finite-element model for the background and a pseudo-particle/PiC model for the correction.

        The fluid background satisfies the full, non-linear, resistive, compressible, Hall MHD equations. \cite{Laakmann_Hu_Farrell_2022} introduces finite-element(-in-space) implicit timesteppers for the incompressible analogue to this system with structure-preserving (SP) properties in the ideal case, alongside parameter-robust preconditioners. We show that these timesteppers can derive from a finite-element-in-time (FET) (and finite-element-in-space) interpretation. The benefits of this reformulation are discussed, including the derivation of timesteppers that are higher order in time, and the quantifiable dissipative SP properties in the non-ideal, resistive case.
        
        We discuss possible options for extending this FET approach to timesteppers for the compressible case.

        The kinetic corrections satisfy linearized Boltzmann equations. Using a Lénard--Bernstein collision operator, these take Fokker--Planck-like forms \cite{Fokker_1914, Planck_1917} wherein pseudo-particles in the numerical model obey the neoclassical transport equations, with particle-independent Brownian drift terms. This offers a rigorous methodology for incorporating collisions into the particle transport model, without coupling the equations of motions for each particle.
        
        Works by Chen, Chacón et al. \cite{Chen_Chacón_Barnes_2011, Chacón_Chen_Barnes_2013, Chen_Chacón_2014, Chen_Chacón_2015} have developed structure-preserving particle pushers for neoclassical transport in the Vlasov equations, derived from Crank--Nicolson integrators. We show these too can can derive from a FET interpretation, similarly offering potential extensions to higher-order-in-time particle pushers. The FET formulation is used also to consider how the stochastic drift terms can be incorporated into the pushers. Stochastic gyrokinetic expansions are also discussed.

        Different options for the numerical implementation of these schemes are considered.

        Due to the efficacy of FET in the development of SP timesteppers for both the fluid and kinetic component, we hope this approach will prove effective in the future for developing SP timesteppers for the full hybrid model. We hope this will give us the opportunity to incorporate previously inaccessible kinetic effects into the highly effective, modern, finite-element MHD models.
    \end{abstract}
    
    
    \newpage
    \tableofcontents
    
    
    \newpage
    \pagenumbering{arabic}
    %\linenumbers\renewcommand\thelinenumber{\color{black!50}\arabic{linenumber}}
            \input{0 - introduction/main.tex}
        \part{Research}
            \input{1 - low-noise PiC models/main.tex}
            \input{2 - kinetic component/main.tex}
            \input{3 - fluid component/main.tex}
            \input{4 - numerical implementation/main.tex}
        \part{Project Overview}
            \input{5 - research plan/main.tex}
            \input{6 - summary/main.tex}
    
    
    %\section{}
    \newpage
    \pagenumbering{gobble}
        \printbibliography


    \newpage
    \pagenumbering{roman}
    \appendix
        \part{Appendices}
            \input{8 - Hilbert complexes/main.tex}
            \input{9 - weak conservation proofs/main.tex}
\end{document}

        \part{Research}
            \documentclass[12pt, a4paper]{report}

\input{template/main.tex}

\title{\BA{Title in Progress...}}
\author{Boris Andrews}
\affil{Mathematical Institute, University of Oxford}
\date{\today}


\begin{document}
    \pagenumbering{gobble}
    \maketitle
    
    
    \begin{abstract}
        Magnetic confinement reactors---in particular tokamaks---offer one of the most promising options for achieving practical nuclear fusion, with the potential to provide virtually limitless, clean energy. The theoretical and numerical modeling of tokamak plasmas is simultaneously an essential component of effective reactor design, and a great research barrier. Tokamak operational conditions exhibit comparatively low Knudsen numbers. Kinetic effects, including kinetic waves and instabilities, Landau damping, bump-on-tail instabilities and more, are therefore highly influential in tokamak plasma dynamics. Purely fluid models are inherently incapable of capturing these effects, whereas the high dimensionality in purely kinetic models render them practically intractable for most relevant purposes.

        We consider a $\delta\!f$ decomposition model, with a macroscopic fluid background and microscopic kinetic correction, both fully coupled to each other. A similar manner of discretization is proposed to that used in the recent \texttt{STRUPHY} code \cite{Holderied_Possanner_Wang_2021, Holderied_2022, Li_et_al_2023} with a finite-element model for the background and a pseudo-particle/PiC model for the correction.

        The fluid background satisfies the full, non-linear, resistive, compressible, Hall MHD equations. \cite{Laakmann_Hu_Farrell_2022} introduces finite-element(-in-space) implicit timesteppers for the incompressible analogue to this system with structure-preserving (SP) properties in the ideal case, alongside parameter-robust preconditioners. We show that these timesteppers can derive from a finite-element-in-time (FET) (and finite-element-in-space) interpretation. The benefits of this reformulation are discussed, including the derivation of timesteppers that are higher order in time, and the quantifiable dissipative SP properties in the non-ideal, resistive case.
        
        We discuss possible options for extending this FET approach to timesteppers for the compressible case.

        The kinetic corrections satisfy linearized Boltzmann equations. Using a Lénard--Bernstein collision operator, these take Fokker--Planck-like forms \cite{Fokker_1914, Planck_1917} wherein pseudo-particles in the numerical model obey the neoclassical transport equations, with particle-independent Brownian drift terms. This offers a rigorous methodology for incorporating collisions into the particle transport model, without coupling the equations of motions for each particle.
        
        Works by Chen, Chacón et al. \cite{Chen_Chacón_Barnes_2011, Chacón_Chen_Barnes_2013, Chen_Chacón_2014, Chen_Chacón_2015} have developed structure-preserving particle pushers for neoclassical transport in the Vlasov equations, derived from Crank--Nicolson integrators. We show these too can can derive from a FET interpretation, similarly offering potential extensions to higher-order-in-time particle pushers. The FET formulation is used also to consider how the stochastic drift terms can be incorporated into the pushers. Stochastic gyrokinetic expansions are also discussed.

        Different options for the numerical implementation of these schemes are considered.

        Due to the efficacy of FET in the development of SP timesteppers for both the fluid and kinetic component, we hope this approach will prove effective in the future for developing SP timesteppers for the full hybrid model. We hope this will give us the opportunity to incorporate previously inaccessible kinetic effects into the highly effective, modern, finite-element MHD models.
    \end{abstract}
    
    
    \newpage
    \tableofcontents
    
    
    \newpage
    \pagenumbering{arabic}
    %\linenumbers\renewcommand\thelinenumber{\color{black!50}\arabic{linenumber}}
            \input{0 - introduction/main.tex}
        \part{Research}
            \input{1 - low-noise PiC models/main.tex}
            \input{2 - kinetic component/main.tex}
            \input{3 - fluid component/main.tex}
            \input{4 - numerical implementation/main.tex}
        \part{Project Overview}
            \input{5 - research plan/main.tex}
            \input{6 - summary/main.tex}
    
    
    %\section{}
    \newpage
    \pagenumbering{gobble}
        \printbibliography


    \newpage
    \pagenumbering{roman}
    \appendix
        \part{Appendices}
            \input{8 - Hilbert complexes/main.tex}
            \input{9 - weak conservation proofs/main.tex}
\end{document}

            \documentclass[12pt, a4paper]{report}

\input{template/main.tex}

\title{\BA{Title in Progress...}}
\author{Boris Andrews}
\affil{Mathematical Institute, University of Oxford}
\date{\today}


\begin{document}
    \pagenumbering{gobble}
    \maketitle
    
    
    \begin{abstract}
        Magnetic confinement reactors---in particular tokamaks---offer one of the most promising options for achieving practical nuclear fusion, with the potential to provide virtually limitless, clean energy. The theoretical and numerical modeling of tokamak plasmas is simultaneously an essential component of effective reactor design, and a great research barrier. Tokamak operational conditions exhibit comparatively low Knudsen numbers. Kinetic effects, including kinetic waves and instabilities, Landau damping, bump-on-tail instabilities and more, are therefore highly influential in tokamak plasma dynamics. Purely fluid models are inherently incapable of capturing these effects, whereas the high dimensionality in purely kinetic models render them practically intractable for most relevant purposes.

        We consider a $\delta\!f$ decomposition model, with a macroscopic fluid background and microscopic kinetic correction, both fully coupled to each other. A similar manner of discretization is proposed to that used in the recent \texttt{STRUPHY} code \cite{Holderied_Possanner_Wang_2021, Holderied_2022, Li_et_al_2023} with a finite-element model for the background and a pseudo-particle/PiC model for the correction.

        The fluid background satisfies the full, non-linear, resistive, compressible, Hall MHD equations. \cite{Laakmann_Hu_Farrell_2022} introduces finite-element(-in-space) implicit timesteppers for the incompressible analogue to this system with structure-preserving (SP) properties in the ideal case, alongside parameter-robust preconditioners. We show that these timesteppers can derive from a finite-element-in-time (FET) (and finite-element-in-space) interpretation. The benefits of this reformulation are discussed, including the derivation of timesteppers that are higher order in time, and the quantifiable dissipative SP properties in the non-ideal, resistive case.
        
        We discuss possible options for extending this FET approach to timesteppers for the compressible case.

        The kinetic corrections satisfy linearized Boltzmann equations. Using a Lénard--Bernstein collision operator, these take Fokker--Planck-like forms \cite{Fokker_1914, Planck_1917} wherein pseudo-particles in the numerical model obey the neoclassical transport equations, with particle-independent Brownian drift terms. This offers a rigorous methodology for incorporating collisions into the particle transport model, without coupling the equations of motions for each particle.
        
        Works by Chen, Chacón et al. \cite{Chen_Chacón_Barnes_2011, Chacón_Chen_Barnes_2013, Chen_Chacón_2014, Chen_Chacón_2015} have developed structure-preserving particle pushers for neoclassical transport in the Vlasov equations, derived from Crank--Nicolson integrators. We show these too can can derive from a FET interpretation, similarly offering potential extensions to higher-order-in-time particle pushers. The FET formulation is used also to consider how the stochastic drift terms can be incorporated into the pushers. Stochastic gyrokinetic expansions are also discussed.

        Different options for the numerical implementation of these schemes are considered.

        Due to the efficacy of FET in the development of SP timesteppers for both the fluid and kinetic component, we hope this approach will prove effective in the future for developing SP timesteppers for the full hybrid model. We hope this will give us the opportunity to incorporate previously inaccessible kinetic effects into the highly effective, modern, finite-element MHD models.
    \end{abstract}
    
    
    \newpage
    \tableofcontents
    
    
    \newpage
    \pagenumbering{arabic}
    %\linenumbers\renewcommand\thelinenumber{\color{black!50}\arabic{linenumber}}
            \input{0 - introduction/main.tex}
        \part{Research}
            \input{1 - low-noise PiC models/main.tex}
            \input{2 - kinetic component/main.tex}
            \input{3 - fluid component/main.tex}
            \input{4 - numerical implementation/main.tex}
        \part{Project Overview}
            \input{5 - research plan/main.tex}
            \input{6 - summary/main.tex}
    
    
    %\section{}
    \newpage
    \pagenumbering{gobble}
        \printbibliography


    \newpage
    \pagenumbering{roman}
    \appendix
        \part{Appendices}
            \input{8 - Hilbert complexes/main.tex}
            \input{9 - weak conservation proofs/main.tex}
\end{document}

            \documentclass[12pt, a4paper]{report}

\input{template/main.tex}

\title{\BA{Title in Progress...}}
\author{Boris Andrews}
\affil{Mathematical Institute, University of Oxford}
\date{\today}


\begin{document}
    \pagenumbering{gobble}
    \maketitle
    
    
    \begin{abstract}
        Magnetic confinement reactors---in particular tokamaks---offer one of the most promising options for achieving practical nuclear fusion, with the potential to provide virtually limitless, clean energy. The theoretical and numerical modeling of tokamak plasmas is simultaneously an essential component of effective reactor design, and a great research barrier. Tokamak operational conditions exhibit comparatively low Knudsen numbers. Kinetic effects, including kinetic waves and instabilities, Landau damping, bump-on-tail instabilities and more, are therefore highly influential in tokamak plasma dynamics. Purely fluid models are inherently incapable of capturing these effects, whereas the high dimensionality in purely kinetic models render them practically intractable for most relevant purposes.

        We consider a $\delta\!f$ decomposition model, with a macroscopic fluid background and microscopic kinetic correction, both fully coupled to each other. A similar manner of discretization is proposed to that used in the recent \texttt{STRUPHY} code \cite{Holderied_Possanner_Wang_2021, Holderied_2022, Li_et_al_2023} with a finite-element model for the background and a pseudo-particle/PiC model for the correction.

        The fluid background satisfies the full, non-linear, resistive, compressible, Hall MHD equations. \cite{Laakmann_Hu_Farrell_2022} introduces finite-element(-in-space) implicit timesteppers for the incompressible analogue to this system with structure-preserving (SP) properties in the ideal case, alongside parameter-robust preconditioners. We show that these timesteppers can derive from a finite-element-in-time (FET) (and finite-element-in-space) interpretation. The benefits of this reformulation are discussed, including the derivation of timesteppers that are higher order in time, and the quantifiable dissipative SP properties in the non-ideal, resistive case.
        
        We discuss possible options for extending this FET approach to timesteppers for the compressible case.

        The kinetic corrections satisfy linearized Boltzmann equations. Using a Lénard--Bernstein collision operator, these take Fokker--Planck-like forms \cite{Fokker_1914, Planck_1917} wherein pseudo-particles in the numerical model obey the neoclassical transport equations, with particle-independent Brownian drift terms. This offers a rigorous methodology for incorporating collisions into the particle transport model, without coupling the equations of motions for each particle.
        
        Works by Chen, Chacón et al. \cite{Chen_Chacón_Barnes_2011, Chacón_Chen_Barnes_2013, Chen_Chacón_2014, Chen_Chacón_2015} have developed structure-preserving particle pushers for neoclassical transport in the Vlasov equations, derived from Crank--Nicolson integrators. We show these too can can derive from a FET interpretation, similarly offering potential extensions to higher-order-in-time particle pushers. The FET formulation is used also to consider how the stochastic drift terms can be incorporated into the pushers. Stochastic gyrokinetic expansions are also discussed.

        Different options for the numerical implementation of these schemes are considered.

        Due to the efficacy of FET in the development of SP timesteppers for both the fluid and kinetic component, we hope this approach will prove effective in the future for developing SP timesteppers for the full hybrid model. We hope this will give us the opportunity to incorporate previously inaccessible kinetic effects into the highly effective, modern, finite-element MHD models.
    \end{abstract}
    
    
    \newpage
    \tableofcontents
    
    
    \newpage
    \pagenumbering{arabic}
    %\linenumbers\renewcommand\thelinenumber{\color{black!50}\arabic{linenumber}}
            \input{0 - introduction/main.tex}
        \part{Research}
            \input{1 - low-noise PiC models/main.tex}
            \input{2 - kinetic component/main.tex}
            \input{3 - fluid component/main.tex}
            \input{4 - numerical implementation/main.tex}
        \part{Project Overview}
            \input{5 - research plan/main.tex}
            \input{6 - summary/main.tex}
    
    
    %\section{}
    \newpage
    \pagenumbering{gobble}
        \printbibliography


    \newpage
    \pagenumbering{roman}
    \appendix
        \part{Appendices}
            \input{8 - Hilbert complexes/main.tex}
            \input{9 - weak conservation proofs/main.tex}
\end{document}

            \documentclass[12pt, a4paper]{report}

\input{template/main.tex}

\title{\BA{Title in Progress...}}
\author{Boris Andrews}
\affil{Mathematical Institute, University of Oxford}
\date{\today}


\begin{document}
    \pagenumbering{gobble}
    \maketitle
    
    
    \begin{abstract}
        Magnetic confinement reactors---in particular tokamaks---offer one of the most promising options for achieving practical nuclear fusion, with the potential to provide virtually limitless, clean energy. The theoretical and numerical modeling of tokamak plasmas is simultaneously an essential component of effective reactor design, and a great research barrier. Tokamak operational conditions exhibit comparatively low Knudsen numbers. Kinetic effects, including kinetic waves and instabilities, Landau damping, bump-on-tail instabilities and more, are therefore highly influential in tokamak plasma dynamics. Purely fluid models are inherently incapable of capturing these effects, whereas the high dimensionality in purely kinetic models render them practically intractable for most relevant purposes.

        We consider a $\delta\!f$ decomposition model, with a macroscopic fluid background and microscopic kinetic correction, both fully coupled to each other. A similar manner of discretization is proposed to that used in the recent \texttt{STRUPHY} code \cite{Holderied_Possanner_Wang_2021, Holderied_2022, Li_et_al_2023} with a finite-element model for the background and a pseudo-particle/PiC model for the correction.

        The fluid background satisfies the full, non-linear, resistive, compressible, Hall MHD equations. \cite{Laakmann_Hu_Farrell_2022} introduces finite-element(-in-space) implicit timesteppers for the incompressible analogue to this system with structure-preserving (SP) properties in the ideal case, alongside parameter-robust preconditioners. We show that these timesteppers can derive from a finite-element-in-time (FET) (and finite-element-in-space) interpretation. The benefits of this reformulation are discussed, including the derivation of timesteppers that are higher order in time, and the quantifiable dissipative SP properties in the non-ideal, resistive case.
        
        We discuss possible options for extending this FET approach to timesteppers for the compressible case.

        The kinetic corrections satisfy linearized Boltzmann equations. Using a Lénard--Bernstein collision operator, these take Fokker--Planck-like forms \cite{Fokker_1914, Planck_1917} wherein pseudo-particles in the numerical model obey the neoclassical transport equations, with particle-independent Brownian drift terms. This offers a rigorous methodology for incorporating collisions into the particle transport model, without coupling the equations of motions for each particle.
        
        Works by Chen, Chacón et al. \cite{Chen_Chacón_Barnes_2011, Chacón_Chen_Barnes_2013, Chen_Chacón_2014, Chen_Chacón_2015} have developed structure-preserving particle pushers for neoclassical transport in the Vlasov equations, derived from Crank--Nicolson integrators. We show these too can can derive from a FET interpretation, similarly offering potential extensions to higher-order-in-time particle pushers. The FET formulation is used also to consider how the stochastic drift terms can be incorporated into the pushers. Stochastic gyrokinetic expansions are also discussed.

        Different options for the numerical implementation of these schemes are considered.

        Due to the efficacy of FET in the development of SP timesteppers for both the fluid and kinetic component, we hope this approach will prove effective in the future for developing SP timesteppers for the full hybrid model. We hope this will give us the opportunity to incorporate previously inaccessible kinetic effects into the highly effective, modern, finite-element MHD models.
    \end{abstract}
    
    
    \newpage
    \tableofcontents
    
    
    \newpage
    \pagenumbering{arabic}
    %\linenumbers\renewcommand\thelinenumber{\color{black!50}\arabic{linenumber}}
            \input{0 - introduction/main.tex}
        \part{Research}
            \input{1 - low-noise PiC models/main.tex}
            \input{2 - kinetic component/main.tex}
            \input{3 - fluid component/main.tex}
            \input{4 - numerical implementation/main.tex}
        \part{Project Overview}
            \input{5 - research plan/main.tex}
            \input{6 - summary/main.tex}
    
    
    %\section{}
    \newpage
    \pagenumbering{gobble}
        \printbibliography


    \newpage
    \pagenumbering{roman}
    \appendix
        \part{Appendices}
            \input{8 - Hilbert complexes/main.tex}
            \input{9 - weak conservation proofs/main.tex}
\end{document}

        \part{Project Overview}
            \documentclass[12pt, a4paper]{report}

\input{template/main.tex}

\title{\BA{Title in Progress...}}
\author{Boris Andrews}
\affil{Mathematical Institute, University of Oxford}
\date{\today}


\begin{document}
    \pagenumbering{gobble}
    \maketitle
    
    
    \begin{abstract}
        Magnetic confinement reactors---in particular tokamaks---offer one of the most promising options for achieving practical nuclear fusion, with the potential to provide virtually limitless, clean energy. The theoretical and numerical modeling of tokamak plasmas is simultaneously an essential component of effective reactor design, and a great research barrier. Tokamak operational conditions exhibit comparatively low Knudsen numbers. Kinetic effects, including kinetic waves and instabilities, Landau damping, bump-on-tail instabilities and more, are therefore highly influential in tokamak plasma dynamics. Purely fluid models are inherently incapable of capturing these effects, whereas the high dimensionality in purely kinetic models render them practically intractable for most relevant purposes.

        We consider a $\delta\!f$ decomposition model, with a macroscopic fluid background and microscopic kinetic correction, both fully coupled to each other. A similar manner of discretization is proposed to that used in the recent \texttt{STRUPHY} code \cite{Holderied_Possanner_Wang_2021, Holderied_2022, Li_et_al_2023} with a finite-element model for the background and a pseudo-particle/PiC model for the correction.

        The fluid background satisfies the full, non-linear, resistive, compressible, Hall MHD equations. \cite{Laakmann_Hu_Farrell_2022} introduces finite-element(-in-space) implicit timesteppers for the incompressible analogue to this system with structure-preserving (SP) properties in the ideal case, alongside parameter-robust preconditioners. We show that these timesteppers can derive from a finite-element-in-time (FET) (and finite-element-in-space) interpretation. The benefits of this reformulation are discussed, including the derivation of timesteppers that are higher order in time, and the quantifiable dissipative SP properties in the non-ideal, resistive case.
        
        We discuss possible options for extending this FET approach to timesteppers for the compressible case.

        The kinetic corrections satisfy linearized Boltzmann equations. Using a Lénard--Bernstein collision operator, these take Fokker--Planck-like forms \cite{Fokker_1914, Planck_1917} wherein pseudo-particles in the numerical model obey the neoclassical transport equations, with particle-independent Brownian drift terms. This offers a rigorous methodology for incorporating collisions into the particle transport model, without coupling the equations of motions for each particle.
        
        Works by Chen, Chacón et al. \cite{Chen_Chacón_Barnes_2011, Chacón_Chen_Barnes_2013, Chen_Chacón_2014, Chen_Chacón_2015} have developed structure-preserving particle pushers for neoclassical transport in the Vlasov equations, derived from Crank--Nicolson integrators. We show these too can can derive from a FET interpretation, similarly offering potential extensions to higher-order-in-time particle pushers. The FET formulation is used also to consider how the stochastic drift terms can be incorporated into the pushers. Stochastic gyrokinetic expansions are also discussed.

        Different options for the numerical implementation of these schemes are considered.

        Due to the efficacy of FET in the development of SP timesteppers for both the fluid and kinetic component, we hope this approach will prove effective in the future for developing SP timesteppers for the full hybrid model. We hope this will give us the opportunity to incorporate previously inaccessible kinetic effects into the highly effective, modern, finite-element MHD models.
    \end{abstract}
    
    
    \newpage
    \tableofcontents
    
    
    \newpage
    \pagenumbering{arabic}
    %\linenumbers\renewcommand\thelinenumber{\color{black!50}\arabic{linenumber}}
            \input{0 - introduction/main.tex}
        \part{Research}
            \input{1 - low-noise PiC models/main.tex}
            \input{2 - kinetic component/main.tex}
            \input{3 - fluid component/main.tex}
            \input{4 - numerical implementation/main.tex}
        \part{Project Overview}
            \input{5 - research plan/main.tex}
            \input{6 - summary/main.tex}
    
    
    %\section{}
    \newpage
    \pagenumbering{gobble}
        \printbibliography


    \newpage
    \pagenumbering{roman}
    \appendix
        \part{Appendices}
            \input{8 - Hilbert complexes/main.tex}
            \input{9 - weak conservation proofs/main.tex}
\end{document}

            \documentclass[12pt, a4paper]{report}

\input{template/main.tex}

\title{\BA{Title in Progress...}}
\author{Boris Andrews}
\affil{Mathematical Institute, University of Oxford}
\date{\today}


\begin{document}
    \pagenumbering{gobble}
    \maketitle
    
    
    \begin{abstract}
        Magnetic confinement reactors---in particular tokamaks---offer one of the most promising options for achieving practical nuclear fusion, with the potential to provide virtually limitless, clean energy. The theoretical and numerical modeling of tokamak plasmas is simultaneously an essential component of effective reactor design, and a great research barrier. Tokamak operational conditions exhibit comparatively low Knudsen numbers. Kinetic effects, including kinetic waves and instabilities, Landau damping, bump-on-tail instabilities and more, are therefore highly influential in tokamak plasma dynamics. Purely fluid models are inherently incapable of capturing these effects, whereas the high dimensionality in purely kinetic models render them practically intractable for most relevant purposes.

        We consider a $\delta\!f$ decomposition model, with a macroscopic fluid background and microscopic kinetic correction, both fully coupled to each other. A similar manner of discretization is proposed to that used in the recent \texttt{STRUPHY} code \cite{Holderied_Possanner_Wang_2021, Holderied_2022, Li_et_al_2023} with a finite-element model for the background and a pseudo-particle/PiC model for the correction.

        The fluid background satisfies the full, non-linear, resistive, compressible, Hall MHD equations. \cite{Laakmann_Hu_Farrell_2022} introduces finite-element(-in-space) implicit timesteppers for the incompressible analogue to this system with structure-preserving (SP) properties in the ideal case, alongside parameter-robust preconditioners. We show that these timesteppers can derive from a finite-element-in-time (FET) (and finite-element-in-space) interpretation. The benefits of this reformulation are discussed, including the derivation of timesteppers that are higher order in time, and the quantifiable dissipative SP properties in the non-ideal, resistive case.
        
        We discuss possible options for extending this FET approach to timesteppers for the compressible case.

        The kinetic corrections satisfy linearized Boltzmann equations. Using a Lénard--Bernstein collision operator, these take Fokker--Planck-like forms \cite{Fokker_1914, Planck_1917} wherein pseudo-particles in the numerical model obey the neoclassical transport equations, with particle-independent Brownian drift terms. This offers a rigorous methodology for incorporating collisions into the particle transport model, without coupling the equations of motions for each particle.
        
        Works by Chen, Chacón et al. \cite{Chen_Chacón_Barnes_2011, Chacón_Chen_Barnes_2013, Chen_Chacón_2014, Chen_Chacón_2015} have developed structure-preserving particle pushers for neoclassical transport in the Vlasov equations, derived from Crank--Nicolson integrators. We show these too can can derive from a FET interpretation, similarly offering potential extensions to higher-order-in-time particle pushers. The FET formulation is used also to consider how the stochastic drift terms can be incorporated into the pushers. Stochastic gyrokinetic expansions are also discussed.

        Different options for the numerical implementation of these schemes are considered.

        Due to the efficacy of FET in the development of SP timesteppers for both the fluid and kinetic component, we hope this approach will prove effective in the future for developing SP timesteppers for the full hybrid model. We hope this will give us the opportunity to incorporate previously inaccessible kinetic effects into the highly effective, modern, finite-element MHD models.
    \end{abstract}
    
    
    \newpage
    \tableofcontents
    
    
    \newpage
    \pagenumbering{arabic}
    %\linenumbers\renewcommand\thelinenumber{\color{black!50}\arabic{linenumber}}
            \input{0 - introduction/main.tex}
        \part{Research}
            \input{1 - low-noise PiC models/main.tex}
            \input{2 - kinetic component/main.tex}
            \input{3 - fluid component/main.tex}
            \input{4 - numerical implementation/main.tex}
        \part{Project Overview}
            \input{5 - research plan/main.tex}
            \input{6 - summary/main.tex}
    
    
    %\section{}
    \newpage
    \pagenumbering{gobble}
        \printbibliography


    \newpage
    \pagenumbering{roman}
    \appendix
        \part{Appendices}
            \input{8 - Hilbert complexes/main.tex}
            \input{9 - weak conservation proofs/main.tex}
\end{document}

    
    
    %\section{}
    \newpage
    \pagenumbering{gobble}
        \printbibliography


    \newpage
    \pagenumbering{roman}
    \appendix
        \part{Appendices}
            \documentclass[12pt, a4paper]{report}

\input{template/main.tex}

\title{\BA{Title in Progress...}}
\author{Boris Andrews}
\affil{Mathematical Institute, University of Oxford}
\date{\today}


\begin{document}
    \pagenumbering{gobble}
    \maketitle
    
    
    \begin{abstract}
        Magnetic confinement reactors---in particular tokamaks---offer one of the most promising options for achieving practical nuclear fusion, with the potential to provide virtually limitless, clean energy. The theoretical and numerical modeling of tokamak plasmas is simultaneously an essential component of effective reactor design, and a great research barrier. Tokamak operational conditions exhibit comparatively low Knudsen numbers. Kinetic effects, including kinetic waves and instabilities, Landau damping, bump-on-tail instabilities and more, are therefore highly influential in tokamak plasma dynamics. Purely fluid models are inherently incapable of capturing these effects, whereas the high dimensionality in purely kinetic models render them practically intractable for most relevant purposes.

        We consider a $\delta\!f$ decomposition model, with a macroscopic fluid background and microscopic kinetic correction, both fully coupled to each other. A similar manner of discretization is proposed to that used in the recent \texttt{STRUPHY} code \cite{Holderied_Possanner_Wang_2021, Holderied_2022, Li_et_al_2023} with a finite-element model for the background and a pseudo-particle/PiC model for the correction.

        The fluid background satisfies the full, non-linear, resistive, compressible, Hall MHD equations. \cite{Laakmann_Hu_Farrell_2022} introduces finite-element(-in-space) implicit timesteppers for the incompressible analogue to this system with structure-preserving (SP) properties in the ideal case, alongside parameter-robust preconditioners. We show that these timesteppers can derive from a finite-element-in-time (FET) (and finite-element-in-space) interpretation. The benefits of this reformulation are discussed, including the derivation of timesteppers that are higher order in time, and the quantifiable dissipative SP properties in the non-ideal, resistive case.
        
        We discuss possible options for extending this FET approach to timesteppers for the compressible case.

        The kinetic corrections satisfy linearized Boltzmann equations. Using a Lénard--Bernstein collision operator, these take Fokker--Planck-like forms \cite{Fokker_1914, Planck_1917} wherein pseudo-particles in the numerical model obey the neoclassical transport equations, with particle-independent Brownian drift terms. This offers a rigorous methodology for incorporating collisions into the particle transport model, without coupling the equations of motions for each particle.
        
        Works by Chen, Chacón et al. \cite{Chen_Chacón_Barnes_2011, Chacón_Chen_Barnes_2013, Chen_Chacón_2014, Chen_Chacón_2015} have developed structure-preserving particle pushers for neoclassical transport in the Vlasov equations, derived from Crank--Nicolson integrators. We show these too can can derive from a FET interpretation, similarly offering potential extensions to higher-order-in-time particle pushers. The FET formulation is used also to consider how the stochastic drift terms can be incorporated into the pushers. Stochastic gyrokinetic expansions are also discussed.

        Different options for the numerical implementation of these schemes are considered.

        Due to the efficacy of FET in the development of SP timesteppers for both the fluid and kinetic component, we hope this approach will prove effective in the future for developing SP timesteppers for the full hybrid model. We hope this will give us the opportunity to incorporate previously inaccessible kinetic effects into the highly effective, modern, finite-element MHD models.
    \end{abstract}
    
    
    \newpage
    \tableofcontents
    
    
    \newpage
    \pagenumbering{arabic}
    %\linenumbers\renewcommand\thelinenumber{\color{black!50}\arabic{linenumber}}
            \input{0 - introduction/main.tex}
        \part{Research}
            \input{1 - low-noise PiC models/main.tex}
            \input{2 - kinetic component/main.tex}
            \input{3 - fluid component/main.tex}
            \input{4 - numerical implementation/main.tex}
        \part{Project Overview}
            \input{5 - research plan/main.tex}
            \input{6 - summary/main.tex}
    
    
    %\section{}
    \newpage
    \pagenumbering{gobble}
        \printbibliography


    \newpage
    \pagenumbering{roman}
    \appendix
        \part{Appendices}
            \input{8 - Hilbert complexes/main.tex}
            \input{9 - weak conservation proofs/main.tex}
\end{document}

            \documentclass[12pt, a4paper]{report}

\input{template/main.tex}

\title{\BA{Title in Progress...}}
\author{Boris Andrews}
\affil{Mathematical Institute, University of Oxford}
\date{\today}


\begin{document}
    \pagenumbering{gobble}
    \maketitle
    
    
    \begin{abstract}
        Magnetic confinement reactors---in particular tokamaks---offer one of the most promising options for achieving practical nuclear fusion, with the potential to provide virtually limitless, clean energy. The theoretical and numerical modeling of tokamak plasmas is simultaneously an essential component of effective reactor design, and a great research barrier. Tokamak operational conditions exhibit comparatively low Knudsen numbers. Kinetic effects, including kinetic waves and instabilities, Landau damping, bump-on-tail instabilities and more, are therefore highly influential in tokamak plasma dynamics. Purely fluid models are inherently incapable of capturing these effects, whereas the high dimensionality in purely kinetic models render them practically intractable for most relevant purposes.

        We consider a $\delta\!f$ decomposition model, with a macroscopic fluid background and microscopic kinetic correction, both fully coupled to each other. A similar manner of discretization is proposed to that used in the recent \texttt{STRUPHY} code \cite{Holderied_Possanner_Wang_2021, Holderied_2022, Li_et_al_2023} with a finite-element model for the background and a pseudo-particle/PiC model for the correction.

        The fluid background satisfies the full, non-linear, resistive, compressible, Hall MHD equations. \cite{Laakmann_Hu_Farrell_2022} introduces finite-element(-in-space) implicit timesteppers for the incompressible analogue to this system with structure-preserving (SP) properties in the ideal case, alongside parameter-robust preconditioners. We show that these timesteppers can derive from a finite-element-in-time (FET) (and finite-element-in-space) interpretation. The benefits of this reformulation are discussed, including the derivation of timesteppers that are higher order in time, and the quantifiable dissipative SP properties in the non-ideal, resistive case.
        
        We discuss possible options for extending this FET approach to timesteppers for the compressible case.

        The kinetic corrections satisfy linearized Boltzmann equations. Using a Lénard--Bernstein collision operator, these take Fokker--Planck-like forms \cite{Fokker_1914, Planck_1917} wherein pseudo-particles in the numerical model obey the neoclassical transport equations, with particle-independent Brownian drift terms. This offers a rigorous methodology for incorporating collisions into the particle transport model, without coupling the equations of motions for each particle.
        
        Works by Chen, Chacón et al. \cite{Chen_Chacón_Barnes_2011, Chacón_Chen_Barnes_2013, Chen_Chacón_2014, Chen_Chacón_2015} have developed structure-preserving particle pushers for neoclassical transport in the Vlasov equations, derived from Crank--Nicolson integrators. We show these too can can derive from a FET interpretation, similarly offering potential extensions to higher-order-in-time particle pushers. The FET formulation is used also to consider how the stochastic drift terms can be incorporated into the pushers. Stochastic gyrokinetic expansions are also discussed.

        Different options for the numerical implementation of these schemes are considered.

        Due to the efficacy of FET in the development of SP timesteppers for both the fluid and kinetic component, we hope this approach will prove effective in the future for developing SP timesteppers for the full hybrid model. We hope this will give us the opportunity to incorporate previously inaccessible kinetic effects into the highly effective, modern, finite-element MHD models.
    \end{abstract}
    
    
    \newpage
    \tableofcontents
    
    
    \newpage
    \pagenumbering{arabic}
    %\linenumbers\renewcommand\thelinenumber{\color{black!50}\arabic{linenumber}}
            \input{0 - introduction/main.tex}
        \part{Research}
            \input{1 - low-noise PiC models/main.tex}
            \input{2 - kinetic component/main.tex}
            \input{3 - fluid component/main.tex}
            \input{4 - numerical implementation/main.tex}
        \part{Project Overview}
            \input{5 - research plan/main.tex}
            \input{6 - summary/main.tex}
    
    
    %\section{}
    \newpage
    \pagenumbering{gobble}
        \printbibliography


    \newpage
    \pagenumbering{roman}
    \appendix
        \part{Appendices}
            \input{8 - Hilbert complexes/main.tex}
            \input{9 - weak conservation proofs/main.tex}
\end{document}

\end{document}

            \documentclass[12pt, a4paper]{report}

\documentclass[12pt, a4paper]{report}

\input{template/main.tex}

\title{\BA{Title in Progress...}}
\author{Boris Andrews}
\affil{Mathematical Institute, University of Oxford}
\date{\today}


\begin{document}
    \pagenumbering{gobble}
    \maketitle
    
    
    \begin{abstract}
        Magnetic confinement reactors---in particular tokamaks---offer one of the most promising options for achieving practical nuclear fusion, with the potential to provide virtually limitless, clean energy. The theoretical and numerical modeling of tokamak plasmas is simultaneously an essential component of effective reactor design, and a great research barrier. Tokamak operational conditions exhibit comparatively low Knudsen numbers. Kinetic effects, including kinetic waves and instabilities, Landau damping, bump-on-tail instabilities and more, are therefore highly influential in tokamak plasma dynamics. Purely fluid models are inherently incapable of capturing these effects, whereas the high dimensionality in purely kinetic models render them practically intractable for most relevant purposes.

        We consider a $\delta\!f$ decomposition model, with a macroscopic fluid background and microscopic kinetic correction, both fully coupled to each other. A similar manner of discretization is proposed to that used in the recent \texttt{STRUPHY} code \cite{Holderied_Possanner_Wang_2021, Holderied_2022, Li_et_al_2023} with a finite-element model for the background and a pseudo-particle/PiC model for the correction.

        The fluid background satisfies the full, non-linear, resistive, compressible, Hall MHD equations. \cite{Laakmann_Hu_Farrell_2022} introduces finite-element(-in-space) implicit timesteppers for the incompressible analogue to this system with structure-preserving (SP) properties in the ideal case, alongside parameter-robust preconditioners. We show that these timesteppers can derive from a finite-element-in-time (FET) (and finite-element-in-space) interpretation. The benefits of this reformulation are discussed, including the derivation of timesteppers that are higher order in time, and the quantifiable dissipative SP properties in the non-ideal, resistive case.
        
        We discuss possible options for extending this FET approach to timesteppers for the compressible case.

        The kinetic corrections satisfy linearized Boltzmann equations. Using a Lénard--Bernstein collision operator, these take Fokker--Planck-like forms \cite{Fokker_1914, Planck_1917} wherein pseudo-particles in the numerical model obey the neoclassical transport equations, with particle-independent Brownian drift terms. This offers a rigorous methodology for incorporating collisions into the particle transport model, without coupling the equations of motions for each particle.
        
        Works by Chen, Chacón et al. \cite{Chen_Chacón_Barnes_2011, Chacón_Chen_Barnes_2013, Chen_Chacón_2014, Chen_Chacón_2015} have developed structure-preserving particle pushers for neoclassical transport in the Vlasov equations, derived from Crank--Nicolson integrators. We show these too can can derive from a FET interpretation, similarly offering potential extensions to higher-order-in-time particle pushers. The FET formulation is used also to consider how the stochastic drift terms can be incorporated into the pushers. Stochastic gyrokinetic expansions are also discussed.

        Different options for the numerical implementation of these schemes are considered.

        Due to the efficacy of FET in the development of SP timesteppers for both the fluid and kinetic component, we hope this approach will prove effective in the future for developing SP timesteppers for the full hybrid model. We hope this will give us the opportunity to incorporate previously inaccessible kinetic effects into the highly effective, modern, finite-element MHD models.
    \end{abstract}
    
    
    \newpage
    \tableofcontents
    
    
    \newpage
    \pagenumbering{arabic}
    %\linenumbers\renewcommand\thelinenumber{\color{black!50}\arabic{linenumber}}
            \input{0 - introduction/main.tex}
        \part{Research}
            \input{1 - low-noise PiC models/main.tex}
            \input{2 - kinetic component/main.tex}
            \input{3 - fluid component/main.tex}
            \input{4 - numerical implementation/main.tex}
        \part{Project Overview}
            \input{5 - research plan/main.tex}
            \input{6 - summary/main.tex}
    
    
    %\section{}
    \newpage
    \pagenumbering{gobble}
        \printbibliography


    \newpage
    \pagenumbering{roman}
    \appendix
        \part{Appendices}
            \input{8 - Hilbert complexes/main.tex}
            \input{9 - weak conservation proofs/main.tex}
\end{document}


\title{\BA{Title in Progress...}}
\author{Boris Andrews}
\affil{Mathematical Institute, University of Oxford}
\date{\today}


\begin{document}
    \pagenumbering{gobble}
    \maketitle
    
    
    \begin{abstract}
        Magnetic confinement reactors---in particular tokamaks---offer one of the most promising options for achieving practical nuclear fusion, with the potential to provide virtually limitless, clean energy. The theoretical and numerical modeling of tokamak plasmas is simultaneously an essential component of effective reactor design, and a great research barrier. Tokamak operational conditions exhibit comparatively low Knudsen numbers. Kinetic effects, including kinetic waves and instabilities, Landau damping, bump-on-tail instabilities and more, are therefore highly influential in tokamak plasma dynamics. Purely fluid models are inherently incapable of capturing these effects, whereas the high dimensionality in purely kinetic models render them practically intractable for most relevant purposes.

        We consider a $\delta\!f$ decomposition model, with a macroscopic fluid background and microscopic kinetic correction, both fully coupled to each other. A similar manner of discretization is proposed to that used in the recent \texttt{STRUPHY} code \cite{Holderied_Possanner_Wang_2021, Holderied_2022, Li_et_al_2023} with a finite-element model for the background and a pseudo-particle/PiC model for the correction.

        The fluid background satisfies the full, non-linear, resistive, compressible, Hall MHD equations. \cite{Laakmann_Hu_Farrell_2022} introduces finite-element(-in-space) implicit timesteppers for the incompressible analogue to this system with structure-preserving (SP) properties in the ideal case, alongside parameter-robust preconditioners. We show that these timesteppers can derive from a finite-element-in-time (FET) (and finite-element-in-space) interpretation. The benefits of this reformulation are discussed, including the derivation of timesteppers that are higher order in time, and the quantifiable dissipative SP properties in the non-ideal, resistive case.
        
        We discuss possible options for extending this FET approach to timesteppers for the compressible case.

        The kinetic corrections satisfy linearized Boltzmann equations. Using a Lénard--Bernstein collision operator, these take Fokker--Planck-like forms \cite{Fokker_1914, Planck_1917} wherein pseudo-particles in the numerical model obey the neoclassical transport equations, with particle-independent Brownian drift terms. This offers a rigorous methodology for incorporating collisions into the particle transport model, without coupling the equations of motions for each particle.
        
        Works by Chen, Chacón et al. \cite{Chen_Chacón_Barnes_2011, Chacón_Chen_Barnes_2013, Chen_Chacón_2014, Chen_Chacón_2015} have developed structure-preserving particle pushers for neoclassical transport in the Vlasov equations, derived from Crank--Nicolson integrators. We show these too can can derive from a FET interpretation, similarly offering potential extensions to higher-order-in-time particle pushers. The FET formulation is used also to consider how the stochastic drift terms can be incorporated into the pushers. Stochastic gyrokinetic expansions are also discussed.

        Different options for the numerical implementation of these schemes are considered.

        Due to the efficacy of FET in the development of SP timesteppers for both the fluid and kinetic component, we hope this approach will prove effective in the future for developing SP timesteppers for the full hybrid model. We hope this will give us the opportunity to incorporate previously inaccessible kinetic effects into the highly effective, modern, finite-element MHD models.
    \end{abstract}
    
    
    \newpage
    \tableofcontents
    
    
    \newpage
    \pagenumbering{arabic}
    %\linenumbers\renewcommand\thelinenumber{\color{black!50}\arabic{linenumber}}
            \documentclass[12pt, a4paper]{report}

\input{template/main.tex}

\title{\BA{Title in Progress...}}
\author{Boris Andrews}
\affil{Mathematical Institute, University of Oxford}
\date{\today}


\begin{document}
    \pagenumbering{gobble}
    \maketitle
    
    
    \begin{abstract}
        Magnetic confinement reactors---in particular tokamaks---offer one of the most promising options for achieving practical nuclear fusion, with the potential to provide virtually limitless, clean energy. The theoretical and numerical modeling of tokamak plasmas is simultaneously an essential component of effective reactor design, and a great research barrier. Tokamak operational conditions exhibit comparatively low Knudsen numbers. Kinetic effects, including kinetic waves and instabilities, Landau damping, bump-on-tail instabilities and more, are therefore highly influential in tokamak plasma dynamics. Purely fluid models are inherently incapable of capturing these effects, whereas the high dimensionality in purely kinetic models render them practically intractable for most relevant purposes.

        We consider a $\delta\!f$ decomposition model, with a macroscopic fluid background and microscopic kinetic correction, both fully coupled to each other. A similar manner of discretization is proposed to that used in the recent \texttt{STRUPHY} code \cite{Holderied_Possanner_Wang_2021, Holderied_2022, Li_et_al_2023} with a finite-element model for the background and a pseudo-particle/PiC model for the correction.

        The fluid background satisfies the full, non-linear, resistive, compressible, Hall MHD equations. \cite{Laakmann_Hu_Farrell_2022} introduces finite-element(-in-space) implicit timesteppers for the incompressible analogue to this system with structure-preserving (SP) properties in the ideal case, alongside parameter-robust preconditioners. We show that these timesteppers can derive from a finite-element-in-time (FET) (and finite-element-in-space) interpretation. The benefits of this reformulation are discussed, including the derivation of timesteppers that are higher order in time, and the quantifiable dissipative SP properties in the non-ideal, resistive case.
        
        We discuss possible options for extending this FET approach to timesteppers for the compressible case.

        The kinetic corrections satisfy linearized Boltzmann equations. Using a Lénard--Bernstein collision operator, these take Fokker--Planck-like forms \cite{Fokker_1914, Planck_1917} wherein pseudo-particles in the numerical model obey the neoclassical transport equations, with particle-independent Brownian drift terms. This offers a rigorous methodology for incorporating collisions into the particle transport model, without coupling the equations of motions for each particle.
        
        Works by Chen, Chacón et al. \cite{Chen_Chacón_Barnes_2011, Chacón_Chen_Barnes_2013, Chen_Chacón_2014, Chen_Chacón_2015} have developed structure-preserving particle pushers for neoclassical transport in the Vlasov equations, derived from Crank--Nicolson integrators. We show these too can can derive from a FET interpretation, similarly offering potential extensions to higher-order-in-time particle pushers. The FET formulation is used also to consider how the stochastic drift terms can be incorporated into the pushers. Stochastic gyrokinetic expansions are also discussed.

        Different options for the numerical implementation of these schemes are considered.

        Due to the efficacy of FET in the development of SP timesteppers for both the fluid and kinetic component, we hope this approach will prove effective in the future for developing SP timesteppers for the full hybrid model. We hope this will give us the opportunity to incorporate previously inaccessible kinetic effects into the highly effective, modern, finite-element MHD models.
    \end{abstract}
    
    
    \newpage
    \tableofcontents
    
    
    \newpage
    \pagenumbering{arabic}
    %\linenumbers\renewcommand\thelinenumber{\color{black!50}\arabic{linenumber}}
            \input{0 - introduction/main.tex}
        \part{Research}
            \input{1 - low-noise PiC models/main.tex}
            \input{2 - kinetic component/main.tex}
            \input{3 - fluid component/main.tex}
            \input{4 - numerical implementation/main.tex}
        \part{Project Overview}
            \input{5 - research plan/main.tex}
            \input{6 - summary/main.tex}
    
    
    %\section{}
    \newpage
    \pagenumbering{gobble}
        \printbibliography


    \newpage
    \pagenumbering{roman}
    \appendix
        \part{Appendices}
            \input{8 - Hilbert complexes/main.tex}
            \input{9 - weak conservation proofs/main.tex}
\end{document}

        \part{Research}
            \documentclass[12pt, a4paper]{report}

\input{template/main.tex}

\title{\BA{Title in Progress...}}
\author{Boris Andrews}
\affil{Mathematical Institute, University of Oxford}
\date{\today}


\begin{document}
    \pagenumbering{gobble}
    \maketitle
    
    
    \begin{abstract}
        Magnetic confinement reactors---in particular tokamaks---offer one of the most promising options for achieving practical nuclear fusion, with the potential to provide virtually limitless, clean energy. The theoretical and numerical modeling of tokamak plasmas is simultaneously an essential component of effective reactor design, and a great research barrier. Tokamak operational conditions exhibit comparatively low Knudsen numbers. Kinetic effects, including kinetic waves and instabilities, Landau damping, bump-on-tail instabilities and more, are therefore highly influential in tokamak plasma dynamics. Purely fluid models are inherently incapable of capturing these effects, whereas the high dimensionality in purely kinetic models render them practically intractable for most relevant purposes.

        We consider a $\delta\!f$ decomposition model, with a macroscopic fluid background and microscopic kinetic correction, both fully coupled to each other. A similar manner of discretization is proposed to that used in the recent \texttt{STRUPHY} code \cite{Holderied_Possanner_Wang_2021, Holderied_2022, Li_et_al_2023} with a finite-element model for the background and a pseudo-particle/PiC model for the correction.

        The fluid background satisfies the full, non-linear, resistive, compressible, Hall MHD equations. \cite{Laakmann_Hu_Farrell_2022} introduces finite-element(-in-space) implicit timesteppers for the incompressible analogue to this system with structure-preserving (SP) properties in the ideal case, alongside parameter-robust preconditioners. We show that these timesteppers can derive from a finite-element-in-time (FET) (and finite-element-in-space) interpretation. The benefits of this reformulation are discussed, including the derivation of timesteppers that are higher order in time, and the quantifiable dissipative SP properties in the non-ideal, resistive case.
        
        We discuss possible options for extending this FET approach to timesteppers for the compressible case.

        The kinetic corrections satisfy linearized Boltzmann equations. Using a Lénard--Bernstein collision operator, these take Fokker--Planck-like forms \cite{Fokker_1914, Planck_1917} wherein pseudo-particles in the numerical model obey the neoclassical transport equations, with particle-independent Brownian drift terms. This offers a rigorous methodology for incorporating collisions into the particle transport model, without coupling the equations of motions for each particle.
        
        Works by Chen, Chacón et al. \cite{Chen_Chacón_Barnes_2011, Chacón_Chen_Barnes_2013, Chen_Chacón_2014, Chen_Chacón_2015} have developed structure-preserving particle pushers for neoclassical transport in the Vlasov equations, derived from Crank--Nicolson integrators. We show these too can can derive from a FET interpretation, similarly offering potential extensions to higher-order-in-time particle pushers. The FET formulation is used also to consider how the stochastic drift terms can be incorporated into the pushers. Stochastic gyrokinetic expansions are also discussed.

        Different options for the numerical implementation of these schemes are considered.

        Due to the efficacy of FET in the development of SP timesteppers for both the fluid and kinetic component, we hope this approach will prove effective in the future for developing SP timesteppers for the full hybrid model. We hope this will give us the opportunity to incorporate previously inaccessible kinetic effects into the highly effective, modern, finite-element MHD models.
    \end{abstract}
    
    
    \newpage
    \tableofcontents
    
    
    \newpage
    \pagenumbering{arabic}
    %\linenumbers\renewcommand\thelinenumber{\color{black!50}\arabic{linenumber}}
            \input{0 - introduction/main.tex}
        \part{Research}
            \input{1 - low-noise PiC models/main.tex}
            \input{2 - kinetic component/main.tex}
            \input{3 - fluid component/main.tex}
            \input{4 - numerical implementation/main.tex}
        \part{Project Overview}
            \input{5 - research plan/main.tex}
            \input{6 - summary/main.tex}
    
    
    %\section{}
    \newpage
    \pagenumbering{gobble}
        \printbibliography


    \newpage
    \pagenumbering{roman}
    \appendix
        \part{Appendices}
            \input{8 - Hilbert complexes/main.tex}
            \input{9 - weak conservation proofs/main.tex}
\end{document}

            \documentclass[12pt, a4paper]{report}

\input{template/main.tex}

\title{\BA{Title in Progress...}}
\author{Boris Andrews}
\affil{Mathematical Institute, University of Oxford}
\date{\today}


\begin{document}
    \pagenumbering{gobble}
    \maketitle
    
    
    \begin{abstract}
        Magnetic confinement reactors---in particular tokamaks---offer one of the most promising options for achieving practical nuclear fusion, with the potential to provide virtually limitless, clean energy. The theoretical and numerical modeling of tokamak plasmas is simultaneously an essential component of effective reactor design, and a great research barrier. Tokamak operational conditions exhibit comparatively low Knudsen numbers. Kinetic effects, including kinetic waves and instabilities, Landau damping, bump-on-tail instabilities and more, are therefore highly influential in tokamak plasma dynamics. Purely fluid models are inherently incapable of capturing these effects, whereas the high dimensionality in purely kinetic models render them practically intractable for most relevant purposes.

        We consider a $\delta\!f$ decomposition model, with a macroscopic fluid background and microscopic kinetic correction, both fully coupled to each other. A similar manner of discretization is proposed to that used in the recent \texttt{STRUPHY} code \cite{Holderied_Possanner_Wang_2021, Holderied_2022, Li_et_al_2023} with a finite-element model for the background and a pseudo-particle/PiC model for the correction.

        The fluid background satisfies the full, non-linear, resistive, compressible, Hall MHD equations. \cite{Laakmann_Hu_Farrell_2022} introduces finite-element(-in-space) implicit timesteppers for the incompressible analogue to this system with structure-preserving (SP) properties in the ideal case, alongside parameter-robust preconditioners. We show that these timesteppers can derive from a finite-element-in-time (FET) (and finite-element-in-space) interpretation. The benefits of this reformulation are discussed, including the derivation of timesteppers that are higher order in time, and the quantifiable dissipative SP properties in the non-ideal, resistive case.
        
        We discuss possible options for extending this FET approach to timesteppers for the compressible case.

        The kinetic corrections satisfy linearized Boltzmann equations. Using a Lénard--Bernstein collision operator, these take Fokker--Planck-like forms \cite{Fokker_1914, Planck_1917} wherein pseudo-particles in the numerical model obey the neoclassical transport equations, with particle-independent Brownian drift terms. This offers a rigorous methodology for incorporating collisions into the particle transport model, without coupling the equations of motions for each particle.
        
        Works by Chen, Chacón et al. \cite{Chen_Chacón_Barnes_2011, Chacón_Chen_Barnes_2013, Chen_Chacón_2014, Chen_Chacón_2015} have developed structure-preserving particle pushers for neoclassical transport in the Vlasov equations, derived from Crank--Nicolson integrators. We show these too can can derive from a FET interpretation, similarly offering potential extensions to higher-order-in-time particle pushers. The FET formulation is used also to consider how the stochastic drift terms can be incorporated into the pushers. Stochastic gyrokinetic expansions are also discussed.

        Different options for the numerical implementation of these schemes are considered.

        Due to the efficacy of FET in the development of SP timesteppers for both the fluid and kinetic component, we hope this approach will prove effective in the future for developing SP timesteppers for the full hybrid model. We hope this will give us the opportunity to incorporate previously inaccessible kinetic effects into the highly effective, modern, finite-element MHD models.
    \end{abstract}
    
    
    \newpage
    \tableofcontents
    
    
    \newpage
    \pagenumbering{arabic}
    %\linenumbers\renewcommand\thelinenumber{\color{black!50}\arabic{linenumber}}
            \input{0 - introduction/main.tex}
        \part{Research}
            \input{1 - low-noise PiC models/main.tex}
            \input{2 - kinetic component/main.tex}
            \input{3 - fluid component/main.tex}
            \input{4 - numerical implementation/main.tex}
        \part{Project Overview}
            \input{5 - research plan/main.tex}
            \input{6 - summary/main.tex}
    
    
    %\section{}
    \newpage
    \pagenumbering{gobble}
        \printbibliography


    \newpage
    \pagenumbering{roman}
    \appendix
        \part{Appendices}
            \input{8 - Hilbert complexes/main.tex}
            \input{9 - weak conservation proofs/main.tex}
\end{document}

            \documentclass[12pt, a4paper]{report}

\input{template/main.tex}

\title{\BA{Title in Progress...}}
\author{Boris Andrews}
\affil{Mathematical Institute, University of Oxford}
\date{\today}


\begin{document}
    \pagenumbering{gobble}
    \maketitle
    
    
    \begin{abstract}
        Magnetic confinement reactors---in particular tokamaks---offer one of the most promising options for achieving practical nuclear fusion, with the potential to provide virtually limitless, clean energy. The theoretical and numerical modeling of tokamak plasmas is simultaneously an essential component of effective reactor design, and a great research barrier. Tokamak operational conditions exhibit comparatively low Knudsen numbers. Kinetic effects, including kinetic waves and instabilities, Landau damping, bump-on-tail instabilities and more, are therefore highly influential in tokamak plasma dynamics. Purely fluid models are inherently incapable of capturing these effects, whereas the high dimensionality in purely kinetic models render them practically intractable for most relevant purposes.

        We consider a $\delta\!f$ decomposition model, with a macroscopic fluid background and microscopic kinetic correction, both fully coupled to each other. A similar manner of discretization is proposed to that used in the recent \texttt{STRUPHY} code \cite{Holderied_Possanner_Wang_2021, Holderied_2022, Li_et_al_2023} with a finite-element model for the background and a pseudo-particle/PiC model for the correction.

        The fluid background satisfies the full, non-linear, resistive, compressible, Hall MHD equations. \cite{Laakmann_Hu_Farrell_2022} introduces finite-element(-in-space) implicit timesteppers for the incompressible analogue to this system with structure-preserving (SP) properties in the ideal case, alongside parameter-robust preconditioners. We show that these timesteppers can derive from a finite-element-in-time (FET) (and finite-element-in-space) interpretation. The benefits of this reformulation are discussed, including the derivation of timesteppers that are higher order in time, and the quantifiable dissipative SP properties in the non-ideal, resistive case.
        
        We discuss possible options for extending this FET approach to timesteppers for the compressible case.

        The kinetic corrections satisfy linearized Boltzmann equations. Using a Lénard--Bernstein collision operator, these take Fokker--Planck-like forms \cite{Fokker_1914, Planck_1917} wherein pseudo-particles in the numerical model obey the neoclassical transport equations, with particle-independent Brownian drift terms. This offers a rigorous methodology for incorporating collisions into the particle transport model, without coupling the equations of motions for each particle.
        
        Works by Chen, Chacón et al. \cite{Chen_Chacón_Barnes_2011, Chacón_Chen_Barnes_2013, Chen_Chacón_2014, Chen_Chacón_2015} have developed structure-preserving particle pushers for neoclassical transport in the Vlasov equations, derived from Crank--Nicolson integrators. We show these too can can derive from a FET interpretation, similarly offering potential extensions to higher-order-in-time particle pushers. The FET formulation is used also to consider how the stochastic drift terms can be incorporated into the pushers. Stochastic gyrokinetic expansions are also discussed.

        Different options for the numerical implementation of these schemes are considered.

        Due to the efficacy of FET in the development of SP timesteppers for both the fluid and kinetic component, we hope this approach will prove effective in the future for developing SP timesteppers for the full hybrid model. We hope this will give us the opportunity to incorporate previously inaccessible kinetic effects into the highly effective, modern, finite-element MHD models.
    \end{abstract}
    
    
    \newpage
    \tableofcontents
    
    
    \newpage
    \pagenumbering{arabic}
    %\linenumbers\renewcommand\thelinenumber{\color{black!50}\arabic{linenumber}}
            \input{0 - introduction/main.tex}
        \part{Research}
            \input{1 - low-noise PiC models/main.tex}
            \input{2 - kinetic component/main.tex}
            \input{3 - fluid component/main.tex}
            \input{4 - numerical implementation/main.tex}
        \part{Project Overview}
            \input{5 - research plan/main.tex}
            \input{6 - summary/main.tex}
    
    
    %\section{}
    \newpage
    \pagenumbering{gobble}
        \printbibliography


    \newpage
    \pagenumbering{roman}
    \appendix
        \part{Appendices}
            \input{8 - Hilbert complexes/main.tex}
            \input{9 - weak conservation proofs/main.tex}
\end{document}

            \documentclass[12pt, a4paper]{report}

\input{template/main.tex}

\title{\BA{Title in Progress...}}
\author{Boris Andrews}
\affil{Mathematical Institute, University of Oxford}
\date{\today}


\begin{document}
    \pagenumbering{gobble}
    \maketitle
    
    
    \begin{abstract}
        Magnetic confinement reactors---in particular tokamaks---offer one of the most promising options for achieving practical nuclear fusion, with the potential to provide virtually limitless, clean energy. The theoretical and numerical modeling of tokamak plasmas is simultaneously an essential component of effective reactor design, and a great research barrier. Tokamak operational conditions exhibit comparatively low Knudsen numbers. Kinetic effects, including kinetic waves and instabilities, Landau damping, bump-on-tail instabilities and more, are therefore highly influential in tokamak plasma dynamics. Purely fluid models are inherently incapable of capturing these effects, whereas the high dimensionality in purely kinetic models render them practically intractable for most relevant purposes.

        We consider a $\delta\!f$ decomposition model, with a macroscopic fluid background and microscopic kinetic correction, both fully coupled to each other. A similar manner of discretization is proposed to that used in the recent \texttt{STRUPHY} code \cite{Holderied_Possanner_Wang_2021, Holderied_2022, Li_et_al_2023} with a finite-element model for the background and a pseudo-particle/PiC model for the correction.

        The fluid background satisfies the full, non-linear, resistive, compressible, Hall MHD equations. \cite{Laakmann_Hu_Farrell_2022} introduces finite-element(-in-space) implicit timesteppers for the incompressible analogue to this system with structure-preserving (SP) properties in the ideal case, alongside parameter-robust preconditioners. We show that these timesteppers can derive from a finite-element-in-time (FET) (and finite-element-in-space) interpretation. The benefits of this reformulation are discussed, including the derivation of timesteppers that are higher order in time, and the quantifiable dissipative SP properties in the non-ideal, resistive case.
        
        We discuss possible options for extending this FET approach to timesteppers for the compressible case.

        The kinetic corrections satisfy linearized Boltzmann equations. Using a Lénard--Bernstein collision operator, these take Fokker--Planck-like forms \cite{Fokker_1914, Planck_1917} wherein pseudo-particles in the numerical model obey the neoclassical transport equations, with particle-independent Brownian drift terms. This offers a rigorous methodology for incorporating collisions into the particle transport model, without coupling the equations of motions for each particle.
        
        Works by Chen, Chacón et al. \cite{Chen_Chacón_Barnes_2011, Chacón_Chen_Barnes_2013, Chen_Chacón_2014, Chen_Chacón_2015} have developed structure-preserving particle pushers for neoclassical transport in the Vlasov equations, derived from Crank--Nicolson integrators. We show these too can can derive from a FET interpretation, similarly offering potential extensions to higher-order-in-time particle pushers. The FET formulation is used also to consider how the stochastic drift terms can be incorporated into the pushers. Stochastic gyrokinetic expansions are also discussed.

        Different options for the numerical implementation of these schemes are considered.

        Due to the efficacy of FET in the development of SP timesteppers for both the fluid and kinetic component, we hope this approach will prove effective in the future for developing SP timesteppers for the full hybrid model. We hope this will give us the opportunity to incorporate previously inaccessible kinetic effects into the highly effective, modern, finite-element MHD models.
    \end{abstract}
    
    
    \newpage
    \tableofcontents
    
    
    \newpage
    \pagenumbering{arabic}
    %\linenumbers\renewcommand\thelinenumber{\color{black!50}\arabic{linenumber}}
            \input{0 - introduction/main.tex}
        \part{Research}
            \input{1 - low-noise PiC models/main.tex}
            \input{2 - kinetic component/main.tex}
            \input{3 - fluid component/main.tex}
            \input{4 - numerical implementation/main.tex}
        \part{Project Overview}
            \input{5 - research plan/main.tex}
            \input{6 - summary/main.tex}
    
    
    %\section{}
    \newpage
    \pagenumbering{gobble}
        \printbibliography


    \newpage
    \pagenumbering{roman}
    \appendix
        \part{Appendices}
            \input{8 - Hilbert complexes/main.tex}
            \input{9 - weak conservation proofs/main.tex}
\end{document}

        \part{Project Overview}
            \documentclass[12pt, a4paper]{report}

\input{template/main.tex}

\title{\BA{Title in Progress...}}
\author{Boris Andrews}
\affil{Mathematical Institute, University of Oxford}
\date{\today}


\begin{document}
    \pagenumbering{gobble}
    \maketitle
    
    
    \begin{abstract}
        Magnetic confinement reactors---in particular tokamaks---offer one of the most promising options for achieving practical nuclear fusion, with the potential to provide virtually limitless, clean energy. The theoretical and numerical modeling of tokamak plasmas is simultaneously an essential component of effective reactor design, and a great research barrier. Tokamak operational conditions exhibit comparatively low Knudsen numbers. Kinetic effects, including kinetic waves and instabilities, Landau damping, bump-on-tail instabilities and more, are therefore highly influential in tokamak plasma dynamics. Purely fluid models are inherently incapable of capturing these effects, whereas the high dimensionality in purely kinetic models render them practically intractable for most relevant purposes.

        We consider a $\delta\!f$ decomposition model, with a macroscopic fluid background and microscopic kinetic correction, both fully coupled to each other. A similar manner of discretization is proposed to that used in the recent \texttt{STRUPHY} code \cite{Holderied_Possanner_Wang_2021, Holderied_2022, Li_et_al_2023} with a finite-element model for the background and a pseudo-particle/PiC model for the correction.

        The fluid background satisfies the full, non-linear, resistive, compressible, Hall MHD equations. \cite{Laakmann_Hu_Farrell_2022} introduces finite-element(-in-space) implicit timesteppers for the incompressible analogue to this system with structure-preserving (SP) properties in the ideal case, alongside parameter-robust preconditioners. We show that these timesteppers can derive from a finite-element-in-time (FET) (and finite-element-in-space) interpretation. The benefits of this reformulation are discussed, including the derivation of timesteppers that are higher order in time, and the quantifiable dissipative SP properties in the non-ideal, resistive case.
        
        We discuss possible options for extending this FET approach to timesteppers for the compressible case.

        The kinetic corrections satisfy linearized Boltzmann equations. Using a Lénard--Bernstein collision operator, these take Fokker--Planck-like forms \cite{Fokker_1914, Planck_1917} wherein pseudo-particles in the numerical model obey the neoclassical transport equations, with particle-independent Brownian drift terms. This offers a rigorous methodology for incorporating collisions into the particle transport model, without coupling the equations of motions for each particle.
        
        Works by Chen, Chacón et al. \cite{Chen_Chacón_Barnes_2011, Chacón_Chen_Barnes_2013, Chen_Chacón_2014, Chen_Chacón_2015} have developed structure-preserving particle pushers for neoclassical transport in the Vlasov equations, derived from Crank--Nicolson integrators. We show these too can can derive from a FET interpretation, similarly offering potential extensions to higher-order-in-time particle pushers. The FET formulation is used also to consider how the stochastic drift terms can be incorporated into the pushers. Stochastic gyrokinetic expansions are also discussed.

        Different options for the numerical implementation of these schemes are considered.

        Due to the efficacy of FET in the development of SP timesteppers for both the fluid and kinetic component, we hope this approach will prove effective in the future for developing SP timesteppers for the full hybrid model. We hope this will give us the opportunity to incorporate previously inaccessible kinetic effects into the highly effective, modern, finite-element MHD models.
    \end{abstract}
    
    
    \newpage
    \tableofcontents
    
    
    \newpage
    \pagenumbering{arabic}
    %\linenumbers\renewcommand\thelinenumber{\color{black!50}\arabic{linenumber}}
            \input{0 - introduction/main.tex}
        \part{Research}
            \input{1 - low-noise PiC models/main.tex}
            \input{2 - kinetic component/main.tex}
            \input{3 - fluid component/main.tex}
            \input{4 - numerical implementation/main.tex}
        \part{Project Overview}
            \input{5 - research plan/main.tex}
            \input{6 - summary/main.tex}
    
    
    %\section{}
    \newpage
    \pagenumbering{gobble}
        \printbibliography


    \newpage
    \pagenumbering{roman}
    \appendix
        \part{Appendices}
            \input{8 - Hilbert complexes/main.tex}
            \input{9 - weak conservation proofs/main.tex}
\end{document}

            \documentclass[12pt, a4paper]{report}

\input{template/main.tex}

\title{\BA{Title in Progress...}}
\author{Boris Andrews}
\affil{Mathematical Institute, University of Oxford}
\date{\today}


\begin{document}
    \pagenumbering{gobble}
    \maketitle
    
    
    \begin{abstract}
        Magnetic confinement reactors---in particular tokamaks---offer one of the most promising options for achieving practical nuclear fusion, with the potential to provide virtually limitless, clean energy. The theoretical and numerical modeling of tokamak plasmas is simultaneously an essential component of effective reactor design, and a great research barrier. Tokamak operational conditions exhibit comparatively low Knudsen numbers. Kinetic effects, including kinetic waves and instabilities, Landau damping, bump-on-tail instabilities and more, are therefore highly influential in tokamak plasma dynamics. Purely fluid models are inherently incapable of capturing these effects, whereas the high dimensionality in purely kinetic models render them practically intractable for most relevant purposes.

        We consider a $\delta\!f$ decomposition model, with a macroscopic fluid background and microscopic kinetic correction, both fully coupled to each other. A similar manner of discretization is proposed to that used in the recent \texttt{STRUPHY} code \cite{Holderied_Possanner_Wang_2021, Holderied_2022, Li_et_al_2023} with a finite-element model for the background and a pseudo-particle/PiC model for the correction.

        The fluid background satisfies the full, non-linear, resistive, compressible, Hall MHD equations. \cite{Laakmann_Hu_Farrell_2022} introduces finite-element(-in-space) implicit timesteppers for the incompressible analogue to this system with structure-preserving (SP) properties in the ideal case, alongside parameter-robust preconditioners. We show that these timesteppers can derive from a finite-element-in-time (FET) (and finite-element-in-space) interpretation. The benefits of this reformulation are discussed, including the derivation of timesteppers that are higher order in time, and the quantifiable dissipative SP properties in the non-ideal, resistive case.
        
        We discuss possible options for extending this FET approach to timesteppers for the compressible case.

        The kinetic corrections satisfy linearized Boltzmann equations. Using a Lénard--Bernstein collision operator, these take Fokker--Planck-like forms \cite{Fokker_1914, Planck_1917} wherein pseudo-particles in the numerical model obey the neoclassical transport equations, with particle-independent Brownian drift terms. This offers a rigorous methodology for incorporating collisions into the particle transport model, without coupling the equations of motions for each particle.
        
        Works by Chen, Chacón et al. \cite{Chen_Chacón_Barnes_2011, Chacón_Chen_Barnes_2013, Chen_Chacón_2014, Chen_Chacón_2015} have developed structure-preserving particle pushers for neoclassical transport in the Vlasov equations, derived from Crank--Nicolson integrators. We show these too can can derive from a FET interpretation, similarly offering potential extensions to higher-order-in-time particle pushers. The FET formulation is used also to consider how the stochastic drift terms can be incorporated into the pushers. Stochastic gyrokinetic expansions are also discussed.

        Different options for the numerical implementation of these schemes are considered.

        Due to the efficacy of FET in the development of SP timesteppers for both the fluid and kinetic component, we hope this approach will prove effective in the future for developing SP timesteppers for the full hybrid model. We hope this will give us the opportunity to incorporate previously inaccessible kinetic effects into the highly effective, modern, finite-element MHD models.
    \end{abstract}
    
    
    \newpage
    \tableofcontents
    
    
    \newpage
    \pagenumbering{arabic}
    %\linenumbers\renewcommand\thelinenumber{\color{black!50}\arabic{linenumber}}
            \input{0 - introduction/main.tex}
        \part{Research}
            \input{1 - low-noise PiC models/main.tex}
            \input{2 - kinetic component/main.tex}
            \input{3 - fluid component/main.tex}
            \input{4 - numerical implementation/main.tex}
        \part{Project Overview}
            \input{5 - research plan/main.tex}
            \input{6 - summary/main.tex}
    
    
    %\section{}
    \newpage
    \pagenumbering{gobble}
        \printbibliography


    \newpage
    \pagenumbering{roman}
    \appendix
        \part{Appendices}
            \input{8 - Hilbert complexes/main.tex}
            \input{9 - weak conservation proofs/main.tex}
\end{document}

    
    
    %\section{}
    \newpage
    \pagenumbering{gobble}
        \printbibliography


    \newpage
    \pagenumbering{roman}
    \appendix
        \part{Appendices}
            \documentclass[12pt, a4paper]{report}

\input{template/main.tex}

\title{\BA{Title in Progress...}}
\author{Boris Andrews}
\affil{Mathematical Institute, University of Oxford}
\date{\today}


\begin{document}
    \pagenumbering{gobble}
    \maketitle
    
    
    \begin{abstract}
        Magnetic confinement reactors---in particular tokamaks---offer one of the most promising options for achieving practical nuclear fusion, with the potential to provide virtually limitless, clean energy. The theoretical and numerical modeling of tokamak plasmas is simultaneously an essential component of effective reactor design, and a great research barrier. Tokamak operational conditions exhibit comparatively low Knudsen numbers. Kinetic effects, including kinetic waves and instabilities, Landau damping, bump-on-tail instabilities and more, are therefore highly influential in tokamak plasma dynamics. Purely fluid models are inherently incapable of capturing these effects, whereas the high dimensionality in purely kinetic models render them practically intractable for most relevant purposes.

        We consider a $\delta\!f$ decomposition model, with a macroscopic fluid background and microscopic kinetic correction, both fully coupled to each other. A similar manner of discretization is proposed to that used in the recent \texttt{STRUPHY} code \cite{Holderied_Possanner_Wang_2021, Holderied_2022, Li_et_al_2023} with a finite-element model for the background and a pseudo-particle/PiC model for the correction.

        The fluid background satisfies the full, non-linear, resistive, compressible, Hall MHD equations. \cite{Laakmann_Hu_Farrell_2022} introduces finite-element(-in-space) implicit timesteppers for the incompressible analogue to this system with structure-preserving (SP) properties in the ideal case, alongside parameter-robust preconditioners. We show that these timesteppers can derive from a finite-element-in-time (FET) (and finite-element-in-space) interpretation. The benefits of this reformulation are discussed, including the derivation of timesteppers that are higher order in time, and the quantifiable dissipative SP properties in the non-ideal, resistive case.
        
        We discuss possible options for extending this FET approach to timesteppers for the compressible case.

        The kinetic corrections satisfy linearized Boltzmann equations. Using a Lénard--Bernstein collision operator, these take Fokker--Planck-like forms \cite{Fokker_1914, Planck_1917} wherein pseudo-particles in the numerical model obey the neoclassical transport equations, with particle-independent Brownian drift terms. This offers a rigorous methodology for incorporating collisions into the particle transport model, without coupling the equations of motions for each particle.
        
        Works by Chen, Chacón et al. \cite{Chen_Chacón_Barnes_2011, Chacón_Chen_Barnes_2013, Chen_Chacón_2014, Chen_Chacón_2015} have developed structure-preserving particle pushers for neoclassical transport in the Vlasov equations, derived from Crank--Nicolson integrators. We show these too can can derive from a FET interpretation, similarly offering potential extensions to higher-order-in-time particle pushers. The FET formulation is used also to consider how the stochastic drift terms can be incorporated into the pushers. Stochastic gyrokinetic expansions are also discussed.

        Different options for the numerical implementation of these schemes are considered.

        Due to the efficacy of FET in the development of SP timesteppers for both the fluid and kinetic component, we hope this approach will prove effective in the future for developing SP timesteppers for the full hybrid model. We hope this will give us the opportunity to incorporate previously inaccessible kinetic effects into the highly effective, modern, finite-element MHD models.
    \end{abstract}
    
    
    \newpage
    \tableofcontents
    
    
    \newpage
    \pagenumbering{arabic}
    %\linenumbers\renewcommand\thelinenumber{\color{black!50}\arabic{linenumber}}
            \input{0 - introduction/main.tex}
        \part{Research}
            \input{1 - low-noise PiC models/main.tex}
            \input{2 - kinetic component/main.tex}
            \input{3 - fluid component/main.tex}
            \input{4 - numerical implementation/main.tex}
        \part{Project Overview}
            \input{5 - research plan/main.tex}
            \input{6 - summary/main.tex}
    
    
    %\section{}
    \newpage
    \pagenumbering{gobble}
        \printbibliography


    \newpage
    \pagenumbering{roman}
    \appendix
        \part{Appendices}
            \input{8 - Hilbert complexes/main.tex}
            \input{9 - weak conservation proofs/main.tex}
\end{document}

            \documentclass[12pt, a4paper]{report}

\input{template/main.tex}

\title{\BA{Title in Progress...}}
\author{Boris Andrews}
\affil{Mathematical Institute, University of Oxford}
\date{\today}


\begin{document}
    \pagenumbering{gobble}
    \maketitle
    
    
    \begin{abstract}
        Magnetic confinement reactors---in particular tokamaks---offer one of the most promising options for achieving practical nuclear fusion, with the potential to provide virtually limitless, clean energy. The theoretical and numerical modeling of tokamak plasmas is simultaneously an essential component of effective reactor design, and a great research barrier. Tokamak operational conditions exhibit comparatively low Knudsen numbers. Kinetic effects, including kinetic waves and instabilities, Landau damping, bump-on-tail instabilities and more, are therefore highly influential in tokamak plasma dynamics. Purely fluid models are inherently incapable of capturing these effects, whereas the high dimensionality in purely kinetic models render them practically intractable for most relevant purposes.

        We consider a $\delta\!f$ decomposition model, with a macroscopic fluid background and microscopic kinetic correction, both fully coupled to each other. A similar manner of discretization is proposed to that used in the recent \texttt{STRUPHY} code \cite{Holderied_Possanner_Wang_2021, Holderied_2022, Li_et_al_2023} with a finite-element model for the background and a pseudo-particle/PiC model for the correction.

        The fluid background satisfies the full, non-linear, resistive, compressible, Hall MHD equations. \cite{Laakmann_Hu_Farrell_2022} introduces finite-element(-in-space) implicit timesteppers for the incompressible analogue to this system with structure-preserving (SP) properties in the ideal case, alongside parameter-robust preconditioners. We show that these timesteppers can derive from a finite-element-in-time (FET) (and finite-element-in-space) interpretation. The benefits of this reformulation are discussed, including the derivation of timesteppers that are higher order in time, and the quantifiable dissipative SP properties in the non-ideal, resistive case.
        
        We discuss possible options for extending this FET approach to timesteppers for the compressible case.

        The kinetic corrections satisfy linearized Boltzmann equations. Using a Lénard--Bernstein collision operator, these take Fokker--Planck-like forms \cite{Fokker_1914, Planck_1917} wherein pseudo-particles in the numerical model obey the neoclassical transport equations, with particle-independent Brownian drift terms. This offers a rigorous methodology for incorporating collisions into the particle transport model, without coupling the equations of motions for each particle.
        
        Works by Chen, Chacón et al. \cite{Chen_Chacón_Barnes_2011, Chacón_Chen_Barnes_2013, Chen_Chacón_2014, Chen_Chacón_2015} have developed structure-preserving particle pushers for neoclassical transport in the Vlasov equations, derived from Crank--Nicolson integrators. We show these too can can derive from a FET interpretation, similarly offering potential extensions to higher-order-in-time particle pushers. The FET formulation is used also to consider how the stochastic drift terms can be incorporated into the pushers. Stochastic gyrokinetic expansions are also discussed.

        Different options for the numerical implementation of these schemes are considered.

        Due to the efficacy of FET in the development of SP timesteppers for both the fluid and kinetic component, we hope this approach will prove effective in the future for developing SP timesteppers for the full hybrid model. We hope this will give us the opportunity to incorporate previously inaccessible kinetic effects into the highly effective, modern, finite-element MHD models.
    \end{abstract}
    
    
    \newpage
    \tableofcontents
    
    
    \newpage
    \pagenumbering{arabic}
    %\linenumbers\renewcommand\thelinenumber{\color{black!50}\arabic{linenumber}}
            \input{0 - introduction/main.tex}
        \part{Research}
            \input{1 - low-noise PiC models/main.tex}
            \input{2 - kinetic component/main.tex}
            \input{3 - fluid component/main.tex}
            \input{4 - numerical implementation/main.tex}
        \part{Project Overview}
            \input{5 - research plan/main.tex}
            \input{6 - summary/main.tex}
    
    
    %\section{}
    \newpage
    \pagenumbering{gobble}
        \printbibliography


    \newpage
    \pagenumbering{roman}
    \appendix
        \part{Appendices}
            \input{8 - Hilbert complexes/main.tex}
            \input{9 - weak conservation proofs/main.tex}
\end{document}

\end{document}

        \part{Project Overview}
            \documentclass[12pt, a4paper]{report}

\documentclass[12pt, a4paper]{report}

\input{template/main.tex}

\title{\BA{Title in Progress...}}
\author{Boris Andrews}
\affil{Mathematical Institute, University of Oxford}
\date{\today}


\begin{document}
    \pagenumbering{gobble}
    \maketitle
    
    
    \begin{abstract}
        Magnetic confinement reactors---in particular tokamaks---offer one of the most promising options for achieving practical nuclear fusion, with the potential to provide virtually limitless, clean energy. The theoretical and numerical modeling of tokamak plasmas is simultaneously an essential component of effective reactor design, and a great research barrier. Tokamak operational conditions exhibit comparatively low Knudsen numbers. Kinetic effects, including kinetic waves and instabilities, Landau damping, bump-on-tail instabilities and more, are therefore highly influential in tokamak plasma dynamics. Purely fluid models are inherently incapable of capturing these effects, whereas the high dimensionality in purely kinetic models render them practically intractable for most relevant purposes.

        We consider a $\delta\!f$ decomposition model, with a macroscopic fluid background and microscopic kinetic correction, both fully coupled to each other. A similar manner of discretization is proposed to that used in the recent \texttt{STRUPHY} code \cite{Holderied_Possanner_Wang_2021, Holderied_2022, Li_et_al_2023} with a finite-element model for the background and a pseudo-particle/PiC model for the correction.

        The fluid background satisfies the full, non-linear, resistive, compressible, Hall MHD equations. \cite{Laakmann_Hu_Farrell_2022} introduces finite-element(-in-space) implicit timesteppers for the incompressible analogue to this system with structure-preserving (SP) properties in the ideal case, alongside parameter-robust preconditioners. We show that these timesteppers can derive from a finite-element-in-time (FET) (and finite-element-in-space) interpretation. The benefits of this reformulation are discussed, including the derivation of timesteppers that are higher order in time, and the quantifiable dissipative SP properties in the non-ideal, resistive case.
        
        We discuss possible options for extending this FET approach to timesteppers for the compressible case.

        The kinetic corrections satisfy linearized Boltzmann equations. Using a Lénard--Bernstein collision operator, these take Fokker--Planck-like forms \cite{Fokker_1914, Planck_1917} wherein pseudo-particles in the numerical model obey the neoclassical transport equations, with particle-independent Brownian drift terms. This offers a rigorous methodology for incorporating collisions into the particle transport model, without coupling the equations of motions for each particle.
        
        Works by Chen, Chacón et al. \cite{Chen_Chacón_Barnes_2011, Chacón_Chen_Barnes_2013, Chen_Chacón_2014, Chen_Chacón_2015} have developed structure-preserving particle pushers for neoclassical transport in the Vlasov equations, derived from Crank--Nicolson integrators. We show these too can can derive from a FET interpretation, similarly offering potential extensions to higher-order-in-time particle pushers. The FET formulation is used also to consider how the stochastic drift terms can be incorporated into the pushers. Stochastic gyrokinetic expansions are also discussed.

        Different options for the numerical implementation of these schemes are considered.

        Due to the efficacy of FET in the development of SP timesteppers for both the fluid and kinetic component, we hope this approach will prove effective in the future for developing SP timesteppers for the full hybrid model. We hope this will give us the opportunity to incorporate previously inaccessible kinetic effects into the highly effective, modern, finite-element MHD models.
    \end{abstract}
    
    
    \newpage
    \tableofcontents
    
    
    \newpage
    \pagenumbering{arabic}
    %\linenumbers\renewcommand\thelinenumber{\color{black!50}\arabic{linenumber}}
            \input{0 - introduction/main.tex}
        \part{Research}
            \input{1 - low-noise PiC models/main.tex}
            \input{2 - kinetic component/main.tex}
            \input{3 - fluid component/main.tex}
            \input{4 - numerical implementation/main.tex}
        \part{Project Overview}
            \input{5 - research plan/main.tex}
            \input{6 - summary/main.tex}
    
    
    %\section{}
    \newpage
    \pagenumbering{gobble}
        \printbibliography


    \newpage
    \pagenumbering{roman}
    \appendix
        \part{Appendices}
            \input{8 - Hilbert complexes/main.tex}
            \input{9 - weak conservation proofs/main.tex}
\end{document}


\title{\BA{Title in Progress...}}
\author{Boris Andrews}
\affil{Mathematical Institute, University of Oxford}
\date{\today}


\begin{document}
    \pagenumbering{gobble}
    \maketitle
    
    
    \begin{abstract}
        Magnetic confinement reactors---in particular tokamaks---offer one of the most promising options for achieving practical nuclear fusion, with the potential to provide virtually limitless, clean energy. The theoretical and numerical modeling of tokamak plasmas is simultaneously an essential component of effective reactor design, and a great research barrier. Tokamak operational conditions exhibit comparatively low Knudsen numbers. Kinetic effects, including kinetic waves and instabilities, Landau damping, bump-on-tail instabilities and more, are therefore highly influential in tokamak plasma dynamics. Purely fluid models are inherently incapable of capturing these effects, whereas the high dimensionality in purely kinetic models render them practically intractable for most relevant purposes.

        We consider a $\delta\!f$ decomposition model, with a macroscopic fluid background and microscopic kinetic correction, both fully coupled to each other. A similar manner of discretization is proposed to that used in the recent \texttt{STRUPHY} code \cite{Holderied_Possanner_Wang_2021, Holderied_2022, Li_et_al_2023} with a finite-element model for the background and a pseudo-particle/PiC model for the correction.

        The fluid background satisfies the full, non-linear, resistive, compressible, Hall MHD equations. \cite{Laakmann_Hu_Farrell_2022} introduces finite-element(-in-space) implicit timesteppers for the incompressible analogue to this system with structure-preserving (SP) properties in the ideal case, alongside parameter-robust preconditioners. We show that these timesteppers can derive from a finite-element-in-time (FET) (and finite-element-in-space) interpretation. The benefits of this reformulation are discussed, including the derivation of timesteppers that are higher order in time, and the quantifiable dissipative SP properties in the non-ideal, resistive case.
        
        We discuss possible options for extending this FET approach to timesteppers for the compressible case.

        The kinetic corrections satisfy linearized Boltzmann equations. Using a Lénard--Bernstein collision operator, these take Fokker--Planck-like forms \cite{Fokker_1914, Planck_1917} wherein pseudo-particles in the numerical model obey the neoclassical transport equations, with particle-independent Brownian drift terms. This offers a rigorous methodology for incorporating collisions into the particle transport model, without coupling the equations of motions for each particle.
        
        Works by Chen, Chacón et al. \cite{Chen_Chacón_Barnes_2011, Chacón_Chen_Barnes_2013, Chen_Chacón_2014, Chen_Chacón_2015} have developed structure-preserving particle pushers for neoclassical transport in the Vlasov equations, derived from Crank--Nicolson integrators. We show these too can can derive from a FET interpretation, similarly offering potential extensions to higher-order-in-time particle pushers. The FET formulation is used also to consider how the stochastic drift terms can be incorporated into the pushers. Stochastic gyrokinetic expansions are also discussed.

        Different options for the numerical implementation of these schemes are considered.

        Due to the efficacy of FET in the development of SP timesteppers for both the fluid and kinetic component, we hope this approach will prove effective in the future for developing SP timesteppers for the full hybrid model. We hope this will give us the opportunity to incorporate previously inaccessible kinetic effects into the highly effective, modern, finite-element MHD models.
    \end{abstract}
    
    
    \newpage
    \tableofcontents
    
    
    \newpage
    \pagenumbering{arabic}
    %\linenumbers\renewcommand\thelinenumber{\color{black!50}\arabic{linenumber}}
            \documentclass[12pt, a4paper]{report}

\input{template/main.tex}

\title{\BA{Title in Progress...}}
\author{Boris Andrews}
\affil{Mathematical Institute, University of Oxford}
\date{\today}


\begin{document}
    \pagenumbering{gobble}
    \maketitle
    
    
    \begin{abstract}
        Magnetic confinement reactors---in particular tokamaks---offer one of the most promising options for achieving practical nuclear fusion, with the potential to provide virtually limitless, clean energy. The theoretical and numerical modeling of tokamak plasmas is simultaneously an essential component of effective reactor design, and a great research barrier. Tokamak operational conditions exhibit comparatively low Knudsen numbers. Kinetic effects, including kinetic waves and instabilities, Landau damping, bump-on-tail instabilities and more, are therefore highly influential in tokamak plasma dynamics. Purely fluid models are inherently incapable of capturing these effects, whereas the high dimensionality in purely kinetic models render them practically intractable for most relevant purposes.

        We consider a $\delta\!f$ decomposition model, with a macroscopic fluid background and microscopic kinetic correction, both fully coupled to each other. A similar manner of discretization is proposed to that used in the recent \texttt{STRUPHY} code \cite{Holderied_Possanner_Wang_2021, Holderied_2022, Li_et_al_2023} with a finite-element model for the background and a pseudo-particle/PiC model for the correction.

        The fluid background satisfies the full, non-linear, resistive, compressible, Hall MHD equations. \cite{Laakmann_Hu_Farrell_2022} introduces finite-element(-in-space) implicit timesteppers for the incompressible analogue to this system with structure-preserving (SP) properties in the ideal case, alongside parameter-robust preconditioners. We show that these timesteppers can derive from a finite-element-in-time (FET) (and finite-element-in-space) interpretation. The benefits of this reformulation are discussed, including the derivation of timesteppers that are higher order in time, and the quantifiable dissipative SP properties in the non-ideal, resistive case.
        
        We discuss possible options for extending this FET approach to timesteppers for the compressible case.

        The kinetic corrections satisfy linearized Boltzmann equations. Using a Lénard--Bernstein collision operator, these take Fokker--Planck-like forms \cite{Fokker_1914, Planck_1917} wherein pseudo-particles in the numerical model obey the neoclassical transport equations, with particle-independent Brownian drift terms. This offers a rigorous methodology for incorporating collisions into the particle transport model, without coupling the equations of motions for each particle.
        
        Works by Chen, Chacón et al. \cite{Chen_Chacón_Barnes_2011, Chacón_Chen_Barnes_2013, Chen_Chacón_2014, Chen_Chacón_2015} have developed structure-preserving particle pushers for neoclassical transport in the Vlasov equations, derived from Crank--Nicolson integrators. We show these too can can derive from a FET interpretation, similarly offering potential extensions to higher-order-in-time particle pushers. The FET formulation is used also to consider how the stochastic drift terms can be incorporated into the pushers. Stochastic gyrokinetic expansions are also discussed.

        Different options for the numerical implementation of these schemes are considered.

        Due to the efficacy of FET in the development of SP timesteppers for both the fluid and kinetic component, we hope this approach will prove effective in the future for developing SP timesteppers for the full hybrid model. We hope this will give us the opportunity to incorporate previously inaccessible kinetic effects into the highly effective, modern, finite-element MHD models.
    \end{abstract}
    
    
    \newpage
    \tableofcontents
    
    
    \newpage
    \pagenumbering{arabic}
    %\linenumbers\renewcommand\thelinenumber{\color{black!50}\arabic{linenumber}}
            \input{0 - introduction/main.tex}
        \part{Research}
            \input{1 - low-noise PiC models/main.tex}
            \input{2 - kinetic component/main.tex}
            \input{3 - fluid component/main.tex}
            \input{4 - numerical implementation/main.tex}
        \part{Project Overview}
            \input{5 - research plan/main.tex}
            \input{6 - summary/main.tex}
    
    
    %\section{}
    \newpage
    \pagenumbering{gobble}
        \printbibliography


    \newpage
    \pagenumbering{roman}
    \appendix
        \part{Appendices}
            \input{8 - Hilbert complexes/main.tex}
            \input{9 - weak conservation proofs/main.tex}
\end{document}

        \part{Research}
            \documentclass[12pt, a4paper]{report}

\input{template/main.tex}

\title{\BA{Title in Progress...}}
\author{Boris Andrews}
\affil{Mathematical Institute, University of Oxford}
\date{\today}


\begin{document}
    \pagenumbering{gobble}
    \maketitle
    
    
    \begin{abstract}
        Magnetic confinement reactors---in particular tokamaks---offer one of the most promising options for achieving practical nuclear fusion, with the potential to provide virtually limitless, clean energy. The theoretical and numerical modeling of tokamak plasmas is simultaneously an essential component of effective reactor design, and a great research barrier. Tokamak operational conditions exhibit comparatively low Knudsen numbers. Kinetic effects, including kinetic waves and instabilities, Landau damping, bump-on-tail instabilities and more, are therefore highly influential in tokamak plasma dynamics. Purely fluid models are inherently incapable of capturing these effects, whereas the high dimensionality in purely kinetic models render them practically intractable for most relevant purposes.

        We consider a $\delta\!f$ decomposition model, with a macroscopic fluid background and microscopic kinetic correction, both fully coupled to each other. A similar manner of discretization is proposed to that used in the recent \texttt{STRUPHY} code \cite{Holderied_Possanner_Wang_2021, Holderied_2022, Li_et_al_2023} with a finite-element model for the background and a pseudo-particle/PiC model for the correction.

        The fluid background satisfies the full, non-linear, resistive, compressible, Hall MHD equations. \cite{Laakmann_Hu_Farrell_2022} introduces finite-element(-in-space) implicit timesteppers for the incompressible analogue to this system with structure-preserving (SP) properties in the ideal case, alongside parameter-robust preconditioners. We show that these timesteppers can derive from a finite-element-in-time (FET) (and finite-element-in-space) interpretation. The benefits of this reformulation are discussed, including the derivation of timesteppers that are higher order in time, and the quantifiable dissipative SP properties in the non-ideal, resistive case.
        
        We discuss possible options for extending this FET approach to timesteppers for the compressible case.

        The kinetic corrections satisfy linearized Boltzmann equations. Using a Lénard--Bernstein collision operator, these take Fokker--Planck-like forms \cite{Fokker_1914, Planck_1917} wherein pseudo-particles in the numerical model obey the neoclassical transport equations, with particle-independent Brownian drift terms. This offers a rigorous methodology for incorporating collisions into the particle transport model, without coupling the equations of motions for each particle.
        
        Works by Chen, Chacón et al. \cite{Chen_Chacón_Barnes_2011, Chacón_Chen_Barnes_2013, Chen_Chacón_2014, Chen_Chacón_2015} have developed structure-preserving particle pushers for neoclassical transport in the Vlasov equations, derived from Crank--Nicolson integrators. We show these too can can derive from a FET interpretation, similarly offering potential extensions to higher-order-in-time particle pushers. The FET formulation is used also to consider how the stochastic drift terms can be incorporated into the pushers. Stochastic gyrokinetic expansions are also discussed.

        Different options for the numerical implementation of these schemes are considered.

        Due to the efficacy of FET in the development of SP timesteppers for both the fluid and kinetic component, we hope this approach will prove effective in the future for developing SP timesteppers for the full hybrid model. We hope this will give us the opportunity to incorporate previously inaccessible kinetic effects into the highly effective, modern, finite-element MHD models.
    \end{abstract}
    
    
    \newpage
    \tableofcontents
    
    
    \newpage
    \pagenumbering{arabic}
    %\linenumbers\renewcommand\thelinenumber{\color{black!50}\arabic{linenumber}}
            \input{0 - introduction/main.tex}
        \part{Research}
            \input{1 - low-noise PiC models/main.tex}
            \input{2 - kinetic component/main.tex}
            \input{3 - fluid component/main.tex}
            \input{4 - numerical implementation/main.tex}
        \part{Project Overview}
            \input{5 - research plan/main.tex}
            \input{6 - summary/main.tex}
    
    
    %\section{}
    \newpage
    \pagenumbering{gobble}
        \printbibliography


    \newpage
    \pagenumbering{roman}
    \appendix
        \part{Appendices}
            \input{8 - Hilbert complexes/main.tex}
            \input{9 - weak conservation proofs/main.tex}
\end{document}

            \documentclass[12pt, a4paper]{report}

\input{template/main.tex}

\title{\BA{Title in Progress...}}
\author{Boris Andrews}
\affil{Mathematical Institute, University of Oxford}
\date{\today}


\begin{document}
    \pagenumbering{gobble}
    \maketitle
    
    
    \begin{abstract}
        Magnetic confinement reactors---in particular tokamaks---offer one of the most promising options for achieving practical nuclear fusion, with the potential to provide virtually limitless, clean energy. The theoretical and numerical modeling of tokamak plasmas is simultaneously an essential component of effective reactor design, and a great research barrier. Tokamak operational conditions exhibit comparatively low Knudsen numbers. Kinetic effects, including kinetic waves and instabilities, Landau damping, bump-on-tail instabilities and more, are therefore highly influential in tokamak plasma dynamics. Purely fluid models are inherently incapable of capturing these effects, whereas the high dimensionality in purely kinetic models render them practically intractable for most relevant purposes.

        We consider a $\delta\!f$ decomposition model, with a macroscopic fluid background and microscopic kinetic correction, both fully coupled to each other. A similar manner of discretization is proposed to that used in the recent \texttt{STRUPHY} code \cite{Holderied_Possanner_Wang_2021, Holderied_2022, Li_et_al_2023} with a finite-element model for the background and a pseudo-particle/PiC model for the correction.

        The fluid background satisfies the full, non-linear, resistive, compressible, Hall MHD equations. \cite{Laakmann_Hu_Farrell_2022} introduces finite-element(-in-space) implicit timesteppers for the incompressible analogue to this system with structure-preserving (SP) properties in the ideal case, alongside parameter-robust preconditioners. We show that these timesteppers can derive from a finite-element-in-time (FET) (and finite-element-in-space) interpretation. The benefits of this reformulation are discussed, including the derivation of timesteppers that are higher order in time, and the quantifiable dissipative SP properties in the non-ideal, resistive case.
        
        We discuss possible options for extending this FET approach to timesteppers for the compressible case.

        The kinetic corrections satisfy linearized Boltzmann equations. Using a Lénard--Bernstein collision operator, these take Fokker--Planck-like forms \cite{Fokker_1914, Planck_1917} wherein pseudo-particles in the numerical model obey the neoclassical transport equations, with particle-independent Brownian drift terms. This offers a rigorous methodology for incorporating collisions into the particle transport model, without coupling the equations of motions for each particle.
        
        Works by Chen, Chacón et al. \cite{Chen_Chacón_Barnes_2011, Chacón_Chen_Barnes_2013, Chen_Chacón_2014, Chen_Chacón_2015} have developed structure-preserving particle pushers for neoclassical transport in the Vlasov equations, derived from Crank--Nicolson integrators. We show these too can can derive from a FET interpretation, similarly offering potential extensions to higher-order-in-time particle pushers. The FET formulation is used also to consider how the stochastic drift terms can be incorporated into the pushers. Stochastic gyrokinetic expansions are also discussed.

        Different options for the numerical implementation of these schemes are considered.

        Due to the efficacy of FET in the development of SP timesteppers for both the fluid and kinetic component, we hope this approach will prove effective in the future for developing SP timesteppers for the full hybrid model. We hope this will give us the opportunity to incorporate previously inaccessible kinetic effects into the highly effective, modern, finite-element MHD models.
    \end{abstract}
    
    
    \newpage
    \tableofcontents
    
    
    \newpage
    \pagenumbering{arabic}
    %\linenumbers\renewcommand\thelinenumber{\color{black!50}\arabic{linenumber}}
            \input{0 - introduction/main.tex}
        \part{Research}
            \input{1 - low-noise PiC models/main.tex}
            \input{2 - kinetic component/main.tex}
            \input{3 - fluid component/main.tex}
            \input{4 - numerical implementation/main.tex}
        \part{Project Overview}
            \input{5 - research plan/main.tex}
            \input{6 - summary/main.tex}
    
    
    %\section{}
    \newpage
    \pagenumbering{gobble}
        \printbibliography


    \newpage
    \pagenumbering{roman}
    \appendix
        \part{Appendices}
            \input{8 - Hilbert complexes/main.tex}
            \input{9 - weak conservation proofs/main.tex}
\end{document}

            \documentclass[12pt, a4paper]{report}

\input{template/main.tex}

\title{\BA{Title in Progress...}}
\author{Boris Andrews}
\affil{Mathematical Institute, University of Oxford}
\date{\today}


\begin{document}
    \pagenumbering{gobble}
    \maketitle
    
    
    \begin{abstract}
        Magnetic confinement reactors---in particular tokamaks---offer one of the most promising options for achieving practical nuclear fusion, with the potential to provide virtually limitless, clean energy. The theoretical and numerical modeling of tokamak plasmas is simultaneously an essential component of effective reactor design, and a great research barrier. Tokamak operational conditions exhibit comparatively low Knudsen numbers. Kinetic effects, including kinetic waves and instabilities, Landau damping, bump-on-tail instabilities and more, are therefore highly influential in tokamak plasma dynamics. Purely fluid models are inherently incapable of capturing these effects, whereas the high dimensionality in purely kinetic models render them practically intractable for most relevant purposes.

        We consider a $\delta\!f$ decomposition model, with a macroscopic fluid background and microscopic kinetic correction, both fully coupled to each other. A similar manner of discretization is proposed to that used in the recent \texttt{STRUPHY} code \cite{Holderied_Possanner_Wang_2021, Holderied_2022, Li_et_al_2023} with a finite-element model for the background and a pseudo-particle/PiC model for the correction.

        The fluid background satisfies the full, non-linear, resistive, compressible, Hall MHD equations. \cite{Laakmann_Hu_Farrell_2022} introduces finite-element(-in-space) implicit timesteppers for the incompressible analogue to this system with structure-preserving (SP) properties in the ideal case, alongside parameter-robust preconditioners. We show that these timesteppers can derive from a finite-element-in-time (FET) (and finite-element-in-space) interpretation. The benefits of this reformulation are discussed, including the derivation of timesteppers that are higher order in time, and the quantifiable dissipative SP properties in the non-ideal, resistive case.
        
        We discuss possible options for extending this FET approach to timesteppers for the compressible case.

        The kinetic corrections satisfy linearized Boltzmann equations. Using a Lénard--Bernstein collision operator, these take Fokker--Planck-like forms \cite{Fokker_1914, Planck_1917} wherein pseudo-particles in the numerical model obey the neoclassical transport equations, with particle-independent Brownian drift terms. This offers a rigorous methodology for incorporating collisions into the particle transport model, without coupling the equations of motions for each particle.
        
        Works by Chen, Chacón et al. \cite{Chen_Chacón_Barnes_2011, Chacón_Chen_Barnes_2013, Chen_Chacón_2014, Chen_Chacón_2015} have developed structure-preserving particle pushers for neoclassical transport in the Vlasov equations, derived from Crank--Nicolson integrators. We show these too can can derive from a FET interpretation, similarly offering potential extensions to higher-order-in-time particle pushers. The FET formulation is used also to consider how the stochastic drift terms can be incorporated into the pushers. Stochastic gyrokinetic expansions are also discussed.

        Different options for the numerical implementation of these schemes are considered.

        Due to the efficacy of FET in the development of SP timesteppers for both the fluid and kinetic component, we hope this approach will prove effective in the future for developing SP timesteppers for the full hybrid model. We hope this will give us the opportunity to incorporate previously inaccessible kinetic effects into the highly effective, modern, finite-element MHD models.
    \end{abstract}
    
    
    \newpage
    \tableofcontents
    
    
    \newpage
    \pagenumbering{arabic}
    %\linenumbers\renewcommand\thelinenumber{\color{black!50}\arabic{linenumber}}
            \input{0 - introduction/main.tex}
        \part{Research}
            \input{1 - low-noise PiC models/main.tex}
            \input{2 - kinetic component/main.tex}
            \input{3 - fluid component/main.tex}
            \input{4 - numerical implementation/main.tex}
        \part{Project Overview}
            \input{5 - research plan/main.tex}
            \input{6 - summary/main.tex}
    
    
    %\section{}
    \newpage
    \pagenumbering{gobble}
        \printbibliography


    \newpage
    \pagenumbering{roman}
    \appendix
        \part{Appendices}
            \input{8 - Hilbert complexes/main.tex}
            \input{9 - weak conservation proofs/main.tex}
\end{document}

            \documentclass[12pt, a4paper]{report}

\input{template/main.tex}

\title{\BA{Title in Progress...}}
\author{Boris Andrews}
\affil{Mathematical Institute, University of Oxford}
\date{\today}


\begin{document}
    \pagenumbering{gobble}
    \maketitle
    
    
    \begin{abstract}
        Magnetic confinement reactors---in particular tokamaks---offer one of the most promising options for achieving practical nuclear fusion, with the potential to provide virtually limitless, clean energy. The theoretical and numerical modeling of tokamak plasmas is simultaneously an essential component of effective reactor design, and a great research barrier. Tokamak operational conditions exhibit comparatively low Knudsen numbers. Kinetic effects, including kinetic waves and instabilities, Landau damping, bump-on-tail instabilities and more, are therefore highly influential in tokamak plasma dynamics. Purely fluid models are inherently incapable of capturing these effects, whereas the high dimensionality in purely kinetic models render them practically intractable for most relevant purposes.

        We consider a $\delta\!f$ decomposition model, with a macroscopic fluid background and microscopic kinetic correction, both fully coupled to each other. A similar manner of discretization is proposed to that used in the recent \texttt{STRUPHY} code \cite{Holderied_Possanner_Wang_2021, Holderied_2022, Li_et_al_2023} with a finite-element model for the background and a pseudo-particle/PiC model for the correction.

        The fluid background satisfies the full, non-linear, resistive, compressible, Hall MHD equations. \cite{Laakmann_Hu_Farrell_2022} introduces finite-element(-in-space) implicit timesteppers for the incompressible analogue to this system with structure-preserving (SP) properties in the ideal case, alongside parameter-robust preconditioners. We show that these timesteppers can derive from a finite-element-in-time (FET) (and finite-element-in-space) interpretation. The benefits of this reformulation are discussed, including the derivation of timesteppers that are higher order in time, and the quantifiable dissipative SP properties in the non-ideal, resistive case.
        
        We discuss possible options for extending this FET approach to timesteppers for the compressible case.

        The kinetic corrections satisfy linearized Boltzmann equations. Using a Lénard--Bernstein collision operator, these take Fokker--Planck-like forms \cite{Fokker_1914, Planck_1917} wherein pseudo-particles in the numerical model obey the neoclassical transport equations, with particle-independent Brownian drift terms. This offers a rigorous methodology for incorporating collisions into the particle transport model, without coupling the equations of motions for each particle.
        
        Works by Chen, Chacón et al. \cite{Chen_Chacón_Barnes_2011, Chacón_Chen_Barnes_2013, Chen_Chacón_2014, Chen_Chacón_2015} have developed structure-preserving particle pushers for neoclassical transport in the Vlasov equations, derived from Crank--Nicolson integrators. We show these too can can derive from a FET interpretation, similarly offering potential extensions to higher-order-in-time particle pushers. The FET formulation is used also to consider how the stochastic drift terms can be incorporated into the pushers. Stochastic gyrokinetic expansions are also discussed.

        Different options for the numerical implementation of these schemes are considered.

        Due to the efficacy of FET in the development of SP timesteppers for both the fluid and kinetic component, we hope this approach will prove effective in the future for developing SP timesteppers for the full hybrid model. We hope this will give us the opportunity to incorporate previously inaccessible kinetic effects into the highly effective, modern, finite-element MHD models.
    \end{abstract}
    
    
    \newpage
    \tableofcontents
    
    
    \newpage
    \pagenumbering{arabic}
    %\linenumbers\renewcommand\thelinenumber{\color{black!50}\arabic{linenumber}}
            \input{0 - introduction/main.tex}
        \part{Research}
            \input{1 - low-noise PiC models/main.tex}
            \input{2 - kinetic component/main.tex}
            \input{3 - fluid component/main.tex}
            \input{4 - numerical implementation/main.tex}
        \part{Project Overview}
            \input{5 - research plan/main.tex}
            \input{6 - summary/main.tex}
    
    
    %\section{}
    \newpage
    \pagenumbering{gobble}
        \printbibliography


    \newpage
    \pagenumbering{roman}
    \appendix
        \part{Appendices}
            \input{8 - Hilbert complexes/main.tex}
            \input{9 - weak conservation proofs/main.tex}
\end{document}

        \part{Project Overview}
            \documentclass[12pt, a4paper]{report}

\input{template/main.tex}

\title{\BA{Title in Progress...}}
\author{Boris Andrews}
\affil{Mathematical Institute, University of Oxford}
\date{\today}


\begin{document}
    \pagenumbering{gobble}
    \maketitle
    
    
    \begin{abstract}
        Magnetic confinement reactors---in particular tokamaks---offer one of the most promising options for achieving practical nuclear fusion, with the potential to provide virtually limitless, clean energy. The theoretical and numerical modeling of tokamak plasmas is simultaneously an essential component of effective reactor design, and a great research barrier. Tokamak operational conditions exhibit comparatively low Knudsen numbers. Kinetic effects, including kinetic waves and instabilities, Landau damping, bump-on-tail instabilities and more, are therefore highly influential in tokamak plasma dynamics. Purely fluid models are inherently incapable of capturing these effects, whereas the high dimensionality in purely kinetic models render them practically intractable for most relevant purposes.

        We consider a $\delta\!f$ decomposition model, with a macroscopic fluid background and microscopic kinetic correction, both fully coupled to each other. A similar manner of discretization is proposed to that used in the recent \texttt{STRUPHY} code \cite{Holderied_Possanner_Wang_2021, Holderied_2022, Li_et_al_2023} with a finite-element model for the background and a pseudo-particle/PiC model for the correction.

        The fluid background satisfies the full, non-linear, resistive, compressible, Hall MHD equations. \cite{Laakmann_Hu_Farrell_2022} introduces finite-element(-in-space) implicit timesteppers for the incompressible analogue to this system with structure-preserving (SP) properties in the ideal case, alongside parameter-robust preconditioners. We show that these timesteppers can derive from a finite-element-in-time (FET) (and finite-element-in-space) interpretation. The benefits of this reformulation are discussed, including the derivation of timesteppers that are higher order in time, and the quantifiable dissipative SP properties in the non-ideal, resistive case.
        
        We discuss possible options for extending this FET approach to timesteppers for the compressible case.

        The kinetic corrections satisfy linearized Boltzmann equations. Using a Lénard--Bernstein collision operator, these take Fokker--Planck-like forms \cite{Fokker_1914, Planck_1917} wherein pseudo-particles in the numerical model obey the neoclassical transport equations, with particle-independent Brownian drift terms. This offers a rigorous methodology for incorporating collisions into the particle transport model, without coupling the equations of motions for each particle.
        
        Works by Chen, Chacón et al. \cite{Chen_Chacón_Barnes_2011, Chacón_Chen_Barnes_2013, Chen_Chacón_2014, Chen_Chacón_2015} have developed structure-preserving particle pushers for neoclassical transport in the Vlasov equations, derived from Crank--Nicolson integrators. We show these too can can derive from a FET interpretation, similarly offering potential extensions to higher-order-in-time particle pushers. The FET formulation is used also to consider how the stochastic drift terms can be incorporated into the pushers. Stochastic gyrokinetic expansions are also discussed.

        Different options for the numerical implementation of these schemes are considered.

        Due to the efficacy of FET in the development of SP timesteppers for both the fluid and kinetic component, we hope this approach will prove effective in the future for developing SP timesteppers for the full hybrid model. We hope this will give us the opportunity to incorporate previously inaccessible kinetic effects into the highly effective, modern, finite-element MHD models.
    \end{abstract}
    
    
    \newpage
    \tableofcontents
    
    
    \newpage
    \pagenumbering{arabic}
    %\linenumbers\renewcommand\thelinenumber{\color{black!50}\arabic{linenumber}}
            \input{0 - introduction/main.tex}
        \part{Research}
            \input{1 - low-noise PiC models/main.tex}
            \input{2 - kinetic component/main.tex}
            \input{3 - fluid component/main.tex}
            \input{4 - numerical implementation/main.tex}
        \part{Project Overview}
            \input{5 - research plan/main.tex}
            \input{6 - summary/main.tex}
    
    
    %\section{}
    \newpage
    \pagenumbering{gobble}
        \printbibliography


    \newpage
    \pagenumbering{roman}
    \appendix
        \part{Appendices}
            \input{8 - Hilbert complexes/main.tex}
            \input{9 - weak conservation proofs/main.tex}
\end{document}

            \documentclass[12pt, a4paper]{report}

\input{template/main.tex}

\title{\BA{Title in Progress...}}
\author{Boris Andrews}
\affil{Mathematical Institute, University of Oxford}
\date{\today}


\begin{document}
    \pagenumbering{gobble}
    \maketitle
    
    
    \begin{abstract}
        Magnetic confinement reactors---in particular tokamaks---offer one of the most promising options for achieving practical nuclear fusion, with the potential to provide virtually limitless, clean energy. The theoretical and numerical modeling of tokamak plasmas is simultaneously an essential component of effective reactor design, and a great research barrier. Tokamak operational conditions exhibit comparatively low Knudsen numbers. Kinetic effects, including kinetic waves and instabilities, Landau damping, bump-on-tail instabilities and more, are therefore highly influential in tokamak plasma dynamics. Purely fluid models are inherently incapable of capturing these effects, whereas the high dimensionality in purely kinetic models render them practically intractable for most relevant purposes.

        We consider a $\delta\!f$ decomposition model, with a macroscopic fluid background and microscopic kinetic correction, both fully coupled to each other. A similar manner of discretization is proposed to that used in the recent \texttt{STRUPHY} code \cite{Holderied_Possanner_Wang_2021, Holderied_2022, Li_et_al_2023} with a finite-element model for the background and a pseudo-particle/PiC model for the correction.

        The fluid background satisfies the full, non-linear, resistive, compressible, Hall MHD equations. \cite{Laakmann_Hu_Farrell_2022} introduces finite-element(-in-space) implicit timesteppers for the incompressible analogue to this system with structure-preserving (SP) properties in the ideal case, alongside parameter-robust preconditioners. We show that these timesteppers can derive from a finite-element-in-time (FET) (and finite-element-in-space) interpretation. The benefits of this reformulation are discussed, including the derivation of timesteppers that are higher order in time, and the quantifiable dissipative SP properties in the non-ideal, resistive case.
        
        We discuss possible options for extending this FET approach to timesteppers for the compressible case.

        The kinetic corrections satisfy linearized Boltzmann equations. Using a Lénard--Bernstein collision operator, these take Fokker--Planck-like forms \cite{Fokker_1914, Planck_1917} wherein pseudo-particles in the numerical model obey the neoclassical transport equations, with particle-independent Brownian drift terms. This offers a rigorous methodology for incorporating collisions into the particle transport model, without coupling the equations of motions for each particle.
        
        Works by Chen, Chacón et al. \cite{Chen_Chacón_Barnes_2011, Chacón_Chen_Barnes_2013, Chen_Chacón_2014, Chen_Chacón_2015} have developed structure-preserving particle pushers for neoclassical transport in the Vlasov equations, derived from Crank--Nicolson integrators. We show these too can can derive from a FET interpretation, similarly offering potential extensions to higher-order-in-time particle pushers. The FET formulation is used also to consider how the stochastic drift terms can be incorporated into the pushers. Stochastic gyrokinetic expansions are also discussed.

        Different options for the numerical implementation of these schemes are considered.

        Due to the efficacy of FET in the development of SP timesteppers for both the fluid and kinetic component, we hope this approach will prove effective in the future for developing SP timesteppers for the full hybrid model. We hope this will give us the opportunity to incorporate previously inaccessible kinetic effects into the highly effective, modern, finite-element MHD models.
    \end{abstract}
    
    
    \newpage
    \tableofcontents
    
    
    \newpage
    \pagenumbering{arabic}
    %\linenumbers\renewcommand\thelinenumber{\color{black!50}\arabic{linenumber}}
            \input{0 - introduction/main.tex}
        \part{Research}
            \input{1 - low-noise PiC models/main.tex}
            \input{2 - kinetic component/main.tex}
            \input{3 - fluid component/main.tex}
            \input{4 - numerical implementation/main.tex}
        \part{Project Overview}
            \input{5 - research plan/main.tex}
            \input{6 - summary/main.tex}
    
    
    %\section{}
    \newpage
    \pagenumbering{gobble}
        \printbibliography


    \newpage
    \pagenumbering{roman}
    \appendix
        \part{Appendices}
            \input{8 - Hilbert complexes/main.tex}
            \input{9 - weak conservation proofs/main.tex}
\end{document}

    
    
    %\section{}
    \newpage
    \pagenumbering{gobble}
        \printbibliography


    \newpage
    \pagenumbering{roman}
    \appendix
        \part{Appendices}
            \documentclass[12pt, a4paper]{report}

\input{template/main.tex}

\title{\BA{Title in Progress...}}
\author{Boris Andrews}
\affil{Mathematical Institute, University of Oxford}
\date{\today}


\begin{document}
    \pagenumbering{gobble}
    \maketitle
    
    
    \begin{abstract}
        Magnetic confinement reactors---in particular tokamaks---offer one of the most promising options for achieving practical nuclear fusion, with the potential to provide virtually limitless, clean energy. The theoretical and numerical modeling of tokamak plasmas is simultaneously an essential component of effective reactor design, and a great research barrier. Tokamak operational conditions exhibit comparatively low Knudsen numbers. Kinetic effects, including kinetic waves and instabilities, Landau damping, bump-on-tail instabilities and more, are therefore highly influential in tokamak plasma dynamics. Purely fluid models are inherently incapable of capturing these effects, whereas the high dimensionality in purely kinetic models render them practically intractable for most relevant purposes.

        We consider a $\delta\!f$ decomposition model, with a macroscopic fluid background and microscopic kinetic correction, both fully coupled to each other. A similar manner of discretization is proposed to that used in the recent \texttt{STRUPHY} code \cite{Holderied_Possanner_Wang_2021, Holderied_2022, Li_et_al_2023} with a finite-element model for the background and a pseudo-particle/PiC model for the correction.

        The fluid background satisfies the full, non-linear, resistive, compressible, Hall MHD equations. \cite{Laakmann_Hu_Farrell_2022} introduces finite-element(-in-space) implicit timesteppers for the incompressible analogue to this system with structure-preserving (SP) properties in the ideal case, alongside parameter-robust preconditioners. We show that these timesteppers can derive from a finite-element-in-time (FET) (and finite-element-in-space) interpretation. The benefits of this reformulation are discussed, including the derivation of timesteppers that are higher order in time, and the quantifiable dissipative SP properties in the non-ideal, resistive case.
        
        We discuss possible options for extending this FET approach to timesteppers for the compressible case.

        The kinetic corrections satisfy linearized Boltzmann equations. Using a Lénard--Bernstein collision operator, these take Fokker--Planck-like forms \cite{Fokker_1914, Planck_1917} wherein pseudo-particles in the numerical model obey the neoclassical transport equations, with particle-independent Brownian drift terms. This offers a rigorous methodology for incorporating collisions into the particle transport model, without coupling the equations of motions for each particle.
        
        Works by Chen, Chacón et al. \cite{Chen_Chacón_Barnes_2011, Chacón_Chen_Barnes_2013, Chen_Chacón_2014, Chen_Chacón_2015} have developed structure-preserving particle pushers for neoclassical transport in the Vlasov equations, derived from Crank--Nicolson integrators. We show these too can can derive from a FET interpretation, similarly offering potential extensions to higher-order-in-time particle pushers. The FET formulation is used also to consider how the stochastic drift terms can be incorporated into the pushers. Stochastic gyrokinetic expansions are also discussed.

        Different options for the numerical implementation of these schemes are considered.

        Due to the efficacy of FET in the development of SP timesteppers for both the fluid and kinetic component, we hope this approach will prove effective in the future for developing SP timesteppers for the full hybrid model. We hope this will give us the opportunity to incorporate previously inaccessible kinetic effects into the highly effective, modern, finite-element MHD models.
    \end{abstract}
    
    
    \newpage
    \tableofcontents
    
    
    \newpage
    \pagenumbering{arabic}
    %\linenumbers\renewcommand\thelinenumber{\color{black!50}\arabic{linenumber}}
            \input{0 - introduction/main.tex}
        \part{Research}
            \input{1 - low-noise PiC models/main.tex}
            \input{2 - kinetic component/main.tex}
            \input{3 - fluid component/main.tex}
            \input{4 - numerical implementation/main.tex}
        \part{Project Overview}
            \input{5 - research plan/main.tex}
            \input{6 - summary/main.tex}
    
    
    %\section{}
    \newpage
    \pagenumbering{gobble}
        \printbibliography


    \newpage
    \pagenumbering{roman}
    \appendix
        \part{Appendices}
            \input{8 - Hilbert complexes/main.tex}
            \input{9 - weak conservation proofs/main.tex}
\end{document}

            \documentclass[12pt, a4paper]{report}

\input{template/main.tex}

\title{\BA{Title in Progress...}}
\author{Boris Andrews}
\affil{Mathematical Institute, University of Oxford}
\date{\today}


\begin{document}
    \pagenumbering{gobble}
    \maketitle
    
    
    \begin{abstract}
        Magnetic confinement reactors---in particular tokamaks---offer one of the most promising options for achieving practical nuclear fusion, with the potential to provide virtually limitless, clean energy. The theoretical and numerical modeling of tokamak plasmas is simultaneously an essential component of effective reactor design, and a great research barrier. Tokamak operational conditions exhibit comparatively low Knudsen numbers. Kinetic effects, including kinetic waves and instabilities, Landau damping, bump-on-tail instabilities and more, are therefore highly influential in tokamak plasma dynamics. Purely fluid models are inherently incapable of capturing these effects, whereas the high dimensionality in purely kinetic models render them practically intractable for most relevant purposes.

        We consider a $\delta\!f$ decomposition model, with a macroscopic fluid background and microscopic kinetic correction, both fully coupled to each other. A similar manner of discretization is proposed to that used in the recent \texttt{STRUPHY} code \cite{Holderied_Possanner_Wang_2021, Holderied_2022, Li_et_al_2023} with a finite-element model for the background and a pseudo-particle/PiC model for the correction.

        The fluid background satisfies the full, non-linear, resistive, compressible, Hall MHD equations. \cite{Laakmann_Hu_Farrell_2022} introduces finite-element(-in-space) implicit timesteppers for the incompressible analogue to this system with structure-preserving (SP) properties in the ideal case, alongside parameter-robust preconditioners. We show that these timesteppers can derive from a finite-element-in-time (FET) (and finite-element-in-space) interpretation. The benefits of this reformulation are discussed, including the derivation of timesteppers that are higher order in time, and the quantifiable dissipative SP properties in the non-ideal, resistive case.
        
        We discuss possible options for extending this FET approach to timesteppers for the compressible case.

        The kinetic corrections satisfy linearized Boltzmann equations. Using a Lénard--Bernstein collision operator, these take Fokker--Planck-like forms \cite{Fokker_1914, Planck_1917} wherein pseudo-particles in the numerical model obey the neoclassical transport equations, with particle-independent Brownian drift terms. This offers a rigorous methodology for incorporating collisions into the particle transport model, without coupling the equations of motions for each particle.
        
        Works by Chen, Chacón et al. \cite{Chen_Chacón_Barnes_2011, Chacón_Chen_Barnes_2013, Chen_Chacón_2014, Chen_Chacón_2015} have developed structure-preserving particle pushers for neoclassical transport in the Vlasov equations, derived from Crank--Nicolson integrators. We show these too can can derive from a FET interpretation, similarly offering potential extensions to higher-order-in-time particle pushers. The FET formulation is used also to consider how the stochastic drift terms can be incorporated into the pushers. Stochastic gyrokinetic expansions are also discussed.

        Different options for the numerical implementation of these schemes are considered.

        Due to the efficacy of FET in the development of SP timesteppers for both the fluid and kinetic component, we hope this approach will prove effective in the future for developing SP timesteppers for the full hybrid model. We hope this will give us the opportunity to incorporate previously inaccessible kinetic effects into the highly effective, modern, finite-element MHD models.
    \end{abstract}
    
    
    \newpage
    \tableofcontents
    
    
    \newpage
    \pagenumbering{arabic}
    %\linenumbers\renewcommand\thelinenumber{\color{black!50}\arabic{linenumber}}
            \input{0 - introduction/main.tex}
        \part{Research}
            \input{1 - low-noise PiC models/main.tex}
            \input{2 - kinetic component/main.tex}
            \input{3 - fluid component/main.tex}
            \input{4 - numerical implementation/main.tex}
        \part{Project Overview}
            \input{5 - research plan/main.tex}
            \input{6 - summary/main.tex}
    
    
    %\section{}
    \newpage
    \pagenumbering{gobble}
        \printbibliography


    \newpage
    \pagenumbering{roman}
    \appendix
        \part{Appendices}
            \input{8 - Hilbert complexes/main.tex}
            \input{9 - weak conservation proofs/main.tex}
\end{document}

\end{document}

            \documentclass[12pt, a4paper]{report}

\documentclass[12pt, a4paper]{report}

\input{template/main.tex}

\title{\BA{Title in Progress...}}
\author{Boris Andrews}
\affil{Mathematical Institute, University of Oxford}
\date{\today}


\begin{document}
    \pagenumbering{gobble}
    \maketitle
    
    
    \begin{abstract}
        Magnetic confinement reactors---in particular tokamaks---offer one of the most promising options for achieving practical nuclear fusion, with the potential to provide virtually limitless, clean energy. The theoretical and numerical modeling of tokamak plasmas is simultaneously an essential component of effective reactor design, and a great research barrier. Tokamak operational conditions exhibit comparatively low Knudsen numbers. Kinetic effects, including kinetic waves and instabilities, Landau damping, bump-on-tail instabilities and more, are therefore highly influential in tokamak plasma dynamics. Purely fluid models are inherently incapable of capturing these effects, whereas the high dimensionality in purely kinetic models render them practically intractable for most relevant purposes.

        We consider a $\delta\!f$ decomposition model, with a macroscopic fluid background and microscopic kinetic correction, both fully coupled to each other. A similar manner of discretization is proposed to that used in the recent \texttt{STRUPHY} code \cite{Holderied_Possanner_Wang_2021, Holderied_2022, Li_et_al_2023} with a finite-element model for the background and a pseudo-particle/PiC model for the correction.

        The fluid background satisfies the full, non-linear, resistive, compressible, Hall MHD equations. \cite{Laakmann_Hu_Farrell_2022} introduces finite-element(-in-space) implicit timesteppers for the incompressible analogue to this system with structure-preserving (SP) properties in the ideal case, alongside parameter-robust preconditioners. We show that these timesteppers can derive from a finite-element-in-time (FET) (and finite-element-in-space) interpretation. The benefits of this reformulation are discussed, including the derivation of timesteppers that are higher order in time, and the quantifiable dissipative SP properties in the non-ideal, resistive case.
        
        We discuss possible options for extending this FET approach to timesteppers for the compressible case.

        The kinetic corrections satisfy linearized Boltzmann equations. Using a Lénard--Bernstein collision operator, these take Fokker--Planck-like forms \cite{Fokker_1914, Planck_1917} wherein pseudo-particles in the numerical model obey the neoclassical transport equations, with particle-independent Brownian drift terms. This offers a rigorous methodology for incorporating collisions into the particle transport model, without coupling the equations of motions for each particle.
        
        Works by Chen, Chacón et al. \cite{Chen_Chacón_Barnes_2011, Chacón_Chen_Barnes_2013, Chen_Chacón_2014, Chen_Chacón_2015} have developed structure-preserving particle pushers for neoclassical transport in the Vlasov equations, derived from Crank--Nicolson integrators. We show these too can can derive from a FET interpretation, similarly offering potential extensions to higher-order-in-time particle pushers. The FET formulation is used also to consider how the stochastic drift terms can be incorporated into the pushers. Stochastic gyrokinetic expansions are also discussed.

        Different options for the numerical implementation of these schemes are considered.

        Due to the efficacy of FET in the development of SP timesteppers for both the fluid and kinetic component, we hope this approach will prove effective in the future for developing SP timesteppers for the full hybrid model. We hope this will give us the opportunity to incorporate previously inaccessible kinetic effects into the highly effective, modern, finite-element MHD models.
    \end{abstract}
    
    
    \newpage
    \tableofcontents
    
    
    \newpage
    \pagenumbering{arabic}
    %\linenumbers\renewcommand\thelinenumber{\color{black!50}\arabic{linenumber}}
            \input{0 - introduction/main.tex}
        \part{Research}
            \input{1 - low-noise PiC models/main.tex}
            \input{2 - kinetic component/main.tex}
            \input{3 - fluid component/main.tex}
            \input{4 - numerical implementation/main.tex}
        \part{Project Overview}
            \input{5 - research plan/main.tex}
            \input{6 - summary/main.tex}
    
    
    %\section{}
    \newpage
    \pagenumbering{gobble}
        \printbibliography


    \newpage
    \pagenumbering{roman}
    \appendix
        \part{Appendices}
            \input{8 - Hilbert complexes/main.tex}
            \input{9 - weak conservation proofs/main.tex}
\end{document}


\title{\BA{Title in Progress...}}
\author{Boris Andrews}
\affil{Mathematical Institute, University of Oxford}
\date{\today}


\begin{document}
    \pagenumbering{gobble}
    \maketitle
    
    
    \begin{abstract}
        Magnetic confinement reactors---in particular tokamaks---offer one of the most promising options for achieving practical nuclear fusion, with the potential to provide virtually limitless, clean energy. The theoretical and numerical modeling of tokamak plasmas is simultaneously an essential component of effective reactor design, and a great research barrier. Tokamak operational conditions exhibit comparatively low Knudsen numbers. Kinetic effects, including kinetic waves and instabilities, Landau damping, bump-on-tail instabilities and more, are therefore highly influential in tokamak plasma dynamics. Purely fluid models are inherently incapable of capturing these effects, whereas the high dimensionality in purely kinetic models render them practically intractable for most relevant purposes.

        We consider a $\delta\!f$ decomposition model, with a macroscopic fluid background and microscopic kinetic correction, both fully coupled to each other. A similar manner of discretization is proposed to that used in the recent \texttt{STRUPHY} code \cite{Holderied_Possanner_Wang_2021, Holderied_2022, Li_et_al_2023} with a finite-element model for the background and a pseudo-particle/PiC model for the correction.

        The fluid background satisfies the full, non-linear, resistive, compressible, Hall MHD equations. \cite{Laakmann_Hu_Farrell_2022} introduces finite-element(-in-space) implicit timesteppers for the incompressible analogue to this system with structure-preserving (SP) properties in the ideal case, alongside parameter-robust preconditioners. We show that these timesteppers can derive from a finite-element-in-time (FET) (and finite-element-in-space) interpretation. The benefits of this reformulation are discussed, including the derivation of timesteppers that are higher order in time, and the quantifiable dissipative SP properties in the non-ideal, resistive case.
        
        We discuss possible options for extending this FET approach to timesteppers for the compressible case.

        The kinetic corrections satisfy linearized Boltzmann equations. Using a Lénard--Bernstein collision operator, these take Fokker--Planck-like forms \cite{Fokker_1914, Planck_1917} wherein pseudo-particles in the numerical model obey the neoclassical transport equations, with particle-independent Brownian drift terms. This offers a rigorous methodology for incorporating collisions into the particle transport model, without coupling the equations of motions for each particle.
        
        Works by Chen, Chacón et al. \cite{Chen_Chacón_Barnes_2011, Chacón_Chen_Barnes_2013, Chen_Chacón_2014, Chen_Chacón_2015} have developed structure-preserving particle pushers for neoclassical transport in the Vlasov equations, derived from Crank--Nicolson integrators. We show these too can can derive from a FET interpretation, similarly offering potential extensions to higher-order-in-time particle pushers. The FET formulation is used also to consider how the stochastic drift terms can be incorporated into the pushers. Stochastic gyrokinetic expansions are also discussed.

        Different options for the numerical implementation of these schemes are considered.

        Due to the efficacy of FET in the development of SP timesteppers for both the fluid and kinetic component, we hope this approach will prove effective in the future for developing SP timesteppers for the full hybrid model. We hope this will give us the opportunity to incorporate previously inaccessible kinetic effects into the highly effective, modern, finite-element MHD models.
    \end{abstract}
    
    
    \newpage
    \tableofcontents
    
    
    \newpage
    \pagenumbering{arabic}
    %\linenumbers\renewcommand\thelinenumber{\color{black!50}\arabic{linenumber}}
            \documentclass[12pt, a4paper]{report}

\input{template/main.tex}

\title{\BA{Title in Progress...}}
\author{Boris Andrews}
\affil{Mathematical Institute, University of Oxford}
\date{\today}


\begin{document}
    \pagenumbering{gobble}
    \maketitle
    
    
    \begin{abstract}
        Magnetic confinement reactors---in particular tokamaks---offer one of the most promising options for achieving practical nuclear fusion, with the potential to provide virtually limitless, clean energy. The theoretical and numerical modeling of tokamak plasmas is simultaneously an essential component of effective reactor design, and a great research barrier. Tokamak operational conditions exhibit comparatively low Knudsen numbers. Kinetic effects, including kinetic waves and instabilities, Landau damping, bump-on-tail instabilities and more, are therefore highly influential in tokamak plasma dynamics. Purely fluid models are inherently incapable of capturing these effects, whereas the high dimensionality in purely kinetic models render them practically intractable for most relevant purposes.

        We consider a $\delta\!f$ decomposition model, with a macroscopic fluid background and microscopic kinetic correction, both fully coupled to each other. A similar manner of discretization is proposed to that used in the recent \texttt{STRUPHY} code \cite{Holderied_Possanner_Wang_2021, Holderied_2022, Li_et_al_2023} with a finite-element model for the background and a pseudo-particle/PiC model for the correction.

        The fluid background satisfies the full, non-linear, resistive, compressible, Hall MHD equations. \cite{Laakmann_Hu_Farrell_2022} introduces finite-element(-in-space) implicit timesteppers for the incompressible analogue to this system with structure-preserving (SP) properties in the ideal case, alongside parameter-robust preconditioners. We show that these timesteppers can derive from a finite-element-in-time (FET) (and finite-element-in-space) interpretation. The benefits of this reformulation are discussed, including the derivation of timesteppers that are higher order in time, and the quantifiable dissipative SP properties in the non-ideal, resistive case.
        
        We discuss possible options for extending this FET approach to timesteppers for the compressible case.

        The kinetic corrections satisfy linearized Boltzmann equations. Using a Lénard--Bernstein collision operator, these take Fokker--Planck-like forms \cite{Fokker_1914, Planck_1917} wherein pseudo-particles in the numerical model obey the neoclassical transport equations, with particle-independent Brownian drift terms. This offers a rigorous methodology for incorporating collisions into the particle transport model, without coupling the equations of motions for each particle.
        
        Works by Chen, Chacón et al. \cite{Chen_Chacón_Barnes_2011, Chacón_Chen_Barnes_2013, Chen_Chacón_2014, Chen_Chacón_2015} have developed structure-preserving particle pushers for neoclassical transport in the Vlasov equations, derived from Crank--Nicolson integrators. We show these too can can derive from a FET interpretation, similarly offering potential extensions to higher-order-in-time particle pushers. The FET formulation is used also to consider how the stochastic drift terms can be incorporated into the pushers. Stochastic gyrokinetic expansions are also discussed.

        Different options for the numerical implementation of these schemes are considered.

        Due to the efficacy of FET in the development of SP timesteppers for both the fluid and kinetic component, we hope this approach will prove effective in the future for developing SP timesteppers for the full hybrid model. We hope this will give us the opportunity to incorporate previously inaccessible kinetic effects into the highly effective, modern, finite-element MHD models.
    \end{abstract}
    
    
    \newpage
    \tableofcontents
    
    
    \newpage
    \pagenumbering{arabic}
    %\linenumbers\renewcommand\thelinenumber{\color{black!50}\arabic{linenumber}}
            \input{0 - introduction/main.tex}
        \part{Research}
            \input{1 - low-noise PiC models/main.tex}
            \input{2 - kinetic component/main.tex}
            \input{3 - fluid component/main.tex}
            \input{4 - numerical implementation/main.tex}
        \part{Project Overview}
            \input{5 - research plan/main.tex}
            \input{6 - summary/main.tex}
    
    
    %\section{}
    \newpage
    \pagenumbering{gobble}
        \printbibliography


    \newpage
    \pagenumbering{roman}
    \appendix
        \part{Appendices}
            \input{8 - Hilbert complexes/main.tex}
            \input{9 - weak conservation proofs/main.tex}
\end{document}

        \part{Research}
            \documentclass[12pt, a4paper]{report}

\input{template/main.tex}

\title{\BA{Title in Progress...}}
\author{Boris Andrews}
\affil{Mathematical Institute, University of Oxford}
\date{\today}


\begin{document}
    \pagenumbering{gobble}
    \maketitle
    
    
    \begin{abstract}
        Magnetic confinement reactors---in particular tokamaks---offer one of the most promising options for achieving practical nuclear fusion, with the potential to provide virtually limitless, clean energy. The theoretical and numerical modeling of tokamak plasmas is simultaneously an essential component of effective reactor design, and a great research barrier. Tokamak operational conditions exhibit comparatively low Knudsen numbers. Kinetic effects, including kinetic waves and instabilities, Landau damping, bump-on-tail instabilities and more, are therefore highly influential in tokamak plasma dynamics. Purely fluid models are inherently incapable of capturing these effects, whereas the high dimensionality in purely kinetic models render them practically intractable for most relevant purposes.

        We consider a $\delta\!f$ decomposition model, with a macroscopic fluid background and microscopic kinetic correction, both fully coupled to each other. A similar manner of discretization is proposed to that used in the recent \texttt{STRUPHY} code \cite{Holderied_Possanner_Wang_2021, Holderied_2022, Li_et_al_2023} with a finite-element model for the background and a pseudo-particle/PiC model for the correction.

        The fluid background satisfies the full, non-linear, resistive, compressible, Hall MHD equations. \cite{Laakmann_Hu_Farrell_2022} introduces finite-element(-in-space) implicit timesteppers for the incompressible analogue to this system with structure-preserving (SP) properties in the ideal case, alongside parameter-robust preconditioners. We show that these timesteppers can derive from a finite-element-in-time (FET) (and finite-element-in-space) interpretation. The benefits of this reformulation are discussed, including the derivation of timesteppers that are higher order in time, and the quantifiable dissipative SP properties in the non-ideal, resistive case.
        
        We discuss possible options for extending this FET approach to timesteppers for the compressible case.

        The kinetic corrections satisfy linearized Boltzmann equations. Using a Lénard--Bernstein collision operator, these take Fokker--Planck-like forms \cite{Fokker_1914, Planck_1917} wherein pseudo-particles in the numerical model obey the neoclassical transport equations, with particle-independent Brownian drift terms. This offers a rigorous methodology for incorporating collisions into the particle transport model, without coupling the equations of motions for each particle.
        
        Works by Chen, Chacón et al. \cite{Chen_Chacón_Barnes_2011, Chacón_Chen_Barnes_2013, Chen_Chacón_2014, Chen_Chacón_2015} have developed structure-preserving particle pushers for neoclassical transport in the Vlasov equations, derived from Crank--Nicolson integrators. We show these too can can derive from a FET interpretation, similarly offering potential extensions to higher-order-in-time particle pushers. The FET formulation is used also to consider how the stochastic drift terms can be incorporated into the pushers. Stochastic gyrokinetic expansions are also discussed.

        Different options for the numerical implementation of these schemes are considered.

        Due to the efficacy of FET in the development of SP timesteppers for both the fluid and kinetic component, we hope this approach will prove effective in the future for developing SP timesteppers for the full hybrid model. We hope this will give us the opportunity to incorporate previously inaccessible kinetic effects into the highly effective, modern, finite-element MHD models.
    \end{abstract}
    
    
    \newpage
    \tableofcontents
    
    
    \newpage
    \pagenumbering{arabic}
    %\linenumbers\renewcommand\thelinenumber{\color{black!50}\arabic{linenumber}}
            \input{0 - introduction/main.tex}
        \part{Research}
            \input{1 - low-noise PiC models/main.tex}
            \input{2 - kinetic component/main.tex}
            \input{3 - fluid component/main.tex}
            \input{4 - numerical implementation/main.tex}
        \part{Project Overview}
            \input{5 - research plan/main.tex}
            \input{6 - summary/main.tex}
    
    
    %\section{}
    \newpage
    \pagenumbering{gobble}
        \printbibliography


    \newpage
    \pagenumbering{roman}
    \appendix
        \part{Appendices}
            \input{8 - Hilbert complexes/main.tex}
            \input{9 - weak conservation proofs/main.tex}
\end{document}

            \documentclass[12pt, a4paper]{report}

\input{template/main.tex}

\title{\BA{Title in Progress...}}
\author{Boris Andrews}
\affil{Mathematical Institute, University of Oxford}
\date{\today}


\begin{document}
    \pagenumbering{gobble}
    \maketitle
    
    
    \begin{abstract}
        Magnetic confinement reactors---in particular tokamaks---offer one of the most promising options for achieving practical nuclear fusion, with the potential to provide virtually limitless, clean energy. The theoretical and numerical modeling of tokamak plasmas is simultaneously an essential component of effective reactor design, and a great research barrier. Tokamak operational conditions exhibit comparatively low Knudsen numbers. Kinetic effects, including kinetic waves and instabilities, Landau damping, bump-on-tail instabilities and more, are therefore highly influential in tokamak plasma dynamics. Purely fluid models are inherently incapable of capturing these effects, whereas the high dimensionality in purely kinetic models render them practically intractable for most relevant purposes.

        We consider a $\delta\!f$ decomposition model, with a macroscopic fluid background and microscopic kinetic correction, both fully coupled to each other. A similar manner of discretization is proposed to that used in the recent \texttt{STRUPHY} code \cite{Holderied_Possanner_Wang_2021, Holderied_2022, Li_et_al_2023} with a finite-element model for the background and a pseudo-particle/PiC model for the correction.

        The fluid background satisfies the full, non-linear, resistive, compressible, Hall MHD equations. \cite{Laakmann_Hu_Farrell_2022} introduces finite-element(-in-space) implicit timesteppers for the incompressible analogue to this system with structure-preserving (SP) properties in the ideal case, alongside parameter-robust preconditioners. We show that these timesteppers can derive from a finite-element-in-time (FET) (and finite-element-in-space) interpretation. The benefits of this reformulation are discussed, including the derivation of timesteppers that are higher order in time, and the quantifiable dissipative SP properties in the non-ideal, resistive case.
        
        We discuss possible options for extending this FET approach to timesteppers for the compressible case.

        The kinetic corrections satisfy linearized Boltzmann equations. Using a Lénard--Bernstein collision operator, these take Fokker--Planck-like forms \cite{Fokker_1914, Planck_1917} wherein pseudo-particles in the numerical model obey the neoclassical transport equations, with particle-independent Brownian drift terms. This offers a rigorous methodology for incorporating collisions into the particle transport model, without coupling the equations of motions for each particle.
        
        Works by Chen, Chacón et al. \cite{Chen_Chacón_Barnes_2011, Chacón_Chen_Barnes_2013, Chen_Chacón_2014, Chen_Chacón_2015} have developed structure-preserving particle pushers for neoclassical transport in the Vlasov equations, derived from Crank--Nicolson integrators. We show these too can can derive from a FET interpretation, similarly offering potential extensions to higher-order-in-time particle pushers. The FET formulation is used also to consider how the stochastic drift terms can be incorporated into the pushers. Stochastic gyrokinetic expansions are also discussed.

        Different options for the numerical implementation of these schemes are considered.

        Due to the efficacy of FET in the development of SP timesteppers for both the fluid and kinetic component, we hope this approach will prove effective in the future for developing SP timesteppers for the full hybrid model. We hope this will give us the opportunity to incorporate previously inaccessible kinetic effects into the highly effective, modern, finite-element MHD models.
    \end{abstract}
    
    
    \newpage
    \tableofcontents
    
    
    \newpage
    \pagenumbering{arabic}
    %\linenumbers\renewcommand\thelinenumber{\color{black!50}\arabic{linenumber}}
            \input{0 - introduction/main.tex}
        \part{Research}
            \input{1 - low-noise PiC models/main.tex}
            \input{2 - kinetic component/main.tex}
            \input{3 - fluid component/main.tex}
            \input{4 - numerical implementation/main.tex}
        \part{Project Overview}
            \input{5 - research plan/main.tex}
            \input{6 - summary/main.tex}
    
    
    %\section{}
    \newpage
    \pagenumbering{gobble}
        \printbibliography


    \newpage
    \pagenumbering{roman}
    \appendix
        \part{Appendices}
            \input{8 - Hilbert complexes/main.tex}
            \input{9 - weak conservation proofs/main.tex}
\end{document}

            \documentclass[12pt, a4paper]{report}

\input{template/main.tex}

\title{\BA{Title in Progress...}}
\author{Boris Andrews}
\affil{Mathematical Institute, University of Oxford}
\date{\today}


\begin{document}
    \pagenumbering{gobble}
    \maketitle
    
    
    \begin{abstract}
        Magnetic confinement reactors---in particular tokamaks---offer one of the most promising options for achieving practical nuclear fusion, with the potential to provide virtually limitless, clean energy. The theoretical and numerical modeling of tokamak plasmas is simultaneously an essential component of effective reactor design, and a great research barrier. Tokamak operational conditions exhibit comparatively low Knudsen numbers. Kinetic effects, including kinetic waves and instabilities, Landau damping, bump-on-tail instabilities and more, are therefore highly influential in tokamak plasma dynamics. Purely fluid models are inherently incapable of capturing these effects, whereas the high dimensionality in purely kinetic models render them practically intractable for most relevant purposes.

        We consider a $\delta\!f$ decomposition model, with a macroscopic fluid background and microscopic kinetic correction, both fully coupled to each other. A similar manner of discretization is proposed to that used in the recent \texttt{STRUPHY} code \cite{Holderied_Possanner_Wang_2021, Holderied_2022, Li_et_al_2023} with a finite-element model for the background and a pseudo-particle/PiC model for the correction.

        The fluid background satisfies the full, non-linear, resistive, compressible, Hall MHD equations. \cite{Laakmann_Hu_Farrell_2022} introduces finite-element(-in-space) implicit timesteppers for the incompressible analogue to this system with structure-preserving (SP) properties in the ideal case, alongside parameter-robust preconditioners. We show that these timesteppers can derive from a finite-element-in-time (FET) (and finite-element-in-space) interpretation. The benefits of this reformulation are discussed, including the derivation of timesteppers that are higher order in time, and the quantifiable dissipative SP properties in the non-ideal, resistive case.
        
        We discuss possible options for extending this FET approach to timesteppers for the compressible case.

        The kinetic corrections satisfy linearized Boltzmann equations. Using a Lénard--Bernstein collision operator, these take Fokker--Planck-like forms \cite{Fokker_1914, Planck_1917} wherein pseudo-particles in the numerical model obey the neoclassical transport equations, with particle-independent Brownian drift terms. This offers a rigorous methodology for incorporating collisions into the particle transport model, without coupling the equations of motions for each particle.
        
        Works by Chen, Chacón et al. \cite{Chen_Chacón_Barnes_2011, Chacón_Chen_Barnes_2013, Chen_Chacón_2014, Chen_Chacón_2015} have developed structure-preserving particle pushers for neoclassical transport in the Vlasov equations, derived from Crank--Nicolson integrators. We show these too can can derive from a FET interpretation, similarly offering potential extensions to higher-order-in-time particle pushers. The FET formulation is used also to consider how the stochastic drift terms can be incorporated into the pushers. Stochastic gyrokinetic expansions are also discussed.

        Different options for the numerical implementation of these schemes are considered.

        Due to the efficacy of FET in the development of SP timesteppers for both the fluid and kinetic component, we hope this approach will prove effective in the future for developing SP timesteppers for the full hybrid model. We hope this will give us the opportunity to incorporate previously inaccessible kinetic effects into the highly effective, modern, finite-element MHD models.
    \end{abstract}
    
    
    \newpage
    \tableofcontents
    
    
    \newpage
    \pagenumbering{arabic}
    %\linenumbers\renewcommand\thelinenumber{\color{black!50}\arabic{linenumber}}
            \input{0 - introduction/main.tex}
        \part{Research}
            \input{1 - low-noise PiC models/main.tex}
            \input{2 - kinetic component/main.tex}
            \input{3 - fluid component/main.tex}
            \input{4 - numerical implementation/main.tex}
        \part{Project Overview}
            \input{5 - research plan/main.tex}
            \input{6 - summary/main.tex}
    
    
    %\section{}
    \newpage
    \pagenumbering{gobble}
        \printbibliography


    \newpage
    \pagenumbering{roman}
    \appendix
        \part{Appendices}
            \input{8 - Hilbert complexes/main.tex}
            \input{9 - weak conservation proofs/main.tex}
\end{document}

            \documentclass[12pt, a4paper]{report}

\input{template/main.tex}

\title{\BA{Title in Progress...}}
\author{Boris Andrews}
\affil{Mathematical Institute, University of Oxford}
\date{\today}


\begin{document}
    \pagenumbering{gobble}
    \maketitle
    
    
    \begin{abstract}
        Magnetic confinement reactors---in particular tokamaks---offer one of the most promising options for achieving practical nuclear fusion, with the potential to provide virtually limitless, clean energy. The theoretical and numerical modeling of tokamak plasmas is simultaneously an essential component of effective reactor design, and a great research barrier. Tokamak operational conditions exhibit comparatively low Knudsen numbers. Kinetic effects, including kinetic waves and instabilities, Landau damping, bump-on-tail instabilities and more, are therefore highly influential in tokamak plasma dynamics. Purely fluid models are inherently incapable of capturing these effects, whereas the high dimensionality in purely kinetic models render them practically intractable for most relevant purposes.

        We consider a $\delta\!f$ decomposition model, with a macroscopic fluid background and microscopic kinetic correction, both fully coupled to each other. A similar manner of discretization is proposed to that used in the recent \texttt{STRUPHY} code \cite{Holderied_Possanner_Wang_2021, Holderied_2022, Li_et_al_2023} with a finite-element model for the background and a pseudo-particle/PiC model for the correction.

        The fluid background satisfies the full, non-linear, resistive, compressible, Hall MHD equations. \cite{Laakmann_Hu_Farrell_2022} introduces finite-element(-in-space) implicit timesteppers for the incompressible analogue to this system with structure-preserving (SP) properties in the ideal case, alongside parameter-robust preconditioners. We show that these timesteppers can derive from a finite-element-in-time (FET) (and finite-element-in-space) interpretation. The benefits of this reformulation are discussed, including the derivation of timesteppers that are higher order in time, and the quantifiable dissipative SP properties in the non-ideal, resistive case.
        
        We discuss possible options for extending this FET approach to timesteppers for the compressible case.

        The kinetic corrections satisfy linearized Boltzmann equations. Using a Lénard--Bernstein collision operator, these take Fokker--Planck-like forms \cite{Fokker_1914, Planck_1917} wherein pseudo-particles in the numerical model obey the neoclassical transport equations, with particle-independent Brownian drift terms. This offers a rigorous methodology for incorporating collisions into the particle transport model, without coupling the equations of motions for each particle.
        
        Works by Chen, Chacón et al. \cite{Chen_Chacón_Barnes_2011, Chacón_Chen_Barnes_2013, Chen_Chacón_2014, Chen_Chacón_2015} have developed structure-preserving particle pushers for neoclassical transport in the Vlasov equations, derived from Crank--Nicolson integrators. We show these too can can derive from a FET interpretation, similarly offering potential extensions to higher-order-in-time particle pushers. The FET formulation is used also to consider how the stochastic drift terms can be incorporated into the pushers. Stochastic gyrokinetic expansions are also discussed.

        Different options for the numerical implementation of these schemes are considered.

        Due to the efficacy of FET in the development of SP timesteppers for both the fluid and kinetic component, we hope this approach will prove effective in the future for developing SP timesteppers for the full hybrid model. We hope this will give us the opportunity to incorporate previously inaccessible kinetic effects into the highly effective, modern, finite-element MHD models.
    \end{abstract}
    
    
    \newpage
    \tableofcontents
    
    
    \newpage
    \pagenumbering{arabic}
    %\linenumbers\renewcommand\thelinenumber{\color{black!50}\arabic{linenumber}}
            \input{0 - introduction/main.tex}
        \part{Research}
            \input{1 - low-noise PiC models/main.tex}
            \input{2 - kinetic component/main.tex}
            \input{3 - fluid component/main.tex}
            \input{4 - numerical implementation/main.tex}
        \part{Project Overview}
            \input{5 - research plan/main.tex}
            \input{6 - summary/main.tex}
    
    
    %\section{}
    \newpage
    \pagenumbering{gobble}
        \printbibliography


    \newpage
    \pagenumbering{roman}
    \appendix
        \part{Appendices}
            \input{8 - Hilbert complexes/main.tex}
            \input{9 - weak conservation proofs/main.tex}
\end{document}

        \part{Project Overview}
            \documentclass[12pt, a4paper]{report}

\input{template/main.tex}

\title{\BA{Title in Progress...}}
\author{Boris Andrews}
\affil{Mathematical Institute, University of Oxford}
\date{\today}


\begin{document}
    \pagenumbering{gobble}
    \maketitle
    
    
    \begin{abstract}
        Magnetic confinement reactors---in particular tokamaks---offer one of the most promising options for achieving practical nuclear fusion, with the potential to provide virtually limitless, clean energy. The theoretical and numerical modeling of tokamak plasmas is simultaneously an essential component of effective reactor design, and a great research barrier. Tokamak operational conditions exhibit comparatively low Knudsen numbers. Kinetic effects, including kinetic waves and instabilities, Landau damping, bump-on-tail instabilities and more, are therefore highly influential in tokamak plasma dynamics. Purely fluid models are inherently incapable of capturing these effects, whereas the high dimensionality in purely kinetic models render them practically intractable for most relevant purposes.

        We consider a $\delta\!f$ decomposition model, with a macroscopic fluid background and microscopic kinetic correction, both fully coupled to each other. A similar manner of discretization is proposed to that used in the recent \texttt{STRUPHY} code \cite{Holderied_Possanner_Wang_2021, Holderied_2022, Li_et_al_2023} with a finite-element model for the background and a pseudo-particle/PiC model for the correction.

        The fluid background satisfies the full, non-linear, resistive, compressible, Hall MHD equations. \cite{Laakmann_Hu_Farrell_2022} introduces finite-element(-in-space) implicit timesteppers for the incompressible analogue to this system with structure-preserving (SP) properties in the ideal case, alongside parameter-robust preconditioners. We show that these timesteppers can derive from a finite-element-in-time (FET) (and finite-element-in-space) interpretation. The benefits of this reformulation are discussed, including the derivation of timesteppers that are higher order in time, and the quantifiable dissipative SP properties in the non-ideal, resistive case.
        
        We discuss possible options for extending this FET approach to timesteppers for the compressible case.

        The kinetic corrections satisfy linearized Boltzmann equations. Using a Lénard--Bernstein collision operator, these take Fokker--Planck-like forms \cite{Fokker_1914, Planck_1917} wherein pseudo-particles in the numerical model obey the neoclassical transport equations, with particle-independent Brownian drift terms. This offers a rigorous methodology for incorporating collisions into the particle transport model, without coupling the equations of motions for each particle.
        
        Works by Chen, Chacón et al. \cite{Chen_Chacón_Barnes_2011, Chacón_Chen_Barnes_2013, Chen_Chacón_2014, Chen_Chacón_2015} have developed structure-preserving particle pushers for neoclassical transport in the Vlasov equations, derived from Crank--Nicolson integrators. We show these too can can derive from a FET interpretation, similarly offering potential extensions to higher-order-in-time particle pushers. The FET formulation is used also to consider how the stochastic drift terms can be incorporated into the pushers. Stochastic gyrokinetic expansions are also discussed.

        Different options for the numerical implementation of these schemes are considered.

        Due to the efficacy of FET in the development of SP timesteppers for both the fluid and kinetic component, we hope this approach will prove effective in the future for developing SP timesteppers for the full hybrid model. We hope this will give us the opportunity to incorporate previously inaccessible kinetic effects into the highly effective, modern, finite-element MHD models.
    \end{abstract}
    
    
    \newpage
    \tableofcontents
    
    
    \newpage
    \pagenumbering{arabic}
    %\linenumbers\renewcommand\thelinenumber{\color{black!50}\arabic{linenumber}}
            \input{0 - introduction/main.tex}
        \part{Research}
            \input{1 - low-noise PiC models/main.tex}
            \input{2 - kinetic component/main.tex}
            \input{3 - fluid component/main.tex}
            \input{4 - numerical implementation/main.tex}
        \part{Project Overview}
            \input{5 - research plan/main.tex}
            \input{6 - summary/main.tex}
    
    
    %\section{}
    \newpage
    \pagenumbering{gobble}
        \printbibliography


    \newpage
    \pagenumbering{roman}
    \appendix
        \part{Appendices}
            \input{8 - Hilbert complexes/main.tex}
            \input{9 - weak conservation proofs/main.tex}
\end{document}

            \documentclass[12pt, a4paper]{report}

\input{template/main.tex}

\title{\BA{Title in Progress...}}
\author{Boris Andrews}
\affil{Mathematical Institute, University of Oxford}
\date{\today}


\begin{document}
    \pagenumbering{gobble}
    \maketitle
    
    
    \begin{abstract}
        Magnetic confinement reactors---in particular tokamaks---offer one of the most promising options for achieving practical nuclear fusion, with the potential to provide virtually limitless, clean energy. The theoretical and numerical modeling of tokamak plasmas is simultaneously an essential component of effective reactor design, and a great research barrier. Tokamak operational conditions exhibit comparatively low Knudsen numbers. Kinetic effects, including kinetic waves and instabilities, Landau damping, bump-on-tail instabilities and more, are therefore highly influential in tokamak plasma dynamics. Purely fluid models are inherently incapable of capturing these effects, whereas the high dimensionality in purely kinetic models render them practically intractable for most relevant purposes.

        We consider a $\delta\!f$ decomposition model, with a macroscopic fluid background and microscopic kinetic correction, both fully coupled to each other. A similar manner of discretization is proposed to that used in the recent \texttt{STRUPHY} code \cite{Holderied_Possanner_Wang_2021, Holderied_2022, Li_et_al_2023} with a finite-element model for the background and a pseudo-particle/PiC model for the correction.

        The fluid background satisfies the full, non-linear, resistive, compressible, Hall MHD equations. \cite{Laakmann_Hu_Farrell_2022} introduces finite-element(-in-space) implicit timesteppers for the incompressible analogue to this system with structure-preserving (SP) properties in the ideal case, alongside parameter-robust preconditioners. We show that these timesteppers can derive from a finite-element-in-time (FET) (and finite-element-in-space) interpretation. The benefits of this reformulation are discussed, including the derivation of timesteppers that are higher order in time, and the quantifiable dissipative SP properties in the non-ideal, resistive case.
        
        We discuss possible options for extending this FET approach to timesteppers for the compressible case.

        The kinetic corrections satisfy linearized Boltzmann equations. Using a Lénard--Bernstein collision operator, these take Fokker--Planck-like forms \cite{Fokker_1914, Planck_1917} wherein pseudo-particles in the numerical model obey the neoclassical transport equations, with particle-independent Brownian drift terms. This offers a rigorous methodology for incorporating collisions into the particle transport model, without coupling the equations of motions for each particle.
        
        Works by Chen, Chacón et al. \cite{Chen_Chacón_Barnes_2011, Chacón_Chen_Barnes_2013, Chen_Chacón_2014, Chen_Chacón_2015} have developed structure-preserving particle pushers for neoclassical transport in the Vlasov equations, derived from Crank--Nicolson integrators. We show these too can can derive from a FET interpretation, similarly offering potential extensions to higher-order-in-time particle pushers. The FET formulation is used also to consider how the stochastic drift terms can be incorporated into the pushers. Stochastic gyrokinetic expansions are also discussed.

        Different options for the numerical implementation of these schemes are considered.

        Due to the efficacy of FET in the development of SP timesteppers for both the fluid and kinetic component, we hope this approach will prove effective in the future for developing SP timesteppers for the full hybrid model. We hope this will give us the opportunity to incorporate previously inaccessible kinetic effects into the highly effective, modern, finite-element MHD models.
    \end{abstract}
    
    
    \newpage
    \tableofcontents
    
    
    \newpage
    \pagenumbering{arabic}
    %\linenumbers\renewcommand\thelinenumber{\color{black!50}\arabic{linenumber}}
            \input{0 - introduction/main.tex}
        \part{Research}
            \input{1 - low-noise PiC models/main.tex}
            \input{2 - kinetic component/main.tex}
            \input{3 - fluid component/main.tex}
            \input{4 - numerical implementation/main.tex}
        \part{Project Overview}
            \input{5 - research plan/main.tex}
            \input{6 - summary/main.tex}
    
    
    %\section{}
    \newpage
    \pagenumbering{gobble}
        \printbibliography


    \newpage
    \pagenumbering{roman}
    \appendix
        \part{Appendices}
            \input{8 - Hilbert complexes/main.tex}
            \input{9 - weak conservation proofs/main.tex}
\end{document}

    
    
    %\section{}
    \newpage
    \pagenumbering{gobble}
        \printbibliography


    \newpage
    \pagenumbering{roman}
    \appendix
        \part{Appendices}
            \documentclass[12pt, a4paper]{report}

\input{template/main.tex}

\title{\BA{Title in Progress...}}
\author{Boris Andrews}
\affil{Mathematical Institute, University of Oxford}
\date{\today}


\begin{document}
    \pagenumbering{gobble}
    \maketitle
    
    
    \begin{abstract}
        Magnetic confinement reactors---in particular tokamaks---offer one of the most promising options for achieving practical nuclear fusion, with the potential to provide virtually limitless, clean energy. The theoretical and numerical modeling of tokamak plasmas is simultaneously an essential component of effective reactor design, and a great research barrier. Tokamak operational conditions exhibit comparatively low Knudsen numbers. Kinetic effects, including kinetic waves and instabilities, Landau damping, bump-on-tail instabilities and more, are therefore highly influential in tokamak plasma dynamics. Purely fluid models are inherently incapable of capturing these effects, whereas the high dimensionality in purely kinetic models render them practically intractable for most relevant purposes.

        We consider a $\delta\!f$ decomposition model, with a macroscopic fluid background and microscopic kinetic correction, both fully coupled to each other. A similar manner of discretization is proposed to that used in the recent \texttt{STRUPHY} code \cite{Holderied_Possanner_Wang_2021, Holderied_2022, Li_et_al_2023} with a finite-element model for the background and a pseudo-particle/PiC model for the correction.

        The fluid background satisfies the full, non-linear, resistive, compressible, Hall MHD equations. \cite{Laakmann_Hu_Farrell_2022} introduces finite-element(-in-space) implicit timesteppers for the incompressible analogue to this system with structure-preserving (SP) properties in the ideal case, alongside parameter-robust preconditioners. We show that these timesteppers can derive from a finite-element-in-time (FET) (and finite-element-in-space) interpretation. The benefits of this reformulation are discussed, including the derivation of timesteppers that are higher order in time, and the quantifiable dissipative SP properties in the non-ideal, resistive case.
        
        We discuss possible options for extending this FET approach to timesteppers for the compressible case.

        The kinetic corrections satisfy linearized Boltzmann equations. Using a Lénard--Bernstein collision operator, these take Fokker--Planck-like forms \cite{Fokker_1914, Planck_1917} wherein pseudo-particles in the numerical model obey the neoclassical transport equations, with particle-independent Brownian drift terms. This offers a rigorous methodology for incorporating collisions into the particle transport model, without coupling the equations of motions for each particle.
        
        Works by Chen, Chacón et al. \cite{Chen_Chacón_Barnes_2011, Chacón_Chen_Barnes_2013, Chen_Chacón_2014, Chen_Chacón_2015} have developed structure-preserving particle pushers for neoclassical transport in the Vlasov equations, derived from Crank--Nicolson integrators. We show these too can can derive from a FET interpretation, similarly offering potential extensions to higher-order-in-time particle pushers. The FET formulation is used also to consider how the stochastic drift terms can be incorporated into the pushers. Stochastic gyrokinetic expansions are also discussed.

        Different options for the numerical implementation of these schemes are considered.

        Due to the efficacy of FET in the development of SP timesteppers for both the fluid and kinetic component, we hope this approach will prove effective in the future for developing SP timesteppers for the full hybrid model. We hope this will give us the opportunity to incorporate previously inaccessible kinetic effects into the highly effective, modern, finite-element MHD models.
    \end{abstract}
    
    
    \newpage
    \tableofcontents
    
    
    \newpage
    \pagenumbering{arabic}
    %\linenumbers\renewcommand\thelinenumber{\color{black!50}\arabic{linenumber}}
            \input{0 - introduction/main.tex}
        \part{Research}
            \input{1 - low-noise PiC models/main.tex}
            \input{2 - kinetic component/main.tex}
            \input{3 - fluid component/main.tex}
            \input{4 - numerical implementation/main.tex}
        \part{Project Overview}
            \input{5 - research plan/main.tex}
            \input{6 - summary/main.tex}
    
    
    %\section{}
    \newpage
    \pagenumbering{gobble}
        \printbibliography


    \newpage
    \pagenumbering{roman}
    \appendix
        \part{Appendices}
            \input{8 - Hilbert complexes/main.tex}
            \input{9 - weak conservation proofs/main.tex}
\end{document}

            \documentclass[12pt, a4paper]{report}

\input{template/main.tex}

\title{\BA{Title in Progress...}}
\author{Boris Andrews}
\affil{Mathematical Institute, University of Oxford}
\date{\today}


\begin{document}
    \pagenumbering{gobble}
    \maketitle
    
    
    \begin{abstract}
        Magnetic confinement reactors---in particular tokamaks---offer one of the most promising options for achieving practical nuclear fusion, with the potential to provide virtually limitless, clean energy. The theoretical and numerical modeling of tokamak plasmas is simultaneously an essential component of effective reactor design, and a great research barrier. Tokamak operational conditions exhibit comparatively low Knudsen numbers. Kinetic effects, including kinetic waves and instabilities, Landau damping, bump-on-tail instabilities and more, are therefore highly influential in tokamak plasma dynamics. Purely fluid models are inherently incapable of capturing these effects, whereas the high dimensionality in purely kinetic models render them practically intractable for most relevant purposes.

        We consider a $\delta\!f$ decomposition model, with a macroscopic fluid background and microscopic kinetic correction, both fully coupled to each other. A similar manner of discretization is proposed to that used in the recent \texttt{STRUPHY} code \cite{Holderied_Possanner_Wang_2021, Holderied_2022, Li_et_al_2023} with a finite-element model for the background and a pseudo-particle/PiC model for the correction.

        The fluid background satisfies the full, non-linear, resistive, compressible, Hall MHD equations. \cite{Laakmann_Hu_Farrell_2022} introduces finite-element(-in-space) implicit timesteppers for the incompressible analogue to this system with structure-preserving (SP) properties in the ideal case, alongside parameter-robust preconditioners. We show that these timesteppers can derive from a finite-element-in-time (FET) (and finite-element-in-space) interpretation. The benefits of this reformulation are discussed, including the derivation of timesteppers that are higher order in time, and the quantifiable dissipative SP properties in the non-ideal, resistive case.
        
        We discuss possible options for extending this FET approach to timesteppers for the compressible case.

        The kinetic corrections satisfy linearized Boltzmann equations. Using a Lénard--Bernstein collision operator, these take Fokker--Planck-like forms \cite{Fokker_1914, Planck_1917} wherein pseudo-particles in the numerical model obey the neoclassical transport equations, with particle-independent Brownian drift terms. This offers a rigorous methodology for incorporating collisions into the particle transport model, without coupling the equations of motions for each particle.
        
        Works by Chen, Chacón et al. \cite{Chen_Chacón_Barnes_2011, Chacón_Chen_Barnes_2013, Chen_Chacón_2014, Chen_Chacón_2015} have developed structure-preserving particle pushers for neoclassical transport in the Vlasov equations, derived from Crank--Nicolson integrators. We show these too can can derive from a FET interpretation, similarly offering potential extensions to higher-order-in-time particle pushers. The FET formulation is used also to consider how the stochastic drift terms can be incorporated into the pushers. Stochastic gyrokinetic expansions are also discussed.

        Different options for the numerical implementation of these schemes are considered.

        Due to the efficacy of FET in the development of SP timesteppers for both the fluid and kinetic component, we hope this approach will prove effective in the future for developing SP timesteppers for the full hybrid model. We hope this will give us the opportunity to incorporate previously inaccessible kinetic effects into the highly effective, modern, finite-element MHD models.
    \end{abstract}
    
    
    \newpage
    \tableofcontents
    
    
    \newpage
    \pagenumbering{arabic}
    %\linenumbers\renewcommand\thelinenumber{\color{black!50}\arabic{linenumber}}
            \input{0 - introduction/main.tex}
        \part{Research}
            \input{1 - low-noise PiC models/main.tex}
            \input{2 - kinetic component/main.tex}
            \input{3 - fluid component/main.tex}
            \input{4 - numerical implementation/main.tex}
        \part{Project Overview}
            \input{5 - research plan/main.tex}
            \input{6 - summary/main.tex}
    
    
    %\section{}
    \newpage
    \pagenumbering{gobble}
        \printbibliography


    \newpage
    \pagenumbering{roman}
    \appendix
        \part{Appendices}
            \input{8 - Hilbert complexes/main.tex}
            \input{9 - weak conservation proofs/main.tex}
\end{document}

\end{document}

    
    
    %\section{}
    \newpage
    \pagenumbering{gobble}
        \printbibliography


    \newpage
    \pagenumbering{roman}
    \appendix
        \part{Appendices}
            \documentclass[12pt, a4paper]{report}

\documentclass[12pt, a4paper]{report}

\input{template/main.tex}

\title{\BA{Title in Progress...}}
\author{Boris Andrews}
\affil{Mathematical Institute, University of Oxford}
\date{\today}


\begin{document}
    \pagenumbering{gobble}
    \maketitle
    
    
    \begin{abstract}
        Magnetic confinement reactors---in particular tokamaks---offer one of the most promising options for achieving practical nuclear fusion, with the potential to provide virtually limitless, clean energy. The theoretical and numerical modeling of tokamak plasmas is simultaneously an essential component of effective reactor design, and a great research barrier. Tokamak operational conditions exhibit comparatively low Knudsen numbers. Kinetic effects, including kinetic waves and instabilities, Landau damping, bump-on-tail instabilities and more, are therefore highly influential in tokamak plasma dynamics. Purely fluid models are inherently incapable of capturing these effects, whereas the high dimensionality in purely kinetic models render them practically intractable for most relevant purposes.

        We consider a $\delta\!f$ decomposition model, with a macroscopic fluid background and microscopic kinetic correction, both fully coupled to each other. A similar manner of discretization is proposed to that used in the recent \texttt{STRUPHY} code \cite{Holderied_Possanner_Wang_2021, Holderied_2022, Li_et_al_2023} with a finite-element model for the background and a pseudo-particle/PiC model for the correction.

        The fluid background satisfies the full, non-linear, resistive, compressible, Hall MHD equations. \cite{Laakmann_Hu_Farrell_2022} introduces finite-element(-in-space) implicit timesteppers for the incompressible analogue to this system with structure-preserving (SP) properties in the ideal case, alongside parameter-robust preconditioners. We show that these timesteppers can derive from a finite-element-in-time (FET) (and finite-element-in-space) interpretation. The benefits of this reformulation are discussed, including the derivation of timesteppers that are higher order in time, and the quantifiable dissipative SP properties in the non-ideal, resistive case.
        
        We discuss possible options for extending this FET approach to timesteppers for the compressible case.

        The kinetic corrections satisfy linearized Boltzmann equations. Using a Lénard--Bernstein collision operator, these take Fokker--Planck-like forms \cite{Fokker_1914, Planck_1917} wherein pseudo-particles in the numerical model obey the neoclassical transport equations, with particle-independent Brownian drift terms. This offers a rigorous methodology for incorporating collisions into the particle transport model, without coupling the equations of motions for each particle.
        
        Works by Chen, Chacón et al. \cite{Chen_Chacón_Barnes_2011, Chacón_Chen_Barnes_2013, Chen_Chacón_2014, Chen_Chacón_2015} have developed structure-preserving particle pushers for neoclassical transport in the Vlasov equations, derived from Crank--Nicolson integrators. We show these too can can derive from a FET interpretation, similarly offering potential extensions to higher-order-in-time particle pushers. The FET formulation is used also to consider how the stochastic drift terms can be incorporated into the pushers. Stochastic gyrokinetic expansions are also discussed.

        Different options for the numerical implementation of these schemes are considered.

        Due to the efficacy of FET in the development of SP timesteppers for both the fluid and kinetic component, we hope this approach will prove effective in the future for developing SP timesteppers for the full hybrid model. We hope this will give us the opportunity to incorporate previously inaccessible kinetic effects into the highly effective, modern, finite-element MHD models.
    \end{abstract}
    
    
    \newpage
    \tableofcontents
    
    
    \newpage
    \pagenumbering{arabic}
    %\linenumbers\renewcommand\thelinenumber{\color{black!50}\arabic{linenumber}}
            \input{0 - introduction/main.tex}
        \part{Research}
            \input{1 - low-noise PiC models/main.tex}
            \input{2 - kinetic component/main.tex}
            \input{3 - fluid component/main.tex}
            \input{4 - numerical implementation/main.tex}
        \part{Project Overview}
            \input{5 - research plan/main.tex}
            \input{6 - summary/main.tex}
    
    
    %\section{}
    \newpage
    \pagenumbering{gobble}
        \printbibliography


    \newpage
    \pagenumbering{roman}
    \appendix
        \part{Appendices}
            \input{8 - Hilbert complexes/main.tex}
            \input{9 - weak conservation proofs/main.tex}
\end{document}


\title{\BA{Title in Progress...}}
\author{Boris Andrews}
\affil{Mathematical Institute, University of Oxford}
\date{\today}


\begin{document}
    \pagenumbering{gobble}
    \maketitle
    
    
    \begin{abstract}
        Magnetic confinement reactors---in particular tokamaks---offer one of the most promising options for achieving practical nuclear fusion, with the potential to provide virtually limitless, clean energy. The theoretical and numerical modeling of tokamak plasmas is simultaneously an essential component of effective reactor design, and a great research barrier. Tokamak operational conditions exhibit comparatively low Knudsen numbers. Kinetic effects, including kinetic waves and instabilities, Landau damping, bump-on-tail instabilities and more, are therefore highly influential in tokamak plasma dynamics. Purely fluid models are inherently incapable of capturing these effects, whereas the high dimensionality in purely kinetic models render them practically intractable for most relevant purposes.

        We consider a $\delta\!f$ decomposition model, with a macroscopic fluid background and microscopic kinetic correction, both fully coupled to each other. A similar manner of discretization is proposed to that used in the recent \texttt{STRUPHY} code \cite{Holderied_Possanner_Wang_2021, Holderied_2022, Li_et_al_2023} with a finite-element model for the background and a pseudo-particle/PiC model for the correction.

        The fluid background satisfies the full, non-linear, resistive, compressible, Hall MHD equations. \cite{Laakmann_Hu_Farrell_2022} introduces finite-element(-in-space) implicit timesteppers for the incompressible analogue to this system with structure-preserving (SP) properties in the ideal case, alongside parameter-robust preconditioners. We show that these timesteppers can derive from a finite-element-in-time (FET) (and finite-element-in-space) interpretation. The benefits of this reformulation are discussed, including the derivation of timesteppers that are higher order in time, and the quantifiable dissipative SP properties in the non-ideal, resistive case.
        
        We discuss possible options for extending this FET approach to timesteppers for the compressible case.

        The kinetic corrections satisfy linearized Boltzmann equations. Using a Lénard--Bernstein collision operator, these take Fokker--Planck-like forms \cite{Fokker_1914, Planck_1917} wherein pseudo-particles in the numerical model obey the neoclassical transport equations, with particle-independent Brownian drift terms. This offers a rigorous methodology for incorporating collisions into the particle transport model, without coupling the equations of motions for each particle.
        
        Works by Chen, Chacón et al. \cite{Chen_Chacón_Barnes_2011, Chacón_Chen_Barnes_2013, Chen_Chacón_2014, Chen_Chacón_2015} have developed structure-preserving particle pushers for neoclassical transport in the Vlasov equations, derived from Crank--Nicolson integrators. We show these too can can derive from a FET interpretation, similarly offering potential extensions to higher-order-in-time particle pushers. The FET formulation is used also to consider how the stochastic drift terms can be incorporated into the pushers. Stochastic gyrokinetic expansions are also discussed.

        Different options for the numerical implementation of these schemes are considered.

        Due to the efficacy of FET in the development of SP timesteppers for both the fluid and kinetic component, we hope this approach will prove effective in the future for developing SP timesteppers for the full hybrid model. We hope this will give us the opportunity to incorporate previously inaccessible kinetic effects into the highly effective, modern, finite-element MHD models.
    \end{abstract}
    
    
    \newpage
    \tableofcontents
    
    
    \newpage
    \pagenumbering{arabic}
    %\linenumbers\renewcommand\thelinenumber{\color{black!50}\arabic{linenumber}}
            \documentclass[12pt, a4paper]{report}

\input{template/main.tex}

\title{\BA{Title in Progress...}}
\author{Boris Andrews}
\affil{Mathematical Institute, University of Oxford}
\date{\today}


\begin{document}
    \pagenumbering{gobble}
    \maketitle
    
    
    \begin{abstract}
        Magnetic confinement reactors---in particular tokamaks---offer one of the most promising options for achieving practical nuclear fusion, with the potential to provide virtually limitless, clean energy. The theoretical and numerical modeling of tokamak plasmas is simultaneously an essential component of effective reactor design, and a great research barrier. Tokamak operational conditions exhibit comparatively low Knudsen numbers. Kinetic effects, including kinetic waves and instabilities, Landau damping, bump-on-tail instabilities and more, are therefore highly influential in tokamak plasma dynamics. Purely fluid models are inherently incapable of capturing these effects, whereas the high dimensionality in purely kinetic models render them practically intractable for most relevant purposes.

        We consider a $\delta\!f$ decomposition model, with a macroscopic fluid background and microscopic kinetic correction, both fully coupled to each other. A similar manner of discretization is proposed to that used in the recent \texttt{STRUPHY} code \cite{Holderied_Possanner_Wang_2021, Holderied_2022, Li_et_al_2023} with a finite-element model for the background and a pseudo-particle/PiC model for the correction.

        The fluid background satisfies the full, non-linear, resistive, compressible, Hall MHD equations. \cite{Laakmann_Hu_Farrell_2022} introduces finite-element(-in-space) implicit timesteppers for the incompressible analogue to this system with structure-preserving (SP) properties in the ideal case, alongside parameter-robust preconditioners. We show that these timesteppers can derive from a finite-element-in-time (FET) (and finite-element-in-space) interpretation. The benefits of this reformulation are discussed, including the derivation of timesteppers that are higher order in time, and the quantifiable dissipative SP properties in the non-ideal, resistive case.
        
        We discuss possible options for extending this FET approach to timesteppers for the compressible case.

        The kinetic corrections satisfy linearized Boltzmann equations. Using a Lénard--Bernstein collision operator, these take Fokker--Planck-like forms \cite{Fokker_1914, Planck_1917} wherein pseudo-particles in the numerical model obey the neoclassical transport equations, with particle-independent Brownian drift terms. This offers a rigorous methodology for incorporating collisions into the particle transport model, without coupling the equations of motions for each particle.
        
        Works by Chen, Chacón et al. \cite{Chen_Chacón_Barnes_2011, Chacón_Chen_Barnes_2013, Chen_Chacón_2014, Chen_Chacón_2015} have developed structure-preserving particle pushers for neoclassical transport in the Vlasov equations, derived from Crank--Nicolson integrators. We show these too can can derive from a FET interpretation, similarly offering potential extensions to higher-order-in-time particle pushers. The FET formulation is used also to consider how the stochastic drift terms can be incorporated into the pushers. Stochastic gyrokinetic expansions are also discussed.

        Different options for the numerical implementation of these schemes are considered.

        Due to the efficacy of FET in the development of SP timesteppers for both the fluid and kinetic component, we hope this approach will prove effective in the future for developing SP timesteppers for the full hybrid model. We hope this will give us the opportunity to incorporate previously inaccessible kinetic effects into the highly effective, modern, finite-element MHD models.
    \end{abstract}
    
    
    \newpage
    \tableofcontents
    
    
    \newpage
    \pagenumbering{arabic}
    %\linenumbers\renewcommand\thelinenumber{\color{black!50}\arabic{linenumber}}
            \input{0 - introduction/main.tex}
        \part{Research}
            \input{1 - low-noise PiC models/main.tex}
            \input{2 - kinetic component/main.tex}
            \input{3 - fluid component/main.tex}
            \input{4 - numerical implementation/main.tex}
        \part{Project Overview}
            \input{5 - research plan/main.tex}
            \input{6 - summary/main.tex}
    
    
    %\section{}
    \newpage
    \pagenumbering{gobble}
        \printbibliography


    \newpage
    \pagenumbering{roman}
    \appendix
        \part{Appendices}
            \input{8 - Hilbert complexes/main.tex}
            \input{9 - weak conservation proofs/main.tex}
\end{document}

        \part{Research}
            \documentclass[12pt, a4paper]{report}

\input{template/main.tex}

\title{\BA{Title in Progress...}}
\author{Boris Andrews}
\affil{Mathematical Institute, University of Oxford}
\date{\today}


\begin{document}
    \pagenumbering{gobble}
    \maketitle
    
    
    \begin{abstract}
        Magnetic confinement reactors---in particular tokamaks---offer one of the most promising options for achieving practical nuclear fusion, with the potential to provide virtually limitless, clean energy. The theoretical and numerical modeling of tokamak plasmas is simultaneously an essential component of effective reactor design, and a great research barrier. Tokamak operational conditions exhibit comparatively low Knudsen numbers. Kinetic effects, including kinetic waves and instabilities, Landau damping, bump-on-tail instabilities and more, are therefore highly influential in tokamak plasma dynamics. Purely fluid models are inherently incapable of capturing these effects, whereas the high dimensionality in purely kinetic models render them practically intractable for most relevant purposes.

        We consider a $\delta\!f$ decomposition model, with a macroscopic fluid background and microscopic kinetic correction, both fully coupled to each other. A similar manner of discretization is proposed to that used in the recent \texttt{STRUPHY} code \cite{Holderied_Possanner_Wang_2021, Holderied_2022, Li_et_al_2023} with a finite-element model for the background and a pseudo-particle/PiC model for the correction.

        The fluid background satisfies the full, non-linear, resistive, compressible, Hall MHD equations. \cite{Laakmann_Hu_Farrell_2022} introduces finite-element(-in-space) implicit timesteppers for the incompressible analogue to this system with structure-preserving (SP) properties in the ideal case, alongside parameter-robust preconditioners. We show that these timesteppers can derive from a finite-element-in-time (FET) (and finite-element-in-space) interpretation. The benefits of this reformulation are discussed, including the derivation of timesteppers that are higher order in time, and the quantifiable dissipative SP properties in the non-ideal, resistive case.
        
        We discuss possible options for extending this FET approach to timesteppers for the compressible case.

        The kinetic corrections satisfy linearized Boltzmann equations. Using a Lénard--Bernstein collision operator, these take Fokker--Planck-like forms \cite{Fokker_1914, Planck_1917} wherein pseudo-particles in the numerical model obey the neoclassical transport equations, with particle-independent Brownian drift terms. This offers a rigorous methodology for incorporating collisions into the particle transport model, without coupling the equations of motions for each particle.
        
        Works by Chen, Chacón et al. \cite{Chen_Chacón_Barnes_2011, Chacón_Chen_Barnes_2013, Chen_Chacón_2014, Chen_Chacón_2015} have developed structure-preserving particle pushers for neoclassical transport in the Vlasov equations, derived from Crank--Nicolson integrators. We show these too can can derive from a FET interpretation, similarly offering potential extensions to higher-order-in-time particle pushers. The FET formulation is used also to consider how the stochastic drift terms can be incorporated into the pushers. Stochastic gyrokinetic expansions are also discussed.

        Different options for the numerical implementation of these schemes are considered.

        Due to the efficacy of FET in the development of SP timesteppers for both the fluid and kinetic component, we hope this approach will prove effective in the future for developing SP timesteppers for the full hybrid model. We hope this will give us the opportunity to incorporate previously inaccessible kinetic effects into the highly effective, modern, finite-element MHD models.
    \end{abstract}
    
    
    \newpage
    \tableofcontents
    
    
    \newpage
    \pagenumbering{arabic}
    %\linenumbers\renewcommand\thelinenumber{\color{black!50}\arabic{linenumber}}
            \input{0 - introduction/main.tex}
        \part{Research}
            \input{1 - low-noise PiC models/main.tex}
            \input{2 - kinetic component/main.tex}
            \input{3 - fluid component/main.tex}
            \input{4 - numerical implementation/main.tex}
        \part{Project Overview}
            \input{5 - research plan/main.tex}
            \input{6 - summary/main.tex}
    
    
    %\section{}
    \newpage
    \pagenumbering{gobble}
        \printbibliography


    \newpage
    \pagenumbering{roman}
    \appendix
        \part{Appendices}
            \input{8 - Hilbert complexes/main.tex}
            \input{9 - weak conservation proofs/main.tex}
\end{document}

            \documentclass[12pt, a4paper]{report}

\input{template/main.tex}

\title{\BA{Title in Progress...}}
\author{Boris Andrews}
\affil{Mathematical Institute, University of Oxford}
\date{\today}


\begin{document}
    \pagenumbering{gobble}
    \maketitle
    
    
    \begin{abstract}
        Magnetic confinement reactors---in particular tokamaks---offer one of the most promising options for achieving practical nuclear fusion, with the potential to provide virtually limitless, clean energy. The theoretical and numerical modeling of tokamak plasmas is simultaneously an essential component of effective reactor design, and a great research barrier. Tokamak operational conditions exhibit comparatively low Knudsen numbers. Kinetic effects, including kinetic waves and instabilities, Landau damping, bump-on-tail instabilities and more, are therefore highly influential in tokamak plasma dynamics. Purely fluid models are inherently incapable of capturing these effects, whereas the high dimensionality in purely kinetic models render them practically intractable for most relevant purposes.

        We consider a $\delta\!f$ decomposition model, with a macroscopic fluid background and microscopic kinetic correction, both fully coupled to each other. A similar manner of discretization is proposed to that used in the recent \texttt{STRUPHY} code \cite{Holderied_Possanner_Wang_2021, Holderied_2022, Li_et_al_2023} with a finite-element model for the background and a pseudo-particle/PiC model for the correction.

        The fluid background satisfies the full, non-linear, resistive, compressible, Hall MHD equations. \cite{Laakmann_Hu_Farrell_2022} introduces finite-element(-in-space) implicit timesteppers for the incompressible analogue to this system with structure-preserving (SP) properties in the ideal case, alongside parameter-robust preconditioners. We show that these timesteppers can derive from a finite-element-in-time (FET) (and finite-element-in-space) interpretation. The benefits of this reformulation are discussed, including the derivation of timesteppers that are higher order in time, and the quantifiable dissipative SP properties in the non-ideal, resistive case.
        
        We discuss possible options for extending this FET approach to timesteppers for the compressible case.

        The kinetic corrections satisfy linearized Boltzmann equations. Using a Lénard--Bernstein collision operator, these take Fokker--Planck-like forms \cite{Fokker_1914, Planck_1917} wherein pseudo-particles in the numerical model obey the neoclassical transport equations, with particle-independent Brownian drift terms. This offers a rigorous methodology for incorporating collisions into the particle transport model, without coupling the equations of motions for each particle.
        
        Works by Chen, Chacón et al. \cite{Chen_Chacón_Barnes_2011, Chacón_Chen_Barnes_2013, Chen_Chacón_2014, Chen_Chacón_2015} have developed structure-preserving particle pushers for neoclassical transport in the Vlasov equations, derived from Crank--Nicolson integrators. We show these too can can derive from a FET interpretation, similarly offering potential extensions to higher-order-in-time particle pushers. The FET formulation is used also to consider how the stochastic drift terms can be incorporated into the pushers. Stochastic gyrokinetic expansions are also discussed.

        Different options for the numerical implementation of these schemes are considered.

        Due to the efficacy of FET in the development of SP timesteppers for both the fluid and kinetic component, we hope this approach will prove effective in the future for developing SP timesteppers for the full hybrid model. We hope this will give us the opportunity to incorporate previously inaccessible kinetic effects into the highly effective, modern, finite-element MHD models.
    \end{abstract}
    
    
    \newpage
    \tableofcontents
    
    
    \newpage
    \pagenumbering{arabic}
    %\linenumbers\renewcommand\thelinenumber{\color{black!50}\arabic{linenumber}}
            \input{0 - introduction/main.tex}
        \part{Research}
            \input{1 - low-noise PiC models/main.tex}
            \input{2 - kinetic component/main.tex}
            \input{3 - fluid component/main.tex}
            \input{4 - numerical implementation/main.tex}
        \part{Project Overview}
            \input{5 - research plan/main.tex}
            \input{6 - summary/main.tex}
    
    
    %\section{}
    \newpage
    \pagenumbering{gobble}
        \printbibliography


    \newpage
    \pagenumbering{roman}
    \appendix
        \part{Appendices}
            \input{8 - Hilbert complexes/main.tex}
            \input{9 - weak conservation proofs/main.tex}
\end{document}

            \documentclass[12pt, a4paper]{report}

\input{template/main.tex}

\title{\BA{Title in Progress...}}
\author{Boris Andrews}
\affil{Mathematical Institute, University of Oxford}
\date{\today}


\begin{document}
    \pagenumbering{gobble}
    \maketitle
    
    
    \begin{abstract}
        Magnetic confinement reactors---in particular tokamaks---offer one of the most promising options for achieving practical nuclear fusion, with the potential to provide virtually limitless, clean energy. The theoretical and numerical modeling of tokamak plasmas is simultaneously an essential component of effective reactor design, and a great research barrier. Tokamak operational conditions exhibit comparatively low Knudsen numbers. Kinetic effects, including kinetic waves and instabilities, Landau damping, bump-on-tail instabilities and more, are therefore highly influential in tokamak plasma dynamics. Purely fluid models are inherently incapable of capturing these effects, whereas the high dimensionality in purely kinetic models render them practically intractable for most relevant purposes.

        We consider a $\delta\!f$ decomposition model, with a macroscopic fluid background and microscopic kinetic correction, both fully coupled to each other. A similar manner of discretization is proposed to that used in the recent \texttt{STRUPHY} code \cite{Holderied_Possanner_Wang_2021, Holderied_2022, Li_et_al_2023} with a finite-element model for the background and a pseudo-particle/PiC model for the correction.

        The fluid background satisfies the full, non-linear, resistive, compressible, Hall MHD equations. \cite{Laakmann_Hu_Farrell_2022} introduces finite-element(-in-space) implicit timesteppers for the incompressible analogue to this system with structure-preserving (SP) properties in the ideal case, alongside parameter-robust preconditioners. We show that these timesteppers can derive from a finite-element-in-time (FET) (and finite-element-in-space) interpretation. The benefits of this reformulation are discussed, including the derivation of timesteppers that are higher order in time, and the quantifiable dissipative SP properties in the non-ideal, resistive case.
        
        We discuss possible options for extending this FET approach to timesteppers for the compressible case.

        The kinetic corrections satisfy linearized Boltzmann equations. Using a Lénard--Bernstein collision operator, these take Fokker--Planck-like forms \cite{Fokker_1914, Planck_1917} wherein pseudo-particles in the numerical model obey the neoclassical transport equations, with particle-independent Brownian drift terms. This offers a rigorous methodology for incorporating collisions into the particle transport model, without coupling the equations of motions for each particle.
        
        Works by Chen, Chacón et al. \cite{Chen_Chacón_Barnes_2011, Chacón_Chen_Barnes_2013, Chen_Chacón_2014, Chen_Chacón_2015} have developed structure-preserving particle pushers for neoclassical transport in the Vlasov equations, derived from Crank--Nicolson integrators. We show these too can can derive from a FET interpretation, similarly offering potential extensions to higher-order-in-time particle pushers. The FET formulation is used also to consider how the stochastic drift terms can be incorporated into the pushers. Stochastic gyrokinetic expansions are also discussed.

        Different options for the numerical implementation of these schemes are considered.

        Due to the efficacy of FET in the development of SP timesteppers for both the fluid and kinetic component, we hope this approach will prove effective in the future for developing SP timesteppers for the full hybrid model. We hope this will give us the opportunity to incorporate previously inaccessible kinetic effects into the highly effective, modern, finite-element MHD models.
    \end{abstract}
    
    
    \newpage
    \tableofcontents
    
    
    \newpage
    \pagenumbering{arabic}
    %\linenumbers\renewcommand\thelinenumber{\color{black!50}\arabic{linenumber}}
            \input{0 - introduction/main.tex}
        \part{Research}
            \input{1 - low-noise PiC models/main.tex}
            \input{2 - kinetic component/main.tex}
            \input{3 - fluid component/main.tex}
            \input{4 - numerical implementation/main.tex}
        \part{Project Overview}
            \input{5 - research plan/main.tex}
            \input{6 - summary/main.tex}
    
    
    %\section{}
    \newpage
    \pagenumbering{gobble}
        \printbibliography


    \newpage
    \pagenumbering{roman}
    \appendix
        \part{Appendices}
            \input{8 - Hilbert complexes/main.tex}
            \input{9 - weak conservation proofs/main.tex}
\end{document}

            \documentclass[12pt, a4paper]{report}

\input{template/main.tex}

\title{\BA{Title in Progress...}}
\author{Boris Andrews}
\affil{Mathematical Institute, University of Oxford}
\date{\today}


\begin{document}
    \pagenumbering{gobble}
    \maketitle
    
    
    \begin{abstract}
        Magnetic confinement reactors---in particular tokamaks---offer one of the most promising options for achieving practical nuclear fusion, with the potential to provide virtually limitless, clean energy. The theoretical and numerical modeling of tokamak plasmas is simultaneously an essential component of effective reactor design, and a great research barrier. Tokamak operational conditions exhibit comparatively low Knudsen numbers. Kinetic effects, including kinetic waves and instabilities, Landau damping, bump-on-tail instabilities and more, are therefore highly influential in tokamak plasma dynamics. Purely fluid models are inherently incapable of capturing these effects, whereas the high dimensionality in purely kinetic models render them practically intractable for most relevant purposes.

        We consider a $\delta\!f$ decomposition model, with a macroscopic fluid background and microscopic kinetic correction, both fully coupled to each other. A similar manner of discretization is proposed to that used in the recent \texttt{STRUPHY} code \cite{Holderied_Possanner_Wang_2021, Holderied_2022, Li_et_al_2023} with a finite-element model for the background and a pseudo-particle/PiC model for the correction.

        The fluid background satisfies the full, non-linear, resistive, compressible, Hall MHD equations. \cite{Laakmann_Hu_Farrell_2022} introduces finite-element(-in-space) implicit timesteppers for the incompressible analogue to this system with structure-preserving (SP) properties in the ideal case, alongside parameter-robust preconditioners. We show that these timesteppers can derive from a finite-element-in-time (FET) (and finite-element-in-space) interpretation. The benefits of this reformulation are discussed, including the derivation of timesteppers that are higher order in time, and the quantifiable dissipative SP properties in the non-ideal, resistive case.
        
        We discuss possible options for extending this FET approach to timesteppers for the compressible case.

        The kinetic corrections satisfy linearized Boltzmann equations. Using a Lénard--Bernstein collision operator, these take Fokker--Planck-like forms \cite{Fokker_1914, Planck_1917} wherein pseudo-particles in the numerical model obey the neoclassical transport equations, with particle-independent Brownian drift terms. This offers a rigorous methodology for incorporating collisions into the particle transport model, without coupling the equations of motions for each particle.
        
        Works by Chen, Chacón et al. \cite{Chen_Chacón_Barnes_2011, Chacón_Chen_Barnes_2013, Chen_Chacón_2014, Chen_Chacón_2015} have developed structure-preserving particle pushers for neoclassical transport in the Vlasov equations, derived from Crank--Nicolson integrators. We show these too can can derive from a FET interpretation, similarly offering potential extensions to higher-order-in-time particle pushers. The FET formulation is used also to consider how the stochastic drift terms can be incorporated into the pushers. Stochastic gyrokinetic expansions are also discussed.

        Different options for the numerical implementation of these schemes are considered.

        Due to the efficacy of FET in the development of SP timesteppers for both the fluid and kinetic component, we hope this approach will prove effective in the future for developing SP timesteppers for the full hybrid model. We hope this will give us the opportunity to incorporate previously inaccessible kinetic effects into the highly effective, modern, finite-element MHD models.
    \end{abstract}
    
    
    \newpage
    \tableofcontents
    
    
    \newpage
    \pagenumbering{arabic}
    %\linenumbers\renewcommand\thelinenumber{\color{black!50}\arabic{linenumber}}
            \input{0 - introduction/main.tex}
        \part{Research}
            \input{1 - low-noise PiC models/main.tex}
            \input{2 - kinetic component/main.tex}
            \input{3 - fluid component/main.tex}
            \input{4 - numerical implementation/main.tex}
        \part{Project Overview}
            \input{5 - research plan/main.tex}
            \input{6 - summary/main.tex}
    
    
    %\section{}
    \newpage
    \pagenumbering{gobble}
        \printbibliography


    \newpage
    \pagenumbering{roman}
    \appendix
        \part{Appendices}
            \input{8 - Hilbert complexes/main.tex}
            \input{9 - weak conservation proofs/main.tex}
\end{document}

        \part{Project Overview}
            \documentclass[12pt, a4paper]{report}

\input{template/main.tex}

\title{\BA{Title in Progress...}}
\author{Boris Andrews}
\affil{Mathematical Institute, University of Oxford}
\date{\today}


\begin{document}
    \pagenumbering{gobble}
    \maketitle
    
    
    \begin{abstract}
        Magnetic confinement reactors---in particular tokamaks---offer one of the most promising options for achieving practical nuclear fusion, with the potential to provide virtually limitless, clean energy. The theoretical and numerical modeling of tokamak plasmas is simultaneously an essential component of effective reactor design, and a great research barrier. Tokamak operational conditions exhibit comparatively low Knudsen numbers. Kinetic effects, including kinetic waves and instabilities, Landau damping, bump-on-tail instabilities and more, are therefore highly influential in tokamak plasma dynamics. Purely fluid models are inherently incapable of capturing these effects, whereas the high dimensionality in purely kinetic models render them practically intractable for most relevant purposes.

        We consider a $\delta\!f$ decomposition model, with a macroscopic fluid background and microscopic kinetic correction, both fully coupled to each other. A similar manner of discretization is proposed to that used in the recent \texttt{STRUPHY} code \cite{Holderied_Possanner_Wang_2021, Holderied_2022, Li_et_al_2023} with a finite-element model for the background and a pseudo-particle/PiC model for the correction.

        The fluid background satisfies the full, non-linear, resistive, compressible, Hall MHD equations. \cite{Laakmann_Hu_Farrell_2022} introduces finite-element(-in-space) implicit timesteppers for the incompressible analogue to this system with structure-preserving (SP) properties in the ideal case, alongside parameter-robust preconditioners. We show that these timesteppers can derive from a finite-element-in-time (FET) (and finite-element-in-space) interpretation. The benefits of this reformulation are discussed, including the derivation of timesteppers that are higher order in time, and the quantifiable dissipative SP properties in the non-ideal, resistive case.
        
        We discuss possible options for extending this FET approach to timesteppers for the compressible case.

        The kinetic corrections satisfy linearized Boltzmann equations. Using a Lénard--Bernstein collision operator, these take Fokker--Planck-like forms \cite{Fokker_1914, Planck_1917} wherein pseudo-particles in the numerical model obey the neoclassical transport equations, with particle-independent Brownian drift terms. This offers a rigorous methodology for incorporating collisions into the particle transport model, without coupling the equations of motions for each particle.
        
        Works by Chen, Chacón et al. \cite{Chen_Chacón_Barnes_2011, Chacón_Chen_Barnes_2013, Chen_Chacón_2014, Chen_Chacón_2015} have developed structure-preserving particle pushers for neoclassical transport in the Vlasov equations, derived from Crank--Nicolson integrators. We show these too can can derive from a FET interpretation, similarly offering potential extensions to higher-order-in-time particle pushers. The FET formulation is used also to consider how the stochastic drift terms can be incorporated into the pushers. Stochastic gyrokinetic expansions are also discussed.

        Different options for the numerical implementation of these schemes are considered.

        Due to the efficacy of FET in the development of SP timesteppers for both the fluid and kinetic component, we hope this approach will prove effective in the future for developing SP timesteppers for the full hybrid model. We hope this will give us the opportunity to incorporate previously inaccessible kinetic effects into the highly effective, modern, finite-element MHD models.
    \end{abstract}
    
    
    \newpage
    \tableofcontents
    
    
    \newpage
    \pagenumbering{arabic}
    %\linenumbers\renewcommand\thelinenumber{\color{black!50}\arabic{linenumber}}
            \input{0 - introduction/main.tex}
        \part{Research}
            \input{1 - low-noise PiC models/main.tex}
            \input{2 - kinetic component/main.tex}
            \input{3 - fluid component/main.tex}
            \input{4 - numerical implementation/main.tex}
        \part{Project Overview}
            \input{5 - research plan/main.tex}
            \input{6 - summary/main.tex}
    
    
    %\section{}
    \newpage
    \pagenumbering{gobble}
        \printbibliography


    \newpage
    \pagenumbering{roman}
    \appendix
        \part{Appendices}
            \input{8 - Hilbert complexes/main.tex}
            \input{9 - weak conservation proofs/main.tex}
\end{document}

            \documentclass[12pt, a4paper]{report}

\input{template/main.tex}

\title{\BA{Title in Progress...}}
\author{Boris Andrews}
\affil{Mathematical Institute, University of Oxford}
\date{\today}


\begin{document}
    \pagenumbering{gobble}
    \maketitle
    
    
    \begin{abstract}
        Magnetic confinement reactors---in particular tokamaks---offer one of the most promising options for achieving practical nuclear fusion, with the potential to provide virtually limitless, clean energy. The theoretical and numerical modeling of tokamak plasmas is simultaneously an essential component of effective reactor design, and a great research barrier. Tokamak operational conditions exhibit comparatively low Knudsen numbers. Kinetic effects, including kinetic waves and instabilities, Landau damping, bump-on-tail instabilities and more, are therefore highly influential in tokamak plasma dynamics. Purely fluid models are inherently incapable of capturing these effects, whereas the high dimensionality in purely kinetic models render them practically intractable for most relevant purposes.

        We consider a $\delta\!f$ decomposition model, with a macroscopic fluid background and microscopic kinetic correction, both fully coupled to each other. A similar manner of discretization is proposed to that used in the recent \texttt{STRUPHY} code \cite{Holderied_Possanner_Wang_2021, Holderied_2022, Li_et_al_2023} with a finite-element model for the background and a pseudo-particle/PiC model for the correction.

        The fluid background satisfies the full, non-linear, resistive, compressible, Hall MHD equations. \cite{Laakmann_Hu_Farrell_2022} introduces finite-element(-in-space) implicit timesteppers for the incompressible analogue to this system with structure-preserving (SP) properties in the ideal case, alongside parameter-robust preconditioners. We show that these timesteppers can derive from a finite-element-in-time (FET) (and finite-element-in-space) interpretation. The benefits of this reformulation are discussed, including the derivation of timesteppers that are higher order in time, and the quantifiable dissipative SP properties in the non-ideal, resistive case.
        
        We discuss possible options for extending this FET approach to timesteppers for the compressible case.

        The kinetic corrections satisfy linearized Boltzmann equations. Using a Lénard--Bernstein collision operator, these take Fokker--Planck-like forms \cite{Fokker_1914, Planck_1917} wherein pseudo-particles in the numerical model obey the neoclassical transport equations, with particle-independent Brownian drift terms. This offers a rigorous methodology for incorporating collisions into the particle transport model, without coupling the equations of motions for each particle.
        
        Works by Chen, Chacón et al. \cite{Chen_Chacón_Barnes_2011, Chacón_Chen_Barnes_2013, Chen_Chacón_2014, Chen_Chacón_2015} have developed structure-preserving particle pushers for neoclassical transport in the Vlasov equations, derived from Crank--Nicolson integrators. We show these too can can derive from a FET interpretation, similarly offering potential extensions to higher-order-in-time particle pushers. The FET formulation is used also to consider how the stochastic drift terms can be incorporated into the pushers. Stochastic gyrokinetic expansions are also discussed.

        Different options for the numerical implementation of these schemes are considered.

        Due to the efficacy of FET in the development of SP timesteppers for both the fluid and kinetic component, we hope this approach will prove effective in the future for developing SP timesteppers for the full hybrid model. We hope this will give us the opportunity to incorporate previously inaccessible kinetic effects into the highly effective, modern, finite-element MHD models.
    \end{abstract}
    
    
    \newpage
    \tableofcontents
    
    
    \newpage
    \pagenumbering{arabic}
    %\linenumbers\renewcommand\thelinenumber{\color{black!50}\arabic{linenumber}}
            \input{0 - introduction/main.tex}
        \part{Research}
            \input{1 - low-noise PiC models/main.tex}
            \input{2 - kinetic component/main.tex}
            \input{3 - fluid component/main.tex}
            \input{4 - numerical implementation/main.tex}
        \part{Project Overview}
            \input{5 - research plan/main.tex}
            \input{6 - summary/main.tex}
    
    
    %\section{}
    \newpage
    \pagenumbering{gobble}
        \printbibliography


    \newpage
    \pagenumbering{roman}
    \appendix
        \part{Appendices}
            \input{8 - Hilbert complexes/main.tex}
            \input{9 - weak conservation proofs/main.tex}
\end{document}

    
    
    %\section{}
    \newpage
    \pagenumbering{gobble}
        \printbibliography


    \newpage
    \pagenumbering{roman}
    \appendix
        \part{Appendices}
            \documentclass[12pt, a4paper]{report}

\input{template/main.tex}

\title{\BA{Title in Progress...}}
\author{Boris Andrews}
\affil{Mathematical Institute, University of Oxford}
\date{\today}


\begin{document}
    \pagenumbering{gobble}
    \maketitle
    
    
    \begin{abstract}
        Magnetic confinement reactors---in particular tokamaks---offer one of the most promising options for achieving practical nuclear fusion, with the potential to provide virtually limitless, clean energy. The theoretical and numerical modeling of tokamak plasmas is simultaneously an essential component of effective reactor design, and a great research barrier. Tokamak operational conditions exhibit comparatively low Knudsen numbers. Kinetic effects, including kinetic waves and instabilities, Landau damping, bump-on-tail instabilities and more, are therefore highly influential in tokamak plasma dynamics. Purely fluid models are inherently incapable of capturing these effects, whereas the high dimensionality in purely kinetic models render them practically intractable for most relevant purposes.

        We consider a $\delta\!f$ decomposition model, with a macroscopic fluid background and microscopic kinetic correction, both fully coupled to each other. A similar manner of discretization is proposed to that used in the recent \texttt{STRUPHY} code \cite{Holderied_Possanner_Wang_2021, Holderied_2022, Li_et_al_2023} with a finite-element model for the background and a pseudo-particle/PiC model for the correction.

        The fluid background satisfies the full, non-linear, resistive, compressible, Hall MHD equations. \cite{Laakmann_Hu_Farrell_2022} introduces finite-element(-in-space) implicit timesteppers for the incompressible analogue to this system with structure-preserving (SP) properties in the ideal case, alongside parameter-robust preconditioners. We show that these timesteppers can derive from a finite-element-in-time (FET) (and finite-element-in-space) interpretation. The benefits of this reformulation are discussed, including the derivation of timesteppers that are higher order in time, and the quantifiable dissipative SP properties in the non-ideal, resistive case.
        
        We discuss possible options for extending this FET approach to timesteppers for the compressible case.

        The kinetic corrections satisfy linearized Boltzmann equations. Using a Lénard--Bernstein collision operator, these take Fokker--Planck-like forms \cite{Fokker_1914, Planck_1917} wherein pseudo-particles in the numerical model obey the neoclassical transport equations, with particle-independent Brownian drift terms. This offers a rigorous methodology for incorporating collisions into the particle transport model, without coupling the equations of motions for each particle.
        
        Works by Chen, Chacón et al. \cite{Chen_Chacón_Barnes_2011, Chacón_Chen_Barnes_2013, Chen_Chacón_2014, Chen_Chacón_2015} have developed structure-preserving particle pushers for neoclassical transport in the Vlasov equations, derived from Crank--Nicolson integrators. We show these too can can derive from a FET interpretation, similarly offering potential extensions to higher-order-in-time particle pushers. The FET formulation is used also to consider how the stochastic drift terms can be incorporated into the pushers. Stochastic gyrokinetic expansions are also discussed.

        Different options for the numerical implementation of these schemes are considered.

        Due to the efficacy of FET in the development of SP timesteppers for both the fluid and kinetic component, we hope this approach will prove effective in the future for developing SP timesteppers for the full hybrid model. We hope this will give us the opportunity to incorporate previously inaccessible kinetic effects into the highly effective, modern, finite-element MHD models.
    \end{abstract}
    
    
    \newpage
    \tableofcontents
    
    
    \newpage
    \pagenumbering{arabic}
    %\linenumbers\renewcommand\thelinenumber{\color{black!50}\arabic{linenumber}}
            \input{0 - introduction/main.tex}
        \part{Research}
            \input{1 - low-noise PiC models/main.tex}
            \input{2 - kinetic component/main.tex}
            \input{3 - fluid component/main.tex}
            \input{4 - numerical implementation/main.tex}
        \part{Project Overview}
            \input{5 - research plan/main.tex}
            \input{6 - summary/main.tex}
    
    
    %\section{}
    \newpage
    \pagenumbering{gobble}
        \printbibliography


    \newpage
    \pagenumbering{roman}
    \appendix
        \part{Appendices}
            \input{8 - Hilbert complexes/main.tex}
            \input{9 - weak conservation proofs/main.tex}
\end{document}

            \documentclass[12pt, a4paper]{report}

\input{template/main.tex}

\title{\BA{Title in Progress...}}
\author{Boris Andrews}
\affil{Mathematical Institute, University of Oxford}
\date{\today}


\begin{document}
    \pagenumbering{gobble}
    \maketitle
    
    
    \begin{abstract}
        Magnetic confinement reactors---in particular tokamaks---offer one of the most promising options for achieving practical nuclear fusion, with the potential to provide virtually limitless, clean energy. The theoretical and numerical modeling of tokamak plasmas is simultaneously an essential component of effective reactor design, and a great research barrier. Tokamak operational conditions exhibit comparatively low Knudsen numbers. Kinetic effects, including kinetic waves and instabilities, Landau damping, bump-on-tail instabilities and more, are therefore highly influential in tokamak plasma dynamics. Purely fluid models are inherently incapable of capturing these effects, whereas the high dimensionality in purely kinetic models render them practically intractable for most relevant purposes.

        We consider a $\delta\!f$ decomposition model, with a macroscopic fluid background and microscopic kinetic correction, both fully coupled to each other. A similar manner of discretization is proposed to that used in the recent \texttt{STRUPHY} code \cite{Holderied_Possanner_Wang_2021, Holderied_2022, Li_et_al_2023} with a finite-element model for the background and a pseudo-particle/PiC model for the correction.

        The fluid background satisfies the full, non-linear, resistive, compressible, Hall MHD equations. \cite{Laakmann_Hu_Farrell_2022} introduces finite-element(-in-space) implicit timesteppers for the incompressible analogue to this system with structure-preserving (SP) properties in the ideal case, alongside parameter-robust preconditioners. We show that these timesteppers can derive from a finite-element-in-time (FET) (and finite-element-in-space) interpretation. The benefits of this reformulation are discussed, including the derivation of timesteppers that are higher order in time, and the quantifiable dissipative SP properties in the non-ideal, resistive case.
        
        We discuss possible options for extending this FET approach to timesteppers for the compressible case.

        The kinetic corrections satisfy linearized Boltzmann equations. Using a Lénard--Bernstein collision operator, these take Fokker--Planck-like forms \cite{Fokker_1914, Planck_1917} wherein pseudo-particles in the numerical model obey the neoclassical transport equations, with particle-independent Brownian drift terms. This offers a rigorous methodology for incorporating collisions into the particle transport model, without coupling the equations of motions for each particle.
        
        Works by Chen, Chacón et al. \cite{Chen_Chacón_Barnes_2011, Chacón_Chen_Barnes_2013, Chen_Chacón_2014, Chen_Chacón_2015} have developed structure-preserving particle pushers for neoclassical transport in the Vlasov equations, derived from Crank--Nicolson integrators. We show these too can can derive from a FET interpretation, similarly offering potential extensions to higher-order-in-time particle pushers. The FET formulation is used also to consider how the stochastic drift terms can be incorporated into the pushers. Stochastic gyrokinetic expansions are also discussed.

        Different options for the numerical implementation of these schemes are considered.

        Due to the efficacy of FET in the development of SP timesteppers for both the fluid and kinetic component, we hope this approach will prove effective in the future for developing SP timesteppers for the full hybrid model. We hope this will give us the opportunity to incorporate previously inaccessible kinetic effects into the highly effective, modern, finite-element MHD models.
    \end{abstract}
    
    
    \newpage
    \tableofcontents
    
    
    \newpage
    \pagenumbering{arabic}
    %\linenumbers\renewcommand\thelinenumber{\color{black!50}\arabic{linenumber}}
            \input{0 - introduction/main.tex}
        \part{Research}
            \input{1 - low-noise PiC models/main.tex}
            \input{2 - kinetic component/main.tex}
            \input{3 - fluid component/main.tex}
            \input{4 - numerical implementation/main.tex}
        \part{Project Overview}
            \input{5 - research plan/main.tex}
            \input{6 - summary/main.tex}
    
    
    %\section{}
    \newpage
    \pagenumbering{gobble}
        \printbibliography


    \newpage
    \pagenumbering{roman}
    \appendix
        \part{Appendices}
            \input{8 - Hilbert complexes/main.tex}
            \input{9 - weak conservation proofs/main.tex}
\end{document}

\end{document}

            \documentclass[12pt, a4paper]{report}

\documentclass[12pt, a4paper]{report}

\input{template/main.tex}

\title{\BA{Title in Progress...}}
\author{Boris Andrews}
\affil{Mathematical Institute, University of Oxford}
\date{\today}


\begin{document}
    \pagenumbering{gobble}
    \maketitle
    
    
    \begin{abstract}
        Magnetic confinement reactors---in particular tokamaks---offer one of the most promising options for achieving practical nuclear fusion, with the potential to provide virtually limitless, clean energy. The theoretical and numerical modeling of tokamak plasmas is simultaneously an essential component of effective reactor design, and a great research barrier. Tokamak operational conditions exhibit comparatively low Knudsen numbers. Kinetic effects, including kinetic waves and instabilities, Landau damping, bump-on-tail instabilities and more, are therefore highly influential in tokamak plasma dynamics. Purely fluid models are inherently incapable of capturing these effects, whereas the high dimensionality in purely kinetic models render them practically intractable for most relevant purposes.

        We consider a $\delta\!f$ decomposition model, with a macroscopic fluid background and microscopic kinetic correction, both fully coupled to each other. A similar manner of discretization is proposed to that used in the recent \texttt{STRUPHY} code \cite{Holderied_Possanner_Wang_2021, Holderied_2022, Li_et_al_2023} with a finite-element model for the background and a pseudo-particle/PiC model for the correction.

        The fluid background satisfies the full, non-linear, resistive, compressible, Hall MHD equations. \cite{Laakmann_Hu_Farrell_2022} introduces finite-element(-in-space) implicit timesteppers for the incompressible analogue to this system with structure-preserving (SP) properties in the ideal case, alongside parameter-robust preconditioners. We show that these timesteppers can derive from a finite-element-in-time (FET) (and finite-element-in-space) interpretation. The benefits of this reformulation are discussed, including the derivation of timesteppers that are higher order in time, and the quantifiable dissipative SP properties in the non-ideal, resistive case.
        
        We discuss possible options for extending this FET approach to timesteppers for the compressible case.

        The kinetic corrections satisfy linearized Boltzmann equations. Using a Lénard--Bernstein collision operator, these take Fokker--Planck-like forms \cite{Fokker_1914, Planck_1917} wherein pseudo-particles in the numerical model obey the neoclassical transport equations, with particle-independent Brownian drift terms. This offers a rigorous methodology for incorporating collisions into the particle transport model, without coupling the equations of motions for each particle.
        
        Works by Chen, Chacón et al. \cite{Chen_Chacón_Barnes_2011, Chacón_Chen_Barnes_2013, Chen_Chacón_2014, Chen_Chacón_2015} have developed structure-preserving particle pushers for neoclassical transport in the Vlasov equations, derived from Crank--Nicolson integrators. We show these too can can derive from a FET interpretation, similarly offering potential extensions to higher-order-in-time particle pushers. The FET formulation is used also to consider how the stochastic drift terms can be incorporated into the pushers. Stochastic gyrokinetic expansions are also discussed.

        Different options for the numerical implementation of these schemes are considered.

        Due to the efficacy of FET in the development of SP timesteppers for both the fluid and kinetic component, we hope this approach will prove effective in the future for developing SP timesteppers for the full hybrid model. We hope this will give us the opportunity to incorporate previously inaccessible kinetic effects into the highly effective, modern, finite-element MHD models.
    \end{abstract}
    
    
    \newpage
    \tableofcontents
    
    
    \newpage
    \pagenumbering{arabic}
    %\linenumbers\renewcommand\thelinenumber{\color{black!50}\arabic{linenumber}}
            \input{0 - introduction/main.tex}
        \part{Research}
            \input{1 - low-noise PiC models/main.tex}
            \input{2 - kinetic component/main.tex}
            \input{3 - fluid component/main.tex}
            \input{4 - numerical implementation/main.tex}
        \part{Project Overview}
            \input{5 - research plan/main.tex}
            \input{6 - summary/main.tex}
    
    
    %\section{}
    \newpage
    \pagenumbering{gobble}
        \printbibliography


    \newpage
    \pagenumbering{roman}
    \appendix
        \part{Appendices}
            \input{8 - Hilbert complexes/main.tex}
            \input{9 - weak conservation proofs/main.tex}
\end{document}


\title{\BA{Title in Progress...}}
\author{Boris Andrews}
\affil{Mathematical Institute, University of Oxford}
\date{\today}


\begin{document}
    \pagenumbering{gobble}
    \maketitle
    
    
    \begin{abstract}
        Magnetic confinement reactors---in particular tokamaks---offer one of the most promising options for achieving practical nuclear fusion, with the potential to provide virtually limitless, clean energy. The theoretical and numerical modeling of tokamak plasmas is simultaneously an essential component of effective reactor design, and a great research barrier. Tokamak operational conditions exhibit comparatively low Knudsen numbers. Kinetic effects, including kinetic waves and instabilities, Landau damping, bump-on-tail instabilities and more, are therefore highly influential in tokamak plasma dynamics. Purely fluid models are inherently incapable of capturing these effects, whereas the high dimensionality in purely kinetic models render them practically intractable for most relevant purposes.

        We consider a $\delta\!f$ decomposition model, with a macroscopic fluid background and microscopic kinetic correction, both fully coupled to each other. A similar manner of discretization is proposed to that used in the recent \texttt{STRUPHY} code \cite{Holderied_Possanner_Wang_2021, Holderied_2022, Li_et_al_2023} with a finite-element model for the background and a pseudo-particle/PiC model for the correction.

        The fluid background satisfies the full, non-linear, resistive, compressible, Hall MHD equations. \cite{Laakmann_Hu_Farrell_2022} introduces finite-element(-in-space) implicit timesteppers for the incompressible analogue to this system with structure-preserving (SP) properties in the ideal case, alongside parameter-robust preconditioners. We show that these timesteppers can derive from a finite-element-in-time (FET) (and finite-element-in-space) interpretation. The benefits of this reformulation are discussed, including the derivation of timesteppers that are higher order in time, and the quantifiable dissipative SP properties in the non-ideal, resistive case.
        
        We discuss possible options for extending this FET approach to timesteppers for the compressible case.

        The kinetic corrections satisfy linearized Boltzmann equations. Using a Lénard--Bernstein collision operator, these take Fokker--Planck-like forms \cite{Fokker_1914, Planck_1917} wherein pseudo-particles in the numerical model obey the neoclassical transport equations, with particle-independent Brownian drift terms. This offers a rigorous methodology for incorporating collisions into the particle transport model, without coupling the equations of motions for each particle.
        
        Works by Chen, Chacón et al. \cite{Chen_Chacón_Barnes_2011, Chacón_Chen_Barnes_2013, Chen_Chacón_2014, Chen_Chacón_2015} have developed structure-preserving particle pushers for neoclassical transport in the Vlasov equations, derived from Crank--Nicolson integrators. We show these too can can derive from a FET interpretation, similarly offering potential extensions to higher-order-in-time particle pushers. The FET formulation is used also to consider how the stochastic drift terms can be incorporated into the pushers. Stochastic gyrokinetic expansions are also discussed.

        Different options for the numerical implementation of these schemes are considered.

        Due to the efficacy of FET in the development of SP timesteppers for both the fluid and kinetic component, we hope this approach will prove effective in the future for developing SP timesteppers for the full hybrid model. We hope this will give us the opportunity to incorporate previously inaccessible kinetic effects into the highly effective, modern, finite-element MHD models.
    \end{abstract}
    
    
    \newpage
    \tableofcontents
    
    
    \newpage
    \pagenumbering{arabic}
    %\linenumbers\renewcommand\thelinenumber{\color{black!50}\arabic{linenumber}}
            \documentclass[12pt, a4paper]{report}

\input{template/main.tex}

\title{\BA{Title in Progress...}}
\author{Boris Andrews}
\affil{Mathematical Institute, University of Oxford}
\date{\today}


\begin{document}
    \pagenumbering{gobble}
    \maketitle
    
    
    \begin{abstract}
        Magnetic confinement reactors---in particular tokamaks---offer one of the most promising options for achieving practical nuclear fusion, with the potential to provide virtually limitless, clean energy. The theoretical and numerical modeling of tokamak plasmas is simultaneously an essential component of effective reactor design, and a great research barrier. Tokamak operational conditions exhibit comparatively low Knudsen numbers. Kinetic effects, including kinetic waves and instabilities, Landau damping, bump-on-tail instabilities and more, are therefore highly influential in tokamak plasma dynamics. Purely fluid models are inherently incapable of capturing these effects, whereas the high dimensionality in purely kinetic models render them practically intractable for most relevant purposes.

        We consider a $\delta\!f$ decomposition model, with a macroscopic fluid background and microscopic kinetic correction, both fully coupled to each other. A similar manner of discretization is proposed to that used in the recent \texttt{STRUPHY} code \cite{Holderied_Possanner_Wang_2021, Holderied_2022, Li_et_al_2023} with a finite-element model for the background and a pseudo-particle/PiC model for the correction.

        The fluid background satisfies the full, non-linear, resistive, compressible, Hall MHD equations. \cite{Laakmann_Hu_Farrell_2022} introduces finite-element(-in-space) implicit timesteppers for the incompressible analogue to this system with structure-preserving (SP) properties in the ideal case, alongside parameter-robust preconditioners. We show that these timesteppers can derive from a finite-element-in-time (FET) (and finite-element-in-space) interpretation. The benefits of this reformulation are discussed, including the derivation of timesteppers that are higher order in time, and the quantifiable dissipative SP properties in the non-ideal, resistive case.
        
        We discuss possible options for extending this FET approach to timesteppers for the compressible case.

        The kinetic corrections satisfy linearized Boltzmann equations. Using a Lénard--Bernstein collision operator, these take Fokker--Planck-like forms \cite{Fokker_1914, Planck_1917} wherein pseudo-particles in the numerical model obey the neoclassical transport equations, with particle-independent Brownian drift terms. This offers a rigorous methodology for incorporating collisions into the particle transport model, without coupling the equations of motions for each particle.
        
        Works by Chen, Chacón et al. \cite{Chen_Chacón_Barnes_2011, Chacón_Chen_Barnes_2013, Chen_Chacón_2014, Chen_Chacón_2015} have developed structure-preserving particle pushers for neoclassical transport in the Vlasov equations, derived from Crank--Nicolson integrators. We show these too can can derive from a FET interpretation, similarly offering potential extensions to higher-order-in-time particle pushers. The FET formulation is used also to consider how the stochastic drift terms can be incorporated into the pushers. Stochastic gyrokinetic expansions are also discussed.

        Different options for the numerical implementation of these schemes are considered.

        Due to the efficacy of FET in the development of SP timesteppers for both the fluid and kinetic component, we hope this approach will prove effective in the future for developing SP timesteppers for the full hybrid model. We hope this will give us the opportunity to incorporate previously inaccessible kinetic effects into the highly effective, modern, finite-element MHD models.
    \end{abstract}
    
    
    \newpage
    \tableofcontents
    
    
    \newpage
    \pagenumbering{arabic}
    %\linenumbers\renewcommand\thelinenumber{\color{black!50}\arabic{linenumber}}
            \input{0 - introduction/main.tex}
        \part{Research}
            \input{1 - low-noise PiC models/main.tex}
            \input{2 - kinetic component/main.tex}
            \input{3 - fluid component/main.tex}
            \input{4 - numerical implementation/main.tex}
        \part{Project Overview}
            \input{5 - research plan/main.tex}
            \input{6 - summary/main.tex}
    
    
    %\section{}
    \newpage
    \pagenumbering{gobble}
        \printbibliography


    \newpage
    \pagenumbering{roman}
    \appendix
        \part{Appendices}
            \input{8 - Hilbert complexes/main.tex}
            \input{9 - weak conservation proofs/main.tex}
\end{document}

        \part{Research}
            \documentclass[12pt, a4paper]{report}

\input{template/main.tex}

\title{\BA{Title in Progress...}}
\author{Boris Andrews}
\affil{Mathematical Institute, University of Oxford}
\date{\today}


\begin{document}
    \pagenumbering{gobble}
    \maketitle
    
    
    \begin{abstract}
        Magnetic confinement reactors---in particular tokamaks---offer one of the most promising options for achieving practical nuclear fusion, with the potential to provide virtually limitless, clean energy. The theoretical and numerical modeling of tokamak plasmas is simultaneously an essential component of effective reactor design, and a great research barrier. Tokamak operational conditions exhibit comparatively low Knudsen numbers. Kinetic effects, including kinetic waves and instabilities, Landau damping, bump-on-tail instabilities and more, are therefore highly influential in tokamak plasma dynamics. Purely fluid models are inherently incapable of capturing these effects, whereas the high dimensionality in purely kinetic models render them practically intractable for most relevant purposes.

        We consider a $\delta\!f$ decomposition model, with a macroscopic fluid background and microscopic kinetic correction, both fully coupled to each other. A similar manner of discretization is proposed to that used in the recent \texttt{STRUPHY} code \cite{Holderied_Possanner_Wang_2021, Holderied_2022, Li_et_al_2023} with a finite-element model for the background and a pseudo-particle/PiC model for the correction.

        The fluid background satisfies the full, non-linear, resistive, compressible, Hall MHD equations. \cite{Laakmann_Hu_Farrell_2022} introduces finite-element(-in-space) implicit timesteppers for the incompressible analogue to this system with structure-preserving (SP) properties in the ideal case, alongside parameter-robust preconditioners. We show that these timesteppers can derive from a finite-element-in-time (FET) (and finite-element-in-space) interpretation. The benefits of this reformulation are discussed, including the derivation of timesteppers that are higher order in time, and the quantifiable dissipative SP properties in the non-ideal, resistive case.
        
        We discuss possible options for extending this FET approach to timesteppers for the compressible case.

        The kinetic corrections satisfy linearized Boltzmann equations. Using a Lénard--Bernstein collision operator, these take Fokker--Planck-like forms \cite{Fokker_1914, Planck_1917} wherein pseudo-particles in the numerical model obey the neoclassical transport equations, with particle-independent Brownian drift terms. This offers a rigorous methodology for incorporating collisions into the particle transport model, without coupling the equations of motions for each particle.
        
        Works by Chen, Chacón et al. \cite{Chen_Chacón_Barnes_2011, Chacón_Chen_Barnes_2013, Chen_Chacón_2014, Chen_Chacón_2015} have developed structure-preserving particle pushers for neoclassical transport in the Vlasov equations, derived from Crank--Nicolson integrators. We show these too can can derive from a FET interpretation, similarly offering potential extensions to higher-order-in-time particle pushers. The FET formulation is used also to consider how the stochastic drift terms can be incorporated into the pushers. Stochastic gyrokinetic expansions are also discussed.

        Different options for the numerical implementation of these schemes are considered.

        Due to the efficacy of FET in the development of SP timesteppers for both the fluid and kinetic component, we hope this approach will prove effective in the future for developing SP timesteppers for the full hybrid model. We hope this will give us the opportunity to incorporate previously inaccessible kinetic effects into the highly effective, modern, finite-element MHD models.
    \end{abstract}
    
    
    \newpage
    \tableofcontents
    
    
    \newpage
    \pagenumbering{arabic}
    %\linenumbers\renewcommand\thelinenumber{\color{black!50}\arabic{linenumber}}
            \input{0 - introduction/main.tex}
        \part{Research}
            \input{1 - low-noise PiC models/main.tex}
            \input{2 - kinetic component/main.tex}
            \input{3 - fluid component/main.tex}
            \input{4 - numerical implementation/main.tex}
        \part{Project Overview}
            \input{5 - research plan/main.tex}
            \input{6 - summary/main.tex}
    
    
    %\section{}
    \newpage
    \pagenumbering{gobble}
        \printbibliography


    \newpage
    \pagenumbering{roman}
    \appendix
        \part{Appendices}
            \input{8 - Hilbert complexes/main.tex}
            \input{9 - weak conservation proofs/main.tex}
\end{document}

            \documentclass[12pt, a4paper]{report}

\input{template/main.tex}

\title{\BA{Title in Progress...}}
\author{Boris Andrews}
\affil{Mathematical Institute, University of Oxford}
\date{\today}


\begin{document}
    \pagenumbering{gobble}
    \maketitle
    
    
    \begin{abstract}
        Magnetic confinement reactors---in particular tokamaks---offer one of the most promising options for achieving practical nuclear fusion, with the potential to provide virtually limitless, clean energy. The theoretical and numerical modeling of tokamak plasmas is simultaneously an essential component of effective reactor design, and a great research barrier. Tokamak operational conditions exhibit comparatively low Knudsen numbers. Kinetic effects, including kinetic waves and instabilities, Landau damping, bump-on-tail instabilities and more, are therefore highly influential in tokamak plasma dynamics. Purely fluid models are inherently incapable of capturing these effects, whereas the high dimensionality in purely kinetic models render them practically intractable for most relevant purposes.

        We consider a $\delta\!f$ decomposition model, with a macroscopic fluid background and microscopic kinetic correction, both fully coupled to each other. A similar manner of discretization is proposed to that used in the recent \texttt{STRUPHY} code \cite{Holderied_Possanner_Wang_2021, Holderied_2022, Li_et_al_2023} with a finite-element model for the background and a pseudo-particle/PiC model for the correction.

        The fluid background satisfies the full, non-linear, resistive, compressible, Hall MHD equations. \cite{Laakmann_Hu_Farrell_2022} introduces finite-element(-in-space) implicit timesteppers for the incompressible analogue to this system with structure-preserving (SP) properties in the ideal case, alongside parameter-robust preconditioners. We show that these timesteppers can derive from a finite-element-in-time (FET) (and finite-element-in-space) interpretation. The benefits of this reformulation are discussed, including the derivation of timesteppers that are higher order in time, and the quantifiable dissipative SP properties in the non-ideal, resistive case.
        
        We discuss possible options for extending this FET approach to timesteppers for the compressible case.

        The kinetic corrections satisfy linearized Boltzmann equations. Using a Lénard--Bernstein collision operator, these take Fokker--Planck-like forms \cite{Fokker_1914, Planck_1917} wherein pseudo-particles in the numerical model obey the neoclassical transport equations, with particle-independent Brownian drift terms. This offers a rigorous methodology for incorporating collisions into the particle transport model, without coupling the equations of motions for each particle.
        
        Works by Chen, Chacón et al. \cite{Chen_Chacón_Barnes_2011, Chacón_Chen_Barnes_2013, Chen_Chacón_2014, Chen_Chacón_2015} have developed structure-preserving particle pushers for neoclassical transport in the Vlasov equations, derived from Crank--Nicolson integrators. We show these too can can derive from a FET interpretation, similarly offering potential extensions to higher-order-in-time particle pushers. The FET formulation is used also to consider how the stochastic drift terms can be incorporated into the pushers. Stochastic gyrokinetic expansions are also discussed.

        Different options for the numerical implementation of these schemes are considered.

        Due to the efficacy of FET in the development of SP timesteppers for both the fluid and kinetic component, we hope this approach will prove effective in the future for developing SP timesteppers for the full hybrid model. We hope this will give us the opportunity to incorporate previously inaccessible kinetic effects into the highly effective, modern, finite-element MHD models.
    \end{abstract}
    
    
    \newpage
    \tableofcontents
    
    
    \newpage
    \pagenumbering{arabic}
    %\linenumbers\renewcommand\thelinenumber{\color{black!50}\arabic{linenumber}}
            \input{0 - introduction/main.tex}
        \part{Research}
            \input{1 - low-noise PiC models/main.tex}
            \input{2 - kinetic component/main.tex}
            \input{3 - fluid component/main.tex}
            \input{4 - numerical implementation/main.tex}
        \part{Project Overview}
            \input{5 - research plan/main.tex}
            \input{6 - summary/main.tex}
    
    
    %\section{}
    \newpage
    \pagenumbering{gobble}
        \printbibliography


    \newpage
    \pagenumbering{roman}
    \appendix
        \part{Appendices}
            \input{8 - Hilbert complexes/main.tex}
            \input{9 - weak conservation proofs/main.tex}
\end{document}

            \documentclass[12pt, a4paper]{report}

\input{template/main.tex}

\title{\BA{Title in Progress...}}
\author{Boris Andrews}
\affil{Mathematical Institute, University of Oxford}
\date{\today}


\begin{document}
    \pagenumbering{gobble}
    \maketitle
    
    
    \begin{abstract}
        Magnetic confinement reactors---in particular tokamaks---offer one of the most promising options for achieving practical nuclear fusion, with the potential to provide virtually limitless, clean energy. The theoretical and numerical modeling of tokamak plasmas is simultaneously an essential component of effective reactor design, and a great research barrier. Tokamak operational conditions exhibit comparatively low Knudsen numbers. Kinetic effects, including kinetic waves and instabilities, Landau damping, bump-on-tail instabilities and more, are therefore highly influential in tokamak plasma dynamics. Purely fluid models are inherently incapable of capturing these effects, whereas the high dimensionality in purely kinetic models render them practically intractable for most relevant purposes.

        We consider a $\delta\!f$ decomposition model, with a macroscopic fluid background and microscopic kinetic correction, both fully coupled to each other. A similar manner of discretization is proposed to that used in the recent \texttt{STRUPHY} code \cite{Holderied_Possanner_Wang_2021, Holderied_2022, Li_et_al_2023} with a finite-element model for the background and a pseudo-particle/PiC model for the correction.

        The fluid background satisfies the full, non-linear, resistive, compressible, Hall MHD equations. \cite{Laakmann_Hu_Farrell_2022} introduces finite-element(-in-space) implicit timesteppers for the incompressible analogue to this system with structure-preserving (SP) properties in the ideal case, alongside parameter-robust preconditioners. We show that these timesteppers can derive from a finite-element-in-time (FET) (and finite-element-in-space) interpretation. The benefits of this reformulation are discussed, including the derivation of timesteppers that are higher order in time, and the quantifiable dissipative SP properties in the non-ideal, resistive case.
        
        We discuss possible options for extending this FET approach to timesteppers for the compressible case.

        The kinetic corrections satisfy linearized Boltzmann equations. Using a Lénard--Bernstein collision operator, these take Fokker--Planck-like forms \cite{Fokker_1914, Planck_1917} wherein pseudo-particles in the numerical model obey the neoclassical transport equations, with particle-independent Brownian drift terms. This offers a rigorous methodology for incorporating collisions into the particle transport model, without coupling the equations of motions for each particle.
        
        Works by Chen, Chacón et al. \cite{Chen_Chacón_Barnes_2011, Chacón_Chen_Barnes_2013, Chen_Chacón_2014, Chen_Chacón_2015} have developed structure-preserving particle pushers for neoclassical transport in the Vlasov equations, derived from Crank--Nicolson integrators. We show these too can can derive from a FET interpretation, similarly offering potential extensions to higher-order-in-time particle pushers. The FET formulation is used also to consider how the stochastic drift terms can be incorporated into the pushers. Stochastic gyrokinetic expansions are also discussed.

        Different options for the numerical implementation of these schemes are considered.

        Due to the efficacy of FET in the development of SP timesteppers for both the fluid and kinetic component, we hope this approach will prove effective in the future for developing SP timesteppers for the full hybrid model. We hope this will give us the opportunity to incorporate previously inaccessible kinetic effects into the highly effective, modern, finite-element MHD models.
    \end{abstract}
    
    
    \newpage
    \tableofcontents
    
    
    \newpage
    \pagenumbering{arabic}
    %\linenumbers\renewcommand\thelinenumber{\color{black!50}\arabic{linenumber}}
            \input{0 - introduction/main.tex}
        \part{Research}
            \input{1 - low-noise PiC models/main.tex}
            \input{2 - kinetic component/main.tex}
            \input{3 - fluid component/main.tex}
            \input{4 - numerical implementation/main.tex}
        \part{Project Overview}
            \input{5 - research plan/main.tex}
            \input{6 - summary/main.tex}
    
    
    %\section{}
    \newpage
    \pagenumbering{gobble}
        \printbibliography


    \newpage
    \pagenumbering{roman}
    \appendix
        \part{Appendices}
            \input{8 - Hilbert complexes/main.tex}
            \input{9 - weak conservation proofs/main.tex}
\end{document}

            \documentclass[12pt, a4paper]{report}

\input{template/main.tex}

\title{\BA{Title in Progress...}}
\author{Boris Andrews}
\affil{Mathematical Institute, University of Oxford}
\date{\today}


\begin{document}
    \pagenumbering{gobble}
    \maketitle
    
    
    \begin{abstract}
        Magnetic confinement reactors---in particular tokamaks---offer one of the most promising options for achieving practical nuclear fusion, with the potential to provide virtually limitless, clean energy. The theoretical and numerical modeling of tokamak plasmas is simultaneously an essential component of effective reactor design, and a great research barrier. Tokamak operational conditions exhibit comparatively low Knudsen numbers. Kinetic effects, including kinetic waves and instabilities, Landau damping, bump-on-tail instabilities and more, are therefore highly influential in tokamak plasma dynamics. Purely fluid models are inherently incapable of capturing these effects, whereas the high dimensionality in purely kinetic models render them practically intractable for most relevant purposes.

        We consider a $\delta\!f$ decomposition model, with a macroscopic fluid background and microscopic kinetic correction, both fully coupled to each other. A similar manner of discretization is proposed to that used in the recent \texttt{STRUPHY} code \cite{Holderied_Possanner_Wang_2021, Holderied_2022, Li_et_al_2023} with a finite-element model for the background and a pseudo-particle/PiC model for the correction.

        The fluid background satisfies the full, non-linear, resistive, compressible, Hall MHD equations. \cite{Laakmann_Hu_Farrell_2022} introduces finite-element(-in-space) implicit timesteppers for the incompressible analogue to this system with structure-preserving (SP) properties in the ideal case, alongside parameter-robust preconditioners. We show that these timesteppers can derive from a finite-element-in-time (FET) (and finite-element-in-space) interpretation. The benefits of this reformulation are discussed, including the derivation of timesteppers that are higher order in time, and the quantifiable dissipative SP properties in the non-ideal, resistive case.
        
        We discuss possible options for extending this FET approach to timesteppers for the compressible case.

        The kinetic corrections satisfy linearized Boltzmann equations. Using a Lénard--Bernstein collision operator, these take Fokker--Planck-like forms \cite{Fokker_1914, Planck_1917} wherein pseudo-particles in the numerical model obey the neoclassical transport equations, with particle-independent Brownian drift terms. This offers a rigorous methodology for incorporating collisions into the particle transport model, without coupling the equations of motions for each particle.
        
        Works by Chen, Chacón et al. \cite{Chen_Chacón_Barnes_2011, Chacón_Chen_Barnes_2013, Chen_Chacón_2014, Chen_Chacón_2015} have developed structure-preserving particle pushers for neoclassical transport in the Vlasov equations, derived from Crank--Nicolson integrators. We show these too can can derive from a FET interpretation, similarly offering potential extensions to higher-order-in-time particle pushers. The FET formulation is used also to consider how the stochastic drift terms can be incorporated into the pushers. Stochastic gyrokinetic expansions are also discussed.

        Different options for the numerical implementation of these schemes are considered.

        Due to the efficacy of FET in the development of SP timesteppers for both the fluid and kinetic component, we hope this approach will prove effective in the future for developing SP timesteppers for the full hybrid model. We hope this will give us the opportunity to incorporate previously inaccessible kinetic effects into the highly effective, modern, finite-element MHD models.
    \end{abstract}
    
    
    \newpage
    \tableofcontents
    
    
    \newpage
    \pagenumbering{arabic}
    %\linenumbers\renewcommand\thelinenumber{\color{black!50}\arabic{linenumber}}
            \input{0 - introduction/main.tex}
        \part{Research}
            \input{1 - low-noise PiC models/main.tex}
            \input{2 - kinetic component/main.tex}
            \input{3 - fluid component/main.tex}
            \input{4 - numerical implementation/main.tex}
        \part{Project Overview}
            \input{5 - research plan/main.tex}
            \input{6 - summary/main.tex}
    
    
    %\section{}
    \newpage
    \pagenumbering{gobble}
        \printbibliography


    \newpage
    \pagenumbering{roman}
    \appendix
        \part{Appendices}
            \input{8 - Hilbert complexes/main.tex}
            \input{9 - weak conservation proofs/main.tex}
\end{document}

        \part{Project Overview}
            \documentclass[12pt, a4paper]{report}

\input{template/main.tex}

\title{\BA{Title in Progress...}}
\author{Boris Andrews}
\affil{Mathematical Institute, University of Oxford}
\date{\today}


\begin{document}
    \pagenumbering{gobble}
    \maketitle
    
    
    \begin{abstract}
        Magnetic confinement reactors---in particular tokamaks---offer one of the most promising options for achieving practical nuclear fusion, with the potential to provide virtually limitless, clean energy. The theoretical and numerical modeling of tokamak plasmas is simultaneously an essential component of effective reactor design, and a great research barrier. Tokamak operational conditions exhibit comparatively low Knudsen numbers. Kinetic effects, including kinetic waves and instabilities, Landau damping, bump-on-tail instabilities and more, are therefore highly influential in tokamak plasma dynamics. Purely fluid models are inherently incapable of capturing these effects, whereas the high dimensionality in purely kinetic models render them practically intractable for most relevant purposes.

        We consider a $\delta\!f$ decomposition model, with a macroscopic fluid background and microscopic kinetic correction, both fully coupled to each other. A similar manner of discretization is proposed to that used in the recent \texttt{STRUPHY} code \cite{Holderied_Possanner_Wang_2021, Holderied_2022, Li_et_al_2023} with a finite-element model for the background and a pseudo-particle/PiC model for the correction.

        The fluid background satisfies the full, non-linear, resistive, compressible, Hall MHD equations. \cite{Laakmann_Hu_Farrell_2022} introduces finite-element(-in-space) implicit timesteppers for the incompressible analogue to this system with structure-preserving (SP) properties in the ideal case, alongside parameter-robust preconditioners. We show that these timesteppers can derive from a finite-element-in-time (FET) (and finite-element-in-space) interpretation. The benefits of this reformulation are discussed, including the derivation of timesteppers that are higher order in time, and the quantifiable dissipative SP properties in the non-ideal, resistive case.
        
        We discuss possible options for extending this FET approach to timesteppers for the compressible case.

        The kinetic corrections satisfy linearized Boltzmann equations. Using a Lénard--Bernstein collision operator, these take Fokker--Planck-like forms \cite{Fokker_1914, Planck_1917} wherein pseudo-particles in the numerical model obey the neoclassical transport equations, with particle-independent Brownian drift terms. This offers a rigorous methodology for incorporating collisions into the particle transport model, without coupling the equations of motions for each particle.
        
        Works by Chen, Chacón et al. \cite{Chen_Chacón_Barnes_2011, Chacón_Chen_Barnes_2013, Chen_Chacón_2014, Chen_Chacón_2015} have developed structure-preserving particle pushers for neoclassical transport in the Vlasov equations, derived from Crank--Nicolson integrators. We show these too can can derive from a FET interpretation, similarly offering potential extensions to higher-order-in-time particle pushers. The FET formulation is used also to consider how the stochastic drift terms can be incorporated into the pushers. Stochastic gyrokinetic expansions are also discussed.

        Different options for the numerical implementation of these schemes are considered.

        Due to the efficacy of FET in the development of SP timesteppers for both the fluid and kinetic component, we hope this approach will prove effective in the future for developing SP timesteppers for the full hybrid model. We hope this will give us the opportunity to incorporate previously inaccessible kinetic effects into the highly effective, modern, finite-element MHD models.
    \end{abstract}
    
    
    \newpage
    \tableofcontents
    
    
    \newpage
    \pagenumbering{arabic}
    %\linenumbers\renewcommand\thelinenumber{\color{black!50}\arabic{linenumber}}
            \input{0 - introduction/main.tex}
        \part{Research}
            \input{1 - low-noise PiC models/main.tex}
            \input{2 - kinetic component/main.tex}
            \input{3 - fluid component/main.tex}
            \input{4 - numerical implementation/main.tex}
        \part{Project Overview}
            \input{5 - research plan/main.tex}
            \input{6 - summary/main.tex}
    
    
    %\section{}
    \newpage
    \pagenumbering{gobble}
        \printbibliography


    \newpage
    \pagenumbering{roman}
    \appendix
        \part{Appendices}
            \input{8 - Hilbert complexes/main.tex}
            \input{9 - weak conservation proofs/main.tex}
\end{document}

            \documentclass[12pt, a4paper]{report}

\input{template/main.tex}

\title{\BA{Title in Progress...}}
\author{Boris Andrews}
\affil{Mathematical Institute, University of Oxford}
\date{\today}


\begin{document}
    \pagenumbering{gobble}
    \maketitle
    
    
    \begin{abstract}
        Magnetic confinement reactors---in particular tokamaks---offer one of the most promising options for achieving practical nuclear fusion, with the potential to provide virtually limitless, clean energy. The theoretical and numerical modeling of tokamak plasmas is simultaneously an essential component of effective reactor design, and a great research barrier. Tokamak operational conditions exhibit comparatively low Knudsen numbers. Kinetic effects, including kinetic waves and instabilities, Landau damping, bump-on-tail instabilities and more, are therefore highly influential in tokamak plasma dynamics. Purely fluid models are inherently incapable of capturing these effects, whereas the high dimensionality in purely kinetic models render them practically intractable for most relevant purposes.

        We consider a $\delta\!f$ decomposition model, with a macroscopic fluid background and microscopic kinetic correction, both fully coupled to each other. A similar manner of discretization is proposed to that used in the recent \texttt{STRUPHY} code \cite{Holderied_Possanner_Wang_2021, Holderied_2022, Li_et_al_2023} with a finite-element model for the background and a pseudo-particle/PiC model for the correction.

        The fluid background satisfies the full, non-linear, resistive, compressible, Hall MHD equations. \cite{Laakmann_Hu_Farrell_2022} introduces finite-element(-in-space) implicit timesteppers for the incompressible analogue to this system with structure-preserving (SP) properties in the ideal case, alongside parameter-robust preconditioners. We show that these timesteppers can derive from a finite-element-in-time (FET) (and finite-element-in-space) interpretation. The benefits of this reformulation are discussed, including the derivation of timesteppers that are higher order in time, and the quantifiable dissipative SP properties in the non-ideal, resistive case.
        
        We discuss possible options for extending this FET approach to timesteppers for the compressible case.

        The kinetic corrections satisfy linearized Boltzmann equations. Using a Lénard--Bernstein collision operator, these take Fokker--Planck-like forms \cite{Fokker_1914, Planck_1917} wherein pseudo-particles in the numerical model obey the neoclassical transport equations, with particle-independent Brownian drift terms. This offers a rigorous methodology for incorporating collisions into the particle transport model, without coupling the equations of motions for each particle.
        
        Works by Chen, Chacón et al. \cite{Chen_Chacón_Barnes_2011, Chacón_Chen_Barnes_2013, Chen_Chacón_2014, Chen_Chacón_2015} have developed structure-preserving particle pushers for neoclassical transport in the Vlasov equations, derived from Crank--Nicolson integrators. We show these too can can derive from a FET interpretation, similarly offering potential extensions to higher-order-in-time particle pushers. The FET formulation is used also to consider how the stochastic drift terms can be incorporated into the pushers. Stochastic gyrokinetic expansions are also discussed.

        Different options for the numerical implementation of these schemes are considered.

        Due to the efficacy of FET in the development of SP timesteppers for both the fluid and kinetic component, we hope this approach will prove effective in the future for developing SP timesteppers for the full hybrid model. We hope this will give us the opportunity to incorporate previously inaccessible kinetic effects into the highly effective, modern, finite-element MHD models.
    \end{abstract}
    
    
    \newpage
    \tableofcontents
    
    
    \newpage
    \pagenumbering{arabic}
    %\linenumbers\renewcommand\thelinenumber{\color{black!50}\arabic{linenumber}}
            \input{0 - introduction/main.tex}
        \part{Research}
            \input{1 - low-noise PiC models/main.tex}
            \input{2 - kinetic component/main.tex}
            \input{3 - fluid component/main.tex}
            \input{4 - numerical implementation/main.tex}
        \part{Project Overview}
            \input{5 - research plan/main.tex}
            \input{6 - summary/main.tex}
    
    
    %\section{}
    \newpage
    \pagenumbering{gobble}
        \printbibliography


    \newpage
    \pagenumbering{roman}
    \appendix
        \part{Appendices}
            \input{8 - Hilbert complexes/main.tex}
            \input{9 - weak conservation proofs/main.tex}
\end{document}

    
    
    %\section{}
    \newpage
    \pagenumbering{gobble}
        \printbibliography


    \newpage
    \pagenumbering{roman}
    \appendix
        \part{Appendices}
            \documentclass[12pt, a4paper]{report}

\input{template/main.tex}

\title{\BA{Title in Progress...}}
\author{Boris Andrews}
\affil{Mathematical Institute, University of Oxford}
\date{\today}


\begin{document}
    \pagenumbering{gobble}
    \maketitle
    
    
    \begin{abstract}
        Magnetic confinement reactors---in particular tokamaks---offer one of the most promising options for achieving practical nuclear fusion, with the potential to provide virtually limitless, clean energy. The theoretical and numerical modeling of tokamak plasmas is simultaneously an essential component of effective reactor design, and a great research barrier. Tokamak operational conditions exhibit comparatively low Knudsen numbers. Kinetic effects, including kinetic waves and instabilities, Landau damping, bump-on-tail instabilities and more, are therefore highly influential in tokamak plasma dynamics. Purely fluid models are inherently incapable of capturing these effects, whereas the high dimensionality in purely kinetic models render them practically intractable for most relevant purposes.

        We consider a $\delta\!f$ decomposition model, with a macroscopic fluid background and microscopic kinetic correction, both fully coupled to each other. A similar manner of discretization is proposed to that used in the recent \texttt{STRUPHY} code \cite{Holderied_Possanner_Wang_2021, Holderied_2022, Li_et_al_2023} with a finite-element model for the background and a pseudo-particle/PiC model for the correction.

        The fluid background satisfies the full, non-linear, resistive, compressible, Hall MHD equations. \cite{Laakmann_Hu_Farrell_2022} introduces finite-element(-in-space) implicit timesteppers for the incompressible analogue to this system with structure-preserving (SP) properties in the ideal case, alongside parameter-robust preconditioners. We show that these timesteppers can derive from a finite-element-in-time (FET) (and finite-element-in-space) interpretation. The benefits of this reformulation are discussed, including the derivation of timesteppers that are higher order in time, and the quantifiable dissipative SP properties in the non-ideal, resistive case.
        
        We discuss possible options for extending this FET approach to timesteppers for the compressible case.

        The kinetic corrections satisfy linearized Boltzmann equations. Using a Lénard--Bernstein collision operator, these take Fokker--Planck-like forms \cite{Fokker_1914, Planck_1917} wherein pseudo-particles in the numerical model obey the neoclassical transport equations, with particle-independent Brownian drift terms. This offers a rigorous methodology for incorporating collisions into the particle transport model, without coupling the equations of motions for each particle.
        
        Works by Chen, Chacón et al. \cite{Chen_Chacón_Barnes_2011, Chacón_Chen_Barnes_2013, Chen_Chacón_2014, Chen_Chacón_2015} have developed structure-preserving particle pushers for neoclassical transport in the Vlasov equations, derived from Crank--Nicolson integrators. We show these too can can derive from a FET interpretation, similarly offering potential extensions to higher-order-in-time particle pushers. The FET formulation is used also to consider how the stochastic drift terms can be incorporated into the pushers. Stochastic gyrokinetic expansions are also discussed.

        Different options for the numerical implementation of these schemes are considered.

        Due to the efficacy of FET in the development of SP timesteppers for both the fluid and kinetic component, we hope this approach will prove effective in the future for developing SP timesteppers for the full hybrid model. We hope this will give us the opportunity to incorporate previously inaccessible kinetic effects into the highly effective, modern, finite-element MHD models.
    \end{abstract}
    
    
    \newpage
    \tableofcontents
    
    
    \newpage
    \pagenumbering{arabic}
    %\linenumbers\renewcommand\thelinenumber{\color{black!50}\arabic{linenumber}}
            \input{0 - introduction/main.tex}
        \part{Research}
            \input{1 - low-noise PiC models/main.tex}
            \input{2 - kinetic component/main.tex}
            \input{3 - fluid component/main.tex}
            \input{4 - numerical implementation/main.tex}
        \part{Project Overview}
            \input{5 - research plan/main.tex}
            \input{6 - summary/main.tex}
    
    
    %\section{}
    \newpage
    \pagenumbering{gobble}
        \printbibliography


    \newpage
    \pagenumbering{roman}
    \appendix
        \part{Appendices}
            \input{8 - Hilbert complexes/main.tex}
            \input{9 - weak conservation proofs/main.tex}
\end{document}

            \documentclass[12pt, a4paper]{report}

\input{template/main.tex}

\title{\BA{Title in Progress...}}
\author{Boris Andrews}
\affil{Mathematical Institute, University of Oxford}
\date{\today}


\begin{document}
    \pagenumbering{gobble}
    \maketitle
    
    
    \begin{abstract}
        Magnetic confinement reactors---in particular tokamaks---offer one of the most promising options for achieving practical nuclear fusion, with the potential to provide virtually limitless, clean energy. The theoretical and numerical modeling of tokamak plasmas is simultaneously an essential component of effective reactor design, and a great research barrier. Tokamak operational conditions exhibit comparatively low Knudsen numbers. Kinetic effects, including kinetic waves and instabilities, Landau damping, bump-on-tail instabilities and more, are therefore highly influential in tokamak plasma dynamics. Purely fluid models are inherently incapable of capturing these effects, whereas the high dimensionality in purely kinetic models render them practically intractable for most relevant purposes.

        We consider a $\delta\!f$ decomposition model, with a macroscopic fluid background and microscopic kinetic correction, both fully coupled to each other. A similar manner of discretization is proposed to that used in the recent \texttt{STRUPHY} code \cite{Holderied_Possanner_Wang_2021, Holderied_2022, Li_et_al_2023} with a finite-element model for the background and a pseudo-particle/PiC model for the correction.

        The fluid background satisfies the full, non-linear, resistive, compressible, Hall MHD equations. \cite{Laakmann_Hu_Farrell_2022} introduces finite-element(-in-space) implicit timesteppers for the incompressible analogue to this system with structure-preserving (SP) properties in the ideal case, alongside parameter-robust preconditioners. We show that these timesteppers can derive from a finite-element-in-time (FET) (and finite-element-in-space) interpretation. The benefits of this reformulation are discussed, including the derivation of timesteppers that are higher order in time, and the quantifiable dissipative SP properties in the non-ideal, resistive case.
        
        We discuss possible options for extending this FET approach to timesteppers for the compressible case.

        The kinetic corrections satisfy linearized Boltzmann equations. Using a Lénard--Bernstein collision operator, these take Fokker--Planck-like forms \cite{Fokker_1914, Planck_1917} wherein pseudo-particles in the numerical model obey the neoclassical transport equations, with particle-independent Brownian drift terms. This offers a rigorous methodology for incorporating collisions into the particle transport model, without coupling the equations of motions for each particle.
        
        Works by Chen, Chacón et al. \cite{Chen_Chacón_Barnes_2011, Chacón_Chen_Barnes_2013, Chen_Chacón_2014, Chen_Chacón_2015} have developed structure-preserving particle pushers for neoclassical transport in the Vlasov equations, derived from Crank--Nicolson integrators. We show these too can can derive from a FET interpretation, similarly offering potential extensions to higher-order-in-time particle pushers. The FET formulation is used also to consider how the stochastic drift terms can be incorporated into the pushers. Stochastic gyrokinetic expansions are also discussed.

        Different options for the numerical implementation of these schemes are considered.

        Due to the efficacy of FET in the development of SP timesteppers for both the fluid and kinetic component, we hope this approach will prove effective in the future for developing SP timesteppers for the full hybrid model. We hope this will give us the opportunity to incorporate previously inaccessible kinetic effects into the highly effective, modern, finite-element MHD models.
    \end{abstract}
    
    
    \newpage
    \tableofcontents
    
    
    \newpage
    \pagenumbering{arabic}
    %\linenumbers\renewcommand\thelinenumber{\color{black!50}\arabic{linenumber}}
            \input{0 - introduction/main.tex}
        \part{Research}
            \input{1 - low-noise PiC models/main.tex}
            \input{2 - kinetic component/main.tex}
            \input{3 - fluid component/main.tex}
            \input{4 - numerical implementation/main.tex}
        \part{Project Overview}
            \input{5 - research plan/main.tex}
            \input{6 - summary/main.tex}
    
    
    %\section{}
    \newpage
    \pagenumbering{gobble}
        \printbibliography


    \newpage
    \pagenumbering{roman}
    \appendix
        \part{Appendices}
            \input{8 - Hilbert complexes/main.tex}
            \input{9 - weak conservation proofs/main.tex}
\end{document}

\end{document}

\end{document}

