\section{Ensuring Structure Preservation in Weak/Variational Formulations}
    When constructing timesteppers for systems with conserved/dissipated quantities, it is natural to seek ones that preserve these conservational/dissipative structures. There are many reasons one might seek to do this:
    \begin{itemize}
        \item  {\bf Qualitative motivations:} The physical origins for such quantities often imply important qualitative interpretations, such as the topology of the magnetic poential characterised by the magnetic helicity as discussed above. Discretization that violate these structure can subsequently be exhibit correspondingly unphysical behavior.
        \item  {\bf Quantiative motivations:} By way of example, the conservation of global quantities can often provide useful bounds on the outputs of the timesteppers, which can be useful in proving many results, such as well-posedness. The conservation of local quantities can alternatively help in providing parameter-robust bounds for convergence results, or allow the modification of the equations without fear of modification of the numerical results, such as in the use of augmented Lagrangians. \cite{FMW19, LFM22}
    \end{itemize}
    Whatever the motivation, it is logical to ensure the constructed timestepper preserved these structures.

    Consider then, a general time-dependent PDE (or system of PDEs) in the generic strong form
    \begin{equation}\label{eqn:general time-dependent PDE}
        \bfzero  =  \bfF\!\left[\bfu, \nabla\bfu, \nabla^{2}\bfu, \cdots; \partial_{t}\bfu\right]\!(\bfx; t),
    \end{equation}
    alongside appropriate boundary conditions (BCs). The physical origins of such systems often imply the existence of certain functionals $\rmE[\bfu](t)$ that are conserved/dissipated over time, e.g. such that $\forall  t$ either $\frac{\rmd}{\rmd t}\rmE  =  0$, $\frac{\rmd}{\rmd t}\rmE  \leq  0$, or $\frac{\rmd}{\rmd t}\rmE  =  \calO[\epsilon]$ for some parameter $\epsilon \ll 1$.
    
    \line
    
    Other than the energies and helicities considered above for the compressible MHD system, other classical examples one might consider include:
    \begin{example}[Heat equation]
        The heat equation,
        \begin{equation}\label{eqn:heat equation}
            \partial_{t}u  =  \frac{1}{\rmPe}\Delta u,
        \end{equation}
        under homogeneous Dirichlet or Neumann BCs, where the energy $\rmE(t)  :=  \int_{\bfOmega}u^{2}$ is dissipated according to
        \begin{equation}\label{eqn:heat equation dissipation}
            \frac{\rmd}{\rmd t}\rmE  =  - \frac{1}{\rmPe}\|\nabla u\|^{2}  \leq  0.
        \end{equation}
    \end{example}
    
    \begin{example}[Hamiltonian systems]
        Hamiltonian systems with Hamiltonian $\calH[p|_{t}, q|_{t}](t)$:
        \begin{align}
            \partial_{t}p  =  - \nabla_{q|_{t}}\calH,  &&
            \partial_{t}q  =  + \nabla_{p|_{t}}\calH,
        \end{align}
        for $L^{2}$ functional derivatives $\nabla_{p|_{t}}$, $\nabla_{q|_{t}}$, on $\bbR^{d}$, which exactly conserve the Hamiltonian,
        \begin{equation}
            \frac{\rmd}{\rmd t}\calH  =  0,
        \end{equation}
        such as:
        \begin{itemize}
            \item  The wave equation,
            \begin{align}
                \partial_{t}v  =  c^{2}\Delta u,  &&
                \partial_{t}u  =  v,
            \end{align}
            from the Hamiltonian
            \begin{equation}
                \calH(t)  :=  c^{2}\|\nabla u\|_{\bfOmega}^{2} + \|v\|_{\bfOmega}^{2}.
            \end{equation}
            \item  The Schrödinger equation,
            \begin{equation}
                i\hbar\partial_{t}\Psi  =  \left[- \frac{\hbar^{2}}{2m}\Delta + V\right]\Psi,
            \end{equation}
            or, from $\Psi  =  q + ip$,
            \begin{align}
                \partial_{t}p  =  - \frac{1}{\hbar}\left[- \frac{\hbar^{2}}{2m}\Delta + V\right]q,  &&
                \partial_{t}q  =  + \frac{1}{\hbar}\left[- \frac{\hbar^{2}}{2m}\Delta + V\right]p,
            \end{align}
            from the Hamiltonian
            \begin{equation}
                \calH(t)  :=  \frac{1}{\hbar}\int_{\bfOmega}\left[\frac{\hbar^{2}}{2m}\left(\|\nabla p\|^{2} + \|\nabla q\|^{2}\right) + V\left(p^{2} + q^{2}\right)\right],
            \end{equation}
            or
            \begin{equation}
                \calH(t)  :=  \frac{1}{\hbar}\int_{\bfOmega}\left[\frac{\hbar^{2}}{2m}\|\nabla\Psi\|^{2} + V|\Psi|^{2}\right].
            \end{equation}
        \end{itemize}
    \end{example}
    \line
    
    To apply the the finite-element method (FEM) to such a system, it must be cast into a weak/variational form.


    \section{Preserved Structures}
    \BA{Introduction.}
    
    Consider first those quantities that are conserved by the transient system, so as to seek discretisations which better represent the physical behaviour of the system by \emph{also} conserved these quantities. 
    
    \cite{LHF22} considers conservation of the following 3 quantities, which the authors define in the incompressible case as: \BA{(Oops I've never defined $\bfA$! That should probably be in the introduction...)}
    \begin{center}\begin{tabular}{ c c c }
        Properties  &  Symbol  &  Definition  \\
        \hline\hline
        Energy  &  $\rmE$  &  $\int_{\bfOmega}\left[\frac{1}{\rmEu\rho}\|\bfp\|^{2} + p + \frac{1}{\beta}\|\bfB\|^{2}\right]$  \\
        Magnetic helicity  &  $\rmH_{\rmM}$  &  $\int_{\bfOmega}\bfA\cdot\bfB$  \\
        Hybrid helicity  &  $\rmH_{\rmH}$  &  $\int_{\bfOmega}(a\bfA + \bfp)\cdot(b\bfB + \nabla\wedge\bfp)$
    \end{tabular}\end{center}
    where $a$, $b$ satisfy the relation $a + b  =  \frac{4}{\beta\rmRH}$. \BA{(What do these represent \emph{physically}? Diagrams!)} Taking the derivatives of these quantities over time (still in the incompressible system) gives
    \begin{align}
        \frac{d\rmE}{dt}  &=  \BA{\cdots}  \\
        \frac{d\rmH_{\rmM}}{dt}  &=  \int_{\bfGamma}(- \varphi\bfB + \bfA\wedge\bfE)\cdot\bfn - \frac{2}{\rmRem}\int_{\bfOmega}\bfB\cdot\bfj  \\
        \frac{d\rmH_{\rmH}}{dt}  &=  \BA{\cdots} \\
    \end{align}

    \BA{Proven that in the \emph{compressible} case, $\frac{d\rmE}{dt}$ evaluates as
    {\small \begin{equation}
        \frac{d\rmE}{dt}  =  \int_{\bfGamma}\left[- \frac{1}{2\rmEu\rho}\|\bfp\|^{2}\bfp - \frac{p}{2\rho}\bfp + \frac{1}{\rmEu\rmRe_{f}}\nabla\left[\frac{1}{\rho}\bfp\right]\cdot\frac{1}{\rho}\bfp - \frac{p}{2\rho}\bfp + \frac{1}{2\rmPe}\nabla\left[\frac{p}{\rho} + \frac{1}{\beta}\bfB\wedge\bfE\right]\right]\cdot\bfn
    \end{equation}}}
    
    \section{Preserved Structures}
    \BA{Introduction.}
    
    Consider first those quantities that are conserved by the transient system, so as to seek discretisations which better represent the physical behaviour of the system by \emph{also} conserved these quantities. 
    
    \cite{LHF22} considers conservation of the following 3 quantities, which the authors define in the incompressible case as: \BA{(Oops I've never defined $\bfA$! That should probably be in the introduction...)}
    \begin{center}\begin{tabular}{ c c c }
        Properties  &  Symbol  &  Definition  \\
        \hline\hline
        Energy  &  $\rmE$  &  $\int_{\bfOmega}\left[\frac{1}{\rmEu\rho}\|\bfp\|^{2} + p + \frac{1}{\beta}\|\bfB\|^{2}\right]$  \\
        Magnetic helicity  &  $\rmH_{\rmM}$  &  $\int_{\bfOmega}\bfA\cdot\bfB$  \\
        Hybrid helicity  &  $\rmH_{\rmH}$  &  $\int_{\bfOmega}(a\bfA + \bfp)\cdot(b\bfB + \nabla\wedge\bfp)$
    \end{tabular}\end{center}
    where $a$, $b$ satisfy the relation $a + b  =  \frac{4}{\beta\rmRH}$. \BA{(What do these represent \emph{physically}? Diagrams!)} Taking the derivatives of these quantities over time (still in the incompressible system) gives
    \begin{align}
        \frac{d\rmE}{dt}  &=  \BA{\cdots}  \\
        \frac{d\rmH_{\rmM}}{dt}  &=  \int_{\bfGamma}(- \varphi\bfB + \bfA\wedge\bfE)\cdot\bfn - \frac{2}{\rmRem}\int_{\bfOmega}\bfB\cdot\bfj  \\
        \frac{d\rmH_{\rmH}}{dt}  &=  \BA{\cdots} \\
    \end{align}

    \BA{Proven that in the \emph{compressible} case, $\frac{d\rmE}{dt}$ evaluates as
    {\small \begin{equation}
        \frac{d\rmE}{dt}  =  \int_{\bfGamma}\left[- \frac{1}{2\rmEu\rho}\|\bfp\|^{2}\bfp - \frac{p}{2\rho}\bfp + \frac{1}{\rmEu\rmRe_{f}}\nabla\left[\frac{1}{\rho}\bfp\right]\cdot\frac{1}{\rho}\bfp - \frac{p}{2\rho}\bfp + \frac{1}{2\rmPe}\nabla\left[\frac{p}{\rho} + \frac{1}{\beta}\bfB\wedge\bfE\right]\right]\cdot\bfn
    \end{equation}}}
    
