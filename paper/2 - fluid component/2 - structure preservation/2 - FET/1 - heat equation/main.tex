\subsubsection{Example 1: The Heat Equation}
    \begin{example}[CG-DG-in-time timestepper for the heat equation]
        Consider the heat equation (\ref{eqn:heat equation}) on $\bfOmega\otimes T^{n}$. We take homogeneous Neumann BCs, and suppose the ICs, $u|_{t = t^{n}}  =  \widehat{u}_{0}$ hold.
        
        In the strong form, the dissipation equation (\ref{eqn:heat equation dissipation}) holds.

        To cast into a chosen weak form \emph{in time}, define the solution space $\calU$, supported on $\bfOmega\otimes T^{n}$. To incorporate the ICs:
        \begin{enumerate}
            \item  Define $\calU_{-}  :=  \{u  \in  \calU  :  u|_{t = t^{n}}  =  0\}$ (with $\calU_{-}  <  \calU$).
            \item  Let $u_{0}  \in  \calU$ be an extension of the ICs $\widehat{u}_{0}$ to the space-time domain, $\bfOmega\otimes T^{n}$.
            \item  Decompose $u$ with a lifting of the ICs, as $u  :=  u_{0} + u_{-}$.
        \end{enumerate}
        One then seeks $u_{-}  \in  \calU_{-}$ such that,
        \begin{equation}
            (\partial_{t}u  =)  \bbQ_{\partial_{t}\calU}[\partial_{t}u]  =  \frac{1}{\rmPe}\bbQ_{\partial_{t}\calU}[\Delta u].
        \end{equation}
        In weak formulation, one can seek $u_{-}  \in  \calU_{-}$ such that $\forall  v  \in  \partial_{t}\calU$,
        \begin{equation}\label{eqn:heat equation weak form}
            \langle v, \partial_{t}u\rangle_{\bfOmega\otimes T^{n}}  =   - \frac{1}{\rmPe}\langle\nabla v, \nabla u\rangle_{\bfOmega\otimes T^{n}},
        \end{equation}
        
        To create an order-$s$ CG-DG FET timestepper, define the Galerkin subspace
        \begin{equation}
            \bbU^{\rmh}  :=  \widehat{\calU}(\bfOmega)\otimes\bbP^{s}\left(T^{n}\right)  <  \calU.
        \end{equation}
        Consequently
        \begin{equation}
            \partial_{t}\bbU^{\rmh}  :=  \widehat{\calU}(\bfOmega)\otimes\bbDP^{s - 1}\left(T^{n}\right)  <  \partial_{t}\calU.
        \end{equation}
        Mapping $\calU_{-}  \mapsto  \bbU^{\rmh}_{-}$ in (\ref{eqn:heat equation weak form}) invokes the Petrov--Galerkin projection. Denote bases for these FE spaces:
        \begin{align}
            (\phi_{j})_{j}     \text{ for }  \bbP^{s}\left(T^{n}\right),         &&
            (\varphi_{i})_{i}  \text{ for }  \bbDP^{s - 1}\left(T^{n}\right),
        \end{align}
        with $(\phi_{j})_{j}$ defined such that:
        \begin{align}
            \phi_{j}\left(t^{n}\right)      =  \delta_{j0},  &&
            \phi_{j}\left(t^{n + 1}\right)  =  \delta_{js}.
        \end{align}
        $u^{\rmh}  \in  \bbU^{\rmh}$, $v^{\rmh}  \in  \partial_{t}\bbU^{\rmh}$ can be expressed in terms of these bases as:
        \begin{align}
            u^{\rmh}(\bfx; t)  =  \sum_{j}\widehat{u}_{j}(\bfx)\phi_{j}(t),  &&
            v^{\rmh}(\bfx; t)  =  \sum_{i}\widehat{v}_{i}(\bfx)\varphi_{i}(t),
        \end{align}
        for $\left(\widehat{u}_{j}\right)_{j}$, $\left(\widehat{v}_{i}\right)_{i}$ in $\widehat{\calU}$. The Petrov--Galerkin-projected variational formulational then seeks $\left(\widehat{u}_{j}\right)_{j \neq 0}$ such that $\forall  \left(\widehat{v}_{i}\right)_{i}$ in $\widehat{\calU}$:
        \begin{align}
            \left\langle\sum_{i}\widehat{v}_{i}\varphi_{i}, \partial_{t}\left[\sum_{j}\widehat{u}_{j}\phi_{j}\right]\right\rangle_{\bfOmega\otimes T^{n}}  &=  - \frac{1}{\rmPe}\left\langle\nabla\left[\sum_{i}\widehat{v}_{i}\varphi_{i}\right], \nabla\left[\sum_{j}\widehat{u}_{j}\phi_{j}\right]\right\rangle_{\bfOmega\otimes T^{n}}  \\
            \sum_{i, j}\left\langle\widehat{v}_{i}, \widehat{u}_{j}\right\rangle_{\bfOmega}\langle\varphi_{i}, \partial_{t}\phi_{j}\rangle_{T^{n}}  &=  - \frac{1}{\rmPe}\sum_{i, j}\left\langle\nabla\widehat{v}_{i}, \nabla\widehat{u}_{j}\right\rangle_{\bfOmega}\langle\varphi_{i}, \phi_{j}\rangle_{T^{n}}
        \end{align}
        \vspace{-8mm}
        \begin{multline}
            \sum_{i, j \neq 0}\left(\left\langle\widehat{v}_{i}, \widehat{u}_{j}\right\rangle_{\bfOmega}\langle\varphi_{i}, \partial_{t}\phi_{j}\rangle_{T^{n}} + \frac{1}{\rmPe}\left\langle\nabla\widehat{v}_{i}, \nabla\widehat{u}_{j}\right\rangle_{\bfOmega}\langle\varphi_{i}, \phi_{j}\rangle_{T^{n}}\right)  \\
            =  - \sum_{i}\left(\left\langle\widehat{v}_{i}, \widehat{u}_{0}\right\rangle_{\bfOmega}\langle\varphi_{i}, \partial_{t}\phi_{0}\rangle_{T^{n}} + \frac{1}{\rmPe}\left\langle\nabla\widehat{v}_{i}, \nabla\widehat{u}_{0}\right\rangle_{\bfOmega}\langle\varphi_{i}, \phi_{0}\rangle_{T^{n}}\right)
        \end{multline}
        The final condition $u|_{t = t^{n + 1}}$ is then given simply as $u|_{t = t^{n + 1}}  =  \widehat{u}_{s}$. This quantifies the CG-DG-in-time timestepper.

        In the case of $s  =  1$---the lowest-order method---one seeks $\widehat{u}_{1}$, such that $\forall \widehat{v}_{1}$,
        \begin{equation}
            \left\langle\widehat{v}_{1}, \widehat{u}_{1}\right\rangle_{\bfOmega} + \frac{\delta t^{n}}{2\rmPe}\left\langle\nabla\widehat{v}_{1}, \nabla\widehat{u}_{1}\right\rangle_{\bfOmega}  =  \left\langle\widehat{v}_{1}, \widehat{u}_{0}\right\rangle_{\bfOmega} - \frac{\delta t^{n}}{2\rmPe}\left\langle\nabla\widehat{v}_{1}, \nabla\widehat{u}_{0}\right\rangle_{\bfOmega}.
        \end{equation}
        By rearranging, this can be seen to be equivalent to the traditional Crank--Nicolson method,
        \begin{equation}
            \left\langle\widehat{v}_{1}, \frac{1}{\delta t}\left(\widehat{u}_{1} - \widehat{u}_{0}\right)\right\rangle_{\bfOmega}  =  - \frac{1}{\rmPe}\left\langle\nabla\widehat{v}_{1}, \nabla\left[\frac{1}{2}\left(\widehat{u}_{1} + \widehat{u}_{0}\right)\right]\right\rangle_{\bfOmega}.
        \end{equation}
        For $s  >  1$---at higher order---the resultant timesteppers are here equivalent to the other higher-order GL methods. As stated earlier, these are all known to be exactly dissipative for the heat equation. Notably however, despite the use of FET such that $u$ is continuously defined over $\bfOmega\otimes T^{n}$, the dissipation in $\rmE$ is \emph{not} that which might be assumed from (\ref{eqn:heat equation dissipation}), $\frac{1}{\rmPe}\int_{\bfOmega\otimes T^{n}}\|\nabla u\|^{2}$, but a close and seemingly related value, e.g. as stated in (\ref{eqn:heat equation dissipation implicit midpoint}) when $s = 1$, $\frac{\delta t^{n}}{\rmPe}\int_{\bfOmega}\left\|\nabla\left[\frac{1}{2}\left(\widehat{u}_{1} + \widehat{u}_{0}\right)\right]\right\|^{2}$.
    \end{example}