\subsubsection{Constructing Structure-Preserving Timesteppers}
    It was observed above that the GL timesteppers derived from a CG-DG Petrov–Galerkin discretization of the heat equation preserved the dissipative properties of the continuous heat equation. Can this be made sense of through the lens of function spaces in time/FET?

    One might initially observe the weak formulation in time (\ref{eqn:heat equation weak form}), wherein one seeks $u_{-}  \in  \calU_{-}$ such that $\forall  v  \in  \calV$,
    \begin{equation*}
        \langle v, \partial_{t}u\rangle_{\bfOmega\otimes T}  =   - \frac{1}{\rmPe}\langle\nabla v, \nabla u\rangle_{\bfOmega\otimes T}.
    \end{equation*}
    One can observe that, from a naïve theoretical standpoint, were $\calU  \leqslant  \calV$, one would necessarily know $u  \in  \calV$. Substituting $v  \mapsto  u$ in (\ref{eqn:heat equation weak form}) gives:
    \begin{align}
        \langle u, \partial_{t}u\rangle_{\bfOmega\otimes T}  &=   - \frac{1}{\rmPe}\langle\nabla u, \nabla u\rangle_{\bfOmega\otimes T}\label{eqn:heat equation dissipation proof 1}  \\
        \partial_{t}\left[\frac{1}{2}\int_{\bfOmega\otimes T}u^{2}\right]  &=   - \frac{1}{\rmPe}\int_{\bfOmega\otimes T}\|\nabla u\|^{2}  \\
        \rmE\left(t^{N}\right)  &\leq  \rmE(0)\label{eqn:heat equation dissipation proof 3}
    \end{align}
    for $T  =  \left(0, t^{N}\right]$, i.e. necessary dissipation on $\rmE$.
    
    The problem here derives from the requirement that $\calU  \leqslant  \calV$. After discretization, the spaces $\bbU^{h}_{-}  <  \bbU^{h}  <  \calU$, $\bbV^{h}  <  \calV$ must be finite-dimensional. The condition $\bbU^{h}  \leqslant  \bbV^{h}$ requires
    \begin{equation}
        \dim\left[\bbU^{h}_{-}\right]  \leq  \dim\left[\bbV^{h}\right].
    \end{equation}
    With $\bbU^{h}_{-}  <  \bbU^{h}$, one finds
    \begin{equation}
        \dim\left[\bbU^{h}_{-}\right]  <  \left(\dim\left[\bbU^{h}_{-}\right]  \leq\right)  \dim\left[\bbV^{h}\right].
    \end{equation}
    For well-posedness however, one requires,
    \begin{equation}
        \dim\left[\bbU^{h}_{-}\right]  =  \dim\left[\bbV^{h}\right],
    \end{equation}
    leading to a necessary contradiction if this condition is to hold after discretization. \contra

    The problem ultimately derives from requiring a ``trial space'' $\leqslant$ ``test space'' condition where one intends to strongly enforce ICs on the given trial space.

    One can often avoid this by the introduction of an auxiliary function for ``problematic'' functions, denoted here as $\widetilde{*}$, where in the strong formulation $\widetilde{*}  =  *$, but in the weak formulation, $\widetilde{*}$ has a lower continuity in time, without the enforced ICs. The initial system can be tested against $\widetilde{*}$, and the auxiliary equation can be tested against $\widetilde{*}  =  *$.

    This generally strategy can be adapted to bring proofs like that in (\ref{eqn:heat equation dissipation proof 1}–\ref{eqn:heat equation dissipation proof 3}) within reach.

    \BA{How much do the introduction of these auxiliary spaces change the convergence/accuracy of the solutions?}

    \BA{Clarify: Much of this not a problem for periodic problems without ICs.}

    