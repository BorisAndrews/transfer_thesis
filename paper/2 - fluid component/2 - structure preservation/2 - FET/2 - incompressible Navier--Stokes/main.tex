\subsubsection*{Example 2: The \emph{Incompressible} Navier--Stokes Equations}
    Consider the \emph{incompressible} Navier--Stokes (NS) equations, in the classical strong form:
    \begin{align}
                       0  &=  \nabla\cdot\bfu,  \label{eqn:incompressible NS mass conservation}  \\
        \partial_{t}\bfu  &=  - \bfu\cdot\nabla\bfu - \nabla p + \frac{1}{\rmRe}\Delta\bfu,  \label{eqn:incompressible NS classical momentum conservation}
    \end{align}
    under homogeneous Dirichlet BCs in $\bfu$ ($\bfzero  =  \bfu|_{\bfGamma}$) with ICs on $\bfu$.
    
    The incompressible NS system can be seen to dissipate (kinetic) energy,
    \begin{equation}\label{eqn:incompressible NS energy definition}
        \rmE[\bfu](t)  :=  \frac{1}{2}\int_{\bfOmega}\|\bfu\|^{2},
    \end{equation}
    with
    \begin{equation}
        \frac{\rmd}{\rmd t}\rmE  =  - \frac{2}{\rmRe}\int_{\bfOmega}\|\nabla_{\rms}\bfu\|^{2}  \leq  0.
    \end{equation}

    By incompressibility, the momentum equation (\ref{eqn:incompressible NS classical momentum conservation}) can be rewritten in the following ``skew-symmetric'' form, to ease the analysis and to motivate a weak form that inherits the energy-dissipative structure,
    \begin{equation}\label{eqn:incompressible NS skew-symmetric momentum equation}
        \partial_{t}\bfu  =  - \frac{1}{2}\left(\nabla\cdot\left[\bfu^{\otimes 2}\right] + \bfu\cdot\nabla\bfu\right) - \nabla p + \frac{2}{\rmRe}\nabla\cdot\nabla_{\rms}\bfu,
    \end{equation}
    since:
    \begin{align}
        &\nabla\cdot\left[\bfu^{\otimes 2}\right]   &&=  (\mst{\nabla\cdot\bfu})\bfu + \bfu\cdot\nabla\bfu  &&=  \bfu\cdot\nabla\bfu  \\
        &\nabla\cdot\left[\nabla\bfu^{\rmT}\right]  &&=  \nabla[\mst{\nabla\cdot\bfu}]                      &&=  0
    \end{align}
    where $\nabla_{\rms}$ again denotes the symmetric gradient.

    \begin{remark}
        Alternative skew-symmetric forms for the convective term, such as those incorporating the vorticity $\bfomega  :=  \nabla\wedge\bfu$, exist. While these too induce timestepper that preserve the energy-dissipative structure, we choose not to consider them further here.
    \end{remark}

    \begin{remark}
        \BA{If the weak form has exact incomp., then it doesn't matter how you write the convective term (classical/skew-symmetric etc.) since it shouldn't affect the solution. Only the skew-symmetric form gives energy-conserving timesteppers even without exact incomp. A similar result holds true for the \emph{compressible} case.}
    \end{remark}

    We can again use the auxiliary function idea from the heat equation to convert (\ref{eqn:incompressible NS mass conservation}, \ref{eqn:incompressible NS skew-symmetric momentum equation}) into a weak form in space-time that invokes an energy-dissipative SP timestepper. Letting $\bfu$ similarly contain a lifting of the ICs, one can seek $\bfu_{-} \in \calU_{-}$, $p \in \calP$ such that: 
    \begin{align}
                                                                            0  &=  \bbQ_{\calP}[\nabla\cdot\bbQ_{\partial_{t}\calU}[\bfu]],  \label{eqn:incompressible NS weak form 1}  \\
        \begin{split}
            (\partial_{t}\bfu  =)  \bbQ_{\partial_{t}\calU}[\partial_{t}\bfu]  &=  - \frac{1}{2}\bbQ_{\partial_{t}\calU}\left[\nabla\cdot\left[\bfu^{*}\otimes\bbQ_{\partial_{t}\calU}[\bfu]\right] + \bfu^{*}\cdot\nabla\bbQ_{\partial_{t}\calU}[\bfu]\right] - \bbQ_{\partial_{t}\calU}[\nabla p]  \\
                                                                               &\qquad\qquad\qquad\qquad\qquad\qquad\qquad\qquad\qquad+ \frac{2}{\rmRe}\bbQ_{\partial_{t}\calU}[\nabla\cdot\nabla_{\rms}\bbQ_{\partial_{t}\calU}[\bfu]],
        \end{split}  \label{eqn:incompressible NS weak form 2}
    \end{align}
    We write $\bfu^{*}$ here to denote \emph{either} $\bfu$ \emph{or} the auxiliary projection $\bbQ_{\partial_{t}\calU}[\bfu]$. The energy-dissipative SP property of the resultant timestepper is indepedent of this choice. In variational form, one can seek $\bfu_{-} \in \calU_{-}$, $\widetilde{\bfu} \in \partial_{t}\calU$, $p \in \calP$ such that:
    \begin{align}
        &\forall                 q  \in  \calP,              &                                                            0  &=  \left\langle q, \nabla\cdot\widetilde{\bfu}\right\rangle  \\
        &\forall  \widetilde{\bfv}  \in  \partial_{t}\calU,  &  \left\langle\widetilde{\bfv}, \partial_{t}\bfu\right\rangle  &=  \calA\left[\bfu^{*}; \widetilde{\bfv}, \widetilde{\bfu}\right] + \left\langle\nabla\cdot\widetilde{\bfv}, p\right\rangle - \frac{2}{\rmRe}\left\langle\nabla_{\rms}\widetilde{\bfv}, \nabla_{\rms}\widetilde{\bfu}\right\rangle  \\
        &\forall              \bfv  \in  \partial_{t}\calU,  &                                                            0  &=  \left\langle\bfv, \widetilde{\bfu}\right\rangle - \left\langle\bfv, \bfu\right\rangle
    \end{align}
    where similarly all inner products $\langle -, -\rangle  =  \langle -, -\rangle_{\bfOmega\otimes T^{n}}$ are taken over the space-time domain, $\bfOmega\otimes T^{n}$, and $\calA[-; -, -]$ is the trilinear convective operator, defined
    \begin{align}
        \calA\left[\bfu^{*}; \widetilde{\bfv}, \widetilde{\bfu}\right]  :=  &\;\frac{1}{2}\left(\left\langle\nabla\widetilde{\bfv}, \widetilde{\bfu}\otimes\bfu^{*}\right\rangle - \left\langle\widetilde{\bfv}, \bfu^{*}\cdot\nabla\widetilde{\bfu}\right\rangle\right)  \label{eqn:convective operator definition}  \\
                                                                         =  &\;\frac{1}{2}\int_{\bfOmega\otimes T}\bfu^{*}\cdot\left(\nabla\widetilde{\bfv}\cdot\widetilde{\bfu} - \nabla\widetilde{\bfu}\cdot\widetilde{\bfv}\right)
    \end{align}
    This has the crucial skew-symmetry property $\calA\left[\bfu^{*}; \widetilde{\bfu}, \widetilde{\bfu}\right]  =  0$, $\forall \bfu^{*}, \widetilde{\bfu}$.

    \begin{remark}
        \BA{Crucial that non-$\bfgrad$-conforming discretizations preserve this skew-symmetry property for energy dissipation! (In fact, is suff. that $\calA  \leq  0$, s.t. various \emph{dissipative} fluxes (e.g. upwinding) can be considered.)}
    \end{remark}

    One can again use the weak form (\ref{eqn:incompressible NS weak form 1}--\ref{eqn:incompressible NS weak form 2}) directly to mimic the proof of energy dissipation:
    \begin{align}
        \rmE\left(t^{n + 1}\right) - \rmE\left(t^{n}\right)  &=  \int_{T^{n}}\frac{\rmd}{\rmd t}\rmE  \\
        &=  \int_{T^{n}}\frac{\rmd}{\rmd t}\left[\frac{1}{2}\int_{\bfOmega}\|\bfu\|^{2}\right]  \\
        &=  \int_{\bfOmega\otimes T^{n}}\bfu\cdot\partial_{t}\bfu  \\
        &=  \int_{\bfOmega\otimes T^{n}}\bfu\cdot\left(- \frac{1}{2}\bbQ_{\partial_{t}\calU}\left[\nabla\cdot\left[\bfu^{*}\otimes\bbQ_{\partial_{t}\calU}[\bfu]\right] + \right. \cdots + \frac{2}{\rmRe}\bbQ_{\partial_{t}\calU}[\nabla\cdot\nabla_{\rms}\bbQ_{\partial_{t}\calU}[\bfu]]\right)  \\
        &=  \int_{\bfOmega\otimes T^{n}}\left[\mst{\calA\left[\bfu^{*}; \bbQ_{\partial_{t}\calU}[\bfu], \bbQ_{\partial_{t}\calU}[\bfu]\right]} + \nabla\cdot\bbQ_{\partial_{t}\calU}[\bfu]p - \frac{2}{\rmRe}\|\nabla_{\rms}\bbQ_{\partial_{t}\calU}[\bfu]\|^{2}\right]  \\
        &=  \int_{\bfOmega\otimes T^{n}}\left[\mst{\bbQ_{\calP}[\nabla\cdot\bbQ_{\partial_{t}\calU}[\bfu]]}p - \frac{2}{\rmRe}\|\nabla_{\rms}\bbQ_{\partial_{t}\calU}[\bfu]\|^{2}\right]  \\
        &=  - \frac{2}{\rmRe}\int_{\bfOmega\otimes T^{n}}\|\nabla_{\rms}\bbQ_{\partial_{t}\calU}[\bfu]\|^{2}  \leq  0
    \end{align}
    Thus, the resultant timestepper preserves the energy-dissipative structure.

    Similar to the classical stationary-state case, if the following diagram commutes,
    \begin{center}\begin{tikzpicture}[align = center, node distance = 4cm, auto]
        \node (Ut*) at (0, 0) {$(\partial_{t}\calU)^{*}$};
        \node (P*)  at (3.5,   0) {$\calP^{*}$};
        \draw[->] (Ut*) -- (P*) node[above, midway] {$\rmdiv$};

        \node (Ut) at (0, - 2) {$\partial_{t}\calU$};
        \node (P)  at (3.5,   - 2) {$\calP$};
        \draw[->] (Ut) -- (P) node[above, midway] {$\rmdiv$};

        \draw[->] (Ut*) -- (Ut) node[left, midway] {$\bbQ_{\partial_{t}\calU}$};
        \draw[->] (P*)  -- (P)  node[left, midway] {$\bbQ_{\calP}$};
    \end{tikzpicture}\end{center}
    then (\ref{eqn:incompressible NS weak form 1}) directly gives pointwise incompressibility on the \emph{auxiliary} space, $\bbQ_{\partial_{t}\calU}[\bfu]$,
    \begin{equation}
        \nabla\cdot\bbQ_{\partial_{t}\calU}[\bfu]  =  0.
    \end{equation}
    
    \begin{remark}
        I believe that, when $\calU$ is a tensor product space and the incompressibility condition $\nabla\cdot\bfu  =  0|_{t = 0}$ holds exactly on the IC, the above result is sufficient to prove pointwise incompressible on $\bfu$ itself, $\nabla\cdot\bfu = 0$. Indeed this is true simply by dimensionality when the time component of such a space-time tensor product construction for $\calU$ is finite-dimensional, as is the case in after discretization. I believe it should hold in the continuous case, but I've yet to sort the technicalities of such a proof. Perhaps some new commutation property is required that is satisfied in the case of tensor product $\calU$?
    \end{remark}

    When using a tensor product space for $\calU$ in the heat equation, we noted that above that the auxiliary function to be computed, $\bbQ_{\partial_{t}\calU}[u]$, could be eliminated from the weak form (\ref{eqn:heat equation auxiliary weak form}). Due to the nonlinearity of the Navier--Stokes system, this is no longer possible here. An alternative possibility is however available. Since, for tensor product spaces, $\bbQ_{\partial_{t}\calU}$ acts on $\calU$ only in the time component, $\bbQ_{\partial_{t}\calU}[\bfu]$ can be computed offline $\forall \bfu (\in \calU)$. To illustrate this, consider the following CG-DG-in-time example discretization.
    
    \BA{Can $\partial_{t}\calU$ ever be a FE space if $\calU$ is not a tensor product space?}

    \line

    \begin{example}[CG-DG-in-time dissipative timestepper for incompressible NS]
        Discretizing in time using an order-$s$ CG discretization for the time-continuous spaces ($\calU$) and an order-$(s - 1)$ DG discretization for the time-discontinuous spaces (everything else), consider the following trial and test spaces:
        \begin{align}
            \calU  &=  \widehat{\calU}(\bfOmega)\otimes\bbP^{s}\left(T^{\rmh}\right),  \\
            \widetilde{\calU}  =  \calV  =  \widetilde{\calV}  &=  \widehat{\calU}(\bfOmega)\otimes\bbDP^{s - 1}\left(T^{\rmh}\right),  \\
            \calP  =  \calQ  &=  \widehat{\calP}(\bfOmega)\otimes\bbDP^{s - 1}\left(T^{\rmh}\right)
        \end{align}
        At lowest order ($s = 1$), after elimination of $\widetilde{\bfu}$ (akin to the CG-DG-in-time timestepper for the heat equation, with $\widetilde{\bfu}^{n}  =  \frac{1}{2}\left(\bfu^{n + 1} + \bfu^{n}\right)$) and taking $\bfu^{*} = \bfu$ (in the convective term, $\calA$) the resultant timestepper, solving for $\left(\widehat{\bfu}^{n}\right)^{n}$, $\left(\widehat{p}^{n}\right)^{n}$ takes the form:
        \begin{align*}
            &\forall  \widehat{q}^{n}     \in  \widehat{\calP},  &                                                                                                               0  &=  \left\langle \widehat{q}^{n}, \frac{1}{2}\left(\widehat{\bfu}^{n + 1} + \widehat{\bfu}^{n}\right)\right\rangle  \\
            &\forall  \widehat{\bfv}^{n}  \in  \widehat{\calU},  &  \frac{1}{\delta t^{n}}\left\langle\widehat{\bfv}^{n}, \widehat{\bfu}^{n + 1} - \widehat{\bfu}^{n}\right\rangle  &=  \calA\left[\frac{1}{2}\left(\widehat{\bfu}^{n + 1} + \widehat{\bfu}^{n}\right); \widehat{\bfv}^{n}, \frac{1}{2}\left(\widehat{\bfu}^{n + 1} + \widehat{\bfu}^{n}\right)\right]  \\
            &                                                    &                                                                                                                  &\;\;\;\;\;\;\;\;\;\;\;\;\;\;\;\;  + \left\langle\nabla\cdot\widehat{\bfv}^{n}, \widehat{p}^{n}\right\rangle  \\
            &                                                    &                                                                                                                  &\;\;\;\;\;\;\;\;\;\;\;\;\;\;\;\;\;\;\;\;\;\;\;\;\;\;\;\;\;\;\;\;  - \frac{2}{\rmRe}\left\langle\nabla_{\rms}\widehat{\bfv}^{n}, \nabla_{\rms}\left[\frac{1}{2}\left(\widehat{\bfu}^{n + 1} + \widehat{\bfu}^{n}\right)\right]\right\rangle
        \end{align*}
        where all inner products $\langle -, -\rangle  =  \langle -, -\rangle_{\bfOmega}$ are taken over $\bfOmega$, and $\calA[-; -, -]$ is the trilinear convective operator as defined in (\ref{eqn:convective operator definition}), resembling the implicit midpoint method.

        From (\ref{eqn:weak NS dissipation}) when $T  =  [t^{n}, t^{n + 1}]$, this has the exact energy dissipation:
        \begin{equation}
            \rmE\left(t^{n + 1}\right)  -  \rmE\left(t^{n}\right)  =  \delta t^{n}\int_{\bfOmega}\left\|\nabla_{\rms}\left[\frac{1}{2}\left(\widehat{\bfu}^{n + 1} + \widehat{\bfu}^{n}\right)\right]\right\|^{2}
        \end{equation}

        \BA{Maybe include an example of the timestepper for $s = 2$, to show that these do in fact scale to arbitrarily high order.}
    \end{example}
