\subsubsection*{Example 2: The \emph{Incompressible} Navier--Stokes Equations}
    Consider the \emph{incompressible} Navier--Stokes (NS) equations, in the classical strong form:
    \begin{align}
                       0  &=  \nabla\cdot\bfu,  \label{eqn:incompressible NS mass conservation}  \\
        \partial_{t}\bfu  &=  - \bfu\cdot\nabla\bfu - \nabla p + \frac{1}{\rmRe}\Delta\bfu,  \label{eqn:incompressible NS classical momentum conservation}
    \end{align}
    under homogeneous Dirichlet BCs in $\bfu$ ($\bfzero  =  \bfu|_{\bfGamma}$) with ICs on $\bfu$.

    By incompressibility, the momentum equation (\ref{eqn:incompressible NS classical momentum conservation}) can be rewritten in the following ``skew-symmetric'' form, to ease the analysis and to motivate a weak form that inherits the energy-dissipative structure,
    \begin{equation}\label{eqn:incompressible NS skew-symmetric momentum equation}
        \partial_{t}\bfu  =  - \frac{1}{2}\left(\nabla\cdot\left[\bfu^{\otimes 2}\right] + \bfu\cdot\nabla\bfu\right) - \nabla p + \frac{2}{\rmRe}\nabla\cdot\nabla_{\rms}\bfu,
    \end{equation}
    where $\nabla_{\rms}$ again denotes the symmetric gradient, since:
    \begin{align}
        &\nabla\cdot\left[\bfu^{\otimes 2}\right]   &&=  (\mst{\nabla\cdot\bfu})\bfu + \bfu\cdot\nabla\bfu  &&=  \bfu\cdot\nabla\bfu  \\
        &\nabla\cdot\left[\nabla\bfu^{\rmT}\right]  &&=  \nabla[\mst{\nabla\cdot\bfu}]                      &&=  0
    \end{align}

    \begin{remark}
        \BA{If the weak form has exact incomp., then it doesn't matter how you write the convective term (classical/skew-symmetric etc.) since it shouldn't affect the solution. Only the skew-symmetric form gives energy-conserving timesteppers even without exact incomp. A similar result holds true for the \emph{compressible} case.}
    \end{remark}

    \begin{remark}
        \BA{Alternative skew-symmetric convective operators exist.}
    \end{remark}
    
    The incompressible NS system can be seen to dissipate (kinetic) energy,
    \begin{equation}\label{eqn:incompressible NS energy definition}
        \rmE[\bfu](t)  :=  \frac{1}{2}\int_{\bfOmega}\|\bfu\|^{2},
    \end{equation}
    with
    \begin{equation}
        \frac{\rmd}{\rmd t}\rmE  =  - \frac{2}{\rmRe}\int_{\bfOmega}\|\nabla_{\rms}\bfu\|^{2}  \leq  0.
    \end{equation}

    We can again use the auxiliary function idea from the heat equation to convert (\ref{eqn:incompressible NS mass conservation}, \ref{eqn:incompressible NS skew-symmetric momentum equation}) into a weak form in space-time that invokes an energy-dissipative SP timestepper. Letting $\bfu$ similarly contain a lifting of the ICs, one can seek $\bfu_{-} \in \calU_{-}$, $p \in \calP$ such that: 
    \begin{align}
                                                                            0  &=  \bbQ_{\calP}[\nabla\cdot\bbQ_{\partial_{t}\calU}[\bfu]],  \\
        \begin{split}
            (\partial_{t}\bfu  =)  \bbQ_{\partial_{t}\calU}[\partial_{t}\bfu]  &=  - \frac{1}{2}\bbQ_{\partial_{t}\calU}\left[\nabla\cdot\left[\bfu^{*}\otimes\bbQ_{\partial_{t}\calU}[\bfu]\right] + \bfu^{*}\cdot\nabla\bbQ_{\partial_{t}\calU}[\bfu]\right] - \bbQ_{\partial_{t}\calU}[\nabla p]  \\
                                                                               &\qquad\qquad\qquad\qquad\qquad\qquad\qquad\qquad\qquad+ \frac{2}{\rmRe}\bbQ_{\partial_{t}\calU}[\nabla\cdot\nabla_{\rms}\bbQ_{\partial_{t}\calU}[\bfu]],
        \end{split}
    \end{align}
    We write $\bfu^{*}$ here to denote \emph{either} $\bfu$ \emph{or} the auxiliary projection $\bbQ_{\partial_{t}\calU}[\bfu]$. The SP property of the resultant timestepper is indepedent of this choice. In variational form, one can seek $\bfu_{-} \in \calU_{-}$, $\widetilde{\bfu} \in \partial_{t}\calU$, $p \in \calP$ such that:
    \begin{align}
        &\forall                 q  \in  \calP,              &                                                0  &=  \left\langle q, \nabla\cdot\widetilde{\bfu}\right\rangle  \\
        &\forall              \bfv  \in  \partial_{t}\calU,  &  \left\langle\bfv, \partial_{t}\bfu\right\rangle  &=  \calA\left[\bfu^{*}; \bfv, \widetilde{\bfu}\right] + \left\langle\nabla\cdot\bfv, p\right\rangle - \frac{2}{\rmRe}\left\langle\nabla_{\rms}\bfv, \nabla_{\rms}\widetilde{\bfu}\right\rangle  \\
        &\forall  \widetilde{\bfv}  \in  \partial_{t}\calU,  &                                                0  &=  \left\langle\widetilde{\bfv}, \widetilde{\bfu}\right\rangle - \left\langle\widetilde{\bfv}, \bfu\right\rangle
    \end{align}
    where similarly all inner products $\langle -, -\rangle  =  \langle -, -\rangle_{\bfOmega\otimes T^{n}}$ are taken over the space-time domain, $\bfOmega\otimes T^{n}$, and $\calA[-; -, -]$ is the trilinear convective operator, defined
    \begin{align}
        \calA\left[\bfu^{*}; \bfv, \widetilde{\bfu}\right]  :=  &\;\frac{1}{2}\left(\left\langle\nabla\bfv, \widetilde{\bfu}\otimes\bfu^{*}\right\rangle - \left\langle\bfv, \bfu^{*}\cdot\nabla\widetilde{\bfu}\right\rangle\right)  \label{eqn:convective operator definition}  \\
                                                             =  &\;\frac{1}{2}\int_{\bfOmega\otimes T}\bfu^{*}\cdot\left(\nabla\bfv\cdot\widetilde{\bfu} - \nabla\widetilde{\bfu}\cdot\bfv\right)
    \end{align}
    with the crucial skew-symmetry property $\calA\left[\bfu^{*}; \widetilde{\bfu}, \widetilde{\bfu}\right] 
     =  0$, $\forall \bfu^{*}, \widetilde{\bfu}$.

    \BA{Local incompressible SP.}

    \BA{$\bbQ_{\partial_{t}\calU}[\bfu]$ can be computed offline in tensor product spaces.}

    Consider now how one might cast this into a weak formulation in time that both:
    \begin{itemize}
        \item  Invokes a timestepper after discretization.
        \item  Inherits this energy-dissipative property.
    \end{itemize}
    Suppose the weak form seeks:
    \begin{align}
           p  \in  \calP,  &&
        \bfu  \in  \calU,
    \end{align}
    and seeks to test (in $L^{2}$):
    \begin{align}
                           \text{(\ref{eqn:incompressible NS mass conservation}) against }    q \in \calQ,  &&
        \text{(\ref{eqn:incompressible NS skew-symmetric momentum equation}) against } \bfv \in \calV.
    \end{align}
    For the results (\ref{eqn:mass tested against p}), (\ref{eqn:momentum tested against u}) to hold, the following subspace criteria are required:
    \begin{align}
        \calP  \leqslant  \calQ,  &&
        \calU  \leqslant  \calV,
    \end{align}
    however, as established above, the latter condition $\calU  \leqslant  \calV$ can \emph{not} invoke a timestepper after discretization, as ICs are to be imposed on $\bfu  \in  \calU$, which is involved in a necessary subspace crietrion. \contra
    
    Again, as above, this can be combated by the introduction of an auxiliary function $\widetilde{\bfu}$, with $\widetilde{\bfu}  =  \bfu$ in the strong formulation. Consider the auxiliary strong form:
    \begin{align}
                       0  &=  \nabla\cdot\widetilde{\bfu},  \\
        \partial_{t}\bfu  &=  - \frac{1}{2}\left(\nabla\cdot\left[\bfu^{*}\otimes\widetilde{\bfu}\right] + \bfu^{*}\cdot\nabla\bfu\right) - \nabla p + \frac{2}{\rmRe}\nabla\cdot[\nabla_{\rms}\widetilde{\bfu}],  \\
                 \bfzero  &=  \widetilde{\bfu} - \bfu,
    \end{align}
    under homogeneous Dirichlet BCs in $\bfu$ \emph{and} $\widetilde{\bfu}$ ($\bfzero  =  \bfu  =  \widetilde{\bfu}|_{\bfGamma}$) and ICs on $\bfu$ \emph{only}. $\bfu^{*}$ can be taken as either $\bfu$ or $\widetilde{\bfu}$ without affecting energy conservation.
    
    In weak form, one seeks:
    \begin{align}
                       p  \in  \calP,  &&
        \widetilde{\bfu}  \in  \widetilde{\calU},  &&
                    \bfu  \in  \calU,
    \end{align}
    such that:
    \begin{align}
        &\forall                 q  \in  \calQ,              &                                                            0  &=  \left\langle q, \nabla\cdot\widetilde{\bfu}\right\rangle  \\
        &\forall  \widetilde{\bfv}  \in  \widetilde{\calV},  &  \left\langle\widetilde{\bfv}, \partial_{t}\bfu\right\rangle  &=  \calA\left[\bfu^{*}; \widetilde{\bfv}, \widetilde{\bfu}\right] + \left\langle\nabla\cdot\widetilde{\bfv}, p\right\rangle - \frac{2}{\rmRe}\left\langle\nabla_{\rms}\widetilde{\bfv}, \nabla_{\rms}\widetilde{\bfu}\right\rangle  \\
        &\forall              \bfv  \in  \calV,              &                                                            0  &=  \left\langle\bfv, \widetilde{\bfu}\right\rangle - \left\langle\bfv, \bfu\right\rangle
    \end{align}

    \begin{remark}
        \BA{Crucial that non-$\bfgrad$-conforming discretizations preserve this skew-symmetry property for energy dissipation! (In fact, is suff. that $\calA  \leq  0$, s.t. various \emph{dissipative} fluxes (e.g. upwinding) can be considered.)}
    \end{remark}

    \begin{theorem}[Energy dissipation for the auxiliary weak formulation of incompressible NS]
        Presuming the following subspace criteria hold (likely with equality for well-posedness):
        \begin{align}
                        \calP  \leqslant  \calQ,  &&
            \widetilde{\calU}  \leqslant  \widetilde{\calV},  &&
            \partial_{t}\calU  \leqslant  \calV,
        \end{align}
        then $\rmE\left(t^{N}\right)  \leq  \rmE(0)$, with
        \begin{equation}
            \rmE\left(t^{N}\right) - \rmE(0)  =  - \frac{2}{\rmRe}\int_{\bfOmega\otimes T}\left\|\nabla_{\rms}\widetilde{\bfu}\right\|^{2}
        \end{equation}
    \end{theorem}
    \begin{proof}
        From the given subspace criteria, the following results necessarily hold:
        \begin{align}
            p                 &\in  (\calP              \leqslant)  \calQ              &&\implies  &                                                              0  &=  \langle p, (\nabla\cdot\widetilde{\bfu})\rangle  \\
            &                                                                          &&          &                                                              0  &=  \int_{\bfOmega\otimes T}p\left(\nabla\cdot\widetilde{\bfu}\right)  \\
            \widetilde{\bfu}  &\in  (\widetilde{\calU}  \leqslant)  \widetilde{\calV}  &&\implies  &    \left\langle\widetilde{\bfu}, \partial_{t}\bfu\right\rangle  &=  \mst{\calA\left[\bfu^{*}; \widetilde{\bfu}, \widetilde{\bfu}\right]} + \left\langle\nabla\cdot\widetilde{\bfu}, p\right\rangle - \frac{2}{\rmRe}\left\langle\nabla_{\rms}\widetilde{\bfu}, \nabla_{\rms}\widetilde{\bfu}\right\rangle  \\
            &                                                                          &&          &  \int_{\bfOmega\otimes T}\widetilde{\bfu}\cdot\partial_{t}\bfu  &=  \int_{\bfOmega\otimes T}\left[\left(\nabla\cdot\widetilde{\bfu}\right)p - \frac{2}{\rmRe}\left\|\nabla_{\rms}\widetilde{\bfu}\right\|^{2}\right]  \\
            \partial_{t}\bfu  &\in  (\partial_{t}\calU  \leqslant)  \calV              &&\implies  &                                                              0  &=  \left\langle\partial_{t}\bfu, \widetilde{\bfu}\right\rangle - \left\langle\partial_{t}\bfu, \bfu\right\rangle  \\
            &                                                                          &&          &                                                              0  &=  \int_{\bfOmega\otimes T}\left[\partial_{t}\bfu\cdot\widetilde{\bfu} - \partial_{t}\bfu\cdot\bfu\right]  \\
        \end{align}
        Summing these:
        \begin{align}
            \int_{\bfOmega\otimes T}\left[\left(\partial_{t}\bfu\cdot\bfu - \mst{\partial_{t}\bfu\cdot\widetilde{\bfu}}\right) + \mst{\widetilde{\bfu}\cdot\partial_{t}\bfu}\right]  &=  \int_{\bfOmega\otimes T}\left[- \mst{p\left(\nabla\cdot\widetilde{\bfu}\right)} + \left(\mst{\left(\nabla\cdot\widetilde{\bfu}\right)p} - \frac{2}{\rmRe}\left\|\nabla_{\rms}\widetilde{\bfu}\right\|^{2}\right)\right]  \\
            \partial_{t}\left[\frac{1}{2}\int_{\bfOmega\otimes T}\|\bfu\|^{2}\right]  &=  - \frac{2}{\rmRe}\int_{\bfOmega\otimes T}\left\|\nabla_{\rms}\widetilde{\bfu}\right\|^{2}  \label{eqn:weak NS dissipation}  \\
            \rmE\left(t^{N}\right)  &\leq  \rmE(0)
        \end{align}
    \end{proof}

    \begin{example}[CG-DG-in-time dissipative timestepper for incompressible NS]
        Discretizing in time using an order-$s$ CG discretization for the time-continuous spaces ($\calU$) and an order-$(s - 1)$ DG discretization for the time-discontinuous spaces (everything else), consider the following trial and test spaces:
        \begin{align}
            \calU  &=  \widehat{\calU}(\bfOmega)\otimes\bbP^{s}\left(T^{\rmh}\right),  \\
            \widetilde{\calU}  =  \calV  =  \widetilde{\calV}  &=  \widehat{\calU}(\bfOmega)\otimes\bbDP^{s - 1}\left(T^{\rmh}\right),  \\
            \calP  =  \calQ  &=  \widehat{\calP}(\bfOmega)\otimes\bbDP^{s - 1}\left(T^{\rmh}\right)
        \end{align}
        At lowest order ($s = 1$), after elimination of $\widetilde{\bfu}$ (akin to the CG-DG-in-time timestepper for the heat equation, with $\widetilde{\bfu}^{n}  =  \frac{1}{2}\left(\bfu^{n + 1} + \bfu^{n}\right)$) and taking $\bfu^{*} = \bfu$ (in the convective term, $\calA$) the resultant timestepper, solving for $\left(\widehat{\bfu}^{n}\right)^{n}$, $\left(\widehat{p}^{n}\right)^{n}$ takes the form:
        \begin{align*}
            &\forall  \widehat{q}^{n}     \in  \widehat{\calP},  &                                                                                                               0  &=  \left\langle \widehat{q}^{n}, \frac{1}{2}\left(\widehat{\bfu}^{n + 1} + \widehat{\bfu}^{n}\right)\right\rangle  \\
            &\forall  \widehat{\bfv}^{n}  \in  \widehat{\calU},  &  \frac{1}{\delta t^{n}}\left\langle\widehat{\bfv}^{n}, \widehat{\bfu}^{n + 1} - \widehat{\bfu}^{n}\right\rangle  &=  \calA\left[\frac{1}{2}\left(\widehat{\bfu}^{n + 1} + \widehat{\bfu}^{n}\right); \widehat{\bfv}^{n}, \frac{1}{2}\left(\widehat{\bfu}^{n + 1} + \widehat{\bfu}^{n}\right)\right]  \\
            &                                                    &                                                                                                                  &\;\;\;\;\;\;\;\;\;\;\;\;\;\;\;\;  + \left\langle\nabla\cdot\widehat{\bfv}^{n}, \widehat{p}^{n}\right\rangle  \\
            &                                                    &                                                                                                                  &\;\;\;\;\;\;\;\;\;\;\;\;\;\;\;\;\;\;\;\;\;\;\;\;\;\;\;\;\;\;\;\;  - \frac{2}{\rmRe}\left\langle\nabla_{\rms}\widehat{\bfv}^{n}, \nabla_{\rms}\left[\frac{1}{2}\left(\widehat{\bfu}^{n + 1} + \widehat{\bfu}^{n}\right)\right]\right\rangle
        \end{align*}
        where all inner products $\langle -, -\rangle  =  \langle -, -\rangle_{\bfOmega}$ are taken over $\bfOmega$, and $\calA[-; -, -]$ is the trilinear convective operator as defined in (\ref{eqn:convective operator definition}), resembling the implicit midpoint method.

        From (\ref{eqn:weak NS dissipation}) when $T  =  [t^{n}, t^{n + 1}]$, this has the exact energy dissipation:
        \begin{equation}
            \rmE\left(t^{n + 1}\right)  -  \rmE\left(t^{n}\right)  =  \delta t^{n}\int_{\bfOmega}\left\|\nabla_{\rms}\left[\frac{1}{2}\left(\widehat{\bfu}^{n + 1} + \widehat{\bfu}^{n}\right)\right]\right\|^{2}
        \end{equation}

        \BA{Maybe include an example of the timestepper for $s = 2$, to show that these do in fact scale to arbitrarily high order.}
    \end{example}
