\subsection{Functions Spaces/Finite Elements in Time (FET)}
    As an alternative approach, one can interpret (\ref{eqn:general time-dependent PDE}) on the space-\emph{time} domain $\bfOmega\otimes T$, where $T$ represents a time interval: potentially the timestep $\left(t^{n}, t^{n + 1}\right]$, potentially the whole simulation interval $\left(0, t^{N}\right]$. One then casts into a weak form by testing directly against test functions defined similarly over the whole space-time domain $\bfOmega\otimes T$.
    
    FET transfers the traditional independent choice of RK method to one of a choice of the behavior of the function spaces and Galerkin discretizations in time, in their continuity, order, etc.
    
    A few key points on this technique:
    \begin{itemize}
        \item  For tensor product domains, $\bfOmega\otimes T$, such spaces can often be constructed via the tensor product of spaces on $d$-dimensional $\bfOmega$ and 1-dimensional $T$, without needing to construct new spaces on the entire $(d + 1)$-dimensional domain $\bfOmega\otimes T$.
        \item  The test function spaces need not be identical to the trial function spaces (i.e. a \emph{Petrov}–Galerkin methods) and often won't be for systems with an odd order in time.
        \item  When using a FET interpretation, the intention is often \emph{not} to solve a highly computationally intense, $(d + 1)$-dimensional FE problem on all of $\bfOmega\otimes T$, but to extract a timestepper from the discretization that need only be solved on the $d$-dimensional spatial domain, $\bfOmega$, by exploiting block-triangularity in the stiffness \BA{(Is this the right terminology here?)} matrix. Such block-triangular matrix are derived from well-posed weak forms where the test space has no strongly-imposed initial conditions, such as $*\otimes H^{1}_{+}(T)$ or $*\otimes L^{2}(T)$. \BA{(Check.)} For linear problems, it is in fact often the case that the resultant timesteppers are identical to certain ones that would be derived from a RK-scheme interpretation.
    \end{itemize}
    There are many benefits to a FET interpretation that are often lacking in traditional RK techniques:
    \begin{itemize}
        \item  Crucially, FET provides a weak formulation that often perfectly aligns with the kind of results needed to prove structure-preserving properties.
        \item  FET interpretations inherit much of the useful results and techniques of traditional spatial FEM, including:
        \begin{itemize}
            \item  Finite-element exterior calculus (FEEC).
            \item  Multigrid (at least for non-Petrov–Galerkin schemes).
            \item  Well-studied interpolation and projection error estimates.
            \item  Language for the analysis and interpretation of non-confoming schemes.
            \item  Agency over bases for the time discretization with good orthogonality properties. \BA{(FDM- I want to make a remark on this later.)}
            \item  Easy extension to high order
        \end{itemize}
        \item  Solutions provided by FET are meaningful at all times. \BA{(This is important for me.)}
        \item  FET is the perfect language for solving problems that are already posed on a space-time domain, such as:
        \begin{itemize}
            \item  Searching for periodic solutions. \BA{(I need to do a good note about this somewhere- about how we solve for time periods for periodic problems, since this is something I really want to consider.)}
            \item  Problems with moving boundaries, where the space-time domain can \emph{not} be written in the tensor-product form $\bfOmega\otimes T$.
        \end{itemize}
    \end{itemize}
    
    \BA{Should go through and add correct punctuation to my equations.}
    

    \documentclass[12pt, a4paper]{report}

\documentclass[12pt, a4paper]{report}

\documentclass[12pt, a4paper]{report}

\input{template/main.tex}

\title{\BA{Title in Progress...}}
\author{Boris Andrews}
\affil{Mathematical Institute, University of Oxford}
\date{\today}


\begin{document}
    \pagenumbering{gobble}
    \maketitle
    
    
    \begin{abstract}
        Magnetic confinement reactors---in particular tokamaks---offer one of the most promising options for achieving practical nuclear fusion, with the potential to provide virtually limitless, clean energy. The theoretical and numerical modeling of tokamak plasmas is simultaneously an essential component of effective reactor design, and a great research barrier. Tokamak operational conditions exhibit comparatively low Knudsen numbers. Kinetic effects, including kinetic waves and instabilities, Landau damping, bump-on-tail instabilities and more, are therefore highly influential in tokamak plasma dynamics. Purely fluid models are inherently incapable of capturing these effects, whereas the high dimensionality in purely kinetic models render them practically intractable for most relevant purposes.

        We consider a $\delta\!f$ decomposition model, with a macroscopic fluid background and microscopic kinetic correction, both fully coupled to each other. A similar manner of discretization is proposed to that used in the recent \texttt{STRUPHY} code \cite{Holderied_Possanner_Wang_2021, Holderied_2022, Li_et_al_2023} with a finite-element model for the background and a pseudo-particle/PiC model for the correction.

        The fluid background satisfies the full, non-linear, resistive, compressible, Hall MHD equations. \cite{Laakmann_Hu_Farrell_2022} introduces finite-element(-in-space) implicit timesteppers for the incompressible analogue to this system with structure-preserving (SP) properties in the ideal case, alongside parameter-robust preconditioners. We show that these timesteppers can derive from a finite-element-in-time (FET) (and finite-element-in-space) interpretation. The benefits of this reformulation are discussed, including the derivation of timesteppers that are higher order in time, and the quantifiable dissipative SP properties in the non-ideal, resistive case.
        
        We discuss possible options for extending this FET approach to timesteppers for the compressible case.

        The kinetic corrections satisfy linearized Boltzmann equations. Using a Lénard--Bernstein collision operator, these take Fokker--Planck-like forms \cite{Fokker_1914, Planck_1917} wherein pseudo-particles in the numerical model obey the neoclassical transport equations, with particle-independent Brownian drift terms. This offers a rigorous methodology for incorporating collisions into the particle transport model, without coupling the equations of motions for each particle.
        
        Works by Chen, Chacón et al. \cite{Chen_Chacón_Barnes_2011, Chacón_Chen_Barnes_2013, Chen_Chacón_2014, Chen_Chacón_2015} have developed structure-preserving particle pushers for neoclassical transport in the Vlasov equations, derived from Crank--Nicolson integrators. We show these too can can derive from a FET interpretation, similarly offering potential extensions to higher-order-in-time particle pushers. The FET formulation is used also to consider how the stochastic drift terms can be incorporated into the pushers. Stochastic gyrokinetic expansions are also discussed.

        Different options for the numerical implementation of these schemes are considered.

        Due to the efficacy of FET in the development of SP timesteppers for both the fluid and kinetic component, we hope this approach will prove effective in the future for developing SP timesteppers for the full hybrid model. We hope this will give us the opportunity to incorporate previously inaccessible kinetic effects into the highly effective, modern, finite-element MHD models.
    \end{abstract}
    
    
    \newpage
    \tableofcontents
    
    
    \newpage
    \pagenumbering{arabic}
    %\linenumbers\renewcommand\thelinenumber{\color{black!50}\arabic{linenumber}}
            \input{0 - introduction/main.tex}
        \part{Research}
            \input{1 - low-noise PiC models/main.tex}
            \input{2 - kinetic component/main.tex}
            \input{3 - fluid component/main.tex}
            \input{4 - numerical implementation/main.tex}
        \part{Project Overview}
            \input{5 - research plan/main.tex}
            \input{6 - summary/main.tex}
    
    
    %\section{}
    \newpage
    \pagenumbering{gobble}
        \printbibliography


    \newpage
    \pagenumbering{roman}
    \appendix
        \part{Appendices}
            \input{8 - Hilbert complexes/main.tex}
            \input{9 - weak conservation proofs/main.tex}
\end{document}


\title{\BA{Title in Progress...}}
\author{Boris Andrews}
\affil{Mathematical Institute, University of Oxford}
\date{\today}


\begin{document}
    \pagenumbering{gobble}
    \maketitle
    
    
    \begin{abstract}
        Magnetic confinement reactors---in particular tokamaks---offer one of the most promising options for achieving practical nuclear fusion, with the potential to provide virtually limitless, clean energy. The theoretical and numerical modeling of tokamak plasmas is simultaneously an essential component of effective reactor design, and a great research barrier. Tokamak operational conditions exhibit comparatively low Knudsen numbers. Kinetic effects, including kinetic waves and instabilities, Landau damping, bump-on-tail instabilities and more, are therefore highly influential in tokamak plasma dynamics. Purely fluid models are inherently incapable of capturing these effects, whereas the high dimensionality in purely kinetic models render them practically intractable for most relevant purposes.

        We consider a $\delta\!f$ decomposition model, with a macroscopic fluid background and microscopic kinetic correction, both fully coupled to each other. A similar manner of discretization is proposed to that used in the recent \texttt{STRUPHY} code \cite{Holderied_Possanner_Wang_2021, Holderied_2022, Li_et_al_2023} with a finite-element model for the background and a pseudo-particle/PiC model for the correction.

        The fluid background satisfies the full, non-linear, resistive, compressible, Hall MHD equations. \cite{Laakmann_Hu_Farrell_2022} introduces finite-element(-in-space) implicit timesteppers for the incompressible analogue to this system with structure-preserving (SP) properties in the ideal case, alongside parameter-robust preconditioners. We show that these timesteppers can derive from a finite-element-in-time (FET) (and finite-element-in-space) interpretation. The benefits of this reformulation are discussed, including the derivation of timesteppers that are higher order in time, and the quantifiable dissipative SP properties in the non-ideal, resistive case.
        
        We discuss possible options for extending this FET approach to timesteppers for the compressible case.

        The kinetic corrections satisfy linearized Boltzmann equations. Using a Lénard--Bernstein collision operator, these take Fokker--Planck-like forms \cite{Fokker_1914, Planck_1917} wherein pseudo-particles in the numerical model obey the neoclassical transport equations, with particle-independent Brownian drift terms. This offers a rigorous methodology for incorporating collisions into the particle transport model, without coupling the equations of motions for each particle.
        
        Works by Chen, Chacón et al. \cite{Chen_Chacón_Barnes_2011, Chacón_Chen_Barnes_2013, Chen_Chacón_2014, Chen_Chacón_2015} have developed structure-preserving particle pushers for neoclassical transport in the Vlasov equations, derived from Crank--Nicolson integrators. We show these too can can derive from a FET interpretation, similarly offering potential extensions to higher-order-in-time particle pushers. The FET formulation is used also to consider how the stochastic drift terms can be incorporated into the pushers. Stochastic gyrokinetic expansions are also discussed.

        Different options for the numerical implementation of these schemes are considered.

        Due to the efficacy of FET in the development of SP timesteppers for both the fluid and kinetic component, we hope this approach will prove effective in the future for developing SP timesteppers for the full hybrid model. We hope this will give us the opportunity to incorporate previously inaccessible kinetic effects into the highly effective, modern, finite-element MHD models.
    \end{abstract}
    
    
    \newpage
    \tableofcontents
    
    
    \newpage
    \pagenumbering{arabic}
    %\linenumbers\renewcommand\thelinenumber{\color{black!50}\arabic{linenumber}}
            \documentclass[12pt, a4paper]{report}

\input{template/main.tex}

\title{\BA{Title in Progress...}}
\author{Boris Andrews}
\affil{Mathematical Institute, University of Oxford}
\date{\today}


\begin{document}
    \pagenumbering{gobble}
    \maketitle
    
    
    \begin{abstract}
        Magnetic confinement reactors---in particular tokamaks---offer one of the most promising options for achieving practical nuclear fusion, with the potential to provide virtually limitless, clean energy. The theoretical and numerical modeling of tokamak plasmas is simultaneously an essential component of effective reactor design, and a great research barrier. Tokamak operational conditions exhibit comparatively low Knudsen numbers. Kinetic effects, including kinetic waves and instabilities, Landau damping, bump-on-tail instabilities and more, are therefore highly influential in tokamak plasma dynamics. Purely fluid models are inherently incapable of capturing these effects, whereas the high dimensionality in purely kinetic models render them practically intractable for most relevant purposes.

        We consider a $\delta\!f$ decomposition model, with a macroscopic fluid background and microscopic kinetic correction, both fully coupled to each other. A similar manner of discretization is proposed to that used in the recent \texttt{STRUPHY} code \cite{Holderied_Possanner_Wang_2021, Holderied_2022, Li_et_al_2023} with a finite-element model for the background and a pseudo-particle/PiC model for the correction.

        The fluid background satisfies the full, non-linear, resistive, compressible, Hall MHD equations. \cite{Laakmann_Hu_Farrell_2022} introduces finite-element(-in-space) implicit timesteppers for the incompressible analogue to this system with structure-preserving (SP) properties in the ideal case, alongside parameter-robust preconditioners. We show that these timesteppers can derive from a finite-element-in-time (FET) (and finite-element-in-space) interpretation. The benefits of this reformulation are discussed, including the derivation of timesteppers that are higher order in time, and the quantifiable dissipative SP properties in the non-ideal, resistive case.
        
        We discuss possible options for extending this FET approach to timesteppers for the compressible case.

        The kinetic corrections satisfy linearized Boltzmann equations. Using a Lénard--Bernstein collision operator, these take Fokker--Planck-like forms \cite{Fokker_1914, Planck_1917} wherein pseudo-particles in the numerical model obey the neoclassical transport equations, with particle-independent Brownian drift terms. This offers a rigorous methodology for incorporating collisions into the particle transport model, without coupling the equations of motions for each particle.
        
        Works by Chen, Chacón et al. \cite{Chen_Chacón_Barnes_2011, Chacón_Chen_Barnes_2013, Chen_Chacón_2014, Chen_Chacón_2015} have developed structure-preserving particle pushers for neoclassical transport in the Vlasov equations, derived from Crank--Nicolson integrators. We show these too can can derive from a FET interpretation, similarly offering potential extensions to higher-order-in-time particle pushers. The FET formulation is used also to consider how the stochastic drift terms can be incorporated into the pushers. Stochastic gyrokinetic expansions are also discussed.

        Different options for the numerical implementation of these schemes are considered.

        Due to the efficacy of FET in the development of SP timesteppers for both the fluid and kinetic component, we hope this approach will prove effective in the future for developing SP timesteppers for the full hybrid model. We hope this will give us the opportunity to incorporate previously inaccessible kinetic effects into the highly effective, modern, finite-element MHD models.
    \end{abstract}
    
    
    \newpage
    \tableofcontents
    
    
    \newpage
    \pagenumbering{arabic}
    %\linenumbers\renewcommand\thelinenumber{\color{black!50}\arabic{linenumber}}
            \input{0 - introduction/main.tex}
        \part{Research}
            \input{1 - low-noise PiC models/main.tex}
            \input{2 - kinetic component/main.tex}
            \input{3 - fluid component/main.tex}
            \input{4 - numerical implementation/main.tex}
        \part{Project Overview}
            \input{5 - research plan/main.tex}
            \input{6 - summary/main.tex}
    
    
    %\section{}
    \newpage
    \pagenumbering{gobble}
        \printbibliography


    \newpage
    \pagenumbering{roman}
    \appendix
        \part{Appendices}
            \input{8 - Hilbert complexes/main.tex}
            \input{9 - weak conservation proofs/main.tex}
\end{document}

        \part{Research}
            \documentclass[12pt, a4paper]{report}

\input{template/main.tex}

\title{\BA{Title in Progress...}}
\author{Boris Andrews}
\affil{Mathematical Institute, University of Oxford}
\date{\today}


\begin{document}
    \pagenumbering{gobble}
    \maketitle
    
    
    \begin{abstract}
        Magnetic confinement reactors---in particular tokamaks---offer one of the most promising options for achieving practical nuclear fusion, with the potential to provide virtually limitless, clean energy. The theoretical and numerical modeling of tokamak plasmas is simultaneously an essential component of effective reactor design, and a great research barrier. Tokamak operational conditions exhibit comparatively low Knudsen numbers. Kinetic effects, including kinetic waves and instabilities, Landau damping, bump-on-tail instabilities and more, are therefore highly influential in tokamak plasma dynamics. Purely fluid models are inherently incapable of capturing these effects, whereas the high dimensionality in purely kinetic models render them practically intractable for most relevant purposes.

        We consider a $\delta\!f$ decomposition model, with a macroscopic fluid background and microscopic kinetic correction, both fully coupled to each other. A similar manner of discretization is proposed to that used in the recent \texttt{STRUPHY} code \cite{Holderied_Possanner_Wang_2021, Holderied_2022, Li_et_al_2023} with a finite-element model for the background and a pseudo-particle/PiC model for the correction.

        The fluid background satisfies the full, non-linear, resistive, compressible, Hall MHD equations. \cite{Laakmann_Hu_Farrell_2022} introduces finite-element(-in-space) implicit timesteppers for the incompressible analogue to this system with structure-preserving (SP) properties in the ideal case, alongside parameter-robust preconditioners. We show that these timesteppers can derive from a finite-element-in-time (FET) (and finite-element-in-space) interpretation. The benefits of this reformulation are discussed, including the derivation of timesteppers that are higher order in time, and the quantifiable dissipative SP properties in the non-ideal, resistive case.
        
        We discuss possible options for extending this FET approach to timesteppers for the compressible case.

        The kinetic corrections satisfy linearized Boltzmann equations. Using a Lénard--Bernstein collision operator, these take Fokker--Planck-like forms \cite{Fokker_1914, Planck_1917} wherein pseudo-particles in the numerical model obey the neoclassical transport equations, with particle-independent Brownian drift terms. This offers a rigorous methodology for incorporating collisions into the particle transport model, without coupling the equations of motions for each particle.
        
        Works by Chen, Chacón et al. \cite{Chen_Chacón_Barnes_2011, Chacón_Chen_Barnes_2013, Chen_Chacón_2014, Chen_Chacón_2015} have developed structure-preserving particle pushers for neoclassical transport in the Vlasov equations, derived from Crank--Nicolson integrators. We show these too can can derive from a FET interpretation, similarly offering potential extensions to higher-order-in-time particle pushers. The FET formulation is used also to consider how the stochastic drift terms can be incorporated into the pushers. Stochastic gyrokinetic expansions are also discussed.

        Different options for the numerical implementation of these schemes are considered.

        Due to the efficacy of FET in the development of SP timesteppers for both the fluid and kinetic component, we hope this approach will prove effective in the future for developing SP timesteppers for the full hybrid model. We hope this will give us the opportunity to incorporate previously inaccessible kinetic effects into the highly effective, modern, finite-element MHD models.
    \end{abstract}
    
    
    \newpage
    \tableofcontents
    
    
    \newpage
    \pagenumbering{arabic}
    %\linenumbers\renewcommand\thelinenumber{\color{black!50}\arabic{linenumber}}
            \input{0 - introduction/main.tex}
        \part{Research}
            \input{1 - low-noise PiC models/main.tex}
            \input{2 - kinetic component/main.tex}
            \input{3 - fluid component/main.tex}
            \input{4 - numerical implementation/main.tex}
        \part{Project Overview}
            \input{5 - research plan/main.tex}
            \input{6 - summary/main.tex}
    
    
    %\section{}
    \newpage
    \pagenumbering{gobble}
        \printbibliography


    \newpage
    \pagenumbering{roman}
    \appendix
        \part{Appendices}
            \input{8 - Hilbert complexes/main.tex}
            \input{9 - weak conservation proofs/main.tex}
\end{document}

            \documentclass[12pt, a4paper]{report}

\input{template/main.tex}

\title{\BA{Title in Progress...}}
\author{Boris Andrews}
\affil{Mathematical Institute, University of Oxford}
\date{\today}


\begin{document}
    \pagenumbering{gobble}
    \maketitle
    
    
    \begin{abstract}
        Magnetic confinement reactors---in particular tokamaks---offer one of the most promising options for achieving practical nuclear fusion, with the potential to provide virtually limitless, clean energy. The theoretical and numerical modeling of tokamak plasmas is simultaneously an essential component of effective reactor design, and a great research barrier. Tokamak operational conditions exhibit comparatively low Knudsen numbers. Kinetic effects, including kinetic waves and instabilities, Landau damping, bump-on-tail instabilities and more, are therefore highly influential in tokamak plasma dynamics. Purely fluid models are inherently incapable of capturing these effects, whereas the high dimensionality in purely kinetic models render them practically intractable for most relevant purposes.

        We consider a $\delta\!f$ decomposition model, with a macroscopic fluid background and microscopic kinetic correction, both fully coupled to each other. A similar manner of discretization is proposed to that used in the recent \texttt{STRUPHY} code \cite{Holderied_Possanner_Wang_2021, Holderied_2022, Li_et_al_2023} with a finite-element model for the background and a pseudo-particle/PiC model for the correction.

        The fluid background satisfies the full, non-linear, resistive, compressible, Hall MHD equations. \cite{Laakmann_Hu_Farrell_2022} introduces finite-element(-in-space) implicit timesteppers for the incompressible analogue to this system with structure-preserving (SP) properties in the ideal case, alongside parameter-robust preconditioners. We show that these timesteppers can derive from a finite-element-in-time (FET) (and finite-element-in-space) interpretation. The benefits of this reformulation are discussed, including the derivation of timesteppers that are higher order in time, and the quantifiable dissipative SP properties in the non-ideal, resistive case.
        
        We discuss possible options for extending this FET approach to timesteppers for the compressible case.

        The kinetic corrections satisfy linearized Boltzmann equations. Using a Lénard--Bernstein collision operator, these take Fokker--Planck-like forms \cite{Fokker_1914, Planck_1917} wherein pseudo-particles in the numerical model obey the neoclassical transport equations, with particle-independent Brownian drift terms. This offers a rigorous methodology for incorporating collisions into the particle transport model, without coupling the equations of motions for each particle.
        
        Works by Chen, Chacón et al. \cite{Chen_Chacón_Barnes_2011, Chacón_Chen_Barnes_2013, Chen_Chacón_2014, Chen_Chacón_2015} have developed structure-preserving particle pushers for neoclassical transport in the Vlasov equations, derived from Crank--Nicolson integrators. We show these too can can derive from a FET interpretation, similarly offering potential extensions to higher-order-in-time particle pushers. The FET formulation is used also to consider how the stochastic drift terms can be incorporated into the pushers. Stochastic gyrokinetic expansions are also discussed.

        Different options for the numerical implementation of these schemes are considered.

        Due to the efficacy of FET in the development of SP timesteppers for both the fluid and kinetic component, we hope this approach will prove effective in the future for developing SP timesteppers for the full hybrid model. We hope this will give us the opportunity to incorporate previously inaccessible kinetic effects into the highly effective, modern, finite-element MHD models.
    \end{abstract}
    
    
    \newpage
    \tableofcontents
    
    
    \newpage
    \pagenumbering{arabic}
    %\linenumbers\renewcommand\thelinenumber{\color{black!50}\arabic{linenumber}}
            \input{0 - introduction/main.tex}
        \part{Research}
            \input{1 - low-noise PiC models/main.tex}
            \input{2 - kinetic component/main.tex}
            \input{3 - fluid component/main.tex}
            \input{4 - numerical implementation/main.tex}
        \part{Project Overview}
            \input{5 - research plan/main.tex}
            \input{6 - summary/main.tex}
    
    
    %\section{}
    \newpage
    \pagenumbering{gobble}
        \printbibliography


    \newpage
    \pagenumbering{roman}
    \appendix
        \part{Appendices}
            \input{8 - Hilbert complexes/main.tex}
            \input{9 - weak conservation proofs/main.tex}
\end{document}

            \documentclass[12pt, a4paper]{report}

\input{template/main.tex}

\title{\BA{Title in Progress...}}
\author{Boris Andrews}
\affil{Mathematical Institute, University of Oxford}
\date{\today}


\begin{document}
    \pagenumbering{gobble}
    \maketitle
    
    
    \begin{abstract}
        Magnetic confinement reactors---in particular tokamaks---offer one of the most promising options for achieving practical nuclear fusion, with the potential to provide virtually limitless, clean energy. The theoretical and numerical modeling of tokamak plasmas is simultaneously an essential component of effective reactor design, and a great research barrier. Tokamak operational conditions exhibit comparatively low Knudsen numbers. Kinetic effects, including kinetic waves and instabilities, Landau damping, bump-on-tail instabilities and more, are therefore highly influential in tokamak plasma dynamics. Purely fluid models are inherently incapable of capturing these effects, whereas the high dimensionality in purely kinetic models render them practically intractable for most relevant purposes.

        We consider a $\delta\!f$ decomposition model, with a macroscopic fluid background and microscopic kinetic correction, both fully coupled to each other. A similar manner of discretization is proposed to that used in the recent \texttt{STRUPHY} code \cite{Holderied_Possanner_Wang_2021, Holderied_2022, Li_et_al_2023} with a finite-element model for the background and a pseudo-particle/PiC model for the correction.

        The fluid background satisfies the full, non-linear, resistive, compressible, Hall MHD equations. \cite{Laakmann_Hu_Farrell_2022} introduces finite-element(-in-space) implicit timesteppers for the incompressible analogue to this system with structure-preserving (SP) properties in the ideal case, alongside parameter-robust preconditioners. We show that these timesteppers can derive from a finite-element-in-time (FET) (and finite-element-in-space) interpretation. The benefits of this reformulation are discussed, including the derivation of timesteppers that are higher order in time, and the quantifiable dissipative SP properties in the non-ideal, resistive case.
        
        We discuss possible options for extending this FET approach to timesteppers for the compressible case.

        The kinetic corrections satisfy linearized Boltzmann equations. Using a Lénard--Bernstein collision operator, these take Fokker--Planck-like forms \cite{Fokker_1914, Planck_1917} wherein pseudo-particles in the numerical model obey the neoclassical transport equations, with particle-independent Brownian drift terms. This offers a rigorous methodology for incorporating collisions into the particle transport model, without coupling the equations of motions for each particle.
        
        Works by Chen, Chacón et al. \cite{Chen_Chacón_Barnes_2011, Chacón_Chen_Barnes_2013, Chen_Chacón_2014, Chen_Chacón_2015} have developed structure-preserving particle pushers for neoclassical transport in the Vlasov equations, derived from Crank--Nicolson integrators. We show these too can can derive from a FET interpretation, similarly offering potential extensions to higher-order-in-time particle pushers. The FET formulation is used also to consider how the stochastic drift terms can be incorporated into the pushers. Stochastic gyrokinetic expansions are also discussed.

        Different options for the numerical implementation of these schemes are considered.

        Due to the efficacy of FET in the development of SP timesteppers for both the fluid and kinetic component, we hope this approach will prove effective in the future for developing SP timesteppers for the full hybrid model. We hope this will give us the opportunity to incorporate previously inaccessible kinetic effects into the highly effective, modern, finite-element MHD models.
    \end{abstract}
    
    
    \newpage
    \tableofcontents
    
    
    \newpage
    \pagenumbering{arabic}
    %\linenumbers\renewcommand\thelinenumber{\color{black!50}\arabic{linenumber}}
            \input{0 - introduction/main.tex}
        \part{Research}
            \input{1 - low-noise PiC models/main.tex}
            \input{2 - kinetic component/main.tex}
            \input{3 - fluid component/main.tex}
            \input{4 - numerical implementation/main.tex}
        \part{Project Overview}
            \input{5 - research plan/main.tex}
            \input{6 - summary/main.tex}
    
    
    %\section{}
    \newpage
    \pagenumbering{gobble}
        \printbibliography


    \newpage
    \pagenumbering{roman}
    \appendix
        \part{Appendices}
            \input{8 - Hilbert complexes/main.tex}
            \input{9 - weak conservation proofs/main.tex}
\end{document}

            \documentclass[12pt, a4paper]{report}

\input{template/main.tex}

\title{\BA{Title in Progress...}}
\author{Boris Andrews}
\affil{Mathematical Institute, University of Oxford}
\date{\today}


\begin{document}
    \pagenumbering{gobble}
    \maketitle
    
    
    \begin{abstract}
        Magnetic confinement reactors---in particular tokamaks---offer one of the most promising options for achieving practical nuclear fusion, with the potential to provide virtually limitless, clean energy. The theoretical and numerical modeling of tokamak plasmas is simultaneously an essential component of effective reactor design, and a great research barrier. Tokamak operational conditions exhibit comparatively low Knudsen numbers. Kinetic effects, including kinetic waves and instabilities, Landau damping, bump-on-tail instabilities and more, are therefore highly influential in tokamak plasma dynamics. Purely fluid models are inherently incapable of capturing these effects, whereas the high dimensionality in purely kinetic models render them practically intractable for most relevant purposes.

        We consider a $\delta\!f$ decomposition model, with a macroscopic fluid background and microscopic kinetic correction, both fully coupled to each other. A similar manner of discretization is proposed to that used in the recent \texttt{STRUPHY} code \cite{Holderied_Possanner_Wang_2021, Holderied_2022, Li_et_al_2023} with a finite-element model for the background and a pseudo-particle/PiC model for the correction.

        The fluid background satisfies the full, non-linear, resistive, compressible, Hall MHD equations. \cite{Laakmann_Hu_Farrell_2022} introduces finite-element(-in-space) implicit timesteppers for the incompressible analogue to this system with structure-preserving (SP) properties in the ideal case, alongside parameter-robust preconditioners. We show that these timesteppers can derive from a finite-element-in-time (FET) (and finite-element-in-space) interpretation. The benefits of this reformulation are discussed, including the derivation of timesteppers that are higher order in time, and the quantifiable dissipative SP properties in the non-ideal, resistive case.
        
        We discuss possible options for extending this FET approach to timesteppers for the compressible case.

        The kinetic corrections satisfy linearized Boltzmann equations. Using a Lénard--Bernstein collision operator, these take Fokker--Planck-like forms \cite{Fokker_1914, Planck_1917} wherein pseudo-particles in the numerical model obey the neoclassical transport equations, with particle-independent Brownian drift terms. This offers a rigorous methodology for incorporating collisions into the particle transport model, without coupling the equations of motions for each particle.
        
        Works by Chen, Chacón et al. \cite{Chen_Chacón_Barnes_2011, Chacón_Chen_Barnes_2013, Chen_Chacón_2014, Chen_Chacón_2015} have developed structure-preserving particle pushers for neoclassical transport in the Vlasov equations, derived from Crank--Nicolson integrators. We show these too can can derive from a FET interpretation, similarly offering potential extensions to higher-order-in-time particle pushers. The FET formulation is used also to consider how the stochastic drift terms can be incorporated into the pushers. Stochastic gyrokinetic expansions are also discussed.

        Different options for the numerical implementation of these schemes are considered.

        Due to the efficacy of FET in the development of SP timesteppers for both the fluid and kinetic component, we hope this approach will prove effective in the future for developing SP timesteppers for the full hybrid model. We hope this will give us the opportunity to incorporate previously inaccessible kinetic effects into the highly effective, modern, finite-element MHD models.
    \end{abstract}
    
    
    \newpage
    \tableofcontents
    
    
    \newpage
    \pagenumbering{arabic}
    %\linenumbers\renewcommand\thelinenumber{\color{black!50}\arabic{linenumber}}
            \input{0 - introduction/main.tex}
        \part{Research}
            \input{1 - low-noise PiC models/main.tex}
            \input{2 - kinetic component/main.tex}
            \input{3 - fluid component/main.tex}
            \input{4 - numerical implementation/main.tex}
        \part{Project Overview}
            \input{5 - research plan/main.tex}
            \input{6 - summary/main.tex}
    
    
    %\section{}
    \newpage
    \pagenumbering{gobble}
        \printbibliography


    \newpage
    \pagenumbering{roman}
    \appendix
        \part{Appendices}
            \input{8 - Hilbert complexes/main.tex}
            \input{9 - weak conservation proofs/main.tex}
\end{document}

        \part{Project Overview}
            \documentclass[12pt, a4paper]{report}

\input{template/main.tex}

\title{\BA{Title in Progress...}}
\author{Boris Andrews}
\affil{Mathematical Institute, University of Oxford}
\date{\today}


\begin{document}
    \pagenumbering{gobble}
    \maketitle
    
    
    \begin{abstract}
        Magnetic confinement reactors---in particular tokamaks---offer one of the most promising options for achieving practical nuclear fusion, with the potential to provide virtually limitless, clean energy. The theoretical and numerical modeling of tokamak plasmas is simultaneously an essential component of effective reactor design, and a great research barrier. Tokamak operational conditions exhibit comparatively low Knudsen numbers. Kinetic effects, including kinetic waves and instabilities, Landau damping, bump-on-tail instabilities and more, are therefore highly influential in tokamak plasma dynamics. Purely fluid models are inherently incapable of capturing these effects, whereas the high dimensionality in purely kinetic models render them practically intractable for most relevant purposes.

        We consider a $\delta\!f$ decomposition model, with a macroscopic fluid background and microscopic kinetic correction, both fully coupled to each other. A similar manner of discretization is proposed to that used in the recent \texttt{STRUPHY} code \cite{Holderied_Possanner_Wang_2021, Holderied_2022, Li_et_al_2023} with a finite-element model for the background and a pseudo-particle/PiC model for the correction.

        The fluid background satisfies the full, non-linear, resistive, compressible, Hall MHD equations. \cite{Laakmann_Hu_Farrell_2022} introduces finite-element(-in-space) implicit timesteppers for the incompressible analogue to this system with structure-preserving (SP) properties in the ideal case, alongside parameter-robust preconditioners. We show that these timesteppers can derive from a finite-element-in-time (FET) (and finite-element-in-space) interpretation. The benefits of this reformulation are discussed, including the derivation of timesteppers that are higher order in time, and the quantifiable dissipative SP properties in the non-ideal, resistive case.
        
        We discuss possible options for extending this FET approach to timesteppers for the compressible case.

        The kinetic corrections satisfy linearized Boltzmann equations. Using a Lénard--Bernstein collision operator, these take Fokker--Planck-like forms \cite{Fokker_1914, Planck_1917} wherein pseudo-particles in the numerical model obey the neoclassical transport equations, with particle-independent Brownian drift terms. This offers a rigorous methodology for incorporating collisions into the particle transport model, without coupling the equations of motions for each particle.
        
        Works by Chen, Chacón et al. \cite{Chen_Chacón_Barnes_2011, Chacón_Chen_Barnes_2013, Chen_Chacón_2014, Chen_Chacón_2015} have developed structure-preserving particle pushers for neoclassical transport in the Vlasov equations, derived from Crank--Nicolson integrators. We show these too can can derive from a FET interpretation, similarly offering potential extensions to higher-order-in-time particle pushers. The FET formulation is used also to consider how the stochastic drift terms can be incorporated into the pushers. Stochastic gyrokinetic expansions are also discussed.

        Different options for the numerical implementation of these schemes are considered.

        Due to the efficacy of FET in the development of SP timesteppers for both the fluid and kinetic component, we hope this approach will prove effective in the future for developing SP timesteppers for the full hybrid model. We hope this will give us the opportunity to incorporate previously inaccessible kinetic effects into the highly effective, modern, finite-element MHD models.
    \end{abstract}
    
    
    \newpage
    \tableofcontents
    
    
    \newpage
    \pagenumbering{arabic}
    %\linenumbers\renewcommand\thelinenumber{\color{black!50}\arabic{linenumber}}
            \input{0 - introduction/main.tex}
        \part{Research}
            \input{1 - low-noise PiC models/main.tex}
            \input{2 - kinetic component/main.tex}
            \input{3 - fluid component/main.tex}
            \input{4 - numerical implementation/main.tex}
        \part{Project Overview}
            \input{5 - research plan/main.tex}
            \input{6 - summary/main.tex}
    
    
    %\section{}
    \newpage
    \pagenumbering{gobble}
        \printbibliography


    \newpage
    \pagenumbering{roman}
    \appendix
        \part{Appendices}
            \input{8 - Hilbert complexes/main.tex}
            \input{9 - weak conservation proofs/main.tex}
\end{document}

            \documentclass[12pt, a4paper]{report}

\input{template/main.tex}

\title{\BA{Title in Progress...}}
\author{Boris Andrews}
\affil{Mathematical Institute, University of Oxford}
\date{\today}


\begin{document}
    \pagenumbering{gobble}
    \maketitle
    
    
    \begin{abstract}
        Magnetic confinement reactors---in particular tokamaks---offer one of the most promising options for achieving practical nuclear fusion, with the potential to provide virtually limitless, clean energy. The theoretical and numerical modeling of tokamak plasmas is simultaneously an essential component of effective reactor design, and a great research barrier. Tokamak operational conditions exhibit comparatively low Knudsen numbers. Kinetic effects, including kinetic waves and instabilities, Landau damping, bump-on-tail instabilities and more, are therefore highly influential in tokamak plasma dynamics. Purely fluid models are inherently incapable of capturing these effects, whereas the high dimensionality in purely kinetic models render them practically intractable for most relevant purposes.

        We consider a $\delta\!f$ decomposition model, with a macroscopic fluid background and microscopic kinetic correction, both fully coupled to each other. A similar manner of discretization is proposed to that used in the recent \texttt{STRUPHY} code \cite{Holderied_Possanner_Wang_2021, Holderied_2022, Li_et_al_2023} with a finite-element model for the background and a pseudo-particle/PiC model for the correction.

        The fluid background satisfies the full, non-linear, resistive, compressible, Hall MHD equations. \cite{Laakmann_Hu_Farrell_2022} introduces finite-element(-in-space) implicit timesteppers for the incompressible analogue to this system with structure-preserving (SP) properties in the ideal case, alongside parameter-robust preconditioners. We show that these timesteppers can derive from a finite-element-in-time (FET) (and finite-element-in-space) interpretation. The benefits of this reformulation are discussed, including the derivation of timesteppers that are higher order in time, and the quantifiable dissipative SP properties in the non-ideal, resistive case.
        
        We discuss possible options for extending this FET approach to timesteppers for the compressible case.

        The kinetic corrections satisfy linearized Boltzmann equations. Using a Lénard--Bernstein collision operator, these take Fokker--Planck-like forms \cite{Fokker_1914, Planck_1917} wherein pseudo-particles in the numerical model obey the neoclassical transport equations, with particle-independent Brownian drift terms. This offers a rigorous methodology for incorporating collisions into the particle transport model, without coupling the equations of motions for each particle.
        
        Works by Chen, Chacón et al. \cite{Chen_Chacón_Barnes_2011, Chacón_Chen_Barnes_2013, Chen_Chacón_2014, Chen_Chacón_2015} have developed structure-preserving particle pushers for neoclassical transport in the Vlasov equations, derived from Crank--Nicolson integrators. We show these too can can derive from a FET interpretation, similarly offering potential extensions to higher-order-in-time particle pushers. The FET formulation is used also to consider how the stochastic drift terms can be incorporated into the pushers. Stochastic gyrokinetic expansions are also discussed.

        Different options for the numerical implementation of these schemes are considered.

        Due to the efficacy of FET in the development of SP timesteppers for both the fluid and kinetic component, we hope this approach will prove effective in the future for developing SP timesteppers for the full hybrid model. We hope this will give us the opportunity to incorporate previously inaccessible kinetic effects into the highly effective, modern, finite-element MHD models.
    \end{abstract}
    
    
    \newpage
    \tableofcontents
    
    
    \newpage
    \pagenumbering{arabic}
    %\linenumbers\renewcommand\thelinenumber{\color{black!50}\arabic{linenumber}}
            \input{0 - introduction/main.tex}
        \part{Research}
            \input{1 - low-noise PiC models/main.tex}
            \input{2 - kinetic component/main.tex}
            \input{3 - fluid component/main.tex}
            \input{4 - numerical implementation/main.tex}
        \part{Project Overview}
            \input{5 - research plan/main.tex}
            \input{6 - summary/main.tex}
    
    
    %\section{}
    \newpage
    \pagenumbering{gobble}
        \printbibliography


    \newpage
    \pagenumbering{roman}
    \appendix
        \part{Appendices}
            \input{8 - Hilbert complexes/main.tex}
            \input{9 - weak conservation proofs/main.tex}
\end{document}

    
    
    %\section{}
    \newpage
    \pagenumbering{gobble}
        \printbibliography


    \newpage
    \pagenumbering{roman}
    \appendix
        \part{Appendices}
            \documentclass[12pt, a4paper]{report}

\input{template/main.tex}

\title{\BA{Title in Progress...}}
\author{Boris Andrews}
\affil{Mathematical Institute, University of Oxford}
\date{\today}


\begin{document}
    \pagenumbering{gobble}
    \maketitle
    
    
    \begin{abstract}
        Magnetic confinement reactors---in particular tokamaks---offer one of the most promising options for achieving practical nuclear fusion, with the potential to provide virtually limitless, clean energy. The theoretical and numerical modeling of tokamak plasmas is simultaneously an essential component of effective reactor design, and a great research barrier. Tokamak operational conditions exhibit comparatively low Knudsen numbers. Kinetic effects, including kinetic waves and instabilities, Landau damping, bump-on-tail instabilities and more, are therefore highly influential in tokamak plasma dynamics. Purely fluid models are inherently incapable of capturing these effects, whereas the high dimensionality in purely kinetic models render them practically intractable for most relevant purposes.

        We consider a $\delta\!f$ decomposition model, with a macroscopic fluid background and microscopic kinetic correction, both fully coupled to each other. A similar manner of discretization is proposed to that used in the recent \texttt{STRUPHY} code \cite{Holderied_Possanner_Wang_2021, Holderied_2022, Li_et_al_2023} with a finite-element model for the background and a pseudo-particle/PiC model for the correction.

        The fluid background satisfies the full, non-linear, resistive, compressible, Hall MHD equations. \cite{Laakmann_Hu_Farrell_2022} introduces finite-element(-in-space) implicit timesteppers for the incompressible analogue to this system with structure-preserving (SP) properties in the ideal case, alongside parameter-robust preconditioners. We show that these timesteppers can derive from a finite-element-in-time (FET) (and finite-element-in-space) interpretation. The benefits of this reformulation are discussed, including the derivation of timesteppers that are higher order in time, and the quantifiable dissipative SP properties in the non-ideal, resistive case.
        
        We discuss possible options for extending this FET approach to timesteppers for the compressible case.

        The kinetic corrections satisfy linearized Boltzmann equations. Using a Lénard--Bernstein collision operator, these take Fokker--Planck-like forms \cite{Fokker_1914, Planck_1917} wherein pseudo-particles in the numerical model obey the neoclassical transport equations, with particle-independent Brownian drift terms. This offers a rigorous methodology for incorporating collisions into the particle transport model, without coupling the equations of motions for each particle.
        
        Works by Chen, Chacón et al. \cite{Chen_Chacón_Barnes_2011, Chacón_Chen_Barnes_2013, Chen_Chacón_2014, Chen_Chacón_2015} have developed structure-preserving particle pushers for neoclassical transport in the Vlasov equations, derived from Crank--Nicolson integrators. We show these too can can derive from a FET interpretation, similarly offering potential extensions to higher-order-in-time particle pushers. The FET formulation is used also to consider how the stochastic drift terms can be incorporated into the pushers. Stochastic gyrokinetic expansions are also discussed.

        Different options for the numerical implementation of these schemes are considered.

        Due to the efficacy of FET in the development of SP timesteppers for both the fluid and kinetic component, we hope this approach will prove effective in the future for developing SP timesteppers for the full hybrid model. We hope this will give us the opportunity to incorporate previously inaccessible kinetic effects into the highly effective, modern, finite-element MHD models.
    \end{abstract}
    
    
    \newpage
    \tableofcontents
    
    
    \newpage
    \pagenumbering{arabic}
    %\linenumbers\renewcommand\thelinenumber{\color{black!50}\arabic{linenumber}}
            \input{0 - introduction/main.tex}
        \part{Research}
            \input{1 - low-noise PiC models/main.tex}
            \input{2 - kinetic component/main.tex}
            \input{3 - fluid component/main.tex}
            \input{4 - numerical implementation/main.tex}
        \part{Project Overview}
            \input{5 - research plan/main.tex}
            \input{6 - summary/main.tex}
    
    
    %\section{}
    \newpage
    \pagenumbering{gobble}
        \printbibliography


    \newpage
    \pagenumbering{roman}
    \appendix
        \part{Appendices}
            \input{8 - Hilbert complexes/main.tex}
            \input{9 - weak conservation proofs/main.tex}
\end{document}

            \documentclass[12pt, a4paper]{report}

\input{template/main.tex}

\title{\BA{Title in Progress...}}
\author{Boris Andrews}
\affil{Mathematical Institute, University of Oxford}
\date{\today}


\begin{document}
    \pagenumbering{gobble}
    \maketitle
    
    
    \begin{abstract}
        Magnetic confinement reactors---in particular tokamaks---offer one of the most promising options for achieving practical nuclear fusion, with the potential to provide virtually limitless, clean energy. The theoretical and numerical modeling of tokamak plasmas is simultaneously an essential component of effective reactor design, and a great research barrier. Tokamak operational conditions exhibit comparatively low Knudsen numbers. Kinetic effects, including kinetic waves and instabilities, Landau damping, bump-on-tail instabilities and more, are therefore highly influential in tokamak plasma dynamics. Purely fluid models are inherently incapable of capturing these effects, whereas the high dimensionality in purely kinetic models render them practically intractable for most relevant purposes.

        We consider a $\delta\!f$ decomposition model, with a macroscopic fluid background and microscopic kinetic correction, both fully coupled to each other. A similar manner of discretization is proposed to that used in the recent \texttt{STRUPHY} code \cite{Holderied_Possanner_Wang_2021, Holderied_2022, Li_et_al_2023} with a finite-element model for the background and a pseudo-particle/PiC model for the correction.

        The fluid background satisfies the full, non-linear, resistive, compressible, Hall MHD equations. \cite{Laakmann_Hu_Farrell_2022} introduces finite-element(-in-space) implicit timesteppers for the incompressible analogue to this system with structure-preserving (SP) properties in the ideal case, alongside parameter-robust preconditioners. We show that these timesteppers can derive from a finite-element-in-time (FET) (and finite-element-in-space) interpretation. The benefits of this reformulation are discussed, including the derivation of timesteppers that are higher order in time, and the quantifiable dissipative SP properties in the non-ideal, resistive case.
        
        We discuss possible options for extending this FET approach to timesteppers for the compressible case.

        The kinetic corrections satisfy linearized Boltzmann equations. Using a Lénard--Bernstein collision operator, these take Fokker--Planck-like forms \cite{Fokker_1914, Planck_1917} wherein pseudo-particles in the numerical model obey the neoclassical transport equations, with particle-independent Brownian drift terms. This offers a rigorous methodology for incorporating collisions into the particle transport model, without coupling the equations of motions for each particle.
        
        Works by Chen, Chacón et al. \cite{Chen_Chacón_Barnes_2011, Chacón_Chen_Barnes_2013, Chen_Chacón_2014, Chen_Chacón_2015} have developed structure-preserving particle pushers for neoclassical transport in the Vlasov equations, derived from Crank--Nicolson integrators. We show these too can can derive from a FET interpretation, similarly offering potential extensions to higher-order-in-time particle pushers. The FET formulation is used also to consider how the stochastic drift terms can be incorporated into the pushers. Stochastic gyrokinetic expansions are also discussed.

        Different options for the numerical implementation of these schemes are considered.

        Due to the efficacy of FET in the development of SP timesteppers for both the fluid and kinetic component, we hope this approach will prove effective in the future for developing SP timesteppers for the full hybrid model. We hope this will give us the opportunity to incorporate previously inaccessible kinetic effects into the highly effective, modern, finite-element MHD models.
    \end{abstract}
    
    
    \newpage
    \tableofcontents
    
    
    \newpage
    \pagenumbering{arabic}
    %\linenumbers\renewcommand\thelinenumber{\color{black!50}\arabic{linenumber}}
            \input{0 - introduction/main.tex}
        \part{Research}
            \input{1 - low-noise PiC models/main.tex}
            \input{2 - kinetic component/main.tex}
            \input{3 - fluid component/main.tex}
            \input{4 - numerical implementation/main.tex}
        \part{Project Overview}
            \input{5 - research plan/main.tex}
            \input{6 - summary/main.tex}
    
    
    %\section{}
    \newpage
    \pagenumbering{gobble}
        \printbibliography


    \newpage
    \pagenumbering{roman}
    \appendix
        \part{Appendices}
            \input{8 - Hilbert complexes/main.tex}
            \input{9 - weak conservation proofs/main.tex}
\end{document}

\end{document}


\title{\BA{Title in Progress...}}
\author{Boris Andrews}
\affil{Mathematical Institute, University of Oxford}
\date{\today}


\begin{document}
    \pagenumbering{gobble}
    \maketitle
    
    
    \begin{abstract}
        Magnetic confinement reactors---in particular tokamaks---offer one of the most promising options for achieving practical nuclear fusion, with the potential to provide virtually limitless, clean energy. The theoretical and numerical modeling of tokamak plasmas is simultaneously an essential component of effective reactor design, and a great research barrier. Tokamak operational conditions exhibit comparatively low Knudsen numbers. Kinetic effects, including kinetic waves and instabilities, Landau damping, bump-on-tail instabilities and more, are therefore highly influential in tokamak plasma dynamics. Purely fluid models are inherently incapable of capturing these effects, whereas the high dimensionality in purely kinetic models render them practically intractable for most relevant purposes.

        We consider a $\delta\!f$ decomposition model, with a macroscopic fluid background and microscopic kinetic correction, both fully coupled to each other. A similar manner of discretization is proposed to that used in the recent \texttt{STRUPHY} code \cite{Holderied_Possanner_Wang_2021, Holderied_2022, Li_et_al_2023} with a finite-element model for the background and a pseudo-particle/PiC model for the correction.

        The fluid background satisfies the full, non-linear, resistive, compressible, Hall MHD equations. \cite{Laakmann_Hu_Farrell_2022} introduces finite-element(-in-space) implicit timesteppers for the incompressible analogue to this system with structure-preserving (SP) properties in the ideal case, alongside parameter-robust preconditioners. We show that these timesteppers can derive from a finite-element-in-time (FET) (and finite-element-in-space) interpretation. The benefits of this reformulation are discussed, including the derivation of timesteppers that are higher order in time, and the quantifiable dissipative SP properties in the non-ideal, resistive case.
        
        We discuss possible options for extending this FET approach to timesteppers for the compressible case.

        The kinetic corrections satisfy linearized Boltzmann equations. Using a Lénard--Bernstein collision operator, these take Fokker--Planck-like forms \cite{Fokker_1914, Planck_1917} wherein pseudo-particles in the numerical model obey the neoclassical transport equations, with particle-independent Brownian drift terms. This offers a rigorous methodology for incorporating collisions into the particle transport model, without coupling the equations of motions for each particle.
        
        Works by Chen, Chacón et al. \cite{Chen_Chacón_Barnes_2011, Chacón_Chen_Barnes_2013, Chen_Chacón_2014, Chen_Chacón_2015} have developed structure-preserving particle pushers for neoclassical transport in the Vlasov equations, derived from Crank--Nicolson integrators. We show these too can can derive from a FET interpretation, similarly offering potential extensions to higher-order-in-time particle pushers. The FET formulation is used also to consider how the stochastic drift terms can be incorporated into the pushers. Stochastic gyrokinetic expansions are also discussed.

        Different options for the numerical implementation of these schemes are considered.

        Due to the efficacy of FET in the development of SP timesteppers for both the fluid and kinetic component, we hope this approach will prove effective in the future for developing SP timesteppers for the full hybrid model. We hope this will give us the opportunity to incorporate previously inaccessible kinetic effects into the highly effective, modern, finite-element MHD models.
    \end{abstract}
    
    
    \newpage
    \tableofcontents
    
    
    \newpage
    \pagenumbering{arabic}
    %\linenumbers\renewcommand\thelinenumber{\color{black!50}\arabic{linenumber}}
            \documentclass[12pt, a4paper]{report}

\documentclass[12pt, a4paper]{report}

\input{template/main.tex}

\title{\BA{Title in Progress...}}
\author{Boris Andrews}
\affil{Mathematical Institute, University of Oxford}
\date{\today}


\begin{document}
    \pagenumbering{gobble}
    \maketitle
    
    
    \begin{abstract}
        Magnetic confinement reactors---in particular tokamaks---offer one of the most promising options for achieving practical nuclear fusion, with the potential to provide virtually limitless, clean energy. The theoretical and numerical modeling of tokamak plasmas is simultaneously an essential component of effective reactor design, and a great research barrier. Tokamak operational conditions exhibit comparatively low Knudsen numbers. Kinetic effects, including kinetic waves and instabilities, Landau damping, bump-on-tail instabilities and more, are therefore highly influential in tokamak plasma dynamics. Purely fluid models are inherently incapable of capturing these effects, whereas the high dimensionality in purely kinetic models render them practically intractable for most relevant purposes.

        We consider a $\delta\!f$ decomposition model, with a macroscopic fluid background and microscopic kinetic correction, both fully coupled to each other. A similar manner of discretization is proposed to that used in the recent \texttt{STRUPHY} code \cite{Holderied_Possanner_Wang_2021, Holderied_2022, Li_et_al_2023} with a finite-element model for the background and a pseudo-particle/PiC model for the correction.

        The fluid background satisfies the full, non-linear, resistive, compressible, Hall MHD equations. \cite{Laakmann_Hu_Farrell_2022} introduces finite-element(-in-space) implicit timesteppers for the incompressible analogue to this system with structure-preserving (SP) properties in the ideal case, alongside parameter-robust preconditioners. We show that these timesteppers can derive from a finite-element-in-time (FET) (and finite-element-in-space) interpretation. The benefits of this reformulation are discussed, including the derivation of timesteppers that are higher order in time, and the quantifiable dissipative SP properties in the non-ideal, resistive case.
        
        We discuss possible options for extending this FET approach to timesteppers for the compressible case.

        The kinetic corrections satisfy linearized Boltzmann equations. Using a Lénard--Bernstein collision operator, these take Fokker--Planck-like forms \cite{Fokker_1914, Planck_1917} wherein pseudo-particles in the numerical model obey the neoclassical transport equations, with particle-independent Brownian drift terms. This offers a rigorous methodology for incorporating collisions into the particle transport model, without coupling the equations of motions for each particle.
        
        Works by Chen, Chacón et al. \cite{Chen_Chacón_Barnes_2011, Chacón_Chen_Barnes_2013, Chen_Chacón_2014, Chen_Chacón_2015} have developed structure-preserving particle pushers for neoclassical transport in the Vlasov equations, derived from Crank--Nicolson integrators. We show these too can can derive from a FET interpretation, similarly offering potential extensions to higher-order-in-time particle pushers. The FET formulation is used also to consider how the stochastic drift terms can be incorporated into the pushers. Stochastic gyrokinetic expansions are also discussed.

        Different options for the numerical implementation of these schemes are considered.

        Due to the efficacy of FET in the development of SP timesteppers for both the fluid and kinetic component, we hope this approach will prove effective in the future for developing SP timesteppers for the full hybrid model. We hope this will give us the opportunity to incorporate previously inaccessible kinetic effects into the highly effective, modern, finite-element MHD models.
    \end{abstract}
    
    
    \newpage
    \tableofcontents
    
    
    \newpage
    \pagenumbering{arabic}
    %\linenumbers\renewcommand\thelinenumber{\color{black!50}\arabic{linenumber}}
            \input{0 - introduction/main.tex}
        \part{Research}
            \input{1 - low-noise PiC models/main.tex}
            \input{2 - kinetic component/main.tex}
            \input{3 - fluid component/main.tex}
            \input{4 - numerical implementation/main.tex}
        \part{Project Overview}
            \input{5 - research plan/main.tex}
            \input{6 - summary/main.tex}
    
    
    %\section{}
    \newpage
    \pagenumbering{gobble}
        \printbibliography


    \newpage
    \pagenumbering{roman}
    \appendix
        \part{Appendices}
            \input{8 - Hilbert complexes/main.tex}
            \input{9 - weak conservation proofs/main.tex}
\end{document}


\title{\BA{Title in Progress...}}
\author{Boris Andrews}
\affil{Mathematical Institute, University of Oxford}
\date{\today}


\begin{document}
    \pagenumbering{gobble}
    \maketitle
    
    
    \begin{abstract}
        Magnetic confinement reactors---in particular tokamaks---offer one of the most promising options for achieving practical nuclear fusion, with the potential to provide virtually limitless, clean energy. The theoretical and numerical modeling of tokamak plasmas is simultaneously an essential component of effective reactor design, and a great research barrier. Tokamak operational conditions exhibit comparatively low Knudsen numbers. Kinetic effects, including kinetic waves and instabilities, Landau damping, bump-on-tail instabilities and more, are therefore highly influential in tokamak plasma dynamics. Purely fluid models are inherently incapable of capturing these effects, whereas the high dimensionality in purely kinetic models render them practically intractable for most relevant purposes.

        We consider a $\delta\!f$ decomposition model, with a macroscopic fluid background and microscopic kinetic correction, both fully coupled to each other. A similar manner of discretization is proposed to that used in the recent \texttt{STRUPHY} code \cite{Holderied_Possanner_Wang_2021, Holderied_2022, Li_et_al_2023} with a finite-element model for the background and a pseudo-particle/PiC model for the correction.

        The fluid background satisfies the full, non-linear, resistive, compressible, Hall MHD equations. \cite{Laakmann_Hu_Farrell_2022} introduces finite-element(-in-space) implicit timesteppers for the incompressible analogue to this system with structure-preserving (SP) properties in the ideal case, alongside parameter-robust preconditioners. We show that these timesteppers can derive from a finite-element-in-time (FET) (and finite-element-in-space) interpretation. The benefits of this reformulation are discussed, including the derivation of timesteppers that are higher order in time, and the quantifiable dissipative SP properties in the non-ideal, resistive case.
        
        We discuss possible options for extending this FET approach to timesteppers for the compressible case.

        The kinetic corrections satisfy linearized Boltzmann equations. Using a Lénard--Bernstein collision operator, these take Fokker--Planck-like forms \cite{Fokker_1914, Planck_1917} wherein pseudo-particles in the numerical model obey the neoclassical transport equations, with particle-independent Brownian drift terms. This offers a rigorous methodology for incorporating collisions into the particle transport model, without coupling the equations of motions for each particle.
        
        Works by Chen, Chacón et al. \cite{Chen_Chacón_Barnes_2011, Chacón_Chen_Barnes_2013, Chen_Chacón_2014, Chen_Chacón_2015} have developed structure-preserving particle pushers for neoclassical transport in the Vlasov equations, derived from Crank--Nicolson integrators. We show these too can can derive from a FET interpretation, similarly offering potential extensions to higher-order-in-time particle pushers. The FET formulation is used also to consider how the stochastic drift terms can be incorporated into the pushers. Stochastic gyrokinetic expansions are also discussed.

        Different options for the numerical implementation of these schemes are considered.

        Due to the efficacy of FET in the development of SP timesteppers for both the fluid and kinetic component, we hope this approach will prove effective in the future for developing SP timesteppers for the full hybrid model. We hope this will give us the opportunity to incorporate previously inaccessible kinetic effects into the highly effective, modern, finite-element MHD models.
    \end{abstract}
    
    
    \newpage
    \tableofcontents
    
    
    \newpage
    \pagenumbering{arabic}
    %\linenumbers\renewcommand\thelinenumber{\color{black!50}\arabic{linenumber}}
            \documentclass[12pt, a4paper]{report}

\input{template/main.tex}

\title{\BA{Title in Progress...}}
\author{Boris Andrews}
\affil{Mathematical Institute, University of Oxford}
\date{\today}


\begin{document}
    \pagenumbering{gobble}
    \maketitle
    
    
    \begin{abstract}
        Magnetic confinement reactors---in particular tokamaks---offer one of the most promising options for achieving practical nuclear fusion, with the potential to provide virtually limitless, clean energy. The theoretical and numerical modeling of tokamak plasmas is simultaneously an essential component of effective reactor design, and a great research barrier. Tokamak operational conditions exhibit comparatively low Knudsen numbers. Kinetic effects, including kinetic waves and instabilities, Landau damping, bump-on-tail instabilities and more, are therefore highly influential in tokamak plasma dynamics. Purely fluid models are inherently incapable of capturing these effects, whereas the high dimensionality in purely kinetic models render them practically intractable for most relevant purposes.

        We consider a $\delta\!f$ decomposition model, with a macroscopic fluid background and microscopic kinetic correction, both fully coupled to each other. A similar manner of discretization is proposed to that used in the recent \texttt{STRUPHY} code \cite{Holderied_Possanner_Wang_2021, Holderied_2022, Li_et_al_2023} with a finite-element model for the background and a pseudo-particle/PiC model for the correction.

        The fluid background satisfies the full, non-linear, resistive, compressible, Hall MHD equations. \cite{Laakmann_Hu_Farrell_2022} introduces finite-element(-in-space) implicit timesteppers for the incompressible analogue to this system with structure-preserving (SP) properties in the ideal case, alongside parameter-robust preconditioners. We show that these timesteppers can derive from a finite-element-in-time (FET) (and finite-element-in-space) interpretation. The benefits of this reformulation are discussed, including the derivation of timesteppers that are higher order in time, and the quantifiable dissipative SP properties in the non-ideal, resistive case.
        
        We discuss possible options for extending this FET approach to timesteppers for the compressible case.

        The kinetic corrections satisfy linearized Boltzmann equations. Using a Lénard--Bernstein collision operator, these take Fokker--Planck-like forms \cite{Fokker_1914, Planck_1917} wherein pseudo-particles in the numerical model obey the neoclassical transport equations, with particle-independent Brownian drift terms. This offers a rigorous methodology for incorporating collisions into the particle transport model, without coupling the equations of motions for each particle.
        
        Works by Chen, Chacón et al. \cite{Chen_Chacón_Barnes_2011, Chacón_Chen_Barnes_2013, Chen_Chacón_2014, Chen_Chacón_2015} have developed structure-preserving particle pushers for neoclassical transport in the Vlasov equations, derived from Crank--Nicolson integrators. We show these too can can derive from a FET interpretation, similarly offering potential extensions to higher-order-in-time particle pushers. The FET formulation is used also to consider how the stochastic drift terms can be incorporated into the pushers. Stochastic gyrokinetic expansions are also discussed.

        Different options for the numerical implementation of these schemes are considered.

        Due to the efficacy of FET in the development of SP timesteppers for both the fluid and kinetic component, we hope this approach will prove effective in the future for developing SP timesteppers for the full hybrid model. We hope this will give us the opportunity to incorporate previously inaccessible kinetic effects into the highly effective, modern, finite-element MHD models.
    \end{abstract}
    
    
    \newpage
    \tableofcontents
    
    
    \newpage
    \pagenumbering{arabic}
    %\linenumbers\renewcommand\thelinenumber{\color{black!50}\arabic{linenumber}}
            \input{0 - introduction/main.tex}
        \part{Research}
            \input{1 - low-noise PiC models/main.tex}
            \input{2 - kinetic component/main.tex}
            \input{3 - fluid component/main.tex}
            \input{4 - numerical implementation/main.tex}
        \part{Project Overview}
            \input{5 - research plan/main.tex}
            \input{6 - summary/main.tex}
    
    
    %\section{}
    \newpage
    \pagenumbering{gobble}
        \printbibliography


    \newpage
    \pagenumbering{roman}
    \appendix
        \part{Appendices}
            \input{8 - Hilbert complexes/main.tex}
            \input{9 - weak conservation proofs/main.tex}
\end{document}

        \part{Research}
            \documentclass[12pt, a4paper]{report}

\input{template/main.tex}

\title{\BA{Title in Progress...}}
\author{Boris Andrews}
\affil{Mathematical Institute, University of Oxford}
\date{\today}


\begin{document}
    \pagenumbering{gobble}
    \maketitle
    
    
    \begin{abstract}
        Magnetic confinement reactors---in particular tokamaks---offer one of the most promising options for achieving practical nuclear fusion, with the potential to provide virtually limitless, clean energy. The theoretical and numerical modeling of tokamak plasmas is simultaneously an essential component of effective reactor design, and a great research barrier. Tokamak operational conditions exhibit comparatively low Knudsen numbers. Kinetic effects, including kinetic waves and instabilities, Landau damping, bump-on-tail instabilities and more, are therefore highly influential in tokamak plasma dynamics. Purely fluid models are inherently incapable of capturing these effects, whereas the high dimensionality in purely kinetic models render them practically intractable for most relevant purposes.

        We consider a $\delta\!f$ decomposition model, with a macroscopic fluid background and microscopic kinetic correction, both fully coupled to each other. A similar manner of discretization is proposed to that used in the recent \texttt{STRUPHY} code \cite{Holderied_Possanner_Wang_2021, Holderied_2022, Li_et_al_2023} with a finite-element model for the background and a pseudo-particle/PiC model for the correction.

        The fluid background satisfies the full, non-linear, resistive, compressible, Hall MHD equations. \cite{Laakmann_Hu_Farrell_2022} introduces finite-element(-in-space) implicit timesteppers for the incompressible analogue to this system with structure-preserving (SP) properties in the ideal case, alongside parameter-robust preconditioners. We show that these timesteppers can derive from a finite-element-in-time (FET) (and finite-element-in-space) interpretation. The benefits of this reformulation are discussed, including the derivation of timesteppers that are higher order in time, and the quantifiable dissipative SP properties in the non-ideal, resistive case.
        
        We discuss possible options for extending this FET approach to timesteppers for the compressible case.

        The kinetic corrections satisfy linearized Boltzmann equations. Using a Lénard--Bernstein collision operator, these take Fokker--Planck-like forms \cite{Fokker_1914, Planck_1917} wherein pseudo-particles in the numerical model obey the neoclassical transport equations, with particle-independent Brownian drift terms. This offers a rigorous methodology for incorporating collisions into the particle transport model, without coupling the equations of motions for each particle.
        
        Works by Chen, Chacón et al. \cite{Chen_Chacón_Barnes_2011, Chacón_Chen_Barnes_2013, Chen_Chacón_2014, Chen_Chacón_2015} have developed structure-preserving particle pushers for neoclassical transport in the Vlasov equations, derived from Crank--Nicolson integrators. We show these too can can derive from a FET interpretation, similarly offering potential extensions to higher-order-in-time particle pushers. The FET formulation is used also to consider how the stochastic drift terms can be incorporated into the pushers. Stochastic gyrokinetic expansions are also discussed.

        Different options for the numerical implementation of these schemes are considered.

        Due to the efficacy of FET in the development of SP timesteppers for both the fluid and kinetic component, we hope this approach will prove effective in the future for developing SP timesteppers for the full hybrid model. We hope this will give us the opportunity to incorporate previously inaccessible kinetic effects into the highly effective, modern, finite-element MHD models.
    \end{abstract}
    
    
    \newpage
    \tableofcontents
    
    
    \newpage
    \pagenumbering{arabic}
    %\linenumbers\renewcommand\thelinenumber{\color{black!50}\arabic{linenumber}}
            \input{0 - introduction/main.tex}
        \part{Research}
            \input{1 - low-noise PiC models/main.tex}
            \input{2 - kinetic component/main.tex}
            \input{3 - fluid component/main.tex}
            \input{4 - numerical implementation/main.tex}
        \part{Project Overview}
            \input{5 - research plan/main.tex}
            \input{6 - summary/main.tex}
    
    
    %\section{}
    \newpage
    \pagenumbering{gobble}
        \printbibliography


    \newpage
    \pagenumbering{roman}
    \appendix
        \part{Appendices}
            \input{8 - Hilbert complexes/main.tex}
            \input{9 - weak conservation proofs/main.tex}
\end{document}

            \documentclass[12pt, a4paper]{report}

\input{template/main.tex}

\title{\BA{Title in Progress...}}
\author{Boris Andrews}
\affil{Mathematical Institute, University of Oxford}
\date{\today}


\begin{document}
    \pagenumbering{gobble}
    \maketitle
    
    
    \begin{abstract}
        Magnetic confinement reactors---in particular tokamaks---offer one of the most promising options for achieving practical nuclear fusion, with the potential to provide virtually limitless, clean energy. The theoretical and numerical modeling of tokamak plasmas is simultaneously an essential component of effective reactor design, and a great research barrier. Tokamak operational conditions exhibit comparatively low Knudsen numbers. Kinetic effects, including kinetic waves and instabilities, Landau damping, bump-on-tail instabilities and more, are therefore highly influential in tokamak plasma dynamics. Purely fluid models are inherently incapable of capturing these effects, whereas the high dimensionality in purely kinetic models render them practically intractable for most relevant purposes.

        We consider a $\delta\!f$ decomposition model, with a macroscopic fluid background and microscopic kinetic correction, both fully coupled to each other. A similar manner of discretization is proposed to that used in the recent \texttt{STRUPHY} code \cite{Holderied_Possanner_Wang_2021, Holderied_2022, Li_et_al_2023} with a finite-element model for the background and a pseudo-particle/PiC model for the correction.

        The fluid background satisfies the full, non-linear, resistive, compressible, Hall MHD equations. \cite{Laakmann_Hu_Farrell_2022} introduces finite-element(-in-space) implicit timesteppers for the incompressible analogue to this system with structure-preserving (SP) properties in the ideal case, alongside parameter-robust preconditioners. We show that these timesteppers can derive from a finite-element-in-time (FET) (and finite-element-in-space) interpretation. The benefits of this reformulation are discussed, including the derivation of timesteppers that are higher order in time, and the quantifiable dissipative SP properties in the non-ideal, resistive case.
        
        We discuss possible options for extending this FET approach to timesteppers for the compressible case.

        The kinetic corrections satisfy linearized Boltzmann equations. Using a Lénard--Bernstein collision operator, these take Fokker--Planck-like forms \cite{Fokker_1914, Planck_1917} wherein pseudo-particles in the numerical model obey the neoclassical transport equations, with particle-independent Brownian drift terms. This offers a rigorous methodology for incorporating collisions into the particle transport model, without coupling the equations of motions for each particle.
        
        Works by Chen, Chacón et al. \cite{Chen_Chacón_Barnes_2011, Chacón_Chen_Barnes_2013, Chen_Chacón_2014, Chen_Chacón_2015} have developed structure-preserving particle pushers for neoclassical transport in the Vlasov equations, derived from Crank--Nicolson integrators. We show these too can can derive from a FET interpretation, similarly offering potential extensions to higher-order-in-time particle pushers. The FET formulation is used also to consider how the stochastic drift terms can be incorporated into the pushers. Stochastic gyrokinetic expansions are also discussed.

        Different options for the numerical implementation of these schemes are considered.

        Due to the efficacy of FET in the development of SP timesteppers for both the fluid and kinetic component, we hope this approach will prove effective in the future for developing SP timesteppers for the full hybrid model. We hope this will give us the opportunity to incorporate previously inaccessible kinetic effects into the highly effective, modern, finite-element MHD models.
    \end{abstract}
    
    
    \newpage
    \tableofcontents
    
    
    \newpage
    \pagenumbering{arabic}
    %\linenumbers\renewcommand\thelinenumber{\color{black!50}\arabic{linenumber}}
            \input{0 - introduction/main.tex}
        \part{Research}
            \input{1 - low-noise PiC models/main.tex}
            \input{2 - kinetic component/main.tex}
            \input{3 - fluid component/main.tex}
            \input{4 - numerical implementation/main.tex}
        \part{Project Overview}
            \input{5 - research plan/main.tex}
            \input{6 - summary/main.tex}
    
    
    %\section{}
    \newpage
    \pagenumbering{gobble}
        \printbibliography


    \newpage
    \pagenumbering{roman}
    \appendix
        \part{Appendices}
            \input{8 - Hilbert complexes/main.tex}
            \input{9 - weak conservation proofs/main.tex}
\end{document}

            \documentclass[12pt, a4paper]{report}

\input{template/main.tex}

\title{\BA{Title in Progress...}}
\author{Boris Andrews}
\affil{Mathematical Institute, University of Oxford}
\date{\today}


\begin{document}
    \pagenumbering{gobble}
    \maketitle
    
    
    \begin{abstract}
        Magnetic confinement reactors---in particular tokamaks---offer one of the most promising options for achieving practical nuclear fusion, with the potential to provide virtually limitless, clean energy. The theoretical and numerical modeling of tokamak plasmas is simultaneously an essential component of effective reactor design, and a great research barrier. Tokamak operational conditions exhibit comparatively low Knudsen numbers. Kinetic effects, including kinetic waves and instabilities, Landau damping, bump-on-tail instabilities and more, are therefore highly influential in tokamak plasma dynamics. Purely fluid models are inherently incapable of capturing these effects, whereas the high dimensionality in purely kinetic models render them practically intractable for most relevant purposes.

        We consider a $\delta\!f$ decomposition model, with a macroscopic fluid background and microscopic kinetic correction, both fully coupled to each other. A similar manner of discretization is proposed to that used in the recent \texttt{STRUPHY} code \cite{Holderied_Possanner_Wang_2021, Holderied_2022, Li_et_al_2023} with a finite-element model for the background and a pseudo-particle/PiC model for the correction.

        The fluid background satisfies the full, non-linear, resistive, compressible, Hall MHD equations. \cite{Laakmann_Hu_Farrell_2022} introduces finite-element(-in-space) implicit timesteppers for the incompressible analogue to this system with structure-preserving (SP) properties in the ideal case, alongside parameter-robust preconditioners. We show that these timesteppers can derive from a finite-element-in-time (FET) (and finite-element-in-space) interpretation. The benefits of this reformulation are discussed, including the derivation of timesteppers that are higher order in time, and the quantifiable dissipative SP properties in the non-ideal, resistive case.
        
        We discuss possible options for extending this FET approach to timesteppers for the compressible case.

        The kinetic corrections satisfy linearized Boltzmann equations. Using a Lénard--Bernstein collision operator, these take Fokker--Planck-like forms \cite{Fokker_1914, Planck_1917} wherein pseudo-particles in the numerical model obey the neoclassical transport equations, with particle-independent Brownian drift terms. This offers a rigorous methodology for incorporating collisions into the particle transport model, without coupling the equations of motions for each particle.
        
        Works by Chen, Chacón et al. \cite{Chen_Chacón_Barnes_2011, Chacón_Chen_Barnes_2013, Chen_Chacón_2014, Chen_Chacón_2015} have developed structure-preserving particle pushers for neoclassical transport in the Vlasov equations, derived from Crank--Nicolson integrators. We show these too can can derive from a FET interpretation, similarly offering potential extensions to higher-order-in-time particle pushers. The FET formulation is used also to consider how the stochastic drift terms can be incorporated into the pushers. Stochastic gyrokinetic expansions are also discussed.

        Different options for the numerical implementation of these schemes are considered.

        Due to the efficacy of FET in the development of SP timesteppers for both the fluid and kinetic component, we hope this approach will prove effective in the future for developing SP timesteppers for the full hybrid model. We hope this will give us the opportunity to incorporate previously inaccessible kinetic effects into the highly effective, modern, finite-element MHD models.
    \end{abstract}
    
    
    \newpage
    \tableofcontents
    
    
    \newpage
    \pagenumbering{arabic}
    %\linenumbers\renewcommand\thelinenumber{\color{black!50}\arabic{linenumber}}
            \input{0 - introduction/main.tex}
        \part{Research}
            \input{1 - low-noise PiC models/main.tex}
            \input{2 - kinetic component/main.tex}
            \input{3 - fluid component/main.tex}
            \input{4 - numerical implementation/main.tex}
        \part{Project Overview}
            \input{5 - research plan/main.tex}
            \input{6 - summary/main.tex}
    
    
    %\section{}
    \newpage
    \pagenumbering{gobble}
        \printbibliography


    \newpage
    \pagenumbering{roman}
    \appendix
        \part{Appendices}
            \input{8 - Hilbert complexes/main.tex}
            \input{9 - weak conservation proofs/main.tex}
\end{document}

            \documentclass[12pt, a4paper]{report}

\input{template/main.tex}

\title{\BA{Title in Progress...}}
\author{Boris Andrews}
\affil{Mathematical Institute, University of Oxford}
\date{\today}


\begin{document}
    \pagenumbering{gobble}
    \maketitle
    
    
    \begin{abstract}
        Magnetic confinement reactors---in particular tokamaks---offer one of the most promising options for achieving practical nuclear fusion, with the potential to provide virtually limitless, clean energy. The theoretical and numerical modeling of tokamak plasmas is simultaneously an essential component of effective reactor design, and a great research barrier. Tokamak operational conditions exhibit comparatively low Knudsen numbers. Kinetic effects, including kinetic waves and instabilities, Landau damping, bump-on-tail instabilities and more, are therefore highly influential in tokamak plasma dynamics. Purely fluid models are inherently incapable of capturing these effects, whereas the high dimensionality in purely kinetic models render them practically intractable for most relevant purposes.

        We consider a $\delta\!f$ decomposition model, with a macroscopic fluid background and microscopic kinetic correction, both fully coupled to each other. A similar manner of discretization is proposed to that used in the recent \texttt{STRUPHY} code \cite{Holderied_Possanner_Wang_2021, Holderied_2022, Li_et_al_2023} with a finite-element model for the background and a pseudo-particle/PiC model for the correction.

        The fluid background satisfies the full, non-linear, resistive, compressible, Hall MHD equations. \cite{Laakmann_Hu_Farrell_2022} introduces finite-element(-in-space) implicit timesteppers for the incompressible analogue to this system with structure-preserving (SP) properties in the ideal case, alongside parameter-robust preconditioners. We show that these timesteppers can derive from a finite-element-in-time (FET) (and finite-element-in-space) interpretation. The benefits of this reformulation are discussed, including the derivation of timesteppers that are higher order in time, and the quantifiable dissipative SP properties in the non-ideal, resistive case.
        
        We discuss possible options for extending this FET approach to timesteppers for the compressible case.

        The kinetic corrections satisfy linearized Boltzmann equations. Using a Lénard--Bernstein collision operator, these take Fokker--Planck-like forms \cite{Fokker_1914, Planck_1917} wherein pseudo-particles in the numerical model obey the neoclassical transport equations, with particle-independent Brownian drift terms. This offers a rigorous methodology for incorporating collisions into the particle transport model, without coupling the equations of motions for each particle.
        
        Works by Chen, Chacón et al. \cite{Chen_Chacón_Barnes_2011, Chacón_Chen_Barnes_2013, Chen_Chacón_2014, Chen_Chacón_2015} have developed structure-preserving particle pushers for neoclassical transport in the Vlasov equations, derived from Crank--Nicolson integrators. We show these too can can derive from a FET interpretation, similarly offering potential extensions to higher-order-in-time particle pushers. The FET formulation is used also to consider how the stochastic drift terms can be incorporated into the pushers. Stochastic gyrokinetic expansions are also discussed.

        Different options for the numerical implementation of these schemes are considered.

        Due to the efficacy of FET in the development of SP timesteppers for both the fluid and kinetic component, we hope this approach will prove effective in the future for developing SP timesteppers for the full hybrid model. We hope this will give us the opportunity to incorporate previously inaccessible kinetic effects into the highly effective, modern, finite-element MHD models.
    \end{abstract}
    
    
    \newpage
    \tableofcontents
    
    
    \newpage
    \pagenumbering{arabic}
    %\linenumbers\renewcommand\thelinenumber{\color{black!50}\arabic{linenumber}}
            \input{0 - introduction/main.tex}
        \part{Research}
            \input{1 - low-noise PiC models/main.tex}
            \input{2 - kinetic component/main.tex}
            \input{3 - fluid component/main.tex}
            \input{4 - numerical implementation/main.tex}
        \part{Project Overview}
            \input{5 - research plan/main.tex}
            \input{6 - summary/main.tex}
    
    
    %\section{}
    \newpage
    \pagenumbering{gobble}
        \printbibliography


    \newpage
    \pagenumbering{roman}
    \appendix
        \part{Appendices}
            \input{8 - Hilbert complexes/main.tex}
            \input{9 - weak conservation proofs/main.tex}
\end{document}

        \part{Project Overview}
            \documentclass[12pt, a4paper]{report}

\input{template/main.tex}

\title{\BA{Title in Progress...}}
\author{Boris Andrews}
\affil{Mathematical Institute, University of Oxford}
\date{\today}


\begin{document}
    \pagenumbering{gobble}
    \maketitle
    
    
    \begin{abstract}
        Magnetic confinement reactors---in particular tokamaks---offer one of the most promising options for achieving practical nuclear fusion, with the potential to provide virtually limitless, clean energy. The theoretical and numerical modeling of tokamak plasmas is simultaneously an essential component of effective reactor design, and a great research barrier. Tokamak operational conditions exhibit comparatively low Knudsen numbers. Kinetic effects, including kinetic waves and instabilities, Landau damping, bump-on-tail instabilities and more, are therefore highly influential in tokamak plasma dynamics. Purely fluid models are inherently incapable of capturing these effects, whereas the high dimensionality in purely kinetic models render them practically intractable for most relevant purposes.

        We consider a $\delta\!f$ decomposition model, with a macroscopic fluid background and microscopic kinetic correction, both fully coupled to each other. A similar manner of discretization is proposed to that used in the recent \texttt{STRUPHY} code \cite{Holderied_Possanner_Wang_2021, Holderied_2022, Li_et_al_2023} with a finite-element model for the background and a pseudo-particle/PiC model for the correction.

        The fluid background satisfies the full, non-linear, resistive, compressible, Hall MHD equations. \cite{Laakmann_Hu_Farrell_2022} introduces finite-element(-in-space) implicit timesteppers for the incompressible analogue to this system with structure-preserving (SP) properties in the ideal case, alongside parameter-robust preconditioners. We show that these timesteppers can derive from a finite-element-in-time (FET) (and finite-element-in-space) interpretation. The benefits of this reformulation are discussed, including the derivation of timesteppers that are higher order in time, and the quantifiable dissipative SP properties in the non-ideal, resistive case.
        
        We discuss possible options for extending this FET approach to timesteppers for the compressible case.

        The kinetic corrections satisfy linearized Boltzmann equations. Using a Lénard--Bernstein collision operator, these take Fokker--Planck-like forms \cite{Fokker_1914, Planck_1917} wherein pseudo-particles in the numerical model obey the neoclassical transport equations, with particle-independent Brownian drift terms. This offers a rigorous methodology for incorporating collisions into the particle transport model, without coupling the equations of motions for each particle.
        
        Works by Chen, Chacón et al. \cite{Chen_Chacón_Barnes_2011, Chacón_Chen_Barnes_2013, Chen_Chacón_2014, Chen_Chacón_2015} have developed structure-preserving particle pushers for neoclassical transport in the Vlasov equations, derived from Crank--Nicolson integrators. We show these too can can derive from a FET interpretation, similarly offering potential extensions to higher-order-in-time particle pushers. The FET formulation is used also to consider how the stochastic drift terms can be incorporated into the pushers. Stochastic gyrokinetic expansions are also discussed.

        Different options for the numerical implementation of these schemes are considered.

        Due to the efficacy of FET in the development of SP timesteppers for both the fluid and kinetic component, we hope this approach will prove effective in the future for developing SP timesteppers for the full hybrid model. We hope this will give us the opportunity to incorporate previously inaccessible kinetic effects into the highly effective, modern, finite-element MHD models.
    \end{abstract}
    
    
    \newpage
    \tableofcontents
    
    
    \newpage
    \pagenumbering{arabic}
    %\linenumbers\renewcommand\thelinenumber{\color{black!50}\arabic{linenumber}}
            \input{0 - introduction/main.tex}
        \part{Research}
            \input{1 - low-noise PiC models/main.tex}
            \input{2 - kinetic component/main.tex}
            \input{3 - fluid component/main.tex}
            \input{4 - numerical implementation/main.tex}
        \part{Project Overview}
            \input{5 - research plan/main.tex}
            \input{6 - summary/main.tex}
    
    
    %\section{}
    \newpage
    \pagenumbering{gobble}
        \printbibliography


    \newpage
    \pagenumbering{roman}
    \appendix
        \part{Appendices}
            \input{8 - Hilbert complexes/main.tex}
            \input{9 - weak conservation proofs/main.tex}
\end{document}

            \documentclass[12pt, a4paper]{report}

\input{template/main.tex}

\title{\BA{Title in Progress...}}
\author{Boris Andrews}
\affil{Mathematical Institute, University of Oxford}
\date{\today}


\begin{document}
    \pagenumbering{gobble}
    \maketitle
    
    
    \begin{abstract}
        Magnetic confinement reactors---in particular tokamaks---offer one of the most promising options for achieving practical nuclear fusion, with the potential to provide virtually limitless, clean energy. The theoretical and numerical modeling of tokamak plasmas is simultaneously an essential component of effective reactor design, and a great research barrier. Tokamak operational conditions exhibit comparatively low Knudsen numbers. Kinetic effects, including kinetic waves and instabilities, Landau damping, bump-on-tail instabilities and more, are therefore highly influential in tokamak plasma dynamics. Purely fluid models are inherently incapable of capturing these effects, whereas the high dimensionality in purely kinetic models render them practically intractable for most relevant purposes.

        We consider a $\delta\!f$ decomposition model, with a macroscopic fluid background and microscopic kinetic correction, both fully coupled to each other. A similar manner of discretization is proposed to that used in the recent \texttt{STRUPHY} code \cite{Holderied_Possanner_Wang_2021, Holderied_2022, Li_et_al_2023} with a finite-element model for the background and a pseudo-particle/PiC model for the correction.

        The fluid background satisfies the full, non-linear, resistive, compressible, Hall MHD equations. \cite{Laakmann_Hu_Farrell_2022} introduces finite-element(-in-space) implicit timesteppers for the incompressible analogue to this system with structure-preserving (SP) properties in the ideal case, alongside parameter-robust preconditioners. We show that these timesteppers can derive from a finite-element-in-time (FET) (and finite-element-in-space) interpretation. The benefits of this reformulation are discussed, including the derivation of timesteppers that are higher order in time, and the quantifiable dissipative SP properties in the non-ideal, resistive case.
        
        We discuss possible options for extending this FET approach to timesteppers for the compressible case.

        The kinetic corrections satisfy linearized Boltzmann equations. Using a Lénard--Bernstein collision operator, these take Fokker--Planck-like forms \cite{Fokker_1914, Planck_1917} wherein pseudo-particles in the numerical model obey the neoclassical transport equations, with particle-independent Brownian drift terms. This offers a rigorous methodology for incorporating collisions into the particle transport model, without coupling the equations of motions for each particle.
        
        Works by Chen, Chacón et al. \cite{Chen_Chacón_Barnes_2011, Chacón_Chen_Barnes_2013, Chen_Chacón_2014, Chen_Chacón_2015} have developed structure-preserving particle pushers for neoclassical transport in the Vlasov equations, derived from Crank--Nicolson integrators. We show these too can can derive from a FET interpretation, similarly offering potential extensions to higher-order-in-time particle pushers. The FET formulation is used also to consider how the stochastic drift terms can be incorporated into the pushers. Stochastic gyrokinetic expansions are also discussed.

        Different options for the numerical implementation of these schemes are considered.

        Due to the efficacy of FET in the development of SP timesteppers for both the fluid and kinetic component, we hope this approach will prove effective in the future for developing SP timesteppers for the full hybrid model. We hope this will give us the opportunity to incorporate previously inaccessible kinetic effects into the highly effective, modern, finite-element MHD models.
    \end{abstract}
    
    
    \newpage
    \tableofcontents
    
    
    \newpage
    \pagenumbering{arabic}
    %\linenumbers\renewcommand\thelinenumber{\color{black!50}\arabic{linenumber}}
            \input{0 - introduction/main.tex}
        \part{Research}
            \input{1 - low-noise PiC models/main.tex}
            \input{2 - kinetic component/main.tex}
            \input{3 - fluid component/main.tex}
            \input{4 - numerical implementation/main.tex}
        \part{Project Overview}
            \input{5 - research plan/main.tex}
            \input{6 - summary/main.tex}
    
    
    %\section{}
    \newpage
    \pagenumbering{gobble}
        \printbibliography


    \newpage
    \pagenumbering{roman}
    \appendix
        \part{Appendices}
            \input{8 - Hilbert complexes/main.tex}
            \input{9 - weak conservation proofs/main.tex}
\end{document}

    
    
    %\section{}
    \newpage
    \pagenumbering{gobble}
        \printbibliography


    \newpage
    \pagenumbering{roman}
    \appendix
        \part{Appendices}
            \documentclass[12pt, a4paper]{report}

\input{template/main.tex}

\title{\BA{Title in Progress...}}
\author{Boris Andrews}
\affil{Mathematical Institute, University of Oxford}
\date{\today}


\begin{document}
    \pagenumbering{gobble}
    \maketitle
    
    
    \begin{abstract}
        Magnetic confinement reactors---in particular tokamaks---offer one of the most promising options for achieving practical nuclear fusion, with the potential to provide virtually limitless, clean energy. The theoretical and numerical modeling of tokamak plasmas is simultaneously an essential component of effective reactor design, and a great research barrier. Tokamak operational conditions exhibit comparatively low Knudsen numbers. Kinetic effects, including kinetic waves and instabilities, Landau damping, bump-on-tail instabilities and more, are therefore highly influential in tokamak plasma dynamics. Purely fluid models are inherently incapable of capturing these effects, whereas the high dimensionality in purely kinetic models render them practically intractable for most relevant purposes.

        We consider a $\delta\!f$ decomposition model, with a macroscopic fluid background and microscopic kinetic correction, both fully coupled to each other. A similar manner of discretization is proposed to that used in the recent \texttt{STRUPHY} code \cite{Holderied_Possanner_Wang_2021, Holderied_2022, Li_et_al_2023} with a finite-element model for the background and a pseudo-particle/PiC model for the correction.

        The fluid background satisfies the full, non-linear, resistive, compressible, Hall MHD equations. \cite{Laakmann_Hu_Farrell_2022} introduces finite-element(-in-space) implicit timesteppers for the incompressible analogue to this system with structure-preserving (SP) properties in the ideal case, alongside parameter-robust preconditioners. We show that these timesteppers can derive from a finite-element-in-time (FET) (and finite-element-in-space) interpretation. The benefits of this reformulation are discussed, including the derivation of timesteppers that are higher order in time, and the quantifiable dissipative SP properties in the non-ideal, resistive case.
        
        We discuss possible options for extending this FET approach to timesteppers for the compressible case.

        The kinetic corrections satisfy linearized Boltzmann equations. Using a Lénard--Bernstein collision operator, these take Fokker--Planck-like forms \cite{Fokker_1914, Planck_1917} wherein pseudo-particles in the numerical model obey the neoclassical transport equations, with particle-independent Brownian drift terms. This offers a rigorous methodology for incorporating collisions into the particle transport model, without coupling the equations of motions for each particle.
        
        Works by Chen, Chacón et al. \cite{Chen_Chacón_Barnes_2011, Chacón_Chen_Barnes_2013, Chen_Chacón_2014, Chen_Chacón_2015} have developed structure-preserving particle pushers for neoclassical transport in the Vlasov equations, derived from Crank--Nicolson integrators. We show these too can can derive from a FET interpretation, similarly offering potential extensions to higher-order-in-time particle pushers. The FET formulation is used also to consider how the stochastic drift terms can be incorporated into the pushers. Stochastic gyrokinetic expansions are also discussed.

        Different options for the numerical implementation of these schemes are considered.

        Due to the efficacy of FET in the development of SP timesteppers for both the fluid and kinetic component, we hope this approach will prove effective in the future for developing SP timesteppers for the full hybrid model. We hope this will give us the opportunity to incorporate previously inaccessible kinetic effects into the highly effective, modern, finite-element MHD models.
    \end{abstract}
    
    
    \newpage
    \tableofcontents
    
    
    \newpage
    \pagenumbering{arabic}
    %\linenumbers\renewcommand\thelinenumber{\color{black!50}\arabic{linenumber}}
            \input{0 - introduction/main.tex}
        \part{Research}
            \input{1 - low-noise PiC models/main.tex}
            \input{2 - kinetic component/main.tex}
            \input{3 - fluid component/main.tex}
            \input{4 - numerical implementation/main.tex}
        \part{Project Overview}
            \input{5 - research plan/main.tex}
            \input{6 - summary/main.tex}
    
    
    %\section{}
    \newpage
    \pagenumbering{gobble}
        \printbibliography


    \newpage
    \pagenumbering{roman}
    \appendix
        \part{Appendices}
            \input{8 - Hilbert complexes/main.tex}
            \input{9 - weak conservation proofs/main.tex}
\end{document}

            \documentclass[12pt, a4paper]{report}

\input{template/main.tex}

\title{\BA{Title in Progress...}}
\author{Boris Andrews}
\affil{Mathematical Institute, University of Oxford}
\date{\today}


\begin{document}
    \pagenumbering{gobble}
    \maketitle
    
    
    \begin{abstract}
        Magnetic confinement reactors---in particular tokamaks---offer one of the most promising options for achieving practical nuclear fusion, with the potential to provide virtually limitless, clean energy. The theoretical and numerical modeling of tokamak plasmas is simultaneously an essential component of effective reactor design, and a great research barrier. Tokamak operational conditions exhibit comparatively low Knudsen numbers. Kinetic effects, including kinetic waves and instabilities, Landau damping, bump-on-tail instabilities and more, are therefore highly influential in tokamak plasma dynamics. Purely fluid models are inherently incapable of capturing these effects, whereas the high dimensionality in purely kinetic models render them practically intractable for most relevant purposes.

        We consider a $\delta\!f$ decomposition model, with a macroscopic fluid background and microscopic kinetic correction, both fully coupled to each other. A similar manner of discretization is proposed to that used in the recent \texttt{STRUPHY} code \cite{Holderied_Possanner_Wang_2021, Holderied_2022, Li_et_al_2023} with a finite-element model for the background and a pseudo-particle/PiC model for the correction.

        The fluid background satisfies the full, non-linear, resistive, compressible, Hall MHD equations. \cite{Laakmann_Hu_Farrell_2022} introduces finite-element(-in-space) implicit timesteppers for the incompressible analogue to this system with structure-preserving (SP) properties in the ideal case, alongside parameter-robust preconditioners. We show that these timesteppers can derive from a finite-element-in-time (FET) (and finite-element-in-space) interpretation. The benefits of this reformulation are discussed, including the derivation of timesteppers that are higher order in time, and the quantifiable dissipative SP properties in the non-ideal, resistive case.
        
        We discuss possible options for extending this FET approach to timesteppers for the compressible case.

        The kinetic corrections satisfy linearized Boltzmann equations. Using a Lénard--Bernstein collision operator, these take Fokker--Planck-like forms \cite{Fokker_1914, Planck_1917} wherein pseudo-particles in the numerical model obey the neoclassical transport equations, with particle-independent Brownian drift terms. This offers a rigorous methodology for incorporating collisions into the particle transport model, without coupling the equations of motions for each particle.
        
        Works by Chen, Chacón et al. \cite{Chen_Chacón_Barnes_2011, Chacón_Chen_Barnes_2013, Chen_Chacón_2014, Chen_Chacón_2015} have developed structure-preserving particle pushers for neoclassical transport in the Vlasov equations, derived from Crank--Nicolson integrators. We show these too can can derive from a FET interpretation, similarly offering potential extensions to higher-order-in-time particle pushers. The FET formulation is used also to consider how the stochastic drift terms can be incorporated into the pushers. Stochastic gyrokinetic expansions are also discussed.

        Different options for the numerical implementation of these schemes are considered.

        Due to the efficacy of FET in the development of SP timesteppers for both the fluid and kinetic component, we hope this approach will prove effective in the future for developing SP timesteppers for the full hybrid model. We hope this will give us the opportunity to incorporate previously inaccessible kinetic effects into the highly effective, modern, finite-element MHD models.
    \end{abstract}
    
    
    \newpage
    \tableofcontents
    
    
    \newpage
    \pagenumbering{arabic}
    %\linenumbers\renewcommand\thelinenumber{\color{black!50}\arabic{linenumber}}
            \input{0 - introduction/main.tex}
        \part{Research}
            \input{1 - low-noise PiC models/main.tex}
            \input{2 - kinetic component/main.tex}
            \input{3 - fluid component/main.tex}
            \input{4 - numerical implementation/main.tex}
        \part{Project Overview}
            \input{5 - research plan/main.tex}
            \input{6 - summary/main.tex}
    
    
    %\section{}
    \newpage
    \pagenumbering{gobble}
        \printbibliography


    \newpage
    \pagenumbering{roman}
    \appendix
        \part{Appendices}
            \input{8 - Hilbert complexes/main.tex}
            \input{9 - weak conservation proofs/main.tex}
\end{document}

\end{document}

        \part{Research}
            \documentclass[12pt, a4paper]{report}

\documentclass[12pt, a4paper]{report}

\input{template/main.tex}

\title{\BA{Title in Progress...}}
\author{Boris Andrews}
\affil{Mathematical Institute, University of Oxford}
\date{\today}


\begin{document}
    \pagenumbering{gobble}
    \maketitle
    
    
    \begin{abstract}
        Magnetic confinement reactors---in particular tokamaks---offer one of the most promising options for achieving practical nuclear fusion, with the potential to provide virtually limitless, clean energy. The theoretical and numerical modeling of tokamak plasmas is simultaneously an essential component of effective reactor design, and a great research barrier. Tokamak operational conditions exhibit comparatively low Knudsen numbers. Kinetic effects, including kinetic waves and instabilities, Landau damping, bump-on-tail instabilities and more, are therefore highly influential in tokamak plasma dynamics. Purely fluid models are inherently incapable of capturing these effects, whereas the high dimensionality in purely kinetic models render them practically intractable for most relevant purposes.

        We consider a $\delta\!f$ decomposition model, with a macroscopic fluid background and microscopic kinetic correction, both fully coupled to each other. A similar manner of discretization is proposed to that used in the recent \texttt{STRUPHY} code \cite{Holderied_Possanner_Wang_2021, Holderied_2022, Li_et_al_2023} with a finite-element model for the background and a pseudo-particle/PiC model for the correction.

        The fluid background satisfies the full, non-linear, resistive, compressible, Hall MHD equations. \cite{Laakmann_Hu_Farrell_2022} introduces finite-element(-in-space) implicit timesteppers for the incompressible analogue to this system with structure-preserving (SP) properties in the ideal case, alongside parameter-robust preconditioners. We show that these timesteppers can derive from a finite-element-in-time (FET) (and finite-element-in-space) interpretation. The benefits of this reformulation are discussed, including the derivation of timesteppers that are higher order in time, and the quantifiable dissipative SP properties in the non-ideal, resistive case.
        
        We discuss possible options for extending this FET approach to timesteppers for the compressible case.

        The kinetic corrections satisfy linearized Boltzmann equations. Using a Lénard--Bernstein collision operator, these take Fokker--Planck-like forms \cite{Fokker_1914, Planck_1917} wherein pseudo-particles in the numerical model obey the neoclassical transport equations, with particle-independent Brownian drift terms. This offers a rigorous methodology for incorporating collisions into the particle transport model, without coupling the equations of motions for each particle.
        
        Works by Chen, Chacón et al. \cite{Chen_Chacón_Barnes_2011, Chacón_Chen_Barnes_2013, Chen_Chacón_2014, Chen_Chacón_2015} have developed structure-preserving particle pushers for neoclassical transport in the Vlasov equations, derived from Crank--Nicolson integrators. We show these too can can derive from a FET interpretation, similarly offering potential extensions to higher-order-in-time particle pushers. The FET formulation is used also to consider how the stochastic drift terms can be incorporated into the pushers. Stochastic gyrokinetic expansions are also discussed.

        Different options for the numerical implementation of these schemes are considered.

        Due to the efficacy of FET in the development of SP timesteppers for both the fluid and kinetic component, we hope this approach will prove effective in the future for developing SP timesteppers for the full hybrid model. We hope this will give us the opportunity to incorporate previously inaccessible kinetic effects into the highly effective, modern, finite-element MHD models.
    \end{abstract}
    
    
    \newpage
    \tableofcontents
    
    
    \newpage
    \pagenumbering{arabic}
    %\linenumbers\renewcommand\thelinenumber{\color{black!50}\arabic{linenumber}}
            \input{0 - introduction/main.tex}
        \part{Research}
            \input{1 - low-noise PiC models/main.tex}
            \input{2 - kinetic component/main.tex}
            \input{3 - fluid component/main.tex}
            \input{4 - numerical implementation/main.tex}
        \part{Project Overview}
            \input{5 - research plan/main.tex}
            \input{6 - summary/main.tex}
    
    
    %\section{}
    \newpage
    \pagenumbering{gobble}
        \printbibliography


    \newpage
    \pagenumbering{roman}
    \appendix
        \part{Appendices}
            \input{8 - Hilbert complexes/main.tex}
            \input{9 - weak conservation proofs/main.tex}
\end{document}


\title{\BA{Title in Progress...}}
\author{Boris Andrews}
\affil{Mathematical Institute, University of Oxford}
\date{\today}


\begin{document}
    \pagenumbering{gobble}
    \maketitle
    
    
    \begin{abstract}
        Magnetic confinement reactors---in particular tokamaks---offer one of the most promising options for achieving practical nuclear fusion, with the potential to provide virtually limitless, clean energy. The theoretical and numerical modeling of tokamak plasmas is simultaneously an essential component of effective reactor design, and a great research barrier. Tokamak operational conditions exhibit comparatively low Knudsen numbers. Kinetic effects, including kinetic waves and instabilities, Landau damping, bump-on-tail instabilities and more, are therefore highly influential in tokamak plasma dynamics. Purely fluid models are inherently incapable of capturing these effects, whereas the high dimensionality in purely kinetic models render them practically intractable for most relevant purposes.

        We consider a $\delta\!f$ decomposition model, with a macroscopic fluid background and microscopic kinetic correction, both fully coupled to each other. A similar manner of discretization is proposed to that used in the recent \texttt{STRUPHY} code \cite{Holderied_Possanner_Wang_2021, Holderied_2022, Li_et_al_2023} with a finite-element model for the background and a pseudo-particle/PiC model for the correction.

        The fluid background satisfies the full, non-linear, resistive, compressible, Hall MHD equations. \cite{Laakmann_Hu_Farrell_2022} introduces finite-element(-in-space) implicit timesteppers for the incompressible analogue to this system with structure-preserving (SP) properties in the ideal case, alongside parameter-robust preconditioners. We show that these timesteppers can derive from a finite-element-in-time (FET) (and finite-element-in-space) interpretation. The benefits of this reformulation are discussed, including the derivation of timesteppers that are higher order in time, and the quantifiable dissipative SP properties in the non-ideal, resistive case.
        
        We discuss possible options for extending this FET approach to timesteppers for the compressible case.

        The kinetic corrections satisfy linearized Boltzmann equations. Using a Lénard--Bernstein collision operator, these take Fokker--Planck-like forms \cite{Fokker_1914, Planck_1917} wherein pseudo-particles in the numerical model obey the neoclassical transport equations, with particle-independent Brownian drift terms. This offers a rigorous methodology for incorporating collisions into the particle transport model, without coupling the equations of motions for each particle.
        
        Works by Chen, Chacón et al. \cite{Chen_Chacón_Barnes_2011, Chacón_Chen_Barnes_2013, Chen_Chacón_2014, Chen_Chacón_2015} have developed structure-preserving particle pushers for neoclassical transport in the Vlasov equations, derived from Crank--Nicolson integrators. We show these too can can derive from a FET interpretation, similarly offering potential extensions to higher-order-in-time particle pushers. The FET formulation is used also to consider how the stochastic drift terms can be incorporated into the pushers. Stochastic gyrokinetic expansions are also discussed.

        Different options for the numerical implementation of these schemes are considered.

        Due to the efficacy of FET in the development of SP timesteppers for both the fluid and kinetic component, we hope this approach will prove effective in the future for developing SP timesteppers for the full hybrid model. We hope this will give us the opportunity to incorporate previously inaccessible kinetic effects into the highly effective, modern, finite-element MHD models.
    \end{abstract}
    
    
    \newpage
    \tableofcontents
    
    
    \newpage
    \pagenumbering{arabic}
    %\linenumbers\renewcommand\thelinenumber{\color{black!50}\arabic{linenumber}}
            \documentclass[12pt, a4paper]{report}

\input{template/main.tex}

\title{\BA{Title in Progress...}}
\author{Boris Andrews}
\affil{Mathematical Institute, University of Oxford}
\date{\today}


\begin{document}
    \pagenumbering{gobble}
    \maketitle
    
    
    \begin{abstract}
        Magnetic confinement reactors---in particular tokamaks---offer one of the most promising options for achieving practical nuclear fusion, with the potential to provide virtually limitless, clean energy. The theoretical and numerical modeling of tokamak plasmas is simultaneously an essential component of effective reactor design, and a great research barrier. Tokamak operational conditions exhibit comparatively low Knudsen numbers. Kinetic effects, including kinetic waves and instabilities, Landau damping, bump-on-tail instabilities and more, are therefore highly influential in tokamak plasma dynamics. Purely fluid models are inherently incapable of capturing these effects, whereas the high dimensionality in purely kinetic models render them practically intractable for most relevant purposes.

        We consider a $\delta\!f$ decomposition model, with a macroscopic fluid background and microscopic kinetic correction, both fully coupled to each other. A similar manner of discretization is proposed to that used in the recent \texttt{STRUPHY} code \cite{Holderied_Possanner_Wang_2021, Holderied_2022, Li_et_al_2023} with a finite-element model for the background and a pseudo-particle/PiC model for the correction.

        The fluid background satisfies the full, non-linear, resistive, compressible, Hall MHD equations. \cite{Laakmann_Hu_Farrell_2022} introduces finite-element(-in-space) implicit timesteppers for the incompressible analogue to this system with structure-preserving (SP) properties in the ideal case, alongside parameter-robust preconditioners. We show that these timesteppers can derive from a finite-element-in-time (FET) (and finite-element-in-space) interpretation. The benefits of this reformulation are discussed, including the derivation of timesteppers that are higher order in time, and the quantifiable dissipative SP properties in the non-ideal, resistive case.
        
        We discuss possible options for extending this FET approach to timesteppers for the compressible case.

        The kinetic corrections satisfy linearized Boltzmann equations. Using a Lénard--Bernstein collision operator, these take Fokker--Planck-like forms \cite{Fokker_1914, Planck_1917} wherein pseudo-particles in the numerical model obey the neoclassical transport equations, with particle-independent Brownian drift terms. This offers a rigorous methodology for incorporating collisions into the particle transport model, without coupling the equations of motions for each particle.
        
        Works by Chen, Chacón et al. \cite{Chen_Chacón_Barnes_2011, Chacón_Chen_Barnes_2013, Chen_Chacón_2014, Chen_Chacón_2015} have developed structure-preserving particle pushers for neoclassical transport in the Vlasov equations, derived from Crank--Nicolson integrators. We show these too can can derive from a FET interpretation, similarly offering potential extensions to higher-order-in-time particle pushers. The FET formulation is used also to consider how the stochastic drift terms can be incorporated into the pushers. Stochastic gyrokinetic expansions are also discussed.

        Different options for the numerical implementation of these schemes are considered.

        Due to the efficacy of FET in the development of SP timesteppers for both the fluid and kinetic component, we hope this approach will prove effective in the future for developing SP timesteppers for the full hybrid model. We hope this will give us the opportunity to incorporate previously inaccessible kinetic effects into the highly effective, modern, finite-element MHD models.
    \end{abstract}
    
    
    \newpage
    \tableofcontents
    
    
    \newpage
    \pagenumbering{arabic}
    %\linenumbers\renewcommand\thelinenumber{\color{black!50}\arabic{linenumber}}
            \input{0 - introduction/main.tex}
        \part{Research}
            \input{1 - low-noise PiC models/main.tex}
            \input{2 - kinetic component/main.tex}
            \input{3 - fluid component/main.tex}
            \input{4 - numerical implementation/main.tex}
        \part{Project Overview}
            \input{5 - research plan/main.tex}
            \input{6 - summary/main.tex}
    
    
    %\section{}
    \newpage
    \pagenumbering{gobble}
        \printbibliography


    \newpage
    \pagenumbering{roman}
    \appendix
        \part{Appendices}
            \input{8 - Hilbert complexes/main.tex}
            \input{9 - weak conservation proofs/main.tex}
\end{document}

        \part{Research}
            \documentclass[12pt, a4paper]{report}

\input{template/main.tex}

\title{\BA{Title in Progress...}}
\author{Boris Andrews}
\affil{Mathematical Institute, University of Oxford}
\date{\today}


\begin{document}
    \pagenumbering{gobble}
    \maketitle
    
    
    \begin{abstract}
        Magnetic confinement reactors---in particular tokamaks---offer one of the most promising options for achieving practical nuclear fusion, with the potential to provide virtually limitless, clean energy. The theoretical and numerical modeling of tokamak plasmas is simultaneously an essential component of effective reactor design, and a great research barrier. Tokamak operational conditions exhibit comparatively low Knudsen numbers. Kinetic effects, including kinetic waves and instabilities, Landau damping, bump-on-tail instabilities and more, are therefore highly influential in tokamak plasma dynamics. Purely fluid models are inherently incapable of capturing these effects, whereas the high dimensionality in purely kinetic models render them practically intractable for most relevant purposes.

        We consider a $\delta\!f$ decomposition model, with a macroscopic fluid background and microscopic kinetic correction, both fully coupled to each other. A similar manner of discretization is proposed to that used in the recent \texttt{STRUPHY} code \cite{Holderied_Possanner_Wang_2021, Holderied_2022, Li_et_al_2023} with a finite-element model for the background and a pseudo-particle/PiC model for the correction.

        The fluid background satisfies the full, non-linear, resistive, compressible, Hall MHD equations. \cite{Laakmann_Hu_Farrell_2022} introduces finite-element(-in-space) implicit timesteppers for the incompressible analogue to this system with structure-preserving (SP) properties in the ideal case, alongside parameter-robust preconditioners. We show that these timesteppers can derive from a finite-element-in-time (FET) (and finite-element-in-space) interpretation. The benefits of this reformulation are discussed, including the derivation of timesteppers that are higher order in time, and the quantifiable dissipative SP properties in the non-ideal, resistive case.
        
        We discuss possible options for extending this FET approach to timesteppers for the compressible case.

        The kinetic corrections satisfy linearized Boltzmann equations. Using a Lénard--Bernstein collision operator, these take Fokker--Planck-like forms \cite{Fokker_1914, Planck_1917} wherein pseudo-particles in the numerical model obey the neoclassical transport equations, with particle-independent Brownian drift terms. This offers a rigorous methodology for incorporating collisions into the particle transport model, without coupling the equations of motions for each particle.
        
        Works by Chen, Chacón et al. \cite{Chen_Chacón_Barnes_2011, Chacón_Chen_Barnes_2013, Chen_Chacón_2014, Chen_Chacón_2015} have developed structure-preserving particle pushers for neoclassical transport in the Vlasov equations, derived from Crank--Nicolson integrators. We show these too can can derive from a FET interpretation, similarly offering potential extensions to higher-order-in-time particle pushers. The FET formulation is used also to consider how the stochastic drift terms can be incorporated into the pushers. Stochastic gyrokinetic expansions are also discussed.

        Different options for the numerical implementation of these schemes are considered.

        Due to the efficacy of FET in the development of SP timesteppers for both the fluid and kinetic component, we hope this approach will prove effective in the future for developing SP timesteppers for the full hybrid model. We hope this will give us the opportunity to incorporate previously inaccessible kinetic effects into the highly effective, modern, finite-element MHD models.
    \end{abstract}
    
    
    \newpage
    \tableofcontents
    
    
    \newpage
    \pagenumbering{arabic}
    %\linenumbers\renewcommand\thelinenumber{\color{black!50}\arabic{linenumber}}
            \input{0 - introduction/main.tex}
        \part{Research}
            \input{1 - low-noise PiC models/main.tex}
            \input{2 - kinetic component/main.tex}
            \input{3 - fluid component/main.tex}
            \input{4 - numerical implementation/main.tex}
        \part{Project Overview}
            \input{5 - research plan/main.tex}
            \input{6 - summary/main.tex}
    
    
    %\section{}
    \newpage
    \pagenumbering{gobble}
        \printbibliography


    \newpage
    \pagenumbering{roman}
    \appendix
        \part{Appendices}
            \input{8 - Hilbert complexes/main.tex}
            \input{9 - weak conservation proofs/main.tex}
\end{document}

            \documentclass[12pt, a4paper]{report}

\input{template/main.tex}

\title{\BA{Title in Progress...}}
\author{Boris Andrews}
\affil{Mathematical Institute, University of Oxford}
\date{\today}


\begin{document}
    \pagenumbering{gobble}
    \maketitle
    
    
    \begin{abstract}
        Magnetic confinement reactors---in particular tokamaks---offer one of the most promising options for achieving practical nuclear fusion, with the potential to provide virtually limitless, clean energy. The theoretical and numerical modeling of tokamak plasmas is simultaneously an essential component of effective reactor design, and a great research barrier. Tokamak operational conditions exhibit comparatively low Knudsen numbers. Kinetic effects, including kinetic waves and instabilities, Landau damping, bump-on-tail instabilities and more, are therefore highly influential in tokamak plasma dynamics. Purely fluid models are inherently incapable of capturing these effects, whereas the high dimensionality in purely kinetic models render them practically intractable for most relevant purposes.

        We consider a $\delta\!f$ decomposition model, with a macroscopic fluid background and microscopic kinetic correction, both fully coupled to each other. A similar manner of discretization is proposed to that used in the recent \texttt{STRUPHY} code \cite{Holderied_Possanner_Wang_2021, Holderied_2022, Li_et_al_2023} with a finite-element model for the background and a pseudo-particle/PiC model for the correction.

        The fluid background satisfies the full, non-linear, resistive, compressible, Hall MHD equations. \cite{Laakmann_Hu_Farrell_2022} introduces finite-element(-in-space) implicit timesteppers for the incompressible analogue to this system with structure-preserving (SP) properties in the ideal case, alongside parameter-robust preconditioners. We show that these timesteppers can derive from a finite-element-in-time (FET) (and finite-element-in-space) interpretation. The benefits of this reformulation are discussed, including the derivation of timesteppers that are higher order in time, and the quantifiable dissipative SP properties in the non-ideal, resistive case.
        
        We discuss possible options for extending this FET approach to timesteppers for the compressible case.

        The kinetic corrections satisfy linearized Boltzmann equations. Using a Lénard--Bernstein collision operator, these take Fokker--Planck-like forms \cite{Fokker_1914, Planck_1917} wherein pseudo-particles in the numerical model obey the neoclassical transport equations, with particle-independent Brownian drift terms. This offers a rigorous methodology for incorporating collisions into the particle transport model, without coupling the equations of motions for each particle.
        
        Works by Chen, Chacón et al. \cite{Chen_Chacón_Barnes_2011, Chacón_Chen_Barnes_2013, Chen_Chacón_2014, Chen_Chacón_2015} have developed structure-preserving particle pushers for neoclassical transport in the Vlasov equations, derived from Crank--Nicolson integrators. We show these too can can derive from a FET interpretation, similarly offering potential extensions to higher-order-in-time particle pushers. The FET formulation is used also to consider how the stochastic drift terms can be incorporated into the pushers. Stochastic gyrokinetic expansions are also discussed.

        Different options for the numerical implementation of these schemes are considered.

        Due to the efficacy of FET in the development of SP timesteppers for both the fluid and kinetic component, we hope this approach will prove effective in the future for developing SP timesteppers for the full hybrid model. We hope this will give us the opportunity to incorporate previously inaccessible kinetic effects into the highly effective, modern, finite-element MHD models.
    \end{abstract}
    
    
    \newpage
    \tableofcontents
    
    
    \newpage
    \pagenumbering{arabic}
    %\linenumbers\renewcommand\thelinenumber{\color{black!50}\arabic{linenumber}}
            \input{0 - introduction/main.tex}
        \part{Research}
            \input{1 - low-noise PiC models/main.tex}
            \input{2 - kinetic component/main.tex}
            \input{3 - fluid component/main.tex}
            \input{4 - numerical implementation/main.tex}
        \part{Project Overview}
            \input{5 - research plan/main.tex}
            \input{6 - summary/main.tex}
    
    
    %\section{}
    \newpage
    \pagenumbering{gobble}
        \printbibliography


    \newpage
    \pagenumbering{roman}
    \appendix
        \part{Appendices}
            \input{8 - Hilbert complexes/main.tex}
            \input{9 - weak conservation proofs/main.tex}
\end{document}

            \documentclass[12pt, a4paper]{report}

\input{template/main.tex}

\title{\BA{Title in Progress...}}
\author{Boris Andrews}
\affil{Mathematical Institute, University of Oxford}
\date{\today}


\begin{document}
    \pagenumbering{gobble}
    \maketitle
    
    
    \begin{abstract}
        Magnetic confinement reactors---in particular tokamaks---offer one of the most promising options for achieving practical nuclear fusion, with the potential to provide virtually limitless, clean energy. The theoretical and numerical modeling of tokamak plasmas is simultaneously an essential component of effective reactor design, and a great research barrier. Tokamak operational conditions exhibit comparatively low Knudsen numbers. Kinetic effects, including kinetic waves and instabilities, Landau damping, bump-on-tail instabilities and more, are therefore highly influential in tokamak plasma dynamics. Purely fluid models are inherently incapable of capturing these effects, whereas the high dimensionality in purely kinetic models render them practically intractable for most relevant purposes.

        We consider a $\delta\!f$ decomposition model, with a macroscopic fluid background and microscopic kinetic correction, both fully coupled to each other. A similar manner of discretization is proposed to that used in the recent \texttt{STRUPHY} code \cite{Holderied_Possanner_Wang_2021, Holderied_2022, Li_et_al_2023} with a finite-element model for the background and a pseudo-particle/PiC model for the correction.

        The fluid background satisfies the full, non-linear, resistive, compressible, Hall MHD equations. \cite{Laakmann_Hu_Farrell_2022} introduces finite-element(-in-space) implicit timesteppers for the incompressible analogue to this system with structure-preserving (SP) properties in the ideal case, alongside parameter-robust preconditioners. We show that these timesteppers can derive from a finite-element-in-time (FET) (and finite-element-in-space) interpretation. The benefits of this reformulation are discussed, including the derivation of timesteppers that are higher order in time, and the quantifiable dissipative SP properties in the non-ideal, resistive case.
        
        We discuss possible options for extending this FET approach to timesteppers for the compressible case.

        The kinetic corrections satisfy linearized Boltzmann equations. Using a Lénard--Bernstein collision operator, these take Fokker--Planck-like forms \cite{Fokker_1914, Planck_1917} wherein pseudo-particles in the numerical model obey the neoclassical transport equations, with particle-independent Brownian drift terms. This offers a rigorous methodology for incorporating collisions into the particle transport model, without coupling the equations of motions for each particle.
        
        Works by Chen, Chacón et al. \cite{Chen_Chacón_Barnes_2011, Chacón_Chen_Barnes_2013, Chen_Chacón_2014, Chen_Chacón_2015} have developed structure-preserving particle pushers for neoclassical transport in the Vlasov equations, derived from Crank--Nicolson integrators. We show these too can can derive from a FET interpretation, similarly offering potential extensions to higher-order-in-time particle pushers. The FET formulation is used also to consider how the stochastic drift terms can be incorporated into the pushers. Stochastic gyrokinetic expansions are also discussed.

        Different options for the numerical implementation of these schemes are considered.

        Due to the efficacy of FET in the development of SP timesteppers for both the fluid and kinetic component, we hope this approach will prove effective in the future for developing SP timesteppers for the full hybrid model. We hope this will give us the opportunity to incorporate previously inaccessible kinetic effects into the highly effective, modern, finite-element MHD models.
    \end{abstract}
    
    
    \newpage
    \tableofcontents
    
    
    \newpage
    \pagenumbering{arabic}
    %\linenumbers\renewcommand\thelinenumber{\color{black!50}\arabic{linenumber}}
            \input{0 - introduction/main.tex}
        \part{Research}
            \input{1 - low-noise PiC models/main.tex}
            \input{2 - kinetic component/main.tex}
            \input{3 - fluid component/main.tex}
            \input{4 - numerical implementation/main.tex}
        \part{Project Overview}
            \input{5 - research plan/main.tex}
            \input{6 - summary/main.tex}
    
    
    %\section{}
    \newpage
    \pagenumbering{gobble}
        \printbibliography


    \newpage
    \pagenumbering{roman}
    \appendix
        \part{Appendices}
            \input{8 - Hilbert complexes/main.tex}
            \input{9 - weak conservation proofs/main.tex}
\end{document}

            \documentclass[12pt, a4paper]{report}

\input{template/main.tex}

\title{\BA{Title in Progress...}}
\author{Boris Andrews}
\affil{Mathematical Institute, University of Oxford}
\date{\today}


\begin{document}
    \pagenumbering{gobble}
    \maketitle
    
    
    \begin{abstract}
        Magnetic confinement reactors---in particular tokamaks---offer one of the most promising options for achieving practical nuclear fusion, with the potential to provide virtually limitless, clean energy. The theoretical and numerical modeling of tokamak plasmas is simultaneously an essential component of effective reactor design, and a great research barrier. Tokamak operational conditions exhibit comparatively low Knudsen numbers. Kinetic effects, including kinetic waves and instabilities, Landau damping, bump-on-tail instabilities and more, are therefore highly influential in tokamak plasma dynamics. Purely fluid models are inherently incapable of capturing these effects, whereas the high dimensionality in purely kinetic models render them practically intractable for most relevant purposes.

        We consider a $\delta\!f$ decomposition model, with a macroscopic fluid background and microscopic kinetic correction, both fully coupled to each other. A similar manner of discretization is proposed to that used in the recent \texttt{STRUPHY} code \cite{Holderied_Possanner_Wang_2021, Holderied_2022, Li_et_al_2023} with a finite-element model for the background and a pseudo-particle/PiC model for the correction.

        The fluid background satisfies the full, non-linear, resistive, compressible, Hall MHD equations. \cite{Laakmann_Hu_Farrell_2022} introduces finite-element(-in-space) implicit timesteppers for the incompressible analogue to this system with structure-preserving (SP) properties in the ideal case, alongside parameter-robust preconditioners. We show that these timesteppers can derive from a finite-element-in-time (FET) (and finite-element-in-space) interpretation. The benefits of this reformulation are discussed, including the derivation of timesteppers that are higher order in time, and the quantifiable dissipative SP properties in the non-ideal, resistive case.
        
        We discuss possible options for extending this FET approach to timesteppers for the compressible case.

        The kinetic corrections satisfy linearized Boltzmann equations. Using a Lénard--Bernstein collision operator, these take Fokker--Planck-like forms \cite{Fokker_1914, Planck_1917} wherein pseudo-particles in the numerical model obey the neoclassical transport equations, with particle-independent Brownian drift terms. This offers a rigorous methodology for incorporating collisions into the particle transport model, without coupling the equations of motions for each particle.
        
        Works by Chen, Chacón et al. \cite{Chen_Chacón_Barnes_2011, Chacón_Chen_Barnes_2013, Chen_Chacón_2014, Chen_Chacón_2015} have developed structure-preserving particle pushers for neoclassical transport in the Vlasov equations, derived from Crank--Nicolson integrators. We show these too can can derive from a FET interpretation, similarly offering potential extensions to higher-order-in-time particle pushers. The FET formulation is used also to consider how the stochastic drift terms can be incorporated into the pushers. Stochastic gyrokinetic expansions are also discussed.

        Different options for the numerical implementation of these schemes are considered.

        Due to the efficacy of FET in the development of SP timesteppers for both the fluid and kinetic component, we hope this approach will prove effective in the future for developing SP timesteppers for the full hybrid model. We hope this will give us the opportunity to incorporate previously inaccessible kinetic effects into the highly effective, modern, finite-element MHD models.
    \end{abstract}
    
    
    \newpage
    \tableofcontents
    
    
    \newpage
    \pagenumbering{arabic}
    %\linenumbers\renewcommand\thelinenumber{\color{black!50}\arabic{linenumber}}
            \input{0 - introduction/main.tex}
        \part{Research}
            \input{1 - low-noise PiC models/main.tex}
            \input{2 - kinetic component/main.tex}
            \input{3 - fluid component/main.tex}
            \input{4 - numerical implementation/main.tex}
        \part{Project Overview}
            \input{5 - research plan/main.tex}
            \input{6 - summary/main.tex}
    
    
    %\section{}
    \newpage
    \pagenumbering{gobble}
        \printbibliography


    \newpage
    \pagenumbering{roman}
    \appendix
        \part{Appendices}
            \input{8 - Hilbert complexes/main.tex}
            \input{9 - weak conservation proofs/main.tex}
\end{document}

        \part{Project Overview}
            \documentclass[12pt, a4paper]{report}

\input{template/main.tex}

\title{\BA{Title in Progress...}}
\author{Boris Andrews}
\affil{Mathematical Institute, University of Oxford}
\date{\today}


\begin{document}
    \pagenumbering{gobble}
    \maketitle
    
    
    \begin{abstract}
        Magnetic confinement reactors---in particular tokamaks---offer one of the most promising options for achieving practical nuclear fusion, with the potential to provide virtually limitless, clean energy. The theoretical and numerical modeling of tokamak plasmas is simultaneously an essential component of effective reactor design, and a great research barrier. Tokamak operational conditions exhibit comparatively low Knudsen numbers. Kinetic effects, including kinetic waves and instabilities, Landau damping, bump-on-tail instabilities and more, are therefore highly influential in tokamak plasma dynamics. Purely fluid models are inherently incapable of capturing these effects, whereas the high dimensionality in purely kinetic models render them practically intractable for most relevant purposes.

        We consider a $\delta\!f$ decomposition model, with a macroscopic fluid background and microscopic kinetic correction, both fully coupled to each other. A similar manner of discretization is proposed to that used in the recent \texttt{STRUPHY} code \cite{Holderied_Possanner_Wang_2021, Holderied_2022, Li_et_al_2023} with a finite-element model for the background and a pseudo-particle/PiC model for the correction.

        The fluid background satisfies the full, non-linear, resistive, compressible, Hall MHD equations. \cite{Laakmann_Hu_Farrell_2022} introduces finite-element(-in-space) implicit timesteppers for the incompressible analogue to this system with structure-preserving (SP) properties in the ideal case, alongside parameter-robust preconditioners. We show that these timesteppers can derive from a finite-element-in-time (FET) (and finite-element-in-space) interpretation. The benefits of this reformulation are discussed, including the derivation of timesteppers that are higher order in time, and the quantifiable dissipative SP properties in the non-ideal, resistive case.
        
        We discuss possible options for extending this FET approach to timesteppers for the compressible case.

        The kinetic corrections satisfy linearized Boltzmann equations. Using a Lénard--Bernstein collision operator, these take Fokker--Planck-like forms \cite{Fokker_1914, Planck_1917} wherein pseudo-particles in the numerical model obey the neoclassical transport equations, with particle-independent Brownian drift terms. This offers a rigorous methodology for incorporating collisions into the particle transport model, without coupling the equations of motions for each particle.
        
        Works by Chen, Chacón et al. \cite{Chen_Chacón_Barnes_2011, Chacón_Chen_Barnes_2013, Chen_Chacón_2014, Chen_Chacón_2015} have developed structure-preserving particle pushers for neoclassical transport in the Vlasov equations, derived from Crank--Nicolson integrators. We show these too can can derive from a FET interpretation, similarly offering potential extensions to higher-order-in-time particle pushers. The FET formulation is used also to consider how the stochastic drift terms can be incorporated into the pushers. Stochastic gyrokinetic expansions are also discussed.

        Different options for the numerical implementation of these schemes are considered.

        Due to the efficacy of FET in the development of SP timesteppers for both the fluid and kinetic component, we hope this approach will prove effective in the future for developing SP timesteppers for the full hybrid model. We hope this will give us the opportunity to incorporate previously inaccessible kinetic effects into the highly effective, modern, finite-element MHD models.
    \end{abstract}
    
    
    \newpage
    \tableofcontents
    
    
    \newpage
    \pagenumbering{arabic}
    %\linenumbers\renewcommand\thelinenumber{\color{black!50}\arabic{linenumber}}
            \input{0 - introduction/main.tex}
        \part{Research}
            \input{1 - low-noise PiC models/main.tex}
            \input{2 - kinetic component/main.tex}
            \input{3 - fluid component/main.tex}
            \input{4 - numerical implementation/main.tex}
        \part{Project Overview}
            \input{5 - research plan/main.tex}
            \input{6 - summary/main.tex}
    
    
    %\section{}
    \newpage
    \pagenumbering{gobble}
        \printbibliography


    \newpage
    \pagenumbering{roman}
    \appendix
        \part{Appendices}
            \input{8 - Hilbert complexes/main.tex}
            \input{9 - weak conservation proofs/main.tex}
\end{document}

            \documentclass[12pt, a4paper]{report}

\input{template/main.tex}

\title{\BA{Title in Progress...}}
\author{Boris Andrews}
\affil{Mathematical Institute, University of Oxford}
\date{\today}


\begin{document}
    \pagenumbering{gobble}
    \maketitle
    
    
    \begin{abstract}
        Magnetic confinement reactors---in particular tokamaks---offer one of the most promising options for achieving practical nuclear fusion, with the potential to provide virtually limitless, clean energy. The theoretical and numerical modeling of tokamak plasmas is simultaneously an essential component of effective reactor design, and a great research barrier. Tokamak operational conditions exhibit comparatively low Knudsen numbers. Kinetic effects, including kinetic waves and instabilities, Landau damping, bump-on-tail instabilities and more, are therefore highly influential in tokamak plasma dynamics. Purely fluid models are inherently incapable of capturing these effects, whereas the high dimensionality in purely kinetic models render them practically intractable for most relevant purposes.

        We consider a $\delta\!f$ decomposition model, with a macroscopic fluid background and microscopic kinetic correction, both fully coupled to each other. A similar manner of discretization is proposed to that used in the recent \texttt{STRUPHY} code \cite{Holderied_Possanner_Wang_2021, Holderied_2022, Li_et_al_2023} with a finite-element model for the background and a pseudo-particle/PiC model for the correction.

        The fluid background satisfies the full, non-linear, resistive, compressible, Hall MHD equations. \cite{Laakmann_Hu_Farrell_2022} introduces finite-element(-in-space) implicit timesteppers for the incompressible analogue to this system with structure-preserving (SP) properties in the ideal case, alongside parameter-robust preconditioners. We show that these timesteppers can derive from a finite-element-in-time (FET) (and finite-element-in-space) interpretation. The benefits of this reformulation are discussed, including the derivation of timesteppers that are higher order in time, and the quantifiable dissipative SP properties in the non-ideal, resistive case.
        
        We discuss possible options for extending this FET approach to timesteppers for the compressible case.

        The kinetic corrections satisfy linearized Boltzmann equations. Using a Lénard--Bernstein collision operator, these take Fokker--Planck-like forms \cite{Fokker_1914, Planck_1917} wherein pseudo-particles in the numerical model obey the neoclassical transport equations, with particle-independent Brownian drift terms. This offers a rigorous methodology for incorporating collisions into the particle transport model, without coupling the equations of motions for each particle.
        
        Works by Chen, Chacón et al. \cite{Chen_Chacón_Barnes_2011, Chacón_Chen_Barnes_2013, Chen_Chacón_2014, Chen_Chacón_2015} have developed structure-preserving particle pushers for neoclassical transport in the Vlasov equations, derived from Crank--Nicolson integrators. We show these too can can derive from a FET interpretation, similarly offering potential extensions to higher-order-in-time particle pushers. The FET formulation is used also to consider how the stochastic drift terms can be incorporated into the pushers. Stochastic gyrokinetic expansions are also discussed.

        Different options for the numerical implementation of these schemes are considered.

        Due to the efficacy of FET in the development of SP timesteppers for both the fluid and kinetic component, we hope this approach will prove effective in the future for developing SP timesteppers for the full hybrid model. We hope this will give us the opportunity to incorporate previously inaccessible kinetic effects into the highly effective, modern, finite-element MHD models.
    \end{abstract}
    
    
    \newpage
    \tableofcontents
    
    
    \newpage
    \pagenumbering{arabic}
    %\linenumbers\renewcommand\thelinenumber{\color{black!50}\arabic{linenumber}}
            \input{0 - introduction/main.tex}
        \part{Research}
            \input{1 - low-noise PiC models/main.tex}
            \input{2 - kinetic component/main.tex}
            \input{3 - fluid component/main.tex}
            \input{4 - numerical implementation/main.tex}
        \part{Project Overview}
            \input{5 - research plan/main.tex}
            \input{6 - summary/main.tex}
    
    
    %\section{}
    \newpage
    \pagenumbering{gobble}
        \printbibliography


    \newpage
    \pagenumbering{roman}
    \appendix
        \part{Appendices}
            \input{8 - Hilbert complexes/main.tex}
            \input{9 - weak conservation proofs/main.tex}
\end{document}

    
    
    %\section{}
    \newpage
    \pagenumbering{gobble}
        \printbibliography


    \newpage
    \pagenumbering{roman}
    \appendix
        \part{Appendices}
            \documentclass[12pt, a4paper]{report}

\input{template/main.tex}

\title{\BA{Title in Progress...}}
\author{Boris Andrews}
\affil{Mathematical Institute, University of Oxford}
\date{\today}


\begin{document}
    \pagenumbering{gobble}
    \maketitle
    
    
    \begin{abstract}
        Magnetic confinement reactors---in particular tokamaks---offer one of the most promising options for achieving practical nuclear fusion, with the potential to provide virtually limitless, clean energy. The theoretical and numerical modeling of tokamak plasmas is simultaneously an essential component of effective reactor design, and a great research barrier. Tokamak operational conditions exhibit comparatively low Knudsen numbers. Kinetic effects, including kinetic waves and instabilities, Landau damping, bump-on-tail instabilities and more, are therefore highly influential in tokamak plasma dynamics. Purely fluid models are inherently incapable of capturing these effects, whereas the high dimensionality in purely kinetic models render them practically intractable for most relevant purposes.

        We consider a $\delta\!f$ decomposition model, with a macroscopic fluid background and microscopic kinetic correction, both fully coupled to each other. A similar manner of discretization is proposed to that used in the recent \texttt{STRUPHY} code \cite{Holderied_Possanner_Wang_2021, Holderied_2022, Li_et_al_2023} with a finite-element model for the background and a pseudo-particle/PiC model for the correction.

        The fluid background satisfies the full, non-linear, resistive, compressible, Hall MHD equations. \cite{Laakmann_Hu_Farrell_2022} introduces finite-element(-in-space) implicit timesteppers for the incompressible analogue to this system with structure-preserving (SP) properties in the ideal case, alongside parameter-robust preconditioners. We show that these timesteppers can derive from a finite-element-in-time (FET) (and finite-element-in-space) interpretation. The benefits of this reformulation are discussed, including the derivation of timesteppers that are higher order in time, and the quantifiable dissipative SP properties in the non-ideal, resistive case.
        
        We discuss possible options for extending this FET approach to timesteppers for the compressible case.

        The kinetic corrections satisfy linearized Boltzmann equations. Using a Lénard--Bernstein collision operator, these take Fokker--Planck-like forms \cite{Fokker_1914, Planck_1917} wherein pseudo-particles in the numerical model obey the neoclassical transport equations, with particle-independent Brownian drift terms. This offers a rigorous methodology for incorporating collisions into the particle transport model, without coupling the equations of motions for each particle.
        
        Works by Chen, Chacón et al. \cite{Chen_Chacón_Barnes_2011, Chacón_Chen_Barnes_2013, Chen_Chacón_2014, Chen_Chacón_2015} have developed structure-preserving particle pushers for neoclassical transport in the Vlasov equations, derived from Crank--Nicolson integrators. We show these too can can derive from a FET interpretation, similarly offering potential extensions to higher-order-in-time particle pushers. The FET formulation is used also to consider how the stochastic drift terms can be incorporated into the pushers. Stochastic gyrokinetic expansions are also discussed.

        Different options for the numerical implementation of these schemes are considered.

        Due to the efficacy of FET in the development of SP timesteppers for both the fluid and kinetic component, we hope this approach will prove effective in the future for developing SP timesteppers for the full hybrid model. We hope this will give us the opportunity to incorporate previously inaccessible kinetic effects into the highly effective, modern, finite-element MHD models.
    \end{abstract}
    
    
    \newpage
    \tableofcontents
    
    
    \newpage
    \pagenumbering{arabic}
    %\linenumbers\renewcommand\thelinenumber{\color{black!50}\arabic{linenumber}}
            \input{0 - introduction/main.tex}
        \part{Research}
            \input{1 - low-noise PiC models/main.tex}
            \input{2 - kinetic component/main.tex}
            \input{3 - fluid component/main.tex}
            \input{4 - numerical implementation/main.tex}
        \part{Project Overview}
            \input{5 - research plan/main.tex}
            \input{6 - summary/main.tex}
    
    
    %\section{}
    \newpage
    \pagenumbering{gobble}
        \printbibliography


    \newpage
    \pagenumbering{roman}
    \appendix
        \part{Appendices}
            \input{8 - Hilbert complexes/main.tex}
            \input{9 - weak conservation proofs/main.tex}
\end{document}

            \documentclass[12pt, a4paper]{report}

\input{template/main.tex}

\title{\BA{Title in Progress...}}
\author{Boris Andrews}
\affil{Mathematical Institute, University of Oxford}
\date{\today}


\begin{document}
    \pagenumbering{gobble}
    \maketitle
    
    
    \begin{abstract}
        Magnetic confinement reactors---in particular tokamaks---offer one of the most promising options for achieving practical nuclear fusion, with the potential to provide virtually limitless, clean energy. The theoretical and numerical modeling of tokamak plasmas is simultaneously an essential component of effective reactor design, and a great research barrier. Tokamak operational conditions exhibit comparatively low Knudsen numbers. Kinetic effects, including kinetic waves and instabilities, Landau damping, bump-on-tail instabilities and more, are therefore highly influential in tokamak plasma dynamics. Purely fluid models are inherently incapable of capturing these effects, whereas the high dimensionality in purely kinetic models render them practically intractable for most relevant purposes.

        We consider a $\delta\!f$ decomposition model, with a macroscopic fluid background and microscopic kinetic correction, both fully coupled to each other. A similar manner of discretization is proposed to that used in the recent \texttt{STRUPHY} code \cite{Holderied_Possanner_Wang_2021, Holderied_2022, Li_et_al_2023} with a finite-element model for the background and a pseudo-particle/PiC model for the correction.

        The fluid background satisfies the full, non-linear, resistive, compressible, Hall MHD equations. \cite{Laakmann_Hu_Farrell_2022} introduces finite-element(-in-space) implicit timesteppers for the incompressible analogue to this system with structure-preserving (SP) properties in the ideal case, alongside parameter-robust preconditioners. We show that these timesteppers can derive from a finite-element-in-time (FET) (and finite-element-in-space) interpretation. The benefits of this reformulation are discussed, including the derivation of timesteppers that are higher order in time, and the quantifiable dissipative SP properties in the non-ideal, resistive case.
        
        We discuss possible options for extending this FET approach to timesteppers for the compressible case.

        The kinetic corrections satisfy linearized Boltzmann equations. Using a Lénard--Bernstein collision operator, these take Fokker--Planck-like forms \cite{Fokker_1914, Planck_1917} wherein pseudo-particles in the numerical model obey the neoclassical transport equations, with particle-independent Brownian drift terms. This offers a rigorous methodology for incorporating collisions into the particle transport model, without coupling the equations of motions for each particle.
        
        Works by Chen, Chacón et al. \cite{Chen_Chacón_Barnes_2011, Chacón_Chen_Barnes_2013, Chen_Chacón_2014, Chen_Chacón_2015} have developed structure-preserving particle pushers for neoclassical transport in the Vlasov equations, derived from Crank--Nicolson integrators. We show these too can can derive from a FET interpretation, similarly offering potential extensions to higher-order-in-time particle pushers. The FET formulation is used also to consider how the stochastic drift terms can be incorporated into the pushers. Stochastic gyrokinetic expansions are also discussed.

        Different options for the numerical implementation of these schemes are considered.

        Due to the efficacy of FET in the development of SP timesteppers for both the fluid and kinetic component, we hope this approach will prove effective in the future for developing SP timesteppers for the full hybrid model. We hope this will give us the opportunity to incorporate previously inaccessible kinetic effects into the highly effective, modern, finite-element MHD models.
    \end{abstract}
    
    
    \newpage
    \tableofcontents
    
    
    \newpage
    \pagenumbering{arabic}
    %\linenumbers\renewcommand\thelinenumber{\color{black!50}\arabic{linenumber}}
            \input{0 - introduction/main.tex}
        \part{Research}
            \input{1 - low-noise PiC models/main.tex}
            \input{2 - kinetic component/main.tex}
            \input{3 - fluid component/main.tex}
            \input{4 - numerical implementation/main.tex}
        \part{Project Overview}
            \input{5 - research plan/main.tex}
            \input{6 - summary/main.tex}
    
    
    %\section{}
    \newpage
    \pagenumbering{gobble}
        \printbibliography


    \newpage
    \pagenumbering{roman}
    \appendix
        \part{Appendices}
            \input{8 - Hilbert complexes/main.tex}
            \input{9 - weak conservation proofs/main.tex}
\end{document}

\end{document}

            \documentclass[12pt, a4paper]{report}

\documentclass[12pt, a4paper]{report}

\input{template/main.tex}

\title{\BA{Title in Progress...}}
\author{Boris Andrews}
\affil{Mathematical Institute, University of Oxford}
\date{\today}


\begin{document}
    \pagenumbering{gobble}
    \maketitle
    
    
    \begin{abstract}
        Magnetic confinement reactors---in particular tokamaks---offer one of the most promising options for achieving practical nuclear fusion, with the potential to provide virtually limitless, clean energy. The theoretical and numerical modeling of tokamak plasmas is simultaneously an essential component of effective reactor design, and a great research barrier. Tokamak operational conditions exhibit comparatively low Knudsen numbers. Kinetic effects, including kinetic waves and instabilities, Landau damping, bump-on-tail instabilities and more, are therefore highly influential in tokamak plasma dynamics. Purely fluid models are inherently incapable of capturing these effects, whereas the high dimensionality in purely kinetic models render them practically intractable for most relevant purposes.

        We consider a $\delta\!f$ decomposition model, with a macroscopic fluid background and microscopic kinetic correction, both fully coupled to each other. A similar manner of discretization is proposed to that used in the recent \texttt{STRUPHY} code \cite{Holderied_Possanner_Wang_2021, Holderied_2022, Li_et_al_2023} with a finite-element model for the background and a pseudo-particle/PiC model for the correction.

        The fluid background satisfies the full, non-linear, resistive, compressible, Hall MHD equations. \cite{Laakmann_Hu_Farrell_2022} introduces finite-element(-in-space) implicit timesteppers for the incompressible analogue to this system with structure-preserving (SP) properties in the ideal case, alongside parameter-robust preconditioners. We show that these timesteppers can derive from a finite-element-in-time (FET) (and finite-element-in-space) interpretation. The benefits of this reformulation are discussed, including the derivation of timesteppers that are higher order in time, and the quantifiable dissipative SP properties in the non-ideal, resistive case.
        
        We discuss possible options for extending this FET approach to timesteppers for the compressible case.

        The kinetic corrections satisfy linearized Boltzmann equations. Using a Lénard--Bernstein collision operator, these take Fokker--Planck-like forms \cite{Fokker_1914, Planck_1917} wherein pseudo-particles in the numerical model obey the neoclassical transport equations, with particle-independent Brownian drift terms. This offers a rigorous methodology for incorporating collisions into the particle transport model, without coupling the equations of motions for each particle.
        
        Works by Chen, Chacón et al. \cite{Chen_Chacón_Barnes_2011, Chacón_Chen_Barnes_2013, Chen_Chacón_2014, Chen_Chacón_2015} have developed structure-preserving particle pushers for neoclassical transport in the Vlasov equations, derived from Crank--Nicolson integrators. We show these too can can derive from a FET interpretation, similarly offering potential extensions to higher-order-in-time particle pushers. The FET formulation is used also to consider how the stochastic drift terms can be incorporated into the pushers. Stochastic gyrokinetic expansions are also discussed.

        Different options for the numerical implementation of these schemes are considered.

        Due to the efficacy of FET in the development of SP timesteppers for both the fluid and kinetic component, we hope this approach will prove effective in the future for developing SP timesteppers for the full hybrid model. We hope this will give us the opportunity to incorporate previously inaccessible kinetic effects into the highly effective, modern, finite-element MHD models.
    \end{abstract}
    
    
    \newpage
    \tableofcontents
    
    
    \newpage
    \pagenumbering{arabic}
    %\linenumbers\renewcommand\thelinenumber{\color{black!50}\arabic{linenumber}}
            \input{0 - introduction/main.tex}
        \part{Research}
            \input{1 - low-noise PiC models/main.tex}
            \input{2 - kinetic component/main.tex}
            \input{3 - fluid component/main.tex}
            \input{4 - numerical implementation/main.tex}
        \part{Project Overview}
            \input{5 - research plan/main.tex}
            \input{6 - summary/main.tex}
    
    
    %\section{}
    \newpage
    \pagenumbering{gobble}
        \printbibliography


    \newpage
    \pagenumbering{roman}
    \appendix
        \part{Appendices}
            \input{8 - Hilbert complexes/main.tex}
            \input{9 - weak conservation proofs/main.tex}
\end{document}


\title{\BA{Title in Progress...}}
\author{Boris Andrews}
\affil{Mathematical Institute, University of Oxford}
\date{\today}


\begin{document}
    \pagenumbering{gobble}
    \maketitle
    
    
    \begin{abstract}
        Magnetic confinement reactors---in particular tokamaks---offer one of the most promising options for achieving practical nuclear fusion, with the potential to provide virtually limitless, clean energy. The theoretical and numerical modeling of tokamak plasmas is simultaneously an essential component of effective reactor design, and a great research barrier. Tokamak operational conditions exhibit comparatively low Knudsen numbers. Kinetic effects, including kinetic waves and instabilities, Landau damping, bump-on-tail instabilities and more, are therefore highly influential in tokamak plasma dynamics. Purely fluid models are inherently incapable of capturing these effects, whereas the high dimensionality in purely kinetic models render them practically intractable for most relevant purposes.

        We consider a $\delta\!f$ decomposition model, with a macroscopic fluid background and microscopic kinetic correction, both fully coupled to each other. A similar manner of discretization is proposed to that used in the recent \texttt{STRUPHY} code \cite{Holderied_Possanner_Wang_2021, Holderied_2022, Li_et_al_2023} with a finite-element model for the background and a pseudo-particle/PiC model for the correction.

        The fluid background satisfies the full, non-linear, resistive, compressible, Hall MHD equations. \cite{Laakmann_Hu_Farrell_2022} introduces finite-element(-in-space) implicit timesteppers for the incompressible analogue to this system with structure-preserving (SP) properties in the ideal case, alongside parameter-robust preconditioners. We show that these timesteppers can derive from a finite-element-in-time (FET) (and finite-element-in-space) interpretation. The benefits of this reformulation are discussed, including the derivation of timesteppers that are higher order in time, and the quantifiable dissipative SP properties in the non-ideal, resistive case.
        
        We discuss possible options for extending this FET approach to timesteppers for the compressible case.

        The kinetic corrections satisfy linearized Boltzmann equations. Using a Lénard--Bernstein collision operator, these take Fokker--Planck-like forms \cite{Fokker_1914, Planck_1917} wherein pseudo-particles in the numerical model obey the neoclassical transport equations, with particle-independent Brownian drift terms. This offers a rigorous methodology for incorporating collisions into the particle transport model, without coupling the equations of motions for each particle.
        
        Works by Chen, Chacón et al. \cite{Chen_Chacón_Barnes_2011, Chacón_Chen_Barnes_2013, Chen_Chacón_2014, Chen_Chacón_2015} have developed structure-preserving particle pushers for neoclassical transport in the Vlasov equations, derived from Crank--Nicolson integrators. We show these too can can derive from a FET interpretation, similarly offering potential extensions to higher-order-in-time particle pushers. The FET formulation is used also to consider how the stochastic drift terms can be incorporated into the pushers. Stochastic gyrokinetic expansions are also discussed.

        Different options for the numerical implementation of these schemes are considered.

        Due to the efficacy of FET in the development of SP timesteppers for both the fluid and kinetic component, we hope this approach will prove effective in the future for developing SP timesteppers for the full hybrid model. We hope this will give us the opportunity to incorporate previously inaccessible kinetic effects into the highly effective, modern, finite-element MHD models.
    \end{abstract}
    
    
    \newpage
    \tableofcontents
    
    
    \newpage
    \pagenumbering{arabic}
    %\linenumbers\renewcommand\thelinenumber{\color{black!50}\arabic{linenumber}}
            \documentclass[12pt, a4paper]{report}

\input{template/main.tex}

\title{\BA{Title in Progress...}}
\author{Boris Andrews}
\affil{Mathematical Institute, University of Oxford}
\date{\today}


\begin{document}
    \pagenumbering{gobble}
    \maketitle
    
    
    \begin{abstract}
        Magnetic confinement reactors---in particular tokamaks---offer one of the most promising options for achieving practical nuclear fusion, with the potential to provide virtually limitless, clean energy. The theoretical and numerical modeling of tokamak plasmas is simultaneously an essential component of effective reactor design, and a great research barrier. Tokamak operational conditions exhibit comparatively low Knudsen numbers. Kinetic effects, including kinetic waves and instabilities, Landau damping, bump-on-tail instabilities and more, are therefore highly influential in tokamak plasma dynamics. Purely fluid models are inherently incapable of capturing these effects, whereas the high dimensionality in purely kinetic models render them practically intractable for most relevant purposes.

        We consider a $\delta\!f$ decomposition model, with a macroscopic fluid background and microscopic kinetic correction, both fully coupled to each other. A similar manner of discretization is proposed to that used in the recent \texttt{STRUPHY} code \cite{Holderied_Possanner_Wang_2021, Holderied_2022, Li_et_al_2023} with a finite-element model for the background and a pseudo-particle/PiC model for the correction.

        The fluid background satisfies the full, non-linear, resistive, compressible, Hall MHD equations. \cite{Laakmann_Hu_Farrell_2022} introduces finite-element(-in-space) implicit timesteppers for the incompressible analogue to this system with structure-preserving (SP) properties in the ideal case, alongside parameter-robust preconditioners. We show that these timesteppers can derive from a finite-element-in-time (FET) (and finite-element-in-space) interpretation. The benefits of this reformulation are discussed, including the derivation of timesteppers that are higher order in time, and the quantifiable dissipative SP properties in the non-ideal, resistive case.
        
        We discuss possible options for extending this FET approach to timesteppers for the compressible case.

        The kinetic corrections satisfy linearized Boltzmann equations. Using a Lénard--Bernstein collision operator, these take Fokker--Planck-like forms \cite{Fokker_1914, Planck_1917} wherein pseudo-particles in the numerical model obey the neoclassical transport equations, with particle-independent Brownian drift terms. This offers a rigorous methodology for incorporating collisions into the particle transport model, without coupling the equations of motions for each particle.
        
        Works by Chen, Chacón et al. \cite{Chen_Chacón_Barnes_2011, Chacón_Chen_Barnes_2013, Chen_Chacón_2014, Chen_Chacón_2015} have developed structure-preserving particle pushers for neoclassical transport in the Vlasov equations, derived from Crank--Nicolson integrators. We show these too can can derive from a FET interpretation, similarly offering potential extensions to higher-order-in-time particle pushers. The FET formulation is used also to consider how the stochastic drift terms can be incorporated into the pushers. Stochastic gyrokinetic expansions are also discussed.

        Different options for the numerical implementation of these schemes are considered.

        Due to the efficacy of FET in the development of SP timesteppers for both the fluid and kinetic component, we hope this approach will prove effective in the future for developing SP timesteppers for the full hybrid model. We hope this will give us the opportunity to incorporate previously inaccessible kinetic effects into the highly effective, modern, finite-element MHD models.
    \end{abstract}
    
    
    \newpage
    \tableofcontents
    
    
    \newpage
    \pagenumbering{arabic}
    %\linenumbers\renewcommand\thelinenumber{\color{black!50}\arabic{linenumber}}
            \input{0 - introduction/main.tex}
        \part{Research}
            \input{1 - low-noise PiC models/main.tex}
            \input{2 - kinetic component/main.tex}
            \input{3 - fluid component/main.tex}
            \input{4 - numerical implementation/main.tex}
        \part{Project Overview}
            \input{5 - research plan/main.tex}
            \input{6 - summary/main.tex}
    
    
    %\section{}
    \newpage
    \pagenumbering{gobble}
        \printbibliography


    \newpage
    \pagenumbering{roman}
    \appendix
        \part{Appendices}
            \input{8 - Hilbert complexes/main.tex}
            \input{9 - weak conservation proofs/main.tex}
\end{document}

        \part{Research}
            \documentclass[12pt, a4paper]{report}

\input{template/main.tex}

\title{\BA{Title in Progress...}}
\author{Boris Andrews}
\affil{Mathematical Institute, University of Oxford}
\date{\today}


\begin{document}
    \pagenumbering{gobble}
    \maketitle
    
    
    \begin{abstract}
        Magnetic confinement reactors---in particular tokamaks---offer one of the most promising options for achieving practical nuclear fusion, with the potential to provide virtually limitless, clean energy. The theoretical and numerical modeling of tokamak plasmas is simultaneously an essential component of effective reactor design, and a great research barrier. Tokamak operational conditions exhibit comparatively low Knudsen numbers. Kinetic effects, including kinetic waves and instabilities, Landau damping, bump-on-tail instabilities and more, are therefore highly influential in tokamak plasma dynamics. Purely fluid models are inherently incapable of capturing these effects, whereas the high dimensionality in purely kinetic models render them practically intractable for most relevant purposes.

        We consider a $\delta\!f$ decomposition model, with a macroscopic fluid background and microscopic kinetic correction, both fully coupled to each other. A similar manner of discretization is proposed to that used in the recent \texttt{STRUPHY} code \cite{Holderied_Possanner_Wang_2021, Holderied_2022, Li_et_al_2023} with a finite-element model for the background and a pseudo-particle/PiC model for the correction.

        The fluid background satisfies the full, non-linear, resistive, compressible, Hall MHD equations. \cite{Laakmann_Hu_Farrell_2022} introduces finite-element(-in-space) implicit timesteppers for the incompressible analogue to this system with structure-preserving (SP) properties in the ideal case, alongside parameter-robust preconditioners. We show that these timesteppers can derive from a finite-element-in-time (FET) (and finite-element-in-space) interpretation. The benefits of this reformulation are discussed, including the derivation of timesteppers that are higher order in time, and the quantifiable dissipative SP properties in the non-ideal, resistive case.
        
        We discuss possible options for extending this FET approach to timesteppers for the compressible case.

        The kinetic corrections satisfy linearized Boltzmann equations. Using a Lénard--Bernstein collision operator, these take Fokker--Planck-like forms \cite{Fokker_1914, Planck_1917} wherein pseudo-particles in the numerical model obey the neoclassical transport equations, with particle-independent Brownian drift terms. This offers a rigorous methodology for incorporating collisions into the particle transport model, without coupling the equations of motions for each particle.
        
        Works by Chen, Chacón et al. \cite{Chen_Chacón_Barnes_2011, Chacón_Chen_Barnes_2013, Chen_Chacón_2014, Chen_Chacón_2015} have developed structure-preserving particle pushers for neoclassical transport in the Vlasov equations, derived from Crank--Nicolson integrators. We show these too can can derive from a FET interpretation, similarly offering potential extensions to higher-order-in-time particle pushers. The FET formulation is used also to consider how the stochastic drift terms can be incorporated into the pushers. Stochastic gyrokinetic expansions are also discussed.

        Different options for the numerical implementation of these schemes are considered.

        Due to the efficacy of FET in the development of SP timesteppers for both the fluid and kinetic component, we hope this approach will prove effective in the future for developing SP timesteppers for the full hybrid model. We hope this will give us the opportunity to incorporate previously inaccessible kinetic effects into the highly effective, modern, finite-element MHD models.
    \end{abstract}
    
    
    \newpage
    \tableofcontents
    
    
    \newpage
    \pagenumbering{arabic}
    %\linenumbers\renewcommand\thelinenumber{\color{black!50}\arabic{linenumber}}
            \input{0 - introduction/main.tex}
        \part{Research}
            \input{1 - low-noise PiC models/main.tex}
            \input{2 - kinetic component/main.tex}
            \input{3 - fluid component/main.tex}
            \input{4 - numerical implementation/main.tex}
        \part{Project Overview}
            \input{5 - research plan/main.tex}
            \input{6 - summary/main.tex}
    
    
    %\section{}
    \newpage
    \pagenumbering{gobble}
        \printbibliography


    \newpage
    \pagenumbering{roman}
    \appendix
        \part{Appendices}
            \input{8 - Hilbert complexes/main.tex}
            \input{9 - weak conservation proofs/main.tex}
\end{document}

            \documentclass[12pt, a4paper]{report}

\input{template/main.tex}

\title{\BA{Title in Progress...}}
\author{Boris Andrews}
\affil{Mathematical Institute, University of Oxford}
\date{\today}


\begin{document}
    \pagenumbering{gobble}
    \maketitle
    
    
    \begin{abstract}
        Magnetic confinement reactors---in particular tokamaks---offer one of the most promising options for achieving practical nuclear fusion, with the potential to provide virtually limitless, clean energy. The theoretical and numerical modeling of tokamak plasmas is simultaneously an essential component of effective reactor design, and a great research barrier. Tokamak operational conditions exhibit comparatively low Knudsen numbers. Kinetic effects, including kinetic waves and instabilities, Landau damping, bump-on-tail instabilities and more, are therefore highly influential in tokamak plasma dynamics. Purely fluid models are inherently incapable of capturing these effects, whereas the high dimensionality in purely kinetic models render them practically intractable for most relevant purposes.

        We consider a $\delta\!f$ decomposition model, with a macroscopic fluid background and microscopic kinetic correction, both fully coupled to each other. A similar manner of discretization is proposed to that used in the recent \texttt{STRUPHY} code \cite{Holderied_Possanner_Wang_2021, Holderied_2022, Li_et_al_2023} with a finite-element model for the background and a pseudo-particle/PiC model for the correction.

        The fluid background satisfies the full, non-linear, resistive, compressible, Hall MHD equations. \cite{Laakmann_Hu_Farrell_2022} introduces finite-element(-in-space) implicit timesteppers for the incompressible analogue to this system with structure-preserving (SP) properties in the ideal case, alongside parameter-robust preconditioners. We show that these timesteppers can derive from a finite-element-in-time (FET) (and finite-element-in-space) interpretation. The benefits of this reformulation are discussed, including the derivation of timesteppers that are higher order in time, and the quantifiable dissipative SP properties in the non-ideal, resistive case.
        
        We discuss possible options for extending this FET approach to timesteppers for the compressible case.

        The kinetic corrections satisfy linearized Boltzmann equations. Using a Lénard--Bernstein collision operator, these take Fokker--Planck-like forms \cite{Fokker_1914, Planck_1917} wherein pseudo-particles in the numerical model obey the neoclassical transport equations, with particle-independent Brownian drift terms. This offers a rigorous methodology for incorporating collisions into the particle transport model, without coupling the equations of motions for each particle.
        
        Works by Chen, Chacón et al. \cite{Chen_Chacón_Barnes_2011, Chacón_Chen_Barnes_2013, Chen_Chacón_2014, Chen_Chacón_2015} have developed structure-preserving particle pushers for neoclassical transport in the Vlasov equations, derived from Crank--Nicolson integrators. We show these too can can derive from a FET interpretation, similarly offering potential extensions to higher-order-in-time particle pushers. The FET formulation is used also to consider how the stochastic drift terms can be incorporated into the pushers. Stochastic gyrokinetic expansions are also discussed.

        Different options for the numerical implementation of these schemes are considered.

        Due to the efficacy of FET in the development of SP timesteppers for both the fluid and kinetic component, we hope this approach will prove effective in the future for developing SP timesteppers for the full hybrid model. We hope this will give us the opportunity to incorporate previously inaccessible kinetic effects into the highly effective, modern, finite-element MHD models.
    \end{abstract}
    
    
    \newpage
    \tableofcontents
    
    
    \newpage
    \pagenumbering{arabic}
    %\linenumbers\renewcommand\thelinenumber{\color{black!50}\arabic{linenumber}}
            \input{0 - introduction/main.tex}
        \part{Research}
            \input{1 - low-noise PiC models/main.tex}
            \input{2 - kinetic component/main.tex}
            \input{3 - fluid component/main.tex}
            \input{4 - numerical implementation/main.tex}
        \part{Project Overview}
            \input{5 - research plan/main.tex}
            \input{6 - summary/main.tex}
    
    
    %\section{}
    \newpage
    \pagenumbering{gobble}
        \printbibliography


    \newpage
    \pagenumbering{roman}
    \appendix
        \part{Appendices}
            \input{8 - Hilbert complexes/main.tex}
            \input{9 - weak conservation proofs/main.tex}
\end{document}

            \documentclass[12pt, a4paper]{report}

\input{template/main.tex}

\title{\BA{Title in Progress...}}
\author{Boris Andrews}
\affil{Mathematical Institute, University of Oxford}
\date{\today}


\begin{document}
    \pagenumbering{gobble}
    \maketitle
    
    
    \begin{abstract}
        Magnetic confinement reactors---in particular tokamaks---offer one of the most promising options for achieving practical nuclear fusion, with the potential to provide virtually limitless, clean energy. The theoretical and numerical modeling of tokamak plasmas is simultaneously an essential component of effective reactor design, and a great research barrier. Tokamak operational conditions exhibit comparatively low Knudsen numbers. Kinetic effects, including kinetic waves and instabilities, Landau damping, bump-on-tail instabilities and more, are therefore highly influential in tokamak plasma dynamics. Purely fluid models are inherently incapable of capturing these effects, whereas the high dimensionality in purely kinetic models render them practically intractable for most relevant purposes.

        We consider a $\delta\!f$ decomposition model, with a macroscopic fluid background and microscopic kinetic correction, both fully coupled to each other. A similar manner of discretization is proposed to that used in the recent \texttt{STRUPHY} code \cite{Holderied_Possanner_Wang_2021, Holderied_2022, Li_et_al_2023} with a finite-element model for the background and a pseudo-particle/PiC model for the correction.

        The fluid background satisfies the full, non-linear, resistive, compressible, Hall MHD equations. \cite{Laakmann_Hu_Farrell_2022} introduces finite-element(-in-space) implicit timesteppers for the incompressible analogue to this system with structure-preserving (SP) properties in the ideal case, alongside parameter-robust preconditioners. We show that these timesteppers can derive from a finite-element-in-time (FET) (and finite-element-in-space) interpretation. The benefits of this reformulation are discussed, including the derivation of timesteppers that are higher order in time, and the quantifiable dissipative SP properties in the non-ideal, resistive case.
        
        We discuss possible options for extending this FET approach to timesteppers for the compressible case.

        The kinetic corrections satisfy linearized Boltzmann equations. Using a Lénard--Bernstein collision operator, these take Fokker--Planck-like forms \cite{Fokker_1914, Planck_1917} wherein pseudo-particles in the numerical model obey the neoclassical transport equations, with particle-independent Brownian drift terms. This offers a rigorous methodology for incorporating collisions into the particle transport model, without coupling the equations of motions for each particle.
        
        Works by Chen, Chacón et al. \cite{Chen_Chacón_Barnes_2011, Chacón_Chen_Barnes_2013, Chen_Chacón_2014, Chen_Chacón_2015} have developed structure-preserving particle pushers for neoclassical transport in the Vlasov equations, derived from Crank--Nicolson integrators. We show these too can can derive from a FET interpretation, similarly offering potential extensions to higher-order-in-time particle pushers. The FET formulation is used also to consider how the stochastic drift terms can be incorporated into the pushers. Stochastic gyrokinetic expansions are also discussed.

        Different options for the numerical implementation of these schemes are considered.

        Due to the efficacy of FET in the development of SP timesteppers for both the fluid and kinetic component, we hope this approach will prove effective in the future for developing SP timesteppers for the full hybrid model. We hope this will give us the opportunity to incorporate previously inaccessible kinetic effects into the highly effective, modern, finite-element MHD models.
    \end{abstract}
    
    
    \newpage
    \tableofcontents
    
    
    \newpage
    \pagenumbering{arabic}
    %\linenumbers\renewcommand\thelinenumber{\color{black!50}\arabic{linenumber}}
            \input{0 - introduction/main.tex}
        \part{Research}
            \input{1 - low-noise PiC models/main.tex}
            \input{2 - kinetic component/main.tex}
            \input{3 - fluid component/main.tex}
            \input{4 - numerical implementation/main.tex}
        \part{Project Overview}
            \input{5 - research plan/main.tex}
            \input{6 - summary/main.tex}
    
    
    %\section{}
    \newpage
    \pagenumbering{gobble}
        \printbibliography


    \newpage
    \pagenumbering{roman}
    \appendix
        \part{Appendices}
            \input{8 - Hilbert complexes/main.tex}
            \input{9 - weak conservation proofs/main.tex}
\end{document}

            \documentclass[12pt, a4paper]{report}

\input{template/main.tex}

\title{\BA{Title in Progress...}}
\author{Boris Andrews}
\affil{Mathematical Institute, University of Oxford}
\date{\today}


\begin{document}
    \pagenumbering{gobble}
    \maketitle
    
    
    \begin{abstract}
        Magnetic confinement reactors---in particular tokamaks---offer one of the most promising options for achieving practical nuclear fusion, with the potential to provide virtually limitless, clean energy. The theoretical and numerical modeling of tokamak plasmas is simultaneously an essential component of effective reactor design, and a great research barrier. Tokamak operational conditions exhibit comparatively low Knudsen numbers. Kinetic effects, including kinetic waves and instabilities, Landau damping, bump-on-tail instabilities and more, are therefore highly influential in tokamak plasma dynamics. Purely fluid models are inherently incapable of capturing these effects, whereas the high dimensionality in purely kinetic models render them practically intractable for most relevant purposes.

        We consider a $\delta\!f$ decomposition model, with a macroscopic fluid background and microscopic kinetic correction, both fully coupled to each other. A similar manner of discretization is proposed to that used in the recent \texttt{STRUPHY} code \cite{Holderied_Possanner_Wang_2021, Holderied_2022, Li_et_al_2023} with a finite-element model for the background and a pseudo-particle/PiC model for the correction.

        The fluid background satisfies the full, non-linear, resistive, compressible, Hall MHD equations. \cite{Laakmann_Hu_Farrell_2022} introduces finite-element(-in-space) implicit timesteppers for the incompressible analogue to this system with structure-preserving (SP) properties in the ideal case, alongside parameter-robust preconditioners. We show that these timesteppers can derive from a finite-element-in-time (FET) (and finite-element-in-space) interpretation. The benefits of this reformulation are discussed, including the derivation of timesteppers that are higher order in time, and the quantifiable dissipative SP properties in the non-ideal, resistive case.
        
        We discuss possible options for extending this FET approach to timesteppers for the compressible case.

        The kinetic corrections satisfy linearized Boltzmann equations. Using a Lénard--Bernstein collision operator, these take Fokker--Planck-like forms \cite{Fokker_1914, Planck_1917} wherein pseudo-particles in the numerical model obey the neoclassical transport equations, with particle-independent Brownian drift terms. This offers a rigorous methodology for incorporating collisions into the particle transport model, without coupling the equations of motions for each particle.
        
        Works by Chen, Chacón et al. \cite{Chen_Chacón_Barnes_2011, Chacón_Chen_Barnes_2013, Chen_Chacón_2014, Chen_Chacón_2015} have developed structure-preserving particle pushers for neoclassical transport in the Vlasov equations, derived from Crank--Nicolson integrators. We show these too can can derive from a FET interpretation, similarly offering potential extensions to higher-order-in-time particle pushers. The FET formulation is used also to consider how the stochastic drift terms can be incorporated into the pushers. Stochastic gyrokinetic expansions are also discussed.

        Different options for the numerical implementation of these schemes are considered.

        Due to the efficacy of FET in the development of SP timesteppers for both the fluid and kinetic component, we hope this approach will prove effective in the future for developing SP timesteppers for the full hybrid model. We hope this will give us the opportunity to incorporate previously inaccessible kinetic effects into the highly effective, modern, finite-element MHD models.
    \end{abstract}
    
    
    \newpage
    \tableofcontents
    
    
    \newpage
    \pagenumbering{arabic}
    %\linenumbers\renewcommand\thelinenumber{\color{black!50}\arabic{linenumber}}
            \input{0 - introduction/main.tex}
        \part{Research}
            \input{1 - low-noise PiC models/main.tex}
            \input{2 - kinetic component/main.tex}
            \input{3 - fluid component/main.tex}
            \input{4 - numerical implementation/main.tex}
        \part{Project Overview}
            \input{5 - research plan/main.tex}
            \input{6 - summary/main.tex}
    
    
    %\section{}
    \newpage
    \pagenumbering{gobble}
        \printbibliography


    \newpage
    \pagenumbering{roman}
    \appendix
        \part{Appendices}
            \input{8 - Hilbert complexes/main.tex}
            \input{9 - weak conservation proofs/main.tex}
\end{document}

        \part{Project Overview}
            \documentclass[12pt, a4paper]{report}

\input{template/main.tex}

\title{\BA{Title in Progress...}}
\author{Boris Andrews}
\affil{Mathematical Institute, University of Oxford}
\date{\today}


\begin{document}
    \pagenumbering{gobble}
    \maketitle
    
    
    \begin{abstract}
        Magnetic confinement reactors---in particular tokamaks---offer one of the most promising options for achieving practical nuclear fusion, with the potential to provide virtually limitless, clean energy. The theoretical and numerical modeling of tokamak plasmas is simultaneously an essential component of effective reactor design, and a great research barrier. Tokamak operational conditions exhibit comparatively low Knudsen numbers. Kinetic effects, including kinetic waves and instabilities, Landau damping, bump-on-tail instabilities and more, are therefore highly influential in tokamak plasma dynamics. Purely fluid models are inherently incapable of capturing these effects, whereas the high dimensionality in purely kinetic models render them practically intractable for most relevant purposes.

        We consider a $\delta\!f$ decomposition model, with a macroscopic fluid background and microscopic kinetic correction, both fully coupled to each other. A similar manner of discretization is proposed to that used in the recent \texttt{STRUPHY} code \cite{Holderied_Possanner_Wang_2021, Holderied_2022, Li_et_al_2023} with a finite-element model for the background and a pseudo-particle/PiC model for the correction.

        The fluid background satisfies the full, non-linear, resistive, compressible, Hall MHD equations. \cite{Laakmann_Hu_Farrell_2022} introduces finite-element(-in-space) implicit timesteppers for the incompressible analogue to this system with structure-preserving (SP) properties in the ideal case, alongside parameter-robust preconditioners. We show that these timesteppers can derive from a finite-element-in-time (FET) (and finite-element-in-space) interpretation. The benefits of this reformulation are discussed, including the derivation of timesteppers that are higher order in time, and the quantifiable dissipative SP properties in the non-ideal, resistive case.
        
        We discuss possible options for extending this FET approach to timesteppers for the compressible case.

        The kinetic corrections satisfy linearized Boltzmann equations. Using a Lénard--Bernstein collision operator, these take Fokker--Planck-like forms \cite{Fokker_1914, Planck_1917} wherein pseudo-particles in the numerical model obey the neoclassical transport equations, with particle-independent Brownian drift terms. This offers a rigorous methodology for incorporating collisions into the particle transport model, without coupling the equations of motions for each particle.
        
        Works by Chen, Chacón et al. \cite{Chen_Chacón_Barnes_2011, Chacón_Chen_Barnes_2013, Chen_Chacón_2014, Chen_Chacón_2015} have developed structure-preserving particle pushers for neoclassical transport in the Vlasov equations, derived from Crank--Nicolson integrators. We show these too can can derive from a FET interpretation, similarly offering potential extensions to higher-order-in-time particle pushers. The FET formulation is used also to consider how the stochastic drift terms can be incorporated into the pushers. Stochastic gyrokinetic expansions are also discussed.

        Different options for the numerical implementation of these schemes are considered.

        Due to the efficacy of FET in the development of SP timesteppers for both the fluid and kinetic component, we hope this approach will prove effective in the future for developing SP timesteppers for the full hybrid model. We hope this will give us the opportunity to incorporate previously inaccessible kinetic effects into the highly effective, modern, finite-element MHD models.
    \end{abstract}
    
    
    \newpage
    \tableofcontents
    
    
    \newpage
    \pagenumbering{arabic}
    %\linenumbers\renewcommand\thelinenumber{\color{black!50}\arabic{linenumber}}
            \input{0 - introduction/main.tex}
        \part{Research}
            \input{1 - low-noise PiC models/main.tex}
            \input{2 - kinetic component/main.tex}
            \input{3 - fluid component/main.tex}
            \input{4 - numerical implementation/main.tex}
        \part{Project Overview}
            \input{5 - research plan/main.tex}
            \input{6 - summary/main.tex}
    
    
    %\section{}
    \newpage
    \pagenumbering{gobble}
        \printbibliography


    \newpage
    \pagenumbering{roman}
    \appendix
        \part{Appendices}
            \input{8 - Hilbert complexes/main.tex}
            \input{9 - weak conservation proofs/main.tex}
\end{document}

            \documentclass[12pt, a4paper]{report}

\input{template/main.tex}

\title{\BA{Title in Progress...}}
\author{Boris Andrews}
\affil{Mathematical Institute, University of Oxford}
\date{\today}


\begin{document}
    \pagenumbering{gobble}
    \maketitle
    
    
    \begin{abstract}
        Magnetic confinement reactors---in particular tokamaks---offer one of the most promising options for achieving practical nuclear fusion, with the potential to provide virtually limitless, clean energy. The theoretical and numerical modeling of tokamak plasmas is simultaneously an essential component of effective reactor design, and a great research barrier. Tokamak operational conditions exhibit comparatively low Knudsen numbers. Kinetic effects, including kinetic waves and instabilities, Landau damping, bump-on-tail instabilities and more, are therefore highly influential in tokamak plasma dynamics. Purely fluid models are inherently incapable of capturing these effects, whereas the high dimensionality in purely kinetic models render them practically intractable for most relevant purposes.

        We consider a $\delta\!f$ decomposition model, with a macroscopic fluid background and microscopic kinetic correction, both fully coupled to each other. A similar manner of discretization is proposed to that used in the recent \texttt{STRUPHY} code \cite{Holderied_Possanner_Wang_2021, Holderied_2022, Li_et_al_2023} with a finite-element model for the background and a pseudo-particle/PiC model for the correction.

        The fluid background satisfies the full, non-linear, resistive, compressible, Hall MHD equations. \cite{Laakmann_Hu_Farrell_2022} introduces finite-element(-in-space) implicit timesteppers for the incompressible analogue to this system with structure-preserving (SP) properties in the ideal case, alongside parameter-robust preconditioners. We show that these timesteppers can derive from a finite-element-in-time (FET) (and finite-element-in-space) interpretation. The benefits of this reformulation are discussed, including the derivation of timesteppers that are higher order in time, and the quantifiable dissipative SP properties in the non-ideal, resistive case.
        
        We discuss possible options for extending this FET approach to timesteppers for the compressible case.

        The kinetic corrections satisfy linearized Boltzmann equations. Using a Lénard--Bernstein collision operator, these take Fokker--Planck-like forms \cite{Fokker_1914, Planck_1917} wherein pseudo-particles in the numerical model obey the neoclassical transport equations, with particle-independent Brownian drift terms. This offers a rigorous methodology for incorporating collisions into the particle transport model, without coupling the equations of motions for each particle.
        
        Works by Chen, Chacón et al. \cite{Chen_Chacón_Barnes_2011, Chacón_Chen_Barnes_2013, Chen_Chacón_2014, Chen_Chacón_2015} have developed structure-preserving particle pushers for neoclassical transport in the Vlasov equations, derived from Crank--Nicolson integrators. We show these too can can derive from a FET interpretation, similarly offering potential extensions to higher-order-in-time particle pushers. The FET formulation is used also to consider how the stochastic drift terms can be incorporated into the pushers. Stochastic gyrokinetic expansions are also discussed.

        Different options for the numerical implementation of these schemes are considered.

        Due to the efficacy of FET in the development of SP timesteppers for both the fluid and kinetic component, we hope this approach will prove effective in the future for developing SP timesteppers for the full hybrid model. We hope this will give us the opportunity to incorporate previously inaccessible kinetic effects into the highly effective, modern, finite-element MHD models.
    \end{abstract}
    
    
    \newpage
    \tableofcontents
    
    
    \newpage
    \pagenumbering{arabic}
    %\linenumbers\renewcommand\thelinenumber{\color{black!50}\arabic{linenumber}}
            \input{0 - introduction/main.tex}
        \part{Research}
            \input{1 - low-noise PiC models/main.tex}
            \input{2 - kinetic component/main.tex}
            \input{3 - fluid component/main.tex}
            \input{4 - numerical implementation/main.tex}
        \part{Project Overview}
            \input{5 - research plan/main.tex}
            \input{6 - summary/main.tex}
    
    
    %\section{}
    \newpage
    \pagenumbering{gobble}
        \printbibliography


    \newpage
    \pagenumbering{roman}
    \appendix
        \part{Appendices}
            \input{8 - Hilbert complexes/main.tex}
            \input{9 - weak conservation proofs/main.tex}
\end{document}

    
    
    %\section{}
    \newpage
    \pagenumbering{gobble}
        \printbibliography


    \newpage
    \pagenumbering{roman}
    \appendix
        \part{Appendices}
            \documentclass[12pt, a4paper]{report}

\input{template/main.tex}

\title{\BA{Title in Progress...}}
\author{Boris Andrews}
\affil{Mathematical Institute, University of Oxford}
\date{\today}


\begin{document}
    \pagenumbering{gobble}
    \maketitle
    
    
    \begin{abstract}
        Magnetic confinement reactors---in particular tokamaks---offer one of the most promising options for achieving practical nuclear fusion, with the potential to provide virtually limitless, clean energy. The theoretical and numerical modeling of tokamak plasmas is simultaneously an essential component of effective reactor design, and a great research barrier. Tokamak operational conditions exhibit comparatively low Knudsen numbers. Kinetic effects, including kinetic waves and instabilities, Landau damping, bump-on-tail instabilities and more, are therefore highly influential in tokamak plasma dynamics. Purely fluid models are inherently incapable of capturing these effects, whereas the high dimensionality in purely kinetic models render them practically intractable for most relevant purposes.

        We consider a $\delta\!f$ decomposition model, with a macroscopic fluid background and microscopic kinetic correction, both fully coupled to each other. A similar manner of discretization is proposed to that used in the recent \texttt{STRUPHY} code \cite{Holderied_Possanner_Wang_2021, Holderied_2022, Li_et_al_2023} with a finite-element model for the background and a pseudo-particle/PiC model for the correction.

        The fluid background satisfies the full, non-linear, resistive, compressible, Hall MHD equations. \cite{Laakmann_Hu_Farrell_2022} introduces finite-element(-in-space) implicit timesteppers for the incompressible analogue to this system with structure-preserving (SP) properties in the ideal case, alongside parameter-robust preconditioners. We show that these timesteppers can derive from a finite-element-in-time (FET) (and finite-element-in-space) interpretation. The benefits of this reformulation are discussed, including the derivation of timesteppers that are higher order in time, and the quantifiable dissipative SP properties in the non-ideal, resistive case.
        
        We discuss possible options for extending this FET approach to timesteppers for the compressible case.

        The kinetic corrections satisfy linearized Boltzmann equations. Using a Lénard--Bernstein collision operator, these take Fokker--Planck-like forms \cite{Fokker_1914, Planck_1917} wherein pseudo-particles in the numerical model obey the neoclassical transport equations, with particle-independent Brownian drift terms. This offers a rigorous methodology for incorporating collisions into the particle transport model, without coupling the equations of motions for each particle.
        
        Works by Chen, Chacón et al. \cite{Chen_Chacón_Barnes_2011, Chacón_Chen_Barnes_2013, Chen_Chacón_2014, Chen_Chacón_2015} have developed structure-preserving particle pushers for neoclassical transport in the Vlasov equations, derived from Crank--Nicolson integrators. We show these too can can derive from a FET interpretation, similarly offering potential extensions to higher-order-in-time particle pushers. The FET formulation is used also to consider how the stochastic drift terms can be incorporated into the pushers. Stochastic gyrokinetic expansions are also discussed.

        Different options for the numerical implementation of these schemes are considered.

        Due to the efficacy of FET in the development of SP timesteppers for both the fluid and kinetic component, we hope this approach will prove effective in the future for developing SP timesteppers for the full hybrid model. We hope this will give us the opportunity to incorporate previously inaccessible kinetic effects into the highly effective, modern, finite-element MHD models.
    \end{abstract}
    
    
    \newpage
    \tableofcontents
    
    
    \newpage
    \pagenumbering{arabic}
    %\linenumbers\renewcommand\thelinenumber{\color{black!50}\arabic{linenumber}}
            \input{0 - introduction/main.tex}
        \part{Research}
            \input{1 - low-noise PiC models/main.tex}
            \input{2 - kinetic component/main.tex}
            \input{3 - fluid component/main.tex}
            \input{4 - numerical implementation/main.tex}
        \part{Project Overview}
            \input{5 - research plan/main.tex}
            \input{6 - summary/main.tex}
    
    
    %\section{}
    \newpage
    \pagenumbering{gobble}
        \printbibliography


    \newpage
    \pagenumbering{roman}
    \appendix
        \part{Appendices}
            \input{8 - Hilbert complexes/main.tex}
            \input{9 - weak conservation proofs/main.tex}
\end{document}

            \documentclass[12pt, a4paper]{report}

\input{template/main.tex}

\title{\BA{Title in Progress...}}
\author{Boris Andrews}
\affil{Mathematical Institute, University of Oxford}
\date{\today}


\begin{document}
    \pagenumbering{gobble}
    \maketitle
    
    
    \begin{abstract}
        Magnetic confinement reactors---in particular tokamaks---offer one of the most promising options for achieving practical nuclear fusion, with the potential to provide virtually limitless, clean energy. The theoretical and numerical modeling of tokamak plasmas is simultaneously an essential component of effective reactor design, and a great research barrier. Tokamak operational conditions exhibit comparatively low Knudsen numbers. Kinetic effects, including kinetic waves and instabilities, Landau damping, bump-on-tail instabilities and more, are therefore highly influential in tokamak plasma dynamics. Purely fluid models are inherently incapable of capturing these effects, whereas the high dimensionality in purely kinetic models render them practically intractable for most relevant purposes.

        We consider a $\delta\!f$ decomposition model, with a macroscopic fluid background and microscopic kinetic correction, both fully coupled to each other. A similar manner of discretization is proposed to that used in the recent \texttt{STRUPHY} code \cite{Holderied_Possanner_Wang_2021, Holderied_2022, Li_et_al_2023} with a finite-element model for the background and a pseudo-particle/PiC model for the correction.

        The fluid background satisfies the full, non-linear, resistive, compressible, Hall MHD equations. \cite{Laakmann_Hu_Farrell_2022} introduces finite-element(-in-space) implicit timesteppers for the incompressible analogue to this system with structure-preserving (SP) properties in the ideal case, alongside parameter-robust preconditioners. We show that these timesteppers can derive from a finite-element-in-time (FET) (and finite-element-in-space) interpretation. The benefits of this reformulation are discussed, including the derivation of timesteppers that are higher order in time, and the quantifiable dissipative SP properties in the non-ideal, resistive case.
        
        We discuss possible options for extending this FET approach to timesteppers for the compressible case.

        The kinetic corrections satisfy linearized Boltzmann equations. Using a Lénard--Bernstein collision operator, these take Fokker--Planck-like forms \cite{Fokker_1914, Planck_1917} wherein pseudo-particles in the numerical model obey the neoclassical transport equations, with particle-independent Brownian drift terms. This offers a rigorous methodology for incorporating collisions into the particle transport model, without coupling the equations of motions for each particle.
        
        Works by Chen, Chacón et al. \cite{Chen_Chacón_Barnes_2011, Chacón_Chen_Barnes_2013, Chen_Chacón_2014, Chen_Chacón_2015} have developed structure-preserving particle pushers for neoclassical transport in the Vlasov equations, derived from Crank--Nicolson integrators. We show these too can can derive from a FET interpretation, similarly offering potential extensions to higher-order-in-time particle pushers. The FET formulation is used also to consider how the stochastic drift terms can be incorporated into the pushers. Stochastic gyrokinetic expansions are also discussed.

        Different options for the numerical implementation of these schemes are considered.

        Due to the efficacy of FET in the development of SP timesteppers for both the fluid and kinetic component, we hope this approach will prove effective in the future for developing SP timesteppers for the full hybrid model. We hope this will give us the opportunity to incorporate previously inaccessible kinetic effects into the highly effective, modern, finite-element MHD models.
    \end{abstract}
    
    
    \newpage
    \tableofcontents
    
    
    \newpage
    \pagenumbering{arabic}
    %\linenumbers\renewcommand\thelinenumber{\color{black!50}\arabic{linenumber}}
            \input{0 - introduction/main.tex}
        \part{Research}
            \input{1 - low-noise PiC models/main.tex}
            \input{2 - kinetic component/main.tex}
            \input{3 - fluid component/main.tex}
            \input{4 - numerical implementation/main.tex}
        \part{Project Overview}
            \input{5 - research plan/main.tex}
            \input{6 - summary/main.tex}
    
    
    %\section{}
    \newpage
    \pagenumbering{gobble}
        \printbibliography


    \newpage
    \pagenumbering{roman}
    \appendix
        \part{Appendices}
            \input{8 - Hilbert complexes/main.tex}
            \input{9 - weak conservation proofs/main.tex}
\end{document}

\end{document}

            \documentclass[12pt, a4paper]{report}

\documentclass[12pt, a4paper]{report}

\input{template/main.tex}

\title{\BA{Title in Progress...}}
\author{Boris Andrews}
\affil{Mathematical Institute, University of Oxford}
\date{\today}


\begin{document}
    \pagenumbering{gobble}
    \maketitle
    
    
    \begin{abstract}
        Magnetic confinement reactors---in particular tokamaks---offer one of the most promising options for achieving practical nuclear fusion, with the potential to provide virtually limitless, clean energy. The theoretical and numerical modeling of tokamak plasmas is simultaneously an essential component of effective reactor design, and a great research barrier. Tokamak operational conditions exhibit comparatively low Knudsen numbers. Kinetic effects, including kinetic waves and instabilities, Landau damping, bump-on-tail instabilities and more, are therefore highly influential in tokamak plasma dynamics. Purely fluid models are inherently incapable of capturing these effects, whereas the high dimensionality in purely kinetic models render them practically intractable for most relevant purposes.

        We consider a $\delta\!f$ decomposition model, with a macroscopic fluid background and microscopic kinetic correction, both fully coupled to each other. A similar manner of discretization is proposed to that used in the recent \texttt{STRUPHY} code \cite{Holderied_Possanner_Wang_2021, Holderied_2022, Li_et_al_2023} with a finite-element model for the background and a pseudo-particle/PiC model for the correction.

        The fluid background satisfies the full, non-linear, resistive, compressible, Hall MHD equations. \cite{Laakmann_Hu_Farrell_2022} introduces finite-element(-in-space) implicit timesteppers for the incompressible analogue to this system with structure-preserving (SP) properties in the ideal case, alongside parameter-robust preconditioners. We show that these timesteppers can derive from a finite-element-in-time (FET) (and finite-element-in-space) interpretation. The benefits of this reformulation are discussed, including the derivation of timesteppers that are higher order in time, and the quantifiable dissipative SP properties in the non-ideal, resistive case.
        
        We discuss possible options for extending this FET approach to timesteppers for the compressible case.

        The kinetic corrections satisfy linearized Boltzmann equations. Using a Lénard--Bernstein collision operator, these take Fokker--Planck-like forms \cite{Fokker_1914, Planck_1917} wherein pseudo-particles in the numerical model obey the neoclassical transport equations, with particle-independent Brownian drift terms. This offers a rigorous methodology for incorporating collisions into the particle transport model, without coupling the equations of motions for each particle.
        
        Works by Chen, Chacón et al. \cite{Chen_Chacón_Barnes_2011, Chacón_Chen_Barnes_2013, Chen_Chacón_2014, Chen_Chacón_2015} have developed structure-preserving particle pushers for neoclassical transport in the Vlasov equations, derived from Crank--Nicolson integrators. We show these too can can derive from a FET interpretation, similarly offering potential extensions to higher-order-in-time particle pushers. The FET formulation is used also to consider how the stochastic drift terms can be incorporated into the pushers. Stochastic gyrokinetic expansions are also discussed.

        Different options for the numerical implementation of these schemes are considered.

        Due to the efficacy of FET in the development of SP timesteppers for both the fluid and kinetic component, we hope this approach will prove effective in the future for developing SP timesteppers for the full hybrid model. We hope this will give us the opportunity to incorporate previously inaccessible kinetic effects into the highly effective, modern, finite-element MHD models.
    \end{abstract}
    
    
    \newpage
    \tableofcontents
    
    
    \newpage
    \pagenumbering{arabic}
    %\linenumbers\renewcommand\thelinenumber{\color{black!50}\arabic{linenumber}}
            \input{0 - introduction/main.tex}
        \part{Research}
            \input{1 - low-noise PiC models/main.tex}
            \input{2 - kinetic component/main.tex}
            \input{3 - fluid component/main.tex}
            \input{4 - numerical implementation/main.tex}
        \part{Project Overview}
            \input{5 - research plan/main.tex}
            \input{6 - summary/main.tex}
    
    
    %\section{}
    \newpage
    \pagenumbering{gobble}
        \printbibliography


    \newpage
    \pagenumbering{roman}
    \appendix
        \part{Appendices}
            \input{8 - Hilbert complexes/main.tex}
            \input{9 - weak conservation proofs/main.tex}
\end{document}


\title{\BA{Title in Progress...}}
\author{Boris Andrews}
\affil{Mathematical Institute, University of Oxford}
\date{\today}


\begin{document}
    \pagenumbering{gobble}
    \maketitle
    
    
    \begin{abstract}
        Magnetic confinement reactors---in particular tokamaks---offer one of the most promising options for achieving practical nuclear fusion, with the potential to provide virtually limitless, clean energy. The theoretical and numerical modeling of tokamak plasmas is simultaneously an essential component of effective reactor design, and a great research barrier. Tokamak operational conditions exhibit comparatively low Knudsen numbers. Kinetic effects, including kinetic waves and instabilities, Landau damping, bump-on-tail instabilities and more, are therefore highly influential in tokamak plasma dynamics. Purely fluid models are inherently incapable of capturing these effects, whereas the high dimensionality in purely kinetic models render them practically intractable for most relevant purposes.

        We consider a $\delta\!f$ decomposition model, with a macroscopic fluid background and microscopic kinetic correction, both fully coupled to each other. A similar manner of discretization is proposed to that used in the recent \texttt{STRUPHY} code \cite{Holderied_Possanner_Wang_2021, Holderied_2022, Li_et_al_2023} with a finite-element model for the background and a pseudo-particle/PiC model for the correction.

        The fluid background satisfies the full, non-linear, resistive, compressible, Hall MHD equations. \cite{Laakmann_Hu_Farrell_2022} introduces finite-element(-in-space) implicit timesteppers for the incompressible analogue to this system with structure-preserving (SP) properties in the ideal case, alongside parameter-robust preconditioners. We show that these timesteppers can derive from a finite-element-in-time (FET) (and finite-element-in-space) interpretation. The benefits of this reformulation are discussed, including the derivation of timesteppers that are higher order in time, and the quantifiable dissipative SP properties in the non-ideal, resistive case.
        
        We discuss possible options for extending this FET approach to timesteppers for the compressible case.

        The kinetic corrections satisfy linearized Boltzmann equations. Using a Lénard--Bernstein collision operator, these take Fokker--Planck-like forms \cite{Fokker_1914, Planck_1917} wherein pseudo-particles in the numerical model obey the neoclassical transport equations, with particle-independent Brownian drift terms. This offers a rigorous methodology for incorporating collisions into the particle transport model, without coupling the equations of motions for each particle.
        
        Works by Chen, Chacón et al. \cite{Chen_Chacón_Barnes_2011, Chacón_Chen_Barnes_2013, Chen_Chacón_2014, Chen_Chacón_2015} have developed structure-preserving particle pushers for neoclassical transport in the Vlasov equations, derived from Crank--Nicolson integrators. We show these too can can derive from a FET interpretation, similarly offering potential extensions to higher-order-in-time particle pushers. The FET formulation is used also to consider how the stochastic drift terms can be incorporated into the pushers. Stochastic gyrokinetic expansions are also discussed.

        Different options for the numerical implementation of these schemes are considered.

        Due to the efficacy of FET in the development of SP timesteppers for both the fluid and kinetic component, we hope this approach will prove effective in the future for developing SP timesteppers for the full hybrid model. We hope this will give us the opportunity to incorporate previously inaccessible kinetic effects into the highly effective, modern, finite-element MHD models.
    \end{abstract}
    
    
    \newpage
    \tableofcontents
    
    
    \newpage
    \pagenumbering{arabic}
    %\linenumbers\renewcommand\thelinenumber{\color{black!50}\arabic{linenumber}}
            \documentclass[12pt, a4paper]{report}

\input{template/main.tex}

\title{\BA{Title in Progress...}}
\author{Boris Andrews}
\affil{Mathematical Institute, University of Oxford}
\date{\today}


\begin{document}
    \pagenumbering{gobble}
    \maketitle
    
    
    \begin{abstract}
        Magnetic confinement reactors---in particular tokamaks---offer one of the most promising options for achieving practical nuclear fusion, with the potential to provide virtually limitless, clean energy. The theoretical and numerical modeling of tokamak plasmas is simultaneously an essential component of effective reactor design, and a great research barrier. Tokamak operational conditions exhibit comparatively low Knudsen numbers. Kinetic effects, including kinetic waves and instabilities, Landau damping, bump-on-tail instabilities and more, are therefore highly influential in tokamak plasma dynamics. Purely fluid models are inherently incapable of capturing these effects, whereas the high dimensionality in purely kinetic models render them practically intractable for most relevant purposes.

        We consider a $\delta\!f$ decomposition model, with a macroscopic fluid background and microscopic kinetic correction, both fully coupled to each other. A similar manner of discretization is proposed to that used in the recent \texttt{STRUPHY} code \cite{Holderied_Possanner_Wang_2021, Holderied_2022, Li_et_al_2023} with a finite-element model for the background and a pseudo-particle/PiC model for the correction.

        The fluid background satisfies the full, non-linear, resistive, compressible, Hall MHD equations. \cite{Laakmann_Hu_Farrell_2022} introduces finite-element(-in-space) implicit timesteppers for the incompressible analogue to this system with structure-preserving (SP) properties in the ideal case, alongside parameter-robust preconditioners. We show that these timesteppers can derive from a finite-element-in-time (FET) (and finite-element-in-space) interpretation. The benefits of this reformulation are discussed, including the derivation of timesteppers that are higher order in time, and the quantifiable dissipative SP properties in the non-ideal, resistive case.
        
        We discuss possible options for extending this FET approach to timesteppers for the compressible case.

        The kinetic corrections satisfy linearized Boltzmann equations. Using a Lénard--Bernstein collision operator, these take Fokker--Planck-like forms \cite{Fokker_1914, Planck_1917} wherein pseudo-particles in the numerical model obey the neoclassical transport equations, with particle-independent Brownian drift terms. This offers a rigorous methodology for incorporating collisions into the particle transport model, without coupling the equations of motions for each particle.
        
        Works by Chen, Chacón et al. \cite{Chen_Chacón_Barnes_2011, Chacón_Chen_Barnes_2013, Chen_Chacón_2014, Chen_Chacón_2015} have developed structure-preserving particle pushers for neoclassical transport in the Vlasov equations, derived from Crank--Nicolson integrators. We show these too can can derive from a FET interpretation, similarly offering potential extensions to higher-order-in-time particle pushers. The FET formulation is used also to consider how the stochastic drift terms can be incorporated into the pushers. Stochastic gyrokinetic expansions are also discussed.

        Different options for the numerical implementation of these schemes are considered.

        Due to the efficacy of FET in the development of SP timesteppers for both the fluid and kinetic component, we hope this approach will prove effective in the future for developing SP timesteppers for the full hybrid model. We hope this will give us the opportunity to incorporate previously inaccessible kinetic effects into the highly effective, modern, finite-element MHD models.
    \end{abstract}
    
    
    \newpage
    \tableofcontents
    
    
    \newpage
    \pagenumbering{arabic}
    %\linenumbers\renewcommand\thelinenumber{\color{black!50}\arabic{linenumber}}
            \input{0 - introduction/main.tex}
        \part{Research}
            \input{1 - low-noise PiC models/main.tex}
            \input{2 - kinetic component/main.tex}
            \input{3 - fluid component/main.tex}
            \input{4 - numerical implementation/main.tex}
        \part{Project Overview}
            \input{5 - research plan/main.tex}
            \input{6 - summary/main.tex}
    
    
    %\section{}
    \newpage
    \pagenumbering{gobble}
        \printbibliography


    \newpage
    \pagenumbering{roman}
    \appendix
        \part{Appendices}
            \input{8 - Hilbert complexes/main.tex}
            \input{9 - weak conservation proofs/main.tex}
\end{document}

        \part{Research}
            \documentclass[12pt, a4paper]{report}

\input{template/main.tex}

\title{\BA{Title in Progress...}}
\author{Boris Andrews}
\affil{Mathematical Institute, University of Oxford}
\date{\today}


\begin{document}
    \pagenumbering{gobble}
    \maketitle
    
    
    \begin{abstract}
        Magnetic confinement reactors---in particular tokamaks---offer one of the most promising options for achieving practical nuclear fusion, with the potential to provide virtually limitless, clean energy. The theoretical and numerical modeling of tokamak plasmas is simultaneously an essential component of effective reactor design, and a great research barrier. Tokamak operational conditions exhibit comparatively low Knudsen numbers. Kinetic effects, including kinetic waves and instabilities, Landau damping, bump-on-tail instabilities and more, are therefore highly influential in tokamak plasma dynamics. Purely fluid models are inherently incapable of capturing these effects, whereas the high dimensionality in purely kinetic models render them practically intractable for most relevant purposes.

        We consider a $\delta\!f$ decomposition model, with a macroscopic fluid background and microscopic kinetic correction, both fully coupled to each other. A similar manner of discretization is proposed to that used in the recent \texttt{STRUPHY} code \cite{Holderied_Possanner_Wang_2021, Holderied_2022, Li_et_al_2023} with a finite-element model for the background and a pseudo-particle/PiC model for the correction.

        The fluid background satisfies the full, non-linear, resistive, compressible, Hall MHD equations. \cite{Laakmann_Hu_Farrell_2022} introduces finite-element(-in-space) implicit timesteppers for the incompressible analogue to this system with structure-preserving (SP) properties in the ideal case, alongside parameter-robust preconditioners. We show that these timesteppers can derive from a finite-element-in-time (FET) (and finite-element-in-space) interpretation. The benefits of this reformulation are discussed, including the derivation of timesteppers that are higher order in time, and the quantifiable dissipative SP properties in the non-ideal, resistive case.
        
        We discuss possible options for extending this FET approach to timesteppers for the compressible case.

        The kinetic corrections satisfy linearized Boltzmann equations. Using a Lénard--Bernstein collision operator, these take Fokker--Planck-like forms \cite{Fokker_1914, Planck_1917} wherein pseudo-particles in the numerical model obey the neoclassical transport equations, with particle-independent Brownian drift terms. This offers a rigorous methodology for incorporating collisions into the particle transport model, without coupling the equations of motions for each particle.
        
        Works by Chen, Chacón et al. \cite{Chen_Chacón_Barnes_2011, Chacón_Chen_Barnes_2013, Chen_Chacón_2014, Chen_Chacón_2015} have developed structure-preserving particle pushers for neoclassical transport in the Vlasov equations, derived from Crank--Nicolson integrators. We show these too can can derive from a FET interpretation, similarly offering potential extensions to higher-order-in-time particle pushers. The FET formulation is used also to consider how the stochastic drift terms can be incorporated into the pushers. Stochastic gyrokinetic expansions are also discussed.

        Different options for the numerical implementation of these schemes are considered.

        Due to the efficacy of FET in the development of SP timesteppers for both the fluid and kinetic component, we hope this approach will prove effective in the future for developing SP timesteppers for the full hybrid model. We hope this will give us the opportunity to incorporate previously inaccessible kinetic effects into the highly effective, modern, finite-element MHD models.
    \end{abstract}
    
    
    \newpage
    \tableofcontents
    
    
    \newpage
    \pagenumbering{arabic}
    %\linenumbers\renewcommand\thelinenumber{\color{black!50}\arabic{linenumber}}
            \input{0 - introduction/main.tex}
        \part{Research}
            \input{1 - low-noise PiC models/main.tex}
            \input{2 - kinetic component/main.tex}
            \input{3 - fluid component/main.tex}
            \input{4 - numerical implementation/main.tex}
        \part{Project Overview}
            \input{5 - research plan/main.tex}
            \input{6 - summary/main.tex}
    
    
    %\section{}
    \newpage
    \pagenumbering{gobble}
        \printbibliography


    \newpage
    \pagenumbering{roman}
    \appendix
        \part{Appendices}
            \input{8 - Hilbert complexes/main.tex}
            \input{9 - weak conservation proofs/main.tex}
\end{document}

            \documentclass[12pt, a4paper]{report}

\input{template/main.tex}

\title{\BA{Title in Progress...}}
\author{Boris Andrews}
\affil{Mathematical Institute, University of Oxford}
\date{\today}


\begin{document}
    \pagenumbering{gobble}
    \maketitle
    
    
    \begin{abstract}
        Magnetic confinement reactors---in particular tokamaks---offer one of the most promising options for achieving practical nuclear fusion, with the potential to provide virtually limitless, clean energy. The theoretical and numerical modeling of tokamak plasmas is simultaneously an essential component of effective reactor design, and a great research barrier. Tokamak operational conditions exhibit comparatively low Knudsen numbers. Kinetic effects, including kinetic waves and instabilities, Landau damping, bump-on-tail instabilities and more, are therefore highly influential in tokamak plasma dynamics. Purely fluid models are inherently incapable of capturing these effects, whereas the high dimensionality in purely kinetic models render them practically intractable for most relevant purposes.

        We consider a $\delta\!f$ decomposition model, with a macroscopic fluid background and microscopic kinetic correction, both fully coupled to each other. A similar manner of discretization is proposed to that used in the recent \texttt{STRUPHY} code \cite{Holderied_Possanner_Wang_2021, Holderied_2022, Li_et_al_2023} with a finite-element model for the background and a pseudo-particle/PiC model for the correction.

        The fluid background satisfies the full, non-linear, resistive, compressible, Hall MHD equations. \cite{Laakmann_Hu_Farrell_2022} introduces finite-element(-in-space) implicit timesteppers for the incompressible analogue to this system with structure-preserving (SP) properties in the ideal case, alongside parameter-robust preconditioners. We show that these timesteppers can derive from a finite-element-in-time (FET) (and finite-element-in-space) interpretation. The benefits of this reformulation are discussed, including the derivation of timesteppers that are higher order in time, and the quantifiable dissipative SP properties in the non-ideal, resistive case.
        
        We discuss possible options for extending this FET approach to timesteppers for the compressible case.

        The kinetic corrections satisfy linearized Boltzmann equations. Using a Lénard--Bernstein collision operator, these take Fokker--Planck-like forms \cite{Fokker_1914, Planck_1917} wherein pseudo-particles in the numerical model obey the neoclassical transport equations, with particle-independent Brownian drift terms. This offers a rigorous methodology for incorporating collisions into the particle transport model, without coupling the equations of motions for each particle.
        
        Works by Chen, Chacón et al. \cite{Chen_Chacón_Barnes_2011, Chacón_Chen_Barnes_2013, Chen_Chacón_2014, Chen_Chacón_2015} have developed structure-preserving particle pushers for neoclassical transport in the Vlasov equations, derived from Crank--Nicolson integrators. We show these too can can derive from a FET interpretation, similarly offering potential extensions to higher-order-in-time particle pushers. The FET formulation is used also to consider how the stochastic drift terms can be incorporated into the pushers. Stochastic gyrokinetic expansions are also discussed.

        Different options for the numerical implementation of these schemes are considered.

        Due to the efficacy of FET in the development of SP timesteppers for both the fluid and kinetic component, we hope this approach will prove effective in the future for developing SP timesteppers for the full hybrid model. We hope this will give us the opportunity to incorporate previously inaccessible kinetic effects into the highly effective, modern, finite-element MHD models.
    \end{abstract}
    
    
    \newpage
    \tableofcontents
    
    
    \newpage
    \pagenumbering{arabic}
    %\linenumbers\renewcommand\thelinenumber{\color{black!50}\arabic{linenumber}}
            \input{0 - introduction/main.tex}
        \part{Research}
            \input{1 - low-noise PiC models/main.tex}
            \input{2 - kinetic component/main.tex}
            \input{3 - fluid component/main.tex}
            \input{4 - numerical implementation/main.tex}
        \part{Project Overview}
            \input{5 - research plan/main.tex}
            \input{6 - summary/main.tex}
    
    
    %\section{}
    \newpage
    \pagenumbering{gobble}
        \printbibliography


    \newpage
    \pagenumbering{roman}
    \appendix
        \part{Appendices}
            \input{8 - Hilbert complexes/main.tex}
            \input{9 - weak conservation proofs/main.tex}
\end{document}

            \documentclass[12pt, a4paper]{report}

\input{template/main.tex}

\title{\BA{Title in Progress...}}
\author{Boris Andrews}
\affil{Mathematical Institute, University of Oxford}
\date{\today}


\begin{document}
    \pagenumbering{gobble}
    \maketitle
    
    
    \begin{abstract}
        Magnetic confinement reactors---in particular tokamaks---offer one of the most promising options for achieving practical nuclear fusion, with the potential to provide virtually limitless, clean energy. The theoretical and numerical modeling of tokamak plasmas is simultaneously an essential component of effective reactor design, and a great research barrier. Tokamak operational conditions exhibit comparatively low Knudsen numbers. Kinetic effects, including kinetic waves and instabilities, Landau damping, bump-on-tail instabilities and more, are therefore highly influential in tokamak plasma dynamics. Purely fluid models are inherently incapable of capturing these effects, whereas the high dimensionality in purely kinetic models render them practically intractable for most relevant purposes.

        We consider a $\delta\!f$ decomposition model, with a macroscopic fluid background and microscopic kinetic correction, both fully coupled to each other. A similar manner of discretization is proposed to that used in the recent \texttt{STRUPHY} code \cite{Holderied_Possanner_Wang_2021, Holderied_2022, Li_et_al_2023} with a finite-element model for the background and a pseudo-particle/PiC model for the correction.

        The fluid background satisfies the full, non-linear, resistive, compressible, Hall MHD equations. \cite{Laakmann_Hu_Farrell_2022} introduces finite-element(-in-space) implicit timesteppers for the incompressible analogue to this system with structure-preserving (SP) properties in the ideal case, alongside parameter-robust preconditioners. We show that these timesteppers can derive from a finite-element-in-time (FET) (and finite-element-in-space) interpretation. The benefits of this reformulation are discussed, including the derivation of timesteppers that are higher order in time, and the quantifiable dissipative SP properties in the non-ideal, resistive case.
        
        We discuss possible options for extending this FET approach to timesteppers for the compressible case.

        The kinetic corrections satisfy linearized Boltzmann equations. Using a Lénard--Bernstein collision operator, these take Fokker--Planck-like forms \cite{Fokker_1914, Planck_1917} wherein pseudo-particles in the numerical model obey the neoclassical transport equations, with particle-independent Brownian drift terms. This offers a rigorous methodology for incorporating collisions into the particle transport model, without coupling the equations of motions for each particle.
        
        Works by Chen, Chacón et al. \cite{Chen_Chacón_Barnes_2011, Chacón_Chen_Barnes_2013, Chen_Chacón_2014, Chen_Chacón_2015} have developed structure-preserving particle pushers for neoclassical transport in the Vlasov equations, derived from Crank--Nicolson integrators. We show these too can can derive from a FET interpretation, similarly offering potential extensions to higher-order-in-time particle pushers. The FET formulation is used also to consider how the stochastic drift terms can be incorporated into the pushers. Stochastic gyrokinetic expansions are also discussed.

        Different options for the numerical implementation of these schemes are considered.

        Due to the efficacy of FET in the development of SP timesteppers for both the fluid and kinetic component, we hope this approach will prove effective in the future for developing SP timesteppers for the full hybrid model. We hope this will give us the opportunity to incorporate previously inaccessible kinetic effects into the highly effective, modern, finite-element MHD models.
    \end{abstract}
    
    
    \newpage
    \tableofcontents
    
    
    \newpage
    \pagenumbering{arabic}
    %\linenumbers\renewcommand\thelinenumber{\color{black!50}\arabic{linenumber}}
            \input{0 - introduction/main.tex}
        \part{Research}
            \input{1 - low-noise PiC models/main.tex}
            \input{2 - kinetic component/main.tex}
            \input{3 - fluid component/main.tex}
            \input{4 - numerical implementation/main.tex}
        \part{Project Overview}
            \input{5 - research plan/main.tex}
            \input{6 - summary/main.tex}
    
    
    %\section{}
    \newpage
    \pagenumbering{gobble}
        \printbibliography


    \newpage
    \pagenumbering{roman}
    \appendix
        \part{Appendices}
            \input{8 - Hilbert complexes/main.tex}
            \input{9 - weak conservation proofs/main.tex}
\end{document}

            \documentclass[12pt, a4paper]{report}

\input{template/main.tex}

\title{\BA{Title in Progress...}}
\author{Boris Andrews}
\affil{Mathematical Institute, University of Oxford}
\date{\today}


\begin{document}
    \pagenumbering{gobble}
    \maketitle
    
    
    \begin{abstract}
        Magnetic confinement reactors---in particular tokamaks---offer one of the most promising options for achieving practical nuclear fusion, with the potential to provide virtually limitless, clean energy. The theoretical and numerical modeling of tokamak plasmas is simultaneously an essential component of effective reactor design, and a great research barrier. Tokamak operational conditions exhibit comparatively low Knudsen numbers. Kinetic effects, including kinetic waves and instabilities, Landau damping, bump-on-tail instabilities and more, are therefore highly influential in tokamak plasma dynamics. Purely fluid models are inherently incapable of capturing these effects, whereas the high dimensionality in purely kinetic models render them practically intractable for most relevant purposes.

        We consider a $\delta\!f$ decomposition model, with a macroscopic fluid background and microscopic kinetic correction, both fully coupled to each other. A similar manner of discretization is proposed to that used in the recent \texttt{STRUPHY} code \cite{Holderied_Possanner_Wang_2021, Holderied_2022, Li_et_al_2023} with a finite-element model for the background and a pseudo-particle/PiC model for the correction.

        The fluid background satisfies the full, non-linear, resistive, compressible, Hall MHD equations. \cite{Laakmann_Hu_Farrell_2022} introduces finite-element(-in-space) implicit timesteppers for the incompressible analogue to this system with structure-preserving (SP) properties in the ideal case, alongside parameter-robust preconditioners. We show that these timesteppers can derive from a finite-element-in-time (FET) (and finite-element-in-space) interpretation. The benefits of this reformulation are discussed, including the derivation of timesteppers that are higher order in time, and the quantifiable dissipative SP properties in the non-ideal, resistive case.
        
        We discuss possible options for extending this FET approach to timesteppers for the compressible case.

        The kinetic corrections satisfy linearized Boltzmann equations. Using a Lénard--Bernstein collision operator, these take Fokker--Planck-like forms \cite{Fokker_1914, Planck_1917} wherein pseudo-particles in the numerical model obey the neoclassical transport equations, with particle-independent Brownian drift terms. This offers a rigorous methodology for incorporating collisions into the particle transport model, without coupling the equations of motions for each particle.
        
        Works by Chen, Chacón et al. \cite{Chen_Chacón_Barnes_2011, Chacón_Chen_Barnes_2013, Chen_Chacón_2014, Chen_Chacón_2015} have developed structure-preserving particle pushers for neoclassical transport in the Vlasov equations, derived from Crank--Nicolson integrators. We show these too can can derive from a FET interpretation, similarly offering potential extensions to higher-order-in-time particle pushers. The FET formulation is used also to consider how the stochastic drift terms can be incorporated into the pushers. Stochastic gyrokinetic expansions are also discussed.

        Different options for the numerical implementation of these schemes are considered.

        Due to the efficacy of FET in the development of SP timesteppers for both the fluid and kinetic component, we hope this approach will prove effective in the future for developing SP timesteppers for the full hybrid model. We hope this will give us the opportunity to incorporate previously inaccessible kinetic effects into the highly effective, modern, finite-element MHD models.
    \end{abstract}
    
    
    \newpage
    \tableofcontents
    
    
    \newpage
    \pagenumbering{arabic}
    %\linenumbers\renewcommand\thelinenumber{\color{black!50}\arabic{linenumber}}
            \input{0 - introduction/main.tex}
        \part{Research}
            \input{1 - low-noise PiC models/main.tex}
            \input{2 - kinetic component/main.tex}
            \input{3 - fluid component/main.tex}
            \input{4 - numerical implementation/main.tex}
        \part{Project Overview}
            \input{5 - research plan/main.tex}
            \input{6 - summary/main.tex}
    
    
    %\section{}
    \newpage
    \pagenumbering{gobble}
        \printbibliography


    \newpage
    \pagenumbering{roman}
    \appendix
        \part{Appendices}
            \input{8 - Hilbert complexes/main.tex}
            \input{9 - weak conservation proofs/main.tex}
\end{document}

        \part{Project Overview}
            \documentclass[12pt, a4paper]{report}

\input{template/main.tex}

\title{\BA{Title in Progress...}}
\author{Boris Andrews}
\affil{Mathematical Institute, University of Oxford}
\date{\today}


\begin{document}
    \pagenumbering{gobble}
    \maketitle
    
    
    \begin{abstract}
        Magnetic confinement reactors---in particular tokamaks---offer one of the most promising options for achieving practical nuclear fusion, with the potential to provide virtually limitless, clean energy. The theoretical and numerical modeling of tokamak plasmas is simultaneously an essential component of effective reactor design, and a great research barrier. Tokamak operational conditions exhibit comparatively low Knudsen numbers. Kinetic effects, including kinetic waves and instabilities, Landau damping, bump-on-tail instabilities and more, are therefore highly influential in tokamak plasma dynamics. Purely fluid models are inherently incapable of capturing these effects, whereas the high dimensionality in purely kinetic models render them practically intractable for most relevant purposes.

        We consider a $\delta\!f$ decomposition model, with a macroscopic fluid background and microscopic kinetic correction, both fully coupled to each other. A similar manner of discretization is proposed to that used in the recent \texttt{STRUPHY} code \cite{Holderied_Possanner_Wang_2021, Holderied_2022, Li_et_al_2023} with a finite-element model for the background and a pseudo-particle/PiC model for the correction.

        The fluid background satisfies the full, non-linear, resistive, compressible, Hall MHD equations. \cite{Laakmann_Hu_Farrell_2022} introduces finite-element(-in-space) implicit timesteppers for the incompressible analogue to this system with structure-preserving (SP) properties in the ideal case, alongside parameter-robust preconditioners. We show that these timesteppers can derive from a finite-element-in-time (FET) (and finite-element-in-space) interpretation. The benefits of this reformulation are discussed, including the derivation of timesteppers that are higher order in time, and the quantifiable dissipative SP properties in the non-ideal, resistive case.
        
        We discuss possible options for extending this FET approach to timesteppers for the compressible case.

        The kinetic corrections satisfy linearized Boltzmann equations. Using a Lénard--Bernstein collision operator, these take Fokker--Planck-like forms \cite{Fokker_1914, Planck_1917} wherein pseudo-particles in the numerical model obey the neoclassical transport equations, with particle-independent Brownian drift terms. This offers a rigorous methodology for incorporating collisions into the particle transport model, without coupling the equations of motions for each particle.
        
        Works by Chen, Chacón et al. \cite{Chen_Chacón_Barnes_2011, Chacón_Chen_Barnes_2013, Chen_Chacón_2014, Chen_Chacón_2015} have developed structure-preserving particle pushers for neoclassical transport in the Vlasov equations, derived from Crank--Nicolson integrators. We show these too can can derive from a FET interpretation, similarly offering potential extensions to higher-order-in-time particle pushers. The FET formulation is used also to consider how the stochastic drift terms can be incorporated into the pushers. Stochastic gyrokinetic expansions are also discussed.

        Different options for the numerical implementation of these schemes are considered.

        Due to the efficacy of FET in the development of SP timesteppers for both the fluid and kinetic component, we hope this approach will prove effective in the future for developing SP timesteppers for the full hybrid model. We hope this will give us the opportunity to incorporate previously inaccessible kinetic effects into the highly effective, modern, finite-element MHD models.
    \end{abstract}
    
    
    \newpage
    \tableofcontents
    
    
    \newpage
    \pagenumbering{arabic}
    %\linenumbers\renewcommand\thelinenumber{\color{black!50}\arabic{linenumber}}
            \input{0 - introduction/main.tex}
        \part{Research}
            \input{1 - low-noise PiC models/main.tex}
            \input{2 - kinetic component/main.tex}
            \input{3 - fluid component/main.tex}
            \input{4 - numerical implementation/main.tex}
        \part{Project Overview}
            \input{5 - research plan/main.tex}
            \input{6 - summary/main.tex}
    
    
    %\section{}
    \newpage
    \pagenumbering{gobble}
        \printbibliography


    \newpage
    \pagenumbering{roman}
    \appendix
        \part{Appendices}
            \input{8 - Hilbert complexes/main.tex}
            \input{9 - weak conservation proofs/main.tex}
\end{document}

            \documentclass[12pt, a4paper]{report}

\input{template/main.tex}

\title{\BA{Title in Progress...}}
\author{Boris Andrews}
\affil{Mathematical Institute, University of Oxford}
\date{\today}


\begin{document}
    \pagenumbering{gobble}
    \maketitle
    
    
    \begin{abstract}
        Magnetic confinement reactors---in particular tokamaks---offer one of the most promising options for achieving practical nuclear fusion, with the potential to provide virtually limitless, clean energy. The theoretical and numerical modeling of tokamak plasmas is simultaneously an essential component of effective reactor design, and a great research barrier. Tokamak operational conditions exhibit comparatively low Knudsen numbers. Kinetic effects, including kinetic waves and instabilities, Landau damping, bump-on-tail instabilities and more, are therefore highly influential in tokamak plasma dynamics. Purely fluid models are inherently incapable of capturing these effects, whereas the high dimensionality in purely kinetic models render them practically intractable for most relevant purposes.

        We consider a $\delta\!f$ decomposition model, with a macroscopic fluid background and microscopic kinetic correction, both fully coupled to each other. A similar manner of discretization is proposed to that used in the recent \texttt{STRUPHY} code \cite{Holderied_Possanner_Wang_2021, Holderied_2022, Li_et_al_2023} with a finite-element model for the background and a pseudo-particle/PiC model for the correction.

        The fluid background satisfies the full, non-linear, resistive, compressible, Hall MHD equations. \cite{Laakmann_Hu_Farrell_2022} introduces finite-element(-in-space) implicit timesteppers for the incompressible analogue to this system with structure-preserving (SP) properties in the ideal case, alongside parameter-robust preconditioners. We show that these timesteppers can derive from a finite-element-in-time (FET) (and finite-element-in-space) interpretation. The benefits of this reformulation are discussed, including the derivation of timesteppers that are higher order in time, and the quantifiable dissipative SP properties in the non-ideal, resistive case.
        
        We discuss possible options for extending this FET approach to timesteppers for the compressible case.

        The kinetic corrections satisfy linearized Boltzmann equations. Using a Lénard--Bernstein collision operator, these take Fokker--Planck-like forms \cite{Fokker_1914, Planck_1917} wherein pseudo-particles in the numerical model obey the neoclassical transport equations, with particle-independent Brownian drift terms. This offers a rigorous methodology for incorporating collisions into the particle transport model, without coupling the equations of motions for each particle.
        
        Works by Chen, Chacón et al. \cite{Chen_Chacón_Barnes_2011, Chacón_Chen_Barnes_2013, Chen_Chacón_2014, Chen_Chacón_2015} have developed structure-preserving particle pushers for neoclassical transport in the Vlasov equations, derived from Crank--Nicolson integrators. We show these too can can derive from a FET interpretation, similarly offering potential extensions to higher-order-in-time particle pushers. The FET formulation is used also to consider how the stochastic drift terms can be incorporated into the pushers. Stochastic gyrokinetic expansions are also discussed.

        Different options for the numerical implementation of these schemes are considered.

        Due to the efficacy of FET in the development of SP timesteppers for both the fluid and kinetic component, we hope this approach will prove effective in the future for developing SP timesteppers for the full hybrid model. We hope this will give us the opportunity to incorporate previously inaccessible kinetic effects into the highly effective, modern, finite-element MHD models.
    \end{abstract}
    
    
    \newpage
    \tableofcontents
    
    
    \newpage
    \pagenumbering{arabic}
    %\linenumbers\renewcommand\thelinenumber{\color{black!50}\arabic{linenumber}}
            \input{0 - introduction/main.tex}
        \part{Research}
            \input{1 - low-noise PiC models/main.tex}
            \input{2 - kinetic component/main.tex}
            \input{3 - fluid component/main.tex}
            \input{4 - numerical implementation/main.tex}
        \part{Project Overview}
            \input{5 - research plan/main.tex}
            \input{6 - summary/main.tex}
    
    
    %\section{}
    \newpage
    \pagenumbering{gobble}
        \printbibliography


    \newpage
    \pagenumbering{roman}
    \appendix
        \part{Appendices}
            \input{8 - Hilbert complexes/main.tex}
            \input{9 - weak conservation proofs/main.tex}
\end{document}

    
    
    %\section{}
    \newpage
    \pagenumbering{gobble}
        \printbibliography


    \newpage
    \pagenumbering{roman}
    \appendix
        \part{Appendices}
            \documentclass[12pt, a4paper]{report}

\input{template/main.tex}

\title{\BA{Title in Progress...}}
\author{Boris Andrews}
\affil{Mathematical Institute, University of Oxford}
\date{\today}


\begin{document}
    \pagenumbering{gobble}
    \maketitle
    
    
    \begin{abstract}
        Magnetic confinement reactors---in particular tokamaks---offer one of the most promising options for achieving practical nuclear fusion, with the potential to provide virtually limitless, clean energy. The theoretical and numerical modeling of tokamak plasmas is simultaneously an essential component of effective reactor design, and a great research barrier. Tokamak operational conditions exhibit comparatively low Knudsen numbers. Kinetic effects, including kinetic waves and instabilities, Landau damping, bump-on-tail instabilities and more, are therefore highly influential in tokamak plasma dynamics. Purely fluid models are inherently incapable of capturing these effects, whereas the high dimensionality in purely kinetic models render them practically intractable for most relevant purposes.

        We consider a $\delta\!f$ decomposition model, with a macroscopic fluid background and microscopic kinetic correction, both fully coupled to each other. A similar manner of discretization is proposed to that used in the recent \texttt{STRUPHY} code \cite{Holderied_Possanner_Wang_2021, Holderied_2022, Li_et_al_2023} with a finite-element model for the background and a pseudo-particle/PiC model for the correction.

        The fluid background satisfies the full, non-linear, resistive, compressible, Hall MHD equations. \cite{Laakmann_Hu_Farrell_2022} introduces finite-element(-in-space) implicit timesteppers for the incompressible analogue to this system with structure-preserving (SP) properties in the ideal case, alongside parameter-robust preconditioners. We show that these timesteppers can derive from a finite-element-in-time (FET) (and finite-element-in-space) interpretation. The benefits of this reformulation are discussed, including the derivation of timesteppers that are higher order in time, and the quantifiable dissipative SP properties in the non-ideal, resistive case.
        
        We discuss possible options for extending this FET approach to timesteppers for the compressible case.

        The kinetic corrections satisfy linearized Boltzmann equations. Using a Lénard--Bernstein collision operator, these take Fokker--Planck-like forms \cite{Fokker_1914, Planck_1917} wherein pseudo-particles in the numerical model obey the neoclassical transport equations, with particle-independent Brownian drift terms. This offers a rigorous methodology for incorporating collisions into the particle transport model, without coupling the equations of motions for each particle.
        
        Works by Chen, Chacón et al. \cite{Chen_Chacón_Barnes_2011, Chacón_Chen_Barnes_2013, Chen_Chacón_2014, Chen_Chacón_2015} have developed structure-preserving particle pushers for neoclassical transport in the Vlasov equations, derived from Crank--Nicolson integrators. We show these too can can derive from a FET interpretation, similarly offering potential extensions to higher-order-in-time particle pushers. The FET formulation is used also to consider how the stochastic drift terms can be incorporated into the pushers. Stochastic gyrokinetic expansions are also discussed.

        Different options for the numerical implementation of these schemes are considered.

        Due to the efficacy of FET in the development of SP timesteppers for both the fluid and kinetic component, we hope this approach will prove effective in the future for developing SP timesteppers for the full hybrid model. We hope this will give us the opportunity to incorporate previously inaccessible kinetic effects into the highly effective, modern, finite-element MHD models.
    \end{abstract}
    
    
    \newpage
    \tableofcontents
    
    
    \newpage
    \pagenumbering{arabic}
    %\linenumbers\renewcommand\thelinenumber{\color{black!50}\arabic{linenumber}}
            \input{0 - introduction/main.tex}
        \part{Research}
            \input{1 - low-noise PiC models/main.tex}
            \input{2 - kinetic component/main.tex}
            \input{3 - fluid component/main.tex}
            \input{4 - numerical implementation/main.tex}
        \part{Project Overview}
            \input{5 - research plan/main.tex}
            \input{6 - summary/main.tex}
    
    
    %\section{}
    \newpage
    \pagenumbering{gobble}
        \printbibliography


    \newpage
    \pagenumbering{roman}
    \appendix
        \part{Appendices}
            \input{8 - Hilbert complexes/main.tex}
            \input{9 - weak conservation proofs/main.tex}
\end{document}

            \documentclass[12pt, a4paper]{report}

\input{template/main.tex}

\title{\BA{Title in Progress...}}
\author{Boris Andrews}
\affil{Mathematical Institute, University of Oxford}
\date{\today}


\begin{document}
    \pagenumbering{gobble}
    \maketitle
    
    
    \begin{abstract}
        Magnetic confinement reactors---in particular tokamaks---offer one of the most promising options for achieving practical nuclear fusion, with the potential to provide virtually limitless, clean energy. The theoretical and numerical modeling of tokamak plasmas is simultaneously an essential component of effective reactor design, and a great research barrier. Tokamak operational conditions exhibit comparatively low Knudsen numbers. Kinetic effects, including kinetic waves and instabilities, Landau damping, bump-on-tail instabilities and more, are therefore highly influential in tokamak plasma dynamics. Purely fluid models are inherently incapable of capturing these effects, whereas the high dimensionality in purely kinetic models render them practically intractable for most relevant purposes.

        We consider a $\delta\!f$ decomposition model, with a macroscopic fluid background and microscopic kinetic correction, both fully coupled to each other. A similar manner of discretization is proposed to that used in the recent \texttt{STRUPHY} code \cite{Holderied_Possanner_Wang_2021, Holderied_2022, Li_et_al_2023} with a finite-element model for the background and a pseudo-particle/PiC model for the correction.

        The fluid background satisfies the full, non-linear, resistive, compressible, Hall MHD equations. \cite{Laakmann_Hu_Farrell_2022} introduces finite-element(-in-space) implicit timesteppers for the incompressible analogue to this system with structure-preserving (SP) properties in the ideal case, alongside parameter-robust preconditioners. We show that these timesteppers can derive from a finite-element-in-time (FET) (and finite-element-in-space) interpretation. The benefits of this reformulation are discussed, including the derivation of timesteppers that are higher order in time, and the quantifiable dissipative SP properties in the non-ideal, resistive case.
        
        We discuss possible options for extending this FET approach to timesteppers for the compressible case.

        The kinetic corrections satisfy linearized Boltzmann equations. Using a Lénard--Bernstein collision operator, these take Fokker--Planck-like forms \cite{Fokker_1914, Planck_1917} wherein pseudo-particles in the numerical model obey the neoclassical transport equations, with particle-independent Brownian drift terms. This offers a rigorous methodology for incorporating collisions into the particle transport model, without coupling the equations of motions for each particle.
        
        Works by Chen, Chacón et al. \cite{Chen_Chacón_Barnes_2011, Chacón_Chen_Barnes_2013, Chen_Chacón_2014, Chen_Chacón_2015} have developed structure-preserving particle pushers for neoclassical transport in the Vlasov equations, derived from Crank--Nicolson integrators. We show these too can can derive from a FET interpretation, similarly offering potential extensions to higher-order-in-time particle pushers. The FET formulation is used also to consider how the stochastic drift terms can be incorporated into the pushers. Stochastic gyrokinetic expansions are also discussed.

        Different options for the numerical implementation of these schemes are considered.

        Due to the efficacy of FET in the development of SP timesteppers for both the fluid and kinetic component, we hope this approach will prove effective in the future for developing SP timesteppers for the full hybrid model. We hope this will give us the opportunity to incorporate previously inaccessible kinetic effects into the highly effective, modern, finite-element MHD models.
    \end{abstract}
    
    
    \newpage
    \tableofcontents
    
    
    \newpage
    \pagenumbering{arabic}
    %\linenumbers\renewcommand\thelinenumber{\color{black!50}\arabic{linenumber}}
            \input{0 - introduction/main.tex}
        \part{Research}
            \input{1 - low-noise PiC models/main.tex}
            \input{2 - kinetic component/main.tex}
            \input{3 - fluid component/main.tex}
            \input{4 - numerical implementation/main.tex}
        \part{Project Overview}
            \input{5 - research plan/main.tex}
            \input{6 - summary/main.tex}
    
    
    %\section{}
    \newpage
    \pagenumbering{gobble}
        \printbibliography


    \newpage
    \pagenumbering{roman}
    \appendix
        \part{Appendices}
            \input{8 - Hilbert complexes/main.tex}
            \input{9 - weak conservation proofs/main.tex}
\end{document}

\end{document}

            \documentclass[12pt, a4paper]{report}

\documentclass[12pt, a4paper]{report}

\input{template/main.tex}

\title{\BA{Title in Progress...}}
\author{Boris Andrews}
\affil{Mathematical Institute, University of Oxford}
\date{\today}


\begin{document}
    \pagenumbering{gobble}
    \maketitle
    
    
    \begin{abstract}
        Magnetic confinement reactors---in particular tokamaks---offer one of the most promising options for achieving practical nuclear fusion, with the potential to provide virtually limitless, clean energy. The theoretical and numerical modeling of tokamak plasmas is simultaneously an essential component of effective reactor design, and a great research barrier. Tokamak operational conditions exhibit comparatively low Knudsen numbers. Kinetic effects, including kinetic waves and instabilities, Landau damping, bump-on-tail instabilities and more, are therefore highly influential in tokamak plasma dynamics. Purely fluid models are inherently incapable of capturing these effects, whereas the high dimensionality in purely kinetic models render them practically intractable for most relevant purposes.

        We consider a $\delta\!f$ decomposition model, with a macroscopic fluid background and microscopic kinetic correction, both fully coupled to each other. A similar manner of discretization is proposed to that used in the recent \texttt{STRUPHY} code \cite{Holderied_Possanner_Wang_2021, Holderied_2022, Li_et_al_2023} with a finite-element model for the background and a pseudo-particle/PiC model for the correction.

        The fluid background satisfies the full, non-linear, resistive, compressible, Hall MHD equations. \cite{Laakmann_Hu_Farrell_2022} introduces finite-element(-in-space) implicit timesteppers for the incompressible analogue to this system with structure-preserving (SP) properties in the ideal case, alongside parameter-robust preconditioners. We show that these timesteppers can derive from a finite-element-in-time (FET) (and finite-element-in-space) interpretation. The benefits of this reformulation are discussed, including the derivation of timesteppers that are higher order in time, and the quantifiable dissipative SP properties in the non-ideal, resistive case.
        
        We discuss possible options for extending this FET approach to timesteppers for the compressible case.

        The kinetic corrections satisfy linearized Boltzmann equations. Using a Lénard--Bernstein collision operator, these take Fokker--Planck-like forms \cite{Fokker_1914, Planck_1917} wherein pseudo-particles in the numerical model obey the neoclassical transport equations, with particle-independent Brownian drift terms. This offers a rigorous methodology for incorporating collisions into the particle transport model, without coupling the equations of motions for each particle.
        
        Works by Chen, Chacón et al. \cite{Chen_Chacón_Barnes_2011, Chacón_Chen_Barnes_2013, Chen_Chacón_2014, Chen_Chacón_2015} have developed structure-preserving particle pushers for neoclassical transport in the Vlasov equations, derived from Crank--Nicolson integrators. We show these too can can derive from a FET interpretation, similarly offering potential extensions to higher-order-in-time particle pushers. The FET formulation is used also to consider how the stochastic drift terms can be incorporated into the pushers. Stochastic gyrokinetic expansions are also discussed.

        Different options for the numerical implementation of these schemes are considered.

        Due to the efficacy of FET in the development of SP timesteppers for both the fluid and kinetic component, we hope this approach will prove effective in the future for developing SP timesteppers for the full hybrid model. We hope this will give us the opportunity to incorporate previously inaccessible kinetic effects into the highly effective, modern, finite-element MHD models.
    \end{abstract}
    
    
    \newpage
    \tableofcontents
    
    
    \newpage
    \pagenumbering{arabic}
    %\linenumbers\renewcommand\thelinenumber{\color{black!50}\arabic{linenumber}}
            \input{0 - introduction/main.tex}
        \part{Research}
            \input{1 - low-noise PiC models/main.tex}
            \input{2 - kinetic component/main.tex}
            \input{3 - fluid component/main.tex}
            \input{4 - numerical implementation/main.tex}
        \part{Project Overview}
            \input{5 - research plan/main.tex}
            \input{6 - summary/main.tex}
    
    
    %\section{}
    \newpage
    \pagenumbering{gobble}
        \printbibliography


    \newpage
    \pagenumbering{roman}
    \appendix
        \part{Appendices}
            \input{8 - Hilbert complexes/main.tex}
            \input{9 - weak conservation proofs/main.tex}
\end{document}


\title{\BA{Title in Progress...}}
\author{Boris Andrews}
\affil{Mathematical Institute, University of Oxford}
\date{\today}


\begin{document}
    \pagenumbering{gobble}
    \maketitle
    
    
    \begin{abstract}
        Magnetic confinement reactors---in particular tokamaks---offer one of the most promising options for achieving practical nuclear fusion, with the potential to provide virtually limitless, clean energy. The theoretical and numerical modeling of tokamak plasmas is simultaneously an essential component of effective reactor design, and a great research barrier. Tokamak operational conditions exhibit comparatively low Knudsen numbers. Kinetic effects, including kinetic waves and instabilities, Landau damping, bump-on-tail instabilities and more, are therefore highly influential in tokamak plasma dynamics. Purely fluid models are inherently incapable of capturing these effects, whereas the high dimensionality in purely kinetic models render them practically intractable for most relevant purposes.

        We consider a $\delta\!f$ decomposition model, with a macroscopic fluid background and microscopic kinetic correction, both fully coupled to each other. A similar manner of discretization is proposed to that used in the recent \texttt{STRUPHY} code \cite{Holderied_Possanner_Wang_2021, Holderied_2022, Li_et_al_2023} with a finite-element model for the background and a pseudo-particle/PiC model for the correction.

        The fluid background satisfies the full, non-linear, resistive, compressible, Hall MHD equations. \cite{Laakmann_Hu_Farrell_2022} introduces finite-element(-in-space) implicit timesteppers for the incompressible analogue to this system with structure-preserving (SP) properties in the ideal case, alongside parameter-robust preconditioners. We show that these timesteppers can derive from a finite-element-in-time (FET) (and finite-element-in-space) interpretation. The benefits of this reformulation are discussed, including the derivation of timesteppers that are higher order in time, and the quantifiable dissipative SP properties in the non-ideal, resistive case.
        
        We discuss possible options for extending this FET approach to timesteppers for the compressible case.

        The kinetic corrections satisfy linearized Boltzmann equations. Using a Lénard--Bernstein collision operator, these take Fokker--Planck-like forms \cite{Fokker_1914, Planck_1917} wherein pseudo-particles in the numerical model obey the neoclassical transport equations, with particle-independent Brownian drift terms. This offers a rigorous methodology for incorporating collisions into the particle transport model, without coupling the equations of motions for each particle.
        
        Works by Chen, Chacón et al. \cite{Chen_Chacón_Barnes_2011, Chacón_Chen_Barnes_2013, Chen_Chacón_2014, Chen_Chacón_2015} have developed structure-preserving particle pushers for neoclassical transport in the Vlasov equations, derived from Crank--Nicolson integrators. We show these too can can derive from a FET interpretation, similarly offering potential extensions to higher-order-in-time particle pushers. The FET formulation is used also to consider how the stochastic drift terms can be incorporated into the pushers. Stochastic gyrokinetic expansions are also discussed.

        Different options for the numerical implementation of these schemes are considered.

        Due to the efficacy of FET in the development of SP timesteppers for both the fluid and kinetic component, we hope this approach will prove effective in the future for developing SP timesteppers for the full hybrid model. We hope this will give us the opportunity to incorporate previously inaccessible kinetic effects into the highly effective, modern, finite-element MHD models.
    \end{abstract}
    
    
    \newpage
    \tableofcontents
    
    
    \newpage
    \pagenumbering{arabic}
    %\linenumbers\renewcommand\thelinenumber{\color{black!50}\arabic{linenumber}}
            \documentclass[12pt, a4paper]{report}

\input{template/main.tex}

\title{\BA{Title in Progress...}}
\author{Boris Andrews}
\affil{Mathematical Institute, University of Oxford}
\date{\today}


\begin{document}
    \pagenumbering{gobble}
    \maketitle
    
    
    \begin{abstract}
        Magnetic confinement reactors---in particular tokamaks---offer one of the most promising options for achieving practical nuclear fusion, with the potential to provide virtually limitless, clean energy. The theoretical and numerical modeling of tokamak plasmas is simultaneously an essential component of effective reactor design, and a great research barrier. Tokamak operational conditions exhibit comparatively low Knudsen numbers. Kinetic effects, including kinetic waves and instabilities, Landau damping, bump-on-tail instabilities and more, are therefore highly influential in tokamak plasma dynamics. Purely fluid models are inherently incapable of capturing these effects, whereas the high dimensionality in purely kinetic models render them practically intractable for most relevant purposes.

        We consider a $\delta\!f$ decomposition model, with a macroscopic fluid background and microscopic kinetic correction, both fully coupled to each other. A similar manner of discretization is proposed to that used in the recent \texttt{STRUPHY} code \cite{Holderied_Possanner_Wang_2021, Holderied_2022, Li_et_al_2023} with a finite-element model for the background and a pseudo-particle/PiC model for the correction.

        The fluid background satisfies the full, non-linear, resistive, compressible, Hall MHD equations. \cite{Laakmann_Hu_Farrell_2022} introduces finite-element(-in-space) implicit timesteppers for the incompressible analogue to this system with structure-preserving (SP) properties in the ideal case, alongside parameter-robust preconditioners. We show that these timesteppers can derive from a finite-element-in-time (FET) (and finite-element-in-space) interpretation. The benefits of this reformulation are discussed, including the derivation of timesteppers that are higher order in time, and the quantifiable dissipative SP properties in the non-ideal, resistive case.
        
        We discuss possible options for extending this FET approach to timesteppers for the compressible case.

        The kinetic corrections satisfy linearized Boltzmann equations. Using a Lénard--Bernstein collision operator, these take Fokker--Planck-like forms \cite{Fokker_1914, Planck_1917} wherein pseudo-particles in the numerical model obey the neoclassical transport equations, with particle-independent Brownian drift terms. This offers a rigorous methodology for incorporating collisions into the particle transport model, without coupling the equations of motions for each particle.
        
        Works by Chen, Chacón et al. \cite{Chen_Chacón_Barnes_2011, Chacón_Chen_Barnes_2013, Chen_Chacón_2014, Chen_Chacón_2015} have developed structure-preserving particle pushers for neoclassical transport in the Vlasov equations, derived from Crank--Nicolson integrators. We show these too can can derive from a FET interpretation, similarly offering potential extensions to higher-order-in-time particle pushers. The FET formulation is used also to consider how the stochastic drift terms can be incorporated into the pushers. Stochastic gyrokinetic expansions are also discussed.

        Different options for the numerical implementation of these schemes are considered.

        Due to the efficacy of FET in the development of SP timesteppers for both the fluid and kinetic component, we hope this approach will prove effective in the future for developing SP timesteppers for the full hybrid model. We hope this will give us the opportunity to incorporate previously inaccessible kinetic effects into the highly effective, modern, finite-element MHD models.
    \end{abstract}
    
    
    \newpage
    \tableofcontents
    
    
    \newpage
    \pagenumbering{arabic}
    %\linenumbers\renewcommand\thelinenumber{\color{black!50}\arabic{linenumber}}
            \input{0 - introduction/main.tex}
        \part{Research}
            \input{1 - low-noise PiC models/main.tex}
            \input{2 - kinetic component/main.tex}
            \input{3 - fluid component/main.tex}
            \input{4 - numerical implementation/main.tex}
        \part{Project Overview}
            \input{5 - research plan/main.tex}
            \input{6 - summary/main.tex}
    
    
    %\section{}
    \newpage
    \pagenumbering{gobble}
        \printbibliography


    \newpage
    \pagenumbering{roman}
    \appendix
        \part{Appendices}
            \input{8 - Hilbert complexes/main.tex}
            \input{9 - weak conservation proofs/main.tex}
\end{document}

        \part{Research}
            \documentclass[12pt, a4paper]{report}

\input{template/main.tex}

\title{\BA{Title in Progress...}}
\author{Boris Andrews}
\affil{Mathematical Institute, University of Oxford}
\date{\today}


\begin{document}
    \pagenumbering{gobble}
    \maketitle
    
    
    \begin{abstract}
        Magnetic confinement reactors---in particular tokamaks---offer one of the most promising options for achieving practical nuclear fusion, with the potential to provide virtually limitless, clean energy. The theoretical and numerical modeling of tokamak plasmas is simultaneously an essential component of effective reactor design, and a great research barrier. Tokamak operational conditions exhibit comparatively low Knudsen numbers. Kinetic effects, including kinetic waves and instabilities, Landau damping, bump-on-tail instabilities and more, are therefore highly influential in tokamak plasma dynamics. Purely fluid models are inherently incapable of capturing these effects, whereas the high dimensionality in purely kinetic models render them practically intractable for most relevant purposes.

        We consider a $\delta\!f$ decomposition model, with a macroscopic fluid background and microscopic kinetic correction, both fully coupled to each other. A similar manner of discretization is proposed to that used in the recent \texttt{STRUPHY} code \cite{Holderied_Possanner_Wang_2021, Holderied_2022, Li_et_al_2023} with a finite-element model for the background and a pseudo-particle/PiC model for the correction.

        The fluid background satisfies the full, non-linear, resistive, compressible, Hall MHD equations. \cite{Laakmann_Hu_Farrell_2022} introduces finite-element(-in-space) implicit timesteppers for the incompressible analogue to this system with structure-preserving (SP) properties in the ideal case, alongside parameter-robust preconditioners. We show that these timesteppers can derive from a finite-element-in-time (FET) (and finite-element-in-space) interpretation. The benefits of this reformulation are discussed, including the derivation of timesteppers that are higher order in time, and the quantifiable dissipative SP properties in the non-ideal, resistive case.
        
        We discuss possible options for extending this FET approach to timesteppers for the compressible case.

        The kinetic corrections satisfy linearized Boltzmann equations. Using a Lénard--Bernstein collision operator, these take Fokker--Planck-like forms \cite{Fokker_1914, Planck_1917} wherein pseudo-particles in the numerical model obey the neoclassical transport equations, with particle-independent Brownian drift terms. This offers a rigorous methodology for incorporating collisions into the particle transport model, without coupling the equations of motions for each particle.
        
        Works by Chen, Chacón et al. \cite{Chen_Chacón_Barnes_2011, Chacón_Chen_Barnes_2013, Chen_Chacón_2014, Chen_Chacón_2015} have developed structure-preserving particle pushers for neoclassical transport in the Vlasov equations, derived from Crank--Nicolson integrators. We show these too can can derive from a FET interpretation, similarly offering potential extensions to higher-order-in-time particle pushers. The FET formulation is used also to consider how the stochastic drift terms can be incorporated into the pushers. Stochastic gyrokinetic expansions are also discussed.

        Different options for the numerical implementation of these schemes are considered.

        Due to the efficacy of FET in the development of SP timesteppers for both the fluid and kinetic component, we hope this approach will prove effective in the future for developing SP timesteppers for the full hybrid model. We hope this will give us the opportunity to incorporate previously inaccessible kinetic effects into the highly effective, modern, finite-element MHD models.
    \end{abstract}
    
    
    \newpage
    \tableofcontents
    
    
    \newpage
    \pagenumbering{arabic}
    %\linenumbers\renewcommand\thelinenumber{\color{black!50}\arabic{linenumber}}
            \input{0 - introduction/main.tex}
        \part{Research}
            \input{1 - low-noise PiC models/main.tex}
            \input{2 - kinetic component/main.tex}
            \input{3 - fluid component/main.tex}
            \input{4 - numerical implementation/main.tex}
        \part{Project Overview}
            \input{5 - research plan/main.tex}
            \input{6 - summary/main.tex}
    
    
    %\section{}
    \newpage
    \pagenumbering{gobble}
        \printbibliography


    \newpage
    \pagenumbering{roman}
    \appendix
        \part{Appendices}
            \input{8 - Hilbert complexes/main.tex}
            \input{9 - weak conservation proofs/main.tex}
\end{document}

            \documentclass[12pt, a4paper]{report}

\input{template/main.tex}

\title{\BA{Title in Progress...}}
\author{Boris Andrews}
\affil{Mathematical Institute, University of Oxford}
\date{\today}


\begin{document}
    \pagenumbering{gobble}
    \maketitle
    
    
    \begin{abstract}
        Magnetic confinement reactors---in particular tokamaks---offer one of the most promising options for achieving practical nuclear fusion, with the potential to provide virtually limitless, clean energy. The theoretical and numerical modeling of tokamak plasmas is simultaneously an essential component of effective reactor design, and a great research barrier. Tokamak operational conditions exhibit comparatively low Knudsen numbers. Kinetic effects, including kinetic waves and instabilities, Landau damping, bump-on-tail instabilities and more, are therefore highly influential in tokamak plasma dynamics. Purely fluid models are inherently incapable of capturing these effects, whereas the high dimensionality in purely kinetic models render them practically intractable for most relevant purposes.

        We consider a $\delta\!f$ decomposition model, with a macroscopic fluid background and microscopic kinetic correction, both fully coupled to each other. A similar manner of discretization is proposed to that used in the recent \texttt{STRUPHY} code \cite{Holderied_Possanner_Wang_2021, Holderied_2022, Li_et_al_2023} with a finite-element model for the background and a pseudo-particle/PiC model for the correction.

        The fluid background satisfies the full, non-linear, resistive, compressible, Hall MHD equations. \cite{Laakmann_Hu_Farrell_2022} introduces finite-element(-in-space) implicit timesteppers for the incompressible analogue to this system with structure-preserving (SP) properties in the ideal case, alongside parameter-robust preconditioners. We show that these timesteppers can derive from a finite-element-in-time (FET) (and finite-element-in-space) interpretation. The benefits of this reformulation are discussed, including the derivation of timesteppers that are higher order in time, and the quantifiable dissipative SP properties in the non-ideal, resistive case.
        
        We discuss possible options for extending this FET approach to timesteppers for the compressible case.

        The kinetic corrections satisfy linearized Boltzmann equations. Using a Lénard--Bernstein collision operator, these take Fokker--Planck-like forms \cite{Fokker_1914, Planck_1917} wherein pseudo-particles in the numerical model obey the neoclassical transport equations, with particle-independent Brownian drift terms. This offers a rigorous methodology for incorporating collisions into the particle transport model, without coupling the equations of motions for each particle.
        
        Works by Chen, Chacón et al. \cite{Chen_Chacón_Barnes_2011, Chacón_Chen_Barnes_2013, Chen_Chacón_2014, Chen_Chacón_2015} have developed structure-preserving particle pushers for neoclassical transport in the Vlasov equations, derived from Crank--Nicolson integrators. We show these too can can derive from a FET interpretation, similarly offering potential extensions to higher-order-in-time particle pushers. The FET formulation is used also to consider how the stochastic drift terms can be incorporated into the pushers. Stochastic gyrokinetic expansions are also discussed.

        Different options for the numerical implementation of these schemes are considered.

        Due to the efficacy of FET in the development of SP timesteppers for both the fluid and kinetic component, we hope this approach will prove effective in the future for developing SP timesteppers for the full hybrid model. We hope this will give us the opportunity to incorporate previously inaccessible kinetic effects into the highly effective, modern, finite-element MHD models.
    \end{abstract}
    
    
    \newpage
    \tableofcontents
    
    
    \newpage
    \pagenumbering{arabic}
    %\linenumbers\renewcommand\thelinenumber{\color{black!50}\arabic{linenumber}}
            \input{0 - introduction/main.tex}
        \part{Research}
            \input{1 - low-noise PiC models/main.tex}
            \input{2 - kinetic component/main.tex}
            \input{3 - fluid component/main.tex}
            \input{4 - numerical implementation/main.tex}
        \part{Project Overview}
            \input{5 - research plan/main.tex}
            \input{6 - summary/main.tex}
    
    
    %\section{}
    \newpage
    \pagenumbering{gobble}
        \printbibliography


    \newpage
    \pagenumbering{roman}
    \appendix
        \part{Appendices}
            \input{8 - Hilbert complexes/main.tex}
            \input{9 - weak conservation proofs/main.tex}
\end{document}

            \documentclass[12pt, a4paper]{report}

\input{template/main.tex}

\title{\BA{Title in Progress...}}
\author{Boris Andrews}
\affil{Mathematical Institute, University of Oxford}
\date{\today}


\begin{document}
    \pagenumbering{gobble}
    \maketitle
    
    
    \begin{abstract}
        Magnetic confinement reactors---in particular tokamaks---offer one of the most promising options for achieving practical nuclear fusion, with the potential to provide virtually limitless, clean energy. The theoretical and numerical modeling of tokamak plasmas is simultaneously an essential component of effective reactor design, and a great research barrier. Tokamak operational conditions exhibit comparatively low Knudsen numbers. Kinetic effects, including kinetic waves and instabilities, Landau damping, bump-on-tail instabilities and more, are therefore highly influential in tokamak plasma dynamics. Purely fluid models are inherently incapable of capturing these effects, whereas the high dimensionality in purely kinetic models render them practically intractable for most relevant purposes.

        We consider a $\delta\!f$ decomposition model, with a macroscopic fluid background and microscopic kinetic correction, both fully coupled to each other. A similar manner of discretization is proposed to that used in the recent \texttt{STRUPHY} code \cite{Holderied_Possanner_Wang_2021, Holderied_2022, Li_et_al_2023} with a finite-element model for the background and a pseudo-particle/PiC model for the correction.

        The fluid background satisfies the full, non-linear, resistive, compressible, Hall MHD equations. \cite{Laakmann_Hu_Farrell_2022} introduces finite-element(-in-space) implicit timesteppers for the incompressible analogue to this system with structure-preserving (SP) properties in the ideal case, alongside parameter-robust preconditioners. We show that these timesteppers can derive from a finite-element-in-time (FET) (and finite-element-in-space) interpretation. The benefits of this reformulation are discussed, including the derivation of timesteppers that are higher order in time, and the quantifiable dissipative SP properties in the non-ideal, resistive case.
        
        We discuss possible options for extending this FET approach to timesteppers for the compressible case.

        The kinetic corrections satisfy linearized Boltzmann equations. Using a Lénard--Bernstein collision operator, these take Fokker--Planck-like forms \cite{Fokker_1914, Planck_1917} wherein pseudo-particles in the numerical model obey the neoclassical transport equations, with particle-independent Brownian drift terms. This offers a rigorous methodology for incorporating collisions into the particle transport model, without coupling the equations of motions for each particle.
        
        Works by Chen, Chacón et al. \cite{Chen_Chacón_Barnes_2011, Chacón_Chen_Barnes_2013, Chen_Chacón_2014, Chen_Chacón_2015} have developed structure-preserving particle pushers for neoclassical transport in the Vlasov equations, derived from Crank--Nicolson integrators. We show these too can can derive from a FET interpretation, similarly offering potential extensions to higher-order-in-time particle pushers. The FET formulation is used also to consider how the stochastic drift terms can be incorporated into the pushers. Stochastic gyrokinetic expansions are also discussed.

        Different options for the numerical implementation of these schemes are considered.

        Due to the efficacy of FET in the development of SP timesteppers for both the fluid and kinetic component, we hope this approach will prove effective in the future for developing SP timesteppers for the full hybrid model. We hope this will give us the opportunity to incorporate previously inaccessible kinetic effects into the highly effective, modern, finite-element MHD models.
    \end{abstract}
    
    
    \newpage
    \tableofcontents
    
    
    \newpage
    \pagenumbering{arabic}
    %\linenumbers\renewcommand\thelinenumber{\color{black!50}\arabic{linenumber}}
            \input{0 - introduction/main.tex}
        \part{Research}
            \input{1 - low-noise PiC models/main.tex}
            \input{2 - kinetic component/main.tex}
            \input{3 - fluid component/main.tex}
            \input{4 - numerical implementation/main.tex}
        \part{Project Overview}
            \input{5 - research plan/main.tex}
            \input{6 - summary/main.tex}
    
    
    %\section{}
    \newpage
    \pagenumbering{gobble}
        \printbibliography


    \newpage
    \pagenumbering{roman}
    \appendix
        \part{Appendices}
            \input{8 - Hilbert complexes/main.tex}
            \input{9 - weak conservation proofs/main.tex}
\end{document}

            \documentclass[12pt, a4paper]{report}

\input{template/main.tex}

\title{\BA{Title in Progress...}}
\author{Boris Andrews}
\affil{Mathematical Institute, University of Oxford}
\date{\today}


\begin{document}
    \pagenumbering{gobble}
    \maketitle
    
    
    \begin{abstract}
        Magnetic confinement reactors---in particular tokamaks---offer one of the most promising options for achieving practical nuclear fusion, with the potential to provide virtually limitless, clean energy. The theoretical and numerical modeling of tokamak plasmas is simultaneously an essential component of effective reactor design, and a great research barrier. Tokamak operational conditions exhibit comparatively low Knudsen numbers. Kinetic effects, including kinetic waves and instabilities, Landau damping, bump-on-tail instabilities and more, are therefore highly influential in tokamak plasma dynamics. Purely fluid models are inherently incapable of capturing these effects, whereas the high dimensionality in purely kinetic models render them practically intractable for most relevant purposes.

        We consider a $\delta\!f$ decomposition model, with a macroscopic fluid background and microscopic kinetic correction, both fully coupled to each other. A similar manner of discretization is proposed to that used in the recent \texttt{STRUPHY} code \cite{Holderied_Possanner_Wang_2021, Holderied_2022, Li_et_al_2023} with a finite-element model for the background and a pseudo-particle/PiC model for the correction.

        The fluid background satisfies the full, non-linear, resistive, compressible, Hall MHD equations. \cite{Laakmann_Hu_Farrell_2022} introduces finite-element(-in-space) implicit timesteppers for the incompressible analogue to this system with structure-preserving (SP) properties in the ideal case, alongside parameter-robust preconditioners. We show that these timesteppers can derive from a finite-element-in-time (FET) (and finite-element-in-space) interpretation. The benefits of this reformulation are discussed, including the derivation of timesteppers that are higher order in time, and the quantifiable dissipative SP properties in the non-ideal, resistive case.
        
        We discuss possible options for extending this FET approach to timesteppers for the compressible case.

        The kinetic corrections satisfy linearized Boltzmann equations. Using a Lénard--Bernstein collision operator, these take Fokker--Planck-like forms \cite{Fokker_1914, Planck_1917} wherein pseudo-particles in the numerical model obey the neoclassical transport equations, with particle-independent Brownian drift terms. This offers a rigorous methodology for incorporating collisions into the particle transport model, without coupling the equations of motions for each particle.
        
        Works by Chen, Chacón et al. \cite{Chen_Chacón_Barnes_2011, Chacón_Chen_Barnes_2013, Chen_Chacón_2014, Chen_Chacón_2015} have developed structure-preserving particle pushers for neoclassical transport in the Vlasov equations, derived from Crank--Nicolson integrators. We show these too can can derive from a FET interpretation, similarly offering potential extensions to higher-order-in-time particle pushers. The FET formulation is used also to consider how the stochastic drift terms can be incorporated into the pushers. Stochastic gyrokinetic expansions are also discussed.

        Different options for the numerical implementation of these schemes are considered.

        Due to the efficacy of FET in the development of SP timesteppers for both the fluid and kinetic component, we hope this approach will prove effective in the future for developing SP timesteppers for the full hybrid model. We hope this will give us the opportunity to incorporate previously inaccessible kinetic effects into the highly effective, modern, finite-element MHD models.
    \end{abstract}
    
    
    \newpage
    \tableofcontents
    
    
    \newpage
    \pagenumbering{arabic}
    %\linenumbers\renewcommand\thelinenumber{\color{black!50}\arabic{linenumber}}
            \input{0 - introduction/main.tex}
        \part{Research}
            \input{1 - low-noise PiC models/main.tex}
            \input{2 - kinetic component/main.tex}
            \input{3 - fluid component/main.tex}
            \input{4 - numerical implementation/main.tex}
        \part{Project Overview}
            \input{5 - research plan/main.tex}
            \input{6 - summary/main.tex}
    
    
    %\section{}
    \newpage
    \pagenumbering{gobble}
        \printbibliography


    \newpage
    \pagenumbering{roman}
    \appendix
        \part{Appendices}
            \input{8 - Hilbert complexes/main.tex}
            \input{9 - weak conservation proofs/main.tex}
\end{document}

        \part{Project Overview}
            \documentclass[12pt, a4paper]{report}

\input{template/main.tex}

\title{\BA{Title in Progress...}}
\author{Boris Andrews}
\affil{Mathematical Institute, University of Oxford}
\date{\today}


\begin{document}
    \pagenumbering{gobble}
    \maketitle
    
    
    \begin{abstract}
        Magnetic confinement reactors---in particular tokamaks---offer one of the most promising options for achieving practical nuclear fusion, with the potential to provide virtually limitless, clean energy. The theoretical and numerical modeling of tokamak plasmas is simultaneously an essential component of effective reactor design, and a great research barrier. Tokamak operational conditions exhibit comparatively low Knudsen numbers. Kinetic effects, including kinetic waves and instabilities, Landau damping, bump-on-tail instabilities and more, are therefore highly influential in tokamak plasma dynamics. Purely fluid models are inherently incapable of capturing these effects, whereas the high dimensionality in purely kinetic models render them practically intractable for most relevant purposes.

        We consider a $\delta\!f$ decomposition model, with a macroscopic fluid background and microscopic kinetic correction, both fully coupled to each other. A similar manner of discretization is proposed to that used in the recent \texttt{STRUPHY} code \cite{Holderied_Possanner_Wang_2021, Holderied_2022, Li_et_al_2023} with a finite-element model for the background and a pseudo-particle/PiC model for the correction.

        The fluid background satisfies the full, non-linear, resistive, compressible, Hall MHD equations. \cite{Laakmann_Hu_Farrell_2022} introduces finite-element(-in-space) implicit timesteppers for the incompressible analogue to this system with structure-preserving (SP) properties in the ideal case, alongside parameter-robust preconditioners. We show that these timesteppers can derive from a finite-element-in-time (FET) (and finite-element-in-space) interpretation. The benefits of this reformulation are discussed, including the derivation of timesteppers that are higher order in time, and the quantifiable dissipative SP properties in the non-ideal, resistive case.
        
        We discuss possible options for extending this FET approach to timesteppers for the compressible case.

        The kinetic corrections satisfy linearized Boltzmann equations. Using a Lénard--Bernstein collision operator, these take Fokker--Planck-like forms \cite{Fokker_1914, Planck_1917} wherein pseudo-particles in the numerical model obey the neoclassical transport equations, with particle-independent Brownian drift terms. This offers a rigorous methodology for incorporating collisions into the particle transport model, without coupling the equations of motions for each particle.
        
        Works by Chen, Chacón et al. \cite{Chen_Chacón_Barnes_2011, Chacón_Chen_Barnes_2013, Chen_Chacón_2014, Chen_Chacón_2015} have developed structure-preserving particle pushers for neoclassical transport in the Vlasov equations, derived from Crank--Nicolson integrators. We show these too can can derive from a FET interpretation, similarly offering potential extensions to higher-order-in-time particle pushers. The FET formulation is used also to consider how the stochastic drift terms can be incorporated into the pushers. Stochastic gyrokinetic expansions are also discussed.

        Different options for the numerical implementation of these schemes are considered.

        Due to the efficacy of FET in the development of SP timesteppers for both the fluid and kinetic component, we hope this approach will prove effective in the future for developing SP timesteppers for the full hybrid model. We hope this will give us the opportunity to incorporate previously inaccessible kinetic effects into the highly effective, modern, finite-element MHD models.
    \end{abstract}
    
    
    \newpage
    \tableofcontents
    
    
    \newpage
    \pagenumbering{arabic}
    %\linenumbers\renewcommand\thelinenumber{\color{black!50}\arabic{linenumber}}
            \input{0 - introduction/main.tex}
        \part{Research}
            \input{1 - low-noise PiC models/main.tex}
            \input{2 - kinetic component/main.tex}
            \input{3 - fluid component/main.tex}
            \input{4 - numerical implementation/main.tex}
        \part{Project Overview}
            \input{5 - research plan/main.tex}
            \input{6 - summary/main.tex}
    
    
    %\section{}
    \newpage
    \pagenumbering{gobble}
        \printbibliography


    \newpage
    \pagenumbering{roman}
    \appendix
        \part{Appendices}
            \input{8 - Hilbert complexes/main.tex}
            \input{9 - weak conservation proofs/main.tex}
\end{document}

            \documentclass[12pt, a4paper]{report}

\input{template/main.tex}

\title{\BA{Title in Progress...}}
\author{Boris Andrews}
\affil{Mathematical Institute, University of Oxford}
\date{\today}


\begin{document}
    \pagenumbering{gobble}
    \maketitle
    
    
    \begin{abstract}
        Magnetic confinement reactors---in particular tokamaks---offer one of the most promising options for achieving practical nuclear fusion, with the potential to provide virtually limitless, clean energy. The theoretical and numerical modeling of tokamak plasmas is simultaneously an essential component of effective reactor design, and a great research barrier. Tokamak operational conditions exhibit comparatively low Knudsen numbers. Kinetic effects, including kinetic waves and instabilities, Landau damping, bump-on-tail instabilities and more, are therefore highly influential in tokamak plasma dynamics. Purely fluid models are inherently incapable of capturing these effects, whereas the high dimensionality in purely kinetic models render them practically intractable for most relevant purposes.

        We consider a $\delta\!f$ decomposition model, with a macroscopic fluid background and microscopic kinetic correction, both fully coupled to each other. A similar manner of discretization is proposed to that used in the recent \texttt{STRUPHY} code \cite{Holderied_Possanner_Wang_2021, Holderied_2022, Li_et_al_2023} with a finite-element model for the background and a pseudo-particle/PiC model for the correction.

        The fluid background satisfies the full, non-linear, resistive, compressible, Hall MHD equations. \cite{Laakmann_Hu_Farrell_2022} introduces finite-element(-in-space) implicit timesteppers for the incompressible analogue to this system with structure-preserving (SP) properties in the ideal case, alongside parameter-robust preconditioners. We show that these timesteppers can derive from a finite-element-in-time (FET) (and finite-element-in-space) interpretation. The benefits of this reformulation are discussed, including the derivation of timesteppers that are higher order in time, and the quantifiable dissipative SP properties in the non-ideal, resistive case.
        
        We discuss possible options for extending this FET approach to timesteppers for the compressible case.

        The kinetic corrections satisfy linearized Boltzmann equations. Using a Lénard--Bernstein collision operator, these take Fokker--Planck-like forms \cite{Fokker_1914, Planck_1917} wherein pseudo-particles in the numerical model obey the neoclassical transport equations, with particle-independent Brownian drift terms. This offers a rigorous methodology for incorporating collisions into the particle transport model, without coupling the equations of motions for each particle.
        
        Works by Chen, Chacón et al. \cite{Chen_Chacón_Barnes_2011, Chacón_Chen_Barnes_2013, Chen_Chacón_2014, Chen_Chacón_2015} have developed structure-preserving particle pushers for neoclassical transport in the Vlasov equations, derived from Crank--Nicolson integrators. We show these too can can derive from a FET interpretation, similarly offering potential extensions to higher-order-in-time particle pushers. The FET formulation is used also to consider how the stochastic drift terms can be incorporated into the pushers. Stochastic gyrokinetic expansions are also discussed.

        Different options for the numerical implementation of these schemes are considered.

        Due to the efficacy of FET in the development of SP timesteppers for both the fluid and kinetic component, we hope this approach will prove effective in the future for developing SP timesteppers for the full hybrid model. We hope this will give us the opportunity to incorporate previously inaccessible kinetic effects into the highly effective, modern, finite-element MHD models.
    \end{abstract}
    
    
    \newpage
    \tableofcontents
    
    
    \newpage
    \pagenumbering{arabic}
    %\linenumbers\renewcommand\thelinenumber{\color{black!50}\arabic{linenumber}}
            \input{0 - introduction/main.tex}
        \part{Research}
            \input{1 - low-noise PiC models/main.tex}
            \input{2 - kinetic component/main.tex}
            \input{3 - fluid component/main.tex}
            \input{4 - numerical implementation/main.tex}
        \part{Project Overview}
            \input{5 - research plan/main.tex}
            \input{6 - summary/main.tex}
    
    
    %\section{}
    \newpage
    \pagenumbering{gobble}
        \printbibliography


    \newpage
    \pagenumbering{roman}
    \appendix
        \part{Appendices}
            \input{8 - Hilbert complexes/main.tex}
            \input{9 - weak conservation proofs/main.tex}
\end{document}

    
    
    %\section{}
    \newpage
    \pagenumbering{gobble}
        \printbibliography


    \newpage
    \pagenumbering{roman}
    \appendix
        \part{Appendices}
            \documentclass[12pt, a4paper]{report}

\input{template/main.tex}

\title{\BA{Title in Progress...}}
\author{Boris Andrews}
\affil{Mathematical Institute, University of Oxford}
\date{\today}


\begin{document}
    \pagenumbering{gobble}
    \maketitle
    
    
    \begin{abstract}
        Magnetic confinement reactors---in particular tokamaks---offer one of the most promising options for achieving practical nuclear fusion, with the potential to provide virtually limitless, clean energy. The theoretical and numerical modeling of tokamak plasmas is simultaneously an essential component of effective reactor design, and a great research barrier. Tokamak operational conditions exhibit comparatively low Knudsen numbers. Kinetic effects, including kinetic waves and instabilities, Landau damping, bump-on-tail instabilities and more, are therefore highly influential in tokamak plasma dynamics. Purely fluid models are inherently incapable of capturing these effects, whereas the high dimensionality in purely kinetic models render them practically intractable for most relevant purposes.

        We consider a $\delta\!f$ decomposition model, with a macroscopic fluid background and microscopic kinetic correction, both fully coupled to each other. A similar manner of discretization is proposed to that used in the recent \texttt{STRUPHY} code \cite{Holderied_Possanner_Wang_2021, Holderied_2022, Li_et_al_2023} with a finite-element model for the background and a pseudo-particle/PiC model for the correction.

        The fluid background satisfies the full, non-linear, resistive, compressible, Hall MHD equations. \cite{Laakmann_Hu_Farrell_2022} introduces finite-element(-in-space) implicit timesteppers for the incompressible analogue to this system with structure-preserving (SP) properties in the ideal case, alongside parameter-robust preconditioners. We show that these timesteppers can derive from a finite-element-in-time (FET) (and finite-element-in-space) interpretation. The benefits of this reformulation are discussed, including the derivation of timesteppers that are higher order in time, and the quantifiable dissipative SP properties in the non-ideal, resistive case.
        
        We discuss possible options for extending this FET approach to timesteppers for the compressible case.

        The kinetic corrections satisfy linearized Boltzmann equations. Using a Lénard--Bernstein collision operator, these take Fokker--Planck-like forms \cite{Fokker_1914, Planck_1917} wherein pseudo-particles in the numerical model obey the neoclassical transport equations, with particle-independent Brownian drift terms. This offers a rigorous methodology for incorporating collisions into the particle transport model, without coupling the equations of motions for each particle.
        
        Works by Chen, Chacón et al. \cite{Chen_Chacón_Barnes_2011, Chacón_Chen_Barnes_2013, Chen_Chacón_2014, Chen_Chacón_2015} have developed structure-preserving particle pushers for neoclassical transport in the Vlasov equations, derived from Crank--Nicolson integrators. We show these too can can derive from a FET interpretation, similarly offering potential extensions to higher-order-in-time particle pushers. The FET formulation is used also to consider how the stochastic drift terms can be incorporated into the pushers. Stochastic gyrokinetic expansions are also discussed.

        Different options for the numerical implementation of these schemes are considered.

        Due to the efficacy of FET in the development of SP timesteppers for both the fluid and kinetic component, we hope this approach will prove effective in the future for developing SP timesteppers for the full hybrid model. We hope this will give us the opportunity to incorporate previously inaccessible kinetic effects into the highly effective, modern, finite-element MHD models.
    \end{abstract}
    
    
    \newpage
    \tableofcontents
    
    
    \newpage
    \pagenumbering{arabic}
    %\linenumbers\renewcommand\thelinenumber{\color{black!50}\arabic{linenumber}}
            \input{0 - introduction/main.tex}
        \part{Research}
            \input{1 - low-noise PiC models/main.tex}
            \input{2 - kinetic component/main.tex}
            \input{3 - fluid component/main.tex}
            \input{4 - numerical implementation/main.tex}
        \part{Project Overview}
            \input{5 - research plan/main.tex}
            \input{6 - summary/main.tex}
    
    
    %\section{}
    \newpage
    \pagenumbering{gobble}
        \printbibliography


    \newpage
    \pagenumbering{roman}
    \appendix
        \part{Appendices}
            \input{8 - Hilbert complexes/main.tex}
            \input{9 - weak conservation proofs/main.tex}
\end{document}

            \documentclass[12pt, a4paper]{report}

\input{template/main.tex}

\title{\BA{Title in Progress...}}
\author{Boris Andrews}
\affil{Mathematical Institute, University of Oxford}
\date{\today}


\begin{document}
    \pagenumbering{gobble}
    \maketitle
    
    
    \begin{abstract}
        Magnetic confinement reactors---in particular tokamaks---offer one of the most promising options for achieving practical nuclear fusion, with the potential to provide virtually limitless, clean energy. The theoretical and numerical modeling of tokamak plasmas is simultaneously an essential component of effective reactor design, and a great research barrier. Tokamak operational conditions exhibit comparatively low Knudsen numbers. Kinetic effects, including kinetic waves and instabilities, Landau damping, bump-on-tail instabilities and more, are therefore highly influential in tokamak plasma dynamics. Purely fluid models are inherently incapable of capturing these effects, whereas the high dimensionality in purely kinetic models render them practically intractable for most relevant purposes.

        We consider a $\delta\!f$ decomposition model, with a macroscopic fluid background and microscopic kinetic correction, both fully coupled to each other. A similar manner of discretization is proposed to that used in the recent \texttt{STRUPHY} code \cite{Holderied_Possanner_Wang_2021, Holderied_2022, Li_et_al_2023} with a finite-element model for the background and a pseudo-particle/PiC model for the correction.

        The fluid background satisfies the full, non-linear, resistive, compressible, Hall MHD equations. \cite{Laakmann_Hu_Farrell_2022} introduces finite-element(-in-space) implicit timesteppers for the incompressible analogue to this system with structure-preserving (SP) properties in the ideal case, alongside parameter-robust preconditioners. We show that these timesteppers can derive from a finite-element-in-time (FET) (and finite-element-in-space) interpretation. The benefits of this reformulation are discussed, including the derivation of timesteppers that are higher order in time, and the quantifiable dissipative SP properties in the non-ideal, resistive case.
        
        We discuss possible options for extending this FET approach to timesteppers for the compressible case.

        The kinetic corrections satisfy linearized Boltzmann equations. Using a Lénard--Bernstein collision operator, these take Fokker--Planck-like forms \cite{Fokker_1914, Planck_1917} wherein pseudo-particles in the numerical model obey the neoclassical transport equations, with particle-independent Brownian drift terms. This offers a rigorous methodology for incorporating collisions into the particle transport model, without coupling the equations of motions for each particle.
        
        Works by Chen, Chacón et al. \cite{Chen_Chacón_Barnes_2011, Chacón_Chen_Barnes_2013, Chen_Chacón_2014, Chen_Chacón_2015} have developed structure-preserving particle pushers for neoclassical transport in the Vlasov equations, derived from Crank--Nicolson integrators. We show these too can can derive from a FET interpretation, similarly offering potential extensions to higher-order-in-time particle pushers. The FET formulation is used also to consider how the stochastic drift terms can be incorporated into the pushers. Stochastic gyrokinetic expansions are also discussed.

        Different options for the numerical implementation of these schemes are considered.

        Due to the efficacy of FET in the development of SP timesteppers for both the fluid and kinetic component, we hope this approach will prove effective in the future for developing SP timesteppers for the full hybrid model. We hope this will give us the opportunity to incorporate previously inaccessible kinetic effects into the highly effective, modern, finite-element MHD models.
    \end{abstract}
    
    
    \newpage
    \tableofcontents
    
    
    \newpage
    \pagenumbering{arabic}
    %\linenumbers\renewcommand\thelinenumber{\color{black!50}\arabic{linenumber}}
            \input{0 - introduction/main.tex}
        \part{Research}
            \input{1 - low-noise PiC models/main.tex}
            \input{2 - kinetic component/main.tex}
            \input{3 - fluid component/main.tex}
            \input{4 - numerical implementation/main.tex}
        \part{Project Overview}
            \input{5 - research plan/main.tex}
            \input{6 - summary/main.tex}
    
    
    %\section{}
    \newpage
    \pagenumbering{gobble}
        \printbibliography


    \newpage
    \pagenumbering{roman}
    \appendix
        \part{Appendices}
            \input{8 - Hilbert complexes/main.tex}
            \input{9 - weak conservation proofs/main.tex}
\end{document}

\end{document}

        \part{Project Overview}
            \documentclass[12pt, a4paper]{report}

\documentclass[12pt, a4paper]{report}

\input{template/main.tex}

\title{\BA{Title in Progress...}}
\author{Boris Andrews}
\affil{Mathematical Institute, University of Oxford}
\date{\today}


\begin{document}
    \pagenumbering{gobble}
    \maketitle
    
    
    \begin{abstract}
        Magnetic confinement reactors---in particular tokamaks---offer one of the most promising options for achieving practical nuclear fusion, with the potential to provide virtually limitless, clean energy. The theoretical and numerical modeling of tokamak plasmas is simultaneously an essential component of effective reactor design, and a great research barrier. Tokamak operational conditions exhibit comparatively low Knudsen numbers. Kinetic effects, including kinetic waves and instabilities, Landau damping, bump-on-tail instabilities and more, are therefore highly influential in tokamak plasma dynamics. Purely fluid models are inherently incapable of capturing these effects, whereas the high dimensionality in purely kinetic models render them practically intractable for most relevant purposes.

        We consider a $\delta\!f$ decomposition model, with a macroscopic fluid background and microscopic kinetic correction, both fully coupled to each other. A similar manner of discretization is proposed to that used in the recent \texttt{STRUPHY} code \cite{Holderied_Possanner_Wang_2021, Holderied_2022, Li_et_al_2023} with a finite-element model for the background and a pseudo-particle/PiC model for the correction.

        The fluid background satisfies the full, non-linear, resistive, compressible, Hall MHD equations. \cite{Laakmann_Hu_Farrell_2022} introduces finite-element(-in-space) implicit timesteppers for the incompressible analogue to this system with structure-preserving (SP) properties in the ideal case, alongside parameter-robust preconditioners. We show that these timesteppers can derive from a finite-element-in-time (FET) (and finite-element-in-space) interpretation. The benefits of this reformulation are discussed, including the derivation of timesteppers that are higher order in time, and the quantifiable dissipative SP properties in the non-ideal, resistive case.
        
        We discuss possible options for extending this FET approach to timesteppers for the compressible case.

        The kinetic corrections satisfy linearized Boltzmann equations. Using a Lénard--Bernstein collision operator, these take Fokker--Planck-like forms \cite{Fokker_1914, Planck_1917} wherein pseudo-particles in the numerical model obey the neoclassical transport equations, with particle-independent Brownian drift terms. This offers a rigorous methodology for incorporating collisions into the particle transport model, without coupling the equations of motions for each particle.
        
        Works by Chen, Chacón et al. \cite{Chen_Chacón_Barnes_2011, Chacón_Chen_Barnes_2013, Chen_Chacón_2014, Chen_Chacón_2015} have developed structure-preserving particle pushers for neoclassical transport in the Vlasov equations, derived from Crank--Nicolson integrators. We show these too can can derive from a FET interpretation, similarly offering potential extensions to higher-order-in-time particle pushers. The FET formulation is used also to consider how the stochastic drift terms can be incorporated into the pushers. Stochastic gyrokinetic expansions are also discussed.

        Different options for the numerical implementation of these schemes are considered.

        Due to the efficacy of FET in the development of SP timesteppers for both the fluid and kinetic component, we hope this approach will prove effective in the future for developing SP timesteppers for the full hybrid model. We hope this will give us the opportunity to incorporate previously inaccessible kinetic effects into the highly effective, modern, finite-element MHD models.
    \end{abstract}
    
    
    \newpage
    \tableofcontents
    
    
    \newpage
    \pagenumbering{arabic}
    %\linenumbers\renewcommand\thelinenumber{\color{black!50}\arabic{linenumber}}
            \input{0 - introduction/main.tex}
        \part{Research}
            \input{1 - low-noise PiC models/main.tex}
            \input{2 - kinetic component/main.tex}
            \input{3 - fluid component/main.tex}
            \input{4 - numerical implementation/main.tex}
        \part{Project Overview}
            \input{5 - research plan/main.tex}
            \input{6 - summary/main.tex}
    
    
    %\section{}
    \newpage
    \pagenumbering{gobble}
        \printbibliography


    \newpage
    \pagenumbering{roman}
    \appendix
        \part{Appendices}
            \input{8 - Hilbert complexes/main.tex}
            \input{9 - weak conservation proofs/main.tex}
\end{document}


\title{\BA{Title in Progress...}}
\author{Boris Andrews}
\affil{Mathematical Institute, University of Oxford}
\date{\today}


\begin{document}
    \pagenumbering{gobble}
    \maketitle
    
    
    \begin{abstract}
        Magnetic confinement reactors---in particular tokamaks---offer one of the most promising options for achieving practical nuclear fusion, with the potential to provide virtually limitless, clean energy. The theoretical and numerical modeling of tokamak plasmas is simultaneously an essential component of effective reactor design, and a great research barrier. Tokamak operational conditions exhibit comparatively low Knudsen numbers. Kinetic effects, including kinetic waves and instabilities, Landau damping, bump-on-tail instabilities and more, are therefore highly influential in tokamak plasma dynamics. Purely fluid models are inherently incapable of capturing these effects, whereas the high dimensionality in purely kinetic models render them practically intractable for most relevant purposes.

        We consider a $\delta\!f$ decomposition model, with a macroscopic fluid background and microscopic kinetic correction, both fully coupled to each other. A similar manner of discretization is proposed to that used in the recent \texttt{STRUPHY} code \cite{Holderied_Possanner_Wang_2021, Holderied_2022, Li_et_al_2023} with a finite-element model for the background and a pseudo-particle/PiC model for the correction.

        The fluid background satisfies the full, non-linear, resistive, compressible, Hall MHD equations. \cite{Laakmann_Hu_Farrell_2022} introduces finite-element(-in-space) implicit timesteppers for the incompressible analogue to this system with structure-preserving (SP) properties in the ideal case, alongside parameter-robust preconditioners. We show that these timesteppers can derive from a finite-element-in-time (FET) (and finite-element-in-space) interpretation. The benefits of this reformulation are discussed, including the derivation of timesteppers that are higher order in time, and the quantifiable dissipative SP properties in the non-ideal, resistive case.
        
        We discuss possible options for extending this FET approach to timesteppers for the compressible case.

        The kinetic corrections satisfy linearized Boltzmann equations. Using a Lénard--Bernstein collision operator, these take Fokker--Planck-like forms \cite{Fokker_1914, Planck_1917} wherein pseudo-particles in the numerical model obey the neoclassical transport equations, with particle-independent Brownian drift terms. This offers a rigorous methodology for incorporating collisions into the particle transport model, without coupling the equations of motions for each particle.
        
        Works by Chen, Chacón et al. \cite{Chen_Chacón_Barnes_2011, Chacón_Chen_Barnes_2013, Chen_Chacón_2014, Chen_Chacón_2015} have developed structure-preserving particle pushers for neoclassical transport in the Vlasov equations, derived from Crank--Nicolson integrators. We show these too can can derive from a FET interpretation, similarly offering potential extensions to higher-order-in-time particle pushers. The FET formulation is used also to consider how the stochastic drift terms can be incorporated into the pushers. Stochastic gyrokinetic expansions are also discussed.

        Different options for the numerical implementation of these schemes are considered.

        Due to the efficacy of FET in the development of SP timesteppers for both the fluid and kinetic component, we hope this approach will prove effective in the future for developing SP timesteppers for the full hybrid model. We hope this will give us the opportunity to incorporate previously inaccessible kinetic effects into the highly effective, modern, finite-element MHD models.
    \end{abstract}
    
    
    \newpage
    \tableofcontents
    
    
    \newpage
    \pagenumbering{arabic}
    %\linenumbers\renewcommand\thelinenumber{\color{black!50}\arabic{linenumber}}
            \documentclass[12pt, a4paper]{report}

\input{template/main.tex}

\title{\BA{Title in Progress...}}
\author{Boris Andrews}
\affil{Mathematical Institute, University of Oxford}
\date{\today}


\begin{document}
    \pagenumbering{gobble}
    \maketitle
    
    
    \begin{abstract}
        Magnetic confinement reactors---in particular tokamaks---offer one of the most promising options for achieving practical nuclear fusion, with the potential to provide virtually limitless, clean energy. The theoretical and numerical modeling of tokamak plasmas is simultaneously an essential component of effective reactor design, and a great research barrier. Tokamak operational conditions exhibit comparatively low Knudsen numbers. Kinetic effects, including kinetic waves and instabilities, Landau damping, bump-on-tail instabilities and more, are therefore highly influential in tokamak plasma dynamics. Purely fluid models are inherently incapable of capturing these effects, whereas the high dimensionality in purely kinetic models render them practically intractable for most relevant purposes.

        We consider a $\delta\!f$ decomposition model, with a macroscopic fluid background and microscopic kinetic correction, both fully coupled to each other. A similar manner of discretization is proposed to that used in the recent \texttt{STRUPHY} code \cite{Holderied_Possanner_Wang_2021, Holderied_2022, Li_et_al_2023} with a finite-element model for the background and a pseudo-particle/PiC model for the correction.

        The fluid background satisfies the full, non-linear, resistive, compressible, Hall MHD equations. \cite{Laakmann_Hu_Farrell_2022} introduces finite-element(-in-space) implicit timesteppers for the incompressible analogue to this system with structure-preserving (SP) properties in the ideal case, alongside parameter-robust preconditioners. We show that these timesteppers can derive from a finite-element-in-time (FET) (and finite-element-in-space) interpretation. The benefits of this reformulation are discussed, including the derivation of timesteppers that are higher order in time, and the quantifiable dissipative SP properties in the non-ideal, resistive case.
        
        We discuss possible options for extending this FET approach to timesteppers for the compressible case.

        The kinetic corrections satisfy linearized Boltzmann equations. Using a Lénard--Bernstein collision operator, these take Fokker--Planck-like forms \cite{Fokker_1914, Planck_1917} wherein pseudo-particles in the numerical model obey the neoclassical transport equations, with particle-independent Brownian drift terms. This offers a rigorous methodology for incorporating collisions into the particle transport model, without coupling the equations of motions for each particle.
        
        Works by Chen, Chacón et al. \cite{Chen_Chacón_Barnes_2011, Chacón_Chen_Barnes_2013, Chen_Chacón_2014, Chen_Chacón_2015} have developed structure-preserving particle pushers for neoclassical transport in the Vlasov equations, derived from Crank--Nicolson integrators. We show these too can can derive from a FET interpretation, similarly offering potential extensions to higher-order-in-time particle pushers. The FET formulation is used also to consider how the stochastic drift terms can be incorporated into the pushers. Stochastic gyrokinetic expansions are also discussed.

        Different options for the numerical implementation of these schemes are considered.

        Due to the efficacy of FET in the development of SP timesteppers for both the fluid and kinetic component, we hope this approach will prove effective in the future for developing SP timesteppers for the full hybrid model. We hope this will give us the opportunity to incorporate previously inaccessible kinetic effects into the highly effective, modern, finite-element MHD models.
    \end{abstract}
    
    
    \newpage
    \tableofcontents
    
    
    \newpage
    \pagenumbering{arabic}
    %\linenumbers\renewcommand\thelinenumber{\color{black!50}\arabic{linenumber}}
            \input{0 - introduction/main.tex}
        \part{Research}
            \input{1 - low-noise PiC models/main.tex}
            \input{2 - kinetic component/main.tex}
            \input{3 - fluid component/main.tex}
            \input{4 - numerical implementation/main.tex}
        \part{Project Overview}
            \input{5 - research plan/main.tex}
            \input{6 - summary/main.tex}
    
    
    %\section{}
    \newpage
    \pagenumbering{gobble}
        \printbibliography


    \newpage
    \pagenumbering{roman}
    \appendix
        \part{Appendices}
            \input{8 - Hilbert complexes/main.tex}
            \input{9 - weak conservation proofs/main.tex}
\end{document}

        \part{Research}
            \documentclass[12pt, a4paper]{report}

\input{template/main.tex}

\title{\BA{Title in Progress...}}
\author{Boris Andrews}
\affil{Mathematical Institute, University of Oxford}
\date{\today}


\begin{document}
    \pagenumbering{gobble}
    \maketitle
    
    
    \begin{abstract}
        Magnetic confinement reactors---in particular tokamaks---offer one of the most promising options for achieving practical nuclear fusion, with the potential to provide virtually limitless, clean energy. The theoretical and numerical modeling of tokamak plasmas is simultaneously an essential component of effective reactor design, and a great research barrier. Tokamak operational conditions exhibit comparatively low Knudsen numbers. Kinetic effects, including kinetic waves and instabilities, Landau damping, bump-on-tail instabilities and more, are therefore highly influential in tokamak plasma dynamics. Purely fluid models are inherently incapable of capturing these effects, whereas the high dimensionality in purely kinetic models render them practically intractable for most relevant purposes.

        We consider a $\delta\!f$ decomposition model, with a macroscopic fluid background and microscopic kinetic correction, both fully coupled to each other. A similar manner of discretization is proposed to that used in the recent \texttt{STRUPHY} code \cite{Holderied_Possanner_Wang_2021, Holderied_2022, Li_et_al_2023} with a finite-element model for the background and a pseudo-particle/PiC model for the correction.

        The fluid background satisfies the full, non-linear, resistive, compressible, Hall MHD equations. \cite{Laakmann_Hu_Farrell_2022} introduces finite-element(-in-space) implicit timesteppers for the incompressible analogue to this system with structure-preserving (SP) properties in the ideal case, alongside parameter-robust preconditioners. We show that these timesteppers can derive from a finite-element-in-time (FET) (and finite-element-in-space) interpretation. The benefits of this reformulation are discussed, including the derivation of timesteppers that are higher order in time, and the quantifiable dissipative SP properties in the non-ideal, resistive case.
        
        We discuss possible options for extending this FET approach to timesteppers for the compressible case.

        The kinetic corrections satisfy linearized Boltzmann equations. Using a Lénard--Bernstein collision operator, these take Fokker--Planck-like forms \cite{Fokker_1914, Planck_1917} wherein pseudo-particles in the numerical model obey the neoclassical transport equations, with particle-independent Brownian drift terms. This offers a rigorous methodology for incorporating collisions into the particle transport model, without coupling the equations of motions for each particle.
        
        Works by Chen, Chacón et al. \cite{Chen_Chacón_Barnes_2011, Chacón_Chen_Barnes_2013, Chen_Chacón_2014, Chen_Chacón_2015} have developed structure-preserving particle pushers for neoclassical transport in the Vlasov equations, derived from Crank--Nicolson integrators. We show these too can can derive from a FET interpretation, similarly offering potential extensions to higher-order-in-time particle pushers. The FET formulation is used also to consider how the stochastic drift terms can be incorporated into the pushers. Stochastic gyrokinetic expansions are also discussed.

        Different options for the numerical implementation of these schemes are considered.

        Due to the efficacy of FET in the development of SP timesteppers for both the fluid and kinetic component, we hope this approach will prove effective in the future for developing SP timesteppers for the full hybrid model. We hope this will give us the opportunity to incorporate previously inaccessible kinetic effects into the highly effective, modern, finite-element MHD models.
    \end{abstract}
    
    
    \newpage
    \tableofcontents
    
    
    \newpage
    \pagenumbering{arabic}
    %\linenumbers\renewcommand\thelinenumber{\color{black!50}\arabic{linenumber}}
            \input{0 - introduction/main.tex}
        \part{Research}
            \input{1 - low-noise PiC models/main.tex}
            \input{2 - kinetic component/main.tex}
            \input{3 - fluid component/main.tex}
            \input{4 - numerical implementation/main.tex}
        \part{Project Overview}
            \input{5 - research plan/main.tex}
            \input{6 - summary/main.tex}
    
    
    %\section{}
    \newpage
    \pagenumbering{gobble}
        \printbibliography


    \newpage
    \pagenumbering{roman}
    \appendix
        \part{Appendices}
            \input{8 - Hilbert complexes/main.tex}
            \input{9 - weak conservation proofs/main.tex}
\end{document}

            \documentclass[12pt, a4paper]{report}

\input{template/main.tex}

\title{\BA{Title in Progress...}}
\author{Boris Andrews}
\affil{Mathematical Institute, University of Oxford}
\date{\today}


\begin{document}
    \pagenumbering{gobble}
    \maketitle
    
    
    \begin{abstract}
        Magnetic confinement reactors---in particular tokamaks---offer one of the most promising options for achieving practical nuclear fusion, with the potential to provide virtually limitless, clean energy. The theoretical and numerical modeling of tokamak plasmas is simultaneously an essential component of effective reactor design, and a great research barrier. Tokamak operational conditions exhibit comparatively low Knudsen numbers. Kinetic effects, including kinetic waves and instabilities, Landau damping, bump-on-tail instabilities and more, are therefore highly influential in tokamak plasma dynamics. Purely fluid models are inherently incapable of capturing these effects, whereas the high dimensionality in purely kinetic models render them practically intractable for most relevant purposes.

        We consider a $\delta\!f$ decomposition model, with a macroscopic fluid background and microscopic kinetic correction, both fully coupled to each other. A similar manner of discretization is proposed to that used in the recent \texttt{STRUPHY} code \cite{Holderied_Possanner_Wang_2021, Holderied_2022, Li_et_al_2023} with a finite-element model for the background and a pseudo-particle/PiC model for the correction.

        The fluid background satisfies the full, non-linear, resistive, compressible, Hall MHD equations. \cite{Laakmann_Hu_Farrell_2022} introduces finite-element(-in-space) implicit timesteppers for the incompressible analogue to this system with structure-preserving (SP) properties in the ideal case, alongside parameter-robust preconditioners. We show that these timesteppers can derive from a finite-element-in-time (FET) (and finite-element-in-space) interpretation. The benefits of this reformulation are discussed, including the derivation of timesteppers that are higher order in time, and the quantifiable dissipative SP properties in the non-ideal, resistive case.
        
        We discuss possible options for extending this FET approach to timesteppers for the compressible case.

        The kinetic corrections satisfy linearized Boltzmann equations. Using a Lénard--Bernstein collision operator, these take Fokker--Planck-like forms \cite{Fokker_1914, Planck_1917} wherein pseudo-particles in the numerical model obey the neoclassical transport equations, with particle-independent Brownian drift terms. This offers a rigorous methodology for incorporating collisions into the particle transport model, without coupling the equations of motions for each particle.
        
        Works by Chen, Chacón et al. \cite{Chen_Chacón_Barnes_2011, Chacón_Chen_Barnes_2013, Chen_Chacón_2014, Chen_Chacón_2015} have developed structure-preserving particle pushers for neoclassical transport in the Vlasov equations, derived from Crank--Nicolson integrators. We show these too can can derive from a FET interpretation, similarly offering potential extensions to higher-order-in-time particle pushers. The FET formulation is used also to consider how the stochastic drift terms can be incorporated into the pushers. Stochastic gyrokinetic expansions are also discussed.

        Different options for the numerical implementation of these schemes are considered.

        Due to the efficacy of FET in the development of SP timesteppers for both the fluid and kinetic component, we hope this approach will prove effective in the future for developing SP timesteppers for the full hybrid model. We hope this will give us the opportunity to incorporate previously inaccessible kinetic effects into the highly effective, modern, finite-element MHD models.
    \end{abstract}
    
    
    \newpage
    \tableofcontents
    
    
    \newpage
    \pagenumbering{arabic}
    %\linenumbers\renewcommand\thelinenumber{\color{black!50}\arabic{linenumber}}
            \input{0 - introduction/main.tex}
        \part{Research}
            \input{1 - low-noise PiC models/main.tex}
            \input{2 - kinetic component/main.tex}
            \input{3 - fluid component/main.tex}
            \input{4 - numerical implementation/main.tex}
        \part{Project Overview}
            \input{5 - research plan/main.tex}
            \input{6 - summary/main.tex}
    
    
    %\section{}
    \newpage
    \pagenumbering{gobble}
        \printbibliography


    \newpage
    \pagenumbering{roman}
    \appendix
        \part{Appendices}
            \input{8 - Hilbert complexes/main.tex}
            \input{9 - weak conservation proofs/main.tex}
\end{document}

            \documentclass[12pt, a4paper]{report}

\input{template/main.tex}

\title{\BA{Title in Progress...}}
\author{Boris Andrews}
\affil{Mathematical Institute, University of Oxford}
\date{\today}


\begin{document}
    \pagenumbering{gobble}
    \maketitle
    
    
    \begin{abstract}
        Magnetic confinement reactors---in particular tokamaks---offer one of the most promising options for achieving practical nuclear fusion, with the potential to provide virtually limitless, clean energy. The theoretical and numerical modeling of tokamak plasmas is simultaneously an essential component of effective reactor design, and a great research barrier. Tokamak operational conditions exhibit comparatively low Knudsen numbers. Kinetic effects, including kinetic waves and instabilities, Landau damping, bump-on-tail instabilities and more, are therefore highly influential in tokamak plasma dynamics. Purely fluid models are inherently incapable of capturing these effects, whereas the high dimensionality in purely kinetic models render them practically intractable for most relevant purposes.

        We consider a $\delta\!f$ decomposition model, with a macroscopic fluid background and microscopic kinetic correction, both fully coupled to each other. A similar manner of discretization is proposed to that used in the recent \texttt{STRUPHY} code \cite{Holderied_Possanner_Wang_2021, Holderied_2022, Li_et_al_2023} with a finite-element model for the background and a pseudo-particle/PiC model for the correction.

        The fluid background satisfies the full, non-linear, resistive, compressible, Hall MHD equations. \cite{Laakmann_Hu_Farrell_2022} introduces finite-element(-in-space) implicit timesteppers for the incompressible analogue to this system with structure-preserving (SP) properties in the ideal case, alongside parameter-robust preconditioners. We show that these timesteppers can derive from a finite-element-in-time (FET) (and finite-element-in-space) interpretation. The benefits of this reformulation are discussed, including the derivation of timesteppers that are higher order in time, and the quantifiable dissipative SP properties in the non-ideal, resistive case.
        
        We discuss possible options for extending this FET approach to timesteppers for the compressible case.

        The kinetic corrections satisfy linearized Boltzmann equations. Using a Lénard--Bernstein collision operator, these take Fokker--Planck-like forms \cite{Fokker_1914, Planck_1917} wherein pseudo-particles in the numerical model obey the neoclassical transport equations, with particle-independent Brownian drift terms. This offers a rigorous methodology for incorporating collisions into the particle transport model, without coupling the equations of motions for each particle.
        
        Works by Chen, Chacón et al. \cite{Chen_Chacón_Barnes_2011, Chacón_Chen_Barnes_2013, Chen_Chacón_2014, Chen_Chacón_2015} have developed structure-preserving particle pushers for neoclassical transport in the Vlasov equations, derived from Crank--Nicolson integrators. We show these too can can derive from a FET interpretation, similarly offering potential extensions to higher-order-in-time particle pushers. The FET formulation is used also to consider how the stochastic drift terms can be incorporated into the pushers. Stochastic gyrokinetic expansions are also discussed.

        Different options for the numerical implementation of these schemes are considered.

        Due to the efficacy of FET in the development of SP timesteppers for both the fluid and kinetic component, we hope this approach will prove effective in the future for developing SP timesteppers for the full hybrid model. We hope this will give us the opportunity to incorporate previously inaccessible kinetic effects into the highly effective, modern, finite-element MHD models.
    \end{abstract}
    
    
    \newpage
    \tableofcontents
    
    
    \newpage
    \pagenumbering{arabic}
    %\linenumbers\renewcommand\thelinenumber{\color{black!50}\arabic{linenumber}}
            \input{0 - introduction/main.tex}
        \part{Research}
            \input{1 - low-noise PiC models/main.tex}
            \input{2 - kinetic component/main.tex}
            \input{3 - fluid component/main.tex}
            \input{4 - numerical implementation/main.tex}
        \part{Project Overview}
            \input{5 - research plan/main.tex}
            \input{6 - summary/main.tex}
    
    
    %\section{}
    \newpage
    \pagenumbering{gobble}
        \printbibliography


    \newpage
    \pagenumbering{roman}
    \appendix
        \part{Appendices}
            \input{8 - Hilbert complexes/main.tex}
            \input{9 - weak conservation proofs/main.tex}
\end{document}

            \documentclass[12pt, a4paper]{report}

\input{template/main.tex}

\title{\BA{Title in Progress...}}
\author{Boris Andrews}
\affil{Mathematical Institute, University of Oxford}
\date{\today}


\begin{document}
    \pagenumbering{gobble}
    \maketitle
    
    
    \begin{abstract}
        Magnetic confinement reactors---in particular tokamaks---offer one of the most promising options for achieving practical nuclear fusion, with the potential to provide virtually limitless, clean energy. The theoretical and numerical modeling of tokamak plasmas is simultaneously an essential component of effective reactor design, and a great research barrier. Tokamak operational conditions exhibit comparatively low Knudsen numbers. Kinetic effects, including kinetic waves and instabilities, Landau damping, bump-on-tail instabilities and more, are therefore highly influential in tokamak plasma dynamics. Purely fluid models are inherently incapable of capturing these effects, whereas the high dimensionality in purely kinetic models render them practically intractable for most relevant purposes.

        We consider a $\delta\!f$ decomposition model, with a macroscopic fluid background and microscopic kinetic correction, both fully coupled to each other. A similar manner of discretization is proposed to that used in the recent \texttt{STRUPHY} code \cite{Holderied_Possanner_Wang_2021, Holderied_2022, Li_et_al_2023} with a finite-element model for the background and a pseudo-particle/PiC model for the correction.

        The fluid background satisfies the full, non-linear, resistive, compressible, Hall MHD equations. \cite{Laakmann_Hu_Farrell_2022} introduces finite-element(-in-space) implicit timesteppers for the incompressible analogue to this system with structure-preserving (SP) properties in the ideal case, alongside parameter-robust preconditioners. We show that these timesteppers can derive from a finite-element-in-time (FET) (and finite-element-in-space) interpretation. The benefits of this reformulation are discussed, including the derivation of timesteppers that are higher order in time, and the quantifiable dissipative SP properties in the non-ideal, resistive case.
        
        We discuss possible options for extending this FET approach to timesteppers for the compressible case.

        The kinetic corrections satisfy linearized Boltzmann equations. Using a Lénard--Bernstein collision operator, these take Fokker--Planck-like forms \cite{Fokker_1914, Planck_1917} wherein pseudo-particles in the numerical model obey the neoclassical transport equations, with particle-independent Brownian drift terms. This offers a rigorous methodology for incorporating collisions into the particle transport model, without coupling the equations of motions for each particle.
        
        Works by Chen, Chacón et al. \cite{Chen_Chacón_Barnes_2011, Chacón_Chen_Barnes_2013, Chen_Chacón_2014, Chen_Chacón_2015} have developed structure-preserving particle pushers for neoclassical transport in the Vlasov equations, derived from Crank--Nicolson integrators. We show these too can can derive from a FET interpretation, similarly offering potential extensions to higher-order-in-time particle pushers. The FET formulation is used also to consider how the stochastic drift terms can be incorporated into the pushers. Stochastic gyrokinetic expansions are also discussed.

        Different options for the numerical implementation of these schemes are considered.

        Due to the efficacy of FET in the development of SP timesteppers for both the fluid and kinetic component, we hope this approach will prove effective in the future for developing SP timesteppers for the full hybrid model. We hope this will give us the opportunity to incorporate previously inaccessible kinetic effects into the highly effective, modern, finite-element MHD models.
    \end{abstract}
    
    
    \newpage
    \tableofcontents
    
    
    \newpage
    \pagenumbering{arabic}
    %\linenumbers\renewcommand\thelinenumber{\color{black!50}\arabic{linenumber}}
            \input{0 - introduction/main.tex}
        \part{Research}
            \input{1 - low-noise PiC models/main.tex}
            \input{2 - kinetic component/main.tex}
            \input{3 - fluid component/main.tex}
            \input{4 - numerical implementation/main.tex}
        \part{Project Overview}
            \input{5 - research plan/main.tex}
            \input{6 - summary/main.tex}
    
    
    %\section{}
    \newpage
    \pagenumbering{gobble}
        \printbibliography


    \newpage
    \pagenumbering{roman}
    \appendix
        \part{Appendices}
            \input{8 - Hilbert complexes/main.tex}
            \input{9 - weak conservation proofs/main.tex}
\end{document}

        \part{Project Overview}
            \documentclass[12pt, a4paper]{report}

\input{template/main.tex}

\title{\BA{Title in Progress...}}
\author{Boris Andrews}
\affil{Mathematical Institute, University of Oxford}
\date{\today}


\begin{document}
    \pagenumbering{gobble}
    \maketitle
    
    
    \begin{abstract}
        Magnetic confinement reactors---in particular tokamaks---offer one of the most promising options for achieving practical nuclear fusion, with the potential to provide virtually limitless, clean energy. The theoretical and numerical modeling of tokamak plasmas is simultaneously an essential component of effective reactor design, and a great research barrier. Tokamak operational conditions exhibit comparatively low Knudsen numbers. Kinetic effects, including kinetic waves and instabilities, Landau damping, bump-on-tail instabilities and more, are therefore highly influential in tokamak plasma dynamics. Purely fluid models are inherently incapable of capturing these effects, whereas the high dimensionality in purely kinetic models render them practically intractable for most relevant purposes.

        We consider a $\delta\!f$ decomposition model, with a macroscopic fluid background and microscopic kinetic correction, both fully coupled to each other. A similar manner of discretization is proposed to that used in the recent \texttt{STRUPHY} code \cite{Holderied_Possanner_Wang_2021, Holderied_2022, Li_et_al_2023} with a finite-element model for the background and a pseudo-particle/PiC model for the correction.

        The fluid background satisfies the full, non-linear, resistive, compressible, Hall MHD equations. \cite{Laakmann_Hu_Farrell_2022} introduces finite-element(-in-space) implicit timesteppers for the incompressible analogue to this system with structure-preserving (SP) properties in the ideal case, alongside parameter-robust preconditioners. We show that these timesteppers can derive from a finite-element-in-time (FET) (and finite-element-in-space) interpretation. The benefits of this reformulation are discussed, including the derivation of timesteppers that are higher order in time, and the quantifiable dissipative SP properties in the non-ideal, resistive case.
        
        We discuss possible options for extending this FET approach to timesteppers for the compressible case.

        The kinetic corrections satisfy linearized Boltzmann equations. Using a Lénard--Bernstein collision operator, these take Fokker--Planck-like forms \cite{Fokker_1914, Planck_1917} wherein pseudo-particles in the numerical model obey the neoclassical transport equations, with particle-independent Brownian drift terms. This offers a rigorous methodology for incorporating collisions into the particle transport model, without coupling the equations of motions for each particle.
        
        Works by Chen, Chacón et al. \cite{Chen_Chacón_Barnes_2011, Chacón_Chen_Barnes_2013, Chen_Chacón_2014, Chen_Chacón_2015} have developed structure-preserving particle pushers for neoclassical transport in the Vlasov equations, derived from Crank--Nicolson integrators. We show these too can can derive from a FET interpretation, similarly offering potential extensions to higher-order-in-time particle pushers. The FET formulation is used also to consider how the stochastic drift terms can be incorporated into the pushers. Stochastic gyrokinetic expansions are also discussed.

        Different options for the numerical implementation of these schemes are considered.

        Due to the efficacy of FET in the development of SP timesteppers for both the fluid and kinetic component, we hope this approach will prove effective in the future for developing SP timesteppers for the full hybrid model. We hope this will give us the opportunity to incorporate previously inaccessible kinetic effects into the highly effective, modern, finite-element MHD models.
    \end{abstract}
    
    
    \newpage
    \tableofcontents
    
    
    \newpage
    \pagenumbering{arabic}
    %\linenumbers\renewcommand\thelinenumber{\color{black!50}\arabic{linenumber}}
            \input{0 - introduction/main.tex}
        \part{Research}
            \input{1 - low-noise PiC models/main.tex}
            \input{2 - kinetic component/main.tex}
            \input{3 - fluid component/main.tex}
            \input{4 - numerical implementation/main.tex}
        \part{Project Overview}
            \input{5 - research plan/main.tex}
            \input{6 - summary/main.tex}
    
    
    %\section{}
    \newpage
    \pagenumbering{gobble}
        \printbibliography


    \newpage
    \pagenumbering{roman}
    \appendix
        \part{Appendices}
            \input{8 - Hilbert complexes/main.tex}
            \input{9 - weak conservation proofs/main.tex}
\end{document}

            \documentclass[12pt, a4paper]{report}

\input{template/main.tex}

\title{\BA{Title in Progress...}}
\author{Boris Andrews}
\affil{Mathematical Institute, University of Oxford}
\date{\today}


\begin{document}
    \pagenumbering{gobble}
    \maketitle
    
    
    \begin{abstract}
        Magnetic confinement reactors---in particular tokamaks---offer one of the most promising options for achieving practical nuclear fusion, with the potential to provide virtually limitless, clean energy. The theoretical and numerical modeling of tokamak plasmas is simultaneously an essential component of effective reactor design, and a great research barrier. Tokamak operational conditions exhibit comparatively low Knudsen numbers. Kinetic effects, including kinetic waves and instabilities, Landau damping, bump-on-tail instabilities and more, are therefore highly influential in tokamak plasma dynamics. Purely fluid models are inherently incapable of capturing these effects, whereas the high dimensionality in purely kinetic models render them practically intractable for most relevant purposes.

        We consider a $\delta\!f$ decomposition model, with a macroscopic fluid background and microscopic kinetic correction, both fully coupled to each other. A similar manner of discretization is proposed to that used in the recent \texttt{STRUPHY} code \cite{Holderied_Possanner_Wang_2021, Holderied_2022, Li_et_al_2023} with a finite-element model for the background and a pseudo-particle/PiC model for the correction.

        The fluid background satisfies the full, non-linear, resistive, compressible, Hall MHD equations. \cite{Laakmann_Hu_Farrell_2022} introduces finite-element(-in-space) implicit timesteppers for the incompressible analogue to this system with structure-preserving (SP) properties in the ideal case, alongside parameter-robust preconditioners. We show that these timesteppers can derive from a finite-element-in-time (FET) (and finite-element-in-space) interpretation. The benefits of this reformulation are discussed, including the derivation of timesteppers that are higher order in time, and the quantifiable dissipative SP properties in the non-ideal, resistive case.
        
        We discuss possible options for extending this FET approach to timesteppers for the compressible case.

        The kinetic corrections satisfy linearized Boltzmann equations. Using a Lénard--Bernstein collision operator, these take Fokker--Planck-like forms \cite{Fokker_1914, Planck_1917} wherein pseudo-particles in the numerical model obey the neoclassical transport equations, with particle-independent Brownian drift terms. This offers a rigorous methodology for incorporating collisions into the particle transport model, without coupling the equations of motions for each particle.
        
        Works by Chen, Chacón et al. \cite{Chen_Chacón_Barnes_2011, Chacón_Chen_Barnes_2013, Chen_Chacón_2014, Chen_Chacón_2015} have developed structure-preserving particle pushers for neoclassical transport in the Vlasov equations, derived from Crank--Nicolson integrators. We show these too can can derive from a FET interpretation, similarly offering potential extensions to higher-order-in-time particle pushers. The FET formulation is used also to consider how the stochastic drift terms can be incorporated into the pushers. Stochastic gyrokinetic expansions are also discussed.

        Different options for the numerical implementation of these schemes are considered.

        Due to the efficacy of FET in the development of SP timesteppers for both the fluid and kinetic component, we hope this approach will prove effective in the future for developing SP timesteppers for the full hybrid model. We hope this will give us the opportunity to incorporate previously inaccessible kinetic effects into the highly effective, modern, finite-element MHD models.
    \end{abstract}
    
    
    \newpage
    \tableofcontents
    
    
    \newpage
    \pagenumbering{arabic}
    %\linenumbers\renewcommand\thelinenumber{\color{black!50}\arabic{linenumber}}
            \input{0 - introduction/main.tex}
        \part{Research}
            \input{1 - low-noise PiC models/main.tex}
            \input{2 - kinetic component/main.tex}
            \input{3 - fluid component/main.tex}
            \input{4 - numerical implementation/main.tex}
        \part{Project Overview}
            \input{5 - research plan/main.tex}
            \input{6 - summary/main.tex}
    
    
    %\section{}
    \newpage
    \pagenumbering{gobble}
        \printbibliography


    \newpage
    \pagenumbering{roman}
    \appendix
        \part{Appendices}
            \input{8 - Hilbert complexes/main.tex}
            \input{9 - weak conservation proofs/main.tex}
\end{document}

    
    
    %\section{}
    \newpage
    \pagenumbering{gobble}
        \printbibliography


    \newpage
    \pagenumbering{roman}
    \appendix
        \part{Appendices}
            \documentclass[12pt, a4paper]{report}

\input{template/main.tex}

\title{\BA{Title in Progress...}}
\author{Boris Andrews}
\affil{Mathematical Institute, University of Oxford}
\date{\today}


\begin{document}
    \pagenumbering{gobble}
    \maketitle
    
    
    \begin{abstract}
        Magnetic confinement reactors---in particular tokamaks---offer one of the most promising options for achieving practical nuclear fusion, with the potential to provide virtually limitless, clean energy. The theoretical and numerical modeling of tokamak plasmas is simultaneously an essential component of effective reactor design, and a great research barrier. Tokamak operational conditions exhibit comparatively low Knudsen numbers. Kinetic effects, including kinetic waves and instabilities, Landau damping, bump-on-tail instabilities and more, are therefore highly influential in tokamak plasma dynamics. Purely fluid models are inherently incapable of capturing these effects, whereas the high dimensionality in purely kinetic models render them practically intractable for most relevant purposes.

        We consider a $\delta\!f$ decomposition model, with a macroscopic fluid background and microscopic kinetic correction, both fully coupled to each other. A similar manner of discretization is proposed to that used in the recent \texttt{STRUPHY} code \cite{Holderied_Possanner_Wang_2021, Holderied_2022, Li_et_al_2023} with a finite-element model for the background and a pseudo-particle/PiC model for the correction.

        The fluid background satisfies the full, non-linear, resistive, compressible, Hall MHD equations. \cite{Laakmann_Hu_Farrell_2022} introduces finite-element(-in-space) implicit timesteppers for the incompressible analogue to this system with structure-preserving (SP) properties in the ideal case, alongside parameter-robust preconditioners. We show that these timesteppers can derive from a finite-element-in-time (FET) (and finite-element-in-space) interpretation. The benefits of this reformulation are discussed, including the derivation of timesteppers that are higher order in time, and the quantifiable dissipative SP properties in the non-ideal, resistive case.
        
        We discuss possible options for extending this FET approach to timesteppers for the compressible case.

        The kinetic corrections satisfy linearized Boltzmann equations. Using a Lénard--Bernstein collision operator, these take Fokker--Planck-like forms \cite{Fokker_1914, Planck_1917} wherein pseudo-particles in the numerical model obey the neoclassical transport equations, with particle-independent Brownian drift terms. This offers a rigorous methodology for incorporating collisions into the particle transport model, without coupling the equations of motions for each particle.
        
        Works by Chen, Chacón et al. \cite{Chen_Chacón_Barnes_2011, Chacón_Chen_Barnes_2013, Chen_Chacón_2014, Chen_Chacón_2015} have developed structure-preserving particle pushers for neoclassical transport in the Vlasov equations, derived from Crank--Nicolson integrators. We show these too can can derive from a FET interpretation, similarly offering potential extensions to higher-order-in-time particle pushers. The FET formulation is used also to consider how the stochastic drift terms can be incorporated into the pushers. Stochastic gyrokinetic expansions are also discussed.

        Different options for the numerical implementation of these schemes are considered.

        Due to the efficacy of FET in the development of SP timesteppers for both the fluid and kinetic component, we hope this approach will prove effective in the future for developing SP timesteppers for the full hybrid model. We hope this will give us the opportunity to incorporate previously inaccessible kinetic effects into the highly effective, modern, finite-element MHD models.
    \end{abstract}
    
    
    \newpage
    \tableofcontents
    
    
    \newpage
    \pagenumbering{arabic}
    %\linenumbers\renewcommand\thelinenumber{\color{black!50}\arabic{linenumber}}
            \input{0 - introduction/main.tex}
        \part{Research}
            \input{1 - low-noise PiC models/main.tex}
            \input{2 - kinetic component/main.tex}
            \input{3 - fluid component/main.tex}
            \input{4 - numerical implementation/main.tex}
        \part{Project Overview}
            \input{5 - research plan/main.tex}
            \input{6 - summary/main.tex}
    
    
    %\section{}
    \newpage
    \pagenumbering{gobble}
        \printbibliography


    \newpage
    \pagenumbering{roman}
    \appendix
        \part{Appendices}
            \input{8 - Hilbert complexes/main.tex}
            \input{9 - weak conservation proofs/main.tex}
\end{document}

            \documentclass[12pt, a4paper]{report}

\input{template/main.tex}

\title{\BA{Title in Progress...}}
\author{Boris Andrews}
\affil{Mathematical Institute, University of Oxford}
\date{\today}


\begin{document}
    \pagenumbering{gobble}
    \maketitle
    
    
    \begin{abstract}
        Magnetic confinement reactors---in particular tokamaks---offer one of the most promising options for achieving practical nuclear fusion, with the potential to provide virtually limitless, clean energy. The theoretical and numerical modeling of tokamak plasmas is simultaneously an essential component of effective reactor design, and a great research barrier. Tokamak operational conditions exhibit comparatively low Knudsen numbers. Kinetic effects, including kinetic waves and instabilities, Landau damping, bump-on-tail instabilities and more, are therefore highly influential in tokamak plasma dynamics. Purely fluid models are inherently incapable of capturing these effects, whereas the high dimensionality in purely kinetic models render them practically intractable for most relevant purposes.

        We consider a $\delta\!f$ decomposition model, with a macroscopic fluid background and microscopic kinetic correction, both fully coupled to each other. A similar manner of discretization is proposed to that used in the recent \texttt{STRUPHY} code \cite{Holderied_Possanner_Wang_2021, Holderied_2022, Li_et_al_2023} with a finite-element model for the background and a pseudo-particle/PiC model for the correction.

        The fluid background satisfies the full, non-linear, resistive, compressible, Hall MHD equations. \cite{Laakmann_Hu_Farrell_2022} introduces finite-element(-in-space) implicit timesteppers for the incompressible analogue to this system with structure-preserving (SP) properties in the ideal case, alongside parameter-robust preconditioners. We show that these timesteppers can derive from a finite-element-in-time (FET) (and finite-element-in-space) interpretation. The benefits of this reformulation are discussed, including the derivation of timesteppers that are higher order in time, and the quantifiable dissipative SP properties in the non-ideal, resistive case.
        
        We discuss possible options for extending this FET approach to timesteppers for the compressible case.

        The kinetic corrections satisfy linearized Boltzmann equations. Using a Lénard--Bernstein collision operator, these take Fokker--Planck-like forms \cite{Fokker_1914, Planck_1917} wherein pseudo-particles in the numerical model obey the neoclassical transport equations, with particle-independent Brownian drift terms. This offers a rigorous methodology for incorporating collisions into the particle transport model, without coupling the equations of motions for each particle.
        
        Works by Chen, Chacón et al. \cite{Chen_Chacón_Barnes_2011, Chacón_Chen_Barnes_2013, Chen_Chacón_2014, Chen_Chacón_2015} have developed structure-preserving particle pushers for neoclassical transport in the Vlasov equations, derived from Crank--Nicolson integrators. We show these too can can derive from a FET interpretation, similarly offering potential extensions to higher-order-in-time particle pushers. The FET formulation is used also to consider how the stochastic drift terms can be incorporated into the pushers. Stochastic gyrokinetic expansions are also discussed.

        Different options for the numerical implementation of these schemes are considered.

        Due to the efficacy of FET in the development of SP timesteppers for both the fluid and kinetic component, we hope this approach will prove effective in the future for developing SP timesteppers for the full hybrid model. We hope this will give us the opportunity to incorporate previously inaccessible kinetic effects into the highly effective, modern, finite-element MHD models.
    \end{abstract}
    
    
    \newpage
    \tableofcontents
    
    
    \newpage
    \pagenumbering{arabic}
    %\linenumbers\renewcommand\thelinenumber{\color{black!50}\arabic{linenumber}}
            \input{0 - introduction/main.tex}
        \part{Research}
            \input{1 - low-noise PiC models/main.tex}
            \input{2 - kinetic component/main.tex}
            \input{3 - fluid component/main.tex}
            \input{4 - numerical implementation/main.tex}
        \part{Project Overview}
            \input{5 - research plan/main.tex}
            \input{6 - summary/main.tex}
    
    
    %\section{}
    \newpage
    \pagenumbering{gobble}
        \printbibliography


    \newpage
    \pagenumbering{roman}
    \appendix
        \part{Appendices}
            \input{8 - Hilbert complexes/main.tex}
            \input{9 - weak conservation proofs/main.tex}
\end{document}

\end{document}

            \documentclass[12pt, a4paper]{report}

\documentclass[12pt, a4paper]{report}

\input{template/main.tex}

\title{\BA{Title in Progress...}}
\author{Boris Andrews}
\affil{Mathematical Institute, University of Oxford}
\date{\today}


\begin{document}
    \pagenumbering{gobble}
    \maketitle
    
    
    \begin{abstract}
        Magnetic confinement reactors---in particular tokamaks---offer one of the most promising options for achieving practical nuclear fusion, with the potential to provide virtually limitless, clean energy. The theoretical and numerical modeling of tokamak plasmas is simultaneously an essential component of effective reactor design, and a great research barrier. Tokamak operational conditions exhibit comparatively low Knudsen numbers. Kinetic effects, including kinetic waves and instabilities, Landau damping, bump-on-tail instabilities and more, are therefore highly influential in tokamak plasma dynamics. Purely fluid models are inherently incapable of capturing these effects, whereas the high dimensionality in purely kinetic models render them practically intractable for most relevant purposes.

        We consider a $\delta\!f$ decomposition model, with a macroscopic fluid background and microscopic kinetic correction, both fully coupled to each other. A similar manner of discretization is proposed to that used in the recent \texttt{STRUPHY} code \cite{Holderied_Possanner_Wang_2021, Holderied_2022, Li_et_al_2023} with a finite-element model for the background and a pseudo-particle/PiC model for the correction.

        The fluid background satisfies the full, non-linear, resistive, compressible, Hall MHD equations. \cite{Laakmann_Hu_Farrell_2022} introduces finite-element(-in-space) implicit timesteppers for the incompressible analogue to this system with structure-preserving (SP) properties in the ideal case, alongside parameter-robust preconditioners. We show that these timesteppers can derive from a finite-element-in-time (FET) (and finite-element-in-space) interpretation. The benefits of this reformulation are discussed, including the derivation of timesteppers that are higher order in time, and the quantifiable dissipative SP properties in the non-ideal, resistive case.
        
        We discuss possible options for extending this FET approach to timesteppers for the compressible case.

        The kinetic corrections satisfy linearized Boltzmann equations. Using a Lénard--Bernstein collision operator, these take Fokker--Planck-like forms \cite{Fokker_1914, Planck_1917} wherein pseudo-particles in the numerical model obey the neoclassical transport equations, with particle-independent Brownian drift terms. This offers a rigorous methodology for incorporating collisions into the particle transport model, without coupling the equations of motions for each particle.
        
        Works by Chen, Chacón et al. \cite{Chen_Chacón_Barnes_2011, Chacón_Chen_Barnes_2013, Chen_Chacón_2014, Chen_Chacón_2015} have developed structure-preserving particle pushers for neoclassical transport in the Vlasov equations, derived from Crank--Nicolson integrators. We show these too can can derive from a FET interpretation, similarly offering potential extensions to higher-order-in-time particle pushers. The FET formulation is used also to consider how the stochastic drift terms can be incorporated into the pushers. Stochastic gyrokinetic expansions are also discussed.

        Different options for the numerical implementation of these schemes are considered.

        Due to the efficacy of FET in the development of SP timesteppers for both the fluid and kinetic component, we hope this approach will prove effective in the future for developing SP timesteppers for the full hybrid model. We hope this will give us the opportunity to incorporate previously inaccessible kinetic effects into the highly effective, modern, finite-element MHD models.
    \end{abstract}
    
    
    \newpage
    \tableofcontents
    
    
    \newpage
    \pagenumbering{arabic}
    %\linenumbers\renewcommand\thelinenumber{\color{black!50}\arabic{linenumber}}
            \input{0 - introduction/main.tex}
        \part{Research}
            \input{1 - low-noise PiC models/main.tex}
            \input{2 - kinetic component/main.tex}
            \input{3 - fluid component/main.tex}
            \input{4 - numerical implementation/main.tex}
        \part{Project Overview}
            \input{5 - research plan/main.tex}
            \input{6 - summary/main.tex}
    
    
    %\section{}
    \newpage
    \pagenumbering{gobble}
        \printbibliography


    \newpage
    \pagenumbering{roman}
    \appendix
        \part{Appendices}
            \input{8 - Hilbert complexes/main.tex}
            \input{9 - weak conservation proofs/main.tex}
\end{document}


\title{\BA{Title in Progress...}}
\author{Boris Andrews}
\affil{Mathematical Institute, University of Oxford}
\date{\today}


\begin{document}
    \pagenumbering{gobble}
    \maketitle
    
    
    \begin{abstract}
        Magnetic confinement reactors---in particular tokamaks---offer one of the most promising options for achieving practical nuclear fusion, with the potential to provide virtually limitless, clean energy. The theoretical and numerical modeling of tokamak plasmas is simultaneously an essential component of effective reactor design, and a great research barrier. Tokamak operational conditions exhibit comparatively low Knudsen numbers. Kinetic effects, including kinetic waves and instabilities, Landau damping, bump-on-tail instabilities and more, are therefore highly influential in tokamak plasma dynamics. Purely fluid models are inherently incapable of capturing these effects, whereas the high dimensionality in purely kinetic models render them practically intractable for most relevant purposes.

        We consider a $\delta\!f$ decomposition model, with a macroscopic fluid background and microscopic kinetic correction, both fully coupled to each other. A similar manner of discretization is proposed to that used in the recent \texttt{STRUPHY} code \cite{Holderied_Possanner_Wang_2021, Holderied_2022, Li_et_al_2023} with a finite-element model for the background and a pseudo-particle/PiC model for the correction.

        The fluid background satisfies the full, non-linear, resistive, compressible, Hall MHD equations. \cite{Laakmann_Hu_Farrell_2022} introduces finite-element(-in-space) implicit timesteppers for the incompressible analogue to this system with structure-preserving (SP) properties in the ideal case, alongside parameter-robust preconditioners. We show that these timesteppers can derive from a finite-element-in-time (FET) (and finite-element-in-space) interpretation. The benefits of this reformulation are discussed, including the derivation of timesteppers that are higher order in time, and the quantifiable dissipative SP properties in the non-ideal, resistive case.
        
        We discuss possible options for extending this FET approach to timesteppers for the compressible case.

        The kinetic corrections satisfy linearized Boltzmann equations. Using a Lénard--Bernstein collision operator, these take Fokker--Planck-like forms \cite{Fokker_1914, Planck_1917} wherein pseudo-particles in the numerical model obey the neoclassical transport equations, with particle-independent Brownian drift terms. This offers a rigorous methodology for incorporating collisions into the particle transport model, without coupling the equations of motions for each particle.
        
        Works by Chen, Chacón et al. \cite{Chen_Chacón_Barnes_2011, Chacón_Chen_Barnes_2013, Chen_Chacón_2014, Chen_Chacón_2015} have developed structure-preserving particle pushers for neoclassical transport in the Vlasov equations, derived from Crank--Nicolson integrators. We show these too can can derive from a FET interpretation, similarly offering potential extensions to higher-order-in-time particle pushers. The FET formulation is used also to consider how the stochastic drift terms can be incorporated into the pushers. Stochastic gyrokinetic expansions are also discussed.

        Different options for the numerical implementation of these schemes are considered.

        Due to the efficacy of FET in the development of SP timesteppers for both the fluid and kinetic component, we hope this approach will prove effective in the future for developing SP timesteppers for the full hybrid model. We hope this will give us the opportunity to incorporate previously inaccessible kinetic effects into the highly effective, modern, finite-element MHD models.
    \end{abstract}
    
    
    \newpage
    \tableofcontents
    
    
    \newpage
    \pagenumbering{arabic}
    %\linenumbers\renewcommand\thelinenumber{\color{black!50}\arabic{linenumber}}
            \documentclass[12pt, a4paper]{report}

\input{template/main.tex}

\title{\BA{Title in Progress...}}
\author{Boris Andrews}
\affil{Mathematical Institute, University of Oxford}
\date{\today}


\begin{document}
    \pagenumbering{gobble}
    \maketitle
    
    
    \begin{abstract}
        Magnetic confinement reactors---in particular tokamaks---offer one of the most promising options for achieving practical nuclear fusion, with the potential to provide virtually limitless, clean energy. The theoretical and numerical modeling of tokamak plasmas is simultaneously an essential component of effective reactor design, and a great research barrier. Tokamak operational conditions exhibit comparatively low Knudsen numbers. Kinetic effects, including kinetic waves and instabilities, Landau damping, bump-on-tail instabilities and more, are therefore highly influential in tokamak plasma dynamics. Purely fluid models are inherently incapable of capturing these effects, whereas the high dimensionality in purely kinetic models render them practically intractable for most relevant purposes.

        We consider a $\delta\!f$ decomposition model, with a macroscopic fluid background and microscopic kinetic correction, both fully coupled to each other. A similar manner of discretization is proposed to that used in the recent \texttt{STRUPHY} code \cite{Holderied_Possanner_Wang_2021, Holderied_2022, Li_et_al_2023} with a finite-element model for the background and a pseudo-particle/PiC model for the correction.

        The fluid background satisfies the full, non-linear, resistive, compressible, Hall MHD equations. \cite{Laakmann_Hu_Farrell_2022} introduces finite-element(-in-space) implicit timesteppers for the incompressible analogue to this system with structure-preserving (SP) properties in the ideal case, alongside parameter-robust preconditioners. We show that these timesteppers can derive from a finite-element-in-time (FET) (and finite-element-in-space) interpretation. The benefits of this reformulation are discussed, including the derivation of timesteppers that are higher order in time, and the quantifiable dissipative SP properties in the non-ideal, resistive case.
        
        We discuss possible options for extending this FET approach to timesteppers for the compressible case.

        The kinetic corrections satisfy linearized Boltzmann equations. Using a Lénard--Bernstein collision operator, these take Fokker--Planck-like forms \cite{Fokker_1914, Planck_1917} wherein pseudo-particles in the numerical model obey the neoclassical transport equations, with particle-independent Brownian drift terms. This offers a rigorous methodology for incorporating collisions into the particle transport model, without coupling the equations of motions for each particle.
        
        Works by Chen, Chacón et al. \cite{Chen_Chacón_Barnes_2011, Chacón_Chen_Barnes_2013, Chen_Chacón_2014, Chen_Chacón_2015} have developed structure-preserving particle pushers for neoclassical transport in the Vlasov equations, derived from Crank--Nicolson integrators. We show these too can can derive from a FET interpretation, similarly offering potential extensions to higher-order-in-time particle pushers. The FET formulation is used also to consider how the stochastic drift terms can be incorporated into the pushers. Stochastic gyrokinetic expansions are also discussed.

        Different options for the numerical implementation of these schemes are considered.

        Due to the efficacy of FET in the development of SP timesteppers for both the fluid and kinetic component, we hope this approach will prove effective in the future for developing SP timesteppers for the full hybrid model. We hope this will give us the opportunity to incorporate previously inaccessible kinetic effects into the highly effective, modern, finite-element MHD models.
    \end{abstract}
    
    
    \newpage
    \tableofcontents
    
    
    \newpage
    \pagenumbering{arabic}
    %\linenumbers\renewcommand\thelinenumber{\color{black!50}\arabic{linenumber}}
            \input{0 - introduction/main.tex}
        \part{Research}
            \input{1 - low-noise PiC models/main.tex}
            \input{2 - kinetic component/main.tex}
            \input{3 - fluid component/main.tex}
            \input{4 - numerical implementation/main.tex}
        \part{Project Overview}
            \input{5 - research plan/main.tex}
            \input{6 - summary/main.tex}
    
    
    %\section{}
    \newpage
    \pagenumbering{gobble}
        \printbibliography


    \newpage
    \pagenumbering{roman}
    \appendix
        \part{Appendices}
            \input{8 - Hilbert complexes/main.tex}
            \input{9 - weak conservation proofs/main.tex}
\end{document}

        \part{Research}
            \documentclass[12pt, a4paper]{report}

\input{template/main.tex}

\title{\BA{Title in Progress...}}
\author{Boris Andrews}
\affil{Mathematical Institute, University of Oxford}
\date{\today}


\begin{document}
    \pagenumbering{gobble}
    \maketitle
    
    
    \begin{abstract}
        Magnetic confinement reactors---in particular tokamaks---offer one of the most promising options for achieving practical nuclear fusion, with the potential to provide virtually limitless, clean energy. The theoretical and numerical modeling of tokamak plasmas is simultaneously an essential component of effective reactor design, and a great research barrier. Tokamak operational conditions exhibit comparatively low Knudsen numbers. Kinetic effects, including kinetic waves and instabilities, Landau damping, bump-on-tail instabilities and more, are therefore highly influential in tokamak plasma dynamics. Purely fluid models are inherently incapable of capturing these effects, whereas the high dimensionality in purely kinetic models render them practically intractable for most relevant purposes.

        We consider a $\delta\!f$ decomposition model, with a macroscopic fluid background and microscopic kinetic correction, both fully coupled to each other. A similar manner of discretization is proposed to that used in the recent \texttt{STRUPHY} code \cite{Holderied_Possanner_Wang_2021, Holderied_2022, Li_et_al_2023} with a finite-element model for the background and a pseudo-particle/PiC model for the correction.

        The fluid background satisfies the full, non-linear, resistive, compressible, Hall MHD equations. \cite{Laakmann_Hu_Farrell_2022} introduces finite-element(-in-space) implicit timesteppers for the incompressible analogue to this system with structure-preserving (SP) properties in the ideal case, alongside parameter-robust preconditioners. We show that these timesteppers can derive from a finite-element-in-time (FET) (and finite-element-in-space) interpretation. The benefits of this reformulation are discussed, including the derivation of timesteppers that are higher order in time, and the quantifiable dissipative SP properties in the non-ideal, resistive case.
        
        We discuss possible options for extending this FET approach to timesteppers for the compressible case.

        The kinetic corrections satisfy linearized Boltzmann equations. Using a Lénard--Bernstein collision operator, these take Fokker--Planck-like forms \cite{Fokker_1914, Planck_1917} wherein pseudo-particles in the numerical model obey the neoclassical transport equations, with particle-independent Brownian drift terms. This offers a rigorous methodology for incorporating collisions into the particle transport model, without coupling the equations of motions for each particle.
        
        Works by Chen, Chacón et al. \cite{Chen_Chacón_Barnes_2011, Chacón_Chen_Barnes_2013, Chen_Chacón_2014, Chen_Chacón_2015} have developed structure-preserving particle pushers for neoclassical transport in the Vlasov equations, derived from Crank--Nicolson integrators. We show these too can can derive from a FET interpretation, similarly offering potential extensions to higher-order-in-time particle pushers. The FET formulation is used also to consider how the stochastic drift terms can be incorporated into the pushers. Stochastic gyrokinetic expansions are also discussed.

        Different options for the numerical implementation of these schemes are considered.

        Due to the efficacy of FET in the development of SP timesteppers for both the fluid and kinetic component, we hope this approach will prove effective in the future for developing SP timesteppers for the full hybrid model. We hope this will give us the opportunity to incorporate previously inaccessible kinetic effects into the highly effective, modern, finite-element MHD models.
    \end{abstract}
    
    
    \newpage
    \tableofcontents
    
    
    \newpage
    \pagenumbering{arabic}
    %\linenumbers\renewcommand\thelinenumber{\color{black!50}\arabic{linenumber}}
            \input{0 - introduction/main.tex}
        \part{Research}
            \input{1 - low-noise PiC models/main.tex}
            \input{2 - kinetic component/main.tex}
            \input{3 - fluid component/main.tex}
            \input{4 - numerical implementation/main.tex}
        \part{Project Overview}
            \input{5 - research plan/main.tex}
            \input{6 - summary/main.tex}
    
    
    %\section{}
    \newpage
    \pagenumbering{gobble}
        \printbibliography


    \newpage
    \pagenumbering{roman}
    \appendix
        \part{Appendices}
            \input{8 - Hilbert complexes/main.tex}
            \input{9 - weak conservation proofs/main.tex}
\end{document}

            \documentclass[12pt, a4paper]{report}

\input{template/main.tex}

\title{\BA{Title in Progress...}}
\author{Boris Andrews}
\affil{Mathematical Institute, University of Oxford}
\date{\today}


\begin{document}
    \pagenumbering{gobble}
    \maketitle
    
    
    \begin{abstract}
        Magnetic confinement reactors---in particular tokamaks---offer one of the most promising options for achieving practical nuclear fusion, with the potential to provide virtually limitless, clean energy. The theoretical and numerical modeling of tokamak plasmas is simultaneously an essential component of effective reactor design, and a great research barrier. Tokamak operational conditions exhibit comparatively low Knudsen numbers. Kinetic effects, including kinetic waves and instabilities, Landau damping, bump-on-tail instabilities and more, are therefore highly influential in tokamak plasma dynamics. Purely fluid models are inherently incapable of capturing these effects, whereas the high dimensionality in purely kinetic models render them practically intractable for most relevant purposes.

        We consider a $\delta\!f$ decomposition model, with a macroscopic fluid background and microscopic kinetic correction, both fully coupled to each other. A similar manner of discretization is proposed to that used in the recent \texttt{STRUPHY} code \cite{Holderied_Possanner_Wang_2021, Holderied_2022, Li_et_al_2023} with a finite-element model for the background and a pseudo-particle/PiC model for the correction.

        The fluid background satisfies the full, non-linear, resistive, compressible, Hall MHD equations. \cite{Laakmann_Hu_Farrell_2022} introduces finite-element(-in-space) implicit timesteppers for the incompressible analogue to this system with structure-preserving (SP) properties in the ideal case, alongside parameter-robust preconditioners. We show that these timesteppers can derive from a finite-element-in-time (FET) (and finite-element-in-space) interpretation. The benefits of this reformulation are discussed, including the derivation of timesteppers that are higher order in time, and the quantifiable dissipative SP properties in the non-ideal, resistive case.
        
        We discuss possible options for extending this FET approach to timesteppers for the compressible case.

        The kinetic corrections satisfy linearized Boltzmann equations. Using a Lénard--Bernstein collision operator, these take Fokker--Planck-like forms \cite{Fokker_1914, Planck_1917} wherein pseudo-particles in the numerical model obey the neoclassical transport equations, with particle-independent Brownian drift terms. This offers a rigorous methodology for incorporating collisions into the particle transport model, without coupling the equations of motions for each particle.
        
        Works by Chen, Chacón et al. \cite{Chen_Chacón_Barnes_2011, Chacón_Chen_Barnes_2013, Chen_Chacón_2014, Chen_Chacón_2015} have developed structure-preserving particle pushers for neoclassical transport in the Vlasov equations, derived from Crank--Nicolson integrators. We show these too can can derive from a FET interpretation, similarly offering potential extensions to higher-order-in-time particle pushers. The FET formulation is used also to consider how the stochastic drift terms can be incorporated into the pushers. Stochastic gyrokinetic expansions are also discussed.

        Different options for the numerical implementation of these schemes are considered.

        Due to the efficacy of FET in the development of SP timesteppers for both the fluid and kinetic component, we hope this approach will prove effective in the future for developing SP timesteppers for the full hybrid model. We hope this will give us the opportunity to incorporate previously inaccessible kinetic effects into the highly effective, modern, finite-element MHD models.
    \end{abstract}
    
    
    \newpage
    \tableofcontents
    
    
    \newpage
    \pagenumbering{arabic}
    %\linenumbers\renewcommand\thelinenumber{\color{black!50}\arabic{linenumber}}
            \input{0 - introduction/main.tex}
        \part{Research}
            \input{1 - low-noise PiC models/main.tex}
            \input{2 - kinetic component/main.tex}
            \input{3 - fluid component/main.tex}
            \input{4 - numerical implementation/main.tex}
        \part{Project Overview}
            \input{5 - research plan/main.tex}
            \input{6 - summary/main.tex}
    
    
    %\section{}
    \newpage
    \pagenumbering{gobble}
        \printbibliography


    \newpage
    \pagenumbering{roman}
    \appendix
        \part{Appendices}
            \input{8 - Hilbert complexes/main.tex}
            \input{9 - weak conservation proofs/main.tex}
\end{document}

            \documentclass[12pt, a4paper]{report}

\input{template/main.tex}

\title{\BA{Title in Progress...}}
\author{Boris Andrews}
\affil{Mathematical Institute, University of Oxford}
\date{\today}


\begin{document}
    \pagenumbering{gobble}
    \maketitle
    
    
    \begin{abstract}
        Magnetic confinement reactors---in particular tokamaks---offer one of the most promising options for achieving practical nuclear fusion, with the potential to provide virtually limitless, clean energy. The theoretical and numerical modeling of tokamak plasmas is simultaneously an essential component of effective reactor design, and a great research barrier. Tokamak operational conditions exhibit comparatively low Knudsen numbers. Kinetic effects, including kinetic waves and instabilities, Landau damping, bump-on-tail instabilities and more, are therefore highly influential in tokamak plasma dynamics. Purely fluid models are inherently incapable of capturing these effects, whereas the high dimensionality in purely kinetic models render them practically intractable for most relevant purposes.

        We consider a $\delta\!f$ decomposition model, with a macroscopic fluid background and microscopic kinetic correction, both fully coupled to each other. A similar manner of discretization is proposed to that used in the recent \texttt{STRUPHY} code \cite{Holderied_Possanner_Wang_2021, Holderied_2022, Li_et_al_2023} with a finite-element model for the background and a pseudo-particle/PiC model for the correction.

        The fluid background satisfies the full, non-linear, resistive, compressible, Hall MHD equations. \cite{Laakmann_Hu_Farrell_2022} introduces finite-element(-in-space) implicit timesteppers for the incompressible analogue to this system with structure-preserving (SP) properties in the ideal case, alongside parameter-robust preconditioners. We show that these timesteppers can derive from a finite-element-in-time (FET) (and finite-element-in-space) interpretation. The benefits of this reformulation are discussed, including the derivation of timesteppers that are higher order in time, and the quantifiable dissipative SP properties in the non-ideal, resistive case.
        
        We discuss possible options for extending this FET approach to timesteppers for the compressible case.

        The kinetic corrections satisfy linearized Boltzmann equations. Using a Lénard--Bernstein collision operator, these take Fokker--Planck-like forms \cite{Fokker_1914, Planck_1917} wherein pseudo-particles in the numerical model obey the neoclassical transport equations, with particle-independent Brownian drift terms. This offers a rigorous methodology for incorporating collisions into the particle transport model, without coupling the equations of motions for each particle.
        
        Works by Chen, Chacón et al. \cite{Chen_Chacón_Barnes_2011, Chacón_Chen_Barnes_2013, Chen_Chacón_2014, Chen_Chacón_2015} have developed structure-preserving particle pushers for neoclassical transport in the Vlasov equations, derived from Crank--Nicolson integrators. We show these too can can derive from a FET interpretation, similarly offering potential extensions to higher-order-in-time particle pushers. The FET formulation is used also to consider how the stochastic drift terms can be incorporated into the pushers. Stochastic gyrokinetic expansions are also discussed.

        Different options for the numerical implementation of these schemes are considered.

        Due to the efficacy of FET in the development of SP timesteppers for both the fluid and kinetic component, we hope this approach will prove effective in the future for developing SP timesteppers for the full hybrid model. We hope this will give us the opportunity to incorporate previously inaccessible kinetic effects into the highly effective, modern, finite-element MHD models.
    \end{abstract}
    
    
    \newpage
    \tableofcontents
    
    
    \newpage
    \pagenumbering{arabic}
    %\linenumbers\renewcommand\thelinenumber{\color{black!50}\arabic{linenumber}}
            \input{0 - introduction/main.tex}
        \part{Research}
            \input{1 - low-noise PiC models/main.tex}
            \input{2 - kinetic component/main.tex}
            \input{3 - fluid component/main.tex}
            \input{4 - numerical implementation/main.tex}
        \part{Project Overview}
            \input{5 - research plan/main.tex}
            \input{6 - summary/main.tex}
    
    
    %\section{}
    \newpage
    \pagenumbering{gobble}
        \printbibliography


    \newpage
    \pagenumbering{roman}
    \appendix
        \part{Appendices}
            \input{8 - Hilbert complexes/main.tex}
            \input{9 - weak conservation proofs/main.tex}
\end{document}

            \documentclass[12pt, a4paper]{report}

\input{template/main.tex}

\title{\BA{Title in Progress...}}
\author{Boris Andrews}
\affil{Mathematical Institute, University of Oxford}
\date{\today}


\begin{document}
    \pagenumbering{gobble}
    \maketitle
    
    
    \begin{abstract}
        Magnetic confinement reactors---in particular tokamaks---offer one of the most promising options for achieving practical nuclear fusion, with the potential to provide virtually limitless, clean energy. The theoretical and numerical modeling of tokamak plasmas is simultaneously an essential component of effective reactor design, and a great research barrier. Tokamak operational conditions exhibit comparatively low Knudsen numbers. Kinetic effects, including kinetic waves and instabilities, Landau damping, bump-on-tail instabilities and more, are therefore highly influential in tokamak plasma dynamics. Purely fluid models are inherently incapable of capturing these effects, whereas the high dimensionality in purely kinetic models render them practically intractable for most relevant purposes.

        We consider a $\delta\!f$ decomposition model, with a macroscopic fluid background and microscopic kinetic correction, both fully coupled to each other. A similar manner of discretization is proposed to that used in the recent \texttt{STRUPHY} code \cite{Holderied_Possanner_Wang_2021, Holderied_2022, Li_et_al_2023} with a finite-element model for the background and a pseudo-particle/PiC model for the correction.

        The fluid background satisfies the full, non-linear, resistive, compressible, Hall MHD equations. \cite{Laakmann_Hu_Farrell_2022} introduces finite-element(-in-space) implicit timesteppers for the incompressible analogue to this system with structure-preserving (SP) properties in the ideal case, alongside parameter-robust preconditioners. We show that these timesteppers can derive from a finite-element-in-time (FET) (and finite-element-in-space) interpretation. The benefits of this reformulation are discussed, including the derivation of timesteppers that are higher order in time, and the quantifiable dissipative SP properties in the non-ideal, resistive case.
        
        We discuss possible options for extending this FET approach to timesteppers for the compressible case.

        The kinetic corrections satisfy linearized Boltzmann equations. Using a Lénard--Bernstein collision operator, these take Fokker--Planck-like forms \cite{Fokker_1914, Planck_1917} wherein pseudo-particles in the numerical model obey the neoclassical transport equations, with particle-independent Brownian drift terms. This offers a rigorous methodology for incorporating collisions into the particle transport model, without coupling the equations of motions for each particle.
        
        Works by Chen, Chacón et al. \cite{Chen_Chacón_Barnes_2011, Chacón_Chen_Barnes_2013, Chen_Chacón_2014, Chen_Chacón_2015} have developed structure-preserving particle pushers for neoclassical transport in the Vlasov equations, derived from Crank--Nicolson integrators. We show these too can can derive from a FET interpretation, similarly offering potential extensions to higher-order-in-time particle pushers. The FET formulation is used also to consider how the stochastic drift terms can be incorporated into the pushers. Stochastic gyrokinetic expansions are also discussed.

        Different options for the numerical implementation of these schemes are considered.

        Due to the efficacy of FET in the development of SP timesteppers for both the fluid and kinetic component, we hope this approach will prove effective in the future for developing SP timesteppers for the full hybrid model. We hope this will give us the opportunity to incorporate previously inaccessible kinetic effects into the highly effective, modern, finite-element MHD models.
    \end{abstract}
    
    
    \newpage
    \tableofcontents
    
    
    \newpage
    \pagenumbering{arabic}
    %\linenumbers\renewcommand\thelinenumber{\color{black!50}\arabic{linenumber}}
            \input{0 - introduction/main.tex}
        \part{Research}
            \input{1 - low-noise PiC models/main.tex}
            \input{2 - kinetic component/main.tex}
            \input{3 - fluid component/main.tex}
            \input{4 - numerical implementation/main.tex}
        \part{Project Overview}
            \input{5 - research plan/main.tex}
            \input{6 - summary/main.tex}
    
    
    %\section{}
    \newpage
    \pagenumbering{gobble}
        \printbibliography


    \newpage
    \pagenumbering{roman}
    \appendix
        \part{Appendices}
            \input{8 - Hilbert complexes/main.tex}
            \input{9 - weak conservation proofs/main.tex}
\end{document}

        \part{Project Overview}
            \documentclass[12pt, a4paper]{report}

\input{template/main.tex}

\title{\BA{Title in Progress...}}
\author{Boris Andrews}
\affil{Mathematical Institute, University of Oxford}
\date{\today}


\begin{document}
    \pagenumbering{gobble}
    \maketitle
    
    
    \begin{abstract}
        Magnetic confinement reactors---in particular tokamaks---offer one of the most promising options for achieving practical nuclear fusion, with the potential to provide virtually limitless, clean energy. The theoretical and numerical modeling of tokamak plasmas is simultaneously an essential component of effective reactor design, and a great research barrier. Tokamak operational conditions exhibit comparatively low Knudsen numbers. Kinetic effects, including kinetic waves and instabilities, Landau damping, bump-on-tail instabilities and more, are therefore highly influential in tokamak plasma dynamics. Purely fluid models are inherently incapable of capturing these effects, whereas the high dimensionality in purely kinetic models render them practically intractable for most relevant purposes.

        We consider a $\delta\!f$ decomposition model, with a macroscopic fluid background and microscopic kinetic correction, both fully coupled to each other. A similar manner of discretization is proposed to that used in the recent \texttt{STRUPHY} code \cite{Holderied_Possanner_Wang_2021, Holderied_2022, Li_et_al_2023} with a finite-element model for the background and a pseudo-particle/PiC model for the correction.

        The fluid background satisfies the full, non-linear, resistive, compressible, Hall MHD equations. \cite{Laakmann_Hu_Farrell_2022} introduces finite-element(-in-space) implicit timesteppers for the incompressible analogue to this system with structure-preserving (SP) properties in the ideal case, alongside parameter-robust preconditioners. We show that these timesteppers can derive from a finite-element-in-time (FET) (and finite-element-in-space) interpretation. The benefits of this reformulation are discussed, including the derivation of timesteppers that are higher order in time, and the quantifiable dissipative SP properties in the non-ideal, resistive case.
        
        We discuss possible options for extending this FET approach to timesteppers for the compressible case.

        The kinetic corrections satisfy linearized Boltzmann equations. Using a Lénard--Bernstein collision operator, these take Fokker--Planck-like forms \cite{Fokker_1914, Planck_1917} wherein pseudo-particles in the numerical model obey the neoclassical transport equations, with particle-independent Brownian drift terms. This offers a rigorous methodology for incorporating collisions into the particle transport model, without coupling the equations of motions for each particle.
        
        Works by Chen, Chacón et al. \cite{Chen_Chacón_Barnes_2011, Chacón_Chen_Barnes_2013, Chen_Chacón_2014, Chen_Chacón_2015} have developed structure-preserving particle pushers for neoclassical transport in the Vlasov equations, derived from Crank--Nicolson integrators. We show these too can can derive from a FET interpretation, similarly offering potential extensions to higher-order-in-time particle pushers. The FET formulation is used also to consider how the stochastic drift terms can be incorporated into the pushers. Stochastic gyrokinetic expansions are also discussed.

        Different options for the numerical implementation of these schemes are considered.

        Due to the efficacy of FET in the development of SP timesteppers for both the fluid and kinetic component, we hope this approach will prove effective in the future for developing SP timesteppers for the full hybrid model. We hope this will give us the opportunity to incorporate previously inaccessible kinetic effects into the highly effective, modern, finite-element MHD models.
    \end{abstract}
    
    
    \newpage
    \tableofcontents
    
    
    \newpage
    \pagenumbering{arabic}
    %\linenumbers\renewcommand\thelinenumber{\color{black!50}\arabic{linenumber}}
            \input{0 - introduction/main.tex}
        \part{Research}
            \input{1 - low-noise PiC models/main.tex}
            \input{2 - kinetic component/main.tex}
            \input{3 - fluid component/main.tex}
            \input{4 - numerical implementation/main.tex}
        \part{Project Overview}
            \input{5 - research plan/main.tex}
            \input{6 - summary/main.tex}
    
    
    %\section{}
    \newpage
    \pagenumbering{gobble}
        \printbibliography


    \newpage
    \pagenumbering{roman}
    \appendix
        \part{Appendices}
            \input{8 - Hilbert complexes/main.tex}
            \input{9 - weak conservation proofs/main.tex}
\end{document}

            \documentclass[12pt, a4paper]{report}

\input{template/main.tex}

\title{\BA{Title in Progress...}}
\author{Boris Andrews}
\affil{Mathematical Institute, University of Oxford}
\date{\today}


\begin{document}
    \pagenumbering{gobble}
    \maketitle
    
    
    \begin{abstract}
        Magnetic confinement reactors---in particular tokamaks---offer one of the most promising options for achieving practical nuclear fusion, with the potential to provide virtually limitless, clean energy. The theoretical and numerical modeling of tokamak plasmas is simultaneously an essential component of effective reactor design, and a great research barrier. Tokamak operational conditions exhibit comparatively low Knudsen numbers. Kinetic effects, including kinetic waves and instabilities, Landau damping, bump-on-tail instabilities and more, are therefore highly influential in tokamak plasma dynamics. Purely fluid models are inherently incapable of capturing these effects, whereas the high dimensionality in purely kinetic models render them practically intractable for most relevant purposes.

        We consider a $\delta\!f$ decomposition model, with a macroscopic fluid background and microscopic kinetic correction, both fully coupled to each other. A similar manner of discretization is proposed to that used in the recent \texttt{STRUPHY} code \cite{Holderied_Possanner_Wang_2021, Holderied_2022, Li_et_al_2023} with a finite-element model for the background and a pseudo-particle/PiC model for the correction.

        The fluid background satisfies the full, non-linear, resistive, compressible, Hall MHD equations. \cite{Laakmann_Hu_Farrell_2022} introduces finite-element(-in-space) implicit timesteppers for the incompressible analogue to this system with structure-preserving (SP) properties in the ideal case, alongside parameter-robust preconditioners. We show that these timesteppers can derive from a finite-element-in-time (FET) (and finite-element-in-space) interpretation. The benefits of this reformulation are discussed, including the derivation of timesteppers that are higher order in time, and the quantifiable dissipative SP properties in the non-ideal, resistive case.
        
        We discuss possible options for extending this FET approach to timesteppers for the compressible case.

        The kinetic corrections satisfy linearized Boltzmann equations. Using a Lénard--Bernstein collision operator, these take Fokker--Planck-like forms \cite{Fokker_1914, Planck_1917} wherein pseudo-particles in the numerical model obey the neoclassical transport equations, with particle-independent Brownian drift terms. This offers a rigorous methodology for incorporating collisions into the particle transport model, without coupling the equations of motions for each particle.
        
        Works by Chen, Chacón et al. \cite{Chen_Chacón_Barnes_2011, Chacón_Chen_Barnes_2013, Chen_Chacón_2014, Chen_Chacón_2015} have developed structure-preserving particle pushers for neoclassical transport in the Vlasov equations, derived from Crank--Nicolson integrators. We show these too can can derive from a FET interpretation, similarly offering potential extensions to higher-order-in-time particle pushers. The FET formulation is used also to consider how the stochastic drift terms can be incorporated into the pushers. Stochastic gyrokinetic expansions are also discussed.

        Different options for the numerical implementation of these schemes are considered.

        Due to the efficacy of FET in the development of SP timesteppers for both the fluid and kinetic component, we hope this approach will prove effective in the future for developing SP timesteppers for the full hybrid model. We hope this will give us the opportunity to incorporate previously inaccessible kinetic effects into the highly effective, modern, finite-element MHD models.
    \end{abstract}
    
    
    \newpage
    \tableofcontents
    
    
    \newpage
    \pagenumbering{arabic}
    %\linenumbers\renewcommand\thelinenumber{\color{black!50}\arabic{linenumber}}
            \input{0 - introduction/main.tex}
        \part{Research}
            \input{1 - low-noise PiC models/main.tex}
            \input{2 - kinetic component/main.tex}
            \input{3 - fluid component/main.tex}
            \input{4 - numerical implementation/main.tex}
        \part{Project Overview}
            \input{5 - research plan/main.tex}
            \input{6 - summary/main.tex}
    
    
    %\section{}
    \newpage
    \pagenumbering{gobble}
        \printbibliography


    \newpage
    \pagenumbering{roman}
    \appendix
        \part{Appendices}
            \input{8 - Hilbert complexes/main.tex}
            \input{9 - weak conservation proofs/main.tex}
\end{document}

    
    
    %\section{}
    \newpage
    \pagenumbering{gobble}
        \printbibliography


    \newpage
    \pagenumbering{roman}
    \appendix
        \part{Appendices}
            \documentclass[12pt, a4paper]{report}

\input{template/main.tex}

\title{\BA{Title in Progress...}}
\author{Boris Andrews}
\affil{Mathematical Institute, University of Oxford}
\date{\today}


\begin{document}
    \pagenumbering{gobble}
    \maketitle
    
    
    \begin{abstract}
        Magnetic confinement reactors---in particular tokamaks---offer one of the most promising options for achieving practical nuclear fusion, with the potential to provide virtually limitless, clean energy. The theoretical and numerical modeling of tokamak plasmas is simultaneously an essential component of effective reactor design, and a great research barrier. Tokamak operational conditions exhibit comparatively low Knudsen numbers. Kinetic effects, including kinetic waves and instabilities, Landau damping, bump-on-tail instabilities and more, are therefore highly influential in tokamak plasma dynamics. Purely fluid models are inherently incapable of capturing these effects, whereas the high dimensionality in purely kinetic models render them practically intractable for most relevant purposes.

        We consider a $\delta\!f$ decomposition model, with a macroscopic fluid background and microscopic kinetic correction, both fully coupled to each other. A similar manner of discretization is proposed to that used in the recent \texttt{STRUPHY} code \cite{Holderied_Possanner_Wang_2021, Holderied_2022, Li_et_al_2023} with a finite-element model for the background and a pseudo-particle/PiC model for the correction.

        The fluid background satisfies the full, non-linear, resistive, compressible, Hall MHD equations. \cite{Laakmann_Hu_Farrell_2022} introduces finite-element(-in-space) implicit timesteppers for the incompressible analogue to this system with structure-preserving (SP) properties in the ideal case, alongside parameter-robust preconditioners. We show that these timesteppers can derive from a finite-element-in-time (FET) (and finite-element-in-space) interpretation. The benefits of this reformulation are discussed, including the derivation of timesteppers that are higher order in time, and the quantifiable dissipative SP properties in the non-ideal, resistive case.
        
        We discuss possible options for extending this FET approach to timesteppers for the compressible case.

        The kinetic corrections satisfy linearized Boltzmann equations. Using a Lénard--Bernstein collision operator, these take Fokker--Planck-like forms \cite{Fokker_1914, Planck_1917} wherein pseudo-particles in the numerical model obey the neoclassical transport equations, with particle-independent Brownian drift terms. This offers a rigorous methodology for incorporating collisions into the particle transport model, without coupling the equations of motions for each particle.
        
        Works by Chen, Chacón et al. \cite{Chen_Chacón_Barnes_2011, Chacón_Chen_Barnes_2013, Chen_Chacón_2014, Chen_Chacón_2015} have developed structure-preserving particle pushers for neoclassical transport in the Vlasov equations, derived from Crank--Nicolson integrators. We show these too can can derive from a FET interpretation, similarly offering potential extensions to higher-order-in-time particle pushers. The FET formulation is used also to consider how the stochastic drift terms can be incorporated into the pushers. Stochastic gyrokinetic expansions are also discussed.

        Different options for the numerical implementation of these schemes are considered.

        Due to the efficacy of FET in the development of SP timesteppers for both the fluid and kinetic component, we hope this approach will prove effective in the future for developing SP timesteppers for the full hybrid model. We hope this will give us the opportunity to incorporate previously inaccessible kinetic effects into the highly effective, modern, finite-element MHD models.
    \end{abstract}
    
    
    \newpage
    \tableofcontents
    
    
    \newpage
    \pagenumbering{arabic}
    %\linenumbers\renewcommand\thelinenumber{\color{black!50}\arabic{linenumber}}
            \input{0 - introduction/main.tex}
        \part{Research}
            \input{1 - low-noise PiC models/main.tex}
            \input{2 - kinetic component/main.tex}
            \input{3 - fluid component/main.tex}
            \input{4 - numerical implementation/main.tex}
        \part{Project Overview}
            \input{5 - research plan/main.tex}
            \input{6 - summary/main.tex}
    
    
    %\section{}
    \newpage
    \pagenumbering{gobble}
        \printbibliography


    \newpage
    \pagenumbering{roman}
    \appendix
        \part{Appendices}
            \input{8 - Hilbert complexes/main.tex}
            \input{9 - weak conservation proofs/main.tex}
\end{document}

            \documentclass[12pt, a4paper]{report}

\input{template/main.tex}

\title{\BA{Title in Progress...}}
\author{Boris Andrews}
\affil{Mathematical Institute, University of Oxford}
\date{\today}


\begin{document}
    \pagenumbering{gobble}
    \maketitle
    
    
    \begin{abstract}
        Magnetic confinement reactors---in particular tokamaks---offer one of the most promising options for achieving practical nuclear fusion, with the potential to provide virtually limitless, clean energy. The theoretical and numerical modeling of tokamak plasmas is simultaneously an essential component of effective reactor design, and a great research barrier. Tokamak operational conditions exhibit comparatively low Knudsen numbers. Kinetic effects, including kinetic waves and instabilities, Landau damping, bump-on-tail instabilities and more, are therefore highly influential in tokamak plasma dynamics. Purely fluid models are inherently incapable of capturing these effects, whereas the high dimensionality in purely kinetic models render them practically intractable for most relevant purposes.

        We consider a $\delta\!f$ decomposition model, with a macroscopic fluid background and microscopic kinetic correction, both fully coupled to each other. A similar manner of discretization is proposed to that used in the recent \texttt{STRUPHY} code \cite{Holderied_Possanner_Wang_2021, Holderied_2022, Li_et_al_2023} with a finite-element model for the background and a pseudo-particle/PiC model for the correction.

        The fluid background satisfies the full, non-linear, resistive, compressible, Hall MHD equations. \cite{Laakmann_Hu_Farrell_2022} introduces finite-element(-in-space) implicit timesteppers for the incompressible analogue to this system with structure-preserving (SP) properties in the ideal case, alongside parameter-robust preconditioners. We show that these timesteppers can derive from a finite-element-in-time (FET) (and finite-element-in-space) interpretation. The benefits of this reformulation are discussed, including the derivation of timesteppers that are higher order in time, and the quantifiable dissipative SP properties in the non-ideal, resistive case.
        
        We discuss possible options for extending this FET approach to timesteppers for the compressible case.

        The kinetic corrections satisfy linearized Boltzmann equations. Using a Lénard--Bernstein collision operator, these take Fokker--Planck-like forms \cite{Fokker_1914, Planck_1917} wherein pseudo-particles in the numerical model obey the neoclassical transport equations, with particle-independent Brownian drift terms. This offers a rigorous methodology for incorporating collisions into the particle transport model, without coupling the equations of motions for each particle.
        
        Works by Chen, Chacón et al. \cite{Chen_Chacón_Barnes_2011, Chacón_Chen_Barnes_2013, Chen_Chacón_2014, Chen_Chacón_2015} have developed structure-preserving particle pushers for neoclassical transport in the Vlasov equations, derived from Crank--Nicolson integrators. We show these too can can derive from a FET interpretation, similarly offering potential extensions to higher-order-in-time particle pushers. The FET formulation is used also to consider how the stochastic drift terms can be incorporated into the pushers. Stochastic gyrokinetic expansions are also discussed.

        Different options for the numerical implementation of these schemes are considered.

        Due to the efficacy of FET in the development of SP timesteppers for both the fluid and kinetic component, we hope this approach will prove effective in the future for developing SP timesteppers for the full hybrid model. We hope this will give us the opportunity to incorporate previously inaccessible kinetic effects into the highly effective, modern, finite-element MHD models.
    \end{abstract}
    
    
    \newpage
    \tableofcontents
    
    
    \newpage
    \pagenumbering{arabic}
    %\linenumbers\renewcommand\thelinenumber{\color{black!50}\arabic{linenumber}}
            \input{0 - introduction/main.tex}
        \part{Research}
            \input{1 - low-noise PiC models/main.tex}
            \input{2 - kinetic component/main.tex}
            \input{3 - fluid component/main.tex}
            \input{4 - numerical implementation/main.tex}
        \part{Project Overview}
            \input{5 - research plan/main.tex}
            \input{6 - summary/main.tex}
    
    
    %\section{}
    \newpage
    \pagenumbering{gobble}
        \printbibliography


    \newpage
    \pagenumbering{roman}
    \appendix
        \part{Appendices}
            \input{8 - Hilbert complexes/main.tex}
            \input{9 - weak conservation proofs/main.tex}
\end{document}

\end{document}

    
    
    %\section{}
    \newpage
    \pagenumbering{gobble}
        \printbibliography


    \newpage
    \pagenumbering{roman}
    \appendix
        \part{Appendices}
            \documentclass[12pt, a4paper]{report}

\documentclass[12pt, a4paper]{report}

\input{template/main.tex}

\title{\BA{Title in Progress...}}
\author{Boris Andrews}
\affil{Mathematical Institute, University of Oxford}
\date{\today}


\begin{document}
    \pagenumbering{gobble}
    \maketitle
    
    
    \begin{abstract}
        Magnetic confinement reactors---in particular tokamaks---offer one of the most promising options for achieving practical nuclear fusion, with the potential to provide virtually limitless, clean energy. The theoretical and numerical modeling of tokamak plasmas is simultaneously an essential component of effective reactor design, and a great research barrier. Tokamak operational conditions exhibit comparatively low Knudsen numbers. Kinetic effects, including kinetic waves and instabilities, Landau damping, bump-on-tail instabilities and more, are therefore highly influential in tokamak plasma dynamics. Purely fluid models are inherently incapable of capturing these effects, whereas the high dimensionality in purely kinetic models render them practically intractable for most relevant purposes.

        We consider a $\delta\!f$ decomposition model, with a macroscopic fluid background and microscopic kinetic correction, both fully coupled to each other. A similar manner of discretization is proposed to that used in the recent \texttt{STRUPHY} code \cite{Holderied_Possanner_Wang_2021, Holderied_2022, Li_et_al_2023} with a finite-element model for the background and a pseudo-particle/PiC model for the correction.

        The fluid background satisfies the full, non-linear, resistive, compressible, Hall MHD equations. \cite{Laakmann_Hu_Farrell_2022} introduces finite-element(-in-space) implicit timesteppers for the incompressible analogue to this system with structure-preserving (SP) properties in the ideal case, alongside parameter-robust preconditioners. We show that these timesteppers can derive from a finite-element-in-time (FET) (and finite-element-in-space) interpretation. The benefits of this reformulation are discussed, including the derivation of timesteppers that are higher order in time, and the quantifiable dissipative SP properties in the non-ideal, resistive case.
        
        We discuss possible options for extending this FET approach to timesteppers for the compressible case.

        The kinetic corrections satisfy linearized Boltzmann equations. Using a Lénard--Bernstein collision operator, these take Fokker--Planck-like forms \cite{Fokker_1914, Planck_1917} wherein pseudo-particles in the numerical model obey the neoclassical transport equations, with particle-independent Brownian drift terms. This offers a rigorous methodology for incorporating collisions into the particle transport model, without coupling the equations of motions for each particle.
        
        Works by Chen, Chacón et al. \cite{Chen_Chacón_Barnes_2011, Chacón_Chen_Barnes_2013, Chen_Chacón_2014, Chen_Chacón_2015} have developed structure-preserving particle pushers for neoclassical transport in the Vlasov equations, derived from Crank--Nicolson integrators. We show these too can can derive from a FET interpretation, similarly offering potential extensions to higher-order-in-time particle pushers. The FET formulation is used also to consider how the stochastic drift terms can be incorporated into the pushers. Stochastic gyrokinetic expansions are also discussed.

        Different options for the numerical implementation of these schemes are considered.

        Due to the efficacy of FET in the development of SP timesteppers for both the fluid and kinetic component, we hope this approach will prove effective in the future for developing SP timesteppers for the full hybrid model. We hope this will give us the opportunity to incorporate previously inaccessible kinetic effects into the highly effective, modern, finite-element MHD models.
    \end{abstract}
    
    
    \newpage
    \tableofcontents
    
    
    \newpage
    \pagenumbering{arabic}
    %\linenumbers\renewcommand\thelinenumber{\color{black!50}\arabic{linenumber}}
            \input{0 - introduction/main.tex}
        \part{Research}
            \input{1 - low-noise PiC models/main.tex}
            \input{2 - kinetic component/main.tex}
            \input{3 - fluid component/main.tex}
            \input{4 - numerical implementation/main.tex}
        \part{Project Overview}
            \input{5 - research plan/main.tex}
            \input{6 - summary/main.tex}
    
    
    %\section{}
    \newpage
    \pagenumbering{gobble}
        \printbibliography


    \newpage
    \pagenumbering{roman}
    \appendix
        \part{Appendices}
            \input{8 - Hilbert complexes/main.tex}
            \input{9 - weak conservation proofs/main.tex}
\end{document}


\title{\BA{Title in Progress...}}
\author{Boris Andrews}
\affil{Mathematical Institute, University of Oxford}
\date{\today}


\begin{document}
    \pagenumbering{gobble}
    \maketitle
    
    
    \begin{abstract}
        Magnetic confinement reactors---in particular tokamaks---offer one of the most promising options for achieving practical nuclear fusion, with the potential to provide virtually limitless, clean energy. The theoretical and numerical modeling of tokamak plasmas is simultaneously an essential component of effective reactor design, and a great research barrier. Tokamak operational conditions exhibit comparatively low Knudsen numbers. Kinetic effects, including kinetic waves and instabilities, Landau damping, bump-on-tail instabilities and more, are therefore highly influential in tokamak plasma dynamics. Purely fluid models are inherently incapable of capturing these effects, whereas the high dimensionality in purely kinetic models render them practically intractable for most relevant purposes.

        We consider a $\delta\!f$ decomposition model, with a macroscopic fluid background and microscopic kinetic correction, both fully coupled to each other. A similar manner of discretization is proposed to that used in the recent \texttt{STRUPHY} code \cite{Holderied_Possanner_Wang_2021, Holderied_2022, Li_et_al_2023} with a finite-element model for the background and a pseudo-particle/PiC model for the correction.

        The fluid background satisfies the full, non-linear, resistive, compressible, Hall MHD equations. \cite{Laakmann_Hu_Farrell_2022} introduces finite-element(-in-space) implicit timesteppers for the incompressible analogue to this system with structure-preserving (SP) properties in the ideal case, alongside parameter-robust preconditioners. We show that these timesteppers can derive from a finite-element-in-time (FET) (and finite-element-in-space) interpretation. The benefits of this reformulation are discussed, including the derivation of timesteppers that are higher order in time, and the quantifiable dissipative SP properties in the non-ideal, resistive case.
        
        We discuss possible options for extending this FET approach to timesteppers for the compressible case.

        The kinetic corrections satisfy linearized Boltzmann equations. Using a Lénard--Bernstein collision operator, these take Fokker--Planck-like forms \cite{Fokker_1914, Planck_1917} wherein pseudo-particles in the numerical model obey the neoclassical transport equations, with particle-independent Brownian drift terms. This offers a rigorous methodology for incorporating collisions into the particle transport model, without coupling the equations of motions for each particle.
        
        Works by Chen, Chacón et al. \cite{Chen_Chacón_Barnes_2011, Chacón_Chen_Barnes_2013, Chen_Chacón_2014, Chen_Chacón_2015} have developed structure-preserving particle pushers for neoclassical transport in the Vlasov equations, derived from Crank--Nicolson integrators. We show these too can can derive from a FET interpretation, similarly offering potential extensions to higher-order-in-time particle pushers. The FET formulation is used also to consider how the stochastic drift terms can be incorporated into the pushers. Stochastic gyrokinetic expansions are also discussed.

        Different options for the numerical implementation of these schemes are considered.

        Due to the efficacy of FET in the development of SP timesteppers for both the fluid and kinetic component, we hope this approach will prove effective in the future for developing SP timesteppers for the full hybrid model. We hope this will give us the opportunity to incorporate previously inaccessible kinetic effects into the highly effective, modern, finite-element MHD models.
    \end{abstract}
    
    
    \newpage
    \tableofcontents
    
    
    \newpage
    \pagenumbering{arabic}
    %\linenumbers\renewcommand\thelinenumber{\color{black!50}\arabic{linenumber}}
            \documentclass[12pt, a4paper]{report}

\input{template/main.tex}

\title{\BA{Title in Progress...}}
\author{Boris Andrews}
\affil{Mathematical Institute, University of Oxford}
\date{\today}


\begin{document}
    \pagenumbering{gobble}
    \maketitle
    
    
    \begin{abstract}
        Magnetic confinement reactors---in particular tokamaks---offer one of the most promising options for achieving practical nuclear fusion, with the potential to provide virtually limitless, clean energy. The theoretical and numerical modeling of tokamak plasmas is simultaneously an essential component of effective reactor design, and a great research barrier. Tokamak operational conditions exhibit comparatively low Knudsen numbers. Kinetic effects, including kinetic waves and instabilities, Landau damping, bump-on-tail instabilities and more, are therefore highly influential in tokamak plasma dynamics. Purely fluid models are inherently incapable of capturing these effects, whereas the high dimensionality in purely kinetic models render them practically intractable for most relevant purposes.

        We consider a $\delta\!f$ decomposition model, with a macroscopic fluid background and microscopic kinetic correction, both fully coupled to each other. A similar manner of discretization is proposed to that used in the recent \texttt{STRUPHY} code \cite{Holderied_Possanner_Wang_2021, Holderied_2022, Li_et_al_2023} with a finite-element model for the background and a pseudo-particle/PiC model for the correction.

        The fluid background satisfies the full, non-linear, resistive, compressible, Hall MHD equations. \cite{Laakmann_Hu_Farrell_2022} introduces finite-element(-in-space) implicit timesteppers for the incompressible analogue to this system with structure-preserving (SP) properties in the ideal case, alongside parameter-robust preconditioners. We show that these timesteppers can derive from a finite-element-in-time (FET) (and finite-element-in-space) interpretation. The benefits of this reformulation are discussed, including the derivation of timesteppers that are higher order in time, and the quantifiable dissipative SP properties in the non-ideal, resistive case.
        
        We discuss possible options for extending this FET approach to timesteppers for the compressible case.

        The kinetic corrections satisfy linearized Boltzmann equations. Using a Lénard--Bernstein collision operator, these take Fokker--Planck-like forms \cite{Fokker_1914, Planck_1917} wherein pseudo-particles in the numerical model obey the neoclassical transport equations, with particle-independent Brownian drift terms. This offers a rigorous methodology for incorporating collisions into the particle transport model, without coupling the equations of motions for each particle.
        
        Works by Chen, Chacón et al. \cite{Chen_Chacón_Barnes_2011, Chacón_Chen_Barnes_2013, Chen_Chacón_2014, Chen_Chacón_2015} have developed structure-preserving particle pushers for neoclassical transport in the Vlasov equations, derived from Crank--Nicolson integrators. We show these too can can derive from a FET interpretation, similarly offering potential extensions to higher-order-in-time particle pushers. The FET formulation is used also to consider how the stochastic drift terms can be incorporated into the pushers. Stochastic gyrokinetic expansions are also discussed.

        Different options for the numerical implementation of these schemes are considered.

        Due to the efficacy of FET in the development of SP timesteppers for both the fluid and kinetic component, we hope this approach will prove effective in the future for developing SP timesteppers for the full hybrid model. We hope this will give us the opportunity to incorporate previously inaccessible kinetic effects into the highly effective, modern, finite-element MHD models.
    \end{abstract}
    
    
    \newpage
    \tableofcontents
    
    
    \newpage
    \pagenumbering{arabic}
    %\linenumbers\renewcommand\thelinenumber{\color{black!50}\arabic{linenumber}}
            \input{0 - introduction/main.tex}
        \part{Research}
            \input{1 - low-noise PiC models/main.tex}
            \input{2 - kinetic component/main.tex}
            \input{3 - fluid component/main.tex}
            \input{4 - numerical implementation/main.tex}
        \part{Project Overview}
            \input{5 - research plan/main.tex}
            \input{6 - summary/main.tex}
    
    
    %\section{}
    \newpage
    \pagenumbering{gobble}
        \printbibliography


    \newpage
    \pagenumbering{roman}
    \appendix
        \part{Appendices}
            \input{8 - Hilbert complexes/main.tex}
            \input{9 - weak conservation proofs/main.tex}
\end{document}

        \part{Research}
            \documentclass[12pt, a4paper]{report}

\input{template/main.tex}

\title{\BA{Title in Progress...}}
\author{Boris Andrews}
\affil{Mathematical Institute, University of Oxford}
\date{\today}


\begin{document}
    \pagenumbering{gobble}
    \maketitle
    
    
    \begin{abstract}
        Magnetic confinement reactors---in particular tokamaks---offer one of the most promising options for achieving practical nuclear fusion, with the potential to provide virtually limitless, clean energy. The theoretical and numerical modeling of tokamak plasmas is simultaneously an essential component of effective reactor design, and a great research barrier. Tokamak operational conditions exhibit comparatively low Knudsen numbers. Kinetic effects, including kinetic waves and instabilities, Landau damping, bump-on-tail instabilities and more, are therefore highly influential in tokamak plasma dynamics. Purely fluid models are inherently incapable of capturing these effects, whereas the high dimensionality in purely kinetic models render them practically intractable for most relevant purposes.

        We consider a $\delta\!f$ decomposition model, with a macroscopic fluid background and microscopic kinetic correction, both fully coupled to each other. A similar manner of discretization is proposed to that used in the recent \texttt{STRUPHY} code \cite{Holderied_Possanner_Wang_2021, Holderied_2022, Li_et_al_2023} with a finite-element model for the background and a pseudo-particle/PiC model for the correction.

        The fluid background satisfies the full, non-linear, resistive, compressible, Hall MHD equations. \cite{Laakmann_Hu_Farrell_2022} introduces finite-element(-in-space) implicit timesteppers for the incompressible analogue to this system with structure-preserving (SP) properties in the ideal case, alongside parameter-robust preconditioners. We show that these timesteppers can derive from a finite-element-in-time (FET) (and finite-element-in-space) interpretation. The benefits of this reformulation are discussed, including the derivation of timesteppers that are higher order in time, and the quantifiable dissipative SP properties in the non-ideal, resistive case.
        
        We discuss possible options for extending this FET approach to timesteppers for the compressible case.

        The kinetic corrections satisfy linearized Boltzmann equations. Using a Lénard--Bernstein collision operator, these take Fokker--Planck-like forms \cite{Fokker_1914, Planck_1917} wherein pseudo-particles in the numerical model obey the neoclassical transport equations, with particle-independent Brownian drift terms. This offers a rigorous methodology for incorporating collisions into the particle transport model, without coupling the equations of motions for each particle.
        
        Works by Chen, Chacón et al. \cite{Chen_Chacón_Barnes_2011, Chacón_Chen_Barnes_2013, Chen_Chacón_2014, Chen_Chacón_2015} have developed structure-preserving particle pushers for neoclassical transport in the Vlasov equations, derived from Crank--Nicolson integrators. We show these too can can derive from a FET interpretation, similarly offering potential extensions to higher-order-in-time particle pushers. The FET formulation is used also to consider how the stochastic drift terms can be incorporated into the pushers. Stochastic gyrokinetic expansions are also discussed.

        Different options for the numerical implementation of these schemes are considered.

        Due to the efficacy of FET in the development of SP timesteppers for both the fluid and kinetic component, we hope this approach will prove effective in the future for developing SP timesteppers for the full hybrid model. We hope this will give us the opportunity to incorporate previously inaccessible kinetic effects into the highly effective, modern, finite-element MHD models.
    \end{abstract}
    
    
    \newpage
    \tableofcontents
    
    
    \newpage
    \pagenumbering{arabic}
    %\linenumbers\renewcommand\thelinenumber{\color{black!50}\arabic{linenumber}}
            \input{0 - introduction/main.tex}
        \part{Research}
            \input{1 - low-noise PiC models/main.tex}
            \input{2 - kinetic component/main.tex}
            \input{3 - fluid component/main.tex}
            \input{4 - numerical implementation/main.tex}
        \part{Project Overview}
            \input{5 - research plan/main.tex}
            \input{6 - summary/main.tex}
    
    
    %\section{}
    \newpage
    \pagenumbering{gobble}
        \printbibliography


    \newpage
    \pagenumbering{roman}
    \appendix
        \part{Appendices}
            \input{8 - Hilbert complexes/main.tex}
            \input{9 - weak conservation proofs/main.tex}
\end{document}

            \documentclass[12pt, a4paper]{report}

\input{template/main.tex}

\title{\BA{Title in Progress...}}
\author{Boris Andrews}
\affil{Mathematical Institute, University of Oxford}
\date{\today}


\begin{document}
    \pagenumbering{gobble}
    \maketitle
    
    
    \begin{abstract}
        Magnetic confinement reactors---in particular tokamaks---offer one of the most promising options for achieving practical nuclear fusion, with the potential to provide virtually limitless, clean energy. The theoretical and numerical modeling of tokamak plasmas is simultaneously an essential component of effective reactor design, and a great research barrier. Tokamak operational conditions exhibit comparatively low Knudsen numbers. Kinetic effects, including kinetic waves and instabilities, Landau damping, bump-on-tail instabilities and more, are therefore highly influential in tokamak plasma dynamics. Purely fluid models are inherently incapable of capturing these effects, whereas the high dimensionality in purely kinetic models render them practically intractable for most relevant purposes.

        We consider a $\delta\!f$ decomposition model, with a macroscopic fluid background and microscopic kinetic correction, both fully coupled to each other. A similar manner of discretization is proposed to that used in the recent \texttt{STRUPHY} code \cite{Holderied_Possanner_Wang_2021, Holderied_2022, Li_et_al_2023} with a finite-element model for the background and a pseudo-particle/PiC model for the correction.

        The fluid background satisfies the full, non-linear, resistive, compressible, Hall MHD equations. \cite{Laakmann_Hu_Farrell_2022} introduces finite-element(-in-space) implicit timesteppers for the incompressible analogue to this system with structure-preserving (SP) properties in the ideal case, alongside parameter-robust preconditioners. We show that these timesteppers can derive from a finite-element-in-time (FET) (and finite-element-in-space) interpretation. The benefits of this reformulation are discussed, including the derivation of timesteppers that are higher order in time, and the quantifiable dissipative SP properties in the non-ideal, resistive case.
        
        We discuss possible options for extending this FET approach to timesteppers for the compressible case.

        The kinetic corrections satisfy linearized Boltzmann equations. Using a Lénard--Bernstein collision operator, these take Fokker--Planck-like forms \cite{Fokker_1914, Planck_1917} wherein pseudo-particles in the numerical model obey the neoclassical transport equations, with particle-independent Brownian drift terms. This offers a rigorous methodology for incorporating collisions into the particle transport model, without coupling the equations of motions for each particle.
        
        Works by Chen, Chacón et al. \cite{Chen_Chacón_Barnes_2011, Chacón_Chen_Barnes_2013, Chen_Chacón_2014, Chen_Chacón_2015} have developed structure-preserving particle pushers for neoclassical transport in the Vlasov equations, derived from Crank--Nicolson integrators. We show these too can can derive from a FET interpretation, similarly offering potential extensions to higher-order-in-time particle pushers. The FET formulation is used also to consider how the stochastic drift terms can be incorporated into the pushers. Stochastic gyrokinetic expansions are also discussed.

        Different options for the numerical implementation of these schemes are considered.

        Due to the efficacy of FET in the development of SP timesteppers for both the fluid and kinetic component, we hope this approach will prove effective in the future for developing SP timesteppers for the full hybrid model. We hope this will give us the opportunity to incorporate previously inaccessible kinetic effects into the highly effective, modern, finite-element MHD models.
    \end{abstract}
    
    
    \newpage
    \tableofcontents
    
    
    \newpage
    \pagenumbering{arabic}
    %\linenumbers\renewcommand\thelinenumber{\color{black!50}\arabic{linenumber}}
            \input{0 - introduction/main.tex}
        \part{Research}
            \input{1 - low-noise PiC models/main.tex}
            \input{2 - kinetic component/main.tex}
            \input{3 - fluid component/main.tex}
            \input{4 - numerical implementation/main.tex}
        \part{Project Overview}
            \input{5 - research plan/main.tex}
            \input{6 - summary/main.tex}
    
    
    %\section{}
    \newpage
    \pagenumbering{gobble}
        \printbibliography


    \newpage
    \pagenumbering{roman}
    \appendix
        \part{Appendices}
            \input{8 - Hilbert complexes/main.tex}
            \input{9 - weak conservation proofs/main.tex}
\end{document}

            \documentclass[12pt, a4paper]{report}

\input{template/main.tex}

\title{\BA{Title in Progress...}}
\author{Boris Andrews}
\affil{Mathematical Institute, University of Oxford}
\date{\today}


\begin{document}
    \pagenumbering{gobble}
    \maketitle
    
    
    \begin{abstract}
        Magnetic confinement reactors---in particular tokamaks---offer one of the most promising options for achieving practical nuclear fusion, with the potential to provide virtually limitless, clean energy. The theoretical and numerical modeling of tokamak plasmas is simultaneously an essential component of effective reactor design, and a great research barrier. Tokamak operational conditions exhibit comparatively low Knudsen numbers. Kinetic effects, including kinetic waves and instabilities, Landau damping, bump-on-tail instabilities and more, are therefore highly influential in tokamak plasma dynamics. Purely fluid models are inherently incapable of capturing these effects, whereas the high dimensionality in purely kinetic models render them practically intractable for most relevant purposes.

        We consider a $\delta\!f$ decomposition model, with a macroscopic fluid background and microscopic kinetic correction, both fully coupled to each other. A similar manner of discretization is proposed to that used in the recent \texttt{STRUPHY} code \cite{Holderied_Possanner_Wang_2021, Holderied_2022, Li_et_al_2023} with a finite-element model for the background and a pseudo-particle/PiC model for the correction.

        The fluid background satisfies the full, non-linear, resistive, compressible, Hall MHD equations. \cite{Laakmann_Hu_Farrell_2022} introduces finite-element(-in-space) implicit timesteppers for the incompressible analogue to this system with structure-preserving (SP) properties in the ideal case, alongside parameter-robust preconditioners. We show that these timesteppers can derive from a finite-element-in-time (FET) (and finite-element-in-space) interpretation. The benefits of this reformulation are discussed, including the derivation of timesteppers that are higher order in time, and the quantifiable dissipative SP properties in the non-ideal, resistive case.
        
        We discuss possible options for extending this FET approach to timesteppers for the compressible case.

        The kinetic corrections satisfy linearized Boltzmann equations. Using a Lénard--Bernstein collision operator, these take Fokker--Planck-like forms \cite{Fokker_1914, Planck_1917} wherein pseudo-particles in the numerical model obey the neoclassical transport equations, with particle-independent Brownian drift terms. This offers a rigorous methodology for incorporating collisions into the particle transport model, without coupling the equations of motions for each particle.
        
        Works by Chen, Chacón et al. \cite{Chen_Chacón_Barnes_2011, Chacón_Chen_Barnes_2013, Chen_Chacón_2014, Chen_Chacón_2015} have developed structure-preserving particle pushers for neoclassical transport in the Vlasov equations, derived from Crank--Nicolson integrators. We show these too can can derive from a FET interpretation, similarly offering potential extensions to higher-order-in-time particle pushers. The FET formulation is used also to consider how the stochastic drift terms can be incorporated into the pushers. Stochastic gyrokinetic expansions are also discussed.

        Different options for the numerical implementation of these schemes are considered.

        Due to the efficacy of FET in the development of SP timesteppers for both the fluid and kinetic component, we hope this approach will prove effective in the future for developing SP timesteppers for the full hybrid model. We hope this will give us the opportunity to incorporate previously inaccessible kinetic effects into the highly effective, modern, finite-element MHD models.
    \end{abstract}
    
    
    \newpage
    \tableofcontents
    
    
    \newpage
    \pagenumbering{arabic}
    %\linenumbers\renewcommand\thelinenumber{\color{black!50}\arabic{linenumber}}
            \input{0 - introduction/main.tex}
        \part{Research}
            \input{1 - low-noise PiC models/main.tex}
            \input{2 - kinetic component/main.tex}
            \input{3 - fluid component/main.tex}
            \input{4 - numerical implementation/main.tex}
        \part{Project Overview}
            \input{5 - research plan/main.tex}
            \input{6 - summary/main.tex}
    
    
    %\section{}
    \newpage
    \pagenumbering{gobble}
        \printbibliography


    \newpage
    \pagenumbering{roman}
    \appendix
        \part{Appendices}
            \input{8 - Hilbert complexes/main.tex}
            \input{9 - weak conservation proofs/main.tex}
\end{document}

            \documentclass[12pt, a4paper]{report}

\input{template/main.tex}

\title{\BA{Title in Progress...}}
\author{Boris Andrews}
\affil{Mathematical Institute, University of Oxford}
\date{\today}


\begin{document}
    \pagenumbering{gobble}
    \maketitle
    
    
    \begin{abstract}
        Magnetic confinement reactors---in particular tokamaks---offer one of the most promising options for achieving practical nuclear fusion, with the potential to provide virtually limitless, clean energy. The theoretical and numerical modeling of tokamak plasmas is simultaneously an essential component of effective reactor design, and a great research barrier. Tokamak operational conditions exhibit comparatively low Knudsen numbers. Kinetic effects, including kinetic waves and instabilities, Landau damping, bump-on-tail instabilities and more, are therefore highly influential in tokamak plasma dynamics. Purely fluid models are inherently incapable of capturing these effects, whereas the high dimensionality in purely kinetic models render them practically intractable for most relevant purposes.

        We consider a $\delta\!f$ decomposition model, with a macroscopic fluid background and microscopic kinetic correction, both fully coupled to each other. A similar manner of discretization is proposed to that used in the recent \texttt{STRUPHY} code \cite{Holderied_Possanner_Wang_2021, Holderied_2022, Li_et_al_2023} with a finite-element model for the background and a pseudo-particle/PiC model for the correction.

        The fluid background satisfies the full, non-linear, resistive, compressible, Hall MHD equations. \cite{Laakmann_Hu_Farrell_2022} introduces finite-element(-in-space) implicit timesteppers for the incompressible analogue to this system with structure-preserving (SP) properties in the ideal case, alongside parameter-robust preconditioners. We show that these timesteppers can derive from a finite-element-in-time (FET) (and finite-element-in-space) interpretation. The benefits of this reformulation are discussed, including the derivation of timesteppers that are higher order in time, and the quantifiable dissipative SP properties in the non-ideal, resistive case.
        
        We discuss possible options for extending this FET approach to timesteppers for the compressible case.

        The kinetic corrections satisfy linearized Boltzmann equations. Using a Lénard--Bernstein collision operator, these take Fokker--Planck-like forms \cite{Fokker_1914, Planck_1917} wherein pseudo-particles in the numerical model obey the neoclassical transport equations, with particle-independent Brownian drift terms. This offers a rigorous methodology for incorporating collisions into the particle transport model, without coupling the equations of motions for each particle.
        
        Works by Chen, Chacón et al. \cite{Chen_Chacón_Barnes_2011, Chacón_Chen_Barnes_2013, Chen_Chacón_2014, Chen_Chacón_2015} have developed structure-preserving particle pushers for neoclassical transport in the Vlasov equations, derived from Crank--Nicolson integrators. We show these too can can derive from a FET interpretation, similarly offering potential extensions to higher-order-in-time particle pushers. The FET formulation is used also to consider how the stochastic drift terms can be incorporated into the pushers. Stochastic gyrokinetic expansions are also discussed.

        Different options for the numerical implementation of these schemes are considered.

        Due to the efficacy of FET in the development of SP timesteppers for both the fluid and kinetic component, we hope this approach will prove effective in the future for developing SP timesteppers for the full hybrid model. We hope this will give us the opportunity to incorporate previously inaccessible kinetic effects into the highly effective, modern, finite-element MHD models.
    \end{abstract}
    
    
    \newpage
    \tableofcontents
    
    
    \newpage
    \pagenumbering{arabic}
    %\linenumbers\renewcommand\thelinenumber{\color{black!50}\arabic{linenumber}}
            \input{0 - introduction/main.tex}
        \part{Research}
            \input{1 - low-noise PiC models/main.tex}
            \input{2 - kinetic component/main.tex}
            \input{3 - fluid component/main.tex}
            \input{4 - numerical implementation/main.tex}
        \part{Project Overview}
            \input{5 - research plan/main.tex}
            \input{6 - summary/main.tex}
    
    
    %\section{}
    \newpage
    \pagenumbering{gobble}
        \printbibliography


    \newpage
    \pagenumbering{roman}
    \appendix
        \part{Appendices}
            \input{8 - Hilbert complexes/main.tex}
            \input{9 - weak conservation proofs/main.tex}
\end{document}

        \part{Project Overview}
            \documentclass[12pt, a4paper]{report}

\input{template/main.tex}

\title{\BA{Title in Progress...}}
\author{Boris Andrews}
\affil{Mathematical Institute, University of Oxford}
\date{\today}


\begin{document}
    \pagenumbering{gobble}
    \maketitle
    
    
    \begin{abstract}
        Magnetic confinement reactors---in particular tokamaks---offer one of the most promising options for achieving practical nuclear fusion, with the potential to provide virtually limitless, clean energy. The theoretical and numerical modeling of tokamak plasmas is simultaneously an essential component of effective reactor design, and a great research barrier. Tokamak operational conditions exhibit comparatively low Knudsen numbers. Kinetic effects, including kinetic waves and instabilities, Landau damping, bump-on-tail instabilities and more, are therefore highly influential in tokamak plasma dynamics. Purely fluid models are inherently incapable of capturing these effects, whereas the high dimensionality in purely kinetic models render them practically intractable for most relevant purposes.

        We consider a $\delta\!f$ decomposition model, with a macroscopic fluid background and microscopic kinetic correction, both fully coupled to each other. A similar manner of discretization is proposed to that used in the recent \texttt{STRUPHY} code \cite{Holderied_Possanner_Wang_2021, Holderied_2022, Li_et_al_2023} with a finite-element model for the background and a pseudo-particle/PiC model for the correction.

        The fluid background satisfies the full, non-linear, resistive, compressible, Hall MHD equations. \cite{Laakmann_Hu_Farrell_2022} introduces finite-element(-in-space) implicit timesteppers for the incompressible analogue to this system with structure-preserving (SP) properties in the ideal case, alongside parameter-robust preconditioners. We show that these timesteppers can derive from a finite-element-in-time (FET) (and finite-element-in-space) interpretation. The benefits of this reformulation are discussed, including the derivation of timesteppers that are higher order in time, and the quantifiable dissipative SP properties in the non-ideal, resistive case.
        
        We discuss possible options for extending this FET approach to timesteppers for the compressible case.

        The kinetic corrections satisfy linearized Boltzmann equations. Using a Lénard--Bernstein collision operator, these take Fokker--Planck-like forms \cite{Fokker_1914, Planck_1917} wherein pseudo-particles in the numerical model obey the neoclassical transport equations, with particle-independent Brownian drift terms. This offers a rigorous methodology for incorporating collisions into the particle transport model, without coupling the equations of motions for each particle.
        
        Works by Chen, Chacón et al. \cite{Chen_Chacón_Barnes_2011, Chacón_Chen_Barnes_2013, Chen_Chacón_2014, Chen_Chacón_2015} have developed structure-preserving particle pushers for neoclassical transport in the Vlasov equations, derived from Crank--Nicolson integrators. We show these too can can derive from a FET interpretation, similarly offering potential extensions to higher-order-in-time particle pushers. The FET formulation is used also to consider how the stochastic drift terms can be incorporated into the pushers. Stochastic gyrokinetic expansions are also discussed.

        Different options for the numerical implementation of these schemes are considered.

        Due to the efficacy of FET in the development of SP timesteppers for both the fluid and kinetic component, we hope this approach will prove effective in the future for developing SP timesteppers for the full hybrid model. We hope this will give us the opportunity to incorporate previously inaccessible kinetic effects into the highly effective, modern, finite-element MHD models.
    \end{abstract}
    
    
    \newpage
    \tableofcontents
    
    
    \newpage
    \pagenumbering{arabic}
    %\linenumbers\renewcommand\thelinenumber{\color{black!50}\arabic{linenumber}}
            \input{0 - introduction/main.tex}
        \part{Research}
            \input{1 - low-noise PiC models/main.tex}
            \input{2 - kinetic component/main.tex}
            \input{3 - fluid component/main.tex}
            \input{4 - numerical implementation/main.tex}
        \part{Project Overview}
            \input{5 - research plan/main.tex}
            \input{6 - summary/main.tex}
    
    
    %\section{}
    \newpage
    \pagenumbering{gobble}
        \printbibliography


    \newpage
    \pagenumbering{roman}
    \appendix
        \part{Appendices}
            \input{8 - Hilbert complexes/main.tex}
            \input{9 - weak conservation proofs/main.tex}
\end{document}

            \documentclass[12pt, a4paper]{report}

\input{template/main.tex}

\title{\BA{Title in Progress...}}
\author{Boris Andrews}
\affil{Mathematical Institute, University of Oxford}
\date{\today}


\begin{document}
    \pagenumbering{gobble}
    \maketitle
    
    
    \begin{abstract}
        Magnetic confinement reactors---in particular tokamaks---offer one of the most promising options for achieving practical nuclear fusion, with the potential to provide virtually limitless, clean energy. The theoretical and numerical modeling of tokamak plasmas is simultaneously an essential component of effective reactor design, and a great research barrier. Tokamak operational conditions exhibit comparatively low Knudsen numbers. Kinetic effects, including kinetic waves and instabilities, Landau damping, bump-on-tail instabilities and more, are therefore highly influential in tokamak plasma dynamics. Purely fluid models are inherently incapable of capturing these effects, whereas the high dimensionality in purely kinetic models render them practically intractable for most relevant purposes.

        We consider a $\delta\!f$ decomposition model, with a macroscopic fluid background and microscopic kinetic correction, both fully coupled to each other. A similar manner of discretization is proposed to that used in the recent \texttt{STRUPHY} code \cite{Holderied_Possanner_Wang_2021, Holderied_2022, Li_et_al_2023} with a finite-element model for the background and a pseudo-particle/PiC model for the correction.

        The fluid background satisfies the full, non-linear, resistive, compressible, Hall MHD equations. \cite{Laakmann_Hu_Farrell_2022} introduces finite-element(-in-space) implicit timesteppers for the incompressible analogue to this system with structure-preserving (SP) properties in the ideal case, alongside parameter-robust preconditioners. We show that these timesteppers can derive from a finite-element-in-time (FET) (and finite-element-in-space) interpretation. The benefits of this reformulation are discussed, including the derivation of timesteppers that are higher order in time, and the quantifiable dissipative SP properties in the non-ideal, resistive case.
        
        We discuss possible options for extending this FET approach to timesteppers for the compressible case.

        The kinetic corrections satisfy linearized Boltzmann equations. Using a Lénard--Bernstein collision operator, these take Fokker--Planck-like forms \cite{Fokker_1914, Planck_1917} wherein pseudo-particles in the numerical model obey the neoclassical transport equations, with particle-independent Brownian drift terms. This offers a rigorous methodology for incorporating collisions into the particle transport model, without coupling the equations of motions for each particle.
        
        Works by Chen, Chacón et al. \cite{Chen_Chacón_Barnes_2011, Chacón_Chen_Barnes_2013, Chen_Chacón_2014, Chen_Chacón_2015} have developed structure-preserving particle pushers for neoclassical transport in the Vlasov equations, derived from Crank--Nicolson integrators. We show these too can can derive from a FET interpretation, similarly offering potential extensions to higher-order-in-time particle pushers. The FET formulation is used also to consider how the stochastic drift terms can be incorporated into the pushers. Stochastic gyrokinetic expansions are also discussed.

        Different options for the numerical implementation of these schemes are considered.

        Due to the efficacy of FET in the development of SP timesteppers for both the fluid and kinetic component, we hope this approach will prove effective in the future for developing SP timesteppers for the full hybrid model. We hope this will give us the opportunity to incorporate previously inaccessible kinetic effects into the highly effective, modern, finite-element MHD models.
    \end{abstract}
    
    
    \newpage
    \tableofcontents
    
    
    \newpage
    \pagenumbering{arabic}
    %\linenumbers\renewcommand\thelinenumber{\color{black!50}\arabic{linenumber}}
            \input{0 - introduction/main.tex}
        \part{Research}
            \input{1 - low-noise PiC models/main.tex}
            \input{2 - kinetic component/main.tex}
            \input{3 - fluid component/main.tex}
            \input{4 - numerical implementation/main.tex}
        \part{Project Overview}
            \input{5 - research plan/main.tex}
            \input{6 - summary/main.tex}
    
    
    %\section{}
    \newpage
    \pagenumbering{gobble}
        \printbibliography


    \newpage
    \pagenumbering{roman}
    \appendix
        \part{Appendices}
            \input{8 - Hilbert complexes/main.tex}
            \input{9 - weak conservation proofs/main.tex}
\end{document}

    
    
    %\section{}
    \newpage
    \pagenumbering{gobble}
        \printbibliography


    \newpage
    \pagenumbering{roman}
    \appendix
        \part{Appendices}
            \documentclass[12pt, a4paper]{report}

\input{template/main.tex}

\title{\BA{Title in Progress...}}
\author{Boris Andrews}
\affil{Mathematical Institute, University of Oxford}
\date{\today}


\begin{document}
    \pagenumbering{gobble}
    \maketitle
    
    
    \begin{abstract}
        Magnetic confinement reactors---in particular tokamaks---offer one of the most promising options for achieving practical nuclear fusion, with the potential to provide virtually limitless, clean energy. The theoretical and numerical modeling of tokamak plasmas is simultaneously an essential component of effective reactor design, and a great research barrier. Tokamak operational conditions exhibit comparatively low Knudsen numbers. Kinetic effects, including kinetic waves and instabilities, Landau damping, bump-on-tail instabilities and more, are therefore highly influential in tokamak plasma dynamics. Purely fluid models are inherently incapable of capturing these effects, whereas the high dimensionality in purely kinetic models render them practically intractable for most relevant purposes.

        We consider a $\delta\!f$ decomposition model, with a macroscopic fluid background and microscopic kinetic correction, both fully coupled to each other. A similar manner of discretization is proposed to that used in the recent \texttt{STRUPHY} code \cite{Holderied_Possanner_Wang_2021, Holderied_2022, Li_et_al_2023} with a finite-element model for the background and a pseudo-particle/PiC model for the correction.

        The fluid background satisfies the full, non-linear, resistive, compressible, Hall MHD equations. \cite{Laakmann_Hu_Farrell_2022} introduces finite-element(-in-space) implicit timesteppers for the incompressible analogue to this system with structure-preserving (SP) properties in the ideal case, alongside parameter-robust preconditioners. We show that these timesteppers can derive from a finite-element-in-time (FET) (and finite-element-in-space) interpretation. The benefits of this reformulation are discussed, including the derivation of timesteppers that are higher order in time, and the quantifiable dissipative SP properties in the non-ideal, resistive case.
        
        We discuss possible options for extending this FET approach to timesteppers for the compressible case.

        The kinetic corrections satisfy linearized Boltzmann equations. Using a Lénard--Bernstein collision operator, these take Fokker--Planck-like forms \cite{Fokker_1914, Planck_1917} wherein pseudo-particles in the numerical model obey the neoclassical transport equations, with particle-independent Brownian drift terms. This offers a rigorous methodology for incorporating collisions into the particle transport model, without coupling the equations of motions for each particle.
        
        Works by Chen, Chacón et al. \cite{Chen_Chacón_Barnes_2011, Chacón_Chen_Barnes_2013, Chen_Chacón_2014, Chen_Chacón_2015} have developed structure-preserving particle pushers for neoclassical transport in the Vlasov equations, derived from Crank--Nicolson integrators. We show these too can can derive from a FET interpretation, similarly offering potential extensions to higher-order-in-time particle pushers. The FET formulation is used also to consider how the stochastic drift terms can be incorporated into the pushers. Stochastic gyrokinetic expansions are also discussed.

        Different options for the numerical implementation of these schemes are considered.

        Due to the efficacy of FET in the development of SP timesteppers for both the fluid and kinetic component, we hope this approach will prove effective in the future for developing SP timesteppers for the full hybrid model. We hope this will give us the opportunity to incorporate previously inaccessible kinetic effects into the highly effective, modern, finite-element MHD models.
    \end{abstract}
    
    
    \newpage
    \tableofcontents
    
    
    \newpage
    \pagenumbering{arabic}
    %\linenumbers\renewcommand\thelinenumber{\color{black!50}\arabic{linenumber}}
            \input{0 - introduction/main.tex}
        \part{Research}
            \input{1 - low-noise PiC models/main.tex}
            \input{2 - kinetic component/main.tex}
            \input{3 - fluid component/main.tex}
            \input{4 - numerical implementation/main.tex}
        \part{Project Overview}
            \input{5 - research plan/main.tex}
            \input{6 - summary/main.tex}
    
    
    %\section{}
    \newpage
    \pagenumbering{gobble}
        \printbibliography


    \newpage
    \pagenumbering{roman}
    \appendix
        \part{Appendices}
            \input{8 - Hilbert complexes/main.tex}
            \input{9 - weak conservation proofs/main.tex}
\end{document}

            \documentclass[12pt, a4paper]{report}

\input{template/main.tex}

\title{\BA{Title in Progress...}}
\author{Boris Andrews}
\affil{Mathematical Institute, University of Oxford}
\date{\today}


\begin{document}
    \pagenumbering{gobble}
    \maketitle
    
    
    \begin{abstract}
        Magnetic confinement reactors---in particular tokamaks---offer one of the most promising options for achieving practical nuclear fusion, with the potential to provide virtually limitless, clean energy. The theoretical and numerical modeling of tokamak plasmas is simultaneously an essential component of effective reactor design, and a great research barrier. Tokamak operational conditions exhibit comparatively low Knudsen numbers. Kinetic effects, including kinetic waves and instabilities, Landau damping, bump-on-tail instabilities and more, are therefore highly influential in tokamak plasma dynamics. Purely fluid models are inherently incapable of capturing these effects, whereas the high dimensionality in purely kinetic models render them practically intractable for most relevant purposes.

        We consider a $\delta\!f$ decomposition model, with a macroscopic fluid background and microscopic kinetic correction, both fully coupled to each other. A similar manner of discretization is proposed to that used in the recent \texttt{STRUPHY} code \cite{Holderied_Possanner_Wang_2021, Holderied_2022, Li_et_al_2023} with a finite-element model for the background and a pseudo-particle/PiC model for the correction.

        The fluid background satisfies the full, non-linear, resistive, compressible, Hall MHD equations. \cite{Laakmann_Hu_Farrell_2022} introduces finite-element(-in-space) implicit timesteppers for the incompressible analogue to this system with structure-preserving (SP) properties in the ideal case, alongside parameter-robust preconditioners. We show that these timesteppers can derive from a finite-element-in-time (FET) (and finite-element-in-space) interpretation. The benefits of this reformulation are discussed, including the derivation of timesteppers that are higher order in time, and the quantifiable dissipative SP properties in the non-ideal, resistive case.
        
        We discuss possible options for extending this FET approach to timesteppers for the compressible case.

        The kinetic corrections satisfy linearized Boltzmann equations. Using a Lénard--Bernstein collision operator, these take Fokker--Planck-like forms \cite{Fokker_1914, Planck_1917} wherein pseudo-particles in the numerical model obey the neoclassical transport equations, with particle-independent Brownian drift terms. This offers a rigorous methodology for incorporating collisions into the particle transport model, without coupling the equations of motions for each particle.
        
        Works by Chen, Chacón et al. \cite{Chen_Chacón_Barnes_2011, Chacón_Chen_Barnes_2013, Chen_Chacón_2014, Chen_Chacón_2015} have developed structure-preserving particle pushers for neoclassical transport in the Vlasov equations, derived from Crank--Nicolson integrators. We show these too can can derive from a FET interpretation, similarly offering potential extensions to higher-order-in-time particle pushers. The FET formulation is used also to consider how the stochastic drift terms can be incorporated into the pushers. Stochastic gyrokinetic expansions are also discussed.

        Different options for the numerical implementation of these schemes are considered.

        Due to the efficacy of FET in the development of SP timesteppers for both the fluid and kinetic component, we hope this approach will prove effective in the future for developing SP timesteppers for the full hybrid model. We hope this will give us the opportunity to incorporate previously inaccessible kinetic effects into the highly effective, modern, finite-element MHD models.
    \end{abstract}
    
    
    \newpage
    \tableofcontents
    
    
    \newpage
    \pagenumbering{arabic}
    %\linenumbers\renewcommand\thelinenumber{\color{black!50}\arabic{linenumber}}
            \input{0 - introduction/main.tex}
        \part{Research}
            \input{1 - low-noise PiC models/main.tex}
            \input{2 - kinetic component/main.tex}
            \input{3 - fluid component/main.tex}
            \input{4 - numerical implementation/main.tex}
        \part{Project Overview}
            \input{5 - research plan/main.tex}
            \input{6 - summary/main.tex}
    
    
    %\section{}
    \newpage
    \pagenumbering{gobble}
        \printbibliography


    \newpage
    \pagenumbering{roman}
    \appendix
        \part{Appendices}
            \input{8 - Hilbert complexes/main.tex}
            \input{9 - weak conservation proofs/main.tex}
\end{document}

\end{document}

            \documentclass[12pt, a4paper]{report}

\documentclass[12pt, a4paper]{report}

\input{template/main.tex}

\title{\BA{Title in Progress...}}
\author{Boris Andrews}
\affil{Mathematical Institute, University of Oxford}
\date{\today}


\begin{document}
    \pagenumbering{gobble}
    \maketitle
    
    
    \begin{abstract}
        Magnetic confinement reactors---in particular tokamaks---offer one of the most promising options for achieving practical nuclear fusion, with the potential to provide virtually limitless, clean energy. The theoretical and numerical modeling of tokamak plasmas is simultaneously an essential component of effective reactor design, and a great research barrier. Tokamak operational conditions exhibit comparatively low Knudsen numbers. Kinetic effects, including kinetic waves and instabilities, Landau damping, bump-on-tail instabilities and more, are therefore highly influential in tokamak plasma dynamics. Purely fluid models are inherently incapable of capturing these effects, whereas the high dimensionality in purely kinetic models render them practically intractable for most relevant purposes.

        We consider a $\delta\!f$ decomposition model, with a macroscopic fluid background and microscopic kinetic correction, both fully coupled to each other. A similar manner of discretization is proposed to that used in the recent \texttt{STRUPHY} code \cite{Holderied_Possanner_Wang_2021, Holderied_2022, Li_et_al_2023} with a finite-element model for the background and a pseudo-particle/PiC model for the correction.

        The fluid background satisfies the full, non-linear, resistive, compressible, Hall MHD equations. \cite{Laakmann_Hu_Farrell_2022} introduces finite-element(-in-space) implicit timesteppers for the incompressible analogue to this system with structure-preserving (SP) properties in the ideal case, alongside parameter-robust preconditioners. We show that these timesteppers can derive from a finite-element-in-time (FET) (and finite-element-in-space) interpretation. The benefits of this reformulation are discussed, including the derivation of timesteppers that are higher order in time, and the quantifiable dissipative SP properties in the non-ideal, resistive case.
        
        We discuss possible options for extending this FET approach to timesteppers for the compressible case.

        The kinetic corrections satisfy linearized Boltzmann equations. Using a Lénard--Bernstein collision operator, these take Fokker--Planck-like forms \cite{Fokker_1914, Planck_1917} wherein pseudo-particles in the numerical model obey the neoclassical transport equations, with particle-independent Brownian drift terms. This offers a rigorous methodology for incorporating collisions into the particle transport model, without coupling the equations of motions for each particle.
        
        Works by Chen, Chacón et al. \cite{Chen_Chacón_Barnes_2011, Chacón_Chen_Barnes_2013, Chen_Chacón_2014, Chen_Chacón_2015} have developed structure-preserving particle pushers for neoclassical transport in the Vlasov equations, derived from Crank--Nicolson integrators. We show these too can can derive from a FET interpretation, similarly offering potential extensions to higher-order-in-time particle pushers. The FET formulation is used also to consider how the stochastic drift terms can be incorporated into the pushers. Stochastic gyrokinetic expansions are also discussed.

        Different options for the numerical implementation of these schemes are considered.

        Due to the efficacy of FET in the development of SP timesteppers for both the fluid and kinetic component, we hope this approach will prove effective in the future for developing SP timesteppers for the full hybrid model. We hope this will give us the opportunity to incorporate previously inaccessible kinetic effects into the highly effective, modern, finite-element MHD models.
    \end{abstract}
    
    
    \newpage
    \tableofcontents
    
    
    \newpage
    \pagenumbering{arabic}
    %\linenumbers\renewcommand\thelinenumber{\color{black!50}\arabic{linenumber}}
            \input{0 - introduction/main.tex}
        \part{Research}
            \input{1 - low-noise PiC models/main.tex}
            \input{2 - kinetic component/main.tex}
            \input{3 - fluid component/main.tex}
            \input{4 - numerical implementation/main.tex}
        \part{Project Overview}
            \input{5 - research plan/main.tex}
            \input{6 - summary/main.tex}
    
    
    %\section{}
    \newpage
    \pagenumbering{gobble}
        \printbibliography


    \newpage
    \pagenumbering{roman}
    \appendix
        \part{Appendices}
            \input{8 - Hilbert complexes/main.tex}
            \input{9 - weak conservation proofs/main.tex}
\end{document}


\title{\BA{Title in Progress...}}
\author{Boris Andrews}
\affil{Mathematical Institute, University of Oxford}
\date{\today}


\begin{document}
    \pagenumbering{gobble}
    \maketitle
    
    
    \begin{abstract}
        Magnetic confinement reactors---in particular tokamaks---offer one of the most promising options for achieving practical nuclear fusion, with the potential to provide virtually limitless, clean energy. The theoretical and numerical modeling of tokamak plasmas is simultaneously an essential component of effective reactor design, and a great research barrier. Tokamak operational conditions exhibit comparatively low Knudsen numbers. Kinetic effects, including kinetic waves and instabilities, Landau damping, bump-on-tail instabilities and more, are therefore highly influential in tokamak plasma dynamics. Purely fluid models are inherently incapable of capturing these effects, whereas the high dimensionality in purely kinetic models render them practically intractable for most relevant purposes.

        We consider a $\delta\!f$ decomposition model, with a macroscopic fluid background and microscopic kinetic correction, both fully coupled to each other. A similar manner of discretization is proposed to that used in the recent \texttt{STRUPHY} code \cite{Holderied_Possanner_Wang_2021, Holderied_2022, Li_et_al_2023} with a finite-element model for the background and a pseudo-particle/PiC model for the correction.

        The fluid background satisfies the full, non-linear, resistive, compressible, Hall MHD equations. \cite{Laakmann_Hu_Farrell_2022} introduces finite-element(-in-space) implicit timesteppers for the incompressible analogue to this system with structure-preserving (SP) properties in the ideal case, alongside parameter-robust preconditioners. We show that these timesteppers can derive from a finite-element-in-time (FET) (and finite-element-in-space) interpretation. The benefits of this reformulation are discussed, including the derivation of timesteppers that are higher order in time, and the quantifiable dissipative SP properties in the non-ideal, resistive case.
        
        We discuss possible options for extending this FET approach to timesteppers for the compressible case.

        The kinetic corrections satisfy linearized Boltzmann equations. Using a Lénard--Bernstein collision operator, these take Fokker--Planck-like forms \cite{Fokker_1914, Planck_1917} wherein pseudo-particles in the numerical model obey the neoclassical transport equations, with particle-independent Brownian drift terms. This offers a rigorous methodology for incorporating collisions into the particle transport model, without coupling the equations of motions for each particle.
        
        Works by Chen, Chacón et al. \cite{Chen_Chacón_Barnes_2011, Chacón_Chen_Barnes_2013, Chen_Chacón_2014, Chen_Chacón_2015} have developed structure-preserving particle pushers for neoclassical transport in the Vlasov equations, derived from Crank--Nicolson integrators. We show these too can can derive from a FET interpretation, similarly offering potential extensions to higher-order-in-time particle pushers. The FET formulation is used also to consider how the stochastic drift terms can be incorporated into the pushers. Stochastic gyrokinetic expansions are also discussed.

        Different options for the numerical implementation of these schemes are considered.

        Due to the efficacy of FET in the development of SP timesteppers for both the fluid and kinetic component, we hope this approach will prove effective in the future for developing SP timesteppers for the full hybrid model. We hope this will give us the opportunity to incorporate previously inaccessible kinetic effects into the highly effective, modern, finite-element MHD models.
    \end{abstract}
    
    
    \newpage
    \tableofcontents
    
    
    \newpage
    \pagenumbering{arabic}
    %\linenumbers\renewcommand\thelinenumber{\color{black!50}\arabic{linenumber}}
            \documentclass[12pt, a4paper]{report}

\input{template/main.tex}

\title{\BA{Title in Progress...}}
\author{Boris Andrews}
\affil{Mathematical Institute, University of Oxford}
\date{\today}


\begin{document}
    \pagenumbering{gobble}
    \maketitle
    
    
    \begin{abstract}
        Magnetic confinement reactors---in particular tokamaks---offer one of the most promising options for achieving practical nuclear fusion, with the potential to provide virtually limitless, clean energy. The theoretical and numerical modeling of tokamak plasmas is simultaneously an essential component of effective reactor design, and a great research barrier. Tokamak operational conditions exhibit comparatively low Knudsen numbers. Kinetic effects, including kinetic waves and instabilities, Landau damping, bump-on-tail instabilities and more, are therefore highly influential in tokamak plasma dynamics. Purely fluid models are inherently incapable of capturing these effects, whereas the high dimensionality in purely kinetic models render them practically intractable for most relevant purposes.

        We consider a $\delta\!f$ decomposition model, with a macroscopic fluid background and microscopic kinetic correction, both fully coupled to each other. A similar manner of discretization is proposed to that used in the recent \texttt{STRUPHY} code \cite{Holderied_Possanner_Wang_2021, Holderied_2022, Li_et_al_2023} with a finite-element model for the background and a pseudo-particle/PiC model for the correction.

        The fluid background satisfies the full, non-linear, resistive, compressible, Hall MHD equations. \cite{Laakmann_Hu_Farrell_2022} introduces finite-element(-in-space) implicit timesteppers for the incompressible analogue to this system with structure-preserving (SP) properties in the ideal case, alongside parameter-robust preconditioners. We show that these timesteppers can derive from a finite-element-in-time (FET) (and finite-element-in-space) interpretation. The benefits of this reformulation are discussed, including the derivation of timesteppers that are higher order in time, and the quantifiable dissipative SP properties in the non-ideal, resistive case.
        
        We discuss possible options for extending this FET approach to timesteppers for the compressible case.

        The kinetic corrections satisfy linearized Boltzmann equations. Using a Lénard--Bernstein collision operator, these take Fokker--Planck-like forms \cite{Fokker_1914, Planck_1917} wherein pseudo-particles in the numerical model obey the neoclassical transport equations, with particle-independent Brownian drift terms. This offers a rigorous methodology for incorporating collisions into the particle transport model, without coupling the equations of motions for each particle.
        
        Works by Chen, Chacón et al. \cite{Chen_Chacón_Barnes_2011, Chacón_Chen_Barnes_2013, Chen_Chacón_2014, Chen_Chacón_2015} have developed structure-preserving particle pushers for neoclassical transport in the Vlasov equations, derived from Crank--Nicolson integrators. We show these too can can derive from a FET interpretation, similarly offering potential extensions to higher-order-in-time particle pushers. The FET formulation is used also to consider how the stochastic drift terms can be incorporated into the pushers. Stochastic gyrokinetic expansions are also discussed.

        Different options for the numerical implementation of these schemes are considered.

        Due to the efficacy of FET in the development of SP timesteppers for both the fluid and kinetic component, we hope this approach will prove effective in the future for developing SP timesteppers for the full hybrid model. We hope this will give us the opportunity to incorporate previously inaccessible kinetic effects into the highly effective, modern, finite-element MHD models.
    \end{abstract}
    
    
    \newpage
    \tableofcontents
    
    
    \newpage
    \pagenumbering{arabic}
    %\linenumbers\renewcommand\thelinenumber{\color{black!50}\arabic{linenumber}}
            \input{0 - introduction/main.tex}
        \part{Research}
            \input{1 - low-noise PiC models/main.tex}
            \input{2 - kinetic component/main.tex}
            \input{3 - fluid component/main.tex}
            \input{4 - numerical implementation/main.tex}
        \part{Project Overview}
            \input{5 - research plan/main.tex}
            \input{6 - summary/main.tex}
    
    
    %\section{}
    \newpage
    \pagenumbering{gobble}
        \printbibliography


    \newpage
    \pagenumbering{roman}
    \appendix
        \part{Appendices}
            \input{8 - Hilbert complexes/main.tex}
            \input{9 - weak conservation proofs/main.tex}
\end{document}

        \part{Research}
            \documentclass[12pt, a4paper]{report}

\input{template/main.tex}

\title{\BA{Title in Progress...}}
\author{Boris Andrews}
\affil{Mathematical Institute, University of Oxford}
\date{\today}


\begin{document}
    \pagenumbering{gobble}
    \maketitle
    
    
    \begin{abstract}
        Magnetic confinement reactors---in particular tokamaks---offer one of the most promising options for achieving practical nuclear fusion, with the potential to provide virtually limitless, clean energy. The theoretical and numerical modeling of tokamak plasmas is simultaneously an essential component of effective reactor design, and a great research barrier. Tokamak operational conditions exhibit comparatively low Knudsen numbers. Kinetic effects, including kinetic waves and instabilities, Landau damping, bump-on-tail instabilities and more, are therefore highly influential in tokamak plasma dynamics. Purely fluid models are inherently incapable of capturing these effects, whereas the high dimensionality in purely kinetic models render them practically intractable for most relevant purposes.

        We consider a $\delta\!f$ decomposition model, with a macroscopic fluid background and microscopic kinetic correction, both fully coupled to each other. A similar manner of discretization is proposed to that used in the recent \texttt{STRUPHY} code \cite{Holderied_Possanner_Wang_2021, Holderied_2022, Li_et_al_2023} with a finite-element model for the background and a pseudo-particle/PiC model for the correction.

        The fluid background satisfies the full, non-linear, resistive, compressible, Hall MHD equations. \cite{Laakmann_Hu_Farrell_2022} introduces finite-element(-in-space) implicit timesteppers for the incompressible analogue to this system with structure-preserving (SP) properties in the ideal case, alongside parameter-robust preconditioners. We show that these timesteppers can derive from a finite-element-in-time (FET) (and finite-element-in-space) interpretation. The benefits of this reformulation are discussed, including the derivation of timesteppers that are higher order in time, and the quantifiable dissipative SP properties in the non-ideal, resistive case.
        
        We discuss possible options for extending this FET approach to timesteppers for the compressible case.

        The kinetic corrections satisfy linearized Boltzmann equations. Using a Lénard--Bernstein collision operator, these take Fokker--Planck-like forms \cite{Fokker_1914, Planck_1917} wherein pseudo-particles in the numerical model obey the neoclassical transport equations, with particle-independent Brownian drift terms. This offers a rigorous methodology for incorporating collisions into the particle transport model, without coupling the equations of motions for each particle.
        
        Works by Chen, Chacón et al. \cite{Chen_Chacón_Barnes_2011, Chacón_Chen_Barnes_2013, Chen_Chacón_2014, Chen_Chacón_2015} have developed structure-preserving particle pushers for neoclassical transport in the Vlasov equations, derived from Crank--Nicolson integrators. We show these too can can derive from a FET interpretation, similarly offering potential extensions to higher-order-in-time particle pushers. The FET formulation is used also to consider how the stochastic drift terms can be incorporated into the pushers. Stochastic gyrokinetic expansions are also discussed.

        Different options for the numerical implementation of these schemes are considered.

        Due to the efficacy of FET in the development of SP timesteppers for both the fluid and kinetic component, we hope this approach will prove effective in the future for developing SP timesteppers for the full hybrid model. We hope this will give us the opportunity to incorporate previously inaccessible kinetic effects into the highly effective, modern, finite-element MHD models.
    \end{abstract}
    
    
    \newpage
    \tableofcontents
    
    
    \newpage
    \pagenumbering{arabic}
    %\linenumbers\renewcommand\thelinenumber{\color{black!50}\arabic{linenumber}}
            \input{0 - introduction/main.tex}
        \part{Research}
            \input{1 - low-noise PiC models/main.tex}
            \input{2 - kinetic component/main.tex}
            \input{3 - fluid component/main.tex}
            \input{4 - numerical implementation/main.tex}
        \part{Project Overview}
            \input{5 - research plan/main.tex}
            \input{6 - summary/main.tex}
    
    
    %\section{}
    \newpage
    \pagenumbering{gobble}
        \printbibliography


    \newpage
    \pagenumbering{roman}
    \appendix
        \part{Appendices}
            \input{8 - Hilbert complexes/main.tex}
            \input{9 - weak conservation proofs/main.tex}
\end{document}

            \documentclass[12pt, a4paper]{report}

\input{template/main.tex}

\title{\BA{Title in Progress...}}
\author{Boris Andrews}
\affil{Mathematical Institute, University of Oxford}
\date{\today}


\begin{document}
    \pagenumbering{gobble}
    \maketitle
    
    
    \begin{abstract}
        Magnetic confinement reactors---in particular tokamaks---offer one of the most promising options for achieving practical nuclear fusion, with the potential to provide virtually limitless, clean energy. The theoretical and numerical modeling of tokamak plasmas is simultaneously an essential component of effective reactor design, and a great research barrier. Tokamak operational conditions exhibit comparatively low Knudsen numbers. Kinetic effects, including kinetic waves and instabilities, Landau damping, bump-on-tail instabilities and more, are therefore highly influential in tokamak plasma dynamics. Purely fluid models are inherently incapable of capturing these effects, whereas the high dimensionality in purely kinetic models render them practically intractable for most relevant purposes.

        We consider a $\delta\!f$ decomposition model, with a macroscopic fluid background and microscopic kinetic correction, both fully coupled to each other. A similar manner of discretization is proposed to that used in the recent \texttt{STRUPHY} code \cite{Holderied_Possanner_Wang_2021, Holderied_2022, Li_et_al_2023} with a finite-element model for the background and a pseudo-particle/PiC model for the correction.

        The fluid background satisfies the full, non-linear, resistive, compressible, Hall MHD equations. \cite{Laakmann_Hu_Farrell_2022} introduces finite-element(-in-space) implicit timesteppers for the incompressible analogue to this system with structure-preserving (SP) properties in the ideal case, alongside parameter-robust preconditioners. We show that these timesteppers can derive from a finite-element-in-time (FET) (and finite-element-in-space) interpretation. The benefits of this reformulation are discussed, including the derivation of timesteppers that are higher order in time, and the quantifiable dissipative SP properties in the non-ideal, resistive case.
        
        We discuss possible options for extending this FET approach to timesteppers for the compressible case.

        The kinetic corrections satisfy linearized Boltzmann equations. Using a Lénard--Bernstein collision operator, these take Fokker--Planck-like forms \cite{Fokker_1914, Planck_1917} wherein pseudo-particles in the numerical model obey the neoclassical transport equations, with particle-independent Brownian drift terms. This offers a rigorous methodology for incorporating collisions into the particle transport model, without coupling the equations of motions for each particle.
        
        Works by Chen, Chacón et al. \cite{Chen_Chacón_Barnes_2011, Chacón_Chen_Barnes_2013, Chen_Chacón_2014, Chen_Chacón_2015} have developed structure-preserving particle pushers for neoclassical transport in the Vlasov equations, derived from Crank--Nicolson integrators. We show these too can can derive from a FET interpretation, similarly offering potential extensions to higher-order-in-time particle pushers. The FET formulation is used also to consider how the stochastic drift terms can be incorporated into the pushers. Stochastic gyrokinetic expansions are also discussed.

        Different options for the numerical implementation of these schemes are considered.

        Due to the efficacy of FET in the development of SP timesteppers for both the fluid and kinetic component, we hope this approach will prove effective in the future for developing SP timesteppers for the full hybrid model. We hope this will give us the opportunity to incorporate previously inaccessible kinetic effects into the highly effective, modern, finite-element MHD models.
    \end{abstract}
    
    
    \newpage
    \tableofcontents
    
    
    \newpage
    \pagenumbering{arabic}
    %\linenumbers\renewcommand\thelinenumber{\color{black!50}\arabic{linenumber}}
            \input{0 - introduction/main.tex}
        \part{Research}
            \input{1 - low-noise PiC models/main.tex}
            \input{2 - kinetic component/main.tex}
            \input{3 - fluid component/main.tex}
            \input{4 - numerical implementation/main.tex}
        \part{Project Overview}
            \input{5 - research plan/main.tex}
            \input{6 - summary/main.tex}
    
    
    %\section{}
    \newpage
    \pagenumbering{gobble}
        \printbibliography


    \newpage
    \pagenumbering{roman}
    \appendix
        \part{Appendices}
            \input{8 - Hilbert complexes/main.tex}
            \input{9 - weak conservation proofs/main.tex}
\end{document}

            \documentclass[12pt, a4paper]{report}

\input{template/main.tex}

\title{\BA{Title in Progress...}}
\author{Boris Andrews}
\affil{Mathematical Institute, University of Oxford}
\date{\today}


\begin{document}
    \pagenumbering{gobble}
    \maketitle
    
    
    \begin{abstract}
        Magnetic confinement reactors---in particular tokamaks---offer one of the most promising options for achieving practical nuclear fusion, with the potential to provide virtually limitless, clean energy. The theoretical and numerical modeling of tokamak plasmas is simultaneously an essential component of effective reactor design, and a great research barrier. Tokamak operational conditions exhibit comparatively low Knudsen numbers. Kinetic effects, including kinetic waves and instabilities, Landau damping, bump-on-tail instabilities and more, are therefore highly influential in tokamak plasma dynamics. Purely fluid models are inherently incapable of capturing these effects, whereas the high dimensionality in purely kinetic models render them practically intractable for most relevant purposes.

        We consider a $\delta\!f$ decomposition model, with a macroscopic fluid background and microscopic kinetic correction, both fully coupled to each other. A similar manner of discretization is proposed to that used in the recent \texttt{STRUPHY} code \cite{Holderied_Possanner_Wang_2021, Holderied_2022, Li_et_al_2023} with a finite-element model for the background and a pseudo-particle/PiC model for the correction.

        The fluid background satisfies the full, non-linear, resistive, compressible, Hall MHD equations. \cite{Laakmann_Hu_Farrell_2022} introduces finite-element(-in-space) implicit timesteppers for the incompressible analogue to this system with structure-preserving (SP) properties in the ideal case, alongside parameter-robust preconditioners. We show that these timesteppers can derive from a finite-element-in-time (FET) (and finite-element-in-space) interpretation. The benefits of this reformulation are discussed, including the derivation of timesteppers that are higher order in time, and the quantifiable dissipative SP properties in the non-ideal, resistive case.
        
        We discuss possible options for extending this FET approach to timesteppers for the compressible case.

        The kinetic corrections satisfy linearized Boltzmann equations. Using a Lénard--Bernstein collision operator, these take Fokker--Planck-like forms \cite{Fokker_1914, Planck_1917} wherein pseudo-particles in the numerical model obey the neoclassical transport equations, with particle-independent Brownian drift terms. This offers a rigorous methodology for incorporating collisions into the particle transport model, without coupling the equations of motions for each particle.
        
        Works by Chen, Chacón et al. \cite{Chen_Chacón_Barnes_2011, Chacón_Chen_Barnes_2013, Chen_Chacón_2014, Chen_Chacón_2015} have developed structure-preserving particle pushers for neoclassical transport in the Vlasov equations, derived from Crank--Nicolson integrators. We show these too can can derive from a FET interpretation, similarly offering potential extensions to higher-order-in-time particle pushers. The FET formulation is used also to consider how the stochastic drift terms can be incorporated into the pushers. Stochastic gyrokinetic expansions are also discussed.

        Different options for the numerical implementation of these schemes are considered.

        Due to the efficacy of FET in the development of SP timesteppers for both the fluid and kinetic component, we hope this approach will prove effective in the future for developing SP timesteppers for the full hybrid model. We hope this will give us the opportunity to incorporate previously inaccessible kinetic effects into the highly effective, modern, finite-element MHD models.
    \end{abstract}
    
    
    \newpage
    \tableofcontents
    
    
    \newpage
    \pagenumbering{arabic}
    %\linenumbers\renewcommand\thelinenumber{\color{black!50}\arabic{linenumber}}
            \input{0 - introduction/main.tex}
        \part{Research}
            \input{1 - low-noise PiC models/main.tex}
            \input{2 - kinetic component/main.tex}
            \input{3 - fluid component/main.tex}
            \input{4 - numerical implementation/main.tex}
        \part{Project Overview}
            \input{5 - research plan/main.tex}
            \input{6 - summary/main.tex}
    
    
    %\section{}
    \newpage
    \pagenumbering{gobble}
        \printbibliography


    \newpage
    \pagenumbering{roman}
    \appendix
        \part{Appendices}
            \input{8 - Hilbert complexes/main.tex}
            \input{9 - weak conservation proofs/main.tex}
\end{document}

            \documentclass[12pt, a4paper]{report}

\input{template/main.tex}

\title{\BA{Title in Progress...}}
\author{Boris Andrews}
\affil{Mathematical Institute, University of Oxford}
\date{\today}


\begin{document}
    \pagenumbering{gobble}
    \maketitle
    
    
    \begin{abstract}
        Magnetic confinement reactors---in particular tokamaks---offer one of the most promising options for achieving practical nuclear fusion, with the potential to provide virtually limitless, clean energy. The theoretical and numerical modeling of tokamak plasmas is simultaneously an essential component of effective reactor design, and a great research barrier. Tokamak operational conditions exhibit comparatively low Knudsen numbers. Kinetic effects, including kinetic waves and instabilities, Landau damping, bump-on-tail instabilities and more, are therefore highly influential in tokamak plasma dynamics. Purely fluid models are inherently incapable of capturing these effects, whereas the high dimensionality in purely kinetic models render them practically intractable for most relevant purposes.

        We consider a $\delta\!f$ decomposition model, with a macroscopic fluid background and microscopic kinetic correction, both fully coupled to each other. A similar manner of discretization is proposed to that used in the recent \texttt{STRUPHY} code \cite{Holderied_Possanner_Wang_2021, Holderied_2022, Li_et_al_2023} with a finite-element model for the background and a pseudo-particle/PiC model for the correction.

        The fluid background satisfies the full, non-linear, resistive, compressible, Hall MHD equations. \cite{Laakmann_Hu_Farrell_2022} introduces finite-element(-in-space) implicit timesteppers for the incompressible analogue to this system with structure-preserving (SP) properties in the ideal case, alongside parameter-robust preconditioners. We show that these timesteppers can derive from a finite-element-in-time (FET) (and finite-element-in-space) interpretation. The benefits of this reformulation are discussed, including the derivation of timesteppers that are higher order in time, and the quantifiable dissipative SP properties in the non-ideal, resistive case.
        
        We discuss possible options for extending this FET approach to timesteppers for the compressible case.

        The kinetic corrections satisfy linearized Boltzmann equations. Using a Lénard--Bernstein collision operator, these take Fokker--Planck-like forms \cite{Fokker_1914, Planck_1917} wherein pseudo-particles in the numerical model obey the neoclassical transport equations, with particle-independent Brownian drift terms. This offers a rigorous methodology for incorporating collisions into the particle transport model, without coupling the equations of motions for each particle.
        
        Works by Chen, Chacón et al. \cite{Chen_Chacón_Barnes_2011, Chacón_Chen_Barnes_2013, Chen_Chacón_2014, Chen_Chacón_2015} have developed structure-preserving particle pushers for neoclassical transport in the Vlasov equations, derived from Crank--Nicolson integrators. We show these too can can derive from a FET interpretation, similarly offering potential extensions to higher-order-in-time particle pushers. The FET formulation is used also to consider how the stochastic drift terms can be incorporated into the pushers. Stochastic gyrokinetic expansions are also discussed.

        Different options for the numerical implementation of these schemes are considered.

        Due to the efficacy of FET in the development of SP timesteppers for both the fluid and kinetic component, we hope this approach will prove effective in the future for developing SP timesteppers for the full hybrid model. We hope this will give us the opportunity to incorporate previously inaccessible kinetic effects into the highly effective, modern, finite-element MHD models.
    \end{abstract}
    
    
    \newpage
    \tableofcontents
    
    
    \newpage
    \pagenumbering{arabic}
    %\linenumbers\renewcommand\thelinenumber{\color{black!50}\arabic{linenumber}}
            \input{0 - introduction/main.tex}
        \part{Research}
            \input{1 - low-noise PiC models/main.tex}
            \input{2 - kinetic component/main.tex}
            \input{3 - fluid component/main.tex}
            \input{4 - numerical implementation/main.tex}
        \part{Project Overview}
            \input{5 - research plan/main.tex}
            \input{6 - summary/main.tex}
    
    
    %\section{}
    \newpage
    \pagenumbering{gobble}
        \printbibliography


    \newpage
    \pagenumbering{roman}
    \appendix
        \part{Appendices}
            \input{8 - Hilbert complexes/main.tex}
            \input{9 - weak conservation proofs/main.tex}
\end{document}

        \part{Project Overview}
            \documentclass[12pt, a4paper]{report}

\input{template/main.tex}

\title{\BA{Title in Progress...}}
\author{Boris Andrews}
\affil{Mathematical Institute, University of Oxford}
\date{\today}


\begin{document}
    \pagenumbering{gobble}
    \maketitle
    
    
    \begin{abstract}
        Magnetic confinement reactors---in particular tokamaks---offer one of the most promising options for achieving practical nuclear fusion, with the potential to provide virtually limitless, clean energy. The theoretical and numerical modeling of tokamak plasmas is simultaneously an essential component of effective reactor design, and a great research barrier. Tokamak operational conditions exhibit comparatively low Knudsen numbers. Kinetic effects, including kinetic waves and instabilities, Landau damping, bump-on-tail instabilities and more, are therefore highly influential in tokamak plasma dynamics. Purely fluid models are inherently incapable of capturing these effects, whereas the high dimensionality in purely kinetic models render them practically intractable for most relevant purposes.

        We consider a $\delta\!f$ decomposition model, with a macroscopic fluid background and microscopic kinetic correction, both fully coupled to each other. A similar manner of discretization is proposed to that used in the recent \texttt{STRUPHY} code \cite{Holderied_Possanner_Wang_2021, Holderied_2022, Li_et_al_2023} with a finite-element model for the background and a pseudo-particle/PiC model for the correction.

        The fluid background satisfies the full, non-linear, resistive, compressible, Hall MHD equations. \cite{Laakmann_Hu_Farrell_2022} introduces finite-element(-in-space) implicit timesteppers for the incompressible analogue to this system with structure-preserving (SP) properties in the ideal case, alongside parameter-robust preconditioners. We show that these timesteppers can derive from a finite-element-in-time (FET) (and finite-element-in-space) interpretation. The benefits of this reformulation are discussed, including the derivation of timesteppers that are higher order in time, and the quantifiable dissipative SP properties in the non-ideal, resistive case.
        
        We discuss possible options for extending this FET approach to timesteppers for the compressible case.

        The kinetic corrections satisfy linearized Boltzmann equations. Using a Lénard--Bernstein collision operator, these take Fokker--Planck-like forms \cite{Fokker_1914, Planck_1917} wherein pseudo-particles in the numerical model obey the neoclassical transport equations, with particle-independent Brownian drift terms. This offers a rigorous methodology for incorporating collisions into the particle transport model, without coupling the equations of motions for each particle.
        
        Works by Chen, Chacón et al. \cite{Chen_Chacón_Barnes_2011, Chacón_Chen_Barnes_2013, Chen_Chacón_2014, Chen_Chacón_2015} have developed structure-preserving particle pushers for neoclassical transport in the Vlasov equations, derived from Crank--Nicolson integrators. We show these too can can derive from a FET interpretation, similarly offering potential extensions to higher-order-in-time particle pushers. The FET formulation is used also to consider how the stochastic drift terms can be incorporated into the pushers. Stochastic gyrokinetic expansions are also discussed.

        Different options for the numerical implementation of these schemes are considered.

        Due to the efficacy of FET in the development of SP timesteppers for both the fluid and kinetic component, we hope this approach will prove effective in the future for developing SP timesteppers for the full hybrid model. We hope this will give us the opportunity to incorporate previously inaccessible kinetic effects into the highly effective, modern, finite-element MHD models.
    \end{abstract}
    
    
    \newpage
    \tableofcontents
    
    
    \newpage
    \pagenumbering{arabic}
    %\linenumbers\renewcommand\thelinenumber{\color{black!50}\arabic{linenumber}}
            \input{0 - introduction/main.tex}
        \part{Research}
            \input{1 - low-noise PiC models/main.tex}
            \input{2 - kinetic component/main.tex}
            \input{3 - fluid component/main.tex}
            \input{4 - numerical implementation/main.tex}
        \part{Project Overview}
            \input{5 - research plan/main.tex}
            \input{6 - summary/main.tex}
    
    
    %\section{}
    \newpage
    \pagenumbering{gobble}
        \printbibliography


    \newpage
    \pagenumbering{roman}
    \appendix
        \part{Appendices}
            \input{8 - Hilbert complexes/main.tex}
            \input{9 - weak conservation proofs/main.tex}
\end{document}

            \documentclass[12pt, a4paper]{report}

\input{template/main.tex}

\title{\BA{Title in Progress...}}
\author{Boris Andrews}
\affil{Mathematical Institute, University of Oxford}
\date{\today}


\begin{document}
    \pagenumbering{gobble}
    \maketitle
    
    
    \begin{abstract}
        Magnetic confinement reactors---in particular tokamaks---offer one of the most promising options for achieving practical nuclear fusion, with the potential to provide virtually limitless, clean energy. The theoretical and numerical modeling of tokamak plasmas is simultaneously an essential component of effective reactor design, and a great research barrier. Tokamak operational conditions exhibit comparatively low Knudsen numbers. Kinetic effects, including kinetic waves and instabilities, Landau damping, bump-on-tail instabilities and more, are therefore highly influential in tokamak plasma dynamics. Purely fluid models are inherently incapable of capturing these effects, whereas the high dimensionality in purely kinetic models render them practically intractable for most relevant purposes.

        We consider a $\delta\!f$ decomposition model, with a macroscopic fluid background and microscopic kinetic correction, both fully coupled to each other. A similar manner of discretization is proposed to that used in the recent \texttt{STRUPHY} code \cite{Holderied_Possanner_Wang_2021, Holderied_2022, Li_et_al_2023} with a finite-element model for the background and a pseudo-particle/PiC model for the correction.

        The fluid background satisfies the full, non-linear, resistive, compressible, Hall MHD equations. \cite{Laakmann_Hu_Farrell_2022} introduces finite-element(-in-space) implicit timesteppers for the incompressible analogue to this system with structure-preserving (SP) properties in the ideal case, alongside parameter-robust preconditioners. We show that these timesteppers can derive from a finite-element-in-time (FET) (and finite-element-in-space) interpretation. The benefits of this reformulation are discussed, including the derivation of timesteppers that are higher order in time, and the quantifiable dissipative SP properties in the non-ideal, resistive case.
        
        We discuss possible options for extending this FET approach to timesteppers for the compressible case.

        The kinetic corrections satisfy linearized Boltzmann equations. Using a Lénard--Bernstein collision operator, these take Fokker--Planck-like forms \cite{Fokker_1914, Planck_1917} wherein pseudo-particles in the numerical model obey the neoclassical transport equations, with particle-independent Brownian drift terms. This offers a rigorous methodology for incorporating collisions into the particle transport model, without coupling the equations of motions for each particle.
        
        Works by Chen, Chacón et al. \cite{Chen_Chacón_Barnes_2011, Chacón_Chen_Barnes_2013, Chen_Chacón_2014, Chen_Chacón_2015} have developed structure-preserving particle pushers for neoclassical transport in the Vlasov equations, derived from Crank--Nicolson integrators. We show these too can can derive from a FET interpretation, similarly offering potential extensions to higher-order-in-time particle pushers. The FET formulation is used also to consider how the stochastic drift terms can be incorporated into the pushers. Stochastic gyrokinetic expansions are also discussed.

        Different options for the numerical implementation of these schemes are considered.

        Due to the efficacy of FET in the development of SP timesteppers for both the fluid and kinetic component, we hope this approach will prove effective in the future for developing SP timesteppers for the full hybrid model. We hope this will give us the opportunity to incorporate previously inaccessible kinetic effects into the highly effective, modern, finite-element MHD models.
    \end{abstract}
    
    
    \newpage
    \tableofcontents
    
    
    \newpage
    \pagenumbering{arabic}
    %\linenumbers\renewcommand\thelinenumber{\color{black!50}\arabic{linenumber}}
            \input{0 - introduction/main.tex}
        \part{Research}
            \input{1 - low-noise PiC models/main.tex}
            \input{2 - kinetic component/main.tex}
            \input{3 - fluid component/main.tex}
            \input{4 - numerical implementation/main.tex}
        \part{Project Overview}
            \input{5 - research plan/main.tex}
            \input{6 - summary/main.tex}
    
    
    %\section{}
    \newpage
    \pagenumbering{gobble}
        \printbibliography


    \newpage
    \pagenumbering{roman}
    \appendix
        \part{Appendices}
            \input{8 - Hilbert complexes/main.tex}
            \input{9 - weak conservation proofs/main.tex}
\end{document}

    
    
    %\section{}
    \newpage
    \pagenumbering{gobble}
        \printbibliography


    \newpage
    \pagenumbering{roman}
    \appendix
        \part{Appendices}
            \documentclass[12pt, a4paper]{report}

\input{template/main.tex}

\title{\BA{Title in Progress...}}
\author{Boris Andrews}
\affil{Mathematical Institute, University of Oxford}
\date{\today}


\begin{document}
    \pagenumbering{gobble}
    \maketitle
    
    
    \begin{abstract}
        Magnetic confinement reactors---in particular tokamaks---offer one of the most promising options for achieving practical nuclear fusion, with the potential to provide virtually limitless, clean energy. The theoretical and numerical modeling of tokamak plasmas is simultaneously an essential component of effective reactor design, and a great research barrier. Tokamak operational conditions exhibit comparatively low Knudsen numbers. Kinetic effects, including kinetic waves and instabilities, Landau damping, bump-on-tail instabilities and more, are therefore highly influential in tokamak plasma dynamics. Purely fluid models are inherently incapable of capturing these effects, whereas the high dimensionality in purely kinetic models render them practically intractable for most relevant purposes.

        We consider a $\delta\!f$ decomposition model, with a macroscopic fluid background and microscopic kinetic correction, both fully coupled to each other. A similar manner of discretization is proposed to that used in the recent \texttt{STRUPHY} code \cite{Holderied_Possanner_Wang_2021, Holderied_2022, Li_et_al_2023} with a finite-element model for the background and a pseudo-particle/PiC model for the correction.

        The fluid background satisfies the full, non-linear, resistive, compressible, Hall MHD equations. \cite{Laakmann_Hu_Farrell_2022} introduces finite-element(-in-space) implicit timesteppers for the incompressible analogue to this system with structure-preserving (SP) properties in the ideal case, alongside parameter-robust preconditioners. We show that these timesteppers can derive from a finite-element-in-time (FET) (and finite-element-in-space) interpretation. The benefits of this reformulation are discussed, including the derivation of timesteppers that are higher order in time, and the quantifiable dissipative SP properties in the non-ideal, resistive case.
        
        We discuss possible options for extending this FET approach to timesteppers for the compressible case.

        The kinetic corrections satisfy linearized Boltzmann equations. Using a Lénard--Bernstein collision operator, these take Fokker--Planck-like forms \cite{Fokker_1914, Planck_1917} wherein pseudo-particles in the numerical model obey the neoclassical transport equations, with particle-independent Brownian drift terms. This offers a rigorous methodology for incorporating collisions into the particle transport model, without coupling the equations of motions for each particle.
        
        Works by Chen, Chacón et al. \cite{Chen_Chacón_Barnes_2011, Chacón_Chen_Barnes_2013, Chen_Chacón_2014, Chen_Chacón_2015} have developed structure-preserving particle pushers for neoclassical transport in the Vlasov equations, derived from Crank--Nicolson integrators. We show these too can can derive from a FET interpretation, similarly offering potential extensions to higher-order-in-time particle pushers. The FET formulation is used also to consider how the stochastic drift terms can be incorporated into the pushers. Stochastic gyrokinetic expansions are also discussed.

        Different options for the numerical implementation of these schemes are considered.

        Due to the efficacy of FET in the development of SP timesteppers for both the fluid and kinetic component, we hope this approach will prove effective in the future for developing SP timesteppers for the full hybrid model. We hope this will give us the opportunity to incorporate previously inaccessible kinetic effects into the highly effective, modern, finite-element MHD models.
    \end{abstract}
    
    
    \newpage
    \tableofcontents
    
    
    \newpage
    \pagenumbering{arabic}
    %\linenumbers\renewcommand\thelinenumber{\color{black!50}\arabic{linenumber}}
            \input{0 - introduction/main.tex}
        \part{Research}
            \input{1 - low-noise PiC models/main.tex}
            \input{2 - kinetic component/main.tex}
            \input{3 - fluid component/main.tex}
            \input{4 - numerical implementation/main.tex}
        \part{Project Overview}
            \input{5 - research plan/main.tex}
            \input{6 - summary/main.tex}
    
    
    %\section{}
    \newpage
    \pagenumbering{gobble}
        \printbibliography


    \newpage
    \pagenumbering{roman}
    \appendix
        \part{Appendices}
            \input{8 - Hilbert complexes/main.tex}
            \input{9 - weak conservation proofs/main.tex}
\end{document}

            \documentclass[12pt, a4paper]{report}

\input{template/main.tex}

\title{\BA{Title in Progress...}}
\author{Boris Andrews}
\affil{Mathematical Institute, University of Oxford}
\date{\today}


\begin{document}
    \pagenumbering{gobble}
    \maketitle
    
    
    \begin{abstract}
        Magnetic confinement reactors---in particular tokamaks---offer one of the most promising options for achieving practical nuclear fusion, with the potential to provide virtually limitless, clean energy. The theoretical and numerical modeling of tokamak plasmas is simultaneously an essential component of effective reactor design, and a great research barrier. Tokamak operational conditions exhibit comparatively low Knudsen numbers. Kinetic effects, including kinetic waves and instabilities, Landau damping, bump-on-tail instabilities and more, are therefore highly influential in tokamak plasma dynamics. Purely fluid models are inherently incapable of capturing these effects, whereas the high dimensionality in purely kinetic models render them practically intractable for most relevant purposes.

        We consider a $\delta\!f$ decomposition model, with a macroscopic fluid background and microscopic kinetic correction, both fully coupled to each other. A similar manner of discretization is proposed to that used in the recent \texttt{STRUPHY} code \cite{Holderied_Possanner_Wang_2021, Holderied_2022, Li_et_al_2023} with a finite-element model for the background and a pseudo-particle/PiC model for the correction.

        The fluid background satisfies the full, non-linear, resistive, compressible, Hall MHD equations. \cite{Laakmann_Hu_Farrell_2022} introduces finite-element(-in-space) implicit timesteppers for the incompressible analogue to this system with structure-preserving (SP) properties in the ideal case, alongside parameter-robust preconditioners. We show that these timesteppers can derive from a finite-element-in-time (FET) (and finite-element-in-space) interpretation. The benefits of this reformulation are discussed, including the derivation of timesteppers that are higher order in time, and the quantifiable dissipative SP properties in the non-ideal, resistive case.
        
        We discuss possible options for extending this FET approach to timesteppers for the compressible case.

        The kinetic corrections satisfy linearized Boltzmann equations. Using a Lénard--Bernstein collision operator, these take Fokker--Planck-like forms \cite{Fokker_1914, Planck_1917} wherein pseudo-particles in the numerical model obey the neoclassical transport equations, with particle-independent Brownian drift terms. This offers a rigorous methodology for incorporating collisions into the particle transport model, without coupling the equations of motions for each particle.
        
        Works by Chen, Chacón et al. \cite{Chen_Chacón_Barnes_2011, Chacón_Chen_Barnes_2013, Chen_Chacón_2014, Chen_Chacón_2015} have developed structure-preserving particle pushers for neoclassical transport in the Vlasov equations, derived from Crank--Nicolson integrators. We show these too can can derive from a FET interpretation, similarly offering potential extensions to higher-order-in-time particle pushers. The FET formulation is used also to consider how the stochastic drift terms can be incorporated into the pushers. Stochastic gyrokinetic expansions are also discussed.

        Different options for the numerical implementation of these schemes are considered.

        Due to the efficacy of FET in the development of SP timesteppers for both the fluid and kinetic component, we hope this approach will prove effective in the future for developing SP timesteppers for the full hybrid model. We hope this will give us the opportunity to incorporate previously inaccessible kinetic effects into the highly effective, modern, finite-element MHD models.
    \end{abstract}
    
    
    \newpage
    \tableofcontents
    
    
    \newpage
    \pagenumbering{arabic}
    %\linenumbers\renewcommand\thelinenumber{\color{black!50}\arabic{linenumber}}
            \input{0 - introduction/main.tex}
        \part{Research}
            \input{1 - low-noise PiC models/main.tex}
            \input{2 - kinetic component/main.tex}
            \input{3 - fluid component/main.tex}
            \input{4 - numerical implementation/main.tex}
        \part{Project Overview}
            \input{5 - research plan/main.tex}
            \input{6 - summary/main.tex}
    
    
    %\section{}
    \newpage
    \pagenumbering{gobble}
        \printbibliography


    \newpage
    \pagenumbering{roman}
    \appendix
        \part{Appendices}
            \input{8 - Hilbert complexes/main.tex}
            \input{9 - weak conservation proofs/main.tex}
\end{document}

\end{document}

\end{document}

    \documentclass[12pt, a4paper]{report}

\documentclass[12pt, a4paper]{report}

\documentclass[12pt, a4paper]{report}

\input{template/main.tex}

\title{\BA{Title in Progress...}}
\author{Boris Andrews}
\affil{Mathematical Institute, University of Oxford}
\date{\today}


\begin{document}
    \pagenumbering{gobble}
    \maketitle
    
    
    \begin{abstract}
        Magnetic confinement reactors---in particular tokamaks---offer one of the most promising options for achieving practical nuclear fusion, with the potential to provide virtually limitless, clean energy. The theoretical and numerical modeling of tokamak plasmas is simultaneously an essential component of effective reactor design, and a great research barrier. Tokamak operational conditions exhibit comparatively low Knudsen numbers. Kinetic effects, including kinetic waves and instabilities, Landau damping, bump-on-tail instabilities and more, are therefore highly influential in tokamak plasma dynamics. Purely fluid models are inherently incapable of capturing these effects, whereas the high dimensionality in purely kinetic models render them practically intractable for most relevant purposes.

        We consider a $\delta\!f$ decomposition model, with a macroscopic fluid background and microscopic kinetic correction, both fully coupled to each other. A similar manner of discretization is proposed to that used in the recent \texttt{STRUPHY} code \cite{Holderied_Possanner_Wang_2021, Holderied_2022, Li_et_al_2023} with a finite-element model for the background and a pseudo-particle/PiC model for the correction.

        The fluid background satisfies the full, non-linear, resistive, compressible, Hall MHD equations. \cite{Laakmann_Hu_Farrell_2022} introduces finite-element(-in-space) implicit timesteppers for the incompressible analogue to this system with structure-preserving (SP) properties in the ideal case, alongside parameter-robust preconditioners. We show that these timesteppers can derive from a finite-element-in-time (FET) (and finite-element-in-space) interpretation. The benefits of this reformulation are discussed, including the derivation of timesteppers that are higher order in time, and the quantifiable dissipative SP properties in the non-ideal, resistive case.
        
        We discuss possible options for extending this FET approach to timesteppers for the compressible case.

        The kinetic corrections satisfy linearized Boltzmann equations. Using a Lénard--Bernstein collision operator, these take Fokker--Planck-like forms \cite{Fokker_1914, Planck_1917} wherein pseudo-particles in the numerical model obey the neoclassical transport equations, with particle-independent Brownian drift terms. This offers a rigorous methodology for incorporating collisions into the particle transport model, without coupling the equations of motions for each particle.
        
        Works by Chen, Chacón et al. \cite{Chen_Chacón_Barnes_2011, Chacón_Chen_Barnes_2013, Chen_Chacón_2014, Chen_Chacón_2015} have developed structure-preserving particle pushers for neoclassical transport in the Vlasov equations, derived from Crank--Nicolson integrators. We show these too can can derive from a FET interpretation, similarly offering potential extensions to higher-order-in-time particle pushers. The FET formulation is used also to consider how the stochastic drift terms can be incorporated into the pushers. Stochastic gyrokinetic expansions are also discussed.

        Different options for the numerical implementation of these schemes are considered.

        Due to the efficacy of FET in the development of SP timesteppers for both the fluid and kinetic component, we hope this approach will prove effective in the future for developing SP timesteppers for the full hybrid model. We hope this will give us the opportunity to incorporate previously inaccessible kinetic effects into the highly effective, modern, finite-element MHD models.
    \end{abstract}
    
    
    \newpage
    \tableofcontents
    
    
    \newpage
    \pagenumbering{arabic}
    %\linenumbers\renewcommand\thelinenumber{\color{black!50}\arabic{linenumber}}
            \input{0 - introduction/main.tex}
        \part{Research}
            \input{1 - low-noise PiC models/main.tex}
            \input{2 - kinetic component/main.tex}
            \input{3 - fluid component/main.tex}
            \input{4 - numerical implementation/main.tex}
        \part{Project Overview}
            \input{5 - research plan/main.tex}
            \input{6 - summary/main.tex}
    
    
    %\section{}
    \newpage
    \pagenumbering{gobble}
        \printbibliography


    \newpage
    \pagenumbering{roman}
    \appendix
        \part{Appendices}
            \input{8 - Hilbert complexes/main.tex}
            \input{9 - weak conservation proofs/main.tex}
\end{document}


\title{\BA{Title in Progress...}}
\author{Boris Andrews}
\affil{Mathematical Institute, University of Oxford}
\date{\today}


\begin{document}
    \pagenumbering{gobble}
    \maketitle
    
    
    \begin{abstract}
        Magnetic confinement reactors---in particular tokamaks---offer one of the most promising options for achieving practical nuclear fusion, with the potential to provide virtually limitless, clean energy. The theoretical and numerical modeling of tokamak plasmas is simultaneously an essential component of effective reactor design, and a great research barrier. Tokamak operational conditions exhibit comparatively low Knudsen numbers. Kinetic effects, including kinetic waves and instabilities, Landau damping, bump-on-tail instabilities and more, are therefore highly influential in tokamak plasma dynamics. Purely fluid models are inherently incapable of capturing these effects, whereas the high dimensionality in purely kinetic models render them practically intractable for most relevant purposes.

        We consider a $\delta\!f$ decomposition model, with a macroscopic fluid background and microscopic kinetic correction, both fully coupled to each other. A similar manner of discretization is proposed to that used in the recent \texttt{STRUPHY} code \cite{Holderied_Possanner_Wang_2021, Holderied_2022, Li_et_al_2023} with a finite-element model for the background and a pseudo-particle/PiC model for the correction.

        The fluid background satisfies the full, non-linear, resistive, compressible, Hall MHD equations. \cite{Laakmann_Hu_Farrell_2022} introduces finite-element(-in-space) implicit timesteppers for the incompressible analogue to this system with structure-preserving (SP) properties in the ideal case, alongside parameter-robust preconditioners. We show that these timesteppers can derive from a finite-element-in-time (FET) (and finite-element-in-space) interpretation. The benefits of this reformulation are discussed, including the derivation of timesteppers that are higher order in time, and the quantifiable dissipative SP properties in the non-ideal, resistive case.
        
        We discuss possible options for extending this FET approach to timesteppers for the compressible case.

        The kinetic corrections satisfy linearized Boltzmann equations. Using a Lénard--Bernstein collision operator, these take Fokker--Planck-like forms \cite{Fokker_1914, Planck_1917} wherein pseudo-particles in the numerical model obey the neoclassical transport equations, with particle-independent Brownian drift terms. This offers a rigorous methodology for incorporating collisions into the particle transport model, without coupling the equations of motions for each particle.
        
        Works by Chen, Chacón et al. \cite{Chen_Chacón_Barnes_2011, Chacón_Chen_Barnes_2013, Chen_Chacón_2014, Chen_Chacón_2015} have developed structure-preserving particle pushers for neoclassical transport in the Vlasov equations, derived from Crank--Nicolson integrators. We show these too can can derive from a FET interpretation, similarly offering potential extensions to higher-order-in-time particle pushers. The FET formulation is used also to consider how the stochastic drift terms can be incorporated into the pushers. Stochastic gyrokinetic expansions are also discussed.

        Different options for the numerical implementation of these schemes are considered.

        Due to the efficacy of FET in the development of SP timesteppers for both the fluid and kinetic component, we hope this approach will prove effective in the future for developing SP timesteppers for the full hybrid model. We hope this will give us the opportunity to incorporate previously inaccessible kinetic effects into the highly effective, modern, finite-element MHD models.
    \end{abstract}
    
    
    \newpage
    \tableofcontents
    
    
    \newpage
    \pagenumbering{arabic}
    %\linenumbers\renewcommand\thelinenumber{\color{black!50}\arabic{linenumber}}
            \documentclass[12pt, a4paper]{report}

\input{template/main.tex}

\title{\BA{Title in Progress...}}
\author{Boris Andrews}
\affil{Mathematical Institute, University of Oxford}
\date{\today}


\begin{document}
    \pagenumbering{gobble}
    \maketitle
    
    
    \begin{abstract}
        Magnetic confinement reactors---in particular tokamaks---offer one of the most promising options for achieving practical nuclear fusion, with the potential to provide virtually limitless, clean energy. The theoretical and numerical modeling of tokamak plasmas is simultaneously an essential component of effective reactor design, and a great research barrier. Tokamak operational conditions exhibit comparatively low Knudsen numbers. Kinetic effects, including kinetic waves and instabilities, Landau damping, bump-on-tail instabilities and more, are therefore highly influential in tokamak plasma dynamics. Purely fluid models are inherently incapable of capturing these effects, whereas the high dimensionality in purely kinetic models render them practically intractable for most relevant purposes.

        We consider a $\delta\!f$ decomposition model, with a macroscopic fluid background and microscopic kinetic correction, both fully coupled to each other. A similar manner of discretization is proposed to that used in the recent \texttt{STRUPHY} code \cite{Holderied_Possanner_Wang_2021, Holderied_2022, Li_et_al_2023} with a finite-element model for the background and a pseudo-particle/PiC model for the correction.

        The fluid background satisfies the full, non-linear, resistive, compressible, Hall MHD equations. \cite{Laakmann_Hu_Farrell_2022} introduces finite-element(-in-space) implicit timesteppers for the incompressible analogue to this system with structure-preserving (SP) properties in the ideal case, alongside parameter-robust preconditioners. We show that these timesteppers can derive from a finite-element-in-time (FET) (and finite-element-in-space) interpretation. The benefits of this reformulation are discussed, including the derivation of timesteppers that are higher order in time, and the quantifiable dissipative SP properties in the non-ideal, resistive case.
        
        We discuss possible options for extending this FET approach to timesteppers for the compressible case.

        The kinetic corrections satisfy linearized Boltzmann equations. Using a Lénard--Bernstein collision operator, these take Fokker--Planck-like forms \cite{Fokker_1914, Planck_1917} wherein pseudo-particles in the numerical model obey the neoclassical transport equations, with particle-independent Brownian drift terms. This offers a rigorous methodology for incorporating collisions into the particle transport model, without coupling the equations of motions for each particle.
        
        Works by Chen, Chacón et al. \cite{Chen_Chacón_Barnes_2011, Chacón_Chen_Barnes_2013, Chen_Chacón_2014, Chen_Chacón_2015} have developed structure-preserving particle pushers for neoclassical transport in the Vlasov equations, derived from Crank--Nicolson integrators. We show these too can can derive from a FET interpretation, similarly offering potential extensions to higher-order-in-time particle pushers. The FET formulation is used also to consider how the stochastic drift terms can be incorporated into the pushers. Stochastic gyrokinetic expansions are also discussed.

        Different options for the numerical implementation of these schemes are considered.

        Due to the efficacy of FET in the development of SP timesteppers for both the fluid and kinetic component, we hope this approach will prove effective in the future for developing SP timesteppers for the full hybrid model. We hope this will give us the opportunity to incorporate previously inaccessible kinetic effects into the highly effective, modern, finite-element MHD models.
    \end{abstract}
    
    
    \newpage
    \tableofcontents
    
    
    \newpage
    \pagenumbering{arabic}
    %\linenumbers\renewcommand\thelinenumber{\color{black!50}\arabic{linenumber}}
            \input{0 - introduction/main.tex}
        \part{Research}
            \input{1 - low-noise PiC models/main.tex}
            \input{2 - kinetic component/main.tex}
            \input{3 - fluid component/main.tex}
            \input{4 - numerical implementation/main.tex}
        \part{Project Overview}
            \input{5 - research plan/main.tex}
            \input{6 - summary/main.tex}
    
    
    %\section{}
    \newpage
    \pagenumbering{gobble}
        \printbibliography


    \newpage
    \pagenumbering{roman}
    \appendix
        \part{Appendices}
            \input{8 - Hilbert complexes/main.tex}
            \input{9 - weak conservation proofs/main.tex}
\end{document}

        \part{Research}
            \documentclass[12pt, a4paper]{report}

\input{template/main.tex}

\title{\BA{Title in Progress...}}
\author{Boris Andrews}
\affil{Mathematical Institute, University of Oxford}
\date{\today}


\begin{document}
    \pagenumbering{gobble}
    \maketitle
    
    
    \begin{abstract}
        Magnetic confinement reactors---in particular tokamaks---offer one of the most promising options for achieving practical nuclear fusion, with the potential to provide virtually limitless, clean energy. The theoretical and numerical modeling of tokamak plasmas is simultaneously an essential component of effective reactor design, and a great research barrier. Tokamak operational conditions exhibit comparatively low Knudsen numbers. Kinetic effects, including kinetic waves and instabilities, Landau damping, bump-on-tail instabilities and more, are therefore highly influential in tokamak plasma dynamics. Purely fluid models are inherently incapable of capturing these effects, whereas the high dimensionality in purely kinetic models render them practically intractable for most relevant purposes.

        We consider a $\delta\!f$ decomposition model, with a macroscopic fluid background and microscopic kinetic correction, both fully coupled to each other. A similar manner of discretization is proposed to that used in the recent \texttt{STRUPHY} code \cite{Holderied_Possanner_Wang_2021, Holderied_2022, Li_et_al_2023} with a finite-element model for the background and a pseudo-particle/PiC model for the correction.

        The fluid background satisfies the full, non-linear, resistive, compressible, Hall MHD equations. \cite{Laakmann_Hu_Farrell_2022} introduces finite-element(-in-space) implicit timesteppers for the incompressible analogue to this system with structure-preserving (SP) properties in the ideal case, alongside parameter-robust preconditioners. We show that these timesteppers can derive from a finite-element-in-time (FET) (and finite-element-in-space) interpretation. The benefits of this reformulation are discussed, including the derivation of timesteppers that are higher order in time, and the quantifiable dissipative SP properties in the non-ideal, resistive case.
        
        We discuss possible options for extending this FET approach to timesteppers for the compressible case.

        The kinetic corrections satisfy linearized Boltzmann equations. Using a Lénard--Bernstein collision operator, these take Fokker--Planck-like forms \cite{Fokker_1914, Planck_1917} wherein pseudo-particles in the numerical model obey the neoclassical transport equations, with particle-independent Brownian drift terms. This offers a rigorous methodology for incorporating collisions into the particle transport model, without coupling the equations of motions for each particle.
        
        Works by Chen, Chacón et al. \cite{Chen_Chacón_Barnes_2011, Chacón_Chen_Barnes_2013, Chen_Chacón_2014, Chen_Chacón_2015} have developed structure-preserving particle pushers for neoclassical transport in the Vlasov equations, derived from Crank--Nicolson integrators. We show these too can can derive from a FET interpretation, similarly offering potential extensions to higher-order-in-time particle pushers. The FET formulation is used also to consider how the stochastic drift terms can be incorporated into the pushers. Stochastic gyrokinetic expansions are also discussed.

        Different options for the numerical implementation of these schemes are considered.

        Due to the efficacy of FET in the development of SP timesteppers for both the fluid and kinetic component, we hope this approach will prove effective in the future for developing SP timesteppers for the full hybrid model. We hope this will give us the opportunity to incorporate previously inaccessible kinetic effects into the highly effective, modern, finite-element MHD models.
    \end{abstract}
    
    
    \newpage
    \tableofcontents
    
    
    \newpage
    \pagenumbering{arabic}
    %\linenumbers\renewcommand\thelinenumber{\color{black!50}\arabic{linenumber}}
            \input{0 - introduction/main.tex}
        \part{Research}
            \input{1 - low-noise PiC models/main.tex}
            \input{2 - kinetic component/main.tex}
            \input{3 - fluid component/main.tex}
            \input{4 - numerical implementation/main.tex}
        \part{Project Overview}
            \input{5 - research plan/main.tex}
            \input{6 - summary/main.tex}
    
    
    %\section{}
    \newpage
    \pagenumbering{gobble}
        \printbibliography


    \newpage
    \pagenumbering{roman}
    \appendix
        \part{Appendices}
            \input{8 - Hilbert complexes/main.tex}
            \input{9 - weak conservation proofs/main.tex}
\end{document}

            \documentclass[12pt, a4paper]{report}

\input{template/main.tex}

\title{\BA{Title in Progress...}}
\author{Boris Andrews}
\affil{Mathematical Institute, University of Oxford}
\date{\today}


\begin{document}
    \pagenumbering{gobble}
    \maketitle
    
    
    \begin{abstract}
        Magnetic confinement reactors---in particular tokamaks---offer one of the most promising options for achieving practical nuclear fusion, with the potential to provide virtually limitless, clean energy. The theoretical and numerical modeling of tokamak plasmas is simultaneously an essential component of effective reactor design, and a great research barrier. Tokamak operational conditions exhibit comparatively low Knudsen numbers. Kinetic effects, including kinetic waves and instabilities, Landau damping, bump-on-tail instabilities and more, are therefore highly influential in tokamak plasma dynamics. Purely fluid models are inherently incapable of capturing these effects, whereas the high dimensionality in purely kinetic models render them practically intractable for most relevant purposes.

        We consider a $\delta\!f$ decomposition model, with a macroscopic fluid background and microscopic kinetic correction, both fully coupled to each other. A similar manner of discretization is proposed to that used in the recent \texttt{STRUPHY} code \cite{Holderied_Possanner_Wang_2021, Holderied_2022, Li_et_al_2023} with a finite-element model for the background and a pseudo-particle/PiC model for the correction.

        The fluid background satisfies the full, non-linear, resistive, compressible, Hall MHD equations. \cite{Laakmann_Hu_Farrell_2022} introduces finite-element(-in-space) implicit timesteppers for the incompressible analogue to this system with structure-preserving (SP) properties in the ideal case, alongside parameter-robust preconditioners. We show that these timesteppers can derive from a finite-element-in-time (FET) (and finite-element-in-space) interpretation. The benefits of this reformulation are discussed, including the derivation of timesteppers that are higher order in time, and the quantifiable dissipative SP properties in the non-ideal, resistive case.
        
        We discuss possible options for extending this FET approach to timesteppers for the compressible case.

        The kinetic corrections satisfy linearized Boltzmann equations. Using a Lénard--Bernstein collision operator, these take Fokker--Planck-like forms \cite{Fokker_1914, Planck_1917} wherein pseudo-particles in the numerical model obey the neoclassical transport equations, with particle-independent Brownian drift terms. This offers a rigorous methodology for incorporating collisions into the particle transport model, without coupling the equations of motions for each particle.
        
        Works by Chen, Chacón et al. \cite{Chen_Chacón_Barnes_2011, Chacón_Chen_Barnes_2013, Chen_Chacón_2014, Chen_Chacón_2015} have developed structure-preserving particle pushers for neoclassical transport in the Vlasov equations, derived from Crank--Nicolson integrators. We show these too can can derive from a FET interpretation, similarly offering potential extensions to higher-order-in-time particle pushers. The FET formulation is used also to consider how the stochastic drift terms can be incorporated into the pushers. Stochastic gyrokinetic expansions are also discussed.

        Different options for the numerical implementation of these schemes are considered.

        Due to the efficacy of FET in the development of SP timesteppers for both the fluid and kinetic component, we hope this approach will prove effective in the future for developing SP timesteppers for the full hybrid model. We hope this will give us the opportunity to incorporate previously inaccessible kinetic effects into the highly effective, modern, finite-element MHD models.
    \end{abstract}
    
    
    \newpage
    \tableofcontents
    
    
    \newpage
    \pagenumbering{arabic}
    %\linenumbers\renewcommand\thelinenumber{\color{black!50}\arabic{linenumber}}
            \input{0 - introduction/main.tex}
        \part{Research}
            \input{1 - low-noise PiC models/main.tex}
            \input{2 - kinetic component/main.tex}
            \input{3 - fluid component/main.tex}
            \input{4 - numerical implementation/main.tex}
        \part{Project Overview}
            \input{5 - research plan/main.tex}
            \input{6 - summary/main.tex}
    
    
    %\section{}
    \newpage
    \pagenumbering{gobble}
        \printbibliography


    \newpage
    \pagenumbering{roman}
    \appendix
        \part{Appendices}
            \input{8 - Hilbert complexes/main.tex}
            \input{9 - weak conservation proofs/main.tex}
\end{document}

            \documentclass[12pt, a4paper]{report}

\input{template/main.tex}

\title{\BA{Title in Progress...}}
\author{Boris Andrews}
\affil{Mathematical Institute, University of Oxford}
\date{\today}


\begin{document}
    \pagenumbering{gobble}
    \maketitle
    
    
    \begin{abstract}
        Magnetic confinement reactors---in particular tokamaks---offer one of the most promising options for achieving practical nuclear fusion, with the potential to provide virtually limitless, clean energy. The theoretical and numerical modeling of tokamak plasmas is simultaneously an essential component of effective reactor design, and a great research barrier. Tokamak operational conditions exhibit comparatively low Knudsen numbers. Kinetic effects, including kinetic waves and instabilities, Landau damping, bump-on-tail instabilities and more, are therefore highly influential in tokamak plasma dynamics. Purely fluid models are inherently incapable of capturing these effects, whereas the high dimensionality in purely kinetic models render them practically intractable for most relevant purposes.

        We consider a $\delta\!f$ decomposition model, with a macroscopic fluid background and microscopic kinetic correction, both fully coupled to each other. A similar manner of discretization is proposed to that used in the recent \texttt{STRUPHY} code \cite{Holderied_Possanner_Wang_2021, Holderied_2022, Li_et_al_2023} with a finite-element model for the background and a pseudo-particle/PiC model for the correction.

        The fluid background satisfies the full, non-linear, resistive, compressible, Hall MHD equations. \cite{Laakmann_Hu_Farrell_2022} introduces finite-element(-in-space) implicit timesteppers for the incompressible analogue to this system with structure-preserving (SP) properties in the ideal case, alongside parameter-robust preconditioners. We show that these timesteppers can derive from a finite-element-in-time (FET) (and finite-element-in-space) interpretation. The benefits of this reformulation are discussed, including the derivation of timesteppers that are higher order in time, and the quantifiable dissipative SP properties in the non-ideal, resistive case.
        
        We discuss possible options for extending this FET approach to timesteppers for the compressible case.

        The kinetic corrections satisfy linearized Boltzmann equations. Using a Lénard--Bernstein collision operator, these take Fokker--Planck-like forms \cite{Fokker_1914, Planck_1917} wherein pseudo-particles in the numerical model obey the neoclassical transport equations, with particle-independent Brownian drift terms. This offers a rigorous methodology for incorporating collisions into the particle transport model, without coupling the equations of motions for each particle.
        
        Works by Chen, Chacón et al. \cite{Chen_Chacón_Barnes_2011, Chacón_Chen_Barnes_2013, Chen_Chacón_2014, Chen_Chacón_2015} have developed structure-preserving particle pushers for neoclassical transport in the Vlasov equations, derived from Crank--Nicolson integrators. We show these too can can derive from a FET interpretation, similarly offering potential extensions to higher-order-in-time particle pushers. The FET formulation is used also to consider how the stochastic drift terms can be incorporated into the pushers. Stochastic gyrokinetic expansions are also discussed.

        Different options for the numerical implementation of these schemes are considered.

        Due to the efficacy of FET in the development of SP timesteppers for both the fluid and kinetic component, we hope this approach will prove effective in the future for developing SP timesteppers for the full hybrid model. We hope this will give us the opportunity to incorporate previously inaccessible kinetic effects into the highly effective, modern, finite-element MHD models.
    \end{abstract}
    
    
    \newpage
    \tableofcontents
    
    
    \newpage
    \pagenumbering{arabic}
    %\linenumbers\renewcommand\thelinenumber{\color{black!50}\arabic{linenumber}}
            \input{0 - introduction/main.tex}
        \part{Research}
            \input{1 - low-noise PiC models/main.tex}
            \input{2 - kinetic component/main.tex}
            \input{3 - fluid component/main.tex}
            \input{4 - numerical implementation/main.tex}
        \part{Project Overview}
            \input{5 - research plan/main.tex}
            \input{6 - summary/main.tex}
    
    
    %\section{}
    \newpage
    \pagenumbering{gobble}
        \printbibliography


    \newpage
    \pagenumbering{roman}
    \appendix
        \part{Appendices}
            \input{8 - Hilbert complexes/main.tex}
            \input{9 - weak conservation proofs/main.tex}
\end{document}

            \documentclass[12pt, a4paper]{report}

\input{template/main.tex}

\title{\BA{Title in Progress...}}
\author{Boris Andrews}
\affil{Mathematical Institute, University of Oxford}
\date{\today}


\begin{document}
    \pagenumbering{gobble}
    \maketitle
    
    
    \begin{abstract}
        Magnetic confinement reactors---in particular tokamaks---offer one of the most promising options for achieving practical nuclear fusion, with the potential to provide virtually limitless, clean energy. The theoretical and numerical modeling of tokamak plasmas is simultaneously an essential component of effective reactor design, and a great research barrier. Tokamak operational conditions exhibit comparatively low Knudsen numbers. Kinetic effects, including kinetic waves and instabilities, Landau damping, bump-on-tail instabilities and more, are therefore highly influential in tokamak plasma dynamics. Purely fluid models are inherently incapable of capturing these effects, whereas the high dimensionality in purely kinetic models render them practically intractable for most relevant purposes.

        We consider a $\delta\!f$ decomposition model, with a macroscopic fluid background and microscopic kinetic correction, both fully coupled to each other. A similar manner of discretization is proposed to that used in the recent \texttt{STRUPHY} code \cite{Holderied_Possanner_Wang_2021, Holderied_2022, Li_et_al_2023} with a finite-element model for the background and a pseudo-particle/PiC model for the correction.

        The fluid background satisfies the full, non-linear, resistive, compressible, Hall MHD equations. \cite{Laakmann_Hu_Farrell_2022} introduces finite-element(-in-space) implicit timesteppers for the incompressible analogue to this system with structure-preserving (SP) properties in the ideal case, alongside parameter-robust preconditioners. We show that these timesteppers can derive from a finite-element-in-time (FET) (and finite-element-in-space) interpretation. The benefits of this reformulation are discussed, including the derivation of timesteppers that are higher order in time, and the quantifiable dissipative SP properties in the non-ideal, resistive case.
        
        We discuss possible options for extending this FET approach to timesteppers for the compressible case.

        The kinetic corrections satisfy linearized Boltzmann equations. Using a Lénard--Bernstein collision operator, these take Fokker--Planck-like forms \cite{Fokker_1914, Planck_1917} wherein pseudo-particles in the numerical model obey the neoclassical transport equations, with particle-independent Brownian drift terms. This offers a rigorous methodology for incorporating collisions into the particle transport model, without coupling the equations of motions for each particle.
        
        Works by Chen, Chacón et al. \cite{Chen_Chacón_Barnes_2011, Chacón_Chen_Barnes_2013, Chen_Chacón_2014, Chen_Chacón_2015} have developed structure-preserving particle pushers for neoclassical transport in the Vlasov equations, derived from Crank--Nicolson integrators. We show these too can can derive from a FET interpretation, similarly offering potential extensions to higher-order-in-time particle pushers. The FET formulation is used also to consider how the stochastic drift terms can be incorporated into the pushers. Stochastic gyrokinetic expansions are also discussed.

        Different options for the numerical implementation of these schemes are considered.

        Due to the efficacy of FET in the development of SP timesteppers for both the fluid and kinetic component, we hope this approach will prove effective in the future for developing SP timesteppers for the full hybrid model. We hope this will give us the opportunity to incorporate previously inaccessible kinetic effects into the highly effective, modern, finite-element MHD models.
    \end{abstract}
    
    
    \newpage
    \tableofcontents
    
    
    \newpage
    \pagenumbering{arabic}
    %\linenumbers\renewcommand\thelinenumber{\color{black!50}\arabic{linenumber}}
            \input{0 - introduction/main.tex}
        \part{Research}
            \input{1 - low-noise PiC models/main.tex}
            \input{2 - kinetic component/main.tex}
            \input{3 - fluid component/main.tex}
            \input{4 - numerical implementation/main.tex}
        \part{Project Overview}
            \input{5 - research plan/main.tex}
            \input{6 - summary/main.tex}
    
    
    %\section{}
    \newpage
    \pagenumbering{gobble}
        \printbibliography


    \newpage
    \pagenumbering{roman}
    \appendix
        \part{Appendices}
            \input{8 - Hilbert complexes/main.tex}
            \input{9 - weak conservation proofs/main.tex}
\end{document}

        \part{Project Overview}
            \documentclass[12pt, a4paper]{report}

\input{template/main.tex}

\title{\BA{Title in Progress...}}
\author{Boris Andrews}
\affil{Mathematical Institute, University of Oxford}
\date{\today}


\begin{document}
    \pagenumbering{gobble}
    \maketitle
    
    
    \begin{abstract}
        Magnetic confinement reactors---in particular tokamaks---offer one of the most promising options for achieving practical nuclear fusion, with the potential to provide virtually limitless, clean energy. The theoretical and numerical modeling of tokamak plasmas is simultaneously an essential component of effective reactor design, and a great research barrier. Tokamak operational conditions exhibit comparatively low Knudsen numbers. Kinetic effects, including kinetic waves and instabilities, Landau damping, bump-on-tail instabilities and more, are therefore highly influential in tokamak plasma dynamics. Purely fluid models are inherently incapable of capturing these effects, whereas the high dimensionality in purely kinetic models render them practically intractable for most relevant purposes.

        We consider a $\delta\!f$ decomposition model, with a macroscopic fluid background and microscopic kinetic correction, both fully coupled to each other. A similar manner of discretization is proposed to that used in the recent \texttt{STRUPHY} code \cite{Holderied_Possanner_Wang_2021, Holderied_2022, Li_et_al_2023} with a finite-element model for the background and a pseudo-particle/PiC model for the correction.

        The fluid background satisfies the full, non-linear, resistive, compressible, Hall MHD equations. \cite{Laakmann_Hu_Farrell_2022} introduces finite-element(-in-space) implicit timesteppers for the incompressible analogue to this system with structure-preserving (SP) properties in the ideal case, alongside parameter-robust preconditioners. We show that these timesteppers can derive from a finite-element-in-time (FET) (and finite-element-in-space) interpretation. The benefits of this reformulation are discussed, including the derivation of timesteppers that are higher order in time, and the quantifiable dissipative SP properties in the non-ideal, resistive case.
        
        We discuss possible options for extending this FET approach to timesteppers for the compressible case.

        The kinetic corrections satisfy linearized Boltzmann equations. Using a Lénard--Bernstein collision operator, these take Fokker--Planck-like forms \cite{Fokker_1914, Planck_1917} wherein pseudo-particles in the numerical model obey the neoclassical transport equations, with particle-independent Brownian drift terms. This offers a rigorous methodology for incorporating collisions into the particle transport model, without coupling the equations of motions for each particle.
        
        Works by Chen, Chacón et al. \cite{Chen_Chacón_Barnes_2011, Chacón_Chen_Barnes_2013, Chen_Chacón_2014, Chen_Chacón_2015} have developed structure-preserving particle pushers for neoclassical transport in the Vlasov equations, derived from Crank--Nicolson integrators. We show these too can can derive from a FET interpretation, similarly offering potential extensions to higher-order-in-time particle pushers. The FET formulation is used also to consider how the stochastic drift terms can be incorporated into the pushers. Stochastic gyrokinetic expansions are also discussed.

        Different options for the numerical implementation of these schemes are considered.

        Due to the efficacy of FET in the development of SP timesteppers for both the fluid and kinetic component, we hope this approach will prove effective in the future for developing SP timesteppers for the full hybrid model. We hope this will give us the opportunity to incorporate previously inaccessible kinetic effects into the highly effective, modern, finite-element MHD models.
    \end{abstract}
    
    
    \newpage
    \tableofcontents
    
    
    \newpage
    \pagenumbering{arabic}
    %\linenumbers\renewcommand\thelinenumber{\color{black!50}\arabic{linenumber}}
            \input{0 - introduction/main.tex}
        \part{Research}
            \input{1 - low-noise PiC models/main.tex}
            \input{2 - kinetic component/main.tex}
            \input{3 - fluid component/main.tex}
            \input{4 - numerical implementation/main.tex}
        \part{Project Overview}
            \input{5 - research plan/main.tex}
            \input{6 - summary/main.tex}
    
    
    %\section{}
    \newpage
    \pagenumbering{gobble}
        \printbibliography


    \newpage
    \pagenumbering{roman}
    \appendix
        \part{Appendices}
            \input{8 - Hilbert complexes/main.tex}
            \input{9 - weak conservation proofs/main.tex}
\end{document}

            \documentclass[12pt, a4paper]{report}

\input{template/main.tex}

\title{\BA{Title in Progress...}}
\author{Boris Andrews}
\affil{Mathematical Institute, University of Oxford}
\date{\today}


\begin{document}
    \pagenumbering{gobble}
    \maketitle
    
    
    \begin{abstract}
        Magnetic confinement reactors---in particular tokamaks---offer one of the most promising options for achieving practical nuclear fusion, with the potential to provide virtually limitless, clean energy. The theoretical and numerical modeling of tokamak plasmas is simultaneously an essential component of effective reactor design, and a great research barrier. Tokamak operational conditions exhibit comparatively low Knudsen numbers. Kinetic effects, including kinetic waves and instabilities, Landau damping, bump-on-tail instabilities and more, are therefore highly influential in tokamak plasma dynamics. Purely fluid models are inherently incapable of capturing these effects, whereas the high dimensionality in purely kinetic models render them practically intractable for most relevant purposes.

        We consider a $\delta\!f$ decomposition model, with a macroscopic fluid background and microscopic kinetic correction, both fully coupled to each other. A similar manner of discretization is proposed to that used in the recent \texttt{STRUPHY} code \cite{Holderied_Possanner_Wang_2021, Holderied_2022, Li_et_al_2023} with a finite-element model for the background and a pseudo-particle/PiC model for the correction.

        The fluid background satisfies the full, non-linear, resistive, compressible, Hall MHD equations. \cite{Laakmann_Hu_Farrell_2022} introduces finite-element(-in-space) implicit timesteppers for the incompressible analogue to this system with structure-preserving (SP) properties in the ideal case, alongside parameter-robust preconditioners. We show that these timesteppers can derive from a finite-element-in-time (FET) (and finite-element-in-space) interpretation. The benefits of this reformulation are discussed, including the derivation of timesteppers that are higher order in time, and the quantifiable dissipative SP properties in the non-ideal, resistive case.
        
        We discuss possible options for extending this FET approach to timesteppers for the compressible case.

        The kinetic corrections satisfy linearized Boltzmann equations. Using a Lénard--Bernstein collision operator, these take Fokker--Planck-like forms \cite{Fokker_1914, Planck_1917} wherein pseudo-particles in the numerical model obey the neoclassical transport equations, with particle-independent Brownian drift terms. This offers a rigorous methodology for incorporating collisions into the particle transport model, without coupling the equations of motions for each particle.
        
        Works by Chen, Chacón et al. \cite{Chen_Chacón_Barnes_2011, Chacón_Chen_Barnes_2013, Chen_Chacón_2014, Chen_Chacón_2015} have developed structure-preserving particle pushers for neoclassical transport in the Vlasov equations, derived from Crank--Nicolson integrators. We show these too can can derive from a FET interpretation, similarly offering potential extensions to higher-order-in-time particle pushers. The FET formulation is used also to consider how the stochastic drift terms can be incorporated into the pushers. Stochastic gyrokinetic expansions are also discussed.

        Different options for the numerical implementation of these schemes are considered.

        Due to the efficacy of FET in the development of SP timesteppers for both the fluid and kinetic component, we hope this approach will prove effective in the future for developing SP timesteppers for the full hybrid model. We hope this will give us the opportunity to incorporate previously inaccessible kinetic effects into the highly effective, modern, finite-element MHD models.
    \end{abstract}
    
    
    \newpage
    \tableofcontents
    
    
    \newpage
    \pagenumbering{arabic}
    %\linenumbers\renewcommand\thelinenumber{\color{black!50}\arabic{linenumber}}
            \input{0 - introduction/main.tex}
        \part{Research}
            \input{1 - low-noise PiC models/main.tex}
            \input{2 - kinetic component/main.tex}
            \input{3 - fluid component/main.tex}
            \input{4 - numerical implementation/main.tex}
        \part{Project Overview}
            \input{5 - research plan/main.tex}
            \input{6 - summary/main.tex}
    
    
    %\section{}
    \newpage
    \pagenumbering{gobble}
        \printbibliography


    \newpage
    \pagenumbering{roman}
    \appendix
        \part{Appendices}
            \input{8 - Hilbert complexes/main.tex}
            \input{9 - weak conservation proofs/main.tex}
\end{document}

    
    
    %\section{}
    \newpage
    \pagenumbering{gobble}
        \printbibliography


    \newpage
    \pagenumbering{roman}
    \appendix
        \part{Appendices}
            \documentclass[12pt, a4paper]{report}

\input{template/main.tex}

\title{\BA{Title in Progress...}}
\author{Boris Andrews}
\affil{Mathematical Institute, University of Oxford}
\date{\today}


\begin{document}
    \pagenumbering{gobble}
    \maketitle
    
    
    \begin{abstract}
        Magnetic confinement reactors---in particular tokamaks---offer one of the most promising options for achieving practical nuclear fusion, with the potential to provide virtually limitless, clean energy. The theoretical and numerical modeling of tokamak plasmas is simultaneously an essential component of effective reactor design, and a great research barrier. Tokamak operational conditions exhibit comparatively low Knudsen numbers. Kinetic effects, including kinetic waves and instabilities, Landau damping, bump-on-tail instabilities and more, are therefore highly influential in tokamak plasma dynamics. Purely fluid models are inherently incapable of capturing these effects, whereas the high dimensionality in purely kinetic models render them practically intractable for most relevant purposes.

        We consider a $\delta\!f$ decomposition model, with a macroscopic fluid background and microscopic kinetic correction, both fully coupled to each other. A similar manner of discretization is proposed to that used in the recent \texttt{STRUPHY} code \cite{Holderied_Possanner_Wang_2021, Holderied_2022, Li_et_al_2023} with a finite-element model for the background and a pseudo-particle/PiC model for the correction.

        The fluid background satisfies the full, non-linear, resistive, compressible, Hall MHD equations. \cite{Laakmann_Hu_Farrell_2022} introduces finite-element(-in-space) implicit timesteppers for the incompressible analogue to this system with structure-preserving (SP) properties in the ideal case, alongside parameter-robust preconditioners. We show that these timesteppers can derive from a finite-element-in-time (FET) (and finite-element-in-space) interpretation. The benefits of this reformulation are discussed, including the derivation of timesteppers that are higher order in time, and the quantifiable dissipative SP properties in the non-ideal, resistive case.
        
        We discuss possible options for extending this FET approach to timesteppers for the compressible case.

        The kinetic corrections satisfy linearized Boltzmann equations. Using a Lénard--Bernstein collision operator, these take Fokker--Planck-like forms \cite{Fokker_1914, Planck_1917} wherein pseudo-particles in the numerical model obey the neoclassical transport equations, with particle-independent Brownian drift terms. This offers a rigorous methodology for incorporating collisions into the particle transport model, without coupling the equations of motions for each particle.
        
        Works by Chen, Chacón et al. \cite{Chen_Chacón_Barnes_2011, Chacón_Chen_Barnes_2013, Chen_Chacón_2014, Chen_Chacón_2015} have developed structure-preserving particle pushers for neoclassical transport in the Vlasov equations, derived from Crank--Nicolson integrators. We show these too can can derive from a FET interpretation, similarly offering potential extensions to higher-order-in-time particle pushers. The FET formulation is used also to consider how the stochastic drift terms can be incorporated into the pushers. Stochastic gyrokinetic expansions are also discussed.

        Different options for the numerical implementation of these schemes are considered.

        Due to the efficacy of FET in the development of SP timesteppers for both the fluid and kinetic component, we hope this approach will prove effective in the future for developing SP timesteppers for the full hybrid model. We hope this will give us the opportunity to incorporate previously inaccessible kinetic effects into the highly effective, modern, finite-element MHD models.
    \end{abstract}
    
    
    \newpage
    \tableofcontents
    
    
    \newpage
    \pagenumbering{arabic}
    %\linenumbers\renewcommand\thelinenumber{\color{black!50}\arabic{linenumber}}
            \input{0 - introduction/main.tex}
        \part{Research}
            \input{1 - low-noise PiC models/main.tex}
            \input{2 - kinetic component/main.tex}
            \input{3 - fluid component/main.tex}
            \input{4 - numerical implementation/main.tex}
        \part{Project Overview}
            \input{5 - research plan/main.tex}
            \input{6 - summary/main.tex}
    
    
    %\section{}
    \newpage
    \pagenumbering{gobble}
        \printbibliography


    \newpage
    \pagenumbering{roman}
    \appendix
        \part{Appendices}
            \input{8 - Hilbert complexes/main.tex}
            \input{9 - weak conservation proofs/main.tex}
\end{document}

            \documentclass[12pt, a4paper]{report}

\input{template/main.tex}

\title{\BA{Title in Progress...}}
\author{Boris Andrews}
\affil{Mathematical Institute, University of Oxford}
\date{\today}


\begin{document}
    \pagenumbering{gobble}
    \maketitle
    
    
    \begin{abstract}
        Magnetic confinement reactors---in particular tokamaks---offer one of the most promising options for achieving practical nuclear fusion, with the potential to provide virtually limitless, clean energy. The theoretical and numerical modeling of tokamak plasmas is simultaneously an essential component of effective reactor design, and a great research barrier. Tokamak operational conditions exhibit comparatively low Knudsen numbers. Kinetic effects, including kinetic waves and instabilities, Landau damping, bump-on-tail instabilities and more, are therefore highly influential in tokamak plasma dynamics. Purely fluid models are inherently incapable of capturing these effects, whereas the high dimensionality in purely kinetic models render them practically intractable for most relevant purposes.

        We consider a $\delta\!f$ decomposition model, with a macroscopic fluid background and microscopic kinetic correction, both fully coupled to each other. A similar manner of discretization is proposed to that used in the recent \texttt{STRUPHY} code \cite{Holderied_Possanner_Wang_2021, Holderied_2022, Li_et_al_2023} with a finite-element model for the background and a pseudo-particle/PiC model for the correction.

        The fluid background satisfies the full, non-linear, resistive, compressible, Hall MHD equations. \cite{Laakmann_Hu_Farrell_2022} introduces finite-element(-in-space) implicit timesteppers for the incompressible analogue to this system with structure-preserving (SP) properties in the ideal case, alongside parameter-robust preconditioners. We show that these timesteppers can derive from a finite-element-in-time (FET) (and finite-element-in-space) interpretation. The benefits of this reformulation are discussed, including the derivation of timesteppers that are higher order in time, and the quantifiable dissipative SP properties in the non-ideal, resistive case.
        
        We discuss possible options for extending this FET approach to timesteppers for the compressible case.

        The kinetic corrections satisfy linearized Boltzmann equations. Using a Lénard--Bernstein collision operator, these take Fokker--Planck-like forms \cite{Fokker_1914, Planck_1917} wherein pseudo-particles in the numerical model obey the neoclassical transport equations, with particle-independent Brownian drift terms. This offers a rigorous methodology for incorporating collisions into the particle transport model, without coupling the equations of motions for each particle.
        
        Works by Chen, Chacón et al. \cite{Chen_Chacón_Barnes_2011, Chacón_Chen_Barnes_2013, Chen_Chacón_2014, Chen_Chacón_2015} have developed structure-preserving particle pushers for neoclassical transport in the Vlasov equations, derived from Crank--Nicolson integrators. We show these too can can derive from a FET interpretation, similarly offering potential extensions to higher-order-in-time particle pushers. The FET formulation is used also to consider how the stochastic drift terms can be incorporated into the pushers. Stochastic gyrokinetic expansions are also discussed.

        Different options for the numerical implementation of these schemes are considered.

        Due to the efficacy of FET in the development of SP timesteppers for both the fluid and kinetic component, we hope this approach will prove effective in the future for developing SP timesteppers for the full hybrid model. We hope this will give us the opportunity to incorporate previously inaccessible kinetic effects into the highly effective, modern, finite-element MHD models.
    \end{abstract}
    
    
    \newpage
    \tableofcontents
    
    
    \newpage
    \pagenumbering{arabic}
    %\linenumbers\renewcommand\thelinenumber{\color{black!50}\arabic{linenumber}}
            \input{0 - introduction/main.tex}
        \part{Research}
            \input{1 - low-noise PiC models/main.tex}
            \input{2 - kinetic component/main.tex}
            \input{3 - fluid component/main.tex}
            \input{4 - numerical implementation/main.tex}
        \part{Project Overview}
            \input{5 - research plan/main.tex}
            \input{6 - summary/main.tex}
    
    
    %\section{}
    \newpage
    \pagenumbering{gobble}
        \printbibliography


    \newpage
    \pagenumbering{roman}
    \appendix
        \part{Appendices}
            \input{8 - Hilbert complexes/main.tex}
            \input{9 - weak conservation proofs/main.tex}
\end{document}

\end{document}


\title{\BA{Title in Progress...}}
\author{Boris Andrews}
\affil{Mathematical Institute, University of Oxford}
\date{\today}


\begin{document}
    \pagenumbering{gobble}
    \maketitle
    
    
    \begin{abstract}
        Magnetic confinement reactors---in particular tokamaks---offer one of the most promising options for achieving practical nuclear fusion, with the potential to provide virtually limitless, clean energy. The theoretical and numerical modeling of tokamak plasmas is simultaneously an essential component of effective reactor design, and a great research barrier. Tokamak operational conditions exhibit comparatively low Knudsen numbers. Kinetic effects, including kinetic waves and instabilities, Landau damping, bump-on-tail instabilities and more, are therefore highly influential in tokamak plasma dynamics. Purely fluid models are inherently incapable of capturing these effects, whereas the high dimensionality in purely kinetic models render them practically intractable for most relevant purposes.

        We consider a $\delta\!f$ decomposition model, with a macroscopic fluid background and microscopic kinetic correction, both fully coupled to each other. A similar manner of discretization is proposed to that used in the recent \texttt{STRUPHY} code \cite{Holderied_Possanner_Wang_2021, Holderied_2022, Li_et_al_2023} with a finite-element model for the background and a pseudo-particle/PiC model for the correction.

        The fluid background satisfies the full, non-linear, resistive, compressible, Hall MHD equations. \cite{Laakmann_Hu_Farrell_2022} introduces finite-element(-in-space) implicit timesteppers for the incompressible analogue to this system with structure-preserving (SP) properties in the ideal case, alongside parameter-robust preconditioners. We show that these timesteppers can derive from a finite-element-in-time (FET) (and finite-element-in-space) interpretation. The benefits of this reformulation are discussed, including the derivation of timesteppers that are higher order in time, and the quantifiable dissipative SP properties in the non-ideal, resistive case.
        
        We discuss possible options for extending this FET approach to timesteppers for the compressible case.

        The kinetic corrections satisfy linearized Boltzmann equations. Using a Lénard--Bernstein collision operator, these take Fokker--Planck-like forms \cite{Fokker_1914, Planck_1917} wherein pseudo-particles in the numerical model obey the neoclassical transport equations, with particle-independent Brownian drift terms. This offers a rigorous methodology for incorporating collisions into the particle transport model, without coupling the equations of motions for each particle.
        
        Works by Chen, Chacón et al. \cite{Chen_Chacón_Barnes_2011, Chacón_Chen_Barnes_2013, Chen_Chacón_2014, Chen_Chacón_2015} have developed structure-preserving particle pushers for neoclassical transport in the Vlasov equations, derived from Crank--Nicolson integrators. We show these too can can derive from a FET interpretation, similarly offering potential extensions to higher-order-in-time particle pushers. The FET formulation is used also to consider how the stochastic drift terms can be incorporated into the pushers. Stochastic gyrokinetic expansions are also discussed.

        Different options for the numerical implementation of these schemes are considered.

        Due to the efficacy of FET in the development of SP timesteppers for both the fluid and kinetic component, we hope this approach will prove effective in the future for developing SP timesteppers for the full hybrid model. We hope this will give us the opportunity to incorporate previously inaccessible kinetic effects into the highly effective, modern, finite-element MHD models.
    \end{abstract}
    
    
    \newpage
    \tableofcontents
    
    
    \newpage
    \pagenumbering{arabic}
    %\linenumbers\renewcommand\thelinenumber{\color{black!50}\arabic{linenumber}}
            \documentclass[12pt, a4paper]{report}

\documentclass[12pt, a4paper]{report}

\input{template/main.tex}

\title{\BA{Title in Progress...}}
\author{Boris Andrews}
\affil{Mathematical Institute, University of Oxford}
\date{\today}


\begin{document}
    \pagenumbering{gobble}
    \maketitle
    
    
    \begin{abstract}
        Magnetic confinement reactors---in particular tokamaks---offer one of the most promising options for achieving practical nuclear fusion, with the potential to provide virtually limitless, clean energy. The theoretical and numerical modeling of tokamak plasmas is simultaneously an essential component of effective reactor design, and a great research barrier. Tokamak operational conditions exhibit comparatively low Knudsen numbers. Kinetic effects, including kinetic waves and instabilities, Landau damping, bump-on-tail instabilities and more, are therefore highly influential in tokamak plasma dynamics. Purely fluid models are inherently incapable of capturing these effects, whereas the high dimensionality in purely kinetic models render them practically intractable for most relevant purposes.

        We consider a $\delta\!f$ decomposition model, with a macroscopic fluid background and microscopic kinetic correction, both fully coupled to each other. A similar manner of discretization is proposed to that used in the recent \texttt{STRUPHY} code \cite{Holderied_Possanner_Wang_2021, Holderied_2022, Li_et_al_2023} with a finite-element model for the background and a pseudo-particle/PiC model for the correction.

        The fluid background satisfies the full, non-linear, resistive, compressible, Hall MHD equations. \cite{Laakmann_Hu_Farrell_2022} introduces finite-element(-in-space) implicit timesteppers for the incompressible analogue to this system with structure-preserving (SP) properties in the ideal case, alongside parameter-robust preconditioners. We show that these timesteppers can derive from a finite-element-in-time (FET) (and finite-element-in-space) interpretation. The benefits of this reformulation are discussed, including the derivation of timesteppers that are higher order in time, and the quantifiable dissipative SP properties in the non-ideal, resistive case.
        
        We discuss possible options for extending this FET approach to timesteppers for the compressible case.

        The kinetic corrections satisfy linearized Boltzmann equations. Using a Lénard--Bernstein collision operator, these take Fokker--Planck-like forms \cite{Fokker_1914, Planck_1917} wherein pseudo-particles in the numerical model obey the neoclassical transport equations, with particle-independent Brownian drift terms. This offers a rigorous methodology for incorporating collisions into the particle transport model, without coupling the equations of motions for each particle.
        
        Works by Chen, Chacón et al. \cite{Chen_Chacón_Barnes_2011, Chacón_Chen_Barnes_2013, Chen_Chacón_2014, Chen_Chacón_2015} have developed structure-preserving particle pushers for neoclassical transport in the Vlasov equations, derived from Crank--Nicolson integrators. We show these too can can derive from a FET interpretation, similarly offering potential extensions to higher-order-in-time particle pushers. The FET formulation is used also to consider how the stochastic drift terms can be incorporated into the pushers. Stochastic gyrokinetic expansions are also discussed.

        Different options for the numerical implementation of these schemes are considered.

        Due to the efficacy of FET in the development of SP timesteppers for both the fluid and kinetic component, we hope this approach will prove effective in the future for developing SP timesteppers for the full hybrid model. We hope this will give us the opportunity to incorporate previously inaccessible kinetic effects into the highly effective, modern, finite-element MHD models.
    \end{abstract}
    
    
    \newpage
    \tableofcontents
    
    
    \newpage
    \pagenumbering{arabic}
    %\linenumbers\renewcommand\thelinenumber{\color{black!50}\arabic{linenumber}}
            \input{0 - introduction/main.tex}
        \part{Research}
            \input{1 - low-noise PiC models/main.tex}
            \input{2 - kinetic component/main.tex}
            \input{3 - fluid component/main.tex}
            \input{4 - numerical implementation/main.tex}
        \part{Project Overview}
            \input{5 - research plan/main.tex}
            \input{6 - summary/main.tex}
    
    
    %\section{}
    \newpage
    \pagenumbering{gobble}
        \printbibliography


    \newpage
    \pagenumbering{roman}
    \appendix
        \part{Appendices}
            \input{8 - Hilbert complexes/main.tex}
            \input{9 - weak conservation proofs/main.tex}
\end{document}


\title{\BA{Title in Progress...}}
\author{Boris Andrews}
\affil{Mathematical Institute, University of Oxford}
\date{\today}


\begin{document}
    \pagenumbering{gobble}
    \maketitle
    
    
    \begin{abstract}
        Magnetic confinement reactors---in particular tokamaks---offer one of the most promising options for achieving practical nuclear fusion, with the potential to provide virtually limitless, clean energy. The theoretical and numerical modeling of tokamak plasmas is simultaneously an essential component of effective reactor design, and a great research barrier. Tokamak operational conditions exhibit comparatively low Knudsen numbers. Kinetic effects, including kinetic waves and instabilities, Landau damping, bump-on-tail instabilities and more, are therefore highly influential in tokamak plasma dynamics. Purely fluid models are inherently incapable of capturing these effects, whereas the high dimensionality in purely kinetic models render them practically intractable for most relevant purposes.

        We consider a $\delta\!f$ decomposition model, with a macroscopic fluid background and microscopic kinetic correction, both fully coupled to each other. A similar manner of discretization is proposed to that used in the recent \texttt{STRUPHY} code \cite{Holderied_Possanner_Wang_2021, Holderied_2022, Li_et_al_2023} with a finite-element model for the background and a pseudo-particle/PiC model for the correction.

        The fluid background satisfies the full, non-linear, resistive, compressible, Hall MHD equations. \cite{Laakmann_Hu_Farrell_2022} introduces finite-element(-in-space) implicit timesteppers for the incompressible analogue to this system with structure-preserving (SP) properties in the ideal case, alongside parameter-robust preconditioners. We show that these timesteppers can derive from a finite-element-in-time (FET) (and finite-element-in-space) interpretation. The benefits of this reformulation are discussed, including the derivation of timesteppers that are higher order in time, and the quantifiable dissipative SP properties in the non-ideal, resistive case.
        
        We discuss possible options for extending this FET approach to timesteppers for the compressible case.

        The kinetic corrections satisfy linearized Boltzmann equations. Using a Lénard--Bernstein collision operator, these take Fokker--Planck-like forms \cite{Fokker_1914, Planck_1917} wherein pseudo-particles in the numerical model obey the neoclassical transport equations, with particle-independent Brownian drift terms. This offers a rigorous methodology for incorporating collisions into the particle transport model, without coupling the equations of motions for each particle.
        
        Works by Chen, Chacón et al. \cite{Chen_Chacón_Barnes_2011, Chacón_Chen_Barnes_2013, Chen_Chacón_2014, Chen_Chacón_2015} have developed structure-preserving particle pushers for neoclassical transport in the Vlasov equations, derived from Crank--Nicolson integrators. We show these too can can derive from a FET interpretation, similarly offering potential extensions to higher-order-in-time particle pushers. The FET formulation is used also to consider how the stochastic drift terms can be incorporated into the pushers. Stochastic gyrokinetic expansions are also discussed.

        Different options for the numerical implementation of these schemes are considered.

        Due to the efficacy of FET in the development of SP timesteppers for both the fluid and kinetic component, we hope this approach will prove effective in the future for developing SP timesteppers for the full hybrid model. We hope this will give us the opportunity to incorporate previously inaccessible kinetic effects into the highly effective, modern, finite-element MHD models.
    \end{abstract}
    
    
    \newpage
    \tableofcontents
    
    
    \newpage
    \pagenumbering{arabic}
    %\linenumbers\renewcommand\thelinenumber{\color{black!50}\arabic{linenumber}}
            \documentclass[12pt, a4paper]{report}

\input{template/main.tex}

\title{\BA{Title in Progress...}}
\author{Boris Andrews}
\affil{Mathematical Institute, University of Oxford}
\date{\today}


\begin{document}
    \pagenumbering{gobble}
    \maketitle
    
    
    \begin{abstract}
        Magnetic confinement reactors---in particular tokamaks---offer one of the most promising options for achieving practical nuclear fusion, with the potential to provide virtually limitless, clean energy. The theoretical and numerical modeling of tokamak plasmas is simultaneously an essential component of effective reactor design, and a great research barrier. Tokamak operational conditions exhibit comparatively low Knudsen numbers. Kinetic effects, including kinetic waves and instabilities, Landau damping, bump-on-tail instabilities and more, are therefore highly influential in tokamak plasma dynamics. Purely fluid models are inherently incapable of capturing these effects, whereas the high dimensionality in purely kinetic models render them practically intractable for most relevant purposes.

        We consider a $\delta\!f$ decomposition model, with a macroscopic fluid background and microscopic kinetic correction, both fully coupled to each other. A similar manner of discretization is proposed to that used in the recent \texttt{STRUPHY} code \cite{Holderied_Possanner_Wang_2021, Holderied_2022, Li_et_al_2023} with a finite-element model for the background and a pseudo-particle/PiC model for the correction.

        The fluid background satisfies the full, non-linear, resistive, compressible, Hall MHD equations. \cite{Laakmann_Hu_Farrell_2022} introduces finite-element(-in-space) implicit timesteppers for the incompressible analogue to this system with structure-preserving (SP) properties in the ideal case, alongside parameter-robust preconditioners. We show that these timesteppers can derive from a finite-element-in-time (FET) (and finite-element-in-space) interpretation. The benefits of this reformulation are discussed, including the derivation of timesteppers that are higher order in time, and the quantifiable dissipative SP properties in the non-ideal, resistive case.
        
        We discuss possible options for extending this FET approach to timesteppers for the compressible case.

        The kinetic corrections satisfy linearized Boltzmann equations. Using a Lénard--Bernstein collision operator, these take Fokker--Planck-like forms \cite{Fokker_1914, Planck_1917} wherein pseudo-particles in the numerical model obey the neoclassical transport equations, with particle-independent Brownian drift terms. This offers a rigorous methodology for incorporating collisions into the particle transport model, without coupling the equations of motions for each particle.
        
        Works by Chen, Chacón et al. \cite{Chen_Chacón_Barnes_2011, Chacón_Chen_Barnes_2013, Chen_Chacón_2014, Chen_Chacón_2015} have developed structure-preserving particle pushers for neoclassical transport in the Vlasov equations, derived from Crank--Nicolson integrators. We show these too can can derive from a FET interpretation, similarly offering potential extensions to higher-order-in-time particle pushers. The FET formulation is used also to consider how the stochastic drift terms can be incorporated into the pushers. Stochastic gyrokinetic expansions are also discussed.

        Different options for the numerical implementation of these schemes are considered.

        Due to the efficacy of FET in the development of SP timesteppers for both the fluid and kinetic component, we hope this approach will prove effective in the future for developing SP timesteppers for the full hybrid model. We hope this will give us the opportunity to incorporate previously inaccessible kinetic effects into the highly effective, modern, finite-element MHD models.
    \end{abstract}
    
    
    \newpage
    \tableofcontents
    
    
    \newpage
    \pagenumbering{arabic}
    %\linenumbers\renewcommand\thelinenumber{\color{black!50}\arabic{linenumber}}
            \input{0 - introduction/main.tex}
        \part{Research}
            \input{1 - low-noise PiC models/main.tex}
            \input{2 - kinetic component/main.tex}
            \input{3 - fluid component/main.tex}
            \input{4 - numerical implementation/main.tex}
        \part{Project Overview}
            \input{5 - research plan/main.tex}
            \input{6 - summary/main.tex}
    
    
    %\section{}
    \newpage
    \pagenumbering{gobble}
        \printbibliography


    \newpage
    \pagenumbering{roman}
    \appendix
        \part{Appendices}
            \input{8 - Hilbert complexes/main.tex}
            \input{9 - weak conservation proofs/main.tex}
\end{document}

        \part{Research}
            \documentclass[12pt, a4paper]{report}

\input{template/main.tex}

\title{\BA{Title in Progress...}}
\author{Boris Andrews}
\affil{Mathematical Institute, University of Oxford}
\date{\today}


\begin{document}
    \pagenumbering{gobble}
    \maketitle
    
    
    \begin{abstract}
        Magnetic confinement reactors---in particular tokamaks---offer one of the most promising options for achieving practical nuclear fusion, with the potential to provide virtually limitless, clean energy. The theoretical and numerical modeling of tokamak plasmas is simultaneously an essential component of effective reactor design, and a great research barrier. Tokamak operational conditions exhibit comparatively low Knudsen numbers. Kinetic effects, including kinetic waves and instabilities, Landau damping, bump-on-tail instabilities and more, are therefore highly influential in tokamak plasma dynamics. Purely fluid models are inherently incapable of capturing these effects, whereas the high dimensionality in purely kinetic models render them practically intractable for most relevant purposes.

        We consider a $\delta\!f$ decomposition model, with a macroscopic fluid background and microscopic kinetic correction, both fully coupled to each other. A similar manner of discretization is proposed to that used in the recent \texttt{STRUPHY} code \cite{Holderied_Possanner_Wang_2021, Holderied_2022, Li_et_al_2023} with a finite-element model for the background and a pseudo-particle/PiC model for the correction.

        The fluid background satisfies the full, non-linear, resistive, compressible, Hall MHD equations. \cite{Laakmann_Hu_Farrell_2022} introduces finite-element(-in-space) implicit timesteppers for the incompressible analogue to this system with structure-preserving (SP) properties in the ideal case, alongside parameter-robust preconditioners. We show that these timesteppers can derive from a finite-element-in-time (FET) (and finite-element-in-space) interpretation. The benefits of this reformulation are discussed, including the derivation of timesteppers that are higher order in time, and the quantifiable dissipative SP properties in the non-ideal, resistive case.
        
        We discuss possible options for extending this FET approach to timesteppers for the compressible case.

        The kinetic corrections satisfy linearized Boltzmann equations. Using a Lénard--Bernstein collision operator, these take Fokker--Planck-like forms \cite{Fokker_1914, Planck_1917} wherein pseudo-particles in the numerical model obey the neoclassical transport equations, with particle-independent Brownian drift terms. This offers a rigorous methodology for incorporating collisions into the particle transport model, without coupling the equations of motions for each particle.
        
        Works by Chen, Chacón et al. \cite{Chen_Chacón_Barnes_2011, Chacón_Chen_Barnes_2013, Chen_Chacón_2014, Chen_Chacón_2015} have developed structure-preserving particle pushers for neoclassical transport in the Vlasov equations, derived from Crank--Nicolson integrators. We show these too can can derive from a FET interpretation, similarly offering potential extensions to higher-order-in-time particle pushers. The FET formulation is used also to consider how the stochastic drift terms can be incorporated into the pushers. Stochastic gyrokinetic expansions are also discussed.

        Different options for the numerical implementation of these schemes are considered.

        Due to the efficacy of FET in the development of SP timesteppers for both the fluid and kinetic component, we hope this approach will prove effective in the future for developing SP timesteppers for the full hybrid model. We hope this will give us the opportunity to incorporate previously inaccessible kinetic effects into the highly effective, modern, finite-element MHD models.
    \end{abstract}
    
    
    \newpage
    \tableofcontents
    
    
    \newpage
    \pagenumbering{arabic}
    %\linenumbers\renewcommand\thelinenumber{\color{black!50}\arabic{linenumber}}
            \input{0 - introduction/main.tex}
        \part{Research}
            \input{1 - low-noise PiC models/main.tex}
            \input{2 - kinetic component/main.tex}
            \input{3 - fluid component/main.tex}
            \input{4 - numerical implementation/main.tex}
        \part{Project Overview}
            \input{5 - research plan/main.tex}
            \input{6 - summary/main.tex}
    
    
    %\section{}
    \newpage
    \pagenumbering{gobble}
        \printbibliography


    \newpage
    \pagenumbering{roman}
    \appendix
        \part{Appendices}
            \input{8 - Hilbert complexes/main.tex}
            \input{9 - weak conservation proofs/main.tex}
\end{document}

            \documentclass[12pt, a4paper]{report}

\input{template/main.tex}

\title{\BA{Title in Progress...}}
\author{Boris Andrews}
\affil{Mathematical Institute, University of Oxford}
\date{\today}


\begin{document}
    \pagenumbering{gobble}
    \maketitle
    
    
    \begin{abstract}
        Magnetic confinement reactors---in particular tokamaks---offer one of the most promising options for achieving practical nuclear fusion, with the potential to provide virtually limitless, clean energy. The theoretical and numerical modeling of tokamak plasmas is simultaneously an essential component of effective reactor design, and a great research barrier. Tokamak operational conditions exhibit comparatively low Knudsen numbers. Kinetic effects, including kinetic waves and instabilities, Landau damping, bump-on-tail instabilities and more, are therefore highly influential in tokamak plasma dynamics. Purely fluid models are inherently incapable of capturing these effects, whereas the high dimensionality in purely kinetic models render them practically intractable for most relevant purposes.

        We consider a $\delta\!f$ decomposition model, with a macroscopic fluid background and microscopic kinetic correction, both fully coupled to each other. A similar manner of discretization is proposed to that used in the recent \texttt{STRUPHY} code \cite{Holderied_Possanner_Wang_2021, Holderied_2022, Li_et_al_2023} with a finite-element model for the background and a pseudo-particle/PiC model for the correction.

        The fluid background satisfies the full, non-linear, resistive, compressible, Hall MHD equations. \cite{Laakmann_Hu_Farrell_2022} introduces finite-element(-in-space) implicit timesteppers for the incompressible analogue to this system with structure-preserving (SP) properties in the ideal case, alongside parameter-robust preconditioners. We show that these timesteppers can derive from a finite-element-in-time (FET) (and finite-element-in-space) interpretation. The benefits of this reformulation are discussed, including the derivation of timesteppers that are higher order in time, and the quantifiable dissipative SP properties in the non-ideal, resistive case.
        
        We discuss possible options for extending this FET approach to timesteppers for the compressible case.

        The kinetic corrections satisfy linearized Boltzmann equations. Using a Lénard--Bernstein collision operator, these take Fokker--Planck-like forms \cite{Fokker_1914, Planck_1917} wherein pseudo-particles in the numerical model obey the neoclassical transport equations, with particle-independent Brownian drift terms. This offers a rigorous methodology for incorporating collisions into the particle transport model, without coupling the equations of motions for each particle.
        
        Works by Chen, Chacón et al. \cite{Chen_Chacón_Barnes_2011, Chacón_Chen_Barnes_2013, Chen_Chacón_2014, Chen_Chacón_2015} have developed structure-preserving particle pushers for neoclassical transport in the Vlasov equations, derived from Crank--Nicolson integrators. We show these too can can derive from a FET interpretation, similarly offering potential extensions to higher-order-in-time particle pushers. The FET formulation is used also to consider how the stochastic drift terms can be incorporated into the pushers. Stochastic gyrokinetic expansions are also discussed.

        Different options for the numerical implementation of these schemes are considered.

        Due to the efficacy of FET in the development of SP timesteppers for both the fluid and kinetic component, we hope this approach will prove effective in the future for developing SP timesteppers for the full hybrid model. We hope this will give us the opportunity to incorporate previously inaccessible kinetic effects into the highly effective, modern, finite-element MHD models.
    \end{abstract}
    
    
    \newpage
    \tableofcontents
    
    
    \newpage
    \pagenumbering{arabic}
    %\linenumbers\renewcommand\thelinenumber{\color{black!50}\arabic{linenumber}}
            \input{0 - introduction/main.tex}
        \part{Research}
            \input{1 - low-noise PiC models/main.tex}
            \input{2 - kinetic component/main.tex}
            \input{3 - fluid component/main.tex}
            \input{4 - numerical implementation/main.tex}
        \part{Project Overview}
            \input{5 - research plan/main.tex}
            \input{6 - summary/main.tex}
    
    
    %\section{}
    \newpage
    \pagenumbering{gobble}
        \printbibliography


    \newpage
    \pagenumbering{roman}
    \appendix
        \part{Appendices}
            \input{8 - Hilbert complexes/main.tex}
            \input{9 - weak conservation proofs/main.tex}
\end{document}

            \documentclass[12pt, a4paper]{report}

\input{template/main.tex}

\title{\BA{Title in Progress...}}
\author{Boris Andrews}
\affil{Mathematical Institute, University of Oxford}
\date{\today}


\begin{document}
    \pagenumbering{gobble}
    \maketitle
    
    
    \begin{abstract}
        Magnetic confinement reactors---in particular tokamaks---offer one of the most promising options for achieving practical nuclear fusion, with the potential to provide virtually limitless, clean energy. The theoretical and numerical modeling of tokamak plasmas is simultaneously an essential component of effective reactor design, and a great research barrier. Tokamak operational conditions exhibit comparatively low Knudsen numbers. Kinetic effects, including kinetic waves and instabilities, Landau damping, bump-on-tail instabilities and more, are therefore highly influential in tokamak plasma dynamics. Purely fluid models are inherently incapable of capturing these effects, whereas the high dimensionality in purely kinetic models render them practically intractable for most relevant purposes.

        We consider a $\delta\!f$ decomposition model, with a macroscopic fluid background and microscopic kinetic correction, both fully coupled to each other. A similar manner of discretization is proposed to that used in the recent \texttt{STRUPHY} code \cite{Holderied_Possanner_Wang_2021, Holderied_2022, Li_et_al_2023} with a finite-element model for the background and a pseudo-particle/PiC model for the correction.

        The fluid background satisfies the full, non-linear, resistive, compressible, Hall MHD equations. \cite{Laakmann_Hu_Farrell_2022} introduces finite-element(-in-space) implicit timesteppers for the incompressible analogue to this system with structure-preserving (SP) properties in the ideal case, alongside parameter-robust preconditioners. We show that these timesteppers can derive from a finite-element-in-time (FET) (and finite-element-in-space) interpretation. The benefits of this reformulation are discussed, including the derivation of timesteppers that are higher order in time, and the quantifiable dissipative SP properties in the non-ideal, resistive case.
        
        We discuss possible options for extending this FET approach to timesteppers for the compressible case.

        The kinetic corrections satisfy linearized Boltzmann equations. Using a Lénard--Bernstein collision operator, these take Fokker--Planck-like forms \cite{Fokker_1914, Planck_1917} wherein pseudo-particles in the numerical model obey the neoclassical transport equations, with particle-independent Brownian drift terms. This offers a rigorous methodology for incorporating collisions into the particle transport model, without coupling the equations of motions for each particle.
        
        Works by Chen, Chacón et al. \cite{Chen_Chacón_Barnes_2011, Chacón_Chen_Barnes_2013, Chen_Chacón_2014, Chen_Chacón_2015} have developed structure-preserving particle pushers for neoclassical transport in the Vlasov equations, derived from Crank--Nicolson integrators. We show these too can can derive from a FET interpretation, similarly offering potential extensions to higher-order-in-time particle pushers. The FET formulation is used also to consider how the stochastic drift terms can be incorporated into the pushers. Stochastic gyrokinetic expansions are also discussed.

        Different options for the numerical implementation of these schemes are considered.

        Due to the efficacy of FET in the development of SP timesteppers for both the fluid and kinetic component, we hope this approach will prove effective in the future for developing SP timesteppers for the full hybrid model. We hope this will give us the opportunity to incorporate previously inaccessible kinetic effects into the highly effective, modern, finite-element MHD models.
    \end{abstract}
    
    
    \newpage
    \tableofcontents
    
    
    \newpage
    \pagenumbering{arabic}
    %\linenumbers\renewcommand\thelinenumber{\color{black!50}\arabic{linenumber}}
            \input{0 - introduction/main.tex}
        \part{Research}
            \input{1 - low-noise PiC models/main.tex}
            \input{2 - kinetic component/main.tex}
            \input{3 - fluid component/main.tex}
            \input{4 - numerical implementation/main.tex}
        \part{Project Overview}
            \input{5 - research plan/main.tex}
            \input{6 - summary/main.tex}
    
    
    %\section{}
    \newpage
    \pagenumbering{gobble}
        \printbibliography


    \newpage
    \pagenumbering{roman}
    \appendix
        \part{Appendices}
            \input{8 - Hilbert complexes/main.tex}
            \input{9 - weak conservation proofs/main.tex}
\end{document}

            \documentclass[12pt, a4paper]{report}

\input{template/main.tex}

\title{\BA{Title in Progress...}}
\author{Boris Andrews}
\affil{Mathematical Institute, University of Oxford}
\date{\today}


\begin{document}
    \pagenumbering{gobble}
    \maketitle
    
    
    \begin{abstract}
        Magnetic confinement reactors---in particular tokamaks---offer one of the most promising options for achieving practical nuclear fusion, with the potential to provide virtually limitless, clean energy. The theoretical and numerical modeling of tokamak plasmas is simultaneously an essential component of effective reactor design, and a great research barrier. Tokamak operational conditions exhibit comparatively low Knudsen numbers. Kinetic effects, including kinetic waves and instabilities, Landau damping, bump-on-tail instabilities and more, are therefore highly influential in tokamak plasma dynamics. Purely fluid models are inherently incapable of capturing these effects, whereas the high dimensionality in purely kinetic models render them practically intractable for most relevant purposes.

        We consider a $\delta\!f$ decomposition model, with a macroscopic fluid background and microscopic kinetic correction, both fully coupled to each other. A similar manner of discretization is proposed to that used in the recent \texttt{STRUPHY} code \cite{Holderied_Possanner_Wang_2021, Holderied_2022, Li_et_al_2023} with a finite-element model for the background and a pseudo-particle/PiC model for the correction.

        The fluid background satisfies the full, non-linear, resistive, compressible, Hall MHD equations. \cite{Laakmann_Hu_Farrell_2022} introduces finite-element(-in-space) implicit timesteppers for the incompressible analogue to this system with structure-preserving (SP) properties in the ideal case, alongside parameter-robust preconditioners. We show that these timesteppers can derive from a finite-element-in-time (FET) (and finite-element-in-space) interpretation. The benefits of this reformulation are discussed, including the derivation of timesteppers that are higher order in time, and the quantifiable dissipative SP properties in the non-ideal, resistive case.
        
        We discuss possible options for extending this FET approach to timesteppers for the compressible case.

        The kinetic corrections satisfy linearized Boltzmann equations. Using a Lénard--Bernstein collision operator, these take Fokker--Planck-like forms \cite{Fokker_1914, Planck_1917} wherein pseudo-particles in the numerical model obey the neoclassical transport equations, with particle-independent Brownian drift terms. This offers a rigorous methodology for incorporating collisions into the particle transport model, without coupling the equations of motions for each particle.
        
        Works by Chen, Chacón et al. \cite{Chen_Chacón_Barnes_2011, Chacón_Chen_Barnes_2013, Chen_Chacón_2014, Chen_Chacón_2015} have developed structure-preserving particle pushers for neoclassical transport in the Vlasov equations, derived from Crank--Nicolson integrators. We show these too can can derive from a FET interpretation, similarly offering potential extensions to higher-order-in-time particle pushers. The FET formulation is used also to consider how the stochastic drift terms can be incorporated into the pushers. Stochastic gyrokinetic expansions are also discussed.

        Different options for the numerical implementation of these schemes are considered.

        Due to the efficacy of FET in the development of SP timesteppers for both the fluid and kinetic component, we hope this approach will prove effective in the future for developing SP timesteppers for the full hybrid model. We hope this will give us the opportunity to incorporate previously inaccessible kinetic effects into the highly effective, modern, finite-element MHD models.
    \end{abstract}
    
    
    \newpage
    \tableofcontents
    
    
    \newpage
    \pagenumbering{arabic}
    %\linenumbers\renewcommand\thelinenumber{\color{black!50}\arabic{linenumber}}
            \input{0 - introduction/main.tex}
        \part{Research}
            \input{1 - low-noise PiC models/main.tex}
            \input{2 - kinetic component/main.tex}
            \input{3 - fluid component/main.tex}
            \input{4 - numerical implementation/main.tex}
        \part{Project Overview}
            \input{5 - research plan/main.tex}
            \input{6 - summary/main.tex}
    
    
    %\section{}
    \newpage
    \pagenumbering{gobble}
        \printbibliography


    \newpage
    \pagenumbering{roman}
    \appendix
        \part{Appendices}
            \input{8 - Hilbert complexes/main.tex}
            \input{9 - weak conservation proofs/main.tex}
\end{document}

        \part{Project Overview}
            \documentclass[12pt, a4paper]{report}

\input{template/main.tex}

\title{\BA{Title in Progress...}}
\author{Boris Andrews}
\affil{Mathematical Institute, University of Oxford}
\date{\today}


\begin{document}
    \pagenumbering{gobble}
    \maketitle
    
    
    \begin{abstract}
        Magnetic confinement reactors---in particular tokamaks---offer one of the most promising options for achieving practical nuclear fusion, with the potential to provide virtually limitless, clean energy. The theoretical and numerical modeling of tokamak plasmas is simultaneously an essential component of effective reactor design, and a great research barrier. Tokamak operational conditions exhibit comparatively low Knudsen numbers. Kinetic effects, including kinetic waves and instabilities, Landau damping, bump-on-tail instabilities and more, are therefore highly influential in tokamak plasma dynamics. Purely fluid models are inherently incapable of capturing these effects, whereas the high dimensionality in purely kinetic models render them practically intractable for most relevant purposes.

        We consider a $\delta\!f$ decomposition model, with a macroscopic fluid background and microscopic kinetic correction, both fully coupled to each other. A similar manner of discretization is proposed to that used in the recent \texttt{STRUPHY} code \cite{Holderied_Possanner_Wang_2021, Holderied_2022, Li_et_al_2023} with a finite-element model for the background and a pseudo-particle/PiC model for the correction.

        The fluid background satisfies the full, non-linear, resistive, compressible, Hall MHD equations. \cite{Laakmann_Hu_Farrell_2022} introduces finite-element(-in-space) implicit timesteppers for the incompressible analogue to this system with structure-preserving (SP) properties in the ideal case, alongside parameter-robust preconditioners. We show that these timesteppers can derive from a finite-element-in-time (FET) (and finite-element-in-space) interpretation. The benefits of this reformulation are discussed, including the derivation of timesteppers that are higher order in time, and the quantifiable dissipative SP properties in the non-ideal, resistive case.
        
        We discuss possible options for extending this FET approach to timesteppers for the compressible case.

        The kinetic corrections satisfy linearized Boltzmann equations. Using a Lénard--Bernstein collision operator, these take Fokker--Planck-like forms \cite{Fokker_1914, Planck_1917} wherein pseudo-particles in the numerical model obey the neoclassical transport equations, with particle-independent Brownian drift terms. This offers a rigorous methodology for incorporating collisions into the particle transport model, without coupling the equations of motions for each particle.
        
        Works by Chen, Chacón et al. \cite{Chen_Chacón_Barnes_2011, Chacón_Chen_Barnes_2013, Chen_Chacón_2014, Chen_Chacón_2015} have developed structure-preserving particle pushers for neoclassical transport in the Vlasov equations, derived from Crank--Nicolson integrators. We show these too can can derive from a FET interpretation, similarly offering potential extensions to higher-order-in-time particle pushers. The FET formulation is used also to consider how the stochastic drift terms can be incorporated into the pushers. Stochastic gyrokinetic expansions are also discussed.

        Different options for the numerical implementation of these schemes are considered.

        Due to the efficacy of FET in the development of SP timesteppers for both the fluid and kinetic component, we hope this approach will prove effective in the future for developing SP timesteppers for the full hybrid model. We hope this will give us the opportunity to incorporate previously inaccessible kinetic effects into the highly effective, modern, finite-element MHD models.
    \end{abstract}
    
    
    \newpage
    \tableofcontents
    
    
    \newpage
    \pagenumbering{arabic}
    %\linenumbers\renewcommand\thelinenumber{\color{black!50}\arabic{linenumber}}
            \input{0 - introduction/main.tex}
        \part{Research}
            \input{1 - low-noise PiC models/main.tex}
            \input{2 - kinetic component/main.tex}
            \input{3 - fluid component/main.tex}
            \input{4 - numerical implementation/main.tex}
        \part{Project Overview}
            \input{5 - research plan/main.tex}
            \input{6 - summary/main.tex}
    
    
    %\section{}
    \newpage
    \pagenumbering{gobble}
        \printbibliography


    \newpage
    \pagenumbering{roman}
    \appendix
        \part{Appendices}
            \input{8 - Hilbert complexes/main.tex}
            \input{9 - weak conservation proofs/main.tex}
\end{document}

            \documentclass[12pt, a4paper]{report}

\input{template/main.tex}

\title{\BA{Title in Progress...}}
\author{Boris Andrews}
\affil{Mathematical Institute, University of Oxford}
\date{\today}


\begin{document}
    \pagenumbering{gobble}
    \maketitle
    
    
    \begin{abstract}
        Magnetic confinement reactors---in particular tokamaks---offer one of the most promising options for achieving practical nuclear fusion, with the potential to provide virtually limitless, clean energy. The theoretical and numerical modeling of tokamak plasmas is simultaneously an essential component of effective reactor design, and a great research barrier. Tokamak operational conditions exhibit comparatively low Knudsen numbers. Kinetic effects, including kinetic waves and instabilities, Landau damping, bump-on-tail instabilities and more, are therefore highly influential in tokamak plasma dynamics. Purely fluid models are inherently incapable of capturing these effects, whereas the high dimensionality in purely kinetic models render them practically intractable for most relevant purposes.

        We consider a $\delta\!f$ decomposition model, with a macroscopic fluid background and microscopic kinetic correction, both fully coupled to each other. A similar manner of discretization is proposed to that used in the recent \texttt{STRUPHY} code \cite{Holderied_Possanner_Wang_2021, Holderied_2022, Li_et_al_2023} with a finite-element model for the background and a pseudo-particle/PiC model for the correction.

        The fluid background satisfies the full, non-linear, resistive, compressible, Hall MHD equations. \cite{Laakmann_Hu_Farrell_2022} introduces finite-element(-in-space) implicit timesteppers for the incompressible analogue to this system with structure-preserving (SP) properties in the ideal case, alongside parameter-robust preconditioners. We show that these timesteppers can derive from a finite-element-in-time (FET) (and finite-element-in-space) interpretation. The benefits of this reformulation are discussed, including the derivation of timesteppers that are higher order in time, and the quantifiable dissipative SP properties in the non-ideal, resistive case.
        
        We discuss possible options for extending this FET approach to timesteppers for the compressible case.

        The kinetic corrections satisfy linearized Boltzmann equations. Using a Lénard--Bernstein collision operator, these take Fokker--Planck-like forms \cite{Fokker_1914, Planck_1917} wherein pseudo-particles in the numerical model obey the neoclassical transport equations, with particle-independent Brownian drift terms. This offers a rigorous methodology for incorporating collisions into the particle transport model, without coupling the equations of motions for each particle.
        
        Works by Chen, Chacón et al. \cite{Chen_Chacón_Barnes_2011, Chacón_Chen_Barnes_2013, Chen_Chacón_2014, Chen_Chacón_2015} have developed structure-preserving particle pushers for neoclassical transport in the Vlasov equations, derived from Crank--Nicolson integrators. We show these too can can derive from a FET interpretation, similarly offering potential extensions to higher-order-in-time particle pushers. The FET formulation is used also to consider how the stochastic drift terms can be incorporated into the pushers. Stochastic gyrokinetic expansions are also discussed.

        Different options for the numerical implementation of these schemes are considered.

        Due to the efficacy of FET in the development of SP timesteppers for both the fluid and kinetic component, we hope this approach will prove effective in the future for developing SP timesteppers for the full hybrid model. We hope this will give us the opportunity to incorporate previously inaccessible kinetic effects into the highly effective, modern, finite-element MHD models.
    \end{abstract}
    
    
    \newpage
    \tableofcontents
    
    
    \newpage
    \pagenumbering{arabic}
    %\linenumbers\renewcommand\thelinenumber{\color{black!50}\arabic{linenumber}}
            \input{0 - introduction/main.tex}
        \part{Research}
            \input{1 - low-noise PiC models/main.tex}
            \input{2 - kinetic component/main.tex}
            \input{3 - fluid component/main.tex}
            \input{4 - numerical implementation/main.tex}
        \part{Project Overview}
            \input{5 - research plan/main.tex}
            \input{6 - summary/main.tex}
    
    
    %\section{}
    \newpage
    \pagenumbering{gobble}
        \printbibliography


    \newpage
    \pagenumbering{roman}
    \appendix
        \part{Appendices}
            \input{8 - Hilbert complexes/main.tex}
            \input{9 - weak conservation proofs/main.tex}
\end{document}

    
    
    %\section{}
    \newpage
    \pagenumbering{gobble}
        \printbibliography


    \newpage
    \pagenumbering{roman}
    \appendix
        \part{Appendices}
            \documentclass[12pt, a4paper]{report}

\input{template/main.tex}

\title{\BA{Title in Progress...}}
\author{Boris Andrews}
\affil{Mathematical Institute, University of Oxford}
\date{\today}


\begin{document}
    \pagenumbering{gobble}
    \maketitle
    
    
    \begin{abstract}
        Magnetic confinement reactors---in particular tokamaks---offer one of the most promising options for achieving practical nuclear fusion, with the potential to provide virtually limitless, clean energy. The theoretical and numerical modeling of tokamak plasmas is simultaneously an essential component of effective reactor design, and a great research barrier. Tokamak operational conditions exhibit comparatively low Knudsen numbers. Kinetic effects, including kinetic waves and instabilities, Landau damping, bump-on-tail instabilities and more, are therefore highly influential in tokamak plasma dynamics. Purely fluid models are inherently incapable of capturing these effects, whereas the high dimensionality in purely kinetic models render them practically intractable for most relevant purposes.

        We consider a $\delta\!f$ decomposition model, with a macroscopic fluid background and microscopic kinetic correction, both fully coupled to each other. A similar manner of discretization is proposed to that used in the recent \texttt{STRUPHY} code \cite{Holderied_Possanner_Wang_2021, Holderied_2022, Li_et_al_2023} with a finite-element model for the background and a pseudo-particle/PiC model for the correction.

        The fluid background satisfies the full, non-linear, resistive, compressible, Hall MHD equations. \cite{Laakmann_Hu_Farrell_2022} introduces finite-element(-in-space) implicit timesteppers for the incompressible analogue to this system with structure-preserving (SP) properties in the ideal case, alongside parameter-robust preconditioners. We show that these timesteppers can derive from a finite-element-in-time (FET) (and finite-element-in-space) interpretation. The benefits of this reformulation are discussed, including the derivation of timesteppers that are higher order in time, and the quantifiable dissipative SP properties in the non-ideal, resistive case.
        
        We discuss possible options for extending this FET approach to timesteppers for the compressible case.

        The kinetic corrections satisfy linearized Boltzmann equations. Using a Lénard--Bernstein collision operator, these take Fokker--Planck-like forms \cite{Fokker_1914, Planck_1917} wherein pseudo-particles in the numerical model obey the neoclassical transport equations, with particle-independent Brownian drift terms. This offers a rigorous methodology for incorporating collisions into the particle transport model, without coupling the equations of motions for each particle.
        
        Works by Chen, Chacón et al. \cite{Chen_Chacón_Barnes_2011, Chacón_Chen_Barnes_2013, Chen_Chacón_2014, Chen_Chacón_2015} have developed structure-preserving particle pushers for neoclassical transport in the Vlasov equations, derived from Crank--Nicolson integrators. We show these too can can derive from a FET interpretation, similarly offering potential extensions to higher-order-in-time particle pushers. The FET formulation is used also to consider how the stochastic drift terms can be incorporated into the pushers. Stochastic gyrokinetic expansions are also discussed.

        Different options for the numerical implementation of these schemes are considered.

        Due to the efficacy of FET in the development of SP timesteppers for both the fluid and kinetic component, we hope this approach will prove effective in the future for developing SP timesteppers for the full hybrid model. We hope this will give us the opportunity to incorporate previously inaccessible kinetic effects into the highly effective, modern, finite-element MHD models.
    \end{abstract}
    
    
    \newpage
    \tableofcontents
    
    
    \newpage
    \pagenumbering{arabic}
    %\linenumbers\renewcommand\thelinenumber{\color{black!50}\arabic{linenumber}}
            \input{0 - introduction/main.tex}
        \part{Research}
            \input{1 - low-noise PiC models/main.tex}
            \input{2 - kinetic component/main.tex}
            \input{3 - fluid component/main.tex}
            \input{4 - numerical implementation/main.tex}
        \part{Project Overview}
            \input{5 - research plan/main.tex}
            \input{6 - summary/main.tex}
    
    
    %\section{}
    \newpage
    \pagenumbering{gobble}
        \printbibliography


    \newpage
    \pagenumbering{roman}
    \appendix
        \part{Appendices}
            \input{8 - Hilbert complexes/main.tex}
            \input{9 - weak conservation proofs/main.tex}
\end{document}

            \documentclass[12pt, a4paper]{report}

\input{template/main.tex}

\title{\BA{Title in Progress...}}
\author{Boris Andrews}
\affil{Mathematical Institute, University of Oxford}
\date{\today}


\begin{document}
    \pagenumbering{gobble}
    \maketitle
    
    
    \begin{abstract}
        Magnetic confinement reactors---in particular tokamaks---offer one of the most promising options for achieving practical nuclear fusion, with the potential to provide virtually limitless, clean energy. The theoretical and numerical modeling of tokamak plasmas is simultaneously an essential component of effective reactor design, and a great research barrier. Tokamak operational conditions exhibit comparatively low Knudsen numbers. Kinetic effects, including kinetic waves and instabilities, Landau damping, bump-on-tail instabilities and more, are therefore highly influential in tokamak plasma dynamics. Purely fluid models are inherently incapable of capturing these effects, whereas the high dimensionality in purely kinetic models render them practically intractable for most relevant purposes.

        We consider a $\delta\!f$ decomposition model, with a macroscopic fluid background and microscopic kinetic correction, both fully coupled to each other. A similar manner of discretization is proposed to that used in the recent \texttt{STRUPHY} code \cite{Holderied_Possanner_Wang_2021, Holderied_2022, Li_et_al_2023} with a finite-element model for the background and a pseudo-particle/PiC model for the correction.

        The fluid background satisfies the full, non-linear, resistive, compressible, Hall MHD equations. \cite{Laakmann_Hu_Farrell_2022} introduces finite-element(-in-space) implicit timesteppers for the incompressible analogue to this system with structure-preserving (SP) properties in the ideal case, alongside parameter-robust preconditioners. We show that these timesteppers can derive from a finite-element-in-time (FET) (and finite-element-in-space) interpretation. The benefits of this reformulation are discussed, including the derivation of timesteppers that are higher order in time, and the quantifiable dissipative SP properties in the non-ideal, resistive case.
        
        We discuss possible options for extending this FET approach to timesteppers for the compressible case.

        The kinetic corrections satisfy linearized Boltzmann equations. Using a Lénard--Bernstein collision operator, these take Fokker--Planck-like forms \cite{Fokker_1914, Planck_1917} wherein pseudo-particles in the numerical model obey the neoclassical transport equations, with particle-independent Brownian drift terms. This offers a rigorous methodology for incorporating collisions into the particle transport model, without coupling the equations of motions for each particle.
        
        Works by Chen, Chacón et al. \cite{Chen_Chacón_Barnes_2011, Chacón_Chen_Barnes_2013, Chen_Chacón_2014, Chen_Chacón_2015} have developed structure-preserving particle pushers for neoclassical transport in the Vlasov equations, derived from Crank--Nicolson integrators. We show these too can can derive from a FET interpretation, similarly offering potential extensions to higher-order-in-time particle pushers. The FET formulation is used also to consider how the stochastic drift terms can be incorporated into the pushers. Stochastic gyrokinetic expansions are also discussed.

        Different options for the numerical implementation of these schemes are considered.

        Due to the efficacy of FET in the development of SP timesteppers for both the fluid and kinetic component, we hope this approach will prove effective in the future for developing SP timesteppers for the full hybrid model. We hope this will give us the opportunity to incorporate previously inaccessible kinetic effects into the highly effective, modern, finite-element MHD models.
    \end{abstract}
    
    
    \newpage
    \tableofcontents
    
    
    \newpage
    \pagenumbering{arabic}
    %\linenumbers\renewcommand\thelinenumber{\color{black!50}\arabic{linenumber}}
            \input{0 - introduction/main.tex}
        \part{Research}
            \input{1 - low-noise PiC models/main.tex}
            \input{2 - kinetic component/main.tex}
            \input{3 - fluid component/main.tex}
            \input{4 - numerical implementation/main.tex}
        \part{Project Overview}
            \input{5 - research plan/main.tex}
            \input{6 - summary/main.tex}
    
    
    %\section{}
    \newpage
    \pagenumbering{gobble}
        \printbibliography


    \newpage
    \pagenumbering{roman}
    \appendix
        \part{Appendices}
            \input{8 - Hilbert complexes/main.tex}
            \input{9 - weak conservation proofs/main.tex}
\end{document}

\end{document}

        \part{Research}
            \documentclass[12pt, a4paper]{report}

\documentclass[12pt, a4paper]{report}

\input{template/main.tex}

\title{\BA{Title in Progress...}}
\author{Boris Andrews}
\affil{Mathematical Institute, University of Oxford}
\date{\today}


\begin{document}
    \pagenumbering{gobble}
    \maketitle
    
    
    \begin{abstract}
        Magnetic confinement reactors---in particular tokamaks---offer one of the most promising options for achieving practical nuclear fusion, with the potential to provide virtually limitless, clean energy. The theoretical and numerical modeling of tokamak plasmas is simultaneously an essential component of effective reactor design, and a great research barrier. Tokamak operational conditions exhibit comparatively low Knudsen numbers. Kinetic effects, including kinetic waves and instabilities, Landau damping, bump-on-tail instabilities and more, are therefore highly influential in tokamak plasma dynamics. Purely fluid models are inherently incapable of capturing these effects, whereas the high dimensionality in purely kinetic models render them practically intractable for most relevant purposes.

        We consider a $\delta\!f$ decomposition model, with a macroscopic fluid background and microscopic kinetic correction, both fully coupled to each other. A similar manner of discretization is proposed to that used in the recent \texttt{STRUPHY} code \cite{Holderied_Possanner_Wang_2021, Holderied_2022, Li_et_al_2023} with a finite-element model for the background and a pseudo-particle/PiC model for the correction.

        The fluid background satisfies the full, non-linear, resistive, compressible, Hall MHD equations. \cite{Laakmann_Hu_Farrell_2022} introduces finite-element(-in-space) implicit timesteppers for the incompressible analogue to this system with structure-preserving (SP) properties in the ideal case, alongside parameter-robust preconditioners. We show that these timesteppers can derive from a finite-element-in-time (FET) (and finite-element-in-space) interpretation. The benefits of this reformulation are discussed, including the derivation of timesteppers that are higher order in time, and the quantifiable dissipative SP properties in the non-ideal, resistive case.
        
        We discuss possible options for extending this FET approach to timesteppers for the compressible case.

        The kinetic corrections satisfy linearized Boltzmann equations. Using a Lénard--Bernstein collision operator, these take Fokker--Planck-like forms \cite{Fokker_1914, Planck_1917} wherein pseudo-particles in the numerical model obey the neoclassical transport equations, with particle-independent Brownian drift terms. This offers a rigorous methodology for incorporating collisions into the particle transport model, without coupling the equations of motions for each particle.
        
        Works by Chen, Chacón et al. \cite{Chen_Chacón_Barnes_2011, Chacón_Chen_Barnes_2013, Chen_Chacón_2014, Chen_Chacón_2015} have developed structure-preserving particle pushers for neoclassical transport in the Vlasov equations, derived from Crank--Nicolson integrators. We show these too can can derive from a FET interpretation, similarly offering potential extensions to higher-order-in-time particle pushers. The FET formulation is used also to consider how the stochastic drift terms can be incorporated into the pushers. Stochastic gyrokinetic expansions are also discussed.

        Different options for the numerical implementation of these schemes are considered.

        Due to the efficacy of FET in the development of SP timesteppers for both the fluid and kinetic component, we hope this approach will prove effective in the future for developing SP timesteppers for the full hybrid model. We hope this will give us the opportunity to incorporate previously inaccessible kinetic effects into the highly effective, modern, finite-element MHD models.
    \end{abstract}
    
    
    \newpage
    \tableofcontents
    
    
    \newpage
    \pagenumbering{arabic}
    %\linenumbers\renewcommand\thelinenumber{\color{black!50}\arabic{linenumber}}
            \input{0 - introduction/main.tex}
        \part{Research}
            \input{1 - low-noise PiC models/main.tex}
            \input{2 - kinetic component/main.tex}
            \input{3 - fluid component/main.tex}
            \input{4 - numerical implementation/main.tex}
        \part{Project Overview}
            \input{5 - research plan/main.tex}
            \input{6 - summary/main.tex}
    
    
    %\section{}
    \newpage
    \pagenumbering{gobble}
        \printbibliography


    \newpage
    \pagenumbering{roman}
    \appendix
        \part{Appendices}
            \input{8 - Hilbert complexes/main.tex}
            \input{9 - weak conservation proofs/main.tex}
\end{document}


\title{\BA{Title in Progress...}}
\author{Boris Andrews}
\affil{Mathematical Institute, University of Oxford}
\date{\today}


\begin{document}
    \pagenumbering{gobble}
    \maketitle
    
    
    \begin{abstract}
        Magnetic confinement reactors---in particular tokamaks---offer one of the most promising options for achieving practical nuclear fusion, with the potential to provide virtually limitless, clean energy. The theoretical and numerical modeling of tokamak plasmas is simultaneously an essential component of effective reactor design, and a great research barrier. Tokamak operational conditions exhibit comparatively low Knudsen numbers. Kinetic effects, including kinetic waves and instabilities, Landau damping, bump-on-tail instabilities and more, are therefore highly influential in tokamak plasma dynamics. Purely fluid models are inherently incapable of capturing these effects, whereas the high dimensionality in purely kinetic models render them practically intractable for most relevant purposes.

        We consider a $\delta\!f$ decomposition model, with a macroscopic fluid background and microscopic kinetic correction, both fully coupled to each other. A similar manner of discretization is proposed to that used in the recent \texttt{STRUPHY} code \cite{Holderied_Possanner_Wang_2021, Holderied_2022, Li_et_al_2023} with a finite-element model for the background and a pseudo-particle/PiC model for the correction.

        The fluid background satisfies the full, non-linear, resistive, compressible, Hall MHD equations. \cite{Laakmann_Hu_Farrell_2022} introduces finite-element(-in-space) implicit timesteppers for the incompressible analogue to this system with structure-preserving (SP) properties in the ideal case, alongside parameter-robust preconditioners. We show that these timesteppers can derive from a finite-element-in-time (FET) (and finite-element-in-space) interpretation. The benefits of this reformulation are discussed, including the derivation of timesteppers that are higher order in time, and the quantifiable dissipative SP properties in the non-ideal, resistive case.
        
        We discuss possible options for extending this FET approach to timesteppers for the compressible case.

        The kinetic corrections satisfy linearized Boltzmann equations. Using a Lénard--Bernstein collision operator, these take Fokker--Planck-like forms \cite{Fokker_1914, Planck_1917} wherein pseudo-particles in the numerical model obey the neoclassical transport equations, with particle-independent Brownian drift terms. This offers a rigorous methodology for incorporating collisions into the particle transport model, without coupling the equations of motions for each particle.
        
        Works by Chen, Chacón et al. \cite{Chen_Chacón_Barnes_2011, Chacón_Chen_Barnes_2013, Chen_Chacón_2014, Chen_Chacón_2015} have developed structure-preserving particle pushers for neoclassical transport in the Vlasov equations, derived from Crank--Nicolson integrators. We show these too can can derive from a FET interpretation, similarly offering potential extensions to higher-order-in-time particle pushers. The FET formulation is used also to consider how the stochastic drift terms can be incorporated into the pushers. Stochastic gyrokinetic expansions are also discussed.

        Different options for the numerical implementation of these schemes are considered.

        Due to the efficacy of FET in the development of SP timesteppers for both the fluid and kinetic component, we hope this approach will prove effective in the future for developing SP timesteppers for the full hybrid model. We hope this will give us the opportunity to incorporate previously inaccessible kinetic effects into the highly effective, modern, finite-element MHD models.
    \end{abstract}
    
    
    \newpage
    \tableofcontents
    
    
    \newpage
    \pagenumbering{arabic}
    %\linenumbers\renewcommand\thelinenumber{\color{black!50}\arabic{linenumber}}
            \documentclass[12pt, a4paper]{report}

\input{template/main.tex}

\title{\BA{Title in Progress...}}
\author{Boris Andrews}
\affil{Mathematical Institute, University of Oxford}
\date{\today}


\begin{document}
    \pagenumbering{gobble}
    \maketitle
    
    
    \begin{abstract}
        Magnetic confinement reactors---in particular tokamaks---offer one of the most promising options for achieving practical nuclear fusion, with the potential to provide virtually limitless, clean energy. The theoretical and numerical modeling of tokamak plasmas is simultaneously an essential component of effective reactor design, and a great research barrier. Tokamak operational conditions exhibit comparatively low Knudsen numbers. Kinetic effects, including kinetic waves and instabilities, Landau damping, bump-on-tail instabilities and more, are therefore highly influential in tokamak plasma dynamics. Purely fluid models are inherently incapable of capturing these effects, whereas the high dimensionality in purely kinetic models render them practically intractable for most relevant purposes.

        We consider a $\delta\!f$ decomposition model, with a macroscopic fluid background and microscopic kinetic correction, both fully coupled to each other. A similar manner of discretization is proposed to that used in the recent \texttt{STRUPHY} code \cite{Holderied_Possanner_Wang_2021, Holderied_2022, Li_et_al_2023} with a finite-element model for the background and a pseudo-particle/PiC model for the correction.

        The fluid background satisfies the full, non-linear, resistive, compressible, Hall MHD equations. \cite{Laakmann_Hu_Farrell_2022} introduces finite-element(-in-space) implicit timesteppers for the incompressible analogue to this system with structure-preserving (SP) properties in the ideal case, alongside parameter-robust preconditioners. We show that these timesteppers can derive from a finite-element-in-time (FET) (and finite-element-in-space) interpretation. The benefits of this reformulation are discussed, including the derivation of timesteppers that are higher order in time, and the quantifiable dissipative SP properties in the non-ideal, resistive case.
        
        We discuss possible options for extending this FET approach to timesteppers for the compressible case.

        The kinetic corrections satisfy linearized Boltzmann equations. Using a Lénard--Bernstein collision operator, these take Fokker--Planck-like forms \cite{Fokker_1914, Planck_1917} wherein pseudo-particles in the numerical model obey the neoclassical transport equations, with particle-independent Brownian drift terms. This offers a rigorous methodology for incorporating collisions into the particle transport model, without coupling the equations of motions for each particle.
        
        Works by Chen, Chacón et al. \cite{Chen_Chacón_Barnes_2011, Chacón_Chen_Barnes_2013, Chen_Chacón_2014, Chen_Chacón_2015} have developed structure-preserving particle pushers for neoclassical transport in the Vlasov equations, derived from Crank--Nicolson integrators. We show these too can can derive from a FET interpretation, similarly offering potential extensions to higher-order-in-time particle pushers. The FET formulation is used also to consider how the stochastic drift terms can be incorporated into the pushers. Stochastic gyrokinetic expansions are also discussed.

        Different options for the numerical implementation of these schemes are considered.

        Due to the efficacy of FET in the development of SP timesteppers for both the fluid and kinetic component, we hope this approach will prove effective in the future for developing SP timesteppers for the full hybrid model. We hope this will give us the opportunity to incorporate previously inaccessible kinetic effects into the highly effective, modern, finite-element MHD models.
    \end{abstract}
    
    
    \newpage
    \tableofcontents
    
    
    \newpage
    \pagenumbering{arabic}
    %\linenumbers\renewcommand\thelinenumber{\color{black!50}\arabic{linenumber}}
            \input{0 - introduction/main.tex}
        \part{Research}
            \input{1 - low-noise PiC models/main.tex}
            \input{2 - kinetic component/main.tex}
            \input{3 - fluid component/main.tex}
            \input{4 - numerical implementation/main.tex}
        \part{Project Overview}
            \input{5 - research plan/main.tex}
            \input{6 - summary/main.tex}
    
    
    %\section{}
    \newpage
    \pagenumbering{gobble}
        \printbibliography


    \newpage
    \pagenumbering{roman}
    \appendix
        \part{Appendices}
            \input{8 - Hilbert complexes/main.tex}
            \input{9 - weak conservation proofs/main.tex}
\end{document}

        \part{Research}
            \documentclass[12pt, a4paper]{report}

\input{template/main.tex}

\title{\BA{Title in Progress...}}
\author{Boris Andrews}
\affil{Mathematical Institute, University of Oxford}
\date{\today}


\begin{document}
    \pagenumbering{gobble}
    \maketitle
    
    
    \begin{abstract}
        Magnetic confinement reactors---in particular tokamaks---offer one of the most promising options for achieving practical nuclear fusion, with the potential to provide virtually limitless, clean energy. The theoretical and numerical modeling of tokamak plasmas is simultaneously an essential component of effective reactor design, and a great research barrier. Tokamak operational conditions exhibit comparatively low Knudsen numbers. Kinetic effects, including kinetic waves and instabilities, Landau damping, bump-on-tail instabilities and more, are therefore highly influential in tokamak plasma dynamics. Purely fluid models are inherently incapable of capturing these effects, whereas the high dimensionality in purely kinetic models render them practically intractable for most relevant purposes.

        We consider a $\delta\!f$ decomposition model, with a macroscopic fluid background and microscopic kinetic correction, both fully coupled to each other. A similar manner of discretization is proposed to that used in the recent \texttt{STRUPHY} code \cite{Holderied_Possanner_Wang_2021, Holderied_2022, Li_et_al_2023} with a finite-element model for the background and a pseudo-particle/PiC model for the correction.

        The fluid background satisfies the full, non-linear, resistive, compressible, Hall MHD equations. \cite{Laakmann_Hu_Farrell_2022} introduces finite-element(-in-space) implicit timesteppers for the incompressible analogue to this system with structure-preserving (SP) properties in the ideal case, alongside parameter-robust preconditioners. We show that these timesteppers can derive from a finite-element-in-time (FET) (and finite-element-in-space) interpretation. The benefits of this reformulation are discussed, including the derivation of timesteppers that are higher order in time, and the quantifiable dissipative SP properties in the non-ideal, resistive case.
        
        We discuss possible options for extending this FET approach to timesteppers for the compressible case.

        The kinetic corrections satisfy linearized Boltzmann equations. Using a Lénard--Bernstein collision operator, these take Fokker--Planck-like forms \cite{Fokker_1914, Planck_1917} wherein pseudo-particles in the numerical model obey the neoclassical transport equations, with particle-independent Brownian drift terms. This offers a rigorous methodology for incorporating collisions into the particle transport model, without coupling the equations of motions for each particle.
        
        Works by Chen, Chacón et al. \cite{Chen_Chacón_Barnes_2011, Chacón_Chen_Barnes_2013, Chen_Chacón_2014, Chen_Chacón_2015} have developed structure-preserving particle pushers for neoclassical transport in the Vlasov equations, derived from Crank--Nicolson integrators. We show these too can can derive from a FET interpretation, similarly offering potential extensions to higher-order-in-time particle pushers. The FET formulation is used also to consider how the stochastic drift terms can be incorporated into the pushers. Stochastic gyrokinetic expansions are also discussed.

        Different options for the numerical implementation of these schemes are considered.

        Due to the efficacy of FET in the development of SP timesteppers for both the fluid and kinetic component, we hope this approach will prove effective in the future for developing SP timesteppers for the full hybrid model. We hope this will give us the opportunity to incorporate previously inaccessible kinetic effects into the highly effective, modern, finite-element MHD models.
    \end{abstract}
    
    
    \newpage
    \tableofcontents
    
    
    \newpage
    \pagenumbering{arabic}
    %\linenumbers\renewcommand\thelinenumber{\color{black!50}\arabic{linenumber}}
            \input{0 - introduction/main.tex}
        \part{Research}
            \input{1 - low-noise PiC models/main.tex}
            \input{2 - kinetic component/main.tex}
            \input{3 - fluid component/main.tex}
            \input{4 - numerical implementation/main.tex}
        \part{Project Overview}
            \input{5 - research plan/main.tex}
            \input{6 - summary/main.tex}
    
    
    %\section{}
    \newpage
    \pagenumbering{gobble}
        \printbibliography


    \newpage
    \pagenumbering{roman}
    \appendix
        \part{Appendices}
            \input{8 - Hilbert complexes/main.tex}
            \input{9 - weak conservation proofs/main.tex}
\end{document}

            \documentclass[12pt, a4paper]{report}

\input{template/main.tex}

\title{\BA{Title in Progress...}}
\author{Boris Andrews}
\affil{Mathematical Institute, University of Oxford}
\date{\today}


\begin{document}
    \pagenumbering{gobble}
    \maketitle
    
    
    \begin{abstract}
        Magnetic confinement reactors---in particular tokamaks---offer one of the most promising options for achieving practical nuclear fusion, with the potential to provide virtually limitless, clean energy. The theoretical and numerical modeling of tokamak plasmas is simultaneously an essential component of effective reactor design, and a great research barrier. Tokamak operational conditions exhibit comparatively low Knudsen numbers. Kinetic effects, including kinetic waves and instabilities, Landau damping, bump-on-tail instabilities and more, are therefore highly influential in tokamak plasma dynamics. Purely fluid models are inherently incapable of capturing these effects, whereas the high dimensionality in purely kinetic models render them practically intractable for most relevant purposes.

        We consider a $\delta\!f$ decomposition model, with a macroscopic fluid background and microscopic kinetic correction, both fully coupled to each other. A similar manner of discretization is proposed to that used in the recent \texttt{STRUPHY} code \cite{Holderied_Possanner_Wang_2021, Holderied_2022, Li_et_al_2023} with a finite-element model for the background and a pseudo-particle/PiC model for the correction.

        The fluid background satisfies the full, non-linear, resistive, compressible, Hall MHD equations. \cite{Laakmann_Hu_Farrell_2022} introduces finite-element(-in-space) implicit timesteppers for the incompressible analogue to this system with structure-preserving (SP) properties in the ideal case, alongside parameter-robust preconditioners. We show that these timesteppers can derive from a finite-element-in-time (FET) (and finite-element-in-space) interpretation. The benefits of this reformulation are discussed, including the derivation of timesteppers that are higher order in time, and the quantifiable dissipative SP properties in the non-ideal, resistive case.
        
        We discuss possible options for extending this FET approach to timesteppers for the compressible case.

        The kinetic corrections satisfy linearized Boltzmann equations. Using a Lénard--Bernstein collision operator, these take Fokker--Planck-like forms \cite{Fokker_1914, Planck_1917} wherein pseudo-particles in the numerical model obey the neoclassical transport equations, with particle-independent Brownian drift terms. This offers a rigorous methodology for incorporating collisions into the particle transport model, without coupling the equations of motions for each particle.
        
        Works by Chen, Chacón et al. \cite{Chen_Chacón_Barnes_2011, Chacón_Chen_Barnes_2013, Chen_Chacón_2014, Chen_Chacón_2015} have developed structure-preserving particle pushers for neoclassical transport in the Vlasov equations, derived from Crank--Nicolson integrators. We show these too can can derive from a FET interpretation, similarly offering potential extensions to higher-order-in-time particle pushers. The FET formulation is used also to consider how the stochastic drift terms can be incorporated into the pushers. Stochastic gyrokinetic expansions are also discussed.

        Different options for the numerical implementation of these schemes are considered.

        Due to the efficacy of FET in the development of SP timesteppers for both the fluid and kinetic component, we hope this approach will prove effective in the future for developing SP timesteppers for the full hybrid model. We hope this will give us the opportunity to incorporate previously inaccessible kinetic effects into the highly effective, modern, finite-element MHD models.
    \end{abstract}
    
    
    \newpage
    \tableofcontents
    
    
    \newpage
    \pagenumbering{arabic}
    %\linenumbers\renewcommand\thelinenumber{\color{black!50}\arabic{linenumber}}
            \input{0 - introduction/main.tex}
        \part{Research}
            \input{1 - low-noise PiC models/main.tex}
            \input{2 - kinetic component/main.tex}
            \input{3 - fluid component/main.tex}
            \input{4 - numerical implementation/main.tex}
        \part{Project Overview}
            \input{5 - research plan/main.tex}
            \input{6 - summary/main.tex}
    
    
    %\section{}
    \newpage
    \pagenumbering{gobble}
        \printbibliography


    \newpage
    \pagenumbering{roman}
    \appendix
        \part{Appendices}
            \input{8 - Hilbert complexes/main.tex}
            \input{9 - weak conservation proofs/main.tex}
\end{document}

            \documentclass[12pt, a4paper]{report}

\input{template/main.tex}

\title{\BA{Title in Progress...}}
\author{Boris Andrews}
\affil{Mathematical Institute, University of Oxford}
\date{\today}


\begin{document}
    \pagenumbering{gobble}
    \maketitle
    
    
    \begin{abstract}
        Magnetic confinement reactors---in particular tokamaks---offer one of the most promising options for achieving practical nuclear fusion, with the potential to provide virtually limitless, clean energy. The theoretical and numerical modeling of tokamak plasmas is simultaneously an essential component of effective reactor design, and a great research barrier. Tokamak operational conditions exhibit comparatively low Knudsen numbers. Kinetic effects, including kinetic waves and instabilities, Landau damping, bump-on-tail instabilities and more, are therefore highly influential in tokamak plasma dynamics. Purely fluid models are inherently incapable of capturing these effects, whereas the high dimensionality in purely kinetic models render them practically intractable for most relevant purposes.

        We consider a $\delta\!f$ decomposition model, with a macroscopic fluid background and microscopic kinetic correction, both fully coupled to each other. A similar manner of discretization is proposed to that used in the recent \texttt{STRUPHY} code \cite{Holderied_Possanner_Wang_2021, Holderied_2022, Li_et_al_2023} with a finite-element model for the background and a pseudo-particle/PiC model for the correction.

        The fluid background satisfies the full, non-linear, resistive, compressible, Hall MHD equations. \cite{Laakmann_Hu_Farrell_2022} introduces finite-element(-in-space) implicit timesteppers for the incompressible analogue to this system with structure-preserving (SP) properties in the ideal case, alongside parameter-robust preconditioners. We show that these timesteppers can derive from a finite-element-in-time (FET) (and finite-element-in-space) interpretation. The benefits of this reformulation are discussed, including the derivation of timesteppers that are higher order in time, and the quantifiable dissipative SP properties in the non-ideal, resistive case.
        
        We discuss possible options for extending this FET approach to timesteppers for the compressible case.

        The kinetic corrections satisfy linearized Boltzmann equations. Using a Lénard--Bernstein collision operator, these take Fokker--Planck-like forms \cite{Fokker_1914, Planck_1917} wherein pseudo-particles in the numerical model obey the neoclassical transport equations, with particle-independent Brownian drift terms. This offers a rigorous methodology for incorporating collisions into the particle transport model, without coupling the equations of motions for each particle.
        
        Works by Chen, Chacón et al. \cite{Chen_Chacón_Barnes_2011, Chacón_Chen_Barnes_2013, Chen_Chacón_2014, Chen_Chacón_2015} have developed structure-preserving particle pushers for neoclassical transport in the Vlasov equations, derived from Crank--Nicolson integrators. We show these too can can derive from a FET interpretation, similarly offering potential extensions to higher-order-in-time particle pushers. The FET formulation is used also to consider how the stochastic drift terms can be incorporated into the pushers. Stochastic gyrokinetic expansions are also discussed.

        Different options for the numerical implementation of these schemes are considered.

        Due to the efficacy of FET in the development of SP timesteppers for both the fluid and kinetic component, we hope this approach will prove effective in the future for developing SP timesteppers for the full hybrid model. We hope this will give us the opportunity to incorporate previously inaccessible kinetic effects into the highly effective, modern, finite-element MHD models.
    \end{abstract}
    
    
    \newpage
    \tableofcontents
    
    
    \newpage
    \pagenumbering{arabic}
    %\linenumbers\renewcommand\thelinenumber{\color{black!50}\arabic{linenumber}}
            \input{0 - introduction/main.tex}
        \part{Research}
            \input{1 - low-noise PiC models/main.tex}
            \input{2 - kinetic component/main.tex}
            \input{3 - fluid component/main.tex}
            \input{4 - numerical implementation/main.tex}
        \part{Project Overview}
            \input{5 - research plan/main.tex}
            \input{6 - summary/main.tex}
    
    
    %\section{}
    \newpage
    \pagenumbering{gobble}
        \printbibliography


    \newpage
    \pagenumbering{roman}
    \appendix
        \part{Appendices}
            \input{8 - Hilbert complexes/main.tex}
            \input{9 - weak conservation proofs/main.tex}
\end{document}

            \documentclass[12pt, a4paper]{report}

\input{template/main.tex}

\title{\BA{Title in Progress...}}
\author{Boris Andrews}
\affil{Mathematical Institute, University of Oxford}
\date{\today}


\begin{document}
    \pagenumbering{gobble}
    \maketitle
    
    
    \begin{abstract}
        Magnetic confinement reactors---in particular tokamaks---offer one of the most promising options for achieving practical nuclear fusion, with the potential to provide virtually limitless, clean energy. The theoretical and numerical modeling of tokamak plasmas is simultaneously an essential component of effective reactor design, and a great research barrier. Tokamak operational conditions exhibit comparatively low Knudsen numbers. Kinetic effects, including kinetic waves and instabilities, Landau damping, bump-on-tail instabilities and more, are therefore highly influential in tokamak plasma dynamics. Purely fluid models are inherently incapable of capturing these effects, whereas the high dimensionality in purely kinetic models render them practically intractable for most relevant purposes.

        We consider a $\delta\!f$ decomposition model, with a macroscopic fluid background and microscopic kinetic correction, both fully coupled to each other. A similar manner of discretization is proposed to that used in the recent \texttt{STRUPHY} code \cite{Holderied_Possanner_Wang_2021, Holderied_2022, Li_et_al_2023} with a finite-element model for the background and a pseudo-particle/PiC model for the correction.

        The fluid background satisfies the full, non-linear, resistive, compressible, Hall MHD equations. \cite{Laakmann_Hu_Farrell_2022} introduces finite-element(-in-space) implicit timesteppers for the incompressible analogue to this system with structure-preserving (SP) properties in the ideal case, alongside parameter-robust preconditioners. We show that these timesteppers can derive from a finite-element-in-time (FET) (and finite-element-in-space) interpretation. The benefits of this reformulation are discussed, including the derivation of timesteppers that are higher order in time, and the quantifiable dissipative SP properties in the non-ideal, resistive case.
        
        We discuss possible options for extending this FET approach to timesteppers for the compressible case.

        The kinetic corrections satisfy linearized Boltzmann equations. Using a Lénard--Bernstein collision operator, these take Fokker--Planck-like forms \cite{Fokker_1914, Planck_1917} wherein pseudo-particles in the numerical model obey the neoclassical transport equations, with particle-independent Brownian drift terms. This offers a rigorous methodology for incorporating collisions into the particle transport model, without coupling the equations of motions for each particle.
        
        Works by Chen, Chacón et al. \cite{Chen_Chacón_Barnes_2011, Chacón_Chen_Barnes_2013, Chen_Chacón_2014, Chen_Chacón_2015} have developed structure-preserving particle pushers for neoclassical transport in the Vlasov equations, derived from Crank--Nicolson integrators. We show these too can can derive from a FET interpretation, similarly offering potential extensions to higher-order-in-time particle pushers. The FET formulation is used also to consider how the stochastic drift terms can be incorporated into the pushers. Stochastic gyrokinetic expansions are also discussed.

        Different options for the numerical implementation of these schemes are considered.

        Due to the efficacy of FET in the development of SP timesteppers for both the fluid and kinetic component, we hope this approach will prove effective in the future for developing SP timesteppers for the full hybrid model. We hope this will give us the opportunity to incorporate previously inaccessible kinetic effects into the highly effective, modern, finite-element MHD models.
    \end{abstract}
    
    
    \newpage
    \tableofcontents
    
    
    \newpage
    \pagenumbering{arabic}
    %\linenumbers\renewcommand\thelinenumber{\color{black!50}\arabic{linenumber}}
            \input{0 - introduction/main.tex}
        \part{Research}
            \input{1 - low-noise PiC models/main.tex}
            \input{2 - kinetic component/main.tex}
            \input{3 - fluid component/main.tex}
            \input{4 - numerical implementation/main.tex}
        \part{Project Overview}
            \input{5 - research plan/main.tex}
            \input{6 - summary/main.tex}
    
    
    %\section{}
    \newpage
    \pagenumbering{gobble}
        \printbibliography


    \newpage
    \pagenumbering{roman}
    \appendix
        \part{Appendices}
            \input{8 - Hilbert complexes/main.tex}
            \input{9 - weak conservation proofs/main.tex}
\end{document}

        \part{Project Overview}
            \documentclass[12pt, a4paper]{report}

\input{template/main.tex}

\title{\BA{Title in Progress...}}
\author{Boris Andrews}
\affil{Mathematical Institute, University of Oxford}
\date{\today}


\begin{document}
    \pagenumbering{gobble}
    \maketitle
    
    
    \begin{abstract}
        Magnetic confinement reactors---in particular tokamaks---offer one of the most promising options for achieving practical nuclear fusion, with the potential to provide virtually limitless, clean energy. The theoretical and numerical modeling of tokamak plasmas is simultaneously an essential component of effective reactor design, and a great research barrier. Tokamak operational conditions exhibit comparatively low Knudsen numbers. Kinetic effects, including kinetic waves and instabilities, Landau damping, bump-on-tail instabilities and more, are therefore highly influential in tokamak plasma dynamics. Purely fluid models are inherently incapable of capturing these effects, whereas the high dimensionality in purely kinetic models render them practically intractable for most relevant purposes.

        We consider a $\delta\!f$ decomposition model, with a macroscopic fluid background and microscopic kinetic correction, both fully coupled to each other. A similar manner of discretization is proposed to that used in the recent \texttt{STRUPHY} code \cite{Holderied_Possanner_Wang_2021, Holderied_2022, Li_et_al_2023} with a finite-element model for the background and a pseudo-particle/PiC model for the correction.

        The fluid background satisfies the full, non-linear, resistive, compressible, Hall MHD equations. \cite{Laakmann_Hu_Farrell_2022} introduces finite-element(-in-space) implicit timesteppers for the incompressible analogue to this system with structure-preserving (SP) properties in the ideal case, alongside parameter-robust preconditioners. We show that these timesteppers can derive from a finite-element-in-time (FET) (and finite-element-in-space) interpretation. The benefits of this reformulation are discussed, including the derivation of timesteppers that are higher order in time, and the quantifiable dissipative SP properties in the non-ideal, resistive case.
        
        We discuss possible options for extending this FET approach to timesteppers for the compressible case.

        The kinetic corrections satisfy linearized Boltzmann equations. Using a Lénard--Bernstein collision operator, these take Fokker--Planck-like forms \cite{Fokker_1914, Planck_1917} wherein pseudo-particles in the numerical model obey the neoclassical transport equations, with particle-independent Brownian drift terms. This offers a rigorous methodology for incorporating collisions into the particle transport model, without coupling the equations of motions for each particle.
        
        Works by Chen, Chacón et al. \cite{Chen_Chacón_Barnes_2011, Chacón_Chen_Barnes_2013, Chen_Chacón_2014, Chen_Chacón_2015} have developed structure-preserving particle pushers for neoclassical transport in the Vlasov equations, derived from Crank--Nicolson integrators. We show these too can can derive from a FET interpretation, similarly offering potential extensions to higher-order-in-time particle pushers. The FET formulation is used also to consider how the stochastic drift terms can be incorporated into the pushers. Stochastic gyrokinetic expansions are also discussed.

        Different options for the numerical implementation of these schemes are considered.

        Due to the efficacy of FET in the development of SP timesteppers for both the fluid and kinetic component, we hope this approach will prove effective in the future for developing SP timesteppers for the full hybrid model. We hope this will give us the opportunity to incorporate previously inaccessible kinetic effects into the highly effective, modern, finite-element MHD models.
    \end{abstract}
    
    
    \newpage
    \tableofcontents
    
    
    \newpage
    \pagenumbering{arabic}
    %\linenumbers\renewcommand\thelinenumber{\color{black!50}\arabic{linenumber}}
            \input{0 - introduction/main.tex}
        \part{Research}
            \input{1 - low-noise PiC models/main.tex}
            \input{2 - kinetic component/main.tex}
            \input{3 - fluid component/main.tex}
            \input{4 - numerical implementation/main.tex}
        \part{Project Overview}
            \input{5 - research plan/main.tex}
            \input{6 - summary/main.tex}
    
    
    %\section{}
    \newpage
    \pagenumbering{gobble}
        \printbibliography


    \newpage
    \pagenumbering{roman}
    \appendix
        \part{Appendices}
            \input{8 - Hilbert complexes/main.tex}
            \input{9 - weak conservation proofs/main.tex}
\end{document}

            \documentclass[12pt, a4paper]{report}

\input{template/main.tex}

\title{\BA{Title in Progress...}}
\author{Boris Andrews}
\affil{Mathematical Institute, University of Oxford}
\date{\today}


\begin{document}
    \pagenumbering{gobble}
    \maketitle
    
    
    \begin{abstract}
        Magnetic confinement reactors---in particular tokamaks---offer one of the most promising options for achieving practical nuclear fusion, with the potential to provide virtually limitless, clean energy. The theoretical and numerical modeling of tokamak plasmas is simultaneously an essential component of effective reactor design, and a great research barrier. Tokamak operational conditions exhibit comparatively low Knudsen numbers. Kinetic effects, including kinetic waves and instabilities, Landau damping, bump-on-tail instabilities and more, are therefore highly influential in tokamak plasma dynamics. Purely fluid models are inherently incapable of capturing these effects, whereas the high dimensionality in purely kinetic models render them practically intractable for most relevant purposes.

        We consider a $\delta\!f$ decomposition model, with a macroscopic fluid background and microscopic kinetic correction, both fully coupled to each other. A similar manner of discretization is proposed to that used in the recent \texttt{STRUPHY} code \cite{Holderied_Possanner_Wang_2021, Holderied_2022, Li_et_al_2023} with a finite-element model for the background and a pseudo-particle/PiC model for the correction.

        The fluid background satisfies the full, non-linear, resistive, compressible, Hall MHD equations. \cite{Laakmann_Hu_Farrell_2022} introduces finite-element(-in-space) implicit timesteppers for the incompressible analogue to this system with structure-preserving (SP) properties in the ideal case, alongside parameter-robust preconditioners. We show that these timesteppers can derive from a finite-element-in-time (FET) (and finite-element-in-space) interpretation. The benefits of this reformulation are discussed, including the derivation of timesteppers that are higher order in time, and the quantifiable dissipative SP properties in the non-ideal, resistive case.
        
        We discuss possible options for extending this FET approach to timesteppers for the compressible case.

        The kinetic corrections satisfy linearized Boltzmann equations. Using a Lénard--Bernstein collision operator, these take Fokker--Planck-like forms \cite{Fokker_1914, Planck_1917} wherein pseudo-particles in the numerical model obey the neoclassical transport equations, with particle-independent Brownian drift terms. This offers a rigorous methodology for incorporating collisions into the particle transport model, without coupling the equations of motions for each particle.
        
        Works by Chen, Chacón et al. \cite{Chen_Chacón_Barnes_2011, Chacón_Chen_Barnes_2013, Chen_Chacón_2014, Chen_Chacón_2015} have developed structure-preserving particle pushers for neoclassical transport in the Vlasov equations, derived from Crank--Nicolson integrators. We show these too can can derive from a FET interpretation, similarly offering potential extensions to higher-order-in-time particle pushers. The FET formulation is used also to consider how the stochastic drift terms can be incorporated into the pushers. Stochastic gyrokinetic expansions are also discussed.

        Different options for the numerical implementation of these schemes are considered.

        Due to the efficacy of FET in the development of SP timesteppers for both the fluid and kinetic component, we hope this approach will prove effective in the future for developing SP timesteppers for the full hybrid model. We hope this will give us the opportunity to incorporate previously inaccessible kinetic effects into the highly effective, modern, finite-element MHD models.
    \end{abstract}
    
    
    \newpage
    \tableofcontents
    
    
    \newpage
    \pagenumbering{arabic}
    %\linenumbers\renewcommand\thelinenumber{\color{black!50}\arabic{linenumber}}
            \input{0 - introduction/main.tex}
        \part{Research}
            \input{1 - low-noise PiC models/main.tex}
            \input{2 - kinetic component/main.tex}
            \input{3 - fluid component/main.tex}
            \input{4 - numerical implementation/main.tex}
        \part{Project Overview}
            \input{5 - research plan/main.tex}
            \input{6 - summary/main.tex}
    
    
    %\section{}
    \newpage
    \pagenumbering{gobble}
        \printbibliography


    \newpage
    \pagenumbering{roman}
    \appendix
        \part{Appendices}
            \input{8 - Hilbert complexes/main.tex}
            \input{9 - weak conservation proofs/main.tex}
\end{document}

    
    
    %\section{}
    \newpage
    \pagenumbering{gobble}
        \printbibliography


    \newpage
    \pagenumbering{roman}
    \appendix
        \part{Appendices}
            \documentclass[12pt, a4paper]{report}

\input{template/main.tex}

\title{\BA{Title in Progress...}}
\author{Boris Andrews}
\affil{Mathematical Institute, University of Oxford}
\date{\today}


\begin{document}
    \pagenumbering{gobble}
    \maketitle
    
    
    \begin{abstract}
        Magnetic confinement reactors---in particular tokamaks---offer one of the most promising options for achieving practical nuclear fusion, with the potential to provide virtually limitless, clean energy. The theoretical and numerical modeling of tokamak plasmas is simultaneously an essential component of effective reactor design, and a great research barrier. Tokamak operational conditions exhibit comparatively low Knudsen numbers. Kinetic effects, including kinetic waves and instabilities, Landau damping, bump-on-tail instabilities and more, are therefore highly influential in tokamak plasma dynamics. Purely fluid models are inherently incapable of capturing these effects, whereas the high dimensionality in purely kinetic models render them practically intractable for most relevant purposes.

        We consider a $\delta\!f$ decomposition model, with a macroscopic fluid background and microscopic kinetic correction, both fully coupled to each other. A similar manner of discretization is proposed to that used in the recent \texttt{STRUPHY} code \cite{Holderied_Possanner_Wang_2021, Holderied_2022, Li_et_al_2023} with a finite-element model for the background and a pseudo-particle/PiC model for the correction.

        The fluid background satisfies the full, non-linear, resistive, compressible, Hall MHD equations. \cite{Laakmann_Hu_Farrell_2022} introduces finite-element(-in-space) implicit timesteppers for the incompressible analogue to this system with structure-preserving (SP) properties in the ideal case, alongside parameter-robust preconditioners. We show that these timesteppers can derive from a finite-element-in-time (FET) (and finite-element-in-space) interpretation. The benefits of this reformulation are discussed, including the derivation of timesteppers that are higher order in time, and the quantifiable dissipative SP properties in the non-ideal, resistive case.
        
        We discuss possible options for extending this FET approach to timesteppers for the compressible case.

        The kinetic corrections satisfy linearized Boltzmann equations. Using a Lénard--Bernstein collision operator, these take Fokker--Planck-like forms \cite{Fokker_1914, Planck_1917} wherein pseudo-particles in the numerical model obey the neoclassical transport equations, with particle-independent Brownian drift terms. This offers a rigorous methodology for incorporating collisions into the particle transport model, without coupling the equations of motions for each particle.
        
        Works by Chen, Chacón et al. \cite{Chen_Chacón_Barnes_2011, Chacón_Chen_Barnes_2013, Chen_Chacón_2014, Chen_Chacón_2015} have developed structure-preserving particle pushers for neoclassical transport in the Vlasov equations, derived from Crank--Nicolson integrators. We show these too can can derive from a FET interpretation, similarly offering potential extensions to higher-order-in-time particle pushers. The FET formulation is used also to consider how the stochastic drift terms can be incorporated into the pushers. Stochastic gyrokinetic expansions are also discussed.

        Different options for the numerical implementation of these schemes are considered.

        Due to the efficacy of FET in the development of SP timesteppers for both the fluid and kinetic component, we hope this approach will prove effective in the future for developing SP timesteppers for the full hybrid model. We hope this will give us the opportunity to incorporate previously inaccessible kinetic effects into the highly effective, modern, finite-element MHD models.
    \end{abstract}
    
    
    \newpage
    \tableofcontents
    
    
    \newpage
    \pagenumbering{arabic}
    %\linenumbers\renewcommand\thelinenumber{\color{black!50}\arabic{linenumber}}
            \input{0 - introduction/main.tex}
        \part{Research}
            \input{1 - low-noise PiC models/main.tex}
            \input{2 - kinetic component/main.tex}
            \input{3 - fluid component/main.tex}
            \input{4 - numerical implementation/main.tex}
        \part{Project Overview}
            \input{5 - research plan/main.tex}
            \input{6 - summary/main.tex}
    
    
    %\section{}
    \newpage
    \pagenumbering{gobble}
        \printbibliography


    \newpage
    \pagenumbering{roman}
    \appendix
        \part{Appendices}
            \input{8 - Hilbert complexes/main.tex}
            \input{9 - weak conservation proofs/main.tex}
\end{document}

            \documentclass[12pt, a4paper]{report}

\input{template/main.tex}

\title{\BA{Title in Progress...}}
\author{Boris Andrews}
\affil{Mathematical Institute, University of Oxford}
\date{\today}


\begin{document}
    \pagenumbering{gobble}
    \maketitle
    
    
    \begin{abstract}
        Magnetic confinement reactors---in particular tokamaks---offer one of the most promising options for achieving practical nuclear fusion, with the potential to provide virtually limitless, clean energy. The theoretical and numerical modeling of tokamak plasmas is simultaneously an essential component of effective reactor design, and a great research barrier. Tokamak operational conditions exhibit comparatively low Knudsen numbers. Kinetic effects, including kinetic waves and instabilities, Landau damping, bump-on-tail instabilities and more, are therefore highly influential in tokamak plasma dynamics. Purely fluid models are inherently incapable of capturing these effects, whereas the high dimensionality in purely kinetic models render them practically intractable for most relevant purposes.

        We consider a $\delta\!f$ decomposition model, with a macroscopic fluid background and microscopic kinetic correction, both fully coupled to each other. A similar manner of discretization is proposed to that used in the recent \texttt{STRUPHY} code \cite{Holderied_Possanner_Wang_2021, Holderied_2022, Li_et_al_2023} with a finite-element model for the background and a pseudo-particle/PiC model for the correction.

        The fluid background satisfies the full, non-linear, resistive, compressible, Hall MHD equations. \cite{Laakmann_Hu_Farrell_2022} introduces finite-element(-in-space) implicit timesteppers for the incompressible analogue to this system with structure-preserving (SP) properties in the ideal case, alongside parameter-robust preconditioners. We show that these timesteppers can derive from a finite-element-in-time (FET) (and finite-element-in-space) interpretation. The benefits of this reformulation are discussed, including the derivation of timesteppers that are higher order in time, and the quantifiable dissipative SP properties in the non-ideal, resistive case.
        
        We discuss possible options for extending this FET approach to timesteppers for the compressible case.

        The kinetic corrections satisfy linearized Boltzmann equations. Using a Lénard--Bernstein collision operator, these take Fokker--Planck-like forms \cite{Fokker_1914, Planck_1917} wherein pseudo-particles in the numerical model obey the neoclassical transport equations, with particle-independent Brownian drift terms. This offers a rigorous methodology for incorporating collisions into the particle transport model, without coupling the equations of motions for each particle.
        
        Works by Chen, Chacón et al. \cite{Chen_Chacón_Barnes_2011, Chacón_Chen_Barnes_2013, Chen_Chacón_2014, Chen_Chacón_2015} have developed structure-preserving particle pushers for neoclassical transport in the Vlasov equations, derived from Crank--Nicolson integrators. We show these too can can derive from a FET interpretation, similarly offering potential extensions to higher-order-in-time particle pushers. The FET formulation is used also to consider how the stochastic drift terms can be incorporated into the pushers. Stochastic gyrokinetic expansions are also discussed.

        Different options for the numerical implementation of these schemes are considered.

        Due to the efficacy of FET in the development of SP timesteppers for both the fluid and kinetic component, we hope this approach will prove effective in the future for developing SP timesteppers for the full hybrid model. We hope this will give us the opportunity to incorporate previously inaccessible kinetic effects into the highly effective, modern, finite-element MHD models.
    \end{abstract}
    
    
    \newpage
    \tableofcontents
    
    
    \newpage
    \pagenumbering{arabic}
    %\linenumbers\renewcommand\thelinenumber{\color{black!50}\arabic{linenumber}}
            \input{0 - introduction/main.tex}
        \part{Research}
            \input{1 - low-noise PiC models/main.tex}
            \input{2 - kinetic component/main.tex}
            \input{3 - fluid component/main.tex}
            \input{4 - numerical implementation/main.tex}
        \part{Project Overview}
            \input{5 - research plan/main.tex}
            \input{6 - summary/main.tex}
    
    
    %\section{}
    \newpage
    \pagenumbering{gobble}
        \printbibliography


    \newpage
    \pagenumbering{roman}
    \appendix
        \part{Appendices}
            \input{8 - Hilbert complexes/main.tex}
            \input{9 - weak conservation proofs/main.tex}
\end{document}

\end{document}

            \documentclass[12pt, a4paper]{report}

\documentclass[12pt, a4paper]{report}

\input{template/main.tex}

\title{\BA{Title in Progress...}}
\author{Boris Andrews}
\affil{Mathematical Institute, University of Oxford}
\date{\today}


\begin{document}
    \pagenumbering{gobble}
    \maketitle
    
    
    \begin{abstract}
        Magnetic confinement reactors---in particular tokamaks---offer one of the most promising options for achieving practical nuclear fusion, with the potential to provide virtually limitless, clean energy. The theoretical and numerical modeling of tokamak plasmas is simultaneously an essential component of effective reactor design, and a great research barrier. Tokamak operational conditions exhibit comparatively low Knudsen numbers. Kinetic effects, including kinetic waves and instabilities, Landau damping, bump-on-tail instabilities and more, are therefore highly influential in tokamak plasma dynamics. Purely fluid models are inherently incapable of capturing these effects, whereas the high dimensionality in purely kinetic models render them practically intractable for most relevant purposes.

        We consider a $\delta\!f$ decomposition model, with a macroscopic fluid background and microscopic kinetic correction, both fully coupled to each other. A similar manner of discretization is proposed to that used in the recent \texttt{STRUPHY} code \cite{Holderied_Possanner_Wang_2021, Holderied_2022, Li_et_al_2023} with a finite-element model for the background and a pseudo-particle/PiC model for the correction.

        The fluid background satisfies the full, non-linear, resistive, compressible, Hall MHD equations. \cite{Laakmann_Hu_Farrell_2022} introduces finite-element(-in-space) implicit timesteppers for the incompressible analogue to this system with structure-preserving (SP) properties in the ideal case, alongside parameter-robust preconditioners. We show that these timesteppers can derive from a finite-element-in-time (FET) (and finite-element-in-space) interpretation. The benefits of this reformulation are discussed, including the derivation of timesteppers that are higher order in time, and the quantifiable dissipative SP properties in the non-ideal, resistive case.
        
        We discuss possible options for extending this FET approach to timesteppers for the compressible case.

        The kinetic corrections satisfy linearized Boltzmann equations. Using a Lénard--Bernstein collision operator, these take Fokker--Planck-like forms \cite{Fokker_1914, Planck_1917} wherein pseudo-particles in the numerical model obey the neoclassical transport equations, with particle-independent Brownian drift terms. This offers a rigorous methodology for incorporating collisions into the particle transport model, without coupling the equations of motions for each particle.
        
        Works by Chen, Chacón et al. \cite{Chen_Chacón_Barnes_2011, Chacón_Chen_Barnes_2013, Chen_Chacón_2014, Chen_Chacón_2015} have developed structure-preserving particle pushers for neoclassical transport in the Vlasov equations, derived from Crank--Nicolson integrators. We show these too can can derive from a FET interpretation, similarly offering potential extensions to higher-order-in-time particle pushers. The FET formulation is used also to consider how the stochastic drift terms can be incorporated into the pushers. Stochastic gyrokinetic expansions are also discussed.

        Different options for the numerical implementation of these schemes are considered.

        Due to the efficacy of FET in the development of SP timesteppers for both the fluid and kinetic component, we hope this approach will prove effective in the future for developing SP timesteppers for the full hybrid model. We hope this will give us the opportunity to incorporate previously inaccessible kinetic effects into the highly effective, modern, finite-element MHD models.
    \end{abstract}
    
    
    \newpage
    \tableofcontents
    
    
    \newpage
    \pagenumbering{arabic}
    %\linenumbers\renewcommand\thelinenumber{\color{black!50}\arabic{linenumber}}
            \input{0 - introduction/main.tex}
        \part{Research}
            \input{1 - low-noise PiC models/main.tex}
            \input{2 - kinetic component/main.tex}
            \input{3 - fluid component/main.tex}
            \input{4 - numerical implementation/main.tex}
        \part{Project Overview}
            \input{5 - research plan/main.tex}
            \input{6 - summary/main.tex}
    
    
    %\section{}
    \newpage
    \pagenumbering{gobble}
        \printbibliography


    \newpage
    \pagenumbering{roman}
    \appendix
        \part{Appendices}
            \input{8 - Hilbert complexes/main.tex}
            \input{9 - weak conservation proofs/main.tex}
\end{document}


\title{\BA{Title in Progress...}}
\author{Boris Andrews}
\affil{Mathematical Institute, University of Oxford}
\date{\today}


\begin{document}
    \pagenumbering{gobble}
    \maketitle
    
    
    \begin{abstract}
        Magnetic confinement reactors---in particular tokamaks---offer one of the most promising options for achieving practical nuclear fusion, with the potential to provide virtually limitless, clean energy. The theoretical and numerical modeling of tokamak plasmas is simultaneously an essential component of effective reactor design, and a great research barrier. Tokamak operational conditions exhibit comparatively low Knudsen numbers. Kinetic effects, including kinetic waves and instabilities, Landau damping, bump-on-tail instabilities and more, are therefore highly influential in tokamak plasma dynamics. Purely fluid models are inherently incapable of capturing these effects, whereas the high dimensionality in purely kinetic models render them practically intractable for most relevant purposes.

        We consider a $\delta\!f$ decomposition model, with a macroscopic fluid background and microscopic kinetic correction, both fully coupled to each other. A similar manner of discretization is proposed to that used in the recent \texttt{STRUPHY} code \cite{Holderied_Possanner_Wang_2021, Holderied_2022, Li_et_al_2023} with a finite-element model for the background and a pseudo-particle/PiC model for the correction.

        The fluid background satisfies the full, non-linear, resistive, compressible, Hall MHD equations. \cite{Laakmann_Hu_Farrell_2022} introduces finite-element(-in-space) implicit timesteppers for the incompressible analogue to this system with structure-preserving (SP) properties in the ideal case, alongside parameter-robust preconditioners. We show that these timesteppers can derive from a finite-element-in-time (FET) (and finite-element-in-space) interpretation. The benefits of this reformulation are discussed, including the derivation of timesteppers that are higher order in time, and the quantifiable dissipative SP properties in the non-ideal, resistive case.
        
        We discuss possible options for extending this FET approach to timesteppers for the compressible case.

        The kinetic corrections satisfy linearized Boltzmann equations. Using a Lénard--Bernstein collision operator, these take Fokker--Planck-like forms \cite{Fokker_1914, Planck_1917} wherein pseudo-particles in the numerical model obey the neoclassical transport equations, with particle-independent Brownian drift terms. This offers a rigorous methodology for incorporating collisions into the particle transport model, without coupling the equations of motions for each particle.
        
        Works by Chen, Chacón et al. \cite{Chen_Chacón_Barnes_2011, Chacón_Chen_Barnes_2013, Chen_Chacón_2014, Chen_Chacón_2015} have developed structure-preserving particle pushers for neoclassical transport in the Vlasov equations, derived from Crank--Nicolson integrators. We show these too can can derive from a FET interpretation, similarly offering potential extensions to higher-order-in-time particle pushers. The FET formulation is used also to consider how the stochastic drift terms can be incorporated into the pushers. Stochastic gyrokinetic expansions are also discussed.

        Different options for the numerical implementation of these schemes are considered.

        Due to the efficacy of FET in the development of SP timesteppers for both the fluid and kinetic component, we hope this approach will prove effective in the future for developing SP timesteppers for the full hybrid model. We hope this will give us the opportunity to incorporate previously inaccessible kinetic effects into the highly effective, modern, finite-element MHD models.
    \end{abstract}
    
    
    \newpage
    \tableofcontents
    
    
    \newpage
    \pagenumbering{arabic}
    %\linenumbers\renewcommand\thelinenumber{\color{black!50}\arabic{linenumber}}
            \documentclass[12pt, a4paper]{report}

\input{template/main.tex}

\title{\BA{Title in Progress...}}
\author{Boris Andrews}
\affil{Mathematical Institute, University of Oxford}
\date{\today}


\begin{document}
    \pagenumbering{gobble}
    \maketitle
    
    
    \begin{abstract}
        Magnetic confinement reactors---in particular tokamaks---offer one of the most promising options for achieving practical nuclear fusion, with the potential to provide virtually limitless, clean energy. The theoretical and numerical modeling of tokamak plasmas is simultaneously an essential component of effective reactor design, and a great research barrier. Tokamak operational conditions exhibit comparatively low Knudsen numbers. Kinetic effects, including kinetic waves and instabilities, Landau damping, bump-on-tail instabilities and more, are therefore highly influential in tokamak plasma dynamics. Purely fluid models are inherently incapable of capturing these effects, whereas the high dimensionality in purely kinetic models render them practically intractable for most relevant purposes.

        We consider a $\delta\!f$ decomposition model, with a macroscopic fluid background and microscopic kinetic correction, both fully coupled to each other. A similar manner of discretization is proposed to that used in the recent \texttt{STRUPHY} code \cite{Holderied_Possanner_Wang_2021, Holderied_2022, Li_et_al_2023} with a finite-element model for the background and a pseudo-particle/PiC model for the correction.

        The fluid background satisfies the full, non-linear, resistive, compressible, Hall MHD equations. \cite{Laakmann_Hu_Farrell_2022} introduces finite-element(-in-space) implicit timesteppers for the incompressible analogue to this system with structure-preserving (SP) properties in the ideal case, alongside parameter-robust preconditioners. We show that these timesteppers can derive from a finite-element-in-time (FET) (and finite-element-in-space) interpretation. The benefits of this reformulation are discussed, including the derivation of timesteppers that are higher order in time, and the quantifiable dissipative SP properties in the non-ideal, resistive case.
        
        We discuss possible options for extending this FET approach to timesteppers for the compressible case.

        The kinetic corrections satisfy linearized Boltzmann equations. Using a Lénard--Bernstein collision operator, these take Fokker--Planck-like forms \cite{Fokker_1914, Planck_1917} wherein pseudo-particles in the numerical model obey the neoclassical transport equations, with particle-independent Brownian drift terms. This offers a rigorous methodology for incorporating collisions into the particle transport model, without coupling the equations of motions for each particle.
        
        Works by Chen, Chacón et al. \cite{Chen_Chacón_Barnes_2011, Chacón_Chen_Barnes_2013, Chen_Chacón_2014, Chen_Chacón_2015} have developed structure-preserving particle pushers for neoclassical transport in the Vlasov equations, derived from Crank--Nicolson integrators. We show these too can can derive from a FET interpretation, similarly offering potential extensions to higher-order-in-time particle pushers. The FET formulation is used also to consider how the stochastic drift terms can be incorporated into the pushers. Stochastic gyrokinetic expansions are also discussed.

        Different options for the numerical implementation of these schemes are considered.

        Due to the efficacy of FET in the development of SP timesteppers for both the fluid and kinetic component, we hope this approach will prove effective in the future for developing SP timesteppers for the full hybrid model. We hope this will give us the opportunity to incorporate previously inaccessible kinetic effects into the highly effective, modern, finite-element MHD models.
    \end{abstract}
    
    
    \newpage
    \tableofcontents
    
    
    \newpage
    \pagenumbering{arabic}
    %\linenumbers\renewcommand\thelinenumber{\color{black!50}\arabic{linenumber}}
            \input{0 - introduction/main.tex}
        \part{Research}
            \input{1 - low-noise PiC models/main.tex}
            \input{2 - kinetic component/main.tex}
            \input{3 - fluid component/main.tex}
            \input{4 - numerical implementation/main.tex}
        \part{Project Overview}
            \input{5 - research plan/main.tex}
            \input{6 - summary/main.tex}
    
    
    %\section{}
    \newpage
    \pagenumbering{gobble}
        \printbibliography


    \newpage
    \pagenumbering{roman}
    \appendix
        \part{Appendices}
            \input{8 - Hilbert complexes/main.tex}
            \input{9 - weak conservation proofs/main.tex}
\end{document}

        \part{Research}
            \documentclass[12pt, a4paper]{report}

\input{template/main.tex}

\title{\BA{Title in Progress...}}
\author{Boris Andrews}
\affil{Mathematical Institute, University of Oxford}
\date{\today}


\begin{document}
    \pagenumbering{gobble}
    \maketitle
    
    
    \begin{abstract}
        Magnetic confinement reactors---in particular tokamaks---offer one of the most promising options for achieving practical nuclear fusion, with the potential to provide virtually limitless, clean energy. The theoretical and numerical modeling of tokamak plasmas is simultaneously an essential component of effective reactor design, and a great research barrier. Tokamak operational conditions exhibit comparatively low Knudsen numbers. Kinetic effects, including kinetic waves and instabilities, Landau damping, bump-on-tail instabilities and more, are therefore highly influential in tokamak plasma dynamics. Purely fluid models are inherently incapable of capturing these effects, whereas the high dimensionality in purely kinetic models render them practically intractable for most relevant purposes.

        We consider a $\delta\!f$ decomposition model, with a macroscopic fluid background and microscopic kinetic correction, both fully coupled to each other. A similar manner of discretization is proposed to that used in the recent \texttt{STRUPHY} code \cite{Holderied_Possanner_Wang_2021, Holderied_2022, Li_et_al_2023} with a finite-element model for the background and a pseudo-particle/PiC model for the correction.

        The fluid background satisfies the full, non-linear, resistive, compressible, Hall MHD equations. \cite{Laakmann_Hu_Farrell_2022} introduces finite-element(-in-space) implicit timesteppers for the incompressible analogue to this system with structure-preserving (SP) properties in the ideal case, alongside parameter-robust preconditioners. We show that these timesteppers can derive from a finite-element-in-time (FET) (and finite-element-in-space) interpretation. The benefits of this reformulation are discussed, including the derivation of timesteppers that are higher order in time, and the quantifiable dissipative SP properties in the non-ideal, resistive case.
        
        We discuss possible options for extending this FET approach to timesteppers for the compressible case.

        The kinetic corrections satisfy linearized Boltzmann equations. Using a Lénard--Bernstein collision operator, these take Fokker--Planck-like forms \cite{Fokker_1914, Planck_1917} wherein pseudo-particles in the numerical model obey the neoclassical transport equations, with particle-independent Brownian drift terms. This offers a rigorous methodology for incorporating collisions into the particle transport model, without coupling the equations of motions for each particle.
        
        Works by Chen, Chacón et al. \cite{Chen_Chacón_Barnes_2011, Chacón_Chen_Barnes_2013, Chen_Chacón_2014, Chen_Chacón_2015} have developed structure-preserving particle pushers for neoclassical transport in the Vlasov equations, derived from Crank--Nicolson integrators. We show these too can can derive from a FET interpretation, similarly offering potential extensions to higher-order-in-time particle pushers. The FET formulation is used also to consider how the stochastic drift terms can be incorporated into the pushers. Stochastic gyrokinetic expansions are also discussed.

        Different options for the numerical implementation of these schemes are considered.

        Due to the efficacy of FET in the development of SP timesteppers for both the fluid and kinetic component, we hope this approach will prove effective in the future for developing SP timesteppers for the full hybrid model. We hope this will give us the opportunity to incorporate previously inaccessible kinetic effects into the highly effective, modern, finite-element MHD models.
    \end{abstract}
    
    
    \newpage
    \tableofcontents
    
    
    \newpage
    \pagenumbering{arabic}
    %\linenumbers\renewcommand\thelinenumber{\color{black!50}\arabic{linenumber}}
            \input{0 - introduction/main.tex}
        \part{Research}
            \input{1 - low-noise PiC models/main.tex}
            \input{2 - kinetic component/main.tex}
            \input{3 - fluid component/main.tex}
            \input{4 - numerical implementation/main.tex}
        \part{Project Overview}
            \input{5 - research plan/main.tex}
            \input{6 - summary/main.tex}
    
    
    %\section{}
    \newpage
    \pagenumbering{gobble}
        \printbibliography


    \newpage
    \pagenumbering{roman}
    \appendix
        \part{Appendices}
            \input{8 - Hilbert complexes/main.tex}
            \input{9 - weak conservation proofs/main.tex}
\end{document}

            \documentclass[12pt, a4paper]{report}

\input{template/main.tex}

\title{\BA{Title in Progress...}}
\author{Boris Andrews}
\affil{Mathematical Institute, University of Oxford}
\date{\today}


\begin{document}
    \pagenumbering{gobble}
    \maketitle
    
    
    \begin{abstract}
        Magnetic confinement reactors---in particular tokamaks---offer one of the most promising options for achieving practical nuclear fusion, with the potential to provide virtually limitless, clean energy. The theoretical and numerical modeling of tokamak plasmas is simultaneously an essential component of effective reactor design, and a great research barrier. Tokamak operational conditions exhibit comparatively low Knudsen numbers. Kinetic effects, including kinetic waves and instabilities, Landau damping, bump-on-tail instabilities and more, are therefore highly influential in tokamak plasma dynamics. Purely fluid models are inherently incapable of capturing these effects, whereas the high dimensionality in purely kinetic models render them practically intractable for most relevant purposes.

        We consider a $\delta\!f$ decomposition model, with a macroscopic fluid background and microscopic kinetic correction, both fully coupled to each other. A similar manner of discretization is proposed to that used in the recent \texttt{STRUPHY} code \cite{Holderied_Possanner_Wang_2021, Holderied_2022, Li_et_al_2023} with a finite-element model for the background and a pseudo-particle/PiC model for the correction.

        The fluid background satisfies the full, non-linear, resistive, compressible, Hall MHD equations. \cite{Laakmann_Hu_Farrell_2022} introduces finite-element(-in-space) implicit timesteppers for the incompressible analogue to this system with structure-preserving (SP) properties in the ideal case, alongside parameter-robust preconditioners. We show that these timesteppers can derive from a finite-element-in-time (FET) (and finite-element-in-space) interpretation. The benefits of this reformulation are discussed, including the derivation of timesteppers that are higher order in time, and the quantifiable dissipative SP properties in the non-ideal, resistive case.
        
        We discuss possible options for extending this FET approach to timesteppers for the compressible case.

        The kinetic corrections satisfy linearized Boltzmann equations. Using a Lénard--Bernstein collision operator, these take Fokker--Planck-like forms \cite{Fokker_1914, Planck_1917} wherein pseudo-particles in the numerical model obey the neoclassical transport equations, with particle-independent Brownian drift terms. This offers a rigorous methodology for incorporating collisions into the particle transport model, without coupling the equations of motions for each particle.
        
        Works by Chen, Chacón et al. \cite{Chen_Chacón_Barnes_2011, Chacón_Chen_Barnes_2013, Chen_Chacón_2014, Chen_Chacón_2015} have developed structure-preserving particle pushers for neoclassical transport in the Vlasov equations, derived from Crank--Nicolson integrators. We show these too can can derive from a FET interpretation, similarly offering potential extensions to higher-order-in-time particle pushers. The FET formulation is used also to consider how the stochastic drift terms can be incorporated into the pushers. Stochastic gyrokinetic expansions are also discussed.

        Different options for the numerical implementation of these schemes are considered.

        Due to the efficacy of FET in the development of SP timesteppers for both the fluid and kinetic component, we hope this approach will prove effective in the future for developing SP timesteppers for the full hybrid model. We hope this will give us the opportunity to incorporate previously inaccessible kinetic effects into the highly effective, modern, finite-element MHD models.
    \end{abstract}
    
    
    \newpage
    \tableofcontents
    
    
    \newpage
    \pagenumbering{arabic}
    %\linenumbers\renewcommand\thelinenumber{\color{black!50}\arabic{linenumber}}
            \input{0 - introduction/main.tex}
        \part{Research}
            \input{1 - low-noise PiC models/main.tex}
            \input{2 - kinetic component/main.tex}
            \input{3 - fluid component/main.tex}
            \input{4 - numerical implementation/main.tex}
        \part{Project Overview}
            \input{5 - research plan/main.tex}
            \input{6 - summary/main.tex}
    
    
    %\section{}
    \newpage
    \pagenumbering{gobble}
        \printbibliography


    \newpage
    \pagenumbering{roman}
    \appendix
        \part{Appendices}
            \input{8 - Hilbert complexes/main.tex}
            \input{9 - weak conservation proofs/main.tex}
\end{document}

            \documentclass[12pt, a4paper]{report}

\input{template/main.tex}

\title{\BA{Title in Progress...}}
\author{Boris Andrews}
\affil{Mathematical Institute, University of Oxford}
\date{\today}


\begin{document}
    \pagenumbering{gobble}
    \maketitle
    
    
    \begin{abstract}
        Magnetic confinement reactors---in particular tokamaks---offer one of the most promising options for achieving practical nuclear fusion, with the potential to provide virtually limitless, clean energy. The theoretical and numerical modeling of tokamak plasmas is simultaneously an essential component of effective reactor design, and a great research barrier. Tokamak operational conditions exhibit comparatively low Knudsen numbers. Kinetic effects, including kinetic waves and instabilities, Landau damping, bump-on-tail instabilities and more, are therefore highly influential in tokamak plasma dynamics. Purely fluid models are inherently incapable of capturing these effects, whereas the high dimensionality in purely kinetic models render them practically intractable for most relevant purposes.

        We consider a $\delta\!f$ decomposition model, with a macroscopic fluid background and microscopic kinetic correction, both fully coupled to each other. A similar manner of discretization is proposed to that used in the recent \texttt{STRUPHY} code \cite{Holderied_Possanner_Wang_2021, Holderied_2022, Li_et_al_2023} with a finite-element model for the background and a pseudo-particle/PiC model for the correction.

        The fluid background satisfies the full, non-linear, resistive, compressible, Hall MHD equations. \cite{Laakmann_Hu_Farrell_2022} introduces finite-element(-in-space) implicit timesteppers for the incompressible analogue to this system with structure-preserving (SP) properties in the ideal case, alongside parameter-robust preconditioners. We show that these timesteppers can derive from a finite-element-in-time (FET) (and finite-element-in-space) interpretation. The benefits of this reformulation are discussed, including the derivation of timesteppers that are higher order in time, and the quantifiable dissipative SP properties in the non-ideal, resistive case.
        
        We discuss possible options for extending this FET approach to timesteppers for the compressible case.

        The kinetic corrections satisfy linearized Boltzmann equations. Using a Lénard--Bernstein collision operator, these take Fokker--Planck-like forms \cite{Fokker_1914, Planck_1917} wherein pseudo-particles in the numerical model obey the neoclassical transport equations, with particle-independent Brownian drift terms. This offers a rigorous methodology for incorporating collisions into the particle transport model, without coupling the equations of motions for each particle.
        
        Works by Chen, Chacón et al. \cite{Chen_Chacón_Barnes_2011, Chacón_Chen_Barnes_2013, Chen_Chacón_2014, Chen_Chacón_2015} have developed structure-preserving particle pushers for neoclassical transport in the Vlasov equations, derived from Crank--Nicolson integrators. We show these too can can derive from a FET interpretation, similarly offering potential extensions to higher-order-in-time particle pushers. The FET formulation is used also to consider how the stochastic drift terms can be incorporated into the pushers. Stochastic gyrokinetic expansions are also discussed.

        Different options for the numerical implementation of these schemes are considered.

        Due to the efficacy of FET in the development of SP timesteppers for both the fluid and kinetic component, we hope this approach will prove effective in the future for developing SP timesteppers for the full hybrid model. We hope this will give us the opportunity to incorporate previously inaccessible kinetic effects into the highly effective, modern, finite-element MHD models.
    \end{abstract}
    
    
    \newpage
    \tableofcontents
    
    
    \newpage
    \pagenumbering{arabic}
    %\linenumbers\renewcommand\thelinenumber{\color{black!50}\arabic{linenumber}}
            \input{0 - introduction/main.tex}
        \part{Research}
            \input{1 - low-noise PiC models/main.tex}
            \input{2 - kinetic component/main.tex}
            \input{3 - fluid component/main.tex}
            \input{4 - numerical implementation/main.tex}
        \part{Project Overview}
            \input{5 - research plan/main.tex}
            \input{6 - summary/main.tex}
    
    
    %\section{}
    \newpage
    \pagenumbering{gobble}
        \printbibliography


    \newpage
    \pagenumbering{roman}
    \appendix
        \part{Appendices}
            \input{8 - Hilbert complexes/main.tex}
            \input{9 - weak conservation proofs/main.tex}
\end{document}

            \documentclass[12pt, a4paper]{report}

\input{template/main.tex}

\title{\BA{Title in Progress...}}
\author{Boris Andrews}
\affil{Mathematical Institute, University of Oxford}
\date{\today}


\begin{document}
    \pagenumbering{gobble}
    \maketitle
    
    
    \begin{abstract}
        Magnetic confinement reactors---in particular tokamaks---offer one of the most promising options for achieving practical nuclear fusion, with the potential to provide virtually limitless, clean energy. The theoretical and numerical modeling of tokamak plasmas is simultaneously an essential component of effective reactor design, and a great research barrier. Tokamak operational conditions exhibit comparatively low Knudsen numbers. Kinetic effects, including kinetic waves and instabilities, Landau damping, bump-on-tail instabilities and more, are therefore highly influential in tokamak plasma dynamics. Purely fluid models are inherently incapable of capturing these effects, whereas the high dimensionality in purely kinetic models render them practically intractable for most relevant purposes.

        We consider a $\delta\!f$ decomposition model, with a macroscopic fluid background and microscopic kinetic correction, both fully coupled to each other. A similar manner of discretization is proposed to that used in the recent \texttt{STRUPHY} code \cite{Holderied_Possanner_Wang_2021, Holderied_2022, Li_et_al_2023} with a finite-element model for the background and a pseudo-particle/PiC model for the correction.

        The fluid background satisfies the full, non-linear, resistive, compressible, Hall MHD equations. \cite{Laakmann_Hu_Farrell_2022} introduces finite-element(-in-space) implicit timesteppers for the incompressible analogue to this system with structure-preserving (SP) properties in the ideal case, alongside parameter-robust preconditioners. We show that these timesteppers can derive from a finite-element-in-time (FET) (and finite-element-in-space) interpretation. The benefits of this reformulation are discussed, including the derivation of timesteppers that are higher order in time, and the quantifiable dissipative SP properties in the non-ideal, resistive case.
        
        We discuss possible options for extending this FET approach to timesteppers for the compressible case.

        The kinetic corrections satisfy linearized Boltzmann equations. Using a Lénard--Bernstein collision operator, these take Fokker--Planck-like forms \cite{Fokker_1914, Planck_1917} wherein pseudo-particles in the numerical model obey the neoclassical transport equations, with particle-independent Brownian drift terms. This offers a rigorous methodology for incorporating collisions into the particle transport model, without coupling the equations of motions for each particle.
        
        Works by Chen, Chacón et al. \cite{Chen_Chacón_Barnes_2011, Chacón_Chen_Barnes_2013, Chen_Chacón_2014, Chen_Chacón_2015} have developed structure-preserving particle pushers for neoclassical transport in the Vlasov equations, derived from Crank--Nicolson integrators. We show these too can can derive from a FET interpretation, similarly offering potential extensions to higher-order-in-time particle pushers. The FET formulation is used also to consider how the stochastic drift terms can be incorporated into the pushers. Stochastic gyrokinetic expansions are also discussed.

        Different options for the numerical implementation of these schemes are considered.

        Due to the efficacy of FET in the development of SP timesteppers for both the fluid and kinetic component, we hope this approach will prove effective in the future for developing SP timesteppers for the full hybrid model. We hope this will give us the opportunity to incorporate previously inaccessible kinetic effects into the highly effective, modern, finite-element MHD models.
    \end{abstract}
    
    
    \newpage
    \tableofcontents
    
    
    \newpage
    \pagenumbering{arabic}
    %\linenumbers\renewcommand\thelinenumber{\color{black!50}\arabic{linenumber}}
            \input{0 - introduction/main.tex}
        \part{Research}
            \input{1 - low-noise PiC models/main.tex}
            \input{2 - kinetic component/main.tex}
            \input{3 - fluid component/main.tex}
            \input{4 - numerical implementation/main.tex}
        \part{Project Overview}
            \input{5 - research plan/main.tex}
            \input{6 - summary/main.tex}
    
    
    %\section{}
    \newpage
    \pagenumbering{gobble}
        \printbibliography


    \newpage
    \pagenumbering{roman}
    \appendix
        \part{Appendices}
            \input{8 - Hilbert complexes/main.tex}
            \input{9 - weak conservation proofs/main.tex}
\end{document}

        \part{Project Overview}
            \documentclass[12pt, a4paper]{report}

\input{template/main.tex}

\title{\BA{Title in Progress...}}
\author{Boris Andrews}
\affil{Mathematical Institute, University of Oxford}
\date{\today}


\begin{document}
    \pagenumbering{gobble}
    \maketitle
    
    
    \begin{abstract}
        Magnetic confinement reactors---in particular tokamaks---offer one of the most promising options for achieving practical nuclear fusion, with the potential to provide virtually limitless, clean energy. The theoretical and numerical modeling of tokamak plasmas is simultaneously an essential component of effective reactor design, and a great research barrier. Tokamak operational conditions exhibit comparatively low Knudsen numbers. Kinetic effects, including kinetic waves and instabilities, Landau damping, bump-on-tail instabilities and more, are therefore highly influential in tokamak plasma dynamics. Purely fluid models are inherently incapable of capturing these effects, whereas the high dimensionality in purely kinetic models render them practically intractable for most relevant purposes.

        We consider a $\delta\!f$ decomposition model, with a macroscopic fluid background and microscopic kinetic correction, both fully coupled to each other. A similar manner of discretization is proposed to that used in the recent \texttt{STRUPHY} code \cite{Holderied_Possanner_Wang_2021, Holderied_2022, Li_et_al_2023} with a finite-element model for the background and a pseudo-particle/PiC model for the correction.

        The fluid background satisfies the full, non-linear, resistive, compressible, Hall MHD equations. \cite{Laakmann_Hu_Farrell_2022} introduces finite-element(-in-space) implicit timesteppers for the incompressible analogue to this system with structure-preserving (SP) properties in the ideal case, alongside parameter-robust preconditioners. We show that these timesteppers can derive from a finite-element-in-time (FET) (and finite-element-in-space) interpretation. The benefits of this reformulation are discussed, including the derivation of timesteppers that are higher order in time, and the quantifiable dissipative SP properties in the non-ideal, resistive case.
        
        We discuss possible options for extending this FET approach to timesteppers for the compressible case.

        The kinetic corrections satisfy linearized Boltzmann equations. Using a Lénard--Bernstein collision operator, these take Fokker--Planck-like forms \cite{Fokker_1914, Planck_1917} wherein pseudo-particles in the numerical model obey the neoclassical transport equations, with particle-independent Brownian drift terms. This offers a rigorous methodology for incorporating collisions into the particle transport model, without coupling the equations of motions for each particle.
        
        Works by Chen, Chacón et al. \cite{Chen_Chacón_Barnes_2011, Chacón_Chen_Barnes_2013, Chen_Chacón_2014, Chen_Chacón_2015} have developed structure-preserving particle pushers for neoclassical transport in the Vlasov equations, derived from Crank--Nicolson integrators. We show these too can can derive from a FET interpretation, similarly offering potential extensions to higher-order-in-time particle pushers. The FET formulation is used also to consider how the stochastic drift terms can be incorporated into the pushers. Stochastic gyrokinetic expansions are also discussed.

        Different options for the numerical implementation of these schemes are considered.

        Due to the efficacy of FET in the development of SP timesteppers for both the fluid and kinetic component, we hope this approach will prove effective in the future for developing SP timesteppers for the full hybrid model. We hope this will give us the opportunity to incorporate previously inaccessible kinetic effects into the highly effective, modern, finite-element MHD models.
    \end{abstract}
    
    
    \newpage
    \tableofcontents
    
    
    \newpage
    \pagenumbering{arabic}
    %\linenumbers\renewcommand\thelinenumber{\color{black!50}\arabic{linenumber}}
            \input{0 - introduction/main.tex}
        \part{Research}
            \input{1 - low-noise PiC models/main.tex}
            \input{2 - kinetic component/main.tex}
            \input{3 - fluid component/main.tex}
            \input{4 - numerical implementation/main.tex}
        \part{Project Overview}
            \input{5 - research plan/main.tex}
            \input{6 - summary/main.tex}
    
    
    %\section{}
    \newpage
    \pagenumbering{gobble}
        \printbibliography


    \newpage
    \pagenumbering{roman}
    \appendix
        \part{Appendices}
            \input{8 - Hilbert complexes/main.tex}
            \input{9 - weak conservation proofs/main.tex}
\end{document}

            \documentclass[12pt, a4paper]{report}

\input{template/main.tex}

\title{\BA{Title in Progress...}}
\author{Boris Andrews}
\affil{Mathematical Institute, University of Oxford}
\date{\today}


\begin{document}
    \pagenumbering{gobble}
    \maketitle
    
    
    \begin{abstract}
        Magnetic confinement reactors---in particular tokamaks---offer one of the most promising options for achieving practical nuclear fusion, with the potential to provide virtually limitless, clean energy. The theoretical and numerical modeling of tokamak plasmas is simultaneously an essential component of effective reactor design, and a great research barrier. Tokamak operational conditions exhibit comparatively low Knudsen numbers. Kinetic effects, including kinetic waves and instabilities, Landau damping, bump-on-tail instabilities and more, are therefore highly influential in tokamak plasma dynamics. Purely fluid models are inherently incapable of capturing these effects, whereas the high dimensionality in purely kinetic models render them practically intractable for most relevant purposes.

        We consider a $\delta\!f$ decomposition model, with a macroscopic fluid background and microscopic kinetic correction, both fully coupled to each other. A similar manner of discretization is proposed to that used in the recent \texttt{STRUPHY} code \cite{Holderied_Possanner_Wang_2021, Holderied_2022, Li_et_al_2023} with a finite-element model for the background and a pseudo-particle/PiC model for the correction.

        The fluid background satisfies the full, non-linear, resistive, compressible, Hall MHD equations. \cite{Laakmann_Hu_Farrell_2022} introduces finite-element(-in-space) implicit timesteppers for the incompressible analogue to this system with structure-preserving (SP) properties in the ideal case, alongside parameter-robust preconditioners. We show that these timesteppers can derive from a finite-element-in-time (FET) (and finite-element-in-space) interpretation. The benefits of this reformulation are discussed, including the derivation of timesteppers that are higher order in time, and the quantifiable dissipative SP properties in the non-ideal, resistive case.
        
        We discuss possible options for extending this FET approach to timesteppers for the compressible case.

        The kinetic corrections satisfy linearized Boltzmann equations. Using a Lénard--Bernstein collision operator, these take Fokker--Planck-like forms \cite{Fokker_1914, Planck_1917} wherein pseudo-particles in the numerical model obey the neoclassical transport equations, with particle-independent Brownian drift terms. This offers a rigorous methodology for incorporating collisions into the particle transport model, without coupling the equations of motions for each particle.
        
        Works by Chen, Chacón et al. \cite{Chen_Chacón_Barnes_2011, Chacón_Chen_Barnes_2013, Chen_Chacón_2014, Chen_Chacón_2015} have developed structure-preserving particle pushers for neoclassical transport in the Vlasov equations, derived from Crank--Nicolson integrators. We show these too can can derive from a FET interpretation, similarly offering potential extensions to higher-order-in-time particle pushers. The FET formulation is used also to consider how the stochastic drift terms can be incorporated into the pushers. Stochastic gyrokinetic expansions are also discussed.

        Different options for the numerical implementation of these schemes are considered.

        Due to the efficacy of FET in the development of SP timesteppers for both the fluid and kinetic component, we hope this approach will prove effective in the future for developing SP timesteppers for the full hybrid model. We hope this will give us the opportunity to incorporate previously inaccessible kinetic effects into the highly effective, modern, finite-element MHD models.
    \end{abstract}
    
    
    \newpage
    \tableofcontents
    
    
    \newpage
    \pagenumbering{arabic}
    %\linenumbers\renewcommand\thelinenumber{\color{black!50}\arabic{linenumber}}
            \input{0 - introduction/main.tex}
        \part{Research}
            \input{1 - low-noise PiC models/main.tex}
            \input{2 - kinetic component/main.tex}
            \input{3 - fluid component/main.tex}
            \input{4 - numerical implementation/main.tex}
        \part{Project Overview}
            \input{5 - research plan/main.tex}
            \input{6 - summary/main.tex}
    
    
    %\section{}
    \newpage
    \pagenumbering{gobble}
        \printbibliography


    \newpage
    \pagenumbering{roman}
    \appendix
        \part{Appendices}
            \input{8 - Hilbert complexes/main.tex}
            \input{9 - weak conservation proofs/main.tex}
\end{document}

    
    
    %\section{}
    \newpage
    \pagenumbering{gobble}
        \printbibliography


    \newpage
    \pagenumbering{roman}
    \appendix
        \part{Appendices}
            \documentclass[12pt, a4paper]{report}

\input{template/main.tex}

\title{\BA{Title in Progress...}}
\author{Boris Andrews}
\affil{Mathematical Institute, University of Oxford}
\date{\today}


\begin{document}
    \pagenumbering{gobble}
    \maketitle
    
    
    \begin{abstract}
        Magnetic confinement reactors---in particular tokamaks---offer one of the most promising options for achieving practical nuclear fusion, with the potential to provide virtually limitless, clean energy. The theoretical and numerical modeling of tokamak plasmas is simultaneously an essential component of effective reactor design, and a great research barrier. Tokamak operational conditions exhibit comparatively low Knudsen numbers. Kinetic effects, including kinetic waves and instabilities, Landau damping, bump-on-tail instabilities and more, are therefore highly influential in tokamak plasma dynamics. Purely fluid models are inherently incapable of capturing these effects, whereas the high dimensionality in purely kinetic models render them practically intractable for most relevant purposes.

        We consider a $\delta\!f$ decomposition model, with a macroscopic fluid background and microscopic kinetic correction, both fully coupled to each other. A similar manner of discretization is proposed to that used in the recent \texttt{STRUPHY} code \cite{Holderied_Possanner_Wang_2021, Holderied_2022, Li_et_al_2023} with a finite-element model for the background and a pseudo-particle/PiC model for the correction.

        The fluid background satisfies the full, non-linear, resistive, compressible, Hall MHD equations. \cite{Laakmann_Hu_Farrell_2022} introduces finite-element(-in-space) implicit timesteppers for the incompressible analogue to this system with structure-preserving (SP) properties in the ideal case, alongside parameter-robust preconditioners. We show that these timesteppers can derive from a finite-element-in-time (FET) (and finite-element-in-space) interpretation. The benefits of this reformulation are discussed, including the derivation of timesteppers that are higher order in time, and the quantifiable dissipative SP properties in the non-ideal, resistive case.
        
        We discuss possible options for extending this FET approach to timesteppers for the compressible case.

        The kinetic corrections satisfy linearized Boltzmann equations. Using a Lénard--Bernstein collision operator, these take Fokker--Planck-like forms \cite{Fokker_1914, Planck_1917} wherein pseudo-particles in the numerical model obey the neoclassical transport equations, with particle-independent Brownian drift terms. This offers a rigorous methodology for incorporating collisions into the particle transport model, without coupling the equations of motions for each particle.
        
        Works by Chen, Chacón et al. \cite{Chen_Chacón_Barnes_2011, Chacón_Chen_Barnes_2013, Chen_Chacón_2014, Chen_Chacón_2015} have developed structure-preserving particle pushers for neoclassical transport in the Vlasov equations, derived from Crank--Nicolson integrators. We show these too can can derive from a FET interpretation, similarly offering potential extensions to higher-order-in-time particle pushers. The FET formulation is used also to consider how the stochastic drift terms can be incorporated into the pushers. Stochastic gyrokinetic expansions are also discussed.

        Different options for the numerical implementation of these schemes are considered.

        Due to the efficacy of FET in the development of SP timesteppers for both the fluid and kinetic component, we hope this approach will prove effective in the future for developing SP timesteppers for the full hybrid model. We hope this will give us the opportunity to incorporate previously inaccessible kinetic effects into the highly effective, modern, finite-element MHD models.
    \end{abstract}
    
    
    \newpage
    \tableofcontents
    
    
    \newpage
    \pagenumbering{arabic}
    %\linenumbers\renewcommand\thelinenumber{\color{black!50}\arabic{linenumber}}
            \input{0 - introduction/main.tex}
        \part{Research}
            \input{1 - low-noise PiC models/main.tex}
            \input{2 - kinetic component/main.tex}
            \input{3 - fluid component/main.tex}
            \input{4 - numerical implementation/main.tex}
        \part{Project Overview}
            \input{5 - research plan/main.tex}
            \input{6 - summary/main.tex}
    
    
    %\section{}
    \newpage
    \pagenumbering{gobble}
        \printbibliography


    \newpage
    \pagenumbering{roman}
    \appendix
        \part{Appendices}
            \input{8 - Hilbert complexes/main.tex}
            \input{9 - weak conservation proofs/main.tex}
\end{document}

            \documentclass[12pt, a4paper]{report}

\input{template/main.tex}

\title{\BA{Title in Progress...}}
\author{Boris Andrews}
\affil{Mathematical Institute, University of Oxford}
\date{\today}


\begin{document}
    \pagenumbering{gobble}
    \maketitle
    
    
    \begin{abstract}
        Magnetic confinement reactors---in particular tokamaks---offer one of the most promising options for achieving practical nuclear fusion, with the potential to provide virtually limitless, clean energy. The theoretical and numerical modeling of tokamak plasmas is simultaneously an essential component of effective reactor design, and a great research barrier. Tokamak operational conditions exhibit comparatively low Knudsen numbers. Kinetic effects, including kinetic waves and instabilities, Landau damping, bump-on-tail instabilities and more, are therefore highly influential in tokamak plasma dynamics. Purely fluid models are inherently incapable of capturing these effects, whereas the high dimensionality in purely kinetic models render them practically intractable for most relevant purposes.

        We consider a $\delta\!f$ decomposition model, with a macroscopic fluid background and microscopic kinetic correction, both fully coupled to each other. A similar manner of discretization is proposed to that used in the recent \texttt{STRUPHY} code \cite{Holderied_Possanner_Wang_2021, Holderied_2022, Li_et_al_2023} with a finite-element model for the background and a pseudo-particle/PiC model for the correction.

        The fluid background satisfies the full, non-linear, resistive, compressible, Hall MHD equations. \cite{Laakmann_Hu_Farrell_2022} introduces finite-element(-in-space) implicit timesteppers for the incompressible analogue to this system with structure-preserving (SP) properties in the ideal case, alongside parameter-robust preconditioners. We show that these timesteppers can derive from a finite-element-in-time (FET) (and finite-element-in-space) interpretation. The benefits of this reformulation are discussed, including the derivation of timesteppers that are higher order in time, and the quantifiable dissipative SP properties in the non-ideal, resistive case.
        
        We discuss possible options for extending this FET approach to timesteppers for the compressible case.

        The kinetic corrections satisfy linearized Boltzmann equations. Using a Lénard--Bernstein collision operator, these take Fokker--Planck-like forms \cite{Fokker_1914, Planck_1917} wherein pseudo-particles in the numerical model obey the neoclassical transport equations, with particle-independent Brownian drift terms. This offers a rigorous methodology for incorporating collisions into the particle transport model, without coupling the equations of motions for each particle.
        
        Works by Chen, Chacón et al. \cite{Chen_Chacón_Barnes_2011, Chacón_Chen_Barnes_2013, Chen_Chacón_2014, Chen_Chacón_2015} have developed structure-preserving particle pushers for neoclassical transport in the Vlasov equations, derived from Crank--Nicolson integrators. We show these too can can derive from a FET interpretation, similarly offering potential extensions to higher-order-in-time particle pushers. The FET formulation is used also to consider how the stochastic drift terms can be incorporated into the pushers. Stochastic gyrokinetic expansions are also discussed.

        Different options for the numerical implementation of these schemes are considered.

        Due to the efficacy of FET in the development of SP timesteppers for both the fluid and kinetic component, we hope this approach will prove effective in the future for developing SP timesteppers for the full hybrid model. We hope this will give us the opportunity to incorporate previously inaccessible kinetic effects into the highly effective, modern, finite-element MHD models.
    \end{abstract}
    
    
    \newpage
    \tableofcontents
    
    
    \newpage
    \pagenumbering{arabic}
    %\linenumbers\renewcommand\thelinenumber{\color{black!50}\arabic{linenumber}}
            \input{0 - introduction/main.tex}
        \part{Research}
            \input{1 - low-noise PiC models/main.tex}
            \input{2 - kinetic component/main.tex}
            \input{3 - fluid component/main.tex}
            \input{4 - numerical implementation/main.tex}
        \part{Project Overview}
            \input{5 - research plan/main.tex}
            \input{6 - summary/main.tex}
    
    
    %\section{}
    \newpage
    \pagenumbering{gobble}
        \printbibliography


    \newpage
    \pagenumbering{roman}
    \appendix
        \part{Appendices}
            \input{8 - Hilbert complexes/main.tex}
            \input{9 - weak conservation proofs/main.tex}
\end{document}

\end{document}

            \documentclass[12pt, a4paper]{report}

\documentclass[12pt, a4paper]{report}

\input{template/main.tex}

\title{\BA{Title in Progress...}}
\author{Boris Andrews}
\affil{Mathematical Institute, University of Oxford}
\date{\today}


\begin{document}
    \pagenumbering{gobble}
    \maketitle
    
    
    \begin{abstract}
        Magnetic confinement reactors---in particular tokamaks---offer one of the most promising options for achieving practical nuclear fusion, with the potential to provide virtually limitless, clean energy. The theoretical and numerical modeling of tokamak plasmas is simultaneously an essential component of effective reactor design, and a great research barrier. Tokamak operational conditions exhibit comparatively low Knudsen numbers. Kinetic effects, including kinetic waves and instabilities, Landau damping, bump-on-tail instabilities and more, are therefore highly influential in tokamak plasma dynamics. Purely fluid models are inherently incapable of capturing these effects, whereas the high dimensionality in purely kinetic models render them practically intractable for most relevant purposes.

        We consider a $\delta\!f$ decomposition model, with a macroscopic fluid background and microscopic kinetic correction, both fully coupled to each other. A similar manner of discretization is proposed to that used in the recent \texttt{STRUPHY} code \cite{Holderied_Possanner_Wang_2021, Holderied_2022, Li_et_al_2023} with a finite-element model for the background and a pseudo-particle/PiC model for the correction.

        The fluid background satisfies the full, non-linear, resistive, compressible, Hall MHD equations. \cite{Laakmann_Hu_Farrell_2022} introduces finite-element(-in-space) implicit timesteppers for the incompressible analogue to this system with structure-preserving (SP) properties in the ideal case, alongside parameter-robust preconditioners. We show that these timesteppers can derive from a finite-element-in-time (FET) (and finite-element-in-space) interpretation. The benefits of this reformulation are discussed, including the derivation of timesteppers that are higher order in time, and the quantifiable dissipative SP properties in the non-ideal, resistive case.
        
        We discuss possible options for extending this FET approach to timesteppers for the compressible case.

        The kinetic corrections satisfy linearized Boltzmann equations. Using a Lénard--Bernstein collision operator, these take Fokker--Planck-like forms \cite{Fokker_1914, Planck_1917} wherein pseudo-particles in the numerical model obey the neoclassical transport equations, with particle-independent Brownian drift terms. This offers a rigorous methodology for incorporating collisions into the particle transport model, without coupling the equations of motions for each particle.
        
        Works by Chen, Chacón et al. \cite{Chen_Chacón_Barnes_2011, Chacón_Chen_Barnes_2013, Chen_Chacón_2014, Chen_Chacón_2015} have developed structure-preserving particle pushers for neoclassical transport in the Vlasov equations, derived from Crank--Nicolson integrators. We show these too can can derive from a FET interpretation, similarly offering potential extensions to higher-order-in-time particle pushers. The FET formulation is used also to consider how the stochastic drift terms can be incorporated into the pushers. Stochastic gyrokinetic expansions are also discussed.

        Different options for the numerical implementation of these schemes are considered.

        Due to the efficacy of FET in the development of SP timesteppers for both the fluid and kinetic component, we hope this approach will prove effective in the future for developing SP timesteppers for the full hybrid model. We hope this will give us the opportunity to incorporate previously inaccessible kinetic effects into the highly effective, modern, finite-element MHD models.
    \end{abstract}
    
    
    \newpage
    \tableofcontents
    
    
    \newpage
    \pagenumbering{arabic}
    %\linenumbers\renewcommand\thelinenumber{\color{black!50}\arabic{linenumber}}
            \input{0 - introduction/main.tex}
        \part{Research}
            \input{1 - low-noise PiC models/main.tex}
            \input{2 - kinetic component/main.tex}
            \input{3 - fluid component/main.tex}
            \input{4 - numerical implementation/main.tex}
        \part{Project Overview}
            \input{5 - research plan/main.tex}
            \input{6 - summary/main.tex}
    
    
    %\section{}
    \newpage
    \pagenumbering{gobble}
        \printbibliography


    \newpage
    \pagenumbering{roman}
    \appendix
        \part{Appendices}
            \input{8 - Hilbert complexes/main.tex}
            \input{9 - weak conservation proofs/main.tex}
\end{document}


\title{\BA{Title in Progress...}}
\author{Boris Andrews}
\affil{Mathematical Institute, University of Oxford}
\date{\today}


\begin{document}
    \pagenumbering{gobble}
    \maketitle
    
    
    \begin{abstract}
        Magnetic confinement reactors---in particular tokamaks---offer one of the most promising options for achieving practical nuclear fusion, with the potential to provide virtually limitless, clean energy. The theoretical and numerical modeling of tokamak plasmas is simultaneously an essential component of effective reactor design, and a great research barrier. Tokamak operational conditions exhibit comparatively low Knudsen numbers. Kinetic effects, including kinetic waves and instabilities, Landau damping, bump-on-tail instabilities and more, are therefore highly influential in tokamak plasma dynamics. Purely fluid models are inherently incapable of capturing these effects, whereas the high dimensionality in purely kinetic models render them practically intractable for most relevant purposes.

        We consider a $\delta\!f$ decomposition model, with a macroscopic fluid background and microscopic kinetic correction, both fully coupled to each other. A similar manner of discretization is proposed to that used in the recent \texttt{STRUPHY} code \cite{Holderied_Possanner_Wang_2021, Holderied_2022, Li_et_al_2023} with a finite-element model for the background and a pseudo-particle/PiC model for the correction.

        The fluid background satisfies the full, non-linear, resistive, compressible, Hall MHD equations. \cite{Laakmann_Hu_Farrell_2022} introduces finite-element(-in-space) implicit timesteppers for the incompressible analogue to this system with structure-preserving (SP) properties in the ideal case, alongside parameter-robust preconditioners. We show that these timesteppers can derive from a finite-element-in-time (FET) (and finite-element-in-space) interpretation. The benefits of this reformulation are discussed, including the derivation of timesteppers that are higher order in time, and the quantifiable dissipative SP properties in the non-ideal, resistive case.
        
        We discuss possible options for extending this FET approach to timesteppers for the compressible case.

        The kinetic corrections satisfy linearized Boltzmann equations. Using a Lénard--Bernstein collision operator, these take Fokker--Planck-like forms \cite{Fokker_1914, Planck_1917} wherein pseudo-particles in the numerical model obey the neoclassical transport equations, with particle-independent Brownian drift terms. This offers a rigorous methodology for incorporating collisions into the particle transport model, without coupling the equations of motions for each particle.
        
        Works by Chen, Chacón et al. \cite{Chen_Chacón_Barnes_2011, Chacón_Chen_Barnes_2013, Chen_Chacón_2014, Chen_Chacón_2015} have developed structure-preserving particle pushers for neoclassical transport in the Vlasov equations, derived from Crank--Nicolson integrators. We show these too can can derive from a FET interpretation, similarly offering potential extensions to higher-order-in-time particle pushers. The FET formulation is used also to consider how the stochastic drift terms can be incorporated into the pushers. Stochastic gyrokinetic expansions are also discussed.

        Different options for the numerical implementation of these schemes are considered.

        Due to the efficacy of FET in the development of SP timesteppers for both the fluid and kinetic component, we hope this approach will prove effective in the future for developing SP timesteppers for the full hybrid model. We hope this will give us the opportunity to incorporate previously inaccessible kinetic effects into the highly effective, modern, finite-element MHD models.
    \end{abstract}
    
    
    \newpage
    \tableofcontents
    
    
    \newpage
    \pagenumbering{arabic}
    %\linenumbers\renewcommand\thelinenumber{\color{black!50}\arabic{linenumber}}
            \documentclass[12pt, a4paper]{report}

\input{template/main.tex}

\title{\BA{Title in Progress...}}
\author{Boris Andrews}
\affil{Mathematical Institute, University of Oxford}
\date{\today}


\begin{document}
    \pagenumbering{gobble}
    \maketitle
    
    
    \begin{abstract}
        Magnetic confinement reactors---in particular tokamaks---offer one of the most promising options for achieving practical nuclear fusion, with the potential to provide virtually limitless, clean energy. The theoretical and numerical modeling of tokamak plasmas is simultaneously an essential component of effective reactor design, and a great research barrier. Tokamak operational conditions exhibit comparatively low Knudsen numbers. Kinetic effects, including kinetic waves and instabilities, Landau damping, bump-on-tail instabilities and more, are therefore highly influential in tokamak plasma dynamics. Purely fluid models are inherently incapable of capturing these effects, whereas the high dimensionality in purely kinetic models render them practically intractable for most relevant purposes.

        We consider a $\delta\!f$ decomposition model, with a macroscopic fluid background and microscopic kinetic correction, both fully coupled to each other. A similar manner of discretization is proposed to that used in the recent \texttt{STRUPHY} code \cite{Holderied_Possanner_Wang_2021, Holderied_2022, Li_et_al_2023} with a finite-element model for the background and a pseudo-particle/PiC model for the correction.

        The fluid background satisfies the full, non-linear, resistive, compressible, Hall MHD equations. \cite{Laakmann_Hu_Farrell_2022} introduces finite-element(-in-space) implicit timesteppers for the incompressible analogue to this system with structure-preserving (SP) properties in the ideal case, alongside parameter-robust preconditioners. We show that these timesteppers can derive from a finite-element-in-time (FET) (and finite-element-in-space) interpretation. The benefits of this reformulation are discussed, including the derivation of timesteppers that are higher order in time, and the quantifiable dissipative SP properties in the non-ideal, resistive case.
        
        We discuss possible options for extending this FET approach to timesteppers for the compressible case.

        The kinetic corrections satisfy linearized Boltzmann equations. Using a Lénard--Bernstein collision operator, these take Fokker--Planck-like forms \cite{Fokker_1914, Planck_1917} wherein pseudo-particles in the numerical model obey the neoclassical transport equations, with particle-independent Brownian drift terms. This offers a rigorous methodology for incorporating collisions into the particle transport model, without coupling the equations of motions for each particle.
        
        Works by Chen, Chacón et al. \cite{Chen_Chacón_Barnes_2011, Chacón_Chen_Barnes_2013, Chen_Chacón_2014, Chen_Chacón_2015} have developed structure-preserving particle pushers for neoclassical transport in the Vlasov equations, derived from Crank--Nicolson integrators. We show these too can can derive from a FET interpretation, similarly offering potential extensions to higher-order-in-time particle pushers. The FET formulation is used also to consider how the stochastic drift terms can be incorporated into the pushers. Stochastic gyrokinetic expansions are also discussed.

        Different options for the numerical implementation of these schemes are considered.

        Due to the efficacy of FET in the development of SP timesteppers for both the fluid and kinetic component, we hope this approach will prove effective in the future for developing SP timesteppers for the full hybrid model. We hope this will give us the opportunity to incorporate previously inaccessible kinetic effects into the highly effective, modern, finite-element MHD models.
    \end{abstract}
    
    
    \newpage
    \tableofcontents
    
    
    \newpage
    \pagenumbering{arabic}
    %\linenumbers\renewcommand\thelinenumber{\color{black!50}\arabic{linenumber}}
            \input{0 - introduction/main.tex}
        \part{Research}
            \input{1 - low-noise PiC models/main.tex}
            \input{2 - kinetic component/main.tex}
            \input{3 - fluid component/main.tex}
            \input{4 - numerical implementation/main.tex}
        \part{Project Overview}
            \input{5 - research plan/main.tex}
            \input{6 - summary/main.tex}
    
    
    %\section{}
    \newpage
    \pagenumbering{gobble}
        \printbibliography


    \newpage
    \pagenumbering{roman}
    \appendix
        \part{Appendices}
            \input{8 - Hilbert complexes/main.tex}
            \input{9 - weak conservation proofs/main.tex}
\end{document}

        \part{Research}
            \documentclass[12pt, a4paper]{report}

\input{template/main.tex}

\title{\BA{Title in Progress...}}
\author{Boris Andrews}
\affil{Mathematical Institute, University of Oxford}
\date{\today}


\begin{document}
    \pagenumbering{gobble}
    \maketitle
    
    
    \begin{abstract}
        Magnetic confinement reactors---in particular tokamaks---offer one of the most promising options for achieving practical nuclear fusion, with the potential to provide virtually limitless, clean energy. The theoretical and numerical modeling of tokamak plasmas is simultaneously an essential component of effective reactor design, and a great research barrier. Tokamak operational conditions exhibit comparatively low Knudsen numbers. Kinetic effects, including kinetic waves and instabilities, Landau damping, bump-on-tail instabilities and more, are therefore highly influential in tokamak plasma dynamics. Purely fluid models are inherently incapable of capturing these effects, whereas the high dimensionality in purely kinetic models render them practically intractable for most relevant purposes.

        We consider a $\delta\!f$ decomposition model, with a macroscopic fluid background and microscopic kinetic correction, both fully coupled to each other. A similar manner of discretization is proposed to that used in the recent \texttt{STRUPHY} code \cite{Holderied_Possanner_Wang_2021, Holderied_2022, Li_et_al_2023} with a finite-element model for the background and a pseudo-particle/PiC model for the correction.

        The fluid background satisfies the full, non-linear, resistive, compressible, Hall MHD equations. \cite{Laakmann_Hu_Farrell_2022} introduces finite-element(-in-space) implicit timesteppers for the incompressible analogue to this system with structure-preserving (SP) properties in the ideal case, alongside parameter-robust preconditioners. We show that these timesteppers can derive from a finite-element-in-time (FET) (and finite-element-in-space) interpretation. The benefits of this reformulation are discussed, including the derivation of timesteppers that are higher order in time, and the quantifiable dissipative SP properties in the non-ideal, resistive case.
        
        We discuss possible options for extending this FET approach to timesteppers for the compressible case.

        The kinetic corrections satisfy linearized Boltzmann equations. Using a Lénard--Bernstein collision operator, these take Fokker--Planck-like forms \cite{Fokker_1914, Planck_1917} wherein pseudo-particles in the numerical model obey the neoclassical transport equations, with particle-independent Brownian drift terms. This offers a rigorous methodology for incorporating collisions into the particle transport model, without coupling the equations of motions for each particle.
        
        Works by Chen, Chacón et al. \cite{Chen_Chacón_Barnes_2011, Chacón_Chen_Barnes_2013, Chen_Chacón_2014, Chen_Chacón_2015} have developed structure-preserving particle pushers for neoclassical transport in the Vlasov equations, derived from Crank--Nicolson integrators. We show these too can can derive from a FET interpretation, similarly offering potential extensions to higher-order-in-time particle pushers. The FET formulation is used also to consider how the stochastic drift terms can be incorporated into the pushers. Stochastic gyrokinetic expansions are also discussed.

        Different options for the numerical implementation of these schemes are considered.

        Due to the efficacy of FET in the development of SP timesteppers for both the fluid and kinetic component, we hope this approach will prove effective in the future for developing SP timesteppers for the full hybrid model. We hope this will give us the opportunity to incorporate previously inaccessible kinetic effects into the highly effective, modern, finite-element MHD models.
    \end{abstract}
    
    
    \newpage
    \tableofcontents
    
    
    \newpage
    \pagenumbering{arabic}
    %\linenumbers\renewcommand\thelinenumber{\color{black!50}\arabic{linenumber}}
            \input{0 - introduction/main.tex}
        \part{Research}
            \input{1 - low-noise PiC models/main.tex}
            \input{2 - kinetic component/main.tex}
            \input{3 - fluid component/main.tex}
            \input{4 - numerical implementation/main.tex}
        \part{Project Overview}
            \input{5 - research plan/main.tex}
            \input{6 - summary/main.tex}
    
    
    %\section{}
    \newpage
    \pagenumbering{gobble}
        \printbibliography


    \newpage
    \pagenumbering{roman}
    \appendix
        \part{Appendices}
            \input{8 - Hilbert complexes/main.tex}
            \input{9 - weak conservation proofs/main.tex}
\end{document}

            \documentclass[12pt, a4paper]{report}

\input{template/main.tex}

\title{\BA{Title in Progress...}}
\author{Boris Andrews}
\affil{Mathematical Institute, University of Oxford}
\date{\today}


\begin{document}
    \pagenumbering{gobble}
    \maketitle
    
    
    \begin{abstract}
        Magnetic confinement reactors---in particular tokamaks---offer one of the most promising options for achieving practical nuclear fusion, with the potential to provide virtually limitless, clean energy. The theoretical and numerical modeling of tokamak plasmas is simultaneously an essential component of effective reactor design, and a great research barrier. Tokamak operational conditions exhibit comparatively low Knudsen numbers. Kinetic effects, including kinetic waves and instabilities, Landau damping, bump-on-tail instabilities and more, are therefore highly influential in tokamak plasma dynamics. Purely fluid models are inherently incapable of capturing these effects, whereas the high dimensionality in purely kinetic models render them practically intractable for most relevant purposes.

        We consider a $\delta\!f$ decomposition model, with a macroscopic fluid background and microscopic kinetic correction, both fully coupled to each other. A similar manner of discretization is proposed to that used in the recent \texttt{STRUPHY} code \cite{Holderied_Possanner_Wang_2021, Holderied_2022, Li_et_al_2023} with a finite-element model for the background and a pseudo-particle/PiC model for the correction.

        The fluid background satisfies the full, non-linear, resistive, compressible, Hall MHD equations. \cite{Laakmann_Hu_Farrell_2022} introduces finite-element(-in-space) implicit timesteppers for the incompressible analogue to this system with structure-preserving (SP) properties in the ideal case, alongside parameter-robust preconditioners. We show that these timesteppers can derive from a finite-element-in-time (FET) (and finite-element-in-space) interpretation. The benefits of this reformulation are discussed, including the derivation of timesteppers that are higher order in time, and the quantifiable dissipative SP properties in the non-ideal, resistive case.
        
        We discuss possible options for extending this FET approach to timesteppers for the compressible case.

        The kinetic corrections satisfy linearized Boltzmann equations. Using a Lénard--Bernstein collision operator, these take Fokker--Planck-like forms \cite{Fokker_1914, Planck_1917} wherein pseudo-particles in the numerical model obey the neoclassical transport equations, with particle-independent Brownian drift terms. This offers a rigorous methodology for incorporating collisions into the particle transport model, without coupling the equations of motions for each particle.
        
        Works by Chen, Chacón et al. \cite{Chen_Chacón_Barnes_2011, Chacón_Chen_Barnes_2013, Chen_Chacón_2014, Chen_Chacón_2015} have developed structure-preserving particle pushers for neoclassical transport in the Vlasov equations, derived from Crank--Nicolson integrators. We show these too can can derive from a FET interpretation, similarly offering potential extensions to higher-order-in-time particle pushers. The FET formulation is used also to consider how the stochastic drift terms can be incorporated into the pushers. Stochastic gyrokinetic expansions are also discussed.

        Different options for the numerical implementation of these schemes are considered.

        Due to the efficacy of FET in the development of SP timesteppers for both the fluid and kinetic component, we hope this approach will prove effective in the future for developing SP timesteppers for the full hybrid model. We hope this will give us the opportunity to incorporate previously inaccessible kinetic effects into the highly effective, modern, finite-element MHD models.
    \end{abstract}
    
    
    \newpage
    \tableofcontents
    
    
    \newpage
    \pagenumbering{arabic}
    %\linenumbers\renewcommand\thelinenumber{\color{black!50}\arabic{linenumber}}
            \input{0 - introduction/main.tex}
        \part{Research}
            \input{1 - low-noise PiC models/main.tex}
            \input{2 - kinetic component/main.tex}
            \input{3 - fluid component/main.tex}
            \input{4 - numerical implementation/main.tex}
        \part{Project Overview}
            \input{5 - research plan/main.tex}
            \input{6 - summary/main.tex}
    
    
    %\section{}
    \newpage
    \pagenumbering{gobble}
        \printbibliography


    \newpage
    \pagenumbering{roman}
    \appendix
        \part{Appendices}
            \input{8 - Hilbert complexes/main.tex}
            \input{9 - weak conservation proofs/main.tex}
\end{document}

            \documentclass[12pt, a4paper]{report}

\input{template/main.tex}

\title{\BA{Title in Progress...}}
\author{Boris Andrews}
\affil{Mathematical Institute, University of Oxford}
\date{\today}


\begin{document}
    \pagenumbering{gobble}
    \maketitle
    
    
    \begin{abstract}
        Magnetic confinement reactors---in particular tokamaks---offer one of the most promising options for achieving practical nuclear fusion, with the potential to provide virtually limitless, clean energy. The theoretical and numerical modeling of tokamak plasmas is simultaneously an essential component of effective reactor design, and a great research barrier. Tokamak operational conditions exhibit comparatively low Knudsen numbers. Kinetic effects, including kinetic waves and instabilities, Landau damping, bump-on-tail instabilities and more, are therefore highly influential in tokamak plasma dynamics. Purely fluid models are inherently incapable of capturing these effects, whereas the high dimensionality in purely kinetic models render them practically intractable for most relevant purposes.

        We consider a $\delta\!f$ decomposition model, with a macroscopic fluid background and microscopic kinetic correction, both fully coupled to each other. A similar manner of discretization is proposed to that used in the recent \texttt{STRUPHY} code \cite{Holderied_Possanner_Wang_2021, Holderied_2022, Li_et_al_2023} with a finite-element model for the background and a pseudo-particle/PiC model for the correction.

        The fluid background satisfies the full, non-linear, resistive, compressible, Hall MHD equations. \cite{Laakmann_Hu_Farrell_2022} introduces finite-element(-in-space) implicit timesteppers for the incompressible analogue to this system with structure-preserving (SP) properties in the ideal case, alongside parameter-robust preconditioners. We show that these timesteppers can derive from a finite-element-in-time (FET) (and finite-element-in-space) interpretation. The benefits of this reformulation are discussed, including the derivation of timesteppers that are higher order in time, and the quantifiable dissipative SP properties in the non-ideal, resistive case.
        
        We discuss possible options for extending this FET approach to timesteppers for the compressible case.

        The kinetic corrections satisfy linearized Boltzmann equations. Using a Lénard--Bernstein collision operator, these take Fokker--Planck-like forms \cite{Fokker_1914, Planck_1917} wherein pseudo-particles in the numerical model obey the neoclassical transport equations, with particle-independent Brownian drift terms. This offers a rigorous methodology for incorporating collisions into the particle transport model, without coupling the equations of motions for each particle.
        
        Works by Chen, Chacón et al. \cite{Chen_Chacón_Barnes_2011, Chacón_Chen_Barnes_2013, Chen_Chacón_2014, Chen_Chacón_2015} have developed structure-preserving particle pushers for neoclassical transport in the Vlasov equations, derived from Crank--Nicolson integrators. We show these too can can derive from a FET interpretation, similarly offering potential extensions to higher-order-in-time particle pushers. The FET formulation is used also to consider how the stochastic drift terms can be incorporated into the pushers. Stochastic gyrokinetic expansions are also discussed.

        Different options for the numerical implementation of these schemes are considered.

        Due to the efficacy of FET in the development of SP timesteppers for both the fluid and kinetic component, we hope this approach will prove effective in the future for developing SP timesteppers for the full hybrid model. We hope this will give us the opportunity to incorporate previously inaccessible kinetic effects into the highly effective, modern, finite-element MHD models.
    \end{abstract}
    
    
    \newpage
    \tableofcontents
    
    
    \newpage
    \pagenumbering{arabic}
    %\linenumbers\renewcommand\thelinenumber{\color{black!50}\arabic{linenumber}}
            \input{0 - introduction/main.tex}
        \part{Research}
            \input{1 - low-noise PiC models/main.tex}
            \input{2 - kinetic component/main.tex}
            \input{3 - fluid component/main.tex}
            \input{4 - numerical implementation/main.tex}
        \part{Project Overview}
            \input{5 - research plan/main.tex}
            \input{6 - summary/main.tex}
    
    
    %\section{}
    \newpage
    \pagenumbering{gobble}
        \printbibliography


    \newpage
    \pagenumbering{roman}
    \appendix
        \part{Appendices}
            \input{8 - Hilbert complexes/main.tex}
            \input{9 - weak conservation proofs/main.tex}
\end{document}

            \documentclass[12pt, a4paper]{report}

\input{template/main.tex}

\title{\BA{Title in Progress...}}
\author{Boris Andrews}
\affil{Mathematical Institute, University of Oxford}
\date{\today}


\begin{document}
    \pagenumbering{gobble}
    \maketitle
    
    
    \begin{abstract}
        Magnetic confinement reactors---in particular tokamaks---offer one of the most promising options for achieving practical nuclear fusion, with the potential to provide virtually limitless, clean energy. The theoretical and numerical modeling of tokamak plasmas is simultaneously an essential component of effective reactor design, and a great research barrier. Tokamak operational conditions exhibit comparatively low Knudsen numbers. Kinetic effects, including kinetic waves and instabilities, Landau damping, bump-on-tail instabilities and more, are therefore highly influential in tokamak plasma dynamics. Purely fluid models are inherently incapable of capturing these effects, whereas the high dimensionality in purely kinetic models render them practically intractable for most relevant purposes.

        We consider a $\delta\!f$ decomposition model, with a macroscopic fluid background and microscopic kinetic correction, both fully coupled to each other. A similar manner of discretization is proposed to that used in the recent \texttt{STRUPHY} code \cite{Holderied_Possanner_Wang_2021, Holderied_2022, Li_et_al_2023} with a finite-element model for the background and a pseudo-particle/PiC model for the correction.

        The fluid background satisfies the full, non-linear, resistive, compressible, Hall MHD equations. \cite{Laakmann_Hu_Farrell_2022} introduces finite-element(-in-space) implicit timesteppers for the incompressible analogue to this system with structure-preserving (SP) properties in the ideal case, alongside parameter-robust preconditioners. We show that these timesteppers can derive from a finite-element-in-time (FET) (and finite-element-in-space) interpretation. The benefits of this reformulation are discussed, including the derivation of timesteppers that are higher order in time, and the quantifiable dissipative SP properties in the non-ideal, resistive case.
        
        We discuss possible options for extending this FET approach to timesteppers for the compressible case.

        The kinetic corrections satisfy linearized Boltzmann equations. Using a Lénard--Bernstein collision operator, these take Fokker--Planck-like forms \cite{Fokker_1914, Planck_1917} wherein pseudo-particles in the numerical model obey the neoclassical transport equations, with particle-independent Brownian drift terms. This offers a rigorous methodology for incorporating collisions into the particle transport model, without coupling the equations of motions for each particle.
        
        Works by Chen, Chacón et al. \cite{Chen_Chacón_Barnes_2011, Chacón_Chen_Barnes_2013, Chen_Chacón_2014, Chen_Chacón_2015} have developed structure-preserving particle pushers for neoclassical transport in the Vlasov equations, derived from Crank--Nicolson integrators. We show these too can can derive from a FET interpretation, similarly offering potential extensions to higher-order-in-time particle pushers. The FET formulation is used also to consider how the stochastic drift terms can be incorporated into the pushers. Stochastic gyrokinetic expansions are also discussed.

        Different options for the numerical implementation of these schemes are considered.

        Due to the efficacy of FET in the development of SP timesteppers for both the fluid and kinetic component, we hope this approach will prove effective in the future for developing SP timesteppers for the full hybrid model. We hope this will give us the opportunity to incorporate previously inaccessible kinetic effects into the highly effective, modern, finite-element MHD models.
    \end{abstract}
    
    
    \newpage
    \tableofcontents
    
    
    \newpage
    \pagenumbering{arabic}
    %\linenumbers\renewcommand\thelinenumber{\color{black!50}\arabic{linenumber}}
            \input{0 - introduction/main.tex}
        \part{Research}
            \input{1 - low-noise PiC models/main.tex}
            \input{2 - kinetic component/main.tex}
            \input{3 - fluid component/main.tex}
            \input{4 - numerical implementation/main.tex}
        \part{Project Overview}
            \input{5 - research plan/main.tex}
            \input{6 - summary/main.tex}
    
    
    %\section{}
    \newpage
    \pagenumbering{gobble}
        \printbibliography


    \newpage
    \pagenumbering{roman}
    \appendix
        \part{Appendices}
            \input{8 - Hilbert complexes/main.tex}
            \input{9 - weak conservation proofs/main.tex}
\end{document}

        \part{Project Overview}
            \documentclass[12pt, a4paper]{report}

\input{template/main.tex}

\title{\BA{Title in Progress...}}
\author{Boris Andrews}
\affil{Mathematical Institute, University of Oxford}
\date{\today}


\begin{document}
    \pagenumbering{gobble}
    \maketitle
    
    
    \begin{abstract}
        Magnetic confinement reactors---in particular tokamaks---offer one of the most promising options for achieving practical nuclear fusion, with the potential to provide virtually limitless, clean energy. The theoretical and numerical modeling of tokamak plasmas is simultaneously an essential component of effective reactor design, and a great research barrier. Tokamak operational conditions exhibit comparatively low Knudsen numbers. Kinetic effects, including kinetic waves and instabilities, Landau damping, bump-on-tail instabilities and more, are therefore highly influential in tokamak plasma dynamics. Purely fluid models are inherently incapable of capturing these effects, whereas the high dimensionality in purely kinetic models render them practically intractable for most relevant purposes.

        We consider a $\delta\!f$ decomposition model, with a macroscopic fluid background and microscopic kinetic correction, both fully coupled to each other. A similar manner of discretization is proposed to that used in the recent \texttt{STRUPHY} code \cite{Holderied_Possanner_Wang_2021, Holderied_2022, Li_et_al_2023} with a finite-element model for the background and a pseudo-particle/PiC model for the correction.

        The fluid background satisfies the full, non-linear, resistive, compressible, Hall MHD equations. \cite{Laakmann_Hu_Farrell_2022} introduces finite-element(-in-space) implicit timesteppers for the incompressible analogue to this system with structure-preserving (SP) properties in the ideal case, alongside parameter-robust preconditioners. We show that these timesteppers can derive from a finite-element-in-time (FET) (and finite-element-in-space) interpretation. The benefits of this reformulation are discussed, including the derivation of timesteppers that are higher order in time, and the quantifiable dissipative SP properties in the non-ideal, resistive case.
        
        We discuss possible options for extending this FET approach to timesteppers for the compressible case.

        The kinetic corrections satisfy linearized Boltzmann equations. Using a Lénard--Bernstein collision operator, these take Fokker--Planck-like forms \cite{Fokker_1914, Planck_1917} wherein pseudo-particles in the numerical model obey the neoclassical transport equations, with particle-independent Brownian drift terms. This offers a rigorous methodology for incorporating collisions into the particle transport model, without coupling the equations of motions for each particle.
        
        Works by Chen, Chacón et al. \cite{Chen_Chacón_Barnes_2011, Chacón_Chen_Barnes_2013, Chen_Chacón_2014, Chen_Chacón_2015} have developed structure-preserving particle pushers for neoclassical transport in the Vlasov equations, derived from Crank--Nicolson integrators. We show these too can can derive from a FET interpretation, similarly offering potential extensions to higher-order-in-time particle pushers. The FET formulation is used also to consider how the stochastic drift terms can be incorporated into the pushers. Stochastic gyrokinetic expansions are also discussed.

        Different options for the numerical implementation of these schemes are considered.

        Due to the efficacy of FET in the development of SP timesteppers for both the fluid and kinetic component, we hope this approach will prove effective in the future for developing SP timesteppers for the full hybrid model. We hope this will give us the opportunity to incorporate previously inaccessible kinetic effects into the highly effective, modern, finite-element MHD models.
    \end{abstract}
    
    
    \newpage
    \tableofcontents
    
    
    \newpage
    \pagenumbering{arabic}
    %\linenumbers\renewcommand\thelinenumber{\color{black!50}\arabic{linenumber}}
            \input{0 - introduction/main.tex}
        \part{Research}
            \input{1 - low-noise PiC models/main.tex}
            \input{2 - kinetic component/main.tex}
            \input{3 - fluid component/main.tex}
            \input{4 - numerical implementation/main.tex}
        \part{Project Overview}
            \input{5 - research plan/main.tex}
            \input{6 - summary/main.tex}
    
    
    %\section{}
    \newpage
    \pagenumbering{gobble}
        \printbibliography


    \newpage
    \pagenumbering{roman}
    \appendix
        \part{Appendices}
            \input{8 - Hilbert complexes/main.tex}
            \input{9 - weak conservation proofs/main.tex}
\end{document}

            \documentclass[12pt, a4paper]{report}

\input{template/main.tex}

\title{\BA{Title in Progress...}}
\author{Boris Andrews}
\affil{Mathematical Institute, University of Oxford}
\date{\today}


\begin{document}
    \pagenumbering{gobble}
    \maketitle
    
    
    \begin{abstract}
        Magnetic confinement reactors---in particular tokamaks---offer one of the most promising options for achieving practical nuclear fusion, with the potential to provide virtually limitless, clean energy. The theoretical and numerical modeling of tokamak plasmas is simultaneously an essential component of effective reactor design, and a great research barrier. Tokamak operational conditions exhibit comparatively low Knudsen numbers. Kinetic effects, including kinetic waves and instabilities, Landau damping, bump-on-tail instabilities and more, are therefore highly influential in tokamak plasma dynamics. Purely fluid models are inherently incapable of capturing these effects, whereas the high dimensionality in purely kinetic models render them practically intractable for most relevant purposes.

        We consider a $\delta\!f$ decomposition model, with a macroscopic fluid background and microscopic kinetic correction, both fully coupled to each other. A similar manner of discretization is proposed to that used in the recent \texttt{STRUPHY} code \cite{Holderied_Possanner_Wang_2021, Holderied_2022, Li_et_al_2023} with a finite-element model for the background and a pseudo-particle/PiC model for the correction.

        The fluid background satisfies the full, non-linear, resistive, compressible, Hall MHD equations. \cite{Laakmann_Hu_Farrell_2022} introduces finite-element(-in-space) implicit timesteppers for the incompressible analogue to this system with structure-preserving (SP) properties in the ideal case, alongside parameter-robust preconditioners. We show that these timesteppers can derive from a finite-element-in-time (FET) (and finite-element-in-space) interpretation. The benefits of this reformulation are discussed, including the derivation of timesteppers that are higher order in time, and the quantifiable dissipative SP properties in the non-ideal, resistive case.
        
        We discuss possible options for extending this FET approach to timesteppers for the compressible case.

        The kinetic corrections satisfy linearized Boltzmann equations. Using a Lénard--Bernstein collision operator, these take Fokker--Planck-like forms \cite{Fokker_1914, Planck_1917} wherein pseudo-particles in the numerical model obey the neoclassical transport equations, with particle-independent Brownian drift terms. This offers a rigorous methodology for incorporating collisions into the particle transport model, without coupling the equations of motions for each particle.
        
        Works by Chen, Chacón et al. \cite{Chen_Chacón_Barnes_2011, Chacón_Chen_Barnes_2013, Chen_Chacón_2014, Chen_Chacón_2015} have developed structure-preserving particle pushers for neoclassical transport in the Vlasov equations, derived from Crank--Nicolson integrators. We show these too can can derive from a FET interpretation, similarly offering potential extensions to higher-order-in-time particle pushers. The FET formulation is used also to consider how the stochastic drift terms can be incorporated into the pushers. Stochastic gyrokinetic expansions are also discussed.

        Different options for the numerical implementation of these schemes are considered.

        Due to the efficacy of FET in the development of SP timesteppers for both the fluid and kinetic component, we hope this approach will prove effective in the future for developing SP timesteppers for the full hybrid model. We hope this will give us the opportunity to incorporate previously inaccessible kinetic effects into the highly effective, modern, finite-element MHD models.
    \end{abstract}
    
    
    \newpage
    \tableofcontents
    
    
    \newpage
    \pagenumbering{arabic}
    %\linenumbers\renewcommand\thelinenumber{\color{black!50}\arabic{linenumber}}
            \input{0 - introduction/main.tex}
        \part{Research}
            \input{1 - low-noise PiC models/main.tex}
            \input{2 - kinetic component/main.tex}
            \input{3 - fluid component/main.tex}
            \input{4 - numerical implementation/main.tex}
        \part{Project Overview}
            \input{5 - research plan/main.tex}
            \input{6 - summary/main.tex}
    
    
    %\section{}
    \newpage
    \pagenumbering{gobble}
        \printbibliography


    \newpage
    \pagenumbering{roman}
    \appendix
        \part{Appendices}
            \input{8 - Hilbert complexes/main.tex}
            \input{9 - weak conservation proofs/main.tex}
\end{document}

    
    
    %\section{}
    \newpage
    \pagenumbering{gobble}
        \printbibliography


    \newpage
    \pagenumbering{roman}
    \appendix
        \part{Appendices}
            \documentclass[12pt, a4paper]{report}

\input{template/main.tex}

\title{\BA{Title in Progress...}}
\author{Boris Andrews}
\affil{Mathematical Institute, University of Oxford}
\date{\today}


\begin{document}
    \pagenumbering{gobble}
    \maketitle
    
    
    \begin{abstract}
        Magnetic confinement reactors---in particular tokamaks---offer one of the most promising options for achieving practical nuclear fusion, with the potential to provide virtually limitless, clean energy. The theoretical and numerical modeling of tokamak plasmas is simultaneously an essential component of effective reactor design, and a great research barrier. Tokamak operational conditions exhibit comparatively low Knudsen numbers. Kinetic effects, including kinetic waves and instabilities, Landau damping, bump-on-tail instabilities and more, are therefore highly influential in tokamak plasma dynamics. Purely fluid models are inherently incapable of capturing these effects, whereas the high dimensionality in purely kinetic models render them practically intractable for most relevant purposes.

        We consider a $\delta\!f$ decomposition model, with a macroscopic fluid background and microscopic kinetic correction, both fully coupled to each other. A similar manner of discretization is proposed to that used in the recent \texttt{STRUPHY} code \cite{Holderied_Possanner_Wang_2021, Holderied_2022, Li_et_al_2023} with a finite-element model for the background and a pseudo-particle/PiC model for the correction.

        The fluid background satisfies the full, non-linear, resistive, compressible, Hall MHD equations. \cite{Laakmann_Hu_Farrell_2022} introduces finite-element(-in-space) implicit timesteppers for the incompressible analogue to this system with structure-preserving (SP) properties in the ideal case, alongside parameter-robust preconditioners. We show that these timesteppers can derive from a finite-element-in-time (FET) (and finite-element-in-space) interpretation. The benefits of this reformulation are discussed, including the derivation of timesteppers that are higher order in time, and the quantifiable dissipative SP properties in the non-ideal, resistive case.
        
        We discuss possible options for extending this FET approach to timesteppers for the compressible case.

        The kinetic corrections satisfy linearized Boltzmann equations. Using a Lénard--Bernstein collision operator, these take Fokker--Planck-like forms \cite{Fokker_1914, Planck_1917} wherein pseudo-particles in the numerical model obey the neoclassical transport equations, with particle-independent Brownian drift terms. This offers a rigorous methodology for incorporating collisions into the particle transport model, without coupling the equations of motions for each particle.
        
        Works by Chen, Chacón et al. \cite{Chen_Chacón_Barnes_2011, Chacón_Chen_Barnes_2013, Chen_Chacón_2014, Chen_Chacón_2015} have developed structure-preserving particle pushers for neoclassical transport in the Vlasov equations, derived from Crank--Nicolson integrators. We show these too can can derive from a FET interpretation, similarly offering potential extensions to higher-order-in-time particle pushers. The FET formulation is used also to consider how the stochastic drift terms can be incorporated into the pushers. Stochastic gyrokinetic expansions are also discussed.

        Different options for the numerical implementation of these schemes are considered.

        Due to the efficacy of FET in the development of SP timesteppers for both the fluid and kinetic component, we hope this approach will prove effective in the future for developing SP timesteppers for the full hybrid model. We hope this will give us the opportunity to incorporate previously inaccessible kinetic effects into the highly effective, modern, finite-element MHD models.
    \end{abstract}
    
    
    \newpage
    \tableofcontents
    
    
    \newpage
    \pagenumbering{arabic}
    %\linenumbers\renewcommand\thelinenumber{\color{black!50}\arabic{linenumber}}
            \input{0 - introduction/main.tex}
        \part{Research}
            \input{1 - low-noise PiC models/main.tex}
            \input{2 - kinetic component/main.tex}
            \input{3 - fluid component/main.tex}
            \input{4 - numerical implementation/main.tex}
        \part{Project Overview}
            \input{5 - research plan/main.tex}
            \input{6 - summary/main.tex}
    
    
    %\section{}
    \newpage
    \pagenumbering{gobble}
        \printbibliography


    \newpage
    \pagenumbering{roman}
    \appendix
        \part{Appendices}
            \input{8 - Hilbert complexes/main.tex}
            \input{9 - weak conservation proofs/main.tex}
\end{document}

            \documentclass[12pt, a4paper]{report}

\input{template/main.tex}

\title{\BA{Title in Progress...}}
\author{Boris Andrews}
\affil{Mathematical Institute, University of Oxford}
\date{\today}


\begin{document}
    \pagenumbering{gobble}
    \maketitle
    
    
    \begin{abstract}
        Magnetic confinement reactors---in particular tokamaks---offer one of the most promising options for achieving practical nuclear fusion, with the potential to provide virtually limitless, clean energy. The theoretical and numerical modeling of tokamak plasmas is simultaneously an essential component of effective reactor design, and a great research barrier. Tokamak operational conditions exhibit comparatively low Knudsen numbers. Kinetic effects, including kinetic waves and instabilities, Landau damping, bump-on-tail instabilities and more, are therefore highly influential in tokamak plasma dynamics. Purely fluid models are inherently incapable of capturing these effects, whereas the high dimensionality in purely kinetic models render them practically intractable for most relevant purposes.

        We consider a $\delta\!f$ decomposition model, with a macroscopic fluid background and microscopic kinetic correction, both fully coupled to each other. A similar manner of discretization is proposed to that used in the recent \texttt{STRUPHY} code \cite{Holderied_Possanner_Wang_2021, Holderied_2022, Li_et_al_2023} with a finite-element model for the background and a pseudo-particle/PiC model for the correction.

        The fluid background satisfies the full, non-linear, resistive, compressible, Hall MHD equations. \cite{Laakmann_Hu_Farrell_2022} introduces finite-element(-in-space) implicit timesteppers for the incompressible analogue to this system with structure-preserving (SP) properties in the ideal case, alongside parameter-robust preconditioners. We show that these timesteppers can derive from a finite-element-in-time (FET) (and finite-element-in-space) interpretation. The benefits of this reformulation are discussed, including the derivation of timesteppers that are higher order in time, and the quantifiable dissipative SP properties in the non-ideal, resistive case.
        
        We discuss possible options for extending this FET approach to timesteppers for the compressible case.

        The kinetic corrections satisfy linearized Boltzmann equations. Using a Lénard--Bernstein collision operator, these take Fokker--Planck-like forms \cite{Fokker_1914, Planck_1917} wherein pseudo-particles in the numerical model obey the neoclassical transport equations, with particle-independent Brownian drift terms. This offers a rigorous methodology for incorporating collisions into the particle transport model, without coupling the equations of motions for each particle.
        
        Works by Chen, Chacón et al. \cite{Chen_Chacón_Barnes_2011, Chacón_Chen_Barnes_2013, Chen_Chacón_2014, Chen_Chacón_2015} have developed structure-preserving particle pushers for neoclassical transport in the Vlasov equations, derived from Crank--Nicolson integrators. We show these too can can derive from a FET interpretation, similarly offering potential extensions to higher-order-in-time particle pushers. The FET formulation is used also to consider how the stochastic drift terms can be incorporated into the pushers. Stochastic gyrokinetic expansions are also discussed.

        Different options for the numerical implementation of these schemes are considered.

        Due to the efficacy of FET in the development of SP timesteppers for both the fluid and kinetic component, we hope this approach will prove effective in the future for developing SP timesteppers for the full hybrid model. We hope this will give us the opportunity to incorporate previously inaccessible kinetic effects into the highly effective, modern, finite-element MHD models.
    \end{abstract}
    
    
    \newpage
    \tableofcontents
    
    
    \newpage
    \pagenumbering{arabic}
    %\linenumbers\renewcommand\thelinenumber{\color{black!50}\arabic{linenumber}}
            \input{0 - introduction/main.tex}
        \part{Research}
            \input{1 - low-noise PiC models/main.tex}
            \input{2 - kinetic component/main.tex}
            \input{3 - fluid component/main.tex}
            \input{4 - numerical implementation/main.tex}
        \part{Project Overview}
            \input{5 - research plan/main.tex}
            \input{6 - summary/main.tex}
    
    
    %\section{}
    \newpage
    \pagenumbering{gobble}
        \printbibliography


    \newpage
    \pagenumbering{roman}
    \appendix
        \part{Appendices}
            \input{8 - Hilbert complexes/main.tex}
            \input{9 - weak conservation proofs/main.tex}
\end{document}

\end{document}

            \documentclass[12pt, a4paper]{report}

\documentclass[12pt, a4paper]{report}

\input{template/main.tex}

\title{\BA{Title in Progress...}}
\author{Boris Andrews}
\affil{Mathematical Institute, University of Oxford}
\date{\today}


\begin{document}
    \pagenumbering{gobble}
    \maketitle
    
    
    \begin{abstract}
        Magnetic confinement reactors---in particular tokamaks---offer one of the most promising options for achieving practical nuclear fusion, with the potential to provide virtually limitless, clean energy. The theoretical and numerical modeling of tokamak plasmas is simultaneously an essential component of effective reactor design, and a great research barrier. Tokamak operational conditions exhibit comparatively low Knudsen numbers. Kinetic effects, including kinetic waves and instabilities, Landau damping, bump-on-tail instabilities and more, are therefore highly influential in tokamak plasma dynamics. Purely fluid models are inherently incapable of capturing these effects, whereas the high dimensionality in purely kinetic models render them practically intractable for most relevant purposes.

        We consider a $\delta\!f$ decomposition model, with a macroscopic fluid background and microscopic kinetic correction, both fully coupled to each other. A similar manner of discretization is proposed to that used in the recent \texttt{STRUPHY} code \cite{Holderied_Possanner_Wang_2021, Holderied_2022, Li_et_al_2023} with a finite-element model for the background and a pseudo-particle/PiC model for the correction.

        The fluid background satisfies the full, non-linear, resistive, compressible, Hall MHD equations. \cite{Laakmann_Hu_Farrell_2022} introduces finite-element(-in-space) implicit timesteppers for the incompressible analogue to this system with structure-preserving (SP) properties in the ideal case, alongside parameter-robust preconditioners. We show that these timesteppers can derive from a finite-element-in-time (FET) (and finite-element-in-space) interpretation. The benefits of this reformulation are discussed, including the derivation of timesteppers that are higher order in time, and the quantifiable dissipative SP properties in the non-ideal, resistive case.
        
        We discuss possible options for extending this FET approach to timesteppers for the compressible case.

        The kinetic corrections satisfy linearized Boltzmann equations. Using a Lénard--Bernstein collision operator, these take Fokker--Planck-like forms \cite{Fokker_1914, Planck_1917} wherein pseudo-particles in the numerical model obey the neoclassical transport equations, with particle-independent Brownian drift terms. This offers a rigorous methodology for incorporating collisions into the particle transport model, without coupling the equations of motions for each particle.
        
        Works by Chen, Chacón et al. \cite{Chen_Chacón_Barnes_2011, Chacón_Chen_Barnes_2013, Chen_Chacón_2014, Chen_Chacón_2015} have developed structure-preserving particle pushers for neoclassical transport in the Vlasov equations, derived from Crank--Nicolson integrators. We show these too can can derive from a FET interpretation, similarly offering potential extensions to higher-order-in-time particle pushers. The FET formulation is used also to consider how the stochastic drift terms can be incorporated into the pushers. Stochastic gyrokinetic expansions are also discussed.

        Different options for the numerical implementation of these schemes are considered.

        Due to the efficacy of FET in the development of SP timesteppers for both the fluid and kinetic component, we hope this approach will prove effective in the future for developing SP timesteppers for the full hybrid model. We hope this will give us the opportunity to incorporate previously inaccessible kinetic effects into the highly effective, modern, finite-element MHD models.
    \end{abstract}
    
    
    \newpage
    \tableofcontents
    
    
    \newpage
    \pagenumbering{arabic}
    %\linenumbers\renewcommand\thelinenumber{\color{black!50}\arabic{linenumber}}
            \input{0 - introduction/main.tex}
        \part{Research}
            \input{1 - low-noise PiC models/main.tex}
            \input{2 - kinetic component/main.tex}
            \input{3 - fluid component/main.tex}
            \input{4 - numerical implementation/main.tex}
        \part{Project Overview}
            \input{5 - research plan/main.tex}
            \input{6 - summary/main.tex}
    
    
    %\section{}
    \newpage
    \pagenumbering{gobble}
        \printbibliography


    \newpage
    \pagenumbering{roman}
    \appendix
        \part{Appendices}
            \input{8 - Hilbert complexes/main.tex}
            \input{9 - weak conservation proofs/main.tex}
\end{document}


\title{\BA{Title in Progress...}}
\author{Boris Andrews}
\affil{Mathematical Institute, University of Oxford}
\date{\today}


\begin{document}
    \pagenumbering{gobble}
    \maketitle
    
    
    \begin{abstract}
        Magnetic confinement reactors---in particular tokamaks---offer one of the most promising options for achieving practical nuclear fusion, with the potential to provide virtually limitless, clean energy. The theoretical and numerical modeling of tokamak plasmas is simultaneously an essential component of effective reactor design, and a great research barrier. Tokamak operational conditions exhibit comparatively low Knudsen numbers. Kinetic effects, including kinetic waves and instabilities, Landau damping, bump-on-tail instabilities and more, are therefore highly influential in tokamak plasma dynamics. Purely fluid models are inherently incapable of capturing these effects, whereas the high dimensionality in purely kinetic models render them practically intractable for most relevant purposes.

        We consider a $\delta\!f$ decomposition model, with a macroscopic fluid background and microscopic kinetic correction, both fully coupled to each other. A similar manner of discretization is proposed to that used in the recent \texttt{STRUPHY} code \cite{Holderied_Possanner_Wang_2021, Holderied_2022, Li_et_al_2023} with a finite-element model for the background and a pseudo-particle/PiC model for the correction.

        The fluid background satisfies the full, non-linear, resistive, compressible, Hall MHD equations. \cite{Laakmann_Hu_Farrell_2022} introduces finite-element(-in-space) implicit timesteppers for the incompressible analogue to this system with structure-preserving (SP) properties in the ideal case, alongside parameter-robust preconditioners. We show that these timesteppers can derive from a finite-element-in-time (FET) (and finite-element-in-space) interpretation. The benefits of this reformulation are discussed, including the derivation of timesteppers that are higher order in time, and the quantifiable dissipative SP properties in the non-ideal, resistive case.
        
        We discuss possible options for extending this FET approach to timesteppers for the compressible case.

        The kinetic corrections satisfy linearized Boltzmann equations. Using a Lénard--Bernstein collision operator, these take Fokker--Planck-like forms \cite{Fokker_1914, Planck_1917} wherein pseudo-particles in the numerical model obey the neoclassical transport equations, with particle-independent Brownian drift terms. This offers a rigorous methodology for incorporating collisions into the particle transport model, without coupling the equations of motions for each particle.
        
        Works by Chen, Chacón et al. \cite{Chen_Chacón_Barnes_2011, Chacón_Chen_Barnes_2013, Chen_Chacón_2014, Chen_Chacón_2015} have developed structure-preserving particle pushers for neoclassical transport in the Vlasov equations, derived from Crank--Nicolson integrators. We show these too can can derive from a FET interpretation, similarly offering potential extensions to higher-order-in-time particle pushers. The FET formulation is used also to consider how the stochastic drift terms can be incorporated into the pushers. Stochastic gyrokinetic expansions are also discussed.

        Different options for the numerical implementation of these schemes are considered.

        Due to the efficacy of FET in the development of SP timesteppers for both the fluid and kinetic component, we hope this approach will prove effective in the future for developing SP timesteppers for the full hybrid model. We hope this will give us the opportunity to incorporate previously inaccessible kinetic effects into the highly effective, modern, finite-element MHD models.
    \end{abstract}
    
    
    \newpage
    \tableofcontents
    
    
    \newpage
    \pagenumbering{arabic}
    %\linenumbers\renewcommand\thelinenumber{\color{black!50}\arabic{linenumber}}
            \documentclass[12pt, a4paper]{report}

\input{template/main.tex}

\title{\BA{Title in Progress...}}
\author{Boris Andrews}
\affil{Mathematical Institute, University of Oxford}
\date{\today}


\begin{document}
    \pagenumbering{gobble}
    \maketitle
    
    
    \begin{abstract}
        Magnetic confinement reactors---in particular tokamaks---offer one of the most promising options for achieving practical nuclear fusion, with the potential to provide virtually limitless, clean energy. The theoretical and numerical modeling of tokamak plasmas is simultaneously an essential component of effective reactor design, and a great research barrier. Tokamak operational conditions exhibit comparatively low Knudsen numbers. Kinetic effects, including kinetic waves and instabilities, Landau damping, bump-on-tail instabilities and more, are therefore highly influential in tokamak plasma dynamics. Purely fluid models are inherently incapable of capturing these effects, whereas the high dimensionality in purely kinetic models render them practically intractable for most relevant purposes.

        We consider a $\delta\!f$ decomposition model, with a macroscopic fluid background and microscopic kinetic correction, both fully coupled to each other. A similar manner of discretization is proposed to that used in the recent \texttt{STRUPHY} code \cite{Holderied_Possanner_Wang_2021, Holderied_2022, Li_et_al_2023} with a finite-element model for the background and a pseudo-particle/PiC model for the correction.

        The fluid background satisfies the full, non-linear, resistive, compressible, Hall MHD equations. \cite{Laakmann_Hu_Farrell_2022} introduces finite-element(-in-space) implicit timesteppers for the incompressible analogue to this system with structure-preserving (SP) properties in the ideal case, alongside parameter-robust preconditioners. We show that these timesteppers can derive from a finite-element-in-time (FET) (and finite-element-in-space) interpretation. The benefits of this reformulation are discussed, including the derivation of timesteppers that are higher order in time, and the quantifiable dissipative SP properties in the non-ideal, resistive case.
        
        We discuss possible options for extending this FET approach to timesteppers for the compressible case.

        The kinetic corrections satisfy linearized Boltzmann equations. Using a Lénard--Bernstein collision operator, these take Fokker--Planck-like forms \cite{Fokker_1914, Planck_1917} wherein pseudo-particles in the numerical model obey the neoclassical transport equations, with particle-independent Brownian drift terms. This offers a rigorous methodology for incorporating collisions into the particle transport model, without coupling the equations of motions for each particle.
        
        Works by Chen, Chacón et al. \cite{Chen_Chacón_Barnes_2011, Chacón_Chen_Barnes_2013, Chen_Chacón_2014, Chen_Chacón_2015} have developed structure-preserving particle pushers for neoclassical transport in the Vlasov equations, derived from Crank--Nicolson integrators. We show these too can can derive from a FET interpretation, similarly offering potential extensions to higher-order-in-time particle pushers. The FET formulation is used also to consider how the stochastic drift terms can be incorporated into the pushers. Stochastic gyrokinetic expansions are also discussed.

        Different options for the numerical implementation of these schemes are considered.

        Due to the efficacy of FET in the development of SP timesteppers for both the fluid and kinetic component, we hope this approach will prove effective in the future for developing SP timesteppers for the full hybrid model. We hope this will give us the opportunity to incorporate previously inaccessible kinetic effects into the highly effective, modern, finite-element MHD models.
    \end{abstract}
    
    
    \newpage
    \tableofcontents
    
    
    \newpage
    \pagenumbering{arabic}
    %\linenumbers\renewcommand\thelinenumber{\color{black!50}\arabic{linenumber}}
            \input{0 - introduction/main.tex}
        \part{Research}
            \input{1 - low-noise PiC models/main.tex}
            \input{2 - kinetic component/main.tex}
            \input{3 - fluid component/main.tex}
            \input{4 - numerical implementation/main.tex}
        \part{Project Overview}
            \input{5 - research plan/main.tex}
            \input{6 - summary/main.tex}
    
    
    %\section{}
    \newpage
    \pagenumbering{gobble}
        \printbibliography


    \newpage
    \pagenumbering{roman}
    \appendix
        \part{Appendices}
            \input{8 - Hilbert complexes/main.tex}
            \input{9 - weak conservation proofs/main.tex}
\end{document}

        \part{Research}
            \documentclass[12pt, a4paper]{report}

\input{template/main.tex}

\title{\BA{Title in Progress...}}
\author{Boris Andrews}
\affil{Mathematical Institute, University of Oxford}
\date{\today}


\begin{document}
    \pagenumbering{gobble}
    \maketitle
    
    
    \begin{abstract}
        Magnetic confinement reactors---in particular tokamaks---offer one of the most promising options for achieving practical nuclear fusion, with the potential to provide virtually limitless, clean energy. The theoretical and numerical modeling of tokamak plasmas is simultaneously an essential component of effective reactor design, and a great research barrier. Tokamak operational conditions exhibit comparatively low Knudsen numbers. Kinetic effects, including kinetic waves and instabilities, Landau damping, bump-on-tail instabilities and more, are therefore highly influential in tokamak plasma dynamics. Purely fluid models are inherently incapable of capturing these effects, whereas the high dimensionality in purely kinetic models render them practically intractable for most relevant purposes.

        We consider a $\delta\!f$ decomposition model, with a macroscopic fluid background and microscopic kinetic correction, both fully coupled to each other. A similar manner of discretization is proposed to that used in the recent \texttt{STRUPHY} code \cite{Holderied_Possanner_Wang_2021, Holderied_2022, Li_et_al_2023} with a finite-element model for the background and a pseudo-particle/PiC model for the correction.

        The fluid background satisfies the full, non-linear, resistive, compressible, Hall MHD equations. \cite{Laakmann_Hu_Farrell_2022} introduces finite-element(-in-space) implicit timesteppers for the incompressible analogue to this system with structure-preserving (SP) properties in the ideal case, alongside parameter-robust preconditioners. We show that these timesteppers can derive from a finite-element-in-time (FET) (and finite-element-in-space) interpretation. The benefits of this reformulation are discussed, including the derivation of timesteppers that are higher order in time, and the quantifiable dissipative SP properties in the non-ideal, resistive case.
        
        We discuss possible options for extending this FET approach to timesteppers for the compressible case.

        The kinetic corrections satisfy linearized Boltzmann equations. Using a Lénard--Bernstein collision operator, these take Fokker--Planck-like forms \cite{Fokker_1914, Planck_1917} wherein pseudo-particles in the numerical model obey the neoclassical transport equations, with particle-independent Brownian drift terms. This offers a rigorous methodology for incorporating collisions into the particle transport model, without coupling the equations of motions for each particle.
        
        Works by Chen, Chacón et al. \cite{Chen_Chacón_Barnes_2011, Chacón_Chen_Barnes_2013, Chen_Chacón_2014, Chen_Chacón_2015} have developed structure-preserving particle pushers for neoclassical transport in the Vlasov equations, derived from Crank--Nicolson integrators. We show these too can can derive from a FET interpretation, similarly offering potential extensions to higher-order-in-time particle pushers. The FET formulation is used also to consider how the stochastic drift terms can be incorporated into the pushers. Stochastic gyrokinetic expansions are also discussed.

        Different options for the numerical implementation of these schemes are considered.

        Due to the efficacy of FET in the development of SP timesteppers for both the fluid and kinetic component, we hope this approach will prove effective in the future for developing SP timesteppers for the full hybrid model. We hope this will give us the opportunity to incorporate previously inaccessible kinetic effects into the highly effective, modern, finite-element MHD models.
    \end{abstract}
    
    
    \newpage
    \tableofcontents
    
    
    \newpage
    \pagenumbering{arabic}
    %\linenumbers\renewcommand\thelinenumber{\color{black!50}\arabic{linenumber}}
            \input{0 - introduction/main.tex}
        \part{Research}
            \input{1 - low-noise PiC models/main.tex}
            \input{2 - kinetic component/main.tex}
            \input{3 - fluid component/main.tex}
            \input{4 - numerical implementation/main.tex}
        \part{Project Overview}
            \input{5 - research plan/main.tex}
            \input{6 - summary/main.tex}
    
    
    %\section{}
    \newpage
    \pagenumbering{gobble}
        \printbibliography


    \newpage
    \pagenumbering{roman}
    \appendix
        \part{Appendices}
            \input{8 - Hilbert complexes/main.tex}
            \input{9 - weak conservation proofs/main.tex}
\end{document}

            \documentclass[12pt, a4paper]{report}

\input{template/main.tex}

\title{\BA{Title in Progress...}}
\author{Boris Andrews}
\affil{Mathematical Institute, University of Oxford}
\date{\today}


\begin{document}
    \pagenumbering{gobble}
    \maketitle
    
    
    \begin{abstract}
        Magnetic confinement reactors---in particular tokamaks---offer one of the most promising options for achieving practical nuclear fusion, with the potential to provide virtually limitless, clean energy. The theoretical and numerical modeling of tokamak plasmas is simultaneously an essential component of effective reactor design, and a great research barrier. Tokamak operational conditions exhibit comparatively low Knudsen numbers. Kinetic effects, including kinetic waves and instabilities, Landau damping, bump-on-tail instabilities and more, are therefore highly influential in tokamak plasma dynamics. Purely fluid models are inherently incapable of capturing these effects, whereas the high dimensionality in purely kinetic models render them practically intractable for most relevant purposes.

        We consider a $\delta\!f$ decomposition model, with a macroscopic fluid background and microscopic kinetic correction, both fully coupled to each other. A similar manner of discretization is proposed to that used in the recent \texttt{STRUPHY} code \cite{Holderied_Possanner_Wang_2021, Holderied_2022, Li_et_al_2023} with a finite-element model for the background and a pseudo-particle/PiC model for the correction.

        The fluid background satisfies the full, non-linear, resistive, compressible, Hall MHD equations. \cite{Laakmann_Hu_Farrell_2022} introduces finite-element(-in-space) implicit timesteppers for the incompressible analogue to this system with structure-preserving (SP) properties in the ideal case, alongside parameter-robust preconditioners. We show that these timesteppers can derive from a finite-element-in-time (FET) (and finite-element-in-space) interpretation. The benefits of this reformulation are discussed, including the derivation of timesteppers that are higher order in time, and the quantifiable dissipative SP properties in the non-ideal, resistive case.
        
        We discuss possible options for extending this FET approach to timesteppers for the compressible case.

        The kinetic corrections satisfy linearized Boltzmann equations. Using a Lénard--Bernstein collision operator, these take Fokker--Planck-like forms \cite{Fokker_1914, Planck_1917} wherein pseudo-particles in the numerical model obey the neoclassical transport equations, with particle-independent Brownian drift terms. This offers a rigorous methodology for incorporating collisions into the particle transport model, without coupling the equations of motions for each particle.
        
        Works by Chen, Chacón et al. \cite{Chen_Chacón_Barnes_2011, Chacón_Chen_Barnes_2013, Chen_Chacón_2014, Chen_Chacón_2015} have developed structure-preserving particle pushers for neoclassical transport in the Vlasov equations, derived from Crank--Nicolson integrators. We show these too can can derive from a FET interpretation, similarly offering potential extensions to higher-order-in-time particle pushers. The FET formulation is used also to consider how the stochastic drift terms can be incorporated into the pushers. Stochastic gyrokinetic expansions are also discussed.

        Different options for the numerical implementation of these schemes are considered.

        Due to the efficacy of FET in the development of SP timesteppers for both the fluid and kinetic component, we hope this approach will prove effective in the future for developing SP timesteppers for the full hybrid model. We hope this will give us the opportunity to incorporate previously inaccessible kinetic effects into the highly effective, modern, finite-element MHD models.
    \end{abstract}
    
    
    \newpage
    \tableofcontents
    
    
    \newpage
    \pagenumbering{arabic}
    %\linenumbers\renewcommand\thelinenumber{\color{black!50}\arabic{linenumber}}
            \input{0 - introduction/main.tex}
        \part{Research}
            \input{1 - low-noise PiC models/main.tex}
            \input{2 - kinetic component/main.tex}
            \input{3 - fluid component/main.tex}
            \input{4 - numerical implementation/main.tex}
        \part{Project Overview}
            \input{5 - research plan/main.tex}
            \input{6 - summary/main.tex}
    
    
    %\section{}
    \newpage
    \pagenumbering{gobble}
        \printbibliography


    \newpage
    \pagenumbering{roman}
    \appendix
        \part{Appendices}
            \input{8 - Hilbert complexes/main.tex}
            \input{9 - weak conservation proofs/main.tex}
\end{document}

            \documentclass[12pt, a4paper]{report}

\input{template/main.tex}

\title{\BA{Title in Progress...}}
\author{Boris Andrews}
\affil{Mathematical Institute, University of Oxford}
\date{\today}


\begin{document}
    \pagenumbering{gobble}
    \maketitle
    
    
    \begin{abstract}
        Magnetic confinement reactors---in particular tokamaks---offer one of the most promising options for achieving practical nuclear fusion, with the potential to provide virtually limitless, clean energy. The theoretical and numerical modeling of tokamak plasmas is simultaneously an essential component of effective reactor design, and a great research barrier. Tokamak operational conditions exhibit comparatively low Knudsen numbers. Kinetic effects, including kinetic waves and instabilities, Landau damping, bump-on-tail instabilities and more, are therefore highly influential in tokamak plasma dynamics. Purely fluid models are inherently incapable of capturing these effects, whereas the high dimensionality in purely kinetic models render them practically intractable for most relevant purposes.

        We consider a $\delta\!f$ decomposition model, with a macroscopic fluid background and microscopic kinetic correction, both fully coupled to each other. A similar manner of discretization is proposed to that used in the recent \texttt{STRUPHY} code \cite{Holderied_Possanner_Wang_2021, Holderied_2022, Li_et_al_2023} with a finite-element model for the background and a pseudo-particle/PiC model for the correction.

        The fluid background satisfies the full, non-linear, resistive, compressible, Hall MHD equations. \cite{Laakmann_Hu_Farrell_2022} introduces finite-element(-in-space) implicit timesteppers for the incompressible analogue to this system with structure-preserving (SP) properties in the ideal case, alongside parameter-robust preconditioners. We show that these timesteppers can derive from a finite-element-in-time (FET) (and finite-element-in-space) interpretation. The benefits of this reformulation are discussed, including the derivation of timesteppers that are higher order in time, and the quantifiable dissipative SP properties in the non-ideal, resistive case.
        
        We discuss possible options for extending this FET approach to timesteppers for the compressible case.

        The kinetic corrections satisfy linearized Boltzmann equations. Using a Lénard--Bernstein collision operator, these take Fokker--Planck-like forms \cite{Fokker_1914, Planck_1917} wherein pseudo-particles in the numerical model obey the neoclassical transport equations, with particle-independent Brownian drift terms. This offers a rigorous methodology for incorporating collisions into the particle transport model, without coupling the equations of motions for each particle.
        
        Works by Chen, Chacón et al. \cite{Chen_Chacón_Barnes_2011, Chacón_Chen_Barnes_2013, Chen_Chacón_2014, Chen_Chacón_2015} have developed structure-preserving particle pushers for neoclassical transport in the Vlasov equations, derived from Crank--Nicolson integrators. We show these too can can derive from a FET interpretation, similarly offering potential extensions to higher-order-in-time particle pushers. The FET formulation is used also to consider how the stochastic drift terms can be incorporated into the pushers. Stochastic gyrokinetic expansions are also discussed.

        Different options for the numerical implementation of these schemes are considered.

        Due to the efficacy of FET in the development of SP timesteppers for both the fluid and kinetic component, we hope this approach will prove effective in the future for developing SP timesteppers for the full hybrid model. We hope this will give us the opportunity to incorporate previously inaccessible kinetic effects into the highly effective, modern, finite-element MHD models.
    \end{abstract}
    
    
    \newpage
    \tableofcontents
    
    
    \newpage
    \pagenumbering{arabic}
    %\linenumbers\renewcommand\thelinenumber{\color{black!50}\arabic{linenumber}}
            \input{0 - introduction/main.tex}
        \part{Research}
            \input{1 - low-noise PiC models/main.tex}
            \input{2 - kinetic component/main.tex}
            \input{3 - fluid component/main.tex}
            \input{4 - numerical implementation/main.tex}
        \part{Project Overview}
            \input{5 - research plan/main.tex}
            \input{6 - summary/main.tex}
    
    
    %\section{}
    \newpage
    \pagenumbering{gobble}
        \printbibliography


    \newpage
    \pagenumbering{roman}
    \appendix
        \part{Appendices}
            \input{8 - Hilbert complexes/main.tex}
            \input{9 - weak conservation proofs/main.tex}
\end{document}

            \documentclass[12pt, a4paper]{report}

\input{template/main.tex}

\title{\BA{Title in Progress...}}
\author{Boris Andrews}
\affil{Mathematical Institute, University of Oxford}
\date{\today}


\begin{document}
    \pagenumbering{gobble}
    \maketitle
    
    
    \begin{abstract}
        Magnetic confinement reactors---in particular tokamaks---offer one of the most promising options for achieving practical nuclear fusion, with the potential to provide virtually limitless, clean energy. The theoretical and numerical modeling of tokamak plasmas is simultaneously an essential component of effective reactor design, and a great research barrier. Tokamak operational conditions exhibit comparatively low Knudsen numbers. Kinetic effects, including kinetic waves and instabilities, Landau damping, bump-on-tail instabilities and more, are therefore highly influential in tokamak plasma dynamics. Purely fluid models are inherently incapable of capturing these effects, whereas the high dimensionality in purely kinetic models render them practically intractable for most relevant purposes.

        We consider a $\delta\!f$ decomposition model, with a macroscopic fluid background and microscopic kinetic correction, both fully coupled to each other. A similar manner of discretization is proposed to that used in the recent \texttt{STRUPHY} code \cite{Holderied_Possanner_Wang_2021, Holderied_2022, Li_et_al_2023} with a finite-element model for the background and a pseudo-particle/PiC model for the correction.

        The fluid background satisfies the full, non-linear, resistive, compressible, Hall MHD equations. \cite{Laakmann_Hu_Farrell_2022} introduces finite-element(-in-space) implicit timesteppers for the incompressible analogue to this system with structure-preserving (SP) properties in the ideal case, alongside parameter-robust preconditioners. We show that these timesteppers can derive from a finite-element-in-time (FET) (and finite-element-in-space) interpretation. The benefits of this reformulation are discussed, including the derivation of timesteppers that are higher order in time, and the quantifiable dissipative SP properties in the non-ideal, resistive case.
        
        We discuss possible options for extending this FET approach to timesteppers for the compressible case.

        The kinetic corrections satisfy linearized Boltzmann equations. Using a Lénard--Bernstein collision operator, these take Fokker--Planck-like forms \cite{Fokker_1914, Planck_1917} wherein pseudo-particles in the numerical model obey the neoclassical transport equations, with particle-independent Brownian drift terms. This offers a rigorous methodology for incorporating collisions into the particle transport model, without coupling the equations of motions for each particle.
        
        Works by Chen, Chacón et al. \cite{Chen_Chacón_Barnes_2011, Chacón_Chen_Barnes_2013, Chen_Chacón_2014, Chen_Chacón_2015} have developed structure-preserving particle pushers for neoclassical transport in the Vlasov equations, derived from Crank--Nicolson integrators. We show these too can can derive from a FET interpretation, similarly offering potential extensions to higher-order-in-time particle pushers. The FET formulation is used also to consider how the stochastic drift terms can be incorporated into the pushers. Stochastic gyrokinetic expansions are also discussed.

        Different options for the numerical implementation of these schemes are considered.

        Due to the efficacy of FET in the development of SP timesteppers for both the fluid and kinetic component, we hope this approach will prove effective in the future for developing SP timesteppers for the full hybrid model. We hope this will give us the opportunity to incorporate previously inaccessible kinetic effects into the highly effective, modern, finite-element MHD models.
    \end{abstract}
    
    
    \newpage
    \tableofcontents
    
    
    \newpage
    \pagenumbering{arabic}
    %\linenumbers\renewcommand\thelinenumber{\color{black!50}\arabic{linenumber}}
            \input{0 - introduction/main.tex}
        \part{Research}
            \input{1 - low-noise PiC models/main.tex}
            \input{2 - kinetic component/main.tex}
            \input{3 - fluid component/main.tex}
            \input{4 - numerical implementation/main.tex}
        \part{Project Overview}
            \input{5 - research plan/main.tex}
            \input{6 - summary/main.tex}
    
    
    %\section{}
    \newpage
    \pagenumbering{gobble}
        \printbibliography


    \newpage
    \pagenumbering{roman}
    \appendix
        \part{Appendices}
            \input{8 - Hilbert complexes/main.tex}
            \input{9 - weak conservation proofs/main.tex}
\end{document}

        \part{Project Overview}
            \documentclass[12pt, a4paper]{report}

\input{template/main.tex}

\title{\BA{Title in Progress...}}
\author{Boris Andrews}
\affil{Mathematical Institute, University of Oxford}
\date{\today}


\begin{document}
    \pagenumbering{gobble}
    \maketitle
    
    
    \begin{abstract}
        Magnetic confinement reactors---in particular tokamaks---offer one of the most promising options for achieving practical nuclear fusion, with the potential to provide virtually limitless, clean energy. The theoretical and numerical modeling of tokamak plasmas is simultaneously an essential component of effective reactor design, and a great research barrier. Tokamak operational conditions exhibit comparatively low Knudsen numbers. Kinetic effects, including kinetic waves and instabilities, Landau damping, bump-on-tail instabilities and more, are therefore highly influential in tokamak plasma dynamics. Purely fluid models are inherently incapable of capturing these effects, whereas the high dimensionality in purely kinetic models render them practically intractable for most relevant purposes.

        We consider a $\delta\!f$ decomposition model, with a macroscopic fluid background and microscopic kinetic correction, both fully coupled to each other. A similar manner of discretization is proposed to that used in the recent \texttt{STRUPHY} code \cite{Holderied_Possanner_Wang_2021, Holderied_2022, Li_et_al_2023} with a finite-element model for the background and a pseudo-particle/PiC model for the correction.

        The fluid background satisfies the full, non-linear, resistive, compressible, Hall MHD equations. \cite{Laakmann_Hu_Farrell_2022} introduces finite-element(-in-space) implicit timesteppers for the incompressible analogue to this system with structure-preserving (SP) properties in the ideal case, alongside parameter-robust preconditioners. We show that these timesteppers can derive from a finite-element-in-time (FET) (and finite-element-in-space) interpretation. The benefits of this reformulation are discussed, including the derivation of timesteppers that are higher order in time, and the quantifiable dissipative SP properties in the non-ideal, resistive case.
        
        We discuss possible options for extending this FET approach to timesteppers for the compressible case.

        The kinetic corrections satisfy linearized Boltzmann equations. Using a Lénard--Bernstein collision operator, these take Fokker--Planck-like forms \cite{Fokker_1914, Planck_1917} wherein pseudo-particles in the numerical model obey the neoclassical transport equations, with particle-independent Brownian drift terms. This offers a rigorous methodology for incorporating collisions into the particle transport model, without coupling the equations of motions for each particle.
        
        Works by Chen, Chacón et al. \cite{Chen_Chacón_Barnes_2011, Chacón_Chen_Barnes_2013, Chen_Chacón_2014, Chen_Chacón_2015} have developed structure-preserving particle pushers for neoclassical transport in the Vlasov equations, derived from Crank--Nicolson integrators. We show these too can can derive from a FET interpretation, similarly offering potential extensions to higher-order-in-time particle pushers. The FET formulation is used also to consider how the stochastic drift terms can be incorporated into the pushers. Stochastic gyrokinetic expansions are also discussed.

        Different options for the numerical implementation of these schemes are considered.

        Due to the efficacy of FET in the development of SP timesteppers for both the fluid and kinetic component, we hope this approach will prove effective in the future for developing SP timesteppers for the full hybrid model. We hope this will give us the opportunity to incorporate previously inaccessible kinetic effects into the highly effective, modern, finite-element MHD models.
    \end{abstract}
    
    
    \newpage
    \tableofcontents
    
    
    \newpage
    \pagenumbering{arabic}
    %\linenumbers\renewcommand\thelinenumber{\color{black!50}\arabic{linenumber}}
            \input{0 - introduction/main.tex}
        \part{Research}
            \input{1 - low-noise PiC models/main.tex}
            \input{2 - kinetic component/main.tex}
            \input{3 - fluid component/main.tex}
            \input{4 - numerical implementation/main.tex}
        \part{Project Overview}
            \input{5 - research plan/main.tex}
            \input{6 - summary/main.tex}
    
    
    %\section{}
    \newpage
    \pagenumbering{gobble}
        \printbibliography


    \newpage
    \pagenumbering{roman}
    \appendix
        \part{Appendices}
            \input{8 - Hilbert complexes/main.tex}
            \input{9 - weak conservation proofs/main.tex}
\end{document}

            \documentclass[12pt, a4paper]{report}

\input{template/main.tex}

\title{\BA{Title in Progress...}}
\author{Boris Andrews}
\affil{Mathematical Institute, University of Oxford}
\date{\today}


\begin{document}
    \pagenumbering{gobble}
    \maketitle
    
    
    \begin{abstract}
        Magnetic confinement reactors---in particular tokamaks---offer one of the most promising options for achieving practical nuclear fusion, with the potential to provide virtually limitless, clean energy. The theoretical and numerical modeling of tokamak plasmas is simultaneously an essential component of effective reactor design, and a great research barrier. Tokamak operational conditions exhibit comparatively low Knudsen numbers. Kinetic effects, including kinetic waves and instabilities, Landau damping, bump-on-tail instabilities and more, are therefore highly influential in tokamak plasma dynamics. Purely fluid models are inherently incapable of capturing these effects, whereas the high dimensionality in purely kinetic models render them practically intractable for most relevant purposes.

        We consider a $\delta\!f$ decomposition model, with a macroscopic fluid background and microscopic kinetic correction, both fully coupled to each other. A similar manner of discretization is proposed to that used in the recent \texttt{STRUPHY} code \cite{Holderied_Possanner_Wang_2021, Holderied_2022, Li_et_al_2023} with a finite-element model for the background and a pseudo-particle/PiC model for the correction.

        The fluid background satisfies the full, non-linear, resistive, compressible, Hall MHD equations. \cite{Laakmann_Hu_Farrell_2022} introduces finite-element(-in-space) implicit timesteppers for the incompressible analogue to this system with structure-preserving (SP) properties in the ideal case, alongside parameter-robust preconditioners. We show that these timesteppers can derive from a finite-element-in-time (FET) (and finite-element-in-space) interpretation. The benefits of this reformulation are discussed, including the derivation of timesteppers that are higher order in time, and the quantifiable dissipative SP properties in the non-ideal, resistive case.
        
        We discuss possible options for extending this FET approach to timesteppers for the compressible case.

        The kinetic corrections satisfy linearized Boltzmann equations. Using a Lénard--Bernstein collision operator, these take Fokker--Planck-like forms \cite{Fokker_1914, Planck_1917} wherein pseudo-particles in the numerical model obey the neoclassical transport equations, with particle-independent Brownian drift terms. This offers a rigorous methodology for incorporating collisions into the particle transport model, without coupling the equations of motions for each particle.
        
        Works by Chen, Chacón et al. \cite{Chen_Chacón_Barnes_2011, Chacón_Chen_Barnes_2013, Chen_Chacón_2014, Chen_Chacón_2015} have developed structure-preserving particle pushers for neoclassical transport in the Vlasov equations, derived from Crank--Nicolson integrators. We show these too can can derive from a FET interpretation, similarly offering potential extensions to higher-order-in-time particle pushers. The FET formulation is used also to consider how the stochastic drift terms can be incorporated into the pushers. Stochastic gyrokinetic expansions are also discussed.

        Different options for the numerical implementation of these schemes are considered.

        Due to the efficacy of FET in the development of SP timesteppers for both the fluid and kinetic component, we hope this approach will prove effective in the future for developing SP timesteppers for the full hybrid model. We hope this will give us the opportunity to incorporate previously inaccessible kinetic effects into the highly effective, modern, finite-element MHD models.
    \end{abstract}
    
    
    \newpage
    \tableofcontents
    
    
    \newpage
    \pagenumbering{arabic}
    %\linenumbers\renewcommand\thelinenumber{\color{black!50}\arabic{linenumber}}
            \input{0 - introduction/main.tex}
        \part{Research}
            \input{1 - low-noise PiC models/main.tex}
            \input{2 - kinetic component/main.tex}
            \input{3 - fluid component/main.tex}
            \input{4 - numerical implementation/main.tex}
        \part{Project Overview}
            \input{5 - research plan/main.tex}
            \input{6 - summary/main.tex}
    
    
    %\section{}
    \newpage
    \pagenumbering{gobble}
        \printbibliography


    \newpage
    \pagenumbering{roman}
    \appendix
        \part{Appendices}
            \input{8 - Hilbert complexes/main.tex}
            \input{9 - weak conservation proofs/main.tex}
\end{document}

    
    
    %\section{}
    \newpage
    \pagenumbering{gobble}
        \printbibliography


    \newpage
    \pagenumbering{roman}
    \appendix
        \part{Appendices}
            \documentclass[12pt, a4paper]{report}

\input{template/main.tex}

\title{\BA{Title in Progress...}}
\author{Boris Andrews}
\affil{Mathematical Institute, University of Oxford}
\date{\today}


\begin{document}
    \pagenumbering{gobble}
    \maketitle
    
    
    \begin{abstract}
        Magnetic confinement reactors---in particular tokamaks---offer one of the most promising options for achieving practical nuclear fusion, with the potential to provide virtually limitless, clean energy. The theoretical and numerical modeling of tokamak plasmas is simultaneously an essential component of effective reactor design, and a great research barrier. Tokamak operational conditions exhibit comparatively low Knudsen numbers. Kinetic effects, including kinetic waves and instabilities, Landau damping, bump-on-tail instabilities and more, are therefore highly influential in tokamak plasma dynamics. Purely fluid models are inherently incapable of capturing these effects, whereas the high dimensionality in purely kinetic models render them practically intractable for most relevant purposes.

        We consider a $\delta\!f$ decomposition model, with a macroscopic fluid background and microscopic kinetic correction, both fully coupled to each other. A similar manner of discretization is proposed to that used in the recent \texttt{STRUPHY} code \cite{Holderied_Possanner_Wang_2021, Holderied_2022, Li_et_al_2023} with a finite-element model for the background and a pseudo-particle/PiC model for the correction.

        The fluid background satisfies the full, non-linear, resistive, compressible, Hall MHD equations. \cite{Laakmann_Hu_Farrell_2022} introduces finite-element(-in-space) implicit timesteppers for the incompressible analogue to this system with structure-preserving (SP) properties in the ideal case, alongside parameter-robust preconditioners. We show that these timesteppers can derive from a finite-element-in-time (FET) (and finite-element-in-space) interpretation. The benefits of this reformulation are discussed, including the derivation of timesteppers that are higher order in time, and the quantifiable dissipative SP properties in the non-ideal, resistive case.
        
        We discuss possible options for extending this FET approach to timesteppers for the compressible case.

        The kinetic corrections satisfy linearized Boltzmann equations. Using a Lénard--Bernstein collision operator, these take Fokker--Planck-like forms \cite{Fokker_1914, Planck_1917} wherein pseudo-particles in the numerical model obey the neoclassical transport equations, with particle-independent Brownian drift terms. This offers a rigorous methodology for incorporating collisions into the particle transport model, without coupling the equations of motions for each particle.
        
        Works by Chen, Chacón et al. \cite{Chen_Chacón_Barnes_2011, Chacón_Chen_Barnes_2013, Chen_Chacón_2014, Chen_Chacón_2015} have developed structure-preserving particle pushers for neoclassical transport in the Vlasov equations, derived from Crank--Nicolson integrators. We show these too can can derive from a FET interpretation, similarly offering potential extensions to higher-order-in-time particle pushers. The FET formulation is used also to consider how the stochastic drift terms can be incorporated into the pushers. Stochastic gyrokinetic expansions are also discussed.

        Different options for the numerical implementation of these schemes are considered.

        Due to the efficacy of FET in the development of SP timesteppers for both the fluid and kinetic component, we hope this approach will prove effective in the future for developing SP timesteppers for the full hybrid model. We hope this will give us the opportunity to incorporate previously inaccessible kinetic effects into the highly effective, modern, finite-element MHD models.
    \end{abstract}
    
    
    \newpage
    \tableofcontents
    
    
    \newpage
    \pagenumbering{arabic}
    %\linenumbers\renewcommand\thelinenumber{\color{black!50}\arabic{linenumber}}
            \input{0 - introduction/main.tex}
        \part{Research}
            \input{1 - low-noise PiC models/main.tex}
            \input{2 - kinetic component/main.tex}
            \input{3 - fluid component/main.tex}
            \input{4 - numerical implementation/main.tex}
        \part{Project Overview}
            \input{5 - research plan/main.tex}
            \input{6 - summary/main.tex}
    
    
    %\section{}
    \newpage
    \pagenumbering{gobble}
        \printbibliography


    \newpage
    \pagenumbering{roman}
    \appendix
        \part{Appendices}
            \input{8 - Hilbert complexes/main.tex}
            \input{9 - weak conservation proofs/main.tex}
\end{document}

            \documentclass[12pt, a4paper]{report}

\input{template/main.tex}

\title{\BA{Title in Progress...}}
\author{Boris Andrews}
\affil{Mathematical Institute, University of Oxford}
\date{\today}


\begin{document}
    \pagenumbering{gobble}
    \maketitle
    
    
    \begin{abstract}
        Magnetic confinement reactors---in particular tokamaks---offer one of the most promising options for achieving practical nuclear fusion, with the potential to provide virtually limitless, clean energy. The theoretical and numerical modeling of tokamak plasmas is simultaneously an essential component of effective reactor design, and a great research barrier. Tokamak operational conditions exhibit comparatively low Knudsen numbers. Kinetic effects, including kinetic waves and instabilities, Landau damping, bump-on-tail instabilities and more, are therefore highly influential in tokamak plasma dynamics. Purely fluid models are inherently incapable of capturing these effects, whereas the high dimensionality in purely kinetic models render them practically intractable for most relevant purposes.

        We consider a $\delta\!f$ decomposition model, with a macroscopic fluid background and microscopic kinetic correction, both fully coupled to each other. A similar manner of discretization is proposed to that used in the recent \texttt{STRUPHY} code \cite{Holderied_Possanner_Wang_2021, Holderied_2022, Li_et_al_2023} with a finite-element model for the background and a pseudo-particle/PiC model for the correction.

        The fluid background satisfies the full, non-linear, resistive, compressible, Hall MHD equations. \cite{Laakmann_Hu_Farrell_2022} introduces finite-element(-in-space) implicit timesteppers for the incompressible analogue to this system with structure-preserving (SP) properties in the ideal case, alongside parameter-robust preconditioners. We show that these timesteppers can derive from a finite-element-in-time (FET) (and finite-element-in-space) interpretation. The benefits of this reformulation are discussed, including the derivation of timesteppers that are higher order in time, and the quantifiable dissipative SP properties in the non-ideal, resistive case.
        
        We discuss possible options for extending this FET approach to timesteppers for the compressible case.

        The kinetic corrections satisfy linearized Boltzmann equations. Using a Lénard--Bernstein collision operator, these take Fokker--Planck-like forms \cite{Fokker_1914, Planck_1917} wherein pseudo-particles in the numerical model obey the neoclassical transport equations, with particle-independent Brownian drift terms. This offers a rigorous methodology for incorporating collisions into the particle transport model, without coupling the equations of motions for each particle.
        
        Works by Chen, Chacón et al. \cite{Chen_Chacón_Barnes_2011, Chacón_Chen_Barnes_2013, Chen_Chacón_2014, Chen_Chacón_2015} have developed structure-preserving particle pushers for neoclassical transport in the Vlasov equations, derived from Crank--Nicolson integrators. We show these too can can derive from a FET interpretation, similarly offering potential extensions to higher-order-in-time particle pushers. The FET formulation is used also to consider how the stochastic drift terms can be incorporated into the pushers. Stochastic gyrokinetic expansions are also discussed.

        Different options for the numerical implementation of these schemes are considered.

        Due to the efficacy of FET in the development of SP timesteppers for both the fluid and kinetic component, we hope this approach will prove effective in the future for developing SP timesteppers for the full hybrid model. We hope this will give us the opportunity to incorporate previously inaccessible kinetic effects into the highly effective, modern, finite-element MHD models.
    \end{abstract}
    
    
    \newpage
    \tableofcontents
    
    
    \newpage
    \pagenumbering{arabic}
    %\linenumbers\renewcommand\thelinenumber{\color{black!50}\arabic{linenumber}}
            \input{0 - introduction/main.tex}
        \part{Research}
            \input{1 - low-noise PiC models/main.tex}
            \input{2 - kinetic component/main.tex}
            \input{3 - fluid component/main.tex}
            \input{4 - numerical implementation/main.tex}
        \part{Project Overview}
            \input{5 - research plan/main.tex}
            \input{6 - summary/main.tex}
    
    
    %\section{}
    \newpage
    \pagenumbering{gobble}
        \printbibliography


    \newpage
    \pagenumbering{roman}
    \appendix
        \part{Appendices}
            \input{8 - Hilbert complexes/main.tex}
            \input{9 - weak conservation proofs/main.tex}
\end{document}

\end{document}

        \part{Project Overview}
            \documentclass[12pt, a4paper]{report}

\documentclass[12pt, a4paper]{report}

\input{template/main.tex}

\title{\BA{Title in Progress...}}
\author{Boris Andrews}
\affil{Mathematical Institute, University of Oxford}
\date{\today}


\begin{document}
    \pagenumbering{gobble}
    \maketitle
    
    
    \begin{abstract}
        Magnetic confinement reactors---in particular tokamaks---offer one of the most promising options for achieving practical nuclear fusion, with the potential to provide virtually limitless, clean energy. The theoretical and numerical modeling of tokamak plasmas is simultaneously an essential component of effective reactor design, and a great research barrier. Tokamak operational conditions exhibit comparatively low Knudsen numbers. Kinetic effects, including kinetic waves and instabilities, Landau damping, bump-on-tail instabilities and more, are therefore highly influential in tokamak plasma dynamics. Purely fluid models are inherently incapable of capturing these effects, whereas the high dimensionality in purely kinetic models render them practically intractable for most relevant purposes.

        We consider a $\delta\!f$ decomposition model, with a macroscopic fluid background and microscopic kinetic correction, both fully coupled to each other. A similar manner of discretization is proposed to that used in the recent \texttt{STRUPHY} code \cite{Holderied_Possanner_Wang_2021, Holderied_2022, Li_et_al_2023} with a finite-element model for the background and a pseudo-particle/PiC model for the correction.

        The fluid background satisfies the full, non-linear, resistive, compressible, Hall MHD equations. \cite{Laakmann_Hu_Farrell_2022} introduces finite-element(-in-space) implicit timesteppers for the incompressible analogue to this system with structure-preserving (SP) properties in the ideal case, alongside parameter-robust preconditioners. We show that these timesteppers can derive from a finite-element-in-time (FET) (and finite-element-in-space) interpretation. The benefits of this reformulation are discussed, including the derivation of timesteppers that are higher order in time, and the quantifiable dissipative SP properties in the non-ideal, resistive case.
        
        We discuss possible options for extending this FET approach to timesteppers for the compressible case.

        The kinetic corrections satisfy linearized Boltzmann equations. Using a Lénard--Bernstein collision operator, these take Fokker--Planck-like forms \cite{Fokker_1914, Planck_1917} wherein pseudo-particles in the numerical model obey the neoclassical transport equations, with particle-independent Brownian drift terms. This offers a rigorous methodology for incorporating collisions into the particle transport model, without coupling the equations of motions for each particle.
        
        Works by Chen, Chacón et al. \cite{Chen_Chacón_Barnes_2011, Chacón_Chen_Barnes_2013, Chen_Chacón_2014, Chen_Chacón_2015} have developed structure-preserving particle pushers for neoclassical transport in the Vlasov equations, derived from Crank--Nicolson integrators. We show these too can can derive from a FET interpretation, similarly offering potential extensions to higher-order-in-time particle pushers. The FET formulation is used also to consider how the stochastic drift terms can be incorporated into the pushers. Stochastic gyrokinetic expansions are also discussed.

        Different options for the numerical implementation of these schemes are considered.

        Due to the efficacy of FET in the development of SP timesteppers for both the fluid and kinetic component, we hope this approach will prove effective in the future for developing SP timesteppers for the full hybrid model. We hope this will give us the opportunity to incorporate previously inaccessible kinetic effects into the highly effective, modern, finite-element MHD models.
    \end{abstract}
    
    
    \newpage
    \tableofcontents
    
    
    \newpage
    \pagenumbering{arabic}
    %\linenumbers\renewcommand\thelinenumber{\color{black!50}\arabic{linenumber}}
            \input{0 - introduction/main.tex}
        \part{Research}
            \input{1 - low-noise PiC models/main.tex}
            \input{2 - kinetic component/main.tex}
            \input{3 - fluid component/main.tex}
            \input{4 - numerical implementation/main.tex}
        \part{Project Overview}
            \input{5 - research plan/main.tex}
            \input{6 - summary/main.tex}
    
    
    %\section{}
    \newpage
    \pagenumbering{gobble}
        \printbibliography


    \newpage
    \pagenumbering{roman}
    \appendix
        \part{Appendices}
            \input{8 - Hilbert complexes/main.tex}
            \input{9 - weak conservation proofs/main.tex}
\end{document}


\title{\BA{Title in Progress...}}
\author{Boris Andrews}
\affil{Mathematical Institute, University of Oxford}
\date{\today}


\begin{document}
    \pagenumbering{gobble}
    \maketitle
    
    
    \begin{abstract}
        Magnetic confinement reactors---in particular tokamaks---offer one of the most promising options for achieving practical nuclear fusion, with the potential to provide virtually limitless, clean energy. The theoretical and numerical modeling of tokamak plasmas is simultaneously an essential component of effective reactor design, and a great research barrier. Tokamak operational conditions exhibit comparatively low Knudsen numbers. Kinetic effects, including kinetic waves and instabilities, Landau damping, bump-on-tail instabilities and more, are therefore highly influential in tokamak plasma dynamics. Purely fluid models are inherently incapable of capturing these effects, whereas the high dimensionality in purely kinetic models render them practically intractable for most relevant purposes.

        We consider a $\delta\!f$ decomposition model, with a macroscopic fluid background and microscopic kinetic correction, both fully coupled to each other. A similar manner of discretization is proposed to that used in the recent \texttt{STRUPHY} code \cite{Holderied_Possanner_Wang_2021, Holderied_2022, Li_et_al_2023} with a finite-element model for the background and a pseudo-particle/PiC model for the correction.

        The fluid background satisfies the full, non-linear, resistive, compressible, Hall MHD equations. \cite{Laakmann_Hu_Farrell_2022} introduces finite-element(-in-space) implicit timesteppers for the incompressible analogue to this system with structure-preserving (SP) properties in the ideal case, alongside parameter-robust preconditioners. We show that these timesteppers can derive from a finite-element-in-time (FET) (and finite-element-in-space) interpretation. The benefits of this reformulation are discussed, including the derivation of timesteppers that are higher order in time, and the quantifiable dissipative SP properties in the non-ideal, resistive case.
        
        We discuss possible options for extending this FET approach to timesteppers for the compressible case.

        The kinetic corrections satisfy linearized Boltzmann equations. Using a Lénard--Bernstein collision operator, these take Fokker--Planck-like forms \cite{Fokker_1914, Planck_1917} wherein pseudo-particles in the numerical model obey the neoclassical transport equations, with particle-independent Brownian drift terms. This offers a rigorous methodology for incorporating collisions into the particle transport model, without coupling the equations of motions for each particle.
        
        Works by Chen, Chacón et al. \cite{Chen_Chacón_Barnes_2011, Chacón_Chen_Barnes_2013, Chen_Chacón_2014, Chen_Chacón_2015} have developed structure-preserving particle pushers for neoclassical transport in the Vlasov equations, derived from Crank--Nicolson integrators. We show these too can can derive from a FET interpretation, similarly offering potential extensions to higher-order-in-time particle pushers. The FET formulation is used also to consider how the stochastic drift terms can be incorporated into the pushers. Stochastic gyrokinetic expansions are also discussed.

        Different options for the numerical implementation of these schemes are considered.

        Due to the efficacy of FET in the development of SP timesteppers for both the fluid and kinetic component, we hope this approach will prove effective in the future for developing SP timesteppers for the full hybrid model. We hope this will give us the opportunity to incorporate previously inaccessible kinetic effects into the highly effective, modern, finite-element MHD models.
    \end{abstract}
    
    
    \newpage
    \tableofcontents
    
    
    \newpage
    \pagenumbering{arabic}
    %\linenumbers\renewcommand\thelinenumber{\color{black!50}\arabic{linenumber}}
            \documentclass[12pt, a4paper]{report}

\input{template/main.tex}

\title{\BA{Title in Progress...}}
\author{Boris Andrews}
\affil{Mathematical Institute, University of Oxford}
\date{\today}


\begin{document}
    \pagenumbering{gobble}
    \maketitle
    
    
    \begin{abstract}
        Magnetic confinement reactors---in particular tokamaks---offer one of the most promising options for achieving practical nuclear fusion, with the potential to provide virtually limitless, clean energy. The theoretical and numerical modeling of tokamak plasmas is simultaneously an essential component of effective reactor design, and a great research barrier. Tokamak operational conditions exhibit comparatively low Knudsen numbers. Kinetic effects, including kinetic waves and instabilities, Landau damping, bump-on-tail instabilities and more, are therefore highly influential in tokamak plasma dynamics. Purely fluid models are inherently incapable of capturing these effects, whereas the high dimensionality in purely kinetic models render them practically intractable for most relevant purposes.

        We consider a $\delta\!f$ decomposition model, with a macroscopic fluid background and microscopic kinetic correction, both fully coupled to each other. A similar manner of discretization is proposed to that used in the recent \texttt{STRUPHY} code \cite{Holderied_Possanner_Wang_2021, Holderied_2022, Li_et_al_2023} with a finite-element model for the background and a pseudo-particle/PiC model for the correction.

        The fluid background satisfies the full, non-linear, resistive, compressible, Hall MHD equations. \cite{Laakmann_Hu_Farrell_2022} introduces finite-element(-in-space) implicit timesteppers for the incompressible analogue to this system with structure-preserving (SP) properties in the ideal case, alongside parameter-robust preconditioners. We show that these timesteppers can derive from a finite-element-in-time (FET) (and finite-element-in-space) interpretation. The benefits of this reformulation are discussed, including the derivation of timesteppers that are higher order in time, and the quantifiable dissipative SP properties in the non-ideal, resistive case.
        
        We discuss possible options for extending this FET approach to timesteppers for the compressible case.

        The kinetic corrections satisfy linearized Boltzmann equations. Using a Lénard--Bernstein collision operator, these take Fokker--Planck-like forms \cite{Fokker_1914, Planck_1917} wherein pseudo-particles in the numerical model obey the neoclassical transport equations, with particle-independent Brownian drift terms. This offers a rigorous methodology for incorporating collisions into the particle transport model, without coupling the equations of motions for each particle.
        
        Works by Chen, Chacón et al. \cite{Chen_Chacón_Barnes_2011, Chacón_Chen_Barnes_2013, Chen_Chacón_2014, Chen_Chacón_2015} have developed structure-preserving particle pushers for neoclassical transport in the Vlasov equations, derived from Crank--Nicolson integrators. We show these too can can derive from a FET interpretation, similarly offering potential extensions to higher-order-in-time particle pushers. The FET formulation is used also to consider how the stochastic drift terms can be incorporated into the pushers. Stochastic gyrokinetic expansions are also discussed.

        Different options for the numerical implementation of these schemes are considered.

        Due to the efficacy of FET in the development of SP timesteppers for both the fluid and kinetic component, we hope this approach will prove effective in the future for developing SP timesteppers for the full hybrid model. We hope this will give us the opportunity to incorporate previously inaccessible kinetic effects into the highly effective, modern, finite-element MHD models.
    \end{abstract}
    
    
    \newpage
    \tableofcontents
    
    
    \newpage
    \pagenumbering{arabic}
    %\linenumbers\renewcommand\thelinenumber{\color{black!50}\arabic{linenumber}}
            \input{0 - introduction/main.tex}
        \part{Research}
            \input{1 - low-noise PiC models/main.tex}
            \input{2 - kinetic component/main.tex}
            \input{3 - fluid component/main.tex}
            \input{4 - numerical implementation/main.tex}
        \part{Project Overview}
            \input{5 - research plan/main.tex}
            \input{6 - summary/main.tex}
    
    
    %\section{}
    \newpage
    \pagenumbering{gobble}
        \printbibliography


    \newpage
    \pagenumbering{roman}
    \appendix
        \part{Appendices}
            \input{8 - Hilbert complexes/main.tex}
            \input{9 - weak conservation proofs/main.tex}
\end{document}

        \part{Research}
            \documentclass[12pt, a4paper]{report}

\input{template/main.tex}

\title{\BA{Title in Progress...}}
\author{Boris Andrews}
\affil{Mathematical Institute, University of Oxford}
\date{\today}


\begin{document}
    \pagenumbering{gobble}
    \maketitle
    
    
    \begin{abstract}
        Magnetic confinement reactors---in particular tokamaks---offer one of the most promising options for achieving practical nuclear fusion, with the potential to provide virtually limitless, clean energy. The theoretical and numerical modeling of tokamak plasmas is simultaneously an essential component of effective reactor design, and a great research barrier. Tokamak operational conditions exhibit comparatively low Knudsen numbers. Kinetic effects, including kinetic waves and instabilities, Landau damping, bump-on-tail instabilities and more, are therefore highly influential in tokamak plasma dynamics. Purely fluid models are inherently incapable of capturing these effects, whereas the high dimensionality in purely kinetic models render them practically intractable for most relevant purposes.

        We consider a $\delta\!f$ decomposition model, with a macroscopic fluid background and microscopic kinetic correction, both fully coupled to each other. A similar manner of discretization is proposed to that used in the recent \texttt{STRUPHY} code \cite{Holderied_Possanner_Wang_2021, Holderied_2022, Li_et_al_2023} with a finite-element model for the background and a pseudo-particle/PiC model for the correction.

        The fluid background satisfies the full, non-linear, resistive, compressible, Hall MHD equations. \cite{Laakmann_Hu_Farrell_2022} introduces finite-element(-in-space) implicit timesteppers for the incompressible analogue to this system with structure-preserving (SP) properties in the ideal case, alongside parameter-robust preconditioners. We show that these timesteppers can derive from a finite-element-in-time (FET) (and finite-element-in-space) interpretation. The benefits of this reformulation are discussed, including the derivation of timesteppers that are higher order in time, and the quantifiable dissipative SP properties in the non-ideal, resistive case.
        
        We discuss possible options for extending this FET approach to timesteppers for the compressible case.

        The kinetic corrections satisfy linearized Boltzmann equations. Using a Lénard--Bernstein collision operator, these take Fokker--Planck-like forms \cite{Fokker_1914, Planck_1917} wherein pseudo-particles in the numerical model obey the neoclassical transport equations, with particle-independent Brownian drift terms. This offers a rigorous methodology for incorporating collisions into the particle transport model, without coupling the equations of motions for each particle.
        
        Works by Chen, Chacón et al. \cite{Chen_Chacón_Barnes_2011, Chacón_Chen_Barnes_2013, Chen_Chacón_2014, Chen_Chacón_2015} have developed structure-preserving particle pushers for neoclassical transport in the Vlasov equations, derived from Crank--Nicolson integrators. We show these too can can derive from a FET interpretation, similarly offering potential extensions to higher-order-in-time particle pushers. The FET formulation is used also to consider how the stochastic drift terms can be incorporated into the pushers. Stochastic gyrokinetic expansions are also discussed.

        Different options for the numerical implementation of these schemes are considered.

        Due to the efficacy of FET in the development of SP timesteppers for both the fluid and kinetic component, we hope this approach will prove effective in the future for developing SP timesteppers for the full hybrid model. We hope this will give us the opportunity to incorporate previously inaccessible kinetic effects into the highly effective, modern, finite-element MHD models.
    \end{abstract}
    
    
    \newpage
    \tableofcontents
    
    
    \newpage
    \pagenumbering{arabic}
    %\linenumbers\renewcommand\thelinenumber{\color{black!50}\arabic{linenumber}}
            \input{0 - introduction/main.tex}
        \part{Research}
            \input{1 - low-noise PiC models/main.tex}
            \input{2 - kinetic component/main.tex}
            \input{3 - fluid component/main.tex}
            \input{4 - numerical implementation/main.tex}
        \part{Project Overview}
            \input{5 - research plan/main.tex}
            \input{6 - summary/main.tex}
    
    
    %\section{}
    \newpage
    \pagenumbering{gobble}
        \printbibliography


    \newpage
    \pagenumbering{roman}
    \appendix
        \part{Appendices}
            \input{8 - Hilbert complexes/main.tex}
            \input{9 - weak conservation proofs/main.tex}
\end{document}

            \documentclass[12pt, a4paper]{report}

\input{template/main.tex}

\title{\BA{Title in Progress...}}
\author{Boris Andrews}
\affil{Mathematical Institute, University of Oxford}
\date{\today}


\begin{document}
    \pagenumbering{gobble}
    \maketitle
    
    
    \begin{abstract}
        Magnetic confinement reactors---in particular tokamaks---offer one of the most promising options for achieving practical nuclear fusion, with the potential to provide virtually limitless, clean energy. The theoretical and numerical modeling of tokamak plasmas is simultaneously an essential component of effective reactor design, and a great research barrier. Tokamak operational conditions exhibit comparatively low Knudsen numbers. Kinetic effects, including kinetic waves and instabilities, Landau damping, bump-on-tail instabilities and more, are therefore highly influential in tokamak plasma dynamics. Purely fluid models are inherently incapable of capturing these effects, whereas the high dimensionality in purely kinetic models render them practically intractable for most relevant purposes.

        We consider a $\delta\!f$ decomposition model, with a macroscopic fluid background and microscopic kinetic correction, both fully coupled to each other. A similar manner of discretization is proposed to that used in the recent \texttt{STRUPHY} code \cite{Holderied_Possanner_Wang_2021, Holderied_2022, Li_et_al_2023} with a finite-element model for the background and a pseudo-particle/PiC model for the correction.

        The fluid background satisfies the full, non-linear, resistive, compressible, Hall MHD equations. \cite{Laakmann_Hu_Farrell_2022} introduces finite-element(-in-space) implicit timesteppers for the incompressible analogue to this system with structure-preserving (SP) properties in the ideal case, alongside parameter-robust preconditioners. We show that these timesteppers can derive from a finite-element-in-time (FET) (and finite-element-in-space) interpretation. The benefits of this reformulation are discussed, including the derivation of timesteppers that are higher order in time, and the quantifiable dissipative SP properties in the non-ideal, resistive case.
        
        We discuss possible options for extending this FET approach to timesteppers for the compressible case.

        The kinetic corrections satisfy linearized Boltzmann equations. Using a Lénard--Bernstein collision operator, these take Fokker--Planck-like forms \cite{Fokker_1914, Planck_1917} wherein pseudo-particles in the numerical model obey the neoclassical transport equations, with particle-independent Brownian drift terms. This offers a rigorous methodology for incorporating collisions into the particle transport model, without coupling the equations of motions for each particle.
        
        Works by Chen, Chacón et al. \cite{Chen_Chacón_Barnes_2011, Chacón_Chen_Barnes_2013, Chen_Chacón_2014, Chen_Chacón_2015} have developed structure-preserving particle pushers for neoclassical transport in the Vlasov equations, derived from Crank--Nicolson integrators. We show these too can can derive from a FET interpretation, similarly offering potential extensions to higher-order-in-time particle pushers. The FET formulation is used also to consider how the stochastic drift terms can be incorporated into the pushers. Stochastic gyrokinetic expansions are also discussed.

        Different options for the numerical implementation of these schemes are considered.

        Due to the efficacy of FET in the development of SP timesteppers for both the fluid and kinetic component, we hope this approach will prove effective in the future for developing SP timesteppers for the full hybrid model. We hope this will give us the opportunity to incorporate previously inaccessible kinetic effects into the highly effective, modern, finite-element MHD models.
    \end{abstract}
    
    
    \newpage
    \tableofcontents
    
    
    \newpage
    \pagenumbering{arabic}
    %\linenumbers\renewcommand\thelinenumber{\color{black!50}\arabic{linenumber}}
            \input{0 - introduction/main.tex}
        \part{Research}
            \input{1 - low-noise PiC models/main.tex}
            \input{2 - kinetic component/main.tex}
            \input{3 - fluid component/main.tex}
            \input{4 - numerical implementation/main.tex}
        \part{Project Overview}
            \input{5 - research plan/main.tex}
            \input{6 - summary/main.tex}
    
    
    %\section{}
    \newpage
    \pagenumbering{gobble}
        \printbibliography


    \newpage
    \pagenumbering{roman}
    \appendix
        \part{Appendices}
            \input{8 - Hilbert complexes/main.tex}
            \input{9 - weak conservation proofs/main.tex}
\end{document}

            \documentclass[12pt, a4paper]{report}

\input{template/main.tex}

\title{\BA{Title in Progress...}}
\author{Boris Andrews}
\affil{Mathematical Institute, University of Oxford}
\date{\today}


\begin{document}
    \pagenumbering{gobble}
    \maketitle
    
    
    \begin{abstract}
        Magnetic confinement reactors---in particular tokamaks---offer one of the most promising options for achieving practical nuclear fusion, with the potential to provide virtually limitless, clean energy. The theoretical and numerical modeling of tokamak plasmas is simultaneously an essential component of effective reactor design, and a great research barrier. Tokamak operational conditions exhibit comparatively low Knudsen numbers. Kinetic effects, including kinetic waves and instabilities, Landau damping, bump-on-tail instabilities and more, are therefore highly influential in tokamak plasma dynamics. Purely fluid models are inherently incapable of capturing these effects, whereas the high dimensionality in purely kinetic models render them practically intractable for most relevant purposes.

        We consider a $\delta\!f$ decomposition model, with a macroscopic fluid background and microscopic kinetic correction, both fully coupled to each other. A similar manner of discretization is proposed to that used in the recent \texttt{STRUPHY} code \cite{Holderied_Possanner_Wang_2021, Holderied_2022, Li_et_al_2023} with a finite-element model for the background and a pseudo-particle/PiC model for the correction.

        The fluid background satisfies the full, non-linear, resistive, compressible, Hall MHD equations. \cite{Laakmann_Hu_Farrell_2022} introduces finite-element(-in-space) implicit timesteppers for the incompressible analogue to this system with structure-preserving (SP) properties in the ideal case, alongside parameter-robust preconditioners. We show that these timesteppers can derive from a finite-element-in-time (FET) (and finite-element-in-space) interpretation. The benefits of this reformulation are discussed, including the derivation of timesteppers that are higher order in time, and the quantifiable dissipative SP properties in the non-ideal, resistive case.
        
        We discuss possible options for extending this FET approach to timesteppers for the compressible case.

        The kinetic corrections satisfy linearized Boltzmann equations. Using a Lénard--Bernstein collision operator, these take Fokker--Planck-like forms \cite{Fokker_1914, Planck_1917} wherein pseudo-particles in the numerical model obey the neoclassical transport equations, with particle-independent Brownian drift terms. This offers a rigorous methodology for incorporating collisions into the particle transport model, without coupling the equations of motions for each particle.
        
        Works by Chen, Chacón et al. \cite{Chen_Chacón_Barnes_2011, Chacón_Chen_Barnes_2013, Chen_Chacón_2014, Chen_Chacón_2015} have developed structure-preserving particle pushers for neoclassical transport in the Vlasov equations, derived from Crank--Nicolson integrators. We show these too can can derive from a FET interpretation, similarly offering potential extensions to higher-order-in-time particle pushers. The FET formulation is used also to consider how the stochastic drift terms can be incorporated into the pushers. Stochastic gyrokinetic expansions are also discussed.

        Different options for the numerical implementation of these schemes are considered.

        Due to the efficacy of FET in the development of SP timesteppers for both the fluid and kinetic component, we hope this approach will prove effective in the future for developing SP timesteppers for the full hybrid model. We hope this will give us the opportunity to incorporate previously inaccessible kinetic effects into the highly effective, modern, finite-element MHD models.
    \end{abstract}
    
    
    \newpage
    \tableofcontents
    
    
    \newpage
    \pagenumbering{arabic}
    %\linenumbers\renewcommand\thelinenumber{\color{black!50}\arabic{linenumber}}
            \input{0 - introduction/main.tex}
        \part{Research}
            \input{1 - low-noise PiC models/main.tex}
            \input{2 - kinetic component/main.tex}
            \input{3 - fluid component/main.tex}
            \input{4 - numerical implementation/main.tex}
        \part{Project Overview}
            \input{5 - research plan/main.tex}
            \input{6 - summary/main.tex}
    
    
    %\section{}
    \newpage
    \pagenumbering{gobble}
        \printbibliography


    \newpage
    \pagenumbering{roman}
    \appendix
        \part{Appendices}
            \input{8 - Hilbert complexes/main.tex}
            \input{9 - weak conservation proofs/main.tex}
\end{document}

            \documentclass[12pt, a4paper]{report}

\input{template/main.tex}

\title{\BA{Title in Progress...}}
\author{Boris Andrews}
\affil{Mathematical Institute, University of Oxford}
\date{\today}


\begin{document}
    \pagenumbering{gobble}
    \maketitle
    
    
    \begin{abstract}
        Magnetic confinement reactors---in particular tokamaks---offer one of the most promising options for achieving practical nuclear fusion, with the potential to provide virtually limitless, clean energy. The theoretical and numerical modeling of tokamak plasmas is simultaneously an essential component of effective reactor design, and a great research barrier. Tokamak operational conditions exhibit comparatively low Knudsen numbers. Kinetic effects, including kinetic waves and instabilities, Landau damping, bump-on-tail instabilities and more, are therefore highly influential in tokamak plasma dynamics. Purely fluid models are inherently incapable of capturing these effects, whereas the high dimensionality in purely kinetic models render them practically intractable for most relevant purposes.

        We consider a $\delta\!f$ decomposition model, with a macroscopic fluid background and microscopic kinetic correction, both fully coupled to each other. A similar manner of discretization is proposed to that used in the recent \texttt{STRUPHY} code \cite{Holderied_Possanner_Wang_2021, Holderied_2022, Li_et_al_2023} with a finite-element model for the background and a pseudo-particle/PiC model for the correction.

        The fluid background satisfies the full, non-linear, resistive, compressible, Hall MHD equations. \cite{Laakmann_Hu_Farrell_2022} introduces finite-element(-in-space) implicit timesteppers for the incompressible analogue to this system with structure-preserving (SP) properties in the ideal case, alongside parameter-robust preconditioners. We show that these timesteppers can derive from a finite-element-in-time (FET) (and finite-element-in-space) interpretation. The benefits of this reformulation are discussed, including the derivation of timesteppers that are higher order in time, and the quantifiable dissipative SP properties in the non-ideal, resistive case.
        
        We discuss possible options for extending this FET approach to timesteppers for the compressible case.

        The kinetic corrections satisfy linearized Boltzmann equations. Using a Lénard--Bernstein collision operator, these take Fokker--Planck-like forms \cite{Fokker_1914, Planck_1917} wherein pseudo-particles in the numerical model obey the neoclassical transport equations, with particle-independent Brownian drift terms. This offers a rigorous methodology for incorporating collisions into the particle transport model, without coupling the equations of motions for each particle.
        
        Works by Chen, Chacón et al. \cite{Chen_Chacón_Barnes_2011, Chacón_Chen_Barnes_2013, Chen_Chacón_2014, Chen_Chacón_2015} have developed structure-preserving particle pushers for neoclassical transport in the Vlasov equations, derived from Crank--Nicolson integrators. We show these too can can derive from a FET interpretation, similarly offering potential extensions to higher-order-in-time particle pushers. The FET formulation is used also to consider how the stochastic drift terms can be incorporated into the pushers. Stochastic gyrokinetic expansions are also discussed.

        Different options for the numerical implementation of these schemes are considered.

        Due to the efficacy of FET in the development of SP timesteppers for both the fluid and kinetic component, we hope this approach will prove effective in the future for developing SP timesteppers for the full hybrid model. We hope this will give us the opportunity to incorporate previously inaccessible kinetic effects into the highly effective, modern, finite-element MHD models.
    \end{abstract}
    
    
    \newpage
    \tableofcontents
    
    
    \newpage
    \pagenumbering{arabic}
    %\linenumbers\renewcommand\thelinenumber{\color{black!50}\arabic{linenumber}}
            \input{0 - introduction/main.tex}
        \part{Research}
            \input{1 - low-noise PiC models/main.tex}
            \input{2 - kinetic component/main.tex}
            \input{3 - fluid component/main.tex}
            \input{4 - numerical implementation/main.tex}
        \part{Project Overview}
            \input{5 - research plan/main.tex}
            \input{6 - summary/main.tex}
    
    
    %\section{}
    \newpage
    \pagenumbering{gobble}
        \printbibliography


    \newpage
    \pagenumbering{roman}
    \appendix
        \part{Appendices}
            \input{8 - Hilbert complexes/main.tex}
            \input{9 - weak conservation proofs/main.tex}
\end{document}

        \part{Project Overview}
            \documentclass[12pt, a4paper]{report}

\input{template/main.tex}

\title{\BA{Title in Progress...}}
\author{Boris Andrews}
\affil{Mathematical Institute, University of Oxford}
\date{\today}


\begin{document}
    \pagenumbering{gobble}
    \maketitle
    
    
    \begin{abstract}
        Magnetic confinement reactors---in particular tokamaks---offer one of the most promising options for achieving practical nuclear fusion, with the potential to provide virtually limitless, clean energy. The theoretical and numerical modeling of tokamak plasmas is simultaneously an essential component of effective reactor design, and a great research barrier. Tokamak operational conditions exhibit comparatively low Knudsen numbers. Kinetic effects, including kinetic waves and instabilities, Landau damping, bump-on-tail instabilities and more, are therefore highly influential in tokamak plasma dynamics. Purely fluid models are inherently incapable of capturing these effects, whereas the high dimensionality in purely kinetic models render them practically intractable for most relevant purposes.

        We consider a $\delta\!f$ decomposition model, with a macroscopic fluid background and microscopic kinetic correction, both fully coupled to each other. A similar manner of discretization is proposed to that used in the recent \texttt{STRUPHY} code \cite{Holderied_Possanner_Wang_2021, Holderied_2022, Li_et_al_2023} with a finite-element model for the background and a pseudo-particle/PiC model for the correction.

        The fluid background satisfies the full, non-linear, resistive, compressible, Hall MHD equations. \cite{Laakmann_Hu_Farrell_2022} introduces finite-element(-in-space) implicit timesteppers for the incompressible analogue to this system with structure-preserving (SP) properties in the ideal case, alongside parameter-robust preconditioners. We show that these timesteppers can derive from a finite-element-in-time (FET) (and finite-element-in-space) interpretation. The benefits of this reformulation are discussed, including the derivation of timesteppers that are higher order in time, and the quantifiable dissipative SP properties in the non-ideal, resistive case.
        
        We discuss possible options for extending this FET approach to timesteppers for the compressible case.

        The kinetic corrections satisfy linearized Boltzmann equations. Using a Lénard--Bernstein collision operator, these take Fokker--Planck-like forms \cite{Fokker_1914, Planck_1917} wherein pseudo-particles in the numerical model obey the neoclassical transport equations, with particle-independent Brownian drift terms. This offers a rigorous methodology for incorporating collisions into the particle transport model, without coupling the equations of motions for each particle.
        
        Works by Chen, Chacón et al. \cite{Chen_Chacón_Barnes_2011, Chacón_Chen_Barnes_2013, Chen_Chacón_2014, Chen_Chacón_2015} have developed structure-preserving particle pushers for neoclassical transport in the Vlasov equations, derived from Crank--Nicolson integrators. We show these too can can derive from a FET interpretation, similarly offering potential extensions to higher-order-in-time particle pushers. The FET formulation is used also to consider how the stochastic drift terms can be incorporated into the pushers. Stochastic gyrokinetic expansions are also discussed.

        Different options for the numerical implementation of these schemes are considered.

        Due to the efficacy of FET in the development of SP timesteppers for both the fluid and kinetic component, we hope this approach will prove effective in the future for developing SP timesteppers for the full hybrid model. We hope this will give us the opportunity to incorporate previously inaccessible kinetic effects into the highly effective, modern, finite-element MHD models.
    \end{abstract}
    
    
    \newpage
    \tableofcontents
    
    
    \newpage
    \pagenumbering{arabic}
    %\linenumbers\renewcommand\thelinenumber{\color{black!50}\arabic{linenumber}}
            \input{0 - introduction/main.tex}
        \part{Research}
            \input{1 - low-noise PiC models/main.tex}
            \input{2 - kinetic component/main.tex}
            \input{3 - fluid component/main.tex}
            \input{4 - numerical implementation/main.tex}
        \part{Project Overview}
            \input{5 - research plan/main.tex}
            \input{6 - summary/main.tex}
    
    
    %\section{}
    \newpage
    \pagenumbering{gobble}
        \printbibliography


    \newpage
    \pagenumbering{roman}
    \appendix
        \part{Appendices}
            \input{8 - Hilbert complexes/main.tex}
            \input{9 - weak conservation proofs/main.tex}
\end{document}

            \documentclass[12pt, a4paper]{report}

\input{template/main.tex}

\title{\BA{Title in Progress...}}
\author{Boris Andrews}
\affil{Mathematical Institute, University of Oxford}
\date{\today}


\begin{document}
    \pagenumbering{gobble}
    \maketitle
    
    
    \begin{abstract}
        Magnetic confinement reactors---in particular tokamaks---offer one of the most promising options for achieving practical nuclear fusion, with the potential to provide virtually limitless, clean energy. The theoretical and numerical modeling of tokamak plasmas is simultaneously an essential component of effective reactor design, and a great research barrier. Tokamak operational conditions exhibit comparatively low Knudsen numbers. Kinetic effects, including kinetic waves and instabilities, Landau damping, bump-on-tail instabilities and more, are therefore highly influential in tokamak plasma dynamics. Purely fluid models are inherently incapable of capturing these effects, whereas the high dimensionality in purely kinetic models render them practically intractable for most relevant purposes.

        We consider a $\delta\!f$ decomposition model, with a macroscopic fluid background and microscopic kinetic correction, both fully coupled to each other. A similar manner of discretization is proposed to that used in the recent \texttt{STRUPHY} code \cite{Holderied_Possanner_Wang_2021, Holderied_2022, Li_et_al_2023} with a finite-element model for the background and a pseudo-particle/PiC model for the correction.

        The fluid background satisfies the full, non-linear, resistive, compressible, Hall MHD equations. \cite{Laakmann_Hu_Farrell_2022} introduces finite-element(-in-space) implicit timesteppers for the incompressible analogue to this system with structure-preserving (SP) properties in the ideal case, alongside parameter-robust preconditioners. We show that these timesteppers can derive from a finite-element-in-time (FET) (and finite-element-in-space) interpretation. The benefits of this reformulation are discussed, including the derivation of timesteppers that are higher order in time, and the quantifiable dissipative SP properties in the non-ideal, resistive case.
        
        We discuss possible options for extending this FET approach to timesteppers for the compressible case.

        The kinetic corrections satisfy linearized Boltzmann equations. Using a Lénard--Bernstein collision operator, these take Fokker--Planck-like forms \cite{Fokker_1914, Planck_1917} wherein pseudo-particles in the numerical model obey the neoclassical transport equations, with particle-independent Brownian drift terms. This offers a rigorous methodology for incorporating collisions into the particle transport model, without coupling the equations of motions for each particle.
        
        Works by Chen, Chacón et al. \cite{Chen_Chacón_Barnes_2011, Chacón_Chen_Barnes_2013, Chen_Chacón_2014, Chen_Chacón_2015} have developed structure-preserving particle pushers for neoclassical transport in the Vlasov equations, derived from Crank--Nicolson integrators. We show these too can can derive from a FET interpretation, similarly offering potential extensions to higher-order-in-time particle pushers. The FET formulation is used also to consider how the stochastic drift terms can be incorporated into the pushers. Stochastic gyrokinetic expansions are also discussed.

        Different options for the numerical implementation of these schemes are considered.

        Due to the efficacy of FET in the development of SP timesteppers for both the fluid and kinetic component, we hope this approach will prove effective in the future for developing SP timesteppers for the full hybrid model. We hope this will give us the opportunity to incorporate previously inaccessible kinetic effects into the highly effective, modern, finite-element MHD models.
    \end{abstract}
    
    
    \newpage
    \tableofcontents
    
    
    \newpage
    \pagenumbering{arabic}
    %\linenumbers\renewcommand\thelinenumber{\color{black!50}\arabic{linenumber}}
            \input{0 - introduction/main.tex}
        \part{Research}
            \input{1 - low-noise PiC models/main.tex}
            \input{2 - kinetic component/main.tex}
            \input{3 - fluid component/main.tex}
            \input{4 - numerical implementation/main.tex}
        \part{Project Overview}
            \input{5 - research plan/main.tex}
            \input{6 - summary/main.tex}
    
    
    %\section{}
    \newpage
    \pagenumbering{gobble}
        \printbibliography


    \newpage
    \pagenumbering{roman}
    \appendix
        \part{Appendices}
            \input{8 - Hilbert complexes/main.tex}
            \input{9 - weak conservation proofs/main.tex}
\end{document}

    
    
    %\section{}
    \newpage
    \pagenumbering{gobble}
        \printbibliography


    \newpage
    \pagenumbering{roman}
    \appendix
        \part{Appendices}
            \documentclass[12pt, a4paper]{report}

\input{template/main.tex}

\title{\BA{Title in Progress...}}
\author{Boris Andrews}
\affil{Mathematical Institute, University of Oxford}
\date{\today}


\begin{document}
    \pagenumbering{gobble}
    \maketitle
    
    
    \begin{abstract}
        Magnetic confinement reactors---in particular tokamaks---offer one of the most promising options for achieving practical nuclear fusion, with the potential to provide virtually limitless, clean energy. The theoretical and numerical modeling of tokamak plasmas is simultaneously an essential component of effective reactor design, and a great research barrier. Tokamak operational conditions exhibit comparatively low Knudsen numbers. Kinetic effects, including kinetic waves and instabilities, Landau damping, bump-on-tail instabilities and more, are therefore highly influential in tokamak plasma dynamics. Purely fluid models are inherently incapable of capturing these effects, whereas the high dimensionality in purely kinetic models render them practically intractable for most relevant purposes.

        We consider a $\delta\!f$ decomposition model, with a macroscopic fluid background and microscopic kinetic correction, both fully coupled to each other. A similar manner of discretization is proposed to that used in the recent \texttt{STRUPHY} code \cite{Holderied_Possanner_Wang_2021, Holderied_2022, Li_et_al_2023} with a finite-element model for the background and a pseudo-particle/PiC model for the correction.

        The fluid background satisfies the full, non-linear, resistive, compressible, Hall MHD equations. \cite{Laakmann_Hu_Farrell_2022} introduces finite-element(-in-space) implicit timesteppers for the incompressible analogue to this system with structure-preserving (SP) properties in the ideal case, alongside parameter-robust preconditioners. We show that these timesteppers can derive from a finite-element-in-time (FET) (and finite-element-in-space) interpretation. The benefits of this reformulation are discussed, including the derivation of timesteppers that are higher order in time, and the quantifiable dissipative SP properties in the non-ideal, resistive case.
        
        We discuss possible options for extending this FET approach to timesteppers for the compressible case.

        The kinetic corrections satisfy linearized Boltzmann equations. Using a Lénard--Bernstein collision operator, these take Fokker--Planck-like forms \cite{Fokker_1914, Planck_1917} wherein pseudo-particles in the numerical model obey the neoclassical transport equations, with particle-independent Brownian drift terms. This offers a rigorous methodology for incorporating collisions into the particle transport model, without coupling the equations of motions for each particle.
        
        Works by Chen, Chacón et al. \cite{Chen_Chacón_Barnes_2011, Chacón_Chen_Barnes_2013, Chen_Chacón_2014, Chen_Chacón_2015} have developed structure-preserving particle pushers for neoclassical transport in the Vlasov equations, derived from Crank--Nicolson integrators. We show these too can can derive from a FET interpretation, similarly offering potential extensions to higher-order-in-time particle pushers. The FET formulation is used also to consider how the stochastic drift terms can be incorporated into the pushers. Stochastic gyrokinetic expansions are also discussed.

        Different options for the numerical implementation of these schemes are considered.

        Due to the efficacy of FET in the development of SP timesteppers for both the fluid and kinetic component, we hope this approach will prove effective in the future for developing SP timesteppers for the full hybrid model. We hope this will give us the opportunity to incorporate previously inaccessible kinetic effects into the highly effective, modern, finite-element MHD models.
    \end{abstract}
    
    
    \newpage
    \tableofcontents
    
    
    \newpage
    \pagenumbering{arabic}
    %\linenumbers\renewcommand\thelinenumber{\color{black!50}\arabic{linenumber}}
            \input{0 - introduction/main.tex}
        \part{Research}
            \input{1 - low-noise PiC models/main.tex}
            \input{2 - kinetic component/main.tex}
            \input{3 - fluid component/main.tex}
            \input{4 - numerical implementation/main.tex}
        \part{Project Overview}
            \input{5 - research plan/main.tex}
            \input{6 - summary/main.tex}
    
    
    %\section{}
    \newpage
    \pagenumbering{gobble}
        \printbibliography


    \newpage
    \pagenumbering{roman}
    \appendix
        \part{Appendices}
            \input{8 - Hilbert complexes/main.tex}
            \input{9 - weak conservation proofs/main.tex}
\end{document}

            \documentclass[12pt, a4paper]{report}

\input{template/main.tex}

\title{\BA{Title in Progress...}}
\author{Boris Andrews}
\affil{Mathematical Institute, University of Oxford}
\date{\today}


\begin{document}
    \pagenumbering{gobble}
    \maketitle
    
    
    \begin{abstract}
        Magnetic confinement reactors---in particular tokamaks---offer one of the most promising options for achieving practical nuclear fusion, with the potential to provide virtually limitless, clean energy. The theoretical and numerical modeling of tokamak plasmas is simultaneously an essential component of effective reactor design, and a great research barrier. Tokamak operational conditions exhibit comparatively low Knudsen numbers. Kinetic effects, including kinetic waves and instabilities, Landau damping, bump-on-tail instabilities and more, are therefore highly influential in tokamak plasma dynamics. Purely fluid models are inherently incapable of capturing these effects, whereas the high dimensionality in purely kinetic models render them practically intractable for most relevant purposes.

        We consider a $\delta\!f$ decomposition model, with a macroscopic fluid background and microscopic kinetic correction, both fully coupled to each other. A similar manner of discretization is proposed to that used in the recent \texttt{STRUPHY} code \cite{Holderied_Possanner_Wang_2021, Holderied_2022, Li_et_al_2023} with a finite-element model for the background and a pseudo-particle/PiC model for the correction.

        The fluid background satisfies the full, non-linear, resistive, compressible, Hall MHD equations. \cite{Laakmann_Hu_Farrell_2022} introduces finite-element(-in-space) implicit timesteppers for the incompressible analogue to this system with structure-preserving (SP) properties in the ideal case, alongside parameter-robust preconditioners. We show that these timesteppers can derive from a finite-element-in-time (FET) (and finite-element-in-space) interpretation. The benefits of this reformulation are discussed, including the derivation of timesteppers that are higher order in time, and the quantifiable dissipative SP properties in the non-ideal, resistive case.
        
        We discuss possible options for extending this FET approach to timesteppers for the compressible case.

        The kinetic corrections satisfy linearized Boltzmann equations. Using a Lénard--Bernstein collision operator, these take Fokker--Planck-like forms \cite{Fokker_1914, Planck_1917} wherein pseudo-particles in the numerical model obey the neoclassical transport equations, with particle-independent Brownian drift terms. This offers a rigorous methodology for incorporating collisions into the particle transport model, without coupling the equations of motions for each particle.
        
        Works by Chen, Chacón et al. \cite{Chen_Chacón_Barnes_2011, Chacón_Chen_Barnes_2013, Chen_Chacón_2014, Chen_Chacón_2015} have developed structure-preserving particle pushers for neoclassical transport in the Vlasov equations, derived from Crank--Nicolson integrators. We show these too can can derive from a FET interpretation, similarly offering potential extensions to higher-order-in-time particle pushers. The FET formulation is used also to consider how the stochastic drift terms can be incorporated into the pushers. Stochastic gyrokinetic expansions are also discussed.

        Different options for the numerical implementation of these schemes are considered.

        Due to the efficacy of FET in the development of SP timesteppers for both the fluid and kinetic component, we hope this approach will prove effective in the future for developing SP timesteppers for the full hybrid model. We hope this will give us the opportunity to incorporate previously inaccessible kinetic effects into the highly effective, modern, finite-element MHD models.
    \end{abstract}
    
    
    \newpage
    \tableofcontents
    
    
    \newpage
    \pagenumbering{arabic}
    %\linenumbers\renewcommand\thelinenumber{\color{black!50}\arabic{linenumber}}
            \input{0 - introduction/main.tex}
        \part{Research}
            \input{1 - low-noise PiC models/main.tex}
            \input{2 - kinetic component/main.tex}
            \input{3 - fluid component/main.tex}
            \input{4 - numerical implementation/main.tex}
        \part{Project Overview}
            \input{5 - research plan/main.tex}
            \input{6 - summary/main.tex}
    
    
    %\section{}
    \newpage
    \pagenumbering{gobble}
        \printbibliography


    \newpage
    \pagenumbering{roman}
    \appendix
        \part{Appendices}
            \input{8 - Hilbert complexes/main.tex}
            \input{9 - weak conservation proofs/main.tex}
\end{document}

\end{document}

            \documentclass[12pt, a4paper]{report}

\documentclass[12pt, a4paper]{report}

\input{template/main.tex}

\title{\BA{Title in Progress...}}
\author{Boris Andrews}
\affil{Mathematical Institute, University of Oxford}
\date{\today}


\begin{document}
    \pagenumbering{gobble}
    \maketitle
    
    
    \begin{abstract}
        Magnetic confinement reactors---in particular tokamaks---offer one of the most promising options for achieving practical nuclear fusion, with the potential to provide virtually limitless, clean energy. The theoretical and numerical modeling of tokamak plasmas is simultaneously an essential component of effective reactor design, and a great research barrier. Tokamak operational conditions exhibit comparatively low Knudsen numbers. Kinetic effects, including kinetic waves and instabilities, Landau damping, bump-on-tail instabilities and more, are therefore highly influential in tokamak plasma dynamics. Purely fluid models are inherently incapable of capturing these effects, whereas the high dimensionality in purely kinetic models render them practically intractable for most relevant purposes.

        We consider a $\delta\!f$ decomposition model, with a macroscopic fluid background and microscopic kinetic correction, both fully coupled to each other. A similar manner of discretization is proposed to that used in the recent \texttt{STRUPHY} code \cite{Holderied_Possanner_Wang_2021, Holderied_2022, Li_et_al_2023} with a finite-element model for the background and a pseudo-particle/PiC model for the correction.

        The fluid background satisfies the full, non-linear, resistive, compressible, Hall MHD equations. \cite{Laakmann_Hu_Farrell_2022} introduces finite-element(-in-space) implicit timesteppers for the incompressible analogue to this system with structure-preserving (SP) properties in the ideal case, alongside parameter-robust preconditioners. We show that these timesteppers can derive from a finite-element-in-time (FET) (and finite-element-in-space) interpretation. The benefits of this reformulation are discussed, including the derivation of timesteppers that are higher order in time, and the quantifiable dissipative SP properties in the non-ideal, resistive case.
        
        We discuss possible options for extending this FET approach to timesteppers for the compressible case.

        The kinetic corrections satisfy linearized Boltzmann equations. Using a Lénard--Bernstein collision operator, these take Fokker--Planck-like forms \cite{Fokker_1914, Planck_1917} wherein pseudo-particles in the numerical model obey the neoclassical transport equations, with particle-independent Brownian drift terms. This offers a rigorous methodology for incorporating collisions into the particle transport model, without coupling the equations of motions for each particle.
        
        Works by Chen, Chacón et al. \cite{Chen_Chacón_Barnes_2011, Chacón_Chen_Barnes_2013, Chen_Chacón_2014, Chen_Chacón_2015} have developed structure-preserving particle pushers for neoclassical transport in the Vlasov equations, derived from Crank--Nicolson integrators. We show these too can can derive from a FET interpretation, similarly offering potential extensions to higher-order-in-time particle pushers. The FET formulation is used also to consider how the stochastic drift terms can be incorporated into the pushers. Stochastic gyrokinetic expansions are also discussed.

        Different options for the numerical implementation of these schemes are considered.

        Due to the efficacy of FET in the development of SP timesteppers for both the fluid and kinetic component, we hope this approach will prove effective in the future for developing SP timesteppers for the full hybrid model. We hope this will give us the opportunity to incorporate previously inaccessible kinetic effects into the highly effective, modern, finite-element MHD models.
    \end{abstract}
    
    
    \newpage
    \tableofcontents
    
    
    \newpage
    \pagenumbering{arabic}
    %\linenumbers\renewcommand\thelinenumber{\color{black!50}\arabic{linenumber}}
            \input{0 - introduction/main.tex}
        \part{Research}
            \input{1 - low-noise PiC models/main.tex}
            \input{2 - kinetic component/main.tex}
            \input{3 - fluid component/main.tex}
            \input{4 - numerical implementation/main.tex}
        \part{Project Overview}
            \input{5 - research plan/main.tex}
            \input{6 - summary/main.tex}
    
    
    %\section{}
    \newpage
    \pagenumbering{gobble}
        \printbibliography


    \newpage
    \pagenumbering{roman}
    \appendix
        \part{Appendices}
            \input{8 - Hilbert complexes/main.tex}
            \input{9 - weak conservation proofs/main.tex}
\end{document}


\title{\BA{Title in Progress...}}
\author{Boris Andrews}
\affil{Mathematical Institute, University of Oxford}
\date{\today}


\begin{document}
    \pagenumbering{gobble}
    \maketitle
    
    
    \begin{abstract}
        Magnetic confinement reactors---in particular tokamaks---offer one of the most promising options for achieving practical nuclear fusion, with the potential to provide virtually limitless, clean energy. The theoretical and numerical modeling of tokamak plasmas is simultaneously an essential component of effective reactor design, and a great research barrier. Tokamak operational conditions exhibit comparatively low Knudsen numbers. Kinetic effects, including kinetic waves and instabilities, Landau damping, bump-on-tail instabilities and more, are therefore highly influential in tokamak plasma dynamics. Purely fluid models are inherently incapable of capturing these effects, whereas the high dimensionality in purely kinetic models render them practically intractable for most relevant purposes.

        We consider a $\delta\!f$ decomposition model, with a macroscopic fluid background and microscopic kinetic correction, both fully coupled to each other. A similar manner of discretization is proposed to that used in the recent \texttt{STRUPHY} code \cite{Holderied_Possanner_Wang_2021, Holderied_2022, Li_et_al_2023} with a finite-element model for the background and a pseudo-particle/PiC model for the correction.

        The fluid background satisfies the full, non-linear, resistive, compressible, Hall MHD equations. \cite{Laakmann_Hu_Farrell_2022} introduces finite-element(-in-space) implicit timesteppers for the incompressible analogue to this system with structure-preserving (SP) properties in the ideal case, alongside parameter-robust preconditioners. We show that these timesteppers can derive from a finite-element-in-time (FET) (and finite-element-in-space) interpretation. The benefits of this reformulation are discussed, including the derivation of timesteppers that are higher order in time, and the quantifiable dissipative SP properties in the non-ideal, resistive case.
        
        We discuss possible options for extending this FET approach to timesteppers for the compressible case.

        The kinetic corrections satisfy linearized Boltzmann equations. Using a Lénard--Bernstein collision operator, these take Fokker--Planck-like forms \cite{Fokker_1914, Planck_1917} wherein pseudo-particles in the numerical model obey the neoclassical transport equations, with particle-independent Brownian drift terms. This offers a rigorous methodology for incorporating collisions into the particle transport model, without coupling the equations of motions for each particle.
        
        Works by Chen, Chacón et al. \cite{Chen_Chacón_Barnes_2011, Chacón_Chen_Barnes_2013, Chen_Chacón_2014, Chen_Chacón_2015} have developed structure-preserving particle pushers for neoclassical transport in the Vlasov equations, derived from Crank--Nicolson integrators. We show these too can can derive from a FET interpretation, similarly offering potential extensions to higher-order-in-time particle pushers. The FET formulation is used also to consider how the stochastic drift terms can be incorporated into the pushers. Stochastic gyrokinetic expansions are also discussed.

        Different options for the numerical implementation of these schemes are considered.

        Due to the efficacy of FET in the development of SP timesteppers for both the fluid and kinetic component, we hope this approach will prove effective in the future for developing SP timesteppers for the full hybrid model. We hope this will give us the opportunity to incorporate previously inaccessible kinetic effects into the highly effective, modern, finite-element MHD models.
    \end{abstract}
    
    
    \newpage
    \tableofcontents
    
    
    \newpage
    \pagenumbering{arabic}
    %\linenumbers\renewcommand\thelinenumber{\color{black!50}\arabic{linenumber}}
            \documentclass[12pt, a4paper]{report}

\input{template/main.tex}

\title{\BA{Title in Progress...}}
\author{Boris Andrews}
\affil{Mathematical Institute, University of Oxford}
\date{\today}


\begin{document}
    \pagenumbering{gobble}
    \maketitle
    
    
    \begin{abstract}
        Magnetic confinement reactors---in particular tokamaks---offer one of the most promising options for achieving practical nuclear fusion, with the potential to provide virtually limitless, clean energy. The theoretical and numerical modeling of tokamak plasmas is simultaneously an essential component of effective reactor design, and a great research barrier. Tokamak operational conditions exhibit comparatively low Knudsen numbers. Kinetic effects, including kinetic waves and instabilities, Landau damping, bump-on-tail instabilities and more, are therefore highly influential in tokamak plasma dynamics. Purely fluid models are inherently incapable of capturing these effects, whereas the high dimensionality in purely kinetic models render them practically intractable for most relevant purposes.

        We consider a $\delta\!f$ decomposition model, with a macroscopic fluid background and microscopic kinetic correction, both fully coupled to each other. A similar manner of discretization is proposed to that used in the recent \texttt{STRUPHY} code \cite{Holderied_Possanner_Wang_2021, Holderied_2022, Li_et_al_2023} with a finite-element model for the background and a pseudo-particle/PiC model for the correction.

        The fluid background satisfies the full, non-linear, resistive, compressible, Hall MHD equations. \cite{Laakmann_Hu_Farrell_2022} introduces finite-element(-in-space) implicit timesteppers for the incompressible analogue to this system with structure-preserving (SP) properties in the ideal case, alongside parameter-robust preconditioners. We show that these timesteppers can derive from a finite-element-in-time (FET) (and finite-element-in-space) interpretation. The benefits of this reformulation are discussed, including the derivation of timesteppers that are higher order in time, and the quantifiable dissipative SP properties in the non-ideal, resistive case.
        
        We discuss possible options for extending this FET approach to timesteppers for the compressible case.

        The kinetic corrections satisfy linearized Boltzmann equations. Using a Lénard--Bernstein collision operator, these take Fokker--Planck-like forms \cite{Fokker_1914, Planck_1917} wherein pseudo-particles in the numerical model obey the neoclassical transport equations, with particle-independent Brownian drift terms. This offers a rigorous methodology for incorporating collisions into the particle transport model, without coupling the equations of motions for each particle.
        
        Works by Chen, Chacón et al. \cite{Chen_Chacón_Barnes_2011, Chacón_Chen_Barnes_2013, Chen_Chacón_2014, Chen_Chacón_2015} have developed structure-preserving particle pushers for neoclassical transport in the Vlasov equations, derived from Crank--Nicolson integrators. We show these too can can derive from a FET interpretation, similarly offering potential extensions to higher-order-in-time particle pushers. The FET formulation is used also to consider how the stochastic drift terms can be incorporated into the pushers. Stochastic gyrokinetic expansions are also discussed.

        Different options for the numerical implementation of these schemes are considered.

        Due to the efficacy of FET in the development of SP timesteppers for both the fluid and kinetic component, we hope this approach will prove effective in the future for developing SP timesteppers for the full hybrid model. We hope this will give us the opportunity to incorporate previously inaccessible kinetic effects into the highly effective, modern, finite-element MHD models.
    \end{abstract}
    
    
    \newpage
    \tableofcontents
    
    
    \newpage
    \pagenumbering{arabic}
    %\linenumbers\renewcommand\thelinenumber{\color{black!50}\arabic{linenumber}}
            \input{0 - introduction/main.tex}
        \part{Research}
            \input{1 - low-noise PiC models/main.tex}
            \input{2 - kinetic component/main.tex}
            \input{3 - fluid component/main.tex}
            \input{4 - numerical implementation/main.tex}
        \part{Project Overview}
            \input{5 - research plan/main.tex}
            \input{6 - summary/main.tex}
    
    
    %\section{}
    \newpage
    \pagenumbering{gobble}
        \printbibliography


    \newpage
    \pagenumbering{roman}
    \appendix
        \part{Appendices}
            \input{8 - Hilbert complexes/main.tex}
            \input{9 - weak conservation proofs/main.tex}
\end{document}

        \part{Research}
            \documentclass[12pt, a4paper]{report}

\input{template/main.tex}

\title{\BA{Title in Progress...}}
\author{Boris Andrews}
\affil{Mathematical Institute, University of Oxford}
\date{\today}


\begin{document}
    \pagenumbering{gobble}
    \maketitle
    
    
    \begin{abstract}
        Magnetic confinement reactors---in particular tokamaks---offer one of the most promising options for achieving practical nuclear fusion, with the potential to provide virtually limitless, clean energy. The theoretical and numerical modeling of tokamak plasmas is simultaneously an essential component of effective reactor design, and a great research barrier. Tokamak operational conditions exhibit comparatively low Knudsen numbers. Kinetic effects, including kinetic waves and instabilities, Landau damping, bump-on-tail instabilities and more, are therefore highly influential in tokamak plasma dynamics. Purely fluid models are inherently incapable of capturing these effects, whereas the high dimensionality in purely kinetic models render them practically intractable for most relevant purposes.

        We consider a $\delta\!f$ decomposition model, with a macroscopic fluid background and microscopic kinetic correction, both fully coupled to each other. A similar manner of discretization is proposed to that used in the recent \texttt{STRUPHY} code \cite{Holderied_Possanner_Wang_2021, Holderied_2022, Li_et_al_2023} with a finite-element model for the background and a pseudo-particle/PiC model for the correction.

        The fluid background satisfies the full, non-linear, resistive, compressible, Hall MHD equations. \cite{Laakmann_Hu_Farrell_2022} introduces finite-element(-in-space) implicit timesteppers for the incompressible analogue to this system with structure-preserving (SP) properties in the ideal case, alongside parameter-robust preconditioners. We show that these timesteppers can derive from a finite-element-in-time (FET) (and finite-element-in-space) interpretation. The benefits of this reformulation are discussed, including the derivation of timesteppers that are higher order in time, and the quantifiable dissipative SP properties in the non-ideal, resistive case.
        
        We discuss possible options for extending this FET approach to timesteppers for the compressible case.

        The kinetic corrections satisfy linearized Boltzmann equations. Using a Lénard--Bernstein collision operator, these take Fokker--Planck-like forms \cite{Fokker_1914, Planck_1917} wherein pseudo-particles in the numerical model obey the neoclassical transport equations, with particle-independent Brownian drift terms. This offers a rigorous methodology for incorporating collisions into the particle transport model, without coupling the equations of motions for each particle.
        
        Works by Chen, Chacón et al. \cite{Chen_Chacón_Barnes_2011, Chacón_Chen_Barnes_2013, Chen_Chacón_2014, Chen_Chacón_2015} have developed structure-preserving particle pushers for neoclassical transport in the Vlasov equations, derived from Crank--Nicolson integrators. We show these too can can derive from a FET interpretation, similarly offering potential extensions to higher-order-in-time particle pushers. The FET formulation is used also to consider how the stochastic drift terms can be incorporated into the pushers. Stochastic gyrokinetic expansions are also discussed.

        Different options for the numerical implementation of these schemes are considered.

        Due to the efficacy of FET in the development of SP timesteppers for both the fluid and kinetic component, we hope this approach will prove effective in the future for developing SP timesteppers for the full hybrid model. We hope this will give us the opportunity to incorporate previously inaccessible kinetic effects into the highly effective, modern, finite-element MHD models.
    \end{abstract}
    
    
    \newpage
    \tableofcontents
    
    
    \newpage
    \pagenumbering{arabic}
    %\linenumbers\renewcommand\thelinenumber{\color{black!50}\arabic{linenumber}}
            \input{0 - introduction/main.tex}
        \part{Research}
            \input{1 - low-noise PiC models/main.tex}
            \input{2 - kinetic component/main.tex}
            \input{3 - fluid component/main.tex}
            \input{4 - numerical implementation/main.tex}
        \part{Project Overview}
            \input{5 - research plan/main.tex}
            \input{6 - summary/main.tex}
    
    
    %\section{}
    \newpage
    \pagenumbering{gobble}
        \printbibliography


    \newpage
    \pagenumbering{roman}
    \appendix
        \part{Appendices}
            \input{8 - Hilbert complexes/main.tex}
            \input{9 - weak conservation proofs/main.tex}
\end{document}

            \documentclass[12pt, a4paper]{report}

\input{template/main.tex}

\title{\BA{Title in Progress...}}
\author{Boris Andrews}
\affil{Mathematical Institute, University of Oxford}
\date{\today}


\begin{document}
    \pagenumbering{gobble}
    \maketitle
    
    
    \begin{abstract}
        Magnetic confinement reactors---in particular tokamaks---offer one of the most promising options for achieving practical nuclear fusion, with the potential to provide virtually limitless, clean energy. The theoretical and numerical modeling of tokamak plasmas is simultaneously an essential component of effective reactor design, and a great research barrier. Tokamak operational conditions exhibit comparatively low Knudsen numbers. Kinetic effects, including kinetic waves and instabilities, Landau damping, bump-on-tail instabilities and more, are therefore highly influential in tokamak plasma dynamics. Purely fluid models are inherently incapable of capturing these effects, whereas the high dimensionality in purely kinetic models render them practically intractable for most relevant purposes.

        We consider a $\delta\!f$ decomposition model, with a macroscopic fluid background and microscopic kinetic correction, both fully coupled to each other. A similar manner of discretization is proposed to that used in the recent \texttt{STRUPHY} code \cite{Holderied_Possanner_Wang_2021, Holderied_2022, Li_et_al_2023} with a finite-element model for the background and a pseudo-particle/PiC model for the correction.

        The fluid background satisfies the full, non-linear, resistive, compressible, Hall MHD equations. \cite{Laakmann_Hu_Farrell_2022} introduces finite-element(-in-space) implicit timesteppers for the incompressible analogue to this system with structure-preserving (SP) properties in the ideal case, alongside parameter-robust preconditioners. We show that these timesteppers can derive from a finite-element-in-time (FET) (and finite-element-in-space) interpretation. The benefits of this reformulation are discussed, including the derivation of timesteppers that are higher order in time, and the quantifiable dissipative SP properties in the non-ideal, resistive case.
        
        We discuss possible options for extending this FET approach to timesteppers for the compressible case.

        The kinetic corrections satisfy linearized Boltzmann equations. Using a Lénard--Bernstein collision operator, these take Fokker--Planck-like forms \cite{Fokker_1914, Planck_1917} wherein pseudo-particles in the numerical model obey the neoclassical transport equations, with particle-independent Brownian drift terms. This offers a rigorous methodology for incorporating collisions into the particle transport model, without coupling the equations of motions for each particle.
        
        Works by Chen, Chacón et al. \cite{Chen_Chacón_Barnes_2011, Chacón_Chen_Barnes_2013, Chen_Chacón_2014, Chen_Chacón_2015} have developed structure-preserving particle pushers for neoclassical transport in the Vlasov equations, derived from Crank--Nicolson integrators. We show these too can can derive from a FET interpretation, similarly offering potential extensions to higher-order-in-time particle pushers. The FET formulation is used also to consider how the stochastic drift terms can be incorporated into the pushers. Stochastic gyrokinetic expansions are also discussed.

        Different options for the numerical implementation of these schemes are considered.

        Due to the efficacy of FET in the development of SP timesteppers for both the fluid and kinetic component, we hope this approach will prove effective in the future for developing SP timesteppers for the full hybrid model. We hope this will give us the opportunity to incorporate previously inaccessible kinetic effects into the highly effective, modern, finite-element MHD models.
    \end{abstract}
    
    
    \newpage
    \tableofcontents
    
    
    \newpage
    \pagenumbering{arabic}
    %\linenumbers\renewcommand\thelinenumber{\color{black!50}\arabic{linenumber}}
            \input{0 - introduction/main.tex}
        \part{Research}
            \input{1 - low-noise PiC models/main.tex}
            \input{2 - kinetic component/main.tex}
            \input{3 - fluid component/main.tex}
            \input{4 - numerical implementation/main.tex}
        \part{Project Overview}
            \input{5 - research plan/main.tex}
            \input{6 - summary/main.tex}
    
    
    %\section{}
    \newpage
    \pagenumbering{gobble}
        \printbibliography


    \newpage
    \pagenumbering{roman}
    \appendix
        \part{Appendices}
            \input{8 - Hilbert complexes/main.tex}
            \input{9 - weak conservation proofs/main.tex}
\end{document}

            \documentclass[12pt, a4paper]{report}

\input{template/main.tex}

\title{\BA{Title in Progress...}}
\author{Boris Andrews}
\affil{Mathematical Institute, University of Oxford}
\date{\today}


\begin{document}
    \pagenumbering{gobble}
    \maketitle
    
    
    \begin{abstract}
        Magnetic confinement reactors---in particular tokamaks---offer one of the most promising options for achieving practical nuclear fusion, with the potential to provide virtually limitless, clean energy. The theoretical and numerical modeling of tokamak plasmas is simultaneously an essential component of effective reactor design, and a great research barrier. Tokamak operational conditions exhibit comparatively low Knudsen numbers. Kinetic effects, including kinetic waves and instabilities, Landau damping, bump-on-tail instabilities and more, are therefore highly influential in tokamak plasma dynamics. Purely fluid models are inherently incapable of capturing these effects, whereas the high dimensionality in purely kinetic models render them practically intractable for most relevant purposes.

        We consider a $\delta\!f$ decomposition model, with a macroscopic fluid background and microscopic kinetic correction, both fully coupled to each other. A similar manner of discretization is proposed to that used in the recent \texttt{STRUPHY} code \cite{Holderied_Possanner_Wang_2021, Holderied_2022, Li_et_al_2023} with a finite-element model for the background and a pseudo-particle/PiC model for the correction.

        The fluid background satisfies the full, non-linear, resistive, compressible, Hall MHD equations. \cite{Laakmann_Hu_Farrell_2022} introduces finite-element(-in-space) implicit timesteppers for the incompressible analogue to this system with structure-preserving (SP) properties in the ideal case, alongside parameter-robust preconditioners. We show that these timesteppers can derive from a finite-element-in-time (FET) (and finite-element-in-space) interpretation. The benefits of this reformulation are discussed, including the derivation of timesteppers that are higher order in time, and the quantifiable dissipative SP properties in the non-ideal, resistive case.
        
        We discuss possible options for extending this FET approach to timesteppers for the compressible case.

        The kinetic corrections satisfy linearized Boltzmann equations. Using a Lénard--Bernstein collision operator, these take Fokker--Planck-like forms \cite{Fokker_1914, Planck_1917} wherein pseudo-particles in the numerical model obey the neoclassical transport equations, with particle-independent Brownian drift terms. This offers a rigorous methodology for incorporating collisions into the particle transport model, without coupling the equations of motions for each particle.
        
        Works by Chen, Chacón et al. \cite{Chen_Chacón_Barnes_2011, Chacón_Chen_Barnes_2013, Chen_Chacón_2014, Chen_Chacón_2015} have developed structure-preserving particle pushers for neoclassical transport in the Vlasov equations, derived from Crank--Nicolson integrators. We show these too can can derive from a FET interpretation, similarly offering potential extensions to higher-order-in-time particle pushers. The FET formulation is used also to consider how the stochastic drift terms can be incorporated into the pushers. Stochastic gyrokinetic expansions are also discussed.

        Different options for the numerical implementation of these schemes are considered.

        Due to the efficacy of FET in the development of SP timesteppers for both the fluid and kinetic component, we hope this approach will prove effective in the future for developing SP timesteppers for the full hybrid model. We hope this will give us the opportunity to incorporate previously inaccessible kinetic effects into the highly effective, modern, finite-element MHD models.
    \end{abstract}
    
    
    \newpage
    \tableofcontents
    
    
    \newpage
    \pagenumbering{arabic}
    %\linenumbers\renewcommand\thelinenumber{\color{black!50}\arabic{linenumber}}
            \input{0 - introduction/main.tex}
        \part{Research}
            \input{1 - low-noise PiC models/main.tex}
            \input{2 - kinetic component/main.tex}
            \input{3 - fluid component/main.tex}
            \input{4 - numerical implementation/main.tex}
        \part{Project Overview}
            \input{5 - research plan/main.tex}
            \input{6 - summary/main.tex}
    
    
    %\section{}
    \newpage
    \pagenumbering{gobble}
        \printbibliography


    \newpage
    \pagenumbering{roman}
    \appendix
        \part{Appendices}
            \input{8 - Hilbert complexes/main.tex}
            \input{9 - weak conservation proofs/main.tex}
\end{document}

            \documentclass[12pt, a4paper]{report}

\input{template/main.tex}

\title{\BA{Title in Progress...}}
\author{Boris Andrews}
\affil{Mathematical Institute, University of Oxford}
\date{\today}


\begin{document}
    \pagenumbering{gobble}
    \maketitle
    
    
    \begin{abstract}
        Magnetic confinement reactors---in particular tokamaks---offer one of the most promising options for achieving practical nuclear fusion, with the potential to provide virtually limitless, clean energy. The theoretical and numerical modeling of tokamak plasmas is simultaneously an essential component of effective reactor design, and a great research barrier. Tokamak operational conditions exhibit comparatively low Knudsen numbers. Kinetic effects, including kinetic waves and instabilities, Landau damping, bump-on-tail instabilities and more, are therefore highly influential in tokamak plasma dynamics. Purely fluid models are inherently incapable of capturing these effects, whereas the high dimensionality in purely kinetic models render them practically intractable for most relevant purposes.

        We consider a $\delta\!f$ decomposition model, with a macroscopic fluid background and microscopic kinetic correction, both fully coupled to each other. A similar manner of discretization is proposed to that used in the recent \texttt{STRUPHY} code \cite{Holderied_Possanner_Wang_2021, Holderied_2022, Li_et_al_2023} with a finite-element model for the background and a pseudo-particle/PiC model for the correction.

        The fluid background satisfies the full, non-linear, resistive, compressible, Hall MHD equations. \cite{Laakmann_Hu_Farrell_2022} introduces finite-element(-in-space) implicit timesteppers for the incompressible analogue to this system with structure-preserving (SP) properties in the ideal case, alongside parameter-robust preconditioners. We show that these timesteppers can derive from a finite-element-in-time (FET) (and finite-element-in-space) interpretation. The benefits of this reformulation are discussed, including the derivation of timesteppers that are higher order in time, and the quantifiable dissipative SP properties in the non-ideal, resistive case.
        
        We discuss possible options for extending this FET approach to timesteppers for the compressible case.

        The kinetic corrections satisfy linearized Boltzmann equations. Using a Lénard--Bernstein collision operator, these take Fokker--Planck-like forms \cite{Fokker_1914, Planck_1917} wherein pseudo-particles in the numerical model obey the neoclassical transport equations, with particle-independent Brownian drift terms. This offers a rigorous methodology for incorporating collisions into the particle transport model, without coupling the equations of motions for each particle.
        
        Works by Chen, Chacón et al. \cite{Chen_Chacón_Barnes_2011, Chacón_Chen_Barnes_2013, Chen_Chacón_2014, Chen_Chacón_2015} have developed structure-preserving particle pushers for neoclassical transport in the Vlasov equations, derived from Crank--Nicolson integrators. We show these too can can derive from a FET interpretation, similarly offering potential extensions to higher-order-in-time particle pushers. The FET formulation is used also to consider how the stochastic drift terms can be incorporated into the pushers. Stochastic gyrokinetic expansions are also discussed.

        Different options for the numerical implementation of these schemes are considered.

        Due to the efficacy of FET in the development of SP timesteppers for both the fluid and kinetic component, we hope this approach will prove effective in the future for developing SP timesteppers for the full hybrid model. We hope this will give us the opportunity to incorporate previously inaccessible kinetic effects into the highly effective, modern, finite-element MHD models.
    \end{abstract}
    
    
    \newpage
    \tableofcontents
    
    
    \newpage
    \pagenumbering{arabic}
    %\linenumbers\renewcommand\thelinenumber{\color{black!50}\arabic{linenumber}}
            \input{0 - introduction/main.tex}
        \part{Research}
            \input{1 - low-noise PiC models/main.tex}
            \input{2 - kinetic component/main.tex}
            \input{3 - fluid component/main.tex}
            \input{4 - numerical implementation/main.tex}
        \part{Project Overview}
            \input{5 - research plan/main.tex}
            \input{6 - summary/main.tex}
    
    
    %\section{}
    \newpage
    \pagenumbering{gobble}
        \printbibliography


    \newpage
    \pagenumbering{roman}
    \appendix
        \part{Appendices}
            \input{8 - Hilbert complexes/main.tex}
            \input{9 - weak conservation proofs/main.tex}
\end{document}

        \part{Project Overview}
            \documentclass[12pt, a4paper]{report}

\input{template/main.tex}

\title{\BA{Title in Progress...}}
\author{Boris Andrews}
\affil{Mathematical Institute, University of Oxford}
\date{\today}


\begin{document}
    \pagenumbering{gobble}
    \maketitle
    
    
    \begin{abstract}
        Magnetic confinement reactors---in particular tokamaks---offer one of the most promising options for achieving practical nuclear fusion, with the potential to provide virtually limitless, clean energy. The theoretical and numerical modeling of tokamak plasmas is simultaneously an essential component of effective reactor design, and a great research barrier. Tokamak operational conditions exhibit comparatively low Knudsen numbers. Kinetic effects, including kinetic waves and instabilities, Landau damping, bump-on-tail instabilities and more, are therefore highly influential in tokamak plasma dynamics. Purely fluid models are inherently incapable of capturing these effects, whereas the high dimensionality in purely kinetic models render them practically intractable for most relevant purposes.

        We consider a $\delta\!f$ decomposition model, with a macroscopic fluid background and microscopic kinetic correction, both fully coupled to each other. A similar manner of discretization is proposed to that used in the recent \texttt{STRUPHY} code \cite{Holderied_Possanner_Wang_2021, Holderied_2022, Li_et_al_2023} with a finite-element model for the background and a pseudo-particle/PiC model for the correction.

        The fluid background satisfies the full, non-linear, resistive, compressible, Hall MHD equations. \cite{Laakmann_Hu_Farrell_2022} introduces finite-element(-in-space) implicit timesteppers for the incompressible analogue to this system with structure-preserving (SP) properties in the ideal case, alongside parameter-robust preconditioners. We show that these timesteppers can derive from a finite-element-in-time (FET) (and finite-element-in-space) interpretation. The benefits of this reformulation are discussed, including the derivation of timesteppers that are higher order in time, and the quantifiable dissipative SP properties in the non-ideal, resistive case.
        
        We discuss possible options for extending this FET approach to timesteppers for the compressible case.

        The kinetic corrections satisfy linearized Boltzmann equations. Using a Lénard--Bernstein collision operator, these take Fokker--Planck-like forms \cite{Fokker_1914, Planck_1917} wherein pseudo-particles in the numerical model obey the neoclassical transport equations, with particle-independent Brownian drift terms. This offers a rigorous methodology for incorporating collisions into the particle transport model, without coupling the equations of motions for each particle.
        
        Works by Chen, Chacón et al. \cite{Chen_Chacón_Barnes_2011, Chacón_Chen_Barnes_2013, Chen_Chacón_2014, Chen_Chacón_2015} have developed structure-preserving particle pushers for neoclassical transport in the Vlasov equations, derived from Crank--Nicolson integrators. We show these too can can derive from a FET interpretation, similarly offering potential extensions to higher-order-in-time particle pushers. The FET formulation is used also to consider how the stochastic drift terms can be incorporated into the pushers. Stochastic gyrokinetic expansions are also discussed.

        Different options for the numerical implementation of these schemes are considered.

        Due to the efficacy of FET in the development of SP timesteppers for both the fluid and kinetic component, we hope this approach will prove effective in the future for developing SP timesteppers for the full hybrid model. We hope this will give us the opportunity to incorporate previously inaccessible kinetic effects into the highly effective, modern, finite-element MHD models.
    \end{abstract}
    
    
    \newpage
    \tableofcontents
    
    
    \newpage
    \pagenumbering{arabic}
    %\linenumbers\renewcommand\thelinenumber{\color{black!50}\arabic{linenumber}}
            \input{0 - introduction/main.tex}
        \part{Research}
            \input{1 - low-noise PiC models/main.tex}
            \input{2 - kinetic component/main.tex}
            \input{3 - fluid component/main.tex}
            \input{4 - numerical implementation/main.tex}
        \part{Project Overview}
            \input{5 - research plan/main.tex}
            \input{6 - summary/main.tex}
    
    
    %\section{}
    \newpage
    \pagenumbering{gobble}
        \printbibliography


    \newpage
    \pagenumbering{roman}
    \appendix
        \part{Appendices}
            \input{8 - Hilbert complexes/main.tex}
            \input{9 - weak conservation proofs/main.tex}
\end{document}

            \documentclass[12pt, a4paper]{report}

\input{template/main.tex}

\title{\BA{Title in Progress...}}
\author{Boris Andrews}
\affil{Mathematical Institute, University of Oxford}
\date{\today}


\begin{document}
    \pagenumbering{gobble}
    \maketitle
    
    
    \begin{abstract}
        Magnetic confinement reactors---in particular tokamaks---offer one of the most promising options for achieving practical nuclear fusion, with the potential to provide virtually limitless, clean energy. The theoretical and numerical modeling of tokamak plasmas is simultaneously an essential component of effective reactor design, and a great research barrier. Tokamak operational conditions exhibit comparatively low Knudsen numbers. Kinetic effects, including kinetic waves and instabilities, Landau damping, bump-on-tail instabilities and more, are therefore highly influential in tokamak plasma dynamics. Purely fluid models are inherently incapable of capturing these effects, whereas the high dimensionality in purely kinetic models render them practically intractable for most relevant purposes.

        We consider a $\delta\!f$ decomposition model, with a macroscopic fluid background and microscopic kinetic correction, both fully coupled to each other. A similar manner of discretization is proposed to that used in the recent \texttt{STRUPHY} code \cite{Holderied_Possanner_Wang_2021, Holderied_2022, Li_et_al_2023} with a finite-element model for the background and a pseudo-particle/PiC model for the correction.

        The fluid background satisfies the full, non-linear, resistive, compressible, Hall MHD equations. \cite{Laakmann_Hu_Farrell_2022} introduces finite-element(-in-space) implicit timesteppers for the incompressible analogue to this system with structure-preserving (SP) properties in the ideal case, alongside parameter-robust preconditioners. We show that these timesteppers can derive from a finite-element-in-time (FET) (and finite-element-in-space) interpretation. The benefits of this reformulation are discussed, including the derivation of timesteppers that are higher order in time, and the quantifiable dissipative SP properties in the non-ideal, resistive case.
        
        We discuss possible options for extending this FET approach to timesteppers for the compressible case.

        The kinetic corrections satisfy linearized Boltzmann equations. Using a Lénard--Bernstein collision operator, these take Fokker--Planck-like forms \cite{Fokker_1914, Planck_1917} wherein pseudo-particles in the numerical model obey the neoclassical transport equations, with particle-independent Brownian drift terms. This offers a rigorous methodology for incorporating collisions into the particle transport model, without coupling the equations of motions for each particle.
        
        Works by Chen, Chacón et al. \cite{Chen_Chacón_Barnes_2011, Chacón_Chen_Barnes_2013, Chen_Chacón_2014, Chen_Chacón_2015} have developed structure-preserving particle pushers for neoclassical transport in the Vlasov equations, derived from Crank--Nicolson integrators. We show these too can can derive from a FET interpretation, similarly offering potential extensions to higher-order-in-time particle pushers. The FET formulation is used also to consider how the stochastic drift terms can be incorporated into the pushers. Stochastic gyrokinetic expansions are also discussed.

        Different options for the numerical implementation of these schemes are considered.

        Due to the efficacy of FET in the development of SP timesteppers for both the fluid and kinetic component, we hope this approach will prove effective in the future for developing SP timesteppers for the full hybrid model. We hope this will give us the opportunity to incorporate previously inaccessible kinetic effects into the highly effective, modern, finite-element MHD models.
    \end{abstract}
    
    
    \newpage
    \tableofcontents
    
    
    \newpage
    \pagenumbering{arabic}
    %\linenumbers\renewcommand\thelinenumber{\color{black!50}\arabic{linenumber}}
            \input{0 - introduction/main.tex}
        \part{Research}
            \input{1 - low-noise PiC models/main.tex}
            \input{2 - kinetic component/main.tex}
            \input{3 - fluid component/main.tex}
            \input{4 - numerical implementation/main.tex}
        \part{Project Overview}
            \input{5 - research plan/main.tex}
            \input{6 - summary/main.tex}
    
    
    %\section{}
    \newpage
    \pagenumbering{gobble}
        \printbibliography


    \newpage
    \pagenumbering{roman}
    \appendix
        \part{Appendices}
            \input{8 - Hilbert complexes/main.tex}
            \input{9 - weak conservation proofs/main.tex}
\end{document}

    
    
    %\section{}
    \newpage
    \pagenumbering{gobble}
        \printbibliography


    \newpage
    \pagenumbering{roman}
    \appendix
        \part{Appendices}
            \documentclass[12pt, a4paper]{report}

\input{template/main.tex}

\title{\BA{Title in Progress...}}
\author{Boris Andrews}
\affil{Mathematical Institute, University of Oxford}
\date{\today}


\begin{document}
    \pagenumbering{gobble}
    \maketitle
    
    
    \begin{abstract}
        Magnetic confinement reactors---in particular tokamaks---offer one of the most promising options for achieving practical nuclear fusion, with the potential to provide virtually limitless, clean energy. The theoretical and numerical modeling of tokamak plasmas is simultaneously an essential component of effective reactor design, and a great research barrier. Tokamak operational conditions exhibit comparatively low Knudsen numbers. Kinetic effects, including kinetic waves and instabilities, Landau damping, bump-on-tail instabilities and more, are therefore highly influential in tokamak plasma dynamics. Purely fluid models are inherently incapable of capturing these effects, whereas the high dimensionality in purely kinetic models render them practically intractable for most relevant purposes.

        We consider a $\delta\!f$ decomposition model, with a macroscopic fluid background and microscopic kinetic correction, both fully coupled to each other. A similar manner of discretization is proposed to that used in the recent \texttt{STRUPHY} code \cite{Holderied_Possanner_Wang_2021, Holderied_2022, Li_et_al_2023} with a finite-element model for the background and a pseudo-particle/PiC model for the correction.

        The fluid background satisfies the full, non-linear, resistive, compressible, Hall MHD equations. \cite{Laakmann_Hu_Farrell_2022} introduces finite-element(-in-space) implicit timesteppers for the incompressible analogue to this system with structure-preserving (SP) properties in the ideal case, alongside parameter-robust preconditioners. We show that these timesteppers can derive from a finite-element-in-time (FET) (and finite-element-in-space) interpretation. The benefits of this reformulation are discussed, including the derivation of timesteppers that are higher order in time, and the quantifiable dissipative SP properties in the non-ideal, resistive case.
        
        We discuss possible options for extending this FET approach to timesteppers for the compressible case.

        The kinetic corrections satisfy linearized Boltzmann equations. Using a Lénard--Bernstein collision operator, these take Fokker--Planck-like forms \cite{Fokker_1914, Planck_1917} wherein pseudo-particles in the numerical model obey the neoclassical transport equations, with particle-independent Brownian drift terms. This offers a rigorous methodology for incorporating collisions into the particle transport model, without coupling the equations of motions for each particle.
        
        Works by Chen, Chacón et al. \cite{Chen_Chacón_Barnes_2011, Chacón_Chen_Barnes_2013, Chen_Chacón_2014, Chen_Chacón_2015} have developed structure-preserving particle pushers for neoclassical transport in the Vlasov equations, derived from Crank--Nicolson integrators. We show these too can can derive from a FET interpretation, similarly offering potential extensions to higher-order-in-time particle pushers. The FET formulation is used also to consider how the stochastic drift terms can be incorporated into the pushers. Stochastic gyrokinetic expansions are also discussed.

        Different options for the numerical implementation of these schemes are considered.

        Due to the efficacy of FET in the development of SP timesteppers for both the fluid and kinetic component, we hope this approach will prove effective in the future for developing SP timesteppers for the full hybrid model. We hope this will give us the opportunity to incorporate previously inaccessible kinetic effects into the highly effective, modern, finite-element MHD models.
    \end{abstract}
    
    
    \newpage
    \tableofcontents
    
    
    \newpage
    \pagenumbering{arabic}
    %\linenumbers\renewcommand\thelinenumber{\color{black!50}\arabic{linenumber}}
            \input{0 - introduction/main.tex}
        \part{Research}
            \input{1 - low-noise PiC models/main.tex}
            \input{2 - kinetic component/main.tex}
            \input{3 - fluid component/main.tex}
            \input{4 - numerical implementation/main.tex}
        \part{Project Overview}
            \input{5 - research plan/main.tex}
            \input{6 - summary/main.tex}
    
    
    %\section{}
    \newpage
    \pagenumbering{gobble}
        \printbibliography


    \newpage
    \pagenumbering{roman}
    \appendix
        \part{Appendices}
            \input{8 - Hilbert complexes/main.tex}
            \input{9 - weak conservation proofs/main.tex}
\end{document}

            \documentclass[12pt, a4paper]{report}

\input{template/main.tex}

\title{\BA{Title in Progress...}}
\author{Boris Andrews}
\affil{Mathematical Institute, University of Oxford}
\date{\today}


\begin{document}
    \pagenumbering{gobble}
    \maketitle
    
    
    \begin{abstract}
        Magnetic confinement reactors---in particular tokamaks---offer one of the most promising options for achieving practical nuclear fusion, with the potential to provide virtually limitless, clean energy. The theoretical and numerical modeling of tokamak plasmas is simultaneously an essential component of effective reactor design, and a great research barrier. Tokamak operational conditions exhibit comparatively low Knudsen numbers. Kinetic effects, including kinetic waves and instabilities, Landau damping, bump-on-tail instabilities and more, are therefore highly influential in tokamak plasma dynamics. Purely fluid models are inherently incapable of capturing these effects, whereas the high dimensionality in purely kinetic models render them practically intractable for most relevant purposes.

        We consider a $\delta\!f$ decomposition model, with a macroscopic fluid background and microscopic kinetic correction, both fully coupled to each other. A similar manner of discretization is proposed to that used in the recent \texttt{STRUPHY} code \cite{Holderied_Possanner_Wang_2021, Holderied_2022, Li_et_al_2023} with a finite-element model for the background and a pseudo-particle/PiC model for the correction.

        The fluid background satisfies the full, non-linear, resistive, compressible, Hall MHD equations. \cite{Laakmann_Hu_Farrell_2022} introduces finite-element(-in-space) implicit timesteppers for the incompressible analogue to this system with structure-preserving (SP) properties in the ideal case, alongside parameter-robust preconditioners. We show that these timesteppers can derive from a finite-element-in-time (FET) (and finite-element-in-space) interpretation. The benefits of this reformulation are discussed, including the derivation of timesteppers that are higher order in time, and the quantifiable dissipative SP properties in the non-ideal, resistive case.
        
        We discuss possible options for extending this FET approach to timesteppers for the compressible case.

        The kinetic corrections satisfy linearized Boltzmann equations. Using a Lénard--Bernstein collision operator, these take Fokker--Planck-like forms \cite{Fokker_1914, Planck_1917} wherein pseudo-particles in the numerical model obey the neoclassical transport equations, with particle-independent Brownian drift terms. This offers a rigorous methodology for incorporating collisions into the particle transport model, without coupling the equations of motions for each particle.
        
        Works by Chen, Chacón et al. \cite{Chen_Chacón_Barnes_2011, Chacón_Chen_Barnes_2013, Chen_Chacón_2014, Chen_Chacón_2015} have developed structure-preserving particle pushers for neoclassical transport in the Vlasov equations, derived from Crank--Nicolson integrators. We show these too can can derive from a FET interpretation, similarly offering potential extensions to higher-order-in-time particle pushers. The FET formulation is used also to consider how the stochastic drift terms can be incorporated into the pushers. Stochastic gyrokinetic expansions are also discussed.

        Different options for the numerical implementation of these schemes are considered.

        Due to the efficacy of FET in the development of SP timesteppers for both the fluid and kinetic component, we hope this approach will prove effective in the future for developing SP timesteppers for the full hybrid model. We hope this will give us the opportunity to incorporate previously inaccessible kinetic effects into the highly effective, modern, finite-element MHD models.
    \end{abstract}
    
    
    \newpage
    \tableofcontents
    
    
    \newpage
    \pagenumbering{arabic}
    %\linenumbers\renewcommand\thelinenumber{\color{black!50}\arabic{linenumber}}
            \input{0 - introduction/main.tex}
        \part{Research}
            \input{1 - low-noise PiC models/main.tex}
            \input{2 - kinetic component/main.tex}
            \input{3 - fluid component/main.tex}
            \input{4 - numerical implementation/main.tex}
        \part{Project Overview}
            \input{5 - research plan/main.tex}
            \input{6 - summary/main.tex}
    
    
    %\section{}
    \newpage
    \pagenumbering{gobble}
        \printbibliography


    \newpage
    \pagenumbering{roman}
    \appendix
        \part{Appendices}
            \input{8 - Hilbert complexes/main.tex}
            \input{9 - weak conservation proofs/main.tex}
\end{document}

\end{document}

    
    
    %\section{}
    \newpage
    \pagenumbering{gobble}
        \printbibliography


    \newpage
    \pagenumbering{roman}
    \appendix
        \part{Appendices}
            \documentclass[12pt, a4paper]{report}

\documentclass[12pt, a4paper]{report}

\input{template/main.tex}

\title{\BA{Title in Progress...}}
\author{Boris Andrews}
\affil{Mathematical Institute, University of Oxford}
\date{\today}


\begin{document}
    \pagenumbering{gobble}
    \maketitle
    
    
    \begin{abstract}
        Magnetic confinement reactors---in particular tokamaks---offer one of the most promising options for achieving practical nuclear fusion, with the potential to provide virtually limitless, clean energy. The theoretical and numerical modeling of tokamak plasmas is simultaneously an essential component of effective reactor design, and a great research barrier. Tokamak operational conditions exhibit comparatively low Knudsen numbers. Kinetic effects, including kinetic waves and instabilities, Landau damping, bump-on-tail instabilities and more, are therefore highly influential in tokamak plasma dynamics. Purely fluid models are inherently incapable of capturing these effects, whereas the high dimensionality in purely kinetic models render them practically intractable for most relevant purposes.

        We consider a $\delta\!f$ decomposition model, with a macroscopic fluid background and microscopic kinetic correction, both fully coupled to each other. A similar manner of discretization is proposed to that used in the recent \texttt{STRUPHY} code \cite{Holderied_Possanner_Wang_2021, Holderied_2022, Li_et_al_2023} with a finite-element model for the background and a pseudo-particle/PiC model for the correction.

        The fluid background satisfies the full, non-linear, resistive, compressible, Hall MHD equations. \cite{Laakmann_Hu_Farrell_2022} introduces finite-element(-in-space) implicit timesteppers for the incompressible analogue to this system with structure-preserving (SP) properties in the ideal case, alongside parameter-robust preconditioners. We show that these timesteppers can derive from a finite-element-in-time (FET) (and finite-element-in-space) interpretation. The benefits of this reformulation are discussed, including the derivation of timesteppers that are higher order in time, and the quantifiable dissipative SP properties in the non-ideal, resistive case.
        
        We discuss possible options for extending this FET approach to timesteppers for the compressible case.

        The kinetic corrections satisfy linearized Boltzmann equations. Using a Lénard--Bernstein collision operator, these take Fokker--Planck-like forms \cite{Fokker_1914, Planck_1917} wherein pseudo-particles in the numerical model obey the neoclassical transport equations, with particle-independent Brownian drift terms. This offers a rigorous methodology for incorporating collisions into the particle transport model, without coupling the equations of motions for each particle.
        
        Works by Chen, Chacón et al. \cite{Chen_Chacón_Barnes_2011, Chacón_Chen_Barnes_2013, Chen_Chacón_2014, Chen_Chacón_2015} have developed structure-preserving particle pushers for neoclassical transport in the Vlasov equations, derived from Crank--Nicolson integrators. We show these too can can derive from a FET interpretation, similarly offering potential extensions to higher-order-in-time particle pushers. The FET formulation is used also to consider how the stochastic drift terms can be incorporated into the pushers. Stochastic gyrokinetic expansions are also discussed.

        Different options for the numerical implementation of these schemes are considered.

        Due to the efficacy of FET in the development of SP timesteppers for both the fluid and kinetic component, we hope this approach will prove effective in the future for developing SP timesteppers for the full hybrid model. We hope this will give us the opportunity to incorporate previously inaccessible kinetic effects into the highly effective, modern, finite-element MHD models.
    \end{abstract}
    
    
    \newpage
    \tableofcontents
    
    
    \newpage
    \pagenumbering{arabic}
    %\linenumbers\renewcommand\thelinenumber{\color{black!50}\arabic{linenumber}}
            \input{0 - introduction/main.tex}
        \part{Research}
            \input{1 - low-noise PiC models/main.tex}
            \input{2 - kinetic component/main.tex}
            \input{3 - fluid component/main.tex}
            \input{4 - numerical implementation/main.tex}
        \part{Project Overview}
            \input{5 - research plan/main.tex}
            \input{6 - summary/main.tex}
    
    
    %\section{}
    \newpage
    \pagenumbering{gobble}
        \printbibliography


    \newpage
    \pagenumbering{roman}
    \appendix
        \part{Appendices}
            \input{8 - Hilbert complexes/main.tex}
            \input{9 - weak conservation proofs/main.tex}
\end{document}


\title{\BA{Title in Progress...}}
\author{Boris Andrews}
\affil{Mathematical Institute, University of Oxford}
\date{\today}


\begin{document}
    \pagenumbering{gobble}
    \maketitle
    
    
    \begin{abstract}
        Magnetic confinement reactors---in particular tokamaks---offer one of the most promising options for achieving practical nuclear fusion, with the potential to provide virtually limitless, clean energy. The theoretical and numerical modeling of tokamak plasmas is simultaneously an essential component of effective reactor design, and a great research barrier. Tokamak operational conditions exhibit comparatively low Knudsen numbers. Kinetic effects, including kinetic waves and instabilities, Landau damping, bump-on-tail instabilities and more, are therefore highly influential in tokamak plasma dynamics. Purely fluid models are inherently incapable of capturing these effects, whereas the high dimensionality in purely kinetic models render them practically intractable for most relevant purposes.

        We consider a $\delta\!f$ decomposition model, with a macroscopic fluid background and microscopic kinetic correction, both fully coupled to each other. A similar manner of discretization is proposed to that used in the recent \texttt{STRUPHY} code \cite{Holderied_Possanner_Wang_2021, Holderied_2022, Li_et_al_2023} with a finite-element model for the background and a pseudo-particle/PiC model for the correction.

        The fluid background satisfies the full, non-linear, resistive, compressible, Hall MHD equations. \cite{Laakmann_Hu_Farrell_2022} introduces finite-element(-in-space) implicit timesteppers for the incompressible analogue to this system with structure-preserving (SP) properties in the ideal case, alongside parameter-robust preconditioners. We show that these timesteppers can derive from a finite-element-in-time (FET) (and finite-element-in-space) interpretation. The benefits of this reformulation are discussed, including the derivation of timesteppers that are higher order in time, and the quantifiable dissipative SP properties in the non-ideal, resistive case.
        
        We discuss possible options for extending this FET approach to timesteppers for the compressible case.

        The kinetic corrections satisfy linearized Boltzmann equations. Using a Lénard--Bernstein collision operator, these take Fokker--Planck-like forms \cite{Fokker_1914, Planck_1917} wherein pseudo-particles in the numerical model obey the neoclassical transport equations, with particle-independent Brownian drift terms. This offers a rigorous methodology for incorporating collisions into the particle transport model, without coupling the equations of motions for each particle.
        
        Works by Chen, Chacón et al. \cite{Chen_Chacón_Barnes_2011, Chacón_Chen_Barnes_2013, Chen_Chacón_2014, Chen_Chacón_2015} have developed structure-preserving particle pushers for neoclassical transport in the Vlasov equations, derived from Crank--Nicolson integrators. We show these too can can derive from a FET interpretation, similarly offering potential extensions to higher-order-in-time particle pushers. The FET formulation is used also to consider how the stochastic drift terms can be incorporated into the pushers. Stochastic gyrokinetic expansions are also discussed.

        Different options for the numerical implementation of these schemes are considered.

        Due to the efficacy of FET in the development of SP timesteppers for both the fluid and kinetic component, we hope this approach will prove effective in the future for developing SP timesteppers for the full hybrid model. We hope this will give us the opportunity to incorporate previously inaccessible kinetic effects into the highly effective, modern, finite-element MHD models.
    \end{abstract}
    
    
    \newpage
    \tableofcontents
    
    
    \newpage
    \pagenumbering{arabic}
    %\linenumbers\renewcommand\thelinenumber{\color{black!50}\arabic{linenumber}}
            \documentclass[12pt, a4paper]{report}

\input{template/main.tex}

\title{\BA{Title in Progress...}}
\author{Boris Andrews}
\affil{Mathematical Institute, University of Oxford}
\date{\today}


\begin{document}
    \pagenumbering{gobble}
    \maketitle
    
    
    \begin{abstract}
        Magnetic confinement reactors---in particular tokamaks---offer one of the most promising options for achieving practical nuclear fusion, with the potential to provide virtually limitless, clean energy. The theoretical and numerical modeling of tokamak plasmas is simultaneously an essential component of effective reactor design, and a great research barrier. Tokamak operational conditions exhibit comparatively low Knudsen numbers. Kinetic effects, including kinetic waves and instabilities, Landau damping, bump-on-tail instabilities and more, are therefore highly influential in tokamak plasma dynamics. Purely fluid models are inherently incapable of capturing these effects, whereas the high dimensionality in purely kinetic models render them practically intractable for most relevant purposes.

        We consider a $\delta\!f$ decomposition model, with a macroscopic fluid background and microscopic kinetic correction, both fully coupled to each other. A similar manner of discretization is proposed to that used in the recent \texttt{STRUPHY} code \cite{Holderied_Possanner_Wang_2021, Holderied_2022, Li_et_al_2023} with a finite-element model for the background and a pseudo-particle/PiC model for the correction.

        The fluid background satisfies the full, non-linear, resistive, compressible, Hall MHD equations. \cite{Laakmann_Hu_Farrell_2022} introduces finite-element(-in-space) implicit timesteppers for the incompressible analogue to this system with structure-preserving (SP) properties in the ideal case, alongside parameter-robust preconditioners. We show that these timesteppers can derive from a finite-element-in-time (FET) (and finite-element-in-space) interpretation. The benefits of this reformulation are discussed, including the derivation of timesteppers that are higher order in time, and the quantifiable dissipative SP properties in the non-ideal, resistive case.
        
        We discuss possible options for extending this FET approach to timesteppers for the compressible case.

        The kinetic corrections satisfy linearized Boltzmann equations. Using a Lénard--Bernstein collision operator, these take Fokker--Planck-like forms \cite{Fokker_1914, Planck_1917} wherein pseudo-particles in the numerical model obey the neoclassical transport equations, with particle-independent Brownian drift terms. This offers a rigorous methodology for incorporating collisions into the particle transport model, without coupling the equations of motions for each particle.
        
        Works by Chen, Chacón et al. \cite{Chen_Chacón_Barnes_2011, Chacón_Chen_Barnes_2013, Chen_Chacón_2014, Chen_Chacón_2015} have developed structure-preserving particle pushers for neoclassical transport in the Vlasov equations, derived from Crank--Nicolson integrators. We show these too can can derive from a FET interpretation, similarly offering potential extensions to higher-order-in-time particle pushers. The FET formulation is used also to consider how the stochastic drift terms can be incorporated into the pushers. Stochastic gyrokinetic expansions are also discussed.

        Different options for the numerical implementation of these schemes are considered.

        Due to the efficacy of FET in the development of SP timesteppers for both the fluid and kinetic component, we hope this approach will prove effective in the future for developing SP timesteppers for the full hybrid model. We hope this will give us the opportunity to incorporate previously inaccessible kinetic effects into the highly effective, modern, finite-element MHD models.
    \end{abstract}
    
    
    \newpage
    \tableofcontents
    
    
    \newpage
    \pagenumbering{arabic}
    %\linenumbers\renewcommand\thelinenumber{\color{black!50}\arabic{linenumber}}
            \input{0 - introduction/main.tex}
        \part{Research}
            \input{1 - low-noise PiC models/main.tex}
            \input{2 - kinetic component/main.tex}
            \input{3 - fluid component/main.tex}
            \input{4 - numerical implementation/main.tex}
        \part{Project Overview}
            \input{5 - research plan/main.tex}
            \input{6 - summary/main.tex}
    
    
    %\section{}
    \newpage
    \pagenumbering{gobble}
        \printbibliography


    \newpage
    \pagenumbering{roman}
    \appendix
        \part{Appendices}
            \input{8 - Hilbert complexes/main.tex}
            \input{9 - weak conservation proofs/main.tex}
\end{document}

        \part{Research}
            \documentclass[12pt, a4paper]{report}

\input{template/main.tex}

\title{\BA{Title in Progress...}}
\author{Boris Andrews}
\affil{Mathematical Institute, University of Oxford}
\date{\today}


\begin{document}
    \pagenumbering{gobble}
    \maketitle
    
    
    \begin{abstract}
        Magnetic confinement reactors---in particular tokamaks---offer one of the most promising options for achieving practical nuclear fusion, with the potential to provide virtually limitless, clean energy. The theoretical and numerical modeling of tokamak plasmas is simultaneously an essential component of effective reactor design, and a great research barrier. Tokamak operational conditions exhibit comparatively low Knudsen numbers. Kinetic effects, including kinetic waves and instabilities, Landau damping, bump-on-tail instabilities and more, are therefore highly influential in tokamak plasma dynamics. Purely fluid models are inherently incapable of capturing these effects, whereas the high dimensionality in purely kinetic models render them practically intractable for most relevant purposes.

        We consider a $\delta\!f$ decomposition model, with a macroscopic fluid background and microscopic kinetic correction, both fully coupled to each other. A similar manner of discretization is proposed to that used in the recent \texttt{STRUPHY} code \cite{Holderied_Possanner_Wang_2021, Holderied_2022, Li_et_al_2023} with a finite-element model for the background and a pseudo-particle/PiC model for the correction.

        The fluid background satisfies the full, non-linear, resistive, compressible, Hall MHD equations. \cite{Laakmann_Hu_Farrell_2022} introduces finite-element(-in-space) implicit timesteppers for the incompressible analogue to this system with structure-preserving (SP) properties in the ideal case, alongside parameter-robust preconditioners. We show that these timesteppers can derive from a finite-element-in-time (FET) (and finite-element-in-space) interpretation. The benefits of this reformulation are discussed, including the derivation of timesteppers that are higher order in time, and the quantifiable dissipative SP properties in the non-ideal, resistive case.
        
        We discuss possible options for extending this FET approach to timesteppers for the compressible case.

        The kinetic corrections satisfy linearized Boltzmann equations. Using a Lénard--Bernstein collision operator, these take Fokker--Planck-like forms \cite{Fokker_1914, Planck_1917} wherein pseudo-particles in the numerical model obey the neoclassical transport equations, with particle-independent Brownian drift terms. This offers a rigorous methodology for incorporating collisions into the particle transport model, without coupling the equations of motions for each particle.
        
        Works by Chen, Chacón et al. \cite{Chen_Chacón_Barnes_2011, Chacón_Chen_Barnes_2013, Chen_Chacón_2014, Chen_Chacón_2015} have developed structure-preserving particle pushers for neoclassical transport in the Vlasov equations, derived from Crank--Nicolson integrators. We show these too can can derive from a FET interpretation, similarly offering potential extensions to higher-order-in-time particle pushers. The FET formulation is used also to consider how the stochastic drift terms can be incorporated into the pushers. Stochastic gyrokinetic expansions are also discussed.

        Different options for the numerical implementation of these schemes are considered.

        Due to the efficacy of FET in the development of SP timesteppers for both the fluid and kinetic component, we hope this approach will prove effective in the future for developing SP timesteppers for the full hybrid model. We hope this will give us the opportunity to incorporate previously inaccessible kinetic effects into the highly effective, modern, finite-element MHD models.
    \end{abstract}
    
    
    \newpage
    \tableofcontents
    
    
    \newpage
    \pagenumbering{arabic}
    %\linenumbers\renewcommand\thelinenumber{\color{black!50}\arabic{linenumber}}
            \input{0 - introduction/main.tex}
        \part{Research}
            \input{1 - low-noise PiC models/main.tex}
            \input{2 - kinetic component/main.tex}
            \input{3 - fluid component/main.tex}
            \input{4 - numerical implementation/main.tex}
        \part{Project Overview}
            \input{5 - research plan/main.tex}
            \input{6 - summary/main.tex}
    
    
    %\section{}
    \newpage
    \pagenumbering{gobble}
        \printbibliography


    \newpage
    \pagenumbering{roman}
    \appendix
        \part{Appendices}
            \input{8 - Hilbert complexes/main.tex}
            \input{9 - weak conservation proofs/main.tex}
\end{document}

            \documentclass[12pt, a4paper]{report}

\input{template/main.tex}

\title{\BA{Title in Progress...}}
\author{Boris Andrews}
\affil{Mathematical Institute, University of Oxford}
\date{\today}


\begin{document}
    \pagenumbering{gobble}
    \maketitle
    
    
    \begin{abstract}
        Magnetic confinement reactors---in particular tokamaks---offer one of the most promising options for achieving practical nuclear fusion, with the potential to provide virtually limitless, clean energy. The theoretical and numerical modeling of tokamak plasmas is simultaneously an essential component of effective reactor design, and a great research barrier. Tokamak operational conditions exhibit comparatively low Knudsen numbers. Kinetic effects, including kinetic waves and instabilities, Landau damping, bump-on-tail instabilities and more, are therefore highly influential in tokamak plasma dynamics. Purely fluid models are inherently incapable of capturing these effects, whereas the high dimensionality in purely kinetic models render them practically intractable for most relevant purposes.

        We consider a $\delta\!f$ decomposition model, with a macroscopic fluid background and microscopic kinetic correction, both fully coupled to each other. A similar manner of discretization is proposed to that used in the recent \texttt{STRUPHY} code \cite{Holderied_Possanner_Wang_2021, Holderied_2022, Li_et_al_2023} with a finite-element model for the background and a pseudo-particle/PiC model for the correction.

        The fluid background satisfies the full, non-linear, resistive, compressible, Hall MHD equations. \cite{Laakmann_Hu_Farrell_2022} introduces finite-element(-in-space) implicit timesteppers for the incompressible analogue to this system with structure-preserving (SP) properties in the ideal case, alongside parameter-robust preconditioners. We show that these timesteppers can derive from a finite-element-in-time (FET) (and finite-element-in-space) interpretation. The benefits of this reformulation are discussed, including the derivation of timesteppers that are higher order in time, and the quantifiable dissipative SP properties in the non-ideal, resistive case.
        
        We discuss possible options for extending this FET approach to timesteppers for the compressible case.

        The kinetic corrections satisfy linearized Boltzmann equations. Using a Lénard--Bernstein collision operator, these take Fokker--Planck-like forms \cite{Fokker_1914, Planck_1917} wherein pseudo-particles in the numerical model obey the neoclassical transport equations, with particle-independent Brownian drift terms. This offers a rigorous methodology for incorporating collisions into the particle transport model, without coupling the equations of motions for each particle.
        
        Works by Chen, Chacón et al. \cite{Chen_Chacón_Barnes_2011, Chacón_Chen_Barnes_2013, Chen_Chacón_2014, Chen_Chacón_2015} have developed structure-preserving particle pushers for neoclassical transport in the Vlasov equations, derived from Crank--Nicolson integrators. We show these too can can derive from a FET interpretation, similarly offering potential extensions to higher-order-in-time particle pushers. The FET formulation is used also to consider how the stochastic drift terms can be incorporated into the pushers. Stochastic gyrokinetic expansions are also discussed.

        Different options for the numerical implementation of these schemes are considered.

        Due to the efficacy of FET in the development of SP timesteppers for both the fluid and kinetic component, we hope this approach will prove effective in the future for developing SP timesteppers for the full hybrid model. We hope this will give us the opportunity to incorporate previously inaccessible kinetic effects into the highly effective, modern, finite-element MHD models.
    \end{abstract}
    
    
    \newpage
    \tableofcontents
    
    
    \newpage
    \pagenumbering{arabic}
    %\linenumbers\renewcommand\thelinenumber{\color{black!50}\arabic{linenumber}}
            \input{0 - introduction/main.tex}
        \part{Research}
            \input{1 - low-noise PiC models/main.tex}
            \input{2 - kinetic component/main.tex}
            \input{3 - fluid component/main.tex}
            \input{4 - numerical implementation/main.tex}
        \part{Project Overview}
            \input{5 - research plan/main.tex}
            \input{6 - summary/main.tex}
    
    
    %\section{}
    \newpage
    \pagenumbering{gobble}
        \printbibliography


    \newpage
    \pagenumbering{roman}
    \appendix
        \part{Appendices}
            \input{8 - Hilbert complexes/main.tex}
            \input{9 - weak conservation proofs/main.tex}
\end{document}

            \documentclass[12pt, a4paper]{report}

\input{template/main.tex}

\title{\BA{Title in Progress...}}
\author{Boris Andrews}
\affil{Mathematical Institute, University of Oxford}
\date{\today}


\begin{document}
    \pagenumbering{gobble}
    \maketitle
    
    
    \begin{abstract}
        Magnetic confinement reactors---in particular tokamaks---offer one of the most promising options for achieving practical nuclear fusion, with the potential to provide virtually limitless, clean energy. The theoretical and numerical modeling of tokamak plasmas is simultaneously an essential component of effective reactor design, and a great research barrier. Tokamak operational conditions exhibit comparatively low Knudsen numbers. Kinetic effects, including kinetic waves and instabilities, Landau damping, bump-on-tail instabilities and more, are therefore highly influential in tokamak plasma dynamics. Purely fluid models are inherently incapable of capturing these effects, whereas the high dimensionality in purely kinetic models render them practically intractable for most relevant purposes.

        We consider a $\delta\!f$ decomposition model, with a macroscopic fluid background and microscopic kinetic correction, both fully coupled to each other. A similar manner of discretization is proposed to that used in the recent \texttt{STRUPHY} code \cite{Holderied_Possanner_Wang_2021, Holderied_2022, Li_et_al_2023} with a finite-element model for the background and a pseudo-particle/PiC model for the correction.

        The fluid background satisfies the full, non-linear, resistive, compressible, Hall MHD equations. \cite{Laakmann_Hu_Farrell_2022} introduces finite-element(-in-space) implicit timesteppers for the incompressible analogue to this system with structure-preserving (SP) properties in the ideal case, alongside parameter-robust preconditioners. We show that these timesteppers can derive from a finite-element-in-time (FET) (and finite-element-in-space) interpretation. The benefits of this reformulation are discussed, including the derivation of timesteppers that are higher order in time, and the quantifiable dissipative SP properties in the non-ideal, resistive case.
        
        We discuss possible options for extending this FET approach to timesteppers for the compressible case.

        The kinetic corrections satisfy linearized Boltzmann equations. Using a Lénard--Bernstein collision operator, these take Fokker--Planck-like forms \cite{Fokker_1914, Planck_1917} wherein pseudo-particles in the numerical model obey the neoclassical transport equations, with particle-independent Brownian drift terms. This offers a rigorous methodology for incorporating collisions into the particle transport model, without coupling the equations of motions for each particle.
        
        Works by Chen, Chacón et al. \cite{Chen_Chacón_Barnes_2011, Chacón_Chen_Barnes_2013, Chen_Chacón_2014, Chen_Chacón_2015} have developed structure-preserving particle pushers for neoclassical transport in the Vlasov equations, derived from Crank--Nicolson integrators. We show these too can can derive from a FET interpretation, similarly offering potential extensions to higher-order-in-time particle pushers. The FET formulation is used also to consider how the stochastic drift terms can be incorporated into the pushers. Stochastic gyrokinetic expansions are also discussed.

        Different options for the numerical implementation of these schemes are considered.

        Due to the efficacy of FET in the development of SP timesteppers for both the fluid and kinetic component, we hope this approach will prove effective in the future for developing SP timesteppers for the full hybrid model. We hope this will give us the opportunity to incorporate previously inaccessible kinetic effects into the highly effective, modern, finite-element MHD models.
    \end{abstract}
    
    
    \newpage
    \tableofcontents
    
    
    \newpage
    \pagenumbering{arabic}
    %\linenumbers\renewcommand\thelinenumber{\color{black!50}\arabic{linenumber}}
            \input{0 - introduction/main.tex}
        \part{Research}
            \input{1 - low-noise PiC models/main.tex}
            \input{2 - kinetic component/main.tex}
            \input{3 - fluid component/main.tex}
            \input{4 - numerical implementation/main.tex}
        \part{Project Overview}
            \input{5 - research plan/main.tex}
            \input{6 - summary/main.tex}
    
    
    %\section{}
    \newpage
    \pagenumbering{gobble}
        \printbibliography


    \newpage
    \pagenumbering{roman}
    \appendix
        \part{Appendices}
            \input{8 - Hilbert complexes/main.tex}
            \input{9 - weak conservation proofs/main.tex}
\end{document}

            \documentclass[12pt, a4paper]{report}

\input{template/main.tex}

\title{\BA{Title in Progress...}}
\author{Boris Andrews}
\affil{Mathematical Institute, University of Oxford}
\date{\today}


\begin{document}
    \pagenumbering{gobble}
    \maketitle
    
    
    \begin{abstract}
        Magnetic confinement reactors---in particular tokamaks---offer one of the most promising options for achieving practical nuclear fusion, with the potential to provide virtually limitless, clean energy. The theoretical and numerical modeling of tokamak plasmas is simultaneously an essential component of effective reactor design, and a great research barrier. Tokamak operational conditions exhibit comparatively low Knudsen numbers. Kinetic effects, including kinetic waves and instabilities, Landau damping, bump-on-tail instabilities and more, are therefore highly influential in tokamak plasma dynamics. Purely fluid models are inherently incapable of capturing these effects, whereas the high dimensionality in purely kinetic models render them practically intractable for most relevant purposes.

        We consider a $\delta\!f$ decomposition model, with a macroscopic fluid background and microscopic kinetic correction, both fully coupled to each other. A similar manner of discretization is proposed to that used in the recent \texttt{STRUPHY} code \cite{Holderied_Possanner_Wang_2021, Holderied_2022, Li_et_al_2023} with a finite-element model for the background and a pseudo-particle/PiC model for the correction.

        The fluid background satisfies the full, non-linear, resistive, compressible, Hall MHD equations. \cite{Laakmann_Hu_Farrell_2022} introduces finite-element(-in-space) implicit timesteppers for the incompressible analogue to this system with structure-preserving (SP) properties in the ideal case, alongside parameter-robust preconditioners. We show that these timesteppers can derive from a finite-element-in-time (FET) (and finite-element-in-space) interpretation. The benefits of this reformulation are discussed, including the derivation of timesteppers that are higher order in time, and the quantifiable dissipative SP properties in the non-ideal, resistive case.
        
        We discuss possible options for extending this FET approach to timesteppers for the compressible case.

        The kinetic corrections satisfy linearized Boltzmann equations. Using a Lénard--Bernstein collision operator, these take Fokker--Planck-like forms \cite{Fokker_1914, Planck_1917} wherein pseudo-particles in the numerical model obey the neoclassical transport equations, with particle-independent Brownian drift terms. This offers a rigorous methodology for incorporating collisions into the particle transport model, without coupling the equations of motions for each particle.
        
        Works by Chen, Chacón et al. \cite{Chen_Chacón_Barnes_2011, Chacón_Chen_Barnes_2013, Chen_Chacón_2014, Chen_Chacón_2015} have developed structure-preserving particle pushers for neoclassical transport in the Vlasov equations, derived from Crank--Nicolson integrators. We show these too can can derive from a FET interpretation, similarly offering potential extensions to higher-order-in-time particle pushers. The FET formulation is used also to consider how the stochastic drift terms can be incorporated into the pushers. Stochastic gyrokinetic expansions are also discussed.

        Different options for the numerical implementation of these schemes are considered.

        Due to the efficacy of FET in the development of SP timesteppers for both the fluid and kinetic component, we hope this approach will prove effective in the future for developing SP timesteppers for the full hybrid model. We hope this will give us the opportunity to incorporate previously inaccessible kinetic effects into the highly effective, modern, finite-element MHD models.
    \end{abstract}
    
    
    \newpage
    \tableofcontents
    
    
    \newpage
    \pagenumbering{arabic}
    %\linenumbers\renewcommand\thelinenumber{\color{black!50}\arabic{linenumber}}
            \input{0 - introduction/main.tex}
        \part{Research}
            \input{1 - low-noise PiC models/main.tex}
            \input{2 - kinetic component/main.tex}
            \input{3 - fluid component/main.tex}
            \input{4 - numerical implementation/main.tex}
        \part{Project Overview}
            \input{5 - research plan/main.tex}
            \input{6 - summary/main.tex}
    
    
    %\section{}
    \newpage
    \pagenumbering{gobble}
        \printbibliography


    \newpage
    \pagenumbering{roman}
    \appendix
        \part{Appendices}
            \input{8 - Hilbert complexes/main.tex}
            \input{9 - weak conservation proofs/main.tex}
\end{document}

        \part{Project Overview}
            \documentclass[12pt, a4paper]{report}

\input{template/main.tex}

\title{\BA{Title in Progress...}}
\author{Boris Andrews}
\affil{Mathematical Institute, University of Oxford}
\date{\today}


\begin{document}
    \pagenumbering{gobble}
    \maketitle
    
    
    \begin{abstract}
        Magnetic confinement reactors---in particular tokamaks---offer one of the most promising options for achieving practical nuclear fusion, with the potential to provide virtually limitless, clean energy. The theoretical and numerical modeling of tokamak plasmas is simultaneously an essential component of effective reactor design, and a great research barrier. Tokamak operational conditions exhibit comparatively low Knudsen numbers. Kinetic effects, including kinetic waves and instabilities, Landau damping, bump-on-tail instabilities and more, are therefore highly influential in tokamak plasma dynamics. Purely fluid models are inherently incapable of capturing these effects, whereas the high dimensionality in purely kinetic models render them practically intractable for most relevant purposes.

        We consider a $\delta\!f$ decomposition model, with a macroscopic fluid background and microscopic kinetic correction, both fully coupled to each other. A similar manner of discretization is proposed to that used in the recent \texttt{STRUPHY} code \cite{Holderied_Possanner_Wang_2021, Holderied_2022, Li_et_al_2023} with a finite-element model for the background and a pseudo-particle/PiC model for the correction.

        The fluid background satisfies the full, non-linear, resistive, compressible, Hall MHD equations. \cite{Laakmann_Hu_Farrell_2022} introduces finite-element(-in-space) implicit timesteppers for the incompressible analogue to this system with structure-preserving (SP) properties in the ideal case, alongside parameter-robust preconditioners. We show that these timesteppers can derive from a finite-element-in-time (FET) (and finite-element-in-space) interpretation. The benefits of this reformulation are discussed, including the derivation of timesteppers that are higher order in time, and the quantifiable dissipative SP properties in the non-ideal, resistive case.
        
        We discuss possible options for extending this FET approach to timesteppers for the compressible case.

        The kinetic corrections satisfy linearized Boltzmann equations. Using a Lénard--Bernstein collision operator, these take Fokker--Planck-like forms \cite{Fokker_1914, Planck_1917} wherein pseudo-particles in the numerical model obey the neoclassical transport equations, with particle-independent Brownian drift terms. This offers a rigorous methodology for incorporating collisions into the particle transport model, without coupling the equations of motions for each particle.
        
        Works by Chen, Chacón et al. \cite{Chen_Chacón_Barnes_2011, Chacón_Chen_Barnes_2013, Chen_Chacón_2014, Chen_Chacón_2015} have developed structure-preserving particle pushers for neoclassical transport in the Vlasov equations, derived from Crank--Nicolson integrators. We show these too can can derive from a FET interpretation, similarly offering potential extensions to higher-order-in-time particle pushers. The FET formulation is used also to consider how the stochastic drift terms can be incorporated into the pushers. Stochastic gyrokinetic expansions are also discussed.

        Different options for the numerical implementation of these schemes are considered.

        Due to the efficacy of FET in the development of SP timesteppers for both the fluid and kinetic component, we hope this approach will prove effective in the future for developing SP timesteppers for the full hybrid model. We hope this will give us the opportunity to incorporate previously inaccessible kinetic effects into the highly effective, modern, finite-element MHD models.
    \end{abstract}
    
    
    \newpage
    \tableofcontents
    
    
    \newpage
    \pagenumbering{arabic}
    %\linenumbers\renewcommand\thelinenumber{\color{black!50}\arabic{linenumber}}
            \input{0 - introduction/main.tex}
        \part{Research}
            \input{1 - low-noise PiC models/main.tex}
            \input{2 - kinetic component/main.tex}
            \input{3 - fluid component/main.tex}
            \input{4 - numerical implementation/main.tex}
        \part{Project Overview}
            \input{5 - research plan/main.tex}
            \input{6 - summary/main.tex}
    
    
    %\section{}
    \newpage
    \pagenumbering{gobble}
        \printbibliography


    \newpage
    \pagenumbering{roman}
    \appendix
        \part{Appendices}
            \input{8 - Hilbert complexes/main.tex}
            \input{9 - weak conservation proofs/main.tex}
\end{document}

            \documentclass[12pt, a4paper]{report}

\input{template/main.tex}

\title{\BA{Title in Progress...}}
\author{Boris Andrews}
\affil{Mathematical Institute, University of Oxford}
\date{\today}


\begin{document}
    \pagenumbering{gobble}
    \maketitle
    
    
    \begin{abstract}
        Magnetic confinement reactors---in particular tokamaks---offer one of the most promising options for achieving practical nuclear fusion, with the potential to provide virtually limitless, clean energy. The theoretical and numerical modeling of tokamak plasmas is simultaneously an essential component of effective reactor design, and a great research barrier. Tokamak operational conditions exhibit comparatively low Knudsen numbers. Kinetic effects, including kinetic waves and instabilities, Landau damping, bump-on-tail instabilities and more, are therefore highly influential in tokamak plasma dynamics. Purely fluid models are inherently incapable of capturing these effects, whereas the high dimensionality in purely kinetic models render them practically intractable for most relevant purposes.

        We consider a $\delta\!f$ decomposition model, with a macroscopic fluid background and microscopic kinetic correction, both fully coupled to each other. A similar manner of discretization is proposed to that used in the recent \texttt{STRUPHY} code \cite{Holderied_Possanner_Wang_2021, Holderied_2022, Li_et_al_2023} with a finite-element model for the background and a pseudo-particle/PiC model for the correction.

        The fluid background satisfies the full, non-linear, resistive, compressible, Hall MHD equations. \cite{Laakmann_Hu_Farrell_2022} introduces finite-element(-in-space) implicit timesteppers for the incompressible analogue to this system with structure-preserving (SP) properties in the ideal case, alongside parameter-robust preconditioners. We show that these timesteppers can derive from a finite-element-in-time (FET) (and finite-element-in-space) interpretation. The benefits of this reformulation are discussed, including the derivation of timesteppers that are higher order in time, and the quantifiable dissipative SP properties in the non-ideal, resistive case.
        
        We discuss possible options for extending this FET approach to timesteppers for the compressible case.

        The kinetic corrections satisfy linearized Boltzmann equations. Using a Lénard--Bernstein collision operator, these take Fokker--Planck-like forms \cite{Fokker_1914, Planck_1917} wherein pseudo-particles in the numerical model obey the neoclassical transport equations, with particle-independent Brownian drift terms. This offers a rigorous methodology for incorporating collisions into the particle transport model, without coupling the equations of motions for each particle.
        
        Works by Chen, Chacón et al. \cite{Chen_Chacón_Barnes_2011, Chacón_Chen_Barnes_2013, Chen_Chacón_2014, Chen_Chacón_2015} have developed structure-preserving particle pushers for neoclassical transport in the Vlasov equations, derived from Crank--Nicolson integrators. We show these too can can derive from a FET interpretation, similarly offering potential extensions to higher-order-in-time particle pushers. The FET formulation is used also to consider how the stochastic drift terms can be incorporated into the pushers. Stochastic gyrokinetic expansions are also discussed.

        Different options for the numerical implementation of these schemes are considered.

        Due to the efficacy of FET in the development of SP timesteppers for both the fluid and kinetic component, we hope this approach will prove effective in the future for developing SP timesteppers for the full hybrid model. We hope this will give us the opportunity to incorporate previously inaccessible kinetic effects into the highly effective, modern, finite-element MHD models.
    \end{abstract}
    
    
    \newpage
    \tableofcontents
    
    
    \newpage
    \pagenumbering{arabic}
    %\linenumbers\renewcommand\thelinenumber{\color{black!50}\arabic{linenumber}}
            \input{0 - introduction/main.tex}
        \part{Research}
            \input{1 - low-noise PiC models/main.tex}
            \input{2 - kinetic component/main.tex}
            \input{3 - fluid component/main.tex}
            \input{4 - numerical implementation/main.tex}
        \part{Project Overview}
            \input{5 - research plan/main.tex}
            \input{6 - summary/main.tex}
    
    
    %\section{}
    \newpage
    \pagenumbering{gobble}
        \printbibliography


    \newpage
    \pagenumbering{roman}
    \appendix
        \part{Appendices}
            \input{8 - Hilbert complexes/main.tex}
            \input{9 - weak conservation proofs/main.tex}
\end{document}

    
    
    %\section{}
    \newpage
    \pagenumbering{gobble}
        \printbibliography


    \newpage
    \pagenumbering{roman}
    \appendix
        \part{Appendices}
            \documentclass[12pt, a4paper]{report}

\input{template/main.tex}

\title{\BA{Title in Progress...}}
\author{Boris Andrews}
\affil{Mathematical Institute, University of Oxford}
\date{\today}


\begin{document}
    \pagenumbering{gobble}
    \maketitle
    
    
    \begin{abstract}
        Magnetic confinement reactors---in particular tokamaks---offer one of the most promising options for achieving practical nuclear fusion, with the potential to provide virtually limitless, clean energy. The theoretical and numerical modeling of tokamak plasmas is simultaneously an essential component of effective reactor design, and a great research barrier. Tokamak operational conditions exhibit comparatively low Knudsen numbers. Kinetic effects, including kinetic waves and instabilities, Landau damping, bump-on-tail instabilities and more, are therefore highly influential in tokamak plasma dynamics. Purely fluid models are inherently incapable of capturing these effects, whereas the high dimensionality in purely kinetic models render them practically intractable for most relevant purposes.

        We consider a $\delta\!f$ decomposition model, with a macroscopic fluid background and microscopic kinetic correction, both fully coupled to each other. A similar manner of discretization is proposed to that used in the recent \texttt{STRUPHY} code \cite{Holderied_Possanner_Wang_2021, Holderied_2022, Li_et_al_2023} with a finite-element model for the background and a pseudo-particle/PiC model for the correction.

        The fluid background satisfies the full, non-linear, resistive, compressible, Hall MHD equations. \cite{Laakmann_Hu_Farrell_2022} introduces finite-element(-in-space) implicit timesteppers for the incompressible analogue to this system with structure-preserving (SP) properties in the ideal case, alongside parameter-robust preconditioners. We show that these timesteppers can derive from a finite-element-in-time (FET) (and finite-element-in-space) interpretation. The benefits of this reformulation are discussed, including the derivation of timesteppers that are higher order in time, and the quantifiable dissipative SP properties in the non-ideal, resistive case.
        
        We discuss possible options for extending this FET approach to timesteppers for the compressible case.

        The kinetic corrections satisfy linearized Boltzmann equations. Using a Lénard--Bernstein collision operator, these take Fokker--Planck-like forms \cite{Fokker_1914, Planck_1917} wherein pseudo-particles in the numerical model obey the neoclassical transport equations, with particle-independent Brownian drift terms. This offers a rigorous methodology for incorporating collisions into the particle transport model, without coupling the equations of motions for each particle.
        
        Works by Chen, Chacón et al. \cite{Chen_Chacón_Barnes_2011, Chacón_Chen_Barnes_2013, Chen_Chacón_2014, Chen_Chacón_2015} have developed structure-preserving particle pushers for neoclassical transport in the Vlasov equations, derived from Crank--Nicolson integrators. We show these too can can derive from a FET interpretation, similarly offering potential extensions to higher-order-in-time particle pushers. The FET formulation is used also to consider how the stochastic drift terms can be incorporated into the pushers. Stochastic gyrokinetic expansions are also discussed.

        Different options for the numerical implementation of these schemes are considered.

        Due to the efficacy of FET in the development of SP timesteppers for both the fluid and kinetic component, we hope this approach will prove effective in the future for developing SP timesteppers for the full hybrid model. We hope this will give us the opportunity to incorporate previously inaccessible kinetic effects into the highly effective, modern, finite-element MHD models.
    \end{abstract}
    
    
    \newpage
    \tableofcontents
    
    
    \newpage
    \pagenumbering{arabic}
    %\linenumbers\renewcommand\thelinenumber{\color{black!50}\arabic{linenumber}}
            \input{0 - introduction/main.tex}
        \part{Research}
            \input{1 - low-noise PiC models/main.tex}
            \input{2 - kinetic component/main.tex}
            \input{3 - fluid component/main.tex}
            \input{4 - numerical implementation/main.tex}
        \part{Project Overview}
            \input{5 - research plan/main.tex}
            \input{6 - summary/main.tex}
    
    
    %\section{}
    \newpage
    \pagenumbering{gobble}
        \printbibliography


    \newpage
    \pagenumbering{roman}
    \appendix
        \part{Appendices}
            \input{8 - Hilbert complexes/main.tex}
            \input{9 - weak conservation proofs/main.tex}
\end{document}

            \documentclass[12pt, a4paper]{report}

\input{template/main.tex}

\title{\BA{Title in Progress...}}
\author{Boris Andrews}
\affil{Mathematical Institute, University of Oxford}
\date{\today}


\begin{document}
    \pagenumbering{gobble}
    \maketitle
    
    
    \begin{abstract}
        Magnetic confinement reactors---in particular tokamaks---offer one of the most promising options for achieving practical nuclear fusion, with the potential to provide virtually limitless, clean energy. The theoretical and numerical modeling of tokamak plasmas is simultaneously an essential component of effective reactor design, and a great research barrier. Tokamak operational conditions exhibit comparatively low Knudsen numbers. Kinetic effects, including kinetic waves and instabilities, Landau damping, bump-on-tail instabilities and more, are therefore highly influential in tokamak plasma dynamics. Purely fluid models are inherently incapable of capturing these effects, whereas the high dimensionality in purely kinetic models render them practically intractable for most relevant purposes.

        We consider a $\delta\!f$ decomposition model, with a macroscopic fluid background and microscopic kinetic correction, both fully coupled to each other. A similar manner of discretization is proposed to that used in the recent \texttt{STRUPHY} code \cite{Holderied_Possanner_Wang_2021, Holderied_2022, Li_et_al_2023} with a finite-element model for the background and a pseudo-particle/PiC model for the correction.

        The fluid background satisfies the full, non-linear, resistive, compressible, Hall MHD equations. \cite{Laakmann_Hu_Farrell_2022} introduces finite-element(-in-space) implicit timesteppers for the incompressible analogue to this system with structure-preserving (SP) properties in the ideal case, alongside parameter-robust preconditioners. We show that these timesteppers can derive from a finite-element-in-time (FET) (and finite-element-in-space) interpretation. The benefits of this reformulation are discussed, including the derivation of timesteppers that are higher order in time, and the quantifiable dissipative SP properties in the non-ideal, resistive case.
        
        We discuss possible options for extending this FET approach to timesteppers for the compressible case.

        The kinetic corrections satisfy linearized Boltzmann equations. Using a Lénard--Bernstein collision operator, these take Fokker--Planck-like forms \cite{Fokker_1914, Planck_1917} wherein pseudo-particles in the numerical model obey the neoclassical transport equations, with particle-independent Brownian drift terms. This offers a rigorous methodology for incorporating collisions into the particle transport model, without coupling the equations of motions for each particle.
        
        Works by Chen, Chacón et al. \cite{Chen_Chacón_Barnes_2011, Chacón_Chen_Barnes_2013, Chen_Chacón_2014, Chen_Chacón_2015} have developed structure-preserving particle pushers for neoclassical transport in the Vlasov equations, derived from Crank--Nicolson integrators. We show these too can can derive from a FET interpretation, similarly offering potential extensions to higher-order-in-time particle pushers. The FET formulation is used also to consider how the stochastic drift terms can be incorporated into the pushers. Stochastic gyrokinetic expansions are also discussed.

        Different options for the numerical implementation of these schemes are considered.

        Due to the efficacy of FET in the development of SP timesteppers for both the fluid and kinetic component, we hope this approach will prove effective in the future for developing SP timesteppers for the full hybrid model. We hope this will give us the opportunity to incorporate previously inaccessible kinetic effects into the highly effective, modern, finite-element MHD models.
    \end{abstract}
    
    
    \newpage
    \tableofcontents
    
    
    \newpage
    \pagenumbering{arabic}
    %\linenumbers\renewcommand\thelinenumber{\color{black!50}\arabic{linenumber}}
            \input{0 - introduction/main.tex}
        \part{Research}
            \input{1 - low-noise PiC models/main.tex}
            \input{2 - kinetic component/main.tex}
            \input{3 - fluid component/main.tex}
            \input{4 - numerical implementation/main.tex}
        \part{Project Overview}
            \input{5 - research plan/main.tex}
            \input{6 - summary/main.tex}
    
    
    %\section{}
    \newpage
    \pagenumbering{gobble}
        \printbibliography


    \newpage
    \pagenumbering{roman}
    \appendix
        \part{Appendices}
            \input{8 - Hilbert complexes/main.tex}
            \input{9 - weak conservation proofs/main.tex}
\end{document}

\end{document}

            \documentclass[12pt, a4paper]{report}

\documentclass[12pt, a4paper]{report}

\input{template/main.tex}

\title{\BA{Title in Progress...}}
\author{Boris Andrews}
\affil{Mathematical Institute, University of Oxford}
\date{\today}


\begin{document}
    \pagenumbering{gobble}
    \maketitle
    
    
    \begin{abstract}
        Magnetic confinement reactors---in particular tokamaks---offer one of the most promising options for achieving practical nuclear fusion, with the potential to provide virtually limitless, clean energy. The theoretical and numerical modeling of tokamak plasmas is simultaneously an essential component of effective reactor design, and a great research barrier. Tokamak operational conditions exhibit comparatively low Knudsen numbers. Kinetic effects, including kinetic waves and instabilities, Landau damping, bump-on-tail instabilities and more, are therefore highly influential in tokamak plasma dynamics. Purely fluid models are inherently incapable of capturing these effects, whereas the high dimensionality in purely kinetic models render them practically intractable for most relevant purposes.

        We consider a $\delta\!f$ decomposition model, with a macroscopic fluid background and microscopic kinetic correction, both fully coupled to each other. A similar manner of discretization is proposed to that used in the recent \texttt{STRUPHY} code \cite{Holderied_Possanner_Wang_2021, Holderied_2022, Li_et_al_2023} with a finite-element model for the background and a pseudo-particle/PiC model for the correction.

        The fluid background satisfies the full, non-linear, resistive, compressible, Hall MHD equations. \cite{Laakmann_Hu_Farrell_2022} introduces finite-element(-in-space) implicit timesteppers for the incompressible analogue to this system with structure-preserving (SP) properties in the ideal case, alongside parameter-robust preconditioners. We show that these timesteppers can derive from a finite-element-in-time (FET) (and finite-element-in-space) interpretation. The benefits of this reformulation are discussed, including the derivation of timesteppers that are higher order in time, and the quantifiable dissipative SP properties in the non-ideal, resistive case.
        
        We discuss possible options for extending this FET approach to timesteppers for the compressible case.

        The kinetic corrections satisfy linearized Boltzmann equations. Using a Lénard--Bernstein collision operator, these take Fokker--Planck-like forms \cite{Fokker_1914, Planck_1917} wherein pseudo-particles in the numerical model obey the neoclassical transport equations, with particle-independent Brownian drift terms. This offers a rigorous methodology for incorporating collisions into the particle transport model, without coupling the equations of motions for each particle.
        
        Works by Chen, Chacón et al. \cite{Chen_Chacón_Barnes_2011, Chacón_Chen_Barnes_2013, Chen_Chacón_2014, Chen_Chacón_2015} have developed structure-preserving particle pushers for neoclassical transport in the Vlasov equations, derived from Crank--Nicolson integrators. We show these too can can derive from a FET interpretation, similarly offering potential extensions to higher-order-in-time particle pushers. The FET formulation is used also to consider how the stochastic drift terms can be incorporated into the pushers. Stochastic gyrokinetic expansions are also discussed.

        Different options for the numerical implementation of these schemes are considered.

        Due to the efficacy of FET in the development of SP timesteppers for both the fluid and kinetic component, we hope this approach will prove effective in the future for developing SP timesteppers for the full hybrid model. We hope this will give us the opportunity to incorporate previously inaccessible kinetic effects into the highly effective, modern, finite-element MHD models.
    \end{abstract}
    
    
    \newpage
    \tableofcontents
    
    
    \newpage
    \pagenumbering{arabic}
    %\linenumbers\renewcommand\thelinenumber{\color{black!50}\arabic{linenumber}}
            \input{0 - introduction/main.tex}
        \part{Research}
            \input{1 - low-noise PiC models/main.tex}
            \input{2 - kinetic component/main.tex}
            \input{3 - fluid component/main.tex}
            \input{4 - numerical implementation/main.tex}
        \part{Project Overview}
            \input{5 - research plan/main.tex}
            \input{6 - summary/main.tex}
    
    
    %\section{}
    \newpage
    \pagenumbering{gobble}
        \printbibliography


    \newpage
    \pagenumbering{roman}
    \appendix
        \part{Appendices}
            \input{8 - Hilbert complexes/main.tex}
            \input{9 - weak conservation proofs/main.tex}
\end{document}


\title{\BA{Title in Progress...}}
\author{Boris Andrews}
\affil{Mathematical Institute, University of Oxford}
\date{\today}


\begin{document}
    \pagenumbering{gobble}
    \maketitle
    
    
    \begin{abstract}
        Magnetic confinement reactors---in particular tokamaks---offer one of the most promising options for achieving practical nuclear fusion, with the potential to provide virtually limitless, clean energy. The theoretical and numerical modeling of tokamak plasmas is simultaneously an essential component of effective reactor design, and a great research barrier. Tokamak operational conditions exhibit comparatively low Knudsen numbers. Kinetic effects, including kinetic waves and instabilities, Landau damping, bump-on-tail instabilities and more, are therefore highly influential in tokamak plasma dynamics. Purely fluid models are inherently incapable of capturing these effects, whereas the high dimensionality in purely kinetic models render them practically intractable for most relevant purposes.

        We consider a $\delta\!f$ decomposition model, with a macroscopic fluid background and microscopic kinetic correction, both fully coupled to each other. A similar manner of discretization is proposed to that used in the recent \texttt{STRUPHY} code \cite{Holderied_Possanner_Wang_2021, Holderied_2022, Li_et_al_2023} with a finite-element model for the background and a pseudo-particle/PiC model for the correction.

        The fluid background satisfies the full, non-linear, resistive, compressible, Hall MHD equations. \cite{Laakmann_Hu_Farrell_2022} introduces finite-element(-in-space) implicit timesteppers for the incompressible analogue to this system with structure-preserving (SP) properties in the ideal case, alongside parameter-robust preconditioners. We show that these timesteppers can derive from a finite-element-in-time (FET) (and finite-element-in-space) interpretation. The benefits of this reformulation are discussed, including the derivation of timesteppers that are higher order in time, and the quantifiable dissipative SP properties in the non-ideal, resistive case.
        
        We discuss possible options for extending this FET approach to timesteppers for the compressible case.

        The kinetic corrections satisfy linearized Boltzmann equations. Using a Lénard--Bernstein collision operator, these take Fokker--Planck-like forms \cite{Fokker_1914, Planck_1917} wherein pseudo-particles in the numerical model obey the neoclassical transport equations, with particle-independent Brownian drift terms. This offers a rigorous methodology for incorporating collisions into the particle transport model, without coupling the equations of motions for each particle.
        
        Works by Chen, Chacón et al. \cite{Chen_Chacón_Barnes_2011, Chacón_Chen_Barnes_2013, Chen_Chacón_2014, Chen_Chacón_2015} have developed structure-preserving particle pushers for neoclassical transport in the Vlasov equations, derived from Crank--Nicolson integrators. We show these too can can derive from a FET interpretation, similarly offering potential extensions to higher-order-in-time particle pushers. The FET formulation is used also to consider how the stochastic drift terms can be incorporated into the pushers. Stochastic gyrokinetic expansions are also discussed.

        Different options for the numerical implementation of these schemes are considered.

        Due to the efficacy of FET in the development of SP timesteppers for both the fluid and kinetic component, we hope this approach will prove effective in the future for developing SP timesteppers for the full hybrid model. We hope this will give us the opportunity to incorporate previously inaccessible kinetic effects into the highly effective, modern, finite-element MHD models.
    \end{abstract}
    
    
    \newpage
    \tableofcontents
    
    
    \newpage
    \pagenumbering{arabic}
    %\linenumbers\renewcommand\thelinenumber{\color{black!50}\arabic{linenumber}}
            \documentclass[12pt, a4paper]{report}

\input{template/main.tex}

\title{\BA{Title in Progress...}}
\author{Boris Andrews}
\affil{Mathematical Institute, University of Oxford}
\date{\today}


\begin{document}
    \pagenumbering{gobble}
    \maketitle
    
    
    \begin{abstract}
        Magnetic confinement reactors---in particular tokamaks---offer one of the most promising options for achieving practical nuclear fusion, with the potential to provide virtually limitless, clean energy. The theoretical and numerical modeling of tokamak plasmas is simultaneously an essential component of effective reactor design, and a great research barrier. Tokamak operational conditions exhibit comparatively low Knudsen numbers. Kinetic effects, including kinetic waves and instabilities, Landau damping, bump-on-tail instabilities and more, are therefore highly influential in tokamak plasma dynamics. Purely fluid models are inherently incapable of capturing these effects, whereas the high dimensionality in purely kinetic models render them practically intractable for most relevant purposes.

        We consider a $\delta\!f$ decomposition model, with a macroscopic fluid background and microscopic kinetic correction, both fully coupled to each other. A similar manner of discretization is proposed to that used in the recent \texttt{STRUPHY} code \cite{Holderied_Possanner_Wang_2021, Holderied_2022, Li_et_al_2023} with a finite-element model for the background and a pseudo-particle/PiC model for the correction.

        The fluid background satisfies the full, non-linear, resistive, compressible, Hall MHD equations. \cite{Laakmann_Hu_Farrell_2022} introduces finite-element(-in-space) implicit timesteppers for the incompressible analogue to this system with structure-preserving (SP) properties in the ideal case, alongside parameter-robust preconditioners. We show that these timesteppers can derive from a finite-element-in-time (FET) (and finite-element-in-space) interpretation. The benefits of this reformulation are discussed, including the derivation of timesteppers that are higher order in time, and the quantifiable dissipative SP properties in the non-ideal, resistive case.
        
        We discuss possible options for extending this FET approach to timesteppers for the compressible case.

        The kinetic corrections satisfy linearized Boltzmann equations. Using a Lénard--Bernstein collision operator, these take Fokker--Planck-like forms \cite{Fokker_1914, Planck_1917} wherein pseudo-particles in the numerical model obey the neoclassical transport equations, with particle-independent Brownian drift terms. This offers a rigorous methodology for incorporating collisions into the particle transport model, without coupling the equations of motions for each particle.
        
        Works by Chen, Chacón et al. \cite{Chen_Chacón_Barnes_2011, Chacón_Chen_Barnes_2013, Chen_Chacón_2014, Chen_Chacón_2015} have developed structure-preserving particle pushers for neoclassical transport in the Vlasov equations, derived from Crank--Nicolson integrators. We show these too can can derive from a FET interpretation, similarly offering potential extensions to higher-order-in-time particle pushers. The FET formulation is used also to consider how the stochastic drift terms can be incorporated into the pushers. Stochastic gyrokinetic expansions are also discussed.

        Different options for the numerical implementation of these schemes are considered.

        Due to the efficacy of FET in the development of SP timesteppers for both the fluid and kinetic component, we hope this approach will prove effective in the future for developing SP timesteppers for the full hybrid model. We hope this will give us the opportunity to incorporate previously inaccessible kinetic effects into the highly effective, modern, finite-element MHD models.
    \end{abstract}
    
    
    \newpage
    \tableofcontents
    
    
    \newpage
    \pagenumbering{arabic}
    %\linenumbers\renewcommand\thelinenumber{\color{black!50}\arabic{linenumber}}
            \input{0 - introduction/main.tex}
        \part{Research}
            \input{1 - low-noise PiC models/main.tex}
            \input{2 - kinetic component/main.tex}
            \input{3 - fluid component/main.tex}
            \input{4 - numerical implementation/main.tex}
        \part{Project Overview}
            \input{5 - research plan/main.tex}
            \input{6 - summary/main.tex}
    
    
    %\section{}
    \newpage
    \pagenumbering{gobble}
        \printbibliography


    \newpage
    \pagenumbering{roman}
    \appendix
        \part{Appendices}
            \input{8 - Hilbert complexes/main.tex}
            \input{9 - weak conservation proofs/main.tex}
\end{document}

        \part{Research}
            \documentclass[12pt, a4paper]{report}

\input{template/main.tex}

\title{\BA{Title in Progress...}}
\author{Boris Andrews}
\affil{Mathematical Institute, University of Oxford}
\date{\today}


\begin{document}
    \pagenumbering{gobble}
    \maketitle
    
    
    \begin{abstract}
        Magnetic confinement reactors---in particular tokamaks---offer one of the most promising options for achieving practical nuclear fusion, with the potential to provide virtually limitless, clean energy. The theoretical and numerical modeling of tokamak plasmas is simultaneously an essential component of effective reactor design, and a great research barrier. Tokamak operational conditions exhibit comparatively low Knudsen numbers. Kinetic effects, including kinetic waves and instabilities, Landau damping, bump-on-tail instabilities and more, are therefore highly influential in tokamak plasma dynamics. Purely fluid models are inherently incapable of capturing these effects, whereas the high dimensionality in purely kinetic models render them practically intractable for most relevant purposes.

        We consider a $\delta\!f$ decomposition model, with a macroscopic fluid background and microscopic kinetic correction, both fully coupled to each other. A similar manner of discretization is proposed to that used in the recent \texttt{STRUPHY} code \cite{Holderied_Possanner_Wang_2021, Holderied_2022, Li_et_al_2023} with a finite-element model for the background and a pseudo-particle/PiC model for the correction.

        The fluid background satisfies the full, non-linear, resistive, compressible, Hall MHD equations. \cite{Laakmann_Hu_Farrell_2022} introduces finite-element(-in-space) implicit timesteppers for the incompressible analogue to this system with structure-preserving (SP) properties in the ideal case, alongside parameter-robust preconditioners. We show that these timesteppers can derive from a finite-element-in-time (FET) (and finite-element-in-space) interpretation. The benefits of this reformulation are discussed, including the derivation of timesteppers that are higher order in time, and the quantifiable dissipative SP properties in the non-ideal, resistive case.
        
        We discuss possible options for extending this FET approach to timesteppers for the compressible case.

        The kinetic corrections satisfy linearized Boltzmann equations. Using a Lénard--Bernstein collision operator, these take Fokker--Planck-like forms \cite{Fokker_1914, Planck_1917} wherein pseudo-particles in the numerical model obey the neoclassical transport equations, with particle-independent Brownian drift terms. This offers a rigorous methodology for incorporating collisions into the particle transport model, without coupling the equations of motions for each particle.
        
        Works by Chen, Chacón et al. \cite{Chen_Chacón_Barnes_2011, Chacón_Chen_Barnes_2013, Chen_Chacón_2014, Chen_Chacón_2015} have developed structure-preserving particle pushers for neoclassical transport in the Vlasov equations, derived from Crank--Nicolson integrators. We show these too can can derive from a FET interpretation, similarly offering potential extensions to higher-order-in-time particle pushers. The FET formulation is used also to consider how the stochastic drift terms can be incorporated into the pushers. Stochastic gyrokinetic expansions are also discussed.

        Different options for the numerical implementation of these schemes are considered.

        Due to the efficacy of FET in the development of SP timesteppers for both the fluid and kinetic component, we hope this approach will prove effective in the future for developing SP timesteppers for the full hybrid model. We hope this will give us the opportunity to incorporate previously inaccessible kinetic effects into the highly effective, modern, finite-element MHD models.
    \end{abstract}
    
    
    \newpage
    \tableofcontents
    
    
    \newpage
    \pagenumbering{arabic}
    %\linenumbers\renewcommand\thelinenumber{\color{black!50}\arabic{linenumber}}
            \input{0 - introduction/main.tex}
        \part{Research}
            \input{1 - low-noise PiC models/main.tex}
            \input{2 - kinetic component/main.tex}
            \input{3 - fluid component/main.tex}
            \input{4 - numerical implementation/main.tex}
        \part{Project Overview}
            \input{5 - research plan/main.tex}
            \input{6 - summary/main.tex}
    
    
    %\section{}
    \newpage
    \pagenumbering{gobble}
        \printbibliography


    \newpage
    \pagenumbering{roman}
    \appendix
        \part{Appendices}
            \input{8 - Hilbert complexes/main.tex}
            \input{9 - weak conservation proofs/main.tex}
\end{document}

            \documentclass[12pt, a4paper]{report}

\input{template/main.tex}

\title{\BA{Title in Progress...}}
\author{Boris Andrews}
\affil{Mathematical Institute, University of Oxford}
\date{\today}


\begin{document}
    \pagenumbering{gobble}
    \maketitle
    
    
    \begin{abstract}
        Magnetic confinement reactors---in particular tokamaks---offer one of the most promising options for achieving practical nuclear fusion, with the potential to provide virtually limitless, clean energy. The theoretical and numerical modeling of tokamak plasmas is simultaneously an essential component of effective reactor design, and a great research barrier. Tokamak operational conditions exhibit comparatively low Knudsen numbers. Kinetic effects, including kinetic waves and instabilities, Landau damping, bump-on-tail instabilities and more, are therefore highly influential in tokamak plasma dynamics. Purely fluid models are inherently incapable of capturing these effects, whereas the high dimensionality in purely kinetic models render them practically intractable for most relevant purposes.

        We consider a $\delta\!f$ decomposition model, with a macroscopic fluid background and microscopic kinetic correction, both fully coupled to each other. A similar manner of discretization is proposed to that used in the recent \texttt{STRUPHY} code \cite{Holderied_Possanner_Wang_2021, Holderied_2022, Li_et_al_2023} with a finite-element model for the background and a pseudo-particle/PiC model for the correction.

        The fluid background satisfies the full, non-linear, resistive, compressible, Hall MHD equations. \cite{Laakmann_Hu_Farrell_2022} introduces finite-element(-in-space) implicit timesteppers for the incompressible analogue to this system with structure-preserving (SP) properties in the ideal case, alongside parameter-robust preconditioners. We show that these timesteppers can derive from a finite-element-in-time (FET) (and finite-element-in-space) interpretation. The benefits of this reformulation are discussed, including the derivation of timesteppers that are higher order in time, and the quantifiable dissipative SP properties in the non-ideal, resistive case.
        
        We discuss possible options for extending this FET approach to timesteppers for the compressible case.

        The kinetic corrections satisfy linearized Boltzmann equations. Using a Lénard--Bernstein collision operator, these take Fokker--Planck-like forms \cite{Fokker_1914, Planck_1917} wherein pseudo-particles in the numerical model obey the neoclassical transport equations, with particle-independent Brownian drift terms. This offers a rigorous methodology for incorporating collisions into the particle transport model, without coupling the equations of motions for each particle.
        
        Works by Chen, Chacón et al. \cite{Chen_Chacón_Barnes_2011, Chacón_Chen_Barnes_2013, Chen_Chacón_2014, Chen_Chacón_2015} have developed structure-preserving particle pushers for neoclassical transport in the Vlasov equations, derived from Crank--Nicolson integrators. We show these too can can derive from a FET interpretation, similarly offering potential extensions to higher-order-in-time particle pushers. The FET formulation is used also to consider how the stochastic drift terms can be incorporated into the pushers. Stochastic gyrokinetic expansions are also discussed.

        Different options for the numerical implementation of these schemes are considered.

        Due to the efficacy of FET in the development of SP timesteppers for both the fluid and kinetic component, we hope this approach will prove effective in the future for developing SP timesteppers for the full hybrid model. We hope this will give us the opportunity to incorporate previously inaccessible kinetic effects into the highly effective, modern, finite-element MHD models.
    \end{abstract}
    
    
    \newpage
    \tableofcontents
    
    
    \newpage
    \pagenumbering{arabic}
    %\linenumbers\renewcommand\thelinenumber{\color{black!50}\arabic{linenumber}}
            \input{0 - introduction/main.tex}
        \part{Research}
            \input{1 - low-noise PiC models/main.tex}
            \input{2 - kinetic component/main.tex}
            \input{3 - fluid component/main.tex}
            \input{4 - numerical implementation/main.tex}
        \part{Project Overview}
            \input{5 - research plan/main.tex}
            \input{6 - summary/main.tex}
    
    
    %\section{}
    \newpage
    \pagenumbering{gobble}
        \printbibliography


    \newpage
    \pagenumbering{roman}
    \appendix
        \part{Appendices}
            \input{8 - Hilbert complexes/main.tex}
            \input{9 - weak conservation proofs/main.tex}
\end{document}

            \documentclass[12pt, a4paper]{report}

\input{template/main.tex}

\title{\BA{Title in Progress...}}
\author{Boris Andrews}
\affil{Mathematical Institute, University of Oxford}
\date{\today}


\begin{document}
    \pagenumbering{gobble}
    \maketitle
    
    
    \begin{abstract}
        Magnetic confinement reactors---in particular tokamaks---offer one of the most promising options for achieving practical nuclear fusion, with the potential to provide virtually limitless, clean energy. The theoretical and numerical modeling of tokamak plasmas is simultaneously an essential component of effective reactor design, and a great research barrier. Tokamak operational conditions exhibit comparatively low Knudsen numbers. Kinetic effects, including kinetic waves and instabilities, Landau damping, bump-on-tail instabilities and more, are therefore highly influential in tokamak plasma dynamics. Purely fluid models are inherently incapable of capturing these effects, whereas the high dimensionality in purely kinetic models render them practically intractable for most relevant purposes.

        We consider a $\delta\!f$ decomposition model, with a macroscopic fluid background and microscopic kinetic correction, both fully coupled to each other. A similar manner of discretization is proposed to that used in the recent \texttt{STRUPHY} code \cite{Holderied_Possanner_Wang_2021, Holderied_2022, Li_et_al_2023} with a finite-element model for the background and a pseudo-particle/PiC model for the correction.

        The fluid background satisfies the full, non-linear, resistive, compressible, Hall MHD equations. \cite{Laakmann_Hu_Farrell_2022} introduces finite-element(-in-space) implicit timesteppers for the incompressible analogue to this system with structure-preserving (SP) properties in the ideal case, alongside parameter-robust preconditioners. We show that these timesteppers can derive from a finite-element-in-time (FET) (and finite-element-in-space) interpretation. The benefits of this reformulation are discussed, including the derivation of timesteppers that are higher order in time, and the quantifiable dissipative SP properties in the non-ideal, resistive case.
        
        We discuss possible options for extending this FET approach to timesteppers for the compressible case.

        The kinetic corrections satisfy linearized Boltzmann equations. Using a Lénard--Bernstein collision operator, these take Fokker--Planck-like forms \cite{Fokker_1914, Planck_1917} wherein pseudo-particles in the numerical model obey the neoclassical transport equations, with particle-independent Brownian drift terms. This offers a rigorous methodology for incorporating collisions into the particle transport model, without coupling the equations of motions for each particle.
        
        Works by Chen, Chacón et al. \cite{Chen_Chacón_Barnes_2011, Chacón_Chen_Barnes_2013, Chen_Chacón_2014, Chen_Chacón_2015} have developed structure-preserving particle pushers for neoclassical transport in the Vlasov equations, derived from Crank--Nicolson integrators. We show these too can can derive from a FET interpretation, similarly offering potential extensions to higher-order-in-time particle pushers. The FET formulation is used also to consider how the stochastic drift terms can be incorporated into the pushers. Stochastic gyrokinetic expansions are also discussed.

        Different options for the numerical implementation of these schemes are considered.

        Due to the efficacy of FET in the development of SP timesteppers for both the fluid and kinetic component, we hope this approach will prove effective in the future for developing SP timesteppers for the full hybrid model. We hope this will give us the opportunity to incorporate previously inaccessible kinetic effects into the highly effective, modern, finite-element MHD models.
    \end{abstract}
    
    
    \newpage
    \tableofcontents
    
    
    \newpage
    \pagenumbering{arabic}
    %\linenumbers\renewcommand\thelinenumber{\color{black!50}\arabic{linenumber}}
            \input{0 - introduction/main.tex}
        \part{Research}
            \input{1 - low-noise PiC models/main.tex}
            \input{2 - kinetic component/main.tex}
            \input{3 - fluid component/main.tex}
            \input{4 - numerical implementation/main.tex}
        \part{Project Overview}
            \input{5 - research plan/main.tex}
            \input{6 - summary/main.tex}
    
    
    %\section{}
    \newpage
    \pagenumbering{gobble}
        \printbibliography


    \newpage
    \pagenumbering{roman}
    \appendix
        \part{Appendices}
            \input{8 - Hilbert complexes/main.tex}
            \input{9 - weak conservation proofs/main.tex}
\end{document}

            \documentclass[12pt, a4paper]{report}

\input{template/main.tex}

\title{\BA{Title in Progress...}}
\author{Boris Andrews}
\affil{Mathematical Institute, University of Oxford}
\date{\today}


\begin{document}
    \pagenumbering{gobble}
    \maketitle
    
    
    \begin{abstract}
        Magnetic confinement reactors---in particular tokamaks---offer one of the most promising options for achieving practical nuclear fusion, with the potential to provide virtually limitless, clean energy. The theoretical and numerical modeling of tokamak plasmas is simultaneously an essential component of effective reactor design, and a great research barrier. Tokamak operational conditions exhibit comparatively low Knudsen numbers. Kinetic effects, including kinetic waves and instabilities, Landau damping, bump-on-tail instabilities and more, are therefore highly influential in tokamak plasma dynamics. Purely fluid models are inherently incapable of capturing these effects, whereas the high dimensionality in purely kinetic models render them practically intractable for most relevant purposes.

        We consider a $\delta\!f$ decomposition model, with a macroscopic fluid background and microscopic kinetic correction, both fully coupled to each other. A similar manner of discretization is proposed to that used in the recent \texttt{STRUPHY} code \cite{Holderied_Possanner_Wang_2021, Holderied_2022, Li_et_al_2023} with a finite-element model for the background and a pseudo-particle/PiC model for the correction.

        The fluid background satisfies the full, non-linear, resistive, compressible, Hall MHD equations. \cite{Laakmann_Hu_Farrell_2022} introduces finite-element(-in-space) implicit timesteppers for the incompressible analogue to this system with structure-preserving (SP) properties in the ideal case, alongside parameter-robust preconditioners. We show that these timesteppers can derive from a finite-element-in-time (FET) (and finite-element-in-space) interpretation. The benefits of this reformulation are discussed, including the derivation of timesteppers that are higher order in time, and the quantifiable dissipative SP properties in the non-ideal, resistive case.
        
        We discuss possible options for extending this FET approach to timesteppers for the compressible case.

        The kinetic corrections satisfy linearized Boltzmann equations. Using a Lénard--Bernstein collision operator, these take Fokker--Planck-like forms \cite{Fokker_1914, Planck_1917} wherein pseudo-particles in the numerical model obey the neoclassical transport equations, with particle-independent Brownian drift terms. This offers a rigorous methodology for incorporating collisions into the particle transport model, without coupling the equations of motions for each particle.
        
        Works by Chen, Chacón et al. \cite{Chen_Chacón_Barnes_2011, Chacón_Chen_Barnes_2013, Chen_Chacón_2014, Chen_Chacón_2015} have developed structure-preserving particle pushers for neoclassical transport in the Vlasov equations, derived from Crank--Nicolson integrators. We show these too can can derive from a FET interpretation, similarly offering potential extensions to higher-order-in-time particle pushers. The FET formulation is used also to consider how the stochastic drift terms can be incorporated into the pushers. Stochastic gyrokinetic expansions are also discussed.

        Different options for the numerical implementation of these schemes are considered.

        Due to the efficacy of FET in the development of SP timesteppers for both the fluid and kinetic component, we hope this approach will prove effective in the future for developing SP timesteppers for the full hybrid model. We hope this will give us the opportunity to incorporate previously inaccessible kinetic effects into the highly effective, modern, finite-element MHD models.
    \end{abstract}
    
    
    \newpage
    \tableofcontents
    
    
    \newpage
    \pagenumbering{arabic}
    %\linenumbers\renewcommand\thelinenumber{\color{black!50}\arabic{linenumber}}
            \input{0 - introduction/main.tex}
        \part{Research}
            \input{1 - low-noise PiC models/main.tex}
            \input{2 - kinetic component/main.tex}
            \input{3 - fluid component/main.tex}
            \input{4 - numerical implementation/main.tex}
        \part{Project Overview}
            \input{5 - research plan/main.tex}
            \input{6 - summary/main.tex}
    
    
    %\section{}
    \newpage
    \pagenumbering{gobble}
        \printbibliography


    \newpage
    \pagenumbering{roman}
    \appendix
        \part{Appendices}
            \input{8 - Hilbert complexes/main.tex}
            \input{9 - weak conservation proofs/main.tex}
\end{document}

        \part{Project Overview}
            \documentclass[12pt, a4paper]{report}

\input{template/main.tex}

\title{\BA{Title in Progress...}}
\author{Boris Andrews}
\affil{Mathematical Institute, University of Oxford}
\date{\today}


\begin{document}
    \pagenumbering{gobble}
    \maketitle
    
    
    \begin{abstract}
        Magnetic confinement reactors---in particular tokamaks---offer one of the most promising options for achieving practical nuclear fusion, with the potential to provide virtually limitless, clean energy. The theoretical and numerical modeling of tokamak plasmas is simultaneously an essential component of effective reactor design, and a great research barrier. Tokamak operational conditions exhibit comparatively low Knudsen numbers. Kinetic effects, including kinetic waves and instabilities, Landau damping, bump-on-tail instabilities and more, are therefore highly influential in tokamak plasma dynamics. Purely fluid models are inherently incapable of capturing these effects, whereas the high dimensionality in purely kinetic models render them practically intractable for most relevant purposes.

        We consider a $\delta\!f$ decomposition model, with a macroscopic fluid background and microscopic kinetic correction, both fully coupled to each other. A similar manner of discretization is proposed to that used in the recent \texttt{STRUPHY} code \cite{Holderied_Possanner_Wang_2021, Holderied_2022, Li_et_al_2023} with a finite-element model for the background and a pseudo-particle/PiC model for the correction.

        The fluid background satisfies the full, non-linear, resistive, compressible, Hall MHD equations. \cite{Laakmann_Hu_Farrell_2022} introduces finite-element(-in-space) implicit timesteppers for the incompressible analogue to this system with structure-preserving (SP) properties in the ideal case, alongside parameter-robust preconditioners. We show that these timesteppers can derive from a finite-element-in-time (FET) (and finite-element-in-space) interpretation. The benefits of this reformulation are discussed, including the derivation of timesteppers that are higher order in time, and the quantifiable dissipative SP properties in the non-ideal, resistive case.
        
        We discuss possible options for extending this FET approach to timesteppers for the compressible case.

        The kinetic corrections satisfy linearized Boltzmann equations. Using a Lénard--Bernstein collision operator, these take Fokker--Planck-like forms \cite{Fokker_1914, Planck_1917} wherein pseudo-particles in the numerical model obey the neoclassical transport equations, with particle-independent Brownian drift terms. This offers a rigorous methodology for incorporating collisions into the particle transport model, without coupling the equations of motions for each particle.
        
        Works by Chen, Chacón et al. \cite{Chen_Chacón_Barnes_2011, Chacón_Chen_Barnes_2013, Chen_Chacón_2014, Chen_Chacón_2015} have developed structure-preserving particle pushers for neoclassical transport in the Vlasov equations, derived from Crank--Nicolson integrators. We show these too can can derive from a FET interpretation, similarly offering potential extensions to higher-order-in-time particle pushers. The FET formulation is used also to consider how the stochastic drift terms can be incorporated into the pushers. Stochastic gyrokinetic expansions are also discussed.

        Different options for the numerical implementation of these schemes are considered.

        Due to the efficacy of FET in the development of SP timesteppers for both the fluid and kinetic component, we hope this approach will prove effective in the future for developing SP timesteppers for the full hybrid model. We hope this will give us the opportunity to incorporate previously inaccessible kinetic effects into the highly effective, modern, finite-element MHD models.
    \end{abstract}
    
    
    \newpage
    \tableofcontents
    
    
    \newpage
    \pagenumbering{arabic}
    %\linenumbers\renewcommand\thelinenumber{\color{black!50}\arabic{linenumber}}
            \input{0 - introduction/main.tex}
        \part{Research}
            \input{1 - low-noise PiC models/main.tex}
            \input{2 - kinetic component/main.tex}
            \input{3 - fluid component/main.tex}
            \input{4 - numerical implementation/main.tex}
        \part{Project Overview}
            \input{5 - research plan/main.tex}
            \input{6 - summary/main.tex}
    
    
    %\section{}
    \newpage
    \pagenumbering{gobble}
        \printbibliography


    \newpage
    \pagenumbering{roman}
    \appendix
        \part{Appendices}
            \input{8 - Hilbert complexes/main.tex}
            \input{9 - weak conservation proofs/main.tex}
\end{document}

            \documentclass[12pt, a4paper]{report}

\input{template/main.tex}

\title{\BA{Title in Progress...}}
\author{Boris Andrews}
\affil{Mathematical Institute, University of Oxford}
\date{\today}


\begin{document}
    \pagenumbering{gobble}
    \maketitle
    
    
    \begin{abstract}
        Magnetic confinement reactors---in particular tokamaks---offer one of the most promising options for achieving practical nuclear fusion, with the potential to provide virtually limitless, clean energy. The theoretical and numerical modeling of tokamak plasmas is simultaneously an essential component of effective reactor design, and a great research barrier. Tokamak operational conditions exhibit comparatively low Knudsen numbers. Kinetic effects, including kinetic waves and instabilities, Landau damping, bump-on-tail instabilities and more, are therefore highly influential in tokamak plasma dynamics. Purely fluid models are inherently incapable of capturing these effects, whereas the high dimensionality in purely kinetic models render them practically intractable for most relevant purposes.

        We consider a $\delta\!f$ decomposition model, with a macroscopic fluid background and microscopic kinetic correction, both fully coupled to each other. A similar manner of discretization is proposed to that used in the recent \texttt{STRUPHY} code \cite{Holderied_Possanner_Wang_2021, Holderied_2022, Li_et_al_2023} with a finite-element model for the background and a pseudo-particle/PiC model for the correction.

        The fluid background satisfies the full, non-linear, resistive, compressible, Hall MHD equations. \cite{Laakmann_Hu_Farrell_2022} introduces finite-element(-in-space) implicit timesteppers for the incompressible analogue to this system with structure-preserving (SP) properties in the ideal case, alongside parameter-robust preconditioners. We show that these timesteppers can derive from a finite-element-in-time (FET) (and finite-element-in-space) interpretation. The benefits of this reformulation are discussed, including the derivation of timesteppers that are higher order in time, and the quantifiable dissipative SP properties in the non-ideal, resistive case.
        
        We discuss possible options for extending this FET approach to timesteppers for the compressible case.

        The kinetic corrections satisfy linearized Boltzmann equations. Using a Lénard--Bernstein collision operator, these take Fokker--Planck-like forms \cite{Fokker_1914, Planck_1917} wherein pseudo-particles in the numerical model obey the neoclassical transport equations, with particle-independent Brownian drift terms. This offers a rigorous methodology for incorporating collisions into the particle transport model, without coupling the equations of motions for each particle.
        
        Works by Chen, Chacón et al. \cite{Chen_Chacón_Barnes_2011, Chacón_Chen_Barnes_2013, Chen_Chacón_2014, Chen_Chacón_2015} have developed structure-preserving particle pushers for neoclassical transport in the Vlasov equations, derived from Crank--Nicolson integrators. We show these too can can derive from a FET interpretation, similarly offering potential extensions to higher-order-in-time particle pushers. The FET formulation is used also to consider how the stochastic drift terms can be incorporated into the pushers. Stochastic gyrokinetic expansions are also discussed.

        Different options for the numerical implementation of these schemes are considered.

        Due to the efficacy of FET in the development of SP timesteppers for both the fluid and kinetic component, we hope this approach will prove effective in the future for developing SP timesteppers for the full hybrid model. We hope this will give us the opportunity to incorporate previously inaccessible kinetic effects into the highly effective, modern, finite-element MHD models.
    \end{abstract}
    
    
    \newpage
    \tableofcontents
    
    
    \newpage
    \pagenumbering{arabic}
    %\linenumbers\renewcommand\thelinenumber{\color{black!50}\arabic{linenumber}}
            \input{0 - introduction/main.tex}
        \part{Research}
            \input{1 - low-noise PiC models/main.tex}
            \input{2 - kinetic component/main.tex}
            \input{3 - fluid component/main.tex}
            \input{4 - numerical implementation/main.tex}
        \part{Project Overview}
            \input{5 - research plan/main.tex}
            \input{6 - summary/main.tex}
    
    
    %\section{}
    \newpage
    \pagenumbering{gobble}
        \printbibliography


    \newpage
    \pagenumbering{roman}
    \appendix
        \part{Appendices}
            \input{8 - Hilbert complexes/main.tex}
            \input{9 - weak conservation proofs/main.tex}
\end{document}

    
    
    %\section{}
    \newpage
    \pagenumbering{gobble}
        \printbibliography


    \newpage
    \pagenumbering{roman}
    \appendix
        \part{Appendices}
            \documentclass[12pt, a4paper]{report}

\input{template/main.tex}

\title{\BA{Title in Progress...}}
\author{Boris Andrews}
\affil{Mathematical Institute, University of Oxford}
\date{\today}


\begin{document}
    \pagenumbering{gobble}
    \maketitle
    
    
    \begin{abstract}
        Magnetic confinement reactors---in particular tokamaks---offer one of the most promising options for achieving practical nuclear fusion, with the potential to provide virtually limitless, clean energy. The theoretical and numerical modeling of tokamak plasmas is simultaneously an essential component of effective reactor design, and a great research barrier. Tokamak operational conditions exhibit comparatively low Knudsen numbers. Kinetic effects, including kinetic waves and instabilities, Landau damping, bump-on-tail instabilities and more, are therefore highly influential in tokamak plasma dynamics. Purely fluid models are inherently incapable of capturing these effects, whereas the high dimensionality in purely kinetic models render them practically intractable for most relevant purposes.

        We consider a $\delta\!f$ decomposition model, with a macroscopic fluid background and microscopic kinetic correction, both fully coupled to each other. A similar manner of discretization is proposed to that used in the recent \texttt{STRUPHY} code \cite{Holderied_Possanner_Wang_2021, Holderied_2022, Li_et_al_2023} with a finite-element model for the background and a pseudo-particle/PiC model for the correction.

        The fluid background satisfies the full, non-linear, resistive, compressible, Hall MHD equations. \cite{Laakmann_Hu_Farrell_2022} introduces finite-element(-in-space) implicit timesteppers for the incompressible analogue to this system with structure-preserving (SP) properties in the ideal case, alongside parameter-robust preconditioners. We show that these timesteppers can derive from a finite-element-in-time (FET) (and finite-element-in-space) interpretation. The benefits of this reformulation are discussed, including the derivation of timesteppers that are higher order in time, and the quantifiable dissipative SP properties in the non-ideal, resistive case.
        
        We discuss possible options for extending this FET approach to timesteppers for the compressible case.

        The kinetic corrections satisfy linearized Boltzmann equations. Using a Lénard--Bernstein collision operator, these take Fokker--Planck-like forms \cite{Fokker_1914, Planck_1917} wherein pseudo-particles in the numerical model obey the neoclassical transport equations, with particle-independent Brownian drift terms. This offers a rigorous methodology for incorporating collisions into the particle transport model, without coupling the equations of motions for each particle.
        
        Works by Chen, Chacón et al. \cite{Chen_Chacón_Barnes_2011, Chacón_Chen_Barnes_2013, Chen_Chacón_2014, Chen_Chacón_2015} have developed structure-preserving particle pushers for neoclassical transport in the Vlasov equations, derived from Crank--Nicolson integrators. We show these too can can derive from a FET interpretation, similarly offering potential extensions to higher-order-in-time particle pushers. The FET formulation is used also to consider how the stochastic drift terms can be incorporated into the pushers. Stochastic gyrokinetic expansions are also discussed.

        Different options for the numerical implementation of these schemes are considered.

        Due to the efficacy of FET in the development of SP timesteppers for both the fluid and kinetic component, we hope this approach will prove effective in the future for developing SP timesteppers for the full hybrid model. We hope this will give us the opportunity to incorporate previously inaccessible kinetic effects into the highly effective, modern, finite-element MHD models.
    \end{abstract}
    
    
    \newpage
    \tableofcontents
    
    
    \newpage
    \pagenumbering{arabic}
    %\linenumbers\renewcommand\thelinenumber{\color{black!50}\arabic{linenumber}}
            \input{0 - introduction/main.tex}
        \part{Research}
            \input{1 - low-noise PiC models/main.tex}
            \input{2 - kinetic component/main.tex}
            \input{3 - fluid component/main.tex}
            \input{4 - numerical implementation/main.tex}
        \part{Project Overview}
            \input{5 - research plan/main.tex}
            \input{6 - summary/main.tex}
    
    
    %\section{}
    \newpage
    \pagenumbering{gobble}
        \printbibliography


    \newpage
    \pagenumbering{roman}
    \appendix
        \part{Appendices}
            \input{8 - Hilbert complexes/main.tex}
            \input{9 - weak conservation proofs/main.tex}
\end{document}

            \documentclass[12pt, a4paper]{report}

\input{template/main.tex}

\title{\BA{Title in Progress...}}
\author{Boris Andrews}
\affil{Mathematical Institute, University of Oxford}
\date{\today}


\begin{document}
    \pagenumbering{gobble}
    \maketitle
    
    
    \begin{abstract}
        Magnetic confinement reactors---in particular tokamaks---offer one of the most promising options for achieving practical nuclear fusion, with the potential to provide virtually limitless, clean energy. The theoretical and numerical modeling of tokamak plasmas is simultaneously an essential component of effective reactor design, and a great research barrier. Tokamak operational conditions exhibit comparatively low Knudsen numbers. Kinetic effects, including kinetic waves and instabilities, Landau damping, bump-on-tail instabilities and more, are therefore highly influential in tokamak plasma dynamics. Purely fluid models are inherently incapable of capturing these effects, whereas the high dimensionality in purely kinetic models render them practically intractable for most relevant purposes.

        We consider a $\delta\!f$ decomposition model, with a macroscopic fluid background and microscopic kinetic correction, both fully coupled to each other. A similar manner of discretization is proposed to that used in the recent \texttt{STRUPHY} code \cite{Holderied_Possanner_Wang_2021, Holderied_2022, Li_et_al_2023} with a finite-element model for the background and a pseudo-particle/PiC model for the correction.

        The fluid background satisfies the full, non-linear, resistive, compressible, Hall MHD equations. \cite{Laakmann_Hu_Farrell_2022} introduces finite-element(-in-space) implicit timesteppers for the incompressible analogue to this system with structure-preserving (SP) properties in the ideal case, alongside parameter-robust preconditioners. We show that these timesteppers can derive from a finite-element-in-time (FET) (and finite-element-in-space) interpretation. The benefits of this reformulation are discussed, including the derivation of timesteppers that are higher order in time, and the quantifiable dissipative SP properties in the non-ideal, resistive case.
        
        We discuss possible options for extending this FET approach to timesteppers for the compressible case.

        The kinetic corrections satisfy linearized Boltzmann equations. Using a Lénard--Bernstein collision operator, these take Fokker--Planck-like forms \cite{Fokker_1914, Planck_1917} wherein pseudo-particles in the numerical model obey the neoclassical transport equations, with particle-independent Brownian drift terms. This offers a rigorous methodology for incorporating collisions into the particle transport model, without coupling the equations of motions for each particle.
        
        Works by Chen, Chacón et al. \cite{Chen_Chacón_Barnes_2011, Chacón_Chen_Barnes_2013, Chen_Chacón_2014, Chen_Chacón_2015} have developed structure-preserving particle pushers for neoclassical transport in the Vlasov equations, derived from Crank--Nicolson integrators. We show these too can can derive from a FET interpretation, similarly offering potential extensions to higher-order-in-time particle pushers. The FET formulation is used also to consider how the stochastic drift terms can be incorporated into the pushers. Stochastic gyrokinetic expansions are also discussed.

        Different options for the numerical implementation of these schemes are considered.

        Due to the efficacy of FET in the development of SP timesteppers for both the fluid and kinetic component, we hope this approach will prove effective in the future for developing SP timesteppers for the full hybrid model. We hope this will give us the opportunity to incorporate previously inaccessible kinetic effects into the highly effective, modern, finite-element MHD models.
    \end{abstract}
    
    
    \newpage
    \tableofcontents
    
    
    \newpage
    \pagenumbering{arabic}
    %\linenumbers\renewcommand\thelinenumber{\color{black!50}\arabic{linenumber}}
            \input{0 - introduction/main.tex}
        \part{Research}
            \input{1 - low-noise PiC models/main.tex}
            \input{2 - kinetic component/main.tex}
            \input{3 - fluid component/main.tex}
            \input{4 - numerical implementation/main.tex}
        \part{Project Overview}
            \input{5 - research plan/main.tex}
            \input{6 - summary/main.tex}
    
    
    %\section{}
    \newpage
    \pagenumbering{gobble}
        \printbibliography


    \newpage
    \pagenumbering{roman}
    \appendix
        \part{Appendices}
            \input{8 - Hilbert complexes/main.tex}
            \input{9 - weak conservation proofs/main.tex}
\end{document}

\end{document}

\end{document}

    \documentclass[12pt, a4paper]{report}

\documentclass[12pt, a4paper]{report}

\documentclass[12pt, a4paper]{report}

\input{template/main.tex}

\title{\BA{Title in Progress...}}
\author{Boris Andrews}
\affil{Mathematical Institute, University of Oxford}
\date{\today}


\begin{document}
    \pagenumbering{gobble}
    \maketitle
    
    
    \begin{abstract}
        Magnetic confinement reactors---in particular tokamaks---offer one of the most promising options for achieving practical nuclear fusion, with the potential to provide virtually limitless, clean energy. The theoretical and numerical modeling of tokamak plasmas is simultaneously an essential component of effective reactor design, and a great research barrier. Tokamak operational conditions exhibit comparatively low Knudsen numbers. Kinetic effects, including kinetic waves and instabilities, Landau damping, bump-on-tail instabilities and more, are therefore highly influential in tokamak plasma dynamics. Purely fluid models are inherently incapable of capturing these effects, whereas the high dimensionality in purely kinetic models render them practically intractable for most relevant purposes.

        We consider a $\delta\!f$ decomposition model, with a macroscopic fluid background and microscopic kinetic correction, both fully coupled to each other. A similar manner of discretization is proposed to that used in the recent \texttt{STRUPHY} code \cite{Holderied_Possanner_Wang_2021, Holderied_2022, Li_et_al_2023} with a finite-element model for the background and a pseudo-particle/PiC model for the correction.

        The fluid background satisfies the full, non-linear, resistive, compressible, Hall MHD equations. \cite{Laakmann_Hu_Farrell_2022} introduces finite-element(-in-space) implicit timesteppers for the incompressible analogue to this system with structure-preserving (SP) properties in the ideal case, alongside parameter-robust preconditioners. We show that these timesteppers can derive from a finite-element-in-time (FET) (and finite-element-in-space) interpretation. The benefits of this reformulation are discussed, including the derivation of timesteppers that are higher order in time, and the quantifiable dissipative SP properties in the non-ideal, resistive case.
        
        We discuss possible options for extending this FET approach to timesteppers for the compressible case.

        The kinetic corrections satisfy linearized Boltzmann equations. Using a Lénard--Bernstein collision operator, these take Fokker--Planck-like forms \cite{Fokker_1914, Planck_1917} wherein pseudo-particles in the numerical model obey the neoclassical transport equations, with particle-independent Brownian drift terms. This offers a rigorous methodology for incorporating collisions into the particle transport model, without coupling the equations of motions for each particle.
        
        Works by Chen, Chacón et al. \cite{Chen_Chacón_Barnes_2011, Chacón_Chen_Barnes_2013, Chen_Chacón_2014, Chen_Chacón_2015} have developed structure-preserving particle pushers for neoclassical transport in the Vlasov equations, derived from Crank--Nicolson integrators. We show these too can can derive from a FET interpretation, similarly offering potential extensions to higher-order-in-time particle pushers. The FET formulation is used also to consider how the stochastic drift terms can be incorporated into the pushers. Stochastic gyrokinetic expansions are also discussed.

        Different options for the numerical implementation of these schemes are considered.

        Due to the efficacy of FET in the development of SP timesteppers for both the fluid and kinetic component, we hope this approach will prove effective in the future for developing SP timesteppers for the full hybrid model. We hope this will give us the opportunity to incorporate previously inaccessible kinetic effects into the highly effective, modern, finite-element MHD models.
    \end{abstract}
    
    
    \newpage
    \tableofcontents
    
    
    \newpage
    \pagenumbering{arabic}
    %\linenumbers\renewcommand\thelinenumber{\color{black!50}\arabic{linenumber}}
            \input{0 - introduction/main.tex}
        \part{Research}
            \input{1 - low-noise PiC models/main.tex}
            \input{2 - kinetic component/main.tex}
            \input{3 - fluid component/main.tex}
            \input{4 - numerical implementation/main.tex}
        \part{Project Overview}
            \input{5 - research plan/main.tex}
            \input{6 - summary/main.tex}
    
    
    %\section{}
    \newpage
    \pagenumbering{gobble}
        \printbibliography


    \newpage
    \pagenumbering{roman}
    \appendix
        \part{Appendices}
            \input{8 - Hilbert complexes/main.tex}
            \input{9 - weak conservation proofs/main.tex}
\end{document}


\title{\BA{Title in Progress...}}
\author{Boris Andrews}
\affil{Mathematical Institute, University of Oxford}
\date{\today}


\begin{document}
    \pagenumbering{gobble}
    \maketitle
    
    
    \begin{abstract}
        Magnetic confinement reactors---in particular tokamaks---offer one of the most promising options for achieving practical nuclear fusion, with the potential to provide virtually limitless, clean energy. The theoretical and numerical modeling of tokamak plasmas is simultaneously an essential component of effective reactor design, and a great research barrier. Tokamak operational conditions exhibit comparatively low Knudsen numbers. Kinetic effects, including kinetic waves and instabilities, Landau damping, bump-on-tail instabilities and more, are therefore highly influential in tokamak plasma dynamics. Purely fluid models are inherently incapable of capturing these effects, whereas the high dimensionality in purely kinetic models render them practically intractable for most relevant purposes.

        We consider a $\delta\!f$ decomposition model, with a macroscopic fluid background and microscopic kinetic correction, both fully coupled to each other. A similar manner of discretization is proposed to that used in the recent \texttt{STRUPHY} code \cite{Holderied_Possanner_Wang_2021, Holderied_2022, Li_et_al_2023} with a finite-element model for the background and a pseudo-particle/PiC model for the correction.

        The fluid background satisfies the full, non-linear, resistive, compressible, Hall MHD equations. \cite{Laakmann_Hu_Farrell_2022} introduces finite-element(-in-space) implicit timesteppers for the incompressible analogue to this system with structure-preserving (SP) properties in the ideal case, alongside parameter-robust preconditioners. We show that these timesteppers can derive from a finite-element-in-time (FET) (and finite-element-in-space) interpretation. The benefits of this reformulation are discussed, including the derivation of timesteppers that are higher order in time, and the quantifiable dissipative SP properties in the non-ideal, resistive case.
        
        We discuss possible options for extending this FET approach to timesteppers for the compressible case.

        The kinetic corrections satisfy linearized Boltzmann equations. Using a Lénard--Bernstein collision operator, these take Fokker--Planck-like forms \cite{Fokker_1914, Planck_1917} wherein pseudo-particles in the numerical model obey the neoclassical transport equations, with particle-independent Brownian drift terms. This offers a rigorous methodology for incorporating collisions into the particle transport model, without coupling the equations of motions for each particle.
        
        Works by Chen, Chacón et al. \cite{Chen_Chacón_Barnes_2011, Chacón_Chen_Barnes_2013, Chen_Chacón_2014, Chen_Chacón_2015} have developed structure-preserving particle pushers for neoclassical transport in the Vlasov equations, derived from Crank--Nicolson integrators. We show these too can can derive from a FET interpretation, similarly offering potential extensions to higher-order-in-time particle pushers. The FET formulation is used also to consider how the stochastic drift terms can be incorporated into the pushers. Stochastic gyrokinetic expansions are also discussed.

        Different options for the numerical implementation of these schemes are considered.

        Due to the efficacy of FET in the development of SP timesteppers for both the fluid and kinetic component, we hope this approach will prove effective in the future for developing SP timesteppers for the full hybrid model. We hope this will give us the opportunity to incorporate previously inaccessible kinetic effects into the highly effective, modern, finite-element MHD models.
    \end{abstract}
    
    
    \newpage
    \tableofcontents
    
    
    \newpage
    \pagenumbering{arabic}
    %\linenumbers\renewcommand\thelinenumber{\color{black!50}\arabic{linenumber}}
            \documentclass[12pt, a4paper]{report}

\input{template/main.tex}

\title{\BA{Title in Progress...}}
\author{Boris Andrews}
\affil{Mathematical Institute, University of Oxford}
\date{\today}


\begin{document}
    \pagenumbering{gobble}
    \maketitle
    
    
    \begin{abstract}
        Magnetic confinement reactors---in particular tokamaks---offer one of the most promising options for achieving practical nuclear fusion, with the potential to provide virtually limitless, clean energy. The theoretical and numerical modeling of tokamak plasmas is simultaneously an essential component of effective reactor design, and a great research barrier. Tokamak operational conditions exhibit comparatively low Knudsen numbers. Kinetic effects, including kinetic waves and instabilities, Landau damping, bump-on-tail instabilities and more, are therefore highly influential in tokamak plasma dynamics. Purely fluid models are inherently incapable of capturing these effects, whereas the high dimensionality in purely kinetic models render them practically intractable for most relevant purposes.

        We consider a $\delta\!f$ decomposition model, with a macroscopic fluid background and microscopic kinetic correction, both fully coupled to each other. A similar manner of discretization is proposed to that used in the recent \texttt{STRUPHY} code \cite{Holderied_Possanner_Wang_2021, Holderied_2022, Li_et_al_2023} with a finite-element model for the background and a pseudo-particle/PiC model for the correction.

        The fluid background satisfies the full, non-linear, resistive, compressible, Hall MHD equations. \cite{Laakmann_Hu_Farrell_2022} introduces finite-element(-in-space) implicit timesteppers for the incompressible analogue to this system with structure-preserving (SP) properties in the ideal case, alongside parameter-robust preconditioners. We show that these timesteppers can derive from a finite-element-in-time (FET) (and finite-element-in-space) interpretation. The benefits of this reformulation are discussed, including the derivation of timesteppers that are higher order in time, and the quantifiable dissipative SP properties in the non-ideal, resistive case.
        
        We discuss possible options for extending this FET approach to timesteppers for the compressible case.

        The kinetic corrections satisfy linearized Boltzmann equations. Using a Lénard--Bernstein collision operator, these take Fokker--Planck-like forms \cite{Fokker_1914, Planck_1917} wherein pseudo-particles in the numerical model obey the neoclassical transport equations, with particle-independent Brownian drift terms. This offers a rigorous methodology for incorporating collisions into the particle transport model, without coupling the equations of motions for each particle.
        
        Works by Chen, Chacón et al. \cite{Chen_Chacón_Barnes_2011, Chacón_Chen_Barnes_2013, Chen_Chacón_2014, Chen_Chacón_2015} have developed structure-preserving particle pushers for neoclassical transport in the Vlasov equations, derived from Crank--Nicolson integrators. We show these too can can derive from a FET interpretation, similarly offering potential extensions to higher-order-in-time particle pushers. The FET formulation is used also to consider how the stochastic drift terms can be incorporated into the pushers. Stochastic gyrokinetic expansions are also discussed.

        Different options for the numerical implementation of these schemes are considered.

        Due to the efficacy of FET in the development of SP timesteppers for both the fluid and kinetic component, we hope this approach will prove effective in the future for developing SP timesteppers for the full hybrid model. We hope this will give us the opportunity to incorporate previously inaccessible kinetic effects into the highly effective, modern, finite-element MHD models.
    \end{abstract}
    
    
    \newpage
    \tableofcontents
    
    
    \newpage
    \pagenumbering{arabic}
    %\linenumbers\renewcommand\thelinenumber{\color{black!50}\arabic{linenumber}}
            \input{0 - introduction/main.tex}
        \part{Research}
            \input{1 - low-noise PiC models/main.tex}
            \input{2 - kinetic component/main.tex}
            \input{3 - fluid component/main.tex}
            \input{4 - numerical implementation/main.tex}
        \part{Project Overview}
            \input{5 - research plan/main.tex}
            \input{6 - summary/main.tex}
    
    
    %\section{}
    \newpage
    \pagenumbering{gobble}
        \printbibliography


    \newpage
    \pagenumbering{roman}
    \appendix
        \part{Appendices}
            \input{8 - Hilbert complexes/main.tex}
            \input{9 - weak conservation proofs/main.tex}
\end{document}

        \part{Research}
            \documentclass[12pt, a4paper]{report}

\input{template/main.tex}

\title{\BA{Title in Progress...}}
\author{Boris Andrews}
\affil{Mathematical Institute, University of Oxford}
\date{\today}


\begin{document}
    \pagenumbering{gobble}
    \maketitle
    
    
    \begin{abstract}
        Magnetic confinement reactors---in particular tokamaks---offer one of the most promising options for achieving practical nuclear fusion, with the potential to provide virtually limitless, clean energy. The theoretical and numerical modeling of tokamak plasmas is simultaneously an essential component of effective reactor design, and a great research barrier. Tokamak operational conditions exhibit comparatively low Knudsen numbers. Kinetic effects, including kinetic waves and instabilities, Landau damping, bump-on-tail instabilities and more, are therefore highly influential in tokamak plasma dynamics. Purely fluid models are inherently incapable of capturing these effects, whereas the high dimensionality in purely kinetic models render them practically intractable for most relevant purposes.

        We consider a $\delta\!f$ decomposition model, with a macroscopic fluid background and microscopic kinetic correction, both fully coupled to each other. A similar manner of discretization is proposed to that used in the recent \texttt{STRUPHY} code \cite{Holderied_Possanner_Wang_2021, Holderied_2022, Li_et_al_2023} with a finite-element model for the background and a pseudo-particle/PiC model for the correction.

        The fluid background satisfies the full, non-linear, resistive, compressible, Hall MHD equations. \cite{Laakmann_Hu_Farrell_2022} introduces finite-element(-in-space) implicit timesteppers for the incompressible analogue to this system with structure-preserving (SP) properties in the ideal case, alongside parameter-robust preconditioners. We show that these timesteppers can derive from a finite-element-in-time (FET) (and finite-element-in-space) interpretation. The benefits of this reformulation are discussed, including the derivation of timesteppers that are higher order in time, and the quantifiable dissipative SP properties in the non-ideal, resistive case.
        
        We discuss possible options for extending this FET approach to timesteppers for the compressible case.

        The kinetic corrections satisfy linearized Boltzmann equations. Using a Lénard--Bernstein collision operator, these take Fokker--Planck-like forms \cite{Fokker_1914, Planck_1917} wherein pseudo-particles in the numerical model obey the neoclassical transport equations, with particle-independent Brownian drift terms. This offers a rigorous methodology for incorporating collisions into the particle transport model, without coupling the equations of motions for each particle.
        
        Works by Chen, Chacón et al. \cite{Chen_Chacón_Barnes_2011, Chacón_Chen_Barnes_2013, Chen_Chacón_2014, Chen_Chacón_2015} have developed structure-preserving particle pushers for neoclassical transport in the Vlasov equations, derived from Crank--Nicolson integrators. We show these too can can derive from a FET interpretation, similarly offering potential extensions to higher-order-in-time particle pushers. The FET formulation is used also to consider how the stochastic drift terms can be incorporated into the pushers. Stochastic gyrokinetic expansions are also discussed.

        Different options for the numerical implementation of these schemes are considered.

        Due to the efficacy of FET in the development of SP timesteppers for both the fluid and kinetic component, we hope this approach will prove effective in the future for developing SP timesteppers for the full hybrid model. We hope this will give us the opportunity to incorporate previously inaccessible kinetic effects into the highly effective, modern, finite-element MHD models.
    \end{abstract}
    
    
    \newpage
    \tableofcontents
    
    
    \newpage
    \pagenumbering{arabic}
    %\linenumbers\renewcommand\thelinenumber{\color{black!50}\arabic{linenumber}}
            \input{0 - introduction/main.tex}
        \part{Research}
            \input{1 - low-noise PiC models/main.tex}
            \input{2 - kinetic component/main.tex}
            \input{3 - fluid component/main.tex}
            \input{4 - numerical implementation/main.tex}
        \part{Project Overview}
            \input{5 - research plan/main.tex}
            \input{6 - summary/main.tex}
    
    
    %\section{}
    \newpage
    \pagenumbering{gobble}
        \printbibliography


    \newpage
    \pagenumbering{roman}
    \appendix
        \part{Appendices}
            \input{8 - Hilbert complexes/main.tex}
            \input{9 - weak conservation proofs/main.tex}
\end{document}

            \documentclass[12pt, a4paper]{report}

\input{template/main.tex}

\title{\BA{Title in Progress...}}
\author{Boris Andrews}
\affil{Mathematical Institute, University of Oxford}
\date{\today}


\begin{document}
    \pagenumbering{gobble}
    \maketitle
    
    
    \begin{abstract}
        Magnetic confinement reactors---in particular tokamaks---offer one of the most promising options for achieving practical nuclear fusion, with the potential to provide virtually limitless, clean energy. The theoretical and numerical modeling of tokamak plasmas is simultaneously an essential component of effective reactor design, and a great research barrier. Tokamak operational conditions exhibit comparatively low Knudsen numbers. Kinetic effects, including kinetic waves and instabilities, Landau damping, bump-on-tail instabilities and more, are therefore highly influential in tokamak plasma dynamics. Purely fluid models are inherently incapable of capturing these effects, whereas the high dimensionality in purely kinetic models render them practically intractable for most relevant purposes.

        We consider a $\delta\!f$ decomposition model, with a macroscopic fluid background and microscopic kinetic correction, both fully coupled to each other. A similar manner of discretization is proposed to that used in the recent \texttt{STRUPHY} code \cite{Holderied_Possanner_Wang_2021, Holderied_2022, Li_et_al_2023} with a finite-element model for the background and a pseudo-particle/PiC model for the correction.

        The fluid background satisfies the full, non-linear, resistive, compressible, Hall MHD equations. \cite{Laakmann_Hu_Farrell_2022} introduces finite-element(-in-space) implicit timesteppers for the incompressible analogue to this system with structure-preserving (SP) properties in the ideal case, alongside parameter-robust preconditioners. We show that these timesteppers can derive from a finite-element-in-time (FET) (and finite-element-in-space) interpretation. The benefits of this reformulation are discussed, including the derivation of timesteppers that are higher order in time, and the quantifiable dissipative SP properties in the non-ideal, resistive case.
        
        We discuss possible options for extending this FET approach to timesteppers for the compressible case.

        The kinetic corrections satisfy linearized Boltzmann equations. Using a Lénard--Bernstein collision operator, these take Fokker--Planck-like forms \cite{Fokker_1914, Planck_1917} wherein pseudo-particles in the numerical model obey the neoclassical transport equations, with particle-independent Brownian drift terms. This offers a rigorous methodology for incorporating collisions into the particle transport model, without coupling the equations of motions for each particle.
        
        Works by Chen, Chacón et al. \cite{Chen_Chacón_Barnes_2011, Chacón_Chen_Barnes_2013, Chen_Chacón_2014, Chen_Chacón_2015} have developed structure-preserving particle pushers for neoclassical transport in the Vlasov equations, derived from Crank--Nicolson integrators. We show these too can can derive from a FET interpretation, similarly offering potential extensions to higher-order-in-time particle pushers. The FET formulation is used also to consider how the stochastic drift terms can be incorporated into the pushers. Stochastic gyrokinetic expansions are also discussed.

        Different options for the numerical implementation of these schemes are considered.

        Due to the efficacy of FET in the development of SP timesteppers for both the fluid and kinetic component, we hope this approach will prove effective in the future for developing SP timesteppers for the full hybrid model. We hope this will give us the opportunity to incorporate previously inaccessible kinetic effects into the highly effective, modern, finite-element MHD models.
    \end{abstract}
    
    
    \newpage
    \tableofcontents
    
    
    \newpage
    \pagenumbering{arabic}
    %\linenumbers\renewcommand\thelinenumber{\color{black!50}\arabic{linenumber}}
            \input{0 - introduction/main.tex}
        \part{Research}
            \input{1 - low-noise PiC models/main.tex}
            \input{2 - kinetic component/main.tex}
            \input{3 - fluid component/main.tex}
            \input{4 - numerical implementation/main.tex}
        \part{Project Overview}
            \input{5 - research plan/main.tex}
            \input{6 - summary/main.tex}
    
    
    %\section{}
    \newpage
    \pagenumbering{gobble}
        \printbibliography


    \newpage
    \pagenumbering{roman}
    \appendix
        \part{Appendices}
            \input{8 - Hilbert complexes/main.tex}
            \input{9 - weak conservation proofs/main.tex}
\end{document}

            \documentclass[12pt, a4paper]{report}

\input{template/main.tex}

\title{\BA{Title in Progress...}}
\author{Boris Andrews}
\affil{Mathematical Institute, University of Oxford}
\date{\today}


\begin{document}
    \pagenumbering{gobble}
    \maketitle
    
    
    \begin{abstract}
        Magnetic confinement reactors---in particular tokamaks---offer one of the most promising options for achieving practical nuclear fusion, with the potential to provide virtually limitless, clean energy. The theoretical and numerical modeling of tokamak plasmas is simultaneously an essential component of effective reactor design, and a great research barrier. Tokamak operational conditions exhibit comparatively low Knudsen numbers. Kinetic effects, including kinetic waves and instabilities, Landau damping, bump-on-tail instabilities and more, are therefore highly influential in tokamak plasma dynamics. Purely fluid models are inherently incapable of capturing these effects, whereas the high dimensionality in purely kinetic models render them practically intractable for most relevant purposes.

        We consider a $\delta\!f$ decomposition model, with a macroscopic fluid background and microscopic kinetic correction, both fully coupled to each other. A similar manner of discretization is proposed to that used in the recent \texttt{STRUPHY} code \cite{Holderied_Possanner_Wang_2021, Holderied_2022, Li_et_al_2023} with a finite-element model for the background and a pseudo-particle/PiC model for the correction.

        The fluid background satisfies the full, non-linear, resistive, compressible, Hall MHD equations. \cite{Laakmann_Hu_Farrell_2022} introduces finite-element(-in-space) implicit timesteppers for the incompressible analogue to this system with structure-preserving (SP) properties in the ideal case, alongside parameter-robust preconditioners. We show that these timesteppers can derive from a finite-element-in-time (FET) (and finite-element-in-space) interpretation. The benefits of this reformulation are discussed, including the derivation of timesteppers that are higher order in time, and the quantifiable dissipative SP properties in the non-ideal, resistive case.
        
        We discuss possible options for extending this FET approach to timesteppers for the compressible case.

        The kinetic corrections satisfy linearized Boltzmann equations. Using a Lénard--Bernstein collision operator, these take Fokker--Planck-like forms \cite{Fokker_1914, Planck_1917} wherein pseudo-particles in the numerical model obey the neoclassical transport equations, with particle-independent Brownian drift terms. This offers a rigorous methodology for incorporating collisions into the particle transport model, without coupling the equations of motions for each particle.
        
        Works by Chen, Chacón et al. \cite{Chen_Chacón_Barnes_2011, Chacón_Chen_Barnes_2013, Chen_Chacón_2014, Chen_Chacón_2015} have developed structure-preserving particle pushers for neoclassical transport in the Vlasov equations, derived from Crank--Nicolson integrators. We show these too can can derive from a FET interpretation, similarly offering potential extensions to higher-order-in-time particle pushers. The FET formulation is used also to consider how the stochastic drift terms can be incorporated into the pushers. Stochastic gyrokinetic expansions are also discussed.

        Different options for the numerical implementation of these schemes are considered.

        Due to the efficacy of FET in the development of SP timesteppers for both the fluid and kinetic component, we hope this approach will prove effective in the future for developing SP timesteppers for the full hybrid model. We hope this will give us the opportunity to incorporate previously inaccessible kinetic effects into the highly effective, modern, finite-element MHD models.
    \end{abstract}
    
    
    \newpage
    \tableofcontents
    
    
    \newpage
    \pagenumbering{arabic}
    %\linenumbers\renewcommand\thelinenumber{\color{black!50}\arabic{linenumber}}
            \input{0 - introduction/main.tex}
        \part{Research}
            \input{1 - low-noise PiC models/main.tex}
            \input{2 - kinetic component/main.tex}
            \input{3 - fluid component/main.tex}
            \input{4 - numerical implementation/main.tex}
        \part{Project Overview}
            \input{5 - research plan/main.tex}
            \input{6 - summary/main.tex}
    
    
    %\section{}
    \newpage
    \pagenumbering{gobble}
        \printbibliography


    \newpage
    \pagenumbering{roman}
    \appendix
        \part{Appendices}
            \input{8 - Hilbert complexes/main.tex}
            \input{9 - weak conservation proofs/main.tex}
\end{document}

            \documentclass[12pt, a4paper]{report}

\input{template/main.tex}

\title{\BA{Title in Progress...}}
\author{Boris Andrews}
\affil{Mathematical Institute, University of Oxford}
\date{\today}


\begin{document}
    \pagenumbering{gobble}
    \maketitle
    
    
    \begin{abstract}
        Magnetic confinement reactors---in particular tokamaks---offer one of the most promising options for achieving practical nuclear fusion, with the potential to provide virtually limitless, clean energy. The theoretical and numerical modeling of tokamak plasmas is simultaneously an essential component of effective reactor design, and a great research barrier. Tokamak operational conditions exhibit comparatively low Knudsen numbers. Kinetic effects, including kinetic waves and instabilities, Landau damping, bump-on-tail instabilities and more, are therefore highly influential in tokamak plasma dynamics. Purely fluid models are inherently incapable of capturing these effects, whereas the high dimensionality in purely kinetic models render them practically intractable for most relevant purposes.

        We consider a $\delta\!f$ decomposition model, with a macroscopic fluid background and microscopic kinetic correction, both fully coupled to each other. A similar manner of discretization is proposed to that used in the recent \texttt{STRUPHY} code \cite{Holderied_Possanner_Wang_2021, Holderied_2022, Li_et_al_2023} with a finite-element model for the background and a pseudo-particle/PiC model for the correction.

        The fluid background satisfies the full, non-linear, resistive, compressible, Hall MHD equations. \cite{Laakmann_Hu_Farrell_2022} introduces finite-element(-in-space) implicit timesteppers for the incompressible analogue to this system with structure-preserving (SP) properties in the ideal case, alongside parameter-robust preconditioners. We show that these timesteppers can derive from a finite-element-in-time (FET) (and finite-element-in-space) interpretation. The benefits of this reformulation are discussed, including the derivation of timesteppers that are higher order in time, and the quantifiable dissipative SP properties in the non-ideal, resistive case.
        
        We discuss possible options for extending this FET approach to timesteppers for the compressible case.

        The kinetic corrections satisfy linearized Boltzmann equations. Using a Lénard--Bernstein collision operator, these take Fokker--Planck-like forms \cite{Fokker_1914, Planck_1917} wherein pseudo-particles in the numerical model obey the neoclassical transport equations, with particle-independent Brownian drift terms. This offers a rigorous methodology for incorporating collisions into the particle transport model, without coupling the equations of motions for each particle.
        
        Works by Chen, Chacón et al. \cite{Chen_Chacón_Barnes_2011, Chacón_Chen_Barnes_2013, Chen_Chacón_2014, Chen_Chacón_2015} have developed structure-preserving particle pushers for neoclassical transport in the Vlasov equations, derived from Crank--Nicolson integrators. We show these too can can derive from a FET interpretation, similarly offering potential extensions to higher-order-in-time particle pushers. The FET formulation is used also to consider how the stochastic drift terms can be incorporated into the pushers. Stochastic gyrokinetic expansions are also discussed.

        Different options for the numerical implementation of these schemes are considered.

        Due to the efficacy of FET in the development of SP timesteppers for both the fluid and kinetic component, we hope this approach will prove effective in the future for developing SP timesteppers for the full hybrid model. We hope this will give us the opportunity to incorporate previously inaccessible kinetic effects into the highly effective, modern, finite-element MHD models.
    \end{abstract}
    
    
    \newpage
    \tableofcontents
    
    
    \newpage
    \pagenumbering{arabic}
    %\linenumbers\renewcommand\thelinenumber{\color{black!50}\arabic{linenumber}}
            \input{0 - introduction/main.tex}
        \part{Research}
            \input{1 - low-noise PiC models/main.tex}
            \input{2 - kinetic component/main.tex}
            \input{3 - fluid component/main.tex}
            \input{4 - numerical implementation/main.tex}
        \part{Project Overview}
            \input{5 - research plan/main.tex}
            \input{6 - summary/main.tex}
    
    
    %\section{}
    \newpage
    \pagenumbering{gobble}
        \printbibliography


    \newpage
    \pagenumbering{roman}
    \appendix
        \part{Appendices}
            \input{8 - Hilbert complexes/main.tex}
            \input{9 - weak conservation proofs/main.tex}
\end{document}

        \part{Project Overview}
            \documentclass[12pt, a4paper]{report}

\input{template/main.tex}

\title{\BA{Title in Progress...}}
\author{Boris Andrews}
\affil{Mathematical Institute, University of Oxford}
\date{\today}


\begin{document}
    \pagenumbering{gobble}
    \maketitle
    
    
    \begin{abstract}
        Magnetic confinement reactors---in particular tokamaks---offer one of the most promising options for achieving practical nuclear fusion, with the potential to provide virtually limitless, clean energy. The theoretical and numerical modeling of tokamak plasmas is simultaneously an essential component of effective reactor design, and a great research barrier. Tokamak operational conditions exhibit comparatively low Knudsen numbers. Kinetic effects, including kinetic waves and instabilities, Landau damping, bump-on-tail instabilities and more, are therefore highly influential in tokamak plasma dynamics. Purely fluid models are inherently incapable of capturing these effects, whereas the high dimensionality in purely kinetic models render them practically intractable for most relevant purposes.

        We consider a $\delta\!f$ decomposition model, with a macroscopic fluid background and microscopic kinetic correction, both fully coupled to each other. A similar manner of discretization is proposed to that used in the recent \texttt{STRUPHY} code \cite{Holderied_Possanner_Wang_2021, Holderied_2022, Li_et_al_2023} with a finite-element model for the background and a pseudo-particle/PiC model for the correction.

        The fluid background satisfies the full, non-linear, resistive, compressible, Hall MHD equations. \cite{Laakmann_Hu_Farrell_2022} introduces finite-element(-in-space) implicit timesteppers for the incompressible analogue to this system with structure-preserving (SP) properties in the ideal case, alongside parameter-robust preconditioners. We show that these timesteppers can derive from a finite-element-in-time (FET) (and finite-element-in-space) interpretation. The benefits of this reformulation are discussed, including the derivation of timesteppers that are higher order in time, and the quantifiable dissipative SP properties in the non-ideal, resistive case.
        
        We discuss possible options for extending this FET approach to timesteppers for the compressible case.

        The kinetic corrections satisfy linearized Boltzmann equations. Using a Lénard--Bernstein collision operator, these take Fokker--Planck-like forms \cite{Fokker_1914, Planck_1917} wherein pseudo-particles in the numerical model obey the neoclassical transport equations, with particle-independent Brownian drift terms. This offers a rigorous methodology for incorporating collisions into the particle transport model, without coupling the equations of motions for each particle.
        
        Works by Chen, Chacón et al. \cite{Chen_Chacón_Barnes_2011, Chacón_Chen_Barnes_2013, Chen_Chacón_2014, Chen_Chacón_2015} have developed structure-preserving particle pushers for neoclassical transport in the Vlasov equations, derived from Crank--Nicolson integrators. We show these too can can derive from a FET interpretation, similarly offering potential extensions to higher-order-in-time particle pushers. The FET formulation is used also to consider how the stochastic drift terms can be incorporated into the pushers. Stochastic gyrokinetic expansions are also discussed.

        Different options for the numerical implementation of these schemes are considered.

        Due to the efficacy of FET in the development of SP timesteppers for both the fluid and kinetic component, we hope this approach will prove effective in the future for developing SP timesteppers for the full hybrid model. We hope this will give us the opportunity to incorporate previously inaccessible kinetic effects into the highly effective, modern, finite-element MHD models.
    \end{abstract}
    
    
    \newpage
    \tableofcontents
    
    
    \newpage
    \pagenumbering{arabic}
    %\linenumbers\renewcommand\thelinenumber{\color{black!50}\arabic{linenumber}}
            \input{0 - introduction/main.tex}
        \part{Research}
            \input{1 - low-noise PiC models/main.tex}
            \input{2 - kinetic component/main.tex}
            \input{3 - fluid component/main.tex}
            \input{4 - numerical implementation/main.tex}
        \part{Project Overview}
            \input{5 - research plan/main.tex}
            \input{6 - summary/main.tex}
    
    
    %\section{}
    \newpage
    \pagenumbering{gobble}
        \printbibliography


    \newpage
    \pagenumbering{roman}
    \appendix
        \part{Appendices}
            \input{8 - Hilbert complexes/main.tex}
            \input{9 - weak conservation proofs/main.tex}
\end{document}

            \documentclass[12pt, a4paper]{report}

\input{template/main.tex}

\title{\BA{Title in Progress...}}
\author{Boris Andrews}
\affil{Mathematical Institute, University of Oxford}
\date{\today}


\begin{document}
    \pagenumbering{gobble}
    \maketitle
    
    
    \begin{abstract}
        Magnetic confinement reactors---in particular tokamaks---offer one of the most promising options for achieving practical nuclear fusion, with the potential to provide virtually limitless, clean energy. The theoretical and numerical modeling of tokamak plasmas is simultaneously an essential component of effective reactor design, and a great research barrier. Tokamak operational conditions exhibit comparatively low Knudsen numbers. Kinetic effects, including kinetic waves and instabilities, Landau damping, bump-on-tail instabilities and more, are therefore highly influential in tokamak plasma dynamics. Purely fluid models are inherently incapable of capturing these effects, whereas the high dimensionality in purely kinetic models render them practically intractable for most relevant purposes.

        We consider a $\delta\!f$ decomposition model, with a macroscopic fluid background and microscopic kinetic correction, both fully coupled to each other. A similar manner of discretization is proposed to that used in the recent \texttt{STRUPHY} code \cite{Holderied_Possanner_Wang_2021, Holderied_2022, Li_et_al_2023} with a finite-element model for the background and a pseudo-particle/PiC model for the correction.

        The fluid background satisfies the full, non-linear, resistive, compressible, Hall MHD equations. \cite{Laakmann_Hu_Farrell_2022} introduces finite-element(-in-space) implicit timesteppers for the incompressible analogue to this system with structure-preserving (SP) properties in the ideal case, alongside parameter-robust preconditioners. We show that these timesteppers can derive from a finite-element-in-time (FET) (and finite-element-in-space) interpretation. The benefits of this reformulation are discussed, including the derivation of timesteppers that are higher order in time, and the quantifiable dissipative SP properties in the non-ideal, resistive case.
        
        We discuss possible options for extending this FET approach to timesteppers for the compressible case.

        The kinetic corrections satisfy linearized Boltzmann equations. Using a Lénard--Bernstein collision operator, these take Fokker--Planck-like forms \cite{Fokker_1914, Planck_1917} wherein pseudo-particles in the numerical model obey the neoclassical transport equations, with particle-independent Brownian drift terms. This offers a rigorous methodology for incorporating collisions into the particle transport model, without coupling the equations of motions for each particle.
        
        Works by Chen, Chacón et al. \cite{Chen_Chacón_Barnes_2011, Chacón_Chen_Barnes_2013, Chen_Chacón_2014, Chen_Chacón_2015} have developed structure-preserving particle pushers for neoclassical transport in the Vlasov equations, derived from Crank--Nicolson integrators. We show these too can can derive from a FET interpretation, similarly offering potential extensions to higher-order-in-time particle pushers. The FET formulation is used also to consider how the stochastic drift terms can be incorporated into the pushers. Stochastic gyrokinetic expansions are also discussed.

        Different options for the numerical implementation of these schemes are considered.

        Due to the efficacy of FET in the development of SP timesteppers for both the fluid and kinetic component, we hope this approach will prove effective in the future for developing SP timesteppers for the full hybrid model. We hope this will give us the opportunity to incorporate previously inaccessible kinetic effects into the highly effective, modern, finite-element MHD models.
    \end{abstract}
    
    
    \newpage
    \tableofcontents
    
    
    \newpage
    \pagenumbering{arabic}
    %\linenumbers\renewcommand\thelinenumber{\color{black!50}\arabic{linenumber}}
            \input{0 - introduction/main.tex}
        \part{Research}
            \input{1 - low-noise PiC models/main.tex}
            \input{2 - kinetic component/main.tex}
            \input{3 - fluid component/main.tex}
            \input{4 - numerical implementation/main.tex}
        \part{Project Overview}
            \input{5 - research plan/main.tex}
            \input{6 - summary/main.tex}
    
    
    %\section{}
    \newpage
    \pagenumbering{gobble}
        \printbibliography


    \newpage
    \pagenumbering{roman}
    \appendix
        \part{Appendices}
            \input{8 - Hilbert complexes/main.tex}
            \input{9 - weak conservation proofs/main.tex}
\end{document}

    
    
    %\section{}
    \newpage
    \pagenumbering{gobble}
        \printbibliography


    \newpage
    \pagenumbering{roman}
    \appendix
        \part{Appendices}
            \documentclass[12pt, a4paper]{report}

\input{template/main.tex}

\title{\BA{Title in Progress...}}
\author{Boris Andrews}
\affil{Mathematical Institute, University of Oxford}
\date{\today}


\begin{document}
    \pagenumbering{gobble}
    \maketitle
    
    
    \begin{abstract}
        Magnetic confinement reactors---in particular tokamaks---offer one of the most promising options for achieving practical nuclear fusion, with the potential to provide virtually limitless, clean energy. The theoretical and numerical modeling of tokamak plasmas is simultaneously an essential component of effective reactor design, and a great research barrier. Tokamak operational conditions exhibit comparatively low Knudsen numbers. Kinetic effects, including kinetic waves and instabilities, Landau damping, bump-on-tail instabilities and more, are therefore highly influential in tokamak plasma dynamics. Purely fluid models are inherently incapable of capturing these effects, whereas the high dimensionality in purely kinetic models render them practically intractable for most relevant purposes.

        We consider a $\delta\!f$ decomposition model, with a macroscopic fluid background and microscopic kinetic correction, both fully coupled to each other. A similar manner of discretization is proposed to that used in the recent \texttt{STRUPHY} code \cite{Holderied_Possanner_Wang_2021, Holderied_2022, Li_et_al_2023} with a finite-element model for the background and a pseudo-particle/PiC model for the correction.

        The fluid background satisfies the full, non-linear, resistive, compressible, Hall MHD equations. \cite{Laakmann_Hu_Farrell_2022} introduces finite-element(-in-space) implicit timesteppers for the incompressible analogue to this system with structure-preserving (SP) properties in the ideal case, alongside parameter-robust preconditioners. We show that these timesteppers can derive from a finite-element-in-time (FET) (and finite-element-in-space) interpretation. The benefits of this reformulation are discussed, including the derivation of timesteppers that are higher order in time, and the quantifiable dissipative SP properties in the non-ideal, resistive case.
        
        We discuss possible options for extending this FET approach to timesteppers for the compressible case.

        The kinetic corrections satisfy linearized Boltzmann equations. Using a Lénard--Bernstein collision operator, these take Fokker--Planck-like forms \cite{Fokker_1914, Planck_1917} wherein pseudo-particles in the numerical model obey the neoclassical transport equations, with particle-independent Brownian drift terms. This offers a rigorous methodology for incorporating collisions into the particle transport model, without coupling the equations of motions for each particle.
        
        Works by Chen, Chacón et al. \cite{Chen_Chacón_Barnes_2011, Chacón_Chen_Barnes_2013, Chen_Chacón_2014, Chen_Chacón_2015} have developed structure-preserving particle pushers for neoclassical transport in the Vlasov equations, derived from Crank--Nicolson integrators. We show these too can can derive from a FET interpretation, similarly offering potential extensions to higher-order-in-time particle pushers. The FET formulation is used also to consider how the stochastic drift terms can be incorporated into the pushers. Stochastic gyrokinetic expansions are also discussed.

        Different options for the numerical implementation of these schemes are considered.

        Due to the efficacy of FET in the development of SP timesteppers for both the fluid and kinetic component, we hope this approach will prove effective in the future for developing SP timesteppers for the full hybrid model. We hope this will give us the opportunity to incorporate previously inaccessible kinetic effects into the highly effective, modern, finite-element MHD models.
    \end{abstract}
    
    
    \newpage
    \tableofcontents
    
    
    \newpage
    \pagenumbering{arabic}
    %\linenumbers\renewcommand\thelinenumber{\color{black!50}\arabic{linenumber}}
            \input{0 - introduction/main.tex}
        \part{Research}
            \input{1 - low-noise PiC models/main.tex}
            \input{2 - kinetic component/main.tex}
            \input{3 - fluid component/main.tex}
            \input{4 - numerical implementation/main.tex}
        \part{Project Overview}
            \input{5 - research plan/main.tex}
            \input{6 - summary/main.tex}
    
    
    %\section{}
    \newpage
    \pagenumbering{gobble}
        \printbibliography


    \newpage
    \pagenumbering{roman}
    \appendix
        \part{Appendices}
            \input{8 - Hilbert complexes/main.tex}
            \input{9 - weak conservation proofs/main.tex}
\end{document}

            \documentclass[12pt, a4paper]{report}

\input{template/main.tex}

\title{\BA{Title in Progress...}}
\author{Boris Andrews}
\affil{Mathematical Institute, University of Oxford}
\date{\today}


\begin{document}
    \pagenumbering{gobble}
    \maketitle
    
    
    \begin{abstract}
        Magnetic confinement reactors---in particular tokamaks---offer one of the most promising options for achieving practical nuclear fusion, with the potential to provide virtually limitless, clean energy. The theoretical and numerical modeling of tokamak plasmas is simultaneously an essential component of effective reactor design, and a great research barrier. Tokamak operational conditions exhibit comparatively low Knudsen numbers. Kinetic effects, including kinetic waves and instabilities, Landau damping, bump-on-tail instabilities and more, are therefore highly influential in tokamak plasma dynamics. Purely fluid models are inherently incapable of capturing these effects, whereas the high dimensionality in purely kinetic models render them practically intractable for most relevant purposes.

        We consider a $\delta\!f$ decomposition model, with a macroscopic fluid background and microscopic kinetic correction, both fully coupled to each other. A similar manner of discretization is proposed to that used in the recent \texttt{STRUPHY} code \cite{Holderied_Possanner_Wang_2021, Holderied_2022, Li_et_al_2023} with a finite-element model for the background and a pseudo-particle/PiC model for the correction.

        The fluid background satisfies the full, non-linear, resistive, compressible, Hall MHD equations. \cite{Laakmann_Hu_Farrell_2022} introduces finite-element(-in-space) implicit timesteppers for the incompressible analogue to this system with structure-preserving (SP) properties in the ideal case, alongside parameter-robust preconditioners. We show that these timesteppers can derive from a finite-element-in-time (FET) (and finite-element-in-space) interpretation. The benefits of this reformulation are discussed, including the derivation of timesteppers that are higher order in time, and the quantifiable dissipative SP properties in the non-ideal, resistive case.
        
        We discuss possible options for extending this FET approach to timesteppers for the compressible case.

        The kinetic corrections satisfy linearized Boltzmann equations. Using a Lénard--Bernstein collision operator, these take Fokker--Planck-like forms \cite{Fokker_1914, Planck_1917} wherein pseudo-particles in the numerical model obey the neoclassical transport equations, with particle-independent Brownian drift terms. This offers a rigorous methodology for incorporating collisions into the particle transport model, without coupling the equations of motions for each particle.
        
        Works by Chen, Chacón et al. \cite{Chen_Chacón_Barnes_2011, Chacón_Chen_Barnes_2013, Chen_Chacón_2014, Chen_Chacón_2015} have developed structure-preserving particle pushers for neoclassical transport in the Vlasov equations, derived from Crank--Nicolson integrators. We show these too can can derive from a FET interpretation, similarly offering potential extensions to higher-order-in-time particle pushers. The FET formulation is used also to consider how the stochastic drift terms can be incorporated into the pushers. Stochastic gyrokinetic expansions are also discussed.

        Different options for the numerical implementation of these schemes are considered.

        Due to the efficacy of FET in the development of SP timesteppers for both the fluid and kinetic component, we hope this approach will prove effective in the future for developing SP timesteppers for the full hybrid model. We hope this will give us the opportunity to incorporate previously inaccessible kinetic effects into the highly effective, modern, finite-element MHD models.
    \end{abstract}
    
    
    \newpage
    \tableofcontents
    
    
    \newpage
    \pagenumbering{arabic}
    %\linenumbers\renewcommand\thelinenumber{\color{black!50}\arabic{linenumber}}
            \input{0 - introduction/main.tex}
        \part{Research}
            \input{1 - low-noise PiC models/main.tex}
            \input{2 - kinetic component/main.tex}
            \input{3 - fluid component/main.tex}
            \input{4 - numerical implementation/main.tex}
        \part{Project Overview}
            \input{5 - research plan/main.tex}
            \input{6 - summary/main.tex}
    
    
    %\section{}
    \newpage
    \pagenumbering{gobble}
        \printbibliography


    \newpage
    \pagenumbering{roman}
    \appendix
        \part{Appendices}
            \input{8 - Hilbert complexes/main.tex}
            \input{9 - weak conservation proofs/main.tex}
\end{document}

\end{document}


\title{\BA{Title in Progress...}}
\author{Boris Andrews}
\affil{Mathematical Institute, University of Oxford}
\date{\today}


\begin{document}
    \pagenumbering{gobble}
    \maketitle
    
    
    \begin{abstract}
        Magnetic confinement reactors---in particular tokamaks---offer one of the most promising options for achieving practical nuclear fusion, with the potential to provide virtually limitless, clean energy. The theoretical and numerical modeling of tokamak plasmas is simultaneously an essential component of effective reactor design, and a great research barrier. Tokamak operational conditions exhibit comparatively low Knudsen numbers. Kinetic effects, including kinetic waves and instabilities, Landau damping, bump-on-tail instabilities and more, are therefore highly influential in tokamak plasma dynamics. Purely fluid models are inherently incapable of capturing these effects, whereas the high dimensionality in purely kinetic models render them practically intractable for most relevant purposes.

        We consider a $\delta\!f$ decomposition model, with a macroscopic fluid background and microscopic kinetic correction, both fully coupled to each other. A similar manner of discretization is proposed to that used in the recent \texttt{STRUPHY} code \cite{Holderied_Possanner_Wang_2021, Holderied_2022, Li_et_al_2023} with a finite-element model for the background and a pseudo-particle/PiC model for the correction.

        The fluid background satisfies the full, non-linear, resistive, compressible, Hall MHD equations. \cite{Laakmann_Hu_Farrell_2022} introduces finite-element(-in-space) implicit timesteppers for the incompressible analogue to this system with structure-preserving (SP) properties in the ideal case, alongside parameter-robust preconditioners. We show that these timesteppers can derive from a finite-element-in-time (FET) (and finite-element-in-space) interpretation. The benefits of this reformulation are discussed, including the derivation of timesteppers that are higher order in time, and the quantifiable dissipative SP properties in the non-ideal, resistive case.
        
        We discuss possible options for extending this FET approach to timesteppers for the compressible case.

        The kinetic corrections satisfy linearized Boltzmann equations. Using a Lénard--Bernstein collision operator, these take Fokker--Planck-like forms \cite{Fokker_1914, Planck_1917} wherein pseudo-particles in the numerical model obey the neoclassical transport equations, with particle-independent Brownian drift terms. This offers a rigorous methodology for incorporating collisions into the particle transport model, without coupling the equations of motions for each particle.
        
        Works by Chen, Chacón et al. \cite{Chen_Chacón_Barnes_2011, Chacón_Chen_Barnes_2013, Chen_Chacón_2014, Chen_Chacón_2015} have developed structure-preserving particle pushers for neoclassical transport in the Vlasov equations, derived from Crank--Nicolson integrators. We show these too can can derive from a FET interpretation, similarly offering potential extensions to higher-order-in-time particle pushers. The FET formulation is used also to consider how the stochastic drift terms can be incorporated into the pushers. Stochastic gyrokinetic expansions are also discussed.

        Different options for the numerical implementation of these schemes are considered.

        Due to the efficacy of FET in the development of SP timesteppers for both the fluid and kinetic component, we hope this approach will prove effective in the future for developing SP timesteppers for the full hybrid model. We hope this will give us the opportunity to incorporate previously inaccessible kinetic effects into the highly effective, modern, finite-element MHD models.
    \end{abstract}
    
    
    \newpage
    \tableofcontents
    
    
    \newpage
    \pagenumbering{arabic}
    %\linenumbers\renewcommand\thelinenumber{\color{black!50}\arabic{linenumber}}
            \documentclass[12pt, a4paper]{report}

\documentclass[12pt, a4paper]{report}

\input{template/main.tex}

\title{\BA{Title in Progress...}}
\author{Boris Andrews}
\affil{Mathematical Institute, University of Oxford}
\date{\today}


\begin{document}
    \pagenumbering{gobble}
    \maketitle
    
    
    \begin{abstract}
        Magnetic confinement reactors---in particular tokamaks---offer one of the most promising options for achieving practical nuclear fusion, with the potential to provide virtually limitless, clean energy. The theoretical and numerical modeling of tokamak plasmas is simultaneously an essential component of effective reactor design, and a great research barrier. Tokamak operational conditions exhibit comparatively low Knudsen numbers. Kinetic effects, including kinetic waves and instabilities, Landau damping, bump-on-tail instabilities and more, are therefore highly influential in tokamak plasma dynamics. Purely fluid models are inherently incapable of capturing these effects, whereas the high dimensionality in purely kinetic models render them practically intractable for most relevant purposes.

        We consider a $\delta\!f$ decomposition model, with a macroscopic fluid background and microscopic kinetic correction, both fully coupled to each other. A similar manner of discretization is proposed to that used in the recent \texttt{STRUPHY} code \cite{Holderied_Possanner_Wang_2021, Holderied_2022, Li_et_al_2023} with a finite-element model for the background and a pseudo-particle/PiC model for the correction.

        The fluid background satisfies the full, non-linear, resistive, compressible, Hall MHD equations. \cite{Laakmann_Hu_Farrell_2022} introduces finite-element(-in-space) implicit timesteppers for the incompressible analogue to this system with structure-preserving (SP) properties in the ideal case, alongside parameter-robust preconditioners. We show that these timesteppers can derive from a finite-element-in-time (FET) (and finite-element-in-space) interpretation. The benefits of this reformulation are discussed, including the derivation of timesteppers that are higher order in time, and the quantifiable dissipative SP properties in the non-ideal, resistive case.
        
        We discuss possible options for extending this FET approach to timesteppers for the compressible case.

        The kinetic corrections satisfy linearized Boltzmann equations. Using a Lénard--Bernstein collision operator, these take Fokker--Planck-like forms \cite{Fokker_1914, Planck_1917} wherein pseudo-particles in the numerical model obey the neoclassical transport equations, with particle-independent Brownian drift terms. This offers a rigorous methodology for incorporating collisions into the particle transport model, without coupling the equations of motions for each particle.
        
        Works by Chen, Chacón et al. \cite{Chen_Chacón_Barnes_2011, Chacón_Chen_Barnes_2013, Chen_Chacón_2014, Chen_Chacón_2015} have developed structure-preserving particle pushers for neoclassical transport in the Vlasov equations, derived from Crank--Nicolson integrators. We show these too can can derive from a FET interpretation, similarly offering potential extensions to higher-order-in-time particle pushers. The FET formulation is used also to consider how the stochastic drift terms can be incorporated into the pushers. Stochastic gyrokinetic expansions are also discussed.

        Different options for the numerical implementation of these schemes are considered.

        Due to the efficacy of FET in the development of SP timesteppers for both the fluid and kinetic component, we hope this approach will prove effective in the future for developing SP timesteppers for the full hybrid model. We hope this will give us the opportunity to incorporate previously inaccessible kinetic effects into the highly effective, modern, finite-element MHD models.
    \end{abstract}
    
    
    \newpage
    \tableofcontents
    
    
    \newpage
    \pagenumbering{arabic}
    %\linenumbers\renewcommand\thelinenumber{\color{black!50}\arabic{linenumber}}
            \input{0 - introduction/main.tex}
        \part{Research}
            \input{1 - low-noise PiC models/main.tex}
            \input{2 - kinetic component/main.tex}
            \input{3 - fluid component/main.tex}
            \input{4 - numerical implementation/main.tex}
        \part{Project Overview}
            \input{5 - research plan/main.tex}
            \input{6 - summary/main.tex}
    
    
    %\section{}
    \newpage
    \pagenumbering{gobble}
        \printbibliography


    \newpage
    \pagenumbering{roman}
    \appendix
        \part{Appendices}
            \input{8 - Hilbert complexes/main.tex}
            \input{9 - weak conservation proofs/main.tex}
\end{document}


\title{\BA{Title in Progress...}}
\author{Boris Andrews}
\affil{Mathematical Institute, University of Oxford}
\date{\today}


\begin{document}
    \pagenumbering{gobble}
    \maketitle
    
    
    \begin{abstract}
        Magnetic confinement reactors---in particular tokamaks---offer one of the most promising options for achieving practical nuclear fusion, with the potential to provide virtually limitless, clean energy. The theoretical and numerical modeling of tokamak plasmas is simultaneously an essential component of effective reactor design, and a great research barrier. Tokamak operational conditions exhibit comparatively low Knudsen numbers. Kinetic effects, including kinetic waves and instabilities, Landau damping, bump-on-tail instabilities and more, are therefore highly influential in tokamak plasma dynamics. Purely fluid models are inherently incapable of capturing these effects, whereas the high dimensionality in purely kinetic models render them practically intractable for most relevant purposes.

        We consider a $\delta\!f$ decomposition model, with a macroscopic fluid background and microscopic kinetic correction, both fully coupled to each other. A similar manner of discretization is proposed to that used in the recent \texttt{STRUPHY} code \cite{Holderied_Possanner_Wang_2021, Holderied_2022, Li_et_al_2023} with a finite-element model for the background and a pseudo-particle/PiC model for the correction.

        The fluid background satisfies the full, non-linear, resistive, compressible, Hall MHD equations. \cite{Laakmann_Hu_Farrell_2022} introduces finite-element(-in-space) implicit timesteppers for the incompressible analogue to this system with structure-preserving (SP) properties in the ideal case, alongside parameter-robust preconditioners. We show that these timesteppers can derive from a finite-element-in-time (FET) (and finite-element-in-space) interpretation. The benefits of this reformulation are discussed, including the derivation of timesteppers that are higher order in time, and the quantifiable dissipative SP properties in the non-ideal, resistive case.
        
        We discuss possible options for extending this FET approach to timesteppers for the compressible case.

        The kinetic corrections satisfy linearized Boltzmann equations. Using a Lénard--Bernstein collision operator, these take Fokker--Planck-like forms \cite{Fokker_1914, Planck_1917} wherein pseudo-particles in the numerical model obey the neoclassical transport equations, with particle-independent Brownian drift terms. This offers a rigorous methodology for incorporating collisions into the particle transport model, without coupling the equations of motions for each particle.
        
        Works by Chen, Chacón et al. \cite{Chen_Chacón_Barnes_2011, Chacón_Chen_Barnes_2013, Chen_Chacón_2014, Chen_Chacón_2015} have developed structure-preserving particle pushers for neoclassical transport in the Vlasov equations, derived from Crank--Nicolson integrators. We show these too can can derive from a FET interpretation, similarly offering potential extensions to higher-order-in-time particle pushers. The FET formulation is used also to consider how the stochastic drift terms can be incorporated into the pushers. Stochastic gyrokinetic expansions are also discussed.

        Different options for the numerical implementation of these schemes are considered.

        Due to the efficacy of FET in the development of SP timesteppers for both the fluid and kinetic component, we hope this approach will prove effective in the future for developing SP timesteppers for the full hybrid model. We hope this will give us the opportunity to incorporate previously inaccessible kinetic effects into the highly effective, modern, finite-element MHD models.
    \end{abstract}
    
    
    \newpage
    \tableofcontents
    
    
    \newpage
    \pagenumbering{arabic}
    %\linenumbers\renewcommand\thelinenumber{\color{black!50}\arabic{linenumber}}
            \documentclass[12pt, a4paper]{report}

\input{template/main.tex}

\title{\BA{Title in Progress...}}
\author{Boris Andrews}
\affil{Mathematical Institute, University of Oxford}
\date{\today}


\begin{document}
    \pagenumbering{gobble}
    \maketitle
    
    
    \begin{abstract}
        Magnetic confinement reactors---in particular tokamaks---offer one of the most promising options for achieving practical nuclear fusion, with the potential to provide virtually limitless, clean energy. The theoretical and numerical modeling of tokamak plasmas is simultaneously an essential component of effective reactor design, and a great research barrier. Tokamak operational conditions exhibit comparatively low Knudsen numbers. Kinetic effects, including kinetic waves and instabilities, Landau damping, bump-on-tail instabilities and more, are therefore highly influential in tokamak plasma dynamics. Purely fluid models are inherently incapable of capturing these effects, whereas the high dimensionality in purely kinetic models render them practically intractable for most relevant purposes.

        We consider a $\delta\!f$ decomposition model, with a macroscopic fluid background and microscopic kinetic correction, both fully coupled to each other. A similar manner of discretization is proposed to that used in the recent \texttt{STRUPHY} code \cite{Holderied_Possanner_Wang_2021, Holderied_2022, Li_et_al_2023} with a finite-element model for the background and a pseudo-particle/PiC model for the correction.

        The fluid background satisfies the full, non-linear, resistive, compressible, Hall MHD equations. \cite{Laakmann_Hu_Farrell_2022} introduces finite-element(-in-space) implicit timesteppers for the incompressible analogue to this system with structure-preserving (SP) properties in the ideal case, alongside parameter-robust preconditioners. We show that these timesteppers can derive from a finite-element-in-time (FET) (and finite-element-in-space) interpretation. The benefits of this reformulation are discussed, including the derivation of timesteppers that are higher order in time, and the quantifiable dissipative SP properties in the non-ideal, resistive case.
        
        We discuss possible options for extending this FET approach to timesteppers for the compressible case.

        The kinetic corrections satisfy linearized Boltzmann equations. Using a Lénard--Bernstein collision operator, these take Fokker--Planck-like forms \cite{Fokker_1914, Planck_1917} wherein pseudo-particles in the numerical model obey the neoclassical transport equations, with particle-independent Brownian drift terms. This offers a rigorous methodology for incorporating collisions into the particle transport model, without coupling the equations of motions for each particle.
        
        Works by Chen, Chacón et al. \cite{Chen_Chacón_Barnes_2011, Chacón_Chen_Barnes_2013, Chen_Chacón_2014, Chen_Chacón_2015} have developed structure-preserving particle pushers for neoclassical transport in the Vlasov equations, derived from Crank--Nicolson integrators. We show these too can can derive from a FET interpretation, similarly offering potential extensions to higher-order-in-time particle pushers. The FET formulation is used also to consider how the stochastic drift terms can be incorporated into the pushers. Stochastic gyrokinetic expansions are also discussed.

        Different options for the numerical implementation of these schemes are considered.

        Due to the efficacy of FET in the development of SP timesteppers for both the fluid and kinetic component, we hope this approach will prove effective in the future for developing SP timesteppers for the full hybrid model. We hope this will give us the opportunity to incorporate previously inaccessible kinetic effects into the highly effective, modern, finite-element MHD models.
    \end{abstract}
    
    
    \newpage
    \tableofcontents
    
    
    \newpage
    \pagenumbering{arabic}
    %\linenumbers\renewcommand\thelinenumber{\color{black!50}\arabic{linenumber}}
            \input{0 - introduction/main.tex}
        \part{Research}
            \input{1 - low-noise PiC models/main.tex}
            \input{2 - kinetic component/main.tex}
            \input{3 - fluid component/main.tex}
            \input{4 - numerical implementation/main.tex}
        \part{Project Overview}
            \input{5 - research plan/main.tex}
            \input{6 - summary/main.tex}
    
    
    %\section{}
    \newpage
    \pagenumbering{gobble}
        \printbibliography


    \newpage
    \pagenumbering{roman}
    \appendix
        \part{Appendices}
            \input{8 - Hilbert complexes/main.tex}
            \input{9 - weak conservation proofs/main.tex}
\end{document}

        \part{Research}
            \documentclass[12pt, a4paper]{report}

\input{template/main.tex}

\title{\BA{Title in Progress...}}
\author{Boris Andrews}
\affil{Mathematical Institute, University of Oxford}
\date{\today}


\begin{document}
    \pagenumbering{gobble}
    \maketitle
    
    
    \begin{abstract}
        Magnetic confinement reactors---in particular tokamaks---offer one of the most promising options for achieving practical nuclear fusion, with the potential to provide virtually limitless, clean energy. The theoretical and numerical modeling of tokamak plasmas is simultaneously an essential component of effective reactor design, and a great research barrier. Tokamak operational conditions exhibit comparatively low Knudsen numbers. Kinetic effects, including kinetic waves and instabilities, Landau damping, bump-on-tail instabilities and more, are therefore highly influential in tokamak plasma dynamics. Purely fluid models are inherently incapable of capturing these effects, whereas the high dimensionality in purely kinetic models render them practically intractable for most relevant purposes.

        We consider a $\delta\!f$ decomposition model, with a macroscopic fluid background and microscopic kinetic correction, both fully coupled to each other. A similar manner of discretization is proposed to that used in the recent \texttt{STRUPHY} code \cite{Holderied_Possanner_Wang_2021, Holderied_2022, Li_et_al_2023} with a finite-element model for the background and a pseudo-particle/PiC model for the correction.

        The fluid background satisfies the full, non-linear, resistive, compressible, Hall MHD equations. \cite{Laakmann_Hu_Farrell_2022} introduces finite-element(-in-space) implicit timesteppers for the incompressible analogue to this system with structure-preserving (SP) properties in the ideal case, alongside parameter-robust preconditioners. We show that these timesteppers can derive from a finite-element-in-time (FET) (and finite-element-in-space) interpretation. The benefits of this reformulation are discussed, including the derivation of timesteppers that are higher order in time, and the quantifiable dissipative SP properties in the non-ideal, resistive case.
        
        We discuss possible options for extending this FET approach to timesteppers for the compressible case.

        The kinetic corrections satisfy linearized Boltzmann equations. Using a Lénard--Bernstein collision operator, these take Fokker--Planck-like forms \cite{Fokker_1914, Planck_1917} wherein pseudo-particles in the numerical model obey the neoclassical transport equations, with particle-independent Brownian drift terms. This offers a rigorous methodology for incorporating collisions into the particle transport model, without coupling the equations of motions for each particle.
        
        Works by Chen, Chacón et al. \cite{Chen_Chacón_Barnes_2011, Chacón_Chen_Barnes_2013, Chen_Chacón_2014, Chen_Chacón_2015} have developed structure-preserving particle pushers for neoclassical transport in the Vlasov equations, derived from Crank--Nicolson integrators. We show these too can can derive from a FET interpretation, similarly offering potential extensions to higher-order-in-time particle pushers. The FET formulation is used also to consider how the stochastic drift terms can be incorporated into the pushers. Stochastic gyrokinetic expansions are also discussed.

        Different options for the numerical implementation of these schemes are considered.

        Due to the efficacy of FET in the development of SP timesteppers for both the fluid and kinetic component, we hope this approach will prove effective in the future for developing SP timesteppers for the full hybrid model. We hope this will give us the opportunity to incorporate previously inaccessible kinetic effects into the highly effective, modern, finite-element MHD models.
    \end{abstract}
    
    
    \newpage
    \tableofcontents
    
    
    \newpage
    \pagenumbering{arabic}
    %\linenumbers\renewcommand\thelinenumber{\color{black!50}\arabic{linenumber}}
            \input{0 - introduction/main.tex}
        \part{Research}
            \input{1 - low-noise PiC models/main.tex}
            \input{2 - kinetic component/main.tex}
            \input{3 - fluid component/main.tex}
            \input{4 - numerical implementation/main.tex}
        \part{Project Overview}
            \input{5 - research plan/main.tex}
            \input{6 - summary/main.tex}
    
    
    %\section{}
    \newpage
    \pagenumbering{gobble}
        \printbibliography


    \newpage
    \pagenumbering{roman}
    \appendix
        \part{Appendices}
            \input{8 - Hilbert complexes/main.tex}
            \input{9 - weak conservation proofs/main.tex}
\end{document}

            \documentclass[12pt, a4paper]{report}

\input{template/main.tex}

\title{\BA{Title in Progress...}}
\author{Boris Andrews}
\affil{Mathematical Institute, University of Oxford}
\date{\today}


\begin{document}
    \pagenumbering{gobble}
    \maketitle
    
    
    \begin{abstract}
        Magnetic confinement reactors---in particular tokamaks---offer one of the most promising options for achieving practical nuclear fusion, with the potential to provide virtually limitless, clean energy. The theoretical and numerical modeling of tokamak plasmas is simultaneously an essential component of effective reactor design, and a great research barrier. Tokamak operational conditions exhibit comparatively low Knudsen numbers. Kinetic effects, including kinetic waves and instabilities, Landau damping, bump-on-tail instabilities and more, are therefore highly influential in tokamak plasma dynamics. Purely fluid models are inherently incapable of capturing these effects, whereas the high dimensionality in purely kinetic models render them practically intractable for most relevant purposes.

        We consider a $\delta\!f$ decomposition model, with a macroscopic fluid background and microscopic kinetic correction, both fully coupled to each other. A similar manner of discretization is proposed to that used in the recent \texttt{STRUPHY} code \cite{Holderied_Possanner_Wang_2021, Holderied_2022, Li_et_al_2023} with a finite-element model for the background and a pseudo-particle/PiC model for the correction.

        The fluid background satisfies the full, non-linear, resistive, compressible, Hall MHD equations. \cite{Laakmann_Hu_Farrell_2022} introduces finite-element(-in-space) implicit timesteppers for the incompressible analogue to this system with structure-preserving (SP) properties in the ideal case, alongside parameter-robust preconditioners. We show that these timesteppers can derive from a finite-element-in-time (FET) (and finite-element-in-space) interpretation. The benefits of this reformulation are discussed, including the derivation of timesteppers that are higher order in time, and the quantifiable dissipative SP properties in the non-ideal, resistive case.
        
        We discuss possible options for extending this FET approach to timesteppers for the compressible case.

        The kinetic corrections satisfy linearized Boltzmann equations. Using a Lénard--Bernstein collision operator, these take Fokker--Planck-like forms \cite{Fokker_1914, Planck_1917} wherein pseudo-particles in the numerical model obey the neoclassical transport equations, with particle-independent Brownian drift terms. This offers a rigorous methodology for incorporating collisions into the particle transport model, without coupling the equations of motions for each particle.
        
        Works by Chen, Chacón et al. \cite{Chen_Chacón_Barnes_2011, Chacón_Chen_Barnes_2013, Chen_Chacón_2014, Chen_Chacón_2015} have developed structure-preserving particle pushers for neoclassical transport in the Vlasov equations, derived from Crank--Nicolson integrators. We show these too can can derive from a FET interpretation, similarly offering potential extensions to higher-order-in-time particle pushers. The FET formulation is used also to consider how the stochastic drift terms can be incorporated into the pushers. Stochastic gyrokinetic expansions are also discussed.

        Different options for the numerical implementation of these schemes are considered.

        Due to the efficacy of FET in the development of SP timesteppers for both the fluid and kinetic component, we hope this approach will prove effective in the future for developing SP timesteppers for the full hybrid model. We hope this will give us the opportunity to incorporate previously inaccessible kinetic effects into the highly effective, modern, finite-element MHD models.
    \end{abstract}
    
    
    \newpage
    \tableofcontents
    
    
    \newpage
    \pagenumbering{arabic}
    %\linenumbers\renewcommand\thelinenumber{\color{black!50}\arabic{linenumber}}
            \input{0 - introduction/main.tex}
        \part{Research}
            \input{1 - low-noise PiC models/main.tex}
            \input{2 - kinetic component/main.tex}
            \input{3 - fluid component/main.tex}
            \input{4 - numerical implementation/main.tex}
        \part{Project Overview}
            \input{5 - research plan/main.tex}
            \input{6 - summary/main.tex}
    
    
    %\section{}
    \newpage
    \pagenumbering{gobble}
        \printbibliography


    \newpage
    \pagenumbering{roman}
    \appendix
        \part{Appendices}
            \input{8 - Hilbert complexes/main.tex}
            \input{9 - weak conservation proofs/main.tex}
\end{document}

            \documentclass[12pt, a4paper]{report}

\input{template/main.tex}

\title{\BA{Title in Progress...}}
\author{Boris Andrews}
\affil{Mathematical Institute, University of Oxford}
\date{\today}


\begin{document}
    \pagenumbering{gobble}
    \maketitle
    
    
    \begin{abstract}
        Magnetic confinement reactors---in particular tokamaks---offer one of the most promising options for achieving practical nuclear fusion, with the potential to provide virtually limitless, clean energy. The theoretical and numerical modeling of tokamak plasmas is simultaneously an essential component of effective reactor design, and a great research barrier. Tokamak operational conditions exhibit comparatively low Knudsen numbers. Kinetic effects, including kinetic waves and instabilities, Landau damping, bump-on-tail instabilities and more, are therefore highly influential in tokamak plasma dynamics. Purely fluid models are inherently incapable of capturing these effects, whereas the high dimensionality in purely kinetic models render them practically intractable for most relevant purposes.

        We consider a $\delta\!f$ decomposition model, with a macroscopic fluid background and microscopic kinetic correction, both fully coupled to each other. A similar manner of discretization is proposed to that used in the recent \texttt{STRUPHY} code \cite{Holderied_Possanner_Wang_2021, Holderied_2022, Li_et_al_2023} with a finite-element model for the background and a pseudo-particle/PiC model for the correction.

        The fluid background satisfies the full, non-linear, resistive, compressible, Hall MHD equations. \cite{Laakmann_Hu_Farrell_2022} introduces finite-element(-in-space) implicit timesteppers for the incompressible analogue to this system with structure-preserving (SP) properties in the ideal case, alongside parameter-robust preconditioners. We show that these timesteppers can derive from a finite-element-in-time (FET) (and finite-element-in-space) interpretation. The benefits of this reformulation are discussed, including the derivation of timesteppers that are higher order in time, and the quantifiable dissipative SP properties in the non-ideal, resistive case.
        
        We discuss possible options for extending this FET approach to timesteppers for the compressible case.

        The kinetic corrections satisfy linearized Boltzmann equations. Using a Lénard--Bernstein collision operator, these take Fokker--Planck-like forms \cite{Fokker_1914, Planck_1917} wherein pseudo-particles in the numerical model obey the neoclassical transport equations, with particle-independent Brownian drift terms. This offers a rigorous methodology for incorporating collisions into the particle transport model, without coupling the equations of motions for each particle.
        
        Works by Chen, Chacón et al. \cite{Chen_Chacón_Barnes_2011, Chacón_Chen_Barnes_2013, Chen_Chacón_2014, Chen_Chacón_2015} have developed structure-preserving particle pushers for neoclassical transport in the Vlasov equations, derived from Crank--Nicolson integrators. We show these too can can derive from a FET interpretation, similarly offering potential extensions to higher-order-in-time particle pushers. The FET formulation is used also to consider how the stochastic drift terms can be incorporated into the pushers. Stochastic gyrokinetic expansions are also discussed.

        Different options for the numerical implementation of these schemes are considered.

        Due to the efficacy of FET in the development of SP timesteppers for both the fluid and kinetic component, we hope this approach will prove effective in the future for developing SP timesteppers for the full hybrid model. We hope this will give us the opportunity to incorporate previously inaccessible kinetic effects into the highly effective, modern, finite-element MHD models.
    \end{abstract}
    
    
    \newpage
    \tableofcontents
    
    
    \newpage
    \pagenumbering{arabic}
    %\linenumbers\renewcommand\thelinenumber{\color{black!50}\arabic{linenumber}}
            \input{0 - introduction/main.tex}
        \part{Research}
            \input{1 - low-noise PiC models/main.tex}
            \input{2 - kinetic component/main.tex}
            \input{3 - fluid component/main.tex}
            \input{4 - numerical implementation/main.tex}
        \part{Project Overview}
            \input{5 - research plan/main.tex}
            \input{6 - summary/main.tex}
    
    
    %\section{}
    \newpage
    \pagenumbering{gobble}
        \printbibliography


    \newpage
    \pagenumbering{roman}
    \appendix
        \part{Appendices}
            \input{8 - Hilbert complexes/main.tex}
            \input{9 - weak conservation proofs/main.tex}
\end{document}

            \documentclass[12pt, a4paper]{report}

\input{template/main.tex}

\title{\BA{Title in Progress...}}
\author{Boris Andrews}
\affil{Mathematical Institute, University of Oxford}
\date{\today}


\begin{document}
    \pagenumbering{gobble}
    \maketitle
    
    
    \begin{abstract}
        Magnetic confinement reactors---in particular tokamaks---offer one of the most promising options for achieving practical nuclear fusion, with the potential to provide virtually limitless, clean energy. The theoretical and numerical modeling of tokamak plasmas is simultaneously an essential component of effective reactor design, and a great research barrier. Tokamak operational conditions exhibit comparatively low Knudsen numbers. Kinetic effects, including kinetic waves and instabilities, Landau damping, bump-on-tail instabilities and more, are therefore highly influential in tokamak plasma dynamics. Purely fluid models are inherently incapable of capturing these effects, whereas the high dimensionality in purely kinetic models render them practically intractable for most relevant purposes.

        We consider a $\delta\!f$ decomposition model, with a macroscopic fluid background and microscopic kinetic correction, both fully coupled to each other. A similar manner of discretization is proposed to that used in the recent \texttt{STRUPHY} code \cite{Holderied_Possanner_Wang_2021, Holderied_2022, Li_et_al_2023} with a finite-element model for the background and a pseudo-particle/PiC model for the correction.

        The fluid background satisfies the full, non-linear, resistive, compressible, Hall MHD equations. \cite{Laakmann_Hu_Farrell_2022} introduces finite-element(-in-space) implicit timesteppers for the incompressible analogue to this system with structure-preserving (SP) properties in the ideal case, alongside parameter-robust preconditioners. We show that these timesteppers can derive from a finite-element-in-time (FET) (and finite-element-in-space) interpretation. The benefits of this reformulation are discussed, including the derivation of timesteppers that are higher order in time, and the quantifiable dissipative SP properties in the non-ideal, resistive case.
        
        We discuss possible options for extending this FET approach to timesteppers for the compressible case.

        The kinetic corrections satisfy linearized Boltzmann equations. Using a Lénard--Bernstein collision operator, these take Fokker--Planck-like forms \cite{Fokker_1914, Planck_1917} wherein pseudo-particles in the numerical model obey the neoclassical transport equations, with particle-independent Brownian drift terms. This offers a rigorous methodology for incorporating collisions into the particle transport model, without coupling the equations of motions for each particle.
        
        Works by Chen, Chacón et al. \cite{Chen_Chacón_Barnes_2011, Chacón_Chen_Barnes_2013, Chen_Chacón_2014, Chen_Chacón_2015} have developed structure-preserving particle pushers for neoclassical transport in the Vlasov equations, derived from Crank--Nicolson integrators. We show these too can can derive from a FET interpretation, similarly offering potential extensions to higher-order-in-time particle pushers. The FET formulation is used also to consider how the stochastic drift terms can be incorporated into the pushers. Stochastic gyrokinetic expansions are also discussed.

        Different options for the numerical implementation of these schemes are considered.

        Due to the efficacy of FET in the development of SP timesteppers for both the fluid and kinetic component, we hope this approach will prove effective in the future for developing SP timesteppers for the full hybrid model. We hope this will give us the opportunity to incorporate previously inaccessible kinetic effects into the highly effective, modern, finite-element MHD models.
    \end{abstract}
    
    
    \newpage
    \tableofcontents
    
    
    \newpage
    \pagenumbering{arabic}
    %\linenumbers\renewcommand\thelinenumber{\color{black!50}\arabic{linenumber}}
            \input{0 - introduction/main.tex}
        \part{Research}
            \input{1 - low-noise PiC models/main.tex}
            \input{2 - kinetic component/main.tex}
            \input{3 - fluid component/main.tex}
            \input{4 - numerical implementation/main.tex}
        \part{Project Overview}
            \input{5 - research plan/main.tex}
            \input{6 - summary/main.tex}
    
    
    %\section{}
    \newpage
    \pagenumbering{gobble}
        \printbibliography


    \newpage
    \pagenumbering{roman}
    \appendix
        \part{Appendices}
            \input{8 - Hilbert complexes/main.tex}
            \input{9 - weak conservation proofs/main.tex}
\end{document}

        \part{Project Overview}
            \documentclass[12pt, a4paper]{report}

\input{template/main.tex}

\title{\BA{Title in Progress...}}
\author{Boris Andrews}
\affil{Mathematical Institute, University of Oxford}
\date{\today}


\begin{document}
    \pagenumbering{gobble}
    \maketitle
    
    
    \begin{abstract}
        Magnetic confinement reactors---in particular tokamaks---offer one of the most promising options for achieving practical nuclear fusion, with the potential to provide virtually limitless, clean energy. The theoretical and numerical modeling of tokamak plasmas is simultaneously an essential component of effective reactor design, and a great research barrier. Tokamak operational conditions exhibit comparatively low Knudsen numbers. Kinetic effects, including kinetic waves and instabilities, Landau damping, bump-on-tail instabilities and more, are therefore highly influential in tokamak plasma dynamics. Purely fluid models are inherently incapable of capturing these effects, whereas the high dimensionality in purely kinetic models render them practically intractable for most relevant purposes.

        We consider a $\delta\!f$ decomposition model, with a macroscopic fluid background and microscopic kinetic correction, both fully coupled to each other. A similar manner of discretization is proposed to that used in the recent \texttt{STRUPHY} code \cite{Holderied_Possanner_Wang_2021, Holderied_2022, Li_et_al_2023} with a finite-element model for the background and a pseudo-particle/PiC model for the correction.

        The fluid background satisfies the full, non-linear, resistive, compressible, Hall MHD equations. \cite{Laakmann_Hu_Farrell_2022} introduces finite-element(-in-space) implicit timesteppers for the incompressible analogue to this system with structure-preserving (SP) properties in the ideal case, alongside parameter-robust preconditioners. We show that these timesteppers can derive from a finite-element-in-time (FET) (and finite-element-in-space) interpretation. The benefits of this reformulation are discussed, including the derivation of timesteppers that are higher order in time, and the quantifiable dissipative SP properties in the non-ideal, resistive case.
        
        We discuss possible options for extending this FET approach to timesteppers for the compressible case.

        The kinetic corrections satisfy linearized Boltzmann equations. Using a Lénard--Bernstein collision operator, these take Fokker--Planck-like forms \cite{Fokker_1914, Planck_1917} wherein pseudo-particles in the numerical model obey the neoclassical transport equations, with particle-independent Brownian drift terms. This offers a rigorous methodology for incorporating collisions into the particle transport model, without coupling the equations of motions for each particle.
        
        Works by Chen, Chacón et al. \cite{Chen_Chacón_Barnes_2011, Chacón_Chen_Barnes_2013, Chen_Chacón_2014, Chen_Chacón_2015} have developed structure-preserving particle pushers for neoclassical transport in the Vlasov equations, derived from Crank--Nicolson integrators. We show these too can can derive from a FET interpretation, similarly offering potential extensions to higher-order-in-time particle pushers. The FET formulation is used also to consider how the stochastic drift terms can be incorporated into the pushers. Stochastic gyrokinetic expansions are also discussed.

        Different options for the numerical implementation of these schemes are considered.

        Due to the efficacy of FET in the development of SP timesteppers for both the fluid and kinetic component, we hope this approach will prove effective in the future for developing SP timesteppers for the full hybrid model. We hope this will give us the opportunity to incorporate previously inaccessible kinetic effects into the highly effective, modern, finite-element MHD models.
    \end{abstract}
    
    
    \newpage
    \tableofcontents
    
    
    \newpage
    \pagenumbering{arabic}
    %\linenumbers\renewcommand\thelinenumber{\color{black!50}\arabic{linenumber}}
            \input{0 - introduction/main.tex}
        \part{Research}
            \input{1 - low-noise PiC models/main.tex}
            \input{2 - kinetic component/main.tex}
            \input{3 - fluid component/main.tex}
            \input{4 - numerical implementation/main.tex}
        \part{Project Overview}
            \input{5 - research plan/main.tex}
            \input{6 - summary/main.tex}
    
    
    %\section{}
    \newpage
    \pagenumbering{gobble}
        \printbibliography


    \newpage
    \pagenumbering{roman}
    \appendix
        \part{Appendices}
            \input{8 - Hilbert complexes/main.tex}
            \input{9 - weak conservation proofs/main.tex}
\end{document}

            \documentclass[12pt, a4paper]{report}

\input{template/main.tex}

\title{\BA{Title in Progress...}}
\author{Boris Andrews}
\affil{Mathematical Institute, University of Oxford}
\date{\today}


\begin{document}
    \pagenumbering{gobble}
    \maketitle
    
    
    \begin{abstract}
        Magnetic confinement reactors---in particular tokamaks---offer one of the most promising options for achieving practical nuclear fusion, with the potential to provide virtually limitless, clean energy. The theoretical and numerical modeling of tokamak plasmas is simultaneously an essential component of effective reactor design, and a great research barrier. Tokamak operational conditions exhibit comparatively low Knudsen numbers. Kinetic effects, including kinetic waves and instabilities, Landau damping, bump-on-tail instabilities and more, are therefore highly influential in tokamak plasma dynamics. Purely fluid models are inherently incapable of capturing these effects, whereas the high dimensionality in purely kinetic models render them practically intractable for most relevant purposes.

        We consider a $\delta\!f$ decomposition model, with a macroscopic fluid background and microscopic kinetic correction, both fully coupled to each other. A similar manner of discretization is proposed to that used in the recent \texttt{STRUPHY} code \cite{Holderied_Possanner_Wang_2021, Holderied_2022, Li_et_al_2023} with a finite-element model for the background and a pseudo-particle/PiC model for the correction.

        The fluid background satisfies the full, non-linear, resistive, compressible, Hall MHD equations. \cite{Laakmann_Hu_Farrell_2022} introduces finite-element(-in-space) implicit timesteppers for the incompressible analogue to this system with structure-preserving (SP) properties in the ideal case, alongside parameter-robust preconditioners. We show that these timesteppers can derive from a finite-element-in-time (FET) (and finite-element-in-space) interpretation. The benefits of this reformulation are discussed, including the derivation of timesteppers that are higher order in time, and the quantifiable dissipative SP properties in the non-ideal, resistive case.
        
        We discuss possible options for extending this FET approach to timesteppers for the compressible case.

        The kinetic corrections satisfy linearized Boltzmann equations. Using a Lénard--Bernstein collision operator, these take Fokker--Planck-like forms \cite{Fokker_1914, Planck_1917} wherein pseudo-particles in the numerical model obey the neoclassical transport equations, with particle-independent Brownian drift terms. This offers a rigorous methodology for incorporating collisions into the particle transport model, without coupling the equations of motions for each particle.
        
        Works by Chen, Chacón et al. \cite{Chen_Chacón_Barnes_2011, Chacón_Chen_Barnes_2013, Chen_Chacón_2014, Chen_Chacón_2015} have developed structure-preserving particle pushers for neoclassical transport in the Vlasov equations, derived from Crank--Nicolson integrators. We show these too can can derive from a FET interpretation, similarly offering potential extensions to higher-order-in-time particle pushers. The FET formulation is used also to consider how the stochastic drift terms can be incorporated into the pushers. Stochastic gyrokinetic expansions are also discussed.

        Different options for the numerical implementation of these schemes are considered.

        Due to the efficacy of FET in the development of SP timesteppers for both the fluid and kinetic component, we hope this approach will prove effective in the future for developing SP timesteppers for the full hybrid model. We hope this will give us the opportunity to incorporate previously inaccessible kinetic effects into the highly effective, modern, finite-element MHD models.
    \end{abstract}
    
    
    \newpage
    \tableofcontents
    
    
    \newpage
    \pagenumbering{arabic}
    %\linenumbers\renewcommand\thelinenumber{\color{black!50}\arabic{linenumber}}
            \input{0 - introduction/main.tex}
        \part{Research}
            \input{1 - low-noise PiC models/main.tex}
            \input{2 - kinetic component/main.tex}
            \input{3 - fluid component/main.tex}
            \input{4 - numerical implementation/main.tex}
        \part{Project Overview}
            \input{5 - research plan/main.tex}
            \input{6 - summary/main.tex}
    
    
    %\section{}
    \newpage
    \pagenumbering{gobble}
        \printbibliography


    \newpage
    \pagenumbering{roman}
    \appendix
        \part{Appendices}
            \input{8 - Hilbert complexes/main.tex}
            \input{9 - weak conservation proofs/main.tex}
\end{document}

    
    
    %\section{}
    \newpage
    \pagenumbering{gobble}
        \printbibliography


    \newpage
    \pagenumbering{roman}
    \appendix
        \part{Appendices}
            \documentclass[12pt, a4paper]{report}

\input{template/main.tex}

\title{\BA{Title in Progress...}}
\author{Boris Andrews}
\affil{Mathematical Institute, University of Oxford}
\date{\today}


\begin{document}
    \pagenumbering{gobble}
    \maketitle
    
    
    \begin{abstract}
        Magnetic confinement reactors---in particular tokamaks---offer one of the most promising options for achieving practical nuclear fusion, with the potential to provide virtually limitless, clean energy. The theoretical and numerical modeling of tokamak plasmas is simultaneously an essential component of effective reactor design, and a great research barrier. Tokamak operational conditions exhibit comparatively low Knudsen numbers. Kinetic effects, including kinetic waves and instabilities, Landau damping, bump-on-tail instabilities and more, are therefore highly influential in tokamak plasma dynamics. Purely fluid models are inherently incapable of capturing these effects, whereas the high dimensionality in purely kinetic models render them practically intractable for most relevant purposes.

        We consider a $\delta\!f$ decomposition model, with a macroscopic fluid background and microscopic kinetic correction, both fully coupled to each other. A similar manner of discretization is proposed to that used in the recent \texttt{STRUPHY} code \cite{Holderied_Possanner_Wang_2021, Holderied_2022, Li_et_al_2023} with a finite-element model for the background and a pseudo-particle/PiC model for the correction.

        The fluid background satisfies the full, non-linear, resistive, compressible, Hall MHD equations. \cite{Laakmann_Hu_Farrell_2022} introduces finite-element(-in-space) implicit timesteppers for the incompressible analogue to this system with structure-preserving (SP) properties in the ideal case, alongside parameter-robust preconditioners. We show that these timesteppers can derive from a finite-element-in-time (FET) (and finite-element-in-space) interpretation. The benefits of this reformulation are discussed, including the derivation of timesteppers that are higher order in time, and the quantifiable dissipative SP properties in the non-ideal, resistive case.
        
        We discuss possible options for extending this FET approach to timesteppers for the compressible case.

        The kinetic corrections satisfy linearized Boltzmann equations. Using a Lénard--Bernstein collision operator, these take Fokker--Planck-like forms \cite{Fokker_1914, Planck_1917} wherein pseudo-particles in the numerical model obey the neoclassical transport equations, with particle-independent Brownian drift terms. This offers a rigorous methodology for incorporating collisions into the particle transport model, without coupling the equations of motions for each particle.
        
        Works by Chen, Chacón et al. \cite{Chen_Chacón_Barnes_2011, Chacón_Chen_Barnes_2013, Chen_Chacón_2014, Chen_Chacón_2015} have developed structure-preserving particle pushers for neoclassical transport in the Vlasov equations, derived from Crank--Nicolson integrators. We show these too can can derive from a FET interpretation, similarly offering potential extensions to higher-order-in-time particle pushers. The FET formulation is used also to consider how the stochastic drift terms can be incorporated into the pushers. Stochastic gyrokinetic expansions are also discussed.

        Different options for the numerical implementation of these schemes are considered.

        Due to the efficacy of FET in the development of SP timesteppers for both the fluid and kinetic component, we hope this approach will prove effective in the future for developing SP timesteppers for the full hybrid model. We hope this will give us the opportunity to incorporate previously inaccessible kinetic effects into the highly effective, modern, finite-element MHD models.
    \end{abstract}
    
    
    \newpage
    \tableofcontents
    
    
    \newpage
    \pagenumbering{arabic}
    %\linenumbers\renewcommand\thelinenumber{\color{black!50}\arabic{linenumber}}
            \input{0 - introduction/main.tex}
        \part{Research}
            \input{1 - low-noise PiC models/main.tex}
            \input{2 - kinetic component/main.tex}
            \input{3 - fluid component/main.tex}
            \input{4 - numerical implementation/main.tex}
        \part{Project Overview}
            \input{5 - research plan/main.tex}
            \input{6 - summary/main.tex}
    
    
    %\section{}
    \newpage
    \pagenumbering{gobble}
        \printbibliography


    \newpage
    \pagenumbering{roman}
    \appendix
        \part{Appendices}
            \input{8 - Hilbert complexes/main.tex}
            \input{9 - weak conservation proofs/main.tex}
\end{document}

            \documentclass[12pt, a4paper]{report}

\input{template/main.tex}

\title{\BA{Title in Progress...}}
\author{Boris Andrews}
\affil{Mathematical Institute, University of Oxford}
\date{\today}


\begin{document}
    \pagenumbering{gobble}
    \maketitle
    
    
    \begin{abstract}
        Magnetic confinement reactors---in particular tokamaks---offer one of the most promising options for achieving practical nuclear fusion, with the potential to provide virtually limitless, clean energy. The theoretical and numerical modeling of tokamak plasmas is simultaneously an essential component of effective reactor design, and a great research barrier. Tokamak operational conditions exhibit comparatively low Knudsen numbers. Kinetic effects, including kinetic waves and instabilities, Landau damping, bump-on-tail instabilities and more, are therefore highly influential in tokamak plasma dynamics. Purely fluid models are inherently incapable of capturing these effects, whereas the high dimensionality in purely kinetic models render them practically intractable for most relevant purposes.

        We consider a $\delta\!f$ decomposition model, with a macroscopic fluid background and microscopic kinetic correction, both fully coupled to each other. A similar manner of discretization is proposed to that used in the recent \texttt{STRUPHY} code \cite{Holderied_Possanner_Wang_2021, Holderied_2022, Li_et_al_2023} with a finite-element model for the background and a pseudo-particle/PiC model for the correction.

        The fluid background satisfies the full, non-linear, resistive, compressible, Hall MHD equations. \cite{Laakmann_Hu_Farrell_2022} introduces finite-element(-in-space) implicit timesteppers for the incompressible analogue to this system with structure-preserving (SP) properties in the ideal case, alongside parameter-robust preconditioners. We show that these timesteppers can derive from a finite-element-in-time (FET) (and finite-element-in-space) interpretation. The benefits of this reformulation are discussed, including the derivation of timesteppers that are higher order in time, and the quantifiable dissipative SP properties in the non-ideal, resistive case.
        
        We discuss possible options for extending this FET approach to timesteppers for the compressible case.

        The kinetic corrections satisfy linearized Boltzmann equations. Using a Lénard--Bernstein collision operator, these take Fokker--Planck-like forms \cite{Fokker_1914, Planck_1917} wherein pseudo-particles in the numerical model obey the neoclassical transport equations, with particle-independent Brownian drift terms. This offers a rigorous methodology for incorporating collisions into the particle transport model, without coupling the equations of motions for each particle.
        
        Works by Chen, Chacón et al. \cite{Chen_Chacón_Barnes_2011, Chacón_Chen_Barnes_2013, Chen_Chacón_2014, Chen_Chacón_2015} have developed structure-preserving particle pushers for neoclassical transport in the Vlasov equations, derived from Crank--Nicolson integrators. We show these too can can derive from a FET interpretation, similarly offering potential extensions to higher-order-in-time particle pushers. The FET formulation is used also to consider how the stochastic drift terms can be incorporated into the pushers. Stochastic gyrokinetic expansions are also discussed.

        Different options for the numerical implementation of these schemes are considered.

        Due to the efficacy of FET in the development of SP timesteppers for both the fluid and kinetic component, we hope this approach will prove effective in the future for developing SP timesteppers for the full hybrid model. We hope this will give us the opportunity to incorporate previously inaccessible kinetic effects into the highly effective, modern, finite-element MHD models.
    \end{abstract}
    
    
    \newpage
    \tableofcontents
    
    
    \newpage
    \pagenumbering{arabic}
    %\linenumbers\renewcommand\thelinenumber{\color{black!50}\arabic{linenumber}}
            \input{0 - introduction/main.tex}
        \part{Research}
            \input{1 - low-noise PiC models/main.tex}
            \input{2 - kinetic component/main.tex}
            \input{3 - fluid component/main.tex}
            \input{4 - numerical implementation/main.tex}
        \part{Project Overview}
            \input{5 - research plan/main.tex}
            \input{6 - summary/main.tex}
    
    
    %\section{}
    \newpage
    \pagenumbering{gobble}
        \printbibliography


    \newpage
    \pagenumbering{roman}
    \appendix
        \part{Appendices}
            \input{8 - Hilbert complexes/main.tex}
            \input{9 - weak conservation proofs/main.tex}
\end{document}

\end{document}

        \part{Research}
            \documentclass[12pt, a4paper]{report}

\documentclass[12pt, a4paper]{report}

\input{template/main.tex}

\title{\BA{Title in Progress...}}
\author{Boris Andrews}
\affil{Mathematical Institute, University of Oxford}
\date{\today}


\begin{document}
    \pagenumbering{gobble}
    \maketitle
    
    
    \begin{abstract}
        Magnetic confinement reactors---in particular tokamaks---offer one of the most promising options for achieving practical nuclear fusion, with the potential to provide virtually limitless, clean energy. The theoretical and numerical modeling of tokamak plasmas is simultaneously an essential component of effective reactor design, and a great research barrier. Tokamak operational conditions exhibit comparatively low Knudsen numbers. Kinetic effects, including kinetic waves and instabilities, Landau damping, bump-on-tail instabilities and more, are therefore highly influential in tokamak plasma dynamics. Purely fluid models are inherently incapable of capturing these effects, whereas the high dimensionality in purely kinetic models render them practically intractable for most relevant purposes.

        We consider a $\delta\!f$ decomposition model, with a macroscopic fluid background and microscopic kinetic correction, both fully coupled to each other. A similar manner of discretization is proposed to that used in the recent \texttt{STRUPHY} code \cite{Holderied_Possanner_Wang_2021, Holderied_2022, Li_et_al_2023} with a finite-element model for the background and a pseudo-particle/PiC model for the correction.

        The fluid background satisfies the full, non-linear, resistive, compressible, Hall MHD equations. \cite{Laakmann_Hu_Farrell_2022} introduces finite-element(-in-space) implicit timesteppers for the incompressible analogue to this system with structure-preserving (SP) properties in the ideal case, alongside parameter-robust preconditioners. We show that these timesteppers can derive from a finite-element-in-time (FET) (and finite-element-in-space) interpretation. The benefits of this reformulation are discussed, including the derivation of timesteppers that are higher order in time, and the quantifiable dissipative SP properties in the non-ideal, resistive case.
        
        We discuss possible options for extending this FET approach to timesteppers for the compressible case.

        The kinetic corrections satisfy linearized Boltzmann equations. Using a Lénard--Bernstein collision operator, these take Fokker--Planck-like forms \cite{Fokker_1914, Planck_1917} wherein pseudo-particles in the numerical model obey the neoclassical transport equations, with particle-independent Brownian drift terms. This offers a rigorous methodology for incorporating collisions into the particle transport model, without coupling the equations of motions for each particle.
        
        Works by Chen, Chacón et al. \cite{Chen_Chacón_Barnes_2011, Chacón_Chen_Barnes_2013, Chen_Chacón_2014, Chen_Chacón_2015} have developed structure-preserving particle pushers for neoclassical transport in the Vlasov equations, derived from Crank--Nicolson integrators. We show these too can can derive from a FET interpretation, similarly offering potential extensions to higher-order-in-time particle pushers. The FET formulation is used also to consider how the stochastic drift terms can be incorporated into the pushers. Stochastic gyrokinetic expansions are also discussed.

        Different options for the numerical implementation of these schemes are considered.

        Due to the efficacy of FET in the development of SP timesteppers for both the fluid and kinetic component, we hope this approach will prove effective in the future for developing SP timesteppers for the full hybrid model. We hope this will give us the opportunity to incorporate previously inaccessible kinetic effects into the highly effective, modern, finite-element MHD models.
    \end{abstract}
    
    
    \newpage
    \tableofcontents
    
    
    \newpage
    \pagenumbering{arabic}
    %\linenumbers\renewcommand\thelinenumber{\color{black!50}\arabic{linenumber}}
            \input{0 - introduction/main.tex}
        \part{Research}
            \input{1 - low-noise PiC models/main.tex}
            \input{2 - kinetic component/main.tex}
            \input{3 - fluid component/main.tex}
            \input{4 - numerical implementation/main.tex}
        \part{Project Overview}
            \input{5 - research plan/main.tex}
            \input{6 - summary/main.tex}
    
    
    %\section{}
    \newpage
    \pagenumbering{gobble}
        \printbibliography


    \newpage
    \pagenumbering{roman}
    \appendix
        \part{Appendices}
            \input{8 - Hilbert complexes/main.tex}
            \input{9 - weak conservation proofs/main.tex}
\end{document}


\title{\BA{Title in Progress...}}
\author{Boris Andrews}
\affil{Mathematical Institute, University of Oxford}
\date{\today}


\begin{document}
    \pagenumbering{gobble}
    \maketitle
    
    
    \begin{abstract}
        Magnetic confinement reactors---in particular tokamaks---offer one of the most promising options for achieving practical nuclear fusion, with the potential to provide virtually limitless, clean energy. The theoretical and numerical modeling of tokamak plasmas is simultaneously an essential component of effective reactor design, and a great research barrier. Tokamak operational conditions exhibit comparatively low Knudsen numbers. Kinetic effects, including kinetic waves and instabilities, Landau damping, bump-on-tail instabilities and more, are therefore highly influential in tokamak plasma dynamics. Purely fluid models are inherently incapable of capturing these effects, whereas the high dimensionality in purely kinetic models render them practically intractable for most relevant purposes.

        We consider a $\delta\!f$ decomposition model, with a macroscopic fluid background and microscopic kinetic correction, both fully coupled to each other. A similar manner of discretization is proposed to that used in the recent \texttt{STRUPHY} code \cite{Holderied_Possanner_Wang_2021, Holderied_2022, Li_et_al_2023} with a finite-element model for the background and a pseudo-particle/PiC model for the correction.

        The fluid background satisfies the full, non-linear, resistive, compressible, Hall MHD equations. \cite{Laakmann_Hu_Farrell_2022} introduces finite-element(-in-space) implicit timesteppers for the incompressible analogue to this system with structure-preserving (SP) properties in the ideal case, alongside parameter-robust preconditioners. We show that these timesteppers can derive from a finite-element-in-time (FET) (and finite-element-in-space) interpretation. The benefits of this reformulation are discussed, including the derivation of timesteppers that are higher order in time, and the quantifiable dissipative SP properties in the non-ideal, resistive case.
        
        We discuss possible options for extending this FET approach to timesteppers for the compressible case.

        The kinetic corrections satisfy linearized Boltzmann equations. Using a Lénard--Bernstein collision operator, these take Fokker--Planck-like forms \cite{Fokker_1914, Planck_1917} wherein pseudo-particles in the numerical model obey the neoclassical transport equations, with particle-independent Brownian drift terms. This offers a rigorous methodology for incorporating collisions into the particle transport model, without coupling the equations of motions for each particle.
        
        Works by Chen, Chacón et al. \cite{Chen_Chacón_Barnes_2011, Chacón_Chen_Barnes_2013, Chen_Chacón_2014, Chen_Chacón_2015} have developed structure-preserving particle pushers for neoclassical transport in the Vlasov equations, derived from Crank--Nicolson integrators. We show these too can can derive from a FET interpretation, similarly offering potential extensions to higher-order-in-time particle pushers. The FET formulation is used also to consider how the stochastic drift terms can be incorporated into the pushers. Stochastic gyrokinetic expansions are also discussed.

        Different options for the numerical implementation of these schemes are considered.

        Due to the efficacy of FET in the development of SP timesteppers for both the fluid and kinetic component, we hope this approach will prove effective in the future for developing SP timesteppers for the full hybrid model. We hope this will give us the opportunity to incorporate previously inaccessible kinetic effects into the highly effective, modern, finite-element MHD models.
    \end{abstract}
    
    
    \newpage
    \tableofcontents
    
    
    \newpage
    \pagenumbering{arabic}
    %\linenumbers\renewcommand\thelinenumber{\color{black!50}\arabic{linenumber}}
            \documentclass[12pt, a4paper]{report}

\input{template/main.tex}

\title{\BA{Title in Progress...}}
\author{Boris Andrews}
\affil{Mathematical Institute, University of Oxford}
\date{\today}


\begin{document}
    \pagenumbering{gobble}
    \maketitle
    
    
    \begin{abstract}
        Magnetic confinement reactors---in particular tokamaks---offer one of the most promising options for achieving practical nuclear fusion, with the potential to provide virtually limitless, clean energy. The theoretical and numerical modeling of tokamak plasmas is simultaneously an essential component of effective reactor design, and a great research barrier. Tokamak operational conditions exhibit comparatively low Knudsen numbers. Kinetic effects, including kinetic waves and instabilities, Landau damping, bump-on-tail instabilities and more, are therefore highly influential in tokamak plasma dynamics. Purely fluid models are inherently incapable of capturing these effects, whereas the high dimensionality in purely kinetic models render them practically intractable for most relevant purposes.

        We consider a $\delta\!f$ decomposition model, with a macroscopic fluid background and microscopic kinetic correction, both fully coupled to each other. A similar manner of discretization is proposed to that used in the recent \texttt{STRUPHY} code \cite{Holderied_Possanner_Wang_2021, Holderied_2022, Li_et_al_2023} with a finite-element model for the background and a pseudo-particle/PiC model for the correction.

        The fluid background satisfies the full, non-linear, resistive, compressible, Hall MHD equations. \cite{Laakmann_Hu_Farrell_2022} introduces finite-element(-in-space) implicit timesteppers for the incompressible analogue to this system with structure-preserving (SP) properties in the ideal case, alongside parameter-robust preconditioners. We show that these timesteppers can derive from a finite-element-in-time (FET) (and finite-element-in-space) interpretation. The benefits of this reformulation are discussed, including the derivation of timesteppers that are higher order in time, and the quantifiable dissipative SP properties in the non-ideal, resistive case.
        
        We discuss possible options for extending this FET approach to timesteppers for the compressible case.

        The kinetic corrections satisfy linearized Boltzmann equations. Using a Lénard--Bernstein collision operator, these take Fokker--Planck-like forms \cite{Fokker_1914, Planck_1917} wherein pseudo-particles in the numerical model obey the neoclassical transport equations, with particle-independent Brownian drift terms. This offers a rigorous methodology for incorporating collisions into the particle transport model, without coupling the equations of motions for each particle.
        
        Works by Chen, Chacón et al. \cite{Chen_Chacón_Barnes_2011, Chacón_Chen_Barnes_2013, Chen_Chacón_2014, Chen_Chacón_2015} have developed structure-preserving particle pushers for neoclassical transport in the Vlasov equations, derived from Crank--Nicolson integrators. We show these too can can derive from a FET interpretation, similarly offering potential extensions to higher-order-in-time particle pushers. The FET formulation is used also to consider how the stochastic drift terms can be incorporated into the pushers. Stochastic gyrokinetic expansions are also discussed.

        Different options for the numerical implementation of these schemes are considered.

        Due to the efficacy of FET in the development of SP timesteppers for both the fluid and kinetic component, we hope this approach will prove effective in the future for developing SP timesteppers for the full hybrid model. We hope this will give us the opportunity to incorporate previously inaccessible kinetic effects into the highly effective, modern, finite-element MHD models.
    \end{abstract}
    
    
    \newpage
    \tableofcontents
    
    
    \newpage
    \pagenumbering{arabic}
    %\linenumbers\renewcommand\thelinenumber{\color{black!50}\arabic{linenumber}}
            \input{0 - introduction/main.tex}
        \part{Research}
            \input{1 - low-noise PiC models/main.tex}
            \input{2 - kinetic component/main.tex}
            \input{3 - fluid component/main.tex}
            \input{4 - numerical implementation/main.tex}
        \part{Project Overview}
            \input{5 - research plan/main.tex}
            \input{6 - summary/main.tex}
    
    
    %\section{}
    \newpage
    \pagenumbering{gobble}
        \printbibliography


    \newpage
    \pagenumbering{roman}
    \appendix
        \part{Appendices}
            \input{8 - Hilbert complexes/main.tex}
            \input{9 - weak conservation proofs/main.tex}
\end{document}

        \part{Research}
            \documentclass[12pt, a4paper]{report}

\input{template/main.tex}

\title{\BA{Title in Progress...}}
\author{Boris Andrews}
\affil{Mathematical Institute, University of Oxford}
\date{\today}


\begin{document}
    \pagenumbering{gobble}
    \maketitle
    
    
    \begin{abstract}
        Magnetic confinement reactors---in particular tokamaks---offer one of the most promising options for achieving practical nuclear fusion, with the potential to provide virtually limitless, clean energy. The theoretical and numerical modeling of tokamak plasmas is simultaneously an essential component of effective reactor design, and a great research barrier. Tokamak operational conditions exhibit comparatively low Knudsen numbers. Kinetic effects, including kinetic waves and instabilities, Landau damping, bump-on-tail instabilities and more, are therefore highly influential in tokamak plasma dynamics. Purely fluid models are inherently incapable of capturing these effects, whereas the high dimensionality in purely kinetic models render them practically intractable for most relevant purposes.

        We consider a $\delta\!f$ decomposition model, with a macroscopic fluid background and microscopic kinetic correction, both fully coupled to each other. A similar manner of discretization is proposed to that used in the recent \texttt{STRUPHY} code \cite{Holderied_Possanner_Wang_2021, Holderied_2022, Li_et_al_2023} with a finite-element model for the background and a pseudo-particle/PiC model for the correction.

        The fluid background satisfies the full, non-linear, resistive, compressible, Hall MHD equations. \cite{Laakmann_Hu_Farrell_2022} introduces finite-element(-in-space) implicit timesteppers for the incompressible analogue to this system with structure-preserving (SP) properties in the ideal case, alongside parameter-robust preconditioners. We show that these timesteppers can derive from a finite-element-in-time (FET) (and finite-element-in-space) interpretation. The benefits of this reformulation are discussed, including the derivation of timesteppers that are higher order in time, and the quantifiable dissipative SP properties in the non-ideal, resistive case.
        
        We discuss possible options for extending this FET approach to timesteppers for the compressible case.

        The kinetic corrections satisfy linearized Boltzmann equations. Using a Lénard--Bernstein collision operator, these take Fokker--Planck-like forms \cite{Fokker_1914, Planck_1917} wherein pseudo-particles in the numerical model obey the neoclassical transport equations, with particle-independent Brownian drift terms. This offers a rigorous methodology for incorporating collisions into the particle transport model, without coupling the equations of motions for each particle.
        
        Works by Chen, Chacón et al. \cite{Chen_Chacón_Barnes_2011, Chacón_Chen_Barnes_2013, Chen_Chacón_2014, Chen_Chacón_2015} have developed structure-preserving particle pushers for neoclassical transport in the Vlasov equations, derived from Crank--Nicolson integrators. We show these too can can derive from a FET interpretation, similarly offering potential extensions to higher-order-in-time particle pushers. The FET formulation is used also to consider how the stochastic drift terms can be incorporated into the pushers. Stochastic gyrokinetic expansions are also discussed.

        Different options for the numerical implementation of these schemes are considered.

        Due to the efficacy of FET in the development of SP timesteppers for both the fluid and kinetic component, we hope this approach will prove effective in the future for developing SP timesteppers for the full hybrid model. We hope this will give us the opportunity to incorporate previously inaccessible kinetic effects into the highly effective, modern, finite-element MHD models.
    \end{abstract}
    
    
    \newpage
    \tableofcontents
    
    
    \newpage
    \pagenumbering{arabic}
    %\linenumbers\renewcommand\thelinenumber{\color{black!50}\arabic{linenumber}}
            \input{0 - introduction/main.tex}
        \part{Research}
            \input{1 - low-noise PiC models/main.tex}
            \input{2 - kinetic component/main.tex}
            \input{3 - fluid component/main.tex}
            \input{4 - numerical implementation/main.tex}
        \part{Project Overview}
            \input{5 - research plan/main.tex}
            \input{6 - summary/main.tex}
    
    
    %\section{}
    \newpage
    \pagenumbering{gobble}
        \printbibliography


    \newpage
    \pagenumbering{roman}
    \appendix
        \part{Appendices}
            \input{8 - Hilbert complexes/main.tex}
            \input{9 - weak conservation proofs/main.tex}
\end{document}

            \documentclass[12pt, a4paper]{report}

\input{template/main.tex}

\title{\BA{Title in Progress...}}
\author{Boris Andrews}
\affil{Mathematical Institute, University of Oxford}
\date{\today}


\begin{document}
    \pagenumbering{gobble}
    \maketitle
    
    
    \begin{abstract}
        Magnetic confinement reactors---in particular tokamaks---offer one of the most promising options for achieving practical nuclear fusion, with the potential to provide virtually limitless, clean energy. The theoretical and numerical modeling of tokamak plasmas is simultaneously an essential component of effective reactor design, and a great research barrier. Tokamak operational conditions exhibit comparatively low Knudsen numbers. Kinetic effects, including kinetic waves and instabilities, Landau damping, bump-on-tail instabilities and more, are therefore highly influential in tokamak plasma dynamics. Purely fluid models are inherently incapable of capturing these effects, whereas the high dimensionality in purely kinetic models render them practically intractable for most relevant purposes.

        We consider a $\delta\!f$ decomposition model, with a macroscopic fluid background and microscopic kinetic correction, both fully coupled to each other. A similar manner of discretization is proposed to that used in the recent \texttt{STRUPHY} code \cite{Holderied_Possanner_Wang_2021, Holderied_2022, Li_et_al_2023} with a finite-element model for the background and a pseudo-particle/PiC model for the correction.

        The fluid background satisfies the full, non-linear, resistive, compressible, Hall MHD equations. \cite{Laakmann_Hu_Farrell_2022} introduces finite-element(-in-space) implicit timesteppers for the incompressible analogue to this system with structure-preserving (SP) properties in the ideal case, alongside parameter-robust preconditioners. We show that these timesteppers can derive from a finite-element-in-time (FET) (and finite-element-in-space) interpretation. The benefits of this reformulation are discussed, including the derivation of timesteppers that are higher order in time, and the quantifiable dissipative SP properties in the non-ideal, resistive case.
        
        We discuss possible options for extending this FET approach to timesteppers for the compressible case.

        The kinetic corrections satisfy linearized Boltzmann equations. Using a Lénard--Bernstein collision operator, these take Fokker--Planck-like forms \cite{Fokker_1914, Planck_1917} wherein pseudo-particles in the numerical model obey the neoclassical transport equations, with particle-independent Brownian drift terms. This offers a rigorous methodology for incorporating collisions into the particle transport model, without coupling the equations of motions for each particle.
        
        Works by Chen, Chacón et al. \cite{Chen_Chacón_Barnes_2011, Chacón_Chen_Barnes_2013, Chen_Chacón_2014, Chen_Chacón_2015} have developed structure-preserving particle pushers for neoclassical transport in the Vlasov equations, derived from Crank--Nicolson integrators. We show these too can can derive from a FET interpretation, similarly offering potential extensions to higher-order-in-time particle pushers. The FET formulation is used also to consider how the stochastic drift terms can be incorporated into the pushers. Stochastic gyrokinetic expansions are also discussed.

        Different options for the numerical implementation of these schemes are considered.

        Due to the efficacy of FET in the development of SP timesteppers for both the fluid and kinetic component, we hope this approach will prove effective in the future for developing SP timesteppers for the full hybrid model. We hope this will give us the opportunity to incorporate previously inaccessible kinetic effects into the highly effective, modern, finite-element MHD models.
    \end{abstract}
    
    
    \newpage
    \tableofcontents
    
    
    \newpage
    \pagenumbering{arabic}
    %\linenumbers\renewcommand\thelinenumber{\color{black!50}\arabic{linenumber}}
            \input{0 - introduction/main.tex}
        \part{Research}
            \input{1 - low-noise PiC models/main.tex}
            \input{2 - kinetic component/main.tex}
            \input{3 - fluid component/main.tex}
            \input{4 - numerical implementation/main.tex}
        \part{Project Overview}
            \input{5 - research plan/main.tex}
            \input{6 - summary/main.tex}
    
    
    %\section{}
    \newpage
    \pagenumbering{gobble}
        \printbibliography


    \newpage
    \pagenumbering{roman}
    \appendix
        \part{Appendices}
            \input{8 - Hilbert complexes/main.tex}
            \input{9 - weak conservation proofs/main.tex}
\end{document}

            \documentclass[12pt, a4paper]{report}

\input{template/main.tex}

\title{\BA{Title in Progress...}}
\author{Boris Andrews}
\affil{Mathematical Institute, University of Oxford}
\date{\today}


\begin{document}
    \pagenumbering{gobble}
    \maketitle
    
    
    \begin{abstract}
        Magnetic confinement reactors---in particular tokamaks---offer one of the most promising options for achieving practical nuclear fusion, with the potential to provide virtually limitless, clean energy. The theoretical and numerical modeling of tokamak plasmas is simultaneously an essential component of effective reactor design, and a great research barrier. Tokamak operational conditions exhibit comparatively low Knudsen numbers. Kinetic effects, including kinetic waves and instabilities, Landau damping, bump-on-tail instabilities and more, are therefore highly influential in tokamak plasma dynamics. Purely fluid models are inherently incapable of capturing these effects, whereas the high dimensionality in purely kinetic models render them practically intractable for most relevant purposes.

        We consider a $\delta\!f$ decomposition model, with a macroscopic fluid background and microscopic kinetic correction, both fully coupled to each other. A similar manner of discretization is proposed to that used in the recent \texttt{STRUPHY} code \cite{Holderied_Possanner_Wang_2021, Holderied_2022, Li_et_al_2023} with a finite-element model for the background and a pseudo-particle/PiC model for the correction.

        The fluid background satisfies the full, non-linear, resistive, compressible, Hall MHD equations. \cite{Laakmann_Hu_Farrell_2022} introduces finite-element(-in-space) implicit timesteppers for the incompressible analogue to this system with structure-preserving (SP) properties in the ideal case, alongside parameter-robust preconditioners. We show that these timesteppers can derive from a finite-element-in-time (FET) (and finite-element-in-space) interpretation. The benefits of this reformulation are discussed, including the derivation of timesteppers that are higher order in time, and the quantifiable dissipative SP properties in the non-ideal, resistive case.
        
        We discuss possible options for extending this FET approach to timesteppers for the compressible case.

        The kinetic corrections satisfy linearized Boltzmann equations. Using a Lénard--Bernstein collision operator, these take Fokker--Planck-like forms \cite{Fokker_1914, Planck_1917} wherein pseudo-particles in the numerical model obey the neoclassical transport equations, with particle-independent Brownian drift terms. This offers a rigorous methodology for incorporating collisions into the particle transport model, without coupling the equations of motions for each particle.
        
        Works by Chen, Chacón et al. \cite{Chen_Chacón_Barnes_2011, Chacón_Chen_Barnes_2013, Chen_Chacón_2014, Chen_Chacón_2015} have developed structure-preserving particle pushers for neoclassical transport in the Vlasov equations, derived from Crank--Nicolson integrators. We show these too can can derive from a FET interpretation, similarly offering potential extensions to higher-order-in-time particle pushers. The FET formulation is used also to consider how the stochastic drift terms can be incorporated into the pushers. Stochastic gyrokinetic expansions are also discussed.

        Different options for the numerical implementation of these schemes are considered.

        Due to the efficacy of FET in the development of SP timesteppers for both the fluid and kinetic component, we hope this approach will prove effective in the future for developing SP timesteppers for the full hybrid model. We hope this will give us the opportunity to incorporate previously inaccessible kinetic effects into the highly effective, modern, finite-element MHD models.
    \end{abstract}
    
    
    \newpage
    \tableofcontents
    
    
    \newpage
    \pagenumbering{arabic}
    %\linenumbers\renewcommand\thelinenumber{\color{black!50}\arabic{linenumber}}
            \input{0 - introduction/main.tex}
        \part{Research}
            \input{1 - low-noise PiC models/main.tex}
            \input{2 - kinetic component/main.tex}
            \input{3 - fluid component/main.tex}
            \input{4 - numerical implementation/main.tex}
        \part{Project Overview}
            \input{5 - research plan/main.tex}
            \input{6 - summary/main.tex}
    
    
    %\section{}
    \newpage
    \pagenumbering{gobble}
        \printbibliography


    \newpage
    \pagenumbering{roman}
    \appendix
        \part{Appendices}
            \input{8 - Hilbert complexes/main.tex}
            \input{9 - weak conservation proofs/main.tex}
\end{document}

            \documentclass[12pt, a4paper]{report}

\input{template/main.tex}

\title{\BA{Title in Progress...}}
\author{Boris Andrews}
\affil{Mathematical Institute, University of Oxford}
\date{\today}


\begin{document}
    \pagenumbering{gobble}
    \maketitle
    
    
    \begin{abstract}
        Magnetic confinement reactors---in particular tokamaks---offer one of the most promising options for achieving practical nuclear fusion, with the potential to provide virtually limitless, clean energy. The theoretical and numerical modeling of tokamak plasmas is simultaneously an essential component of effective reactor design, and a great research barrier. Tokamak operational conditions exhibit comparatively low Knudsen numbers. Kinetic effects, including kinetic waves and instabilities, Landau damping, bump-on-tail instabilities and more, are therefore highly influential in tokamak plasma dynamics. Purely fluid models are inherently incapable of capturing these effects, whereas the high dimensionality in purely kinetic models render them practically intractable for most relevant purposes.

        We consider a $\delta\!f$ decomposition model, with a macroscopic fluid background and microscopic kinetic correction, both fully coupled to each other. A similar manner of discretization is proposed to that used in the recent \texttt{STRUPHY} code \cite{Holderied_Possanner_Wang_2021, Holderied_2022, Li_et_al_2023} with a finite-element model for the background and a pseudo-particle/PiC model for the correction.

        The fluid background satisfies the full, non-linear, resistive, compressible, Hall MHD equations. \cite{Laakmann_Hu_Farrell_2022} introduces finite-element(-in-space) implicit timesteppers for the incompressible analogue to this system with structure-preserving (SP) properties in the ideal case, alongside parameter-robust preconditioners. We show that these timesteppers can derive from a finite-element-in-time (FET) (and finite-element-in-space) interpretation. The benefits of this reformulation are discussed, including the derivation of timesteppers that are higher order in time, and the quantifiable dissipative SP properties in the non-ideal, resistive case.
        
        We discuss possible options for extending this FET approach to timesteppers for the compressible case.

        The kinetic corrections satisfy linearized Boltzmann equations. Using a Lénard--Bernstein collision operator, these take Fokker--Planck-like forms \cite{Fokker_1914, Planck_1917} wherein pseudo-particles in the numerical model obey the neoclassical transport equations, with particle-independent Brownian drift terms. This offers a rigorous methodology for incorporating collisions into the particle transport model, without coupling the equations of motions for each particle.
        
        Works by Chen, Chacón et al. \cite{Chen_Chacón_Barnes_2011, Chacón_Chen_Barnes_2013, Chen_Chacón_2014, Chen_Chacón_2015} have developed structure-preserving particle pushers for neoclassical transport in the Vlasov equations, derived from Crank--Nicolson integrators. We show these too can can derive from a FET interpretation, similarly offering potential extensions to higher-order-in-time particle pushers. The FET formulation is used also to consider how the stochastic drift terms can be incorporated into the pushers. Stochastic gyrokinetic expansions are also discussed.

        Different options for the numerical implementation of these schemes are considered.

        Due to the efficacy of FET in the development of SP timesteppers for both the fluid and kinetic component, we hope this approach will prove effective in the future for developing SP timesteppers for the full hybrid model. We hope this will give us the opportunity to incorporate previously inaccessible kinetic effects into the highly effective, modern, finite-element MHD models.
    \end{abstract}
    
    
    \newpage
    \tableofcontents
    
    
    \newpage
    \pagenumbering{arabic}
    %\linenumbers\renewcommand\thelinenumber{\color{black!50}\arabic{linenumber}}
            \input{0 - introduction/main.tex}
        \part{Research}
            \input{1 - low-noise PiC models/main.tex}
            \input{2 - kinetic component/main.tex}
            \input{3 - fluid component/main.tex}
            \input{4 - numerical implementation/main.tex}
        \part{Project Overview}
            \input{5 - research plan/main.tex}
            \input{6 - summary/main.tex}
    
    
    %\section{}
    \newpage
    \pagenumbering{gobble}
        \printbibliography


    \newpage
    \pagenumbering{roman}
    \appendix
        \part{Appendices}
            \input{8 - Hilbert complexes/main.tex}
            \input{9 - weak conservation proofs/main.tex}
\end{document}

        \part{Project Overview}
            \documentclass[12pt, a4paper]{report}

\input{template/main.tex}

\title{\BA{Title in Progress...}}
\author{Boris Andrews}
\affil{Mathematical Institute, University of Oxford}
\date{\today}


\begin{document}
    \pagenumbering{gobble}
    \maketitle
    
    
    \begin{abstract}
        Magnetic confinement reactors---in particular tokamaks---offer one of the most promising options for achieving practical nuclear fusion, with the potential to provide virtually limitless, clean energy. The theoretical and numerical modeling of tokamak plasmas is simultaneously an essential component of effective reactor design, and a great research barrier. Tokamak operational conditions exhibit comparatively low Knudsen numbers. Kinetic effects, including kinetic waves and instabilities, Landau damping, bump-on-tail instabilities and more, are therefore highly influential in tokamak plasma dynamics. Purely fluid models are inherently incapable of capturing these effects, whereas the high dimensionality in purely kinetic models render them practically intractable for most relevant purposes.

        We consider a $\delta\!f$ decomposition model, with a macroscopic fluid background and microscopic kinetic correction, both fully coupled to each other. A similar manner of discretization is proposed to that used in the recent \texttt{STRUPHY} code \cite{Holderied_Possanner_Wang_2021, Holderied_2022, Li_et_al_2023} with a finite-element model for the background and a pseudo-particle/PiC model for the correction.

        The fluid background satisfies the full, non-linear, resistive, compressible, Hall MHD equations. \cite{Laakmann_Hu_Farrell_2022} introduces finite-element(-in-space) implicit timesteppers for the incompressible analogue to this system with structure-preserving (SP) properties in the ideal case, alongside parameter-robust preconditioners. We show that these timesteppers can derive from a finite-element-in-time (FET) (and finite-element-in-space) interpretation. The benefits of this reformulation are discussed, including the derivation of timesteppers that are higher order in time, and the quantifiable dissipative SP properties in the non-ideal, resistive case.
        
        We discuss possible options for extending this FET approach to timesteppers for the compressible case.

        The kinetic corrections satisfy linearized Boltzmann equations. Using a Lénard--Bernstein collision operator, these take Fokker--Planck-like forms \cite{Fokker_1914, Planck_1917} wherein pseudo-particles in the numerical model obey the neoclassical transport equations, with particle-independent Brownian drift terms. This offers a rigorous methodology for incorporating collisions into the particle transport model, without coupling the equations of motions for each particle.
        
        Works by Chen, Chacón et al. \cite{Chen_Chacón_Barnes_2011, Chacón_Chen_Barnes_2013, Chen_Chacón_2014, Chen_Chacón_2015} have developed structure-preserving particle pushers for neoclassical transport in the Vlasov equations, derived from Crank--Nicolson integrators. We show these too can can derive from a FET interpretation, similarly offering potential extensions to higher-order-in-time particle pushers. The FET formulation is used also to consider how the stochastic drift terms can be incorporated into the pushers. Stochastic gyrokinetic expansions are also discussed.

        Different options for the numerical implementation of these schemes are considered.

        Due to the efficacy of FET in the development of SP timesteppers for both the fluid and kinetic component, we hope this approach will prove effective in the future for developing SP timesteppers for the full hybrid model. We hope this will give us the opportunity to incorporate previously inaccessible kinetic effects into the highly effective, modern, finite-element MHD models.
    \end{abstract}
    
    
    \newpage
    \tableofcontents
    
    
    \newpage
    \pagenumbering{arabic}
    %\linenumbers\renewcommand\thelinenumber{\color{black!50}\arabic{linenumber}}
            \input{0 - introduction/main.tex}
        \part{Research}
            \input{1 - low-noise PiC models/main.tex}
            \input{2 - kinetic component/main.tex}
            \input{3 - fluid component/main.tex}
            \input{4 - numerical implementation/main.tex}
        \part{Project Overview}
            \input{5 - research plan/main.tex}
            \input{6 - summary/main.tex}
    
    
    %\section{}
    \newpage
    \pagenumbering{gobble}
        \printbibliography


    \newpage
    \pagenumbering{roman}
    \appendix
        \part{Appendices}
            \input{8 - Hilbert complexes/main.tex}
            \input{9 - weak conservation proofs/main.tex}
\end{document}

            \documentclass[12pt, a4paper]{report}

\input{template/main.tex}

\title{\BA{Title in Progress...}}
\author{Boris Andrews}
\affil{Mathematical Institute, University of Oxford}
\date{\today}


\begin{document}
    \pagenumbering{gobble}
    \maketitle
    
    
    \begin{abstract}
        Magnetic confinement reactors---in particular tokamaks---offer one of the most promising options for achieving practical nuclear fusion, with the potential to provide virtually limitless, clean energy. The theoretical and numerical modeling of tokamak plasmas is simultaneously an essential component of effective reactor design, and a great research barrier. Tokamak operational conditions exhibit comparatively low Knudsen numbers. Kinetic effects, including kinetic waves and instabilities, Landau damping, bump-on-tail instabilities and more, are therefore highly influential in tokamak plasma dynamics. Purely fluid models are inherently incapable of capturing these effects, whereas the high dimensionality in purely kinetic models render them practically intractable for most relevant purposes.

        We consider a $\delta\!f$ decomposition model, with a macroscopic fluid background and microscopic kinetic correction, both fully coupled to each other. A similar manner of discretization is proposed to that used in the recent \texttt{STRUPHY} code \cite{Holderied_Possanner_Wang_2021, Holderied_2022, Li_et_al_2023} with a finite-element model for the background and a pseudo-particle/PiC model for the correction.

        The fluid background satisfies the full, non-linear, resistive, compressible, Hall MHD equations. \cite{Laakmann_Hu_Farrell_2022} introduces finite-element(-in-space) implicit timesteppers for the incompressible analogue to this system with structure-preserving (SP) properties in the ideal case, alongside parameter-robust preconditioners. We show that these timesteppers can derive from a finite-element-in-time (FET) (and finite-element-in-space) interpretation. The benefits of this reformulation are discussed, including the derivation of timesteppers that are higher order in time, and the quantifiable dissipative SP properties in the non-ideal, resistive case.
        
        We discuss possible options for extending this FET approach to timesteppers for the compressible case.

        The kinetic corrections satisfy linearized Boltzmann equations. Using a Lénard--Bernstein collision operator, these take Fokker--Planck-like forms \cite{Fokker_1914, Planck_1917} wherein pseudo-particles in the numerical model obey the neoclassical transport equations, with particle-independent Brownian drift terms. This offers a rigorous methodology for incorporating collisions into the particle transport model, without coupling the equations of motions for each particle.
        
        Works by Chen, Chacón et al. \cite{Chen_Chacón_Barnes_2011, Chacón_Chen_Barnes_2013, Chen_Chacón_2014, Chen_Chacón_2015} have developed structure-preserving particle pushers for neoclassical transport in the Vlasov equations, derived from Crank--Nicolson integrators. We show these too can can derive from a FET interpretation, similarly offering potential extensions to higher-order-in-time particle pushers. The FET formulation is used also to consider how the stochastic drift terms can be incorporated into the pushers. Stochastic gyrokinetic expansions are also discussed.

        Different options for the numerical implementation of these schemes are considered.

        Due to the efficacy of FET in the development of SP timesteppers for both the fluid and kinetic component, we hope this approach will prove effective in the future for developing SP timesteppers for the full hybrid model. We hope this will give us the opportunity to incorporate previously inaccessible kinetic effects into the highly effective, modern, finite-element MHD models.
    \end{abstract}
    
    
    \newpage
    \tableofcontents
    
    
    \newpage
    \pagenumbering{arabic}
    %\linenumbers\renewcommand\thelinenumber{\color{black!50}\arabic{linenumber}}
            \input{0 - introduction/main.tex}
        \part{Research}
            \input{1 - low-noise PiC models/main.tex}
            \input{2 - kinetic component/main.tex}
            \input{3 - fluid component/main.tex}
            \input{4 - numerical implementation/main.tex}
        \part{Project Overview}
            \input{5 - research plan/main.tex}
            \input{6 - summary/main.tex}
    
    
    %\section{}
    \newpage
    \pagenumbering{gobble}
        \printbibliography


    \newpage
    \pagenumbering{roman}
    \appendix
        \part{Appendices}
            \input{8 - Hilbert complexes/main.tex}
            \input{9 - weak conservation proofs/main.tex}
\end{document}

    
    
    %\section{}
    \newpage
    \pagenumbering{gobble}
        \printbibliography


    \newpage
    \pagenumbering{roman}
    \appendix
        \part{Appendices}
            \documentclass[12pt, a4paper]{report}

\input{template/main.tex}

\title{\BA{Title in Progress...}}
\author{Boris Andrews}
\affil{Mathematical Institute, University of Oxford}
\date{\today}


\begin{document}
    \pagenumbering{gobble}
    \maketitle
    
    
    \begin{abstract}
        Magnetic confinement reactors---in particular tokamaks---offer one of the most promising options for achieving practical nuclear fusion, with the potential to provide virtually limitless, clean energy. The theoretical and numerical modeling of tokamak plasmas is simultaneously an essential component of effective reactor design, and a great research barrier. Tokamak operational conditions exhibit comparatively low Knudsen numbers. Kinetic effects, including kinetic waves and instabilities, Landau damping, bump-on-tail instabilities and more, are therefore highly influential in tokamak plasma dynamics. Purely fluid models are inherently incapable of capturing these effects, whereas the high dimensionality in purely kinetic models render them practically intractable for most relevant purposes.

        We consider a $\delta\!f$ decomposition model, with a macroscopic fluid background and microscopic kinetic correction, both fully coupled to each other. A similar manner of discretization is proposed to that used in the recent \texttt{STRUPHY} code \cite{Holderied_Possanner_Wang_2021, Holderied_2022, Li_et_al_2023} with a finite-element model for the background and a pseudo-particle/PiC model for the correction.

        The fluid background satisfies the full, non-linear, resistive, compressible, Hall MHD equations. \cite{Laakmann_Hu_Farrell_2022} introduces finite-element(-in-space) implicit timesteppers for the incompressible analogue to this system with structure-preserving (SP) properties in the ideal case, alongside parameter-robust preconditioners. We show that these timesteppers can derive from a finite-element-in-time (FET) (and finite-element-in-space) interpretation. The benefits of this reformulation are discussed, including the derivation of timesteppers that are higher order in time, and the quantifiable dissipative SP properties in the non-ideal, resistive case.
        
        We discuss possible options for extending this FET approach to timesteppers for the compressible case.

        The kinetic corrections satisfy linearized Boltzmann equations. Using a Lénard--Bernstein collision operator, these take Fokker--Planck-like forms \cite{Fokker_1914, Planck_1917} wherein pseudo-particles in the numerical model obey the neoclassical transport equations, with particle-independent Brownian drift terms. This offers a rigorous methodology for incorporating collisions into the particle transport model, without coupling the equations of motions for each particle.
        
        Works by Chen, Chacón et al. \cite{Chen_Chacón_Barnes_2011, Chacón_Chen_Barnes_2013, Chen_Chacón_2014, Chen_Chacón_2015} have developed structure-preserving particle pushers for neoclassical transport in the Vlasov equations, derived from Crank--Nicolson integrators. We show these too can can derive from a FET interpretation, similarly offering potential extensions to higher-order-in-time particle pushers. The FET formulation is used also to consider how the stochastic drift terms can be incorporated into the pushers. Stochastic gyrokinetic expansions are also discussed.

        Different options for the numerical implementation of these schemes are considered.

        Due to the efficacy of FET in the development of SP timesteppers for both the fluid and kinetic component, we hope this approach will prove effective in the future for developing SP timesteppers for the full hybrid model. We hope this will give us the opportunity to incorporate previously inaccessible kinetic effects into the highly effective, modern, finite-element MHD models.
    \end{abstract}
    
    
    \newpage
    \tableofcontents
    
    
    \newpage
    \pagenumbering{arabic}
    %\linenumbers\renewcommand\thelinenumber{\color{black!50}\arabic{linenumber}}
            \input{0 - introduction/main.tex}
        \part{Research}
            \input{1 - low-noise PiC models/main.tex}
            \input{2 - kinetic component/main.tex}
            \input{3 - fluid component/main.tex}
            \input{4 - numerical implementation/main.tex}
        \part{Project Overview}
            \input{5 - research plan/main.tex}
            \input{6 - summary/main.tex}
    
    
    %\section{}
    \newpage
    \pagenumbering{gobble}
        \printbibliography


    \newpage
    \pagenumbering{roman}
    \appendix
        \part{Appendices}
            \input{8 - Hilbert complexes/main.tex}
            \input{9 - weak conservation proofs/main.tex}
\end{document}

            \documentclass[12pt, a4paper]{report}

\input{template/main.tex}

\title{\BA{Title in Progress...}}
\author{Boris Andrews}
\affil{Mathematical Institute, University of Oxford}
\date{\today}


\begin{document}
    \pagenumbering{gobble}
    \maketitle
    
    
    \begin{abstract}
        Magnetic confinement reactors---in particular tokamaks---offer one of the most promising options for achieving practical nuclear fusion, with the potential to provide virtually limitless, clean energy. The theoretical and numerical modeling of tokamak plasmas is simultaneously an essential component of effective reactor design, and a great research barrier. Tokamak operational conditions exhibit comparatively low Knudsen numbers. Kinetic effects, including kinetic waves and instabilities, Landau damping, bump-on-tail instabilities and more, are therefore highly influential in tokamak plasma dynamics. Purely fluid models are inherently incapable of capturing these effects, whereas the high dimensionality in purely kinetic models render them practically intractable for most relevant purposes.

        We consider a $\delta\!f$ decomposition model, with a macroscopic fluid background and microscopic kinetic correction, both fully coupled to each other. A similar manner of discretization is proposed to that used in the recent \texttt{STRUPHY} code \cite{Holderied_Possanner_Wang_2021, Holderied_2022, Li_et_al_2023} with a finite-element model for the background and a pseudo-particle/PiC model for the correction.

        The fluid background satisfies the full, non-linear, resistive, compressible, Hall MHD equations. \cite{Laakmann_Hu_Farrell_2022} introduces finite-element(-in-space) implicit timesteppers for the incompressible analogue to this system with structure-preserving (SP) properties in the ideal case, alongside parameter-robust preconditioners. We show that these timesteppers can derive from a finite-element-in-time (FET) (and finite-element-in-space) interpretation. The benefits of this reformulation are discussed, including the derivation of timesteppers that are higher order in time, and the quantifiable dissipative SP properties in the non-ideal, resistive case.
        
        We discuss possible options for extending this FET approach to timesteppers for the compressible case.

        The kinetic corrections satisfy linearized Boltzmann equations. Using a Lénard--Bernstein collision operator, these take Fokker--Planck-like forms \cite{Fokker_1914, Planck_1917} wherein pseudo-particles in the numerical model obey the neoclassical transport equations, with particle-independent Brownian drift terms. This offers a rigorous methodology for incorporating collisions into the particle transport model, without coupling the equations of motions for each particle.
        
        Works by Chen, Chacón et al. \cite{Chen_Chacón_Barnes_2011, Chacón_Chen_Barnes_2013, Chen_Chacón_2014, Chen_Chacón_2015} have developed structure-preserving particle pushers for neoclassical transport in the Vlasov equations, derived from Crank--Nicolson integrators. We show these too can can derive from a FET interpretation, similarly offering potential extensions to higher-order-in-time particle pushers. The FET formulation is used also to consider how the stochastic drift terms can be incorporated into the pushers. Stochastic gyrokinetic expansions are also discussed.

        Different options for the numerical implementation of these schemes are considered.

        Due to the efficacy of FET in the development of SP timesteppers for both the fluid and kinetic component, we hope this approach will prove effective in the future for developing SP timesteppers for the full hybrid model. We hope this will give us the opportunity to incorporate previously inaccessible kinetic effects into the highly effective, modern, finite-element MHD models.
    \end{abstract}
    
    
    \newpage
    \tableofcontents
    
    
    \newpage
    \pagenumbering{arabic}
    %\linenumbers\renewcommand\thelinenumber{\color{black!50}\arabic{linenumber}}
            \input{0 - introduction/main.tex}
        \part{Research}
            \input{1 - low-noise PiC models/main.tex}
            \input{2 - kinetic component/main.tex}
            \input{3 - fluid component/main.tex}
            \input{4 - numerical implementation/main.tex}
        \part{Project Overview}
            \input{5 - research plan/main.tex}
            \input{6 - summary/main.tex}
    
    
    %\section{}
    \newpage
    \pagenumbering{gobble}
        \printbibliography


    \newpage
    \pagenumbering{roman}
    \appendix
        \part{Appendices}
            \input{8 - Hilbert complexes/main.tex}
            \input{9 - weak conservation proofs/main.tex}
\end{document}

\end{document}

            \documentclass[12pt, a4paper]{report}

\documentclass[12pt, a4paper]{report}

\input{template/main.tex}

\title{\BA{Title in Progress...}}
\author{Boris Andrews}
\affil{Mathematical Institute, University of Oxford}
\date{\today}


\begin{document}
    \pagenumbering{gobble}
    \maketitle
    
    
    \begin{abstract}
        Magnetic confinement reactors---in particular tokamaks---offer one of the most promising options for achieving practical nuclear fusion, with the potential to provide virtually limitless, clean energy. The theoretical and numerical modeling of tokamak plasmas is simultaneously an essential component of effective reactor design, and a great research barrier. Tokamak operational conditions exhibit comparatively low Knudsen numbers. Kinetic effects, including kinetic waves and instabilities, Landau damping, bump-on-tail instabilities and more, are therefore highly influential in tokamak plasma dynamics. Purely fluid models are inherently incapable of capturing these effects, whereas the high dimensionality in purely kinetic models render them practically intractable for most relevant purposes.

        We consider a $\delta\!f$ decomposition model, with a macroscopic fluid background and microscopic kinetic correction, both fully coupled to each other. A similar manner of discretization is proposed to that used in the recent \texttt{STRUPHY} code \cite{Holderied_Possanner_Wang_2021, Holderied_2022, Li_et_al_2023} with a finite-element model for the background and a pseudo-particle/PiC model for the correction.

        The fluid background satisfies the full, non-linear, resistive, compressible, Hall MHD equations. \cite{Laakmann_Hu_Farrell_2022} introduces finite-element(-in-space) implicit timesteppers for the incompressible analogue to this system with structure-preserving (SP) properties in the ideal case, alongside parameter-robust preconditioners. We show that these timesteppers can derive from a finite-element-in-time (FET) (and finite-element-in-space) interpretation. The benefits of this reformulation are discussed, including the derivation of timesteppers that are higher order in time, and the quantifiable dissipative SP properties in the non-ideal, resistive case.
        
        We discuss possible options for extending this FET approach to timesteppers for the compressible case.

        The kinetic corrections satisfy linearized Boltzmann equations. Using a Lénard--Bernstein collision operator, these take Fokker--Planck-like forms \cite{Fokker_1914, Planck_1917} wherein pseudo-particles in the numerical model obey the neoclassical transport equations, with particle-independent Brownian drift terms. This offers a rigorous methodology for incorporating collisions into the particle transport model, without coupling the equations of motions for each particle.
        
        Works by Chen, Chacón et al. \cite{Chen_Chacón_Barnes_2011, Chacón_Chen_Barnes_2013, Chen_Chacón_2014, Chen_Chacón_2015} have developed structure-preserving particle pushers for neoclassical transport in the Vlasov equations, derived from Crank--Nicolson integrators. We show these too can can derive from a FET interpretation, similarly offering potential extensions to higher-order-in-time particle pushers. The FET formulation is used also to consider how the stochastic drift terms can be incorporated into the pushers. Stochastic gyrokinetic expansions are also discussed.

        Different options for the numerical implementation of these schemes are considered.

        Due to the efficacy of FET in the development of SP timesteppers for both the fluid and kinetic component, we hope this approach will prove effective in the future for developing SP timesteppers for the full hybrid model. We hope this will give us the opportunity to incorporate previously inaccessible kinetic effects into the highly effective, modern, finite-element MHD models.
    \end{abstract}
    
    
    \newpage
    \tableofcontents
    
    
    \newpage
    \pagenumbering{arabic}
    %\linenumbers\renewcommand\thelinenumber{\color{black!50}\arabic{linenumber}}
            \input{0 - introduction/main.tex}
        \part{Research}
            \input{1 - low-noise PiC models/main.tex}
            \input{2 - kinetic component/main.tex}
            \input{3 - fluid component/main.tex}
            \input{4 - numerical implementation/main.tex}
        \part{Project Overview}
            \input{5 - research plan/main.tex}
            \input{6 - summary/main.tex}
    
    
    %\section{}
    \newpage
    \pagenumbering{gobble}
        \printbibliography


    \newpage
    \pagenumbering{roman}
    \appendix
        \part{Appendices}
            \input{8 - Hilbert complexes/main.tex}
            \input{9 - weak conservation proofs/main.tex}
\end{document}


\title{\BA{Title in Progress...}}
\author{Boris Andrews}
\affil{Mathematical Institute, University of Oxford}
\date{\today}


\begin{document}
    \pagenumbering{gobble}
    \maketitle
    
    
    \begin{abstract}
        Magnetic confinement reactors---in particular tokamaks---offer one of the most promising options for achieving practical nuclear fusion, with the potential to provide virtually limitless, clean energy. The theoretical and numerical modeling of tokamak plasmas is simultaneously an essential component of effective reactor design, and a great research barrier. Tokamak operational conditions exhibit comparatively low Knudsen numbers. Kinetic effects, including kinetic waves and instabilities, Landau damping, bump-on-tail instabilities and more, are therefore highly influential in tokamak plasma dynamics. Purely fluid models are inherently incapable of capturing these effects, whereas the high dimensionality in purely kinetic models render them practically intractable for most relevant purposes.

        We consider a $\delta\!f$ decomposition model, with a macroscopic fluid background and microscopic kinetic correction, both fully coupled to each other. A similar manner of discretization is proposed to that used in the recent \texttt{STRUPHY} code \cite{Holderied_Possanner_Wang_2021, Holderied_2022, Li_et_al_2023} with a finite-element model for the background and a pseudo-particle/PiC model for the correction.

        The fluid background satisfies the full, non-linear, resistive, compressible, Hall MHD equations. \cite{Laakmann_Hu_Farrell_2022} introduces finite-element(-in-space) implicit timesteppers for the incompressible analogue to this system with structure-preserving (SP) properties in the ideal case, alongside parameter-robust preconditioners. We show that these timesteppers can derive from a finite-element-in-time (FET) (and finite-element-in-space) interpretation. The benefits of this reformulation are discussed, including the derivation of timesteppers that are higher order in time, and the quantifiable dissipative SP properties in the non-ideal, resistive case.
        
        We discuss possible options for extending this FET approach to timesteppers for the compressible case.

        The kinetic corrections satisfy linearized Boltzmann equations. Using a Lénard--Bernstein collision operator, these take Fokker--Planck-like forms \cite{Fokker_1914, Planck_1917} wherein pseudo-particles in the numerical model obey the neoclassical transport equations, with particle-independent Brownian drift terms. This offers a rigorous methodology for incorporating collisions into the particle transport model, without coupling the equations of motions for each particle.
        
        Works by Chen, Chacón et al. \cite{Chen_Chacón_Barnes_2011, Chacón_Chen_Barnes_2013, Chen_Chacón_2014, Chen_Chacón_2015} have developed structure-preserving particle pushers for neoclassical transport in the Vlasov equations, derived from Crank--Nicolson integrators. We show these too can can derive from a FET interpretation, similarly offering potential extensions to higher-order-in-time particle pushers. The FET formulation is used also to consider how the stochastic drift terms can be incorporated into the pushers. Stochastic gyrokinetic expansions are also discussed.

        Different options for the numerical implementation of these schemes are considered.

        Due to the efficacy of FET in the development of SP timesteppers for both the fluid and kinetic component, we hope this approach will prove effective in the future for developing SP timesteppers for the full hybrid model. We hope this will give us the opportunity to incorporate previously inaccessible kinetic effects into the highly effective, modern, finite-element MHD models.
    \end{abstract}
    
    
    \newpage
    \tableofcontents
    
    
    \newpage
    \pagenumbering{arabic}
    %\linenumbers\renewcommand\thelinenumber{\color{black!50}\arabic{linenumber}}
            \documentclass[12pt, a4paper]{report}

\input{template/main.tex}

\title{\BA{Title in Progress...}}
\author{Boris Andrews}
\affil{Mathematical Institute, University of Oxford}
\date{\today}


\begin{document}
    \pagenumbering{gobble}
    \maketitle
    
    
    \begin{abstract}
        Magnetic confinement reactors---in particular tokamaks---offer one of the most promising options for achieving practical nuclear fusion, with the potential to provide virtually limitless, clean energy. The theoretical and numerical modeling of tokamak plasmas is simultaneously an essential component of effective reactor design, and a great research barrier. Tokamak operational conditions exhibit comparatively low Knudsen numbers. Kinetic effects, including kinetic waves and instabilities, Landau damping, bump-on-tail instabilities and more, are therefore highly influential in tokamak plasma dynamics. Purely fluid models are inherently incapable of capturing these effects, whereas the high dimensionality in purely kinetic models render them practically intractable for most relevant purposes.

        We consider a $\delta\!f$ decomposition model, with a macroscopic fluid background and microscopic kinetic correction, both fully coupled to each other. A similar manner of discretization is proposed to that used in the recent \texttt{STRUPHY} code \cite{Holderied_Possanner_Wang_2021, Holderied_2022, Li_et_al_2023} with a finite-element model for the background and a pseudo-particle/PiC model for the correction.

        The fluid background satisfies the full, non-linear, resistive, compressible, Hall MHD equations. \cite{Laakmann_Hu_Farrell_2022} introduces finite-element(-in-space) implicit timesteppers for the incompressible analogue to this system with structure-preserving (SP) properties in the ideal case, alongside parameter-robust preconditioners. We show that these timesteppers can derive from a finite-element-in-time (FET) (and finite-element-in-space) interpretation. The benefits of this reformulation are discussed, including the derivation of timesteppers that are higher order in time, and the quantifiable dissipative SP properties in the non-ideal, resistive case.
        
        We discuss possible options for extending this FET approach to timesteppers for the compressible case.

        The kinetic corrections satisfy linearized Boltzmann equations. Using a Lénard--Bernstein collision operator, these take Fokker--Planck-like forms \cite{Fokker_1914, Planck_1917} wherein pseudo-particles in the numerical model obey the neoclassical transport equations, with particle-independent Brownian drift terms. This offers a rigorous methodology for incorporating collisions into the particle transport model, without coupling the equations of motions for each particle.
        
        Works by Chen, Chacón et al. \cite{Chen_Chacón_Barnes_2011, Chacón_Chen_Barnes_2013, Chen_Chacón_2014, Chen_Chacón_2015} have developed structure-preserving particle pushers for neoclassical transport in the Vlasov equations, derived from Crank--Nicolson integrators. We show these too can can derive from a FET interpretation, similarly offering potential extensions to higher-order-in-time particle pushers. The FET formulation is used also to consider how the stochastic drift terms can be incorporated into the pushers. Stochastic gyrokinetic expansions are also discussed.

        Different options for the numerical implementation of these schemes are considered.

        Due to the efficacy of FET in the development of SP timesteppers for both the fluid and kinetic component, we hope this approach will prove effective in the future for developing SP timesteppers for the full hybrid model. We hope this will give us the opportunity to incorporate previously inaccessible kinetic effects into the highly effective, modern, finite-element MHD models.
    \end{abstract}
    
    
    \newpage
    \tableofcontents
    
    
    \newpage
    \pagenumbering{arabic}
    %\linenumbers\renewcommand\thelinenumber{\color{black!50}\arabic{linenumber}}
            \input{0 - introduction/main.tex}
        \part{Research}
            \input{1 - low-noise PiC models/main.tex}
            \input{2 - kinetic component/main.tex}
            \input{3 - fluid component/main.tex}
            \input{4 - numerical implementation/main.tex}
        \part{Project Overview}
            \input{5 - research plan/main.tex}
            \input{6 - summary/main.tex}
    
    
    %\section{}
    \newpage
    \pagenumbering{gobble}
        \printbibliography


    \newpage
    \pagenumbering{roman}
    \appendix
        \part{Appendices}
            \input{8 - Hilbert complexes/main.tex}
            \input{9 - weak conservation proofs/main.tex}
\end{document}

        \part{Research}
            \documentclass[12pt, a4paper]{report}

\input{template/main.tex}

\title{\BA{Title in Progress...}}
\author{Boris Andrews}
\affil{Mathematical Institute, University of Oxford}
\date{\today}


\begin{document}
    \pagenumbering{gobble}
    \maketitle
    
    
    \begin{abstract}
        Magnetic confinement reactors---in particular tokamaks---offer one of the most promising options for achieving practical nuclear fusion, with the potential to provide virtually limitless, clean energy. The theoretical and numerical modeling of tokamak plasmas is simultaneously an essential component of effective reactor design, and a great research barrier. Tokamak operational conditions exhibit comparatively low Knudsen numbers. Kinetic effects, including kinetic waves and instabilities, Landau damping, bump-on-tail instabilities and more, are therefore highly influential in tokamak plasma dynamics. Purely fluid models are inherently incapable of capturing these effects, whereas the high dimensionality in purely kinetic models render them practically intractable for most relevant purposes.

        We consider a $\delta\!f$ decomposition model, with a macroscopic fluid background and microscopic kinetic correction, both fully coupled to each other. A similar manner of discretization is proposed to that used in the recent \texttt{STRUPHY} code \cite{Holderied_Possanner_Wang_2021, Holderied_2022, Li_et_al_2023} with a finite-element model for the background and a pseudo-particle/PiC model for the correction.

        The fluid background satisfies the full, non-linear, resistive, compressible, Hall MHD equations. \cite{Laakmann_Hu_Farrell_2022} introduces finite-element(-in-space) implicit timesteppers for the incompressible analogue to this system with structure-preserving (SP) properties in the ideal case, alongside parameter-robust preconditioners. We show that these timesteppers can derive from a finite-element-in-time (FET) (and finite-element-in-space) interpretation. The benefits of this reformulation are discussed, including the derivation of timesteppers that are higher order in time, and the quantifiable dissipative SP properties in the non-ideal, resistive case.
        
        We discuss possible options for extending this FET approach to timesteppers for the compressible case.

        The kinetic corrections satisfy linearized Boltzmann equations. Using a Lénard--Bernstein collision operator, these take Fokker--Planck-like forms \cite{Fokker_1914, Planck_1917} wherein pseudo-particles in the numerical model obey the neoclassical transport equations, with particle-independent Brownian drift terms. This offers a rigorous methodology for incorporating collisions into the particle transport model, without coupling the equations of motions for each particle.
        
        Works by Chen, Chacón et al. \cite{Chen_Chacón_Barnes_2011, Chacón_Chen_Barnes_2013, Chen_Chacón_2014, Chen_Chacón_2015} have developed structure-preserving particle pushers for neoclassical transport in the Vlasov equations, derived from Crank--Nicolson integrators. We show these too can can derive from a FET interpretation, similarly offering potential extensions to higher-order-in-time particle pushers. The FET formulation is used also to consider how the stochastic drift terms can be incorporated into the pushers. Stochastic gyrokinetic expansions are also discussed.

        Different options for the numerical implementation of these schemes are considered.

        Due to the efficacy of FET in the development of SP timesteppers for both the fluid and kinetic component, we hope this approach will prove effective in the future for developing SP timesteppers for the full hybrid model. We hope this will give us the opportunity to incorporate previously inaccessible kinetic effects into the highly effective, modern, finite-element MHD models.
    \end{abstract}
    
    
    \newpage
    \tableofcontents
    
    
    \newpage
    \pagenumbering{arabic}
    %\linenumbers\renewcommand\thelinenumber{\color{black!50}\arabic{linenumber}}
            \input{0 - introduction/main.tex}
        \part{Research}
            \input{1 - low-noise PiC models/main.tex}
            \input{2 - kinetic component/main.tex}
            \input{3 - fluid component/main.tex}
            \input{4 - numerical implementation/main.tex}
        \part{Project Overview}
            \input{5 - research plan/main.tex}
            \input{6 - summary/main.tex}
    
    
    %\section{}
    \newpage
    \pagenumbering{gobble}
        \printbibliography


    \newpage
    \pagenumbering{roman}
    \appendix
        \part{Appendices}
            \input{8 - Hilbert complexes/main.tex}
            \input{9 - weak conservation proofs/main.tex}
\end{document}

            \documentclass[12pt, a4paper]{report}

\input{template/main.tex}

\title{\BA{Title in Progress...}}
\author{Boris Andrews}
\affil{Mathematical Institute, University of Oxford}
\date{\today}


\begin{document}
    \pagenumbering{gobble}
    \maketitle
    
    
    \begin{abstract}
        Magnetic confinement reactors---in particular tokamaks---offer one of the most promising options for achieving practical nuclear fusion, with the potential to provide virtually limitless, clean energy. The theoretical and numerical modeling of tokamak plasmas is simultaneously an essential component of effective reactor design, and a great research barrier. Tokamak operational conditions exhibit comparatively low Knudsen numbers. Kinetic effects, including kinetic waves and instabilities, Landau damping, bump-on-tail instabilities and more, are therefore highly influential in tokamak plasma dynamics. Purely fluid models are inherently incapable of capturing these effects, whereas the high dimensionality in purely kinetic models render them practically intractable for most relevant purposes.

        We consider a $\delta\!f$ decomposition model, with a macroscopic fluid background and microscopic kinetic correction, both fully coupled to each other. A similar manner of discretization is proposed to that used in the recent \texttt{STRUPHY} code \cite{Holderied_Possanner_Wang_2021, Holderied_2022, Li_et_al_2023} with a finite-element model for the background and a pseudo-particle/PiC model for the correction.

        The fluid background satisfies the full, non-linear, resistive, compressible, Hall MHD equations. \cite{Laakmann_Hu_Farrell_2022} introduces finite-element(-in-space) implicit timesteppers for the incompressible analogue to this system with structure-preserving (SP) properties in the ideal case, alongside parameter-robust preconditioners. We show that these timesteppers can derive from a finite-element-in-time (FET) (and finite-element-in-space) interpretation. The benefits of this reformulation are discussed, including the derivation of timesteppers that are higher order in time, and the quantifiable dissipative SP properties in the non-ideal, resistive case.
        
        We discuss possible options for extending this FET approach to timesteppers for the compressible case.

        The kinetic corrections satisfy linearized Boltzmann equations. Using a Lénard--Bernstein collision operator, these take Fokker--Planck-like forms \cite{Fokker_1914, Planck_1917} wherein pseudo-particles in the numerical model obey the neoclassical transport equations, with particle-independent Brownian drift terms. This offers a rigorous methodology for incorporating collisions into the particle transport model, without coupling the equations of motions for each particle.
        
        Works by Chen, Chacón et al. \cite{Chen_Chacón_Barnes_2011, Chacón_Chen_Barnes_2013, Chen_Chacón_2014, Chen_Chacón_2015} have developed structure-preserving particle pushers for neoclassical transport in the Vlasov equations, derived from Crank--Nicolson integrators. We show these too can can derive from a FET interpretation, similarly offering potential extensions to higher-order-in-time particle pushers. The FET formulation is used also to consider how the stochastic drift terms can be incorporated into the pushers. Stochastic gyrokinetic expansions are also discussed.

        Different options for the numerical implementation of these schemes are considered.

        Due to the efficacy of FET in the development of SP timesteppers for both the fluid and kinetic component, we hope this approach will prove effective in the future for developing SP timesteppers for the full hybrid model. We hope this will give us the opportunity to incorporate previously inaccessible kinetic effects into the highly effective, modern, finite-element MHD models.
    \end{abstract}
    
    
    \newpage
    \tableofcontents
    
    
    \newpage
    \pagenumbering{arabic}
    %\linenumbers\renewcommand\thelinenumber{\color{black!50}\arabic{linenumber}}
            \input{0 - introduction/main.tex}
        \part{Research}
            \input{1 - low-noise PiC models/main.tex}
            \input{2 - kinetic component/main.tex}
            \input{3 - fluid component/main.tex}
            \input{4 - numerical implementation/main.tex}
        \part{Project Overview}
            \input{5 - research plan/main.tex}
            \input{6 - summary/main.tex}
    
    
    %\section{}
    \newpage
    \pagenumbering{gobble}
        \printbibliography


    \newpage
    \pagenumbering{roman}
    \appendix
        \part{Appendices}
            \input{8 - Hilbert complexes/main.tex}
            \input{9 - weak conservation proofs/main.tex}
\end{document}

            \documentclass[12pt, a4paper]{report}

\input{template/main.tex}

\title{\BA{Title in Progress...}}
\author{Boris Andrews}
\affil{Mathematical Institute, University of Oxford}
\date{\today}


\begin{document}
    \pagenumbering{gobble}
    \maketitle
    
    
    \begin{abstract}
        Magnetic confinement reactors---in particular tokamaks---offer one of the most promising options for achieving practical nuclear fusion, with the potential to provide virtually limitless, clean energy. The theoretical and numerical modeling of tokamak plasmas is simultaneously an essential component of effective reactor design, and a great research barrier. Tokamak operational conditions exhibit comparatively low Knudsen numbers. Kinetic effects, including kinetic waves and instabilities, Landau damping, bump-on-tail instabilities and more, are therefore highly influential in tokamak plasma dynamics. Purely fluid models are inherently incapable of capturing these effects, whereas the high dimensionality in purely kinetic models render them practically intractable for most relevant purposes.

        We consider a $\delta\!f$ decomposition model, with a macroscopic fluid background and microscopic kinetic correction, both fully coupled to each other. A similar manner of discretization is proposed to that used in the recent \texttt{STRUPHY} code \cite{Holderied_Possanner_Wang_2021, Holderied_2022, Li_et_al_2023} with a finite-element model for the background and a pseudo-particle/PiC model for the correction.

        The fluid background satisfies the full, non-linear, resistive, compressible, Hall MHD equations. \cite{Laakmann_Hu_Farrell_2022} introduces finite-element(-in-space) implicit timesteppers for the incompressible analogue to this system with structure-preserving (SP) properties in the ideal case, alongside parameter-robust preconditioners. We show that these timesteppers can derive from a finite-element-in-time (FET) (and finite-element-in-space) interpretation. The benefits of this reformulation are discussed, including the derivation of timesteppers that are higher order in time, and the quantifiable dissipative SP properties in the non-ideal, resistive case.
        
        We discuss possible options for extending this FET approach to timesteppers for the compressible case.

        The kinetic corrections satisfy linearized Boltzmann equations. Using a Lénard--Bernstein collision operator, these take Fokker--Planck-like forms \cite{Fokker_1914, Planck_1917} wherein pseudo-particles in the numerical model obey the neoclassical transport equations, with particle-independent Brownian drift terms. This offers a rigorous methodology for incorporating collisions into the particle transport model, without coupling the equations of motions for each particle.
        
        Works by Chen, Chacón et al. \cite{Chen_Chacón_Barnes_2011, Chacón_Chen_Barnes_2013, Chen_Chacón_2014, Chen_Chacón_2015} have developed structure-preserving particle pushers for neoclassical transport in the Vlasov equations, derived from Crank--Nicolson integrators. We show these too can can derive from a FET interpretation, similarly offering potential extensions to higher-order-in-time particle pushers. The FET formulation is used also to consider how the stochastic drift terms can be incorporated into the pushers. Stochastic gyrokinetic expansions are also discussed.

        Different options for the numerical implementation of these schemes are considered.

        Due to the efficacy of FET in the development of SP timesteppers for both the fluid and kinetic component, we hope this approach will prove effective in the future for developing SP timesteppers for the full hybrid model. We hope this will give us the opportunity to incorporate previously inaccessible kinetic effects into the highly effective, modern, finite-element MHD models.
    \end{abstract}
    
    
    \newpage
    \tableofcontents
    
    
    \newpage
    \pagenumbering{arabic}
    %\linenumbers\renewcommand\thelinenumber{\color{black!50}\arabic{linenumber}}
            \input{0 - introduction/main.tex}
        \part{Research}
            \input{1 - low-noise PiC models/main.tex}
            \input{2 - kinetic component/main.tex}
            \input{3 - fluid component/main.tex}
            \input{4 - numerical implementation/main.tex}
        \part{Project Overview}
            \input{5 - research plan/main.tex}
            \input{6 - summary/main.tex}
    
    
    %\section{}
    \newpage
    \pagenumbering{gobble}
        \printbibliography


    \newpage
    \pagenumbering{roman}
    \appendix
        \part{Appendices}
            \input{8 - Hilbert complexes/main.tex}
            \input{9 - weak conservation proofs/main.tex}
\end{document}

            \documentclass[12pt, a4paper]{report}

\input{template/main.tex}

\title{\BA{Title in Progress...}}
\author{Boris Andrews}
\affil{Mathematical Institute, University of Oxford}
\date{\today}


\begin{document}
    \pagenumbering{gobble}
    \maketitle
    
    
    \begin{abstract}
        Magnetic confinement reactors---in particular tokamaks---offer one of the most promising options for achieving practical nuclear fusion, with the potential to provide virtually limitless, clean energy. The theoretical and numerical modeling of tokamak plasmas is simultaneously an essential component of effective reactor design, and a great research barrier. Tokamak operational conditions exhibit comparatively low Knudsen numbers. Kinetic effects, including kinetic waves and instabilities, Landau damping, bump-on-tail instabilities and more, are therefore highly influential in tokamak plasma dynamics. Purely fluid models are inherently incapable of capturing these effects, whereas the high dimensionality in purely kinetic models render them practically intractable for most relevant purposes.

        We consider a $\delta\!f$ decomposition model, with a macroscopic fluid background and microscopic kinetic correction, both fully coupled to each other. A similar manner of discretization is proposed to that used in the recent \texttt{STRUPHY} code \cite{Holderied_Possanner_Wang_2021, Holderied_2022, Li_et_al_2023} with a finite-element model for the background and a pseudo-particle/PiC model for the correction.

        The fluid background satisfies the full, non-linear, resistive, compressible, Hall MHD equations. \cite{Laakmann_Hu_Farrell_2022} introduces finite-element(-in-space) implicit timesteppers for the incompressible analogue to this system with structure-preserving (SP) properties in the ideal case, alongside parameter-robust preconditioners. We show that these timesteppers can derive from a finite-element-in-time (FET) (and finite-element-in-space) interpretation. The benefits of this reformulation are discussed, including the derivation of timesteppers that are higher order in time, and the quantifiable dissipative SP properties in the non-ideal, resistive case.
        
        We discuss possible options for extending this FET approach to timesteppers for the compressible case.

        The kinetic corrections satisfy linearized Boltzmann equations. Using a Lénard--Bernstein collision operator, these take Fokker--Planck-like forms \cite{Fokker_1914, Planck_1917} wherein pseudo-particles in the numerical model obey the neoclassical transport equations, with particle-independent Brownian drift terms. This offers a rigorous methodology for incorporating collisions into the particle transport model, without coupling the equations of motions for each particle.
        
        Works by Chen, Chacón et al. \cite{Chen_Chacón_Barnes_2011, Chacón_Chen_Barnes_2013, Chen_Chacón_2014, Chen_Chacón_2015} have developed structure-preserving particle pushers for neoclassical transport in the Vlasov equations, derived from Crank--Nicolson integrators. We show these too can can derive from a FET interpretation, similarly offering potential extensions to higher-order-in-time particle pushers. The FET formulation is used also to consider how the stochastic drift terms can be incorporated into the pushers. Stochastic gyrokinetic expansions are also discussed.

        Different options for the numerical implementation of these schemes are considered.

        Due to the efficacy of FET in the development of SP timesteppers for both the fluid and kinetic component, we hope this approach will prove effective in the future for developing SP timesteppers for the full hybrid model. We hope this will give us the opportunity to incorporate previously inaccessible kinetic effects into the highly effective, modern, finite-element MHD models.
    \end{abstract}
    
    
    \newpage
    \tableofcontents
    
    
    \newpage
    \pagenumbering{arabic}
    %\linenumbers\renewcommand\thelinenumber{\color{black!50}\arabic{linenumber}}
            \input{0 - introduction/main.tex}
        \part{Research}
            \input{1 - low-noise PiC models/main.tex}
            \input{2 - kinetic component/main.tex}
            \input{3 - fluid component/main.tex}
            \input{4 - numerical implementation/main.tex}
        \part{Project Overview}
            \input{5 - research plan/main.tex}
            \input{6 - summary/main.tex}
    
    
    %\section{}
    \newpage
    \pagenumbering{gobble}
        \printbibliography


    \newpage
    \pagenumbering{roman}
    \appendix
        \part{Appendices}
            \input{8 - Hilbert complexes/main.tex}
            \input{9 - weak conservation proofs/main.tex}
\end{document}

        \part{Project Overview}
            \documentclass[12pt, a4paper]{report}

\input{template/main.tex}

\title{\BA{Title in Progress...}}
\author{Boris Andrews}
\affil{Mathematical Institute, University of Oxford}
\date{\today}


\begin{document}
    \pagenumbering{gobble}
    \maketitle
    
    
    \begin{abstract}
        Magnetic confinement reactors---in particular tokamaks---offer one of the most promising options for achieving practical nuclear fusion, with the potential to provide virtually limitless, clean energy. The theoretical and numerical modeling of tokamak plasmas is simultaneously an essential component of effective reactor design, and a great research barrier. Tokamak operational conditions exhibit comparatively low Knudsen numbers. Kinetic effects, including kinetic waves and instabilities, Landau damping, bump-on-tail instabilities and more, are therefore highly influential in tokamak plasma dynamics. Purely fluid models are inherently incapable of capturing these effects, whereas the high dimensionality in purely kinetic models render them practically intractable for most relevant purposes.

        We consider a $\delta\!f$ decomposition model, with a macroscopic fluid background and microscopic kinetic correction, both fully coupled to each other. A similar manner of discretization is proposed to that used in the recent \texttt{STRUPHY} code \cite{Holderied_Possanner_Wang_2021, Holderied_2022, Li_et_al_2023} with a finite-element model for the background and a pseudo-particle/PiC model for the correction.

        The fluid background satisfies the full, non-linear, resistive, compressible, Hall MHD equations. \cite{Laakmann_Hu_Farrell_2022} introduces finite-element(-in-space) implicit timesteppers for the incompressible analogue to this system with structure-preserving (SP) properties in the ideal case, alongside parameter-robust preconditioners. We show that these timesteppers can derive from a finite-element-in-time (FET) (and finite-element-in-space) interpretation. The benefits of this reformulation are discussed, including the derivation of timesteppers that are higher order in time, and the quantifiable dissipative SP properties in the non-ideal, resistive case.
        
        We discuss possible options for extending this FET approach to timesteppers for the compressible case.

        The kinetic corrections satisfy linearized Boltzmann equations. Using a Lénard--Bernstein collision operator, these take Fokker--Planck-like forms \cite{Fokker_1914, Planck_1917} wherein pseudo-particles in the numerical model obey the neoclassical transport equations, with particle-independent Brownian drift terms. This offers a rigorous methodology for incorporating collisions into the particle transport model, without coupling the equations of motions for each particle.
        
        Works by Chen, Chacón et al. \cite{Chen_Chacón_Barnes_2011, Chacón_Chen_Barnes_2013, Chen_Chacón_2014, Chen_Chacón_2015} have developed structure-preserving particle pushers for neoclassical transport in the Vlasov equations, derived from Crank--Nicolson integrators. We show these too can can derive from a FET interpretation, similarly offering potential extensions to higher-order-in-time particle pushers. The FET formulation is used also to consider how the stochastic drift terms can be incorporated into the pushers. Stochastic gyrokinetic expansions are also discussed.

        Different options for the numerical implementation of these schemes are considered.

        Due to the efficacy of FET in the development of SP timesteppers for both the fluid and kinetic component, we hope this approach will prove effective in the future for developing SP timesteppers for the full hybrid model. We hope this will give us the opportunity to incorporate previously inaccessible kinetic effects into the highly effective, modern, finite-element MHD models.
    \end{abstract}
    
    
    \newpage
    \tableofcontents
    
    
    \newpage
    \pagenumbering{arabic}
    %\linenumbers\renewcommand\thelinenumber{\color{black!50}\arabic{linenumber}}
            \input{0 - introduction/main.tex}
        \part{Research}
            \input{1 - low-noise PiC models/main.tex}
            \input{2 - kinetic component/main.tex}
            \input{3 - fluid component/main.tex}
            \input{4 - numerical implementation/main.tex}
        \part{Project Overview}
            \input{5 - research plan/main.tex}
            \input{6 - summary/main.tex}
    
    
    %\section{}
    \newpage
    \pagenumbering{gobble}
        \printbibliography


    \newpage
    \pagenumbering{roman}
    \appendix
        \part{Appendices}
            \input{8 - Hilbert complexes/main.tex}
            \input{9 - weak conservation proofs/main.tex}
\end{document}

            \documentclass[12pt, a4paper]{report}

\input{template/main.tex}

\title{\BA{Title in Progress...}}
\author{Boris Andrews}
\affil{Mathematical Institute, University of Oxford}
\date{\today}


\begin{document}
    \pagenumbering{gobble}
    \maketitle
    
    
    \begin{abstract}
        Magnetic confinement reactors---in particular tokamaks---offer one of the most promising options for achieving practical nuclear fusion, with the potential to provide virtually limitless, clean energy. The theoretical and numerical modeling of tokamak plasmas is simultaneously an essential component of effective reactor design, and a great research barrier. Tokamak operational conditions exhibit comparatively low Knudsen numbers. Kinetic effects, including kinetic waves and instabilities, Landau damping, bump-on-tail instabilities and more, are therefore highly influential in tokamak plasma dynamics. Purely fluid models are inherently incapable of capturing these effects, whereas the high dimensionality in purely kinetic models render them practically intractable for most relevant purposes.

        We consider a $\delta\!f$ decomposition model, with a macroscopic fluid background and microscopic kinetic correction, both fully coupled to each other. A similar manner of discretization is proposed to that used in the recent \texttt{STRUPHY} code \cite{Holderied_Possanner_Wang_2021, Holderied_2022, Li_et_al_2023} with a finite-element model for the background and a pseudo-particle/PiC model for the correction.

        The fluid background satisfies the full, non-linear, resistive, compressible, Hall MHD equations. \cite{Laakmann_Hu_Farrell_2022} introduces finite-element(-in-space) implicit timesteppers for the incompressible analogue to this system with structure-preserving (SP) properties in the ideal case, alongside parameter-robust preconditioners. We show that these timesteppers can derive from a finite-element-in-time (FET) (and finite-element-in-space) interpretation. The benefits of this reformulation are discussed, including the derivation of timesteppers that are higher order in time, and the quantifiable dissipative SP properties in the non-ideal, resistive case.
        
        We discuss possible options for extending this FET approach to timesteppers for the compressible case.

        The kinetic corrections satisfy linearized Boltzmann equations. Using a Lénard--Bernstein collision operator, these take Fokker--Planck-like forms \cite{Fokker_1914, Planck_1917} wherein pseudo-particles in the numerical model obey the neoclassical transport equations, with particle-independent Brownian drift terms. This offers a rigorous methodology for incorporating collisions into the particle transport model, without coupling the equations of motions for each particle.
        
        Works by Chen, Chacón et al. \cite{Chen_Chacón_Barnes_2011, Chacón_Chen_Barnes_2013, Chen_Chacón_2014, Chen_Chacón_2015} have developed structure-preserving particle pushers for neoclassical transport in the Vlasov equations, derived from Crank--Nicolson integrators. We show these too can can derive from a FET interpretation, similarly offering potential extensions to higher-order-in-time particle pushers. The FET formulation is used also to consider how the stochastic drift terms can be incorporated into the pushers. Stochastic gyrokinetic expansions are also discussed.

        Different options for the numerical implementation of these schemes are considered.

        Due to the efficacy of FET in the development of SP timesteppers for both the fluid and kinetic component, we hope this approach will prove effective in the future for developing SP timesteppers for the full hybrid model. We hope this will give us the opportunity to incorporate previously inaccessible kinetic effects into the highly effective, modern, finite-element MHD models.
    \end{abstract}
    
    
    \newpage
    \tableofcontents
    
    
    \newpage
    \pagenumbering{arabic}
    %\linenumbers\renewcommand\thelinenumber{\color{black!50}\arabic{linenumber}}
            \input{0 - introduction/main.tex}
        \part{Research}
            \input{1 - low-noise PiC models/main.tex}
            \input{2 - kinetic component/main.tex}
            \input{3 - fluid component/main.tex}
            \input{4 - numerical implementation/main.tex}
        \part{Project Overview}
            \input{5 - research plan/main.tex}
            \input{6 - summary/main.tex}
    
    
    %\section{}
    \newpage
    \pagenumbering{gobble}
        \printbibliography


    \newpage
    \pagenumbering{roman}
    \appendix
        \part{Appendices}
            \input{8 - Hilbert complexes/main.tex}
            \input{9 - weak conservation proofs/main.tex}
\end{document}

    
    
    %\section{}
    \newpage
    \pagenumbering{gobble}
        \printbibliography


    \newpage
    \pagenumbering{roman}
    \appendix
        \part{Appendices}
            \documentclass[12pt, a4paper]{report}

\input{template/main.tex}

\title{\BA{Title in Progress...}}
\author{Boris Andrews}
\affil{Mathematical Institute, University of Oxford}
\date{\today}


\begin{document}
    \pagenumbering{gobble}
    \maketitle
    
    
    \begin{abstract}
        Magnetic confinement reactors---in particular tokamaks---offer one of the most promising options for achieving practical nuclear fusion, with the potential to provide virtually limitless, clean energy. The theoretical and numerical modeling of tokamak plasmas is simultaneously an essential component of effective reactor design, and a great research barrier. Tokamak operational conditions exhibit comparatively low Knudsen numbers. Kinetic effects, including kinetic waves and instabilities, Landau damping, bump-on-tail instabilities and more, are therefore highly influential in tokamak plasma dynamics. Purely fluid models are inherently incapable of capturing these effects, whereas the high dimensionality in purely kinetic models render them practically intractable for most relevant purposes.

        We consider a $\delta\!f$ decomposition model, with a macroscopic fluid background and microscopic kinetic correction, both fully coupled to each other. A similar manner of discretization is proposed to that used in the recent \texttt{STRUPHY} code \cite{Holderied_Possanner_Wang_2021, Holderied_2022, Li_et_al_2023} with a finite-element model for the background and a pseudo-particle/PiC model for the correction.

        The fluid background satisfies the full, non-linear, resistive, compressible, Hall MHD equations. \cite{Laakmann_Hu_Farrell_2022} introduces finite-element(-in-space) implicit timesteppers for the incompressible analogue to this system with structure-preserving (SP) properties in the ideal case, alongside parameter-robust preconditioners. We show that these timesteppers can derive from a finite-element-in-time (FET) (and finite-element-in-space) interpretation. The benefits of this reformulation are discussed, including the derivation of timesteppers that are higher order in time, and the quantifiable dissipative SP properties in the non-ideal, resistive case.
        
        We discuss possible options for extending this FET approach to timesteppers for the compressible case.

        The kinetic corrections satisfy linearized Boltzmann equations. Using a Lénard--Bernstein collision operator, these take Fokker--Planck-like forms \cite{Fokker_1914, Planck_1917} wherein pseudo-particles in the numerical model obey the neoclassical transport equations, with particle-independent Brownian drift terms. This offers a rigorous methodology for incorporating collisions into the particle transport model, without coupling the equations of motions for each particle.
        
        Works by Chen, Chacón et al. \cite{Chen_Chacón_Barnes_2011, Chacón_Chen_Barnes_2013, Chen_Chacón_2014, Chen_Chacón_2015} have developed structure-preserving particle pushers for neoclassical transport in the Vlasov equations, derived from Crank--Nicolson integrators. We show these too can can derive from a FET interpretation, similarly offering potential extensions to higher-order-in-time particle pushers. The FET formulation is used also to consider how the stochastic drift terms can be incorporated into the pushers. Stochastic gyrokinetic expansions are also discussed.

        Different options for the numerical implementation of these schemes are considered.

        Due to the efficacy of FET in the development of SP timesteppers for both the fluid and kinetic component, we hope this approach will prove effective in the future for developing SP timesteppers for the full hybrid model. We hope this will give us the opportunity to incorporate previously inaccessible kinetic effects into the highly effective, modern, finite-element MHD models.
    \end{abstract}
    
    
    \newpage
    \tableofcontents
    
    
    \newpage
    \pagenumbering{arabic}
    %\linenumbers\renewcommand\thelinenumber{\color{black!50}\arabic{linenumber}}
            \input{0 - introduction/main.tex}
        \part{Research}
            \input{1 - low-noise PiC models/main.tex}
            \input{2 - kinetic component/main.tex}
            \input{3 - fluid component/main.tex}
            \input{4 - numerical implementation/main.tex}
        \part{Project Overview}
            \input{5 - research plan/main.tex}
            \input{6 - summary/main.tex}
    
    
    %\section{}
    \newpage
    \pagenumbering{gobble}
        \printbibliography


    \newpage
    \pagenumbering{roman}
    \appendix
        \part{Appendices}
            \input{8 - Hilbert complexes/main.tex}
            \input{9 - weak conservation proofs/main.tex}
\end{document}

            \documentclass[12pt, a4paper]{report}

\input{template/main.tex}

\title{\BA{Title in Progress...}}
\author{Boris Andrews}
\affil{Mathematical Institute, University of Oxford}
\date{\today}


\begin{document}
    \pagenumbering{gobble}
    \maketitle
    
    
    \begin{abstract}
        Magnetic confinement reactors---in particular tokamaks---offer one of the most promising options for achieving practical nuclear fusion, with the potential to provide virtually limitless, clean energy. The theoretical and numerical modeling of tokamak plasmas is simultaneously an essential component of effective reactor design, and a great research barrier. Tokamak operational conditions exhibit comparatively low Knudsen numbers. Kinetic effects, including kinetic waves and instabilities, Landau damping, bump-on-tail instabilities and more, are therefore highly influential in tokamak plasma dynamics. Purely fluid models are inherently incapable of capturing these effects, whereas the high dimensionality in purely kinetic models render them practically intractable for most relevant purposes.

        We consider a $\delta\!f$ decomposition model, with a macroscopic fluid background and microscopic kinetic correction, both fully coupled to each other. A similar manner of discretization is proposed to that used in the recent \texttt{STRUPHY} code \cite{Holderied_Possanner_Wang_2021, Holderied_2022, Li_et_al_2023} with a finite-element model for the background and a pseudo-particle/PiC model for the correction.

        The fluid background satisfies the full, non-linear, resistive, compressible, Hall MHD equations. \cite{Laakmann_Hu_Farrell_2022} introduces finite-element(-in-space) implicit timesteppers for the incompressible analogue to this system with structure-preserving (SP) properties in the ideal case, alongside parameter-robust preconditioners. We show that these timesteppers can derive from a finite-element-in-time (FET) (and finite-element-in-space) interpretation. The benefits of this reformulation are discussed, including the derivation of timesteppers that are higher order in time, and the quantifiable dissipative SP properties in the non-ideal, resistive case.
        
        We discuss possible options for extending this FET approach to timesteppers for the compressible case.

        The kinetic corrections satisfy linearized Boltzmann equations. Using a Lénard--Bernstein collision operator, these take Fokker--Planck-like forms \cite{Fokker_1914, Planck_1917} wherein pseudo-particles in the numerical model obey the neoclassical transport equations, with particle-independent Brownian drift terms. This offers a rigorous methodology for incorporating collisions into the particle transport model, without coupling the equations of motions for each particle.
        
        Works by Chen, Chacón et al. \cite{Chen_Chacón_Barnes_2011, Chacón_Chen_Barnes_2013, Chen_Chacón_2014, Chen_Chacón_2015} have developed structure-preserving particle pushers for neoclassical transport in the Vlasov equations, derived from Crank--Nicolson integrators. We show these too can can derive from a FET interpretation, similarly offering potential extensions to higher-order-in-time particle pushers. The FET formulation is used also to consider how the stochastic drift terms can be incorporated into the pushers. Stochastic gyrokinetic expansions are also discussed.

        Different options for the numerical implementation of these schemes are considered.

        Due to the efficacy of FET in the development of SP timesteppers for both the fluid and kinetic component, we hope this approach will prove effective in the future for developing SP timesteppers for the full hybrid model. We hope this will give us the opportunity to incorporate previously inaccessible kinetic effects into the highly effective, modern, finite-element MHD models.
    \end{abstract}
    
    
    \newpage
    \tableofcontents
    
    
    \newpage
    \pagenumbering{arabic}
    %\linenumbers\renewcommand\thelinenumber{\color{black!50}\arabic{linenumber}}
            \input{0 - introduction/main.tex}
        \part{Research}
            \input{1 - low-noise PiC models/main.tex}
            \input{2 - kinetic component/main.tex}
            \input{3 - fluid component/main.tex}
            \input{4 - numerical implementation/main.tex}
        \part{Project Overview}
            \input{5 - research plan/main.tex}
            \input{6 - summary/main.tex}
    
    
    %\section{}
    \newpage
    \pagenumbering{gobble}
        \printbibliography


    \newpage
    \pagenumbering{roman}
    \appendix
        \part{Appendices}
            \input{8 - Hilbert complexes/main.tex}
            \input{9 - weak conservation proofs/main.tex}
\end{document}

\end{document}

            \documentclass[12pt, a4paper]{report}

\documentclass[12pt, a4paper]{report}

\input{template/main.tex}

\title{\BA{Title in Progress...}}
\author{Boris Andrews}
\affil{Mathematical Institute, University of Oxford}
\date{\today}


\begin{document}
    \pagenumbering{gobble}
    \maketitle
    
    
    \begin{abstract}
        Magnetic confinement reactors---in particular tokamaks---offer one of the most promising options for achieving practical nuclear fusion, with the potential to provide virtually limitless, clean energy. The theoretical and numerical modeling of tokamak plasmas is simultaneously an essential component of effective reactor design, and a great research barrier. Tokamak operational conditions exhibit comparatively low Knudsen numbers. Kinetic effects, including kinetic waves and instabilities, Landau damping, bump-on-tail instabilities and more, are therefore highly influential in tokamak plasma dynamics. Purely fluid models are inherently incapable of capturing these effects, whereas the high dimensionality in purely kinetic models render them practically intractable for most relevant purposes.

        We consider a $\delta\!f$ decomposition model, with a macroscopic fluid background and microscopic kinetic correction, both fully coupled to each other. A similar manner of discretization is proposed to that used in the recent \texttt{STRUPHY} code \cite{Holderied_Possanner_Wang_2021, Holderied_2022, Li_et_al_2023} with a finite-element model for the background and a pseudo-particle/PiC model for the correction.

        The fluid background satisfies the full, non-linear, resistive, compressible, Hall MHD equations. \cite{Laakmann_Hu_Farrell_2022} introduces finite-element(-in-space) implicit timesteppers for the incompressible analogue to this system with structure-preserving (SP) properties in the ideal case, alongside parameter-robust preconditioners. We show that these timesteppers can derive from a finite-element-in-time (FET) (and finite-element-in-space) interpretation. The benefits of this reformulation are discussed, including the derivation of timesteppers that are higher order in time, and the quantifiable dissipative SP properties in the non-ideal, resistive case.
        
        We discuss possible options for extending this FET approach to timesteppers for the compressible case.

        The kinetic corrections satisfy linearized Boltzmann equations. Using a Lénard--Bernstein collision operator, these take Fokker--Planck-like forms \cite{Fokker_1914, Planck_1917} wherein pseudo-particles in the numerical model obey the neoclassical transport equations, with particle-independent Brownian drift terms. This offers a rigorous methodology for incorporating collisions into the particle transport model, without coupling the equations of motions for each particle.
        
        Works by Chen, Chacón et al. \cite{Chen_Chacón_Barnes_2011, Chacón_Chen_Barnes_2013, Chen_Chacón_2014, Chen_Chacón_2015} have developed structure-preserving particle pushers for neoclassical transport in the Vlasov equations, derived from Crank--Nicolson integrators. We show these too can can derive from a FET interpretation, similarly offering potential extensions to higher-order-in-time particle pushers. The FET formulation is used also to consider how the stochastic drift terms can be incorporated into the pushers. Stochastic gyrokinetic expansions are also discussed.

        Different options for the numerical implementation of these schemes are considered.

        Due to the efficacy of FET in the development of SP timesteppers for both the fluid and kinetic component, we hope this approach will prove effective in the future for developing SP timesteppers for the full hybrid model. We hope this will give us the opportunity to incorporate previously inaccessible kinetic effects into the highly effective, modern, finite-element MHD models.
    \end{abstract}
    
    
    \newpage
    \tableofcontents
    
    
    \newpage
    \pagenumbering{arabic}
    %\linenumbers\renewcommand\thelinenumber{\color{black!50}\arabic{linenumber}}
            \input{0 - introduction/main.tex}
        \part{Research}
            \input{1 - low-noise PiC models/main.tex}
            \input{2 - kinetic component/main.tex}
            \input{3 - fluid component/main.tex}
            \input{4 - numerical implementation/main.tex}
        \part{Project Overview}
            \input{5 - research plan/main.tex}
            \input{6 - summary/main.tex}
    
    
    %\section{}
    \newpage
    \pagenumbering{gobble}
        \printbibliography


    \newpage
    \pagenumbering{roman}
    \appendix
        \part{Appendices}
            \input{8 - Hilbert complexes/main.tex}
            \input{9 - weak conservation proofs/main.tex}
\end{document}


\title{\BA{Title in Progress...}}
\author{Boris Andrews}
\affil{Mathematical Institute, University of Oxford}
\date{\today}


\begin{document}
    \pagenumbering{gobble}
    \maketitle
    
    
    \begin{abstract}
        Magnetic confinement reactors---in particular tokamaks---offer one of the most promising options for achieving practical nuclear fusion, with the potential to provide virtually limitless, clean energy. The theoretical and numerical modeling of tokamak plasmas is simultaneously an essential component of effective reactor design, and a great research barrier. Tokamak operational conditions exhibit comparatively low Knudsen numbers. Kinetic effects, including kinetic waves and instabilities, Landau damping, bump-on-tail instabilities and more, are therefore highly influential in tokamak plasma dynamics. Purely fluid models are inherently incapable of capturing these effects, whereas the high dimensionality in purely kinetic models render them practically intractable for most relevant purposes.

        We consider a $\delta\!f$ decomposition model, with a macroscopic fluid background and microscopic kinetic correction, both fully coupled to each other. A similar manner of discretization is proposed to that used in the recent \texttt{STRUPHY} code \cite{Holderied_Possanner_Wang_2021, Holderied_2022, Li_et_al_2023} with a finite-element model for the background and a pseudo-particle/PiC model for the correction.

        The fluid background satisfies the full, non-linear, resistive, compressible, Hall MHD equations. \cite{Laakmann_Hu_Farrell_2022} introduces finite-element(-in-space) implicit timesteppers for the incompressible analogue to this system with structure-preserving (SP) properties in the ideal case, alongside parameter-robust preconditioners. We show that these timesteppers can derive from a finite-element-in-time (FET) (and finite-element-in-space) interpretation. The benefits of this reformulation are discussed, including the derivation of timesteppers that are higher order in time, and the quantifiable dissipative SP properties in the non-ideal, resistive case.
        
        We discuss possible options for extending this FET approach to timesteppers for the compressible case.

        The kinetic corrections satisfy linearized Boltzmann equations. Using a Lénard--Bernstein collision operator, these take Fokker--Planck-like forms \cite{Fokker_1914, Planck_1917} wherein pseudo-particles in the numerical model obey the neoclassical transport equations, with particle-independent Brownian drift terms. This offers a rigorous methodology for incorporating collisions into the particle transport model, without coupling the equations of motions for each particle.
        
        Works by Chen, Chacón et al. \cite{Chen_Chacón_Barnes_2011, Chacón_Chen_Barnes_2013, Chen_Chacón_2014, Chen_Chacón_2015} have developed structure-preserving particle pushers for neoclassical transport in the Vlasov equations, derived from Crank--Nicolson integrators. We show these too can can derive from a FET interpretation, similarly offering potential extensions to higher-order-in-time particle pushers. The FET formulation is used also to consider how the stochastic drift terms can be incorporated into the pushers. Stochastic gyrokinetic expansions are also discussed.

        Different options for the numerical implementation of these schemes are considered.

        Due to the efficacy of FET in the development of SP timesteppers for both the fluid and kinetic component, we hope this approach will prove effective in the future for developing SP timesteppers for the full hybrid model. We hope this will give us the opportunity to incorporate previously inaccessible kinetic effects into the highly effective, modern, finite-element MHD models.
    \end{abstract}
    
    
    \newpage
    \tableofcontents
    
    
    \newpage
    \pagenumbering{arabic}
    %\linenumbers\renewcommand\thelinenumber{\color{black!50}\arabic{linenumber}}
            \documentclass[12pt, a4paper]{report}

\input{template/main.tex}

\title{\BA{Title in Progress...}}
\author{Boris Andrews}
\affil{Mathematical Institute, University of Oxford}
\date{\today}


\begin{document}
    \pagenumbering{gobble}
    \maketitle
    
    
    \begin{abstract}
        Magnetic confinement reactors---in particular tokamaks---offer one of the most promising options for achieving practical nuclear fusion, with the potential to provide virtually limitless, clean energy. The theoretical and numerical modeling of tokamak plasmas is simultaneously an essential component of effective reactor design, and a great research barrier. Tokamak operational conditions exhibit comparatively low Knudsen numbers. Kinetic effects, including kinetic waves and instabilities, Landau damping, bump-on-tail instabilities and more, are therefore highly influential in tokamak plasma dynamics. Purely fluid models are inherently incapable of capturing these effects, whereas the high dimensionality in purely kinetic models render them practically intractable for most relevant purposes.

        We consider a $\delta\!f$ decomposition model, with a macroscopic fluid background and microscopic kinetic correction, both fully coupled to each other. A similar manner of discretization is proposed to that used in the recent \texttt{STRUPHY} code \cite{Holderied_Possanner_Wang_2021, Holderied_2022, Li_et_al_2023} with a finite-element model for the background and a pseudo-particle/PiC model for the correction.

        The fluid background satisfies the full, non-linear, resistive, compressible, Hall MHD equations. \cite{Laakmann_Hu_Farrell_2022} introduces finite-element(-in-space) implicit timesteppers for the incompressible analogue to this system with structure-preserving (SP) properties in the ideal case, alongside parameter-robust preconditioners. We show that these timesteppers can derive from a finite-element-in-time (FET) (and finite-element-in-space) interpretation. The benefits of this reformulation are discussed, including the derivation of timesteppers that are higher order in time, and the quantifiable dissipative SP properties in the non-ideal, resistive case.
        
        We discuss possible options for extending this FET approach to timesteppers for the compressible case.

        The kinetic corrections satisfy linearized Boltzmann equations. Using a Lénard--Bernstein collision operator, these take Fokker--Planck-like forms \cite{Fokker_1914, Planck_1917} wherein pseudo-particles in the numerical model obey the neoclassical transport equations, with particle-independent Brownian drift terms. This offers a rigorous methodology for incorporating collisions into the particle transport model, without coupling the equations of motions for each particle.
        
        Works by Chen, Chacón et al. \cite{Chen_Chacón_Barnes_2011, Chacón_Chen_Barnes_2013, Chen_Chacón_2014, Chen_Chacón_2015} have developed structure-preserving particle pushers for neoclassical transport in the Vlasov equations, derived from Crank--Nicolson integrators. We show these too can can derive from a FET interpretation, similarly offering potential extensions to higher-order-in-time particle pushers. The FET formulation is used also to consider how the stochastic drift terms can be incorporated into the pushers. Stochastic gyrokinetic expansions are also discussed.

        Different options for the numerical implementation of these schemes are considered.

        Due to the efficacy of FET in the development of SP timesteppers for both the fluid and kinetic component, we hope this approach will prove effective in the future for developing SP timesteppers for the full hybrid model. We hope this will give us the opportunity to incorporate previously inaccessible kinetic effects into the highly effective, modern, finite-element MHD models.
    \end{abstract}
    
    
    \newpage
    \tableofcontents
    
    
    \newpage
    \pagenumbering{arabic}
    %\linenumbers\renewcommand\thelinenumber{\color{black!50}\arabic{linenumber}}
            \input{0 - introduction/main.tex}
        \part{Research}
            \input{1 - low-noise PiC models/main.tex}
            \input{2 - kinetic component/main.tex}
            \input{3 - fluid component/main.tex}
            \input{4 - numerical implementation/main.tex}
        \part{Project Overview}
            \input{5 - research plan/main.tex}
            \input{6 - summary/main.tex}
    
    
    %\section{}
    \newpage
    \pagenumbering{gobble}
        \printbibliography


    \newpage
    \pagenumbering{roman}
    \appendix
        \part{Appendices}
            \input{8 - Hilbert complexes/main.tex}
            \input{9 - weak conservation proofs/main.tex}
\end{document}

        \part{Research}
            \documentclass[12pt, a4paper]{report}

\input{template/main.tex}

\title{\BA{Title in Progress...}}
\author{Boris Andrews}
\affil{Mathematical Institute, University of Oxford}
\date{\today}


\begin{document}
    \pagenumbering{gobble}
    \maketitle
    
    
    \begin{abstract}
        Magnetic confinement reactors---in particular tokamaks---offer one of the most promising options for achieving practical nuclear fusion, with the potential to provide virtually limitless, clean energy. The theoretical and numerical modeling of tokamak plasmas is simultaneously an essential component of effective reactor design, and a great research barrier. Tokamak operational conditions exhibit comparatively low Knudsen numbers. Kinetic effects, including kinetic waves and instabilities, Landau damping, bump-on-tail instabilities and more, are therefore highly influential in tokamak plasma dynamics. Purely fluid models are inherently incapable of capturing these effects, whereas the high dimensionality in purely kinetic models render them practically intractable for most relevant purposes.

        We consider a $\delta\!f$ decomposition model, with a macroscopic fluid background and microscopic kinetic correction, both fully coupled to each other. A similar manner of discretization is proposed to that used in the recent \texttt{STRUPHY} code \cite{Holderied_Possanner_Wang_2021, Holderied_2022, Li_et_al_2023} with a finite-element model for the background and a pseudo-particle/PiC model for the correction.

        The fluid background satisfies the full, non-linear, resistive, compressible, Hall MHD equations. \cite{Laakmann_Hu_Farrell_2022} introduces finite-element(-in-space) implicit timesteppers for the incompressible analogue to this system with structure-preserving (SP) properties in the ideal case, alongside parameter-robust preconditioners. We show that these timesteppers can derive from a finite-element-in-time (FET) (and finite-element-in-space) interpretation. The benefits of this reformulation are discussed, including the derivation of timesteppers that are higher order in time, and the quantifiable dissipative SP properties in the non-ideal, resistive case.
        
        We discuss possible options for extending this FET approach to timesteppers for the compressible case.

        The kinetic corrections satisfy linearized Boltzmann equations. Using a Lénard--Bernstein collision operator, these take Fokker--Planck-like forms \cite{Fokker_1914, Planck_1917} wherein pseudo-particles in the numerical model obey the neoclassical transport equations, with particle-independent Brownian drift terms. This offers a rigorous methodology for incorporating collisions into the particle transport model, without coupling the equations of motions for each particle.
        
        Works by Chen, Chacón et al. \cite{Chen_Chacón_Barnes_2011, Chacón_Chen_Barnes_2013, Chen_Chacón_2014, Chen_Chacón_2015} have developed structure-preserving particle pushers for neoclassical transport in the Vlasov equations, derived from Crank--Nicolson integrators. We show these too can can derive from a FET interpretation, similarly offering potential extensions to higher-order-in-time particle pushers. The FET formulation is used also to consider how the stochastic drift terms can be incorporated into the pushers. Stochastic gyrokinetic expansions are also discussed.

        Different options for the numerical implementation of these schemes are considered.

        Due to the efficacy of FET in the development of SP timesteppers for both the fluid and kinetic component, we hope this approach will prove effective in the future for developing SP timesteppers for the full hybrid model. We hope this will give us the opportunity to incorporate previously inaccessible kinetic effects into the highly effective, modern, finite-element MHD models.
    \end{abstract}
    
    
    \newpage
    \tableofcontents
    
    
    \newpage
    \pagenumbering{arabic}
    %\linenumbers\renewcommand\thelinenumber{\color{black!50}\arabic{linenumber}}
            \input{0 - introduction/main.tex}
        \part{Research}
            \input{1 - low-noise PiC models/main.tex}
            \input{2 - kinetic component/main.tex}
            \input{3 - fluid component/main.tex}
            \input{4 - numerical implementation/main.tex}
        \part{Project Overview}
            \input{5 - research plan/main.tex}
            \input{6 - summary/main.tex}
    
    
    %\section{}
    \newpage
    \pagenumbering{gobble}
        \printbibliography


    \newpage
    \pagenumbering{roman}
    \appendix
        \part{Appendices}
            \input{8 - Hilbert complexes/main.tex}
            \input{9 - weak conservation proofs/main.tex}
\end{document}

            \documentclass[12pt, a4paper]{report}

\input{template/main.tex}

\title{\BA{Title in Progress...}}
\author{Boris Andrews}
\affil{Mathematical Institute, University of Oxford}
\date{\today}


\begin{document}
    \pagenumbering{gobble}
    \maketitle
    
    
    \begin{abstract}
        Magnetic confinement reactors---in particular tokamaks---offer one of the most promising options for achieving practical nuclear fusion, with the potential to provide virtually limitless, clean energy. The theoretical and numerical modeling of tokamak plasmas is simultaneously an essential component of effective reactor design, and a great research barrier. Tokamak operational conditions exhibit comparatively low Knudsen numbers. Kinetic effects, including kinetic waves and instabilities, Landau damping, bump-on-tail instabilities and more, are therefore highly influential in tokamak plasma dynamics. Purely fluid models are inherently incapable of capturing these effects, whereas the high dimensionality in purely kinetic models render them practically intractable for most relevant purposes.

        We consider a $\delta\!f$ decomposition model, with a macroscopic fluid background and microscopic kinetic correction, both fully coupled to each other. A similar manner of discretization is proposed to that used in the recent \texttt{STRUPHY} code \cite{Holderied_Possanner_Wang_2021, Holderied_2022, Li_et_al_2023} with a finite-element model for the background and a pseudo-particle/PiC model for the correction.

        The fluid background satisfies the full, non-linear, resistive, compressible, Hall MHD equations. \cite{Laakmann_Hu_Farrell_2022} introduces finite-element(-in-space) implicit timesteppers for the incompressible analogue to this system with structure-preserving (SP) properties in the ideal case, alongside parameter-robust preconditioners. We show that these timesteppers can derive from a finite-element-in-time (FET) (and finite-element-in-space) interpretation. The benefits of this reformulation are discussed, including the derivation of timesteppers that are higher order in time, and the quantifiable dissipative SP properties in the non-ideal, resistive case.
        
        We discuss possible options for extending this FET approach to timesteppers for the compressible case.

        The kinetic corrections satisfy linearized Boltzmann equations. Using a Lénard--Bernstein collision operator, these take Fokker--Planck-like forms \cite{Fokker_1914, Planck_1917} wherein pseudo-particles in the numerical model obey the neoclassical transport equations, with particle-independent Brownian drift terms. This offers a rigorous methodology for incorporating collisions into the particle transport model, without coupling the equations of motions for each particle.
        
        Works by Chen, Chacón et al. \cite{Chen_Chacón_Barnes_2011, Chacón_Chen_Barnes_2013, Chen_Chacón_2014, Chen_Chacón_2015} have developed structure-preserving particle pushers for neoclassical transport in the Vlasov equations, derived from Crank--Nicolson integrators. We show these too can can derive from a FET interpretation, similarly offering potential extensions to higher-order-in-time particle pushers. The FET formulation is used also to consider how the stochastic drift terms can be incorporated into the pushers. Stochastic gyrokinetic expansions are also discussed.

        Different options for the numerical implementation of these schemes are considered.

        Due to the efficacy of FET in the development of SP timesteppers for both the fluid and kinetic component, we hope this approach will prove effective in the future for developing SP timesteppers for the full hybrid model. We hope this will give us the opportunity to incorporate previously inaccessible kinetic effects into the highly effective, modern, finite-element MHD models.
    \end{abstract}
    
    
    \newpage
    \tableofcontents
    
    
    \newpage
    \pagenumbering{arabic}
    %\linenumbers\renewcommand\thelinenumber{\color{black!50}\arabic{linenumber}}
            \input{0 - introduction/main.tex}
        \part{Research}
            \input{1 - low-noise PiC models/main.tex}
            \input{2 - kinetic component/main.tex}
            \input{3 - fluid component/main.tex}
            \input{4 - numerical implementation/main.tex}
        \part{Project Overview}
            \input{5 - research plan/main.tex}
            \input{6 - summary/main.tex}
    
    
    %\section{}
    \newpage
    \pagenumbering{gobble}
        \printbibliography


    \newpage
    \pagenumbering{roman}
    \appendix
        \part{Appendices}
            \input{8 - Hilbert complexes/main.tex}
            \input{9 - weak conservation proofs/main.tex}
\end{document}

            \documentclass[12pt, a4paper]{report}

\input{template/main.tex}

\title{\BA{Title in Progress...}}
\author{Boris Andrews}
\affil{Mathematical Institute, University of Oxford}
\date{\today}


\begin{document}
    \pagenumbering{gobble}
    \maketitle
    
    
    \begin{abstract}
        Magnetic confinement reactors---in particular tokamaks---offer one of the most promising options for achieving practical nuclear fusion, with the potential to provide virtually limitless, clean energy. The theoretical and numerical modeling of tokamak plasmas is simultaneously an essential component of effective reactor design, and a great research barrier. Tokamak operational conditions exhibit comparatively low Knudsen numbers. Kinetic effects, including kinetic waves and instabilities, Landau damping, bump-on-tail instabilities and more, are therefore highly influential in tokamak plasma dynamics. Purely fluid models are inherently incapable of capturing these effects, whereas the high dimensionality in purely kinetic models render them practically intractable for most relevant purposes.

        We consider a $\delta\!f$ decomposition model, with a macroscopic fluid background and microscopic kinetic correction, both fully coupled to each other. A similar manner of discretization is proposed to that used in the recent \texttt{STRUPHY} code \cite{Holderied_Possanner_Wang_2021, Holderied_2022, Li_et_al_2023} with a finite-element model for the background and a pseudo-particle/PiC model for the correction.

        The fluid background satisfies the full, non-linear, resistive, compressible, Hall MHD equations. \cite{Laakmann_Hu_Farrell_2022} introduces finite-element(-in-space) implicit timesteppers for the incompressible analogue to this system with structure-preserving (SP) properties in the ideal case, alongside parameter-robust preconditioners. We show that these timesteppers can derive from a finite-element-in-time (FET) (and finite-element-in-space) interpretation. The benefits of this reformulation are discussed, including the derivation of timesteppers that are higher order in time, and the quantifiable dissipative SP properties in the non-ideal, resistive case.
        
        We discuss possible options for extending this FET approach to timesteppers for the compressible case.

        The kinetic corrections satisfy linearized Boltzmann equations. Using a Lénard--Bernstein collision operator, these take Fokker--Planck-like forms \cite{Fokker_1914, Planck_1917} wherein pseudo-particles in the numerical model obey the neoclassical transport equations, with particle-independent Brownian drift terms. This offers a rigorous methodology for incorporating collisions into the particle transport model, without coupling the equations of motions for each particle.
        
        Works by Chen, Chacón et al. \cite{Chen_Chacón_Barnes_2011, Chacón_Chen_Barnes_2013, Chen_Chacón_2014, Chen_Chacón_2015} have developed structure-preserving particle pushers for neoclassical transport in the Vlasov equations, derived from Crank--Nicolson integrators. We show these too can can derive from a FET interpretation, similarly offering potential extensions to higher-order-in-time particle pushers. The FET formulation is used also to consider how the stochastic drift terms can be incorporated into the pushers. Stochastic gyrokinetic expansions are also discussed.

        Different options for the numerical implementation of these schemes are considered.

        Due to the efficacy of FET in the development of SP timesteppers for both the fluid and kinetic component, we hope this approach will prove effective in the future for developing SP timesteppers for the full hybrid model. We hope this will give us the opportunity to incorporate previously inaccessible kinetic effects into the highly effective, modern, finite-element MHD models.
    \end{abstract}
    
    
    \newpage
    \tableofcontents
    
    
    \newpage
    \pagenumbering{arabic}
    %\linenumbers\renewcommand\thelinenumber{\color{black!50}\arabic{linenumber}}
            \input{0 - introduction/main.tex}
        \part{Research}
            \input{1 - low-noise PiC models/main.tex}
            \input{2 - kinetic component/main.tex}
            \input{3 - fluid component/main.tex}
            \input{4 - numerical implementation/main.tex}
        \part{Project Overview}
            \input{5 - research plan/main.tex}
            \input{6 - summary/main.tex}
    
    
    %\section{}
    \newpage
    \pagenumbering{gobble}
        \printbibliography


    \newpage
    \pagenumbering{roman}
    \appendix
        \part{Appendices}
            \input{8 - Hilbert complexes/main.tex}
            \input{9 - weak conservation proofs/main.tex}
\end{document}

            \documentclass[12pt, a4paper]{report}

\input{template/main.tex}

\title{\BA{Title in Progress...}}
\author{Boris Andrews}
\affil{Mathematical Institute, University of Oxford}
\date{\today}


\begin{document}
    \pagenumbering{gobble}
    \maketitle
    
    
    \begin{abstract}
        Magnetic confinement reactors---in particular tokamaks---offer one of the most promising options for achieving practical nuclear fusion, with the potential to provide virtually limitless, clean energy. The theoretical and numerical modeling of tokamak plasmas is simultaneously an essential component of effective reactor design, and a great research barrier. Tokamak operational conditions exhibit comparatively low Knudsen numbers. Kinetic effects, including kinetic waves and instabilities, Landau damping, bump-on-tail instabilities and more, are therefore highly influential in tokamak plasma dynamics. Purely fluid models are inherently incapable of capturing these effects, whereas the high dimensionality in purely kinetic models render them practically intractable for most relevant purposes.

        We consider a $\delta\!f$ decomposition model, with a macroscopic fluid background and microscopic kinetic correction, both fully coupled to each other. A similar manner of discretization is proposed to that used in the recent \texttt{STRUPHY} code \cite{Holderied_Possanner_Wang_2021, Holderied_2022, Li_et_al_2023} with a finite-element model for the background and a pseudo-particle/PiC model for the correction.

        The fluid background satisfies the full, non-linear, resistive, compressible, Hall MHD equations. \cite{Laakmann_Hu_Farrell_2022} introduces finite-element(-in-space) implicit timesteppers for the incompressible analogue to this system with structure-preserving (SP) properties in the ideal case, alongside parameter-robust preconditioners. We show that these timesteppers can derive from a finite-element-in-time (FET) (and finite-element-in-space) interpretation. The benefits of this reformulation are discussed, including the derivation of timesteppers that are higher order in time, and the quantifiable dissipative SP properties in the non-ideal, resistive case.
        
        We discuss possible options for extending this FET approach to timesteppers for the compressible case.

        The kinetic corrections satisfy linearized Boltzmann equations. Using a Lénard--Bernstein collision operator, these take Fokker--Planck-like forms \cite{Fokker_1914, Planck_1917} wherein pseudo-particles in the numerical model obey the neoclassical transport equations, with particle-independent Brownian drift terms. This offers a rigorous methodology for incorporating collisions into the particle transport model, without coupling the equations of motions for each particle.
        
        Works by Chen, Chacón et al. \cite{Chen_Chacón_Barnes_2011, Chacón_Chen_Barnes_2013, Chen_Chacón_2014, Chen_Chacón_2015} have developed structure-preserving particle pushers for neoclassical transport in the Vlasov equations, derived from Crank--Nicolson integrators. We show these too can can derive from a FET interpretation, similarly offering potential extensions to higher-order-in-time particle pushers. The FET formulation is used also to consider how the stochastic drift terms can be incorporated into the pushers. Stochastic gyrokinetic expansions are also discussed.

        Different options for the numerical implementation of these schemes are considered.

        Due to the efficacy of FET in the development of SP timesteppers for both the fluid and kinetic component, we hope this approach will prove effective in the future for developing SP timesteppers for the full hybrid model. We hope this will give us the opportunity to incorporate previously inaccessible kinetic effects into the highly effective, modern, finite-element MHD models.
    \end{abstract}
    
    
    \newpage
    \tableofcontents
    
    
    \newpage
    \pagenumbering{arabic}
    %\linenumbers\renewcommand\thelinenumber{\color{black!50}\arabic{linenumber}}
            \input{0 - introduction/main.tex}
        \part{Research}
            \input{1 - low-noise PiC models/main.tex}
            \input{2 - kinetic component/main.tex}
            \input{3 - fluid component/main.tex}
            \input{4 - numerical implementation/main.tex}
        \part{Project Overview}
            \input{5 - research plan/main.tex}
            \input{6 - summary/main.tex}
    
    
    %\section{}
    \newpage
    \pagenumbering{gobble}
        \printbibliography


    \newpage
    \pagenumbering{roman}
    \appendix
        \part{Appendices}
            \input{8 - Hilbert complexes/main.tex}
            \input{9 - weak conservation proofs/main.tex}
\end{document}

        \part{Project Overview}
            \documentclass[12pt, a4paper]{report}

\input{template/main.tex}

\title{\BA{Title in Progress...}}
\author{Boris Andrews}
\affil{Mathematical Institute, University of Oxford}
\date{\today}


\begin{document}
    \pagenumbering{gobble}
    \maketitle
    
    
    \begin{abstract}
        Magnetic confinement reactors---in particular tokamaks---offer one of the most promising options for achieving practical nuclear fusion, with the potential to provide virtually limitless, clean energy. The theoretical and numerical modeling of tokamak plasmas is simultaneously an essential component of effective reactor design, and a great research barrier. Tokamak operational conditions exhibit comparatively low Knudsen numbers. Kinetic effects, including kinetic waves and instabilities, Landau damping, bump-on-tail instabilities and more, are therefore highly influential in tokamak plasma dynamics. Purely fluid models are inherently incapable of capturing these effects, whereas the high dimensionality in purely kinetic models render them practically intractable for most relevant purposes.

        We consider a $\delta\!f$ decomposition model, with a macroscopic fluid background and microscopic kinetic correction, both fully coupled to each other. A similar manner of discretization is proposed to that used in the recent \texttt{STRUPHY} code \cite{Holderied_Possanner_Wang_2021, Holderied_2022, Li_et_al_2023} with a finite-element model for the background and a pseudo-particle/PiC model for the correction.

        The fluid background satisfies the full, non-linear, resistive, compressible, Hall MHD equations. \cite{Laakmann_Hu_Farrell_2022} introduces finite-element(-in-space) implicit timesteppers for the incompressible analogue to this system with structure-preserving (SP) properties in the ideal case, alongside parameter-robust preconditioners. We show that these timesteppers can derive from a finite-element-in-time (FET) (and finite-element-in-space) interpretation. The benefits of this reformulation are discussed, including the derivation of timesteppers that are higher order in time, and the quantifiable dissipative SP properties in the non-ideal, resistive case.
        
        We discuss possible options for extending this FET approach to timesteppers for the compressible case.

        The kinetic corrections satisfy linearized Boltzmann equations. Using a Lénard--Bernstein collision operator, these take Fokker--Planck-like forms \cite{Fokker_1914, Planck_1917} wherein pseudo-particles in the numerical model obey the neoclassical transport equations, with particle-independent Brownian drift terms. This offers a rigorous methodology for incorporating collisions into the particle transport model, without coupling the equations of motions for each particle.
        
        Works by Chen, Chacón et al. \cite{Chen_Chacón_Barnes_2011, Chacón_Chen_Barnes_2013, Chen_Chacón_2014, Chen_Chacón_2015} have developed structure-preserving particle pushers for neoclassical transport in the Vlasov equations, derived from Crank--Nicolson integrators. We show these too can can derive from a FET interpretation, similarly offering potential extensions to higher-order-in-time particle pushers. The FET formulation is used also to consider how the stochastic drift terms can be incorporated into the pushers. Stochastic gyrokinetic expansions are also discussed.

        Different options for the numerical implementation of these schemes are considered.

        Due to the efficacy of FET in the development of SP timesteppers for both the fluid and kinetic component, we hope this approach will prove effective in the future for developing SP timesteppers for the full hybrid model. We hope this will give us the opportunity to incorporate previously inaccessible kinetic effects into the highly effective, modern, finite-element MHD models.
    \end{abstract}
    
    
    \newpage
    \tableofcontents
    
    
    \newpage
    \pagenumbering{arabic}
    %\linenumbers\renewcommand\thelinenumber{\color{black!50}\arabic{linenumber}}
            \input{0 - introduction/main.tex}
        \part{Research}
            \input{1 - low-noise PiC models/main.tex}
            \input{2 - kinetic component/main.tex}
            \input{3 - fluid component/main.tex}
            \input{4 - numerical implementation/main.tex}
        \part{Project Overview}
            \input{5 - research plan/main.tex}
            \input{6 - summary/main.tex}
    
    
    %\section{}
    \newpage
    \pagenumbering{gobble}
        \printbibliography


    \newpage
    \pagenumbering{roman}
    \appendix
        \part{Appendices}
            \input{8 - Hilbert complexes/main.tex}
            \input{9 - weak conservation proofs/main.tex}
\end{document}

            \documentclass[12pt, a4paper]{report}

\input{template/main.tex}

\title{\BA{Title in Progress...}}
\author{Boris Andrews}
\affil{Mathematical Institute, University of Oxford}
\date{\today}


\begin{document}
    \pagenumbering{gobble}
    \maketitle
    
    
    \begin{abstract}
        Magnetic confinement reactors---in particular tokamaks---offer one of the most promising options for achieving practical nuclear fusion, with the potential to provide virtually limitless, clean energy. The theoretical and numerical modeling of tokamak plasmas is simultaneously an essential component of effective reactor design, and a great research barrier. Tokamak operational conditions exhibit comparatively low Knudsen numbers. Kinetic effects, including kinetic waves and instabilities, Landau damping, bump-on-tail instabilities and more, are therefore highly influential in tokamak plasma dynamics. Purely fluid models are inherently incapable of capturing these effects, whereas the high dimensionality in purely kinetic models render them practically intractable for most relevant purposes.

        We consider a $\delta\!f$ decomposition model, with a macroscopic fluid background and microscopic kinetic correction, both fully coupled to each other. A similar manner of discretization is proposed to that used in the recent \texttt{STRUPHY} code \cite{Holderied_Possanner_Wang_2021, Holderied_2022, Li_et_al_2023} with a finite-element model for the background and a pseudo-particle/PiC model for the correction.

        The fluid background satisfies the full, non-linear, resistive, compressible, Hall MHD equations. \cite{Laakmann_Hu_Farrell_2022} introduces finite-element(-in-space) implicit timesteppers for the incompressible analogue to this system with structure-preserving (SP) properties in the ideal case, alongside parameter-robust preconditioners. We show that these timesteppers can derive from a finite-element-in-time (FET) (and finite-element-in-space) interpretation. The benefits of this reformulation are discussed, including the derivation of timesteppers that are higher order in time, and the quantifiable dissipative SP properties in the non-ideal, resistive case.
        
        We discuss possible options for extending this FET approach to timesteppers for the compressible case.

        The kinetic corrections satisfy linearized Boltzmann equations. Using a Lénard--Bernstein collision operator, these take Fokker--Planck-like forms \cite{Fokker_1914, Planck_1917} wherein pseudo-particles in the numerical model obey the neoclassical transport equations, with particle-independent Brownian drift terms. This offers a rigorous methodology for incorporating collisions into the particle transport model, without coupling the equations of motions for each particle.
        
        Works by Chen, Chacón et al. \cite{Chen_Chacón_Barnes_2011, Chacón_Chen_Barnes_2013, Chen_Chacón_2014, Chen_Chacón_2015} have developed structure-preserving particle pushers for neoclassical transport in the Vlasov equations, derived from Crank--Nicolson integrators. We show these too can can derive from a FET interpretation, similarly offering potential extensions to higher-order-in-time particle pushers. The FET formulation is used also to consider how the stochastic drift terms can be incorporated into the pushers. Stochastic gyrokinetic expansions are also discussed.

        Different options for the numerical implementation of these schemes are considered.

        Due to the efficacy of FET in the development of SP timesteppers for both the fluid and kinetic component, we hope this approach will prove effective in the future for developing SP timesteppers for the full hybrid model. We hope this will give us the opportunity to incorporate previously inaccessible kinetic effects into the highly effective, modern, finite-element MHD models.
    \end{abstract}
    
    
    \newpage
    \tableofcontents
    
    
    \newpage
    \pagenumbering{arabic}
    %\linenumbers\renewcommand\thelinenumber{\color{black!50}\arabic{linenumber}}
            \input{0 - introduction/main.tex}
        \part{Research}
            \input{1 - low-noise PiC models/main.tex}
            \input{2 - kinetic component/main.tex}
            \input{3 - fluid component/main.tex}
            \input{4 - numerical implementation/main.tex}
        \part{Project Overview}
            \input{5 - research plan/main.tex}
            \input{6 - summary/main.tex}
    
    
    %\section{}
    \newpage
    \pagenumbering{gobble}
        \printbibliography


    \newpage
    \pagenumbering{roman}
    \appendix
        \part{Appendices}
            \input{8 - Hilbert complexes/main.tex}
            \input{9 - weak conservation proofs/main.tex}
\end{document}

    
    
    %\section{}
    \newpage
    \pagenumbering{gobble}
        \printbibliography


    \newpage
    \pagenumbering{roman}
    \appendix
        \part{Appendices}
            \documentclass[12pt, a4paper]{report}

\input{template/main.tex}

\title{\BA{Title in Progress...}}
\author{Boris Andrews}
\affil{Mathematical Institute, University of Oxford}
\date{\today}


\begin{document}
    \pagenumbering{gobble}
    \maketitle
    
    
    \begin{abstract}
        Magnetic confinement reactors---in particular tokamaks---offer one of the most promising options for achieving practical nuclear fusion, with the potential to provide virtually limitless, clean energy. The theoretical and numerical modeling of tokamak plasmas is simultaneously an essential component of effective reactor design, and a great research barrier. Tokamak operational conditions exhibit comparatively low Knudsen numbers. Kinetic effects, including kinetic waves and instabilities, Landau damping, bump-on-tail instabilities and more, are therefore highly influential in tokamak plasma dynamics. Purely fluid models are inherently incapable of capturing these effects, whereas the high dimensionality in purely kinetic models render them practically intractable for most relevant purposes.

        We consider a $\delta\!f$ decomposition model, with a macroscopic fluid background and microscopic kinetic correction, both fully coupled to each other. A similar manner of discretization is proposed to that used in the recent \texttt{STRUPHY} code \cite{Holderied_Possanner_Wang_2021, Holderied_2022, Li_et_al_2023} with a finite-element model for the background and a pseudo-particle/PiC model for the correction.

        The fluid background satisfies the full, non-linear, resistive, compressible, Hall MHD equations. \cite{Laakmann_Hu_Farrell_2022} introduces finite-element(-in-space) implicit timesteppers for the incompressible analogue to this system with structure-preserving (SP) properties in the ideal case, alongside parameter-robust preconditioners. We show that these timesteppers can derive from a finite-element-in-time (FET) (and finite-element-in-space) interpretation. The benefits of this reformulation are discussed, including the derivation of timesteppers that are higher order in time, and the quantifiable dissipative SP properties in the non-ideal, resistive case.
        
        We discuss possible options for extending this FET approach to timesteppers for the compressible case.

        The kinetic corrections satisfy linearized Boltzmann equations. Using a Lénard--Bernstein collision operator, these take Fokker--Planck-like forms \cite{Fokker_1914, Planck_1917} wherein pseudo-particles in the numerical model obey the neoclassical transport equations, with particle-independent Brownian drift terms. This offers a rigorous methodology for incorporating collisions into the particle transport model, without coupling the equations of motions for each particle.
        
        Works by Chen, Chacón et al. \cite{Chen_Chacón_Barnes_2011, Chacón_Chen_Barnes_2013, Chen_Chacón_2014, Chen_Chacón_2015} have developed structure-preserving particle pushers for neoclassical transport in the Vlasov equations, derived from Crank--Nicolson integrators. We show these too can can derive from a FET interpretation, similarly offering potential extensions to higher-order-in-time particle pushers. The FET formulation is used also to consider how the stochastic drift terms can be incorporated into the pushers. Stochastic gyrokinetic expansions are also discussed.

        Different options for the numerical implementation of these schemes are considered.

        Due to the efficacy of FET in the development of SP timesteppers for both the fluid and kinetic component, we hope this approach will prove effective in the future for developing SP timesteppers for the full hybrid model. We hope this will give us the opportunity to incorporate previously inaccessible kinetic effects into the highly effective, modern, finite-element MHD models.
    \end{abstract}
    
    
    \newpage
    \tableofcontents
    
    
    \newpage
    \pagenumbering{arabic}
    %\linenumbers\renewcommand\thelinenumber{\color{black!50}\arabic{linenumber}}
            \input{0 - introduction/main.tex}
        \part{Research}
            \input{1 - low-noise PiC models/main.tex}
            \input{2 - kinetic component/main.tex}
            \input{3 - fluid component/main.tex}
            \input{4 - numerical implementation/main.tex}
        \part{Project Overview}
            \input{5 - research plan/main.tex}
            \input{6 - summary/main.tex}
    
    
    %\section{}
    \newpage
    \pagenumbering{gobble}
        \printbibliography


    \newpage
    \pagenumbering{roman}
    \appendix
        \part{Appendices}
            \input{8 - Hilbert complexes/main.tex}
            \input{9 - weak conservation proofs/main.tex}
\end{document}

            \documentclass[12pt, a4paper]{report}

\input{template/main.tex}

\title{\BA{Title in Progress...}}
\author{Boris Andrews}
\affil{Mathematical Institute, University of Oxford}
\date{\today}


\begin{document}
    \pagenumbering{gobble}
    \maketitle
    
    
    \begin{abstract}
        Magnetic confinement reactors---in particular tokamaks---offer one of the most promising options for achieving practical nuclear fusion, with the potential to provide virtually limitless, clean energy. The theoretical and numerical modeling of tokamak plasmas is simultaneously an essential component of effective reactor design, and a great research barrier. Tokamak operational conditions exhibit comparatively low Knudsen numbers. Kinetic effects, including kinetic waves and instabilities, Landau damping, bump-on-tail instabilities and more, are therefore highly influential in tokamak plasma dynamics. Purely fluid models are inherently incapable of capturing these effects, whereas the high dimensionality in purely kinetic models render them practically intractable for most relevant purposes.

        We consider a $\delta\!f$ decomposition model, with a macroscopic fluid background and microscopic kinetic correction, both fully coupled to each other. A similar manner of discretization is proposed to that used in the recent \texttt{STRUPHY} code \cite{Holderied_Possanner_Wang_2021, Holderied_2022, Li_et_al_2023} with a finite-element model for the background and a pseudo-particle/PiC model for the correction.

        The fluid background satisfies the full, non-linear, resistive, compressible, Hall MHD equations. \cite{Laakmann_Hu_Farrell_2022} introduces finite-element(-in-space) implicit timesteppers for the incompressible analogue to this system with structure-preserving (SP) properties in the ideal case, alongside parameter-robust preconditioners. We show that these timesteppers can derive from a finite-element-in-time (FET) (and finite-element-in-space) interpretation. The benefits of this reformulation are discussed, including the derivation of timesteppers that are higher order in time, and the quantifiable dissipative SP properties in the non-ideal, resistive case.
        
        We discuss possible options for extending this FET approach to timesteppers for the compressible case.

        The kinetic corrections satisfy linearized Boltzmann equations. Using a Lénard--Bernstein collision operator, these take Fokker--Planck-like forms \cite{Fokker_1914, Planck_1917} wherein pseudo-particles in the numerical model obey the neoclassical transport equations, with particle-independent Brownian drift terms. This offers a rigorous methodology for incorporating collisions into the particle transport model, without coupling the equations of motions for each particle.
        
        Works by Chen, Chacón et al. \cite{Chen_Chacón_Barnes_2011, Chacón_Chen_Barnes_2013, Chen_Chacón_2014, Chen_Chacón_2015} have developed structure-preserving particle pushers for neoclassical transport in the Vlasov equations, derived from Crank--Nicolson integrators. We show these too can can derive from a FET interpretation, similarly offering potential extensions to higher-order-in-time particle pushers. The FET formulation is used also to consider how the stochastic drift terms can be incorporated into the pushers. Stochastic gyrokinetic expansions are also discussed.

        Different options for the numerical implementation of these schemes are considered.

        Due to the efficacy of FET in the development of SP timesteppers for both the fluid and kinetic component, we hope this approach will prove effective in the future for developing SP timesteppers for the full hybrid model. We hope this will give us the opportunity to incorporate previously inaccessible kinetic effects into the highly effective, modern, finite-element MHD models.
    \end{abstract}
    
    
    \newpage
    \tableofcontents
    
    
    \newpage
    \pagenumbering{arabic}
    %\linenumbers\renewcommand\thelinenumber{\color{black!50}\arabic{linenumber}}
            \input{0 - introduction/main.tex}
        \part{Research}
            \input{1 - low-noise PiC models/main.tex}
            \input{2 - kinetic component/main.tex}
            \input{3 - fluid component/main.tex}
            \input{4 - numerical implementation/main.tex}
        \part{Project Overview}
            \input{5 - research plan/main.tex}
            \input{6 - summary/main.tex}
    
    
    %\section{}
    \newpage
    \pagenumbering{gobble}
        \printbibliography


    \newpage
    \pagenumbering{roman}
    \appendix
        \part{Appendices}
            \input{8 - Hilbert complexes/main.tex}
            \input{9 - weak conservation proofs/main.tex}
\end{document}

\end{document}

            \documentclass[12pt, a4paper]{report}

\documentclass[12pt, a4paper]{report}

\input{template/main.tex}

\title{\BA{Title in Progress...}}
\author{Boris Andrews}
\affil{Mathematical Institute, University of Oxford}
\date{\today}


\begin{document}
    \pagenumbering{gobble}
    \maketitle
    
    
    \begin{abstract}
        Magnetic confinement reactors---in particular tokamaks---offer one of the most promising options for achieving practical nuclear fusion, with the potential to provide virtually limitless, clean energy. The theoretical and numerical modeling of tokamak plasmas is simultaneously an essential component of effective reactor design, and a great research barrier. Tokamak operational conditions exhibit comparatively low Knudsen numbers. Kinetic effects, including kinetic waves and instabilities, Landau damping, bump-on-tail instabilities and more, are therefore highly influential in tokamak plasma dynamics. Purely fluid models are inherently incapable of capturing these effects, whereas the high dimensionality in purely kinetic models render them practically intractable for most relevant purposes.

        We consider a $\delta\!f$ decomposition model, with a macroscopic fluid background and microscopic kinetic correction, both fully coupled to each other. A similar manner of discretization is proposed to that used in the recent \texttt{STRUPHY} code \cite{Holderied_Possanner_Wang_2021, Holderied_2022, Li_et_al_2023} with a finite-element model for the background and a pseudo-particle/PiC model for the correction.

        The fluid background satisfies the full, non-linear, resistive, compressible, Hall MHD equations. \cite{Laakmann_Hu_Farrell_2022} introduces finite-element(-in-space) implicit timesteppers for the incompressible analogue to this system with structure-preserving (SP) properties in the ideal case, alongside parameter-robust preconditioners. We show that these timesteppers can derive from a finite-element-in-time (FET) (and finite-element-in-space) interpretation. The benefits of this reformulation are discussed, including the derivation of timesteppers that are higher order in time, and the quantifiable dissipative SP properties in the non-ideal, resistive case.
        
        We discuss possible options for extending this FET approach to timesteppers for the compressible case.

        The kinetic corrections satisfy linearized Boltzmann equations. Using a Lénard--Bernstein collision operator, these take Fokker--Planck-like forms \cite{Fokker_1914, Planck_1917} wherein pseudo-particles in the numerical model obey the neoclassical transport equations, with particle-independent Brownian drift terms. This offers a rigorous methodology for incorporating collisions into the particle transport model, without coupling the equations of motions for each particle.
        
        Works by Chen, Chacón et al. \cite{Chen_Chacón_Barnes_2011, Chacón_Chen_Barnes_2013, Chen_Chacón_2014, Chen_Chacón_2015} have developed structure-preserving particle pushers for neoclassical transport in the Vlasov equations, derived from Crank--Nicolson integrators. We show these too can can derive from a FET interpretation, similarly offering potential extensions to higher-order-in-time particle pushers. The FET formulation is used also to consider how the stochastic drift terms can be incorporated into the pushers. Stochastic gyrokinetic expansions are also discussed.

        Different options for the numerical implementation of these schemes are considered.

        Due to the efficacy of FET in the development of SP timesteppers for both the fluid and kinetic component, we hope this approach will prove effective in the future for developing SP timesteppers for the full hybrid model. We hope this will give us the opportunity to incorporate previously inaccessible kinetic effects into the highly effective, modern, finite-element MHD models.
    \end{abstract}
    
    
    \newpage
    \tableofcontents
    
    
    \newpage
    \pagenumbering{arabic}
    %\linenumbers\renewcommand\thelinenumber{\color{black!50}\arabic{linenumber}}
            \input{0 - introduction/main.tex}
        \part{Research}
            \input{1 - low-noise PiC models/main.tex}
            \input{2 - kinetic component/main.tex}
            \input{3 - fluid component/main.tex}
            \input{4 - numerical implementation/main.tex}
        \part{Project Overview}
            \input{5 - research plan/main.tex}
            \input{6 - summary/main.tex}
    
    
    %\section{}
    \newpage
    \pagenumbering{gobble}
        \printbibliography


    \newpage
    \pagenumbering{roman}
    \appendix
        \part{Appendices}
            \input{8 - Hilbert complexes/main.tex}
            \input{9 - weak conservation proofs/main.tex}
\end{document}


\title{\BA{Title in Progress...}}
\author{Boris Andrews}
\affil{Mathematical Institute, University of Oxford}
\date{\today}


\begin{document}
    \pagenumbering{gobble}
    \maketitle
    
    
    \begin{abstract}
        Magnetic confinement reactors---in particular tokamaks---offer one of the most promising options for achieving practical nuclear fusion, with the potential to provide virtually limitless, clean energy. The theoretical and numerical modeling of tokamak plasmas is simultaneously an essential component of effective reactor design, and a great research barrier. Tokamak operational conditions exhibit comparatively low Knudsen numbers. Kinetic effects, including kinetic waves and instabilities, Landau damping, bump-on-tail instabilities and more, are therefore highly influential in tokamak plasma dynamics. Purely fluid models are inherently incapable of capturing these effects, whereas the high dimensionality in purely kinetic models render them practically intractable for most relevant purposes.

        We consider a $\delta\!f$ decomposition model, with a macroscopic fluid background and microscopic kinetic correction, both fully coupled to each other. A similar manner of discretization is proposed to that used in the recent \texttt{STRUPHY} code \cite{Holderied_Possanner_Wang_2021, Holderied_2022, Li_et_al_2023} with a finite-element model for the background and a pseudo-particle/PiC model for the correction.

        The fluid background satisfies the full, non-linear, resistive, compressible, Hall MHD equations. \cite{Laakmann_Hu_Farrell_2022} introduces finite-element(-in-space) implicit timesteppers for the incompressible analogue to this system with structure-preserving (SP) properties in the ideal case, alongside parameter-robust preconditioners. We show that these timesteppers can derive from a finite-element-in-time (FET) (and finite-element-in-space) interpretation. The benefits of this reformulation are discussed, including the derivation of timesteppers that are higher order in time, and the quantifiable dissipative SP properties in the non-ideal, resistive case.
        
        We discuss possible options for extending this FET approach to timesteppers for the compressible case.

        The kinetic corrections satisfy linearized Boltzmann equations. Using a Lénard--Bernstein collision operator, these take Fokker--Planck-like forms \cite{Fokker_1914, Planck_1917} wherein pseudo-particles in the numerical model obey the neoclassical transport equations, with particle-independent Brownian drift terms. This offers a rigorous methodology for incorporating collisions into the particle transport model, without coupling the equations of motions for each particle.
        
        Works by Chen, Chacón et al. \cite{Chen_Chacón_Barnes_2011, Chacón_Chen_Barnes_2013, Chen_Chacón_2014, Chen_Chacón_2015} have developed structure-preserving particle pushers for neoclassical transport in the Vlasov equations, derived from Crank--Nicolson integrators. We show these too can can derive from a FET interpretation, similarly offering potential extensions to higher-order-in-time particle pushers. The FET formulation is used also to consider how the stochastic drift terms can be incorporated into the pushers. Stochastic gyrokinetic expansions are also discussed.

        Different options for the numerical implementation of these schemes are considered.

        Due to the efficacy of FET in the development of SP timesteppers for both the fluid and kinetic component, we hope this approach will prove effective in the future for developing SP timesteppers for the full hybrid model. We hope this will give us the opportunity to incorporate previously inaccessible kinetic effects into the highly effective, modern, finite-element MHD models.
    \end{abstract}
    
    
    \newpage
    \tableofcontents
    
    
    \newpage
    \pagenumbering{arabic}
    %\linenumbers\renewcommand\thelinenumber{\color{black!50}\arabic{linenumber}}
            \documentclass[12pt, a4paper]{report}

\input{template/main.tex}

\title{\BA{Title in Progress...}}
\author{Boris Andrews}
\affil{Mathematical Institute, University of Oxford}
\date{\today}


\begin{document}
    \pagenumbering{gobble}
    \maketitle
    
    
    \begin{abstract}
        Magnetic confinement reactors---in particular tokamaks---offer one of the most promising options for achieving practical nuclear fusion, with the potential to provide virtually limitless, clean energy. The theoretical and numerical modeling of tokamak plasmas is simultaneously an essential component of effective reactor design, and a great research barrier. Tokamak operational conditions exhibit comparatively low Knudsen numbers. Kinetic effects, including kinetic waves and instabilities, Landau damping, bump-on-tail instabilities and more, are therefore highly influential in tokamak plasma dynamics. Purely fluid models are inherently incapable of capturing these effects, whereas the high dimensionality in purely kinetic models render them practically intractable for most relevant purposes.

        We consider a $\delta\!f$ decomposition model, with a macroscopic fluid background and microscopic kinetic correction, both fully coupled to each other. A similar manner of discretization is proposed to that used in the recent \texttt{STRUPHY} code \cite{Holderied_Possanner_Wang_2021, Holderied_2022, Li_et_al_2023} with a finite-element model for the background and a pseudo-particle/PiC model for the correction.

        The fluid background satisfies the full, non-linear, resistive, compressible, Hall MHD equations. \cite{Laakmann_Hu_Farrell_2022} introduces finite-element(-in-space) implicit timesteppers for the incompressible analogue to this system with structure-preserving (SP) properties in the ideal case, alongside parameter-robust preconditioners. We show that these timesteppers can derive from a finite-element-in-time (FET) (and finite-element-in-space) interpretation. The benefits of this reformulation are discussed, including the derivation of timesteppers that are higher order in time, and the quantifiable dissipative SP properties in the non-ideal, resistive case.
        
        We discuss possible options for extending this FET approach to timesteppers for the compressible case.

        The kinetic corrections satisfy linearized Boltzmann equations. Using a Lénard--Bernstein collision operator, these take Fokker--Planck-like forms \cite{Fokker_1914, Planck_1917} wherein pseudo-particles in the numerical model obey the neoclassical transport equations, with particle-independent Brownian drift terms. This offers a rigorous methodology for incorporating collisions into the particle transport model, without coupling the equations of motions for each particle.
        
        Works by Chen, Chacón et al. \cite{Chen_Chacón_Barnes_2011, Chacón_Chen_Barnes_2013, Chen_Chacón_2014, Chen_Chacón_2015} have developed structure-preserving particle pushers for neoclassical transport in the Vlasov equations, derived from Crank--Nicolson integrators. We show these too can can derive from a FET interpretation, similarly offering potential extensions to higher-order-in-time particle pushers. The FET formulation is used also to consider how the stochastic drift terms can be incorporated into the pushers. Stochastic gyrokinetic expansions are also discussed.

        Different options for the numerical implementation of these schemes are considered.

        Due to the efficacy of FET in the development of SP timesteppers for both the fluid and kinetic component, we hope this approach will prove effective in the future for developing SP timesteppers for the full hybrid model. We hope this will give us the opportunity to incorporate previously inaccessible kinetic effects into the highly effective, modern, finite-element MHD models.
    \end{abstract}
    
    
    \newpage
    \tableofcontents
    
    
    \newpage
    \pagenumbering{arabic}
    %\linenumbers\renewcommand\thelinenumber{\color{black!50}\arabic{linenumber}}
            \input{0 - introduction/main.tex}
        \part{Research}
            \input{1 - low-noise PiC models/main.tex}
            \input{2 - kinetic component/main.tex}
            \input{3 - fluid component/main.tex}
            \input{4 - numerical implementation/main.tex}
        \part{Project Overview}
            \input{5 - research plan/main.tex}
            \input{6 - summary/main.tex}
    
    
    %\section{}
    \newpage
    \pagenumbering{gobble}
        \printbibliography


    \newpage
    \pagenumbering{roman}
    \appendix
        \part{Appendices}
            \input{8 - Hilbert complexes/main.tex}
            \input{9 - weak conservation proofs/main.tex}
\end{document}

        \part{Research}
            \documentclass[12pt, a4paper]{report}

\input{template/main.tex}

\title{\BA{Title in Progress...}}
\author{Boris Andrews}
\affil{Mathematical Institute, University of Oxford}
\date{\today}


\begin{document}
    \pagenumbering{gobble}
    \maketitle
    
    
    \begin{abstract}
        Magnetic confinement reactors---in particular tokamaks---offer one of the most promising options for achieving practical nuclear fusion, with the potential to provide virtually limitless, clean energy. The theoretical and numerical modeling of tokamak plasmas is simultaneously an essential component of effective reactor design, and a great research barrier. Tokamak operational conditions exhibit comparatively low Knudsen numbers. Kinetic effects, including kinetic waves and instabilities, Landau damping, bump-on-tail instabilities and more, are therefore highly influential in tokamak plasma dynamics. Purely fluid models are inherently incapable of capturing these effects, whereas the high dimensionality in purely kinetic models render them practically intractable for most relevant purposes.

        We consider a $\delta\!f$ decomposition model, with a macroscopic fluid background and microscopic kinetic correction, both fully coupled to each other. A similar manner of discretization is proposed to that used in the recent \texttt{STRUPHY} code \cite{Holderied_Possanner_Wang_2021, Holderied_2022, Li_et_al_2023} with a finite-element model for the background and a pseudo-particle/PiC model for the correction.

        The fluid background satisfies the full, non-linear, resistive, compressible, Hall MHD equations. \cite{Laakmann_Hu_Farrell_2022} introduces finite-element(-in-space) implicit timesteppers for the incompressible analogue to this system with structure-preserving (SP) properties in the ideal case, alongside parameter-robust preconditioners. We show that these timesteppers can derive from a finite-element-in-time (FET) (and finite-element-in-space) interpretation. The benefits of this reformulation are discussed, including the derivation of timesteppers that are higher order in time, and the quantifiable dissipative SP properties in the non-ideal, resistive case.
        
        We discuss possible options for extending this FET approach to timesteppers for the compressible case.

        The kinetic corrections satisfy linearized Boltzmann equations. Using a Lénard--Bernstein collision operator, these take Fokker--Planck-like forms \cite{Fokker_1914, Planck_1917} wherein pseudo-particles in the numerical model obey the neoclassical transport equations, with particle-independent Brownian drift terms. This offers a rigorous methodology for incorporating collisions into the particle transport model, without coupling the equations of motions for each particle.
        
        Works by Chen, Chacón et al. \cite{Chen_Chacón_Barnes_2011, Chacón_Chen_Barnes_2013, Chen_Chacón_2014, Chen_Chacón_2015} have developed structure-preserving particle pushers for neoclassical transport in the Vlasov equations, derived from Crank--Nicolson integrators. We show these too can can derive from a FET interpretation, similarly offering potential extensions to higher-order-in-time particle pushers. The FET formulation is used also to consider how the stochastic drift terms can be incorporated into the pushers. Stochastic gyrokinetic expansions are also discussed.

        Different options for the numerical implementation of these schemes are considered.

        Due to the efficacy of FET in the development of SP timesteppers for both the fluid and kinetic component, we hope this approach will prove effective in the future for developing SP timesteppers for the full hybrid model. We hope this will give us the opportunity to incorporate previously inaccessible kinetic effects into the highly effective, modern, finite-element MHD models.
    \end{abstract}
    
    
    \newpage
    \tableofcontents
    
    
    \newpage
    \pagenumbering{arabic}
    %\linenumbers\renewcommand\thelinenumber{\color{black!50}\arabic{linenumber}}
            \input{0 - introduction/main.tex}
        \part{Research}
            \input{1 - low-noise PiC models/main.tex}
            \input{2 - kinetic component/main.tex}
            \input{3 - fluid component/main.tex}
            \input{4 - numerical implementation/main.tex}
        \part{Project Overview}
            \input{5 - research plan/main.tex}
            \input{6 - summary/main.tex}
    
    
    %\section{}
    \newpage
    \pagenumbering{gobble}
        \printbibliography


    \newpage
    \pagenumbering{roman}
    \appendix
        \part{Appendices}
            \input{8 - Hilbert complexes/main.tex}
            \input{9 - weak conservation proofs/main.tex}
\end{document}

            \documentclass[12pt, a4paper]{report}

\input{template/main.tex}

\title{\BA{Title in Progress...}}
\author{Boris Andrews}
\affil{Mathematical Institute, University of Oxford}
\date{\today}


\begin{document}
    \pagenumbering{gobble}
    \maketitle
    
    
    \begin{abstract}
        Magnetic confinement reactors---in particular tokamaks---offer one of the most promising options for achieving practical nuclear fusion, with the potential to provide virtually limitless, clean energy. The theoretical and numerical modeling of tokamak plasmas is simultaneously an essential component of effective reactor design, and a great research barrier. Tokamak operational conditions exhibit comparatively low Knudsen numbers. Kinetic effects, including kinetic waves and instabilities, Landau damping, bump-on-tail instabilities and more, are therefore highly influential in tokamak plasma dynamics. Purely fluid models are inherently incapable of capturing these effects, whereas the high dimensionality in purely kinetic models render them practically intractable for most relevant purposes.

        We consider a $\delta\!f$ decomposition model, with a macroscopic fluid background and microscopic kinetic correction, both fully coupled to each other. A similar manner of discretization is proposed to that used in the recent \texttt{STRUPHY} code \cite{Holderied_Possanner_Wang_2021, Holderied_2022, Li_et_al_2023} with a finite-element model for the background and a pseudo-particle/PiC model for the correction.

        The fluid background satisfies the full, non-linear, resistive, compressible, Hall MHD equations. \cite{Laakmann_Hu_Farrell_2022} introduces finite-element(-in-space) implicit timesteppers for the incompressible analogue to this system with structure-preserving (SP) properties in the ideal case, alongside parameter-robust preconditioners. We show that these timesteppers can derive from a finite-element-in-time (FET) (and finite-element-in-space) interpretation. The benefits of this reformulation are discussed, including the derivation of timesteppers that are higher order in time, and the quantifiable dissipative SP properties in the non-ideal, resistive case.
        
        We discuss possible options for extending this FET approach to timesteppers for the compressible case.

        The kinetic corrections satisfy linearized Boltzmann equations. Using a Lénard--Bernstein collision operator, these take Fokker--Planck-like forms \cite{Fokker_1914, Planck_1917} wherein pseudo-particles in the numerical model obey the neoclassical transport equations, with particle-independent Brownian drift terms. This offers a rigorous methodology for incorporating collisions into the particle transport model, without coupling the equations of motions for each particle.
        
        Works by Chen, Chacón et al. \cite{Chen_Chacón_Barnes_2011, Chacón_Chen_Barnes_2013, Chen_Chacón_2014, Chen_Chacón_2015} have developed structure-preserving particle pushers for neoclassical transport in the Vlasov equations, derived from Crank--Nicolson integrators. We show these too can can derive from a FET interpretation, similarly offering potential extensions to higher-order-in-time particle pushers. The FET formulation is used also to consider how the stochastic drift terms can be incorporated into the pushers. Stochastic gyrokinetic expansions are also discussed.

        Different options for the numerical implementation of these schemes are considered.

        Due to the efficacy of FET in the development of SP timesteppers for both the fluid and kinetic component, we hope this approach will prove effective in the future for developing SP timesteppers for the full hybrid model. We hope this will give us the opportunity to incorporate previously inaccessible kinetic effects into the highly effective, modern, finite-element MHD models.
    \end{abstract}
    
    
    \newpage
    \tableofcontents
    
    
    \newpage
    \pagenumbering{arabic}
    %\linenumbers\renewcommand\thelinenumber{\color{black!50}\arabic{linenumber}}
            \input{0 - introduction/main.tex}
        \part{Research}
            \input{1 - low-noise PiC models/main.tex}
            \input{2 - kinetic component/main.tex}
            \input{3 - fluid component/main.tex}
            \input{4 - numerical implementation/main.tex}
        \part{Project Overview}
            \input{5 - research plan/main.tex}
            \input{6 - summary/main.tex}
    
    
    %\section{}
    \newpage
    \pagenumbering{gobble}
        \printbibliography


    \newpage
    \pagenumbering{roman}
    \appendix
        \part{Appendices}
            \input{8 - Hilbert complexes/main.tex}
            \input{9 - weak conservation proofs/main.tex}
\end{document}

            \documentclass[12pt, a4paper]{report}

\input{template/main.tex}

\title{\BA{Title in Progress...}}
\author{Boris Andrews}
\affil{Mathematical Institute, University of Oxford}
\date{\today}


\begin{document}
    \pagenumbering{gobble}
    \maketitle
    
    
    \begin{abstract}
        Magnetic confinement reactors---in particular tokamaks---offer one of the most promising options for achieving practical nuclear fusion, with the potential to provide virtually limitless, clean energy. The theoretical and numerical modeling of tokamak plasmas is simultaneously an essential component of effective reactor design, and a great research barrier. Tokamak operational conditions exhibit comparatively low Knudsen numbers. Kinetic effects, including kinetic waves and instabilities, Landau damping, bump-on-tail instabilities and more, are therefore highly influential in tokamak plasma dynamics. Purely fluid models are inherently incapable of capturing these effects, whereas the high dimensionality in purely kinetic models render them practically intractable for most relevant purposes.

        We consider a $\delta\!f$ decomposition model, with a macroscopic fluid background and microscopic kinetic correction, both fully coupled to each other. A similar manner of discretization is proposed to that used in the recent \texttt{STRUPHY} code \cite{Holderied_Possanner_Wang_2021, Holderied_2022, Li_et_al_2023} with a finite-element model for the background and a pseudo-particle/PiC model for the correction.

        The fluid background satisfies the full, non-linear, resistive, compressible, Hall MHD equations. \cite{Laakmann_Hu_Farrell_2022} introduces finite-element(-in-space) implicit timesteppers for the incompressible analogue to this system with structure-preserving (SP) properties in the ideal case, alongside parameter-robust preconditioners. We show that these timesteppers can derive from a finite-element-in-time (FET) (and finite-element-in-space) interpretation. The benefits of this reformulation are discussed, including the derivation of timesteppers that are higher order in time, and the quantifiable dissipative SP properties in the non-ideal, resistive case.
        
        We discuss possible options for extending this FET approach to timesteppers for the compressible case.

        The kinetic corrections satisfy linearized Boltzmann equations. Using a Lénard--Bernstein collision operator, these take Fokker--Planck-like forms \cite{Fokker_1914, Planck_1917} wherein pseudo-particles in the numerical model obey the neoclassical transport equations, with particle-independent Brownian drift terms. This offers a rigorous methodology for incorporating collisions into the particle transport model, without coupling the equations of motions for each particle.
        
        Works by Chen, Chacón et al. \cite{Chen_Chacón_Barnes_2011, Chacón_Chen_Barnes_2013, Chen_Chacón_2014, Chen_Chacón_2015} have developed structure-preserving particle pushers for neoclassical transport in the Vlasov equations, derived from Crank--Nicolson integrators. We show these too can can derive from a FET interpretation, similarly offering potential extensions to higher-order-in-time particle pushers. The FET formulation is used also to consider how the stochastic drift terms can be incorporated into the pushers. Stochastic gyrokinetic expansions are also discussed.

        Different options for the numerical implementation of these schemes are considered.

        Due to the efficacy of FET in the development of SP timesteppers for both the fluid and kinetic component, we hope this approach will prove effective in the future for developing SP timesteppers for the full hybrid model. We hope this will give us the opportunity to incorporate previously inaccessible kinetic effects into the highly effective, modern, finite-element MHD models.
    \end{abstract}
    
    
    \newpage
    \tableofcontents
    
    
    \newpage
    \pagenumbering{arabic}
    %\linenumbers\renewcommand\thelinenumber{\color{black!50}\arabic{linenumber}}
            \input{0 - introduction/main.tex}
        \part{Research}
            \input{1 - low-noise PiC models/main.tex}
            \input{2 - kinetic component/main.tex}
            \input{3 - fluid component/main.tex}
            \input{4 - numerical implementation/main.tex}
        \part{Project Overview}
            \input{5 - research plan/main.tex}
            \input{6 - summary/main.tex}
    
    
    %\section{}
    \newpage
    \pagenumbering{gobble}
        \printbibliography


    \newpage
    \pagenumbering{roman}
    \appendix
        \part{Appendices}
            \input{8 - Hilbert complexes/main.tex}
            \input{9 - weak conservation proofs/main.tex}
\end{document}

            \documentclass[12pt, a4paper]{report}

\input{template/main.tex}

\title{\BA{Title in Progress...}}
\author{Boris Andrews}
\affil{Mathematical Institute, University of Oxford}
\date{\today}


\begin{document}
    \pagenumbering{gobble}
    \maketitle
    
    
    \begin{abstract}
        Magnetic confinement reactors---in particular tokamaks---offer one of the most promising options for achieving practical nuclear fusion, with the potential to provide virtually limitless, clean energy. The theoretical and numerical modeling of tokamak plasmas is simultaneously an essential component of effective reactor design, and a great research barrier. Tokamak operational conditions exhibit comparatively low Knudsen numbers. Kinetic effects, including kinetic waves and instabilities, Landau damping, bump-on-tail instabilities and more, are therefore highly influential in tokamak plasma dynamics. Purely fluid models are inherently incapable of capturing these effects, whereas the high dimensionality in purely kinetic models render them practically intractable for most relevant purposes.

        We consider a $\delta\!f$ decomposition model, with a macroscopic fluid background and microscopic kinetic correction, both fully coupled to each other. A similar manner of discretization is proposed to that used in the recent \texttt{STRUPHY} code \cite{Holderied_Possanner_Wang_2021, Holderied_2022, Li_et_al_2023} with a finite-element model for the background and a pseudo-particle/PiC model for the correction.

        The fluid background satisfies the full, non-linear, resistive, compressible, Hall MHD equations. \cite{Laakmann_Hu_Farrell_2022} introduces finite-element(-in-space) implicit timesteppers for the incompressible analogue to this system with structure-preserving (SP) properties in the ideal case, alongside parameter-robust preconditioners. We show that these timesteppers can derive from a finite-element-in-time (FET) (and finite-element-in-space) interpretation. The benefits of this reformulation are discussed, including the derivation of timesteppers that are higher order in time, and the quantifiable dissipative SP properties in the non-ideal, resistive case.
        
        We discuss possible options for extending this FET approach to timesteppers for the compressible case.

        The kinetic corrections satisfy linearized Boltzmann equations. Using a Lénard--Bernstein collision operator, these take Fokker--Planck-like forms \cite{Fokker_1914, Planck_1917} wherein pseudo-particles in the numerical model obey the neoclassical transport equations, with particle-independent Brownian drift terms. This offers a rigorous methodology for incorporating collisions into the particle transport model, without coupling the equations of motions for each particle.
        
        Works by Chen, Chacón et al. \cite{Chen_Chacón_Barnes_2011, Chacón_Chen_Barnes_2013, Chen_Chacón_2014, Chen_Chacón_2015} have developed structure-preserving particle pushers for neoclassical transport in the Vlasov equations, derived from Crank--Nicolson integrators. We show these too can can derive from a FET interpretation, similarly offering potential extensions to higher-order-in-time particle pushers. The FET formulation is used also to consider how the stochastic drift terms can be incorporated into the pushers. Stochastic gyrokinetic expansions are also discussed.

        Different options for the numerical implementation of these schemes are considered.

        Due to the efficacy of FET in the development of SP timesteppers for both the fluid and kinetic component, we hope this approach will prove effective in the future for developing SP timesteppers for the full hybrid model. We hope this will give us the opportunity to incorporate previously inaccessible kinetic effects into the highly effective, modern, finite-element MHD models.
    \end{abstract}
    
    
    \newpage
    \tableofcontents
    
    
    \newpage
    \pagenumbering{arabic}
    %\linenumbers\renewcommand\thelinenumber{\color{black!50}\arabic{linenumber}}
            \input{0 - introduction/main.tex}
        \part{Research}
            \input{1 - low-noise PiC models/main.tex}
            \input{2 - kinetic component/main.tex}
            \input{3 - fluid component/main.tex}
            \input{4 - numerical implementation/main.tex}
        \part{Project Overview}
            \input{5 - research plan/main.tex}
            \input{6 - summary/main.tex}
    
    
    %\section{}
    \newpage
    \pagenumbering{gobble}
        \printbibliography


    \newpage
    \pagenumbering{roman}
    \appendix
        \part{Appendices}
            \input{8 - Hilbert complexes/main.tex}
            \input{9 - weak conservation proofs/main.tex}
\end{document}

        \part{Project Overview}
            \documentclass[12pt, a4paper]{report}

\input{template/main.tex}

\title{\BA{Title in Progress...}}
\author{Boris Andrews}
\affil{Mathematical Institute, University of Oxford}
\date{\today}


\begin{document}
    \pagenumbering{gobble}
    \maketitle
    
    
    \begin{abstract}
        Magnetic confinement reactors---in particular tokamaks---offer one of the most promising options for achieving practical nuclear fusion, with the potential to provide virtually limitless, clean energy. The theoretical and numerical modeling of tokamak plasmas is simultaneously an essential component of effective reactor design, and a great research barrier. Tokamak operational conditions exhibit comparatively low Knudsen numbers. Kinetic effects, including kinetic waves and instabilities, Landau damping, bump-on-tail instabilities and more, are therefore highly influential in tokamak plasma dynamics. Purely fluid models are inherently incapable of capturing these effects, whereas the high dimensionality in purely kinetic models render them practically intractable for most relevant purposes.

        We consider a $\delta\!f$ decomposition model, with a macroscopic fluid background and microscopic kinetic correction, both fully coupled to each other. A similar manner of discretization is proposed to that used in the recent \texttt{STRUPHY} code \cite{Holderied_Possanner_Wang_2021, Holderied_2022, Li_et_al_2023} with a finite-element model for the background and a pseudo-particle/PiC model for the correction.

        The fluid background satisfies the full, non-linear, resistive, compressible, Hall MHD equations. \cite{Laakmann_Hu_Farrell_2022} introduces finite-element(-in-space) implicit timesteppers for the incompressible analogue to this system with structure-preserving (SP) properties in the ideal case, alongside parameter-robust preconditioners. We show that these timesteppers can derive from a finite-element-in-time (FET) (and finite-element-in-space) interpretation. The benefits of this reformulation are discussed, including the derivation of timesteppers that are higher order in time, and the quantifiable dissipative SP properties in the non-ideal, resistive case.
        
        We discuss possible options for extending this FET approach to timesteppers for the compressible case.

        The kinetic corrections satisfy linearized Boltzmann equations. Using a Lénard--Bernstein collision operator, these take Fokker--Planck-like forms \cite{Fokker_1914, Planck_1917} wherein pseudo-particles in the numerical model obey the neoclassical transport equations, with particle-independent Brownian drift terms. This offers a rigorous methodology for incorporating collisions into the particle transport model, without coupling the equations of motions for each particle.
        
        Works by Chen, Chacón et al. \cite{Chen_Chacón_Barnes_2011, Chacón_Chen_Barnes_2013, Chen_Chacón_2014, Chen_Chacón_2015} have developed structure-preserving particle pushers for neoclassical transport in the Vlasov equations, derived from Crank--Nicolson integrators. We show these too can can derive from a FET interpretation, similarly offering potential extensions to higher-order-in-time particle pushers. The FET formulation is used also to consider how the stochastic drift terms can be incorporated into the pushers. Stochastic gyrokinetic expansions are also discussed.

        Different options for the numerical implementation of these schemes are considered.

        Due to the efficacy of FET in the development of SP timesteppers for both the fluid and kinetic component, we hope this approach will prove effective in the future for developing SP timesteppers for the full hybrid model. We hope this will give us the opportunity to incorporate previously inaccessible kinetic effects into the highly effective, modern, finite-element MHD models.
    \end{abstract}
    
    
    \newpage
    \tableofcontents
    
    
    \newpage
    \pagenumbering{arabic}
    %\linenumbers\renewcommand\thelinenumber{\color{black!50}\arabic{linenumber}}
            \input{0 - introduction/main.tex}
        \part{Research}
            \input{1 - low-noise PiC models/main.tex}
            \input{2 - kinetic component/main.tex}
            \input{3 - fluid component/main.tex}
            \input{4 - numerical implementation/main.tex}
        \part{Project Overview}
            \input{5 - research plan/main.tex}
            \input{6 - summary/main.tex}
    
    
    %\section{}
    \newpage
    \pagenumbering{gobble}
        \printbibliography


    \newpage
    \pagenumbering{roman}
    \appendix
        \part{Appendices}
            \input{8 - Hilbert complexes/main.tex}
            \input{9 - weak conservation proofs/main.tex}
\end{document}

            \documentclass[12pt, a4paper]{report}

\input{template/main.tex}

\title{\BA{Title in Progress...}}
\author{Boris Andrews}
\affil{Mathematical Institute, University of Oxford}
\date{\today}


\begin{document}
    \pagenumbering{gobble}
    \maketitle
    
    
    \begin{abstract}
        Magnetic confinement reactors---in particular tokamaks---offer one of the most promising options for achieving practical nuclear fusion, with the potential to provide virtually limitless, clean energy. The theoretical and numerical modeling of tokamak plasmas is simultaneously an essential component of effective reactor design, and a great research barrier. Tokamak operational conditions exhibit comparatively low Knudsen numbers. Kinetic effects, including kinetic waves and instabilities, Landau damping, bump-on-tail instabilities and more, are therefore highly influential in tokamak plasma dynamics. Purely fluid models are inherently incapable of capturing these effects, whereas the high dimensionality in purely kinetic models render them practically intractable for most relevant purposes.

        We consider a $\delta\!f$ decomposition model, with a macroscopic fluid background and microscopic kinetic correction, both fully coupled to each other. A similar manner of discretization is proposed to that used in the recent \texttt{STRUPHY} code \cite{Holderied_Possanner_Wang_2021, Holderied_2022, Li_et_al_2023} with a finite-element model for the background and a pseudo-particle/PiC model for the correction.

        The fluid background satisfies the full, non-linear, resistive, compressible, Hall MHD equations. \cite{Laakmann_Hu_Farrell_2022} introduces finite-element(-in-space) implicit timesteppers for the incompressible analogue to this system with structure-preserving (SP) properties in the ideal case, alongside parameter-robust preconditioners. We show that these timesteppers can derive from a finite-element-in-time (FET) (and finite-element-in-space) interpretation. The benefits of this reformulation are discussed, including the derivation of timesteppers that are higher order in time, and the quantifiable dissipative SP properties in the non-ideal, resistive case.
        
        We discuss possible options for extending this FET approach to timesteppers for the compressible case.

        The kinetic corrections satisfy linearized Boltzmann equations. Using a Lénard--Bernstein collision operator, these take Fokker--Planck-like forms \cite{Fokker_1914, Planck_1917} wherein pseudo-particles in the numerical model obey the neoclassical transport equations, with particle-independent Brownian drift terms. This offers a rigorous methodology for incorporating collisions into the particle transport model, without coupling the equations of motions for each particle.
        
        Works by Chen, Chacón et al. \cite{Chen_Chacón_Barnes_2011, Chacón_Chen_Barnes_2013, Chen_Chacón_2014, Chen_Chacón_2015} have developed structure-preserving particle pushers for neoclassical transport in the Vlasov equations, derived from Crank--Nicolson integrators. We show these too can can derive from a FET interpretation, similarly offering potential extensions to higher-order-in-time particle pushers. The FET formulation is used also to consider how the stochastic drift terms can be incorporated into the pushers. Stochastic gyrokinetic expansions are also discussed.

        Different options for the numerical implementation of these schemes are considered.

        Due to the efficacy of FET in the development of SP timesteppers for both the fluid and kinetic component, we hope this approach will prove effective in the future for developing SP timesteppers for the full hybrid model. We hope this will give us the opportunity to incorporate previously inaccessible kinetic effects into the highly effective, modern, finite-element MHD models.
    \end{abstract}
    
    
    \newpage
    \tableofcontents
    
    
    \newpage
    \pagenumbering{arabic}
    %\linenumbers\renewcommand\thelinenumber{\color{black!50}\arabic{linenumber}}
            \input{0 - introduction/main.tex}
        \part{Research}
            \input{1 - low-noise PiC models/main.tex}
            \input{2 - kinetic component/main.tex}
            \input{3 - fluid component/main.tex}
            \input{4 - numerical implementation/main.tex}
        \part{Project Overview}
            \input{5 - research plan/main.tex}
            \input{6 - summary/main.tex}
    
    
    %\section{}
    \newpage
    \pagenumbering{gobble}
        \printbibliography


    \newpage
    \pagenumbering{roman}
    \appendix
        \part{Appendices}
            \input{8 - Hilbert complexes/main.tex}
            \input{9 - weak conservation proofs/main.tex}
\end{document}

    
    
    %\section{}
    \newpage
    \pagenumbering{gobble}
        \printbibliography


    \newpage
    \pagenumbering{roman}
    \appendix
        \part{Appendices}
            \documentclass[12pt, a4paper]{report}

\input{template/main.tex}

\title{\BA{Title in Progress...}}
\author{Boris Andrews}
\affil{Mathematical Institute, University of Oxford}
\date{\today}


\begin{document}
    \pagenumbering{gobble}
    \maketitle
    
    
    \begin{abstract}
        Magnetic confinement reactors---in particular tokamaks---offer one of the most promising options for achieving practical nuclear fusion, with the potential to provide virtually limitless, clean energy. The theoretical and numerical modeling of tokamak plasmas is simultaneously an essential component of effective reactor design, and a great research barrier. Tokamak operational conditions exhibit comparatively low Knudsen numbers. Kinetic effects, including kinetic waves and instabilities, Landau damping, bump-on-tail instabilities and more, are therefore highly influential in tokamak plasma dynamics. Purely fluid models are inherently incapable of capturing these effects, whereas the high dimensionality in purely kinetic models render them practically intractable for most relevant purposes.

        We consider a $\delta\!f$ decomposition model, with a macroscopic fluid background and microscopic kinetic correction, both fully coupled to each other. A similar manner of discretization is proposed to that used in the recent \texttt{STRUPHY} code \cite{Holderied_Possanner_Wang_2021, Holderied_2022, Li_et_al_2023} with a finite-element model for the background and a pseudo-particle/PiC model for the correction.

        The fluid background satisfies the full, non-linear, resistive, compressible, Hall MHD equations. \cite{Laakmann_Hu_Farrell_2022} introduces finite-element(-in-space) implicit timesteppers for the incompressible analogue to this system with structure-preserving (SP) properties in the ideal case, alongside parameter-robust preconditioners. We show that these timesteppers can derive from a finite-element-in-time (FET) (and finite-element-in-space) interpretation. The benefits of this reformulation are discussed, including the derivation of timesteppers that are higher order in time, and the quantifiable dissipative SP properties in the non-ideal, resistive case.
        
        We discuss possible options for extending this FET approach to timesteppers for the compressible case.

        The kinetic corrections satisfy linearized Boltzmann equations. Using a Lénard--Bernstein collision operator, these take Fokker--Planck-like forms \cite{Fokker_1914, Planck_1917} wherein pseudo-particles in the numerical model obey the neoclassical transport equations, with particle-independent Brownian drift terms. This offers a rigorous methodology for incorporating collisions into the particle transport model, without coupling the equations of motions for each particle.
        
        Works by Chen, Chacón et al. \cite{Chen_Chacón_Barnes_2011, Chacón_Chen_Barnes_2013, Chen_Chacón_2014, Chen_Chacón_2015} have developed structure-preserving particle pushers for neoclassical transport in the Vlasov equations, derived from Crank--Nicolson integrators. We show these too can can derive from a FET interpretation, similarly offering potential extensions to higher-order-in-time particle pushers. The FET formulation is used also to consider how the stochastic drift terms can be incorporated into the pushers. Stochastic gyrokinetic expansions are also discussed.

        Different options for the numerical implementation of these schemes are considered.

        Due to the efficacy of FET in the development of SP timesteppers for both the fluid and kinetic component, we hope this approach will prove effective in the future for developing SP timesteppers for the full hybrid model. We hope this will give us the opportunity to incorporate previously inaccessible kinetic effects into the highly effective, modern, finite-element MHD models.
    \end{abstract}
    
    
    \newpage
    \tableofcontents
    
    
    \newpage
    \pagenumbering{arabic}
    %\linenumbers\renewcommand\thelinenumber{\color{black!50}\arabic{linenumber}}
            \input{0 - introduction/main.tex}
        \part{Research}
            \input{1 - low-noise PiC models/main.tex}
            \input{2 - kinetic component/main.tex}
            \input{3 - fluid component/main.tex}
            \input{4 - numerical implementation/main.tex}
        \part{Project Overview}
            \input{5 - research plan/main.tex}
            \input{6 - summary/main.tex}
    
    
    %\section{}
    \newpage
    \pagenumbering{gobble}
        \printbibliography


    \newpage
    \pagenumbering{roman}
    \appendix
        \part{Appendices}
            \input{8 - Hilbert complexes/main.tex}
            \input{9 - weak conservation proofs/main.tex}
\end{document}

            \documentclass[12pt, a4paper]{report}

\input{template/main.tex}

\title{\BA{Title in Progress...}}
\author{Boris Andrews}
\affil{Mathematical Institute, University of Oxford}
\date{\today}


\begin{document}
    \pagenumbering{gobble}
    \maketitle
    
    
    \begin{abstract}
        Magnetic confinement reactors---in particular tokamaks---offer one of the most promising options for achieving practical nuclear fusion, with the potential to provide virtually limitless, clean energy. The theoretical and numerical modeling of tokamak plasmas is simultaneously an essential component of effective reactor design, and a great research barrier. Tokamak operational conditions exhibit comparatively low Knudsen numbers. Kinetic effects, including kinetic waves and instabilities, Landau damping, bump-on-tail instabilities and more, are therefore highly influential in tokamak plasma dynamics. Purely fluid models are inherently incapable of capturing these effects, whereas the high dimensionality in purely kinetic models render them practically intractable for most relevant purposes.

        We consider a $\delta\!f$ decomposition model, with a macroscopic fluid background and microscopic kinetic correction, both fully coupled to each other. A similar manner of discretization is proposed to that used in the recent \texttt{STRUPHY} code \cite{Holderied_Possanner_Wang_2021, Holderied_2022, Li_et_al_2023} with a finite-element model for the background and a pseudo-particle/PiC model for the correction.

        The fluid background satisfies the full, non-linear, resistive, compressible, Hall MHD equations. \cite{Laakmann_Hu_Farrell_2022} introduces finite-element(-in-space) implicit timesteppers for the incompressible analogue to this system with structure-preserving (SP) properties in the ideal case, alongside parameter-robust preconditioners. We show that these timesteppers can derive from a finite-element-in-time (FET) (and finite-element-in-space) interpretation. The benefits of this reformulation are discussed, including the derivation of timesteppers that are higher order in time, and the quantifiable dissipative SP properties in the non-ideal, resistive case.
        
        We discuss possible options for extending this FET approach to timesteppers for the compressible case.

        The kinetic corrections satisfy linearized Boltzmann equations. Using a Lénard--Bernstein collision operator, these take Fokker--Planck-like forms \cite{Fokker_1914, Planck_1917} wherein pseudo-particles in the numerical model obey the neoclassical transport equations, with particle-independent Brownian drift terms. This offers a rigorous methodology for incorporating collisions into the particle transport model, without coupling the equations of motions for each particle.
        
        Works by Chen, Chacón et al. \cite{Chen_Chacón_Barnes_2011, Chacón_Chen_Barnes_2013, Chen_Chacón_2014, Chen_Chacón_2015} have developed structure-preserving particle pushers for neoclassical transport in the Vlasov equations, derived from Crank--Nicolson integrators. We show these too can can derive from a FET interpretation, similarly offering potential extensions to higher-order-in-time particle pushers. The FET formulation is used also to consider how the stochastic drift terms can be incorporated into the pushers. Stochastic gyrokinetic expansions are also discussed.

        Different options for the numerical implementation of these schemes are considered.

        Due to the efficacy of FET in the development of SP timesteppers for both the fluid and kinetic component, we hope this approach will prove effective in the future for developing SP timesteppers for the full hybrid model. We hope this will give us the opportunity to incorporate previously inaccessible kinetic effects into the highly effective, modern, finite-element MHD models.
    \end{abstract}
    
    
    \newpage
    \tableofcontents
    
    
    \newpage
    \pagenumbering{arabic}
    %\linenumbers\renewcommand\thelinenumber{\color{black!50}\arabic{linenumber}}
            \input{0 - introduction/main.tex}
        \part{Research}
            \input{1 - low-noise PiC models/main.tex}
            \input{2 - kinetic component/main.tex}
            \input{3 - fluid component/main.tex}
            \input{4 - numerical implementation/main.tex}
        \part{Project Overview}
            \input{5 - research plan/main.tex}
            \input{6 - summary/main.tex}
    
    
    %\section{}
    \newpage
    \pagenumbering{gobble}
        \printbibliography


    \newpage
    \pagenumbering{roman}
    \appendix
        \part{Appendices}
            \input{8 - Hilbert complexes/main.tex}
            \input{9 - weak conservation proofs/main.tex}
\end{document}

\end{document}

        \part{Project Overview}
            \documentclass[12pt, a4paper]{report}

\documentclass[12pt, a4paper]{report}

\input{template/main.tex}

\title{\BA{Title in Progress...}}
\author{Boris Andrews}
\affil{Mathematical Institute, University of Oxford}
\date{\today}


\begin{document}
    \pagenumbering{gobble}
    \maketitle
    
    
    \begin{abstract}
        Magnetic confinement reactors---in particular tokamaks---offer one of the most promising options for achieving practical nuclear fusion, with the potential to provide virtually limitless, clean energy. The theoretical and numerical modeling of tokamak plasmas is simultaneously an essential component of effective reactor design, and a great research barrier. Tokamak operational conditions exhibit comparatively low Knudsen numbers. Kinetic effects, including kinetic waves and instabilities, Landau damping, bump-on-tail instabilities and more, are therefore highly influential in tokamak plasma dynamics. Purely fluid models are inherently incapable of capturing these effects, whereas the high dimensionality in purely kinetic models render them practically intractable for most relevant purposes.

        We consider a $\delta\!f$ decomposition model, with a macroscopic fluid background and microscopic kinetic correction, both fully coupled to each other. A similar manner of discretization is proposed to that used in the recent \texttt{STRUPHY} code \cite{Holderied_Possanner_Wang_2021, Holderied_2022, Li_et_al_2023} with a finite-element model for the background and a pseudo-particle/PiC model for the correction.

        The fluid background satisfies the full, non-linear, resistive, compressible, Hall MHD equations. \cite{Laakmann_Hu_Farrell_2022} introduces finite-element(-in-space) implicit timesteppers for the incompressible analogue to this system with structure-preserving (SP) properties in the ideal case, alongside parameter-robust preconditioners. We show that these timesteppers can derive from a finite-element-in-time (FET) (and finite-element-in-space) interpretation. The benefits of this reformulation are discussed, including the derivation of timesteppers that are higher order in time, and the quantifiable dissipative SP properties in the non-ideal, resistive case.
        
        We discuss possible options for extending this FET approach to timesteppers for the compressible case.

        The kinetic corrections satisfy linearized Boltzmann equations. Using a Lénard--Bernstein collision operator, these take Fokker--Planck-like forms \cite{Fokker_1914, Planck_1917} wherein pseudo-particles in the numerical model obey the neoclassical transport equations, with particle-independent Brownian drift terms. This offers a rigorous methodology for incorporating collisions into the particle transport model, without coupling the equations of motions for each particle.
        
        Works by Chen, Chacón et al. \cite{Chen_Chacón_Barnes_2011, Chacón_Chen_Barnes_2013, Chen_Chacón_2014, Chen_Chacón_2015} have developed structure-preserving particle pushers for neoclassical transport in the Vlasov equations, derived from Crank--Nicolson integrators. We show these too can can derive from a FET interpretation, similarly offering potential extensions to higher-order-in-time particle pushers. The FET formulation is used also to consider how the stochastic drift terms can be incorporated into the pushers. Stochastic gyrokinetic expansions are also discussed.

        Different options for the numerical implementation of these schemes are considered.

        Due to the efficacy of FET in the development of SP timesteppers for both the fluid and kinetic component, we hope this approach will prove effective in the future for developing SP timesteppers for the full hybrid model. We hope this will give us the opportunity to incorporate previously inaccessible kinetic effects into the highly effective, modern, finite-element MHD models.
    \end{abstract}
    
    
    \newpage
    \tableofcontents
    
    
    \newpage
    \pagenumbering{arabic}
    %\linenumbers\renewcommand\thelinenumber{\color{black!50}\arabic{linenumber}}
            \input{0 - introduction/main.tex}
        \part{Research}
            \input{1 - low-noise PiC models/main.tex}
            \input{2 - kinetic component/main.tex}
            \input{3 - fluid component/main.tex}
            \input{4 - numerical implementation/main.tex}
        \part{Project Overview}
            \input{5 - research plan/main.tex}
            \input{6 - summary/main.tex}
    
    
    %\section{}
    \newpage
    \pagenumbering{gobble}
        \printbibliography


    \newpage
    \pagenumbering{roman}
    \appendix
        \part{Appendices}
            \input{8 - Hilbert complexes/main.tex}
            \input{9 - weak conservation proofs/main.tex}
\end{document}


\title{\BA{Title in Progress...}}
\author{Boris Andrews}
\affil{Mathematical Institute, University of Oxford}
\date{\today}


\begin{document}
    \pagenumbering{gobble}
    \maketitle
    
    
    \begin{abstract}
        Magnetic confinement reactors---in particular tokamaks---offer one of the most promising options for achieving practical nuclear fusion, with the potential to provide virtually limitless, clean energy. The theoretical and numerical modeling of tokamak plasmas is simultaneously an essential component of effective reactor design, and a great research barrier. Tokamak operational conditions exhibit comparatively low Knudsen numbers. Kinetic effects, including kinetic waves and instabilities, Landau damping, bump-on-tail instabilities and more, are therefore highly influential in tokamak plasma dynamics. Purely fluid models are inherently incapable of capturing these effects, whereas the high dimensionality in purely kinetic models render them practically intractable for most relevant purposes.

        We consider a $\delta\!f$ decomposition model, with a macroscopic fluid background and microscopic kinetic correction, both fully coupled to each other. A similar manner of discretization is proposed to that used in the recent \texttt{STRUPHY} code \cite{Holderied_Possanner_Wang_2021, Holderied_2022, Li_et_al_2023} with a finite-element model for the background and a pseudo-particle/PiC model for the correction.

        The fluid background satisfies the full, non-linear, resistive, compressible, Hall MHD equations. \cite{Laakmann_Hu_Farrell_2022} introduces finite-element(-in-space) implicit timesteppers for the incompressible analogue to this system with structure-preserving (SP) properties in the ideal case, alongside parameter-robust preconditioners. We show that these timesteppers can derive from a finite-element-in-time (FET) (and finite-element-in-space) interpretation. The benefits of this reformulation are discussed, including the derivation of timesteppers that are higher order in time, and the quantifiable dissipative SP properties in the non-ideal, resistive case.
        
        We discuss possible options for extending this FET approach to timesteppers for the compressible case.

        The kinetic corrections satisfy linearized Boltzmann equations. Using a Lénard--Bernstein collision operator, these take Fokker--Planck-like forms \cite{Fokker_1914, Planck_1917} wherein pseudo-particles in the numerical model obey the neoclassical transport equations, with particle-independent Brownian drift terms. This offers a rigorous methodology for incorporating collisions into the particle transport model, without coupling the equations of motions for each particle.
        
        Works by Chen, Chacón et al. \cite{Chen_Chacón_Barnes_2011, Chacón_Chen_Barnes_2013, Chen_Chacón_2014, Chen_Chacón_2015} have developed structure-preserving particle pushers for neoclassical transport in the Vlasov equations, derived from Crank--Nicolson integrators. We show these too can can derive from a FET interpretation, similarly offering potential extensions to higher-order-in-time particle pushers. The FET formulation is used also to consider how the stochastic drift terms can be incorporated into the pushers. Stochastic gyrokinetic expansions are also discussed.

        Different options for the numerical implementation of these schemes are considered.

        Due to the efficacy of FET in the development of SP timesteppers for both the fluid and kinetic component, we hope this approach will prove effective in the future for developing SP timesteppers for the full hybrid model. We hope this will give us the opportunity to incorporate previously inaccessible kinetic effects into the highly effective, modern, finite-element MHD models.
    \end{abstract}
    
    
    \newpage
    \tableofcontents
    
    
    \newpage
    \pagenumbering{arabic}
    %\linenumbers\renewcommand\thelinenumber{\color{black!50}\arabic{linenumber}}
            \documentclass[12pt, a4paper]{report}

\input{template/main.tex}

\title{\BA{Title in Progress...}}
\author{Boris Andrews}
\affil{Mathematical Institute, University of Oxford}
\date{\today}


\begin{document}
    \pagenumbering{gobble}
    \maketitle
    
    
    \begin{abstract}
        Magnetic confinement reactors---in particular tokamaks---offer one of the most promising options for achieving practical nuclear fusion, with the potential to provide virtually limitless, clean energy. The theoretical and numerical modeling of tokamak plasmas is simultaneously an essential component of effective reactor design, and a great research barrier. Tokamak operational conditions exhibit comparatively low Knudsen numbers. Kinetic effects, including kinetic waves and instabilities, Landau damping, bump-on-tail instabilities and more, are therefore highly influential in tokamak plasma dynamics. Purely fluid models are inherently incapable of capturing these effects, whereas the high dimensionality in purely kinetic models render them practically intractable for most relevant purposes.

        We consider a $\delta\!f$ decomposition model, with a macroscopic fluid background and microscopic kinetic correction, both fully coupled to each other. A similar manner of discretization is proposed to that used in the recent \texttt{STRUPHY} code \cite{Holderied_Possanner_Wang_2021, Holderied_2022, Li_et_al_2023} with a finite-element model for the background and a pseudo-particle/PiC model for the correction.

        The fluid background satisfies the full, non-linear, resistive, compressible, Hall MHD equations. \cite{Laakmann_Hu_Farrell_2022} introduces finite-element(-in-space) implicit timesteppers for the incompressible analogue to this system with structure-preserving (SP) properties in the ideal case, alongside parameter-robust preconditioners. We show that these timesteppers can derive from a finite-element-in-time (FET) (and finite-element-in-space) interpretation. The benefits of this reformulation are discussed, including the derivation of timesteppers that are higher order in time, and the quantifiable dissipative SP properties in the non-ideal, resistive case.
        
        We discuss possible options for extending this FET approach to timesteppers for the compressible case.

        The kinetic corrections satisfy linearized Boltzmann equations. Using a Lénard--Bernstein collision operator, these take Fokker--Planck-like forms \cite{Fokker_1914, Planck_1917} wherein pseudo-particles in the numerical model obey the neoclassical transport equations, with particle-independent Brownian drift terms. This offers a rigorous methodology for incorporating collisions into the particle transport model, without coupling the equations of motions for each particle.
        
        Works by Chen, Chacón et al. \cite{Chen_Chacón_Barnes_2011, Chacón_Chen_Barnes_2013, Chen_Chacón_2014, Chen_Chacón_2015} have developed structure-preserving particle pushers for neoclassical transport in the Vlasov equations, derived from Crank--Nicolson integrators. We show these too can can derive from a FET interpretation, similarly offering potential extensions to higher-order-in-time particle pushers. The FET formulation is used also to consider how the stochastic drift terms can be incorporated into the pushers. Stochastic gyrokinetic expansions are also discussed.

        Different options for the numerical implementation of these schemes are considered.

        Due to the efficacy of FET in the development of SP timesteppers for both the fluid and kinetic component, we hope this approach will prove effective in the future for developing SP timesteppers for the full hybrid model. We hope this will give us the opportunity to incorporate previously inaccessible kinetic effects into the highly effective, modern, finite-element MHD models.
    \end{abstract}
    
    
    \newpage
    \tableofcontents
    
    
    \newpage
    \pagenumbering{arabic}
    %\linenumbers\renewcommand\thelinenumber{\color{black!50}\arabic{linenumber}}
            \input{0 - introduction/main.tex}
        \part{Research}
            \input{1 - low-noise PiC models/main.tex}
            \input{2 - kinetic component/main.tex}
            \input{3 - fluid component/main.tex}
            \input{4 - numerical implementation/main.tex}
        \part{Project Overview}
            \input{5 - research plan/main.tex}
            \input{6 - summary/main.tex}
    
    
    %\section{}
    \newpage
    \pagenumbering{gobble}
        \printbibliography


    \newpage
    \pagenumbering{roman}
    \appendix
        \part{Appendices}
            \input{8 - Hilbert complexes/main.tex}
            \input{9 - weak conservation proofs/main.tex}
\end{document}

        \part{Research}
            \documentclass[12pt, a4paper]{report}

\input{template/main.tex}

\title{\BA{Title in Progress...}}
\author{Boris Andrews}
\affil{Mathematical Institute, University of Oxford}
\date{\today}


\begin{document}
    \pagenumbering{gobble}
    \maketitle
    
    
    \begin{abstract}
        Magnetic confinement reactors---in particular tokamaks---offer one of the most promising options for achieving practical nuclear fusion, with the potential to provide virtually limitless, clean energy. The theoretical and numerical modeling of tokamak plasmas is simultaneously an essential component of effective reactor design, and a great research barrier. Tokamak operational conditions exhibit comparatively low Knudsen numbers. Kinetic effects, including kinetic waves and instabilities, Landau damping, bump-on-tail instabilities and more, are therefore highly influential in tokamak plasma dynamics. Purely fluid models are inherently incapable of capturing these effects, whereas the high dimensionality in purely kinetic models render them practically intractable for most relevant purposes.

        We consider a $\delta\!f$ decomposition model, with a macroscopic fluid background and microscopic kinetic correction, both fully coupled to each other. A similar manner of discretization is proposed to that used in the recent \texttt{STRUPHY} code \cite{Holderied_Possanner_Wang_2021, Holderied_2022, Li_et_al_2023} with a finite-element model for the background and a pseudo-particle/PiC model for the correction.

        The fluid background satisfies the full, non-linear, resistive, compressible, Hall MHD equations. \cite{Laakmann_Hu_Farrell_2022} introduces finite-element(-in-space) implicit timesteppers for the incompressible analogue to this system with structure-preserving (SP) properties in the ideal case, alongside parameter-robust preconditioners. We show that these timesteppers can derive from a finite-element-in-time (FET) (and finite-element-in-space) interpretation. The benefits of this reformulation are discussed, including the derivation of timesteppers that are higher order in time, and the quantifiable dissipative SP properties in the non-ideal, resistive case.
        
        We discuss possible options for extending this FET approach to timesteppers for the compressible case.

        The kinetic corrections satisfy linearized Boltzmann equations. Using a Lénard--Bernstein collision operator, these take Fokker--Planck-like forms \cite{Fokker_1914, Planck_1917} wherein pseudo-particles in the numerical model obey the neoclassical transport equations, with particle-independent Brownian drift terms. This offers a rigorous methodology for incorporating collisions into the particle transport model, without coupling the equations of motions for each particle.
        
        Works by Chen, Chacón et al. \cite{Chen_Chacón_Barnes_2011, Chacón_Chen_Barnes_2013, Chen_Chacón_2014, Chen_Chacón_2015} have developed structure-preserving particle pushers for neoclassical transport in the Vlasov equations, derived from Crank--Nicolson integrators. We show these too can can derive from a FET interpretation, similarly offering potential extensions to higher-order-in-time particle pushers. The FET formulation is used also to consider how the stochastic drift terms can be incorporated into the pushers. Stochastic gyrokinetic expansions are also discussed.

        Different options for the numerical implementation of these schemes are considered.

        Due to the efficacy of FET in the development of SP timesteppers for both the fluid and kinetic component, we hope this approach will prove effective in the future for developing SP timesteppers for the full hybrid model. We hope this will give us the opportunity to incorporate previously inaccessible kinetic effects into the highly effective, modern, finite-element MHD models.
    \end{abstract}
    
    
    \newpage
    \tableofcontents
    
    
    \newpage
    \pagenumbering{arabic}
    %\linenumbers\renewcommand\thelinenumber{\color{black!50}\arabic{linenumber}}
            \input{0 - introduction/main.tex}
        \part{Research}
            \input{1 - low-noise PiC models/main.tex}
            \input{2 - kinetic component/main.tex}
            \input{3 - fluid component/main.tex}
            \input{4 - numerical implementation/main.tex}
        \part{Project Overview}
            \input{5 - research plan/main.tex}
            \input{6 - summary/main.tex}
    
    
    %\section{}
    \newpage
    \pagenumbering{gobble}
        \printbibliography


    \newpage
    \pagenumbering{roman}
    \appendix
        \part{Appendices}
            \input{8 - Hilbert complexes/main.tex}
            \input{9 - weak conservation proofs/main.tex}
\end{document}

            \documentclass[12pt, a4paper]{report}

\input{template/main.tex}

\title{\BA{Title in Progress...}}
\author{Boris Andrews}
\affil{Mathematical Institute, University of Oxford}
\date{\today}


\begin{document}
    \pagenumbering{gobble}
    \maketitle
    
    
    \begin{abstract}
        Magnetic confinement reactors---in particular tokamaks---offer one of the most promising options for achieving practical nuclear fusion, with the potential to provide virtually limitless, clean energy. The theoretical and numerical modeling of tokamak plasmas is simultaneously an essential component of effective reactor design, and a great research barrier. Tokamak operational conditions exhibit comparatively low Knudsen numbers. Kinetic effects, including kinetic waves and instabilities, Landau damping, bump-on-tail instabilities and more, are therefore highly influential in tokamak plasma dynamics. Purely fluid models are inherently incapable of capturing these effects, whereas the high dimensionality in purely kinetic models render them practically intractable for most relevant purposes.

        We consider a $\delta\!f$ decomposition model, with a macroscopic fluid background and microscopic kinetic correction, both fully coupled to each other. A similar manner of discretization is proposed to that used in the recent \texttt{STRUPHY} code \cite{Holderied_Possanner_Wang_2021, Holderied_2022, Li_et_al_2023} with a finite-element model for the background and a pseudo-particle/PiC model for the correction.

        The fluid background satisfies the full, non-linear, resistive, compressible, Hall MHD equations. \cite{Laakmann_Hu_Farrell_2022} introduces finite-element(-in-space) implicit timesteppers for the incompressible analogue to this system with structure-preserving (SP) properties in the ideal case, alongside parameter-robust preconditioners. We show that these timesteppers can derive from a finite-element-in-time (FET) (and finite-element-in-space) interpretation. The benefits of this reformulation are discussed, including the derivation of timesteppers that are higher order in time, and the quantifiable dissipative SP properties in the non-ideal, resistive case.
        
        We discuss possible options for extending this FET approach to timesteppers for the compressible case.

        The kinetic corrections satisfy linearized Boltzmann equations. Using a Lénard--Bernstein collision operator, these take Fokker--Planck-like forms \cite{Fokker_1914, Planck_1917} wherein pseudo-particles in the numerical model obey the neoclassical transport equations, with particle-independent Brownian drift terms. This offers a rigorous methodology for incorporating collisions into the particle transport model, without coupling the equations of motions for each particle.
        
        Works by Chen, Chacón et al. \cite{Chen_Chacón_Barnes_2011, Chacón_Chen_Barnes_2013, Chen_Chacón_2014, Chen_Chacón_2015} have developed structure-preserving particle pushers for neoclassical transport in the Vlasov equations, derived from Crank--Nicolson integrators. We show these too can can derive from a FET interpretation, similarly offering potential extensions to higher-order-in-time particle pushers. The FET formulation is used also to consider how the stochastic drift terms can be incorporated into the pushers. Stochastic gyrokinetic expansions are also discussed.

        Different options for the numerical implementation of these schemes are considered.

        Due to the efficacy of FET in the development of SP timesteppers for both the fluid and kinetic component, we hope this approach will prove effective in the future for developing SP timesteppers for the full hybrid model. We hope this will give us the opportunity to incorporate previously inaccessible kinetic effects into the highly effective, modern, finite-element MHD models.
    \end{abstract}
    
    
    \newpage
    \tableofcontents
    
    
    \newpage
    \pagenumbering{arabic}
    %\linenumbers\renewcommand\thelinenumber{\color{black!50}\arabic{linenumber}}
            \input{0 - introduction/main.tex}
        \part{Research}
            \input{1 - low-noise PiC models/main.tex}
            \input{2 - kinetic component/main.tex}
            \input{3 - fluid component/main.tex}
            \input{4 - numerical implementation/main.tex}
        \part{Project Overview}
            \input{5 - research plan/main.tex}
            \input{6 - summary/main.tex}
    
    
    %\section{}
    \newpage
    \pagenumbering{gobble}
        \printbibliography


    \newpage
    \pagenumbering{roman}
    \appendix
        \part{Appendices}
            \input{8 - Hilbert complexes/main.tex}
            \input{9 - weak conservation proofs/main.tex}
\end{document}

            \documentclass[12pt, a4paper]{report}

\input{template/main.tex}

\title{\BA{Title in Progress...}}
\author{Boris Andrews}
\affil{Mathematical Institute, University of Oxford}
\date{\today}


\begin{document}
    \pagenumbering{gobble}
    \maketitle
    
    
    \begin{abstract}
        Magnetic confinement reactors---in particular tokamaks---offer one of the most promising options for achieving practical nuclear fusion, with the potential to provide virtually limitless, clean energy. The theoretical and numerical modeling of tokamak plasmas is simultaneously an essential component of effective reactor design, and a great research barrier. Tokamak operational conditions exhibit comparatively low Knudsen numbers. Kinetic effects, including kinetic waves and instabilities, Landau damping, bump-on-tail instabilities and more, are therefore highly influential in tokamak plasma dynamics. Purely fluid models are inherently incapable of capturing these effects, whereas the high dimensionality in purely kinetic models render them practically intractable for most relevant purposes.

        We consider a $\delta\!f$ decomposition model, with a macroscopic fluid background and microscopic kinetic correction, both fully coupled to each other. A similar manner of discretization is proposed to that used in the recent \texttt{STRUPHY} code \cite{Holderied_Possanner_Wang_2021, Holderied_2022, Li_et_al_2023} with a finite-element model for the background and a pseudo-particle/PiC model for the correction.

        The fluid background satisfies the full, non-linear, resistive, compressible, Hall MHD equations. \cite{Laakmann_Hu_Farrell_2022} introduces finite-element(-in-space) implicit timesteppers for the incompressible analogue to this system with structure-preserving (SP) properties in the ideal case, alongside parameter-robust preconditioners. We show that these timesteppers can derive from a finite-element-in-time (FET) (and finite-element-in-space) interpretation. The benefits of this reformulation are discussed, including the derivation of timesteppers that are higher order in time, and the quantifiable dissipative SP properties in the non-ideal, resistive case.
        
        We discuss possible options for extending this FET approach to timesteppers for the compressible case.

        The kinetic corrections satisfy linearized Boltzmann equations. Using a Lénard--Bernstein collision operator, these take Fokker--Planck-like forms \cite{Fokker_1914, Planck_1917} wherein pseudo-particles in the numerical model obey the neoclassical transport equations, with particle-independent Brownian drift terms. This offers a rigorous methodology for incorporating collisions into the particle transport model, without coupling the equations of motions for each particle.
        
        Works by Chen, Chacón et al. \cite{Chen_Chacón_Barnes_2011, Chacón_Chen_Barnes_2013, Chen_Chacón_2014, Chen_Chacón_2015} have developed structure-preserving particle pushers for neoclassical transport in the Vlasov equations, derived from Crank--Nicolson integrators. We show these too can can derive from a FET interpretation, similarly offering potential extensions to higher-order-in-time particle pushers. The FET formulation is used also to consider how the stochastic drift terms can be incorporated into the pushers. Stochastic gyrokinetic expansions are also discussed.

        Different options for the numerical implementation of these schemes are considered.

        Due to the efficacy of FET in the development of SP timesteppers for both the fluid and kinetic component, we hope this approach will prove effective in the future for developing SP timesteppers for the full hybrid model. We hope this will give us the opportunity to incorporate previously inaccessible kinetic effects into the highly effective, modern, finite-element MHD models.
    \end{abstract}
    
    
    \newpage
    \tableofcontents
    
    
    \newpage
    \pagenumbering{arabic}
    %\linenumbers\renewcommand\thelinenumber{\color{black!50}\arabic{linenumber}}
            \input{0 - introduction/main.tex}
        \part{Research}
            \input{1 - low-noise PiC models/main.tex}
            \input{2 - kinetic component/main.tex}
            \input{3 - fluid component/main.tex}
            \input{4 - numerical implementation/main.tex}
        \part{Project Overview}
            \input{5 - research plan/main.tex}
            \input{6 - summary/main.tex}
    
    
    %\section{}
    \newpage
    \pagenumbering{gobble}
        \printbibliography


    \newpage
    \pagenumbering{roman}
    \appendix
        \part{Appendices}
            \input{8 - Hilbert complexes/main.tex}
            \input{9 - weak conservation proofs/main.tex}
\end{document}

            \documentclass[12pt, a4paper]{report}

\input{template/main.tex}

\title{\BA{Title in Progress...}}
\author{Boris Andrews}
\affil{Mathematical Institute, University of Oxford}
\date{\today}


\begin{document}
    \pagenumbering{gobble}
    \maketitle
    
    
    \begin{abstract}
        Magnetic confinement reactors---in particular tokamaks---offer one of the most promising options for achieving practical nuclear fusion, with the potential to provide virtually limitless, clean energy. The theoretical and numerical modeling of tokamak plasmas is simultaneously an essential component of effective reactor design, and a great research barrier. Tokamak operational conditions exhibit comparatively low Knudsen numbers. Kinetic effects, including kinetic waves and instabilities, Landau damping, bump-on-tail instabilities and more, are therefore highly influential in tokamak plasma dynamics. Purely fluid models are inherently incapable of capturing these effects, whereas the high dimensionality in purely kinetic models render them practically intractable for most relevant purposes.

        We consider a $\delta\!f$ decomposition model, with a macroscopic fluid background and microscopic kinetic correction, both fully coupled to each other. A similar manner of discretization is proposed to that used in the recent \texttt{STRUPHY} code \cite{Holderied_Possanner_Wang_2021, Holderied_2022, Li_et_al_2023} with a finite-element model for the background and a pseudo-particle/PiC model for the correction.

        The fluid background satisfies the full, non-linear, resistive, compressible, Hall MHD equations. \cite{Laakmann_Hu_Farrell_2022} introduces finite-element(-in-space) implicit timesteppers for the incompressible analogue to this system with structure-preserving (SP) properties in the ideal case, alongside parameter-robust preconditioners. We show that these timesteppers can derive from a finite-element-in-time (FET) (and finite-element-in-space) interpretation. The benefits of this reformulation are discussed, including the derivation of timesteppers that are higher order in time, and the quantifiable dissipative SP properties in the non-ideal, resistive case.
        
        We discuss possible options for extending this FET approach to timesteppers for the compressible case.

        The kinetic corrections satisfy linearized Boltzmann equations. Using a Lénard--Bernstein collision operator, these take Fokker--Planck-like forms \cite{Fokker_1914, Planck_1917} wherein pseudo-particles in the numerical model obey the neoclassical transport equations, with particle-independent Brownian drift terms. This offers a rigorous methodology for incorporating collisions into the particle transport model, without coupling the equations of motions for each particle.
        
        Works by Chen, Chacón et al. \cite{Chen_Chacón_Barnes_2011, Chacón_Chen_Barnes_2013, Chen_Chacón_2014, Chen_Chacón_2015} have developed structure-preserving particle pushers for neoclassical transport in the Vlasov equations, derived from Crank--Nicolson integrators. We show these too can can derive from a FET interpretation, similarly offering potential extensions to higher-order-in-time particle pushers. The FET formulation is used also to consider how the stochastic drift terms can be incorporated into the pushers. Stochastic gyrokinetic expansions are also discussed.

        Different options for the numerical implementation of these schemes are considered.

        Due to the efficacy of FET in the development of SP timesteppers for both the fluid and kinetic component, we hope this approach will prove effective in the future for developing SP timesteppers for the full hybrid model. We hope this will give us the opportunity to incorporate previously inaccessible kinetic effects into the highly effective, modern, finite-element MHD models.
    \end{abstract}
    
    
    \newpage
    \tableofcontents
    
    
    \newpage
    \pagenumbering{arabic}
    %\linenumbers\renewcommand\thelinenumber{\color{black!50}\arabic{linenumber}}
            \input{0 - introduction/main.tex}
        \part{Research}
            \input{1 - low-noise PiC models/main.tex}
            \input{2 - kinetic component/main.tex}
            \input{3 - fluid component/main.tex}
            \input{4 - numerical implementation/main.tex}
        \part{Project Overview}
            \input{5 - research plan/main.tex}
            \input{6 - summary/main.tex}
    
    
    %\section{}
    \newpage
    \pagenumbering{gobble}
        \printbibliography


    \newpage
    \pagenumbering{roman}
    \appendix
        \part{Appendices}
            \input{8 - Hilbert complexes/main.tex}
            \input{9 - weak conservation proofs/main.tex}
\end{document}

        \part{Project Overview}
            \documentclass[12pt, a4paper]{report}

\input{template/main.tex}

\title{\BA{Title in Progress...}}
\author{Boris Andrews}
\affil{Mathematical Institute, University of Oxford}
\date{\today}


\begin{document}
    \pagenumbering{gobble}
    \maketitle
    
    
    \begin{abstract}
        Magnetic confinement reactors---in particular tokamaks---offer one of the most promising options for achieving practical nuclear fusion, with the potential to provide virtually limitless, clean energy. The theoretical and numerical modeling of tokamak plasmas is simultaneously an essential component of effective reactor design, and a great research barrier. Tokamak operational conditions exhibit comparatively low Knudsen numbers. Kinetic effects, including kinetic waves and instabilities, Landau damping, bump-on-tail instabilities and more, are therefore highly influential in tokamak plasma dynamics. Purely fluid models are inherently incapable of capturing these effects, whereas the high dimensionality in purely kinetic models render them practically intractable for most relevant purposes.

        We consider a $\delta\!f$ decomposition model, with a macroscopic fluid background and microscopic kinetic correction, both fully coupled to each other. A similar manner of discretization is proposed to that used in the recent \texttt{STRUPHY} code \cite{Holderied_Possanner_Wang_2021, Holderied_2022, Li_et_al_2023} with a finite-element model for the background and a pseudo-particle/PiC model for the correction.

        The fluid background satisfies the full, non-linear, resistive, compressible, Hall MHD equations. \cite{Laakmann_Hu_Farrell_2022} introduces finite-element(-in-space) implicit timesteppers for the incompressible analogue to this system with structure-preserving (SP) properties in the ideal case, alongside parameter-robust preconditioners. We show that these timesteppers can derive from a finite-element-in-time (FET) (and finite-element-in-space) interpretation. The benefits of this reformulation are discussed, including the derivation of timesteppers that are higher order in time, and the quantifiable dissipative SP properties in the non-ideal, resistive case.
        
        We discuss possible options for extending this FET approach to timesteppers for the compressible case.

        The kinetic corrections satisfy linearized Boltzmann equations. Using a Lénard--Bernstein collision operator, these take Fokker--Planck-like forms \cite{Fokker_1914, Planck_1917} wherein pseudo-particles in the numerical model obey the neoclassical transport equations, with particle-independent Brownian drift terms. This offers a rigorous methodology for incorporating collisions into the particle transport model, without coupling the equations of motions for each particle.
        
        Works by Chen, Chacón et al. \cite{Chen_Chacón_Barnes_2011, Chacón_Chen_Barnes_2013, Chen_Chacón_2014, Chen_Chacón_2015} have developed structure-preserving particle pushers for neoclassical transport in the Vlasov equations, derived from Crank--Nicolson integrators. We show these too can can derive from a FET interpretation, similarly offering potential extensions to higher-order-in-time particle pushers. The FET formulation is used also to consider how the stochastic drift terms can be incorporated into the pushers. Stochastic gyrokinetic expansions are also discussed.

        Different options for the numerical implementation of these schemes are considered.

        Due to the efficacy of FET in the development of SP timesteppers for both the fluid and kinetic component, we hope this approach will prove effective in the future for developing SP timesteppers for the full hybrid model. We hope this will give us the opportunity to incorporate previously inaccessible kinetic effects into the highly effective, modern, finite-element MHD models.
    \end{abstract}
    
    
    \newpage
    \tableofcontents
    
    
    \newpage
    \pagenumbering{arabic}
    %\linenumbers\renewcommand\thelinenumber{\color{black!50}\arabic{linenumber}}
            \input{0 - introduction/main.tex}
        \part{Research}
            \input{1 - low-noise PiC models/main.tex}
            \input{2 - kinetic component/main.tex}
            \input{3 - fluid component/main.tex}
            \input{4 - numerical implementation/main.tex}
        \part{Project Overview}
            \input{5 - research plan/main.tex}
            \input{6 - summary/main.tex}
    
    
    %\section{}
    \newpage
    \pagenumbering{gobble}
        \printbibliography


    \newpage
    \pagenumbering{roman}
    \appendix
        \part{Appendices}
            \input{8 - Hilbert complexes/main.tex}
            \input{9 - weak conservation proofs/main.tex}
\end{document}

            \documentclass[12pt, a4paper]{report}

\input{template/main.tex}

\title{\BA{Title in Progress...}}
\author{Boris Andrews}
\affil{Mathematical Institute, University of Oxford}
\date{\today}


\begin{document}
    \pagenumbering{gobble}
    \maketitle
    
    
    \begin{abstract}
        Magnetic confinement reactors---in particular tokamaks---offer one of the most promising options for achieving practical nuclear fusion, with the potential to provide virtually limitless, clean energy. The theoretical and numerical modeling of tokamak plasmas is simultaneously an essential component of effective reactor design, and a great research barrier. Tokamak operational conditions exhibit comparatively low Knudsen numbers. Kinetic effects, including kinetic waves and instabilities, Landau damping, bump-on-tail instabilities and more, are therefore highly influential in tokamak plasma dynamics. Purely fluid models are inherently incapable of capturing these effects, whereas the high dimensionality in purely kinetic models render them practically intractable for most relevant purposes.

        We consider a $\delta\!f$ decomposition model, with a macroscopic fluid background and microscopic kinetic correction, both fully coupled to each other. A similar manner of discretization is proposed to that used in the recent \texttt{STRUPHY} code \cite{Holderied_Possanner_Wang_2021, Holderied_2022, Li_et_al_2023} with a finite-element model for the background and a pseudo-particle/PiC model for the correction.

        The fluid background satisfies the full, non-linear, resistive, compressible, Hall MHD equations. \cite{Laakmann_Hu_Farrell_2022} introduces finite-element(-in-space) implicit timesteppers for the incompressible analogue to this system with structure-preserving (SP) properties in the ideal case, alongside parameter-robust preconditioners. We show that these timesteppers can derive from a finite-element-in-time (FET) (and finite-element-in-space) interpretation. The benefits of this reformulation are discussed, including the derivation of timesteppers that are higher order in time, and the quantifiable dissipative SP properties in the non-ideal, resistive case.
        
        We discuss possible options for extending this FET approach to timesteppers for the compressible case.

        The kinetic corrections satisfy linearized Boltzmann equations. Using a Lénard--Bernstein collision operator, these take Fokker--Planck-like forms \cite{Fokker_1914, Planck_1917} wherein pseudo-particles in the numerical model obey the neoclassical transport equations, with particle-independent Brownian drift terms. This offers a rigorous methodology for incorporating collisions into the particle transport model, without coupling the equations of motions for each particle.
        
        Works by Chen, Chacón et al. \cite{Chen_Chacón_Barnes_2011, Chacón_Chen_Barnes_2013, Chen_Chacón_2014, Chen_Chacón_2015} have developed structure-preserving particle pushers for neoclassical transport in the Vlasov equations, derived from Crank--Nicolson integrators. We show these too can can derive from a FET interpretation, similarly offering potential extensions to higher-order-in-time particle pushers. The FET formulation is used also to consider how the stochastic drift terms can be incorporated into the pushers. Stochastic gyrokinetic expansions are also discussed.

        Different options for the numerical implementation of these schemes are considered.

        Due to the efficacy of FET in the development of SP timesteppers for both the fluid and kinetic component, we hope this approach will prove effective in the future for developing SP timesteppers for the full hybrid model. We hope this will give us the opportunity to incorporate previously inaccessible kinetic effects into the highly effective, modern, finite-element MHD models.
    \end{abstract}
    
    
    \newpage
    \tableofcontents
    
    
    \newpage
    \pagenumbering{arabic}
    %\linenumbers\renewcommand\thelinenumber{\color{black!50}\arabic{linenumber}}
            \input{0 - introduction/main.tex}
        \part{Research}
            \input{1 - low-noise PiC models/main.tex}
            \input{2 - kinetic component/main.tex}
            \input{3 - fluid component/main.tex}
            \input{4 - numerical implementation/main.tex}
        \part{Project Overview}
            \input{5 - research plan/main.tex}
            \input{6 - summary/main.tex}
    
    
    %\section{}
    \newpage
    \pagenumbering{gobble}
        \printbibliography


    \newpage
    \pagenumbering{roman}
    \appendix
        \part{Appendices}
            \input{8 - Hilbert complexes/main.tex}
            \input{9 - weak conservation proofs/main.tex}
\end{document}

    
    
    %\section{}
    \newpage
    \pagenumbering{gobble}
        \printbibliography


    \newpage
    \pagenumbering{roman}
    \appendix
        \part{Appendices}
            \documentclass[12pt, a4paper]{report}

\input{template/main.tex}

\title{\BA{Title in Progress...}}
\author{Boris Andrews}
\affil{Mathematical Institute, University of Oxford}
\date{\today}


\begin{document}
    \pagenumbering{gobble}
    \maketitle
    
    
    \begin{abstract}
        Magnetic confinement reactors---in particular tokamaks---offer one of the most promising options for achieving practical nuclear fusion, with the potential to provide virtually limitless, clean energy. The theoretical and numerical modeling of tokamak plasmas is simultaneously an essential component of effective reactor design, and a great research barrier. Tokamak operational conditions exhibit comparatively low Knudsen numbers. Kinetic effects, including kinetic waves and instabilities, Landau damping, bump-on-tail instabilities and more, are therefore highly influential in tokamak plasma dynamics. Purely fluid models are inherently incapable of capturing these effects, whereas the high dimensionality in purely kinetic models render them practically intractable for most relevant purposes.

        We consider a $\delta\!f$ decomposition model, with a macroscopic fluid background and microscopic kinetic correction, both fully coupled to each other. A similar manner of discretization is proposed to that used in the recent \texttt{STRUPHY} code \cite{Holderied_Possanner_Wang_2021, Holderied_2022, Li_et_al_2023} with a finite-element model for the background and a pseudo-particle/PiC model for the correction.

        The fluid background satisfies the full, non-linear, resistive, compressible, Hall MHD equations. \cite{Laakmann_Hu_Farrell_2022} introduces finite-element(-in-space) implicit timesteppers for the incompressible analogue to this system with structure-preserving (SP) properties in the ideal case, alongside parameter-robust preconditioners. We show that these timesteppers can derive from a finite-element-in-time (FET) (and finite-element-in-space) interpretation. The benefits of this reformulation are discussed, including the derivation of timesteppers that are higher order in time, and the quantifiable dissipative SP properties in the non-ideal, resistive case.
        
        We discuss possible options for extending this FET approach to timesteppers for the compressible case.

        The kinetic corrections satisfy linearized Boltzmann equations. Using a Lénard--Bernstein collision operator, these take Fokker--Planck-like forms \cite{Fokker_1914, Planck_1917} wherein pseudo-particles in the numerical model obey the neoclassical transport equations, with particle-independent Brownian drift terms. This offers a rigorous methodology for incorporating collisions into the particle transport model, without coupling the equations of motions for each particle.
        
        Works by Chen, Chacón et al. \cite{Chen_Chacón_Barnes_2011, Chacón_Chen_Barnes_2013, Chen_Chacón_2014, Chen_Chacón_2015} have developed structure-preserving particle pushers for neoclassical transport in the Vlasov equations, derived from Crank--Nicolson integrators. We show these too can can derive from a FET interpretation, similarly offering potential extensions to higher-order-in-time particle pushers. The FET formulation is used also to consider how the stochastic drift terms can be incorporated into the pushers. Stochastic gyrokinetic expansions are also discussed.

        Different options for the numerical implementation of these schemes are considered.

        Due to the efficacy of FET in the development of SP timesteppers for both the fluid and kinetic component, we hope this approach will prove effective in the future for developing SP timesteppers for the full hybrid model. We hope this will give us the opportunity to incorporate previously inaccessible kinetic effects into the highly effective, modern, finite-element MHD models.
    \end{abstract}
    
    
    \newpage
    \tableofcontents
    
    
    \newpage
    \pagenumbering{arabic}
    %\linenumbers\renewcommand\thelinenumber{\color{black!50}\arabic{linenumber}}
            \input{0 - introduction/main.tex}
        \part{Research}
            \input{1 - low-noise PiC models/main.tex}
            \input{2 - kinetic component/main.tex}
            \input{3 - fluid component/main.tex}
            \input{4 - numerical implementation/main.tex}
        \part{Project Overview}
            \input{5 - research plan/main.tex}
            \input{6 - summary/main.tex}
    
    
    %\section{}
    \newpage
    \pagenumbering{gobble}
        \printbibliography


    \newpage
    \pagenumbering{roman}
    \appendix
        \part{Appendices}
            \input{8 - Hilbert complexes/main.tex}
            \input{9 - weak conservation proofs/main.tex}
\end{document}

            \documentclass[12pt, a4paper]{report}

\input{template/main.tex}

\title{\BA{Title in Progress...}}
\author{Boris Andrews}
\affil{Mathematical Institute, University of Oxford}
\date{\today}


\begin{document}
    \pagenumbering{gobble}
    \maketitle
    
    
    \begin{abstract}
        Magnetic confinement reactors---in particular tokamaks---offer one of the most promising options for achieving practical nuclear fusion, with the potential to provide virtually limitless, clean energy. The theoretical and numerical modeling of tokamak plasmas is simultaneously an essential component of effective reactor design, and a great research barrier. Tokamak operational conditions exhibit comparatively low Knudsen numbers. Kinetic effects, including kinetic waves and instabilities, Landau damping, bump-on-tail instabilities and more, are therefore highly influential in tokamak plasma dynamics. Purely fluid models are inherently incapable of capturing these effects, whereas the high dimensionality in purely kinetic models render them practically intractable for most relevant purposes.

        We consider a $\delta\!f$ decomposition model, with a macroscopic fluid background and microscopic kinetic correction, both fully coupled to each other. A similar manner of discretization is proposed to that used in the recent \texttt{STRUPHY} code \cite{Holderied_Possanner_Wang_2021, Holderied_2022, Li_et_al_2023} with a finite-element model for the background and a pseudo-particle/PiC model for the correction.

        The fluid background satisfies the full, non-linear, resistive, compressible, Hall MHD equations. \cite{Laakmann_Hu_Farrell_2022} introduces finite-element(-in-space) implicit timesteppers for the incompressible analogue to this system with structure-preserving (SP) properties in the ideal case, alongside parameter-robust preconditioners. We show that these timesteppers can derive from a finite-element-in-time (FET) (and finite-element-in-space) interpretation. The benefits of this reformulation are discussed, including the derivation of timesteppers that are higher order in time, and the quantifiable dissipative SP properties in the non-ideal, resistive case.
        
        We discuss possible options for extending this FET approach to timesteppers for the compressible case.

        The kinetic corrections satisfy linearized Boltzmann equations. Using a Lénard--Bernstein collision operator, these take Fokker--Planck-like forms \cite{Fokker_1914, Planck_1917} wherein pseudo-particles in the numerical model obey the neoclassical transport equations, with particle-independent Brownian drift terms. This offers a rigorous methodology for incorporating collisions into the particle transport model, without coupling the equations of motions for each particle.
        
        Works by Chen, Chacón et al. \cite{Chen_Chacón_Barnes_2011, Chacón_Chen_Barnes_2013, Chen_Chacón_2014, Chen_Chacón_2015} have developed structure-preserving particle pushers for neoclassical transport in the Vlasov equations, derived from Crank--Nicolson integrators. We show these too can can derive from a FET interpretation, similarly offering potential extensions to higher-order-in-time particle pushers. The FET formulation is used also to consider how the stochastic drift terms can be incorporated into the pushers. Stochastic gyrokinetic expansions are also discussed.

        Different options for the numerical implementation of these schemes are considered.

        Due to the efficacy of FET in the development of SP timesteppers for both the fluid and kinetic component, we hope this approach will prove effective in the future for developing SP timesteppers for the full hybrid model. We hope this will give us the opportunity to incorporate previously inaccessible kinetic effects into the highly effective, modern, finite-element MHD models.
    \end{abstract}
    
    
    \newpage
    \tableofcontents
    
    
    \newpage
    \pagenumbering{arabic}
    %\linenumbers\renewcommand\thelinenumber{\color{black!50}\arabic{linenumber}}
            \input{0 - introduction/main.tex}
        \part{Research}
            \input{1 - low-noise PiC models/main.tex}
            \input{2 - kinetic component/main.tex}
            \input{3 - fluid component/main.tex}
            \input{4 - numerical implementation/main.tex}
        \part{Project Overview}
            \input{5 - research plan/main.tex}
            \input{6 - summary/main.tex}
    
    
    %\section{}
    \newpage
    \pagenumbering{gobble}
        \printbibliography


    \newpage
    \pagenumbering{roman}
    \appendix
        \part{Appendices}
            \input{8 - Hilbert complexes/main.tex}
            \input{9 - weak conservation proofs/main.tex}
\end{document}

\end{document}

            \documentclass[12pt, a4paper]{report}

\documentclass[12pt, a4paper]{report}

\input{template/main.tex}

\title{\BA{Title in Progress...}}
\author{Boris Andrews}
\affil{Mathematical Institute, University of Oxford}
\date{\today}


\begin{document}
    \pagenumbering{gobble}
    \maketitle
    
    
    \begin{abstract}
        Magnetic confinement reactors---in particular tokamaks---offer one of the most promising options for achieving practical nuclear fusion, with the potential to provide virtually limitless, clean energy. The theoretical and numerical modeling of tokamak plasmas is simultaneously an essential component of effective reactor design, and a great research barrier. Tokamak operational conditions exhibit comparatively low Knudsen numbers. Kinetic effects, including kinetic waves and instabilities, Landau damping, bump-on-tail instabilities and more, are therefore highly influential in tokamak plasma dynamics. Purely fluid models are inherently incapable of capturing these effects, whereas the high dimensionality in purely kinetic models render them practically intractable for most relevant purposes.

        We consider a $\delta\!f$ decomposition model, with a macroscopic fluid background and microscopic kinetic correction, both fully coupled to each other. A similar manner of discretization is proposed to that used in the recent \texttt{STRUPHY} code \cite{Holderied_Possanner_Wang_2021, Holderied_2022, Li_et_al_2023} with a finite-element model for the background and a pseudo-particle/PiC model for the correction.

        The fluid background satisfies the full, non-linear, resistive, compressible, Hall MHD equations. \cite{Laakmann_Hu_Farrell_2022} introduces finite-element(-in-space) implicit timesteppers for the incompressible analogue to this system with structure-preserving (SP) properties in the ideal case, alongside parameter-robust preconditioners. We show that these timesteppers can derive from a finite-element-in-time (FET) (and finite-element-in-space) interpretation. The benefits of this reformulation are discussed, including the derivation of timesteppers that are higher order in time, and the quantifiable dissipative SP properties in the non-ideal, resistive case.
        
        We discuss possible options for extending this FET approach to timesteppers for the compressible case.

        The kinetic corrections satisfy linearized Boltzmann equations. Using a Lénard--Bernstein collision operator, these take Fokker--Planck-like forms \cite{Fokker_1914, Planck_1917} wherein pseudo-particles in the numerical model obey the neoclassical transport equations, with particle-independent Brownian drift terms. This offers a rigorous methodology for incorporating collisions into the particle transport model, without coupling the equations of motions for each particle.
        
        Works by Chen, Chacón et al. \cite{Chen_Chacón_Barnes_2011, Chacón_Chen_Barnes_2013, Chen_Chacón_2014, Chen_Chacón_2015} have developed structure-preserving particle pushers for neoclassical transport in the Vlasov equations, derived from Crank--Nicolson integrators. We show these too can can derive from a FET interpretation, similarly offering potential extensions to higher-order-in-time particle pushers. The FET formulation is used also to consider how the stochastic drift terms can be incorporated into the pushers. Stochastic gyrokinetic expansions are also discussed.

        Different options for the numerical implementation of these schemes are considered.

        Due to the efficacy of FET in the development of SP timesteppers for both the fluid and kinetic component, we hope this approach will prove effective in the future for developing SP timesteppers for the full hybrid model. We hope this will give us the opportunity to incorporate previously inaccessible kinetic effects into the highly effective, modern, finite-element MHD models.
    \end{abstract}
    
    
    \newpage
    \tableofcontents
    
    
    \newpage
    \pagenumbering{arabic}
    %\linenumbers\renewcommand\thelinenumber{\color{black!50}\arabic{linenumber}}
            \input{0 - introduction/main.tex}
        \part{Research}
            \input{1 - low-noise PiC models/main.tex}
            \input{2 - kinetic component/main.tex}
            \input{3 - fluid component/main.tex}
            \input{4 - numerical implementation/main.tex}
        \part{Project Overview}
            \input{5 - research plan/main.tex}
            \input{6 - summary/main.tex}
    
    
    %\section{}
    \newpage
    \pagenumbering{gobble}
        \printbibliography


    \newpage
    \pagenumbering{roman}
    \appendix
        \part{Appendices}
            \input{8 - Hilbert complexes/main.tex}
            \input{9 - weak conservation proofs/main.tex}
\end{document}


\title{\BA{Title in Progress...}}
\author{Boris Andrews}
\affil{Mathematical Institute, University of Oxford}
\date{\today}


\begin{document}
    \pagenumbering{gobble}
    \maketitle
    
    
    \begin{abstract}
        Magnetic confinement reactors---in particular tokamaks---offer one of the most promising options for achieving practical nuclear fusion, with the potential to provide virtually limitless, clean energy. The theoretical and numerical modeling of tokamak plasmas is simultaneously an essential component of effective reactor design, and a great research barrier. Tokamak operational conditions exhibit comparatively low Knudsen numbers. Kinetic effects, including kinetic waves and instabilities, Landau damping, bump-on-tail instabilities and more, are therefore highly influential in tokamak plasma dynamics. Purely fluid models are inherently incapable of capturing these effects, whereas the high dimensionality in purely kinetic models render them practically intractable for most relevant purposes.

        We consider a $\delta\!f$ decomposition model, with a macroscopic fluid background and microscopic kinetic correction, both fully coupled to each other. A similar manner of discretization is proposed to that used in the recent \texttt{STRUPHY} code \cite{Holderied_Possanner_Wang_2021, Holderied_2022, Li_et_al_2023} with a finite-element model for the background and a pseudo-particle/PiC model for the correction.

        The fluid background satisfies the full, non-linear, resistive, compressible, Hall MHD equations. \cite{Laakmann_Hu_Farrell_2022} introduces finite-element(-in-space) implicit timesteppers for the incompressible analogue to this system with structure-preserving (SP) properties in the ideal case, alongside parameter-robust preconditioners. We show that these timesteppers can derive from a finite-element-in-time (FET) (and finite-element-in-space) interpretation. The benefits of this reformulation are discussed, including the derivation of timesteppers that are higher order in time, and the quantifiable dissipative SP properties in the non-ideal, resistive case.
        
        We discuss possible options for extending this FET approach to timesteppers for the compressible case.

        The kinetic corrections satisfy linearized Boltzmann equations. Using a Lénard--Bernstein collision operator, these take Fokker--Planck-like forms \cite{Fokker_1914, Planck_1917} wherein pseudo-particles in the numerical model obey the neoclassical transport equations, with particle-independent Brownian drift terms. This offers a rigorous methodology for incorporating collisions into the particle transport model, without coupling the equations of motions for each particle.
        
        Works by Chen, Chacón et al. \cite{Chen_Chacón_Barnes_2011, Chacón_Chen_Barnes_2013, Chen_Chacón_2014, Chen_Chacón_2015} have developed structure-preserving particle pushers for neoclassical transport in the Vlasov equations, derived from Crank--Nicolson integrators. We show these too can can derive from a FET interpretation, similarly offering potential extensions to higher-order-in-time particle pushers. The FET formulation is used also to consider how the stochastic drift terms can be incorporated into the pushers. Stochastic gyrokinetic expansions are also discussed.

        Different options for the numerical implementation of these schemes are considered.

        Due to the efficacy of FET in the development of SP timesteppers for both the fluid and kinetic component, we hope this approach will prove effective in the future for developing SP timesteppers for the full hybrid model. We hope this will give us the opportunity to incorporate previously inaccessible kinetic effects into the highly effective, modern, finite-element MHD models.
    \end{abstract}
    
    
    \newpage
    \tableofcontents
    
    
    \newpage
    \pagenumbering{arabic}
    %\linenumbers\renewcommand\thelinenumber{\color{black!50}\arabic{linenumber}}
            \documentclass[12pt, a4paper]{report}

\input{template/main.tex}

\title{\BA{Title in Progress...}}
\author{Boris Andrews}
\affil{Mathematical Institute, University of Oxford}
\date{\today}


\begin{document}
    \pagenumbering{gobble}
    \maketitle
    
    
    \begin{abstract}
        Magnetic confinement reactors---in particular tokamaks---offer one of the most promising options for achieving practical nuclear fusion, with the potential to provide virtually limitless, clean energy. The theoretical and numerical modeling of tokamak plasmas is simultaneously an essential component of effective reactor design, and a great research barrier. Tokamak operational conditions exhibit comparatively low Knudsen numbers. Kinetic effects, including kinetic waves and instabilities, Landau damping, bump-on-tail instabilities and more, are therefore highly influential in tokamak plasma dynamics. Purely fluid models are inherently incapable of capturing these effects, whereas the high dimensionality in purely kinetic models render them practically intractable for most relevant purposes.

        We consider a $\delta\!f$ decomposition model, with a macroscopic fluid background and microscopic kinetic correction, both fully coupled to each other. A similar manner of discretization is proposed to that used in the recent \texttt{STRUPHY} code \cite{Holderied_Possanner_Wang_2021, Holderied_2022, Li_et_al_2023} with a finite-element model for the background and a pseudo-particle/PiC model for the correction.

        The fluid background satisfies the full, non-linear, resistive, compressible, Hall MHD equations. \cite{Laakmann_Hu_Farrell_2022} introduces finite-element(-in-space) implicit timesteppers for the incompressible analogue to this system with structure-preserving (SP) properties in the ideal case, alongside parameter-robust preconditioners. We show that these timesteppers can derive from a finite-element-in-time (FET) (and finite-element-in-space) interpretation. The benefits of this reformulation are discussed, including the derivation of timesteppers that are higher order in time, and the quantifiable dissipative SP properties in the non-ideal, resistive case.
        
        We discuss possible options for extending this FET approach to timesteppers for the compressible case.

        The kinetic corrections satisfy linearized Boltzmann equations. Using a Lénard--Bernstein collision operator, these take Fokker--Planck-like forms \cite{Fokker_1914, Planck_1917} wherein pseudo-particles in the numerical model obey the neoclassical transport equations, with particle-independent Brownian drift terms. This offers a rigorous methodology for incorporating collisions into the particle transport model, without coupling the equations of motions for each particle.
        
        Works by Chen, Chacón et al. \cite{Chen_Chacón_Barnes_2011, Chacón_Chen_Barnes_2013, Chen_Chacón_2014, Chen_Chacón_2015} have developed structure-preserving particle pushers for neoclassical transport in the Vlasov equations, derived from Crank--Nicolson integrators. We show these too can can derive from a FET interpretation, similarly offering potential extensions to higher-order-in-time particle pushers. The FET formulation is used also to consider how the stochastic drift terms can be incorporated into the pushers. Stochastic gyrokinetic expansions are also discussed.

        Different options for the numerical implementation of these schemes are considered.

        Due to the efficacy of FET in the development of SP timesteppers for both the fluid and kinetic component, we hope this approach will prove effective in the future for developing SP timesteppers for the full hybrid model. We hope this will give us the opportunity to incorporate previously inaccessible kinetic effects into the highly effective, modern, finite-element MHD models.
    \end{abstract}
    
    
    \newpage
    \tableofcontents
    
    
    \newpage
    \pagenumbering{arabic}
    %\linenumbers\renewcommand\thelinenumber{\color{black!50}\arabic{linenumber}}
            \input{0 - introduction/main.tex}
        \part{Research}
            \input{1 - low-noise PiC models/main.tex}
            \input{2 - kinetic component/main.tex}
            \input{3 - fluid component/main.tex}
            \input{4 - numerical implementation/main.tex}
        \part{Project Overview}
            \input{5 - research plan/main.tex}
            \input{6 - summary/main.tex}
    
    
    %\section{}
    \newpage
    \pagenumbering{gobble}
        \printbibliography


    \newpage
    \pagenumbering{roman}
    \appendix
        \part{Appendices}
            \input{8 - Hilbert complexes/main.tex}
            \input{9 - weak conservation proofs/main.tex}
\end{document}

        \part{Research}
            \documentclass[12pt, a4paper]{report}

\input{template/main.tex}

\title{\BA{Title in Progress...}}
\author{Boris Andrews}
\affil{Mathematical Institute, University of Oxford}
\date{\today}


\begin{document}
    \pagenumbering{gobble}
    \maketitle
    
    
    \begin{abstract}
        Magnetic confinement reactors---in particular tokamaks---offer one of the most promising options for achieving practical nuclear fusion, with the potential to provide virtually limitless, clean energy. The theoretical and numerical modeling of tokamak plasmas is simultaneously an essential component of effective reactor design, and a great research barrier. Tokamak operational conditions exhibit comparatively low Knudsen numbers. Kinetic effects, including kinetic waves and instabilities, Landau damping, bump-on-tail instabilities and more, are therefore highly influential in tokamak plasma dynamics. Purely fluid models are inherently incapable of capturing these effects, whereas the high dimensionality in purely kinetic models render them practically intractable for most relevant purposes.

        We consider a $\delta\!f$ decomposition model, with a macroscopic fluid background and microscopic kinetic correction, both fully coupled to each other. A similar manner of discretization is proposed to that used in the recent \texttt{STRUPHY} code \cite{Holderied_Possanner_Wang_2021, Holderied_2022, Li_et_al_2023} with a finite-element model for the background and a pseudo-particle/PiC model for the correction.

        The fluid background satisfies the full, non-linear, resistive, compressible, Hall MHD equations. \cite{Laakmann_Hu_Farrell_2022} introduces finite-element(-in-space) implicit timesteppers for the incompressible analogue to this system with structure-preserving (SP) properties in the ideal case, alongside parameter-robust preconditioners. We show that these timesteppers can derive from a finite-element-in-time (FET) (and finite-element-in-space) interpretation. The benefits of this reformulation are discussed, including the derivation of timesteppers that are higher order in time, and the quantifiable dissipative SP properties in the non-ideal, resistive case.
        
        We discuss possible options for extending this FET approach to timesteppers for the compressible case.

        The kinetic corrections satisfy linearized Boltzmann equations. Using a Lénard--Bernstein collision operator, these take Fokker--Planck-like forms \cite{Fokker_1914, Planck_1917} wherein pseudo-particles in the numerical model obey the neoclassical transport equations, with particle-independent Brownian drift terms. This offers a rigorous methodology for incorporating collisions into the particle transport model, without coupling the equations of motions for each particle.
        
        Works by Chen, Chacón et al. \cite{Chen_Chacón_Barnes_2011, Chacón_Chen_Barnes_2013, Chen_Chacón_2014, Chen_Chacón_2015} have developed structure-preserving particle pushers for neoclassical transport in the Vlasov equations, derived from Crank--Nicolson integrators. We show these too can can derive from a FET interpretation, similarly offering potential extensions to higher-order-in-time particle pushers. The FET formulation is used also to consider how the stochastic drift terms can be incorporated into the pushers. Stochastic gyrokinetic expansions are also discussed.

        Different options for the numerical implementation of these schemes are considered.

        Due to the efficacy of FET in the development of SP timesteppers for both the fluid and kinetic component, we hope this approach will prove effective in the future for developing SP timesteppers for the full hybrid model. We hope this will give us the opportunity to incorporate previously inaccessible kinetic effects into the highly effective, modern, finite-element MHD models.
    \end{abstract}
    
    
    \newpage
    \tableofcontents
    
    
    \newpage
    \pagenumbering{arabic}
    %\linenumbers\renewcommand\thelinenumber{\color{black!50}\arabic{linenumber}}
            \input{0 - introduction/main.tex}
        \part{Research}
            \input{1 - low-noise PiC models/main.tex}
            \input{2 - kinetic component/main.tex}
            \input{3 - fluid component/main.tex}
            \input{4 - numerical implementation/main.tex}
        \part{Project Overview}
            \input{5 - research plan/main.tex}
            \input{6 - summary/main.tex}
    
    
    %\section{}
    \newpage
    \pagenumbering{gobble}
        \printbibliography


    \newpage
    \pagenumbering{roman}
    \appendix
        \part{Appendices}
            \input{8 - Hilbert complexes/main.tex}
            \input{9 - weak conservation proofs/main.tex}
\end{document}

            \documentclass[12pt, a4paper]{report}

\input{template/main.tex}

\title{\BA{Title in Progress...}}
\author{Boris Andrews}
\affil{Mathematical Institute, University of Oxford}
\date{\today}


\begin{document}
    \pagenumbering{gobble}
    \maketitle
    
    
    \begin{abstract}
        Magnetic confinement reactors---in particular tokamaks---offer one of the most promising options for achieving practical nuclear fusion, with the potential to provide virtually limitless, clean energy. The theoretical and numerical modeling of tokamak plasmas is simultaneously an essential component of effective reactor design, and a great research barrier. Tokamak operational conditions exhibit comparatively low Knudsen numbers. Kinetic effects, including kinetic waves and instabilities, Landau damping, bump-on-tail instabilities and more, are therefore highly influential in tokamak plasma dynamics. Purely fluid models are inherently incapable of capturing these effects, whereas the high dimensionality in purely kinetic models render them practically intractable for most relevant purposes.

        We consider a $\delta\!f$ decomposition model, with a macroscopic fluid background and microscopic kinetic correction, both fully coupled to each other. A similar manner of discretization is proposed to that used in the recent \texttt{STRUPHY} code \cite{Holderied_Possanner_Wang_2021, Holderied_2022, Li_et_al_2023} with a finite-element model for the background and a pseudo-particle/PiC model for the correction.

        The fluid background satisfies the full, non-linear, resistive, compressible, Hall MHD equations. \cite{Laakmann_Hu_Farrell_2022} introduces finite-element(-in-space) implicit timesteppers for the incompressible analogue to this system with structure-preserving (SP) properties in the ideal case, alongside parameter-robust preconditioners. We show that these timesteppers can derive from a finite-element-in-time (FET) (and finite-element-in-space) interpretation. The benefits of this reformulation are discussed, including the derivation of timesteppers that are higher order in time, and the quantifiable dissipative SP properties in the non-ideal, resistive case.
        
        We discuss possible options for extending this FET approach to timesteppers for the compressible case.

        The kinetic corrections satisfy linearized Boltzmann equations. Using a Lénard--Bernstein collision operator, these take Fokker--Planck-like forms \cite{Fokker_1914, Planck_1917} wherein pseudo-particles in the numerical model obey the neoclassical transport equations, with particle-independent Brownian drift terms. This offers a rigorous methodology for incorporating collisions into the particle transport model, without coupling the equations of motions for each particle.
        
        Works by Chen, Chacón et al. \cite{Chen_Chacón_Barnes_2011, Chacón_Chen_Barnes_2013, Chen_Chacón_2014, Chen_Chacón_2015} have developed structure-preserving particle pushers for neoclassical transport in the Vlasov equations, derived from Crank--Nicolson integrators. We show these too can can derive from a FET interpretation, similarly offering potential extensions to higher-order-in-time particle pushers. The FET formulation is used also to consider how the stochastic drift terms can be incorporated into the pushers. Stochastic gyrokinetic expansions are also discussed.

        Different options for the numerical implementation of these schemes are considered.

        Due to the efficacy of FET in the development of SP timesteppers for both the fluid and kinetic component, we hope this approach will prove effective in the future for developing SP timesteppers for the full hybrid model. We hope this will give us the opportunity to incorporate previously inaccessible kinetic effects into the highly effective, modern, finite-element MHD models.
    \end{abstract}
    
    
    \newpage
    \tableofcontents
    
    
    \newpage
    \pagenumbering{arabic}
    %\linenumbers\renewcommand\thelinenumber{\color{black!50}\arabic{linenumber}}
            \input{0 - introduction/main.tex}
        \part{Research}
            \input{1 - low-noise PiC models/main.tex}
            \input{2 - kinetic component/main.tex}
            \input{3 - fluid component/main.tex}
            \input{4 - numerical implementation/main.tex}
        \part{Project Overview}
            \input{5 - research plan/main.tex}
            \input{6 - summary/main.tex}
    
    
    %\section{}
    \newpage
    \pagenumbering{gobble}
        \printbibliography


    \newpage
    \pagenumbering{roman}
    \appendix
        \part{Appendices}
            \input{8 - Hilbert complexes/main.tex}
            \input{9 - weak conservation proofs/main.tex}
\end{document}

            \documentclass[12pt, a4paper]{report}

\input{template/main.tex}

\title{\BA{Title in Progress...}}
\author{Boris Andrews}
\affil{Mathematical Institute, University of Oxford}
\date{\today}


\begin{document}
    \pagenumbering{gobble}
    \maketitle
    
    
    \begin{abstract}
        Magnetic confinement reactors---in particular tokamaks---offer one of the most promising options for achieving practical nuclear fusion, with the potential to provide virtually limitless, clean energy. The theoretical and numerical modeling of tokamak plasmas is simultaneously an essential component of effective reactor design, and a great research barrier. Tokamak operational conditions exhibit comparatively low Knudsen numbers. Kinetic effects, including kinetic waves and instabilities, Landau damping, bump-on-tail instabilities and more, are therefore highly influential in tokamak plasma dynamics. Purely fluid models are inherently incapable of capturing these effects, whereas the high dimensionality in purely kinetic models render them practically intractable for most relevant purposes.

        We consider a $\delta\!f$ decomposition model, with a macroscopic fluid background and microscopic kinetic correction, both fully coupled to each other. A similar manner of discretization is proposed to that used in the recent \texttt{STRUPHY} code \cite{Holderied_Possanner_Wang_2021, Holderied_2022, Li_et_al_2023} with a finite-element model for the background and a pseudo-particle/PiC model for the correction.

        The fluid background satisfies the full, non-linear, resistive, compressible, Hall MHD equations. \cite{Laakmann_Hu_Farrell_2022} introduces finite-element(-in-space) implicit timesteppers for the incompressible analogue to this system with structure-preserving (SP) properties in the ideal case, alongside parameter-robust preconditioners. We show that these timesteppers can derive from a finite-element-in-time (FET) (and finite-element-in-space) interpretation. The benefits of this reformulation are discussed, including the derivation of timesteppers that are higher order in time, and the quantifiable dissipative SP properties in the non-ideal, resistive case.
        
        We discuss possible options for extending this FET approach to timesteppers for the compressible case.

        The kinetic corrections satisfy linearized Boltzmann equations. Using a Lénard--Bernstein collision operator, these take Fokker--Planck-like forms \cite{Fokker_1914, Planck_1917} wherein pseudo-particles in the numerical model obey the neoclassical transport equations, with particle-independent Brownian drift terms. This offers a rigorous methodology for incorporating collisions into the particle transport model, without coupling the equations of motions for each particle.
        
        Works by Chen, Chacón et al. \cite{Chen_Chacón_Barnes_2011, Chacón_Chen_Barnes_2013, Chen_Chacón_2014, Chen_Chacón_2015} have developed structure-preserving particle pushers for neoclassical transport in the Vlasov equations, derived from Crank--Nicolson integrators. We show these too can can derive from a FET interpretation, similarly offering potential extensions to higher-order-in-time particle pushers. The FET formulation is used also to consider how the stochastic drift terms can be incorporated into the pushers. Stochastic gyrokinetic expansions are also discussed.

        Different options for the numerical implementation of these schemes are considered.

        Due to the efficacy of FET in the development of SP timesteppers for both the fluid and kinetic component, we hope this approach will prove effective in the future for developing SP timesteppers for the full hybrid model. We hope this will give us the opportunity to incorporate previously inaccessible kinetic effects into the highly effective, modern, finite-element MHD models.
    \end{abstract}
    
    
    \newpage
    \tableofcontents
    
    
    \newpage
    \pagenumbering{arabic}
    %\linenumbers\renewcommand\thelinenumber{\color{black!50}\arabic{linenumber}}
            \input{0 - introduction/main.tex}
        \part{Research}
            \input{1 - low-noise PiC models/main.tex}
            \input{2 - kinetic component/main.tex}
            \input{3 - fluid component/main.tex}
            \input{4 - numerical implementation/main.tex}
        \part{Project Overview}
            \input{5 - research plan/main.tex}
            \input{6 - summary/main.tex}
    
    
    %\section{}
    \newpage
    \pagenumbering{gobble}
        \printbibliography


    \newpage
    \pagenumbering{roman}
    \appendix
        \part{Appendices}
            \input{8 - Hilbert complexes/main.tex}
            \input{9 - weak conservation proofs/main.tex}
\end{document}

            \documentclass[12pt, a4paper]{report}

\input{template/main.tex}

\title{\BA{Title in Progress...}}
\author{Boris Andrews}
\affil{Mathematical Institute, University of Oxford}
\date{\today}


\begin{document}
    \pagenumbering{gobble}
    \maketitle
    
    
    \begin{abstract}
        Magnetic confinement reactors---in particular tokamaks---offer one of the most promising options for achieving practical nuclear fusion, with the potential to provide virtually limitless, clean energy. The theoretical and numerical modeling of tokamak plasmas is simultaneously an essential component of effective reactor design, and a great research barrier. Tokamak operational conditions exhibit comparatively low Knudsen numbers. Kinetic effects, including kinetic waves and instabilities, Landau damping, bump-on-tail instabilities and more, are therefore highly influential in tokamak plasma dynamics. Purely fluid models are inherently incapable of capturing these effects, whereas the high dimensionality in purely kinetic models render them practically intractable for most relevant purposes.

        We consider a $\delta\!f$ decomposition model, with a macroscopic fluid background and microscopic kinetic correction, both fully coupled to each other. A similar manner of discretization is proposed to that used in the recent \texttt{STRUPHY} code \cite{Holderied_Possanner_Wang_2021, Holderied_2022, Li_et_al_2023} with a finite-element model for the background and a pseudo-particle/PiC model for the correction.

        The fluid background satisfies the full, non-linear, resistive, compressible, Hall MHD equations. \cite{Laakmann_Hu_Farrell_2022} introduces finite-element(-in-space) implicit timesteppers for the incompressible analogue to this system with structure-preserving (SP) properties in the ideal case, alongside parameter-robust preconditioners. We show that these timesteppers can derive from a finite-element-in-time (FET) (and finite-element-in-space) interpretation. The benefits of this reformulation are discussed, including the derivation of timesteppers that are higher order in time, and the quantifiable dissipative SP properties in the non-ideal, resistive case.
        
        We discuss possible options for extending this FET approach to timesteppers for the compressible case.

        The kinetic corrections satisfy linearized Boltzmann equations. Using a Lénard--Bernstein collision operator, these take Fokker--Planck-like forms \cite{Fokker_1914, Planck_1917} wherein pseudo-particles in the numerical model obey the neoclassical transport equations, with particle-independent Brownian drift terms. This offers a rigorous methodology for incorporating collisions into the particle transport model, without coupling the equations of motions for each particle.
        
        Works by Chen, Chacón et al. \cite{Chen_Chacón_Barnes_2011, Chacón_Chen_Barnes_2013, Chen_Chacón_2014, Chen_Chacón_2015} have developed structure-preserving particle pushers for neoclassical transport in the Vlasov equations, derived from Crank--Nicolson integrators. We show these too can can derive from a FET interpretation, similarly offering potential extensions to higher-order-in-time particle pushers. The FET formulation is used also to consider how the stochastic drift terms can be incorporated into the pushers. Stochastic gyrokinetic expansions are also discussed.

        Different options for the numerical implementation of these schemes are considered.

        Due to the efficacy of FET in the development of SP timesteppers for both the fluid and kinetic component, we hope this approach will prove effective in the future for developing SP timesteppers for the full hybrid model. We hope this will give us the opportunity to incorporate previously inaccessible kinetic effects into the highly effective, modern, finite-element MHD models.
    \end{abstract}
    
    
    \newpage
    \tableofcontents
    
    
    \newpage
    \pagenumbering{arabic}
    %\linenumbers\renewcommand\thelinenumber{\color{black!50}\arabic{linenumber}}
            \input{0 - introduction/main.tex}
        \part{Research}
            \input{1 - low-noise PiC models/main.tex}
            \input{2 - kinetic component/main.tex}
            \input{3 - fluid component/main.tex}
            \input{4 - numerical implementation/main.tex}
        \part{Project Overview}
            \input{5 - research plan/main.tex}
            \input{6 - summary/main.tex}
    
    
    %\section{}
    \newpage
    \pagenumbering{gobble}
        \printbibliography


    \newpage
    \pagenumbering{roman}
    \appendix
        \part{Appendices}
            \input{8 - Hilbert complexes/main.tex}
            \input{9 - weak conservation proofs/main.tex}
\end{document}

        \part{Project Overview}
            \documentclass[12pt, a4paper]{report}

\input{template/main.tex}

\title{\BA{Title in Progress...}}
\author{Boris Andrews}
\affil{Mathematical Institute, University of Oxford}
\date{\today}


\begin{document}
    \pagenumbering{gobble}
    \maketitle
    
    
    \begin{abstract}
        Magnetic confinement reactors---in particular tokamaks---offer one of the most promising options for achieving practical nuclear fusion, with the potential to provide virtually limitless, clean energy. The theoretical and numerical modeling of tokamak plasmas is simultaneously an essential component of effective reactor design, and a great research barrier. Tokamak operational conditions exhibit comparatively low Knudsen numbers. Kinetic effects, including kinetic waves and instabilities, Landau damping, bump-on-tail instabilities and more, are therefore highly influential in tokamak plasma dynamics. Purely fluid models are inherently incapable of capturing these effects, whereas the high dimensionality in purely kinetic models render them practically intractable for most relevant purposes.

        We consider a $\delta\!f$ decomposition model, with a macroscopic fluid background and microscopic kinetic correction, both fully coupled to each other. A similar manner of discretization is proposed to that used in the recent \texttt{STRUPHY} code \cite{Holderied_Possanner_Wang_2021, Holderied_2022, Li_et_al_2023} with a finite-element model for the background and a pseudo-particle/PiC model for the correction.

        The fluid background satisfies the full, non-linear, resistive, compressible, Hall MHD equations. \cite{Laakmann_Hu_Farrell_2022} introduces finite-element(-in-space) implicit timesteppers for the incompressible analogue to this system with structure-preserving (SP) properties in the ideal case, alongside parameter-robust preconditioners. We show that these timesteppers can derive from a finite-element-in-time (FET) (and finite-element-in-space) interpretation. The benefits of this reformulation are discussed, including the derivation of timesteppers that are higher order in time, and the quantifiable dissipative SP properties in the non-ideal, resistive case.
        
        We discuss possible options for extending this FET approach to timesteppers for the compressible case.

        The kinetic corrections satisfy linearized Boltzmann equations. Using a Lénard--Bernstein collision operator, these take Fokker--Planck-like forms \cite{Fokker_1914, Planck_1917} wherein pseudo-particles in the numerical model obey the neoclassical transport equations, with particle-independent Brownian drift terms. This offers a rigorous methodology for incorporating collisions into the particle transport model, without coupling the equations of motions for each particle.
        
        Works by Chen, Chacón et al. \cite{Chen_Chacón_Barnes_2011, Chacón_Chen_Barnes_2013, Chen_Chacón_2014, Chen_Chacón_2015} have developed structure-preserving particle pushers for neoclassical transport in the Vlasov equations, derived from Crank--Nicolson integrators. We show these too can can derive from a FET interpretation, similarly offering potential extensions to higher-order-in-time particle pushers. The FET formulation is used also to consider how the stochastic drift terms can be incorporated into the pushers. Stochastic gyrokinetic expansions are also discussed.

        Different options for the numerical implementation of these schemes are considered.

        Due to the efficacy of FET in the development of SP timesteppers for both the fluid and kinetic component, we hope this approach will prove effective in the future for developing SP timesteppers for the full hybrid model. We hope this will give us the opportunity to incorporate previously inaccessible kinetic effects into the highly effective, modern, finite-element MHD models.
    \end{abstract}
    
    
    \newpage
    \tableofcontents
    
    
    \newpage
    \pagenumbering{arabic}
    %\linenumbers\renewcommand\thelinenumber{\color{black!50}\arabic{linenumber}}
            \input{0 - introduction/main.tex}
        \part{Research}
            \input{1 - low-noise PiC models/main.tex}
            \input{2 - kinetic component/main.tex}
            \input{3 - fluid component/main.tex}
            \input{4 - numerical implementation/main.tex}
        \part{Project Overview}
            \input{5 - research plan/main.tex}
            \input{6 - summary/main.tex}
    
    
    %\section{}
    \newpage
    \pagenumbering{gobble}
        \printbibliography


    \newpage
    \pagenumbering{roman}
    \appendix
        \part{Appendices}
            \input{8 - Hilbert complexes/main.tex}
            \input{9 - weak conservation proofs/main.tex}
\end{document}

            \documentclass[12pt, a4paper]{report}

\input{template/main.tex}

\title{\BA{Title in Progress...}}
\author{Boris Andrews}
\affil{Mathematical Institute, University of Oxford}
\date{\today}


\begin{document}
    \pagenumbering{gobble}
    \maketitle
    
    
    \begin{abstract}
        Magnetic confinement reactors---in particular tokamaks---offer one of the most promising options for achieving practical nuclear fusion, with the potential to provide virtually limitless, clean energy. The theoretical and numerical modeling of tokamak plasmas is simultaneously an essential component of effective reactor design, and a great research barrier. Tokamak operational conditions exhibit comparatively low Knudsen numbers. Kinetic effects, including kinetic waves and instabilities, Landau damping, bump-on-tail instabilities and more, are therefore highly influential in tokamak plasma dynamics. Purely fluid models are inherently incapable of capturing these effects, whereas the high dimensionality in purely kinetic models render them practically intractable for most relevant purposes.

        We consider a $\delta\!f$ decomposition model, with a macroscopic fluid background and microscopic kinetic correction, both fully coupled to each other. A similar manner of discretization is proposed to that used in the recent \texttt{STRUPHY} code \cite{Holderied_Possanner_Wang_2021, Holderied_2022, Li_et_al_2023} with a finite-element model for the background and a pseudo-particle/PiC model for the correction.

        The fluid background satisfies the full, non-linear, resistive, compressible, Hall MHD equations. \cite{Laakmann_Hu_Farrell_2022} introduces finite-element(-in-space) implicit timesteppers for the incompressible analogue to this system with structure-preserving (SP) properties in the ideal case, alongside parameter-robust preconditioners. We show that these timesteppers can derive from a finite-element-in-time (FET) (and finite-element-in-space) interpretation. The benefits of this reformulation are discussed, including the derivation of timesteppers that are higher order in time, and the quantifiable dissipative SP properties in the non-ideal, resistive case.
        
        We discuss possible options for extending this FET approach to timesteppers for the compressible case.

        The kinetic corrections satisfy linearized Boltzmann equations. Using a Lénard--Bernstein collision operator, these take Fokker--Planck-like forms \cite{Fokker_1914, Planck_1917} wherein pseudo-particles in the numerical model obey the neoclassical transport equations, with particle-independent Brownian drift terms. This offers a rigorous methodology for incorporating collisions into the particle transport model, without coupling the equations of motions for each particle.
        
        Works by Chen, Chacón et al. \cite{Chen_Chacón_Barnes_2011, Chacón_Chen_Barnes_2013, Chen_Chacón_2014, Chen_Chacón_2015} have developed structure-preserving particle pushers for neoclassical transport in the Vlasov equations, derived from Crank--Nicolson integrators. We show these too can can derive from a FET interpretation, similarly offering potential extensions to higher-order-in-time particle pushers. The FET formulation is used also to consider how the stochastic drift terms can be incorporated into the pushers. Stochastic gyrokinetic expansions are also discussed.

        Different options for the numerical implementation of these schemes are considered.

        Due to the efficacy of FET in the development of SP timesteppers for both the fluid and kinetic component, we hope this approach will prove effective in the future for developing SP timesteppers for the full hybrid model. We hope this will give us the opportunity to incorporate previously inaccessible kinetic effects into the highly effective, modern, finite-element MHD models.
    \end{abstract}
    
    
    \newpage
    \tableofcontents
    
    
    \newpage
    \pagenumbering{arabic}
    %\linenumbers\renewcommand\thelinenumber{\color{black!50}\arabic{linenumber}}
            \input{0 - introduction/main.tex}
        \part{Research}
            \input{1 - low-noise PiC models/main.tex}
            \input{2 - kinetic component/main.tex}
            \input{3 - fluid component/main.tex}
            \input{4 - numerical implementation/main.tex}
        \part{Project Overview}
            \input{5 - research plan/main.tex}
            \input{6 - summary/main.tex}
    
    
    %\section{}
    \newpage
    \pagenumbering{gobble}
        \printbibliography


    \newpage
    \pagenumbering{roman}
    \appendix
        \part{Appendices}
            \input{8 - Hilbert complexes/main.tex}
            \input{9 - weak conservation proofs/main.tex}
\end{document}

    
    
    %\section{}
    \newpage
    \pagenumbering{gobble}
        \printbibliography


    \newpage
    \pagenumbering{roman}
    \appendix
        \part{Appendices}
            \documentclass[12pt, a4paper]{report}

\input{template/main.tex}

\title{\BA{Title in Progress...}}
\author{Boris Andrews}
\affil{Mathematical Institute, University of Oxford}
\date{\today}


\begin{document}
    \pagenumbering{gobble}
    \maketitle
    
    
    \begin{abstract}
        Magnetic confinement reactors---in particular tokamaks---offer one of the most promising options for achieving practical nuclear fusion, with the potential to provide virtually limitless, clean energy. The theoretical and numerical modeling of tokamak plasmas is simultaneously an essential component of effective reactor design, and a great research barrier. Tokamak operational conditions exhibit comparatively low Knudsen numbers. Kinetic effects, including kinetic waves and instabilities, Landau damping, bump-on-tail instabilities and more, are therefore highly influential in tokamak plasma dynamics. Purely fluid models are inherently incapable of capturing these effects, whereas the high dimensionality in purely kinetic models render them practically intractable for most relevant purposes.

        We consider a $\delta\!f$ decomposition model, with a macroscopic fluid background and microscopic kinetic correction, both fully coupled to each other. A similar manner of discretization is proposed to that used in the recent \texttt{STRUPHY} code \cite{Holderied_Possanner_Wang_2021, Holderied_2022, Li_et_al_2023} with a finite-element model for the background and a pseudo-particle/PiC model for the correction.

        The fluid background satisfies the full, non-linear, resistive, compressible, Hall MHD equations. \cite{Laakmann_Hu_Farrell_2022} introduces finite-element(-in-space) implicit timesteppers for the incompressible analogue to this system with structure-preserving (SP) properties in the ideal case, alongside parameter-robust preconditioners. We show that these timesteppers can derive from a finite-element-in-time (FET) (and finite-element-in-space) interpretation. The benefits of this reformulation are discussed, including the derivation of timesteppers that are higher order in time, and the quantifiable dissipative SP properties in the non-ideal, resistive case.
        
        We discuss possible options for extending this FET approach to timesteppers for the compressible case.

        The kinetic corrections satisfy linearized Boltzmann equations. Using a Lénard--Bernstein collision operator, these take Fokker--Planck-like forms \cite{Fokker_1914, Planck_1917} wherein pseudo-particles in the numerical model obey the neoclassical transport equations, with particle-independent Brownian drift terms. This offers a rigorous methodology for incorporating collisions into the particle transport model, without coupling the equations of motions for each particle.
        
        Works by Chen, Chacón et al. \cite{Chen_Chacón_Barnes_2011, Chacón_Chen_Barnes_2013, Chen_Chacón_2014, Chen_Chacón_2015} have developed structure-preserving particle pushers for neoclassical transport in the Vlasov equations, derived from Crank--Nicolson integrators. We show these too can can derive from a FET interpretation, similarly offering potential extensions to higher-order-in-time particle pushers. The FET formulation is used also to consider how the stochastic drift terms can be incorporated into the pushers. Stochastic gyrokinetic expansions are also discussed.

        Different options for the numerical implementation of these schemes are considered.

        Due to the efficacy of FET in the development of SP timesteppers for both the fluid and kinetic component, we hope this approach will prove effective in the future for developing SP timesteppers for the full hybrid model. We hope this will give us the opportunity to incorporate previously inaccessible kinetic effects into the highly effective, modern, finite-element MHD models.
    \end{abstract}
    
    
    \newpage
    \tableofcontents
    
    
    \newpage
    \pagenumbering{arabic}
    %\linenumbers\renewcommand\thelinenumber{\color{black!50}\arabic{linenumber}}
            \input{0 - introduction/main.tex}
        \part{Research}
            \input{1 - low-noise PiC models/main.tex}
            \input{2 - kinetic component/main.tex}
            \input{3 - fluid component/main.tex}
            \input{4 - numerical implementation/main.tex}
        \part{Project Overview}
            \input{5 - research plan/main.tex}
            \input{6 - summary/main.tex}
    
    
    %\section{}
    \newpage
    \pagenumbering{gobble}
        \printbibliography


    \newpage
    \pagenumbering{roman}
    \appendix
        \part{Appendices}
            \input{8 - Hilbert complexes/main.tex}
            \input{9 - weak conservation proofs/main.tex}
\end{document}

            \documentclass[12pt, a4paper]{report}

\input{template/main.tex}

\title{\BA{Title in Progress...}}
\author{Boris Andrews}
\affil{Mathematical Institute, University of Oxford}
\date{\today}


\begin{document}
    \pagenumbering{gobble}
    \maketitle
    
    
    \begin{abstract}
        Magnetic confinement reactors---in particular tokamaks---offer one of the most promising options for achieving practical nuclear fusion, with the potential to provide virtually limitless, clean energy. The theoretical and numerical modeling of tokamak plasmas is simultaneously an essential component of effective reactor design, and a great research barrier. Tokamak operational conditions exhibit comparatively low Knudsen numbers. Kinetic effects, including kinetic waves and instabilities, Landau damping, bump-on-tail instabilities and more, are therefore highly influential in tokamak plasma dynamics. Purely fluid models are inherently incapable of capturing these effects, whereas the high dimensionality in purely kinetic models render them practically intractable for most relevant purposes.

        We consider a $\delta\!f$ decomposition model, with a macroscopic fluid background and microscopic kinetic correction, both fully coupled to each other. A similar manner of discretization is proposed to that used in the recent \texttt{STRUPHY} code \cite{Holderied_Possanner_Wang_2021, Holderied_2022, Li_et_al_2023} with a finite-element model for the background and a pseudo-particle/PiC model for the correction.

        The fluid background satisfies the full, non-linear, resistive, compressible, Hall MHD equations. \cite{Laakmann_Hu_Farrell_2022} introduces finite-element(-in-space) implicit timesteppers for the incompressible analogue to this system with structure-preserving (SP) properties in the ideal case, alongside parameter-robust preconditioners. We show that these timesteppers can derive from a finite-element-in-time (FET) (and finite-element-in-space) interpretation. The benefits of this reformulation are discussed, including the derivation of timesteppers that are higher order in time, and the quantifiable dissipative SP properties in the non-ideal, resistive case.
        
        We discuss possible options for extending this FET approach to timesteppers for the compressible case.

        The kinetic corrections satisfy linearized Boltzmann equations. Using a Lénard--Bernstein collision operator, these take Fokker--Planck-like forms \cite{Fokker_1914, Planck_1917} wherein pseudo-particles in the numerical model obey the neoclassical transport equations, with particle-independent Brownian drift terms. This offers a rigorous methodology for incorporating collisions into the particle transport model, without coupling the equations of motions for each particle.
        
        Works by Chen, Chacón et al. \cite{Chen_Chacón_Barnes_2011, Chacón_Chen_Barnes_2013, Chen_Chacón_2014, Chen_Chacón_2015} have developed structure-preserving particle pushers for neoclassical transport in the Vlasov equations, derived from Crank--Nicolson integrators. We show these too can can derive from a FET interpretation, similarly offering potential extensions to higher-order-in-time particle pushers. The FET formulation is used also to consider how the stochastic drift terms can be incorporated into the pushers. Stochastic gyrokinetic expansions are also discussed.

        Different options for the numerical implementation of these schemes are considered.

        Due to the efficacy of FET in the development of SP timesteppers for both the fluid and kinetic component, we hope this approach will prove effective in the future for developing SP timesteppers for the full hybrid model. We hope this will give us the opportunity to incorporate previously inaccessible kinetic effects into the highly effective, modern, finite-element MHD models.
    \end{abstract}
    
    
    \newpage
    \tableofcontents
    
    
    \newpage
    \pagenumbering{arabic}
    %\linenumbers\renewcommand\thelinenumber{\color{black!50}\arabic{linenumber}}
            \input{0 - introduction/main.tex}
        \part{Research}
            \input{1 - low-noise PiC models/main.tex}
            \input{2 - kinetic component/main.tex}
            \input{3 - fluid component/main.tex}
            \input{4 - numerical implementation/main.tex}
        \part{Project Overview}
            \input{5 - research plan/main.tex}
            \input{6 - summary/main.tex}
    
    
    %\section{}
    \newpage
    \pagenumbering{gobble}
        \printbibliography


    \newpage
    \pagenumbering{roman}
    \appendix
        \part{Appendices}
            \input{8 - Hilbert complexes/main.tex}
            \input{9 - weak conservation proofs/main.tex}
\end{document}

\end{document}

    
    
    %\section{}
    \newpage
    \pagenumbering{gobble}
        \printbibliography


    \newpage
    \pagenumbering{roman}
    \appendix
        \part{Appendices}
            \documentclass[12pt, a4paper]{report}

\documentclass[12pt, a4paper]{report}

\input{template/main.tex}

\title{\BA{Title in Progress...}}
\author{Boris Andrews}
\affil{Mathematical Institute, University of Oxford}
\date{\today}


\begin{document}
    \pagenumbering{gobble}
    \maketitle
    
    
    \begin{abstract}
        Magnetic confinement reactors---in particular tokamaks---offer one of the most promising options for achieving practical nuclear fusion, with the potential to provide virtually limitless, clean energy. The theoretical and numerical modeling of tokamak plasmas is simultaneously an essential component of effective reactor design, and a great research barrier. Tokamak operational conditions exhibit comparatively low Knudsen numbers. Kinetic effects, including kinetic waves and instabilities, Landau damping, bump-on-tail instabilities and more, are therefore highly influential in tokamak plasma dynamics. Purely fluid models are inherently incapable of capturing these effects, whereas the high dimensionality in purely kinetic models render them practically intractable for most relevant purposes.

        We consider a $\delta\!f$ decomposition model, with a macroscopic fluid background and microscopic kinetic correction, both fully coupled to each other. A similar manner of discretization is proposed to that used in the recent \texttt{STRUPHY} code \cite{Holderied_Possanner_Wang_2021, Holderied_2022, Li_et_al_2023} with a finite-element model for the background and a pseudo-particle/PiC model for the correction.

        The fluid background satisfies the full, non-linear, resistive, compressible, Hall MHD equations. \cite{Laakmann_Hu_Farrell_2022} introduces finite-element(-in-space) implicit timesteppers for the incompressible analogue to this system with structure-preserving (SP) properties in the ideal case, alongside parameter-robust preconditioners. We show that these timesteppers can derive from a finite-element-in-time (FET) (and finite-element-in-space) interpretation. The benefits of this reformulation are discussed, including the derivation of timesteppers that are higher order in time, and the quantifiable dissipative SP properties in the non-ideal, resistive case.
        
        We discuss possible options for extending this FET approach to timesteppers for the compressible case.

        The kinetic corrections satisfy linearized Boltzmann equations. Using a Lénard--Bernstein collision operator, these take Fokker--Planck-like forms \cite{Fokker_1914, Planck_1917} wherein pseudo-particles in the numerical model obey the neoclassical transport equations, with particle-independent Brownian drift terms. This offers a rigorous methodology for incorporating collisions into the particle transport model, without coupling the equations of motions for each particle.
        
        Works by Chen, Chacón et al. \cite{Chen_Chacón_Barnes_2011, Chacón_Chen_Barnes_2013, Chen_Chacón_2014, Chen_Chacón_2015} have developed structure-preserving particle pushers for neoclassical transport in the Vlasov equations, derived from Crank--Nicolson integrators. We show these too can can derive from a FET interpretation, similarly offering potential extensions to higher-order-in-time particle pushers. The FET formulation is used also to consider how the stochastic drift terms can be incorporated into the pushers. Stochastic gyrokinetic expansions are also discussed.

        Different options for the numerical implementation of these schemes are considered.

        Due to the efficacy of FET in the development of SP timesteppers for both the fluid and kinetic component, we hope this approach will prove effective in the future for developing SP timesteppers for the full hybrid model. We hope this will give us the opportunity to incorporate previously inaccessible kinetic effects into the highly effective, modern, finite-element MHD models.
    \end{abstract}
    
    
    \newpage
    \tableofcontents
    
    
    \newpage
    \pagenumbering{arabic}
    %\linenumbers\renewcommand\thelinenumber{\color{black!50}\arabic{linenumber}}
            \input{0 - introduction/main.tex}
        \part{Research}
            \input{1 - low-noise PiC models/main.tex}
            \input{2 - kinetic component/main.tex}
            \input{3 - fluid component/main.tex}
            \input{4 - numerical implementation/main.tex}
        \part{Project Overview}
            \input{5 - research plan/main.tex}
            \input{6 - summary/main.tex}
    
    
    %\section{}
    \newpage
    \pagenumbering{gobble}
        \printbibliography


    \newpage
    \pagenumbering{roman}
    \appendix
        \part{Appendices}
            \input{8 - Hilbert complexes/main.tex}
            \input{9 - weak conservation proofs/main.tex}
\end{document}


\title{\BA{Title in Progress...}}
\author{Boris Andrews}
\affil{Mathematical Institute, University of Oxford}
\date{\today}


\begin{document}
    \pagenumbering{gobble}
    \maketitle
    
    
    \begin{abstract}
        Magnetic confinement reactors---in particular tokamaks---offer one of the most promising options for achieving practical nuclear fusion, with the potential to provide virtually limitless, clean energy. The theoretical and numerical modeling of tokamak plasmas is simultaneously an essential component of effective reactor design, and a great research barrier. Tokamak operational conditions exhibit comparatively low Knudsen numbers. Kinetic effects, including kinetic waves and instabilities, Landau damping, bump-on-tail instabilities and more, are therefore highly influential in tokamak plasma dynamics. Purely fluid models are inherently incapable of capturing these effects, whereas the high dimensionality in purely kinetic models render them practically intractable for most relevant purposes.

        We consider a $\delta\!f$ decomposition model, with a macroscopic fluid background and microscopic kinetic correction, both fully coupled to each other. A similar manner of discretization is proposed to that used in the recent \texttt{STRUPHY} code \cite{Holderied_Possanner_Wang_2021, Holderied_2022, Li_et_al_2023} with a finite-element model for the background and a pseudo-particle/PiC model for the correction.

        The fluid background satisfies the full, non-linear, resistive, compressible, Hall MHD equations. \cite{Laakmann_Hu_Farrell_2022} introduces finite-element(-in-space) implicit timesteppers for the incompressible analogue to this system with structure-preserving (SP) properties in the ideal case, alongside parameter-robust preconditioners. We show that these timesteppers can derive from a finite-element-in-time (FET) (and finite-element-in-space) interpretation. The benefits of this reformulation are discussed, including the derivation of timesteppers that are higher order in time, and the quantifiable dissipative SP properties in the non-ideal, resistive case.
        
        We discuss possible options for extending this FET approach to timesteppers for the compressible case.

        The kinetic corrections satisfy linearized Boltzmann equations. Using a Lénard--Bernstein collision operator, these take Fokker--Planck-like forms \cite{Fokker_1914, Planck_1917} wherein pseudo-particles in the numerical model obey the neoclassical transport equations, with particle-independent Brownian drift terms. This offers a rigorous methodology for incorporating collisions into the particle transport model, without coupling the equations of motions for each particle.
        
        Works by Chen, Chacón et al. \cite{Chen_Chacón_Barnes_2011, Chacón_Chen_Barnes_2013, Chen_Chacón_2014, Chen_Chacón_2015} have developed structure-preserving particle pushers for neoclassical transport in the Vlasov equations, derived from Crank--Nicolson integrators. We show these too can can derive from a FET interpretation, similarly offering potential extensions to higher-order-in-time particle pushers. The FET formulation is used also to consider how the stochastic drift terms can be incorporated into the pushers. Stochastic gyrokinetic expansions are also discussed.

        Different options for the numerical implementation of these schemes are considered.

        Due to the efficacy of FET in the development of SP timesteppers for both the fluid and kinetic component, we hope this approach will prove effective in the future for developing SP timesteppers for the full hybrid model. We hope this will give us the opportunity to incorporate previously inaccessible kinetic effects into the highly effective, modern, finite-element MHD models.
    \end{abstract}
    
    
    \newpage
    \tableofcontents
    
    
    \newpage
    \pagenumbering{arabic}
    %\linenumbers\renewcommand\thelinenumber{\color{black!50}\arabic{linenumber}}
            \documentclass[12pt, a4paper]{report}

\input{template/main.tex}

\title{\BA{Title in Progress...}}
\author{Boris Andrews}
\affil{Mathematical Institute, University of Oxford}
\date{\today}


\begin{document}
    \pagenumbering{gobble}
    \maketitle
    
    
    \begin{abstract}
        Magnetic confinement reactors---in particular tokamaks---offer one of the most promising options for achieving practical nuclear fusion, with the potential to provide virtually limitless, clean energy. The theoretical and numerical modeling of tokamak plasmas is simultaneously an essential component of effective reactor design, and a great research barrier. Tokamak operational conditions exhibit comparatively low Knudsen numbers. Kinetic effects, including kinetic waves and instabilities, Landau damping, bump-on-tail instabilities and more, are therefore highly influential in tokamak plasma dynamics. Purely fluid models are inherently incapable of capturing these effects, whereas the high dimensionality in purely kinetic models render them practically intractable for most relevant purposes.

        We consider a $\delta\!f$ decomposition model, with a macroscopic fluid background and microscopic kinetic correction, both fully coupled to each other. A similar manner of discretization is proposed to that used in the recent \texttt{STRUPHY} code \cite{Holderied_Possanner_Wang_2021, Holderied_2022, Li_et_al_2023} with a finite-element model for the background and a pseudo-particle/PiC model for the correction.

        The fluid background satisfies the full, non-linear, resistive, compressible, Hall MHD equations. \cite{Laakmann_Hu_Farrell_2022} introduces finite-element(-in-space) implicit timesteppers for the incompressible analogue to this system with structure-preserving (SP) properties in the ideal case, alongside parameter-robust preconditioners. We show that these timesteppers can derive from a finite-element-in-time (FET) (and finite-element-in-space) interpretation. The benefits of this reformulation are discussed, including the derivation of timesteppers that are higher order in time, and the quantifiable dissipative SP properties in the non-ideal, resistive case.
        
        We discuss possible options for extending this FET approach to timesteppers for the compressible case.

        The kinetic corrections satisfy linearized Boltzmann equations. Using a Lénard--Bernstein collision operator, these take Fokker--Planck-like forms \cite{Fokker_1914, Planck_1917} wherein pseudo-particles in the numerical model obey the neoclassical transport equations, with particle-independent Brownian drift terms. This offers a rigorous methodology for incorporating collisions into the particle transport model, without coupling the equations of motions for each particle.
        
        Works by Chen, Chacón et al. \cite{Chen_Chacón_Barnes_2011, Chacón_Chen_Barnes_2013, Chen_Chacón_2014, Chen_Chacón_2015} have developed structure-preserving particle pushers for neoclassical transport in the Vlasov equations, derived from Crank--Nicolson integrators. We show these too can can derive from a FET interpretation, similarly offering potential extensions to higher-order-in-time particle pushers. The FET formulation is used also to consider how the stochastic drift terms can be incorporated into the pushers. Stochastic gyrokinetic expansions are also discussed.

        Different options for the numerical implementation of these schemes are considered.

        Due to the efficacy of FET in the development of SP timesteppers for both the fluid and kinetic component, we hope this approach will prove effective in the future for developing SP timesteppers for the full hybrid model. We hope this will give us the opportunity to incorporate previously inaccessible kinetic effects into the highly effective, modern, finite-element MHD models.
    \end{abstract}
    
    
    \newpage
    \tableofcontents
    
    
    \newpage
    \pagenumbering{arabic}
    %\linenumbers\renewcommand\thelinenumber{\color{black!50}\arabic{linenumber}}
            \input{0 - introduction/main.tex}
        \part{Research}
            \input{1 - low-noise PiC models/main.tex}
            \input{2 - kinetic component/main.tex}
            \input{3 - fluid component/main.tex}
            \input{4 - numerical implementation/main.tex}
        \part{Project Overview}
            \input{5 - research plan/main.tex}
            \input{6 - summary/main.tex}
    
    
    %\section{}
    \newpage
    \pagenumbering{gobble}
        \printbibliography


    \newpage
    \pagenumbering{roman}
    \appendix
        \part{Appendices}
            \input{8 - Hilbert complexes/main.tex}
            \input{9 - weak conservation proofs/main.tex}
\end{document}

        \part{Research}
            \documentclass[12pt, a4paper]{report}

\input{template/main.tex}

\title{\BA{Title in Progress...}}
\author{Boris Andrews}
\affil{Mathematical Institute, University of Oxford}
\date{\today}


\begin{document}
    \pagenumbering{gobble}
    \maketitle
    
    
    \begin{abstract}
        Magnetic confinement reactors---in particular tokamaks---offer one of the most promising options for achieving practical nuclear fusion, with the potential to provide virtually limitless, clean energy. The theoretical and numerical modeling of tokamak plasmas is simultaneously an essential component of effective reactor design, and a great research barrier. Tokamak operational conditions exhibit comparatively low Knudsen numbers. Kinetic effects, including kinetic waves and instabilities, Landau damping, bump-on-tail instabilities and more, are therefore highly influential in tokamak plasma dynamics. Purely fluid models are inherently incapable of capturing these effects, whereas the high dimensionality in purely kinetic models render them practically intractable for most relevant purposes.

        We consider a $\delta\!f$ decomposition model, with a macroscopic fluid background and microscopic kinetic correction, both fully coupled to each other. A similar manner of discretization is proposed to that used in the recent \texttt{STRUPHY} code \cite{Holderied_Possanner_Wang_2021, Holderied_2022, Li_et_al_2023} with a finite-element model for the background and a pseudo-particle/PiC model for the correction.

        The fluid background satisfies the full, non-linear, resistive, compressible, Hall MHD equations. \cite{Laakmann_Hu_Farrell_2022} introduces finite-element(-in-space) implicit timesteppers for the incompressible analogue to this system with structure-preserving (SP) properties in the ideal case, alongside parameter-robust preconditioners. We show that these timesteppers can derive from a finite-element-in-time (FET) (and finite-element-in-space) interpretation. The benefits of this reformulation are discussed, including the derivation of timesteppers that are higher order in time, and the quantifiable dissipative SP properties in the non-ideal, resistive case.
        
        We discuss possible options for extending this FET approach to timesteppers for the compressible case.

        The kinetic corrections satisfy linearized Boltzmann equations. Using a Lénard--Bernstein collision operator, these take Fokker--Planck-like forms \cite{Fokker_1914, Planck_1917} wherein pseudo-particles in the numerical model obey the neoclassical transport equations, with particle-independent Brownian drift terms. This offers a rigorous methodology for incorporating collisions into the particle transport model, without coupling the equations of motions for each particle.
        
        Works by Chen, Chacón et al. \cite{Chen_Chacón_Barnes_2011, Chacón_Chen_Barnes_2013, Chen_Chacón_2014, Chen_Chacón_2015} have developed structure-preserving particle pushers for neoclassical transport in the Vlasov equations, derived from Crank--Nicolson integrators. We show these too can can derive from a FET interpretation, similarly offering potential extensions to higher-order-in-time particle pushers. The FET formulation is used also to consider how the stochastic drift terms can be incorporated into the pushers. Stochastic gyrokinetic expansions are also discussed.

        Different options for the numerical implementation of these schemes are considered.

        Due to the efficacy of FET in the development of SP timesteppers for both the fluid and kinetic component, we hope this approach will prove effective in the future for developing SP timesteppers for the full hybrid model. We hope this will give us the opportunity to incorporate previously inaccessible kinetic effects into the highly effective, modern, finite-element MHD models.
    \end{abstract}
    
    
    \newpage
    \tableofcontents
    
    
    \newpage
    \pagenumbering{arabic}
    %\linenumbers\renewcommand\thelinenumber{\color{black!50}\arabic{linenumber}}
            \input{0 - introduction/main.tex}
        \part{Research}
            \input{1 - low-noise PiC models/main.tex}
            \input{2 - kinetic component/main.tex}
            \input{3 - fluid component/main.tex}
            \input{4 - numerical implementation/main.tex}
        \part{Project Overview}
            \input{5 - research plan/main.tex}
            \input{6 - summary/main.tex}
    
    
    %\section{}
    \newpage
    \pagenumbering{gobble}
        \printbibliography


    \newpage
    \pagenumbering{roman}
    \appendix
        \part{Appendices}
            \input{8 - Hilbert complexes/main.tex}
            \input{9 - weak conservation proofs/main.tex}
\end{document}

            \documentclass[12pt, a4paper]{report}

\input{template/main.tex}

\title{\BA{Title in Progress...}}
\author{Boris Andrews}
\affil{Mathematical Institute, University of Oxford}
\date{\today}


\begin{document}
    \pagenumbering{gobble}
    \maketitle
    
    
    \begin{abstract}
        Magnetic confinement reactors---in particular tokamaks---offer one of the most promising options for achieving practical nuclear fusion, with the potential to provide virtually limitless, clean energy. The theoretical and numerical modeling of tokamak plasmas is simultaneously an essential component of effective reactor design, and a great research barrier. Tokamak operational conditions exhibit comparatively low Knudsen numbers. Kinetic effects, including kinetic waves and instabilities, Landau damping, bump-on-tail instabilities and more, are therefore highly influential in tokamak plasma dynamics. Purely fluid models are inherently incapable of capturing these effects, whereas the high dimensionality in purely kinetic models render them practically intractable for most relevant purposes.

        We consider a $\delta\!f$ decomposition model, with a macroscopic fluid background and microscopic kinetic correction, both fully coupled to each other. A similar manner of discretization is proposed to that used in the recent \texttt{STRUPHY} code \cite{Holderied_Possanner_Wang_2021, Holderied_2022, Li_et_al_2023} with a finite-element model for the background and a pseudo-particle/PiC model for the correction.

        The fluid background satisfies the full, non-linear, resistive, compressible, Hall MHD equations. \cite{Laakmann_Hu_Farrell_2022} introduces finite-element(-in-space) implicit timesteppers for the incompressible analogue to this system with structure-preserving (SP) properties in the ideal case, alongside parameter-robust preconditioners. We show that these timesteppers can derive from a finite-element-in-time (FET) (and finite-element-in-space) interpretation. The benefits of this reformulation are discussed, including the derivation of timesteppers that are higher order in time, and the quantifiable dissipative SP properties in the non-ideal, resistive case.
        
        We discuss possible options for extending this FET approach to timesteppers for the compressible case.

        The kinetic corrections satisfy linearized Boltzmann equations. Using a Lénard--Bernstein collision operator, these take Fokker--Planck-like forms \cite{Fokker_1914, Planck_1917} wherein pseudo-particles in the numerical model obey the neoclassical transport equations, with particle-independent Brownian drift terms. This offers a rigorous methodology for incorporating collisions into the particle transport model, without coupling the equations of motions for each particle.
        
        Works by Chen, Chacón et al. \cite{Chen_Chacón_Barnes_2011, Chacón_Chen_Barnes_2013, Chen_Chacón_2014, Chen_Chacón_2015} have developed structure-preserving particle pushers for neoclassical transport in the Vlasov equations, derived from Crank--Nicolson integrators. We show these too can can derive from a FET interpretation, similarly offering potential extensions to higher-order-in-time particle pushers. The FET formulation is used also to consider how the stochastic drift terms can be incorporated into the pushers. Stochastic gyrokinetic expansions are also discussed.

        Different options for the numerical implementation of these schemes are considered.

        Due to the efficacy of FET in the development of SP timesteppers for both the fluid and kinetic component, we hope this approach will prove effective in the future for developing SP timesteppers for the full hybrid model. We hope this will give us the opportunity to incorporate previously inaccessible kinetic effects into the highly effective, modern, finite-element MHD models.
    \end{abstract}
    
    
    \newpage
    \tableofcontents
    
    
    \newpage
    \pagenumbering{arabic}
    %\linenumbers\renewcommand\thelinenumber{\color{black!50}\arabic{linenumber}}
            \input{0 - introduction/main.tex}
        \part{Research}
            \input{1 - low-noise PiC models/main.tex}
            \input{2 - kinetic component/main.tex}
            \input{3 - fluid component/main.tex}
            \input{4 - numerical implementation/main.tex}
        \part{Project Overview}
            \input{5 - research plan/main.tex}
            \input{6 - summary/main.tex}
    
    
    %\section{}
    \newpage
    \pagenumbering{gobble}
        \printbibliography


    \newpage
    \pagenumbering{roman}
    \appendix
        \part{Appendices}
            \input{8 - Hilbert complexes/main.tex}
            \input{9 - weak conservation proofs/main.tex}
\end{document}

            \documentclass[12pt, a4paper]{report}

\input{template/main.tex}

\title{\BA{Title in Progress...}}
\author{Boris Andrews}
\affil{Mathematical Institute, University of Oxford}
\date{\today}


\begin{document}
    \pagenumbering{gobble}
    \maketitle
    
    
    \begin{abstract}
        Magnetic confinement reactors---in particular tokamaks---offer one of the most promising options for achieving practical nuclear fusion, with the potential to provide virtually limitless, clean energy. The theoretical and numerical modeling of tokamak plasmas is simultaneously an essential component of effective reactor design, and a great research barrier. Tokamak operational conditions exhibit comparatively low Knudsen numbers. Kinetic effects, including kinetic waves and instabilities, Landau damping, bump-on-tail instabilities and more, are therefore highly influential in tokamak plasma dynamics. Purely fluid models are inherently incapable of capturing these effects, whereas the high dimensionality in purely kinetic models render them practically intractable for most relevant purposes.

        We consider a $\delta\!f$ decomposition model, with a macroscopic fluid background and microscopic kinetic correction, both fully coupled to each other. A similar manner of discretization is proposed to that used in the recent \texttt{STRUPHY} code \cite{Holderied_Possanner_Wang_2021, Holderied_2022, Li_et_al_2023} with a finite-element model for the background and a pseudo-particle/PiC model for the correction.

        The fluid background satisfies the full, non-linear, resistive, compressible, Hall MHD equations. \cite{Laakmann_Hu_Farrell_2022} introduces finite-element(-in-space) implicit timesteppers for the incompressible analogue to this system with structure-preserving (SP) properties in the ideal case, alongside parameter-robust preconditioners. We show that these timesteppers can derive from a finite-element-in-time (FET) (and finite-element-in-space) interpretation. The benefits of this reformulation are discussed, including the derivation of timesteppers that are higher order in time, and the quantifiable dissipative SP properties in the non-ideal, resistive case.
        
        We discuss possible options for extending this FET approach to timesteppers for the compressible case.

        The kinetic corrections satisfy linearized Boltzmann equations. Using a Lénard--Bernstein collision operator, these take Fokker--Planck-like forms \cite{Fokker_1914, Planck_1917} wherein pseudo-particles in the numerical model obey the neoclassical transport equations, with particle-independent Brownian drift terms. This offers a rigorous methodology for incorporating collisions into the particle transport model, without coupling the equations of motions for each particle.
        
        Works by Chen, Chacón et al. \cite{Chen_Chacón_Barnes_2011, Chacón_Chen_Barnes_2013, Chen_Chacón_2014, Chen_Chacón_2015} have developed structure-preserving particle pushers for neoclassical transport in the Vlasov equations, derived from Crank--Nicolson integrators. We show these too can can derive from a FET interpretation, similarly offering potential extensions to higher-order-in-time particle pushers. The FET formulation is used also to consider how the stochastic drift terms can be incorporated into the pushers. Stochastic gyrokinetic expansions are also discussed.

        Different options for the numerical implementation of these schemes are considered.

        Due to the efficacy of FET in the development of SP timesteppers for both the fluid and kinetic component, we hope this approach will prove effective in the future for developing SP timesteppers for the full hybrid model. We hope this will give us the opportunity to incorporate previously inaccessible kinetic effects into the highly effective, modern, finite-element MHD models.
    \end{abstract}
    
    
    \newpage
    \tableofcontents
    
    
    \newpage
    \pagenumbering{arabic}
    %\linenumbers\renewcommand\thelinenumber{\color{black!50}\arabic{linenumber}}
            \input{0 - introduction/main.tex}
        \part{Research}
            \input{1 - low-noise PiC models/main.tex}
            \input{2 - kinetic component/main.tex}
            \input{3 - fluid component/main.tex}
            \input{4 - numerical implementation/main.tex}
        \part{Project Overview}
            \input{5 - research plan/main.tex}
            \input{6 - summary/main.tex}
    
    
    %\section{}
    \newpage
    \pagenumbering{gobble}
        \printbibliography


    \newpage
    \pagenumbering{roman}
    \appendix
        \part{Appendices}
            \input{8 - Hilbert complexes/main.tex}
            \input{9 - weak conservation proofs/main.tex}
\end{document}

            \documentclass[12pt, a4paper]{report}

\input{template/main.tex}

\title{\BA{Title in Progress...}}
\author{Boris Andrews}
\affil{Mathematical Institute, University of Oxford}
\date{\today}


\begin{document}
    \pagenumbering{gobble}
    \maketitle
    
    
    \begin{abstract}
        Magnetic confinement reactors---in particular tokamaks---offer one of the most promising options for achieving practical nuclear fusion, with the potential to provide virtually limitless, clean energy. The theoretical and numerical modeling of tokamak plasmas is simultaneously an essential component of effective reactor design, and a great research barrier. Tokamak operational conditions exhibit comparatively low Knudsen numbers. Kinetic effects, including kinetic waves and instabilities, Landau damping, bump-on-tail instabilities and more, are therefore highly influential in tokamak plasma dynamics. Purely fluid models are inherently incapable of capturing these effects, whereas the high dimensionality in purely kinetic models render them practically intractable for most relevant purposes.

        We consider a $\delta\!f$ decomposition model, with a macroscopic fluid background and microscopic kinetic correction, both fully coupled to each other. A similar manner of discretization is proposed to that used in the recent \texttt{STRUPHY} code \cite{Holderied_Possanner_Wang_2021, Holderied_2022, Li_et_al_2023} with a finite-element model for the background and a pseudo-particle/PiC model for the correction.

        The fluid background satisfies the full, non-linear, resistive, compressible, Hall MHD equations. \cite{Laakmann_Hu_Farrell_2022} introduces finite-element(-in-space) implicit timesteppers for the incompressible analogue to this system with structure-preserving (SP) properties in the ideal case, alongside parameter-robust preconditioners. We show that these timesteppers can derive from a finite-element-in-time (FET) (and finite-element-in-space) interpretation. The benefits of this reformulation are discussed, including the derivation of timesteppers that are higher order in time, and the quantifiable dissipative SP properties in the non-ideal, resistive case.
        
        We discuss possible options for extending this FET approach to timesteppers for the compressible case.

        The kinetic corrections satisfy linearized Boltzmann equations. Using a Lénard--Bernstein collision operator, these take Fokker--Planck-like forms \cite{Fokker_1914, Planck_1917} wherein pseudo-particles in the numerical model obey the neoclassical transport equations, with particle-independent Brownian drift terms. This offers a rigorous methodology for incorporating collisions into the particle transport model, without coupling the equations of motions for each particle.
        
        Works by Chen, Chacón et al. \cite{Chen_Chacón_Barnes_2011, Chacón_Chen_Barnes_2013, Chen_Chacón_2014, Chen_Chacón_2015} have developed structure-preserving particle pushers for neoclassical transport in the Vlasov equations, derived from Crank--Nicolson integrators. We show these too can can derive from a FET interpretation, similarly offering potential extensions to higher-order-in-time particle pushers. The FET formulation is used also to consider how the stochastic drift terms can be incorporated into the pushers. Stochastic gyrokinetic expansions are also discussed.

        Different options for the numerical implementation of these schemes are considered.

        Due to the efficacy of FET in the development of SP timesteppers for both the fluid and kinetic component, we hope this approach will prove effective in the future for developing SP timesteppers for the full hybrid model. We hope this will give us the opportunity to incorporate previously inaccessible kinetic effects into the highly effective, modern, finite-element MHD models.
    \end{abstract}
    
    
    \newpage
    \tableofcontents
    
    
    \newpage
    \pagenumbering{arabic}
    %\linenumbers\renewcommand\thelinenumber{\color{black!50}\arabic{linenumber}}
            \input{0 - introduction/main.tex}
        \part{Research}
            \input{1 - low-noise PiC models/main.tex}
            \input{2 - kinetic component/main.tex}
            \input{3 - fluid component/main.tex}
            \input{4 - numerical implementation/main.tex}
        \part{Project Overview}
            \input{5 - research plan/main.tex}
            \input{6 - summary/main.tex}
    
    
    %\section{}
    \newpage
    \pagenumbering{gobble}
        \printbibliography


    \newpage
    \pagenumbering{roman}
    \appendix
        \part{Appendices}
            \input{8 - Hilbert complexes/main.tex}
            \input{9 - weak conservation proofs/main.tex}
\end{document}

        \part{Project Overview}
            \documentclass[12pt, a4paper]{report}

\input{template/main.tex}

\title{\BA{Title in Progress...}}
\author{Boris Andrews}
\affil{Mathematical Institute, University of Oxford}
\date{\today}


\begin{document}
    \pagenumbering{gobble}
    \maketitle
    
    
    \begin{abstract}
        Magnetic confinement reactors---in particular tokamaks---offer one of the most promising options for achieving practical nuclear fusion, with the potential to provide virtually limitless, clean energy. The theoretical and numerical modeling of tokamak plasmas is simultaneously an essential component of effective reactor design, and a great research barrier. Tokamak operational conditions exhibit comparatively low Knudsen numbers. Kinetic effects, including kinetic waves and instabilities, Landau damping, bump-on-tail instabilities and more, are therefore highly influential in tokamak plasma dynamics. Purely fluid models are inherently incapable of capturing these effects, whereas the high dimensionality in purely kinetic models render them practically intractable for most relevant purposes.

        We consider a $\delta\!f$ decomposition model, with a macroscopic fluid background and microscopic kinetic correction, both fully coupled to each other. A similar manner of discretization is proposed to that used in the recent \texttt{STRUPHY} code \cite{Holderied_Possanner_Wang_2021, Holderied_2022, Li_et_al_2023} with a finite-element model for the background and a pseudo-particle/PiC model for the correction.

        The fluid background satisfies the full, non-linear, resistive, compressible, Hall MHD equations. \cite{Laakmann_Hu_Farrell_2022} introduces finite-element(-in-space) implicit timesteppers for the incompressible analogue to this system with structure-preserving (SP) properties in the ideal case, alongside parameter-robust preconditioners. We show that these timesteppers can derive from a finite-element-in-time (FET) (and finite-element-in-space) interpretation. The benefits of this reformulation are discussed, including the derivation of timesteppers that are higher order in time, and the quantifiable dissipative SP properties in the non-ideal, resistive case.
        
        We discuss possible options for extending this FET approach to timesteppers for the compressible case.

        The kinetic corrections satisfy linearized Boltzmann equations. Using a Lénard--Bernstein collision operator, these take Fokker--Planck-like forms \cite{Fokker_1914, Planck_1917} wherein pseudo-particles in the numerical model obey the neoclassical transport equations, with particle-independent Brownian drift terms. This offers a rigorous methodology for incorporating collisions into the particle transport model, without coupling the equations of motions for each particle.
        
        Works by Chen, Chacón et al. \cite{Chen_Chacón_Barnes_2011, Chacón_Chen_Barnes_2013, Chen_Chacón_2014, Chen_Chacón_2015} have developed structure-preserving particle pushers for neoclassical transport in the Vlasov equations, derived from Crank--Nicolson integrators. We show these too can can derive from a FET interpretation, similarly offering potential extensions to higher-order-in-time particle pushers. The FET formulation is used also to consider how the stochastic drift terms can be incorporated into the pushers. Stochastic gyrokinetic expansions are also discussed.

        Different options for the numerical implementation of these schemes are considered.

        Due to the efficacy of FET in the development of SP timesteppers for both the fluid and kinetic component, we hope this approach will prove effective in the future for developing SP timesteppers for the full hybrid model. We hope this will give us the opportunity to incorporate previously inaccessible kinetic effects into the highly effective, modern, finite-element MHD models.
    \end{abstract}
    
    
    \newpage
    \tableofcontents
    
    
    \newpage
    \pagenumbering{arabic}
    %\linenumbers\renewcommand\thelinenumber{\color{black!50}\arabic{linenumber}}
            \input{0 - introduction/main.tex}
        \part{Research}
            \input{1 - low-noise PiC models/main.tex}
            \input{2 - kinetic component/main.tex}
            \input{3 - fluid component/main.tex}
            \input{4 - numerical implementation/main.tex}
        \part{Project Overview}
            \input{5 - research plan/main.tex}
            \input{6 - summary/main.tex}
    
    
    %\section{}
    \newpage
    \pagenumbering{gobble}
        \printbibliography


    \newpage
    \pagenumbering{roman}
    \appendix
        \part{Appendices}
            \input{8 - Hilbert complexes/main.tex}
            \input{9 - weak conservation proofs/main.tex}
\end{document}

            \documentclass[12pt, a4paper]{report}

\input{template/main.tex}

\title{\BA{Title in Progress...}}
\author{Boris Andrews}
\affil{Mathematical Institute, University of Oxford}
\date{\today}


\begin{document}
    \pagenumbering{gobble}
    \maketitle
    
    
    \begin{abstract}
        Magnetic confinement reactors---in particular tokamaks---offer one of the most promising options for achieving practical nuclear fusion, with the potential to provide virtually limitless, clean energy. The theoretical and numerical modeling of tokamak plasmas is simultaneously an essential component of effective reactor design, and a great research barrier. Tokamak operational conditions exhibit comparatively low Knudsen numbers. Kinetic effects, including kinetic waves and instabilities, Landau damping, bump-on-tail instabilities and more, are therefore highly influential in tokamak plasma dynamics. Purely fluid models are inherently incapable of capturing these effects, whereas the high dimensionality in purely kinetic models render them practically intractable for most relevant purposes.

        We consider a $\delta\!f$ decomposition model, with a macroscopic fluid background and microscopic kinetic correction, both fully coupled to each other. A similar manner of discretization is proposed to that used in the recent \texttt{STRUPHY} code \cite{Holderied_Possanner_Wang_2021, Holderied_2022, Li_et_al_2023} with a finite-element model for the background and a pseudo-particle/PiC model for the correction.

        The fluid background satisfies the full, non-linear, resistive, compressible, Hall MHD equations. \cite{Laakmann_Hu_Farrell_2022} introduces finite-element(-in-space) implicit timesteppers for the incompressible analogue to this system with structure-preserving (SP) properties in the ideal case, alongside parameter-robust preconditioners. We show that these timesteppers can derive from a finite-element-in-time (FET) (and finite-element-in-space) interpretation. The benefits of this reformulation are discussed, including the derivation of timesteppers that are higher order in time, and the quantifiable dissipative SP properties in the non-ideal, resistive case.
        
        We discuss possible options for extending this FET approach to timesteppers for the compressible case.

        The kinetic corrections satisfy linearized Boltzmann equations. Using a Lénard--Bernstein collision operator, these take Fokker--Planck-like forms \cite{Fokker_1914, Planck_1917} wherein pseudo-particles in the numerical model obey the neoclassical transport equations, with particle-independent Brownian drift terms. This offers a rigorous methodology for incorporating collisions into the particle transport model, without coupling the equations of motions for each particle.
        
        Works by Chen, Chacón et al. \cite{Chen_Chacón_Barnes_2011, Chacón_Chen_Barnes_2013, Chen_Chacón_2014, Chen_Chacón_2015} have developed structure-preserving particle pushers for neoclassical transport in the Vlasov equations, derived from Crank--Nicolson integrators. We show these too can can derive from a FET interpretation, similarly offering potential extensions to higher-order-in-time particle pushers. The FET formulation is used also to consider how the stochastic drift terms can be incorporated into the pushers. Stochastic gyrokinetic expansions are also discussed.

        Different options for the numerical implementation of these schemes are considered.

        Due to the efficacy of FET in the development of SP timesteppers for both the fluid and kinetic component, we hope this approach will prove effective in the future for developing SP timesteppers for the full hybrid model. We hope this will give us the opportunity to incorporate previously inaccessible kinetic effects into the highly effective, modern, finite-element MHD models.
    \end{abstract}
    
    
    \newpage
    \tableofcontents
    
    
    \newpage
    \pagenumbering{arabic}
    %\linenumbers\renewcommand\thelinenumber{\color{black!50}\arabic{linenumber}}
            \input{0 - introduction/main.tex}
        \part{Research}
            \input{1 - low-noise PiC models/main.tex}
            \input{2 - kinetic component/main.tex}
            \input{3 - fluid component/main.tex}
            \input{4 - numerical implementation/main.tex}
        \part{Project Overview}
            \input{5 - research plan/main.tex}
            \input{6 - summary/main.tex}
    
    
    %\section{}
    \newpage
    \pagenumbering{gobble}
        \printbibliography


    \newpage
    \pagenumbering{roman}
    \appendix
        \part{Appendices}
            \input{8 - Hilbert complexes/main.tex}
            \input{9 - weak conservation proofs/main.tex}
\end{document}

    
    
    %\section{}
    \newpage
    \pagenumbering{gobble}
        \printbibliography


    \newpage
    \pagenumbering{roman}
    \appendix
        \part{Appendices}
            \documentclass[12pt, a4paper]{report}

\input{template/main.tex}

\title{\BA{Title in Progress...}}
\author{Boris Andrews}
\affil{Mathematical Institute, University of Oxford}
\date{\today}


\begin{document}
    \pagenumbering{gobble}
    \maketitle
    
    
    \begin{abstract}
        Magnetic confinement reactors---in particular tokamaks---offer one of the most promising options for achieving practical nuclear fusion, with the potential to provide virtually limitless, clean energy. The theoretical and numerical modeling of tokamak plasmas is simultaneously an essential component of effective reactor design, and a great research barrier. Tokamak operational conditions exhibit comparatively low Knudsen numbers. Kinetic effects, including kinetic waves and instabilities, Landau damping, bump-on-tail instabilities and more, are therefore highly influential in tokamak plasma dynamics. Purely fluid models are inherently incapable of capturing these effects, whereas the high dimensionality in purely kinetic models render them practically intractable for most relevant purposes.

        We consider a $\delta\!f$ decomposition model, with a macroscopic fluid background and microscopic kinetic correction, both fully coupled to each other. A similar manner of discretization is proposed to that used in the recent \texttt{STRUPHY} code \cite{Holderied_Possanner_Wang_2021, Holderied_2022, Li_et_al_2023} with a finite-element model for the background and a pseudo-particle/PiC model for the correction.

        The fluid background satisfies the full, non-linear, resistive, compressible, Hall MHD equations. \cite{Laakmann_Hu_Farrell_2022} introduces finite-element(-in-space) implicit timesteppers for the incompressible analogue to this system with structure-preserving (SP) properties in the ideal case, alongside parameter-robust preconditioners. We show that these timesteppers can derive from a finite-element-in-time (FET) (and finite-element-in-space) interpretation. The benefits of this reformulation are discussed, including the derivation of timesteppers that are higher order in time, and the quantifiable dissipative SP properties in the non-ideal, resistive case.
        
        We discuss possible options for extending this FET approach to timesteppers for the compressible case.

        The kinetic corrections satisfy linearized Boltzmann equations. Using a Lénard--Bernstein collision operator, these take Fokker--Planck-like forms \cite{Fokker_1914, Planck_1917} wherein pseudo-particles in the numerical model obey the neoclassical transport equations, with particle-independent Brownian drift terms. This offers a rigorous methodology for incorporating collisions into the particle transport model, without coupling the equations of motions for each particle.
        
        Works by Chen, Chacón et al. \cite{Chen_Chacón_Barnes_2011, Chacón_Chen_Barnes_2013, Chen_Chacón_2014, Chen_Chacón_2015} have developed structure-preserving particle pushers for neoclassical transport in the Vlasov equations, derived from Crank--Nicolson integrators. We show these too can can derive from a FET interpretation, similarly offering potential extensions to higher-order-in-time particle pushers. The FET formulation is used also to consider how the stochastic drift terms can be incorporated into the pushers. Stochastic gyrokinetic expansions are also discussed.

        Different options for the numerical implementation of these schemes are considered.

        Due to the efficacy of FET in the development of SP timesteppers for both the fluid and kinetic component, we hope this approach will prove effective in the future for developing SP timesteppers for the full hybrid model. We hope this will give us the opportunity to incorporate previously inaccessible kinetic effects into the highly effective, modern, finite-element MHD models.
    \end{abstract}
    
    
    \newpage
    \tableofcontents
    
    
    \newpage
    \pagenumbering{arabic}
    %\linenumbers\renewcommand\thelinenumber{\color{black!50}\arabic{linenumber}}
            \input{0 - introduction/main.tex}
        \part{Research}
            \input{1 - low-noise PiC models/main.tex}
            \input{2 - kinetic component/main.tex}
            \input{3 - fluid component/main.tex}
            \input{4 - numerical implementation/main.tex}
        \part{Project Overview}
            \input{5 - research plan/main.tex}
            \input{6 - summary/main.tex}
    
    
    %\section{}
    \newpage
    \pagenumbering{gobble}
        \printbibliography


    \newpage
    \pagenumbering{roman}
    \appendix
        \part{Appendices}
            \input{8 - Hilbert complexes/main.tex}
            \input{9 - weak conservation proofs/main.tex}
\end{document}

            \documentclass[12pt, a4paper]{report}

\input{template/main.tex}

\title{\BA{Title in Progress...}}
\author{Boris Andrews}
\affil{Mathematical Institute, University of Oxford}
\date{\today}


\begin{document}
    \pagenumbering{gobble}
    \maketitle
    
    
    \begin{abstract}
        Magnetic confinement reactors---in particular tokamaks---offer one of the most promising options for achieving practical nuclear fusion, with the potential to provide virtually limitless, clean energy. The theoretical and numerical modeling of tokamak plasmas is simultaneously an essential component of effective reactor design, and a great research barrier. Tokamak operational conditions exhibit comparatively low Knudsen numbers. Kinetic effects, including kinetic waves and instabilities, Landau damping, bump-on-tail instabilities and more, are therefore highly influential in tokamak plasma dynamics. Purely fluid models are inherently incapable of capturing these effects, whereas the high dimensionality in purely kinetic models render them practically intractable for most relevant purposes.

        We consider a $\delta\!f$ decomposition model, with a macroscopic fluid background and microscopic kinetic correction, both fully coupled to each other. A similar manner of discretization is proposed to that used in the recent \texttt{STRUPHY} code \cite{Holderied_Possanner_Wang_2021, Holderied_2022, Li_et_al_2023} with a finite-element model for the background and a pseudo-particle/PiC model for the correction.

        The fluid background satisfies the full, non-linear, resistive, compressible, Hall MHD equations. \cite{Laakmann_Hu_Farrell_2022} introduces finite-element(-in-space) implicit timesteppers for the incompressible analogue to this system with structure-preserving (SP) properties in the ideal case, alongside parameter-robust preconditioners. We show that these timesteppers can derive from a finite-element-in-time (FET) (and finite-element-in-space) interpretation. The benefits of this reformulation are discussed, including the derivation of timesteppers that are higher order in time, and the quantifiable dissipative SP properties in the non-ideal, resistive case.
        
        We discuss possible options for extending this FET approach to timesteppers for the compressible case.

        The kinetic corrections satisfy linearized Boltzmann equations. Using a Lénard--Bernstein collision operator, these take Fokker--Planck-like forms \cite{Fokker_1914, Planck_1917} wherein pseudo-particles in the numerical model obey the neoclassical transport equations, with particle-independent Brownian drift terms. This offers a rigorous methodology for incorporating collisions into the particle transport model, without coupling the equations of motions for each particle.
        
        Works by Chen, Chacón et al. \cite{Chen_Chacón_Barnes_2011, Chacón_Chen_Barnes_2013, Chen_Chacón_2014, Chen_Chacón_2015} have developed structure-preserving particle pushers for neoclassical transport in the Vlasov equations, derived from Crank--Nicolson integrators. We show these too can can derive from a FET interpretation, similarly offering potential extensions to higher-order-in-time particle pushers. The FET formulation is used also to consider how the stochastic drift terms can be incorporated into the pushers. Stochastic gyrokinetic expansions are also discussed.

        Different options for the numerical implementation of these schemes are considered.

        Due to the efficacy of FET in the development of SP timesteppers for both the fluid and kinetic component, we hope this approach will prove effective in the future for developing SP timesteppers for the full hybrid model. We hope this will give us the opportunity to incorporate previously inaccessible kinetic effects into the highly effective, modern, finite-element MHD models.
    \end{abstract}
    
    
    \newpage
    \tableofcontents
    
    
    \newpage
    \pagenumbering{arabic}
    %\linenumbers\renewcommand\thelinenumber{\color{black!50}\arabic{linenumber}}
            \input{0 - introduction/main.tex}
        \part{Research}
            \input{1 - low-noise PiC models/main.tex}
            \input{2 - kinetic component/main.tex}
            \input{3 - fluid component/main.tex}
            \input{4 - numerical implementation/main.tex}
        \part{Project Overview}
            \input{5 - research plan/main.tex}
            \input{6 - summary/main.tex}
    
    
    %\section{}
    \newpage
    \pagenumbering{gobble}
        \printbibliography


    \newpage
    \pagenumbering{roman}
    \appendix
        \part{Appendices}
            \input{8 - Hilbert complexes/main.tex}
            \input{9 - weak conservation proofs/main.tex}
\end{document}

\end{document}

            \documentclass[12pt, a4paper]{report}

\documentclass[12pt, a4paper]{report}

\input{template/main.tex}

\title{\BA{Title in Progress...}}
\author{Boris Andrews}
\affil{Mathematical Institute, University of Oxford}
\date{\today}


\begin{document}
    \pagenumbering{gobble}
    \maketitle
    
    
    \begin{abstract}
        Magnetic confinement reactors---in particular tokamaks---offer one of the most promising options for achieving practical nuclear fusion, with the potential to provide virtually limitless, clean energy. The theoretical and numerical modeling of tokamak plasmas is simultaneously an essential component of effective reactor design, and a great research barrier. Tokamak operational conditions exhibit comparatively low Knudsen numbers. Kinetic effects, including kinetic waves and instabilities, Landau damping, bump-on-tail instabilities and more, are therefore highly influential in tokamak plasma dynamics. Purely fluid models are inherently incapable of capturing these effects, whereas the high dimensionality in purely kinetic models render them practically intractable for most relevant purposes.

        We consider a $\delta\!f$ decomposition model, with a macroscopic fluid background and microscopic kinetic correction, both fully coupled to each other. A similar manner of discretization is proposed to that used in the recent \texttt{STRUPHY} code \cite{Holderied_Possanner_Wang_2021, Holderied_2022, Li_et_al_2023} with a finite-element model for the background and a pseudo-particle/PiC model for the correction.

        The fluid background satisfies the full, non-linear, resistive, compressible, Hall MHD equations. \cite{Laakmann_Hu_Farrell_2022} introduces finite-element(-in-space) implicit timesteppers for the incompressible analogue to this system with structure-preserving (SP) properties in the ideal case, alongside parameter-robust preconditioners. We show that these timesteppers can derive from a finite-element-in-time (FET) (and finite-element-in-space) interpretation. The benefits of this reformulation are discussed, including the derivation of timesteppers that are higher order in time, and the quantifiable dissipative SP properties in the non-ideal, resistive case.
        
        We discuss possible options for extending this FET approach to timesteppers for the compressible case.

        The kinetic corrections satisfy linearized Boltzmann equations. Using a Lénard--Bernstein collision operator, these take Fokker--Planck-like forms \cite{Fokker_1914, Planck_1917} wherein pseudo-particles in the numerical model obey the neoclassical transport equations, with particle-independent Brownian drift terms. This offers a rigorous methodology for incorporating collisions into the particle transport model, without coupling the equations of motions for each particle.
        
        Works by Chen, Chacón et al. \cite{Chen_Chacón_Barnes_2011, Chacón_Chen_Barnes_2013, Chen_Chacón_2014, Chen_Chacón_2015} have developed structure-preserving particle pushers for neoclassical transport in the Vlasov equations, derived from Crank--Nicolson integrators. We show these too can can derive from a FET interpretation, similarly offering potential extensions to higher-order-in-time particle pushers. The FET formulation is used also to consider how the stochastic drift terms can be incorporated into the pushers. Stochastic gyrokinetic expansions are also discussed.

        Different options for the numerical implementation of these schemes are considered.

        Due to the efficacy of FET in the development of SP timesteppers for both the fluid and kinetic component, we hope this approach will prove effective in the future for developing SP timesteppers for the full hybrid model. We hope this will give us the opportunity to incorporate previously inaccessible kinetic effects into the highly effective, modern, finite-element MHD models.
    \end{abstract}
    
    
    \newpage
    \tableofcontents
    
    
    \newpage
    \pagenumbering{arabic}
    %\linenumbers\renewcommand\thelinenumber{\color{black!50}\arabic{linenumber}}
            \input{0 - introduction/main.tex}
        \part{Research}
            \input{1 - low-noise PiC models/main.tex}
            \input{2 - kinetic component/main.tex}
            \input{3 - fluid component/main.tex}
            \input{4 - numerical implementation/main.tex}
        \part{Project Overview}
            \input{5 - research plan/main.tex}
            \input{6 - summary/main.tex}
    
    
    %\section{}
    \newpage
    \pagenumbering{gobble}
        \printbibliography


    \newpage
    \pagenumbering{roman}
    \appendix
        \part{Appendices}
            \input{8 - Hilbert complexes/main.tex}
            \input{9 - weak conservation proofs/main.tex}
\end{document}


\title{\BA{Title in Progress...}}
\author{Boris Andrews}
\affil{Mathematical Institute, University of Oxford}
\date{\today}


\begin{document}
    \pagenumbering{gobble}
    \maketitle
    
    
    \begin{abstract}
        Magnetic confinement reactors---in particular tokamaks---offer one of the most promising options for achieving practical nuclear fusion, with the potential to provide virtually limitless, clean energy. The theoretical and numerical modeling of tokamak plasmas is simultaneously an essential component of effective reactor design, and a great research barrier. Tokamak operational conditions exhibit comparatively low Knudsen numbers. Kinetic effects, including kinetic waves and instabilities, Landau damping, bump-on-tail instabilities and more, are therefore highly influential in tokamak plasma dynamics. Purely fluid models are inherently incapable of capturing these effects, whereas the high dimensionality in purely kinetic models render them practically intractable for most relevant purposes.

        We consider a $\delta\!f$ decomposition model, with a macroscopic fluid background and microscopic kinetic correction, both fully coupled to each other. A similar manner of discretization is proposed to that used in the recent \texttt{STRUPHY} code \cite{Holderied_Possanner_Wang_2021, Holderied_2022, Li_et_al_2023} with a finite-element model for the background and a pseudo-particle/PiC model for the correction.

        The fluid background satisfies the full, non-linear, resistive, compressible, Hall MHD equations. \cite{Laakmann_Hu_Farrell_2022} introduces finite-element(-in-space) implicit timesteppers for the incompressible analogue to this system with structure-preserving (SP) properties in the ideal case, alongside parameter-robust preconditioners. We show that these timesteppers can derive from a finite-element-in-time (FET) (and finite-element-in-space) interpretation. The benefits of this reformulation are discussed, including the derivation of timesteppers that are higher order in time, and the quantifiable dissipative SP properties in the non-ideal, resistive case.
        
        We discuss possible options for extending this FET approach to timesteppers for the compressible case.

        The kinetic corrections satisfy linearized Boltzmann equations. Using a Lénard--Bernstein collision operator, these take Fokker--Planck-like forms \cite{Fokker_1914, Planck_1917} wherein pseudo-particles in the numerical model obey the neoclassical transport equations, with particle-independent Brownian drift terms. This offers a rigorous methodology for incorporating collisions into the particle transport model, without coupling the equations of motions for each particle.
        
        Works by Chen, Chacón et al. \cite{Chen_Chacón_Barnes_2011, Chacón_Chen_Barnes_2013, Chen_Chacón_2014, Chen_Chacón_2015} have developed structure-preserving particle pushers for neoclassical transport in the Vlasov equations, derived from Crank--Nicolson integrators. We show these too can can derive from a FET interpretation, similarly offering potential extensions to higher-order-in-time particle pushers. The FET formulation is used also to consider how the stochastic drift terms can be incorporated into the pushers. Stochastic gyrokinetic expansions are also discussed.

        Different options for the numerical implementation of these schemes are considered.

        Due to the efficacy of FET in the development of SP timesteppers for both the fluid and kinetic component, we hope this approach will prove effective in the future for developing SP timesteppers for the full hybrid model. We hope this will give us the opportunity to incorporate previously inaccessible kinetic effects into the highly effective, modern, finite-element MHD models.
    \end{abstract}
    
    
    \newpage
    \tableofcontents
    
    
    \newpage
    \pagenumbering{arabic}
    %\linenumbers\renewcommand\thelinenumber{\color{black!50}\arabic{linenumber}}
            \documentclass[12pt, a4paper]{report}

\input{template/main.tex}

\title{\BA{Title in Progress...}}
\author{Boris Andrews}
\affil{Mathematical Institute, University of Oxford}
\date{\today}


\begin{document}
    \pagenumbering{gobble}
    \maketitle
    
    
    \begin{abstract}
        Magnetic confinement reactors---in particular tokamaks---offer one of the most promising options for achieving practical nuclear fusion, with the potential to provide virtually limitless, clean energy. The theoretical and numerical modeling of tokamak plasmas is simultaneously an essential component of effective reactor design, and a great research barrier. Tokamak operational conditions exhibit comparatively low Knudsen numbers. Kinetic effects, including kinetic waves and instabilities, Landau damping, bump-on-tail instabilities and more, are therefore highly influential in tokamak plasma dynamics. Purely fluid models are inherently incapable of capturing these effects, whereas the high dimensionality in purely kinetic models render them practically intractable for most relevant purposes.

        We consider a $\delta\!f$ decomposition model, with a macroscopic fluid background and microscopic kinetic correction, both fully coupled to each other. A similar manner of discretization is proposed to that used in the recent \texttt{STRUPHY} code \cite{Holderied_Possanner_Wang_2021, Holderied_2022, Li_et_al_2023} with a finite-element model for the background and a pseudo-particle/PiC model for the correction.

        The fluid background satisfies the full, non-linear, resistive, compressible, Hall MHD equations. \cite{Laakmann_Hu_Farrell_2022} introduces finite-element(-in-space) implicit timesteppers for the incompressible analogue to this system with structure-preserving (SP) properties in the ideal case, alongside parameter-robust preconditioners. We show that these timesteppers can derive from a finite-element-in-time (FET) (and finite-element-in-space) interpretation. The benefits of this reformulation are discussed, including the derivation of timesteppers that are higher order in time, and the quantifiable dissipative SP properties in the non-ideal, resistive case.
        
        We discuss possible options for extending this FET approach to timesteppers for the compressible case.

        The kinetic corrections satisfy linearized Boltzmann equations. Using a Lénard--Bernstein collision operator, these take Fokker--Planck-like forms \cite{Fokker_1914, Planck_1917} wherein pseudo-particles in the numerical model obey the neoclassical transport equations, with particle-independent Brownian drift terms. This offers a rigorous methodology for incorporating collisions into the particle transport model, without coupling the equations of motions for each particle.
        
        Works by Chen, Chacón et al. \cite{Chen_Chacón_Barnes_2011, Chacón_Chen_Barnes_2013, Chen_Chacón_2014, Chen_Chacón_2015} have developed structure-preserving particle pushers for neoclassical transport in the Vlasov equations, derived from Crank--Nicolson integrators. We show these too can can derive from a FET interpretation, similarly offering potential extensions to higher-order-in-time particle pushers. The FET formulation is used also to consider how the stochastic drift terms can be incorporated into the pushers. Stochastic gyrokinetic expansions are also discussed.

        Different options for the numerical implementation of these schemes are considered.

        Due to the efficacy of FET in the development of SP timesteppers for both the fluid and kinetic component, we hope this approach will prove effective in the future for developing SP timesteppers for the full hybrid model. We hope this will give us the opportunity to incorporate previously inaccessible kinetic effects into the highly effective, modern, finite-element MHD models.
    \end{abstract}
    
    
    \newpage
    \tableofcontents
    
    
    \newpage
    \pagenumbering{arabic}
    %\linenumbers\renewcommand\thelinenumber{\color{black!50}\arabic{linenumber}}
            \input{0 - introduction/main.tex}
        \part{Research}
            \input{1 - low-noise PiC models/main.tex}
            \input{2 - kinetic component/main.tex}
            \input{3 - fluid component/main.tex}
            \input{4 - numerical implementation/main.tex}
        \part{Project Overview}
            \input{5 - research plan/main.tex}
            \input{6 - summary/main.tex}
    
    
    %\section{}
    \newpage
    \pagenumbering{gobble}
        \printbibliography


    \newpage
    \pagenumbering{roman}
    \appendix
        \part{Appendices}
            \input{8 - Hilbert complexes/main.tex}
            \input{9 - weak conservation proofs/main.tex}
\end{document}

        \part{Research}
            \documentclass[12pt, a4paper]{report}

\input{template/main.tex}

\title{\BA{Title in Progress...}}
\author{Boris Andrews}
\affil{Mathematical Institute, University of Oxford}
\date{\today}


\begin{document}
    \pagenumbering{gobble}
    \maketitle
    
    
    \begin{abstract}
        Magnetic confinement reactors---in particular tokamaks---offer one of the most promising options for achieving practical nuclear fusion, with the potential to provide virtually limitless, clean energy. The theoretical and numerical modeling of tokamak plasmas is simultaneously an essential component of effective reactor design, and a great research barrier. Tokamak operational conditions exhibit comparatively low Knudsen numbers. Kinetic effects, including kinetic waves and instabilities, Landau damping, bump-on-tail instabilities and more, are therefore highly influential in tokamak plasma dynamics. Purely fluid models are inherently incapable of capturing these effects, whereas the high dimensionality in purely kinetic models render them practically intractable for most relevant purposes.

        We consider a $\delta\!f$ decomposition model, with a macroscopic fluid background and microscopic kinetic correction, both fully coupled to each other. A similar manner of discretization is proposed to that used in the recent \texttt{STRUPHY} code \cite{Holderied_Possanner_Wang_2021, Holderied_2022, Li_et_al_2023} with a finite-element model for the background and a pseudo-particle/PiC model for the correction.

        The fluid background satisfies the full, non-linear, resistive, compressible, Hall MHD equations. \cite{Laakmann_Hu_Farrell_2022} introduces finite-element(-in-space) implicit timesteppers for the incompressible analogue to this system with structure-preserving (SP) properties in the ideal case, alongside parameter-robust preconditioners. We show that these timesteppers can derive from a finite-element-in-time (FET) (and finite-element-in-space) interpretation. The benefits of this reformulation are discussed, including the derivation of timesteppers that are higher order in time, and the quantifiable dissipative SP properties in the non-ideal, resistive case.
        
        We discuss possible options for extending this FET approach to timesteppers for the compressible case.

        The kinetic corrections satisfy linearized Boltzmann equations. Using a Lénard--Bernstein collision operator, these take Fokker--Planck-like forms \cite{Fokker_1914, Planck_1917} wherein pseudo-particles in the numerical model obey the neoclassical transport equations, with particle-independent Brownian drift terms. This offers a rigorous methodology for incorporating collisions into the particle transport model, without coupling the equations of motions for each particle.
        
        Works by Chen, Chacón et al. \cite{Chen_Chacón_Barnes_2011, Chacón_Chen_Barnes_2013, Chen_Chacón_2014, Chen_Chacón_2015} have developed structure-preserving particle pushers for neoclassical transport in the Vlasov equations, derived from Crank--Nicolson integrators. We show these too can can derive from a FET interpretation, similarly offering potential extensions to higher-order-in-time particle pushers. The FET formulation is used also to consider how the stochastic drift terms can be incorporated into the pushers. Stochastic gyrokinetic expansions are also discussed.

        Different options for the numerical implementation of these schemes are considered.

        Due to the efficacy of FET in the development of SP timesteppers for both the fluid and kinetic component, we hope this approach will prove effective in the future for developing SP timesteppers for the full hybrid model. We hope this will give us the opportunity to incorporate previously inaccessible kinetic effects into the highly effective, modern, finite-element MHD models.
    \end{abstract}
    
    
    \newpage
    \tableofcontents
    
    
    \newpage
    \pagenumbering{arabic}
    %\linenumbers\renewcommand\thelinenumber{\color{black!50}\arabic{linenumber}}
            \input{0 - introduction/main.tex}
        \part{Research}
            \input{1 - low-noise PiC models/main.tex}
            \input{2 - kinetic component/main.tex}
            \input{3 - fluid component/main.tex}
            \input{4 - numerical implementation/main.tex}
        \part{Project Overview}
            \input{5 - research plan/main.tex}
            \input{6 - summary/main.tex}
    
    
    %\section{}
    \newpage
    \pagenumbering{gobble}
        \printbibliography


    \newpage
    \pagenumbering{roman}
    \appendix
        \part{Appendices}
            \input{8 - Hilbert complexes/main.tex}
            \input{9 - weak conservation proofs/main.tex}
\end{document}

            \documentclass[12pt, a4paper]{report}

\input{template/main.tex}

\title{\BA{Title in Progress...}}
\author{Boris Andrews}
\affil{Mathematical Institute, University of Oxford}
\date{\today}


\begin{document}
    \pagenumbering{gobble}
    \maketitle
    
    
    \begin{abstract}
        Magnetic confinement reactors---in particular tokamaks---offer one of the most promising options for achieving practical nuclear fusion, with the potential to provide virtually limitless, clean energy. The theoretical and numerical modeling of tokamak plasmas is simultaneously an essential component of effective reactor design, and a great research barrier. Tokamak operational conditions exhibit comparatively low Knudsen numbers. Kinetic effects, including kinetic waves and instabilities, Landau damping, bump-on-tail instabilities and more, are therefore highly influential in tokamak plasma dynamics. Purely fluid models are inherently incapable of capturing these effects, whereas the high dimensionality in purely kinetic models render them practically intractable for most relevant purposes.

        We consider a $\delta\!f$ decomposition model, with a macroscopic fluid background and microscopic kinetic correction, both fully coupled to each other. A similar manner of discretization is proposed to that used in the recent \texttt{STRUPHY} code \cite{Holderied_Possanner_Wang_2021, Holderied_2022, Li_et_al_2023} with a finite-element model for the background and a pseudo-particle/PiC model for the correction.

        The fluid background satisfies the full, non-linear, resistive, compressible, Hall MHD equations. \cite{Laakmann_Hu_Farrell_2022} introduces finite-element(-in-space) implicit timesteppers for the incompressible analogue to this system with structure-preserving (SP) properties in the ideal case, alongside parameter-robust preconditioners. We show that these timesteppers can derive from a finite-element-in-time (FET) (and finite-element-in-space) interpretation. The benefits of this reformulation are discussed, including the derivation of timesteppers that are higher order in time, and the quantifiable dissipative SP properties in the non-ideal, resistive case.
        
        We discuss possible options for extending this FET approach to timesteppers for the compressible case.

        The kinetic corrections satisfy linearized Boltzmann equations. Using a Lénard--Bernstein collision operator, these take Fokker--Planck-like forms \cite{Fokker_1914, Planck_1917} wherein pseudo-particles in the numerical model obey the neoclassical transport equations, with particle-independent Brownian drift terms. This offers a rigorous methodology for incorporating collisions into the particle transport model, without coupling the equations of motions for each particle.
        
        Works by Chen, Chacón et al. \cite{Chen_Chacón_Barnes_2011, Chacón_Chen_Barnes_2013, Chen_Chacón_2014, Chen_Chacón_2015} have developed structure-preserving particle pushers for neoclassical transport in the Vlasov equations, derived from Crank--Nicolson integrators. We show these too can can derive from a FET interpretation, similarly offering potential extensions to higher-order-in-time particle pushers. The FET formulation is used also to consider how the stochastic drift terms can be incorporated into the pushers. Stochastic gyrokinetic expansions are also discussed.

        Different options for the numerical implementation of these schemes are considered.

        Due to the efficacy of FET in the development of SP timesteppers for both the fluid and kinetic component, we hope this approach will prove effective in the future for developing SP timesteppers for the full hybrid model. We hope this will give us the opportunity to incorporate previously inaccessible kinetic effects into the highly effective, modern, finite-element MHD models.
    \end{abstract}
    
    
    \newpage
    \tableofcontents
    
    
    \newpage
    \pagenumbering{arabic}
    %\linenumbers\renewcommand\thelinenumber{\color{black!50}\arabic{linenumber}}
            \input{0 - introduction/main.tex}
        \part{Research}
            \input{1 - low-noise PiC models/main.tex}
            \input{2 - kinetic component/main.tex}
            \input{3 - fluid component/main.tex}
            \input{4 - numerical implementation/main.tex}
        \part{Project Overview}
            \input{5 - research plan/main.tex}
            \input{6 - summary/main.tex}
    
    
    %\section{}
    \newpage
    \pagenumbering{gobble}
        \printbibliography


    \newpage
    \pagenumbering{roman}
    \appendix
        \part{Appendices}
            \input{8 - Hilbert complexes/main.tex}
            \input{9 - weak conservation proofs/main.tex}
\end{document}

            \documentclass[12pt, a4paper]{report}

\input{template/main.tex}

\title{\BA{Title in Progress...}}
\author{Boris Andrews}
\affil{Mathematical Institute, University of Oxford}
\date{\today}


\begin{document}
    \pagenumbering{gobble}
    \maketitle
    
    
    \begin{abstract}
        Magnetic confinement reactors---in particular tokamaks---offer one of the most promising options for achieving practical nuclear fusion, with the potential to provide virtually limitless, clean energy. The theoretical and numerical modeling of tokamak plasmas is simultaneously an essential component of effective reactor design, and a great research barrier. Tokamak operational conditions exhibit comparatively low Knudsen numbers. Kinetic effects, including kinetic waves and instabilities, Landau damping, bump-on-tail instabilities and more, are therefore highly influential in tokamak plasma dynamics. Purely fluid models are inherently incapable of capturing these effects, whereas the high dimensionality in purely kinetic models render them practically intractable for most relevant purposes.

        We consider a $\delta\!f$ decomposition model, with a macroscopic fluid background and microscopic kinetic correction, both fully coupled to each other. A similar manner of discretization is proposed to that used in the recent \texttt{STRUPHY} code \cite{Holderied_Possanner_Wang_2021, Holderied_2022, Li_et_al_2023} with a finite-element model for the background and a pseudo-particle/PiC model for the correction.

        The fluid background satisfies the full, non-linear, resistive, compressible, Hall MHD equations. \cite{Laakmann_Hu_Farrell_2022} introduces finite-element(-in-space) implicit timesteppers for the incompressible analogue to this system with structure-preserving (SP) properties in the ideal case, alongside parameter-robust preconditioners. We show that these timesteppers can derive from a finite-element-in-time (FET) (and finite-element-in-space) interpretation. The benefits of this reformulation are discussed, including the derivation of timesteppers that are higher order in time, and the quantifiable dissipative SP properties in the non-ideal, resistive case.
        
        We discuss possible options for extending this FET approach to timesteppers for the compressible case.

        The kinetic corrections satisfy linearized Boltzmann equations. Using a Lénard--Bernstein collision operator, these take Fokker--Planck-like forms \cite{Fokker_1914, Planck_1917} wherein pseudo-particles in the numerical model obey the neoclassical transport equations, with particle-independent Brownian drift terms. This offers a rigorous methodology for incorporating collisions into the particle transport model, without coupling the equations of motions for each particle.
        
        Works by Chen, Chacón et al. \cite{Chen_Chacón_Barnes_2011, Chacón_Chen_Barnes_2013, Chen_Chacón_2014, Chen_Chacón_2015} have developed structure-preserving particle pushers for neoclassical transport in the Vlasov equations, derived from Crank--Nicolson integrators. We show these too can can derive from a FET interpretation, similarly offering potential extensions to higher-order-in-time particle pushers. The FET formulation is used also to consider how the stochastic drift terms can be incorporated into the pushers. Stochastic gyrokinetic expansions are also discussed.

        Different options for the numerical implementation of these schemes are considered.

        Due to the efficacy of FET in the development of SP timesteppers for both the fluid and kinetic component, we hope this approach will prove effective in the future for developing SP timesteppers for the full hybrid model. We hope this will give us the opportunity to incorporate previously inaccessible kinetic effects into the highly effective, modern, finite-element MHD models.
    \end{abstract}
    
    
    \newpage
    \tableofcontents
    
    
    \newpage
    \pagenumbering{arabic}
    %\linenumbers\renewcommand\thelinenumber{\color{black!50}\arabic{linenumber}}
            \input{0 - introduction/main.tex}
        \part{Research}
            \input{1 - low-noise PiC models/main.tex}
            \input{2 - kinetic component/main.tex}
            \input{3 - fluid component/main.tex}
            \input{4 - numerical implementation/main.tex}
        \part{Project Overview}
            \input{5 - research plan/main.tex}
            \input{6 - summary/main.tex}
    
    
    %\section{}
    \newpage
    \pagenumbering{gobble}
        \printbibliography


    \newpage
    \pagenumbering{roman}
    \appendix
        \part{Appendices}
            \input{8 - Hilbert complexes/main.tex}
            \input{9 - weak conservation proofs/main.tex}
\end{document}

            \documentclass[12pt, a4paper]{report}

\input{template/main.tex}

\title{\BA{Title in Progress...}}
\author{Boris Andrews}
\affil{Mathematical Institute, University of Oxford}
\date{\today}


\begin{document}
    \pagenumbering{gobble}
    \maketitle
    
    
    \begin{abstract}
        Magnetic confinement reactors---in particular tokamaks---offer one of the most promising options for achieving practical nuclear fusion, with the potential to provide virtually limitless, clean energy. The theoretical and numerical modeling of tokamak plasmas is simultaneously an essential component of effective reactor design, and a great research barrier. Tokamak operational conditions exhibit comparatively low Knudsen numbers. Kinetic effects, including kinetic waves and instabilities, Landau damping, bump-on-tail instabilities and more, are therefore highly influential in tokamak plasma dynamics. Purely fluid models are inherently incapable of capturing these effects, whereas the high dimensionality in purely kinetic models render them practically intractable for most relevant purposes.

        We consider a $\delta\!f$ decomposition model, with a macroscopic fluid background and microscopic kinetic correction, both fully coupled to each other. A similar manner of discretization is proposed to that used in the recent \texttt{STRUPHY} code \cite{Holderied_Possanner_Wang_2021, Holderied_2022, Li_et_al_2023} with a finite-element model for the background and a pseudo-particle/PiC model for the correction.

        The fluid background satisfies the full, non-linear, resistive, compressible, Hall MHD equations. \cite{Laakmann_Hu_Farrell_2022} introduces finite-element(-in-space) implicit timesteppers for the incompressible analogue to this system with structure-preserving (SP) properties in the ideal case, alongside parameter-robust preconditioners. We show that these timesteppers can derive from a finite-element-in-time (FET) (and finite-element-in-space) interpretation. The benefits of this reformulation are discussed, including the derivation of timesteppers that are higher order in time, and the quantifiable dissipative SP properties in the non-ideal, resistive case.
        
        We discuss possible options for extending this FET approach to timesteppers for the compressible case.

        The kinetic corrections satisfy linearized Boltzmann equations. Using a Lénard--Bernstein collision operator, these take Fokker--Planck-like forms \cite{Fokker_1914, Planck_1917} wherein pseudo-particles in the numerical model obey the neoclassical transport equations, with particle-independent Brownian drift terms. This offers a rigorous methodology for incorporating collisions into the particle transport model, without coupling the equations of motions for each particle.
        
        Works by Chen, Chacón et al. \cite{Chen_Chacón_Barnes_2011, Chacón_Chen_Barnes_2013, Chen_Chacón_2014, Chen_Chacón_2015} have developed structure-preserving particle pushers for neoclassical transport in the Vlasov equations, derived from Crank--Nicolson integrators. We show these too can can derive from a FET interpretation, similarly offering potential extensions to higher-order-in-time particle pushers. The FET formulation is used also to consider how the stochastic drift terms can be incorporated into the pushers. Stochastic gyrokinetic expansions are also discussed.

        Different options for the numerical implementation of these schemes are considered.

        Due to the efficacy of FET in the development of SP timesteppers for both the fluid and kinetic component, we hope this approach will prove effective in the future for developing SP timesteppers for the full hybrid model. We hope this will give us the opportunity to incorporate previously inaccessible kinetic effects into the highly effective, modern, finite-element MHD models.
    \end{abstract}
    
    
    \newpage
    \tableofcontents
    
    
    \newpage
    \pagenumbering{arabic}
    %\linenumbers\renewcommand\thelinenumber{\color{black!50}\arabic{linenumber}}
            \input{0 - introduction/main.tex}
        \part{Research}
            \input{1 - low-noise PiC models/main.tex}
            \input{2 - kinetic component/main.tex}
            \input{3 - fluid component/main.tex}
            \input{4 - numerical implementation/main.tex}
        \part{Project Overview}
            \input{5 - research plan/main.tex}
            \input{6 - summary/main.tex}
    
    
    %\section{}
    \newpage
    \pagenumbering{gobble}
        \printbibliography


    \newpage
    \pagenumbering{roman}
    \appendix
        \part{Appendices}
            \input{8 - Hilbert complexes/main.tex}
            \input{9 - weak conservation proofs/main.tex}
\end{document}

        \part{Project Overview}
            \documentclass[12pt, a4paper]{report}

\input{template/main.tex}

\title{\BA{Title in Progress...}}
\author{Boris Andrews}
\affil{Mathematical Institute, University of Oxford}
\date{\today}


\begin{document}
    \pagenumbering{gobble}
    \maketitle
    
    
    \begin{abstract}
        Magnetic confinement reactors---in particular tokamaks---offer one of the most promising options for achieving practical nuclear fusion, with the potential to provide virtually limitless, clean energy. The theoretical and numerical modeling of tokamak plasmas is simultaneously an essential component of effective reactor design, and a great research barrier. Tokamak operational conditions exhibit comparatively low Knudsen numbers. Kinetic effects, including kinetic waves and instabilities, Landau damping, bump-on-tail instabilities and more, are therefore highly influential in tokamak plasma dynamics. Purely fluid models are inherently incapable of capturing these effects, whereas the high dimensionality in purely kinetic models render them practically intractable for most relevant purposes.

        We consider a $\delta\!f$ decomposition model, with a macroscopic fluid background and microscopic kinetic correction, both fully coupled to each other. A similar manner of discretization is proposed to that used in the recent \texttt{STRUPHY} code \cite{Holderied_Possanner_Wang_2021, Holderied_2022, Li_et_al_2023} with a finite-element model for the background and a pseudo-particle/PiC model for the correction.

        The fluid background satisfies the full, non-linear, resistive, compressible, Hall MHD equations. \cite{Laakmann_Hu_Farrell_2022} introduces finite-element(-in-space) implicit timesteppers for the incompressible analogue to this system with structure-preserving (SP) properties in the ideal case, alongside parameter-robust preconditioners. We show that these timesteppers can derive from a finite-element-in-time (FET) (and finite-element-in-space) interpretation. The benefits of this reformulation are discussed, including the derivation of timesteppers that are higher order in time, and the quantifiable dissipative SP properties in the non-ideal, resistive case.
        
        We discuss possible options for extending this FET approach to timesteppers for the compressible case.

        The kinetic corrections satisfy linearized Boltzmann equations. Using a Lénard--Bernstein collision operator, these take Fokker--Planck-like forms \cite{Fokker_1914, Planck_1917} wherein pseudo-particles in the numerical model obey the neoclassical transport equations, with particle-independent Brownian drift terms. This offers a rigorous methodology for incorporating collisions into the particle transport model, without coupling the equations of motions for each particle.
        
        Works by Chen, Chacón et al. \cite{Chen_Chacón_Barnes_2011, Chacón_Chen_Barnes_2013, Chen_Chacón_2014, Chen_Chacón_2015} have developed structure-preserving particle pushers for neoclassical transport in the Vlasov equations, derived from Crank--Nicolson integrators. We show these too can can derive from a FET interpretation, similarly offering potential extensions to higher-order-in-time particle pushers. The FET formulation is used also to consider how the stochastic drift terms can be incorporated into the pushers. Stochastic gyrokinetic expansions are also discussed.

        Different options for the numerical implementation of these schemes are considered.

        Due to the efficacy of FET in the development of SP timesteppers for both the fluid and kinetic component, we hope this approach will prove effective in the future for developing SP timesteppers for the full hybrid model. We hope this will give us the opportunity to incorporate previously inaccessible kinetic effects into the highly effective, modern, finite-element MHD models.
    \end{abstract}
    
    
    \newpage
    \tableofcontents
    
    
    \newpage
    \pagenumbering{arabic}
    %\linenumbers\renewcommand\thelinenumber{\color{black!50}\arabic{linenumber}}
            \input{0 - introduction/main.tex}
        \part{Research}
            \input{1 - low-noise PiC models/main.tex}
            \input{2 - kinetic component/main.tex}
            \input{3 - fluid component/main.tex}
            \input{4 - numerical implementation/main.tex}
        \part{Project Overview}
            \input{5 - research plan/main.tex}
            \input{6 - summary/main.tex}
    
    
    %\section{}
    \newpage
    \pagenumbering{gobble}
        \printbibliography


    \newpage
    \pagenumbering{roman}
    \appendix
        \part{Appendices}
            \input{8 - Hilbert complexes/main.tex}
            \input{9 - weak conservation proofs/main.tex}
\end{document}

            \documentclass[12pt, a4paper]{report}

\input{template/main.tex}

\title{\BA{Title in Progress...}}
\author{Boris Andrews}
\affil{Mathematical Institute, University of Oxford}
\date{\today}


\begin{document}
    \pagenumbering{gobble}
    \maketitle
    
    
    \begin{abstract}
        Magnetic confinement reactors---in particular tokamaks---offer one of the most promising options for achieving practical nuclear fusion, with the potential to provide virtually limitless, clean energy. The theoretical and numerical modeling of tokamak plasmas is simultaneously an essential component of effective reactor design, and a great research barrier. Tokamak operational conditions exhibit comparatively low Knudsen numbers. Kinetic effects, including kinetic waves and instabilities, Landau damping, bump-on-tail instabilities and more, are therefore highly influential in tokamak plasma dynamics. Purely fluid models are inherently incapable of capturing these effects, whereas the high dimensionality in purely kinetic models render them practically intractable for most relevant purposes.

        We consider a $\delta\!f$ decomposition model, with a macroscopic fluid background and microscopic kinetic correction, both fully coupled to each other. A similar manner of discretization is proposed to that used in the recent \texttt{STRUPHY} code \cite{Holderied_Possanner_Wang_2021, Holderied_2022, Li_et_al_2023} with a finite-element model for the background and a pseudo-particle/PiC model for the correction.

        The fluid background satisfies the full, non-linear, resistive, compressible, Hall MHD equations. \cite{Laakmann_Hu_Farrell_2022} introduces finite-element(-in-space) implicit timesteppers for the incompressible analogue to this system with structure-preserving (SP) properties in the ideal case, alongside parameter-robust preconditioners. We show that these timesteppers can derive from a finite-element-in-time (FET) (and finite-element-in-space) interpretation. The benefits of this reformulation are discussed, including the derivation of timesteppers that are higher order in time, and the quantifiable dissipative SP properties in the non-ideal, resistive case.
        
        We discuss possible options for extending this FET approach to timesteppers for the compressible case.

        The kinetic corrections satisfy linearized Boltzmann equations. Using a Lénard--Bernstein collision operator, these take Fokker--Planck-like forms \cite{Fokker_1914, Planck_1917} wherein pseudo-particles in the numerical model obey the neoclassical transport equations, with particle-independent Brownian drift terms. This offers a rigorous methodology for incorporating collisions into the particle transport model, without coupling the equations of motions for each particle.
        
        Works by Chen, Chacón et al. \cite{Chen_Chacón_Barnes_2011, Chacón_Chen_Barnes_2013, Chen_Chacón_2014, Chen_Chacón_2015} have developed structure-preserving particle pushers for neoclassical transport in the Vlasov equations, derived from Crank--Nicolson integrators. We show these too can can derive from a FET interpretation, similarly offering potential extensions to higher-order-in-time particle pushers. The FET formulation is used also to consider how the stochastic drift terms can be incorporated into the pushers. Stochastic gyrokinetic expansions are also discussed.

        Different options for the numerical implementation of these schemes are considered.

        Due to the efficacy of FET in the development of SP timesteppers for both the fluid and kinetic component, we hope this approach will prove effective in the future for developing SP timesteppers for the full hybrid model. We hope this will give us the opportunity to incorporate previously inaccessible kinetic effects into the highly effective, modern, finite-element MHD models.
    \end{abstract}
    
    
    \newpage
    \tableofcontents
    
    
    \newpage
    \pagenumbering{arabic}
    %\linenumbers\renewcommand\thelinenumber{\color{black!50}\arabic{linenumber}}
            \input{0 - introduction/main.tex}
        \part{Research}
            \input{1 - low-noise PiC models/main.tex}
            \input{2 - kinetic component/main.tex}
            \input{3 - fluid component/main.tex}
            \input{4 - numerical implementation/main.tex}
        \part{Project Overview}
            \input{5 - research plan/main.tex}
            \input{6 - summary/main.tex}
    
    
    %\section{}
    \newpage
    \pagenumbering{gobble}
        \printbibliography


    \newpage
    \pagenumbering{roman}
    \appendix
        \part{Appendices}
            \input{8 - Hilbert complexes/main.tex}
            \input{9 - weak conservation proofs/main.tex}
\end{document}

    
    
    %\section{}
    \newpage
    \pagenumbering{gobble}
        \printbibliography


    \newpage
    \pagenumbering{roman}
    \appendix
        \part{Appendices}
            \documentclass[12pt, a4paper]{report}

\input{template/main.tex}

\title{\BA{Title in Progress...}}
\author{Boris Andrews}
\affil{Mathematical Institute, University of Oxford}
\date{\today}


\begin{document}
    \pagenumbering{gobble}
    \maketitle
    
    
    \begin{abstract}
        Magnetic confinement reactors---in particular tokamaks---offer one of the most promising options for achieving practical nuclear fusion, with the potential to provide virtually limitless, clean energy. The theoretical and numerical modeling of tokamak plasmas is simultaneously an essential component of effective reactor design, and a great research barrier. Tokamak operational conditions exhibit comparatively low Knudsen numbers. Kinetic effects, including kinetic waves and instabilities, Landau damping, bump-on-tail instabilities and more, are therefore highly influential in tokamak plasma dynamics. Purely fluid models are inherently incapable of capturing these effects, whereas the high dimensionality in purely kinetic models render them practically intractable for most relevant purposes.

        We consider a $\delta\!f$ decomposition model, with a macroscopic fluid background and microscopic kinetic correction, both fully coupled to each other. A similar manner of discretization is proposed to that used in the recent \texttt{STRUPHY} code \cite{Holderied_Possanner_Wang_2021, Holderied_2022, Li_et_al_2023} with a finite-element model for the background and a pseudo-particle/PiC model for the correction.

        The fluid background satisfies the full, non-linear, resistive, compressible, Hall MHD equations. \cite{Laakmann_Hu_Farrell_2022} introduces finite-element(-in-space) implicit timesteppers for the incompressible analogue to this system with structure-preserving (SP) properties in the ideal case, alongside parameter-robust preconditioners. We show that these timesteppers can derive from a finite-element-in-time (FET) (and finite-element-in-space) interpretation. The benefits of this reformulation are discussed, including the derivation of timesteppers that are higher order in time, and the quantifiable dissipative SP properties in the non-ideal, resistive case.
        
        We discuss possible options for extending this FET approach to timesteppers for the compressible case.

        The kinetic corrections satisfy linearized Boltzmann equations. Using a Lénard--Bernstein collision operator, these take Fokker--Planck-like forms \cite{Fokker_1914, Planck_1917} wherein pseudo-particles in the numerical model obey the neoclassical transport equations, with particle-independent Brownian drift terms. This offers a rigorous methodology for incorporating collisions into the particle transport model, without coupling the equations of motions for each particle.
        
        Works by Chen, Chacón et al. \cite{Chen_Chacón_Barnes_2011, Chacón_Chen_Barnes_2013, Chen_Chacón_2014, Chen_Chacón_2015} have developed structure-preserving particle pushers for neoclassical transport in the Vlasov equations, derived from Crank--Nicolson integrators. We show these too can can derive from a FET interpretation, similarly offering potential extensions to higher-order-in-time particle pushers. The FET formulation is used also to consider how the stochastic drift terms can be incorporated into the pushers. Stochastic gyrokinetic expansions are also discussed.

        Different options for the numerical implementation of these schemes are considered.

        Due to the efficacy of FET in the development of SP timesteppers for both the fluid and kinetic component, we hope this approach will prove effective in the future for developing SP timesteppers for the full hybrid model. We hope this will give us the opportunity to incorporate previously inaccessible kinetic effects into the highly effective, modern, finite-element MHD models.
    \end{abstract}
    
    
    \newpage
    \tableofcontents
    
    
    \newpage
    \pagenumbering{arabic}
    %\linenumbers\renewcommand\thelinenumber{\color{black!50}\arabic{linenumber}}
            \input{0 - introduction/main.tex}
        \part{Research}
            \input{1 - low-noise PiC models/main.tex}
            \input{2 - kinetic component/main.tex}
            \input{3 - fluid component/main.tex}
            \input{4 - numerical implementation/main.tex}
        \part{Project Overview}
            \input{5 - research plan/main.tex}
            \input{6 - summary/main.tex}
    
    
    %\section{}
    \newpage
    \pagenumbering{gobble}
        \printbibliography


    \newpage
    \pagenumbering{roman}
    \appendix
        \part{Appendices}
            \input{8 - Hilbert complexes/main.tex}
            \input{9 - weak conservation proofs/main.tex}
\end{document}

            \documentclass[12pt, a4paper]{report}

\input{template/main.tex}

\title{\BA{Title in Progress...}}
\author{Boris Andrews}
\affil{Mathematical Institute, University of Oxford}
\date{\today}


\begin{document}
    \pagenumbering{gobble}
    \maketitle
    
    
    \begin{abstract}
        Magnetic confinement reactors---in particular tokamaks---offer one of the most promising options for achieving practical nuclear fusion, with the potential to provide virtually limitless, clean energy. The theoretical and numerical modeling of tokamak plasmas is simultaneously an essential component of effective reactor design, and a great research barrier. Tokamak operational conditions exhibit comparatively low Knudsen numbers. Kinetic effects, including kinetic waves and instabilities, Landau damping, bump-on-tail instabilities and more, are therefore highly influential in tokamak plasma dynamics. Purely fluid models are inherently incapable of capturing these effects, whereas the high dimensionality in purely kinetic models render them practically intractable for most relevant purposes.

        We consider a $\delta\!f$ decomposition model, with a macroscopic fluid background and microscopic kinetic correction, both fully coupled to each other. A similar manner of discretization is proposed to that used in the recent \texttt{STRUPHY} code \cite{Holderied_Possanner_Wang_2021, Holderied_2022, Li_et_al_2023} with a finite-element model for the background and a pseudo-particle/PiC model for the correction.

        The fluid background satisfies the full, non-linear, resistive, compressible, Hall MHD equations. \cite{Laakmann_Hu_Farrell_2022} introduces finite-element(-in-space) implicit timesteppers for the incompressible analogue to this system with structure-preserving (SP) properties in the ideal case, alongside parameter-robust preconditioners. We show that these timesteppers can derive from a finite-element-in-time (FET) (and finite-element-in-space) interpretation. The benefits of this reformulation are discussed, including the derivation of timesteppers that are higher order in time, and the quantifiable dissipative SP properties in the non-ideal, resistive case.
        
        We discuss possible options for extending this FET approach to timesteppers for the compressible case.

        The kinetic corrections satisfy linearized Boltzmann equations. Using a Lénard--Bernstein collision operator, these take Fokker--Planck-like forms \cite{Fokker_1914, Planck_1917} wherein pseudo-particles in the numerical model obey the neoclassical transport equations, with particle-independent Brownian drift terms. This offers a rigorous methodology for incorporating collisions into the particle transport model, without coupling the equations of motions for each particle.
        
        Works by Chen, Chacón et al. \cite{Chen_Chacón_Barnes_2011, Chacón_Chen_Barnes_2013, Chen_Chacón_2014, Chen_Chacón_2015} have developed structure-preserving particle pushers for neoclassical transport in the Vlasov equations, derived from Crank--Nicolson integrators. We show these too can can derive from a FET interpretation, similarly offering potential extensions to higher-order-in-time particle pushers. The FET formulation is used also to consider how the stochastic drift terms can be incorporated into the pushers. Stochastic gyrokinetic expansions are also discussed.

        Different options for the numerical implementation of these schemes are considered.

        Due to the efficacy of FET in the development of SP timesteppers for both the fluid and kinetic component, we hope this approach will prove effective in the future for developing SP timesteppers for the full hybrid model. We hope this will give us the opportunity to incorporate previously inaccessible kinetic effects into the highly effective, modern, finite-element MHD models.
    \end{abstract}
    
    
    \newpage
    \tableofcontents
    
    
    \newpage
    \pagenumbering{arabic}
    %\linenumbers\renewcommand\thelinenumber{\color{black!50}\arabic{linenumber}}
            \input{0 - introduction/main.tex}
        \part{Research}
            \input{1 - low-noise PiC models/main.tex}
            \input{2 - kinetic component/main.tex}
            \input{3 - fluid component/main.tex}
            \input{4 - numerical implementation/main.tex}
        \part{Project Overview}
            \input{5 - research plan/main.tex}
            \input{6 - summary/main.tex}
    
    
    %\section{}
    \newpage
    \pagenumbering{gobble}
        \printbibliography


    \newpage
    \pagenumbering{roman}
    \appendix
        \part{Appendices}
            \input{8 - Hilbert complexes/main.tex}
            \input{9 - weak conservation proofs/main.tex}
\end{document}

\end{document}

\end{document}

    \documentclass[12pt, a4paper]{report}

\documentclass[12pt, a4paper]{report}

\documentclass[12pt, a4paper]{report}

\input{template/main.tex}

\title{\BA{Title in Progress...}}
\author{Boris Andrews}
\affil{Mathematical Institute, University of Oxford}
\date{\today}


\begin{document}
    \pagenumbering{gobble}
    \maketitle
    
    
    \begin{abstract}
        Magnetic confinement reactors---in particular tokamaks---offer one of the most promising options for achieving practical nuclear fusion, with the potential to provide virtually limitless, clean energy. The theoretical and numerical modeling of tokamak plasmas is simultaneously an essential component of effective reactor design, and a great research barrier. Tokamak operational conditions exhibit comparatively low Knudsen numbers. Kinetic effects, including kinetic waves and instabilities, Landau damping, bump-on-tail instabilities and more, are therefore highly influential in tokamak plasma dynamics. Purely fluid models are inherently incapable of capturing these effects, whereas the high dimensionality in purely kinetic models render them practically intractable for most relevant purposes.

        We consider a $\delta\!f$ decomposition model, with a macroscopic fluid background and microscopic kinetic correction, both fully coupled to each other. A similar manner of discretization is proposed to that used in the recent \texttt{STRUPHY} code \cite{Holderied_Possanner_Wang_2021, Holderied_2022, Li_et_al_2023} with a finite-element model for the background and a pseudo-particle/PiC model for the correction.

        The fluid background satisfies the full, non-linear, resistive, compressible, Hall MHD equations. \cite{Laakmann_Hu_Farrell_2022} introduces finite-element(-in-space) implicit timesteppers for the incompressible analogue to this system with structure-preserving (SP) properties in the ideal case, alongside parameter-robust preconditioners. We show that these timesteppers can derive from a finite-element-in-time (FET) (and finite-element-in-space) interpretation. The benefits of this reformulation are discussed, including the derivation of timesteppers that are higher order in time, and the quantifiable dissipative SP properties in the non-ideal, resistive case.
        
        We discuss possible options for extending this FET approach to timesteppers for the compressible case.

        The kinetic corrections satisfy linearized Boltzmann equations. Using a Lénard--Bernstein collision operator, these take Fokker--Planck-like forms \cite{Fokker_1914, Planck_1917} wherein pseudo-particles in the numerical model obey the neoclassical transport equations, with particle-independent Brownian drift terms. This offers a rigorous methodology for incorporating collisions into the particle transport model, without coupling the equations of motions for each particle.
        
        Works by Chen, Chacón et al. \cite{Chen_Chacón_Barnes_2011, Chacón_Chen_Barnes_2013, Chen_Chacón_2014, Chen_Chacón_2015} have developed structure-preserving particle pushers for neoclassical transport in the Vlasov equations, derived from Crank--Nicolson integrators. We show these too can can derive from a FET interpretation, similarly offering potential extensions to higher-order-in-time particle pushers. The FET formulation is used also to consider how the stochastic drift terms can be incorporated into the pushers. Stochastic gyrokinetic expansions are also discussed.

        Different options for the numerical implementation of these schemes are considered.

        Due to the efficacy of FET in the development of SP timesteppers for both the fluid and kinetic component, we hope this approach will prove effective in the future for developing SP timesteppers for the full hybrid model. We hope this will give us the opportunity to incorporate previously inaccessible kinetic effects into the highly effective, modern, finite-element MHD models.
    \end{abstract}
    
    
    \newpage
    \tableofcontents
    
    
    \newpage
    \pagenumbering{arabic}
    %\linenumbers\renewcommand\thelinenumber{\color{black!50}\arabic{linenumber}}
            \input{0 - introduction/main.tex}
        \part{Research}
            \input{1 - low-noise PiC models/main.tex}
            \input{2 - kinetic component/main.tex}
            \input{3 - fluid component/main.tex}
            \input{4 - numerical implementation/main.tex}
        \part{Project Overview}
            \input{5 - research plan/main.tex}
            \input{6 - summary/main.tex}
    
    
    %\section{}
    \newpage
    \pagenumbering{gobble}
        \printbibliography


    \newpage
    \pagenumbering{roman}
    \appendix
        \part{Appendices}
            \input{8 - Hilbert complexes/main.tex}
            \input{9 - weak conservation proofs/main.tex}
\end{document}


\title{\BA{Title in Progress...}}
\author{Boris Andrews}
\affil{Mathematical Institute, University of Oxford}
\date{\today}


\begin{document}
    \pagenumbering{gobble}
    \maketitle
    
    
    \begin{abstract}
        Magnetic confinement reactors---in particular tokamaks---offer one of the most promising options for achieving practical nuclear fusion, with the potential to provide virtually limitless, clean energy. The theoretical and numerical modeling of tokamak plasmas is simultaneously an essential component of effective reactor design, and a great research barrier. Tokamak operational conditions exhibit comparatively low Knudsen numbers. Kinetic effects, including kinetic waves and instabilities, Landau damping, bump-on-tail instabilities and more, are therefore highly influential in tokamak plasma dynamics. Purely fluid models are inherently incapable of capturing these effects, whereas the high dimensionality in purely kinetic models render them practically intractable for most relevant purposes.

        We consider a $\delta\!f$ decomposition model, with a macroscopic fluid background and microscopic kinetic correction, both fully coupled to each other. A similar manner of discretization is proposed to that used in the recent \texttt{STRUPHY} code \cite{Holderied_Possanner_Wang_2021, Holderied_2022, Li_et_al_2023} with a finite-element model for the background and a pseudo-particle/PiC model for the correction.

        The fluid background satisfies the full, non-linear, resistive, compressible, Hall MHD equations. \cite{Laakmann_Hu_Farrell_2022} introduces finite-element(-in-space) implicit timesteppers for the incompressible analogue to this system with structure-preserving (SP) properties in the ideal case, alongside parameter-robust preconditioners. We show that these timesteppers can derive from a finite-element-in-time (FET) (and finite-element-in-space) interpretation. The benefits of this reformulation are discussed, including the derivation of timesteppers that are higher order in time, and the quantifiable dissipative SP properties in the non-ideal, resistive case.
        
        We discuss possible options for extending this FET approach to timesteppers for the compressible case.

        The kinetic corrections satisfy linearized Boltzmann equations. Using a Lénard--Bernstein collision operator, these take Fokker--Planck-like forms \cite{Fokker_1914, Planck_1917} wherein pseudo-particles in the numerical model obey the neoclassical transport equations, with particle-independent Brownian drift terms. This offers a rigorous methodology for incorporating collisions into the particle transport model, without coupling the equations of motions for each particle.
        
        Works by Chen, Chacón et al. \cite{Chen_Chacón_Barnes_2011, Chacón_Chen_Barnes_2013, Chen_Chacón_2014, Chen_Chacón_2015} have developed structure-preserving particle pushers for neoclassical transport in the Vlasov equations, derived from Crank--Nicolson integrators. We show these too can can derive from a FET interpretation, similarly offering potential extensions to higher-order-in-time particle pushers. The FET formulation is used also to consider how the stochastic drift terms can be incorporated into the pushers. Stochastic gyrokinetic expansions are also discussed.

        Different options for the numerical implementation of these schemes are considered.

        Due to the efficacy of FET in the development of SP timesteppers for both the fluid and kinetic component, we hope this approach will prove effective in the future for developing SP timesteppers for the full hybrid model. We hope this will give us the opportunity to incorporate previously inaccessible kinetic effects into the highly effective, modern, finite-element MHD models.
    \end{abstract}
    
    
    \newpage
    \tableofcontents
    
    
    \newpage
    \pagenumbering{arabic}
    %\linenumbers\renewcommand\thelinenumber{\color{black!50}\arabic{linenumber}}
            \documentclass[12pt, a4paper]{report}

\input{template/main.tex}

\title{\BA{Title in Progress...}}
\author{Boris Andrews}
\affil{Mathematical Institute, University of Oxford}
\date{\today}


\begin{document}
    \pagenumbering{gobble}
    \maketitle
    
    
    \begin{abstract}
        Magnetic confinement reactors---in particular tokamaks---offer one of the most promising options for achieving practical nuclear fusion, with the potential to provide virtually limitless, clean energy. The theoretical and numerical modeling of tokamak plasmas is simultaneously an essential component of effective reactor design, and a great research barrier. Tokamak operational conditions exhibit comparatively low Knudsen numbers. Kinetic effects, including kinetic waves and instabilities, Landau damping, bump-on-tail instabilities and more, are therefore highly influential in tokamak plasma dynamics. Purely fluid models are inherently incapable of capturing these effects, whereas the high dimensionality in purely kinetic models render them practically intractable for most relevant purposes.

        We consider a $\delta\!f$ decomposition model, with a macroscopic fluid background and microscopic kinetic correction, both fully coupled to each other. A similar manner of discretization is proposed to that used in the recent \texttt{STRUPHY} code \cite{Holderied_Possanner_Wang_2021, Holderied_2022, Li_et_al_2023} with a finite-element model for the background and a pseudo-particle/PiC model for the correction.

        The fluid background satisfies the full, non-linear, resistive, compressible, Hall MHD equations. \cite{Laakmann_Hu_Farrell_2022} introduces finite-element(-in-space) implicit timesteppers for the incompressible analogue to this system with structure-preserving (SP) properties in the ideal case, alongside parameter-robust preconditioners. We show that these timesteppers can derive from a finite-element-in-time (FET) (and finite-element-in-space) interpretation. The benefits of this reformulation are discussed, including the derivation of timesteppers that are higher order in time, and the quantifiable dissipative SP properties in the non-ideal, resistive case.
        
        We discuss possible options for extending this FET approach to timesteppers for the compressible case.

        The kinetic corrections satisfy linearized Boltzmann equations. Using a Lénard--Bernstein collision operator, these take Fokker--Planck-like forms \cite{Fokker_1914, Planck_1917} wherein pseudo-particles in the numerical model obey the neoclassical transport equations, with particle-independent Brownian drift terms. This offers a rigorous methodology for incorporating collisions into the particle transport model, without coupling the equations of motions for each particle.
        
        Works by Chen, Chacón et al. \cite{Chen_Chacón_Barnes_2011, Chacón_Chen_Barnes_2013, Chen_Chacón_2014, Chen_Chacón_2015} have developed structure-preserving particle pushers for neoclassical transport in the Vlasov equations, derived from Crank--Nicolson integrators. We show these too can can derive from a FET interpretation, similarly offering potential extensions to higher-order-in-time particle pushers. The FET formulation is used also to consider how the stochastic drift terms can be incorporated into the pushers. Stochastic gyrokinetic expansions are also discussed.

        Different options for the numerical implementation of these schemes are considered.

        Due to the efficacy of FET in the development of SP timesteppers for both the fluid and kinetic component, we hope this approach will prove effective in the future for developing SP timesteppers for the full hybrid model. We hope this will give us the opportunity to incorporate previously inaccessible kinetic effects into the highly effective, modern, finite-element MHD models.
    \end{abstract}
    
    
    \newpage
    \tableofcontents
    
    
    \newpage
    \pagenumbering{arabic}
    %\linenumbers\renewcommand\thelinenumber{\color{black!50}\arabic{linenumber}}
            \input{0 - introduction/main.tex}
        \part{Research}
            \input{1 - low-noise PiC models/main.tex}
            \input{2 - kinetic component/main.tex}
            \input{3 - fluid component/main.tex}
            \input{4 - numerical implementation/main.tex}
        \part{Project Overview}
            \input{5 - research plan/main.tex}
            \input{6 - summary/main.tex}
    
    
    %\section{}
    \newpage
    \pagenumbering{gobble}
        \printbibliography


    \newpage
    \pagenumbering{roman}
    \appendix
        \part{Appendices}
            \input{8 - Hilbert complexes/main.tex}
            \input{9 - weak conservation proofs/main.tex}
\end{document}

        \part{Research}
            \documentclass[12pt, a4paper]{report}

\input{template/main.tex}

\title{\BA{Title in Progress...}}
\author{Boris Andrews}
\affil{Mathematical Institute, University of Oxford}
\date{\today}


\begin{document}
    \pagenumbering{gobble}
    \maketitle
    
    
    \begin{abstract}
        Magnetic confinement reactors---in particular tokamaks---offer one of the most promising options for achieving practical nuclear fusion, with the potential to provide virtually limitless, clean energy. The theoretical and numerical modeling of tokamak plasmas is simultaneously an essential component of effective reactor design, and a great research barrier. Tokamak operational conditions exhibit comparatively low Knudsen numbers. Kinetic effects, including kinetic waves and instabilities, Landau damping, bump-on-tail instabilities and more, are therefore highly influential in tokamak plasma dynamics. Purely fluid models are inherently incapable of capturing these effects, whereas the high dimensionality in purely kinetic models render them practically intractable for most relevant purposes.

        We consider a $\delta\!f$ decomposition model, with a macroscopic fluid background and microscopic kinetic correction, both fully coupled to each other. A similar manner of discretization is proposed to that used in the recent \texttt{STRUPHY} code \cite{Holderied_Possanner_Wang_2021, Holderied_2022, Li_et_al_2023} with a finite-element model for the background and a pseudo-particle/PiC model for the correction.

        The fluid background satisfies the full, non-linear, resistive, compressible, Hall MHD equations. \cite{Laakmann_Hu_Farrell_2022} introduces finite-element(-in-space) implicit timesteppers for the incompressible analogue to this system with structure-preserving (SP) properties in the ideal case, alongside parameter-robust preconditioners. We show that these timesteppers can derive from a finite-element-in-time (FET) (and finite-element-in-space) interpretation. The benefits of this reformulation are discussed, including the derivation of timesteppers that are higher order in time, and the quantifiable dissipative SP properties in the non-ideal, resistive case.
        
        We discuss possible options for extending this FET approach to timesteppers for the compressible case.

        The kinetic corrections satisfy linearized Boltzmann equations. Using a Lénard--Bernstein collision operator, these take Fokker--Planck-like forms \cite{Fokker_1914, Planck_1917} wherein pseudo-particles in the numerical model obey the neoclassical transport equations, with particle-independent Brownian drift terms. This offers a rigorous methodology for incorporating collisions into the particle transport model, without coupling the equations of motions for each particle.
        
        Works by Chen, Chacón et al. \cite{Chen_Chacón_Barnes_2011, Chacón_Chen_Barnes_2013, Chen_Chacón_2014, Chen_Chacón_2015} have developed structure-preserving particle pushers for neoclassical transport in the Vlasov equations, derived from Crank--Nicolson integrators. We show these too can can derive from a FET interpretation, similarly offering potential extensions to higher-order-in-time particle pushers. The FET formulation is used also to consider how the stochastic drift terms can be incorporated into the pushers. Stochastic gyrokinetic expansions are also discussed.

        Different options for the numerical implementation of these schemes are considered.

        Due to the efficacy of FET in the development of SP timesteppers for both the fluid and kinetic component, we hope this approach will prove effective in the future for developing SP timesteppers for the full hybrid model. We hope this will give us the opportunity to incorporate previously inaccessible kinetic effects into the highly effective, modern, finite-element MHD models.
    \end{abstract}
    
    
    \newpage
    \tableofcontents
    
    
    \newpage
    \pagenumbering{arabic}
    %\linenumbers\renewcommand\thelinenumber{\color{black!50}\arabic{linenumber}}
            \input{0 - introduction/main.tex}
        \part{Research}
            \input{1 - low-noise PiC models/main.tex}
            \input{2 - kinetic component/main.tex}
            \input{3 - fluid component/main.tex}
            \input{4 - numerical implementation/main.tex}
        \part{Project Overview}
            \input{5 - research plan/main.tex}
            \input{6 - summary/main.tex}
    
    
    %\section{}
    \newpage
    \pagenumbering{gobble}
        \printbibliography


    \newpage
    \pagenumbering{roman}
    \appendix
        \part{Appendices}
            \input{8 - Hilbert complexes/main.tex}
            \input{9 - weak conservation proofs/main.tex}
\end{document}

            \documentclass[12pt, a4paper]{report}

\input{template/main.tex}

\title{\BA{Title in Progress...}}
\author{Boris Andrews}
\affil{Mathematical Institute, University of Oxford}
\date{\today}


\begin{document}
    \pagenumbering{gobble}
    \maketitle
    
    
    \begin{abstract}
        Magnetic confinement reactors---in particular tokamaks---offer one of the most promising options for achieving practical nuclear fusion, with the potential to provide virtually limitless, clean energy. The theoretical and numerical modeling of tokamak plasmas is simultaneously an essential component of effective reactor design, and a great research barrier. Tokamak operational conditions exhibit comparatively low Knudsen numbers. Kinetic effects, including kinetic waves and instabilities, Landau damping, bump-on-tail instabilities and more, are therefore highly influential in tokamak plasma dynamics. Purely fluid models are inherently incapable of capturing these effects, whereas the high dimensionality in purely kinetic models render them practically intractable for most relevant purposes.

        We consider a $\delta\!f$ decomposition model, with a macroscopic fluid background and microscopic kinetic correction, both fully coupled to each other. A similar manner of discretization is proposed to that used in the recent \texttt{STRUPHY} code \cite{Holderied_Possanner_Wang_2021, Holderied_2022, Li_et_al_2023} with a finite-element model for the background and a pseudo-particle/PiC model for the correction.

        The fluid background satisfies the full, non-linear, resistive, compressible, Hall MHD equations. \cite{Laakmann_Hu_Farrell_2022} introduces finite-element(-in-space) implicit timesteppers for the incompressible analogue to this system with structure-preserving (SP) properties in the ideal case, alongside parameter-robust preconditioners. We show that these timesteppers can derive from a finite-element-in-time (FET) (and finite-element-in-space) interpretation. The benefits of this reformulation are discussed, including the derivation of timesteppers that are higher order in time, and the quantifiable dissipative SP properties in the non-ideal, resistive case.
        
        We discuss possible options for extending this FET approach to timesteppers for the compressible case.

        The kinetic corrections satisfy linearized Boltzmann equations. Using a Lénard--Bernstein collision operator, these take Fokker--Planck-like forms \cite{Fokker_1914, Planck_1917} wherein pseudo-particles in the numerical model obey the neoclassical transport equations, with particle-independent Brownian drift terms. This offers a rigorous methodology for incorporating collisions into the particle transport model, without coupling the equations of motions for each particle.
        
        Works by Chen, Chacón et al. \cite{Chen_Chacón_Barnes_2011, Chacón_Chen_Barnes_2013, Chen_Chacón_2014, Chen_Chacón_2015} have developed structure-preserving particle pushers for neoclassical transport in the Vlasov equations, derived from Crank--Nicolson integrators. We show these too can can derive from a FET interpretation, similarly offering potential extensions to higher-order-in-time particle pushers. The FET formulation is used also to consider how the stochastic drift terms can be incorporated into the pushers. Stochastic gyrokinetic expansions are also discussed.

        Different options for the numerical implementation of these schemes are considered.

        Due to the efficacy of FET in the development of SP timesteppers for both the fluid and kinetic component, we hope this approach will prove effective in the future for developing SP timesteppers for the full hybrid model. We hope this will give us the opportunity to incorporate previously inaccessible kinetic effects into the highly effective, modern, finite-element MHD models.
    \end{abstract}
    
    
    \newpage
    \tableofcontents
    
    
    \newpage
    \pagenumbering{arabic}
    %\linenumbers\renewcommand\thelinenumber{\color{black!50}\arabic{linenumber}}
            \input{0 - introduction/main.tex}
        \part{Research}
            \input{1 - low-noise PiC models/main.tex}
            \input{2 - kinetic component/main.tex}
            \input{3 - fluid component/main.tex}
            \input{4 - numerical implementation/main.tex}
        \part{Project Overview}
            \input{5 - research plan/main.tex}
            \input{6 - summary/main.tex}
    
    
    %\section{}
    \newpage
    \pagenumbering{gobble}
        \printbibliography


    \newpage
    \pagenumbering{roman}
    \appendix
        \part{Appendices}
            \input{8 - Hilbert complexes/main.tex}
            \input{9 - weak conservation proofs/main.tex}
\end{document}

            \documentclass[12pt, a4paper]{report}

\input{template/main.tex}

\title{\BA{Title in Progress...}}
\author{Boris Andrews}
\affil{Mathematical Institute, University of Oxford}
\date{\today}


\begin{document}
    \pagenumbering{gobble}
    \maketitle
    
    
    \begin{abstract}
        Magnetic confinement reactors---in particular tokamaks---offer one of the most promising options for achieving practical nuclear fusion, with the potential to provide virtually limitless, clean energy. The theoretical and numerical modeling of tokamak plasmas is simultaneously an essential component of effective reactor design, and a great research barrier. Tokamak operational conditions exhibit comparatively low Knudsen numbers. Kinetic effects, including kinetic waves and instabilities, Landau damping, bump-on-tail instabilities and more, are therefore highly influential in tokamak plasma dynamics. Purely fluid models are inherently incapable of capturing these effects, whereas the high dimensionality in purely kinetic models render them practically intractable for most relevant purposes.

        We consider a $\delta\!f$ decomposition model, with a macroscopic fluid background and microscopic kinetic correction, both fully coupled to each other. A similar manner of discretization is proposed to that used in the recent \texttt{STRUPHY} code \cite{Holderied_Possanner_Wang_2021, Holderied_2022, Li_et_al_2023} with a finite-element model for the background and a pseudo-particle/PiC model for the correction.

        The fluid background satisfies the full, non-linear, resistive, compressible, Hall MHD equations. \cite{Laakmann_Hu_Farrell_2022} introduces finite-element(-in-space) implicit timesteppers for the incompressible analogue to this system with structure-preserving (SP) properties in the ideal case, alongside parameter-robust preconditioners. We show that these timesteppers can derive from a finite-element-in-time (FET) (and finite-element-in-space) interpretation. The benefits of this reformulation are discussed, including the derivation of timesteppers that are higher order in time, and the quantifiable dissipative SP properties in the non-ideal, resistive case.
        
        We discuss possible options for extending this FET approach to timesteppers for the compressible case.

        The kinetic corrections satisfy linearized Boltzmann equations. Using a Lénard--Bernstein collision operator, these take Fokker--Planck-like forms \cite{Fokker_1914, Planck_1917} wherein pseudo-particles in the numerical model obey the neoclassical transport equations, with particle-independent Brownian drift terms. This offers a rigorous methodology for incorporating collisions into the particle transport model, without coupling the equations of motions for each particle.
        
        Works by Chen, Chacón et al. \cite{Chen_Chacón_Barnes_2011, Chacón_Chen_Barnes_2013, Chen_Chacón_2014, Chen_Chacón_2015} have developed structure-preserving particle pushers for neoclassical transport in the Vlasov equations, derived from Crank--Nicolson integrators. We show these too can can derive from a FET interpretation, similarly offering potential extensions to higher-order-in-time particle pushers. The FET formulation is used also to consider how the stochastic drift terms can be incorporated into the pushers. Stochastic gyrokinetic expansions are also discussed.

        Different options for the numerical implementation of these schemes are considered.

        Due to the efficacy of FET in the development of SP timesteppers for both the fluid and kinetic component, we hope this approach will prove effective in the future for developing SP timesteppers for the full hybrid model. We hope this will give us the opportunity to incorporate previously inaccessible kinetic effects into the highly effective, modern, finite-element MHD models.
    \end{abstract}
    
    
    \newpage
    \tableofcontents
    
    
    \newpage
    \pagenumbering{arabic}
    %\linenumbers\renewcommand\thelinenumber{\color{black!50}\arabic{linenumber}}
            \input{0 - introduction/main.tex}
        \part{Research}
            \input{1 - low-noise PiC models/main.tex}
            \input{2 - kinetic component/main.tex}
            \input{3 - fluid component/main.tex}
            \input{4 - numerical implementation/main.tex}
        \part{Project Overview}
            \input{5 - research plan/main.tex}
            \input{6 - summary/main.tex}
    
    
    %\section{}
    \newpage
    \pagenumbering{gobble}
        \printbibliography


    \newpage
    \pagenumbering{roman}
    \appendix
        \part{Appendices}
            \input{8 - Hilbert complexes/main.tex}
            \input{9 - weak conservation proofs/main.tex}
\end{document}

            \documentclass[12pt, a4paper]{report}

\input{template/main.tex}

\title{\BA{Title in Progress...}}
\author{Boris Andrews}
\affil{Mathematical Institute, University of Oxford}
\date{\today}


\begin{document}
    \pagenumbering{gobble}
    \maketitle
    
    
    \begin{abstract}
        Magnetic confinement reactors---in particular tokamaks---offer one of the most promising options for achieving practical nuclear fusion, with the potential to provide virtually limitless, clean energy. The theoretical and numerical modeling of tokamak plasmas is simultaneously an essential component of effective reactor design, and a great research barrier. Tokamak operational conditions exhibit comparatively low Knudsen numbers. Kinetic effects, including kinetic waves and instabilities, Landau damping, bump-on-tail instabilities and more, are therefore highly influential in tokamak plasma dynamics. Purely fluid models are inherently incapable of capturing these effects, whereas the high dimensionality in purely kinetic models render them practically intractable for most relevant purposes.

        We consider a $\delta\!f$ decomposition model, with a macroscopic fluid background and microscopic kinetic correction, both fully coupled to each other. A similar manner of discretization is proposed to that used in the recent \texttt{STRUPHY} code \cite{Holderied_Possanner_Wang_2021, Holderied_2022, Li_et_al_2023} with a finite-element model for the background and a pseudo-particle/PiC model for the correction.

        The fluid background satisfies the full, non-linear, resistive, compressible, Hall MHD equations. \cite{Laakmann_Hu_Farrell_2022} introduces finite-element(-in-space) implicit timesteppers for the incompressible analogue to this system with structure-preserving (SP) properties in the ideal case, alongside parameter-robust preconditioners. We show that these timesteppers can derive from a finite-element-in-time (FET) (and finite-element-in-space) interpretation. The benefits of this reformulation are discussed, including the derivation of timesteppers that are higher order in time, and the quantifiable dissipative SP properties in the non-ideal, resistive case.
        
        We discuss possible options for extending this FET approach to timesteppers for the compressible case.

        The kinetic corrections satisfy linearized Boltzmann equations. Using a Lénard--Bernstein collision operator, these take Fokker--Planck-like forms \cite{Fokker_1914, Planck_1917} wherein pseudo-particles in the numerical model obey the neoclassical transport equations, with particle-independent Brownian drift terms. This offers a rigorous methodology for incorporating collisions into the particle transport model, without coupling the equations of motions for each particle.
        
        Works by Chen, Chacón et al. \cite{Chen_Chacón_Barnes_2011, Chacón_Chen_Barnes_2013, Chen_Chacón_2014, Chen_Chacón_2015} have developed structure-preserving particle pushers for neoclassical transport in the Vlasov equations, derived from Crank--Nicolson integrators. We show these too can can derive from a FET interpretation, similarly offering potential extensions to higher-order-in-time particle pushers. The FET formulation is used also to consider how the stochastic drift terms can be incorporated into the pushers. Stochastic gyrokinetic expansions are also discussed.

        Different options for the numerical implementation of these schemes are considered.

        Due to the efficacy of FET in the development of SP timesteppers for both the fluid and kinetic component, we hope this approach will prove effective in the future for developing SP timesteppers for the full hybrid model. We hope this will give us the opportunity to incorporate previously inaccessible kinetic effects into the highly effective, modern, finite-element MHD models.
    \end{abstract}
    
    
    \newpage
    \tableofcontents
    
    
    \newpage
    \pagenumbering{arabic}
    %\linenumbers\renewcommand\thelinenumber{\color{black!50}\arabic{linenumber}}
            \input{0 - introduction/main.tex}
        \part{Research}
            \input{1 - low-noise PiC models/main.tex}
            \input{2 - kinetic component/main.tex}
            \input{3 - fluid component/main.tex}
            \input{4 - numerical implementation/main.tex}
        \part{Project Overview}
            \input{5 - research plan/main.tex}
            \input{6 - summary/main.tex}
    
    
    %\section{}
    \newpage
    \pagenumbering{gobble}
        \printbibliography


    \newpage
    \pagenumbering{roman}
    \appendix
        \part{Appendices}
            \input{8 - Hilbert complexes/main.tex}
            \input{9 - weak conservation proofs/main.tex}
\end{document}

        \part{Project Overview}
            \documentclass[12pt, a4paper]{report}

\input{template/main.tex}

\title{\BA{Title in Progress...}}
\author{Boris Andrews}
\affil{Mathematical Institute, University of Oxford}
\date{\today}


\begin{document}
    \pagenumbering{gobble}
    \maketitle
    
    
    \begin{abstract}
        Magnetic confinement reactors---in particular tokamaks---offer one of the most promising options for achieving practical nuclear fusion, with the potential to provide virtually limitless, clean energy. The theoretical and numerical modeling of tokamak plasmas is simultaneously an essential component of effective reactor design, and a great research barrier. Tokamak operational conditions exhibit comparatively low Knudsen numbers. Kinetic effects, including kinetic waves and instabilities, Landau damping, bump-on-tail instabilities and more, are therefore highly influential in tokamak plasma dynamics. Purely fluid models are inherently incapable of capturing these effects, whereas the high dimensionality in purely kinetic models render them practically intractable for most relevant purposes.

        We consider a $\delta\!f$ decomposition model, with a macroscopic fluid background and microscopic kinetic correction, both fully coupled to each other. A similar manner of discretization is proposed to that used in the recent \texttt{STRUPHY} code \cite{Holderied_Possanner_Wang_2021, Holderied_2022, Li_et_al_2023} with a finite-element model for the background and a pseudo-particle/PiC model for the correction.

        The fluid background satisfies the full, non-linear, resistive, compressible, Hall MHD equations. \cite{Laakmann_Hu_Farrell_2022} introduces finite-element(-in-space) implicit timesteppers for the incompressible analogue to this system with structure-preserving (SP) properties in the ideal case, alongside parameter-robust preconditioners. We show that these timesteppers can derive from a finite-element-in-time (FET) (and finite-element-in-space) interpretation. The benefits of this reformulation are discussed, including the derivation of timesteppers that are higher order in time, and the quantifiable dissipative SP properties in the non-ideal, resistive case.
        
        We discuss possible options for extending this FET approach to timesteppers for the compressible case.

        The kinetic corrections satisfy linearized Boltzmann equations. Using a Lénard--Bernstein collision operator, these take Fokker--Planck-like forms \cite{Fokker_1914, Planck_1917} wherein pseudo-particles in the numerical model obey the neoclassical transport equations, with particle-independent Brownian drift terms. This offers a rigorous methodology for incorporating collisions into the particle transport model, without coupling the equations of motions for each particle.
        
        Works by Chen, Chacón et al. \cite{Chen_Chacón_Barnes_2011, Chacón_Chen_Barnes_2013, Chen_Chacón_2014, Chen_Chacón_2015} have developed structure-preserving particle pushers for neoclassical transport in the Vlasov equations, derived from Crank--Nicolson integrators. We show these too can can derive from a FET interpretation, similarly offering potential extensions to higher-order-in-time particle pushers. The FET formulation is used also to consider how the stochastic drift terms can be incorporated into the pushers. Stochastic gyrokinetic expansions are also discussed.

        Different options for the numerical implementation of these schemes are considered.

        Due to the efficacy of FET in the development of SP timesteppers for both the fluid and kinetic component, we hope this approach will prove effective in the future for developing SP timesteppers for the full hybrid model. We hope this will give us the opportunity to incorporate previously inaccessible kinetic effects into the highly effective, modern, finite-element MHD models.
    \end{abstract}
    
    
    \newpage
    \tableofcontents
    
    
    \newpage
    \pagenumbering{arabic}
    %\linenumbers\renewcommand\thelinenumber{\color{black!50}\arabic{linenumber}}
            \input{0 - introduction/main.tex}
        \part{Research}
            \input{1 - low-noise PiC models/main.tex}
            \input{2 - kinetic component/main.tex}
            \input{3 - fluid component/main.tex}
            \input{4 - numerical implementation/main.tex}
        \part{Project Overview}
            \input{5 - research plan/main.tex}
            \input{6 - summary/main.tex}
    
    
    %\section{}
    \newpage
    \pagenumbering{gobble}
        \printbibliography


    \newpage
    \pagenumbering{roman}
    \appendix
        \part{Appendices}
            \input{8 - Hilbert complexes/main.tex}
            \input{9 - weak conservation proofs/main.tex}
\end{document}

            \documentclass[12pt, a4paper]{report}

\input{template/main.tex}

\title{\BA{Title in Progress...}}
\author{Boris Andrews}
\affil{Mathematical Institute, University of Oxford}
\date{\today}


\begin{document}
    \pagenumbering{gobble}
    \maketitle
    
    
    \begin{abstract}
        Magnetic confinement reactors---in particular tokamaks---offer one of the most promising options for achieving practical nuclear fusion, with the potential to provide virtually limitless, clean energy. The theoretical and numerical modeling of tokamak plasmas is simultaneously an essential component of effective reactor design, and a great research barrier. Tokamak operational conditions exhibit comparatively low Knudsen numbers. Kinetic effects, including kinetic waves and instabilities, Landau damping, bump-on-tail instabilities and more, are therefore highly influential in tokamak plasma dynamics. Purely fluid models are inherently incapable of capturing these effects, whereas the high dimensionality in purely kinetic models render them practically intractable for most relevant purposes.

        We consider a $\delta\!f$ decomposition model, with a macroscopic fluid background and microscopic kinetic correction, both fully coupled to each other. A similar manner of discretization is proposed to that used in the recent \texttt{STRUPHY} code \cite{Holderied_Possanner_Wang_2021, Holderied_2022, Li_et_al_2023} with a finite-element model for the background and a pseudo-particle/PiC model for the correction.

        The fluid background satisfies the full, non-linear, resistive, compressible, Hall MHD equations. \cite{Laakmann_Hu_Farrell_2022} introduces finite-element(-in-space) implicit timesteppers for the incompressible analogue to this system with structure-preserving (SP) properties in the ideal case, alongside parameter-robust preconditioners. We show that these timesteppers can derive from a finite-element-in-time (FET) (and finite-element-in-space) interpretation. The benefits of this reformulation are discussed, including the derivation of timesteppers that are higher order in time, and the quantifiable dissipative SP properties in the non-ideal, resistive case.
        
        We discuss possible options for extending this FET approach to timesteppers for the compressible case.

        The kinetic corrections satisfy linearized Boltzmann equations. Using a Lénard--Bernstein collision operator, these take Fokker--Planck-like forms \cite{Fokker_1914, Planck_1917} wherein pseudo-particles in the numerical model obey the neoclassical transport equations, with particle-independent Brownian drift terms. This offers a rigorous methodology for incorporating collisions into the particle transport model, without coupling the equations of motions for each particle.
        
        Works by Chen, Chacón et al. \cite{Chen_Chacón_Barnes_2011, Chacón_Chen_Barnes_2013, Chen_Chacón_2014, Chen_Chacón_2015} have developed structure-preserving particle pushers for neoclassical transport in the Vlasov equations, derived from Crank--Nicolson integrators. We show these too can can derive from a FET interpretation, similarly offering potential extensions to higher-order-in-time particle pushers. The FET formulation is used also to consider how the stochastic drift terms can be incorporated into the pushers. Stochastic gyrokinetic expansions are also discussed.

        Different options for the numerical implementation of these schemes are considered.

        Due to the efficacy of FET in the development of SP timesteppers for both the fluid and kinetic component, we hope this approach will prove effective in the future for developing SP timesteppers for the full hybrid model. We hope this will give us the opportunity to incorporate previously inaccessible kinetic effects into the highly effective, modern, finite-element MHD models.
    \end{abstract}
    
    
    \newpage
    \tableofcontents
    
    
    \newpage
    \pagenumbering{arabic}
    %\linenumbers\renewcommand\thelinenumber{\color{black!50}\arabic{linenumber}}
            \input{0 - introduction/main.tex}
        \part{Research}
            \input{1 - low-noise PiC models/main.tex}
            \input{2 - kinetic component/main.tex}
            \input{3 - fluid component/main.tex}
            \input{4 - numerical implementation/main.tex}
        \part{Project Overview}
            \input{5 - research plan/main.tex}
            \input{6 - summary/main.tex}
    
    
    %\section{}
    \newpage
    \pagenumbering{gobble}
        \printbibliography


    \newpage
    \pagenumbering{roman}
    \appendix
        \part{Appendices}
            \input{8 - Hilbert complexes/main.tex}
            \input{9 - weak conservation proofs/main.tex}
\end{document}

    
    
    %\section{}
    \newpage
    \pagenumbering{gobble}
        \printbibliography


    \newpage
    \pagenumbering{roman}
    \appendix
        \part{Appendices}
            \documentclass[12pt, a4paper]{report}

\input{template/main.tex}

\title{\BA{Title in Progress...}}
\author{Boris Andrews}
\affil{Mathematical Institute, University of Oxford}
\date{\today}


\begin{document}
    \pagenumbering{gobble}
    \maketitle
    
    
    \begin{abstract}
        Magnetic confinement reactors---in particular tokamaks---offer one of the most promising options for achieving practical nuclear fusion, with the potential to provide virtually limitless, clean energy. The theoretical and numerical modeling of tokamak plasmas is simultaneously an essential component of effective reactor design, and a great research barrier. Tokamak operational conditions exhibit comparatively low Knudsen numbers. Kinetic effects, including kinetic waves and instabilities, Landau damping, bump-on-tail instabilities and more, are therefore highly influential in tokamak plasma dynamics. Purely fluid models are inherently incapable of capturing these effects, whereas the high dimensionality in purely kinetic models render them practically intractable for most relevant purposes.

        We consider a $\delta\!f$ decomposition model, with a macroscopic fluid background and microscopic kinetic correction, both fully coupled to each other. A similar manner of discretization is proposed to that used in the recent \texttt{STRUPHY} code \cite{Holderied_Possanner_Wang_2021, Holderied_2022, Li_et_al_2023} with a finite-element model for the background and a pseudo-particle/PiC model for the correction.

        The fluid background satisfies the full, non-linear, resistive, compressible, Hall MHD equations. \cite{Laakmann_Hu_Farrell_2022} introduces finite-element(-in-space) implicit timesteppers for the incompressible analogue to this system with structure-preserving (SP) properties in the ideal case, alongside parameter-robust preconditioners. We show that these timesteppers can derive from a finite-element-in-time (FET) (and finite-element-in-space) interpretation. The benefits of this reformulation are discussed, including the derivation of timesteppers that are higher order in time, and the quantifiable dissipative SP properties in the non-ideal, resistive case.
        
        We discuss possible options for extending this FET approach to timesteppers for the compressible case.

        The kinetic corrections satisfy linearized Boltzmann equations. Using a Lénard--Bernstein collision operator, these take Fokker--Planck-like forms \cite{Fokker_1914, Planck_1917} wherein pseudo-particles in the numerical model obey the neoclassical transport equations, with particle-independent Brownian drift terms. This offers a rigorous methodology for incorporating collisions into the particle transport model, without coupling the equations of motions for each particle.
        
        Works by Chen, Chacón et al. \cite{Chen_Chacón_Barnes_2011, Chacón_Chen_Barnes_2013, Chen_Chacón_2014, Chen_Chacón_2015} have developed structure-preserving particle pushers for neoclassical transport in the Vlasov equations, derived from Crank--Nicolson integrators. We show these too can can derive from a FET interpretation, similarly offering potential extensions to higher-order-in-time particle pushers. The FET formulation is used also to consider how the stochastic drift terms can be incorporated into the pushers. Stochastic gyrokinetic expansions are also discussed.

        Different options for the numerical implementation of these schemes are considered.

        Due to the efficacy of FET in the development of SP timesteppers for both the fluid and kinetic component, we hope this approach will prove effective in the future for developing SP timesteppers for the full hybrid model. We hope this will give us the opportunity to incorporate previously inaccessible kinetic effects into the highly effective, modern, finite-element MHD models.
    \end{abstract}
    
    
    \newpage
    \tableofcontents
    
    
    \newpage
    \pagenumbering{arabic}
    %\linenumbers\renewcommand\thelinenumber{\color{black!50}\arabic{linenumber}}
            \input{0 - introduction/main.tex}
        \part{Research}
            \input{1 - low-noise PiC models/main.tex}
            \input{2 - kinetic component/main.tex}
            \input{3 - fluid component/main.tex}
            \input{4 - numerical implementation/main.tex}
        \part{Project Overview}
            \input{5 - research plan/main.tex}
            \input{6 - summary/main.tex}
    
    
    %\section{}
    \newpage
    \pagenumbering{gobble}
        \printbibliography


    \newpage
    \pagenumbering{roman}
    \appendix
        \part{Appendices}
            \input{8 - Hilbert complexes/main.tex}
            \input{9 - weak conservation proofs/main.tex}
\end{document}

            \documentclass[12pt, a4paper]{report}

\input{template/main.tex}

\title{\BA{Title in Progress...}}
\author{Boris Andrews}
\affil{Mathematical Institute, University of Oxford}
\date{\today}


\begin{document}
    \pagenumbering{gobble}
    \maketitle
    
    
    \begin{abstract}
        Magnetic confinement reactors---in particular tokamaks---offer one of the most promising options for achieving practical nuclear fusion, with the potential to provide virtually limitless, clean energy. The theoretical and numerical modeling of tokamak plasmas is simultaneously an essential component of effective reactor design, and a great research barrier. Tokamak operational conditions exhibit comparatively low Knudsen numbers. Kinetic effects, including kinetic waves and instabilities, Landau damping, bump-on-tail instabilities and more, are therefore highly influential in tokamak plasma dynamics. Purely fluid models are inherently incapable of capturing these effects, whereas the high dimensionality in purely kinetic models render them practically intractable for most relevant purposes.

        We consider a $\delta\!f$ decomposition model, with a macroscopic fluid background and microscopic kinetic correction, both fully coupled to each other. A similar manner of discretization is proposed to that used in the recent \texttt{STRUPHY} code \cite{Holderied_Possanner_Wang_2021, Holderied_2022, Li_et_al_2023} with a finite-element model for the background and a pseudo-particle/PiC model for the correction.

        The fluid background satisfies the full, non-linear, resistive, compressible, Hall MHD equations. \cite{Laakmann_Hu_Farrell_2022} introduces finite-element(-in-space) implicit timesteppers for the incompressible analogue to this system with structure-preserving (SP) properties in the ideal case, alongside parameter-robust preconditioners. We show that these timesteppers can derive from a finite-element-in-time (FET) (and finite-element-in-space) interpretation. The benefits of this reformulation are discussed, including the derivation of timesteppers that are higher order in time, and the quantifiable dissipative SP properties in the non-ideal, resistive case.
        
        We discuss possible options for extending this FET approach to timesteppers for the compressible case.

        The kinetic corrections satisfy linearized Boltzmann equations. Using a Lénard--Bernstein collision operator, these take Fokker--Planck-like forms \cite{Fokker_1914, Planck_1917} wherein pseudo-particles in the numerical model obey the neoclassical transport equations, with particle-independent Brownian drift terms. This offers a rigorous methodology for incorporating collisions into the particle transport model, without coupling the equations of motions for each particle.
        
        Works by Chen, Chacón et al. \cite{Chen_Chacón_Barnes_2011, Chacón_Chen_Barnes_2013, Chen_Chacón_2014, Chen_Chacón_2015} have developed structure-preserving particle pushers for neoclassical transport in the Vlasov equations, derived from Crank--Nicolson integrators. We show these too can can derive from a FET interpretation, similarly offering potential extensions to higher-order-in-time particle pushers. The FET formulation is used also to consider how the stochastic drift terms can be incorporated into the pushers. Stochastic gyrokinetic expansions are also discussed.

        Different options for the numerical implementation of these schemes are considered.

        Due to the efficacy of FET in the development of SP timesteppers for both the fluid and kinetic component, we hope this approach will prove effective in the future for developing SP timesteppers for the full hybrid model. We hope this will give us the opportunity to incorporate previously inaccessible kinetic effects into the highly effective, modern, finite-element MHD models.
    \end{abstract}
    
    
    \newpage
    \tableofcontents
    
    
    \newpage
    \pagenumbering{arabic}
    %\linenumbers\renewcommand\thelinenumber{\color{black!50}\arabic{linenumber}}
            \input{0 - introduction/main.tex}
        \part{Research}
            \input{1 - low-noise PiC models/main.tex}
            \input{2 - kinetic component/main.tex}
            \input{3 - fluid component/main.tex}
            \input{4 - numerical implementation/main.tex}
        \part{Project Overview}
            \input{5 - research plan/main.tex}
            \input{6 - summary/main.tex}
    
    
    %\section{}
    \newpage
    \pagenumbering{gobble}
        \printbibliography


    \newpage
    \pagenumbering{roman}
    \appendix
        \part{Appendices}
            \input{8 - Hilbert complexes/main.tex}
            \input{9 - weak conservation proofs/main.tex}
\end{document}

\end{document}


\title{\BA{Title in Progress...}}
\author{Boris Andrews}
\affil{Mathematical Institute, University of Oxford}
\date{\today}


\begin{document}
    \pagenumbering{gobble}
    \maketitle
    
    
    \begin{abstract}
        Magnetic confinement reactors---in particular tokamaks---offer one of the most promising options for achieving practical nuclear fusion, with the potential to provide virtually limitless, clean energy. The theoretical and numerical modeling of tokamak plasmas is simultaneously an essential component of effective reactor design, and a great research barrier. Tokamak operational conditions exhibit comparatively low Knudsen numbers. Kinetic effects, including kinetic waves and instabilities, Landau damping, bump-on-tail instabilities and more, are therefore highly influential in tokamak plasma dynamics. Purely fluid models are inherently incapable of capturing these effects, whereas the high dimensionality in purely kinetic models render them practically intractable for most relevant purposes.

        We consider a $\delta\!f$ decomposition model, with a macroscopic fluid background and microscopic kinetic correction, both fully coupled to each other. A similar manner of discretization is proposed to that used in the recent \texttt{STRUPHY} code \cite{Holderied_Possanner_Wang_2021, Holderied_2022, Li_et_al_2023} with a finite-element model for the background and a pseudo-particle/PiC model for the correction.

        The fluid background satisfies the full, non-linear, resistive, compressible, Hall MHD equations. \cite{Laakmann_Hu_Farrell_2022} introduces finite-element(-in-space) implicit timesteppers for the incompressible analogue to this system with structure-preserving (SP) properties in the ideal case, alongside parameter-robust preconditioners. We show that these timesteppers can derive from a finite-element-in-time (FET) (and finite-element-in-space) interpretation. The benefits of this reformulation are discussed, including the derivation of timesteppers that are higher order in time, and the quantifiable dissipative SP properties in the non-ideal, resistive case.
        
        We discuss possible options for extending this FET approach to timesteppers for the compressible case.

        The kinetic corrections satisfy linearized Boltzmann equations. Using a Lénard--Bernstein collision operator, these take Fokker--Planck-like forms \cite{Fokker_1914, Planck_1917} wherein pseudo-particles in the numerical model obey the neoclassical transport equations, with particle-independent Brownian drift terms. This offers a rigorous methodology for incorporating collisions into the particle transport model, without coupling the equations of motions for each particle.
        
        Works by Chen, Chacón et al. \cite{Chen_Chacón_Barnes_2011, Chacón_Chen_Barnes_2013, Chen_Chacón_2014, Chen_Chacón_2015} have developed structure-preserving particle pushers for neoclassical transport in the Vlasov equations, derived from Crank--Nicolson integrators. We show these too can can derive from a FET interpretation, similarly offering potential extensions to higher-order-in-time particle pushers. The FET formulation is used also to consider how the stochastic drift terms can be incorporated into the pushers. Stochastic gyrokinetic expansions are also discussed.

        Different options for the numerical implementation of these schemes are considered.

        Due to the efficacy of FET in the development of SP timesteppers for both the fluid and kinetic component, we hope this approach will prove effective in the future for developing SP timesteppers for the full hybrid model. We hope this will give us the opportunity to incorporate previously inaccessible kinetic effects into the highly effective, modern, finite-element MHD models.
    \end{abstract}
    
    
    \newpage
    \tableofcontents
    
    
    \newpage
    \pagenumbering{arabic}
    %\linenumbers\renewcommand\thelinenumber{\color{black!50}\arabic{linenumber}}
            \documentclass[12pt, a4paper]{report}

\documentclass[12pt, a4paper]{report}

\input{template/main.tex}

\title{\BA{Title in Progress...}}
\author{Boris Andrews}
\affil{Mathematical Institute, University of Oxford}
\date{\today}


\begin{document}
    \pagenumbering{gobble}
    \maketitle
    
    
    \begin{abstract}
        Magnetic confinement reactors---in particular tokamaks---offer one of the most promising options for achieving practical nuclear fusion, with the potential to provide virtually limitless, clean energy. The theoretical and numerical modeling of tokamak plasmas is simultaneously an essential component of effective reactor design, and a great research barrier. Tokamak operational conditions exhibit comparatively low Knudsen numbers. Kinetic effects, including kinetic waves and instabilities, Landau damping, bump-on-tail instabilities and more, are therefore highly influential in tokamak plasma dynamics. Purely fluid models are inherently incapable of capturing these effects, whereas the high dimensionality in purely kinetic models render them practically intractable for most relevant purposes.

        We consider a $\delta\!f$ decomposition model, with a macroscopic fluid background and microscopic kinetic correction, both fully coupled to each other. A similar manner of discretization is proposed to that used in the recent \texttt{STRUPHY} code \cite{Holderied_Possanner_Wang_2021, Holderied_2022, Li_et_al_2023} with a finite-element model for the background and a pseudo-particle/PiC model for the correction.

        The fluid background satisfies the full, non-linear, resistive, compressible, Hall MHD equations. \cite{Laakmann_Hu_Farrell_2022} introduces finite-element(-in-space) implicit timesteppers for the incompressible analogue to this system with structure-preserving (SP) properties in the ideal case, alongside parameter-robust preconditioners. We show that these timesteppers can derive from a finite-element-in-time (FET) (and finite-element-in-space) interpretation. The benefits of this reformulation are discussed, including the derivation of timesteppers that are higher order in time, and the quantifiable dissipative SP properties in the non-ideal, resistive case.
        
        We discuss possible options for extending this FET approach to timesteppers for the compressible case.

        The kinetic corrections satisfy linearized Boltzmann equations. Using a Lénard--Bernstein collision operator, these take Fokker--Planck-like forms \cite{Fokker_1914, Planck_1917} wherein pseudo-particles in the numerical model obey the neoclassical transport equations, with particle-independent Brownian drift terms. This offers a rigorous methodology for incorporating collisions into the particle transport model, without coupling the equations of motions for each particle.
        
        Works by Chen, Chacón et al. \cite{Chen_Chacón_Barnes_2011, Chacón_Chen_Barnes_2013, Chen_Chacón_2014, Chen_Chacón_2015} have developed structure-preserving particle pushers for neoclassical transport in the Vlasov equations, derived from Crank--Nicolson integrators. We show these too can can derive from a FET interpretation, similarly offering potential extensions to higher-order-in-time particle pushers. The FET formulation is used also to consider how the stochastic drift terms can be incorporated into the pushers. Stochastic gyrokinetic expansions are also discussed.

        Different options for the numerical implementation of these schemes are considered.

        Due to the efficacy of FET in the development of SP timesteppers for both the fluid and kinetic component, we hope this approach will prove effective in the future for developing SP timesteppers for the full hybrid model. We hope this will give us the opportunity to incorporate previously inaccessible kinetic effects into the highly effective, modern, finite-element MHD models.
    \end{abstract}
    
    
    \newpage
    \tableofcontents
    
    
    \newpage
    \pagenumbering{arabic}
    %\linenumbers\renewcommand\thelinenumber{\color{black!50}\arabic{linenumber}}
            \input{0 - introduction/main.tex}
        \part{Research}
            \input{1 - low-noise PiC models/main.tex}
            \input{2 - kinetic component/main.tex}
            \input{3 - fluid component/main.tex}
            \input{4 - numerical implementation/main.tex}
        \part{Project Overview}
            \input{5 - research plan/main.tex}
            \input{6 - summary/main.tex}
    
    
    %\section{}
    \newpage
    \pagenumbering{gobble}
        \printbibliography


    \newpage
    \pagenumbering{roman}
    \appendix
        \part{Appendices}
            \input{8 - Hilbert complexes/main.tex}
            \input{9 - weak conservation proofs/main.tex}
\end{document}


\title{\BA{Title in Progress...}}
\author{Boris Andrews}
\affil{Mathematical Institute, University of Oxford}
\date{\today}


\begin{document}
    \pagenumbering{gobble}
    \maketitle
    
    
    \begin{abstract}
        Magnetic confinement reactors---in particular tokamaks---offer one of the most promising options for achieving practical nuclear fusion, with the potential to provide virtually limitless, clean energy. The theoretical and numerical modeling of tokamak plasmas is simultaneously an essential component of effective reactor design, and a great research barrier. Tokamak operational conditions exhibit comparatively low Knudsen numbers. Kinetic effects, including kinetic waves and instabilities, Landau damping, bump-on-tail instabilities and more, are therefore highly influential in tokamak plasma dynamics. Purely fluid models are inherently incapable of capturing these effects, whereas the high dimensionality in purely kinetic models render them practically intractable for most relevant purposes.

        We consider a $\delta\!f$ decomposition model, with a macroscopic fluid background and microscopic kinetic correction, both fully coupled to each other. A similar manner of discretization is proposed to that used in the recent \texttt{STRUPHY} code \cite{Holderied_Possanner_Wang_2021, Holderied_2022, Li_et_al_2023} with a finite-element model for the background and a pseudo-particle/PiC model for the correction.

        The fluid background satisfies the full, non-linear, resistive, compressible, Hall MHD equations. \cite{Laakmann_Hu_Farrell_2022} introduces finite-element(-in-space) implicit timesteppers for the incompressible analogue to this system with structure-preserving (SP) properties in the ideal case, alongside parameter-robust preconditioners. We show that these timesteppers can derive from a finite-element-in-time (FET) (and finite-element-in-space) interpretation. The benefits of this reformulation are discussed, including the derivation of timesteppers that are higher order in time, and the quantifiable dissipative SP properties in the non-ideal, resistive case.
        
        We discuss possible options for extending this FET approach to timesteppers for the compressible case.

        The kinetic corrections satisfy linearized Boltzmann equations. Using a Lénard--Bernstein collision operator, these take Fokker--Planck-like forms \cite{Fokker_1914, Planck_1917} wherein pseudo-particles in the numerical model obey the neoclassical transport equations, with particle-independent Brownian drift terms. This offers a rigorous methodology for incorporating collisions into the particle transport model, without coupling the equations of motions for each particle.
        
        Works by Chen, Chacón et al. \cite{Chen_Chacón_Barnes_2011, Chacón_Chen_Barnes_2013, Chen_Chacón_2014, Chen_Chacón_2015} have developed structure-preserving particle pushers for neoclassical transport in the Vlasov equations, derived from Crank--Nicolson integrators. We show these too can can derive from a FET interpretation, similarly offering potential extensions to higher-order-in-time particle pushers. The FET formulation is used also to consider how the stochastic drift terms can be incorporated into the pushers. Stochastic gyrokinetic expansions are also discussed.

        Different options for the numerical implementation of these schemes are considered.

        Due to the efficacy of FET in the development of SP timesteppers for both the fluid and kinetic component, we hope this approach will prove effective in the future for developing SP timesteppers for the full hybrid model. We hope this will give us the opportunity to incorporate previously inaccessible kinetic effects into the highly effective, modern, finite-element MHD models.
    \end{abstract}
    
    
    \newpage
    \tableofcontents
    
    
    \newpage
    \pagenumbering{arabic}
    %\linenumbers\renewcommand\thelinenumber{\color{black!50}\arabic{linenumber}}
            \documentclass[12pt, a4paper]{report}

\input{template/main.tex}

\title{\BA{Title in Progress...}}
\author{Boris Andrews}
\affil{Mathematical Institute, University of Oxford}
\date{\today}


\begin{document}
    \pagenumbering{gobble}
    \maketitle
    
    
    \begin{abstract}
        Magnetic confinement reactors---in particular tokamaks---offer one of the most promising options for achieving practical nuclear fusion, with the potential to provide virtually limitless, clean energy. The theoretical and numerical modeling of tokamak plasmas is simultaneously an essential component of effective reactor design, and a great research barrier. Tokamak operational conditions exhibit comparatively low Knudsen numbers. Kinetic effects, including kinetic waves and instabilities, Landau damping, bump-on-tail instabilities and more, are therefore highly influential in tokamak plasma dynamics. Purely fluid models are inherently incapable of capturing these effects, whereas the high dimensionality in purely kinetic models render them practically intractable for most relevant purposes.

        We consider a $\delta\!f$ decomposition model, with a macroscopic fluid background and microscopic kinetic correction, both fully coupled to each other. A similar manner of discretization is proposed to that used in the recent \texttt{STRUPHY} code \cite{Holderied_Possanner_Wang_2021, Holderied_2022, Li_et_al_2023} with a finite-element model for the background and a pseudo-particle/PiC model for the correction.

        The fluid background satisfies the full, non-linear, resistive, compressible, Hall MHD equations. \cite{Laakmann_Hu_Farrell_2022} introduces finite-element(-in-space) implicit timesteppers for the incompressible analogue to this system with structure-preserving (SP) properties in the ideal case, alongside parameter-robust preconditioners. We show that these timesteppers can derive from a finite-element-in-time (FET) (and finite-element-in-space) interpretation. The benefits of this reformulation are discussed, including the derivation of timesteppers that are higher order in time, and the quantifiable dissipative SP properties in the non-ideal, resistive case.
        
        We discuss possible options for extending this FET approach to timesteppers for the compressible case.

        The kinetic corrections satisfy linearized Boltzmann equations. Using a Lénard--Bernstein collision operator, these take Fokker--Planck-like forms \cite{Fokker_1914, Planck_1917} wherein pseudo-particles in the numerical model obey the neoclassical transport equations, with particle-independent Brownian drift terms. This offers a rigorous methodology for incorporating collisions into the particle transport model, without coupling the equations of motions for each particle.
        
        Works by Chen, Chacón et al. \cite{Chen_Chacón_Barnes_2011, Chacón_Chen_Barnes_2013, Chen_Chacón_2014, Chen_Chacón_2015} have developed structure-preserving particle pushers for neoclassical transport in the Vlasov equations, derived from Crank--Nicolson integrators. We show these too can can derive from a FET interpretation, similarly offering potential extensions to higher-order-in-time particle pushers. The FET formulation is used also to consider how the stochastic drift terms can be incorporated into the pushers. Stochastic gyrokinetic expansions are also discussed.

        Different options for the numerical implementation of these schemes are considered.

        Due to the efficacy of FET in the development of SP timesteppers for both the fluid and kinetic component, we hope this approach will prove effective in the future for developing SP timesteppers for the full hybrid model. We hope this will give us the opportunity to incorporate previously inaccessible kinetic effects into the highly effective, modern, finite-element MHD models.
    \end{abstract}
    
    
    \newpage
    \tableofcontents
    
    
    \newpage
    \pagenumbering{arabic}
    %\linenumbers\renewcommand\thelinenumber{\color{black!50}\arabic{linenumber}}
            \input{0 - introduction/main.tex}
        \part{Research}
            \input{1 - low-noise PiC models/main.tex}
            \input{2 - kinetic component/main.tex}
            \input{3 - fluid component/main.tex}
            \input{4 - numerical implementation/main.tex}
        \part{Project Overview}
            \input{5 - research plan/main.tex}
            \input{6 - summary/main.tex}
    
    
    %\section{}
    \newpage
    \pagenumbering{gobble}
        \printbibliography


    \newpage
    \pagenumbering{roman}
    \appendix
        \part{Appendices}
            \input{8 - Hilbert complexes/main.tex}
            \input{9 - weak conservation proofs/main.tex}
\end{document}

        \part{Research}
            \documentclass[12pt, a4paper]{report}

\input{template/main.tex}

\title{\BA{Title in Progress...}}
\author{Boris Andrews}
\affil{Mathematical Institute, University of Oxford}
\date{\today}


\begin{document}
    \pagenumbering{gobble}
    \maketitle
    
    
    \begin{abstract}
        Magnetic confinement reactors---in particular tokamaks---offer one of the most promising options for achieving practical nuclear fusion, with the potential to provide virtually limitless, clean energy. The theoretical and numerical modeling of tokamak plasmas is simultaneously an essential component of effective reactor design, and a great research barrier. Tokamak operational conditions exhibit comparatively low Knudsen numbers. Kinetic effects, including kinetic waves and instabilities, Landau damping, bump-on-tail instabilities and more, are therefore highly influential in tokamak plasma dynamics. Purely fluid models are inherently incapable of capturing these effects, whereas the high dimensionality in purely kinetic models render them practically intractable for most relevant purposes.

        We consider a $\delta\!f$ decomposition model, with a macroscopic fluid background and microscopic kinetic correction, both fully coupled to each other. A similar manner of discretization is proposed to that used in the recent \texttt{STRUPHY} code \cite{Holderied_Possanner_Wang_2021, Holderied_2022, Li_et_al_2023} with a finite-element model for the background and a pseudo-particle/PiC model for the correction.

        The fluid background satisfies the full, non-linear, resistive, compressible, Hall MHD equations. \cite{Laakmann_Hu_Farrell_2022} introduces finite-element(-in-space) implicit timesteppers for the incompressible analogue to this system with structure-preserving (SP) properties in the ideal case, alongside parameter-robust preconditioners. We show that these timesteppers can derive from a finite-element-in-time (FET) (and finite-element-in-space) interpretation. The benefits of this reformulation are discussed, including the derivation of timesteppers that are higher order in time, and the quantifiable dissipative SP properties in the non-ideal, resistive case.
        
        We discuss possible options for extending this FET approach to timesteppers for the compressible case.

        The kinetic corrections satisfy linearized Boltzmann equations. Using a Lénard--Bernstein collision operator, these take Fokker--Planck-like forms \cite{Fokker_1914, Planck_1917} wherein pseudo-particles in the numerical model obey the neoclassical transport equations, with particle-independent Brownian drift terms. This offers a rigorous methodology for incorporating collisions into the particle transport model, without coupling the equations of motions for each particle.
        
        Works by Chen, Chacón et al. \cite{Chen_Chacón_Barnes_2011, Chacón_Chen_Barnes_2013, Chen_Chacón_2014, Chen_Chacón_2015} have developed structure-preserving particle pushers for neoclassical transport in the Vlasov equations, derived from Crank--Nicolson integrators. We show these too can can derive from a FET interpretation, similarly offering potential extensions to higher-order-in-time particle pushers. The FET formulation is used also to consider how the stochastic drift terms can be incorporated into the pushers. Stochastic gyrokinetic expansions are also discussed.

        Different options for the numerical implementation of these schemes are considered.

        Due to the efficacy of FET in the development of SP timesteppers for both the fluid and kinetic component, we hope this approach will prove effective in the future for developing SP timesteppers for the full hybrid model. We hope this will give us the opportunity to incorporate previously inaccessible kinetic effects into the highly effective, modern, finite-element MHD models.
    \end{abstract}
    
    
    \newpage
    \tableofcontents
    
    
    \newpage
    \pagenumbering{arabic}
    %\linenumbers\renewcommand\thelinenumber{\color{black!50}\arabic{linenumber}}
            \input{0 - introduction/main.tex}
        \part{Research}
            \input{1 - low-noise PiC models/main.tex}
            \input{2 - kinetic component/main.tex}
            \input{3 - fluid component/main.tex}
            \input{4 - numerical implementation/main.tex}
        \part{Project Overview}
            \input{5 - research plan/main.tex}
            \input{6 - summary/main.tex}
    
    
    %\section{}
    \newpage
    \pagenumbering{gobble}
        \printbibliography


    \newpage
    \pagenumbering{roman}
    \appendix
        \part{Appendices}
            \input{8 - Hilbert complexes/main.tex}
            \input{9 - weak conservation proofs/main.tex}
\end{document}

            \documentclass[12pt, a4paper]{report}

\input{template/main.tex}

\title{\BA{Title in Progress...}}
\author{Boris Andrews}
\affil{Mathematical Institute, University of Oxford}
\date{\today}


\begin{document}
    \pagenumbering{gobble}
    \maketitle
    
    
    \begin{abstract}
        Magnetic confinement reactors---in particular tokamaks---offer one of the most promising options for achieving practical nuclear fusion, with the potential to provide virtually limitless, clean energy. The theoretical and numerical modeling of tokamak plasmas is simultaneously an essential component of effective reactor design, and a great research barrier. Tokamak operational conditions exhibit comparatively low Knudsen numbers. Kinetic effects, including kinetic waves and instabilities, Landau damping, bump-on-tail instabilities and more, are therefore highly influential in tokamak plasma dynamics. Purely fluid models are inherently incapable of capturing these effects, whereas the high dimensionality in purely kinetic models render them practically intractable for most relevant purposes.

        We consider a $\delta\!f$ decomposition model, with a macroscopic fluid background and microscopic kinetic correction, both fully coupled to each other. A similar manner of discretization is proposed to that used in the recent \texttt{STRUPHY} code \cite{Holderied_Possanner_Wang_2021, Holderied_2022, Li_et_al_2023} with a finite-element model for the background and a pseudo-particle/PiC model for the correction.

        The fluid background satisfies the full, non-linear, resistive, compressible, Hall MHD equations. \cite{Laakmann_Hu_Farrell_2022} introduces finite-element(-in-space) implicit timesteppers for the incompressible analogue to this system with structure-preserving (SP) properties in the ideal case, alongside parameter-robust preconditioners. We show that these timesteppers can derive from a finite-element-in-time (FET) (and finite-element-in-space) interpretation. The benefits of this reformulation are discussed, including the derivation of timesteppers that are higher order in time, and the quantifiable dissipative SP properties in the non-ideal, resistive case.
        
        We discuss possible options for extending this FET approach to timesteppers for the compressible case.

        The kinetic corrections satisfy linearized Boltzmann equations. Using a Lénard--Bernstein collision operator, these take Fokker--Planck-like forms \cite{Fokker_1914, Planck_1917} wherein pseudo-particles in the numerical model obey the neoclassical transport equations, with particle-independent Brownian drift terms. This offers a rigorous methodology for incorporating collisions into the particle transport model, without coupling the equations of motions for each particle.
        
        Works by Chen, Chacón et al. \cite{Chen_Chacón_Barnes_2011, Chacón_Chen_Barnes_2013, Chen_Chacón_2014, Chen_Chacón_2015} have developed structure-preserving particle pushers for neoclassical transport in the Vlasov equations, derived from Crank--Nicolson integrators. We show these too can can derive from a FET interpretation, similarly offering potential extensions to higher-order-in-time particle pushers. The FET formulation is used also to consider how the stochastic drift terms can be incorporated into the pushers. Stochastic gyrokinetic expansions are also discussed.

        Different options for the numerical implementation of these schemes are considered.

        Due to the efficacy of FET in the development of SP timesteppers for both the fluid and kinetic component, we hope this approach will prove effective in the future for developing SP timesteppers for the full hybrid model. We hope this will give us the opportunity to incorporate previously inaccessible kinetic effects into the highly effective, modern, finite-element MHD models.
    \end{abstract}
    
    
    \newpage
    \tableofcontents
    
    
    \newpage
    \pagenumbering{arabic}
    %\linenumbers\renewcommand\thelinenumber{\color{black!50}\arabic{linenumber}}
            \input{0 - introduction/main.tex}
        \part{Research}
            \input{1 - low-noise PiC models/main.tex}
            \input{2 - kinetic component/main.tex}
            \input{3 - fluid component/main.tex}
            \input{4 - numerical implementation/main.tex}
        \part{Project Overview}
            \input{5 - research plan/main.tex}
            \input{6 - summary/main.tex}
    
    
    %\section{}
    \newpage
    \pagenumbering{gobble}
        \printbibliography


    \newpage
    \pagenumbering{roman}
    \appendix
        \part{Appendices}
            \input{8 - Hilbert complexes/main.tex}
            \input{9 - weak conservation proofs/main.tex}
\end{document}

            \documentclass[12pt, a4paper]{report}

\input{template/main.tex}

\title{\BA{Title in Progress...}}
\author{Boris Andrews}
\affil{Mathematical Institute, University of Oxford}
\date{\today}


\begin{document}
    \pagenumbering{gobble}
    \maketitle
    
    
    \begin{abstract}
        Magnetic confinement reactors---in particular tokamaks---offer one of the most promising options for achieving practical nuclear fusion, with the potential to provide virtually limitless, clean energy. The theoretical and numerical modeling of tokamak plasmas is simultaneously an essential component of effective reactor design, and a great research barrier. Tokamak operational conditions exhibit comparatively low Knudsen numbers. Kinetic effects, including kinetic waves and instabilities, Landau damping, bump-on-tail instabilities and more, are therefore highly influential in tokamak plasma dynamics. Purely fluid models are inherently incapable of capturing these effects, whereas the high dimensionality in purely kinetic models render them practically intractable for most relevant purposes.

        We consider a $\delta\!f$ decomposition model, with a macroscopic fluid background and microscopic kinetic correction, both fully coupled to each other. A similar manner of discretization is proposed to that used in the recent \texttt{STRUPHY} code \cite{Holderied_Possanner_Wang_2021, Holderied_2022, Li_et_al_2023} with a finite-element model for the background and a pseudo-particle/PiC model for the correction.

        The fluid background satisfies the full, non-linear, resistive, compressible, Hall MHD equations. \cite{Laakmann_Hu_Farrell_2022} introduces finite-element(-in-space) implicit timesteppers for the incompressible analogue to this system with structure-preserving (SP) properties in the ideal case, alongside parameter-robust preconditioners. We show that these timesteppers can derive from a finite-element-in-time (FET) (and finite-element-in-space) interpretation. The benefits of this reformulation are discussed, including the derivation of timesteppers that are higher order in time, and the quantifiable dissipative SP properties in the non-ideal, resistive case.
        
        We discuss possible options for extending this FET approach to timesteppers for the compressible case.

        The kinetic corrections satisfy linearized Boltzmann equations. Using a Lénard--Bernstein collision operator, these take Fokker--Planck-like forms \cite{Fokker_1914, Planck_1917} wherein pseudo-particles in the numerical model obey the neoclassical transport equations, with particle-independent Brownian drift terms. This offers a rigorous methodology for incorporating collisions into the particle transport model, without coupling the equations of motions for each particle.
        
        Works by Chen, Chacón et al. \cite{Chen_Chacón_Barnes_2011, Chacón_Chen_Barnes_2013, Chen_Chacón_2014, Chen_Chacón_2015} have developed structure-preserving particle pushers for neoclassical transport in the Vlasov equations, derived from Crank--Nicolson integrators. We show these too can can derive from a FET interpretation, similarly offering potential extensions to higher-order-in-time particle pushers. The FET formulation is used also to consider how the stochastic drift terms can be incorporated into the pushers. Stochastic gyrokinetic expansions are also discussed.

        Different options for the numerical implementation of these schemes are considered.

        Due to the efficacy of FET in the development of SP timesteppers for both the fluid and kinetic component, we hope this approach will prove effective in the future for developing SP timesteppers for the full hybrid model. We hope this will give us the opportunity to incorporate previously inaccessible kinetic effects into the highly effective, modern, finite-element MHD models.
    \end{abstract}
    
    
    \newpage
    \tableofcontents
    
    
    \newpage
    \pagenumbering{arabic}
    %\linenumbers\renewcommand\thelinenumber{\color{black!50}\arabic{linenumber}}
            \input{0 - introduction/main.tex}
        \part{Research}
            \input{1 - low-noise PiC models/main.tex}
            \input{2 - kinetic component/main.tex}
            \input{3 - fluid component/main.tex}
            \input{4 - numerical implementation/main.tex}
        \part{Project Overview}
            \input{5 - research plan/main.tex}
            \input{6 - summary/main.tex}
    
    
    %\section{}
    \newpage
    \pagenumbering{gobble}
        \printbibliography


    \newpage
    \pagenumbering{roman}
    \appendix
        \part{Appendices}
            \input{8 - Hilbert complexes/main.tex}
            \input{9 - weak conservation proofs/main.tex}
\end{document}

            \documentclass[12pt, a4paper]{report}

\input{template/main.tex}

\title{\BA{Title in Progress...}}
\author{Boris Andrews}
\affil{Mathematical Institute, University of Oxford}
\date{\today}


\begin{document}
    \pagenumbering{gobble}
    \maketitle
    
    
    \begin{abstract}
        Magnetic confinement reactors---in particular tokamaks---offer one of the most promising options for achieving practical nuclear fusion, with the potential to provide virtually limitless, clean energy. The theoretical and numerical modeling of tokamak plasmas is simultaneously an essential component of effective reactor design, and a great research barrier. Tokamak operational conditions exhibit comparatively low Knudsen numbers. Kinetic effects, including kinetic waves and instabilities, Landau damping, bump-on-tail instabilities and more, are therefore highly influential in tokamak plasma dynamics. Purely fluid models are inherently incapable of capturing these effects, whereas the high dimensionality in purely kinetic models render them practically intractable for most relevant purposes.

        We consider a $\delta\!f$ decomposition model, with a macroscopic fluid background and microscopic kinetic correction, both fully coupled to each other. A similar manner of discretization is proposed to that used in the recent \texttt{STRUPHY} code \cite{Holderied_Possanner_Wang_2021, Holderied_2022, Li_et_al_2023} with a finite-element model for the background and a pseudo-particle/PiC model for the correction.

        The fluid background satisfies the full, non-linear, resistive, compressible, Hall MHD equations. \cite{Laakmann_Hu_Farrell_2022} introduces finite-element(-in-space) implicit timesteppers for the incompressible analogue to this system with structure-preserving (SP) properties in the ideal case, alongside parameter-robust preconditioners. We show that these timesteppers can derive from a finite-element-in-time (FET) (and finite-element-in-space) interpretation. The benefits of this reformulation are discussed, including the derivation of timesteppers that are higher order in time, and the quantifiable dissipative SP properties in the non-ideal, resistive case.
        
        We discuss possible options for extending this FET approach to timesteppers for the compressible case.

        The kinetic corrections satisfy linearized Boltzmann equations. Using a Lénard--Bernstein collision operator, these take Fokker--Planck-like forms \cite{Fokker_1914, Planck_1917} wherein pseudo-particles in the numerical model obey the neoclassical transport equations, with particle-independent Brownian drift terms. This offers a rigorous methodology for incorporating collisions into the particle transport model, without coupling the equations of motions for each particle.
        
        Works by Chen, Chacón et al. \cite{Chen_Chacón_Barnes_2011, Chacón_Chen_Barnes_2013, Chen_Chacón_2014, Chen_Chacón_2015} have developed structure-preserving particle pushers for neoclassical transport in the Vlasov equations, derived from Crank--Nicolson integrators. We show these too can can derive from a FET interpretation, similarly offering potential extensions to higher-order-in-time particle pushers. The FET formulation is used also to consider how the stochastic drift terms can be incorporated into the pushers. Stochastic gyrokinetic expansions are also discussed.

        Different options for the numerical implementation of these schemes are considered.

        Due to the efficacy of FET in the development of SP timesteppers for both the fluid and kinetic component, we hope this approach will prove effective in the future for developing SP timesteppers for the full hybrid model. We hope this will give us the opportunity to incorporate previously inaccessible kinetic effects into the highly effective, modern, finite-element MHD models.
    \end{abstract}
    
    
    \newpage
    \tableofcontents
    
    
    \newpage
    \pagenumbering{arabic}
    %\linenumbers\renewcommand\thelinenumber{\color{black!50}\arabic{linenumber}}
            \input{0 - introduction/main.tex}
        \part{Research}
            \input{1 - low-noise PiC models/main.tex}
            \input{2 - kinetic component/main.tex}
            \input{3 - fluid component/main.tex}
            \input{4 - numerical implementation/main.tex}
        \part{Project Overview}
            \input{5 - research plan/main.tex}
            \input{6 - summary/main.tex}
    
    
    %\section{}
    \newpage
    \pagenumbering{gobble}
        \printbibliography


    \newpage
    \pagenumbering{roman}
    \appendix
        \part{Appendices}
            \input{8 - Hilbert complexes/main.tex}
            \input{9 - weak conservation proofs/main.tex}
\end{document}

        \part{Project Overview}
            \documentclass[12pt, a4paper]{report}

\input{template/main.tex}

\title{\BA{Title in Progress...}}
\author{Boris Andrews}
\affil{Mathematical Institute, University of Oxford}
\date{\today}


\begin{document}
    \pagenumbering{gobble}
    \maketitle
    
    
    \begin{abstract}
        Magnetic confinement reactors---in particular tokamaks---offer one of the most promising options for achieving practical nuclear fusion, with the potential to provide virtually limitless, clean energy. The theoretical and numerical modeling of tokamak plasmas is simultaneously an essential component of effective reactor design, and a great research barrier. Tokamak operational conditions exhibit comparatively low Knudsen numbers. Kinetic effects, including kinetic waves and instabilities, Landau damping, bump-on-tail instabilities and more, are therefore highly influential in tokamak plasma dynamics. Purely fluid models are inherently incapable of capturing these effects, whereas the high dimensionality in purely kinetic models render them practically intractable for most relevant purposes.

        We consider a $\delta\!f$ decomposition model, with a macroscopic fluid background and microscopic kinetic correction, both fully coupled to each other. A similar manner of discretization is proposed to that used in the recent \texttt{STRUPHY} code \cite{Holderied_Possanner_Wang_2021, Holderied_2022, Li_et_al_2023} with a finite-element model for the background and a pseudo-particle/PiC model for the correction.

        The fluid background satisfies the full, non-linear, resistive, compressible, Hall MHD equations. \cite{Laakmann_Hu_Farrell_2022} introduces finite-element(-in-space) implicit timesteppers for the incompressible analogue to this system with structure-preserving (SP) properties in the ideal case, alongside parameter-robust preconditioners. We show that these timesteppers can derive from a finite-element-in-time (FET) (and finite-element-in-space) interpretation. The benefits of this reformulation are discussed, including the derivation of timesteppers that are higher order in time, and the quantifiable dissipative SP properties in the non-ideal, resistive case.
        
        We discuss possible options for extending this FET approach to timesteppers for the compressible case.

        The kinetic corrections satisfy linearized Boltzmann equations. Using a Lénard--Bernstein collision operator, these take Fokker--Planck-like forms \cite{Fokker_1914, Planck_1917} wherein pseudo-particles in the numerical model obey the neoclassical transport equations, with particle-independent Brownian drift terms. This offers a rigorous methodology for incorporating collisions into the particle transport model, without coupling the equations of motions for each particle.
        
        Works by Chen, Chacón et al. \cite{Chen_Chacón_Barnes_2011, Chacón_Chen_Barnes_2013, Chen_Chacón_2014, Chen_Chacón_2015} have developed structure-preserving particle pushers for neoclassical transport in the Vlasov equations, derived from Crank--Nicolson integrators. We show these too can can derive from a FET interpretation, similarly offering potential extensions to higher-order-in-time particle pushers. The FET formulation is used also to consider how the stochastic drift terms can be incorporated into the pushers. Stochastic gyrokinetic expansions are also discussed.

        Different options for the numerical implementation of these schemes are considered.

        Due to the efficacy of FET in the development of SP timesteppers for both the fluid and kinetic component, we hope this approach will prove effective in the future for developing SP timesteppers for the full hybrid model. We hope this will give us the opportunity to incorporate previously inaccessible kinetic effects into the highly effective, modern, finite-element MHD models.
    \end{abstract}
    
    
    \newpage
    \tableofcontents
    
    
    \newpage
    \pagenumbering{arabic}
    %\linenumbers\renewcommand\thelinenumber{\color{black!50}\arabic{linenumber}}
            \input{0 - introduction/main.tex}
        \part{Research}
            \input{1 - low-noise PiC models/main.tex}
            \input{2 - kinetic component/main.tex}
            \input{3 - fluid component/main.tex}
            \input{4 - numerical implementation/main.tex}
        \part{Project Overview}
            \input{5 - research plan/main.tex}
            \input{6 - summary/main.tex}
    
    
    %\section{}
    \newpage
    \pagenumbering{gobble}
        \printbibliography


    \newpage
    \pagenumbering{roman}
    \appendix
        \part{Appendices}
            \input{8 - Hilbert complexes/main.tex}
            \input{9 - weak conservation proofs/main.tex}
\end{document}

            \documentclass[12pt, a4paper]{report}

\input{template/main.tex}

\title{\BA{Title in Progress...}}
\author{Boris Andrews}
\affil{Mathematical Institute, University of Oxford}
\date{\today}


\begin{document}
    \pagenumbering{gobble}
    \maketitle
    
    
    \begin{abstract}
        Magnetic confinement reactors---in particular tokamaks---offer one of the most promising options for achieving practical nuclear fusion, with the potential to provide virtually limitless, clean energy. The theoretical and numerical modeling of tokamak plasmas is simultaneously an essential component of effective reactor design, and a great research barrier. Tokamak operational conditions exhibit comparatively low Knudsen numbers. Kinetic effects, including kinetic waves and instabilities, Landau damping, bump-on-tail instabilities and more, are therefore highly influential in tokamak plasma dynamics. Purely fluid models are inherently incapable of capturing these effects, whereas the high dimensionality in purely kinetic models render them practically intractable for most relevant purposes.

        We consider a $\delta\!f$ decomposition model, with a macroscopic fluid background and microscopic kinetic correction, both fully coupled to each other. A similar manner of discretization is proposed to that used in the recent \texttt{STRUPHY} code \cite{Holderied_Possanner_Wang_2021, Holderied_2022, Li_et_al_2023} with a finite-element model for the background and a pseudo-particle/PiC model for the correction.

        The fluid background satisfies the full, non-linear, resistive, compressible, Hall MHD equations. \cite{Laakmann_Hu_Farrell_2022} introduces finite-element(-in-space) implicit timesteppers for the incompressible analogue to this system with structure-preserving (SP) properties in the ideal case, alongside parameter-robust preconditioners. We show that these timesteppers can derive from a finite-element-in-time (FET) (and finite-element-in-space) interpretation. The benefits of this reformulation are discussed, including the derivation of timesteppers that are higher order in time, and the quantifiable dissipative SP properties in the non-ideal, resistive case.
        
        We discuss possible options for extending this FET approach to timesteppers for the compressible case.

        The kinetic corrections satisfy linearized Boltzmann equations. Using a Lénard--Bernstein collision operator, these take Fokker--Planck-like forms \cite{Fokker_1914, Planck_1917} wherein pseudo-particles in the numerical model obey the neoclassical transport equations, with particle-independent Brownian drift terms. This offers a rigorous methodology for incorporating collisions into the particle transport model, without coupling the equations of motions for each particle.
        
        Works by Chen, Chacón et al. \cite{Chen_Chacón_Barnes_2011, Chacón_Chen_Barnes_2013, Chen_Chacón_2014, Chen_Chacón_2015} have developed structure-preserving particle pushers for neoclassical transport in the Vlasov equations, derived from Crank--Nicolson integrators. We show these too can can derive from a FET interpretation, similarly offering potential extensions to higher-order-in-time particle pushers. The FET formulation is used also to consider how the stochastic drift terms can be incorporated into the pushers. Stochastic gyrokinetic expansions are also discussed.

        Different options for the numerical implementation of these schemes are considered.

        Due to the efficacy of FET in the development of SP timesteppers for both the fluid and kinetic component, we hope this approach will prove effective in the future for developing SP timesteppers for the full hybrid model. We hope this will give us the opportunity to incorporate previously inaccessible kinetic effects into the highly effective, modern, finite-element MHD models.
    \end{abstract}
    
    
    \newpage
    \tableofcontents
    
    
    \newpage
    \pagenumbering{arabic}
    %\linenumbers\renewcommand\thelinenumber{\color{black!50}\arabic{linenumber}}
            \input{0 - introduction/main.tex}
        \part{Research}
            \input{1 - low-noise PiC models/main.tex}
            \input{2 - kinetic component/main.tex}
            \input{3 - fluid component/main.tex}
            \input{4 - numerical implementation/main.tex}
        \part{Project Overview}
            \input{5 - research plan/main.tex}
            \input{6 - summary/main.tex}
    
    
    %\section{}
    \newpage
    \pagenumbering{gobble}
        \printbibliography


    \newpage
    \pagenumbering{roman}
    \appendix
        \part{Appendices}
            \input{8 - Hilbert complexes/main.tex}
            \input{9 - weak conservation proofs/main.tex}
\end{document}

    
    
    %\section{}
    \newpage
    \pagenumbering{gobble}
        \printbibliography


    \newpage
    \pagenumbering{roman}
    \appendix
        \part{Appendices}
            \documentclass[12pt, a4paper]{report}

\input{template/main.tex}

\title{\BA{Title in Progress...}}
\author{Boris Andrews}
\affil{Mathematical Institute, University of Oxford}
\date{\today}


\begin{document}
    \pagenumbering{gobble}
    \maketitle
    
    
    \begin{abstract}
        Magnetic confinement reactors---in particular tokamaks---offer one of the most promising options for achieving practical nuclear fusion, with the potential to provide virtually limitless, clean energy. The theoretical and numerical modeling of tokamak plasmas is simultaneously an essential component of effective reactor design, and a great research barrier. Tokamak operational conditions exhibit comparatively low Knudsen numbers. Kinetic effects, including kinetic waves and instabilities, Landau damping, bump-on-tail instabilities and more, are therefore highly influential in tokamak plasma dynamics. Purely fluid models are inherently incapable of capturing these effects, whereas the high dimensionality in purely kinetic models render them practically intractable for most relevant purposes.

        We consider a $\delta\!f$ decomposition model, with a macroscopic fluid background and microscopic kinetic correction, both fully coupled to each other. A similar manner of discretization is proposed to that used in the recent \texttt{STRUPHY} code \cite{Holderied_Possanner_Wang_2021, Holderied_2022, Li_et_al_2023} with a finite-element model for the background and a pseudo-particle/PiC model for the correction.

        The fluid background satisfies the full, non-linear, resistive, compressible, Hall MHD equations. \cite{Laakmann_Hu_Farrell_2022} introduces finite-element(-in-space) implicit timesteppers for the incompressible analogue to this system with structure-preserving (SP) properties in the ideal case, alongside parameter-robust preconditioners. We show that these timesteppers can derive from a finite-element-in-time (FET) (and finite-element-in-space) interpretation. The benefits of this reformulation are discussed, including the derivation of timesteppers that are higher order in time, and the quantifiable dissipative SP properties in the non-ideal, resistive case.
        
        We discuss possible options for extending this FET approach to timesteppers for the compressible case.

        The kinetic corrections satisfy linearized Boltzmann equations. Using a Lénard--Bernstein collision operator, these take Fokker--Planck-like forms \cite{Fokker_1914, Planck_1917} wherein pseudo-particles in the numerical model obey the neoclassical transport equations, with particle-independent Brownian drift terms. This offers a rigorous methodology for incorporating collisions into the particle transport model, without coupling the equations of motions for each particle.
        
        Works by Chen, Chacón et al. \cite{Chen_Chacón_Barnes_2011, Chacón_Chen_Barnes_2013, Chen_Chacón_2014, Chen_Chacón_2015} have developed structure-preserving particle pushers for neoclassical transport in the Vlasov equations, derived from Crank--Nicolson integrators. We show these too can can derive from a FET interpretation, similarly offering potential extensions to higher-order-in-time particle pushers. The FET formulation is used also to consider how the stochastic drift terms can be incorporated into the pushers. Stochastic gyrokinetic expansions are also discussed.

        Different options for the numerical implementation of these schemes are considered.

        Due to the efficacy of FET in the development of SP timesteppers for both the fluid and kinetic component, we hope this approach will prove effective in the future for developing SP timesteppers for the full hybrid model. We hope this will give us the opportunity to incorporate previously inaccessible kinetic effects into the highly effective, modern, finite-element MHD models.
    \end{abstract}
    
    
    \newpage
    \tableofcontents
    
    
    \newpage
    \pagenumbering{arabic}
    %\linenumbers\renewcommand\thelinenumber{\color{black!50}\arabic{linenumber}}
            \input{0 - introduction/main.tex}
        \part{Research}
            \input{1 - low-noise PiC models/main.tex}
            \input{2 - kinetic component/main.tex}
            \input{3 - fluid component/main.tex}
            \input{4 - numerical implementation/main.tex}
        \part{Project Overview}
            \input{5 - research plan/main.tex}
            \input{6 - summary/main.tex}
    
    
    %\section{}
    \newpage
    \pagenumbering{gobble}
        \printbibliography


    \newpage
    \pagenumbering{roman}
    \appendix
        \part{Appendices}
            \input{8 - Hilbert complexes/main.tex}
            \input{9 - weak conservation proofs/main.tex}
\end{document}

            \documentclass[12pt, a4paper]{report}

\input{template/main.tex}

\title{\BA{Title in Progress...}}
\author{Boris Andrews}
\affil{Mathematical Institute, University of Oxford}
\date{\today}


\begin{document}
    \pagenumbering{gobble}
    \maketitle
    
    
    \begin{abstract}
        Magnetic confinement reactors---in particular tokamaks---offer one of the most promising options for achieving practical nuclear fusion, with the potential to provide virtually limitless, clean energy. The theoretical and numerical modeling of tokamak plasmas is simultaneously an essential component of effective reactor design, and a great research barrier. Tokamak operational conditions exhibit comparatively low Knudsen numbers. Kinetic effects, including kinetic waves and instabilities, Landau damping, bump-on-tail instabilities and more, are therefore highly influential in tokamak plasma dynamics. Purely fluid models are inherently incapable of capturing these effects, whereas the high dimensionality in purely kinetic models render them practically intractable for most relevant purposes.

        We consider a $\delta\!f$ decomposition model, with a macroscopic fluid background and microscopic kinetic correction, both fully coupled to each other. A similar manner of discretization is proposed to that used in the recent \texttt{STRUPHY} code \cite{Holderied_Possanner_Wang_2021, Holderied_2022, Li_et_al_2023} with a finite-element model for the background and a pseudo-particle/PiC model for the correction.

        The fluid background satisfies the full, non-linear, resistive, compressible, Hall MHD equations. \cite{Laakmann_Hu_Farrell_2022} introduces finite-element(-in-space) implicit timesteppers for the incompressible analogue to this system with structure-preserving (SP) properties in the ideal case, alongside parameter-robust preconditioners. We show that these timesteppers can derive from a finite-element-in-time (FET) (and finite-element-in-space) interpretation. The benefits of this reformulation are discussed, including the derivation of timesteppers that are higher order in time, and the quantifiable dissipative SP properties in the non-ideal, resistive case.
        
        We discuss possible options for extending this FET approach to timesteppers for the compressible case.

        The kinetic corrections satisfy linearized Boltzmann equations. Using a Lénard--Bernstein collision operator, these take Fokker--Planck-like forms \cite{Fokker_1914, Planck_1917} wherein pseudo-particles in the numerical model obey the neoclassical transport equations, with particle-independent Brownian drift terms. This offers a rigorous methodology for incorporating collisions into the particle transport model, without coupling the equations of motions for each particle.
        
        Works by Chen, Chacón et al. \cite{Chen_Chacón_Barnes_2011, Chacón_Chen_Barnes_2013, Chen_Chacón_2014, Chen_Chacón_2015} have developed structure-preserving particle pushers for neoclassical transport in the Vlasov equations, derived from Crank--Nicolson integrators. We show these too can can derive from a FET interpretation, similarly offering potential extensions to higher-order-in-time particle pushers. The FET formulation is used also to consider how the stochastic drift terms can be incorporated into the pushers. Stochastic gyrokinetic expansions are also discussed.

        Different options for the numerical implementation of these schemes are considered.

        Due to the efficacy of FET in the development of SP timesteppers for both the fluid and kinetic component, we hope this approach will prove effective in the future for developing SP timesteppers for the full hybrid model. We hope this will give us the opportunity to incorporate previously inaccessible kinetic effects into the highly effective, modern, finite-element MHD models.
    \end{abstract}
    
    
    \newpage
    \tableofcontents
    
    
    \newpage
    \pagenumbering{arabic}
    %\linenumbers\renewcommand\thelinenumber{\color{black!50}\arabic{linenumber}}
            \input{0 - introduction/main.tex}
        \part{Research}
            \input{1 - low-noise PiC models/main.tex}
            \input{2 - kinetic component/main.tex}
            \input{3 - fluid component/main.tex}
            \input{4 - numerical implementation/main.tex}
        \part{Project Overview}
            \input{5 - research plan/main.tex}
            \input{6 - summary/main.tex}
    
    
    %\section{}
    \newpage
    \pagenumbering{gobble}
        \printbibliography


    \newpage
    \pagenumbering{roman}
    \appendix
        \part{Appendices}
            \input{8 - Hilbert complexes/main.tex}
            \input{9 - weak conservation proofs/main.tex}
\end{document}

\end{document}

        \part{Research}
            \documentclass[12pt, a4paper]{report}

\documentclass[12pt, a4paper]{report}

\input{template/main.tex}

\title{\BA{Title in Progress...}}
\author{Boris Andrews}
\affil{Mathematical Institute, University of Oxford}
\date{\today}


\begin{document}
    \pagenumbering{gobble}
    \maketitle
    
    
    \begin{abstract}
        Magnetic confinement reactors---in particular tokamaks---offer one of the most promising options for achieving practical nuclear fusion, with the potential to provide virtually limitless, clean energy. The theoretical and numerical modeling of tokamak plasmas is simultaneously an essential component of effective reactor design, and a great research barrier. Tokamak operational conditions exhibit comparatively low Knudsen numbers. Kinetic effects, including kinetic waves and instabilities, Landau damping, bump-on-tail instabilities and more, are therefore highly influential in tokamak plasma dynamics. Purely fluid models are inherently incapable of capturing these effects, whereas the high dimensionality in purely kinetic models render them practically intractable for most relevant purposes.

        We consider a $\delta\!f$ decomposition model, with a macroscopic fluid background and microscopic kinetic correction, both fully coupled to each other. A similar manner of discretization is proposed to that used in the recent \texttt{STRUPHY} code \cite{Holderied_Possanner_Wang_2021, Holderied_2022, Li_et_al_2023} with a finite-element model for the background and a pseudo-particle/PiC model for the correction.

        The fluid background satisfies the full, non-linear, resistive, compressible, Hall MHD equations. \cite{Laakmann_Hu_Farrell_2022} introduces finite-element(-in-space) implicit timesteppers for the incompressible analogue to this system with structure-preserving (SP) properties in the ideal case, alongside parameter-robust preconditioners. We show that these timesteppers can derive from a finite-element-in-time (FET) (and finite-element-in-space) interpretation. The benefits of this reformulation are discussed, including the derivation of timesteppers that are higher order in time, and the quantifiable dissipative SP properties in the non-ideal, resistive case.
        
        We discuss possible options for extending this FET approach to timesteppers for the compressible case.

        The kinetic corrections satisfy linearized Boltzmann equations. Using a Lénard--Bernstein collision operator, these take Fokker--Planck-like forms \cite{Fokker_1914, Planck_1917} wherein pseudo-particles in the numerical model obey the neoclassical transport equations, with particle-independent Brownian drift terms. This offers a rigorous methodology for incorporating collisions into the particle transport model, without coupling the equations of motions for each particle.
        
        Works by Chen, Chacón et al. \cite{Chen_Chacón_Barnes_2011, Chacón_Chen_Barnes_2013, Chen_Chacón_2014, Chen_Chacón_2015} have developed structure-preserving particle pushers for neoclassical transport in the Vlasov equations, derived from Crank--Nicolson integrators. We show these too can can derive from a FET interpretation, similarly offering potential extensions to higher-order-in-time particle pushers. The FET formulation is used also to consider how the stochastic drift terms can be incorporated into the pushers. Stochastic gyrokinetic expansions are also discussed.

        Different options for the numerical implementation of these schemes are considered.

        Due to the efficacy of FET in the development of SP timesteppers for both the fluid and kinetic component, we hope this approach will prove effective in the future for developing SP timesteppers for the full hybrid model. We hope this will give us the opportunity to incorporate previously inaccessible kinetic effects into the highly effective, modern, finite-element MHD models.
    \end{abstract}
    
    
    \newpage
    \tableofcontents
    
    
    \newpage
    \pagenumbering{arabic}
    %\linenumbers\renewcommand\thelinenumber{\color{black!50}\arabic{linenumber}}
            \input{0 - introduction/main.tex}
        \part{Research}
            \input{1 - low-noise PiC models/main.tex}
            \input{2 - kinetic component/main.tex}
            \input{3 - fluid component/main.tex}
            \input{4 - numerical implementation/main.tex}
        \part{Project Overview}
            \input{5 - research plan/main.tex}
            \input{6 - summary/main.tex}
    
    
    %\section{}
    \newpage
    \pagenumbering{gobble}
        \printbibliography


    \newpage
    \pagenumbering{roman}
    \appendix
        \part{Appendices}
            \input{8 - Hilbert complexes/main.tex}
            \input{9 - weak conservation proofs/main.tex}
\end{document}


\title{\BA{Title in Progress...}}
\author{Boris Andrews}
\affil{Mathematical Institute, University of Oxford}
\date{\today}


\begin{document}
    \pagenumbering{gobble}
    \maketitle
    
    
    \begin{abstract}
        Magnetic confinement reactors---in particular tokamaks---offer one of the most promising options for achieving practical nuclear fusion, with the potential to provide virtually limitless, clean energy. The theoretical and numerical modeling of tokamak plasmas is simultaneously an essential component of effective reactor design, and a great research barrier. Tokamak operational conditions exhibit comparatively low Knudsen numbers. Kinetic effects, including kinetic waves and instabilities, Landau damping, bump-on-tail instabilities and more, are therefore highly influential in tokamak plasma dynamics. Purely fluid models are inherently incapable of capturing these effects, whereas the high dimensionality in purely kinetic models render them practically intractable for most relevant purposes.

        We consider a $\delta\!f$ decomposition model, with a macroscopic fluid background and microscopic kinetic correction, both fully coupled to each other. A similar manner of discretization is proposed to that used in the recent \texttt{STRUPHY} code \cite{Holderied_Possanner_Wang_2021, Holderied_2022, Li_et_al_2023} with a finite-element model for the background and a pseudo-particle/PiC model for the correction.

        The fluid background satisfies the full, non-linear, resistive, compressible, Hall MHD equations. \cite{Laakmann_Hu_Farrell_2022} introduces finite-element(-in-space) implicit timesteppers for the incompressible analogue to this system with structure-preserving (SP) properties in the ideal case, alongside parameter-robust preconditioners. We show that these timesteppers can derive from a finite-element-in-time (FET) (and finite-element-in-space) interpretation. The benefits of this reformulation are discussed, including the derivation of timesteppers that are higher order in time, and the quantifiable dissipative SP properties in the non-ideal, resistive case.
        
        We discuss possible options for extending this FET approach to timesteppers for the compressible case.

        The kinetic corrections satisfy linearized Boltzmann equations. Using a Lénard--Bernstein collision operator, these take Fokker--Planck-like forms \cite{Fokker_1914, Planck_1917} wherein pseudo-particles in the numerical model obey the neoclassical transport equations, with particle-independent Brownian drift terms. This offers a rigorous methodology for incorporating collisions into the particle transport model, without coupling the equations of motions for each particle.
        
        Works by Chen, Chacón et al. \cite{Chen_Chacón_Barnes_2011, Chacón_Chen_Barnes_2013, Chen_Chacón_2014, Chen_Chacón_2015} have developed structure-preserving particle pushers for neoclassical transport in the Vlasov equations, derived from Crank--Nicolson integrators. We show these too can can derive from a FET interpretation, similarly offering potential extensions to higher-order-in-time particle pushers. The FET formulation is used also to consider how the stochastic drift terms can be incorporated into the pushers. Stochastic gyrokinetic expansions are also discussed.

        Different options for the numerical implementation of these schemes are considered.

        Due to the efficacy of FET in the development of SP timesteppers for both the fluid and kinetic component, we hope this approach will prove effective in the future for developing SP timesteppers for the full hybrid model. We hope this will give us the opportunity to incorporate previously inaccessible kinetic effects into the highly effective, modern, finite-element MHD models.
    \end{abstract}
    
    
    \newpage
    \tableofcontents
    
    
    \newpage
    \pagenumbering{arabic}
    %\linenumbers\renewcommand\thelinenumber{\color{black!50}\arabic{linenumber}}
            \documentclass[12pt, a4paper]{report}

\input{template/main.tex}

\title{\BA{Title in Progress...}}
\author{Boris Andrews}
\affil{Mathematical Institute, University of Oxford}
\date{\today}


\begin{document}
    \pagenumbering{gobble}
    \maketitle
    
    
    \begin{abstract}
        Magnetic confinement reactors---in particular tokamaks---offer one of the most promising options for achieving practical nuclear fusion, with the potential to provide virtually limitless, clean energy. The theoretical and numerical modeling of tokamak plasmas is simultaneously an essential component of effective reactor design, and a great research barrier. Tokamak operational conditions exhibit comparatively low Knudsen numbers. Kinetic effects, including kinetic waves and instabilities, Landau damping, bump-on-tail instabilities and more, are therefore highly influential in tokamak plasma dynamics. Purely fluid models are inherently incapable of capturing these effects, whereas the high dimensionality in purely kinetic models render them practically intractable for most relevant purposes.

        We consider a $\delta\!f$ decomposition model, with a macroscopic fluid background and microscopic kinetic correction, both fully coupled to each other. A similar manner of discretization is proposed to that used in the recent \texttt{STRUPHY} code \cite{Holderied_Possanner_Wang_2021, Holderied_2022, Li_et_al_2023} with a finite-element model for the background and a pseudo-particle/PiC model for the correction.

        The fluid background satisfies the full, non-linear, resistive, compressible, Hall MHD equations. \cite{Laakmann_Hu_Farrell_2022} introduces finite-element(-in-space) implicit timesteppers for the incompressible analogue to this system with structure-preserving (SP) properties in the ideal case, alongside parameter-robust preconditioners. We show that these timesteppers can derive from a finite-element-in-time (FET) (and finite-element-in-space) interpretation. The benefits of this reformulation are discussed, including the derivation of timesteppers that are higher order in time, and the quantifiable dissipative SP properties in the non-ideal, resistive case.
        
        We discuss possible options for extending this FET approach to timesteppers for the compressible case.

        The kinetic corrections satisfy linearized Boltzmann equations. Using a Lénard--Bernstein collision operator, these take Fokker--Planck-like forms \cite{Fokker_1914, Planck_1917} wherein pseudo-particles in the numerical model obey the neoclassical transport equations, with particle-independent Brownian drift terms. This offers a rigorous methodology for incorporating collisions into the particle transport model, without coupling the equations of motions for each particle.
        
        Works by Chen, Chacón et al. \cite{Chen_Chacón_Barnes_2011, Chacón_Chen_Barnes_2013, Chen_Chacón_2014, Chen_Chacón_2015} have developed structure-preserving particle pushers for neoclassical transport in the Vlasov equations, derived from Crank--Nicolson integrators. We show these too can can derive from a FET interpretation, similarly offering potential extensions to higher-order-in-time particle pushers. The FET formulation is used also to consider how the stochastic drift terms can be incorporated into the pushers. Stochastic gyrokinetic expansions are also discussed.

        Different options for the numerical implementation of these schemes are considered.

        Due to the efficacy of FET in the development of SP timesteppers for both the fluid and kinetic component, we hope this approach will prove effective in the future for developing SP timesteppers for the full hybrid model. We hope this will give us the opportunity to incorporate previously inaccessible kinetic effects into the highly effective, modern, finite-element MHD models.
    \end{abstract}
    
    
    \newpage
    \tableofcontents
    
    
    \newpage
    \pagenumbering{arabic}
    %\linenumbers\renewcommand\thelinenumber{\color{black!50}\arabic{linenumber}}
            \input{0 - introduction/main.tex}
        \part{Research}
            \input{1 - low-noise PiC models/main.tex}
            \input{2 - kinetic component/main.tex}
            \input{3 - fluid component/main.tex}
            \input{4 - numerical implementation/main.tex}
        \part{Project Overview}
            \input{5 - research plan/main.tex}
            \input{6 - summary/main.tex}
    
    
    %\section{}
    \newpage
    \pagenumbering{gobble}
        \printbibliography


    \newpage
    \pagenumbering{roman}
    \appendix
        \part{Appendices}
            \input{8 - Hilbert complexes/main.tex}
            \input{9 - weak conservation proofs/main.tex}
\end{document}

        \part{Research}
            \documentclass[12pt, a4paper]{report}

\input{template/main.tex}

\title{\BA{Title in Progress...}}
\author{Boris Andrews}
\affil{Mathematical Institute, University of Oxford}
\date{\today}


\begin{document}
    \pagenumbering{gobble}
    \maketitle
    
    
    \begin{abstract}
        Magnetic confinement reactors---in particular tokamaks---offer one of the most promising options for achieving practical nuclear fusion, with the potential to provide virtually limitless, clean energy. The theoretical and numerical modeling of tokamak plasmas is simultaneously an essential component of effective reactor design, and a great research barrier. Tokamak operational conditions exhibit comparatively low Knudsen numbers. Kinetic effects, including kinetic waves and instabilities, Landau damping, bump-on-tail instabilities and more, are therefore highly influential in tokamak plasma dynamics. Purely fluid models are inherently incapable of capturing these effects, whereas the high dimensionality in purely kinetic models render them practically intractable for most relevant purposes.

        We consider a $\delta\!f$ decomposition model, with a macroscopic fluid background and microscopic kinetic correction, both fully coupled to each other. A similar manner of discretization is proposed to that used in the recent \texttt{STRUPHY} code \cite{Holderied_Possanner_Wang_2021, Holderied_2022, Li_et_al_2023} with a finite-element model for the background and a pseudo-particle/PiC model for the correction.

        The fluid background satisfies the full, non-linear, resistive, compressible, Hall MHD equations. \cite{Laakmann_Hu_Farrell_2022} introduces finite-element(-in-space) implicit timesteppers for the incompressible analogue to this system with structure-preserving (SP) properties in the ideal case, alongside parameter-robust preconditioners. We show that these timesteppers can derive from a finite-element-in-time (FET) (and finite-element-in-space) interpretation. The benefits of this reformulation are discussed, including the derivation of timesteppers that are higher order in time, and the quantifiable dissipative SP properties in the non-ideal, resistive case.
        
        We discuss possible options for extending this FET approach to timesteppers for the compressible case.

        The kinetic corrections satisfy linearized Boltzmann equations. Using a Lénard--Bernstein collision operator, these take Fokker--Planck-like forms \cite{Fokker_1914, Planck_1917} wherein pseudo-particles in the numerical model obey the neoclassical transport equations, with particle-independent Brownian drift terms. This offers a rigorous methodology for incorporating collisions into the particle transport model, without coupling the equations of motions for each particle.
        
        Works by Chen, Chacón et al. \cite{Chen_Chacón_Barnes_2011, Chacón_Chen_Barnes_2013, Chen_Chacón_2014, Chen_Chacón_2015} have developed structure-preserving particle pushers for neoclassical transport in the Vlasov equations, derived from Crank--Nicolson integrators. We show these too can can derive from a FET interpretation, similarly offering potential extensions to higher-order-in-time particle pushers. The FET formulation is used also to consider how the stochastic drift terms can be incorporated into the pushers. Stochastic gyrokinetic expansions are also discussed.

        Different options for the numerical implementation of these schemes are considered.

        Due to the efficacy of FET in the development of SP timesteppers for both the fluid and kinetic component, we hope this approach will prove effective in the future for developing SP timesteppers for the full hybrid model. We hope this will give us the opportunity to incorporate previously inaccessible kinetic effects into the highly effective, modern, finite-element MHD models.
    \end{abstract}
    
    
    \newpage
    \tableofcontents
    
    
    \newpage
    \pagenumbering{arabic}
    %\linenumbers\renewcommand\thelinenumber{\color{black!50}\arabic{linenumber}}
            \input{0 - introduction/main.tex}
        \part{Research}
            \input{1 - low-noise PiC models/main.tex}
            \input{2 - kinetic component/main.tex}
            \input{3 - fluid component/main.tex}
            \input{4 - numerical implementation/main.tex}
        \part{Project Overview}
            \input{5 - research plan/main.tex}
            \input{6 - summary/main.tex}
    
    
    %\section{}
    \newpage
    \pagenumbering{gobble}
        \printbibliography


    \newpage
    \pagenumbering{roman}
    \appendix
        \part{Appendices}
            \input{8 - Hilbert complexes/main.tex}
            \input{9 - weak conservation proofs/main.tex}
\end{document}

            \documentclass[12pt, a4paper]{report}

\input{template/main.tex}

\title{\BA{Title in Progress...}}
\author{Boris Andrews}
\affil{Mathematical Institute, University of Oxford}
\date{\today}


\begin{document}
    \pagenumbering{gobble}
    \maketitle
    
    
    \begin{abstract}
        Magnetic confinement reactors---in particular tokamaks---offer one of the most promising options for achieving practical nuclear fusion, with the potential to provide virtually limitless, clean energy. The theoretical and numerical modeling of tokamak plasmas is simultaneously an essential component of effective reactor design, and a great research barrier. Tokamak operational conditions exhibit comparatively low Knudsen numbers. Kinetic effects, including kinetic waves and instabilities, Landau damping, bump-on-tail instabilities and more, are therefore highly influential in tokamak plasma dynamics. Purely fluid models are inherently incapable of capturing these effects, whereas the high dimensionality in purely kinetic models render them practically intractable for most relevant purposes.

        We consider a $\delta\!f$ decomposition model, with a macroscopic fluid background and microscopic kinetic correction, both fully coupled to each other. A similar manner of discretization is proposed to that used in the recent \texttt{STRUPHY} code \cite{Holderied_Possanner_Wang_2021, Holderied_2022, Li_et_al_2023} with a finite-element model for the background and a pseudo-particle/PiC model for the correction.

        The fluid background satisfies the full, non-linear, resistive, compressible, Hall MHD equations. \cite{Laakmann_Hu_Farrell_2022} introduces finite-element(-in-space) implicit timesteppers for the incompressible analogue to this system with structure-preserving (SP) properties in the ideal case, alongside parameter-robust preconditioners. We show that these timesteppers can derive from a finite-element-in-time (FET) (and finite-element-in-space) interpretation. The benefits of this reformulation are discussed, including the derivation of timesteppers that are higher order in time, and the quantifiable dissipative SP properties in the non-ideal, resistive case.
        
        We discuss possible options for extending this FET approach to timesteppers for the compressible case.

        The kinetic corrections satisfy linearized Boltzmann equations. Using a Lénard--Bernstein collision operator, these take Fokker--Planck-like forms \cite{Fokker_1914, Planck_1917} wherein pseudo-particles in the numerical model obey the neoclassical transport equations, with particle-independent Brownian drift terms. This offers a rigorous methodology for incorporating collisions into the particle transport model, without coupling the equations of motions for each particle.
        
        Works by Chen, Chacón et al. \cite{Chen_Chacón_Barnes_2011, Chacón_Chen_Barnes_2013, Chen_Chacón_2014, Chen_Chacón_2015} have developed structure-preserving particle pushers for neoclassical transport in the Vlasov equations, derived from Crank--Nicolson integrators. We show these too can can derive from a FET interpretation, similarly offering potential extensions to higher-order-in-time particle pushers. The FET formulation is used also to consider how the stochastic drift terms can be incorporated into the pushers. Stochastic gyrokinetic expansions are also discussed.

        Different options for the numerical implementation of these schemes are considered.

        Due to the efficacy of FET in the development of SP timesteppers for both the fluid and kinetic component, we hope this approach will prove effective in the future for developing SP timesteppers for the full hybrid model. We hope this will give us the opportunity to incorporate previously inaccessible kinetic effects into the highly effective, modern, finite-element MHD models.
    \end{abstract}
    
    
    \newpage
    \tableofcontents
    
    
    \newpage
    \pagenumbering{arabic}
    %\linenumbers\renewcommand\thelinenumber{\color{black!50}\arabic{linenumber}}
            \input{0 - introduction/main.tex}
        \part{Research}
            \input{1 - low-noise PiC models/main.tex}
            \input{2 - kinetic component/main.tex}
            \input{3 - fluid component/main.tex}
            \input{4 - numerical implementation/main.tex}
        \part{Project Overview}
            \input{5 - research plan/main.tex}
            \input{6 - summary/main.tex}
    
    
    %\section{}
    \newpage
    \pagenumbering{gobble}
        \printbibliography


    \newpage
    \pagenumbering{roman}
    \appendix
        \part{Appendices}
            \input{8 - Hilbert complexes/main.tex}
            \input{9 - weak conservation proofs/main.tex}
\end{document}

            \documentclass[12pt, a4paper]{report}

\input{template/main.tex}

\title{\BA{Title in Progress...}}
\author{Boris Andrews}
\affil{Mathematical Institute, University of Oxford}
\date{\today}


\begin{document}
    \pagenumbering{gobble}
    \maketitle
    
    
    \begin{abstract}
        Magnetic confinement reactors---in particular tokamaks---offer one of the most promising options for achieving practical nuclear fusion, with the potential to provide virtually limitless, clean energy. The theoretical and numerical modeling of tokamak plasmas is simultaneously an essential component of effective reactor design, and a great research barrier. Tokamak operational conditions exhibit comparatively low Knudsen numbers. Kinetic effects, including kinetic waves and instabilities, Landau damping, bump-on-tail instabilities and more, are therefore highly influential in tokamak plasma dynamics. Purely fluid models are inherently incapable of capturing these effects, whereas the high dimensionality in purely kinetic models render them practically intractable for most relevant purposes.

        We consider a $\delta\!f$ decomposition model, with a macroscopic fluid background and microscopic kinetic correction, both fully coupled to each other. A similar manner of discretization is proposed to that used in the recent \texttt{STRUPHY} code \cite{Holderied_Possanner_Wang_2021, Holderied_2022, Li_et_al_2023} with a finite-element model for the background and a pseudo-particle/PiC model for the correction.

        The fluid background satisfies the full, non-linear, resistive, compressible, Hall MHD equations. \cite{Laakmann_Hu_Farrell_2022} introduces finite-element(-in-space) implicit timesteppers for the incompressible analogue to this system with structure-preserving (SP) properties in the ideal case, alongside parameter-robust preconditioners. We show that these timesteppers can derive from a finite-element-in-time (FET) (and finite-element-in-space) interpretation. The benefits of this reformulation are discussed, including the derivation of timesteppers that are higher order in time, and the quantifiable dissipative SP properties in the non-ideal, resistive case.
        
        We discuss possible options for extending this FET approach to timesteppers for the compressible case.

        The kinetic corrections satisfy linearized Boltzmann equations. Using a Lénard--Bernstein collision operator, these take Fokker--Planck-like forms \cite{Fokker_1914, Planck_1917} wherein pseudo-particles in the numerical model obey the neoclassical transport equations, with particle-independent Brownian drift terms. This offers a rigorous methodology for incorporating collisions into the particle transport model, without coupling the equations of motions for each particle.
        
        Works by Chen, Chacón et al. \cite{Chen_Chacón_Barnes_2011, Chacón_Chen_Barnes_2013, Chen_Chacón_2014, Chen_Chacón_2015} have developed structure-preserving particle pushers for neoclassical transport in the Vlasov equations, derived from Crank--Nicolson integrators. We show these too can can derive from a FET interpretation, similarly offering potential extensions to higher-order-in-time particle pushers. The FET formulation is used also to consider how the stochastic drift terms can be incorporated into the pushers. Stochastic gyrokinetic expansions are also discussed.

        Different options for the numerical implementation of these schemes are considered.

        Due to the efficacy of FET in the development of SP timesteppers for both the fluid and kinetic component, we hope this approach will prove effective in the future for developing SP timesteppers for the full hybrid model. We hope this will give us the opportunity to incorporate previously inaccessible kinetic effects into the highly effective, modern, finite-element MHD models.
    \end{abstract}
    
    
    \newpage
    \tableofcontents
    
    
    \newpage
    \pagenumbering{arabic}
    %\linenumbers\renewcommand\thelinenumber{\color{black!50}\arabic{linenumber}}
            \input{0 - introduction/main.tex}
        \part{Research}
            \input{1 - low-noise PiC models/main.tex}
            \input{2 - kinetic component/main.tex}
            \input{3 - fluid component/main.tex}
            \input{4 - numerical implementation/main.tex}
        \part{Project Overview}
            \input{5 - research plan/main.tex}
            \input{6 - summary/main.tex}
    
    
    %\section{}
    \newpage
    \pagenumbering{gobble}
        \printbibliography


    \newpage
    \pagenumbering{roman}
    \appendix
        \part{Appendices}
            \input{8 - Hilbert complexes/main.tex}
            \input{9 - weak conservation proofs/main.tex}
\end{document}

            \documentclass[12pt, a4paper]{report}

\input{template/main.tex}

\title{\BA{Title in Progress...}}
\author{Boris Andrews}
\affil{Mathematical Institute, University of Oxford}
\date{\today}


\begin{document}
    \pagenumbering{gobble}
    \maketitle
    
    
    \begin{abstract}
        Magnetic confinement reactors---in particular tokamaks---offer one of the most promising options for achieving practical nuclear fusion, with the potential to provide virtually limitless, clean energy. The theoretical and numerical modeling of tokamak plasmas is simultaneously an essential component of effective reactor design, and a great research barrier. Tokamak operational conditions exhibit comparatively low Knudsen numbers. Kinetic effects, including kinetic waves and instabilities, Landau damping, bump-on-tail instabilities and more, are therefore highly influential in tokamak plasma dynamics. Purely fluid models are inherently incapable of capturing these effects, whereas the high dimensionality in purely kinetic models render them practically intractable for most relevant purposes.

        We consider a $\delta\!f$ decomposition model, with a macroscopic fluid background and microscopic kinetic correction, both fully coupled to each other. A similar manner of discretization is proposed to that used in the recent \texttt{STRUPHY} code \cite{Holderied_Possanner_Wang_2021, Holderied_2022, Li_et_al_2023} with a finite-element model for the background and a pseudo-particle/PiC model for the correction.

        The fluid background satisfies the full, non-linear, resistive, compressible, Hall MHD equations. \cite{Laakmann_Hu_Farrell_2022} introduces finite-element(-in-space) implicit timesteppers for the incompressible analogue to this system with structure-preserving (SP) properties in the ideal case, alongside parameter-robust preconditioners. We show that these timesteppers can derive from a finite-element-in-time (FET) (and finite-element-in-space) interpretation. The benefits of this reformulation are discussed, including the derivation of timesteppers that are higher order in time, and the quantifiable dissipative SP properties in the non-ideal, resistive case.
        
        We discuss possible options for extending this FET approach to timesteppers for the compressible case.

        The kinetic corrections satisfy linearized Boltzmann equations. Using a Lénard--Bernstein collision operator, these take Fokker--Planck-like forms \cite{Fokker_1914, Planck_1917} wherein pseudo-particles in the numerical model obey the neoclassical transport equations, with particle-independent Brownian drift terms. This offers a rigorous methodology for incorporating collisions into the particle transport model, without coupling the equations of motions for each particle.
        
        Works by Chen, Chacón et al. \cite{Chen_Chacón_Barnes_2011, Chacón_Chen_Barnes_2013, Chen_Chacón_2014, Chen_Chacón_2015} have developed structure-preserving particle pushers for neoclassical transport in the Vlasov equations, derived from Crank--Nicolson integrators. We show these too can can derive from a FET interpretation, similarly offering potential extensions to higher-order-in-time particle pushers. The FET formulation is used also to consider how the stochastic drift terms can be incorporated into the pushers. Stochastic gyrokinetic expansions are also discussed.

        Different options for the numerical implementation of these schemes are considered.

        Due to the efficacy of FET in the development of SP timesteppers for both the fluid and kinetic component, we hope this approach will prove effective in the future for developing SP timesteppers for the full hybrid model. We hope this will give us the opportunity to incorporate previously inaccessible kinetic effects into the highly effective, modern, finite-element MHD models.
    \end{abstract}
    
    
    \newpage
    \tableofcontents
    
    
    \newpage
    \pagenumbering{arabic}
    %\linenumbers\renewcommand\thelinenumber{\color{black!50}\arabic{linenumber}}
            \input{0 - introduction/main.tex}
        \part{Research}
            \input{1 - low-noise PiC models/main.tex}
            \input{2 - kinetic component/main.tex}
            \input{3 - fluid component/main.tex}
            \input{4 - numerical implementation/main.tex}
        \part{Project Overview}
            \input{5 - research plan/main.tex}
            \input{6 - summary/main.tex}
    
    
    %\section{}
    \newpage
    \pagenumbering{gobble}
        \printbibliography


    \newpage
    \pagenumbering{roman}
    \appendix
        \part{Appendices}
            \input{8 - Hilbert complexes/main.tex}
            \input{9 - weak conservation proofs/main.tex}
\end{document}

        \part{Project Overview}
            \documentclass[12pt, a4paper]{report}

\input{template/main.tex}

\title{\BA{Title in Progress...}}
\author{Boris Andrews}
\affil{Mathematical Institute, University of Oxford}
\date{\today}


\begin{document}
    \pagenumbering{gobble}
    \maketitle
    
    
    \begin{abstract}
        Magnetic confinement reactors---in particular tokamaks---offer one of the most promising options for achieving practical nuclear fusion, with the potential to provide virtually limitless, clean energy. The theoretical and numerical modeling of tokamak plasmas is simultaneously an essential component of effective reactor design, and a great research barrier. Tokamak operational conditions exhibit comparatively low Knudsen numbers. Kinetic effects, including kinetic waves and instabilities, Landau damping, bump-on-tail instabilities and more, are therefore highly influential in tokamak plasma dynamics. Purely fluid models are inherently incapable of capturing these effects, whereas the high dimensionality in purely kinetic models render them practically intractable for most relevant purposes.

        We consider a $\delta\!f$ decomposition model, with a macroscopic fluid background and microscopic kinetic correction, both fully coupled to each other. A similar manner of discretization is proposed to that used in the recent \texttt{STRUPHY} code \cite{Holderied_Possanner_Wang_2021, Holderied_2022, Li_et_al_2023} with a finite-element model for the background and a pseudo-particle/PiC model for the correction.

        The fluid background satisfies the full, non-linear, resistive, compressible, Hall MHD equations. \cite{Laakmann_Hu_Farrell_2022} introduces finite-element(-in-space) implicit timesteppers for the incompressible analogue to this system with structure-preserving (SP) properties in the ideal case, alongside parameter-robust preconditioners. We show that these timesteppers can derive from a finite-element-in-time (FET) (and finite-element-in-space) interpretation. The benefits of this reformulation are discussed, including the derivation of timesteppers that are higher order in time, and the quantifiable dissipative SP properties in the non-ideal, resistive case.
        
        We discuss possible options for extending this FET approach to timesteppers for the compressible case.

        The kinetic corrections satisfy linearized Boltzmann equations. Using a Lénard--Bernstein collision operator, these take Fokker--Planck-like forms \cite{Fokker_1914, Planck_1917} wherein pseudo-particles in the numerical model obey the neoclassical transport equations, with particle-independent Brownian drift terms. This offers a rigorous methodology for incorporating collisions into the particle transport model, without coupling the equations of motions for each particle.
        
        Works by Chen, Chacón et al. \cite{Chen_Chacón_Barnes_2011, Chacón_Chen_Barnes_2013, Chen_Chacón_2014, Chen_Chacón_2015} have developed structure-preserving particle pushers for neoclassical transport in the Vlasov equations, derived from Crank--Nicolson integrators. We show these too can can derive from a FET interpretation, similarly offering potential extensions to higher-order-in-time particle pushers. The FET formulation is used also to consider how the stochastic drift terms can be incorporated into the pushers. Stochastic gyrokinetic expansions are also discussed.

        Different options for the numerical implementation of these schemes are considered.

        Due to the efficacy of FET in the development of SP timesteppers for both the fluid and kinetic component, we hope this approach will prove effective in the future for developing SP timesteppers for the full hybrid model. We hope this will give us the opportunity to incorporate previously inaccessible kinetic effects into the highly effective, modern, finite-element MHD models.
    \end{abstract}
    
    
    \newpage
    \tableofcontents
    
    
    \newpage
    \pagenumbering{arabic}
    %\linenumbers\renewcommand\thelinenumber{\color{black!50}\arabic{linenumber}}
            \input{0 - introduction/main.tex}
        \part{Research}
            \input{1 - low-noise PiC models/main.tex}
            \input{2 - kinetic component/main.tex}
            \input{3 - fluid component/main.tex}
            \input{4 - numerical implementation/main.tex}
        \part{Project Overview}
            \input{5 - research plan/main.tex}
            \input{6 - summary/main.tex}
    
    
    %\section{}
    \newpage
    \pagenumbering{gobble}
        \printbibliography


    \newpage
    \pagenumbering{roman}
    \appendix
        \part{Appendices}
            \input{8 - Hilbert complexes/main.tex}
            \input{9 - weak conservation proofs/main.tex}
\end{document}

            \documentclass[12pt, a4paper]{report}

\input{template/main.tex}

\title{\BA{Title in Progress...}}
\author{Boris Andrews}
\affil{Mathematical Institute, University of Oxford}
\date{\today}


\begin{document}
    \pagenumbering{gobble}
    \maketitle
    
    
    \begin{abstract}
        Magnetic confinement reactors---in particular tokamaks---offer one of the most promising options for achieving practical nuclear fusion, with the potential to provide virtually limitless, clean energy. The theoretical and numerical modeling of tokamak plasmas is simultaneously an essential component of effective reactor design, and a great research barrier. Tokamak operational conditions exhibit comparatively low Knudsen numbers. Kinetic effects, including kinetic waves and instabilities, Landau damping, bump-on-tail instabilities and more, are therefore highly influential in tokamak plasma dynamics. Purely fluid models are inherently incapable of capturing these effects, whereas the high dimensionality in purely kinetic models render them practically intractable for most relevant purposes.

        We consider a $\delta\!f$ decomposition model, with a macroscopic fluid background and microscopic kinetic correction, both fully coupled to each other. A similar manner of discretization is proposed to that used in the recent \texttt{STRUPHY} code \cite{Holderied_Possanner_Wang_2021, Holderied_2022, Li_et_al_2023} with a finite-element model for the background and a pseudo-particle/PiC model for the correction.

        The fluid background satisfies the full, non-linear, resistive, compressible, Hall MHD equations. \cite{Laakmann_Hu_Farrell_2022} introduces finite-element(-in-space) implicit timesteppers for the incompressible analogue to this system with structure-preserving (SP) properties in the ideal case, alongside parameter-robust preconditioners. We show that these timesteppers can derive from a finite-element-in-time (FET) (and finite-element-in-space) interpretation. The benefits of this reformulation are discussed, including the derivation of timesteppers that are higher order in time, and the quantifiable dissipative SP properties in the non-ideal, resistive case.
        
        We discuss possible options for extending this FET approach to timesteppers for the compressible case.

        The kinetic corrections satisfy linearized Boltzmann equations. Using a Lénard--Bernstein collision operator, these take Fokker--Planck-like forms \cite{Fokker_1914, Planck_1917} wherein pseudo-particles in the numerical model obey the neoclassical transport equations, with particle-independent Brownian drift terms. This offers a rigorous methodology for incorporating collisions into the particle transport model, without coupling the equations of motions for each particle.
        
        Works by Chen, Chacón et al. \cite{Chen_Chacón_Barnes_2011, Chacón_Chen_Barnes_2013, Chen_Chacón_2014, Chen_Chacón_2015} have developed structure-preserving particle pushers for neoclassical transport in the Vlasov equations, derived from Crank--Nicolson integrators. We show these too can can derive from a FET interpretation, similarly offering potential extensions to higher-order-in-time particle pushers. The FET formulation is used also to consider how the stochastic drift terms can be incorporated into the pushers. Stochastic gyrokinetic expansions are also discussed.

        Different options for the numerical implementation of these schemes are considered.

        Due to the efficacy of FET in the development of SP timesteppers for both the fluid and kinetic component, we hope this approach will prove effective in the future for developing SP timesteppers for the full hybrid model. We hope this will give us the opportunity to incorporate previously inaccessible kinetic effects into the highly effective, modern, finite-element MHD models.
    \end{abstract}
    
    
    \newpage
    \tableofcontents
    
    
    \newpage
    \pagenumbering{arabic}
    %\linenumbers\renewcommand\thelinenumber{\color{black!50}\arabic{linenumber}}
            \input{0 - introduction/main.tex}
        \part{Research}
            \input{1 - low-noise PiC models/main.tex}
            \input{2 - kinetic component/main.tex}
            \input{3 - fluid component/main.tex}
            \input{4 - numerical implementation/main.tex}
        \part{Project Overview}
            \input{5 - research plan/main.tex}
            \input{6 - summary/main.tex}
    
    
    %\section{}
    \newpage
    \pagenumbering{gobble}
        \printbibliography


    \newpage
    \pagenumbering{roman}
    \appendix
        \part{Appendices}
            \input{8 - Hilbert complexes/main.tex}
            \input{9 - weak conservation proofs/main.tex}
\end{document}

    
    
    %\section{}
    \newpage
    \pagenumbering{gobble}
        \printbibliography


    \newpage
    \pagenumbering{roman}
    \appendix
        \part{Appendices}
            \documentclass[12pt, a4paper]{report}

\input{template/main.tex}

\title{\BA{Title in Progress...}}
\author{Boris Andrews}
\affil{Mathematical Institute, University of Oxford}
\date{\today}


\begin{document}
    \pagenumbering{gobble}
    \maketitle
    
    
    \begin{abstract}
        Magnetic confinement reactors---in particular tokamaks---offer one of the most promising options for achieving practical nuclear fusion, with the potential to provide virtually limitless, clean energy. The theoretical and numerical modeling of tokamak plasmas is simultaneously an essential component of effective reactor design, and a great research barrier. Tokamak operational conditions exhibit comparatively low Knudsen numbers. Kinetic effects, including kinetic waves and instabilities, Landau damping, bump-on-tail instabilities and more, are therefore highly influential in tokamak plasma dynamics. Purely fluid models are inherently incapable of capturing these effects, whereas the high dimensionality in purely kinetic models render them practically intractable for most relevant purposes.

        We consider a $\delta\!f$ decomposition model, with a macroscopic fluid background and microscopic kinetic correction, both fully coupled to each other. A similar manner of discretization is proposed to that used in the recent \texttt{STRUPHY} code \cite{Holderied_Possanner_Wang_2021, Holderied_2022, Li_et_al_2023} with a finite-element model for the background and a pseudo-particle/PiC model for the correction.

        The fluid background satisfies the full, non-linear, resistive, compressible, Hall MHD equations. \cite{Laakmann_Hu_Farrell_2022} introduces finite-element(-in-space) implicit timesteppers for the incompressible analogue to this system with structure-preserving (SP) properties in the ideal case, alongside parameter-robust preconditioners. We show that these timesteppers can derive from a finite-element-in-time (FET) (and finite-element-in-space) interpretation. The benefits of this reformulation are discussed, including the derivation of timesteppers that are higher order in time, and the quantifiable dissipative SP properties in the non-ideal, resistive case.
        
        We discuss possible options for extending this FET approach to timesteppers for the compressible case.

        The kinetic corrections satisfy linearized Boltzmann equations. Using a Lénard--Bernstein collision operator, these take Fokker--Planck-like forms \cite{Fokker_1914, Planck_1917} wherein pseudo-particles in the numerical model obey the neoclassical transport equations, with particle-independent Brownian drift terms. This offers a rigorous methodology for incorporating collisions into the particle transport model, without coupling the equations of motions for each particle.
        
        Works by Chen, Chacón et al. \cite{Chen_Chacón_Barnes_2011, Chacón_Chen_Barnes_2013, Chen_Chacón_2014, Chen_Chacón_2015} have developed structure-preserving particle pushers for neoclassical transport in the Vlasov equations, derived from Crank--Nicolson integrators. We show these too can can derive from a FET interpretation, similarly offering potential extensions to higher-order-in-time particle pushers. The FET formulation is used also to consider how the stochastic drift terms can be incorporated into the pushers. Stochastic gyrokinetic expansions are also discussed.

        Different options for the numerical implementation of these schemes are considered.

        Due to the efficacy of FET in the development of SP timesteppers for both the fluid and kinetic component, we hope this approach will prove effective in the future for developing SP timesteppers for the full hybrid model. We hope this will give us the opportunity to incorporate previously inaccessible kinetic effects into the highly effective, modern, finite-element MHD models.
    \end{abstract}
    
    
    \newpage
    \tableofcontents
    
    
    \newpage
    \pagenumbering{arabic}
    %\linenumbers\renewcommand\thelinenumber{\color{black!50}\arabic{linenumber}}
            \input{0 - introduction/main.tex}
        \part{Research}
            \input{1 - low-noise PiC models/main.tex}
            \input{2 - kinetic component/main.tex}
            \input{3 - fluid component/main.tex}
            \input{4 - numerical implementation/main.tex}
        \part{Project Overview}
            \input{5 - research plan/main.tex}
            \input{6 - summary/main.tex}
    
    
    %\section{}
    \newpage
    \pagenumbering{gobble}
        \printbibliography


    \newpage
    \pagenumbering{roman}
    \appendix
        \part{Appendices}
            \input{8 - Hilbert complexes/main.tex}
            \input{9 - weak conservation proofs/main.tex}
\end{document}

            \documentclass[12pt, a4paper]{report}

\input{template/main.tex}

\title{\BA{Title in Progress...}}
\author{Boris Andrews}
\affil{Mathematical Institute, University of Oxford}
\date{\today}


\begin{document}
    \pagenumbering{gobble}
    \maketitle
    
    
    \begin{abstract}
        Magnetic confinement reactors---in particular tokamaks---offer one of the most promising options for achieving practical nuclear fusion, with the potential to provide virtually limitless, clean energy. The theoretical and numerical modeling of tokamak plasmas is simultaneously an essential component of effective reactor design, and a great research barrier. Tokamak operational conditions exhibit comparatively low Knudsen numbers. Kinetic effects, including kinetic waves and instabilities, Landau damping, bump-on-tail instabilities and more, are therefore highly influential in tokamak plasma dynamics. Purely fluid models are inherently incapable of capturing these effects, whereas the high dimensionality in purely kinetic models render them practically intractable for most relevant purposes.

        We consider a $\delta\!f$ decomposition model, with a macroscopic fluid background and microscopic kinetic correction, both fully coupled to each other. A similar manner of discretization is proposed to that used in the recent \texttt{STRUPHY} code \cite{Holderied_Possanner_Wang_2021, Holderied_2022, Li_et_al_2023} with a finite-element model for the background and a pseudo-particle/PiC model for the correction.

        The fluid background satisfies the full, non-linear, resistive, compressible, Hall MHD equations. \cite{Laakmann_Hu_Farrell_2022} introduces finite-element(-in-space) implicit timesteppers for the incompressible analogue to this system with structure-preserving (SP) properties in the ideal case, alongside parameter-robust preconditioners. We show that these timesteppers can derive from a finite-element-in-time (FET) (and finite-element-in-space) interpretation. The benefits of this reformulation are discussed, including the derivation of timesteppers that are higher order in time, and the quantifiable dissipative SP properties in the non-ideal, resistive case.
        
        We discuss possible options for extending this FET approach to timesteppers for the compressible case.

        The kinetic corrections satisfy linearized Boltzmann equations. Using a Lénard--Bernstein collision operator, these take Fokker--Planck-like forms \cite{Fokker_1914, Planck_1917} wherein pseudo-particles in the numerical model obey the neoclassical transport equations, with particle-independent Brownian drift terms. This offers a rigorous methodology for incorporating collisions into the particle transport model, without coupling the equations of motions for each particle.
        
        Works by Chen, Chacón et al. \cite{Chen_Chacón_Barnes_2011, Chacón_Chen_Barnes_2013, Chen_Chacón_2014, Chen_Chacón_2015} have developed structure-preserving particle pushers for neoclassical transport in the Vlasov equations, derived from Crank--Nicolson integrators. We show these too can can derive from a FET interpretation, similarly offering potential extensions to higher-order-in-time particle pushers. The FET formulation is used also to consider how the stochastic drift terms can be incorporated into the pushers. Stochastic gyrokinetic expansions are also discussed.

        Different options for the numerical implementation of these schemes are considered.

        Due to the efficacy of FET in the development of SP timesteppers for both the fluid and kinetic component, we hope this approach will prove effective in the future for developing SP timesteppers for the full hybrid model. We hope this will give us the opportunity to incorporate previously inaccessible kinetic effects into the highly effective, modern, finite-element MHD models.
    \end{abstract}
    
    
    \newpage
    \tableofcontents
    
    
    \newpage
    \pagenumbering{arabic}
    %\linenumbers\renewcommand\thelinenumber{\color{black!50}\arabic{linenumber}}
            \input{0 - introduction/main.tex}
        \part{Research}
            \input{1 - low-noise PiC models/main.tex}
            \input{2 - kinetic component/main.tex}
            \input{3 - fluid component/main.tex}
            \input{4 - numerical implementation/main.tex}
        \part{Project Overview}
            \input{5 - research plan/main.tex}
            \input{6 - summary/main.tex}
    
    
    %\section{}
    \newpage
    \pagenumbering{gobble}
        \printbibliography


    \newpage
    \pagenumbering{roman}
    \appendix
        \part{Appendices}
            \input{8 - Hilbert complexes/main.tex}
            \input{9 - weak conservation proofs/main.tex}
\end{document}

\end{document}

            \documentclass[12pt, a4paper]{report}

\documentclass[12pt, a4paper]{report}

\input{template/main.tex}

\title{\BA{Title in Progress...}}
\author{Boris Andrews}
\affil{Mathematical Institute, University of Oxford}
\date{\today}


\begin{document}
    \pagenumbering{gobble}
    \maketitle
    
    
    \begin{abstract}
        Magnetic confinement reactors---in particular tokamaks---offer one of the most promising options for achieving practical nuclear fusion, with the potential to provide virtually limitless, clean energy. The theoretical and numerical modeling of tokamak plasmas is simultaneously an essential component of effective reactor design, and a great research barrier. Tokamak operational conditions exhibit comparatively low Knudsen numbers. Kinetic effects, including kinetic waves and instabilities, Landau damping, bump-on-tail instabilities and more, are therefore highly influential in tokamak plasma dynamics. Purely fluid models are inherently incapable of capturing these effects, whereas the high dimensionality in purely kinetic models render them practically intractable for most relevant purposes.

        We consider a $\delta\!f$ decomposition model, with a macroscopic fluid background and microscopic kinetic correction, both fully coupled to each other. A similar manner of discretization is proposed to that used in the recent \texttt{STRUPHY} code \cite{Holderied_Possanner_Wang_2021, Holderied_2022, Li_et_al_2023} with a finite-element model for the background and a pseudo-particle/PiC model for the correction.

        The fluid background satisfies the full, non-linear, resistive, compressible, Hall MHD equations. \cite{Laakmann_Hu_Farrell_2022} introduces finite-element(-in-space) implicit timesteppers for the incompressible analogue to this system with structure-preserving (SP) properties in the ideal case, alongside parameter-robust preconditioners. We show that these timesteppers can derive from a finite-element-in-time (FET) (and finite-element-in-space) interpretation. The benefits of this reformulation are discussed, including the derivation of timesteppers that are higher order in time, and the quantifiable dissipative SP properties in the non-ideal, resistive case.
        
        We discuss possible options for extending this FET approach to timesteppers for the compressible case.

        The kinetic corrections satisfy linearized Boltzmann equations. Using a Lénard--Bernstein collision operator, these take Fokker--Planck-like forms \cite{Fokker_1914, Planck_1917} wherein pseudo-particles in the numerical model obey the neoclassical transport equations, with particle-independent Brownian drift terms. This offers a rigorous methodology for incorporating collisions into the particle transport model, without coupling the equations of motions for each particle.
        
        Works by Chen, Chacón et al. \cite{Chen_Chacón_Barnes_2011, Chacón_Chen_Barnes_2013, Chen_Chacón_2014, Chen_Chacón_2015} have developed structure-preserving particle pushers for neoclassical transport in the Vlasov equations, derived from Crank--Nicolson integrators. We show these too can can derive from a FET interpretation, similarly offering potential extensions to higher-order-in-time particle pushers. The FET formulation is used also to consider how the stochastic drift terms can be incorporated into the pushers. Stochastic gyrokinetic expansions are also discussed.

        Different options for the numerical implementation of these schemes are considered.

        Due to the efficacy of FET in the development of SP timesteppers for both the fluid and kinetic component, we hope this approach will prove effective in the future for developing SP timesteppers for the full hybrid model. We hope this will give us the opportunity to incorporate previously inaccessible kinetic effects into the highly effective, modern, finite-element MHD models.
    \end{abstract}
    
    
    \newpage
    \tableofcontents
    
    
    \newpage
    \pagenumbering{arabic}
    %\linenumbers\renewcommand\thelinenumber{\color{black!50}\arabic{linenumber}}
            \input{0 - introduction/main.tex}
        \part{Research}
            \input{1 - low-noise PiC models/main.tex}
            \input{2 - kinetic component/main.tex}
            \input{3 - fluid component/main.tex}
            \input{4 - numerical implementation/main.tex}
        \part{Project Overview}
            \input{5 - research plan/main.tex}
            \input{6 - summary/main.tex}
    
    
    %\section{}
    \newpage
    \pagenumbering{gobble}
        \printbibliography


    \newpage
    \pagenumbering{roman}
    \appendix
        \part{Appendices}
            \input{8 - Hilbert complexes/main.tex}
            \input{9 - weak conservation proofs/main.tex}
\end{document}


\title{\BA{Title in Progress...}}
\author{Boris Andrews}
\affil{Mathematical Institute, University of Oxford}
\date{\today}


\begin{document}
    \pagenumbering{gobble}
    \maketitle
    
    
    \begin{abstract}
        Magnetic confinement reactors---in particular tokamaks---offer one of the most promising options for achieving practical nuclear fusion, with the potential to provide virtually limitless, clean energy. The theoretical and numerical modeling of tokamak plasmas is simultaneously an essential component of effective reactor design, and a great research barrier. Tokamak operational conditions exhibit comparatively low Knudsen numbers. Kinetic effects, including kinetic waves and instabilities, Landau damping, bump-on-tail instabilities and more, are therefore highly influential in tokamak plasma dynamics. Purely fluid models are inherently incapable of capturing these effects, whereas the high dimensionality in purely kinetic models render them practically intractable for most relevant purposes.

        We consider a $\delta\!f$ decomposition model, with a macroscopic fluid background and microscopic kinetic correction, both fully coupled to each other. A similar manner of discretization is proposed to that used in the recent \texttt{STRUPHY} code \cite{Holderied_Possanner_Wang_2021, Holderied_2022, Li_et_al_2023} with a finite-element model for the background and a pseudo-particle/PiC model for the correction.

        The fluid background satisfies the full, non-linear, resistive, compressible, Hall MHD equations. \cite{Laakmann_Hu_Farrell_2022} introduces finite-element(-in-space) implicit timesteppers for the incompressible analogue to this system with structure-preserving (SP) properties in the ideal case, alongside parameter-robust preconditioners. We show that these timesteppers can derive from a finite-element-in-time (FET) (and finite-element-in-space) interpretation. The benefits of this reformulation are discussed, including the derivation of timesteppers that are higher order in time, and the quantifiable dissipative SP properties in the non-ideal, resistive case.
        
        We discuss possible options for extending this FET approach to timesteppers for the compressible case.

        The kinetic corrections satisfy linearized Boltzmann equations. Using a Lénard--Bernstein collision operator, these take Fokker--Planck-like forms \cite{Fokker_1914, Planck_1917} wherein pseudo-particles in the numerical model obey the neoclassical transport equations, with particle-independent Brownian drift terms. This offers a rigorous methodology for incorporating collisions into the particle transport model, without coupling the equations of motions for each particle.
        
        Works by Chen, Chacón et al. \cite{Chen_Chacón_Barnes_2011, Chacón_Chen_Barnes_2013, Chen_Chacón_2014, Chen_Chacón_2015} have developed structure-preserving particle pushers for neoclassical transport in the Vlasov equations, derived from Crank--Nicolson integrators. We show these too can can derive from a FET interpretation, similarly offering potential extensions to higher-order-in-time particle pushers. The FET formulation is used also to consider how the stochastic drift terms can be incorporated into the pushers. Stochastic gyrokinetic expansions are also discussed.

        Different options for the numerical implementation of these schemes are considered.

        Due to the efficacy of FET in the development of SP timesteppers for both the fluid and kinetic component, we hope this approach will prove effective in the future for developing SP timesteppers for the full hybrid model. We hope this will give us the opportunity to incorporate previously inaccessible kinetic effects into the highly effective, modern, finite-element MHD models.
    \end{abstract}
    
    
    \newpage
    \tableofcontents
    
    
    \newpage
    \pagenumbering{arabic}
    %\linenumbers\renewcommand\thelinenumber{\color{black!50}\arabic{linenumber}}
            \documentclass[12pt, a4paper]{report}

\input{template/main.tex}

\title{\BA{Title in Progress...}}
\author{Boris Andrews}
\affil{Mathematical Institute, University of Oxford}
\date{\today}


\begin{document}
    \pagenumbering{gobble}
    \maketitle
    
    
    \begin{abstract}
        Magnetic confinement reactors---in particular tokamaks---offer one of the most promising options for achieving practical nuclear fusion, with the potential to provide virtually limitless, clean energy. The theoretical and numerical modeling of tokamak plasmas is simultaneously an essential component of effective reactor design, and a great research barrier. Tokamak operational conditions exhibit comparatively low Knudsen numbers. Kinetic effects, including kinetic waves and instabilities, Landau damping, bump-on-tail instabilities and more, are therefore highly influential in tokamak plasma dynamics. Purely fluid models are inherently incapable of capturing these effects, whereas the high dimensionality in purely kinetic models render them practically intractable for most relevant purposes.

        We consider a $\delta\!f$ decomposition model, with a macroscopic fluid background and microscopic kinetic correction, both fully coupled to each other. A similar manner of discretization is proposed to that used in the recent \texttt{STRUPHY} code \cite{Holderied_Possanner_Wang_2021, Holderied_2022, Li_et_al_2023} with a finite-element model for the background and a pseudo-particle/PiC model for the correction.

        The fluid background satisfies the full, non-linear, resistive, compressible, Hall MHD equations. \cite{Laakmann_Hu_Farrell_2022} introduces finite-element(-in-space) implicit timesteppers for the incompressible analogue to this system with structure-preserving (SP) properties in the ideal case, alongside parameter-robust preconditioners. We show that these timesteppers can derive from a finite-element-in-time (FET) (and finite-element-in-space) interpretation. The benefits of this reformulation are discussed, including the derivation of timesteppers that are higher order in time, and the quantifiable dissipative SP properties in the non-ideal, resistive case.
        
        We discuss possible options for extending this FET approach to timesteppers for the compressible case.

        The kinetic corrections satisfy linearized Boltzmann equations. Using a Lénard--Bernstein collision operator, these take Fokker--Planck-like forms \cite{Fokker_1914, Planck_1917} wherein pseudo-particles in the numerical model obey the neoclassical transport equations, with particle-independent Brownian drift terms. This offers a rigorous methodology for incorporating collisions into the particle transport model, without coupling the equations of motions for each particle.
        
        Works by Chen, Chacón et al. \cite{Chen_Chacón_Barnes_2011, Chacón_Chen_Barnes_2013, Chen_Chacón_2014, Chen_Chacón_2015} have developed structure-preserving particle pushers for neoclassical transport in the Vlasov equations, derived from Crank--Nicolson integrators. We show these too can can derive from a FET interpretation, similarly offering potential extensions to higher-order-in-time particle pushers. The FET formulation is used also to consider how the stochastic drift terms can be incorporated into the pushers. Stochastic gyrokinetic expansions are also discussed.

        Different options for the numerical implementation of these schemes are considered.

        Due to the efficacy of FET in the development of SP timesteppers for both the fluid and kinetic component, we hope this approach will prove effective in the future for developing SP timesteppers for the full hybrid model. We hope this will give us the opportunity to incorporate previously inaccessible kinetic effects into the highly effective, modern, finite-element MHD models.
    \end{abstract}
    
    
    \newpage
    \tableofcontents
    
    
    \newpage
    \pagenumbering{arabic}
    %\linenumbers\renewcommand\thelinenumber{\color{black!50}\arabic{linenumber}}
            \input{0 - introduction/main.tex}
        \part{Research}
            \input{1 - low-noise PiC models/main.tex}
            \input{2 - kinetic component/main.tex}
            \input{3 - fluid component/main.tex}
            \input{4 - numerical implementation/main.tex}
        \part{Project Overview}
            \input{5 - research plan/main.tex}
            \input{6 - summary/main.tex}
    
    
    %\section{}
    \newpage
    \pagenumbering{gobble}
        \printbibliography


    \newpage
    \pagenumbering{roman}
    \appendix
        \part{Appendices}
            \input{8 - Hilbert complexes/main.tex}
            \input{9 - weak conservation proofs/main.tex}
\end{document}

        \part{Research}
            \documentclass[12pt, a4paper]{report}

\input{template/main.tex}

\title{\BA{Title in Progress...}}
\author{Boris Andrews}
\affil{Mathematical Institute, University of Oxford}
\date{\today}


\begin{document}
    \pagenumbering{gobble}
    \maketitle
    
    
    \begin{abstract}
        Magnetic confinement reactors---in particular tokamaks---offer one of the most promising options for achieving practical nuclear fusion, with the potential to provide virtually limitless, clean energy. The theoretical and numerical modeling of tokamak plasmas is simultaneously an essential component of effective reactor design, and a great research barrier. Tokamak operational conditions exhibit comparatively low Knudsen numbers. Kinetic effects, including kinetic waves and instabilities, Landau damping, bump-on-tail instabilities and more, are therefore highly influential in tokamak plasma dynamics. Purely fluid models are inherently incapable of capturing these effects, whereas the high dimensionality in purely kinetic models render them practically intractable for most relevant purposes.

        We consider a $\delta\!f$ decomposition model, with a macroscopic fluid background and microscopic kinetic correction, both fully coupled to each other. A similar manner of discretization is proposed to that used in the recent \texttt{STRUPHY} code \cite{Holderied_Possanner_Wang_2021, Holderied_2022, Li_et_al_2023} with a finite-element model for the background and a pseudo-particle/PiC model for the correction.

        The fluid background satisfies the full, non-linear, resistive, compressible, Hall MHD equations. \cite{Laakmann_Hu_Farrell_2022} introduces finite-element(-in-space) implicit timesteppers for the incompressible analogue to this system with structure-preserving (SP) properties in the ideal case, alongside parameter-robust preconditioners. We show that these timesteppers can derive from a finite-element-in-time (FET) (and finite-element-in-space) interpretation. The benefits of this reformulation are discussed, including the derivation of timesteppers that are higher order in time, and the quantifiable dissipative SP properties in the non-ideal, resistive case.
        
        We discuss possible options for extending this FET approach to timesteppers for the compressible case.

        The kinetic corrections satisfy linearized Boltzmann equations. Using a Lénard--Bernstein collision operator, these take Fokker--Planck-like forms \cite{Fokker_1914, Planck_1917} wherein pseudo-particles in the numerical model obey the neoclassical transport equations, with particle-independent Brownian drift terms. This offers a rigorous methodology for incorporating collisions into the particle transport model, without coupling the equations of motions for each particle.
        
        Works by Chen, Chacón et al. \cite{Chen_Chacón_Barnes_2011, Chacón_Chen_Barnes_2013, Chen_Chacón_2014, Chen_Chacón_2015} have developed structure-preserving particle pushers for neoclassical transport in the Vlasov equations, derived from Crank--Nicolson integrators. We show these too can can derive from a FET interpretation, similarly offering potential extensions to higher-order-in-time particle pushers. The FET formulation is used also to consider how the stochastic drift terms can be incorporated into the pushers. Stochastic gyrokinetic expansions are also discussed.

        Different options for the numerical implementation of these schemes are considered.

        Due to the efficacy of FET in the development of SP timesteppers for both the fluid and kinetic component, we hope this approach will prove effective in the future for developing SP timesteppers for the full hybrid model. We hope this will give us the opportunity to incorporate previously inaccessible kinetic effects into the highly effective, modern, finite-element MHD models.
    \end{abstract}
    
    
    \newpage
    \tableofcontents
    
    
    \newpage
    \pagenumbering{arabic}
    %\linenumbers\renewcommand\thelinenumber{\color{black!50}\arabic{linenumber}}
            \input{0 - introduction/main.tex}
        \part{Research}
            \input{1 - low-noise PiC models/main.tex}
            \input{2 - kinetic component/main.tex}
            \input{3 - fluid component/main.tex}
            \input{4 - numerical implementation/main.tex}
        \part{Project Overview}
            \input{5 - research plan/main.tex}
            \input{6 - summary/main.tex}
    
    
    %\section{}
    \newpage
    \pagenumbering{gobble}
        \printbibliography


    \newpage
    \pagenumbering{roman}
    \appendix
        \part{Appendices}
            \input{8 - Hilbert complexes/main.tex}
            \input{9 - weak conservation proofs/main.tex}
\end{document}

            \documentclass[12pt, a4paper]{report}

\input{template/main.tex}

\title{\BA{Title in Progress...}}
\author{Boris Andrews}
\affil{Mathematical Institute, University of Oxford}
\date{\today}


\begin{document}
    \pagenumbering{gobble}
    \maketitle
    
    
    \begin{abstract}
        Magnetic confinement reactors---in particular tokamaks---offer one of the most promising options for achieving practical nuclear fusion, with the potential to provide virtually limitless, clean energy. The theoretical and numerical modeling of tokamak plasmas is simultaneously an essential component of effective reactor design, and a great research barrier. Tokamak operational conditions exhibit comparatively low Knudsen numbers. Kinetic effects, including kinetic waves and instabilities, Landau damping, bump-on-tail instabilities and more, are therefore highly influential in tokamak plasma dynamics. Purely fluid models are inherently incapable of capturing these effects, whereas the high dimensionality in purely kinetic models render them practically intractable for most relevant purposes.

        We consider a $\delta\!f$ decomposition model, with a macroscopic fluid background and microscopic kinetic correction, both fully coupled to each other. A similar manner of discretization is proposed to that used in the recent \texttt{STRUPHY} code \cite{Holderied_Possanner_Wang_2021, Holderied_2022, Li_et_al_2023} with a finite-element model for the background and a pseudo-particle/PiC model for the correction.

        The fluid background satisfies the full, non-linear, resistive, compressible, Hall MHD equations. \cite{Laakmann_Hu_Farrell_2022} introduces finite-element(-in-space) implicit timesteppers for the incompressible analogue to this system with structure-preserving (SP) properties in the ideal case, alongside parameter-robust preconditioners. We show that these timesteppers can derive from a finite-element-in-time (FET) (and finite-element-in-space) interpretation. The benefits of this reformulation are discussed, including the derivation of timesteppers that are higher order in time, and the quantifiable dissipative SP properties in the non-ideal, resistive case.
        
        We discuss possible options for extending this FET approach to timesteppers for the compressible case.

        The kinetic corrections satisfy linearized Boltzmann equations. Using a Lénard--Bernstein collision operator, these take Fokker--Planck-like forms \cite{Fokker_1914, Planck_1917} wherein pseudo-particles in the numerical model obey the neoclassical transport equations, with particle-independent Brownian drift terms. This offers a rigorous methodology for incorporating collisions into the particle transport model, without coupling the equations of motions for each particle.
        
        Works by Chen, Chacón et al. \cite{Chen_Chacón_Barnes_2011, Chacón_Chen_Barnes_2013, Chen_Chacón_2014, Chen_Chacón_2015} have developed structure-preserving particle pushers for neoclassical transport in the Vlasov equations, derived from Crank--Nicolson integrators. We show these too can can derive from a FET interpretation, similarly offering potential extensions to higher-order-in-time particle pushers. The FET formulation is used also to consider how the stochastic drift terms can be incorporated into the pushers. Stochastic gyrokinetic expansions are also discussed.

        Different options for the numerical implementation of these schemes are considered.

        Due to the efficacy of FET in the development of SP timesteppers for both the fluid and kinetic component, we hope this approach will prove effective in the future for developing SP timesteppers for the full hybrid model. We hope this will give us the opportunity to incorporate previously inaccessible kinetic effects into the highly effective, modern, finite-element MHD models.
    \end{abstract}
    
    
    \newpage
    \tableofcontents
    
    
    \newpage
    \pagenumbering{arabic}
    %\linenumbers\renewcommand\thelinenumber{\color{black!50}\arabic{linenumber}}
            \input{0 - introduction/main.tex}
        \part{Research}
            \input{1 - low-noise PiC models/main.tex}
            \input{2 - kinetic component/main.tex}
            \input{3 - fluid component/main.tex}
            \input{4 - numerical implementation/main.tex}
        \part{Project Overview}
            \input{5 - research plan/main.tex}
            \input{6 - summary/main.tex}
    
    
    %\section{}
    \newpage
    \pagenumbering{gobble}
        \printbibliography


    \newpage
    \pagenumbering{roman}
    \appendix
        \part{Appendices}
            \input{8 - Hilbert complexes/main.tex}
            \input{9 - weak conservation proofs/main.tex}
\end{document}

            \documentclass[12pt, a4paper]{report}

\input{template/main.tex}

\title{\BA{Title in Progress...}}
\author{Boris Andrews}
\affil{Mathematical Institute, University of Oxford}
\date{\today}


\begin{document}
    \pagenumbering{gobble}
    \maketitle
    
    
    \begin{abstract}
        Magnetic confinement reactors---in particular tokamaks---offer one of the most promising options for achieving practical nuclear fusion, with the potential to provide virtually limitless, clean energy. The theoretical and numerical modeling of tokamak plasmas is simultaneously an essential component of effective reactor design, and a great research barrier. Tokamak operational conditions exhibit comparatively low Knudsen numbers. Kinetic effects, including kinetic waves and instabilities, Landau damping, bump-on-tail instabilities and more, are therefore highly influential in tokamak plasma dynamics. Purely fluid models are inherently incapable of capturing these effects, whereas the high dimensionality in purely kinetic models render them practically intractable for most relevant purposes.

        We consider a $\delta\!f$ decomposition model, with a macroscopic fluid background and microscopic kinetic correction, both fully coupled to each other. A similar manner of discretization is proposed to that used in the recent \texttt{STRUPHY} code \cite{Holderied_Possanner_Wang_2021, Holderied_2022, Li_et_al_2023} with a finite-element model for the background and a pseudo-particle/PiC model for the correction.

        The fluid background satisfies the full, non-linear, resistive, compressible, Hall MHD equations. \cite{Laakmann_Hu_Farrell_2022} introduces finite-element(-in-space) implicit timesteppers for the incompressible analogue to this system with structure-preserving (SP) properties in the ideal case, alongside parameter-robust preconditioners. We show that these timesteppers can derive from a finite-element-in-time (FET) (and finite-element-in-space) interpretation. The benefits of this reformulation are discussed, including the derivation of timesteppers that are higher order in time, and the quantifiable dissipative SP properties in the non-ideal, resistive case.
        
        We discuss possible options for extending this FET approach to timesteppers for the compressible case.

        The kinetic corrections satisfy linearized Boltzmann equations. Using a Lénard--Bernstein collision operator, these take Fokker--Planck-like forms \cite{Fokker_1914, Planck_1917} wherein pseudo-particles in the numerical model obey the neoclassical transport equations, with particle-independent Brownian drift terms. This offers a rigorous methodology for incorporating collisions into the particle transport model, without coupling the equations of motions for each particle.
        
        Works by Chen, Chacón et al. \cite{Chen_Chacón_Barnes_2011, Chacón_Chen_Barnes_2013, Chen_Chacón_2014, Chen_Chacón_2015} have developed structure-preserving particle pushers for neoclassical transport in the Vlasov equations, derived from Crank--Nicolson integrators. We show these too can can derive from a FET interpretation, similarly offering potential extensions to higher-order-in-time particle pushers. The FET formulation is used also to consider how the stochastic drift terms can be incorporated into the pushers. Stochastic gyrokinetic expansions are also discussed.

        Different options for the numerical implementation of these schemes are considered.

        Due to the efficacy of FET in the development of SP timesteppers for both the fluid and kinetic component, we hope this approach will prove effective in the future for developing SP timesteppers for the full hybrid model. We hope this will give us the opportunity to incorporate previously inaccessible kinetic effects into the highly effective, modern, finite-element MHD models.
    \end{abstract}
    
    
    \newpage
    \tableofcontents
    
    
    \newpage
    \pagenumbering{arabic}
    %\linenumbers\renewcommand\thelinenumber{\color{black!50}\arabic{linenumber}}
            \input{0 - introduction/main.tex}
        \part{Research}
            \input{1 - low-noise PiC models/main.tex}
            \input{2 - kinetic component/main.tex}
            \input{3 - fluid component/main.tex}
            \input{4 - numerical implementation/main.tex}
        \part{Project Overview}
            \input{5 - research plan/main.tex}
            \input{6 - summary/main.tex}
    
    
    %\section{}
    \newpage
    \pagenumbering{gobble}
        \printbibliography


    \newpage
    \pagenumbering{roman}
    \appendix
        \part{Appendices}
            \input{8 - Hilbert complexes/main.tex}
            \input{9 - weak conservation proofs/main.tex}
\end{document}

            \documentclass[12pt, a4paper]{report}

\input{template/main.tex}

\title{\BA{Title in Progress...}}
\author{Boris Andrews}
\affil{Mathematical Institute, University of Oxford}
\date{\today}


\begin{document}
    \pagenumbering{gobble}
    \maketitle
    
    
    \begin{abstract}
        Magnetic confinement reactors---in particular tokamaks---offer one of the most promising options for achieving practical nuclear fusion, with the potential to provide virtually limitless, clean energy. The theoretical and numerical modeling of tokamak plasmas is simultaneously an essential component of effective reactor design, and a great research barrier. Tokamak operational conditions exhibit comparatively low Knudsen numbers. Kinetic effects, including kinetic waves and instabilities, Landau damping, bump-on-tail instabilities and more, are therefore highly influential in tokamak plasma dynamics. Purely fluid models are inherently incapable of capturing these effects, whereas the high dimensionality in purely kinetic models render them practically intractable for most relevant purposes.

        We consider a $\delta\!f$ decomposition model, with a macroscopic fluid background and microscopic kinetic correction, both fully coupled to each other. A similar manner of discretization is proposed to that used in the recent \texttt{STRUPHY} code \cite{Holderied_Possanner_Wang_2021, Holderied_2022, Li_et_al_2023} with a finite-element model for the background and a pseudo-particle/PiC model for the correction.

        The fluid background satisfies the full, non-linear, resistive, compressible, Hall MHD equations. \cite{Laakmann_Hu_Farrell_2022} introduces finite-element(-in-space) implicit timesteppers for the incompressible analogue to this system with structure-preserving (SP) properties in the ideal case, alongside parameter-robust preconditioners. We show that these timesteppers can derive from a finite-element-in-time (FET) (and finite-element-in-space) interpretation. The benefits of this reformulation are discussed, including the derivation of timesteppers that are higher order in time, and the quantifiable dissipative SP properties in the non-ideal, resistive case.
        
        We discuss possible options for extending this FET approach to timesteppers for the compressible case.

        The kinetic corrections satisfy linearized Boltzmann equations. Using a Lénard--Bernstein collision operator, these take Fokker--Planck-like forms \cite{Fokker_1914, Planck_1917} wherein pseudo-particles in the numerical model obey the neoclassical transport equations, with particle-independent Brownian drift terms. This offers a rigorous methodology for incorporating collisions into the particle transport model, without coupling the equations of motions for each particle.
        
        Works by Chen, Chacón et al. \cite{Chen_Chacón_Barnes_2011, Chacón_Chen_Barnes_2013, Chen_Chacón_2014, Chen_Chacón_2015} have developed structure-preserving particle pushers for neoclassical transport in the Vlasov equations, derived from Crank--Nicolson integrators. We show these too can can derive from a FET interpretation, similarly offering potential extensions to higher-order-in-time particle pushers. The FET formulation is used also to consider how the stochastic drift terms can be incorporated into the pushers. Stochastic gyrokinetic expansions are also discussed.

        Different options for the numerical implementation of these schemes are considered.

        Due to the efficacy of FET in the development of SP timesteppers for both the fluid and kinetic component, we hope this approach will prove effective in the future for developing SP timesteppers for the full hybrid model. We hope this will give us the opportunity to incorporate previously inaccessible kinetic effects into the highly effective, modern, finite-element MHD models.
    \end{abstract}
    
    
    \newpage
    \tableofcontents
    
    
    \newpage
    \pagenumbering{arabic}
    %\linenumbers\renewcommand\thelinenumber{\color{black!50}\arabic{linenumber}}
            \input{0 - introduction/main.tex}
        \part{Research}
            \input{1 - low-noise PiC models/main.tex}
            \input{2 - kinetic component/main.tex}
            \input{3 - fluid component/main.tex}
            \input{4 - numerical implementation/main.tex}
        \part{Project Overview}
            \input{5 - research plan/main.tex}
            \input{6 - summary/main.tex}
    
    
    %\section{}
    \newpage
    \pagenumbering{gobble}
        \printbibliography


    \newpage
    \pagenumbering{roman}
    \appendix
        \part{Appendices}
            \input{8 - Hilbert complexes/main.tex}
            \input{9 - weak conservation proofs/main.tex}
\end{document}

        \part{Project Overview}
            \documentclass[12pt, a4paper]{report}

\input{template/main.tex}

\title{\BA{Title in Progress...}}
\author{Boris Andrews}
\affil{Mathematical Institute, University of Oxford}
\date{\today}


\begin{document}
    \pagenumbering{gobble}
    \maketitle
    
    
    \begin{abstract}
        Magnetic confinement reactors---in particular tokamaks---offer one of the most promising options for achieving practical nuclear fusion, with the potential to provide virtually limitless, clean energy. The theoretical and numerical modeling of tokamak plasmas is simultaneously an essential component of effective reactor design, and a great research barrier. Tokamak operational conditions exhibit comparatively low Knudsen numbers. Kinetic effects, including kinetic waves and instabilities, Landau damping, bump-on-tail instabilities and more, are therefore highly influential in tokamak plasma dynamics. Purely fluid models are inherently incapable of capturing these effects, whereas the high dimensionality in purely kinetic models render them practically intractable for most relevant purposes.

        We consider a $\delta\!f$ decomposition model, with a macroscopic fluid background and microscopic kinetic correction, both fully coupled to each other. A similar manner of discretization is proposed to that used in the recent \texttt{STRUPHY} code \cite{Holderied_Possanner_Wang_2021, Holderied_2022, Li_et_al_2023} with a finite-element model for the background and a pseudo-particle/PiC model for the correction.

        The fluid background satisfies the full, non-linear, resistive, compressible, Hall MHD equations. \cite{Laakmann_Hu_Farrell_2022} introduces finite-element(-in-space) implicit timesteppers for the incompressible analogue to this system with structure-preserving (SP) properties in the ideal case, alongside parameter-robust preconditioners. We show that these timesteppers can derive from a finite-element-in-time (FET) (and finite-element-in-space) interpretation. The benefits of this reformulation are discussed, including the derivation of timesteppers that are higher order in time, and the quantifiable dissipative SP properties in the non-ideal, resistive case.
        
        We discuss possible options for extending this FET approach to timesteppers for the compressible case.

        The kinetic corrections satisfy linearized Boltzmann equations. Using a Lénard--Bernstein collision operator, these take Fokker--Planck-like forms \cite{Fokker_1914, Planck_1917} wherein pseudo-particles in the numerical model obey the neoclassical transport equations, with particle-independent Brownian drift terms. This offers a rigorous methodology for incorporating collisions into the particle transport model, without coupling the equations of motions for each particle.
        
        Works by Chen, Chacón et al. \cite{Chen_Chacón_Barnes_2011, Chacón_Chen_Barnes_2013, Chen_Chacón_2014, Chen_Chacón_2015} have developed structure-preserving particle pushers for neoclassical transport in the Vlasov equations, derived from Crank--Nicolson integrators. We show these too can can derive from a FET interpretation, similarly offering potential extensions to higher-order-in-time particle pushers. The FET formulation is used also to consider how the stochastic drift terms can be incorporated into the pushers. Stochastic gyrokinetic expansions are also discussed.

        Different options for the numerical implementation of these schemes are considered.

        Due to the efficacy of FET in the development of SP timesteppers for both the fluid and kinetic component, we hope this approach will prove effective in the future for developing SP timesteppers for the full hybrid model. We hope this will give us the opportunity to incorporate previously inaccessible kinetic effects into the highly effective, modern, finite-element MHD models.
    \end{abstract}
    
    
    \newpage
    \tableofcontents
    
    
    \newpage
    \pagenumbering{arabic}
    %\linenumbers\renewcommand\thelinenumber{\color{black!50}\arabic{linenumber}}
            \input{0 - introduction/main.tex}
        \part{Research}
            \input{1 - low-noise PiC models/main.tex}
            \input{2 - kinetic component/main.tex}
            \input{3 - fluid component/main.tex}
            \input{4 - numerical implementation/main.tex}
        \part{Project Overview}
            \input{5 - research plan/main.tex}
            \input{6 - summary/main.tex}
    
    
    %\section{}
    \newpage
    \pagenumbering{gobble}
        \printbibliography


    \newpage
    \pagenumbering{roman}
    \appendix
        \part{Appendices}
            \input{8 - Hilbert complexes/main.tex}
            \input{9 - weak conservation proofs/main.tex}
\end{document}

            \documentclass[12pt, a4paper]{report}

\input{template/main.tex}

\title{\BA{Title in Progress...}}
\author{Boris Andrews}
\affil{Mathematical Institute, University of Oxford}
\date{\today}


\begin{document}
    \pagenumbering{gobble}
    \maketitle
    
    
    \begin{abstract}
        Magnetic confinement reactors---in particular tokamaks---offer one of the most promising options for achieving practical nuclear fusion, with the potential to provide virtually limitless, clean energy. The theoretical and numerical modeling of tokamak plasmas is simultaneously an essential component of effective reactor design, and a great research barrier. Tokamak operational conditions exhibit comparatively low Knudsen numbers. Kinetic effects, including kinetic waves and instabilities, Landau damping, bump-on-tail instabilities and more, are therefore highly influential in tokamak plasma dynamics. Purely fluid models are inherently incapable of capturing these effects, whereas the high dimensionality in purely kinetic models render them practically intractable for most relevant purposes.

        We consider a $\delta\!f$ decomposition model, with a macroscopic fluid background and microscopic kinetic correction, both fully coupled to each other. A similar manner of discretization is proposed to that used in the recent \texttt{STRUPHY} code \cite{Holderied_Possanner_Wang_2021, Holderied_2022, Li_et_al_2023} with a finite-element model for the background and a pseudo-particle/PiC model for the correction.

        The fluid background satisfies the full, non-linear, resistive, compressible, Hall MHD equations. \cite{Laakmann_Hu_Farrell_2022} introduces finite-element(-in-space) implicit timesteppers for the incompressible analogue to this system with structure-preserving (SP) properties in the ideal case, alongside parameter-robust preconditioners. We show that these timesteppers can derive from a finite-element-in-time (FET) (and finite-element-in-space) interpretation. The benefits of this reformulation are discussed, including the derivation of timesteppers that are higher order in time, and the quantifiable dissipative SP properties in the non-ideal, resistive case.
        
        We discuss possible options for extending this FET approach to timesteppers for the compressible case.

        The kinetic corrections satisfy linearized Boltzmann equations. Using a Lénard--Bernstein collision operator, these take Fokker--Planck-like forms \cite{Fokker_1914, Planck_1917} wherein pseudo-particles in the numerical model obey the neoclassical transport equations, with particle-independent Brownian drift terms. This offers a rigorous methodology for incorporating collisions into the particle transport model, without coupling the equations of motions for each particle.
        
        Works by Chen, Chacón et al. \cite{Chen_Chacón_Barnes_2011, Chacón_Chen_Barnes_2013, Chen_Chacón_2014, Chen_Chacón_2015} have developed structure-preserving particle pushers for neoclassical transport in the Vlasov equations, derived from Crank--Nicolson integrators. We show these too can can derive from a FET interpretation, similarly offering potential extensions to higher-order-in-time particle pushers. The FET formulation is used also to consider how the stochastic drift terms can be incorporated into the pushers. Stochastic gyrokinetic expansions are also discussed.

        Different options for the numerical implementation of these schemes are considered.

        Due to the efficacy of FET in the development of SP timesteppers for both the fluid and kinetic component, we hope this approach will prove effective in the future for developing SP timesteppers for the full hybrid model. We hope this will give us the opportunity to incorporate previously inaccessible kinetic effects into the highly effective, modern, finite-element MHD models.
    \end{abstract}
    
    
    \newpage
    \tableofcontents
    
    
    \newpage
    \pagenumbering{arabic}
    %\linenumbers\renewcommand\thelinenumber{\color{black!50}\arabic{linenumber}}
            \input{0 - introduction/main.tex}
        \part{Research}
            \input{1 - low-noise PiC models/main.tex}
            \input{2 - kinetic component/main.tex}
            \input{3 - fluid component/main.tex}
            \input{4 - numerical implementation/main.tex}
        \part{Project Overview}
            \input{5 - research plan/main.tex}
            \input{6 - summary/main.tex}
    
    
    %\section{}
    \newpage
    \pagenumbering{gobble}
        \printbibliography


    \newpage
    \pagenumbering{roman}
    \appendix
        \part{Appendices}
            \input{8 - Hilbert complexes/main.tex}
            \input{9 - weak conservation proofs/main.tex}
\end{document}

    
    
    %\section{}
    \newpage
    \pagenumbering{gobble}
        \printbibliography


    \newpage
    \pagenumbering{roman}
    \appendix
        \part{Appendices}
            \documentclass[12pt, a4paper]{report}

\input{template/main.tex}

\title{\BA{Title in Progress...}}
\author{Boris Andrews}
\affil{Mathematical Institute, University of Oxford}
\date{\today}


\begin{document}
    \pagenumbering{gobble}
    \maketitle
    
    
    \begin{abstract}
        Magnetic confinement reactors---in particular tokamaks---offer one of the most promising options for achieving practical nuclear fusion, with the potential to provide virtually limitless, clean energy. The theoretical and numerical modeling of tokamak plasmas is simultaneously an essential component of effective reactor design, and a great research barrier. Tokamak operational conditions exhibit comparatively low Knudsen numbers. Kinetic effects, including kinetic waves and instabilities, Landau damping, bump-on-tail instabilities and more, are therefore highly influential in tokamak plasma dynamics. Purely fluid models are inherently incapable of capturing these effects, whereas the high dimensionality in purely kinetic models render them practically intractable for most relevant purposes.

        We consider a $\delta\!f$ decomposition model, with a macroscopic fluid background and microscopic kinetic correction, both fully coupled to each other. A similar manner of discretization is proposed to that used in the recent \texttt{STRUPHY} code \cite{Holderied_Possanner_Wang_2021, Holderied_2022, Li_et_al_2023} with a finite-element model for the background and a pseudo-particle/PiC model for the correction.

        The fluid background satisfies the full, non-linear, resistive, compressible, Hall MHD equations. \cite{Laakmann_Hu_Farrell_2022} introduces finite-element(-in-space) implicit timesteppers for the incompressible analogue to this system with structure-preserving (SP) properties in the ideal case, alongside parameter-robust preconditioners. We show that these timesteppers can derive from a finite-element-in-time (FET) (and finite-element-in-space) interpretation. The benefits of this reformulation are discussed, including the derivation of timesteppers that are higher order in time, and the quantifiable dissipative SP properties in the non-ideal, resistive case.
        
        We discuss possible options for extending this FET approach to timesteppers for the compressible case.

        The kinetic corrections satisfy linearized Boltzmann equations. Using a Lénard--Bernstein collision operator, these take Fokker--Planck-like forms \cite{Fokker_1914, Planck_1917} wherein pseudo-particles in the numerical model obey the neoclassical transport equations, with particle-independent Brownian drift terms. This offers a rigorous methodology for incorporating collisions into the particle transport model, without coupling the equations of motions for each particle.
        
        Works by Chen, Chacón et al. \cite{Chen_Chacón_Barnes_2011, Chacón_Chen_Barnes_2013, Chen_Chacón_2014, Chen_Chacón_2015} have developed structure-preserving particle pushers for neoclassical transport in the Vlasov equations, derived from Crank--Nicolson integrators. We show these too can can derive from a FET interpretation, similarly offering potential extensions to higher-order-in-time particle pushers. The FET formulation is used also to consider how the stochastic drift terms can be incorporated into the pushers. Stochastic gyrokinetic expansions are also discussed.

        Different options for the numerical implementation of these schemes are considered.

        Due to the efficacy of FET in the development of SP timesteppers for both the fluid and kinetic component, we hope this approach will prove effective in the future for developing SP timesteppers for the full hybrid model. We hope this will give us the opportunity to incorporate previously inaccessible kinetic effects into the highly effective, modern, finite-element MHD models.
    \end{abstract}
    
    
    \newpage
    \tableofcontents
    
    
    \newpage
    \pagenumbering{arabic}
    %\linenumbers\renewcommand\thelinenumber{\color{black!50}\arabic{linenumber}}
            \input{0 - introduction/main.tex}
        \part{Research}
            \input{1 - low-noise PiC models/main.tex}
            \input{2 - kinetic component/main.tex}
            \input{3 - fluid component/main.tex}
            \input{4 - numerical implementation/main.tex}
        \part{Project Overview}
            \input{5 - research plan/main.tex}
            \input{6 - summary/main.tex}
    
    
    %\section{}
    \newpage
    \pagenumbering{gobble}
        \printbibliography


    \newpage
    \pagenumbering{roman}
    \appendix
        \part{Appendices}
            \input{8 - Hilbert complexes/main.tex}
            \input{9 - weak conservation proofs/main.tex}
\end{document}

            \documentclass[12pt, a4paper]{report}

\input{template/main.tex}

\title{\BA{Title in Progress...}}
\author{Boris Andrews}
\affil{Mathematical Institute, University of Oxford}
\date{\today}


\begin{document}
    \pagenumbering{gobble}
    \maketitle
    
    
    \begin{abstract}
        Magnetic confinement reactors---in particular tokamaks---offer one of the most promising options for achieving practical nuclear fusion, with the potential to provide virtually limitless, clean energy. The theoretical and numerical modeling of tokamak plasmas is simultaneously an essential component of effective reactor design, and a great research barrier. Tokamak operational conditions exhibit comparatively low Knudsen numbers. Kinetic effects, including kinetic waves and instabilities, Landau damping, bump-on-tail instabilities and more, are therefore highly influential in tokamak plasma dynamics. Purely fluid models are inherently incapable of capturing these effects, whereas the high dimensionality in purely kinetic models render them practically intractable for most relevant purposes.

        We consider a $\delta\!f$ decomposition model, with a macroscopic fluid background and microscopic kinetic correction, both fully coupled to each other. A similar manner of discretization is proposed to that used in the recent \texttt{STRUPHY} code \cite{Holderied_Possanner_Wang_2021, Holderied_2022, Li_et_al_2023} with a finite-element model for the background and a pseudo-particle/PiC model for the correction.

        The fluid background satisfies the full, non-linear, resistive, compressible, Hall MHD equations. \cite{Laakmann_Hu_Farrell_2022} introduces finite-element(-in-space) implicit timesteppers for the incompressible analogue to this system with structure-preserving (SP) properties in the ideal case, alongside parameter-robust preconditioners. We show that these timesteppers can derive from a finite-element-in-time (FET) (and finite-element-in-space) interpretation. The benefits of this reformulation are discussed, including the derivation of timesteppers that are higher order in time, and the quantifiable dissipative SP properties in the non-ideal, resistive case.
        
        We discuss possible options for extending this FET approach to timesteppers for the compressible case.

        The kinetic corrections satisfy linearized Boltzmann equations. Using a Lénard--Bernstein collision operator, these take Fokker--Planck-like forms \cite{Fokker_1914, Planck_1917} wherein pseudo-particles in the numerical model obey the neoclassical transport equations, with particle-independent Brownian drift terms. This offers a rigorous methodology for incorporating collisions into the particle transport model, without coupling the equations of motions for each particle.
        
        Works by Chen, Chacón et al. \cite{Chen_Chacón_Barnes_2011, Chacón_Chen_Barnes_2013, Chen_Chacón_2014, Chen_Chacón_2015} have developed structure-preserving particle pushers for neoclassical transport in the Vlasov equations, derived from Crank--Nicolson integrators. We show these too can can derive from a FET interpretation, similarly offering potential extensions to higher-order-in-time particle pushers. The FET formulation is used also to consider how the stochastic drift terms can be incorporated into the pushers. Stochastic gyrokinetic expansions are also discussed.

        Different options for the numerical implementation of these schemes are considered.

        Due to the efficacy of FET in the development of SP timesteppers for both the fluid and kinetic component, we hope this approach will prove effective in the future for developing SP timesteppers for the full hybrid model. We hope this will give us the opportunity to incorporate previously inaccessible kinetic effects into the highly effective, modern, finite-element MHD models.
    \end{abstract}
    
    
    \newpage
    \tableofcontents
    
    
    \newpage
    \pagenumbering{arabic}
    %\linenumbers\renewcommand\thelinenumber{\color{black!50}\arabic{linenumber}}
            \input{0 - introduction/main.tex}
        \part{Research}
            \input{1 - low-noise PiC models/main.tex}
            \input{2 - kinetic component/main.tex}
            \input{3 - fluid component/main.tex}
            \input{4 - numerical implementation/main.tex}
        \part{Project Overview}
            \input{5 - research plan/main.tex}
            \input{6 - summary/main.tex}
    
    
    %\section{}
    \newpage
    \pagenumbering{gobble}
        \printbibliography


    \newpage
    \pagenumbering{roman}
    \appendix
        \part{Appendices}
            \input{8 - Hilbert complexes/main.tex}
            \input{9 - weak conservation proofs/main.tex}
\end{document}

\end{document}

            \documentclass[12pt, a4paper]{report}

\documentclass[12pt, a4paper]{report}

\input{template/main.tex}

\title{\BA{Title in Progress...}}
\author{Boris Andrews}
\affil{Mathematical Institute, University of Oxford}
\date{\today}


\begin{document}
    \pagenumbering{gobble}
    \maketitle
    
    
    \begin{abstract}
        Magnetic confinement reactors---in particular tokamaks---offer one of the most promising options for achieving practical nuclear fusion, with the potential to provide virtually limitless, clean energy. The theoretical and numerical modeling of tokamak plasmas is simultaneously an essential component of effective reactor design, and a great research barrier. Tokamak operational conditions exhibit comparatively low Knudsen numbers. Kinetic effects, including kinetic waves and instabilities, Landau damping, bump-on-tail instabilities and more, are therefore highly influential in tokamak plasma dynamics. Purely fluid models are inherently incapable of capturing these effects, whereas the high dimensionality in purely kinetic models render them practically intractable for most relevant purposes.

        We consider a $\delta\!f$ decomposition model, with a macroscopic fluid background and microscopic kinetic correction, both fully coupled to each other. A similar manner of discretization is proposed to that used in the recent \texttt{STRUPHY} code \cite{Holderied_Possanner_Wang_2021, Holderied_2022, Li_et_al_2023} with a finite-element model for the background and a pseudo-particle/PiC model for the correction.

        The fluid background satisfies the full, non-linear, resistive, compressible, Hall MHD equations. \cite{Laakmann_Hu_Farrell_2022} introduces finite-element(-in-space) implicit timesteppers for the incompressible analogue to this system with structure-preserving (SP) properties in the ideal case, alongside parameter-robust preconditioners. We show that these timesteppers can derive from a finite-element-in-time (FET) (and finite-element-in-space) interpretation. The benefits of this reformulation are discussed, including the derivation of timesteppers that are higher order in time, and the quantifiable dissipative SP properties in the non-ideal, resistive case.
        
        We discuss possible options for extending this FET approach to timesteppers for the compressible case.

        The kinetic corrections satisfy linearized Boltzmann equations. Using a Lénard--Bernstein collision operator, these take Fokker--Planck-like forms \cite{Fokker_1914, Planck_1917} wherein pseudo-particles in the numerical model obey the neoclassical transport equations, with particle-independent Brownian drift terms. This offers a rigorous methodology for incorporating collisions into the particle transport model, without coupling the equations of motions for each particle.
        
        Works by Chen, Chacón et al. \cite{Chen_Chacón_Barnes_2011, Chacón_Chen_Barnes_2013, Chen_Chacón_2014, Chen_Chacón_2015} have developed structure-preserving particle pushers for neoclassical transport in the Vlasov equations, derived from Crank--Nicolson integrators. We show these too can can derive from a FET interpretation, similarly offering potential extensions to higher-order-in-time particle pushers. The FET formulation is used also to consider how the stochastic drift terms can be incorporated into the pushers. Stochastic gyrokinetic expansions are also discussed.

        Different options for the numerical implementation of these schemes are considered.

        Due to the efficacy of FET in the development of SP timesteppers for both the fluid and kinetic component, we hope this approach will prove effective in the future for developing SP timesteppers for the full hybrid model. We hope this will give us the opportunity to incorporate previously inaccessible kinetic effects into the highly effective, modern, finite-element MHD models.
    \end{abstract}
    
    
    \newpage
    \tableofcontents
    
    
    \newpage
    \pagenumbering{arabic}
    %\linenumbers\renewcommand\thelinenumber{\color{black!50}\arabic{linenumber}}
            \input{0 - introduction/main.tex}
        \part{Research}
            \input{1 - low-noise PiC models/main.tex}
            \input{2 - kinetic component/main.tex}
            \input{3 - fluid component/main.tex}
            \input{4 - numerical implementation/main.tex}
        \part{Project Overview}
            \input{5 - research plan/main.tex}
            \input{6 - summary/main.tex}
    
    
    %\section{}
    \newpage
    \pagenumbering{gobble}
        \printbibliography


    \newpage
    \pagenumbering{roman}
    \appendix
        \part{Appendices}
            \input{8 - Hilbert complexes/main.tex}
            \input{9 - weak conservation proofs/main.tex}
\end{document}


\title{\BA{Title in Progress...}}
\author{Boris Andrews}
\affil{Mathematical Institute, University of Oxford}
\date{\today}


\begin{document}
    \pagenumbering{gobble}
    \maketitle
    
    
    \begin{abstract}
        Magnetic confinement reactors---in particular tokamaks---offer one of the most promising options for achieving practical nuclear fusion, with the potential to provide virtually limitless, clean energy. The theoretical and numerical modeling of tokamak plasmas is simultaneously an essential component of effective reactor design, and a great research barrier. Tokamak operational conditions exhibit comparatively low Knudsen numbers. Kinetic effects, including kinetic waves and instabilities, Landau damping, bump-on-tail instabilities and more, are therefore highly influential in tokamak plasma dynamics. Purely fluid models are inherently incapable of capturing these effects, whereas the high dimensionality in purely kinetic models render them practically intractable for most relevant purposes.

        We consider a $\delta\!f$ decomposition model, with a macroscopic fluid background and microscopic kinetic correction, both fully coupled to each other. A similar manner of discretization is proposed to that used in the recent \texttt{STRUPHY} code \cite{Holderied_Possanner_Wang_2021, Holderied_2022, Li_et_al_2023} with a finite-element model for the background and a pseudo-particle/PiC model for the correction.

        The fluid background satisfies the full, non-linear, resistive, compressible, Hall MHD equations. \cite{Laakmann_Hu_Farrell_2022} introduces finite-element(-in-space) implicit timesteppers for the incompressible analogue to this system with structure-preserving (SP) properties in the ideal case, alongside parameter-robust preconditioners. We show that these timesteppers can derive from a finite-element-in-time (FET) (and finite-element-in-space) interpretation. The benefits of this reformulation are discussed, including the derivation of timesteppers that are higher order in time, and the quantifiable dissipative SP properties in the non-ideal, resistive case.
        
        We discuss possible options for extending this FET approach to timesteppers for the compressible case.

        The kinetic corrections satisfy linearized Boltzmann equations. Using a Lénard--Bernstein collision operator, these take Fokker--Planck-like forms \cite{Fokker_1914, Planck_1917} wherein pseudo-particles in the numerical model obey the neoclassical transport equations, with particle-independent Brownian drift terms. This offers a rigorous methodology for incorporating collisions into the particle transport model, without coupling the equations of motions for each particle.
        
        Works by Chen, Chacón et al. \cite{Chen_Chacón_Barnes_2011, Chacón_Chen_Barnes_2013, Chen_Chacón_2014, Chen_Chacón_2015} have developed structure-preserving particle pushers for neoclassical transport in the Vlasov equations, derived from Crank--Nicolson integrators. We show these too can can derive from a FET interpretation, similarly offering potential extensions to higher-order-in-time particle pushers. The FET formulation is used also to consider how the stochastic drift terms can be incorporated into the pushers. Stochastic gyrokinetic expansions are also discussed.

        Different options for the numerical implementation of these schemes are considered.

        Due to the efficacy of FET in the development of SP timesteppers for both the fluid and kinetic component, we hope this approach will prove effective in the future for developing SP timesteppers for the full hybrid model. We hope this will give us the opportunity to incorporate previously inaccessible kinetic effects into the highly effective, modern, finite-element MHD models.
    \end{abstract}
    
    
    \newpage
    \tableofcontents
    
    
    \newpage
    \pagenumbering{arabic}
    %\linenumbers\renewcommand\thelinenumber{\color{black!50}\arabic{linenumber}}
            \documentclass[12pt, a4paper]{report}

\input{template/main.tex}

\title{\BA{Title in Progress...}}
\author{Boris Andrews}
\affil{Mathematical Institute, University of Oxford}
\date{\today}


\begin{document}
    \pagenumbering{gobble}
    \maketitle
    
    
    \begin{abstract}
        Magnetic confinement reactors---in particular tokamaks---offer one of the most promising options for achieving practical nuclear fusion, with the potential to provide virtually limitless, clean energy. The theoretical and numerical modeling of tokamak plasmas is simultaneously an essential component of effective reactor design, and a great research barrier. Tokamak operational conditions exhibit comparatively low Knudsen numbers. Kinetic effects, including kinetic waves and instabilities, Landau damping, bump-on-tail instabilities and more, are therefore highly influential in tokamak plasma dynamics. Purely fluid models are inherently incapable of capturing these effects, whereas the high dimensionality in purely kinetic models render them practically intractable for most relevant purposes.

        We consider a $\delta\!f$ decomposition model, with a macroscopic fluid background and microscopic kinetic correction, both fully coupled to each other. A similar manner of discretization is proposed to that used in the recent \texttt{STRUPHY} code \cite{Holderied_Possanner_Wang_2021, Holderied_2022, Li_et_al_2023} with a finite-element model for the background and a pseudo-particle/PiC model for the correction.

        The fluid background satisfies the full, non-linear, resistive, compressible, Hall MHD equations. \cite{Laakmann_Hu_Farrell_2022} introduces finite-element(-in-space) implicit timesteppers for the incompressible analogue to this system with structure-preserving (SP) properties in the ideal case, alongside parameter-robust preconditioners. We show that these timesteppers can derive from a finite-element-in-time (FET) (and finite-element-in-space) interpretation. The benefits of this reformulation are discussed, including the derivation of timesteppers that are higher order in time, and the quantifiable dissipative SP properties in the non-ideal, resistive case.
        
        We discuss possible options for extending this FET approach to timesteppers for the compressible case.

        The kinetic corrections satisfy linearized Boltzmann equations. Using a Lénard--Bernstein collision operator, these take Fokker--Planck-like forms \cite{Fokker_1914, Planck_1917} wherein pseudo-particles in the numerical model obey the neoclassical transport equations, with particle-independent Brownian drift terms. This offers a rigorous methodology for incorporating collisions into the particle transport model, without coupling the equations of motions for each particle.
        
        Works by Chen, Chacón et al. \cite{Chen_Chacón_Barnes_2011, Chacón_Chen_Barnes_2013, Chen_Chacón_2014, Chen_Chacón_2015} have developed structure-preserving particle pushers for neoclassical transport in the Vlasov equations, derived from Crank--Nicolson integrators. We show these too can can derive from a FET interpretation, similarly offering potential extensions to higher-order-in-time particle pushers. The FET formulation is used also to consider how the stochastic drift terms can be incorporated into the pushers. Stochastic gyrokinetic expansions are also discussed.

        Different options for the numerical implementation of these schemes are considered.

        Due to the efficacy of FET in the development of SP timesteppers for both the fluid and kinetic component, we hope this approach will prove effective in the future for developing SP timesteppers for the full hybrid model. We hope this will give us the opportunity to incorporate previously inaccessible kinetic effects into the highly effective, modern, finite-element MHD models.
    \end{abstract}
    
    
    \newpage
    \tableofcontents
    
    
    \newpage
    \pagenumbering{arabic}
    %\linenumbers\renewcommand\thelinenumber{\color{black!50}\arabic{linenumber}}
            \input{0 - introduction/main.tex}
        \part{Research}
            \input{1 - low-noise PiC models/main.tex}
            \input{2 - kinetic component/main.tex}
            \input{3 - fluid component/main.tex}
            \input{4 - numerical implementation/main.tex}
        \part{Project Overview}
            \input{5 - research plan/main.tex}
            \input{6 - summary/main.tex}
    
    
    %\section{}
    \newpage
    \pagenumbering{gobble}
        \printbibliography


    \newpage
    \pagenumbering{roman}
    \appendix
        \part{Appendices}
            \input{8 - Hilbert complexes/main.tex}
            \input{9 - weak conservation proofs/main.tex}
\end{document}

        \part{Research}
            \documentclass[12pt, a4paper]{report}

\input{template/main.tex}

\title{\BA{Title in Progress...}}
\author{Boris Andrews}
\affil{Mathematical Institute, University of Oxford}
\date{\today}


\begin{document}
    \pagenumbering{gobble}
    \maketitle
    
    
    \begin{abstract}
        Magnetic confinement reactors---in particular tokamaks---offer one of the most promising options for achieving practical nuclear fusion, with the potential to provide virtually limitless, clean energy. The theoretical and numerical modeling of tokamak plasmas is simultaneously an essential component of effective reactor design, and a great research barrier. Tokamak operational conditions exhibit comparatively low Knudsen numbers. Kinetic effects, including kinetic waves and instabilities, Landau damping, bump-on-tail instabilities and more, are therefore highly influential in tokamak plasma dynamics. Purely fluid models are inherently incapable of capturing these effects, whereas the high dimensionality in purely kinetic models render them practically intractable for most relevant purposes.

        We consider a $\delta\!f$ decomposition model, with a macroscopic fluid background and microscopic kinetic correction, both fully coupled to each other. A similar manner of discretization is proposed to that used in the recent \texttt{STRUPHY} code \cite{Holderied_Possanner_Wang_2021, Holderied_2022, Li_et_al_2023} with a finite-element model for the background and a pseudo-particle/PiC model for the correction.

        The fluid background satisfies the full, non-linear, resistive, compressible, Hall MHD equations. \cite{Laakmann_Hu_Farrell_2022} introduces finite-element(-in-space) implicit timesteppers for the incompressible analogue to this system with structure-preserving (SP) properties in the ideal case, alongside parameter-robust preconditioners. We show that these timesteppers can derive from a finite-element-in-time (FET) (and finite-element-in-space) interpretation. The benefits of this reformulation are discussed, including the derivation of timesteppers that are higher order in time, and the quantifiable dissipative SP properties in the non-ideal, resistive case.
        
        We discuss possible options for extending this FET approach to timesteppers for the compressible case.

        The kinetic corrections satisfy linearized Boltzmann equations. Using a Lénard--Bernstein collision operator, these take Fokker--Planck-like forms \cite{Fokker_1914, Planck_1917} wherein pseudo-particles in the numerical model obey the neoclassical transport equations, with particle-independent Brownian drift terms. This offers a rigorous methodology for incorporating collisions into the particle transport model, without coupling the equations of motions for each particle.
        
        Works by Chen, Chacón et al. \cite{Chen_Chacón_Barnes_2011, Chacón_Chen_Barnes_2013, Chen_Chacón_2014, Chen_Chacón_2015} have developed structure-preserving particle pushers for neoclassical transport in the Vlasov equations, derived from Crank--Nicolson integrators. We show these too can can derive from a FET interpretation, similarly offering potential extensions to higher-order-in-time particle pushers. The FET formulation is used also to consider how the stochastic drift terms can be incorporated into the pushers. Stochastic gyrokinetic expansions are also discussed.

        Different options for the numerical implementation of these schemes are considered.

        Due to the efficacy of FET in the development of SP timesteppers for both the fluid and kinetic component, we hope this approach will prove effective in the future for developing SP timesteppers for the full hybrid model. We hope this will give us the opportunity to incorporate previously inaccessible kinetic effects into the highly effective, modern, finite-element MHD models.
    \end{abstract}
    
    
    \newpage
    \tableofcontents
    
    
    \newpage
    \pagenumbering{arabic}
    %\linenumbers\renewcommand\thelinenumber{\color{black!50}\arabic{linenumber}}
            \input{0 - introduction/main.tex}
        \part{Research}
            \input{1 - low-noise PiC models/main.tex}
            \input{2 - kinetic component/main.tex}
            \input{3 - fluid component/main.tex}
            \input{4 - numerical implementation/main.tex}
        \part{Project Overview}
            \input{5 - research plan/main.tex}
            \input{6 - summary/main.tex}
    
    
    %\section{}
    \newpage
    \pagenumbering{gobble}
        \printbibliography


    \newpage
    \pagenumbering{roman}
    \appendix
        \part{Appendices}
            \input{8 - Hilbert complexes/main.tex}
            \input{9 - weak conservation proofs/main.tex}
\end{document}

            \documentclass[12pt, a4paper]{report}

\input{template/main.tex}

\title{\BA{Title in Progress...}}
\author{Boris Andrews}
\affil{Mathematical Institute, University of Oxford}
\date{\today}


\begin{document}
    \pagenumbering{gobble}
    \maketitle
    
    
    \begin{abstract}
        Magnetic confinement reactors---in particular tokamaks---offer one of the most promising options for achieving practical nuclear fusion, with the potential to provide virtually limitless, clean energy. The theoretical and numerical modeling of tokamak plasmas is simultaneously an essential component of effective reactor design, and a great research barrier. Tokamak operational conditions exhibit comparatively low Knudsen numbers. Kinetic effects, including kinetic waves and instabilities, Landau damping, bump-on-tail instabilities and more, are therefore highly influential in tokamak plasma dynamics. Purely fluid models are inherently incapable of capturing these effects, whereas the high dimensionality in purely kinetic models render them practically intractable for most relevant purposes.

        We consider a $\delta\!f$ decomposition model, with a macroscopic fluid background and microscopic kinetic correction, both fully coupled to each other. A similar manner of discretization is proposed to that used in the recent \texttt{STRUPHY} code \cite{Holderied_Possanner_Wang_2021, Holderied_2022, Li_et_al_2023} with a finite-element model for the background and a pseudo-particle/PiC model for the correction.

        The fluid background satisfies the full, non-linear, resistive, compressible, Hall MHD equations. \cite{Laakmann_Hu_Farrell_2022} introduces finite-element(-in-space) implicit timesteppers for the incompressible analogue to this system with structure-preserving (SP) properties in the ideal case, alongside parameter-robust preconditioners. We show that these timesteppers can derive from a finite-element-in-time (FET) (and finite-element-in-space) interpretation. The benefits of this reformulation are discussed, including the derivation of timesteppers that are higher order in time, and the quantifiable dissipative SP properties in the non-ideal, resistive case.
        
        We discuss possible options for extending this FET approach to timesteppers for the compressible case.

        The kinetic corrections satisfy linearized Boltzmann equations. Using a Lénard--Bernstein collision operator, these take Fokker--Planck-like forms \cite{Fokker_1914, Planck_1917} wherein pseudo-particles in the numerical model obey the neoclassical transport equations, with particle-independent Brownian drift terms. This offers a rigorous methodology for incorporating collisions into the particle transport model, without coupling the equations of motions for each particle.
        
        Works by Chen, Chacón et al. \cite{Chen_Chacón_Barnes_2011, Chacón_Chen_Barnes_2013, Chen_Chacón_2014, Chen_Chacón_2015} have developed structure-preserving particle pushers for neoclassical transport in the Vlasov equations, derived from Crank--Nicolson integrators. We show these too can can derive from a FET interpretation, similarly offering potential extensions to higher-order-in-time particle pushers. The FET formulation is used also to consider how the stochastic drift terms can be incorporated into the pushers. Stochastic gyrokinetic expansions are also discussed.

        Different options for the numerical implementation of these schemes are considered.

        Due to the efficacy of FET in the development of SP timesteppers for both the fluid and kinetic component, we hope this approach will prove effective in the future for developing SP timesteppers for the full hybrid model. We hope this will give us the opportunity to incorporate previously inaccessible kinetic effects into the highly effective, modern, finite-element MHD models.
    \end{abstract}
    
    
    \newpage
    \tableofcontents
    
    
    \newpage
    \pagenumbering{arabic}
    %\linenumbers\renewcommand\thelinenumber{\color{black!50}\arabic{linenumber}}
            \input{0 - introduction/main.tex}
        \part{Research}
            \input{1 - low-noise PiC models/main.tex}
            \input{2 - kinetic component/main.tex}
            \input{3 - fluid component/main.tex}
            \input{4 - numerical implementation/main.tex}
        \part{Project Overview}
            \input{5 - research plan/main.tex}
            \input{6 - summary/main.tex}
    
    
    %\section{}
    \newpage
    \pagenumbering{gobble}
        \printbibliography


    \newpage
    \pagenumbering{roman}
    \appendix
        \part{Appendices}
            \input{8 - Hilbert complexes/main.tex}
            \input{9 - weak conservation proofs/main.tex}
\end{document}

            \documentclass[12pt, a4paper]{report}

\input{template/main.tex}

\title{\BA{Title in Progress...}}
\author{Boris Andrews}
\affil{Mathematical Institute, University of Oxford}
\date{\today}


\begin{document}
    \pagenumbering{gobble}
    \maketitle
    
    
    \begin{abstract}
        Magnetic confinement reactors---in particular tokamaks---offer one of the most promising options for achieving practical nuclear fusion, with the potential to provide virtually limitless, clean energy. The theoretical and numerical modeling of tokamak plasmas is simultaneously an essential component of effective reactor design, and a great research barrier. Tokamak operational conditions exhibit comparatively low Knudsen numbers. Kinetic effects, including kinetic waves and instabilities, Landau damping, bump-on-tail instabilities and more, are therefore highly influential in tokamak plasma dynamics. Purely fluid models are inherently incapable of capturing these effects, whereas the high dimensionality in purely kinetic models render them practically intractable for most relevant purposes.

        We consider a $\delta\!f$ decomposition model, with a macroscopic fluid background and microscopic kinetic correction, both fully coupled to each other. A similar manner of discretization is proposed to that used in the recent \texttt{STRUPHY} code \cite{Holderied_Possanner_Wang_2021, Holderied_2022, Li_et_al_2023} with a finite-element model for the background and a pseudo-particle/PiC model for the correction.

        The fluid background satisfies the full, non-linear, resistive, compressible, Hall MHD equations. \cite{Laakmann_Hu_Farrell_2022} introduces finite-element(-in-space) implicit timesteppers for the incompressible analogue to this system with structure-preserving (SP) properties in the ideal case, alongside parameter-robust preconditioners. We show that these timesteppers can derive from a finite-element-in-time (FET) (and finite-element-in-space) interpretation. The benefits of this reformulation are discussed, including the derivation of timesteppers that are higher order in time, and the quantifiable dissipative SP properties in the non-ideal, resistive case.
        
        We discuss possible options for extending this FET approach to timesteppers for the compressible case.

        The kinetic corrections satisfy linearized Boltzmann equations. Using a Lénard--Bernstein collision operator, these take Fokker--Planck-like forms \cite{Fokker_1914, Planck_1917} wherein pseudo-particles in the numerical model obey the neoclassical transport equations, with particle-independent Brownian drift terms. This offers a rigorous methodology for incorporating collisions into the particle transport model, without coupling the equations of motions for each particle.
        
        Works by Chen, Chacón et al. \cite{Chen_Chacón_Barnes_2011, Chacón_Chen_Barnes_2013, Chen_Chacón_2014, Chen_Chacón_2015} have developed structure-preserving particle pushers for neoclassical transport in the Vlasov equations, derived from Crank--Nicolson integrators. We show these too can can derive from a FET interpretation, similarly offering potential extensions to higher-order-in-time particle pushers. The FET formulation is used also to consider how the stochastic drift terms can be incorporated into the pushers. Stochastic gyrokinetic expansions are also discussed.

        Different options for the numerical implementation of these schemes are considered.

        Due to the efficacy of FET in the development of SP timesteppers for both the fluid and kinetic component, we hope this approach will prove effective in the future for developing SP timesteppers for the full hybrid model. We hope this will give us the opportunity to incorporate previously inaccessible kinetic effects into the highly effective, modern, finite-element MHD models.
    \end{abstract}
    
    
    \newpage
    \tableofcontents
    
    
    \newpage
    \pagenumbering{arabic}
    %\linenumbers\renewcommand\thelinenumber{\color{black!50}\arabic{linenumber}}
            \input{0 - introduction/main.tex}
        \part{Research}
            \input{1 - low-noise PiC models/main.tex}
            \input{2 - kinetic component/main.tex}
            \input{3 - fluid component/main.tex}
            \input{4 - numerical implementation/main.tex}
        \part{Project Overview}
            \input{5 - research plan/main.tex}
            \input{6 - summary/main.tex}
    
    
    %\section{}
    \newpage
    \pagenumbering{gobble}
        \printbibliography


    \newpage
    \pagenumbering{roman}
    \appendix
        \part{Appendices}
            \input{8 - Hilbert complexes/main.tex}
            \input{9 - weak conservation proofs/main.tex}
\end{document}

            \documentclass[12pt, a4paper]{report}

\input{template/main.tex}

\title{\BA{Title in Progress...}}
\author{Boris Andrews}
\affil{Mathematical Institute, University of Oxford}
\date{\today}


\begin{document}
    \pagenumbering{gobble}
    \maketitle
    
    
    \begin{abstract}
        Magnetic confinement reactors---in particular tokamaks---offer one of the most promising options for achieving practical nuclear fusion, with the potential to provide virtually limitless, clean energy. The theoretical and numerical modeling of tokamak plasmas is simultaneously an essential component of effective reactor design, and a great research barrier. Tokamak operational conditions exhibit comparatively low Knudsen numbers. Kinetic effects, including kinetic waves and instabilities, Landau damping, bump-on-tail instabilities and more, are therefore highly influential in tokamak plasma dynamics. Purely fluid models are inherently incapable of capturing these effects, whereas the high dimensionality in purely kinetic models render them practically intractable for most relevant purposes.

        We consider a $\delta\!f$ decomposition model, with a macroscopic fluid background and microscopic kinetic correction, both fully coupled to each other. A similar manner of discretization is proposed to that used in the recent \texttt{STRUPHY} code \cite{Holderied_Possanner_Wang_2021, Holderied_2022, Li_et_al_2023} with a finite-element model for the background and a pseudo-particle/PiC model for the correction.

        The fluid background satisfies the full, non-linear, resistive, compressible, Hall MHD equations. \cite{Laakmann_Hu_Farrell_2022} introduces finite-element(-in-space) implicit timesteppers for the incompressible analogue to this system with structure-preserving (SP) properties in the ideal case, alongside parameter-robust preconditioners. We show that these timesteppers can derive from a finite-element-in-time (FET) (and finite-element-in-space) interpretation. The benefits of this reformulation are discussed, including the derivation of timesteppers that are higher order in time, and the quantifiable dissipative SP properties in the non-ideal, resistive case.
        
        We discuss possible options for extending this FET approach to timesteppers for the compressible case.

        The kinetic corrections satisfy linearized Boltzmann equations. Using a Lénard--Bernstein collision operator, these take Fokker--Planck-like forms \cite{Fokker_1914, Planck_1917} wherein pseudo-particles in the numerical model obey the neoclassical transport equations, with particle-independent Brownian drift terms. This offers a rigorous methodology for incorporating collisions into the particle transport model, without coupling the equations of motions for each particle.
        
        Works by Chen, Chacón et al. \cite{Chen_Chacón_Barnes_2011, Chacón_Chen_Barnes_2013, Chen_Chacón_2014, Chen_Chacón_2015} have developed structure-preserving particle pushers for neoclassical transport in the Vlasov equations, derived from Crank--Nicolson integrators. We show these too can can derive from a FET interpretation, similarly offering potential extensions to higher-order-in-time particle pushers. The FET formulation is used also to consider how the stochastic drift terms can be incorporated into the pushers. Stochastic gyrokinetic expansions are also discussed.

        Different options for the numerical implementation of these schemes are considered.

        Due to the efficacy of FET in the development of SP timesteppers for both the fluid and kinetic component, we hope this approach will prove effective in the future for developing SP timesteppers for the full hybrid model. We hope this will give us the opportunity to incorporate previously inaccessible kinetic effects into the highly effective, modern, finite-element MHD models.
    \end{abstract}
    
    
    \newpage
    \tableofcontents
    
    
    \newpage
    \pagenumbering{arabic}
    %\linenumbers\renewcommand\thelinenumber{\color{black!50}\arabic{linenumber}}
            \input{0 - introduction/main.tex}
        \part{Research}
            \input{1 - low-noise PiC models/main.tex}
            \input{2 - kinetic component/main.tex}
            \input{3 - fluid component/main.tex}
            \input{4 - numerical implementation/main.tex}
        \part{Project Overview}
            \input{5 - research plan/main.tex}
            \input{6 - summary/main.tex}
    
    
    %\section{}
    \newpage
    \pagenumbering{gobble}
        \printbibliography


    \newpage
    \pagenumbering{roman}
    \appendix
        \part{Appendices}
            \input{8 - Hilbert complexes/main.tex}
            \input{9 - weak conservation proofs/main.tex}
\end{document}

        \part{Project Overview}
            \documentclass[12pt, a4paper]{report}

\input{template/main.tex}

\title{\BA{Title in Progress...}}
\author{Boris Andrews}
\affil{Mathematical Institute, University of Oxford}
\date{\today}


\begin{document}
    \pagenumbering{gobble}
    \maketitle
    
    
    \begin{abstract}
        Magnetic confinement reactors---in particular tokamaks---offer one of the most promising options for achieving practical nuclear fusion, with the potential to provide virtually limitless, clean energy. The theoretical and numerical modeling of tokamak plasmas is simultaneously an essential component of effective reactor design, and a great research barrier. Tokamak operational conditions exhibit comparatively low Knudsen numbers. Kinetic effects, including kinetic waves and instabilities, Landau damping, bump-on-tail instabilities and more, are therefore highly influential in tokamak plasma dynamics. Purely fluid models are inherently incapable of capturing these effects, whereas the high dimensionality in purely kinetic models render them practically intractable for most relevant purposes.

        We consider a $\delta\!f$ decomposition model, with a macroscopic fluid background and microscopic kinetic correction, both fully coupled to each other. A similar manner of discretization is proposed to that used in the recent \texttt{STRUPHY} code \cite{Holderied_Possanner_Wang_2021, Holderied_2022, Li_et_al_2023} with a finite-element model for the background and a pseudo-particle/PiC model for the correction.

        The fluid background satisfies the full, non-linear, resistive, compressible, Hall MHD equations. \cite{Laakmann_Hu_Farrell_2022} introduces finite-element(-in-space) implicit timesteppers for the incompressible analogue to this system with structure-preserving (SP) properties in the ideal case, alongside parameter-robust preconditioners. We show that these timesteppers can derive from a finite-element-in-time (FET) (and finite-element-in-space) interpretation. The benefits of this reformulation are discussed, including the derivation of timesteppers that are higher order in time, and the quantifiable dissipative SP properties in the non-ideal, resistive case.
        
        We discuss possible options for extending this FET approach to timesteppers for the compressible case.

        The kinetic corrections satisfy linearized Boltzmann equations. Using a Lénard--Bernstein collision operator, these take Fokker--Planck-like forms \cite{Fokker_1914, Planck_1917} wherein pseudo-particles in the numerical model obey the neoclassical transport equations, with particle-independent Brownian drift terms. This offers a rigorous methodology for incorporating collisions into the particle transport model, without coupling the equations of motions for each particle.
        
        Works by Chen, Chacón et al. \cite{Chen_Chacón_Barnes_2011, Chacón_Chen_Barnes_2013, Chen_Chacón_2014, Chen_Chacón_2015} have developed structure-preserving particle pushers for neoclassical transport in the Vlasov equations, derived from Crank--Nicolson integrators. We show these too can can derive from a FET interpretation, similarly offering potential extensions to higher-order-in-time particle pushers. The FET formulation is used also to consider how the stochastic drift terms can be incorporated into the pushers. Stochastic gyrokinetic expansions are also discussed.

        Different options for the numerical implementation of these schemes are considered.

        Due to the efficacy of FET in the development of SP timesteppers for both the fluid and kinetic component, we hope this approach will prove effective in the future for developing SP timesteppers for the full hybrid model. We hope this will give us the opportunity to incorporate previously inaccessible kinetic effects into the highly effective, modern, finite-element MHD models.
    \end{abstract}
    
    
    \newpage
    \tableofcontents
    
    
    \newpage
    \pagenumbering{arabic}
    %\linenumbers\renewcommand\thelinenumber{\color{black!50}\arabic{linenumber}}
            \input{0 - introduction/main.tex}
        \part{Research}
            \input{1 - low-noise PiC models/main.tex}
            \input{2 - kinetic component/main.tex}
            \input{3 - fluid component/main.tex}
            \input{4 - numerical implementation/main.tex}
        \part{Project Overview}
            \input{5 - research plan/main.tex}
            \input{6 - summary/main.tex}
    
    
    %\section{}
    \newpage
    \pagenumbering{gobble}
        \printbibliography


    \newpage
    \pagenumbering{roman}
    \appendix
        \part{Appendices}
            \input{8 - Hilbert complexes/main.tex}
            \input{9 - weak conservation proofs/main.tex}
\end{document}

            \documentclass[12pt, a4paper]{report}

\input{template/main.tex}

\title{\BA{Title in Progress...}}
\author{Boris Andrews}
\affil{Mathematical Institute, University of Oxford}
\date{\today}


\begin{document}
    \pagenumbering{gobble}
    \maketitle
    
    
    \begin{abstract}
        Magnetic confinement reactors---in particular tokamaks---offer one of the most promising options for achieving practical nuclear fusion, with the potential to provide virtually limitless, clean energy. The theoretical and numerical modeling of tokamak plasmas is simultaneously an essential component of effective reactor design, and a great research barrier. Tokamak operational conditions exhibit comparatively low Knudsen numbers. Kinetic effects, including kinetic waves and instabilities, Landau damping, bump-on-tail instabilities and more, are therefore highly influential in tokamak plasma dynamics. Purely fluid models are inherently incapable of capturing these effects, whereas the high dimensionality in purely kinetic models render them practically intractable for most relevant purposes.

        We consider a $\delta\!f$ decomposition model, with a macroscopic fluid background and microscopic kinetic correction, both fully coupled to each other. A similar manner of discretization is proposed to that used in the recent \texttt{STRUPHY} code \cite{Holderied_Possanner_Wang_2021, Holderied_2022, Li_et_al_2023} with a finite-element model for the background and a pseudo-particle/PiC model for the correction.

        The fluid background satisfies the full, non-linear, resistive, compressible, Hall MHD equations. \cite{Laakmann_Hu_Farrell_2022} introduces finite-element(-in-space) implicit timesteppers for the incompressible analogue to this system with structure-preserving (SP) properties in the ideal case, alongside parameter-robust preconditioners. We show that these timesteppers can derive from a finite-element-in-time (FET) (and finite-element-in-space) interpretation. The benefits of this reformulation are discussed, including the derivation of timesteppers that are higher order in time, and the quantifiable dissipative SP properties in the non-ideal, resistive case.
        
        We discuss possible options for extending this FET approach to timesteppers for the compressible case.

        The kinetic corrections satisfy linearized Boltzmann equations. Using a Lénard--Bernstein collision operator, these take Fokker--Planck-like forms \cite{Fokker_1914, Planck_1917} wherein pseudo-particles in the numerical model obey the neoclassical transport equations, with particle-independent Brownian drift terms. This offers a rigorous methodology for incorporating collisions into the particle transport model, without coupling the equations of motions for each particle.
        
        Works by Chen, Chacón et al. \cite{Chen_Chacón_Barnes_2011, Chacón_Chen_Barnes_2013, Chen_Chacón_2014, Chen_Chacón_2015} have developed structure-preserving particle pushers for neoclassical transport in the Vlasov equations, derived from Crank--Nicolson integrators. We show these too can can derive from a FET interpretation, similarly offering potential extensions to higher-order-in-time particle pushers. The FET formulation is used also to consider how the stochastic drift terms can be incorporated into the pushers. Stochastic gyrokinetic expansions are also discussed.

        Different options for the numerical implementation of these schemes are considered.

        Due to the efficacy of FET in the development of SP timesteppers for both the fluid and kinetic component, we hope this approach will prove effective in the future for developing SP timesteppers for the full hybrid model. We hope this will give us the opportunity to incorporate previously inaccessible kinetic effects into the highly effective, modern, finite-element MHD models.
    \end{abstract}
    
    
    \newpage
    \tableofcontents
    
    
    \newpage
    \pagenumbering{arabic}
    %\linenumbers\renewcommand\thelinenumber{\color{black!50}\arabic{linenumber}}
            \input{0 - introduction/main.tex}
        \part{Research}
            \input{1 - low-noise PiC models/main.tex}
            \input{2 - kinetic component/main.tex}
            \input{3 - fluid component/main.tex}
            \input{4 - numerical implementation/main.tex}
        \part{Project Overview}
            \input{5 - research plan/main.tex}
            \input{6 - summary/main.tex}
    
    
    %\section{}
    \newpage
    \pagenumbering{gobble}
        \printbibliography


    \newpage
    \pagenumbering{roman}
    \appendix
        \part{Appendices}
            \input{8 - Hilbert complexes/main.tex}
            \input{9 - weak conservation proofs/main.tex}
\end{document}

    
    
    %\section{}
    \newpage
    \pagenumbering{gobble}
        \printbibliography


    \newpage
    \pagenumbering{roman}
    \appendix
        \part{Appendices}
            \documentclass[12pt, a4paper]{report}

\input{template/main.tex}

\title{\BA{Title in Progress...}}
\author{Boris Andrews}
\affil{Mathematical Institute, University of Oxford}
\date{\today}


\begin{document}
    \pagenumbering{gobble}
    \maketitle
    
    
    \begin{abstract}
        Magnetic confinement reactors---in particular tokamaks---offer one of the most promising options for achieving practical nuclear fusion, with the potential to provide virtually limitless, clean energy. The theoretical and numerical modeling of tokamak plasmas is simultaneously an essential component of effective reactor design, and a great research barrier. Tokamak operational conditions exhibit comparatively low Knudsen numbers. Kinetic effects, including kinetic waves and instabilities, Landau damping, bump-on-tail instabilities and more, are therefore highly influential in tokamak plasma dynamics. Purely fluid models are inherently incapable of capturing these effects, whereas the high dimensionality in purely kinetic models render them practically intractable for most relevant purposes.

        We consider a $\delta\!f$ decomposition model, with a macroscopic fluid background and microscopic kinetic correction, both fully coupled to each other. A similar manner of discretization is proposed to that used in the recent \texttt{STRUPHY} code \cite{Holderied_Possanner_Wang_2021, Holderied_2022, Li_et_al_2023} with a finite-element model for the background and a pseudo-particle/PiC model for the correction.

        The fluid background satisfies the full, non-linear, resistive, compressible, Hall MHD equations. \cite{Laakmann_Hu_Farrell_2022} introduces finite-element(-in-space) implicit timesteppers for the incompressible analogue to this system with structure-preserving (SP) properties in the ideal case, alongside parameter-robust preconditioners. We show that these timesteppers can derive from a finite-element-in-time (FET) (and finite-element-in-space) interpretation. The benefits of this reformulation are discussed, including the derivation of timesteppers that are higher order in time, and the quantifiable dissipative SP properties in the non-ideal, resistive case.
        
        We discuss possible options for extending this FET approach to timesteppers for the compressible case.

        The kinetic corrections satisfy linearized Boltzmann equations. Using a Lénard--Bernstein collision operator, these take Fokker--Planck-like forms \cite{Fokker_1914, Planck_1917} wherein pseudo-particles in the numerical model obey the neoclassical transport equations, with particle-independent Brownian drift terms. This offers a rigorous methodology for incorporating collisions into the particle transport model, without coupling the equations of motions for each particle.
        
        Works by Chen, Chacón et al. \cite{Chen_Chacón_Barnes_2011, Chacón_Chen_Barnes_2013, Chen_Chacón_2014, Chen_Chacón_2015} have developed structure-preserving particle pushers for neoclassical transport in the Vlasov equations, derived from Crank--Nicolson integrators. We show these too can can derive from a FET interpretation, similarly offering potential extensions to higher-order-in-time particle pushers. The FET formulation is used also to consider how the stochastic drift terms can be incorporated into the pushers. Stochastic gyrokinetic expansions are also discussed.

        Different options for the numerical implementation of these schemes are considered.

        Due to the efficacy of FET in the development of SP timesteppers for both the fluid and kinetic component, we hope this approach will prove effective in the future for developing SP timesteppers for the full hybrid model. We hope this will give us the opportunity to incorporate previously inaccessible kinetic effects into the highly effective, modern, finite-element MHD models.
    \end{abstract}
    
    
    \newpage
    \tableofcontents
    
    
    \newpage
    \pagenumbering{arabic}
    %\linenumbers\renewcommand\thelinenumber{\color{black!50}\arabic{linenumber}}
            \input{0 - introduction/main.tex}
        \part{Research}
            \input{1 - low-noise PiC models/main.tex}
            \input{2 - kinetic component/main.tex}
            \input{3 - fluid component/main.tex}
            \input{4 - numerical implementation/main.tex}
        \part{Project Overview}
            \input{5 - research plan/main.tex}
            \input{6 - summary/main.tex}
    
    
    %\section{}
    \newpage
    \pagenumbering{gobble}
        \printbibliography


    \newpage
    \pagenumbering{roman}
    \appendix
        \part{Appendices}
            \input{8 - Hilbert complexes/main.tex}
            \input{9 - weak conservation proofs/main.tex}
\end{document}

            \documentclass[12pt, a4paper]{report}

\input{template/main.tex}

\title{\BA{Title in Progress...}}
\author{Boris Andrews}
\affil{Mathematical Institute, University of Oxford}
\date{\today}


\begin{document}
    \pagenumbering{gobble}
    \maketitle
    
    
    \begin{abstract}
        Magnetic confinement reactors---in particular tokamaks---offer one of the most promising options for achieving practical nuclear fusion, with the potential to provide virtually limitless, clean energy. The theoretical and numerical modeling of tokamak plasmas is simultaneously an essential component of effective reactor design, and a great research barrier. Tokamak operational conditions exhibit comparatively low Knudsen numbers. Kinetic effects, including kinetic waves and instabilities, Landau damping, bump-on-tail instabilities and more, are therefore highly influential in tokamak plasma dynamics. Purely fluid models are inherently incapable of capturing these effects, whereas the high dimensionality in purely kinetic models render them practically intractable for most relevant purposes.

        We consider a $\delta\!f$ decomposition model, with a macroscopic fluid background and microscopic kinetic correction, both fully coupled to each other. A similar manner of discretization is proposed to that used in the recent \texttt{STRUPHY} code \cite{Holderied_Possanner_Wang_2021, Holderied_2022, Li_et_al_2023} with a finite-element model for the background and a pseudo-particle/PiC model for the correction.

        The fluid background satisfies the full, non-linear, resistive, compressible, Hall MHD equations. \cite{Laakmann_Hu_Farrell_2022} introduces finite-element(-in-space) implicit timesteppers for the incompressible analogue to this system with structure-preserving (SP) properties in the ideal case, alongside parameter-robust preconditioners. We show that these timesteppers can derive from a finite-element-in-time (FET) (and finite-element-in-space) interpretation. The benefits of this reformulation are discussed, including the derivation of timesteppers that are higher order in time, and the quantifiable dissipative SP properties in the non-ideal, resistive case.
        
        We discuss possible options for extending this FET approach to timesteppers for the compressible case.

        The kinetic corrections satisfy linearized Boltzmann equations. Using a Lénard--Bernstein collision operator, these take Fokker--Planck-like forms \cite{Fokker_1914, Planck_1917} wherein pseudo-particles in the numerical model obey the neoclassical transport equations, with particle-independent Brownian drift terms. This offers a rigorous methodology for incorporating collisions into the particle transport model, without coupling the equations of motions for each particle.
        
        Works by Chen, Chacón et al. \cite{Chen_Chacón_Barnes_2011, Chacón_Chen_Barnes_2013, Chen_Chacón_2014, Chen_Chacón_2015} have developed structure-preserving particle pushers for neoclassical transport in the Vlasov equations, derived from Crank--Nicolson integrators. We show these too can can derive from a FET interpretation, similarly offering potential extensions to higher-order-in-time particle pushers. The FET formulation is used also to consider how the stochastic drift terms can be incorporated into the pushers. Stochastic gyrokinetic expansions are also discussed.

        Different options for the numerical implementation of these schemes are considered.

        Due to the efficacy of FET in the development of SP timesteppers for both the fluid and kinetic component, we hope this approach will prove effective in the future for developing SP timesteppers for the full hybrid model. We hope this will give us the opportunity to incorporate previously inaccessible kinetic effects into the highly effective, modern, finite-element MHD models.
    \end{abstract}
    
    
    \newpage
    \tableofcontents
    
    
    \newpage
    \pagenumbering{arabic}
    %\linenumbers\renewcommand\thelinenumber{\color{black!50}\arabic{linenumber}}
            \input{0 - introduction/main.tex}
        \part{Research}
            \input{1 - low-noise PiC models/main.tex}
            \input{2 - kinetic component/main.tex}
            \input{3 - fluid component/main.tex}
            \input{4 - numerical implementation/main.tex}
        \part{Project Overview}
            \input{5 - research plan/main.tex}
            \input{6 - summary/main.tex}
    
    
    %\section{}
    \newpage
    \pagenumbering{gobble}
        \printbibliography


    \newpage
    \pagenumbering{roman}
    \appendix
        \part{Appendices}
            \input{8 - Hilbert complexes/main.tex}
            \input{9 - weak conservation proofs/main.tex}
\end{document}

\end{document}

            \documentclass[12pt, a4paper]{report}

\documentclass[12pt, a4paper]{report}

\input{template/main.tex}

\title{\BA{Title in Progress...}}
\author{Boris Andrews}
\affil{Mathematical Institute, University of Oxford}
\date{\today}


\begin{document}
    \pagenumbering{gobble}
    \maketitle
    
    
    \begin{abstract}
        Magnetic confinement reactors---in particular tokamaks---offer one of the most promising options for achieving practical nuclear fusion, with the potential to provide virtually limitless, clean energy. The theoretical and numerical modeling of tokamak plasmas is simultaneously an essential component of effective reactor design, and a great research barrier. Tokamak operational conditions exhibit comparatively low Knudsen numbers. Kinetic effects, including kinetic waves and instabilities, Landau damping, bump-on-tail instabilities and more, are therefore highly influential in tokamak plasma dynamics. Purely fluid models are inherently incapable of capturing these effects, whereas the high dimensionality in purely kinetic models render them practically intractable for most relevant purposes.

        We consider a $\delta\!f$ decomposition model, with a macroscopic fluid background and microscopic kinetic correction, both fully coupled to each other. A similar manner of discretization is proposed to that used in the recent \texttt{STRUPHY} code \cite{Holderied_Possanner_Wang_2021, Holderied_2022, Li_et_al_2023} with a finite-element model for the background and a pseudo-particle/PiC model for the correction.

        The fluid background satisfies the full, non-linear, resistive, compressible, Hall MHD equations. \cite{Laakmann_Hu_Farrell_2022} introduces finite-element(-in-space) implicit timesteppers for the incompressible analogue to this system with structure-preserving (SP) properties in the ideal case, alongside parameter-robust preconditioners. We show that these timesteppers can derive from a finite-element-in-time (FET) (and finite-element-in-space) interpretation. The benefits of this reformulation are discussed, including the derivation of timesteppers that are higher order in time, and the quantifiable dissipative SP properties in the non-ideal, resistive case.
        
        We discuss possible options for extending this FET approach to timesteppers for the compressible case.

        The kinetic corrections satisfy linearized Boltzmann equations. Using a Lénard--Bernstein collision operator, these take Fokker--Planck-like forms \cite{Fokker_1914, Planck_1917} wherein pseudo-particles in the numerical model obey the neoclassical transport equations, with particle-independent Brownian drift terms. This offers a rigorous methodology for incorporating collisions into the particle transport model, without coupling the equations of motions for each particle.
        
        Works by Chen, Chacón et al. \cite{Chen_Chacón_Barnes_2011, Chacón_Chen_Barnes_2013, Chen_Chacón_2014, Chen_Chacón_2015} have developed structure-preserving particle pushers for neoclassical transport in the Vlasov equations, derived from Crank--Nicolson integrators. We show these too can can derive from a FET interpretation, similarly offering potential extensions to higher-order-in-time particle pushers. The FET formulation is used also to consider how the stochastic drift terms can be incorporated into the pushers. Stochastic gyrokinetic expansions are also discussed.

        Different options for the numerical implementation of these schemes are considered.

        Due to the efficacy of FET in the development of SP timesteppers for both the fluid and kinetic component, we hope this approach will prove effective in the future for developing SP timesteppers for the full hybrid model. We hope this will give us the opportunity to incorporate previously inaccessible kinetic effects into the highly effective, modern, finite-element MHD models.
    \end{abstract}
    
    
    \newpage
    \tableofcontents
    
    
    \newpage
    \pagenumbering{arabic}
    %\linenumbers\renewcommand\thelinenumber{\color{black!50}\arabic{linenumber}}
            \input{0 - introduction/main.tex}
        \part{Research}
            \input{1 - low-noise PiC models/main.tex}
            \input{2 - kinetic component/main.tex}
            \input{3 - fluid component/main.tex}
            \input{4 - numerical implementation/main.tex}
        \part{Project Overview}
            \input{5 - research plan/main.tex}
            \input{6 - summary/main.tex}
    
    
    %\section{}
    \newpage
    \pagenumbering{gobble}
        \printbibliography


    \newpage
    \pagenumbering{roman}
    \appendix
        \part{Appendices}
            \input{8 - Hilbert complexes/main.tex}
            \input{9 - weak conservation proofs/main.tex}
\end{document}


\title{\BA{Title in Progress...}}
\author{Boris Andrews}
\affil{Mathematical Institute, University of Oxford}
\date{\today}


\begin{document}
    \pagenumbering{gobble}
    \maketitle
    
    
    \begin{abstract}
        Magnetic confinement reactors---in particular tokamaks---offer one of the most promising options for achieving practical nuclear fusion, with the potential to provide virtually limitless, clean energy. The theoretical and numerical modeling of tokamak plasmas is simultaneously an essential component of effective reactor design, and a great research barrier. Tokamak operational conditions exhibit comparatively low Knudsen numbers. Kinetic effects, including kinetic waves and instabilities, Landau damping, bump-on-tail instabilities and more, are therefore highly influential in tokamak plasma dynamics. Purely fluid models are inherently incapable of capturing these effects, whereas the high dimensionality in purely kinetic models render them practically intractable for most relevant purposes.

        We consider a $\delta\!f$ decomposition model, with a macroscopic fluid background and microscopic kinetic correction, both fully coupled to each other. A similar manner of discretization is proposed to that used in the recent \texttt{STRUPHY} code \cite{Holderied_Possanner_Wang_2021, Holderied_2022, Li_et_al_2023} with a finite-element model for the background and a pseudo-particle/PiC model for the correction.

        The fluid background satisfies the full, non-linear, resistive, compressible, Hall MHD equations. \cite{Laakmann_Hu_Farrell_2022} introduces finite-element(-in-space) implicit timesteppers for the incompressible analogue to this system with structure-preserving (SP) properties in the ideal case, alongside parameter-robust preconditioners. We show that these timesteppers can derive from a finite-element-in-time (FET) (and finite-element-in-space) interpretation. The benefits of this reformulation are discussed, including the derivation of timesteppers that are higher order in time, and the quantifiable dissipative SP properties in the non-ideal, resistive case.
        
        We discuss possible options for extending this FET approach to timesteppers for the compressible case.

        The kinetic corrections satisfy linearized Boltzmann equations. Using a Lénard--Bernstein collision operator, these take Fokker--Planck-like forms \cite{Fokker_1914, Planck_1917} wherein pseudo-particles in the numerical model obey the neoclassical transport equations, with particle-independent Brownian drift terms. This offers a rigorous methodology for incorporating collisions into the particle transport model, without coupling the equations of motions for each particle.
        
        Works by Chen, Chacón et al. \cite{Chen_Chacón_Barnes_2011, Chacón_Chen_Barnes_2013, Chen_Chacón_2014, Chen_Chacón_2015} have developed structure-preserving particle pushers for neoclassical transport in the Vlasov equations, derived from Crank--Nicolson integrators. We show these too can can derive from a FET interpretation, similarly offering potential extensions to higher-order-in-time particle pushers. The FET formulation is used also to consider how the stochastic drift terms can be incorporated into the pushers. Stochastic gyrokinetic expansions are also discussed.

        Different options for the numerical implementation of these schemes are considered.

        Due to the efficacy of FET in the development of SP timesteppers for both the fluid and kinetic component, we hope this approach will prove effective in the future for developing SP timesteppers for the full hybrid model. We hope this will give us the opportunity to incorporate previously inaccessible kinetic effects into the highly effective, modern, finite-element MHD models.
    \end{abstract}
    
    
    \newpage
    \tableofcontents
    
    
    \newpage
    \pagenumbering{arabic}
    %\linenumbers\renewcommand\thelinenumber{\color{black!50}\arabic{linenumber}}
            \documentclass[12pt, a4paper]{report}

\input{template/main.tex}

\title{\BA{Title in Progress...}}
\author{Boris Andrews}
\affil{Mathematical Institute, University of Oxford}
\date{\today}


\begin{document}
    \pagenumbering{gobble}
    \maketitle
    
    
    \begin{abstract}
        Magnetic confinement reactors---in particular tokamaks---offer one of the most promising options for achieving practical nuclear fusion, with the potential to provide virtually limitless, clean energy. The theoretical and numerical modeling of tokamak plasmas is simultaneously an essential component of effective reactor design, and a great research barrier. Tokamak operational conditions exhibit comparatively low Knudsen numbers. Kinetic effects, including kinetic waves and instabilities, Landau damping, bump-on-tail instabilities and more, are therefore highly influential in tokamak plasma dynamics. Purely fluid models are inherently incapable of capturing these effects, whereas the high dimensionality in purely kinetic models render them practically intractable for most relevant purposes.

        We consider a $\delta\!f$ decomposition model, with a macroscopic fluid background and microscopic kinetic correction, both fully coupled to each other. A similar manner of discretization is proposed to that used in the recent \texttt{STRUPHY} code \cite{Holderied_Possanner_Wang_2021, Holderied_2022, Li_et_al_2023} with a finite-element model for the background and a pseudo-particle/PiC model for the correction.

        The fluid background satisfies the full, non-linear, resistive, compressible, Hall MHD equations. \cite{Laakmann_Hu_Farrell_2022} introduces finite-element(-in-space) implicit timesteppers for the incompressible analogue to this system with structure-preserving (SP) properties in the ideal case, alongside parameter-robust preconditioners. We show that these timesteppers can derive from a finite-element-in-time (FET) (and finite-element-in-space) interpretation. The benefits of this reformulation are discussed, including the derivation of timesteppers that are higher order in time, and the quantifiable dissipative SP properties in the non-ideal, resistive case.
        
        We discuss possible options for extending this FET approach to timesteppers for the compressible case.

        The kinetic corrections satisfy linearized Boltzmann equations. Using a Lénard--Bernstein collision operator, these take Fokker--Planck-like forms \cite{Fokker_1914, Planck_1917} wherein pseudo-particles in the numerical model obey the neoclassical transport equations, with particle-independent Brownian drift terms. This offers a rigorous methodology for incorporating collisions into the particle transport model, without coupling the equations of motions for each particle.
        
        Works by Chen, Chacón et al. \cite{Chen_Chacón_Barnes_2011, Chacón_Chen_Barnes_2013, Chen_Chacón_2014, Chen_Chacón_2015} have developed structure-preserving particle pushers for neoclassical transport in the Vlasov equations, derived from Crank--Nicolson integrators. We show these too can can derive from a FET interpretation, similarly offering potential extensions to higher-order-in-time particle pushers. The FET formulation is used also to consider how the stochastic drift terms can be incorporated into the pushers. Stochastic gyrokinetic expansions are also discussed.

        Different options for the numerical implementation of these schemes are considered.

        Due to the efficacy of FET in the development of SP timesteppers for both the fluid and kinetic component, we hope this approach will prove effective in the future for developing SP timesteppers for the full hybrid model. We hope this will give us the opportunity to incorporate previously inaccessible kinetic effects into the highly effective, modern, finite-element MHD models.
    \end{abstract}
    
    
    \newpage
    \tableofcontents
    
    
    \newpage
    \pagenumbering{arabic}
    %\linenumbers\renewcommand\thelinenumber{\color{black!50}\arabic{linenumber}}
            \input{0 - introduction/main.tex}
        \part{Research}
            \input{1 - low-noise PiC models/main.tex}
            \input{2 - kinetic component/main.tex}
            \input{3 - fluid component/main.tex}
            \input{4 - numerical implementation/main.tex}
        \part{Project Overview}
            \input{5 - research plan/main.tex}
            \input{6 - summary/main.tex}
    
    
    %\section{}
    \newpage
    \pagenumbering{gobble}
        \printbibliography


    \newpage
    \pagenumbering{roman}
    \appendix
        \part{Appendices}
            \input{8 - Hilbert complexes/main.tex}
            \input{9 - weak conservation proofs/main.tex}
\end{document}

        \part{Research}
            \documentclass[12pt, a4paper]{report}

\input{template/main.tex}

\title{\BA{Title in Progress...}}
\author{Boris Andrews}
\affil{Mathematical Institute, University of Oxford}
\date{\today}


\begin{document}
    \pagenumbering{gobble}
    \maketitle
    
    
    \begin{abstract}
        Magnetic confinement reactors---in particular tokamaks---offer one of the most promising options for achieving practical nuclear fusion, with the potential to provide virtually limitless, clean energy. The theoretical and numerical modeling of tokamak plasmas is simultaneously an essential component of effective reactor design, and a great research barrier. Tokamak operational conditions exhibit comparatively low Knudsen numbers. Kinetic effects, including kinetic waves and instabilities, Landau damping, bump-on-tail instabilities and more, are therefore highly influential in tokamak plasma dynamics. Purely fluid models are inherently incapable of capturing these effects, whereas the high dimensionality in purely kinetic models render them practically intractable for most relevant purposes.

        We consider a $\delta\!f$ decomposition model, with a macroscopic fluid background and microscopic kinetic correction, both fully coupled to each other. A similar manner of discretization is proposed to that used in the recent \texttt{STRUPHY} code \cite{Holderied_Possanner_Wang_2021, Holderied_2022, Li_et_al_2023} with a finite-element model for the background and a pseudo-particle/PiC model for the correction.

        The fluid background satisfies the full, non-linear, resistive, compressible, Hall MHD equations. \cite{Laakmann_Hu_Farrell_2022} introduces finite-element(-in-space) implicit timesteppers for the incompressible analogue to this system with structure-preserving (SP) properties in the ideal case, alongside parameter-robust preconditioners. We show that these timesteppers can derive from a finite-element-in-time (FET) (and finite-element-in-space) interpretation. The benefits of this reformulation are discussed, including the derivation of timesteppers that are higher order in time, and the quantifiable dissipative SP properties in the non-ideal, resistive case.
        
        We discuss possible options for extending this FET approach to timesteppers for the compressible case.

        The kinetic corrections satisfy linearized Boltzmann equations. Using a Lénard--Bernstein collision operator, these take Fokker--Planck-like forms \cite{Fokker_1914, Planck_1917} wherein pseudo-particles in the numerical model obey the neoclassical transport equations, with particle-independent Brownian drift terms. This offers a rigorous methodology for incorporating collisions into the particle transport model, without coupling the equations of motions for each particle.
        
        Works by Chen, Chacón et al. \cite{Chen_Chacón_Barnes_2011, Chacón_Chen_Barnes_2013, Chen_Chacón_2014, Chen_Chacón_2015} have developed structure-preserving particle pushers for neoclassical transport in the Vlasov equations, derived from Crank--Nicolson integrators. We show these too can can derive from a FET interpretation, similarly offering potential extensions to higher-order-in-time particle pushers. The FET formulation is used also to consider how the stochastic drift terms can be incorporated into the pushers. Stochastic gyrokinetic expansions are also discussed.

        Different options for the numerical implementation of these schemes are considered.

        Due to the efficacy of FET in the development of SP timesteppers for both the fluid and kinetic component, we hope this approach will prove effective in the future for developing SP timesteppers for the full hybrid model. We hope this will give us the opportunity to incorporate previously inaccessible kinetic effects into the highly effective, modern, finite-element MHD models.
    \end{abstract}
    
    
    \newpage
    \tableofcontents
    
    
    \newpage
    \pagenumbering{arabic}
    %\linenumbers\renewcommand\thelinenumber{\color{black!50}\arabic{linenumber}}
            \input{0 - introduction/main.tex}
        \part{Research}
            \input{1 - low-noise PiC models/main.tex}
            \input{2 - kinetic component/main.tex}
            \input{3 - fluid component/main.tex}
            \input{4 - numerical implementation/main.tex}
        \part{Project Overview}
            \input{5 - research plan/main.tex}
            \input{6 - summary/main.tex}
    
    
    %\section{}
    \newpage
    \pagenumbering{gobble}
        \printbibliography


    \newpage
    \pagenumbering{roman}
    \appendix
        \part{Appendices}
            \input{8 - Hilbert complexes/main.tex}
            \input{9 - weak conservation proofs/main.tex}
\end{document}

            \documentclass[12pt, a4paper]{report}

\input{template/main.tex}

\title{\BA{Title in Progress...}}
\author{Boris Andrews}
\affil{Mathematical Institute, University of Oxford}
\date{\today}


\begin{document}
    \pagenumbering{gobble}
    \maketitle
    
    
    \begin{abstract}
        Magnetic confinement reactors---in particular tokamaks---offer one of the most promising options for achieving practical nuclear fusion, with the potential to provide virtually limitless, clean energy. The theoretical and numerical modeling of tokamak plasmas is simultaneously an essential component of effective reactor design, and a great research barrier. Tokamak operational conditions exhibit comparatively low Knudsen numbers. Kinetic effects, including kinetic waves and instabilities, Landau damping, bump-on-tail instabilities and more, are therefore highly influential in tokamak plasma dynamics. Purely fluid models are inherently incapable of capturing these effects, whereas the high dimensionality in purely kinetic models render them practically intractable for most relevant purposes.

        We consider a $\delta\!f$ decomposition model, with a macroscopic fluid background and microscopic kinetic correction, both fully coupled to each other. A similar manner of discretization is proposed to that used in the recent \texttt{STRUPHY} code \cite{Holderied_Possanner_Wang_2021, Holderied_2022, Li_et_al_2023} with a finite-element model for the background and a pseudo-particle/PiC model for the correction.

        The fluid background satisfies the full, non-linear, resistive, compressible, Hall MHD equations. \cite{Laakmann_Hu_Farrell_2022} introduces finite-element(-in-space) implicit timesteppers for the incompressible analogue to this system with structure-preserving (SP) properties in the ideal case, alongside parameter-robust preconditioners. We show that these timesteppers can derive from a finite-element-in-time (FET) (and finite-element-in-space) interpretation. The benefits of this reformulation are discussed, including the derivation of timesteppers that are higher order in time, and the quantifiable dissipative SP properties in the non-ideal, resistive case.
        
        We discuss possible options for extending this FET approach to timesteppers for the compressible case.

        The kinetic corrections satisfy linearized Boltzmann equations. Using a Lénard--Bernstein collision operator, these take Fokker--Planck-like forms \cite{Fokker_1914, Planck_1917} wherein pseudo-particles in the numerical model obey the neoclassical transport equations, with particle-independent Brownian drift terms. This offers a rigorous methodology for incorporating collisions into the particle transport model, without coupling the equations of motions for each particle.
        
        Works by Chen, Chacón et al. \cite{Chen_Chacón_Barnes_2011, Chacón_Chen_Barnes_2013, Chen_Chacón_2014, Chen_Chacón_2015} have developed structure-preserving particle pushers for neoclassical transport in the Vlasov equations, derived from Crank--Nicolson integrators. We show these too can can derive from a FET interpretation, similarly offering potential extensions to higher-order-in-time particle pushers. The FET formulation is used also to consider how the stochastic drift terms can be incorporated into the pushers. Stochastic gyrokinetic expansions are also discussed.

        Different options for the numerical implementation of these schemes are considered.

        Due to the efficacy of FET in the development of SP timesteppers for both the fluid and kinetic component, we hope this approach will prove effective in the future for developing SP timesteppers for the full hybrid model. We hope this will give us the opportunity to incorporate previously inaccessible kinetic effects into the highly effective, modern, finite-element MHD models.
    \end{abstract}
    
    
    \newpage
    \tableofcontents
    
    
    \newpage
    \pagenumbering{arabic}
    %\linenumbers\renewcommand\thelinenumber{\color{black!50}\arabic{linenumber}}
            \input{0 - introduction/main.tex}
        \part{Research}
            \input{1 - low-noise PiC models/main.tex}
            \input{2 - kinetic component/main.tex}
            \input{3 - fluid component/main.tex}
            \input{4 - numerical implementation/main.tex}
        \part{Project Overview}
            \input{5 - research plan/main.tex}
            \input{6 - summary/main.tex}
    
    
    %\section{}
    \newpage
    \pagenumbering{gobble}
        \printbibliography


    \newpage
    \pagenumbering{roman}
    \appendix
        \part{Appendices}
            \input{8 - Hilbert complexes/main.tex}
            \input{9 - weak conservation proofs/main.tex}
\end{document}

            \documentclass[12pt, a4paper]{report}

\input{template/main.tex}

\title{\BA{Title in Progress...}}
\author{Boris Andrews}
\affil{Mathematical Institute, University of Oxford}
\date{\today}


\begin{document}
    \pagenumbering{gobble}
    \maketitle
    
    
    \begin{abstract}
        Magnetic confinement reactors---in particular tokamaks---offer one of the most promising options for achieving practical nuclear fusion, with the potential to provide virtually limitless, clean energy. The theoretical and numerical modeling of tokamak plasmas is simultaneously an essential component of effective reactor design, and a great research barrier. Tokamak operational conditions exhibit comparatively low Knudsen numbers. Kinetic effects, including kinetic waves and instabilities, Landau damping, bump-on-tail instabilities and more, are therefore highly influential in tokamak plasma dynamics. Purely fluid models are inherently incapable of capturing these effects, whereas the high dimensionality in purely kinetic models render them practically intractable for most relevant purposes.

        We consider a $\delta\!f$ decomposition model, with a macroscopic fluid background and microscopic kinetic correction, both fully coupled to each other. A similar manner of discretization is proposed to that used in the recent \texttt{STRUPHY} code \cite{Holderied_Possanner_Wang_2021, Holderied_2022, Li_et_al_2023} with a finite-element model for the background and a pseudo-particle/PiC model for the correction.

        The fluid background satisfies the full, non-linear, resistive, compressible, Hall MHD equations. \cite{Laakmann_Hu_Farrell_2022} introduces finite-element(-in-space) implicit timesteppers for the incompressible analogue to this system with structure-preserving (SP) properties in the ideal case, alongside parameter-robust preconditioners. We show that these timesteppers can derive from a finite-element-in-time (FET) (and finite-element-in-space) interpretation. The benefits of this reformulation are discussed, including the derivation of timesteppers that are higher order in time, and the quantifiable dissipative SP properties in the non-ideal, resistive case.
        
        We discuss possible options for extending this FET approach to timesteppers for the compressible case.

        The kinetic corrections satisfy linearized Boltzmann equations. Using a Lénard--Bernstein collision operator, these take Fokker--Planck-like forms \cite{Fokker_1914, Planck_1917} wherein pseudo-particles in the numerical model obey the neoclassical transport equations, with particle-independent Brownian drift terms. This offers a rigorous methodology for incorporating collisions into the particle transport model, without coupling the equations of motions for each particle.
        
        Works by Chen, Chacón et al. \cite{Chen_Chacón_Barnes_2011, Chacón_Chen_Barnes_2013, Chen_Chacón_2014, Chen_Chacón_2015} have developed structure-preserving particle pushers for neoclassical transport in the Vlasov equations, derived from Crank--Nicolson integrators. We show these too can can derive from a FET interpretation, similarly offering potential extensions to higher-order-in-time particle pushers. The FET formulation is used also to consider how the stochastic drift terms can be incorporated into the pushers. Stochastic gyrokinetic expansions are also discussed.

        Different options for the numerical implementation of these schemes are considered.

        Due to the efficacy of FET in the development of SP timesteppers for both the fluid and kinetic component, we hope this approach will prove effective in the future for developing SP timesteppers for the full hybrid model. We hope this will give us the opportunity to incorporate previously inaccessible kinetic effects into the highly effective, modern, finite-element MHD models.
    \end{abstract}
    
    
    \newpage
    \tableofcontents
    
    
    \newpage
    \pagenumbering{arabic}
    %\linenumbers\renewcommand\thelinenumber{\color{black!50}\arabic{linenumber}}
            \input{0 - introduction/main.tex}
        \part{Research}
            \input{1 - low-noise PiC models/main.tex}
            \input{2 - kinetic component/main.tex}
            \input{3 - fluid component/main.tex}
            \input{4 - numerical implementation/main.tex}
        \part{Project Overview}
            \input{5 - research plan/main.tex}
            \input{6 - summary/main.tex}
    
    
    %\section{}
    \newpage
    \pagenumbering{gobble}
        \printbibliography


    \newpage
    \pagenumbering{roman}
    \appendix
        \part{Appendices}
            \input{8 - Hilbert complexes/main.tex}
            \input{9 - weak conservation proofs/main.tex}
\end{document}

            \documentclass[12pt, a4paper]{report}

\input{template/main.tex}

\title{\BA{Title in Progress...}}
\author{Boris Andrews}
\affil{Mathematical Institute, University of Oxford}
\date{\today}


\begin{document}
    \pagenumbering{gobble}
    \maketitle
    
    
    \begin{abstract}
        Magnetic confinement reactors---in particular tokamaks---offer one of the most promising options for achieving practical nuclear fusion, with the potential to provide virtually limitless, clean energy. The theoretical and numerical modeling of tokamak plasmas is simultaneously an essential component of effective reactor design, and a great research barrier. Tokamak operational conditions exhibit comparatively low Knudsen numbers. Kinetic effects, including kinetic waves and instabilities, Landau damping, bump-on-tail instabilities and more, are therefore highly influential in tokamak plasma dynamics. Purely fluid models are inherently incapable of capturing these effects, whereas the high dimensionality in purely kinetic models render them practically intractable for most relevant purposes.

        We consider a $\delta\!f$ decomposition model, with a macroscopic fluid background and microscopic kinetic correction, both fully coupled to each other. A similar manner of discretization is proposed to that used in the recent \texttt{STRUPHY} code \cite{Holderied_Possanner_Wang_2021, Holderied_2022, Li_et_al_2023} with a finite-element model for the background and a pseudo-particle/PiC model for the correction.

        The fluid background satisfies the full, non-linear, resistive, compressible, Hall MHD equations. \cite{Laakmann_Hu_Farrell_2022} introduces finite-element(-in-space) implicit timesteppers for the incompressible analogue to this system with structure-preserving (SP) properties in the ideal case, alongside parameter-robust preconditioners. We show that these timesteppers can derive from a finite-element-in-time (FET) (and finite-element-in-space) interpretation. The benefits of this reformulation are discussed, including the derivation of timesteppers that are higher order in time, and the quantifiable dissipative SP properties in the non-ideal, resistive case.
        
        We discuss possible options for extending this FET approach to timesteppers for the compressible case.

        The kinetic corrections satisfy linearized Boltzmann equations. Using a Lénard--Bernstein collision operator, these take Fokker--Planck-like forms \cite{Fokker_1914, Planck_1917} wherein pseudo-particles in the numerical model obey the neoclassical transport equations, with particle-independent Brownian drift terms. This offers a rigorous methodology for incorporating collisions into the particle transport model, without coupling the equations of motions for each particle.
        
        Works by Chen, Chacón et al. \cite{Chen_Chacón_Barnes_2011, Chacón_Chen_Barnes_2013, Chen_Chacón_2014, Chen_Chacón_2015} have developed structure-preserving particle pushers for neoclassical transport in the Vlasov equations, derived from Crank--Nicolson integrators. We show these too can can derive from a FET interpretation, similarly offering potential extensions to higher-order-in-time particle pushers. The FET formulation is used also to consider how the stochastic drift terms can be incorporated into the pushers. Stochastic gyrokinetic expansions are also discussed.

        Different options for the numerical implementation of these schemes are considered.

        Due to the efficacy of FET in the development of SP timesteppers for both the fluid and kinetic component, we hope this approach will prove effective in the future for developing SP timesteppers for the full hybrid model. We hope this will give us the opportunity to incorporate previously inaccessible kinetic effects into the highly effective, modern, finite-element MHD models.
    \end{abstract}
    
    
    \newpage
    \tableofcontents
    
    
    \newpage
    \pagenumbering{arabic}
    %\linenumbers\renewcommand\thelinenumber{\color{black!50}\arabic{linenumber}}
            \input{0 - introduction/main.tex}
        \part{Research}
            \input{1 - low-noise PiC models/main.tex}
            \input{2 - kinetic component/main.tex}
            \input{3 - fluid component/main.tex}
            \input{4 - numerical implementation/main.tex}
        \part{Project Overview}
            \input{5 - research plan/main.tex}
            \input{6 - summary/main.tex}
    
    
    %\section{}
    \newpage
    \pagenumbering{gobble}
        \printbibliography


    \newpage
    \pagenumbering{roman}
    \appendix
        \part{Appendices}
            \input{8 - Hilbert complexes/main.tex}
            \input{9 - weak conservation proofs/main.tex}
\end{document}

        \part{Project Overview}
            \documentclass[12pt, a4paper]{report}

\input{template/main.tex}

\title{\BA{Title in Progress...}}
\author{Boris Andrews}
\affil{Mathematical Institute, University of Oxford}
\date{\today}


\begin{document}
    \pagenumbering{gobble}
    \maketitle
    
    
    \begin{abstract}
        Magnetic confinement reactors---in particular tokamaks---offer one of the most promising options for achieving practical nuclear fusion, with the potential to provide virtually limitless, clean energy. The theoretical and numerical modeling of tokamak plasmas is simultaneously an essential component of effective reactor design, and a great research barrier. Tokamak operational conditions exhibit comparatively low Knudsen numbers. Kinetic effects, including kinetic waves and instabilities, Landau damping, bump-on-tail instabilities and more, are therefore highly influential in tokamak plasma dynamics. Purely fluid models are inherently incapable of capturing these effects, whereas the high dimensionality in purely kinetic models render them practically intractable for most relevant purposes.

        We consider a $\delta\!f$ decomposition model, with a macroscopic fluid background and microscopic kinetic correction, both fully coupled to each other. A similar manner of discretization is proposed to that used in the recent \texttt{STRUPHY} code \cite{Holderied_Possanner_Wang_2021, Holderied_2022, Li_et_al_2023} with a finite-element model for the background and a pseudo-particle/PiC model for the correction.

        The fluid background satisfies the full, non-linear, resistive, compressible, Hall MHD equations. \cite{Laakmann_Hu_Farrell_2022} introduces finite-element(-in-space) implicit timesteppers for the incompressible analogue to this system with structure-preserving (SP) properties in the ideal case, alongside parameter-robust preconditioners. We show that these timesteppers can derive from a finite-element-in-time (FET) (and finite-element-in-space) interpretation. The benefits of this reformulation are discussed, including the derivation of timesteppers that are higher order in time, and the quantifiable dissipative SP properties in the non-ideal, resistive case.
        
        We discuss possible options for extending this FET approach to timesteppers for the compressible case.

        The kinetic corrections satisfy linearized Boltzmann equations. Using a Lénard--Bernstein collision operator, these take Fokker--Planck-like forms \cite{Fokker_1914, Planck_1917} wherein pseudo-particles in the numerical model obey the neoclassical transport equations, with particle-independent Brownian drift terms. This offers a rigorous methodology for incorporating collisions into the particle transport model, without coupling the equations of motions for each particle.
        
        Works by Chen, Chacón et al. \cite{Chen_Chacón_Barnes_2011, Chacón_Chen_Barnes_2013, Chen_Chacón_2014, Chen_Chacón_2015} have developed structure-preserving particle pushers for neoclassical transport in the Vlasov equations, derived from Crank--Nicolson integrators. We show these too can can derive from a FET interpretation, similarly offering potential extensions to higher-order-in-time particle pushers. The FET formulation is used also to consider how the stochastic drift terms can be incorporated into the pushers. Stochastic gyrokinetic expansions are also discussed.

        Different options for the numerical implementation of these schemes are considered.

        Due to the efficacy of FET in the development of SP timesteppers for both the fluid and kinetic component, we hope this approach will prove effective in the future for developing SP timesteppers for the full hybrid model. We hope this will give us the opportunity to incorporate previously inaccessible kinetic effects into the highly effective, modern, finite-element MHD models.
    \end{abstract}
    
    
    \newpage
    \tableofcontents
    
    
    \newpage
    \pagenumbering{arabic}
    %\linenumbers\renewcommand\thelinenumber{\color{black!50}\arabic{linenumber}}
            \input{0 - introduction/main.tex}
        \part{Research}
            \input{1 - low-noise PiC models/main.tex}
            \input{2 - kinetic component/main.tex}
            \input{3 - fluid component/main.tex}
            \input{4 - numerical implementation/main.tex}
        \part{Project Overview}
            \input{5 - research plan/main.tex}
            \input{6 - summary/main.tex}
    
    
    %\section{}
    \newpage
    \pagenumbering{gobble}
        \printbibliography


    \newpage
    \pagenumbering{roman}
    \appendix
        \part{Appendices}
            \input{8 - Hilbert complexes/main.tex}
            \input{9 - weak conservation proofs/main.tex}
\end{document}

            \documentclass[12pt, a4paper]{report}

\input{template/main.tex}

\title{\BA{Title in Progress...}}
\author{Boris Andrews}
\affil{Mathematical Institute, University of Oxford}
\date{\today}


\begin{document}
    \pagenumbering{gobble}
    \maketitle
    
    
    \begin{abstract}
        Magnetic confinement reactors---in particular tokamaks---offer one of the most promising options for achieving practical nuclear fusion, with the potential to provide virtually limitless, clean energy. The theoretical and numerical modeling of tokamak plasmas is simultaneously an essential component of effective reactor design, and a great research barrier. Tokamak operational conditions exhibit comparatively low Knudsen numbers. Kinetic effects, including kinetic waves and instabilities, Landau damping, bump-on-tail instabilities and more, are therefore highly influential in tokamak plasma dynamics. Purely fluid models are inherently incapable of capturing these effects, whereas the high dimensionality in purely kinetic models render them practically intractable for most relevant purposes.

        We consider a $\delta\!f$ decomposition model, with a macroscopic fluid background and microscopic kinetic correction, both fully coupled to each other. A similar manner of discretization is proposed to that used in the recent \texttt{STRUPHY} code \cite{Holderied_Possanner_Wang_2021, Holderied_2022, Li_et_al_2023} with a finite-element model for the background and a pseudo-particle/PiC model for the correction.

        The fluid background satisfies the full, non-linear, resistive, compressible, Hall MHD equations. \cite{Laakmann_Hu_Farrell_2022} introduces finite-element(-in-space) implicit timesteppers for the incompressible analogue to this system with structure-preserving (SP) properties in the ideal case, alongside parameter-robust preconditioners. We show that these timesteppers can derive from a finite-element-in-time (FET) (and finite-element-in-space) interpretation. The benefits of this reformulation are discussed, including the derivation of timesteppers that are higher order in time, and the quantifiable dissipative SP properties in the non-ideal, resistive case.
        
        We discuss possible options for extending this FET approach to timesteppers for the compressible case.

        The kinetic corrections satisfy linearized Boltzmann equations. Using a Lénard--Bernstein collision operator, these take Fokker--Planck-like forms \cite{Fokker_1914, Planck_1917} wherein pseudo-particles in the numerical model obey the neoclassical transport equations, with particle-independent Brownian drift terms. This offers a rigorous methodology for incorporating collisions into the particle transport model, without coupling the equations of motions for each particle.
        
        Works by Chen, Chacón et al. \cite{Chen_Chacón_Barnes_2011, Chacón_Chen_Barnes_2013, Chen_Chacón_2014, Chen_Chacón_2015} have developed structure-preserving particle pushers for neoclassical transport in the Vlasov equations, derived from Crank--Nicolson integrators. We show these too can can derive from a FET interpretation, similarly offering potential extensions to higher-order-in-time particle pushers. The FET formulation is used also to consider how the stochastic drift terms can be incorporated into the pushers. Stochastic gyrokinetic expansions are also discussed.

        Different options for the numerical implementation of these schemes are considered.

        Due to the efficacy of FET in the development of SP timesteppers for both the fluid and kinetic component, we hope this approach will prove effective in the future for developing SP timesteppers for the full hybrid model. We hope this will give us the opportunity to incorporate previously inaccessible kinetic effects into the highly effective, modern, finite-element MHD models.
    \end{abstract}
    
    
    \newpage
    \tableofcontents
    
    
    \newpage
    \pagenumbering{arabic}
    %\linenumbers\renewcommand\thelinenumber{\color{black!50}\arabic{linenumber}}
            \input{0 - introduction/main.tex}
        \part{Research}
            \input{1 - low-noise PiC models/main.tex}
            \input{2 - kinetic component/main.tex}
            \input{3 - fluid component/main.tex}
            \input{4 - numerical implementation/main.tex}
        \part{Project Overview}
            \input{5 - research plan/main.tex}
            \input{6 - summary/main.tex}
    
    
    %\section{}
    \newpage
    \pagenumbering{gobble}
        \printbibliography


    \newpage
    \pagenumbering{roman}
    \appendix
        \part{Appendices}
            \input{8 - Hilbert complexes/main.tex}
            \input{9 - weak conservation proofs/main.tex}
\end{document}

    
    
    %\section{}
    \newpage
    \pagenumbering{gobble}
        \printbibliography


    \newpage
    \pagenumbering{roman}
    \appendix
        \part{Appendices}
            \documentclass[12pt, a4paper]{report}

\input{template/main.tex}

\title{\BA{Title in Progress...}}
\author{Boris Andrews}
\affil{Mathematical Institute, University of Oxford}
\date{\today}


\begin{document}
    \pagenumbering{gobble}
    \maketitle
    
    
    \begin{abstract}
        Magnetic confinement reactors---in particular tokamaks---offer one of the most promising options for achieving practical nuclear fusion, with the potential to provide virtually limitless, clean energy. The theoretical and numerical modeling of tokamak plasmas is simultaneously an essential component of effective reactor design, and a great research barrier. Tokamak operational conditions exhibit comparatively low Knudsen numbers. Kinetic effects, including kinetic waves and instabilities, Landau damping, bump-on-tail instabilities and more, are therefore highly influential in tokamak plasma dynamics. Purely fluid models are inherently incapable of capturing these effects, whereas the high dimensionality in purely kinetic models render them practically intractable for most relevant purposes.

        We consider a $\delta\!f$ decomposition model, with a macroscopic fluid background and microscopic kinetic correction, both fully coupled to each other. A similar manner of discretization is proposed to that used in the recent \texttt{STRUPHY} code \cite{Holderied_Possanner_Wang_2021, Holderied_2022, Li_et_al_2023} with a finite-element model for the background and a pseudo-particle/PiC model for the correction.

        The fluid background satisfies the full, non-linear, resistive, compressible, Hall MHD equations. \cite{Laakmann_Hu_Farrell_2022} introduces finite-element(-in-space) implicit timesteppers for the incompressible analogue to this system with structure-preserving (SP) properties in the ideal case, alongside parameter-robust preconditioners. We show that these timesteppers can derive from a finite-element-in-time (FET) (and finite-element-in-space) interpretation. The benefits of this reformulation are discussed, including the derivation of timesteppers that are higher order in time, and the quantifiable dissipative SP properties in the non-ideal, resistive case.
        
        We discuss possible options for extending this FET approach to timesteppers for the compressible case.

        The kinetic corrections satisfy linearized Boltzmann equations. Using a Lénard--Bernstein collision operator, these take Fokker--Planck-like forms \cite{Fokker_1914, Planck_1917} wherein pseudo-particles in the numerical model obey the neoclassical transport equations, with particle-independent Brownian drift terms. This offers a rigorous methodology for incorporating collisions into the particle transport model, without coupling the equations of motions for each particle.
        
        Works by Chen, Chacón et al. \cite{Chen_Chacón_Barnes_2011, Chacón_Chen_Barnes_2013, Chen_Chacón_2014, Chen_Chacón_2015} have developed structure-preserving particle pushers for neoclassical transport in the Vlasov equations, derived from Crank--Nicolson integrators. We show these too can can derive from a FET interpretation, similarly offering potential extensions to higher-order-in-time particle pushers. The FET formulation is used also to consider how the stochastic drift terms can be incorporated into the pushers. Stochastic gyrokinetic expansions are also discussed.

        Different options for the numerical implementation of these schemes are considered.

        Due to the efficacy of FET in the development of SP timesteppers for both the fluid and kinetic component, we hope this approach will prove effective in the future for developing SP timesteppers for the full hybrid model. We hope this will give us the opportunity to incorporate previously inaccessible kinetic effects into the highly effective, modern, finite-element MHD models.
    \end{abstract}
    
    
    \newpage
    \tableofcontents
    
    
    \newpage
    \pagenumbering{arabic}
    %\linenumbers\renewcommand\thelinenumber{\color{black!50}\arabic{linenumber}}
            \input{0 - introduction/main.tex}
        \part{Research}
            \input{1 - low-noise PiC models/main.tex}
            \input{2 - kinetic component/main.tex}
            \input{3 - fluid component/main.tex}
            \input{4 - numerical implementation/main.tex}
        \part{Project Overview}
            \input{5 - research plan/main.tex}
            \input{6 - summary/main.tex}
    
    
    %\section{}
    \newpage
    \pagenumbering{gobble}
        \printbibliography


    \newpage
    \pagenumbering{roman}
    \appendix
        \part{Appendices}
            \input{8 - Hilbert complexes/main.tex}
            \input{9 - weak conservation proofs/main.tex}
\end{document}

            \documentclass[12pt, a4paper]{report}

\input{template/main.tex}

\title{\BA{Title in Progress...}}
\author{Boris Andrews}
\affil{Mathematical Institute, University of Oxford}
\date{\today}


\begin{document}
    \pagenumbering{gobble}
    \maketitle
    
    
    \begin{abstract}
        Magnetic confinement reactors---in particular tokamaks---offer one of the most promising options for achieving practical nuclear fusion, with the potential to provide virtually limitless, clean energy. The theoretical and numerical modeling of tokamak plasmas is simultaneously an essential component of effective reactor design, and a great research barrier. Tokamak operational conditions exhibit comparatively low Knudsen numbers. Kinetic effects, including kinetic waves and instabilities, Landau damping, bump-on-tail instabilities and more, are therefore highly influential in tokamak plasma dynamics. Purely fluid models are inherently incapable of capturing these effects, whereas the high dimensionality in purely kinetic models render them practically intractable for most relevant purposes.

        We consider a $\delta\!f$ decomposition model, with a macroscopic fluid background and microscopic kinetic correction, both fully coupled to each other. A similar manner of discretization is proposed to that used in the recent \texttt{STRUPHY} code \cite{Holderied_Possanner_Wang_2021, Holderied_2022, Li_et_al_2023} with a finite-element model for the background and a pseudo-particle/PiC model for the correction.

        The fluid background satisfies the full, non-linear, resistive, compressible, Hall MHD equations. \cite{Laakmann_Hu_Farrell_2022} introduces finite-element(-in-space) implicit timesteppers for the incompressible analogue to this system with structure-preserving (SP) properties in the ideal case, alongside parameter-robust preconditioners. We show that these timesteppers can derive from a finite-element-in-time (FET) (and finite-element-in-space) interpretation. The benefits of this reformulation are discussed, including the derivation of timesteppers that are higher order in time, and the quantifiable dissipative SP properties in the non-ideal, resistive case.
        
        We discuss possible options for extending this FET approach to timesteppers for the compressible case.

        The kinetic corrections satisfy linearized Boltzmann equations. Using a Lénard--Bernstein collision operator, these take Fokker--Planck-like forms \cite{Fokker_1914, Planck_1917} wherein pseudo-particles in the numerical model obey the neoclassical transport equations, with particle-independent Brownian drift terms. This offers a rigorous methodology for incorporating collisions into the particle transport model, without coupling the equations of motions for each particle.
        
        Works by Chen, Chacón et al. \cite{Chen_Chacón_Barnes_2011, Chacón_Chen_Barnes_2013, Chen_Chacón_2014, Chen_Chacón_2015} have developed structure-preserving particle pushers for neoclassical transport in the Vlasov equations, derived from Crank--Nicolson integrators. We show these too can can derive from a FET interpretation, similarly offering potential extensions to higher-order-in-time particle pushers. The FET formulation is used also to consider how the stochastic drift terms can be incorporated into the pushers. Stochastic gyrokinetic expansions are also discussed.

        Different options for the numerical implementation of these schemes are considered.

        Due to the efficacy of FET in the development of SP timesteppers for both the fluid and kinetic component, we hope this approach will prove effective in the future for developing SP timesteppers for the full hybrid model. We hope this will give us the opportunity to incorporate previously inaccessible kinetic effects into the highly effective, modern, finite-element MHD models.
    \end{abstract}
    
    
    \newpage
    \tableofcontents
    
    
    \newpage
    \pagenumbering{arabic}
    %\linenumbers\renewcommand\thelinenumber{\color{black!50}\arabic{linenumber}}
            \input{0 - introduction/main.tex}
        \part{Research}
            \input{1 - low-noise PiC models/main.tex}
            \input{2 - kinetic component/main.tex}
            \input{3 - fluid component/main.tex}
            \input{4 - numerical implementation/main.tex}
        \part{Project Overview}
            \input{5 - research plan/main.tex}
            \input{6 - summary/main.tex}
    
    
    %\section{}
    \newpage
    \pagenumbering{gobble}
        \printbibliography


    \newpage
    \pagenumbering{roman}
    \appendix
        \part{Appendices}
            \input{8 - Hilbert complexes/main.tex}
            \input{9 - weak conservation proofs/main.tex}
\end{document}

\end{document}

        \part{Project Overview}
            \documentclass[12pt, a4paper]{report}

\documentclass[12pt, a4paper]{report}

\input{template/main.tex}

\title{\BA{Title in Progress...}}
\author{Boris Andrews}
\affil{Mathematical Institute, University of Oxford}
\date{\today}


\begin{document}
    \pagenumbering{gobble}
    \maketitle
    
    
    \begin{abstract}
        Magnetic confinement reactors---in particular tokamaks---offer one of the most promising options for achieving practical nuclear fusion, with the potential to provide virtually limitless, clean energy. The theoretical and numerical modeling of tokamak plasmas is simultaneously an essential component of effective reactor design, and a great research barrier. Tokamak operational conditions exhibit comparatively low Knudsen numbers. Kinetic effects, including kinetic waves and instabilities, Landau damping, bump-on-tail instabilities and more, are therefore highly influential in tokamak plasma dynamics. Purely fluid models are inherently incapable of capturing these effects, whereas the high dimensionality in purely kinetic models render them practically intractable for most relevant purposes.

        We consider a $\delta\!f$ decomposition model, with a macroscopic fluid background and microscopic kinetic correction, both fully coupled to each other. A similar manner of discretization is proposed to that used in the recent \texttt{STRUPHY} code \cite{Holderied_Possanner_Wang_2021, Holderied_2022, Li_et_al_2023} with a finite-element model for the background and a pseudo-particle/PiC model for the correction.

        The fluid background satisfies the full, non-linear, resistive, compressible, Hall MHD equations. \cite{Laakmann_Hu_Farrell_2022} introduces finite-element(-in-space) implicit timesteppers for the incompressible analogue to this system with structure-preserving (SP) properties in the ideal case, alongside parameter-robust preconditioners. We show that these timesteppers can derive from a finite-element-in-time (FET) (and finite-element-in-space) interpretation. The benefits of this reformulation are discussed, including the derivation of timesteppers that are higher order in time, and the quantifiable dissipative SP properties in the non-ideal, resistive case.
        
        We discuss possible options for extending this FET approach to timesteppers for the compressible case.

        The kinetic corrections satisfy linearized Boltzmann equations. Using a Lénard--Bernstein collision operator, these take Fokker--Planck-like forms \cite{Fokker_1914, Planck_1917} wherein pseudo-particles in the numerical model obey the neoclassical transport equations, with particle-independent Brownian drift terms. This offers a rigorous methodology for incorporating collisions into the particle transport model, without coupling the equations of motions for each particle.
        
        Works by Chen, Chacón et al. \cite{Chen_Chacón_Barnes_2011, Chacón_Chen_Barnes_2013, Chen_Chacón_2014, Chen_Chacón_2015} have developed structure-preserving particle pushers for neoclassical transport in the Vlasov equations, derived from Crank--Nicolson integrators. We show these too can can derive from a FET interpretation, similarly offering potential extensions to higher-order-in-time particle pushers. The FET formulation is used also to consider how the stochastic drift terms can be incorporated into the pushers. Stochastic gyrokinetic expansions are also discussed.

        Different options for the numerical implementation of these schemes are considered.

        Due to the efficacy of FET in the development of SP timesteppers for both the fluid and kinetic component, we hope this approach will prove effective in the future for developing SP timesteppers for the full hybrid model. We hope this will give us the opportunity to incorporate previously inaccessible kinetic effects into the highly effective, modern, finite-element MHD models.
    \end{abstract}
    
    
    \newpage
    \tableofcontents
    
    
    \newpage
    \pagenumbering{arabic}
    %\linenumbers\renewcommand\thelinenumber{\color{black!50}\arabic{linenumber}}
            \input{0 - introduction/main.tex}
        \part{Research}
            \input{1 - low-noise PiC models/main.tex}
            \input{2 - kinetic component/main.tex}
            \input{3 - fluid component/main.tex}
            \input{4 - numerical implementation/main.tex}
        \part{Project Overview}
            \input{5 - research plan/main.tex}
            \input{6 - summary/main.tex}
    
    
    %\section{}
    \newpage
    \pagenumbering{gobble}
        \printbibliography


    \newpage
    \pagenumbering{roman}
    \appendix
        \part{Appendices}
            \input{8 - Hilbert complexes/main.tex}
            \input{9 - weak conservation proofs/main.tex}
\end{document}


\title{\BA{Title in Progress...}}
\author{Boris Andrews}
\affil{Mathematical Institute, University of Oxford}
\date{\today}


\begin{document}
    \pagenumbering{gobble}
    \maketitle
    
    
    \begin{abstract}
        Magnetic confinement reactors---in particular tokamaks---offer one of the most promising options for achieving practical nuclear fusion, with the potential to provide virtually limitless, clean energy. The theoretical and numerical modeling of tokamak plasmas is simultaneously an essential component of effective reactor design, and a great research barrier. Tokamak operational conditions exhibit comparatively low Knudsen numbers. Kinetic effects, including kinetic waves and instabilities, Landau damping, bump-on-tail instabilities and more, are therefore highly influential in tokamak plasma dynamics. Purely fluid models are inherently incapable of capturing these effects, whereas the high dimensionality in purely kinetic models render them practically intractable for most relevant purposes.

        We consider a $\delta\!f$ decomposition model, with a macroscopic fluid background and microscopic kinetic correction, both fully coupled to each other. A similar manner of discretization is proposed to that used in the recent \texttt{STRUPHY} code \cite{Holderied_Possanner_Wang_2021, Holderied_2022, Li_et_al_2023} with a finite-element model for the background and a pseudo-particle/PiC model for the correction.

        The fluid background satisfies the full, non-linear, resistive, compressible, Hall MHD equations. \cite{Laakmann_Hu_Farrell_2022} introduces finite-element(-in-space) implicit timesteppers for the incompressible analogue to this system with structure-preserving (SP) properties in the ideal case, alongside parameter-robust preconditioners. We show that these timesteppers can derive from a finite-element-in-time (FET) (and finite-element-in-space) interpretation. The benefits of this reformulation are discussed, including the derivation of timesteppers that are higher order in time, and the quantifiable dissipative SP properties in the non-ideal, resistive case.
        
        We discuss possible options for extending this FET approach to timesteppers for the compressible case.

        The kinetic corrections satisfy linearized Boltzmann equations. Using a Lénard--Bernstein collision operator, these take Fokker--Planck-like forms \cite{Fokker_1914, Planck_1917} wherein pseudo-particles in the numerical model obey the neoclassical transport equations, with particle-independent Brownian drift terms. This offers a rigorous methodology for incorporating collisions into the particle transport model, without coupling the equations of motions for each particle.
        
        Works by Chen, Chacón et al. \cite{Chen_Chacón_Barnes_2011, Chacón_Chen_Barnes_2013, Chen_Chacón_2014, Chen_Chacón_2015} have developed structure-preserving particle pushers for neoclassical transport in the Vlasov equations, derived from Crank--Nicolson integrators. We show these too can can derive from a FET interpretation, similarly offering potential extensions to higher-order-in-time particle pushers. The FET formulation is used also to consider how the stochastic drift terms can be incorporated into the pushers. Stochastic gyrokinetic expansions are also discussed.

        Different options for the numerical implementation of these schemes are considered.

        Due to the efficacy of FET in the development of SP timesteppers for both the fluid and kinetic component, we hope this approach will prove effective in the future for developing SP timesteppers for the full hybrid model. We hope this will give us the opportunity to incorporate previously inaccessible kinetic effects into the highly effective, modern, finite-element MHD models.
    \end{abstract}
    
    
    \newpage
    \tableofcontents
    
    
    \newpage
    \pagenumbering{arabic}
    %\linenumbers\renewcommand\thelinenumber{\color{black!50}\arabic{linenumber}}
            \documentclass[12pt, a4paper]{report}

\input{template/main.tex}

\title{\BA{Title in Progress...}}
\author{Boris Andrews}
\affil{Mathematical Institute, University of Oxford}
\date{\today}


\begin{document}
    \pagenumbering{gobble}
    \maketitle
    
    
    \begin{abstract}
        Magnetic confinement reactors---in particular tokamaks---offer one of the most promising options for achieving practical nuclear fusion, with the potential to provide virtually limitless, clean energy. The theoretical and numerical modeling of tokamak plasmas is simultaneously an essential component of effective reactor design, and a great research barrier. Tokamak operational conditions exhibit comparatively low Knudsen numbers. Kinetic effects, including kinetic waves and instabilities, Landau damping, bump-on-tail instabilities and more, are therefore highly influential in tokamak plasma dynamics. Purely fluid models are inherently incapable of capturing these effects, whereas the high dimensionality in purely kinetic models render them practically intractable for most relevant purposes.

        We consider a $\delta\!f$ decomposition model, with a macroscopic fluid background and microscopic kinetic correction, both fully coupled to each other. A similar manner of discretization is proposed to that used in the recent \texttt{STRUPHY} code \cite{Holderied_Possanner_Wang_2021, Holderied_2022, Li_et_al_2023} with a finite-element model for the background and a pseudo-particle/PiC model for the correction.

        The fluid background satisfies the full, non-linear, resistive, compressible, Hall MHD equations. \cite{Laakmann_Hu_Farrell_2022} introduces finite-element(-in-space) implicit timesteppers for the incompressible analogue to this system with structure-preserving (SP) properties in the ideal case, alongside parameter-robust preconditioners. We show that these timesteppers can derive from a finite-element-in-time (FET) (and finite-element-in-space) interpretation. The benefits of this reformulation are discussed, including the derivation of timesteppers that are higher order in time, and the quantifiable dissipative SP properties in the non-ideal, resistive case.
        
        We discuss possible options for extending this FET approach to timesteppers for the compressible case.

        The kinetic corrections satisfy linearized Boltzmann equations. Using a Lénard--Bernstein collision operator, these take Fokker--Planck-like forms \cite{Fokker_1914, Planck_1917} wherein pseudo-particles in the numerical model obey the neoclassical transport equations, with particle-independent Brownian drift terms. This offers a rigorous methodology for incorporating collisions into the particle transport model, without coupling the equations of motions for each particle.
        
        Works by Chen, Chacón et al. \cite{Chen_Chacón_Barnes_2011, Chacón_Chen_Barnes_2013, Chen_Chacón_2014, Chen_Chacón_2015} have developed structure-preserving particle pushers for neoclassical transport in the Vlasov equations, derived from Crank--Nicolson integrators. We show these too can can derive from a FET interpretation, similarly offering potential extensions to higher-order-in-time particle pushers. The FET formulation is used also to consider how the stochastic drift terms can be incorporated into the pushers. Stochastic gyrokinetic expansions are also discussed.

        Different options for the numerical implementation of these schemes are considered.

        Due to the efficacy of FET in the development of SP timesteppers for both the fluid and kinetic component, we hope this approach will prove effective in the future for developing SP timesteppers for the full hybrid model. We hope this will give us the opportunity to incorporate previously inaccessible kinetic effects into the highly effective, modern, finite-element MHD models.
    \end{abstract}
    
    
    \newpage
    \tableofcontents
    
    
    \newpage
    \pagenumbering{arabic}
    %\linenumbers\renewcommand\thelinenumber{\color{black!50}\arabic{linenumber}}
            \input{0 - introduction/main.tex}
        \part{Research}
            \input{1 - low-noise PiC models/main.tex}
            \input{2 - kinetic component/main.tex}
            \input{3 - fluid component/main.tex}
            \input{4 - numerical implementation/main.tex}
        \part{Project Overview}
            \input{5 - research plan/main.tex}
            \input{6 - summary/main.tex}
    
    
    %\section{}
    \newpage
    \pagenumbering{gobble}
        \printbibliography


    \newpage
    \pagenumbering{roman}
    \appendix
        \part{Appendices}
            \input{8 - Hilbert complexes/main.tex}
            \input{9 - weak conservation proofs/main.tex}
\end{document}

        \part{Research}
            \documentclass[12pt, a4paper]{report}

\input{template/main.tex}

\title{\BA{Title in Progress...}}
\author{Boris Andrews}
\affil{Mathematical Institute, University of Oxford}
\date{\today}


\begin{document}
    \pagenumbering{gobble}
    \maketitle
    
    
    \begin{abstract}
        Magnetic confinement reactors---in particular tokamaks---offer one of the most promising options for achieving practical nuclear fusion, with the potential to provide virtually limitless, clean energy. The theoretical and numerical modeling of tokamak plasmas is simultaneously an essential component of effective reactor design, and a great research barrier. Tokamak operational conditions exhibit comparatively low Knudsen numbers. Kinetic effects, including kinetic waves and instabilities, Landau damping, bump-on-tail instabilities and more, are therefore highly influential in tokamak plasma dynamics. Purely fluid models are inherently incapable of capturing these effects, whereas the high dimensionality in purely kinetic models render them practically intractable for most relevant purposes.

        We consider a $\delta\!f$ decomposition model, with a macroscopic fluid background and microscopic kinetic correction, both fully coupled to each other. A similar manner of discretization is proposed to that used in the recent \texttt{STRUPHY} code \cite{Holderied_Possanner_Wang_2021, Holderied_2022, Li_et_al_2023} with a finite-element model for the background and a pseudo-particle/PiC model for the correction.

        The fluid background satisfies the full, non-linear, resistive, compressible, Hall MHD equations. \cite{Laakmann_Hu_Farrell_2022} introduces finite-element(-in-space) implicit timesteppers for the incompressible analogue to this system with structure-preserving (SP) properties in the ideal case, alongside parameter-robust preconditioners. We show that these timesteppers can derive from a finite-element-in-time (FET) (and finite-element-in-space) interpretation. The benefits of this reformulation are discussed, including the derivation of timesteppers that are higher order in time, and the quantifiable dissipative SP properties in the non-ideal, resistive case.
        
        We discuss possible options for extending this FET approach to timesteppers for the compressible case.

        The kinetic corrections satisfy linearized Boltzmann equations. Using a Lénard--Bernstein collision operator, these take Fokker--Planck-like forms \cite{Fokker_1914, Planck_1917} wherein pseudo-particles in the numerical model obey the neoclassical transport equations, with particle-independent Brownian drift terms. This offers a rigorous methodology for incorporating collisions into the particle transport model, without coupling the equations of motions for each particle.
        
        Works by Chen, Chacón et al. \cite{Chen_Chacón_Barnes_2011, Chacón_Chen_Barnes_2013, Chen_Chacón_2014, Chen_Chacón_2015} have developed structure-preserving particle pushers for neoclassical transport in the Vlasov equations, derived from Crank--Nicolson integrators. We show these too can can derive from a FET interpretation, similarly offering potential extensions to higher-order-in-time particle pushers. The FET formulation is used also to consider how the stochastic drift terms can be incorporated into the pushers. Stochastic gyrokinetic expansions are also discussed.

        Different options for the numerical implementation of these schemes are considered.

        Due to the efficacy of FET in the development of SP timesteppers for both the fluid and kinetic component, we hope this approach will prove effective in the future for developing SP timesteppers for the full hybrid model. We hope this will give us the opportunity to incorporate previously inaccessible kinetic effects into the highly effective, modern, finite-element MHD models.
    \end{abstract}
    
    
    \newpage
    \tableofcontents
    
    
    \newpage
    \pagenumbering{arabic}
    %\linenumbers\renewcommand\thelinenumber{\color{black!50}\arabic{linenumber}}
            \input{0 - introduction/main.tex}
        \part{Research}
            \input{1 - low-noise PiC models/main.tex}
            \input{2 - kinetic component/main.tex}
            \input{3 - fluid component/main.tex}
            \input{4 - numerical implementation/main.tex}
        \part{Project Overview}
            \input{5 - research plan/main.tex}
            \input{6 - summary/main.tex}
    
    
    %\section{}
    \newpage
    \pagenumbering{gobble}
        \printbibliography


    \newpage
    \pagenumbering{roman}
    \appendix
        \part{Appendices}
            \input{8 - Hilbert complexes/main.tex}
            \input{9 - weak conservation proofs/main.tex}
\end{document}

            \documentclass[12pt, a4paper]{report}

\input{template/main.tex}

\title{\BA{Title in Progress...}}
\author{Boris Andrews}
\affil{Mathematical Institute, University of Oxford}
\date{\today}


\begin{document}
    \pagenumbering{gobble}
    \maketitle
    
    
    \begin{abstract}
        Magnetic confinement reactors---in particular tokamaks---offer one of the most promising options for achieving practical nuclear fusion, with the potential to provide virtually limitless, clean energy. The theoretical and numerical modeling of tokamak plasmas is simultaneously an essential component of effective reactor design, and a great research barrier. Tokamak operational conditions exhibit comparatively low Knudsen numbers. Kinetic effects, including kinetic waves and instabilities, Landau damping, bump-on-tail instabilities and more, are therefore highly influential in tokamak plasma dynamics. Purely fluid models are inherently incapable of capturing these effects, whereas the high dimensionality in purely kinetic models render them practically intractable for most relevant purposes.

        We consider a $\delta\!f$ decomposition model, with a macroscopic fluid background and microscopic kinetic correction, both fully coupled to each other. A similar manner of discretization is proposed to that used in the recent \texttt{STRUPHY} code \cite{Holderied_Possanner_Wang_2021, Holderied_2022, Li_et_al_2023} with a finite-element model for the background and a pseudo-particle/PiC model for the correction.

        The fluid background satisfies the full, non-linear, resistive, compressible, Hall MHD equations. \cite{Laakmann_Hu_Farrell_2022} introduces finite-element(-in-space) implicit timesteppers for the incompressible analogue to this system with structure-preserving (SP) properties in the ideal case, alongside parameter-robust preconditioners. We show that these timesteppers can derive from a finite-element-in-time (FET) (and finite-element-in-space) interpretation. The benefits of this reformulation are discussed, including the derivation of timesteppers that are higher order in time, and the quantifiable dissipative SP properties in the non-ideal, resistive case.
        
        We discuss possible options for extending this FET approach to timesteppers for the compressible case.

        The kinetic corrections satisfy linearized Boltzmann equations. Using a Lénard--Bernstein collision operator, these take Fokker--Planck-like forms \cite{Fokker_1914, Planck_1917} wherein pseudo-particles in the numerical model obey the neoclassical transport equations, with particle-independent Brownian drift terms. This offers a rigorous methodology for incorporating collisions into the particle transport model, without coupling the equations of motions for each particle.
        
        Works by Chen, Chacón et al. \cite{Chen_Chacón_Barnes_2011, Chacón_Chen_Barnes_2013, Chen_Chacón_2014, Chen_Chacón_2015} have developed structure-preserving particle pushers for neoclassical transport in the Vlasov equations, derived from Crank--Nicolson integrators. We show these too can can derive from a FET interpretation, similarly offering potential extensions to higher-order-in-time particle pushers. The FET formulation is used also to consider how the stochastic drift terms can be incorporated into the pushers. Stochastic gyrokinetic expansions are also discussed.

        Different options for the numerical implementation of these schemes are considered.

        Due to the efficacy of FET in the development of SP timesteppers for both the fluid and kinetic component, we hope this approach will prove effective in the future for developing SP timesteppers for the full hybrid model. We hope this will give us the opportunity to incorporate previously inaccessible kinetic effects into the highly effective, modern, finite-element MHD models.
    \end{abstract}
    
    
    \newpage
    \tableofcontents
    
    
    \newpage
    \pagenumbering{arabic}
    %\linenumbers\renewcommand\thelinenumber{\color{black!50}\arabic{linenumber}}
            \input{0 - introduction/main.tex}
        \part{Research}
            \input{1 - low-noise PiC models/main.tex}
            \input{2 - kinetic component/main.tex}
            \input{3 - fluid component/main.tex}
            \input{4 - numerical implementation/main.tex}
        \part{Project Overview}
            \input{5 - research plan/main.tex}
            \input{6 - summary/main.tex}
    
    
    %\section{}
    \newpage
    \pagenumbering{gobble}
        \printbibliography


    \newpage
    \pagenumbering{roman}
    \appendix
        \part{Appendices}
            \input{8 - Hilbert complexes/main.tex}
            \input{9 - weak conservation proofs/main.tex}
\end{document}

            \documentclass[12pt, a4paper]{report}

\input{template/main.tex}

\title{\BA{Title in Progress...}}
\author{Boris Andrews}
\affil{Mathematical Institute, University of Oxford}
\date{\today}


\begin{document}
    \pagenumbering{gobble}
    \maketitle
    
    
    \begin{abstract}
        Magnetic confinement reactors---in particular tokamaks---offer one of the most promising options for achieving practical nuclear fusion, with the potential to provide virtually limitless, clean energy. The theoretical and numerical modeling of tokamak plasmas is simultaneously an essential component of effective reactor design, and a great research barrier. Tokamak operational conditions exhibit comparatively low Knudsen numbers. Kinetic effects, including kinetic waves and instabilities, Landau damping, bump-on-tail instabilities and more, are therefore highly influential in tokamak plasma dynamics. Purely fluid models are inherently incapable of capturing these effects, whereas the high dimensionality in purely kinetic models render them practically intractable for most relevant purposes.

        We consider a $\delta\!f$ decomposition model, with a macroscopic fluid background and microscopic kinetic correction, both fully coupled to each other. A similar manner of discretization is proposed to that used in the recent \texttt{STRUPHY} code \cite{Holderied_Possanner_Wang_2021, Holderied_2022, Li_et_al_2023} with a finite-element model for the background and a pseudo-particle/PiC model for the correction.

        The fluid background satisfies the full, non-linear, resistive, compressible, Hall MHD equations. \cite{Laakmann_Hu_Farrell_2022} introduces finite-element(-in-space) implicit timesteppers for the incompressible analogue to this system with structure-preserving (SP) properties in the ideal case, alongside parameter-robust preconditioners. We show that these timesteppers can derive from a finite-element-in-time (FET) (and finite-element-in-space) interpretation. The benefits of this reformulation are discussed, including the derivation of timesteppers that are higher order in time, and the quantifiable dissipative SP properties in the non-ideal, resistive case.
        
        We discuss possible options for extending this FET approach to timesteppers for the compressible case.

        The kinetic corrections satisfy linearized Boltzmann equations. Using a Lénard--Bernstein collision operator, these take Fokker--Planck-like forms \cite{Fokker_1914, Planck_1917} wherein pseudo-particles in the numerical model obey the neoclassical transport equations, with particle-independent Brownian drift terms. This offers a rigorous methodology for incorporating collisions into the particle transport model, without coupling the equations of motions for each particle.
        
        Works by Chen, Chacón et al. \cite{Chen_Chacón_Barnes_2011, Chacón_Chen_Barnes_2013, Chen_Chacón_2014, Chen_Chacón_2015} have developed structure-preserving particle pushers for neoclassical transport in the Vlasov equations, derived from Crank--Nicolson integrators. We show these too can can derive from a FET interpretation, similarly offering potential extensions to higher-order-in-time particle pushers. The FET formulation is used also to consider how the stochastic drift terms can be incorporated into the pushers. Stochastic gyrokinetic expansions are also discussed.

        Different options for the numerical implementation of these schemes are considered.

        Due to the efficacy of FET in the development of SP timesteppers for both the fluid and kinetic component, we hope this approach will prove effective in the future for developing SP timesteppers for the full hybrid model. We hope this will give us the opportunity to incorporate previously inaccessible kinetic effects into the highly effective, modern, finite-element MHD models.
    \end{abstract}
    
    
    \newpage
    \tableofcontents
    
    
    \newpage
    \pagenumbering{arabic}
    %\linenumbers\renewcommand\thelinenumber{\color{black!50}\arabic{linenumber}}
            \input{0 - introduction/main.tex}
        \part{Research}
            \input{1 - low-noise PiC models/main.tex}
            \input{2 - kinetic component/main.tex}
            \input{3 - fluid component/main.tex}
            \input{4 - numerical implementation/main.tex}
        \part{Project Overview}
            \input{5 - research plan/main.tex}
            \input{6 - summary/main.tex}
    
    
    %\section{}
    \newpage
    \pagenumbering{gobble}
        \printbibliography


    \newpage
    \pagenumbering{roman}
    \appendix
        \part{Appendices}
            \input{8 - Hilbert complexes/main.tex}
            \input{9 - weak conservation proofs/main.tex}
\end{document}

            \documentclass[12pt, a4paper]{report}

\input{template/main.tex}

\title{\BA{Title in Progress...}}
\author{Boris Andrews}
\affil{Mathematical Institute, University of Oxford}
\date{\today}


\begin{document}
    \pagenumbering{gobble}
    \maketitle
    
    
    \begin{abstract}
        Magnetic confinement reactors---in particular tokamaks---offer one of the most promising options for achieving practical nuclear fusion, with the potential to provide virtually limitless, clean energy. The theoretical and numerical modeling of tokamak plasmas is simultaneously an essential component of effective reactor design, and a great research barrier. Tokamak operational conditions exhibit comparatively low Knudsen numbers. Kinetic effects, including kinetic waves and instabilities, Landau damping, bump-on-tail instabilities and more, are therefore highly influential in tokamak plasma dynamics. Purely fluid models are inherently incapable of capturing these effects, whereas the high dimensionality in purely kinetic models render them practically intractable for most relevant purposes.

        We consider a $\delta\!f$ decomposition model, with a macroscopic fluid background and microscopic kinetic correction, both fully coupled to each other. A similar manner of discretization is proposed to that used in the recent \texttt{STRUPHY} code \cite{Holderied_Possanner_Wang_2021, Holderied_2022, Li_et_al_2023} with a finite-element model for the background and a pseudo-particle/PiC model for the correction.

        The fluid background satisfies the full, non-linear, resistive, compressible, Hall MHD equations. \cite{Laakmann_Hu_Farrell_2022} introduces finite-element(-in-space) implicit timesteppers for the incompressible analogue to this system with structure-preserving (SP) properties in the ideal case, alongside parameter-robust preconditioners. We show that these timesteppers can derive from a finite-element-in-time (FET) (and finite-element-in-space) interpretation. The benefits of this reformulation are discussed, including the derivation of timesteppers that are higher order in time, and the quantifiable dissipative SP properties in the non-ideal, resistive case.
        
        We discuss possible options for extending this FET approach to timesteppers for the compressible case.

        The kinetic corrections satisfy linearized Boltzmann equations. Using a Lénard--Bernstein collision operator, these take Fokker--Planck-like forms \cite{Fokker_1914, Planck_1917} wherein pseudo-particles in the numerical model obey the neoclassical transport equations, with particle-independent Brownian drift terms. This offers a rigorous methodology for incorporating collisions into the particle transport model, without coupling the equations of motions for each particle.
        
        Works by Chen, Chacón et al. \cite{Chen_Chacón_Barnes_2011, Chacón_Chen_Barnes_2013, Chen_Chacón_2014, Chen_Chacón_2015} have developed structure-preserving particle pushers for neoclassical transport in the Vlasov equations, derived from Crank--Nicolson integrators. We show these too can can derive from a FET interpretation, similarly offering potential extensions to higher-order-in-time particle pushers. The FET formulation is used also to consider how the stochastic drift terms can be incorporated into the pushers. Stochastic gyrokinetic expansions are also discussed.

        Different options for the numerical implementation of these schemes are considered.

        Due to the efficacy of FET in the development of SP timesteppers for both the fluid and kinetic component, we hope this approach will prove effective in the future for developing SP timesteppers for the full hybrid model. We hope this will give us the opportunity to incorporate previously inaccessible kinetic effects into the highly effective, modern, finite-element MHD models.
    \end{abstract}
    
    
    \newpage
    \tableofcontents
    
    
    \newpage
    \pagenumbering{arabic}
    %\linenumbers\renewcommand\thelinenumber{\color{black!50}\arabic{linenumber}}
            \input{0 - introduction/main.tex}
        \part{Research}
            \input{1 - low-noise PiC models/main.tex}
            \input{2 - kinetic component/main.tex}
            \input{3 - fluid component/main.tex}
            \input{4 - numerical implementation/main.tex}
        \part{Project Overview}
            \input{5 - research plan/main.tex}
            \input{6 - summary/main.tex}
    
    
    %\section{}
    \newpage
    \pagenumbering{gobble}
        \printbibliography


    \newpage
    \pagenumbering{roman}
    \appendix
        \part{Appendices}
            \input{8 - Hilbert complexes/main.tex}
            \input{9 - weak conservation proofs/main.tex}
\end{document}

        \part{Project Overview}
            \documentclass[12pt, a4paper]{report}

\input{template/main.tex}

\title{\BA{Title in Progress...}}
\author{Boris Andrews}
\affil{Mathematical Institute, University of Oxford}
\date{\today}


\begin{document}
    \pagenumbering{gobble}
    \maketitle
    
    
    \begin{abstract}
        Magnetic confinement reactors---in particular tokamaks---offer one of the most promising options for achieving practical nuclear fusion, with the potential to provide virtually limitless, clean energy. The theoretical and numerical modeling of tokamak plasmas is simultaneously an essential component of effective reactor design, and a great research barrier. Tokamak operational conditions exhibit comparatively low Knudsen numbers. Kinetic effects, including kinetic waves and instabilities, Landau damping, bump-on-tail instabilities and more, are therefore highly influential in tokamak plasma dynamics. Purely fluid models are inherently incapable of capturing these effects, whereas the high dimensionality in purely kinetic models render them practically intractable for most relevant purposes.

        We consider a $\delta\!f$ decomposition model, with a macroscopic fluid background and microscopic kinetic correction, both fully coupled to each other. A similar manner of discretization is proposed to that used in the recent \texttt{STRUPHY} code \cite{Holderied_Possanner_Wang_2021, Holderied_2022, Li_et_al_2023} with a finite-element model for the background and a pseudo-particle/PiC model for the correction.

        The fluid background satisfies the full, non-linear, resistive, compressible, Hall MHD equations. \cite{Laakmann_Hu_Farrell_2022} introduces finite-element(-in-space) implicit timesteppers for the incompressible analogue to this system with structure-preserving (SP) properties in the ideal case, alongside parameter-robust preconditioners. We show that these timesteppers can derive from a finite-element-in-time (FET) (and finite-element-in-space) interpretation. The benefits of this reformulation are discussed, including the derivation of timesteppers that are higher order in time, and the quantifiable dissipative SP properties in the non-ideal, resistive case.
        
        We discuss possible options for extending this FET approach to timesteppers for the compressible case.

        The kinetic corrections satisfy linearized Boltzmann equations. Using a Lénard--Bernstein collision operator, these take Fokker--Planck-like forms \cite{Fokker_1914, Planck_1917} wherein pseudo-particles in the numerical model obey the neoclassical transport equations, with particle-independent Brownian drift terms. This offers a rigorous methodology for incorporating collisions into the particle transport model, without coupling the equations of motions for each particle.
        
        Works by Chen, Chacón et al. \cite{Chen_Chacón_Barnes_2011, Chacón_Chen_Barnes_2013, Chen_Chacón_2014, Chen_Chacón_2015} have developed structure-preserving particle pushers for neoclassical transport in the Vlasov equations, derived from Crank--Nicolson integrators. We show these too can can derive from a FET interpretation, similarly offering potential extensions to higher-order-in-time particle pushers. The FET formulation is used also to consider how the stochastic drift terms can be incorporated into the pushers. Stochastic gyrokinetic expansions are also discussed.

        Different options for the numerical implementation of these schemes are considered.

        Due to the efficacy of FET in the development of SP timesteppers for both the fluid and kinetic component, we hope this approach will prove effective in the future for developing SP timesteppers for the full hybrid model. We hope this will give us the opportunity to incorporate previously inaccessible kinetic effects into the highly effective, modern, finite-element MHD models.
    \end{abstract}
    
    
    \newpage
    \tableofcontents
    
    
    \newpage
    \pagenumbering{arabic}
    %\linenumbers\renewcommand\thelinenumber{\color{black!50}\arabic{linenumber}}
            \input{0 - introduction/main.tex}
        \part{Research}
            \input{1 - low-noise PiC models/main.tex}
            \input{2 - kinetic component/main.tex}
            \input{3 - fluid component/main.tex}
            \input{4 - numerical implementation/main.tex}
        \part{Project Overview}
            \input{5 - research plan/main.tex}
            \input{6 - summary/main.tex}
    
    
    %\section{}
    \newpage
    \pagenumbering{gobble}
        \printbibliography


    \newpage
    \pagenumbering{roman}
    \appendix
        \part{Appendices}
            \input{8 - Hilbert complexes/main.tex}
            \input{9 - weak conservation proofs/main.tex}
\end{document}

            \documentclass[12pt, a4paper]{report}

\input{template/main.tex}

\title{\BA{Title in Progress...}}
\author{Boris Andrews}
\affil{Mathematical Institute, University of Oxford}
\date{\today}


\begin{document}
    \pagenumbering{gobble}
    \maketitle
    
    
    \begin{abstract}
        Magnetic confinement reactors---in particular tokamaks---offer one of the most promising options for achieving practical nuclear fusion, with the potential to provide virtually limitless, clean energy. The theoretical and numerical modeling of tokamak plasmas is simultaneously an essential component of effective reactor design, and a great research barrier. Tokamak operational conditions exhibit comparatively low Knudsen numbers. Kinetic effects, including kinetic waves and instabilities, Landau damping, bump-on-tail instabilities and more, are therefore highly influential in tokamak plasma dynamics. Purely fluid models are inherently incapable of capturing these effects, whereas the high dimensionality in purely kinetic models render them practically intractable for most relevant purposes.

        We consider a $\delta\!f$ decomposition model, with a macroscopic fluid background and microscopic kinetic correction, both fully coupled to each other. A similar manner of discretization is proposed to that used in the recent \texttt{STRUPHY} code \cite{Holderied_Possanner_Wang_2021, Holderied_2022, Li_et_al_2023} with a finite-element model for the background and a pseudo-particle/PiC model for the correction.

        The fluid background satisfies the full, non-linear, resistive, compressible, Hall MHD equations. \cite{Laakmann_Hu_Farrell_2022} introduces finite-element(-in-space) implicit timesteppers for the incompressible analogue to this system with structure-preserving (SP) properties in the ideal case, alongside parameter-robust preconditioners. We show that these timesteppers can derive from a finite-element-in-time (FET) (and finite-element-in-space) interpretation. The benefits of this reformulation are discussed, including the derivation of timesteppers that are higher order in time, and the quantifiable dissipative SP properties in the non-ideal, resistive case.
        
        We discuss possible options for extending this FET approach to timesteppers for the compressible case.

        The kinetic corrections satisfy linearized Boltzmann equations. Using a Lénard--Bernstein collision operator, these take Fokker--Planck-like forms \cite{Fokker_1914, Planck_1917} wherein pseudo-particles in the numerical model obey the neoclassical transport equations, with particle-independent Brownian drift terms. This offers a rigorous methodology for incorporating collisions into the particle transport model, without coupling the equations of motions for each particle.
        
        Works by Chen, Chacón et al. \cite{Chen_Chacón_Barnes_2011, Chacón_Chen_Barnes_2013, Chen_Chacón_2014, Chen_Chacón_2015} have developed structure-preserving particle pushers for neoclassical transport in the Vlasov equations, derived from Crank--Nicolson integrators. We show these too can can derive from a FET interpretation, similarly offering potential extensions to higher-order-in-time particle pushers. The FET formulation is used also to consider how the stochastic drift terms can be incorporated into the pushers. Stochastic gyrokinetic expansions are also discussed.

        Different options for the numerical implementation of these schemes are considered.

        Due to the efficacy of FET in the development of SP timesteppers for both the fluid and kinetic component, we hope this approach will prove effective in the future for developing SP timesteppers for the full hybrid model. We hope this will give us the opportunity to incorporate previously inaccessible kinetic effects into the highly effective, modern, finite-element MHD models.
    \end{abstract}
    
    
    \newpage
    \tableofcontents
    
    
    \newpage
    \pagenumbering{arabic}
    %\linenumbers\renewcommand\thelinenumber{\color{black!50}\arabic{linenumber}}
            \input{0 - introduction/main.tex}
        \part{Research}
            \input{1 - low-noise PiC models/main.tex}
            \input{2 - kinetic component/main.tex}
            \input{3 - fluid component/main.tex}
            \input{4 - numerical implementation/main.tex}
        \part{Project Overview}
            \input{5 - research plan/main.tex}
            \input{6 - summary/main.tex}
    
    
    %\section{}
    \newpage
    \pagenumbering{gobble}
        \printbibliography


    \newpage
    \pagenumbering{roman}
    \appendix
        \part{Appendices}
            \input{8 - Hilbert complexes/main.tex}
            \input{9 - weak conservation proofs/main.tex}
\end{document}

    
    
    %\section{}
    \newpage
    \pagenumbering{gobble}
        \printbibliography


    \newpage
    \pagenumbering{roman}
    \appendix
        \part{Appendices}
            \documentclass[12pt, a4paper]{report}

\input{template/main.tex}

\title{\BA{Title in Progress...}}
\author{Boris Andrews}
\affil{Mathematical Institute, University of Oxford}
\date{\today}


\begin{document}
    \pagenumbering{gobble}
    \maketitle
    
    
    \begin{abstract}
        Magnetic confinement reactors---in particular tokamaks---offer one of the most promising options for achieving practical nuclear fusion, with the potential to provide virtually limitless, clean energy. The theoretical and numerical modeling of tokamak plasmas is simultaneously an essential component of effective reactor design, and a great research barrier. Tokamak operational conditions exhibit comparatively low Knudsen numbers. Kinetic effects, including kinetic waves and instabilities, Landau damping, bump-on-tail instabilities and more, are therefore highly influential in tokamak plasma dynamics. Purely fluid models are inherently incapable of capturing these effects, whereas the high dimensionality in purely kinetic models render them practically intractable for most relevant purposes.

        We consider a $\delta\!f$ decomposition model, with a macroscopic fluid background and microscopic kinetic correction, both fully coupled to each other. A similar manner of discretization is proposed to that used in the recent \texttt{STRUPHY} code \cite{Holderied_Possanner_Wang_2021, Holderied_2022, Li_et_al_2023} with a finite-element model for the background and a pseudo-particle/PiC model for the correction.

        The fluid background satisfies the full, non-linear, resistive, compressible, Hall MHD equations. \cite{Laakmann_Hu_Farrell_2022} introduces finite-element(-in-space) implicit timesteppers for the incompressible analogue to this system with structure-preserving (SP) properties in the ideal case, alongside parameter-robust preconditioners. We show that these timesteppers can derive from a finite-element-in-time (FET) (and finite-element-in-space) interpretation. The benefits of this reformulation are discussed, including the derivation of timesteppers that are higher order in time, and the quantifiable dissipative SP properties in the non-ideal, resistive case.
        
        We discuss possible options for extending this FET approach to timesteppers for the compressible case.

        The kinetic corrections satisfy linearized Boltzmann equations. Using a Lénard--Bernstein collision operator, these take Fokker--Planck-like forms \cite{Fokker_1914, Planck_1917} wherein pseudo-particles in the numerical model obey the neoclassical transport equations, with particle-independent Brownian drift terms. This offers a rigorous methodology for incorporating collisions into the particle transport model, without coupling the equations of motions for each particle.
        
        Works by Chen, Chacón et al. \cite{Chen_Chacón_Barnes_2011, Chacón_Chen_Barnes_2013, Chen_Chacón_2014, Chen_Chacón_2015} have developed structure-preserving particle pushers for neoclassical transport in the Vlasov equations, derived from Crank--Nicolson integrators. We show these too can can derive from a FET interpretation, similarly offering potential extensions to higher-order-in-time particle pushers. The FET formulation is used also to consider how the stochastic drift terms can be incorporated into the pushers. Stochastic gyrokinetic expansions are also discussed.

        Different options for the numerical implementation of these schemes are considered.

        Due to the efficacy of FET in the development of SP timesteppers for both the fluid and kinetic component, we hope this approach will prove effective in the future for developing SP timesteppers for the full hybrid model. We hope this will give us the opportunity to incorporate previously inaccessible kinetic effects into the highly effective, modern, finite-element MHD models.
    \end{abstract}
    
    
    \newpage
    \tableofcontents
    
    
    \newpage
    \pagenumbering{arabic}
    %\linenumbers\renewcommand\thelinenumber{\color{black!50}\arabic{linenumber}}
            \input{0 - introduction/main.tex}
        \part{Research}
            \input{1 - low-noise PiC models/main.tex}
            \input{2 - kinetic component/main.tex}
            \input{3 - fluid component/main.tex}
            \input{4 - numerical implementation/main.tex}
        \part{Project Overview}
            \input{5 - research plan/main.tex}
            \input{6 - summary/main.tex}
    
    
    %\section{}
    \newpage
    \pagenumbering{gobble}
        \printbibliography


    \newpage
    \pagenumbering{roman}
    \appendix
        \part{Appendices}
            \input{8 - Hilbert complexes/main.tex}
            \input{9 - weak conservation proofs/main.tex}
\end{document}

            \documentclass[12pt, a4paper]{report}

\input{template/main.tex}

\title{\BA{Title in Progress...}}
\author{Boris Andrews}
\affil{Mathematical Institute, University of Oxford}
\date{\today}


\begin{document}
    \pagenumbering{gobble}
    \maketitle
    
    
    \begin{abstract}
        Magnetic confinement reactors---in particular tokamaks---offer one of the most promising options for achieving practical nuclear fusion, with the potential to provide virtually limitless, clean energy. The theoretical and numerical modeling of tokamak plasmas is simultaneously an essential component of effective reactor design, and a great research barrier. Tokamak operational conditions exhibit comparatively low Knudsen numbers. Kinetic effects, including kinetic waves and instabilities, Landau damping, bump-on-tail instabilities and more, are therefore highly influential in tokamak plasma dynamics. Purely fluid models are inherently incapable of capturing these effects, whereas the high dimensionality in purely kinetic models render them practically intractable for most relevant purposes.

        We consider a $\delta\!f$ decomposition model, with a macroscopic fluid background and microscopic kinetic correction, both fully coupled to each other. A similar manner of discretization is proposed to that used in the recent \texttt{STRUPHY} code \cite{Holderied_Possanner_Wang_2021, Holderied_2022, Li_et_al_2023} with a finite-element model for the background and a pseudo-particle/PiC model for the correction.

        The fluid background satisfies the full, non-linear, resistive, compressible, Hall MHD equations. \cite{Laakmann_Hu_Farrell_2022} introduces finite-element(-in-space) implicit timesteppers for the incompressible analogue to this system with structure-preserving (SP) properties in the ideal case, alongside parameter-robust preconditioners. We show that these timesteppers can derive from a finite-element-in-time (FET) (and finite-element-in-space) interpretation. The benefits of this reformulation are discussed, including the derivation of timesteppers that are higher order in time, and the quantifiable dissipative SP properties in the non-ideal, resistive case.
        
        We discuss possible options for extending this FET approach to timesteppers for the compressible case.

        The kinetic corrections satisfy linearized Boltzmann equations. Using a Lénard--Bernstein collision operator, these take Fokker--Planck-like forms \cite{Fokker_1914, Planck_1917} wherein pseudo-particles in the numerical model obey the neoclassical transport equations, with particle-independent Brownian drift terms. This offers a rigorous methodology for incorporating collisions into the particle transport model, without coupling the equations of motions for each particle.
        
        Works by Chen, Chacón et al. \cite{Chen_Chacón_Barnes_2011, Chacón_Chen_Barnes_2013, Chen_Chacón_2014, Chen_Chacón_2015} have developed structure-preserving particle pushers for neoclassical transport in the Vlasov equations, derived from Crank--Nicolson integrators. We show these too can can derive from a FET interpretation, similarly offering potential extensions to higher-order-in-time particle pushers. The FET formulation is used also to consider how the stochastic drift terms can be incorporated into the pushers. Stochastic gyrokinetic expansions are also discussed.

        Different options for the numerical implementation of these schemes are considered.

        Due to the efficacy of FET in the development of SP timesteppers for both the fluid and kinetic component, we hope this approach will prove effective in the future for developing SP timesteppers for the full hybrid model. We hope this will give us the opportunity to incorporate previously inaccessible kinetic effects into the highly effective, modern, finite-element MHD models.
    \end{abstract}
    
    
    \newpage
    \tableofcontents
    
    
    \newpage
    \pagenumbering{arabic}
    %\linenumbers\renewcommand\thelinenumber{\color{black!50}\arabic{linenumber}}
            \input{0 - introduction/main.tex}
        \part{Research}
            \input{1 - low-noise PiC models/main.tex}
            \input{2 - kinetic component/main.tex}
            \input{3 - fluid component/main.tex}
            \input{4 - numerical implementation/main.tex}
        \part{Project Overview}
            \input{5 - research plan/main.tex}
            \input{6 - summary/main.tex}
    
    
    %\section{}
    \newpage
    \pagenumbering{gobble}
        \printbibliography


    \newpage
    \pagenumbering{roman}
    \appendix
        \part{Appendices}
            \input{8 - Hilbert complexes/main.tex}
            \input{9 - weak conservation proofs/main.tex}
\end{document}

\end{document}

            \documentclass[12pt, a4paper]{report}

\documentclass[12pt, a4paper]{report}

\input{template/main.tex}

\title{\BA{Title in Progress...}}
\author{Boris Andrews}
\affil{Mathematical Institute, University of Oxford}
\date{\today}


\begin{document}
    \pagenumbering{gobble}
    \maketitle
    
    
    \begin{abstract}
        Magnetic confinement reactors---in particular tokamaks---offer one of the most promising options for achieving practical nuclear fusion, with the potential to provide virtually limitless, clean energy. The theoretical and numerical modeling of tokamak plasmas is simultaneously an essential component of effective reactor design, and a great research barrier. Tokamak operational conditions exhibit comparatively low Knudsen numbers. Kinetic effects, including kinetic waves and instabilities, Landau damping, bump-on-tail instabilities and more, are therefore highly influential in tokamak plasma dynamics. Purely fluid models are inherently incapable of capturing these effects, whereas the high dimensionality in purely kinetic models render them practically intractable for most relevant purposes.

        We consider a $\delta\!f$ decomposition model, with a macroscopic fluid background and microscopic kinetic correction, both fully coupled to each other. A similar manner of discretization is proposed to that used in the recent \texttt{STRUPHY} code \cite{Holderied_Possanner_Wang_2021, Holderied_2022, Li_et_al_2023} with a finite-element model for the background and a pseudo-particle/PiC model for the correction.

        The fluid background satisfies the full, non-linear, resistive, compressible, Hall MHD equations. \cite{Laakmann_Hu_Farrell_2022} introduces finite-element(-in-space) implicit timesteppers for the incompressible analogue to this system with structure-preserving (SP) properties in the ideal case, alongside parameter-robust preconditioners. We show that these timesteppers can derive from a finite-element-in-time (FET) (and finite-element-in-space) interpretation. The benefits of this reformulation are discussed, including the derivation of timesteppers that are higher order in time, and the quantifiable dissipative SP properties in the non-ideal, resistive case.
        
        We discuss possible options for extending this FET approach to timesteppers for the compressible case.

        The kinetic corrections satisfy linearized Boltzmann equations. Using a Lénard--Bernstein collision operator, these take Fokker--Planck-like forms \cite{Fokker_1914, Planck_1917} wherein pseudo-particles in the numerical model obey the neoclassical transport equations, with particle-independent Brownian drift terms. This offers a rigorous methodology for incorporating collisions into the particle transport model, without coupling the equations of motions for each particle.
        
        Works by Chen, Chacón et al. \cite{Chen_Chacón_Barnes_2011, Chacón_Chen_Barnes_2013, Chen_Chacón_2014, Chen_Chacón_2015} have developed structure-preserving particle pushers for neoclassical transport in the Vlasov equations, derived from Crank--Nicolson integrators. We show these too can can derive from a FET interpretation, similarly offering potential extensions to higher-order-in-time particle pushers. The FET formulation is used also to consider how the stochastic drift terms can be incorporated into the pushers. Stochastic gyrokinetic expansions are also discussed.

        Different options for the numerical implementation of these schemes are considered.

        Due to the efficacy of FET in the development of SP timesteppers for both the fluid and kinetic component, we hope this approach will prove effective in the future for developing SP timesteppers for the full hybrid model. We hope this will give us the opportunity to incorporate previously inaccessible kinetic effects into the highly effective, modern, finite-element MHD models.
    \end{abstract}
    
    
    \newpage
    \tableofcontents
    
    
    \newpage
    \pagenumbering{arabic}
    %\linenumbers\renewcommand\thelinenumber{\color{black!50}\arabic{linenumber}}
            \input{0 - introduction/main.tex}
        \part{Research}
            \input{1 - low-noise PiC models/main.tex}
            \input{2 - kinetic component/main.tex}
            \input{3 - fluid component/main.tex}
            \input{4 - numerical implementation/main.tex}
        \part{Project Overview}
            \input{5 - research plan/main.tex}
            \input{6 - summary/main.tex}
    
    
    %\section{}
    \newpage
    \pagenumbering{gobble}
        \printbibliography


    \newpage
    \pagenumbering{roman}
    \appendix
        \part{Appendices}
            \input{8 - Hilbert complexes/main.tex}
            \input{9 - weak conservation proofs/main.tex}
\end{document}


\title{\BA{Title in Progress...}}
\author{Boris Andrews}
\affil{Mathematical Institute, University of Oxford}
\date{\today}


\begin{document}
    \pagenumbering{gobble}
    \maketitle
    
    
    \begin{abstract}
        Magnetic confinement reactors---in particular tokamaks---offer one of the most promising options for achieving practical nuclear fusion, with the potential to provide virtually limitless, clean energy. The theoretical and numerical modeling of tokamak plasmas is simultaneously an essential component of effective reactor design, and a great research barrier. Tokamak operational conditions exhibit comparatively low Knudsen numbers. Kinetic effects, including kinetic waves and instabilities, Landau damping, bump-on-tail instabilities and more, are therefore highly influential in tokamak plasma dynamics. Purely fluid models are inherently incapable of capturing these effects, whereas the high dimensionality in purely kinetic models render them practically intractable for most relevant purposes.

        We consider a $\delta\!f$ decomposition model, with a macroscopic fluid background and microscopic kinetic correction, both fully coupled to each other. A similar manner of discretization is proposed to that used in the recent \texttt{STRUPHY} code \cite{Holderied_Possanner_Wang_2021, Holderied_2022, Li_et_al_2023} with a finite-element model for the background and a pseudo-particle/PiC model for the correction.

        The fluid background satisfies the full, non-linear, resistive, compressible, Hall MHD equations. \cite{Laakmann_Hu_Farrell_2022} introduces finite-element(-in-space) implicit timesteppers for the incompressible analogue to this system with structure-preserving (SP) properties in the ideal case, alongside parameter-robust preconditioners. We show that these timesteppers can derive from a finite-element-in-time (FET) (and finite-element-in-space) interpretation. The benefits of this reformulation are discussed, including the derivation of timesteppers that are higher order in time, and the quantifiable dissipative SP properties in the non-ideal, resistive case.
        
        We discuss possible options for extending this FET approach to timesteppers for the compressible case.

        The kinetic corrections satisfy linearized Boltzmann equations. Using a Lénard--Bernstein collision operator, these take Fokker--Planck-like forms \cite{Fokker_1914, Planck_1917} wherein pseudo-particles in the numerical model obey the neoclassical transport equations, with particle-independent Brownian drift terms. This offers a rigorous methodology for incorporating collisions into the particle transport model, without coupling the equations of motions for each particle.
        
        Works by Chen, Chacón et al. \cite{Chen_Chacón_Barnes_2011, Chacón_Chen_Barnes_2013, Chen_Chacón_2014, Chen_Chacón_2015} have developed structure-preserving particle pushers for neoclassical transport in the Vlasov equations, derived from Crank--Nicolson integrators. We show these too can can derive from a FET interpretation, similarly offering potential extensions to higher-order-in-time particle pushers. The FET formulation is used also to consider how the stochastic drift terms can be incorporated into the pushers. Stochastic gyrokinetic expansions are also discussed.

        Different options for the numerical implementation of these schemes are considered.

        Due to the efficacy of FET in the development of SP timesteppers for both the fluid and kinetic component, we hope this approach will prove effective in the future for developing SP timesteppers for the full hybrid model. We hope this will give us the opportunity to incorporate previously inaccessible kinetic effects into the highly effective, modern, finite-element MHD models.
    \end{abstract}
    
    
    \newpage
    \tableofcontents
    
    
    \newpage
    \pagenumbering{arabic}
    %\linenumbers\renewcommand\thelinenumber{\color{black!50}\arabic{linenumber}}
            \documentclass[12pt, a4paper]{report}

\input{template/main.tex}

\title{\BA{Title in Progress...}}
\author{Boris Andrews}
\affil{Mathematical Institute, University of Oxford}
\date{\today}


\begin{document}
    \pagenumbering{gobble}
    \maketitle
    
    
    \begin{abstract}
        Magnetic confinement reactors---in particular tokamaks---offer one of the most promising options for achieving practical nuclear fusion, with the potential to provide virtually limitless, clean energy. The theoretical and numerical modeling of tokamak plasmas is simultaneously an essential component of effective reactor design, and a great research barrier. Tokamak operational conditions exhibit comparatively low Knudsen numbers. Kinetic effects, including kinetic waves and instabilities, Landau damping, bump-on-tail instabilities and more, are therefore highly influential in tokamak plasma dynamics. Purely fluid models are inherently incapable of capturing these effects, whereas the high dimensionality in purely kinetic models render them practically intractable for most relevant purposes.

        We consider a $\delta\!f$ decomposition model, with a macroscopic fluid background and microscopic kinetic correction, both fully coupled to each other. A similar manner of discretization is proposed to that used in the recent \texttt{STRUPHY} code \cite{Holderied_Possanner_Wang_2021, Holderied_2022, Li_et_al_2023} with a finite-element model for the background and a pseudo-particle/PiC model for the correction.

        The fluid background satisfies the full, non-linear, resistive, compressible, Hall MHD equations. \cite{Laakmann_Hu_Farrell_2022} introduces finite-element(-in-space) implicit timesteppers for the incompressible analogue to this system with structure-preserving (SP) properties in the ideal case, alongside parameter-robust preconditioners. We show that these timesteppers can derive from a finite-element-in-time (FET) (and finite-element-in-space) interpretation. The benefits of this reformulation are discussed, including the derivation of timesteppers that are higher order in time, and the quantifiable dissipative SP properties in the non-ideal, resistive case.
        
        We discuss possible options for extending this FET approach to timesteppers for the compressible case.

        The kinetic corrections satisfy linearized Boltzmann equations. Using a Lénard--Bernstein collision operator, these take Fokker--Planck-like forms \cite{Fokker_1914, Planck_1917} wherein pseudo-particles in the numerical model obey the neoclassical transport equations, with particle-independent Brownian drift terms. This offers a rigorous methodology for incorporating collisions into the particle transport model, without coupling the equations of motions for each particle.
        
        Works by Chen, Chacón et al. \cite{Chen_Chacón_Barnes_2011, Chacón_Chen_Barnes_2013, Chen_Chacón_2014, Chen_Chacón_2015} have developed structure-preserving particle pushers for neoclassical transport in the Vlasov equations, derived from Crank--Nicolson integrators. We show these too can can derive from a FET interpretation, similarly offering potential extensions to higher-order-in-time particle pushers. The FET formulation is used also to consider how the stochastic drift terms can be incorporated into the pushers. Stochastic gyrokinetic expansions are also discussed.

        Different options for the numerical implementation of these schemes are considered.

        Due to the efficacy of FET in the development of SP timesteppers for both the fluid and kinetic component, we hope this approach will prove effective in the future for developing SP timesteppers for the full hybrid model. We hope this will give us the opportunity to incorporate previously inaccessible kinetic effects into the highly effective, modern, finite-element MHD models.
    \end{abstract}
    
    
    \newpage
    \tableofcontents
    
    
    \newpage
    \pagenumbering{arabic}
    %\linenumbers\renewcommand\thelinenumber{\color{black!50}\arabic{linenumber}}
            \input{0 - introduction/main.tex}
        \part{Research}
            \input{1 - low-noise PiC models/main.tex}
            \input{2 - kinetic component/main.tex}
            \input{3 - fluid component/main.tex}
            \input{4 - numerical implementation/main.tex}
        \part{Project Overview}
            \input{5 - research plan/main.tex}
            \input{6 - summary/main.tex}
    
    
    %\section{}
    \newpage
    \pagenumbering{gobble}
        \printbibliography


    \newpage
    \pagenumbering{roman}
    \appendix
        \part{Appendices}
            \input{8 - Hilbert complexes/main.tex}
            \input{9 - weak conservation proofs/main.tex}
\end{document}

        \part{Research}
            \documentclass[12pt, a4paper]{report}

\input{template/main.tex}

\title{\BA{Title in Progress...}}
\author{Boris Andrews}
\affil{Mathematical Institute, University of Oxford}
\date{\today}


\begin{document}
    \pagenumbering{gobble}
    \maketitle
    
    
    \begin{abstract}
        Magnetic confinement reactors---in particular tokamaks---offer one of the most promising options for achieving practical nuclear fusion, with the potential to provide virtually limitless, clean energy. The theoretical and numerical modeling of tokamak plasmas is simultaneously an essential component of effective reactor design, and a great research barrier. Tokamak operational conditions exhibit comparatively low Knudsen numbers. Kinetic effects, including kinetic waves and instabilities, Landau damping, bump-on-tail instabilities and more, are therefore highly influential in tokamak plasma dynamics. Purely fluid models are inherently incapable of capturing these effects, whereas the high dimensionality in purely kinetic models render them practically intractable for most relevant purposes.

        We consider a $\delta\!f$ decomposition model, with a macroscopic fluid background and microscopic kinetic correction, both fully coupled to each other. A similar manner of discretization is proposed to that used in the recent \texttt{STRUPHY} code \cite{Holderied_Possanner_Wang_2021, Holderied_2022, Li_et_al_2023} with a finite-element model for the background and a pseudo-particle/PiC model for the correction.

        The fluid background satisfies the full, non-linear, resistive, compressible, Hall MHD equations. \cite{Laakmann_Hu_Farrell_2022} introduces finite-element(-in-space) implicit timesteppers for the incompressible analogue to this system with structure-preserving (SP) properties in the ideal case, alongside parameter-robust preconditioners. We show that these timesteppers can derive from a finite-element-in-time (FET) (and finite-element-in-space) interpretation. The benefits of this reformulation are discussed, including the derivation of timesteppers that are higher order in time, and the quantifiable dissipative SP properties in the non-ideal, resistive case.
        
        We discuss possible options for extending this FET approach to timesteppers for the compressible case.

        The kinetic corrections satisfy linearized Boltzmann equations. Using a Lénard--Bernstein collision operator, these take Fokker--Planck-like forms \cite{Fokker_1914, Planck_1917} wherein pseudo-particles in the numerical model obey the neoclassical transport equations, with particle-independent Brownian drift terms. This offers a rigorous methodology for incorporating collisions into the particle transport model, without coupling the equations of motions for each particle.
        
        Works by Chen, Chacón et al. \cite{Chen_Chacón_Barnes_2011, Chacón_Chen_Barnes_2013, Chen_Chacón_2014, Chen_Chacón_2015} have developed structure-preserving particle pushers for neoclassical transport in the Vlasov equations, derived from Crank--Nicolson integrators. We show these too can can derive from a FET interpretation, similarly offering potential extensions to higher-order-in-time particle pushers. The FET formulation is used also to consider how the stochastic drift terms can be incorporated into the pushers. Stochastic gyrokinetic expansions are also discussed.

        Different options for the numerical implementation of these schemes are considered.

        Due to the efficacy of FET in the development of SP timesteppers for both the fluid and kinetic component, we hope this approach will prove effective in the future for developing SP timesteppers for the full hybrid model. We hope this will give us the opportunity to incorporate previously inaccessible kinetic effects into the highly effective, modern, finite-element MHD models.
    \end{abstract}
    
    
    \newpage
    \tableofcontents
    
    
    \newpage
    \pagenumbering{arabic}
    %\linenumbers\renewcommand\thelinenumber{\color{black!50}\arabic{linenumber}}
            \input{0 - introduction/main.tex}
        \part{Research}
            \input{1 - low-noise PiC models/main.tex}
            \input{2 - kinetic component/main.tex}
            \input{3 - fluid component/main.tex}
            \input{4 - numerical implementation/main.tex}
        \part{Project Overview}
            \input{5 - research plan/main.tex}
            \input{6 - summary/main.tex}
    
    
    %\section{}
    \newpage
    \pagenumbering{gobble}
        \printbibliography


    \newpage
    \pagenumbering{roman}
    \appendix
        \part{Appendices}
            \input{8 - Hilbert complexes/main.tex}
            \input{9 - weak conservation proofs/main.tex}
\end{document}

            \documentclass[12pt, a4paper]{report}

\input{template/main.tex}

\title{\BA{Title in Progress...}}
\author{Boris Andrews}
\affil{Mathematical Institute, University of Oxford}
\date{\today}


\begin{document}
    \pagenumbering{gobble}
    \maketitle
    
    
    \begin{abstract}
        Magnetic confinement reactors---in particular tokamaks---offer one of the most promising options for achieving practical nuclear fusion, with the potential to provide virtually limitless, clean energy. The theoretical and numerical modeling of tokamak plasmas is simultaneously an essential component of effective reactor design, and a great research barrier. Tokamak operational conditions exhibit comparatively low Knudsen numbers. Kinetic effects, including kinetic waves and instabilities, Landau damping, bump-on-tail instabilities and more, are therefore highly influential in tokamak plasma dynamics. Purely fluid models are inherently incapable of capturing these effects, whereas the high dimensionality in purely kinetic models render them practically intractable for most relevant purposes.

        We consider a $\delta\!f$ decomposition model, with a macroscopic fluid background and microscopic kinetic correction, both fully coupled to each other. A similar manner of discretization is proposed to that used in the recent \texttt{STRUPHY} code \cite{Holderied_Possanner_Wang_2021, Holderied_2022, Li_et_al_2023} with a finite-element model for the background and a pseudo-particle/PiC model for the correction.

        The fluid background satisfies the full, non-linear, resistive, compressible, Hall MHD equations. \cite{Laakmann_Hu_Farrell_2022} introduces finite-element(-in-space) implicit timesteppers for the incompressible analogue to this system with structure-preserving (SP) properties in the ideal case, alongside parameter-robust preconditioners. We show that these timesteppers can derive from a finite-element-in-time (FET) (and finite-element-in-space) interpretation. The benefits of this reformulation are discussed, including the derivation of timesteppers that are higher order in time, and the quantifiable dissipative SP properties in the non-ideal, resistive case.
        
        We discuss possible options for extending this FET approach to timesteppers for the compressible case.

        The kinetic corrections satisfy linearized Boltzmann equations. Using a Lénard--Bernstein collision operator, these take Fokker--Planck-like forms \cite{Fokker_1914, Planck_1917} wherein pseudo-particles in the numerical model obey the neoclassical transport equations, with particle-independent Brownian drift terms. This offers a rigorous methodology for incorporating collisions into the particle transport model, without coupling the equations of motions for each particle.
        
        Works by Chen, Chacón et al. \cite{Chen_Chacón_Barnes_2011, Chacón_Chen_Barnes_2013, Chen_Chacón_2014, Chen_Chacón_2015} have developed structure-preserving particle pushers for neoclassical transport in the Vlasov equations, derived from Crank--Nicolson integrators. We show these too can can derive from a FET interpretation, similarly offering potential extensions to higher-order-in-time particle pushers. The FET formulation is used also to consider how the stochastic drift terms can be incorporated into the pushers. Stochastic gyrokinetic expansions are also discussed.

        Different options for the numerical implementation of these schemes are considered.

        Due to the efficacy of FET in the development of SP timesteppers for both the fluid and kinetic component, we hope this approach will prove effective in the future for developing SP timesteppers for the full hybrid model. We hope this will give us the opportunity to incorporate previously inaccessible kinetic effects into the highly effective, modern, finite-element MHD models.
    \end{abstract}
    
    
    \newpage
    \tableofcontents
    
    
    \newpage
    \pagenumbering{arabic}
    %\linenumbers\renewcommand\thelinenumber{\color{black!50}\arabic{linenumber}}
            \input{0 - introduction/main.tex}
        \part{Research}
            \input{1 - low-noise PiC models/main.tex}
            \input{2 - kinetic component/main.tex}
            \input{3 - fluid component/main.tex}
            \input{4 - numerical implementation/main.tex}
        \part{Project Overview}
            \input{5 - research plan/main.tex}
            \input{6 - summary/main.tex}
    
    
    %\section{}
    \newpage
    \pagenumbering{gobble}
        \printbibliography


    \newpage
    \pagenumbering{roman}
    \appendix
        \part{Appendices}
            \input{8 - Hilbert complexes/main.tex}
            \input{9 - weak conservation proofs/main.tex}
\end{document}

            \documentclass[12pt, a4paper]{report}

\input{template/main.tex}

\title{\BA{Title in Progress...}}
\author{Boris Andrews}
\affil{Mathematical Institute, University of Oxford}
\date{\today}


\begin{document}
    \pagenumbering{gobble}
    \maketitle
    
    
    \begin{abstract}
        Magnetic confinement reactors---in particular tokamaks---offer one of the most promising options for achieving practical nuclear fusion, with the potential to provide virtually limitless, clean energy. The theoretical and numerical modeling of tokamak plasmas is simultaneously an essential component of effective reactor design, and a great research barrier. Tokamak operational conditions exhibit comparatively low Knudsen numbers. Kinetic effects, including kinetic waves and instabilities, Landau damping, bump-on-tail instabilities and more, are therefore highly influential in tokamak plasma dynamics. Purely fluid models are inherently incapable of capturing these effects, whereas the high dimensionality in purely kinetic models render them practically intractable for most relevant purposes.

        We consider a $\delta\!f$ decomposition model, with a macroscopic fluid background and microscopic kinetic correction, both fully coupled to each other. A similar manner of discretization is proposed to that used in the recent \texttt{STRUPHY} code \cite{Holderied_Possanner_Wang_2021, Holderied_2022, Li_et_al_2023} with a finite-element model for the background and a pseudo-particle/PiC model for the correction.

        The fluid background satisfies the full, non-linear, resistive, compressible, Hall MHD equations. \cite{Laakmann_Hu_Farrell_2022} introduces finite-element(-in-space) implicit timesteppers for the incompressible analogue to this system with structure-preserving (SP) properties in the ideal case, alongside parameter-robust preconditioners. We show that these timesteppers can derive from a finite-element-in-time (FET) (and finite-element-in-space) interpretation. The benefits of this reformulation are discussed, including the derivation of timesteppers that are higher order in time, and the quantifiable dissipative SP properties in the non-ideal, resistive case.
        
        We discuss possible options for extending this FET approach to timesteppers for the compressible case.

        The kinetic corrections satisfy linearized Boltzmann equations. Using a Lénard--Bernstein collision operator, these take Fokker--Planck-like forms \cite{Fokker_1914, Planck_1917} wherein pseudo-particles in the numerical model obey the neoclassical transport equations, with particle-independent Brownian drift terms. This offers a rigorous methodology for incorporating collisions into the particle transport model, without coupling the equations of motions for each particle.
        
        Works by Chen, Chacón et al. \cite{Chen_Chacón_Barnes_2011, Chacón_Chen_Barnes_2013, Chen_Chacón_2014, Chen_Chacón_2015} have developed structure-preserving particle pushers for neoclassical transport in the Vlasov equations, derived from Crank--Nicolson integrators. We show these too can can derive from a FET interpretation, similarly offering potential extensions to higher-order-in-time particle pushers. The FET formulation is used also to consider how the stochastic drift terms can be incorporated into the pushers. Stochastic gyrokinetic expansions are also discussed.

        Different options for the numerical implementation of these schemes are considered.

        Due to the efficacy of FET in the development of SP timesteppers for both the fluid and kinetic component, we hope this approach will prove effective in the future for developing SP timesteppers for the full hybrid model. We hope this will give us the opportunity to incorporate previously inaccessible kinetic effects into the highly effective, modern, finite-element MHD models.
    \end{abstract}
    
    
    \newpage
    \tableofcontents
    
    
    \newpage
    \pagenumbering{arabic}
    %\linenumbers\renewcommand\thelinenumber{\color{black!50}\arabic{linenumber}}
            \input{0 - introduction/main.tex}
        \part{Research}
            \input{1 - low-noise PiC models/main.tex}
            \input{2 - kinetic component/main.tex}
            \input{3 - fluid component/main.tex}
            \input{4 - numerical implementation/main.tex}
        \part{Project Overview}
            \input{5 - research plan/main.tex}
            \input{6 - summary/main.tex}
    
    
    %\section{}
    \newpage
    \pagenumbering{gobble}
        \printbibliography


    \newpage
    \pagenumbering{roman}
    \appendix
        \part{Appendices}
            \input{8 - Hilbert complexes/main.tex}
            \input{9 - weak conservation proofs/main.tex}
\end{document}

            \documentclass[12pt, a4paper]{report}

\input{template/main.tex}

\title{\BA{Title in Progress...}}
\author{Boris Andrews}
\affil{Mathematical Institute, University of Oxford}
\date{\today}


\begin{document}
    \pagenumbering{gobble}
    \maketitle
    
    
    \begin{abstract}
        Magnetic confinement reactors---in particular tokamaks---offer one of the most promising options for achieving practical nuclear fusion, with the potential to provide virtually limitless, clean energy. The theoretical and numerical modeling of tokamak plasmas is simultaneously an essential component of effective reactor design, and a great research barrier. Tokamak operational conditions exhibit comparatively low Knudsen numbers. Kinetic effects, including kinetic waves and instabilities, Landau damping, bump-on-tail instabilities and more, are therefore highly influential in tokamak plasma dynamics. Purely fluid models are inherently incapable of capturing these effects, whereas the high dimensionality in purely kinetic models render them practically intractable for most relevant purposes.

        We consider a $\delta\!f$ decomposition model, with a macroscopic fluid background and microscopic kinetic correction, both fully coupled to each other. A similar manner of discretization is proposed to that used in the recent \texttt{STRUPHY} code \cite{Holderied_Possanner_Wang_2021, Holderied_2022, Li_et_al_2023} with a finite-element model for the background and a pseudo-particle/PiC model for the correction.

        The fluid background satisfies the full, non-linear, resistive, compressible, Hall MHD equations. \cite{Laakmann_Hu_Farrell_2022} introduces finite-element(-in-space) implicit timesteppers for the incompressible analogue to this system with structure-preserving (SP) properties in the ideal case, alongside parameter-robust preconditioners. We show that these timesteppers can derive from a finite-element-in-time (FET) (and finite-element-in-space) interpretation. The benefits of this reformulation are discussed, including the derivation of timesteppers that are higher order in time, and the quantifiable dissipative SP properties in the non-ideal, resistive case.
        
        We discuss possible options for extending this FET approach to timesteppers for the compressible case.

        The kinetic corrections satisfy linearized Boltzmann equations. Using a Lénard--Bernstein collision operator, these take Fokker--Planck-like forms \cite{Fokker_1914, Planck_1917} wherein pseudo-particles in the numerical model obey the neoclassical transport equations, with particle-independent Brownian drift terms. This offers a rigorous methodology for incorporating collisions into the particle transport model, without coupling the equations of motions for each particle.
        
        Works by Chen, Chacón et al. \cite{Chen_Chacón_Barnes_2011, Chacón_Chen_Barnes_2013, Chen_Chacón_2014, Chen_Chacón_2015} have developed structure-preserving particle pushers for neoclassical transport in the Vlasov equations, derived from Crank--Nicolson integrators. We show these too can can derive from a FET interpretation, similarly offering potential extensions to higher-order-in-time particle pushers. The FET formulation is used also to consider how the stochastic drift terms can be incorporated into the pushers. Stochastic gyrokinetic expansions are also discussed.

        Different options for the numerical implementation of these schemes are considered.

        Due to the efficacy of FET in the development of SP timesteppers for both the fluid and kinetic component, we hope this approach will prove effective in the future for developing SP timesteppers for the full hybrid model. We hope this will give us the opportunity to incorporate previously inaccessible kinetic effects into the highly effective, modern, finite-element MHD models.
    \end{abstract}
    
    
    \newpage
    \tableofcontents
    
    
    \newpage
    \pagenumbering{arabic}
    %\linenumbers\renewcommand\thelinenumber{\color{black!50}\arabic{linenumber}}
            \input{0 - introduction/main.tex}
        \part{Research}
            \input{1 - low-noise PiC models/main.tex}
            \input{2 - kinetic component/main.tex}
            \input{3 - fluid component/main.tex}
            \input{4 - numerical implementation/main.tex}
        \part{Project Overview}
            \input{5 - research plan/main.tex}
            \input{6 - summary/main.tex}
    
    
    %\section{}
    \newpage
    \pagenumbering{gobble}
        \printbibliography


    \newpage
    \pagenumbering{roman}
    \appendix
        \part{Appendices}
            \input{8 - Hilbert complexes/main.tex}
            \input{9 - weak conservation proofs/main.tex}
\end{document}

        \part{Project Overview}
            \documentclass[12pt, a4paper]{report}

\input{template/main.tex}

\title{\BA{Title in Progress...}}
\author{Boris Andrews}
\affil{Mathematical Institute, University of Oxford}
\date{\today}


\begin{document}
    \pagenumbering{gobble}
    \maketitle
    
    
    \begin{abstract}
        Magnetic confinement reactors---in particular tokamaks---offer one of the most promising options for achieving practical nuclear fusion, with the potential to provide virtually limitless, clean energy. The theoretical and numerical modeling of tokamak plasmas is simultaneously an essential component of effective reactor design, and a great research barrier. Tokamak operational conditions exhibit comparatively low Knudsen numbers. Kinetic effects, including kinetic waves and instabilities, Landau damping, bump-on-tail instabilities and more, are therefore highly influential in tokamak plasma dynamics. Purely fluid models are inherently incapable of capturing these effects, whereas the high dimensionality in purely kinetic models render them practically intractable for most relevant purposes.

        We consider a $\delta\!f$ decomposition model, with a macroscopic fluid background and microscopic kinetic correction, both fully coupled to each other. A similar manner of discretization is proposed to that used in the recent \texttt{STRUPHY} code \cite{Holderied_Possanner_Wang_2021, Holderied_2022, Li_et_al_2023} with a finite-element model for the background and a pseudo-particle/PiC model for the correction.

        The fluid background satisfies the full, non-linear, resistive, compressible, Hall MHD equations. \cite{Laakmann_Hu_Farrell_2022} introduces finite-element(-in-space) implicit timesteppers for the incompressible analogue to this system with structure-preserving (SP) properties in the ideal case, alongside parameter-robust preconditioners. We show that these timesteppers can derive from a finite-element-in-time (FET) (and finite-element-in-space) interpretation. The benefits of this reformulation are discussed, including the derivation of timesteppers that are higher order in time, and the quantifiable dissipative SP properties in the non-ideal, resistive case.
        
        We discuss possible options for extending this FET approach to timesteppers for the compressible case.

        The kinetic corrections satisfy linearized Boltzmann equations. Using a Lénard--Bernstein collision operator, these take Fokker--Planck-like forms \cite{Fokker_1914, Planck_1917} wherein pseudo-particles in the numerical model obey the neoclassical transport equations, with particle-independent Brownian drift terms. This offers a rigorous methodology for incorporating collisions into the particle transport model, without coupling the equations of motions for each particle.
        
        Works by Chen, Chacón et al. \cite{Chen_Chacón_Barnes_2011, Chacón_Chen_Barnes_2013, Chen_Chacón_2014, Chen_Chacón_2015} have developed structure-preserving particle pushers for neoclassical transport in the Vlasov equations, derived from Crank--Nicolson integrators. We show these too can can derive from a FET interpretation, similarly offering potential extensions to higher-order-in-time particle pushers. The FET formulation is used also to consider how the stochastic drift terms can be incorporated into the pushers. Stochastic gyrokinetic expansions are also discussed.

        Different options for the numerical implementation of these schemes are considered.

        Due to the efficacy of FET in the development of SP timesteppers for both the fluid and kinetic component, we hope this approach will prove effective in the future for developing SP timesteppers for the full hybrid model. We hope this will give us the opportunity to incorporate previously inaccessible kinetic effects into the highly effective, modern, finite-element MHD models.
    \end{abstract}
    
    
    \newpage
    \tableofcontents
    
    
    \newpage
    \pagenumbering{arabic}
    %\linenumbers\renewcommand\thelinenumber{\color{black!50}\arabic{linenumber}}
            \input{0 - introduction/main.tex}
        \part{Research}
            \input{1 - low-noise PiC models/main.tex}
            \input{2 - kinetic component/main.tex}
            \input{3 - fluid component/main.tex}
            \input{4 - numerical implementation/main.tex}
        \part{Project Overview}
            \input{5 - research plan/main.tex}
            \input{6 - summary/main.tex}
    
    
    %\section{}
    \newpage
    \pagenumbering{gobble}
        \printbibliography


    \newpage
    \pagenumbering{roman}
    \appendix
        \part{Appendices}
            \input{8 - Hilbert complexes/main.tex}
            \input{9 - weak conservation proofs/main.tex}
\end{document}

            \documentclass[12pt, a4paper]{report}

\input{template/main.tex}

\title{\BA{Title in Progress...}}
\author{Boris Andrews}
\affil{Mathematical Institute, University of Oxford}
\date{\today}


\begin{document}
    \pagenumbering{gobble}
    \maketitle
    
    
    \begin{abstract}
        Magnetic confinement reactors---in particular tokamaks---offer one of the most promising options for achieving practical nuclear fusion, with the potential to provide virtually limitless, clean energy. The theoretical and numerical modeling of tokamak plasmas is simultaneously an essential component of effective reactor design, and a great research barrier. Tokamak operational conditions exhibit comparatively low Knudsen numbers. Kinetic effects, including kinetic waves and instabilities, Landau damping, bump-on-tail instabilities and more, are therefore highly influential in tokamak plasma dynamics. Purely fluid models are inherently incapable of capturing these effects, whereas the high dimensionality in purely kinetic models render them practically intractable for most relevant purposes.

        We consider a $\delta\!f$ decomposition model, with a macroscopic fluid background and microscopic kinetic correction, both fully coupled to each other. A similar manner of discretization is proposed to that used in the recent \texttt{STRUPHY} code \cite{Holderied_Possanner_Wang_2021, Holderied_2022, Li_et_al_2023} with a finite-element model for the background and a pseudo-particle/PiC model for the correction.

        The fluid background satisfies the full, non-linear, resistive, compressible, Hall MHD equations. \cite{Laakmann_Hu_Farrell_2022} introduces finite-element(-in-space) implicit timesteppers for the incompressible analogue to this system with structure-preserving (SP) properties in the ideal case, alongside parameter-robust preconditioners. We show that these timesteppers can derive from a finite-element-in-time (FET) (and finite-element-in-space) interpretation. The benefits of this reformulation are discussed, including the derivation of timesteppers that are higher order in time, and the quantifiable dissipative SP properties in the non-ideal, resistive case.
        
        We discuss possible options for extending this FET approach to timesteppers for the compressible case.

        The kinetic corrections satisfy linearized Boltzmann equations. Using a Lénard--Bernstein collision operator, these take Fokker--Planck-like forms \cite{Fokker_1914, Planck_1917} wherein pseudo-particles in the numerical model obey the neoclassical transport equations, with particle-independent Brownian drift terms. This offers a rigorous methodology for incorporating collisions into the particle transport model, without coupling the equations of motions for each particle.
        
        Works by Chen, Chacón et al. \cite{Chen_Chacón_Barnes_2011, Chacón_Chen_Barnes_2013, Chen_Chacón_2014, Chen_Chacón_2015} have developed structure-preserving particle pushers for neoclassical transport in the Vlasov equations, derived from Crank--Nicolson integrators. We show these too can can derive from a FET interpretation, similarly offering potential extensions to higher-order-in-time particle pushers. The FET formulation is used also to consider how the stochastic drift terms can be incorporated into the pushers. Stochastic gyrokinetic expansions are also discussed.

        Different options for the numerical implementation of these schemes are considered.

        Due to the efficacy of FET in the development of SP timesteppers for both the fluid and kinetic component, we hope this approach will prove effective in the future for developing SP timesteppers for the full hybrid model. We hope this will give us the opportunity to incorporate previously inaccessible kinetic effects into the highly effective, modern, finite-element MHD models.
    \end{abstract}
    
    
    \newpage
    \tableofcontents
    
    
    \newpage
    \pagenumbering{arabic}
    %\linenumbers\renewcommand\thelinenumber{\color{black!50}\arabic{linenumber}}
            \input{0 - introduction/main.tex}
        \part{Research}
            \input{1 - low-noise PiC models/main.tex}
            \input{2 - kinetic component/main.tex}
            \input{3 - fluid component/main.tex}
            \input{4 - numerical implementation/main.tex}
        \part{Project Overview}
            \input{5 - research plan/main.tex}
            \input{6 - summary/main.tex}
    
    
    %\section{}
    \newpage
    \pagenumbering{gobble}
        \printbibliography


    \newpage
    \pagenumbering{roman}
    \appendix
        \part{Appendices}
            \input{8 - Hilbert complexes/main.tex}
            \input{9 - weak conservation proofs/main.tex}
\end{document}

    
    
    %\section{}
    \newpage
    \pagenumbering{gobble}
        \printbibliography


    \newpage
    \pagenumbering{roman}
    \appendix
        \part{Appendices}
            \documentclass[12pt, a4paper]{report}

\input{template/main.tex}

\title{\BA{Title in Progress...}}
\author{Boris Andrews}
\affil{Mathematical Institute, University of Oxford}
\date{\today}


\begin{document}
    \pagenumbering{gobble}
    \maketitle
    
    
    \begin{abstract}
        Magnetic confinement reactors---in particular tokamaks---offer one of the most promising options for achieving practical nuclear fusion, with the potential to provide virtually limitless, clean energy. The theoretical and numerical modeling of tokamak plasmas is simultaneously an essential component of effective reactor design, and a great research barrier. Tokamak operational conditions exhibit comparatively low Knudsen numbers. Kinetic effects, including kinetic waves and instabilities, Landau damping, bump-on-tail instabilities and more, are therefore highly influential in tokamak plasma dynamics. Purely fluid models are inherently incapable of capturing these effects, whereas the high dimensionality in purely kinetic models render them practically intractable for most relevant purposes.

        We consider a $\delta\!f$ decomposition model, with a macroscopic fluid background and microscopic kinetic correction, both fully coupled to each other. A similar manner of discretization is proposed to that used in the recent \texttt{STRUPHY} code \cite{Holderied_Possanner_Wang_2021, Holderied_2022, Li_et_al_2023} with a finite-element model for the background and a pseudo-particle/PiC model for the correction.

        The fluid background satisfies the full, non-linear, resistive, compressible, Hall MHD equations. \cite{Laakmann_Hu_Farrell_2022} introduces finite-element(-in-space) implicit timesteppers for the incompressible analogue to this system with structure-preserving (SP) properties in the ideal case, alongside parameter-robust preconditioners. We show that these timesteppers can derive from a finite-element-in-time (FET) (and finite-element-in-space) interpretation. The benefits of this reformulation are discussed, including the derivation of timesteppers that are higher order in time, and the quantifiable dissipative SP properties in the non-ideal, resistive case.
        
        We discuss possible options for extending this FET approach to timesteppers for the compressible case.

        The kinetic corrections satisfy linearized Boltzmann equations. Using a Lénard--Bernstein collision operator, these take Fokker--Planck-like forms \cite{Fokker_1914, Planck_1917} wherein pseudo-particles in the numerical model obey the neoclassical transport equations, with particle-independent Brownian drift terms. This offers a rigorous methodology for incorporating collisions into the particle transport model, without coupling the equations of motions for each particle.
        
        Works by Chen, Chacón et al. \cite{Chen_Chacón_Barnes_2011, Chacón_Chen_Barnes_2013, Chen_Chacón_2014, Chen_Chacón_2015} have developed structure-preserving particle pushers for neoclassical transport in the Vlasov equations, derived from Crank--Nicolson integrators. We show these too can can derive from a FET interpretation, similarly offering potential extensions to higher-order-in-time particle pushers. The FET formulation is used also to consider how the stochastic drift terms can be incorporated into the pushers. Stochastic gyrokinetic expansions are also discussed.

        Different options for the numerical implementation of these schemes are considered.

        Due to the efficacy of FET in the development of SP timesteppers for both the fluid and kinetic component, we hope this approach will prove effective in the future for developing SP timesteppers for the full hybrid model. We hope this will give us the opportunity to incorporate previously inaccessible kinetic effects into the highly effective, modern, finite-element MHD models.
    \end{abstract}
    
    
    \newpage
    \tableofcontents
    
    
    \newpage
    \pagenumbering{arabic}
    %\linenumbers\renewcommand\thelinenumber{\color{black!50}\arabic{linenumber}}
            \input{0 - introduction/main.tex}
        \part{Research}
            \input{1 - low-noise PiC models/main.tex}
            \input{2 - kinetic component/main.tex}
            \input{3 - fluid component/main.tex}
            \input{4 - numerical implementation/main.tex}
        \part{Project Overview}
            \input{5 - research plan/main.tex}
            \input{6 - summary/main.tex}
    
    
    %\section{}
    \newpage
    \pagenumbering{gobble}
        \printbibliography


    \newpage
    \pagenumbering{roman}
    \appendix
        \part{Appendices}
            \input{8 - Hilbert complexes/main.tex}
            \input{9 - weak conservation proofs/main.tex}
\end{document}

            \documentclass[12pt, a4paper]{report}

\input{template/main.tex}

\title{\BA{Title in Progress...}}
\author{Boris Andrews}
\affil{Mathematical Institute, University of Oxford}
\date{\today}


\begin{document}
    \pagenumbering{gobble}
    \maketitle
    
    
    \begin{abstract}
        Magnetic confinement reactors---in particular tokamaks---offer one of the most promising options for achieving practical nuclear fusion, with the potential to provide virtually limitless, clean energy. The theoretical and numerical modeling of tokamak plasmas is simultaneously an essential component of effective reactor design, and a great research barrier. Tokamak operational conditions exhibit comparatively low Knudsen numbers. Kinetic effects, including kinetic waves and instabilities, Landau damping, bump-on-tail instabilities and more, are therefore highly influential in tokamak plasma dynamics. Purely fluid models are inherently incapable of capturing these effects, whereas the high dimensionality in purely kinetic models render them practically intractable for most relevant purposes.

        We consider a $\delta\!f$ decomposition model, with a macroscopic fluid background and microscopic kinetic correction, both fully coupled to each other. A similar manner of discretization is proposed to that used in the recent \texttt{STRUPHY} code \cite{Holderied_Possanner_Wang_2021, Holderied_2022, Li_et_al_2023} with a finite-element model for the background and a pseudo-particle/PiC model for the correction.

        The fluid background satisfies the full, non-linear, resistive, compressible, Hall MHD equations. \cite{Laakmann_Hu_Farrell_2022} introduces finite-element(-in-space) implicit timesteppers for the incompressible analogue to this system with structure-preserving (SP) properties in the ideal case, alongside parameter-robust preconditioners. We show that these timesteppers can derive from a finite-element-in-time (FET) (and finite-element-in-space) interpretation. The benefits of this reformulation are discussed, including the derivation of timesteppers that are higher order in time, and the quantifiable dissipative SP properties in the non-ideal, resistive case.
        
        We discuss possible options for extending this FET approach to timesteppers for the compressible case.

        The kinetic corrections satisfy linearized Boltzmann equations. Using a Lénard--Bernstein collision operator, these take Fokker--Planck-like forms \cite{Fokker_1914, Planck_1917} wherein pseudo-particles in the numerical model obey the neoclassical transport equations, with particle-independent Brownian drift terms. This offers a rigorous methodology for incorporating collisions into the particle transport model, without coupling the equations of motions for each particle.
        
        Works by Chen, Chacón et al. \cite{Chen_Chacón_Barnes_2011, Chacón_Chen_Barnes_2013, Chen_Chacón_2014, Chen_Chacón_2015} have developed structure-preserving particle pushers for neoclassical transport in the Vlasov equations, derived from Crank--Nicolson integrators. We show these too can can derive from a FET interpretation, similarly offering potential extensions to higher-order-in-time particle pushers. The FET formulation is used also to consider how the stochastic drift terms can be incorporated into the pushers. Stochastic gyrokinetic expansions are also discussed.

        Different options for the numerical implementation of these schemes are considered.

        Due to the efficacy of FET in the development of SP timesteppers for both the fluid and kinetic component, we hope this approach will prove effective in the future for developing SP timesteppers for the full hybrid model. We hope this will give us the opportunity to incorporate previously inaccessible kinetic effects into the highly effective, modern, finite-element MHD models.
    \end{abstract}
    
    
    \newpage
    \tableofcontents
    
    
    \newpage
    \pagenumbering{arabic}
    %\linenumbers\renewcommand\thelinenumber{\color{black!50}\arabic{linenumber}}
            \input{0 - introduction/main.tex}
        \part{Research}
            \input{1 - low-noise PiC models/main.tex}
            \input{2 - kinetic component/main.tex}
            \input{3 - fluid component/main.tex}
            \input{4 - numerical implementation/main.tex}
        \part{Project Overview}
            \input{5 - research plan/main.tex}
            \input{6 - summary/main.tex}
    
    
    %\section{}
    \newpage
    \pagenumbering{gobble}
        \printbibliography


    \newpage
    \pagenumbering{roman}
    \appendix
        \part{Appendices}
            \input{8 - Hilbert complexes/main.tex}
            \input{9 - weak conservation proofs/main.tex}
\end{document}

\end{document}

    
    
    %\section{}
    \newpage
    \pagenumbering{gobble}
        \printbibliography


    \newpage
    \pagenumbering{roman}
    \appendix
        \part{Appendices}
            \documentclass[12pt, a4paper]{report}

\documentclass[12pt, a4paper]{report}

\input{template/main.tex}

\title{\BA{Title in Progress...}}
\author{Boris Andrews}
\affil{Mathematical Institute, University of Oxford}
\date{\today}


\begin{document}
    \pagenumbering{gobble}
    \maketitle
    
    
    \begin{abstract}
        Magnetic confinement reactors---in particular tokamaks---offer one of the most promising options for achieving practical nuclear fusion, with the potential to provide virtually limitless, clean energy. The theoretical and numerical modeling of tokamak plasmas is simultaneously an essential component of effective reactor design, and a great research barrier. Tokamak operational conditions exhibit comparatively low Knudsen numbers. Kinetic effects, including kinetic waves and instabilities, Landau damping, bump-on-tail instabilities and more, are therefore highly influential in tokamak plasma dynamics. Purely fluid models are inherently incapable of capturing these effects, whereas the high dimensionality in purely kinetic models render them practically intractable for most relevant purposes.

        We consider a $\delta\!f$ decomposition model, with a macroscopic fluid background and microscopic kinetic correction, both fully coupled to each other. A similar manner of discretization is proposed to that used in the recent \texttt{STRUPHY} code \cite{Holderied_Possanner_Wang_2021, Holderied_2022, Li_et_al_2023} with a finite-element model for the background and a pseudo-particle/PiC model for the correction.

        The fluid background satisfies the full, non-linear, resistive, compressible, Hall MHD equations. \cite{Laakmann_Hu_Farrell_2022} introduces finite-element(-in-space) implicit timesteppers for the incompressible analogue to this system with structure-preserving (SP) properties in the ideal case, alongside parameter-robust preconditioners. We show that these timesteppers can derive from a finite-element-in-time (FET) (and finite-element-in-space) interpretation. The benefits of this reformulation are discussed, including the derivation of timesteppers that are higher order in time, and the quantifiable dissipative SP properties in the non-ideal, resistive case.
        
        We discuss possible options for extending this FET approach to timesteppers for the compressible case.

        The kinetic corrections satisfy linearized Boltzmann equations. Using a Lénard--Bernstein collision operator, these take Fokker--Planck-like forms \cite{Fokker_1914, Planck_1917} wherein pseudo-particles in the numerical model obey the neoclassical transport equations, with particle-independent Brownian drift terms. This offers a rigorous methodology for incorporating collisions into the particle transport model, without coupling the equations of motions for each particle.
        
        Works by Chen, Chacón et al. \cite{Chen_Chacón_Barnes_2011, Chacón_Chen_Barnes_2013, Chen_Chacón_2014, Chen_Chacón_2015} have developed structure-preserving particle pushers for neoclassical transport in the Vlasov equations, derived from Crank--Nicolson integrators. We show these too can can derive from a FET interpretation, similarly offering potential extensions to higher-order-in-time particle pushers. The FET formulation is used also to consider how the stochastic drift terms can be incorporated into the pushers. Stochastic gyrokinetic expansions are also discussed.

        Different options for the numerical implementation of these schemes are considered.

        Due to the efficacy of FET in the development of SP timesteppers for both the fluid and kinetic component, we hope this approach will prove effective in the future for developing SP timesteppers for the full hybrid model. We hope this will give us the opportunity to incorporate previously inaccessible kinetic effects into the highly effective, modern, finite-element MHD models.
    \end{abstract}
    
    
    \newpage
    \tableofcontents
    
    
    \newpage
    \pagenumbering{arabic}
    %\linenumbers\renewcommand\thelinenumber{\color{black!50}\arabic{linenumber}}
            \input{0 - introduction/main.tex}
        \part{Research}
            \input{1 - low-noise PiC models/main.tex}
            \input{2 - kinetic component/main.tex}
            \input{3 - fluid component/main.tex}
            \input{4 - numerical implementation/main.tex}
        \part{Project Overview}
            \input{5 - research plan/main.tex}
            \input{6 - summary/main.tex}
    
    
    %\section{}
    \newpage
    \pagenumbering{gobble}
        \printbibliography


    \newpage
    \pagenumbering{roman}
    \appendix
        \part{Appendices}
            \input{8 - Hilbert complexes/main.tex}
            \input{9 - weak conservation proofs/main.tex}
\end{document}


\title{\BA{Title in Progress...}}
\author{Boris Andrews}
\affil{Mathematical Institute, University of Oxford}
\date{\today}


\begin{document}
    \pagenumbering{gobble}
    \maketitle
    
    
    \begin{abstract}
        Magnetic confinement reactors---in particular tokamaks---offer one of the most promising options for achieving practical nuclear fusion, with the potential to provide virtually limitless, clean energy. The theoretical and numerical modeling of tokamak plasmas is simultaneously an essential component of effective reactor design, and a great research barrier. Tokamak operational conditions exhibit comparatively low Knudsen numbers. Kinetic effects, including kinetic waves and instabilities, Landau damping, bump-on-tail instabilities and more, are therefore highly influential in tokamak plasma dynamics. Purely fluid models are inherently incapable of capturing these effects, whereas the high dimensionality in purely kinetic models render them practically intractable for most relevant purposes.

        We consider a $\delta\!f$ decomposition model, with a macroscopic fluid background and microscopic kinetic correction, both fully coupled to each other. A similar manner of discretization is proposed to that used in the recent \texttt{STRUPHY} code \cite{Holderied_Possanner_Wang_2021, Holderied_2022, Li_et_al_2023} with a finite-element model for the background and a pseudo-particle/PiC model for the correction.

        The fluid background satisfies the full, non-linear, resistive, compressible, Hall MHD equations. \cite{Laakmann_Hu_Farrell_2022} introduces finite-element(-in-space) implicit timesteppers for the incompressible analogue to this system with structure-preserving (SP) properties in the ideal case, alongside parameter-robust preconditioners. We show that these timesteppers can derive from a finite-element-in-time (FET) (and finite-element-in-space) interpretation. The benefits of this reformulation are discussed, including the derivation of timesteppers that are higher order in time, and the quantifiable dissipative SP properties in the non-ideal, resistive case.
        
        We discuss possible options for extending this FET approach to timesteppers for the compressible case.

        The kinetic corrections satisfy linearized Boltzmann equations. Using a Lénard--Bernstein collision operator, these take Fokker--Planck-like forms \cite{Fokker_1914, Planck_1917} wherein pseudo-particles in the numerical model obey the neoclassical transport equations, with particle-independent Brownian drift terms. This offers a rigorous methodology for incorporating collisions into the particle transport model, without coupling the equations of motions for each particle.
        
        Works by Chen, Chacón et al. \cite{Chen_Chacón_Barnes_2011, Chacón_Chen_Barnes_2013, Chen_Chacón_2014, Chen_Chacón_2015} have developed structure-preserving particle pushers for neoclassical transport in the Vlasov equations, derived from Crank--Nicolson integrators. We show these too can can derive from a FET interpretation, similarly offering potential extensions to higher-order-in-time particle pushers. The FET formulation is used also to consider how the stochastic drift terms can be incorporated into the pushers. Stochastic gyrokinetic expansions are also discussed.

        Different options for the numerical implementation of these schemes are considered.

        Due to the efficacy of FET in the development of SP timesteppers for both the fluid and kinetic component, we hope this approach will prove effective in the future for developing SP timesteppers for the full hybrid model. We hope this will give us the opportunity to incorporate previously inaccessible kinetic effects into the highly effective, modern, finite-element MHD models.
    \end{abstract}
    
    
    \newpage
    \tableofcontents
    
    
    \newpage
    \pagenumbering{arabic}
    %\linenumbers\renewcommand\thelinenumber{\color{black!50}\arabic{linenumber}}
            \documentclass[12pt, a4paper]{report}

\input{template/main.tex}

\title{\BA{Title in Progress...}}
\author{Boris Andrews}
\affil{Mathematical Institute, University of Oxford}
\date{\today}


\begin{document}
    \pagenumbering{gobble}
    \maketitle
    
    
    \begin{abstract}
        Magnetic confinement reactors---in particular tokamaks---offer one of the most promising options for achieving practical nuclear fusion, with the potential to provide virtually limitless, clean energy. The theoretical and numerical modeling of tokamak plasmas is simultaneously an essential component of effective reactor design, and a great research barrier. Tokamak operational conditions exhibit comparatively low Knudsen numbers. Kinetic effects, including kinetic waves and instabilities, Landau damping, bump-on-tail instabilities and more, are therefore highly influential in tokamak plasma dynamics. Purely fluid models are inherently incapable of capturing these effects, whereas the high dimensionality in purely kinetic models render them practically intractable for most relevant purposes.

        We consider a $\delta\!f$ decomposition model, with a macroscopic fluid background and microscopic kinetic correction, both fully coupled to each other. A similar manner of discretization is proposed to that used in the recent \texttt{STRUPHY} code \cite{Holderied_Possanner_Wang_2021, Holderied_2022, Li_et_al_2023} with a finite-element model for the background and a pseudo-particle/PiC model for the correction.

        The fluid background satisfies the full, non-linear, resistive, compressible, Hall MHD equations. \cite{Laakmann_Hu_Farrell_2022} introduces finite-element(-in-space) implicit timesteppers for the incompressible analogue to this system with structure-preserving (SP) properties in the ideal case, alongside parameter-robust preconditioners. We show that these timesteppers can derive from a finite-element-in-time (FET) (and finite-element-in-space) interpretation. The benefits of this reformulation are discussed, including the derivation of timesteppers that are higher order in time, and the quantifiable dissipative SP properties in the non-ideal, resistive case.
        
        We discuss possible options for extending this FET approach to timesteppers for the compressible case.

        The kinetic corrections satisfy linearized Boltzmann equations. Using a Lénard--Bernstein collision operator, these take Fokker--Planck-like forms \cite{Fokker_1914, Planck_1917} wherein pseudo-particles in the numerical model obey the neoclassical transport equations, with particle-independent Brownian drift terms. This offers a rigorous methodology for incorporating collisions into the particle transport model, without coupling the equations of motions for each particle.
        
        Works by Chen, Chacón et al. \cite{Chen_Chacón_Barnes_2011, Chacón_Chen_Barnes_2013, Chen_Chacón_2014, Chen_Chacón_2015} have developed structure-preserving particle pushers for neoclassical transport in the Vlasov equations, derived from Crank--Nicolson integrators. We show these too can can derive from a FET interpretation, similarly offering potential extensions to higher-order-in-time particle pushers. The FET formulation is used also to consider how the stochastic drift terms can be incorporated into the pushers. Stochastic gyrokinetic expansions are also discussed.

        Different options for the numerical implementation of these schemes are considered.

        Due to the efficacy of FET in the development of SP timesteppers for both the fluid and kinetic component, we hope this approach will prove effective in the future for developing SP timesteppers for the full hybrid model. We hope this will give us the opportunity to incorporate previously inaccessible kinetic effects into the highly effective, modern, finite-element MHD models.
    \end{abstract}
    
    
    \newpage
    \tableofcontents
    
    
    \newpage
    \pagenumbering{arabic}
    %\linenumbers\renewcommand\thelinenumber{\color{black!50}\arabic{linenumber}}
            \input{0 - introduction/main.tex}
        \part{Research}
            \input{1 - low-noise PiC models/main.tex}
            \input{2 - kinetic component/main.tex}
            \input{3 - fluid component/main.tex}
            \input{4 - numerical implementation/main.tex}
        \part{Project Overview}
            \input{5 - research plan/main.tex}
            \input{6 - summary/main.tex}
    
    
    %\section{}
    \newpage
    \pagenumbering{gobble}
        \printbibliography


    \newpage
    \pagenumbering{roman}
    \appendix
        \part{Appendices}
            \input{8 - Hilbert complexes/main.tex}
            \input{9 - weak conservation proofs/main.tex}
\end{document}

        \part{Research}
            \documentclass[12pt, a4paper]{report}

\input{template/main.tex}

\title{\BA{Title in Progress...}}
\author{Boris Andrews}
\affil{Mathematical Institute, University of Oxford}
\date{\today}


\begin{document}
    \pagenumbering{gobble}
    \maketitle
    
    
    \begin{abstract}
        Magnetic confinement reactors---in particular tokamaks---offer one of the most promising options for achieving practical nuclear fusion, with the potential to provide virtually limitless, clean energy. The theoretical and numerical modeling of tokamak plasmas is simultaneously an essential component of effective reactor design, and a great research barrier. Tokamak operational conditions exhibit comparatively low Knudsen numbers. Kinetic effects, including kinetic waves and instabilities, Landau damping, bump-on-tail instabilities and more, are therefore highly influential in tokamak plasma dynamics. Purely fluid models are inherently incapable of capturing these effects, whereas the high dimensionality in purely kinetic models render them practically intractable for most relevant purposes.

        We consider a $\delta\!f$ decomposition model, with a macroscopic fluid background and microscopic kinetic correction, both fully coupled to each other. A similar manner of discretization is proposed to that used in the recent \texttt{STRUPHY} code \cite{Holderied_Possanner_Wang_2021, Holderied_2022, Li_et_al_2023} with a finite-element model for the background and a pseudo-particle/PiC model for the correction.

        The fluid background satisfies the full, non-linear, resistive, compressible, Hall MHD equations. \cite{Laakmann_Hu_Farrell_2022} introduces finite-element(-in-space) implicit timesteppers for the incompressible analogue to this system with structure-preserving (SP) properties in the ideal case, alongside parameter-robust preconditioners. We show that these timesteppers can derive from a finite-element-in-time (FET) (and finite-element-in-space) interpretation. The benefits of this reformulation are discussed, including the derivation of timesteppers that are higher order in time, and the quantifiable dissipative SP properties in the non-ideal, resistive case.
        
        We discuss possible options for extending this FET approach to timesteppers for the compressible case.

        The kinetic corrections satisfy linearized Boltzmann equations. Using a Lénard--Bernstein collision operator, these take Fokker--Planck-like forms \cite{Fokker_1914, Planck_1917} wherein pseudo-particles in the numerical model obey the neoclassical transport equations, with particle-independent Brownian drift terms. This offers a rigorous methodology for incorporating collisions into the particle transport model, without coupling the equations of motions for each particle.
        
        Works by Chen, Chacón et al. \cite{Chen_Chacón_Barnes_2011, Chacón_Chen_Barnes_2013, Chen_Chacón_2014, Chen_Chacón_2015} have developed structure-preserving particle pushers for neoclassical transport in the Vlasov equations, derived from Crank--Nicolson integrators. We show these too can can derive from a FET interpretation, similarly offering potential extensions to higher-order-in-time particle pushers. The FET formulation is used also to consider how the stochastic drift terms can be incorporated into the pushers. Stochastic gyrokinetic expansions are also discussed.

        Different options for the numerical implementation of these schemes are considered.

        Due to the efficacy of FET in the development of SP timesteppers for both the fluid and kinetic component, we hope this approach will prove effective in the future for developing SP timesteppers for the full hybrid model. We hope this will give us the opportunity to incorporate previously inaccessible kinetic effects into the highly effective, modern, finite-element MHD models.
    \end{abstract}
    
    
    \newpage
    \tableofcontents
    
    
    \newpage
    \pagenumbering{arabic}
    %\linenumbers\renewcommand\thelinenumber{\color{black!50}\arabic{linenumber}}
            \input{0 - introduction/main.tex}
        \part{Research}
            \input{1 - low-noise PiC models/main.tex}
            \input{2 - kinetic component/main.tex}
            \input{3 - fluid component/main.tex}
            \input{4 - numerical implementation/main.tex}
        \part{Project Overview}
            \input{5 - research plan/main.tex}
            \input{6 - summary/main.tex}
    
    
    %\section{}
    \newpage
    \pagenumbering{gobble}
        \printbibliography


    \newpage
    \pagenumbering{roman}
    \appendix
        \part{Appendices}
            \input{8 - Hilbert complexes/main.tex}
            \input{9 - weak conservation proofs/main.tex}
\end{document}

            \documentclass[12pt, a4paper]{report}

\input{template/main.tex}

\title{\BA{Title in Progress...}}
\author{Boris Andrews}
\affil{Mathematical Institute, University of Oxford}
\date{\today}


\begin{document}
    \pagenumbering{gobble}
    \maketitle
    
    
    \begin{abstract}
        Magnetic confinement reactors---in particular tokamaks---offer one of the most promising options for achieving practical nuclear fusion, with the potential to provide virtually limitless, clean energy. The theoretical and numerical modeling of tokamak plasmas is simultaneously an essential component of effective reactor design, and a great research barrier. Tokamak operational conditions exhibit comparatively low Knudsen numbers. Kinetic effects, including kinetic waves and instabilities, Landau damping, bump-on-tail instabilities and more, are therefore highly influential in tokamak plasma dynamics. Purely fluid models are inherently incapable of capturing these effects, whereas the high dimensionality in purely kinetic models render them practically intractable for most relevant purposes.

        We consider a $\delta\!f$ decomposition model, with a macroscopic fluid background and microscopic kinetic correction, both fully coupled to each other. A similar manner of discretization is proposed to that used in the recent \texttt{STRUPHY} code \cite{Holderied_Possanner_Wang_2021, Holderied_2022, Li_et_al_2023} with a finite-element model for the background and a pseudo-particle/PiC model for the correction.

        The fluid background satisfies the full, non-linear, resistive, compressible, Hall MHD equations. \cite{Laakmann_Hu_Farrell_2022} introduces finite-element(-in-space) implicit timesteppers for the incompressible analogue to this system with structure-preserving (SP) properties in the ideal case, alongside parameter-robust preconditioners. We show that these timesteppers can derive from a finite-element-in-time (FET) (and finite-element-in-space) interpretation. The benefits of this reformulation are discussed, including the derivation of timesteppers that are higher order in time, and the quantifiable dissipative SP properties in the non-ideal, resistive case.
        
        We discuss possible options for extending this FET approach to timesteppers for the compressible case.

        The kinetic corrections satisfy linearized Boltzmann equations. Using a Lénard--Bernstein collision operator, these take Fokker--Planck-like forms \cite{Fokker_1914, Planck_1917} wherein pseudo-particles in the numerical model obey the neoclassical transport equations, with particle-independent Brownian drift terms. This offers a rigorous methodology for incorporating collisions into the particle transport model, without coupling the equations of motions for each particle.
        
        Works by Chen, Chacón et al. \cite{Chen_Chacón_Barnes_2011, Chacón_Chen_Barnes_2013, Chen_Chacón_2014, Chen_Chacón_2015} have developed structure-preserving particle pushers for neoclassical transport in the Vlasov equations, derived from Crank--Nicolson integrators. We show these too can can derive from a FET interpretation, similarly offering potential extensions to higher-order-in-time particle pushers. The FET formulation is used also to consider how the stochastic drift terms can be incorporated into the pushers. Stochastic gyrokinetic expansions are also discussed.

        Different options for the numerical implementation of these schemes are considered.

        Due to the efficacy of FET in the development of SP timesteppers for both the fluid and kinetic component, we hope this approach will prove effective in the future for developing SP timesteppers for the full hybrid model. We hope this will give us the opportunity to incorporate previously inaccessible kinetic effects into the highly effective, modern, finite-element MHD models.
    \end{abstract}
    
    
    \newpage
    \tableofcontents
    
    
    \newpage
    \pagenumbering{arabic}
    %\linenumbers\renewcommand\thelinenumber{\color{black!50}\arabic{linenumber}}
            \input{0 - introduction/main.tex}
        \part{Research}
            \input{1 - low-noise PiC models/main.tex}
            \input{2 - kinetic component/main.tex}
            \input{3 - fluid component/main.tex}
            \input{4 - numerical implementation/main.tex}
        \part{Project Overview}
            \input{5 - research plan/main.tex}
            \input{6 - summary/main.tex}
    
    
    %\section{}
    \newpage
    \pagenumbering{gobble}
        \printbibliography


    \newpage
    \pagenumbering{roman}
    \appendix
        \part{Appendices}
            \input{8 - Hilbert complexes/main.tex}
            \input{9 - weak conservation proofs/main.tex}
\end{document}

            \documentclass[12pt, a4paper]{report}

\input{template/main.tex}

\title{\BA{Title in Progress...}}
\author{Boris Andrews}
\affil{Mathematical Institute, University of Oxford}
\date{\today}


\begin{document}
    \pagenumbering{gobble}
    \maketitle
    
    
    \begin{abstract}
        Magnetic confinement reactors---in particular tokamaks---offer one of the most promising options for achieving practical nuclear fusion, with the potential to provide virtually limitless, clean energy. The theoretical and numerical modeling of tokamak plasmas is simultaneously an essential component of effective reactor design, and a great research barrier. Tokamak operational conditions exhibit comparatively low Knudsen numbers. Kinetic effects, including kinetic waves and instabilities, Landau damping, bump-on-tail instabilities and more, are therefore highly influential in tokamak plasma dynamics. Purely fluid models are inherently incapable of capturing these effects, whereas the high dimensionality in purely kinetic models render them practically intractable for most relevant purposes.

        We consider a $\delta\!f$ decomposition model, with a macroscopic fluid background and microscopic kinetic correction, both fully coupled to each other. A similar manner of discretization is proposed to that used in the recent \texttt{STRUPHY} code \cite{Holderied_Possanner_Wang_2021, Holderied_2022, Li_et_al_2023} with a finite-element model for the background and a pseudo-particle/PiC model for the correction.

        The fluid background satisfies the full, non-linear, resistive, compressible, Hall MHD equations. \cite{Laakmann_Hu_Farrell_2022} introduces finite-element(-in-space) implicit timesteppers for the incompressible analogue to this system with structure-preserving (SP) properties in the ideal case, alongside parameter-robust preconditioners. We show that these timesteppers can derive from a finite-element-in-time (FET) (and finite-element-in-space) interpretation. The benefits of this reformulation are discussed, including the derivation of timesteppers that are higher order in time, and the quantifiable dissipative SP properties in the non-ideal, resistive case.
        
        We discuss possible options for extending this FET approach to timesteppers for the compressible case.

        The kinetic corrections satisfy linearized Boltzmann equations. Using a Lénard--Bernstein collision operator, these take Fokker--Planck-like forms \cite{Fokker_1914, Planck_1917} wherein pseudo-particles in the numerical model obey the neoclassical transport equations, with particle-independent Brownian drift terms. This offers a rigorous methodology for incorporating collisions into the particle transport model, without coupling the equations of motions for each particle.
        
        Works by Chen, Chacón et al. \cite{Chen_Chacón_Barnes_2011, Chacón_Chen_Barnes_2013, Chen_Chacón_2014, Chen_Chacón_2015} have developed structure-preserving particle pushers for neoclassical transport in the Vlasov equations, derived from Crank--Nicolson integrators. We show these too can can derive from a FET interpretation, similarly offering potential extensions to higher-order-in-time particle pushers. The FET formulation is used also to consider how the stochastic drift terms can be incorporated into the pushers. Stochastic gyrokinetic expansions are also discussed.

        Different options for the numerical implementation of these schemes are considered.

        Due to the efficacy of FET in the development of SP timesteppers for both the fluid and kinetic component, we hope this approach will prove effective in the future for developing SP timesteppers for the full hybrid model. We hope this will give us the opportunity to incorporate previously inaccessible kinetic effects into the highly effective, modern, finite-element MHD models.
    \end{abstract}
    
    
    \newpage
    \tableofcontents
    
    
    \newpage
    \pagenumbering{arabic}
    %\linenumbers\renewcommand\thelinenumber{\color{black!50}\arabic{linenumber}}
            \input{0 - introduction/main.tex}
        \part{Research}
            \input{1 - low-noise PiC models/main.tex}
            \input{2 - kinetic component/main.tex}
            \input{3 - fluid component/main.tex}
            \input{4 - numerical implementation/main.tex}
        \part{Project Overview}
            \input{5 - research plan/main.tex}
            \input{6 - summary/main.tex}
    
    
    %\section{}
    \newpage
    \pagenumbering{gobble}
        \printbibliography


    \newpage
    \pagenumbering{roman}
    \appendix
        \part{Appendices}
            \input{8 - Hilbert complexes/main.tex}
            \input{9 - weak conservation proofs/main.tex}
\end{document}

            \documentclass[12pt, a4paper]{report}

\input{template/main.tex}

\title{\BA{Title in Progress...}}
\author{Boris Andrews}
\affil{Mathematical Institute, University of Oxford}
\date{\today}


\begin{document}
    \pagenumbering{gobble}
    \maketitle
    
    
    \begin{abstract}
        Magnetic confinement reactors---in particular tokamaks---offer one of the most promising options for achieving practical nuclear fusion, with the potential to provide virtually limitless, clean energy. The theoretical and numerical modeling of tokamak plasmas is simultaneously an essential component of effective reactor design, and a great research barrier. Tokamak operational conditions exhibit comparatively low Knudsen numbers. Kinetic effects, including kinetic waves and instabilities, Landau damping, bump-on-tail instabilities and more, are therefore highly influential in tokamak plasma dynamics. Purely fluid models are inherently incapable of capturing these effects, whereas the high dimensionality in purely kinetic models render them practically intractable for most relevant purposes.

        We consider a $\delta\!f$ decomposition model, with a macroscopic fluid background and microscopic kinetic correction, both fully coupled to each other. A similar manner of discretization is proposed to that used in the recent \texttt{STRUPHY} code \cite{Holderied_Possanner_Wang_2021, Holderied_2022, Li_et_al_2023} with a finite-element model for the background and a pseudo-particle/PiC model for the correction.

        The fluid background satisfies the full, non-linear, resistive, compressible, Hall MHD equations. \cite{Laakmann_Hu_Farrell_2022} introduces finite-element(-in-space) implicit timesteppers for the incompressible analogue to this system with structure-preserving (SP) properties in the ideal case, alongside parameter-robust preconditioners. We show that these timesteppers can derive from a finite-element-in-time (FET) (and finite-element-in-space) interpretation. The benefits of this reformulation are discussed, including the derivation of timesteppers that are higher order in time, and the quantifiable dissipative SP properties in the non-ideal, resistive case.
        
        We discuss possible options for extending this FET approach to timesteppers for the compressible case.

        The kinetic corrections satisfy linearized Boltzmann equations. Using a Lénard--Bernstein collision operator, these take Fokker--Planck-like forms \cite{Fokker_1914, Planck_1917} wherein pseudo-particles in the numerical model obey the neoclassical transport equations, with particle-independent Brownian drift terms. This offers a rigorous methodology for incorporating collisions into the particle transport model, without coupling the equations of motions for each particle.
        
        Works by Chen, Chacón et al. \cite{Chen_Chacón_Barnes_2011, Chacón_Chen_Barnes_2013, Chen_Chacón_2014, Chen_Chacón_2015} have developed structure-preserving particle pushers for neoclassical transport in the Vlasov equations, derived from Crank--Nicolson integrators. We show these too can can derive from a FET interpretation, similarly offering potential extensions to higher-order-in-time particle pushers. The FET formulation is used also to consider how the stochastic drift terms can be incorporated into the pushers. Stochastic gyrokinetic expansions are also discussed.

        Different options for the numerical implementation of these schemes are considered.

        Due to the efficacy of FET in the development of SP timesteppers for both the fluid and kinetic component, we hope this approach will prove effective in the future for developing SP timesteppers for the full hybrid model. We hope this will give us the opportunity to incorporate previously inaccessible kinetic effects into the highly effective, modern, finite-element MHD models.
    \end{abstract}
    
    
    \newpage
    \tableofcontents
    
    
    \newpage
    \pagenumbering{arabic}
    %\linenumbers\renewcommand\thelinenumber{\color{black!50}\arabic{linenumber}}
            \input{0 - introduction/main.tex}
        \part{Research}
            \input{1 - low-noise PiC models/main.tex}
            \input{2 - kinetic component/main.tex}
            \input{3 - fluid component/main.tex}
            \input{4 - numerical implementation/main.tex}
        \part{Project Overview}
            \input{5 - research plan/main.tex}
            \input{6 - summary/main.tex}
    
    
    %\section{}
    \newpage
    \pagenumbering{gobble}
        \printbibliography


    \newpage
    \pagenumbering{roman}
    \appendix
        \part{Appendices}
            \input{8 - Hilbert complexes/main.tex}
            \input{9 - weak conservation proofs/main.tex}
\end{document}

        \part{Project Overview}
            \documentclass[12pt, a4paper]{report}

\input{template/main.tex}

\title{\BA{Title in Progress...}}
\author{Boris Andrews}
\affil{Mathematical Institute, University of Oxford}
\date{\today}


\begin{document}
    \pagenumbering{gobble}
    \maketitle
    
    
    \begin{abstract}
        Magnetic confinement reactors---in particular tokamaks---offer one of the most promising options for achieving practical nuclear fusion, with the potential to provide virtually limitless, clean energy. The theoretical and numerical modeling of tokamak plasmas is simultaneously an essential component of effective reactor design, and a great research barrier. Tokamak operational conditions exhibit comparatively low Knudsen numbers. Kinetic effects, including kinetic waves and instabilities, Landau damping, bump-on-tail instabilities and more, are therefore highly influential in tokamak plasma dynamics. Purely fluid models are inherently incapable of capturing these effects, whereas the high dimensionality in purely kinetic models render them practically intractable for most relevant purposes.

        We consider a $\delta\!f$ decomposition model, with a macroscopic fluid background and microscopic kinetic correction, both fully coupled to each other. A similar manner of discretization is proposed to that used in the recent \texttt{STRUPHY} code \cite{Holderied_Possanner_Wang_2021, Holderied_2022, Li_et_al_2023} with a finite-element model for the background and a pseudo-particle/PiC model for the correction.

        The fluid background satisfies the full, non-linear, resistive, compressible, Hall MHD equations. \cite{Laakmann_Hu_Farrell_2022} introduces finite-element(-in-space) implicit timesteppers for the incompressible analogue to this system with structure-preserving (SP) properties in the ideal case, alongside parameter-robust preconditioners. We show that these timesteppers can derive from a finite-element-in-time (FET) (and finite-element-in-space) interpretation. The benefits of this reformulation are discussed, including the derivation of timesteppers that are higher order in time, and the quantifiable dissipative SP properties in the non-ideal, resistive case.
        
        We discuss possible options for extending this FET approach to timesteppers for the compressible case.

        The kinetic corrections satisfy linearized Boltzmann equations. Using a Lénard--Bernstein collision operator, these take Fokker--Planck-like forms \cite{Fokker_1914, Planck_1917} wherein pseudo-particles in the numerical model obey the neoclassical transport equations, with particle-independent Brownian drift terms. This offers a rigorous methodology for incorporating collisions into the particle transport model, without coupling the equations of motions for each particle.
        
        Works by Chen, Chacón et al. \cite{Chen_Chacón_Barnes_2011, Chacón_Chen_Barnes_2013, Chen_Chacón_2014, Chen_Chacón_2015} have developed structure-preserving particle pushers for neoclassical transport in the Vlasov equations, derived from Crank--Nicolson integrators. We show these too can can derive from a FET interpretation, similarly offering potential extensions to higher-order-in-time particle pushers. The FET formulation is used also to consider how the stochastic drift terms can be incorporated into the pushers. Stochastic gyrokinetic expansions are also discussed.

        Different options for the numerical implementation of these schemes are considered.

        Due to the efficacy of FET in the development of SP timesteppers for both the fluid and kinetic component, we hope this approach will prove effective in the future for developing SP timesteppers for the full hybrid model. We hope this will give us the opportunity to incorporate previously inaccessible kinetic effects into the highly effective, modern, finite-element MHD models.
    \end{abstract}
    
    
    \newpage
    \tableofcontents
    
    
    \newpage
    \pagenumbering{arabic}
    %\linenumbers\renewcommand\thelinenumber{\color{black!50}\arabic{linenumber}}
            \input{0 - introduction/main.tex}
        \part{Research}
            \input{1 - low-noise PiC models/main.tex}
            \input{2 - kinetic component/main.tex}
            \input{3 - fluid component/main.tex}
            \input{4 - numerical implementation/main.tex}
        \part{Project Overview}
            \input{5 - research plan/main.tex}
            \input{6 - summary/main.tex}
    
    
    %\section{}
    \newpage
    \pagenumbering{gobble}
        \printbibliography


    \newpage
    \pagenumbering{roman}
    \appendix
        \part{Appendices}
            \input{8 - Hilbert complexes/main.tex}
            \input{9 - weak conservation proofs/main.tex}
\end{document}

            \documentclass[12pt, a4paper]{report}

\input{template/main.tex}

\title{\BA{Title in Progress...}}
\author{Boris Andrews}
\affil{Mathematical Institute, University of Oxford}
\date{\today}


\begin{document}
    \pagenumbering{gobble}
    \maketitle
    
    
    \begin{abstract}
        Magnetic confinement reactors---in particular tokamaks---offer one of the most promising options for achieving practical nuclear fusion, with the potential to provide virtually limitless, clean energy. The theoretical and numerical modeling of tokamak plasmas is simultaneously an essential component of effective reactor design, and a great research barrier. Tokamak operational conditions exhibit comparatively low Knudsen numbers. Kinetic effects, including kinetic waves and instabilities, Landau damping, bump-on-tail instabilities and more, are therefore highly influential in tokamak plasma dynamics. Purely fluid models are inherently incapable of capturing these effects, whereas the high dimensionality in purely kinetic models render them practically intractable for most relevant purposes.

        We consider a $\delta\!f$ decomposition model, with a macroscopic fluid background and microscopic kinetic correction, both fully coupled to each other. A similar manner of discretization is proposed to that used in the recent \texttt{STRUPHY} code \cite{Holderied_Possanner_Wang_2021, Holderied_2022, Li_et_al_2023} with a finite-element model for the background and a pseudo-particle/PiC model for the correction.

        The fluid background satisfies the full, non-linear, resistive, compressible, Hall MHD equations. \cite{Laakmann_Hu_Farrell_2022} introduces finite-element(-in-space) implicit timesteppers for the incompressible analogue to this system with structure-preserving (SP) properties in the ideal case, alongside parameter-robust preconditioners. We show that these timesteppers can derive from a finite-element-in-time (FET) (and finite-element-in-space) interpretation. The benefits of this reformulation are discussed, including the derivation of timesteppers that are higher order in time, and the quantifiable dissipative SP properties in the non-ideal, resistive case.
        
        We discuss possible options for extending this FET approach to timesteppers for the compressible case.

        The kinetic corrections satisfy linearized Boltzmann equations. Using a Lénard--Bernstein collision operator, these take Fokker--Planck-like forms \cite{Fokker_1914, Planck_1917} wherein pseudo-particles in the numerical model obey the neoclassical transport equations, with particle-independent Brownian drift terms. This offers a rigorous methodology for incorporating collisions into the particle transport model, without coupling the equations of motions for each particle.
        
        Works by Chen, Chacón et al. \cite{Chen_Chacón_Barnes_2011, Chacón_Chen_Barnes_2013, Chen_Chacón_2014, Chen_Chacón_2015} have developed structure-preserving particle pushers for neoclassical transport in the Vlasov equations, derived from Crank--Nicolson integrators. We show these too can can derive from a FET interpretation, similarly offering potential extensions to higher-order-in-time particle pushers. The FET formulation is used also to consider how the stochastic drift terms can be incorporated into the pushers. Stochastic gyrokinetic expansions are also discussed.

        Different options for the numerical implementation of these schemes are considered.

        Due to the efficacy of FET in the development of SP timesteppers for both the fluid and kinetic component, we hope this approach will prove effective in the future for developing SP timesteppers for the full hybrid model. We hope this will give us the opportunity to incorporate previously inaccessible kinetic effects into the highly effective, modern, finite-element MHD models.
    \end{abstract}
    
    
    \newpage
    \tableofcontents
    
    
    \newpage
    \pagenumbering{arabic}
    %\linenumbers\renewcommand\thelinenumber{\color{black!50}\arabic{linenumber}}
            \input{0 - introduction/main.tex}
        \part{Research}
            \input{1 - low-noise PiC models/main.tex}
            \input{2 - kinetic component/main.tex}
            \input{3 - fluid component/main.tex}
            \input{4 - numerical implementation/main.tex}
        \part{Project Overview}
            \input{5 - research plan/main.tex}
            \input{6 - summary/main.tex}
    
    
    %\section{}
    \newpage
    \pagenumbering{gobble}
        \printbibliography


    \newpage
    \pagenumbering{roman}
    \appendix
        \part{Appendices}
            \input{8 - Hilbert complexes/main.tex}
            \input{9 - weak conservation proofs/main.tex}
\end{document}

    
    
    %\section{}
    \newpage
    \pagenumbering{gobble}
        \printbibliography


    \newpage
    \pagenumbering{roman}
    \appendix
        \part{Appendices}
            \documentclass[12pt, a4paper]{report}

\input{template/main.tex}

\title{\BA{Title in Progress...}}
\author{Boris Andrews}
\affil{Mathematical Institute, University of Oxford}
\date{\today}


\begin{document}
    \pagenumbering{gobble}
    \maketitle
    
    
    \begin{abstract}
        Magnetic confinement reactors---in particular tokamaks---offer one of the most promising options for achieving practical nuclear fusion, with the potential to provide virtually limitless, clean energy. The theoretical and numerical modeling of tokamak plasmas is simultaneously an essential component of effective reactor design, and a great research barrier. Tokamak operational conditions exhibit comparatively low Knudsen numbers. Kinetic effects, including kinetic waves and instabilities, Landau damping, bump-on-tail instabilities and more, are therefore highly influential in tokamak plasma dynamics. Purely fluid models are inherently incapable of capturing these effects, whereas the high dimensionality in purely kinetic models render them practically intractable for most relevant purposes.

        We consider a $\delta\!f$ decomposition model, with a macroscopic fluid background and microscopic kinetic correction, both fully coupled to each other. A similar manner of discretization is proposed to that used in the recent \texttt{STRUPHY} code \cite{Holderied_Possanner_Wang_2021, Holderied_2022, Li_et_al_2023} with a finite-element model for the background and a pseudo-particle/PiC model for the correction.

        The fluid background satisfies the full, non-linear, resistive, compressible, Hall MHD equations. \cite{Laakmann_Hu_Farrell_2022} introduces finite-element(-in-space) implicit timesteppers for the incompressible analogue to this system with structure-preserving (SP) properties in the ideal case, alongside parameter-robust preconditioners. We show that these timesteppers can derive from a finite-element-in-time (FET) (and finite-element-in-space) interpretation. The benefits of this reformulation are discussed, including the derivation of timesteppers that are higher order in time, and the quantifiable dissipative SP properties in the non-ideal, resistive case.
        
        We discuss possible options for extending this FET approach to timesteppers for the compressible case.

        The kinetic corrections satisfy linearized Boltzmann equations. Using a Lénard--Bernstein collision operator, these take Fokker--Planck-like forms \cite{Fokker_1914, Planck_1917} wherein pseudo-particles in the numerical model obey the neoclassical transport equations, with particle-independent Brownian drift terms. This offers a rigorous methodology for incorporating collisions into the particle transport model, without coupling the equations of motions for each particle.
        
        Works by Chen, Chacón et al. \cite{Chen_Chacón_Barnes_2011, Chacón_Chen_Barnes_2013, Chen_Chacón_2014, Chen_Chacón_2015} have developed structure-preserving particle pushers for neoclassical transport in the Vlasov equations, derived from Crank--Nicolson integrators. We show these too can can derive from a FET interpretation, similarly offering potential extensions to higher-order-in-time particle pushers. The FET formulation is used also to consider how the stochastic drift terms can be incorporated into the pushers. Stochastic gyrokinetic expansions are also discussed.

        Different options for the numerical implementation of these schemes are considered.

        Due to the efficacy of FET in the development of SP timesteppers for both the fluid and kinetic component, we hope this approach will prove effective in the future for developing SP timesteppers for the full hybrid model. We hope this will give us the opportunity to incorporate previously inaccessible kinetic effects into the highly effective, modern, finite-element MHD models.
    \end{abstract}
    
    
    \newpage
    \tableofcontents
    
    
    \newpage
    \pagenumbering{arabic}
    %\linenumbers\renewcommand\thelinenumber{\color{black!50}\arabic{linenumber}}
            \input{0 - introduction/main.tex}
        \part{Research}
            \input{1 - low-noise PiC models/main.tex}
            \input{2 - kinetic component/main.tex}
            \input{3 - fluid component/main.tex}
            \input{4 - numerical implementation/main.tex}
        \part{Project Overview}
            \input{5 - research plan/main.tex}
            \input{6 - summary/main.tex}
    
    
    %\section{}
    \newpage
    \pagenumbering{gobble}
        \printbibliography


    \newpage
    \pagenumbering{roman}
    \appendix
        \part{Appendices}
            \input{8 - Hilbert complexes/main.tex}
            \input{9 - weak conservation proofs/main.tex}
\end{document}

            \documentclass[12pt, a4paper]{report}

\input{template/main.tex}

\title{\BA{Title in Progress...}}
\author{Boris Andrews}
\affil{Mathematical Institute, University of Oxford}
\date{\today}


\begin{document}
    \pagenumbering{gobble}
    \maketitle
    
    
    \begin{abstract}
        Magnetic confinement reactors---in particular tokamaks---offer one of the most promising options for achieving practical nuclear fusion, with the potential to provide virtually limitless, clean energy. The theoretical and numerical modeling of tokamak plasmas is simultaneously an essential component of effective reactor design, and a great research barrier. Tokamak operational conditions exhibit comparatively low Knudsen numbers. Kinetic effects, including kinetic waves and instabilities, Landau damping, bump-on-tail instabilities and more, are therefore highly influential in tokamak plasma dynamics. Purely fluid models are inherently incapable of capturing these effects, whereas the high dimensionality in purely kinetic models render them practically intractable for most relevant purposes.

        We consider a $\delta\!f$ decomposition model, with a macroscopic fluid background and microscopic kinetic correction, both fully coupled to each other. A similar manner of discretization is proposed to that used in the recent \texttt{STRUPHY} code \cite{Holderied_Possanner_Wang_2021, Holderied_2022, Li_et_al_2023} with a finite-element model for the background and a pseudo-particle/PiC model for the correction.

        The fluid background satisfies the full, non-linear, resistive, compressible, Hall MHD equations. \cite{Laakmann_Hu_Farrell_2022} introduces finite-element(-in-space) implicit timesteppers for the incompressible analogue to this system with structure-preserving (SP) properties in the ideal case, alongside parameter-robust preconditioners. We show that these timesteppers can derive from a finite-element-in-time (FET) (and finite-element-in-space) interpretation. The benefits of this reformulation are discussed, including the derivation of timesteppers that are higher order in time, and the quantifiable dissipative SP properties in the non-ideal, resistive case.
        
        We discuss possible options for extending this FET approach to timesteppers for the compressible case.

        The kinetic corrections satisfy linearized Boltzmann equations. Using a Lénard--Bernstein collision operator, these take Fokker--Planck-like forms \cite{Fokker_1914, Planck_1917} wherein pseudo-particles in the numerical model obey the neoclassical transport equations, with particle-independent Brownian drift terms. This offers a rigorous methodology for incorporating collisions into the particle transport model, without coupling the equations of motions for each particle.
        
        Works by Chen, Chacón et al. \cite{Chen_Chacón_Barnes_2011, Chacón_Chen_Barnes_2013, Chen_Chacón_2014, Chen_Chacón_2015} have developed structure-preserving particle pushers for neoclassical transport in the Vlasov equations, derived from Crank--Nicolson integrators. We show these too can can derive from a FET interpretation, similarly offering potential extensions to higher-order-in-time particle pushers. The FET formulation is used also to consider how the stochastic drift terms can be incorporated into the pushers. Stochastic gyrokinetic expansions are also discussed.

        Different options for the numerical implementation of these schemes are considered.

        Due to the efficacy of FET in the development of SP timesteppers for both the fluid and kinetic component, we hope this approach will prove effective in the future for developing SP timesteppers for the full hybrid model. We hope this will give us the opportunity to incorporate previously inaccessible kinetic effects into the highly effective, modern, finite-element MHD models.
    \end{abstract}
    
    
    \newpage
    \tableofcontents
    
    
    \newpage
    \pagenumbering{arabic}
    %\linenumbers\renewcommand\thelinenumber{\color{black!50}\arabic{linenumber}}
            \input{0 - introduction/main.tex}
        \part{Research}
            \input{1 - low-noise PiC models/main.tex}
            \input{2 - kinetic component/main.tex}
            \input{3 - fluid component/main.tex}
            \input{4 - numerical implementation/main.tex}
        \part{Project Overview}
            \input{5 - research plan/main.tex}
            \input{6 - summary/main.tex}
    
    
    %\section{}
    \newpage
    \pagenumbering{gobble}
        \printbibliography


    \newpage
    \pagenumbering{roman}
    \appendix
        \part{Appendices}
            \input{8 - Hilbert complexes/main.tex}
            \input{9 - weak conservation proofs/main.tex}
\end{document}

\end{document}

            \documentclass[12pt, a4paper]{report}

\documentclass[12pt, a4paper]{report}

\input{template/main.tex}

\title{\BA{Title in Progress...}}
\author{Boris Andrews}
\affil{Mathematical Institute, University of Oxford}
\date{\today}


\begin{document}
    \pagenumbering{gobble}
    \maketitle
    
    
    \begin{abstract}
        Magnetic confinement reactors---in particular tokamaks---offer one of the most promising options for achieving practical nuclear fusion, with the potential to provide virtually limitless, clean energy. The theoretical and numerical modeling of tokamak plasmas is simultaneously an essential component of effective reactor design, and a great research barrier. Tokamak operational conditions exhibit comparatively low Knudsen numbers. Kinetic effects, including kinetic waves and instabilities, Landau damping, bump-on-tail instabilities and more, are therefore highly influential in tokamak plasma dynamics. Purely fluid models are inherently incapable of capturing these effects, whereas the high dimensionality in purely kinetic models render them practically intractable for most relevant purposes.

        We consider a $\delta\!f$ decomposition model, with a macroscopic fluid background and microscopic kinetic correction, both fully coupled to each other. A similar manner of discretization is proposed to that used in the recent \texttt{STRUPHY} code \cite{Holderied_Possanner_Wang_2021, Holderied_2022, Li_et_al_2023} with a finite-element model for the background and a pseudo-particle/PiC model for the correction.

        The fluid background satisfies the full, non-linear, resistive, compressible, Hall MHD equations. \cite{Laakmann_Hu_Farrell_2022} introduces finite-element(-in-space) implicit timesteppers for the incompressible analogue to this system with structure-preserving (SP) properties in the ideal case, alongside parameter-robust preconditioners. We show that these timesteppers can derive from a finite-element-in-time (FET) (and finite-element-in-space) interpretation. The benefits of this reformulation are discussed, including the derivation of timesteppers that are higher order in time, and the quantifiable dissipative SP properties in the non-ideal, resistive case.
        
        We discuss possible options for extending this FET approach to timesteppers for the compressible case.

        The kinetic corrections satisfy linearized Boltzmann equations. Using a Lénard--Bernstein collision operator, these take Fokker--Planck-like forms \cite{Fokker_1914, Planck_1917} wherein pseudo-particles in the numerical model obey the neoclassical transport equations, with particle-independent Brownian drift terms. This offers a rigorous methodology for incorporating collisions into the particle transport model, without coupling the equations of motions for each particle.
        
        Works by Chen, Chacón et al. \cite{Chen_Chacón_Barnes_2011, Chacón_Chen_Barnes_2013, Chen_Chacón_2014, Chen_Chacón_2015} have developed structure-preserving particle pushers for neoclassical transport in the Vlasov equations, derived from Crank--Nicolson integrators. We show these too can can derive from a FET interpretation, similarly offering potential extensions to higher-order-in-time particle pushers. The FET formulation is used also to consider how the stochastic drift terms can be incorporated into the pushers. Stochastic gyrokinetic expansions are also discussed.

        Different options for the numerical implementation of these schemes are considered.

        Due to the efficacy of FET in the development of SP timesteppers for both the fluid and kinetic component, we hope this approach will prove effective in the future for developing SP timesteppers for the full hybrid model. We hope this will give us the opportunity to incorporate previously inaccessible kinetic effects into the highly effective, modern, finite-element MHD models.
    \end{abstract}
    
    
    \newpage
    \tableofcontents
    
    
    \newpage
    \pagenumbering{arabic}
    %\linenumbers\renewcommand\thelinenumber{\color{black!50}\arabic{linenumber}}
            \input{0 - introduction/main.tex}
        \part{Research}
            \input{1 - low-noise PiC models/main.tex}
            \input{2 - kinetic component/main.tex}
            \input{3 - fluid component/main.tex}
            \input{4 - numerical implementation/main.tex}
        \part{Project Overview}
            \input{5 - research plan/main.tex}
            \input{6 - summary/main.tex}
    
    
    %\section{}
    \newpage
    \pagenumbering{gobble}
        \printbibliography


    \newpage
    \pagenumbering{roman}
    \appendix
        \part{Appendices}
            \input{8 - Hilbert complexes/main.tex}
            \input{9 - weak conservation proofs/main.tex}
\end{document}


\title{\BA{Title in Progress...}}
\author{Boris Andrews}
\affil{Mathematical Institute, University of Oxford}
\date{\today}


\begin{document}
    \pagenumbering{gobble}
    \maketitle
    
    
    \begin{abstract}
        Magnetic confinement reactors---in particular tokamaks---offer one of the most promising options for achieving practical nuclear fusion, with the potential to provide virtually limitless, clean energy. The theoretical and numerical modeling of tokamak plasmas is simultaneously an essential component of effective reactor design, and a great research barrier. Tokamak operational conditions exhibit comparatively low Knudsen numbers. Kinetic effects, including kinetic waves and instabilities, Landau damping, bump-on-tail instabilities and more, are therefore highly influential in tokamak plasma dynamics. Purely fluid models are inherently incapable of capturing these effects, whereas the high dimensionality in purely kinetic models render them practically intractable for most relevant purposes.

        We consider a $\delta\!f$ decomposition model, with a macroscopic fluid background and microscopic kinetic correction, both fully coupled to each other. A similar manner of discretization is proposed to that used in the recent \texttt{STRUPHY} code \cite{Holderied_Possanner_Wang_2021, Holderied_2022, Li_et_al_2023} with a finite-element model for the background and a pseudo-particle/PiC model for the correction.

        The fluid background satisfies the full, non-linear, resistive, compressible, Hall MHD equations. \cite{Laakmann_Hu_Farrell_2022} introduces finite-element(-in-space) implicit timesteppers for the incompressible analogue to this system with structure-preserving (SP) properties in the ideal case, alongside parameter-robust preconditioners. We show that these timesteppers can derive from a finite-element-in-time (FET) (and finite-element-in-space) interpretation. The benefits of this reformulation are discussed, including the derivation of timesteppers that are higher order in time, and the quantifiable dissipative SP properties in the non-ideal, resistive case.
        
        We discuss possible options for extending this FET approach to timesteppers for the compressible case.

        The kinetic corrections satisfy linearized Boltzmann equations. Using a Lénard--Bernstein collision operator, these take Fokker--Planck-like forms \cite{Fokker_1914, Planck_1917} wherein pseudo-particles in the numerical model obey the neoclassical transport equations, with particle-independent Brownian drift terms. This offers a rigorous methodology for incorporating collisions into the particle transport model, without coupling the equations of motions for each particle.
        
        Works by Chen, Chacón et al. \cite{Chen_Chacón_Barnes_2011, Chacón_Chen_Barnes_2013, Chen_Chacón_2014, Chen_Chacón_2015} have developed structure-preserving particle pushers for neoclassical transport in the Vlasov equations, derived from Crank--Nicolson integrators. We show these too can can derive from a FET interpretation, similarly offering potential extensions to higher-order-in-time particle pushers. The FET formulation is used also to consider how the stochastic drift terms can be incorporated into the pushers. Stochastic gyrokinetic expansions are also discussed.

        Different options for the numerical implementation of these schemes are considered.

        Due to the efficacy of FET in the development of SP timesteppers for both the fluid and kinetic component, we hope this approach will prove effective in the future for developing SP timesteppers for the full hybrid model. We hope this will give us the opportunity to incorporate previously inaccessible kinetic effects into the highly effective, modern, finite-element MHD models.
    \end{abstract}
    
    
    \newpage
    \tableofcontents
    
    
    \newpage
    \pagenumbering{arabic}
    %\linenumbers\renewcommand\thelinenumber{\color{black!50}\arabic{linenumber}}
            \documentclass[12pt, a4paper]{report}

\input{template/main.tex}

\title{\BA{Title in Progress...}}
\author{Boris Andrews}
\affil{Mathematical Institute, University of Oxford}
\date{\today}


\begin{document}
    \pagenumbering{gobble}
    \maketitle
    
    
    \begin{abstract}
        Magnetic confinement reactors---in particular tokamaks---offer one of the most promising options for achieving practical nuclear fusion, with the potential to provide virtually limitless, clean energy. The theoretical and numerical modeling of tokamak plasmas is simultaneously an essential component of effective reactor design, and a great research barrier. Tokamak operational conditions exhibit comparatively low Knudsen numbers. Kinetic effects, including kinetic waves and instabilities, Landau damping, bump-on-tail instabilities and more, are therefore highly influential in tokamak plasma dynamics. Purely fluid models are inherently incapable of capturing these effects, whereas the high dimensionality in purely kinetic models render them practically intractable for most relevant purposes.

        We consider a $\delta\!f$ decomposition model, with a macroscopic fluid background and microscopic kinetic correction, both fully coupled to each other. A similar manner of discretization is proposed to that used in the recent \texttt{STRUPHY} code \cite{Holderied_Possanner_Wang_2021, Holderied_2022, Li_et_al_2023} with a finite-element model for the background and a pseudo-particle/PiC model for the correction.

        The fluid background satisfies the full, non-linear, resistive, compressible, Hall MHD equations. \cite{Laakmann_Hu_Farrell_2022} introduces finite-element(-in-space) implicit timesteppers for the incompressible analogue to this system with structure-preserving (SP) properties in the ideal case, alongside parameter-robust preconditioners. We show that these timesteppers can derive from a finite-element-in-time (FET) (and finite-element-in-space) interpretation. The benefits of this reformulation are discussed, including the derivation of timesteppers that are higher order in time, and the quantifiable dissipative SP properties in the non-ideal, resistive case.
        
        We discuss possible options for extending this FET approach to timesteppers for the compressible case.

        The kinetic corrections satisfy linearized Boltzmann equations. Using a Lénard--Bernstein collision operator, these take Fokker--Planck-like forms \cite{Fokker_1914, Planck_1917} wherein pseudo-particles in the numerical model obey the neoclassical transport equations, with particle-independent Brownian drift terms. This offers a rigorous methodology for incorporating collisions into the particle transport model, without coupling the equations of motions for each particle.
        
        Works by Chen, Chacón et al. \cite{Chen_Chacón_Barnes_2011, Chacón_Chen_Barnes_2013, Chen_Chacón_2014, Chen_Chacón_2015} have developed structure-preserving particle pushers for neoclassical transport in the Vlasov equations, derived from Crank--Nicolson integrators. We show these too can can derive from a FET interpretation, similarly offering potential extensions to higher-order-in-time particle pushers. The FET formulation is used also to consider how the stochastic drift terms can be incorporated into the pushers. Stochastic gyrokinetic expansions are also discussed.

        Different options for the numerical implementation of these schemes are considered.

        Due to the efficacy of FET in the development of SP timesteppers for both the fluid and kinetic component, we hope this approach will prove effective in the future for developing SP timesteppers for the full hybrid model. We hope this will give us the opportunity to incorporate previously inaccessible kinetic effects into the highly effective, modern, finite-element MHD models.
    \end{abstract}
    
    
    \newpage
    \tableofcontents
    
    
    \newpage
    \pagenumbering{arabic}
    %\linenumbers\renewcommand\thelinenumber{\color{black!50}\arabic{linenumber}}
            \input{0 - introduction/main.tex}
        \part{Research}
            \input{1 - low-noise PiC models/main.tex}
            \input{2 - kinetic component/main.tex}
            \input{3 - fluid component/main.tex}
            \input{4 - numerical implementation/main.tex}
        \part{Project Overview}
            \input{5 - research plan/main.tex}
            \input{6 - summary/main.tex}
    
    
    %\section{}
    \newpage
    \pagenumbering{gobble}
        \printbibliography


    \newpage
    \pagenumbering{roman}
    \appendix
        \part{Appendices}
            \input{8 - Hilbert complexes/main.tex}
            \input{9 - weak conservation proofs/main.tex}
\end{document}

        \part{Research}
            \documentclass[12pt, a4paper]{report}

\input{template/main.tex}

\title{\BA{Title in Progress...}}
\author{Boris Andrews}
\affil{Mathematical Institute, University of Oxford}
\date{\today}


\begin{document}
    \pagenumbering{gobble}
    \maketitle
    
    
    \begin{abstract}
        Magnetic confinement reactors---in particular tokamaks---offer one of the most promising options for achieving practical nuclear fusion, with the potential to provide virtually limitless, clean energy. The theoretical and numerical modeling of tokamak plasmas is simultaneously an essential component of effective reactor design, and a great research barrier. Tokamak operational conditions exhibit comparatively low Knudsen numbers. Kinetic effects, including kinetic waves and instabilities, Landau damping, bump-on-tail instabilities and more, are therefore highly influential in tokamak plasma dynamics. Purely fluid models are inherently incapable of capturing these effects, whereas the high dimensionality in purely kinetic models render them practically intractable for most relevant purposes.

        We consider a $\delta\!f$ decomposition model, with a macroscopic fluid background and microscopic kinetic correction, both fully coupled to each other. A similar manner of discretization is proposed to that used in the recent \texttt{STRUPHY} code \cite{Holderied_Possanner_Wang_2021, Holderied_2022, Li_et_al_2023} with a finite-element model for the background and a pseudo-particle/PiC model for the correction.

        The fluid background satisfies the full, non-linear, resistive, compressible, Hall MHD equations. \cite{Laakmann_Hu_Farrell_2022} introduces finite-element(-in-space) implicit timesteppers for the incompressible analogue to this system with structure-preserving (SP) properties in the ideal case, alongside parameter-robust preconditioners. We show that these timesteppers can derive from a finite-element-in-time (FET) (and finite-element-in-space) interpretation. The benefits of this reformulation are discussed, including the derivation of timesteppers that are higher order in time, and the quantifiable dissipative SP properties in the non-ideal, resistive case.
        
        We discuss possible options for extending this FET approach to timesteppers for the compressible case.

        The kinetic corrections satisfy linearized Boltzmann equations. Using a Lénard--Bernstein collision operator, these take Fokker--Planck-like forms \cite{Fokker_1914, Planck_1917} wherein pseudo-particles in the numerical model obey the neoclassical transport equations, with particle-independent Brownian drift terms. This offers a rigorous methodology for incorporating collisions into the particle transport model, without coupling the equations of motions for each particle.
        
        Works by Chen, Chacón et al. \cite{Chen_Chacón_Barnes_2011, Chacón_Chen_Barnes_2013, Chen_Chacón_2014, Chen_Chacón_2015} have developed structure-preserving particle pushers for neoclassical transport in the Vlasov equations, derived from Crank--Nicolson integrators. We show these too can can derive from a FET interpretation, similarly offering potential extensions to higher-order-in-time particle pushers. The FET formulation is used also to consider how the stochastic drift terms can be incorporated into the pushers. Stochastic gyrokinetic expansions are also discussed.

        Different options for the numerical implementation of these schemes are considered.

        Due to the efficacy of FET in the development of SP timesteppers for both the fluid and kinetic component, we hope this approach will prove effective in the future for developing SP timesteppers for the full hybrid model. We hope this will give us the opportunity to incorporate previously inaccessible kinetic effects into the highly effective, modern, finite-element MHD models.
    \end{abstract}
    
    
    \newpage
    \tableofcontents
    
    
    \newpage
    \pagenumbering{arabic}
    %\linenumbers\renewcommand\thelinenumber{\color{black!50}\arabic{linenumber}}
            \input{0 - introduction/main.tex}
        \part{Research}
            \input{1 - low-noise PiC models/main.tex}
            \input{2 - kinetic component/main.tex}
            \input{3 - fluid component/main.tex}
            \input{4 - numerical implementation/main.tex}
        \part{Project Overview}
            \input{5 - research plan/main.tex}
            \input{6 - summary/main.tex}
    
    
    %\section{}
    \newpage
    \pagenumbering{gobble}
        \printbibliography


    \newpage
    \pagenumbering{roman}
    \appendix
        \part{Appendices}
            \input{8 - Hilbert complexes/main.tex}
            \input{9 - weak conservation proofs/main.tex}
\end{document}

            \documentclass[12pt, a4paper]{report}

\input{template/main.tex}

\title{\BA{Title in Progress...}}
\author{Boris Andrews}
\affil{Mathematical Institute, University of Oxford}
\date{\today}


\begin{document}
    \pagenumbering{gobble}
    \maketitle
    
    
    \begin{abstract}
        Magnetic confinement reactors---in particular tokamaks---offer one of the most promising options for achieving practical nuclear fusion, with the potential to provide virtually limitless, clean energy. The theoretical and numerical modeling of tokamak plasmas is simultaneously an essential component of effective reactor design, and a great research barrier. Tokamak operational conditions exhibit comparatively low Knudsen numbers. Kinetic effects, including kinetic waves and instabilities, Landau damping, bump-on-tail instabilities and more, are therefore highly influential in tokamak plasma dynamics. Purely fluid models are inherently incapable of capturing these effects, whereas the high dimensionality in purely kinetic models render them practically intractable for most relevant purposes.

        We consider a $\delta\!f$ decomposition model, with a macroscopic fluid background and microscopic kinetic correction, both fully coupled to each other. A similar manner of discretization is proposed to that used in the recent \texttt{STRUPHY} code \cite{Holderied_Possanner_Wang_2021, Holderied_2022, Li_et_al_2023} with a finite-element model for the background and a pseudo-particle/PiC model for the correction.

        The fluid background satisfies the full, non-linear, resistive, compressible, Hall MHD equations. \cite{Laakmann_Hu_Farrell_2022} introduces finite-element(-in-space) implicit timesteppers for the incompressible analogue to this system with structure-preserving (SP) properties in the ideal case, alongside parameter-robust preconditioners. We show that these timesteppers can derive from a finite-element-in-time (FET) (and finite-element-in-space) interpretation. The benefits of this reformulation are discussed, including the derivation of timesteppers that are higher order in time, and the quantifiable dissipative SP properties in the non-ideal, resistive case.
        
        We discuss possible options for extending this FET approach to timesteppers for the compressible case.

        The kinetic corrections satisfy linearized Boltzmann equations. Using a Lénard--Bernstein collision operator, these take Fokker--Planck-like forms \cite{Fokker_1914, Planck_1917} wherein pseudo-particles in the numerical model obey the neoclassical transport equations, with particle-independent Brownian drift terms. This offers a rigorous methodology for incorporating collisions into the particle transport model, without coupling the equations of motions for each particle.
        
        Works by Chen, Chacón et al. \cite{Chen_Chacón_Barnes_2011, Chacón_Chen_Barnes_2013, Chen_Chacón_2014, Chen_Chacón_2015} have developed structure-preserving particle pushers for neoclassical transport in the Vlasov equations, derived from Crank--Nicolson integrators. We show these too can can derive from a FET interpretation, similarly offering potential extensions to higher-order-in-time particle pushers. The FET formulation is used also to consider how the stochastic drift terms can be incorporated into the pushers. Stochastic gyrokinetic expansions are also discussed.

        Different options for the numerical implementation of these schemes are considered.

        Due to the efficacy of FET in the development of SP timesteppers for both the fluid and kinetic component, we hope this approach will prove effective in the future for developing SP timesteppers for the full hybrid model. We hope this will give us the opportunity to incorporate previously inaccessible kinetic effects into the highly effective, modern, finite-element MHD models.
    \end{abstract}
    
    
    \newpage
    \tableofcontents
    
    
    \newpage
    \pagenumbering{arabic}
    %\linenumbers\renewcommand\thelinenumber{\color{black!50}\arabic{linenumber}}
            \input{0 - introduction/main.tex}
        \part{Research}
            \input{1 - low-noise PiC models/main.tex}
            \input{2 - kinetic component/main.tex}
            \input{3 - fluid component/main.tex}
            \input{4 - numerical implementation/main.tex}
        \part{Project Overview}
            \input{5 - research plan/main.tex}
            \input{6 - summary/main.tex}
    
    
    %\section{}
    \newpage
    \pagenumbering{gobble}
        \printbibliography


    \newpage
    \pagenumbering{roman}
    \appendix
        \part{Appendices}
            \input{8 - Hilbert complexes/main.tex}
            \input{9 - weak conservation proofs/main.tex}
\end{document}

            \documentclass[12pt, a4paper]{report}

\input{template/main.tex}

\title{\BA{Title in Progress...}}
\author{Boris Andrews}
\affil{Mathematical Institute, University of Oxford}
\date{\today}


\begin{document}
    \pagenumbering{gobble}
    \maketitle
    
    
    \begin{abstract}
        Magnetic confinement reactors---in particular tokamaks---offer one of the most promising options for achieving practical nuclear fusion, with the potential to provide virtually limitless, clean energy. The theoretical and numerical modeling of tokamak plasmas is simultaneously an essential component of effective reactor design, and a great research barrier. Tokamak operational conditions exhibit comparatively low Knudsen numbers. Kinetic effects, including kinetic waves and instabilities, Landau damping, bump-on-tail instabilities and more, are therefore highly influential in tokamak plasma dynamics. Purely fluid models are inherently incapable of capturing these effects, whereas the high dimensionality in purely kinetic models render them practically intractable for most relevant purposes.

        We consider a $\delta\!f$ decomposition model, with a macroscopic fluid background and microscopic kinetic correction, both fully coupled to each other. A similar manner of discretization is proposed to that used in the recent \texttt{STRUPHY} code \cite{Holderied_Possanner_Wang_2021, Holderied_2022, Li_et_al_2023} with a finite-element model for the background and a pseudo-particle/PiC model for the correction.

        The fluid background satisfies the full, non-linear, resistive, compressible, Hall MHD equations. \cite{Laakmann_Hu_Farrell_2022} introduces finite-element(-in-space) implicit timesteppers for the incompressible analogue to this system with structure-preserving (SP) properties in the ideal case, alongside parameter-robust preconditioners. We show that these timesteppers can derive from a finite-element-in-time (FET) (and finite-element-in-space) interpretation. The benefits of this reformulation are discussed, including the derivation of timesteppers that are higher order in time, and the quantifiable dissipative SP properties in the non-ideal, resistive case.
        
        We discuss possible options for extending this FET approach to timesteppers for the compressible case.

        The kinetic corrections satisfy linearized Boltzmann equations. Using a Lénard--Bernstein collision operator, these take Fokker--Planck-like forms \cite{Fokker_1914, Planck_1917} wherein pseudo-particles in the numerical model obey the neoclassical transport equations, with particle-independent Brownian drift terms. This offers a rigorous methodology for incorporating collisions into the particle transport model, without coupling the equations of motions for each particle.
        
        Works by Chen, Chacón et al. \cite{Chen_Chacón_Barnes_2011, Chacón_Chen_Barnes_2013, Chen_Chacón_2014, Chen_Chacón_2015} have developed structure-preserving particle pushers for neoclassical transport in the Vlasov equations, derived from Crank--Nicolson integrators. We show these too can can derive from a FET interpretation, similarly offering potential extensions to higher-order-in-time particle pushers. The FET formulation is used also to consider how the stochastic drift terms can be incorporated into the pushers. Stochastic gyrokinetic expansions are also discussed.

        Different options for the numerical implementation of these schemes are considered.

        Due to the efficacy of FET in the development of SP timesteppers for both the fluid and kinetic component, we hope this approach will prove effective in the future for developing SP timesteppers for the full hybrid model. We hope this will give us the opportunity to incorporate previously inaccessible kinetic effects into the highly effective, modern, finite-element MHD models.
    \end{abstract}
    
    
    \newpage
    \tableofcontents
    
    
    \newpage
    \pagenumbering{arabic}
    %\linenumbers\renewcommand\thelinenumber{\color{black!50}\arabic{linenumber}}
            \input{0 - introduction/main.tex}
        \part{Research}
            \input{1 - low-noise PiC models/main.tex}
            \input{2 - kinetic component/main.tex}
            \input{3 - fluid component/main.tex}
            \input{4 - numerical implementation/main.tex}
        \part{Project Overview}
            \input{5 - research plan/main.tex}
            \input{6 - summary/main.tex}
    
    
    %\section{}
    \newpage
    \pagenumbering{gobble}
        \printbibliography


    \newpage
    \pagenumbering{roman}
    \appendix
        \part{Appendices}
            \input{8 - Hilbert complexes/main.tex}
            \input{9 - weak conservation proofs/main.tex}
\end{document}

            \documentclass[12pt, a4paper]{report}

\input{template/main.tex}

\title{\BA{Title in Progress...}}
\author{Boris Andrews}
\affil{Mathematical Institute, University of Oxford}
\date{\today}


\begin{document}
    \pagenumbering{gobble}
    \maketitle
    
    
    \begin{abstract}
        Magnetic confinement reactors---in particular tokamaks---offer one of the most promising options for achieving practical nuclear fusion, with the potential to provide virtually limitless, clean energy. The theoretical and numerical modeling of tokamak plasmas is simultaneously an essential component of effective reactor design, and a great research barrier. Tokamak operational conditions exhibit comparatively low Knudsen numbers. Kinetic effects, including kinetic waves and instabilities, Landau damping, bump-on-tail instabilities and more, are therefore highly influential in tokamak plasma dynamics. Purely fluid models are inherently incapable of capturing these effects, whereas the high dimensionality in purely kinetic models render them practically intractable for most relevant purposes.

        We consider a $\delta\!f$ decomposition model, with a macroscopic fluid background and microscopic kinetic correction, both fully coupled to each other. A similar manner of discretization is proposed to that used in the recent \texttt{STRUPHY} code \cite{Holderied_Possanner_Wang_2021, Holderied_2022, Li_et_al_2023} with a finite-element model for the background and a pseudo-particle/PiC model for the correction.

        The fluid background satisfies the full, non-linear, resistive, compressible, Hall MHD equations. \cite{Laakmann_Hu_Farrell_2022} introduces finite-element(-in-space) implicit timesteppers for the incompressible analogue to this system with structure-preserving (SP) properties in the ideal case, alongside parameter-robust preconditioners. We show that these timesteppers can derive from a finite-element-in-time (FET) (and finite-element-in-space) interpretation. The benefits of this reformulation are discussed, including the derivation of timesteppers that are higher order in time, and the quantifiable dissipative SP properties in the non-ideal, resistive case.
        
        We discuss possible options for extending this FET approach to timesteppers for the compressible case.

        The kinetic corrections satisfy linearized Boltzmann equations. Using a Lénard--Bernstein collision operator, these take Fokker--Planck-like forms \cite{Fokker_1914, Planck_1917} wherein pseudo-particles in the numerical model obey the neoclassical transport equations, with particle-independent Brownian drift terms. This offers a rigorous methodology for incorporating collisions into the particle transport model, without coupling the equations of motions for each particle.
        
        Works by Chen, Chacón et al. \cite{Chen_Chacón_Barnes_2011, Chacón_Chen_Barnes_2013, Chen_Chacón_2014, Chen_Chacón_2015} have developed structure-preserving particle pushers for neoclassical transport in the Vlasov equations, derived from Crank--Nicolson integrators. We show these too can can derive from a FET interpretation, similarly offering potential extensions to higher-order-in-time particle pushers. The FET formulation is used also to consider how the stochastic drift terms can be incorporated into the pushers. Stochastic gyrokinetic expansions are also discussed.

        Different options for the numerical implementation of these schemes are considered.

        Due to the efficacy of FET in the development of SP timesteppers for both the fluid and kinetic component, we hope this approach will prove effective in the future for developing SP timesteppers for the full hybrid model. We hope this will give us the opportunity to incorporate previously inaccessible kinetic effects into the highly effective, modern, finite-element MHD models.
    \end{abstract}
    
    
    \newpage
    \tableofcontents
    
    
    \newpage
    \pagenumbering{arabic}
    %\linenumbers\renewcommand\thelinenumber{\color{black!50}\arabic{linenumber}}
            \input{0 - introduction/main.tex}
        \part{Research}
            \input{1 - low-noise PiC models/main.tex}
            \input{2 - kinetic component/main.tex}
            \input{3 - fluid component/main.tex}
            \input{4 - numerical implementation/main.tex}
        \part{Project Overview}
            \input{5 - research plan/main.tex}
            \input{6 - summary/main.tex}
    
    
    %\section{}
    \newpage
    \pagenumbering{gobble}
        \printbibliography


    \newpage
    \pagenumbering{roman}
    \appendix
        \part{Appendices}
            \input{8 - Hilbert complexes/main.tex}
            \input{9 - weak conservation proofs/main.tex}
\end{document}

        \part{Project Overview}
            \documentclass[12pt, a4paper]{report}

\input{template/main.tex}

\title{\BA{Title in Progress...}}
\author{Boris Andrews}
\affil{Mathematical Institute, University of Oxford}
\date{\today}


\begin{document}
    \pagenumbering{gobble}
    \maketitle
    
    
    \begin{abstract}
        Magnetic confinement reactors---in particular tokamaks---offer one of the most promising options for achieving practical nuclear fusion, with the potential to provide virtually limitless, clean energy. The theoretical and numerical modeling of tokamak plasmas is simultaneously an essential component of effective reactor design, and a great research barrier. Tokamak operational conditions exhibit comparatively low Knudsen numbers. Kinetic effects, including kinetic waves and instabilities, Landau damping, bump-on-tail instabilities and more, are therefore highly influential in tokamak plasma dynamics. Purely fluid models are inherently incapable of capturing these effects, whereas the high dimensionality in purely kinetic models render them practically intractable for most relevant purposes.

        We consider a $\delta\!f$ decomposition model, with a macroscopic fluid background and microscopic kinetic correction, both fully coupled to each other. A similar manner of discretization is proposed to that used in the recent \texttt{STRUPHY} code \cite{Holderied_Possanner_Wang_2021, Holderied_2022, Li_et_al_2023} with a finite-element model for the background and a pseudo-particle/PiC model for the correction.

        The fluid background satisfies the full, non-linear, resistive, compressible, Hall MHD equations. \cite{Laakmann_Hu_Farrell_2022} introduces finite-element(-in-space) implicit timesteppers for the incompressible analogue to this system with structure-preserving (SP) properties in the ideal case, alongside parameter-robust preconditioners. We show that these timesteppers can derive from a finite-element-in-time (FET) (and finite-element-in-space) interpretation. The benefits of this reformulation are discussed, including the derivation of timesteppers that are higher order in time, and the quantifiable dissipative SP properties in the non-ideal, resistive case.
        
        We discuss possible options for extending this FET approach to timesteppers for the compressible case.

        The kinetic corrections satisfy linearized Boltzmann equations. Using a Lénard--Bernstein collision operator, these take Fokker--Planck-like forms \cite{Fokker_1914, Planck_1917} wherein pseudo-particles in the numerical model obey the neoclassical transport equations, with particle-independent Brownian drift terms. This offers a rigorous methodology for incorporating collisions into the particle transport model, without coupling the equations of motions for each particle.
        
        Works by Chen, Chacón et al. \cite{Chen_Chacón_Barnes_2011, Chacón_Chen_Barnes_2013, Chen_Chacón_2014, Chen_Chacón_2015} have developed structure-preserving particle pushers for neoclassical transport in the Vlasov equations, derived from Crank--Nicolson integrators. We show these too can can derive from a FET interpretation, similarly offering potential extensions to higher-order-in-time particle pushers. The FET formulation is used also to consider how the stochastic drift terms can be incorporated into the pushers. Stochastic gyrokinetic expansions are also discussed.

        Different options for the numerical implementation of these schemes are considered.

        Due to the efficacy of FET in the development of SP timesteppers for both the fluid and kinetic component, we hope this approach will prove effective in the future for developing SP timesteppers for the full hybrid model. We hope this will give us the opportunity to incorporate previously inaccessible kinetic effects into the highly effective, modern, finite-element MHD models.
    \end{abstract}
    
    
    \newpage
    \tableofcontents
    
    
    \newpage
    \pagenumbering{arabic}
    %\linenumbers\renewcommand\thelinenumber{\color{black!50}\arabic{linenumber}}
            \input{0 - introduction/main.tex}
        \part{Research}
            \input{1 - low-noise PiC models/main.tex}
            \input{2 - kinetic component/main.tex}
            \input{3 - fluid component/main.tex}
            \input{4 - numerical implementation/main.tex}
        \part{Project Overview}
            \input{5 - research plan/main.tex}
            \input{6 - summary/main.tex}
    
    
    %\section{}
    \newpage
    \pagenumbering{gobble}
        \printbibliography


    \newpage
    \pagenumbering{roman}
    \appendix
        \part{Appendices}
            \input{8 - Hilbert complexes/main.tex}
            \input{9 - weak conservation proofs/main.tex}
\end{document}

            \documentclass[12pt, a4paper]{report}

\input{template/main.tex}

\title{\BA{Title in Progress...}}
\author{Boris Andrews}
\affil{Mathematical Institute, University of Oxford}
\date{\today}


\begin{document}
    \pagenumbering{gobble}
    \maketitle
    
    
    \begin{abstract}
        Magnetic confinement reactors---in particular tokamaks---offer one of the most promising options for achieving practical nuclear fusion, with the potential to provide virtually limitless, clean energy. The theoretical and numerical modeling of tokamak plasmas is simultaneously an essential component of effective reactor design, and a great research barrier. Tokamak operational conditions exhibit comparatively low Knudsen numbers. Kinetic effects, including kinetic waves and instabilities, Landau damping, bump-on-tail instabilities and more, are therefore highly influential in tokamak plasma dynamics. Purely fluid models are inherently incapable of capturing these effects, whereas the high dimensionality in purely kinetic models render them practically intractable for most relevant purposes.

        We consider a $\delta\!f$ decomposition model, with a macroscopic fluid background and microscopic kinetic correction, both fully coupled to each other. A similar manner of discretization is proposed to that used in the recent \texttt{STRUPHY} code \cite{Holderied_Possanner_Wang_2021, Holderied_2022, Li_et_al_2023} with a finite-element model for the background and a pseudo-particle/PiC model for the correction.

        The fluid background satisfies the full, non-linear, resistive, compressible, Hall MHD equations. \cite{Laakmann_Hu_Farrell_2022} introduces finite-element(-in-space) implicit timesteppers for the incompressible analogue to this system with structure-preserving (SP) properties in the ideal case, alongside parameter-robust preconditioners. We show that these timesteppers can derive from a finite-element-in-time (FET) (and finite-element-in-space) interpretation. The benefits of this reformulation are discussed, including the derivation of timesteppers that are higher order in time, and the quantifiable dissipative SP properties in the non-ideal, resistive case.
        
        We discuss possible options for extending this FET approach to timesteppers for the compressible case.

        The kinetic corrections satisfy linearized Boltzmann equations. Using a Lénard--Bernstein collision operator, these take Fokker--Planck-like forms \cite{Fokker_1914, Planck_1917} wherein pseudo-particles in the numerical model obey the neoclassical transport equations, with particle-independent Brownian drift terms. This offers a rigorous methodology for incorporating collisions into the particle transport model, without coupling the equations of motions for each particle.
        
        Works by Chen, Chacón et al. \cite{Chen_Chacón_Barnes_2011, Chacón_Chen_Barnes_2013, Chen_Chacón_2014, Chen_Chacón_2015} have developed structure-preserving particle pushers for neoclassical transport in the Vlasov equations, derived from Crank--Nicolson integrators. We show these too can can derive from a FET interpretation, similarly offering potential extensions to higher-order-in-time particle pushers. The FET formulation is used also to consider how the stochastic drift terms can be incorporated into the pushers. Stochastic gyrokinetic expansions are also discussed.

        Different options for the numerical implementation of these schemes are considered.

        Due to the efficacy of FET in the development of SP timesteppers for both the fluid and kinetic component, we hope this approach will prove effective in the future for developing SP timesteppers for the full hybrid model. We hope this will give us the opportunity to incorporate previously inaccessible kinetic effects into the highly effective, modern, finite-element MHD models.
    \end{abstract}
    
    
    \newpage
    \tableofcontents
    
    
    \newpage
    \pagenumbering{arabic}
    %\linenumbers\renewcommand\thelinenumber{\color{black!50}\arabic{linenumber}}
            \input{0 - introduction/main.tex}
        \part{Research}
            \input{1 - low-noise PiC models/main.tex}
            \input{2 - kinetic component/main.tex}
            \input{3 - fluid component/main.tex}
            \input{4 - numerical implementation/main.tex}
        \part{Project Overview}
            \input{5 - research plan/main.tex}
            \input{6 - summary/main.tex}
    
    
    %\section{}
    \newpage
    \pagenumbering{gobble}
        \printbibliography


    \newpage
    \pagenumbering{roman}
    \appendix
        \part{Appendices}
            \input{8 - Hilbert complexes/main.tex}
            \input{9 - weak conservation proofs/main.tex}
\end{document}

    
    
    %\section{}
    \newpage
    \pagenumbering{gobble}
        \printbibliography


    \newpage
    \pagenumbering{roman}
    \appendix
        \part{Appendices}
            \documentclass[12pt, a4paper]{report}

\input{template/main.tex}

\title{\BA{Title in Progress...}}
\author{Boris Andrews}
\affil{Mathematical Institute, University of Oxford}
\date{\today}


\begin{document}
    \pagenumbering{gobble}
    \maketitle
    
    
    \begin{abstract}
        Magnetic confinement reactors---in particular tokamaks---offer one of the most promising options for achieving practical nuclear fusion, with the potential to provide virtually limitless, clean energy. The theoretical and numerical modeling of tokamak plasmas is simultaneously an essential component of effective reactor design, and a great research barrier. Tokamak operational conditions exhibit comparatively low Knudsen numbers. Kinetic effects, including kinetic waves and instabilities, Landau damping, bump-on-tail instabilities and more, are therefore highly influential in tokamak plasma dynamics. Purely fluid models are inherently incapable of capturing these effects, whereas the high dimensionality in purely kinetic models render them practically intractable for most relevant purposes.

        We consider a $\delta\!f$ decomposition model, with a macroscopic fluid background and microscopic kinetic correction, both fully coupled to each other. A similar manner of discretization is proposed to that used in the recent \texttt{STRUPHY} code \cite{Holderied_Possanner_Wang_2021, Holderied_2022, Li_et_al_2023} with a finite-element model for the background and a pseudo-particle/PiC model for the correction.

        The fluid background satisfies the full, non-linear, resistive, compressible, Hall MHD equations. \cite{Laakmann_Hu_Farrell_2022} introduces finite-element(-in-space) implicit timesteppers for the incompressible analogue to this system with structure-preserving (SP) properties in the ideal case, alongside parameter-robust preconditioners. We show that these timesteppers can derive from a finite-element-in-time (FET) (and finite-element-in-space) interpretation. The benefits of this reformulation are discussed, including the derivation of timesteppers that are higher order in time, and the quantifiable dissipative SP properties in the non-ideal, resistive case.
        
        We discuss possible options for extending this FET approach to timesteppers for the compressible case.

        The kinetic corrections satisfy linearized Boltzmann equations. Using a Lénard--Bernstein collision operator, these take Fokker--Planck-like forms \cite{Fokker_1914, Planck_1917} wherein pseudo-particles in the numerical model obey the neoclassical transport equations, with particle-independent Brownian drift terms. This offers a rigorous methodology for incorporating collisions into the particle transport model, without coupling the equations of motions for each particle.
        
        Works by Chen, Chacón et al. \cite{Chen_Chacón_Barnes_2011, Chacón_Chen_Barnes_2013, Chen_Chacón_2014, Chen_Chacón_2015} have developed structure-preserving particle pushers for neoclassical transport in the Vlasov equations, derived from Crank--Nicolson integrators. We show these too can can derive from a FET interpretation, similarly offering potential extensions to higher-order-in-time particle pushers. The FET formulation is used also to consider how the stochastic drift terms can be incorporated into the pushers. Stochastic gyrokinetic expansions are also discussed.

        Different options for the numerical implementation of these schemes are considered.

        Due to the efficacy of FET in the development of SP timesteppers for both the fluid and kinetic component, we hope this approach will prove effective in the future for developing SP timesteppers for the full hybrid model. We hope this will give us the opportunity to incorporate previously inaccessible kinetic effects into the highly effective, modern, finite-element MHD models.
    \end{abstract}
    
    
    \newpage
    \tableofcontents
    
    
    \newpage
    \pagenumbering{arabic}
    %\linenumbers\renewcommand\thelinenumber{\color{black!50}\arabic{linenumber}}
            \input{0 - introduction/main.tex}
        \part{Research}
            \input{1 - low-noise PiC models/main.tex}
            \input{2 - kinetic component/main.tex}
            \input{3 - fluid component/main.tex}
            \input{4 - numerical implementation/main.tex}
        \part{Project Overview}
            \input{5 - research plan/main.tex}
            \input{6 - summary/main.tex}
    
    
    %\section{}
    \newpage
    \pagenumbering{gobble}
        \printbibliography


    \newpage
    \pagenumbering{roman}
    \appendix
        \part{Appendices}
            \input{8 - Hilbert complexes/main.tex}
            \input{9 - weak conservation proofs/main.tex}
\end{document}

            \documentclass[12pt, a4paper]{report}

\input{template/main.tex}

\title{\BA{Title in Progress...}}
\author{Boris Andrews}
\affil{Mathematical Institute, University of Oxford}
\date{\today}


\begin{document}
    \pagenumbering{gobble}
    \maketitle
    
    
    \begin{abstract}
        Magnetic confinement reactors---in particular tokamaks---offer one of the most promising options for achieving practical nuclear fusion, with the potential to provide virtually limitless, clean energy. The theoretical and numerical modeling of tokamak plasmas is simultaneously an essential component of effective reactor design, and a great research barrier. Tokamak operational conditions exhibit comparatively low Knudsen numbers. Kinetic effects, including kinetic waves and instabilities, Landau damping, bump-on-tail instabilities and more, are therefore highly influential in tokamak plasma dynamics. Purely fluid models are inherently incapable of capturing these effects, whereas the high dimensionality in purely kinetic models render them practically intractable for most relevant purposes.

        We consider a $\delta\!f$ decomposition model, with a macroscopic fluid background and microscopic kinetic correction, both fully coupled to each other. A similar manner of discretization is proposed to that used in the recent \texttt{STRUPHY} code \cite{Holderied_Possanner_Wang_2021, Holderied_2022, Li_et_al_2023} with a finite-element model for the background and a pseudo-particle/PiC model for the correction.

        The fluid background satisfies the full, non-linear, resistive, compressible, Hall MHD equations. \cite{Laakmann_Hu_Farrell_2022} introduces finite-element(-in-space) implicit timesteppers for the incompressible analogue to this system with structure-preserving (SP) properties in the ideal case, alongside parameter-robust preconditioners. We show that these timesteppers can derive from a finite-element-in-time (FET) (and finite-element-in-space) interpretation. The benefits of this reformulation are discussed, including the derivation of timesteppers that are higher order in time, and the quantifiable dissipative SP properties in the non-ideal, resistive case.
        
        We discuss possible options for extending this FET approach to timesteppers for the compressible case.

        The kinetic corrections satisfy linearized Boltzmann equations. Using a Lénard--Bernstein collision operator, these take Fokker--Planck-like forms \cite{Fokker_1914, Planck_1917} wherein pseudo-particles in the numerical model obey the neoclassical transport equations, with particle-independent Brownian drift terms. This offers a rigorous methodology for incorporating collisions into the particle transport model, without coupling the equations of motions for each particle.
        
        Works by Chen, Chacón et al. \cite{Chen_Chacón_Barnes_2011, Chacón_Chen_Barnes_2013, Chen_Chacón_2014, Chen_Chacón_2015} have developed structure-preserving particle pushers for neoclassical transport in the Vlasov equations, derived from Crank--Nicolson integrators. We show these too can can derive from a FET interpretation, similarly offering potential extensions to higher-order-in-time particle pushers. The FET formulation is used also to consider how the stochastic drift terms can be incorporated into the pushers. Stochastic gyrokinetic expansions are also discussed.

        Different options for the numerical implementation of these schemes are considered.

        Due to the efficacy of FET in the development of SP timesteppers for both the fluid and kinetic component, we hope this approach will prove effective in the future for developing SP timesteppers for the full hybrid model. We hope this will give us the opportunity to incorporate previously inaccessible kinetic effects into the highly effective, modern, finite-element MHD models.
    \end{abstract}
    
    
    \newpage
    \tableofcontents
    
    
    \newpage
    \pagenumbering{arabic}
    %\linenumbers\renewcommand\thelinenumber{\color{black!50}\arabic{linenumber}}
            \input{0 - introduction/main.tex}
        \part{Research}
            \input{1 - low-noise PiC models/main.tex}
            \input{2 - kinetic component/main.tex}
            \input{3 - fluid component/main.tex}
            \input{4 - numerical implementation/main.tex}
        \part{Project Overview}
            \input{5 - research plan/main.tex}
            \input{6 - summary/main.tex}
    
    
    %\section{}
    \newpage
    \pagenumbering{gobble}
        \printbibliography


    \newpage
    \pagenumbering{roman}
    \appendix
        \part{Appendices}
            \input{8 - Hilbert complexes/main.tex}
            \input{9 - weak conservation proofs/main.tex}
\end{document}

\end{document}

\end{document}

    \documentclass[12pt, a4paper]{report}

\documentclass[12pt, a4paper]{report}

\documentclass[12pt, a4paper]{report}

\input{template/main.tex}

\title{\BA{Title in Progress...}}
\author{Boris Andrews}
\affil{Mathematical Institute, University of Oxford}
\date{\today}


\begin{document}
    \pagenumbering{gobble}
    \maketitle
    
    
    \begin{abstract}
        Magnetic confinement reactors---in particular tokamaks---offer one of the most promising options for achieving practical nuclear fusion, with the potential to provide virtually limitless, clean energy. The theoretical and numerical modeling of tokamak plasmas is simultaneously an essential component of effective reactor design, and a great research barrier. Tokamak operational conditions exhibit comparatively low Knudsen numbers. Kinetic effects, including kinetic waves and instabilities, Landau damping, bump-on-tail instabilities and more, are therefore highly influential in tokamak plasma dynamics. Purely fluid models are inherently incapable of capturing these effects, whereas the high dimensionality in purely kinetic models render them practically intractable for most relevant purposes.

        We consider a $\delta\!f$ decomposition model, with a macroscopic fluid background and microscopic kinetic correction, both fully coupled to each other. A similar manner of discretization is proposed to that used in the recent \texttt{STRUPHY} code \cite{Holderied_Possanner_Wang_2021, Holderied_2022, Li_et_al_2023} with a finite-element model for the background and a pseudo-particle/PiC model for the correction.

        The fluid background satisfies the full, non-linear, resistive, compressible, Hall MHD equations. \cite{Laakmann_Hu_Farrell_2022} introduces finite-element(-in-space) implicit timesteppers for the incompressible analogue to this system with structure-preserving (SP) properties in the ideal case, alongside parameter-robust preconditioners. We show that these timesteppers can derive from a finite-element-in-time (FET) (and finite-element-in-space) interpretation. The benefits of this reformulation are discussed, including the derivation of timesteppers that are higher order in time, and the quantifiable dissipative SP properties in the non-ideal, resistive case.
        
        We discuss possible options for extending this FET approach to timesteppers for the compressible case.

        The kinetic corrections satisfy linearized Boltzmann equations. Using a Lénard--Bernstein collision operator, these take Fokker--Planck-like forms \cite{Fokker_1914, Planck_1917} wherein pseudo-particles in the numerical model obey the neoclassical transport equations, with particle-independent Brownian drift terms. This offers a rigorous methodology for incorporating collisions into the particle transport model, without coupling the equations of motions for each particle.
        
        Works by Chen, Chacón et al. \cite{Chen_Chacón_Barnes_2011, Chacón_Chen_Barnes_2013, Chen_Chacón_2014, Chen_Chacón_2015} have developed structure-preserving particle pushers for neoclassical transport in the Vlasov equations, derived from Crank--Nicolson integrators. We show these too can can derive from a FET interpretation, similarly offering potential extensions to higher-order-in-time particle pushers. The FET formulation is used also to consider how the stochastic drift terms can be incorporated into the pushers. Stochastic gyrokinetic expansions are also discussed.

        Different options for the numerical implementation of these schemes are considered.

        Due to the efficacy of FET in the development of SP timesteppers for both the fluid and kinetic component, we hope this approach will prove effective in the future for developing SP timesteppers for the full hybrid model. We hope this will give us the opportunity to incorporate previously inaccessible kinetic effects into the highly effective, modern, finite-element MHD models.
    \end{abstract}
    
    
    \newpage
    \tableofcontents
    
    
    \newpage
    \pagenumbering{arabic}
    %\linenumbers\renewcommand\thelinenumber{\color{black!50}\arabic{linenumber}}
            \input{0 - introduction/main.tex}
        \part{Research}
            \input{1 - low-noise PiC models/main.tex}
            \input{2 - kinetic component/main.tex}
            \input{3 - fluid component/main.tex}
            \input{4 - numerical implementation/main.tex}
        \part{Project Overview}
            \input{5 - research plan/main.tex}
            \input{6 - summary/main.tex}
    
    
    %\section{}
    \newpage
    \pagenumbering{gobble}
        \printbibliography


    \newpage
    \pagenumbering{roman}
    \appendix
        \part{Appendices}
            \input{8 - Hilbert complexes/main.tex}
            \input{9 - weak conservation proofs/main.tex}
\end{document}


\title{\BA{Title in Progress...}}
\author{Boris Andrews}
\affil{Mathematical Institute, University of Oxford}
\date{\today}


\begin{document}
    \pagenumbering{gobble}
    \maketitle
    
    
    \begin{abstract}
        Magnetic confinement reactors---in particular tokamaks---offer one of the most promising options for achieving practical nuclear fusion, with the potential to provide virtually limitless, clean energy. The theoretical and numerical modeling of tokamak plasmas is simultaneously an essential component of effective reactor design, and a great research barrier. Tokamak operational conditions exhibit comparatively low Knudsen numbers. Kinetic effects, including kinetic waves and instabilities, Landau damping, bump-on-tail instabilities and more, are therefore highly influential in tokamak plasma dynamics. Purely fluid models are inherently incapable of capturing these effects, whereas the high dimensionality in purely kinetic models render them practically intractable for most relevant purposes.

        We consider a $\delta\!f$ decomposition model, with a macroscopic fluid background and microscopic kinetic correction, both fully coupled to each other. A similar manner of discretization is proposed to that used in the recent \texttt{STRUPHY} code \cite{Holderied_Possanner_Wang_2021, Holderied_2022, Li_et_al_2023} with a finite-element model for the background and a pseudo-particle/PiC model for the correction.

        The fluid background satisfies the full, non-linear, resistive, compressible, Hall MHD equations. \cite{Laakmann_Hu_Farrell_2022} introduces finite-element(-in-space) implicit timesteppers for the incompressible analogue to this system with structure-preserving (SP) properties in the ideal case, alongside parameter-robust preconditioners. We show that these timesteppers can derive from a finite-element-in-time (FET) (and finite-element-in-space) interpretation. The benefits of this reformulation are discussed, including the derivation of timesteppers that are higher order in time, and the quantifiable dissipative SP properties in the non-ideal, resistive case.
        
        We discuss possible options for extending this FET approach to timesteppers for the compressible case.

        The kinetic corrections satisfy linearized Boltzmann equations. Using a Lénard--Bernstein collision operator, these take Fokker--Planck-like forms \cite{Fokker_1914, Planck_1917} wherein pseudo-particles in the numerical model obey the neoclassical transport equations, with particle-independent Brownian drift terms. This offers a rigorous methodology for incorporating collisions into the particle transport model, without coupling the equations of motions for each particle.
        
        Works by Chen, Chacón et al. \cite{Chen_Chacón_Barnes_2011, Chacón_Chen_Barnes_2013, Chen_Chacón_2014, Chen_Chacón_2015} have developed structure-preserving particle pushers for neoclassical transport in the Vlasov equations, derived from Crank--Nicolson integrators. We show these too can can derive from a FET interpretation, similarly offering potential extensions to higher-order-in-time particle pushers. The FET formulation is used also to consider how the stochastic drift terms can be incorporated into the pushers. Stochastic gyrokinetic expansions are also discussed.

        Different options for the numerical implementation of these schemes are considered.

        Due to the efficacy of FET in the development of SP timesteppers for both the fluid and kinetic component, we hope this approach will prove effective in the future for developing SP timesteppers for the full hybrid model. We hope this will give us the opportunity to incorporate previously inaccessible kinetic effects into the highly effective, modern, finite-element MHD models.
    \end{abstract}
    
    
    \newpage
    \tableofcontents
    
    
    \newpage
    \pagenumbering{arabic}
    %\linenumbers\renewcommand\thelinenumber{\color{black!50}\arabic{linenumber}}
            \documentclass[12pt, a4paper]{report}

\input{template/main.tex}

\title{\BA{Title in Progress...}}
\author{Boris Andrews}
\affil{Mathematical Institute, University of Oxford}
\date{\today}


\begin{document}
    \pagenumbering{gobble}
    \maketitle
    
    
    \begin{abstract}
        Magnetic confinement reactors---in particular tokamaks---offer one of the most promising options for achieving practical nuclear fusion, with the potential to provide virtually limitless, clean energy. The theoretical and numerical modeling of tokamak plasmas is simultaneously an essential component of effective reactor design, and a great research barrier. Tokamak operational conditions exhibit comparatively low Knudsen numbers. Kinetic effects, including kinetic waves and instabilities, Landau damping, bump-on-tail instabilities and more, are therefore highly influential in tokamak plasma dynamics. Purely fluid models are inherently incapable of capturing these effects, whereas the high dimensionality in purely kinetic models render them practically intractable for most relevant purposes.

        We consider a $\delta\!f$ decomposition model, with a macroscopic fluid background and microscopic kinetic correction, both fully coupled to each other. A similar manner of discretization is proposed to that used in the recent \texttt{STRUPHY} code \cite{Holderied_Possanner_Wang_2021, Holderied_2022, Li_et_al_2023} with a finite-element model for the background and a pseudo-particle/PiC model for the correction.

        The fluid background satisfies the full, non-linear, resistive, compressible, Hall MHD equations. \cite{Laakmann_Hu_Farrell_2022} introduces finite-element(-in-space) implicit timesteppers for the incompressible analogue to this system with structure-preserving (SP) properties in the ideal case, alongside parameter-robust preconditioners. We show that these timesteppers can derive from a finite-element-in-time (FET) (and finite-element-in-space) interpretation. The benefits of this reformulation are discussed, including the derivation of timesteppers that are higher order in time, and the quantifiable dissipative SP properties in the non-ideal, resistive case.
        
        We discuss possible options for extending this FET approach to timesteppers for the compressible case.

        The kinetic corrections satisfy linearized Boltzmann equations. Using a Lénard--Bernstein collision operator, these take Fokker--Planck-like forms \cite{Fokker_1914, Planck_1917} wherein pseudo-particles in the numerical model obey the neoclassical transport equations, with particle-independent Brownian drift terms. This offers a rigorous methodology for incorporating collisions into the particle transport model, without coupling the equations of motions for each particle.
        
        Works by Chen, Chacón et al. \cite{Chen_Chacón_Barnes_2011, Chacón_Chen_Barnes_2013, Chen_Chacón_2014, Chen_Chacón_2015} have developed structure-preserving particle pushers for neoclassical transport in the Vlasov equations, derived from Crank--Nicolson integrators. We show these too can can derive from a FET interpretation, similarly offering potential extensions to higher-order-in-time particle pushers. The FET formulation is used also to consider how the stochastic drift terms can be incorporated into the pushers. Stochastic gyrokinetic expansions are also discussed.

        Different options for the numerical implementation of these schemes are considered.

        Due to the efficacy of FET in the development of SP timesteppers for both the fluid and kinetic component, we hope this approach will prove effective in the future for developing SP timesteppers for the full hybrid model. We hope this will give us the opportunity to incorporate previously inaccessible kinetic effects into the highly effective, modern, finite-element MHD models.
    \end{abstract}
    
    
    \newpage
    \tableofcontents
    
    
    \newpage
    \pagenumbering{arabic}
    %\linenumbers\renewcommand\thelinenumber{\color{black!50}\arabic{linenumber}}
            \input{0 - introduction/main.tex}
        \part{Research}
            \input{1 - low-noise PiC models/main.tex}
            \input{2 - kinetic component/main.tex}
            \input{3 - fluid component/main.tex}
            \input{4 - numerical implementation/main.tex}
        \part{Project Overview}
            \input{5 - research plan/main.tex}
            \input{6 - summary/main.tex}
    
    
    %\section{}
    \newpage
    \pagenumbering{gobble}
        \printbibliography


    \newpage
    \pagenumbering{roman}
    \appendix
        \part{Appendices}
            \input{8 - Hilbert complexes/main.tex}
            \input{9 - weak conservation proofs/main.tex}
\end{document}

        \part{Research}
            \documentclass[12pt, a4paper]{report}

\input{template/main.tex}

\title{\BA{Title in Progress...}}
\author{Boris Andrews}
\affil{Mathematical Institute, University of Oxford}
\date{\today}


\begin{document}
    \pagenumbering{gobble}
    \maketitle
    
    
    \begin{abstract}
        Magnetic confinement reactors---in particular tokamaks---offer one of the most promising options for achieving practical nuclear fusion, with the potential to provide virtually limitless, clean energy. The theoretical and numerical modeling of tokamak plasmas is simultaneously an essential component of effective reactor design, and a great research barrier. Tokamak operational conditions exhibit comparatively low Knudsen numbers. Kinetic effects, including kinetic waves and instabilities, Landau damping, bump-on-tail instabilities and more, are therefore highly influential in tokamak plasma dynamics. Purely fluid models are inherently incapable of capturing these effects, whereas the high dimensionality in purely kinetic models render them practically intractable for most relevant purposes.

        We consider a $\delta\!f$ decomposition model, with a macroscopic fluid background and microscopic kinetic correction, both fully coupled to each other. A similar manner of discretization is proposed to that used in the recent \texttt{STRUPHY} code \cite{Holderied_Possanner_Wang_2021, Holderied_2022, Li_et_al_2023} with a finite-element model for the background and a pseudo-particle/PiC model for the correction.

        The fluid background satisfies the full, non-linear, resistive, compressible, Hall MHD equations. \cite{Laakmann_Hu_Farrell_2022} introduces finite-element(-in-space) implicit timesteppers for the incompressible analogue to this system with structure-preserving (SP) properties in the ideal case, alongside parameter-robust preconditioners. We show that these timesteppers can derive from a finite-element-in-time (FET) (and finite-element-in-space) interpretation. The benefits of this reformulation are discussed, including the derivation of timesteppers that are higher order in time, and the quantifiable dissipative SP properties in the non-ideal, resistive case.
        
        We discuss possible options for extending this FET approach to timesteppers for the compressible case.

        The kinetic corrections satisfy linearized Boltzmann equations. Using a Lénard--Bernstein collision operator, these take Fokker--Planck-like forms \cite{Fokker_1914, Planck_1917} wherein pseudo-particles in the numerical model obey the neoclassical transport equations, with particle-independent Brownian drift terms. This offers a rigorous methodology for incorporating collisions into the particle transport model, without coupling the equations of motions for each particle.
        
        Works by Chen, Chacón et al. \cite{Chen_Chacón_Barnes_2011, Chacón_Chen_Barnes_2013, Chen_Chacón_2014, Chen_Chacón_2015} have developed structure-preserving particle pushers for neoclassical transport in the Vlasov equations, derived from Crank--Nicolson integrators. We show these too can can derive from a FET interpretation, similarly offering potential extensions to higher-order-in-time particle pushers. The FET formulation is used also to consider how the stochastic drift terms can be incorporated into the pushers. Stochastic gyrokinetic expansions are also discussed.

        Different options for the numerical implementation of these schemes are considered.

        Due to the efficacy of FET in the development of SP timesteppers for both the fluid and kinetic component, we hope this approach will prove effective in the future for developing SP timesteppers for the full hybrid model. We hope this will give us the opportunity to incorporate previously inaccessible kinetic effects into the highly effective, modern, finite-element MHD models.
    \end{abstract}
    
    
    \newpage
    \tableofcontents
    
    
    \newpage
    \pagenumbering{arabic}
    %\linenumbers\renewcommand\thelinenumber{\color{black!50}\arabic{linenumber}}
            \input{0 - introduction/main.tex}
        \part{Research}
            \input{1 - low-noise PiC models/main.tex}
            \input{2 - kinetic component/main.tex}
            \input{3 - fluid component/main.tex}
            \input{4 - numerical implementation/main.tex}
        \part{Project Overview}
            \input{5 - research plan/main.tex}
            \input{6 - summary/main.tex}
    
    
    %\section{}
    \newpage
    \pagenumbering{gobble}
        \printbibliography


    \newpage
    \pagenumbering{roman}
    \appendix
        \part{Appendices}
            \input{8 - Hilbert complexes/main.tex}
            \input{9 - weak conservation proofs/main.tex}
\end{document}

            \documentclass[12pt, a4paper]{report}

\input{template/main.tex}

\title{\BA{Title in Progress...}}
\author{Boris Andrews}
\affil{Mathematical Institute, University of Oxford}
\date{\today}


\begin{document}
    \pagenumbering{gobble}
    \maketitle
    
    
    \begin{abstract}
        Magnetic confinement reactors---in particular tokamaks---offer one of the most promising options for achieving practical nuclear fusion, with the potential to provide virtually limitless, clean energy. The theoretical and numerical modeling of tokamak plasmas is simultaneously an essential component of effective reactor design, and a great research barrier. Tokamak operational conditions exhibit comparatively low Knudsen numbers. Kinetic effects, including kinetic waves and instabilities, Landau damping, bump-on-tail instabilities and more, are therefore highly influential in tokamak plasma dynamics. Purely fluid models are inherently incapable of capturing these effects, whereas the high dimensionality in purely kinetic models render them practically intractable for most relevant purposes.

        We consider a $\delta\!f$ decomposition model, with a macroscopic fluid background and microscopic kinetic correction, both fully coupled to each other. A similar manner of discretization is proposed to that used in the recent \texttt{STRUPHY} code \cite{Holderied_Possanner_Wang_2021, Holderied_2022, Li_et_al_2023} with a finite-element model for the background and a pseudo-particle/PiC model for the correction.

        The fluid background satisfies the full, non-linear, resistive, compressible, Hall MHD equations. \cite{Laakmann_Hu_Farrell_2022} introduces finite-element(-in-space) implicit timesteppers for the incompressible analogue to this system with structure-preserving (SP) properties in the ideal case, alongside parameter-robust preconditioners. We show that these timesteppers can derive from a finite-element-in-time (FET) (and finite-element-in-space) interpretation. The benefits of this reformulation are discussed, including the derivation of timesteppers that are higher order in time, and the quantifiable dissipative SP properties in the non-ideal, resistive case.
        
        We discuss possible options for extending this FET approach to timesteppers for the compressible case.

        The kinetic corrections satisfy linearized Boltzmann equations. Using a Lénard--Bernstein collision operator, these take Fokker--Planck-like forms \cite{Fokker_1914, Planck_1917} wherein pseudo-particles in the numerical model obey the neoclassical transport equations, with particle-independent Brownian drift terms. This offers a rigorous methodology for incorporating collisions into the particle transport model, without coupling the equations of motions for each particle.
        
        Works by Chen, Chacón et al. \cite{Chen_Chacón_Barnes_2011, Chacón_Chen_Barnes_2013, Chen_Chacón_2014, Chen_Chacón_2015} have developed structure-preserving particle pushers for neoclassical transport in the Vlasov equations, derived from Crank--Nicolson integrators. We show these too can can derive from a FET interpretation, similarly offering potential extensions to higher-order-in-time particle pushers. The FET formulation is used also to consider how the stochastic drift terms can be incorporated into the pushers. Stochastic gyrokinetic expansions are also discussed.

        Different options for the numerical implementation of these schemes are considered.

        Due to the efficacy of FET in the development of SP timesteppers for both the fluid and kinetic component, we hope this approach will prove effective in the future for developing SP timesteppers for the full hybrid model. We hope this will give us the opportunity to incorporate previously inaccessible kinetic effects into the highly effective, modern, finite-element MHD models.
    \end{abstract}
    
    
    \newpage
    \tableofcontents
    
    
    \newpage
    \pagenumbering{arabic}
    %\linenumbers\renewcommand\thelinenumber{\color{black!50}\arabic{linenumber}}
            \input{0 - introduction/main.tex}
        \part{Research}
            \input{1 - low-noise PiC models/main.tex}
            \input{2 - kinetic component/main.tex}
            \input{3 - fluid component/main.tex}
            \input{4 - numerical implementation/main.tex}
        \part{Project Overview}
            \input{5 - research plan/main.tex}
            \input{6 - summary/main.tex}
    
    
    %\section{}
    \newpage
    \pagenumbering{gobble}
        \printbibliography


    \newpage
    \pagenumbering{roman}
    \appendix
        \part{Appendices}
            \input{8 - Hilbert complexes/main.tex}
            \input{9 - weak conservation proofs/main.tex}
\end{document}

            \documentclass[12pt, a4paper]{report}

\input{template/main.tex}

\title{\BA{Title in Progress...}}
\author{Boris Andrews}
\affil{Mathematical Institute, University of Oxford}
\date{\today}


\begin{document}
    \pagenumbering{gobble}
    \maketitle
    
    
    \begin{abstract}
        Magnetic confinement reactors---in particular tokamaks---offer one of the most promising options for achieving practical nuclear fusion, with the potential to provide virtually limitless, clean energy. The theoretical and numerical modeling of tokamak plasmas is simultaneously an essential component of effective reactor design, and a great research barrier. Tokamak operational conditions exhibit comparatively low Knudsen numbers. Kinetic effects, including kinetic waves and instabilities, Landau damping, bump-on-tail instabilities and more, are therefore highly influential in tokamak plasma dynamics. Purely fluid models are inherently incapable of capturing these effects, whereas the high dimensionality in purely kinetic models render them practically intractable for most relevant purposes.

        We consider a $\delta\!f$ decomposition model, with a macroscopic fluid background and microscopic kinetic correction, both fully coupled to each other. A similar manner of discretization is proposed to that used in the recent \texttt{STRUPHY} code \cite{Holderied_Possanner_Wang_2021, Holderied_2022, Li_et_al_2023} with a finite-element model for the background and a pseudo-particle/PiC model for the correction.

        The fluid background satisfies the full, non-linear, resistive, compressible, Hall MHD equations. \cite{Laakmann_Hu_Farrell_2022} introduces finite-element(-in-space) implicit timesteppers for the incompressible analogue to this system with structure-preserving (SP) properties in the ideal case, alongside parameter-robust preconditioners. We show that these timesteppers can derive from a finite-element-in-time (FET) (and finite-element-in-space) interpretation. The benefits of this reformulation are discussed, including the derivation of timesteppers that are higher order in time, and the quantifiable dissipative SP properties in the non-ideal, resistive case.
        
        We discuss possible options for extending this FET approach to timesteppers for the compressible case.

        The kinetic corrections satisfy linearized Boltzmann equations. Using a Lénard--Bernstein collision operator, these take Fokker--Planck-like forms \cite{Fokker_1914, Planck_1917} wherein pseudo-particles in the numerical model obey the neoclassical transport equations, with particle-independent Brownian drift terms. This offers a rigorous methodology for incorporating collisions into the particle transport model, without coupling the equations of motions for each particle.
        
        Works by Chen, Chacón et al. \cite{Chen_Chacón_Barnes_2011, Chacón_Chen_Barnes_2013, Chen_Chacón_2014, Chen_Chacón_2015} have developed structure-preserving particle pushers for neoclassical transport in the Vlasov equations, derived from Crank--Nicolson integrators. We show these too can can derive from a FET interpretation, similarly offering potential extensions to higher-order-in-time particle pushers. The FET formulation is used also to consider how the stochastic drift terms can be incorporated into the pushers. Stochastic gyrokinetic expansions are also discussed.

        Different options for the numerical implementation of these schemes are considered.

        Due to the efficacy of FET in the development of SP timesteppers for both the fluid and kinetic component, we hope this approach will prove effective in the future for developing SP timesteppers for the full hybrid model. We hope this will give us the opportunity to incorporate previously inaccessible kinetic effects into the highly effective, modern, finite-element MHD models.
    \end{abstract}
    
    
    \newpage
    \tableofcontents
    
    
    \newpage
    \pagenumbering{arabic}
    %\linenumbers\renewcommand\thelinenumber{\color{black!50}\arabic{linenumber}}
            \input{0 - introduction/main.tex}
        \part{Research}
            \input{1 - low-noise PiC models/main.tex}
            \input{2 - kinetic component/main.tex}
            \input{3 - fluid component/main.tex}
            \input{4 - numerical implementation/main.tex}
        \part{Project Overview}
            \input{5 - research plan/main.tex}
            \input{6 - summary/main.tex}
    
    
    %\section{}
    \newpage
    \pagenumbering{gobble}
        \printbibliography


    \newpage
    \pagenumbering{roman}
    \appendix
        \part{Appendices}
            \input{8 - Hilbert complexes/main.tex}
            \input{9 - weak conservation proofs/main.tex}
\end{document}

            \documentclass[12pt, a4paper]{report}

\input{template/main.tex}

\title{\BA{Title in Progress...}}
\author{Boris Andrews}
\affil{Mathematical Institute, University of Oxford}
\date{\today}


\begin{document}
    \pagenumbering{gobble}
    \maketitle
    
    
    \begin{abstract}
        Magnetic confinement reactors---in particular tokamaks---offer one of the most promising options for achieving practical nuclear fusion, with the potential to provide virtually limitless, clean energy. The theoretical and numerical modeling of tokamak plasmas is simultaneously an essential component of effective reactor design, and a great research barrier. Tokamak operational conditions exhibit comparatively low Knudsen numbers. Kinetic effects, including kinetic waves and instabilities, Landau damping, bump-on-tail instabilities and more, are therefore highly influential in tokamak plasma dynamics. Purely fluid models are inherently incapable of capturing these effects, whereas the high dimensionality in purely kinetic models render them practically intractable for most relevant purposes.

        We consider a $\delta\!f$ decomposition model, with a macroscopic fluid background and microscopic kinetic correction, both fully coupled to each other. A similar manner of discretization is proposed to that used in the recent \texttt{STRUPHY} code \cite{Holderied_Possanner_Wang_2021, Holderied_2022, Li_et_al_2023} with a finite-element model for the background and a pseudo-particle/PiC model for the correction.

        The fluid background satisfies the full, non-linear, resistive, compressible, Hall MHD equations. \cite{Laakmann_Hu_Farrell_2022} introduces finite-element(-in-space) implicit timesteppers for the incompressible analogue to this system with structure-preserving (SP) properties in the ideal case, alongside parameter-robust preconditioners. We show that these timesteppers can derive from a finite-element-in-time (FET) (and finite-element-in-space) interpretation. The benefits of this reformulation are discussed, including the derivation of timesteppers that are higher order in time, and the quantifiable dissipative SP properties in the non-ideal, resistive case.
        
        We discuss possible options for extending this FET approach to timesteppers for the compressible case.

        The kinetic corrections satisfy linearized Boltzmann equations. Using a Lénard--Bernstein collision operator, these take Fokker--Planck-like forms \cite{Fokker_1914, Planck_1917} wherein pseudo-particles in the numerical model obey the neoclassical transport equations, with particle-independent Brownian drift terms. This offers a rigorous methodology for incorporating collisions into the particle transport model, without coupling the equations of motions for each particle.
        
        Works by Chen, Chacón et al. \cite{Chen_Chacón_Barnes_2011, Chacón_Chen_Barnes_2013, Chen_Chacón_2014, Chen_Chacón_2015} have developed structure-preserving particle pushers for neoclassical transport in the Vlasov equations, derived from Crank--Nicolson integrators. We show these too can can derive from a FET interpretation, similarly offering potential extensions to higher-order-in-time particle pushers. The FET formulation is used also to consider how the stochastic drift terms can be incorporated into the pushers. Stochastic gyrokinetic expansions are also discussed.

        Different options for the numerical implementation of these schemes are considered.

        Due to the efficacy of FET in the development of SP timesteppers for both the fluid and kinetic component, we hope this approach will prove effective in the future for developing SP timesteppers for the full hybrid model. We hope this will give us the opportunity to incorporate previously inaccessible kinetic effects into the highly effective, modern, finite-element MHD models.
    \end{abstract}
    
    
    \newpage
    \tableofcontents
    
    
    \newpage
    \pagenumbering{arabic}
    %\linenumbers\renewcommand\thelinenumber{\color{black!50}\arabic{linenumber}}
            \input{0 - introduction/main.tex}
        \part{Research}
            \input{1 - low-noise PiC models/main.tex}
            \input{2 - kinetic component/main.tex}
            \input{3 - fluid component/main.tex}
            \input{4 - numerical implementation/main.tex}
        \part{Project Overview}
            \input{5 - research plan/main.tex}
            \input{6 - summary/main.tex}
    
    
    %\section{}
    \newpage
    \pagenumbering{gobble}
        \printbibliography


    \newpage
    \pagenumbering{roman}
    \appendix
        \part{Appendices}
            \input{8 - Hilbert complexes/main.tex}
            \input{9 - weak conservation proofs/main.tex}
\end{document}

        \part{Project Overview}
            \documentclass[12pt, a4paper]{report}

\input{template/main.tex}

\title{\BA{Title in Progress...}}
\author{Boris Andrews}
\affil{Mathematical Institute, University of Oxford}
\date{\today}


\begin{document}
    \pagenumbering{gobble}
    \maketitle
    
    
    \begin{abstract}
        Magnetic confinement reactors---in particular tokamaks---offer one of the most promising options for achieving practical nuclear fusion, with the potential to provide virtually limitless, clean energy. The theoretical and numerical modeling of tokamak plasmas is simultaneously an essential component of effective reactor design, and a great research barrier. Tokamak operational conditions exhibit comparatively low Knudsen numbers. Kinetic effects, including kinetic waves and instabilities, Landau damping, bump-on-tail instabilities and more, are therefore highly influential in tokamak plasma dynamics. Purely fluid models are inherently incapable of capturing these effects, whereas the high dimensionality in purely kinetic models render them practically intractable for most relevant purposes.

        We consider a $\delta\!f$ decomposition model, with a macroscopic fluid background and microscopic kinetic correction, both fully coupled to each other. A similar manner of discretization is proposed to that used in the recent \texttt{STRUPHY} code \cite{Holderied_Possanner_Wang_2021, Holderied_2022, Li_et_al_2023} with a finite-element model for the background and a pseudo-particle/PiC model for the correction.

        The fluid background satisfies the full, non-linear, resistive, compressible, Hall MHD equations. \cite{Laakmann_Hu_Farrell_2022} introduces finite-element(-in-space) implicit timesteppers for the incompressible analogue to this system with structure-preserving (SP) properties in the ideal case, alongside parameter-robust preconditioners. We show that these timesteppers can derive from a finite-element-in-time (FET) (and finite-element-in-space) interpretation. The benefits of this reformulation are discussed, including the derivation of timesteppers that are higher order in time, and the quantifiable dissipative SP properties in the non-ideal, resistive case.
        
        We discuss possible options for extending this FET approach to timesteppers for the compressible case.

        The kinetic corrections satisfy linearized Boltzmann equations. Using a Lénard--Bernstein collision operator, these take Fokker--Planck-like forms \cite{Fokker_1914, Planck_1917} wherein pseudo-particles in the numerical model obey the neoclassical transport equations, with particle-independent Brownian drift terms. This offers a rigorous methodology for incorporating collisions into the particle transport model, without coupling the equations of motions for each particle.
        
        Works by Chen, Chacón et al. \cite{Chen_Chacón_Barnes_2011, Chacón_Chen_Barnes_2013, Chen_Chacón_2014, Chen_Chacón_2015} have developed structure-preserving particle pushers for neoclassical transport in the Vlasov equations, derived from Crank--Nicolson integrators. We show these too can can derive from a FET interpretation, similarly offering potential extensions to higher-order-in-time particle pushers. The FET formulation is used also to consider how the stochastic drift terms can be incorporated into the pushers. Stochastic gyrokinetic expansions are also discussed.

        Different options for the numerical implementation of these schemes are considered.

        Due to the efficacy of FET in the development of SP timesteppers for both the fluid and kinetic component, we hope this approach will prove effective in the future for developing SP timesteppers for the full hybrid model. We hope this will give us the opportunity to incorporate previously inaccessible kinetic effects into the highly effective, modern, finite-element MHD models.
    \end{abstract}
    
    
    \newpage
    \tableofcontents
    
    
    \newpage
    \pagenumbering{arabic}
    %\linenumbers\renewcommand\thelinenumber{\color{black!50}\arabic{linenumber}}
            \input{0 - introduction/main.tex}
        \part{Research}
            \input{1 - low-noise PiC models/main.tex}
            \input{2 - kinetic component/main.tex}
            \input{3 - fluid component/main.tex}
            \input{4 - numerical implementation/main.tex}
        \part{Project Overview}
            \input{5 - research plan/main.tex}
            \input{6 - summary/main.tex}
    
    
    %\section{}
    \newpage
    \pagenumbering{gobble}
        \printbibliography


    \newpage
    \pagenumbering{roman}
    \appendix
        \part{Appendices}
            \input{8 - Hilbert complexes/main.tex}
            \input{9 - weak conservation proofs/main.tex}
\end{document}

            \documentclass[12pt, a4paper]{report}

\input{template/main.tex}

\title{\BA{Title in Progress...}}
\author{Boris Andrews}
\affil{Mathematical Institute, University of Oxford}
\date{\today}


\begin{document}
    \pagenumbering{gobble}
    \maketitle
    
    
    \begin{abstract}
        Magnetic confinement reactors---in particular tokamaks---offer one of the most promising options for achieving practical nuclear fusion, with the potential to provide virtually limitless, clean energy. The theoretical and numerical modeling of tokamak plasmas is simultaneously an essential component of effective reactor design, and a great research barrier. Tokamak operational conditions exhibit comparatively low Knudsen numbers. Kinetic effects, including kinetic waves and instabilities, Landau damping, bump-on-tail instabilities and more, are therefore highly influential in tokamak plasma dynamics. Purely fluid models are inherently incapable of capturing these effects, whereas the high dimensionality in purely kinetic models render them practically intractable for most relevant purposes.

        We consider a $\delta\!f$ decomposition model, with a macroscopic fluid background and microscopic kinetic correction, both fully coupled to each other. A similar manner of discretization is proposed to that used in the recent \texttt{STRUPHY} code \cite{Holderied_Possanner_Wang_2021, Holderied_2022, Li_et_al_2023} with a finite-element model for the background and a pseudo-particle/PiC model for the correction.

        The fluid background satisfies the full, non-linear, resistive, compressible, Hall MHD equations. \cite{Laakmann_Hu_Farrell_2022} introduces finite-element(-in-space) implicit timesteppers for the incompressible analogue to this system with structure-preserving (SP) properties in the ideal case, alongside parameter-robust preconditioners. We show that these timesteppers can derive from a finite-element-in-time (FET) (and finite-element-in-space) interpretation. The benefits of this reformulation are discussed, including the derivation of timesteppers that are higher order in time, and the quantifiable dissipative SP properties in the non-ideal, resistive case.
        
        We discuss possible options for extending this FET approach to timesteppers for the compressible case.

        The kinetic corrections satisfy linearized Boltzmann equations. Using a Lénard--Bernstein collision operator, these take Fokker--Planck-like forms \cite{Fokker_1914, Planck_1917} wherein pseudo-particles in the numerical model obey the neoclassical transport equations, with particle-independent Brownian drift terms. This offers a rigorous methodology for incorporating collisions into the particle transport model, without coupling the equations of motions for each particle.
        
        Works by Chen, Chacón et al. \cite{Chen_Chacón_Barnes_2011, Chacón_Chen_Barnes_2013, Chen_Chacón_2014, Chen_Chacón_2015} have developed structure-preserving particle pushers for neoclassical transport in the Vlasov equations, derived from Crank--Nicolson integrators. We show these too can can derive from a FET interpretation, similarly offering potential extensions to higher-order-in-time particle pushers. The FET formulation is used also to consider how the stochastic drift terms can be incorporated into the pushers. Stochastic gyrokinetic expansions are also discussed.

        Different options for the numerical implementation of these schemes are considered.

        Due to the efficacy of FET in the development of SP timesteppers for both the fluid and kinetic component, we hope this approach will prove effective in the future for developing SP timesteppers for the full hybrid model. We hope this will give us the opportunity to incorporate previously inaccessible kinetic effects into the highly effective, modern, finite-element MHD models.
    \end{abstract}
    
    
    \newpage
    \tableofcontents
    
    
    \newpage
    \pagenumbering{arabic}
    %\linenumbers\renewcommand\thelinenumber{\color{black!50}\arabic{linenumber}}
            \input{0 - introduction/main.tex}
        \part{Research}
            \input{1 - low-noise PiC models/main.tex}
            \input{2 - kinetic component/main.tex}
            \input{3 - fluid component/main.tex}
            \input{4 - numerical implementation/main.tex}
        \part{Project Overview}
            \input{5 - research plan/main.tex}
            \input{6 - summary/main.tex}
    
    
    %\section{}
    \newpage
    \pagenumbering{gobble}
        \printbibliography


    \newpage
    \pagenumbering{roman}
    \appendix
        \part{Appendices}
            \input{8 - Hilbert complexes/main.tex}
            \input{9 - weak conservation proofs/main.tex}
\end{document}

    
    
    %\section{}
    \newpage
    \pagenumbering{gobble}
        \printbibliography


    \newpage
    \pagenumbering{roman}
    \appendix
        \part{Appendices}
            \documentclass[12pt, a4paper]{report}

\input{template/main.tex}

\title{\BA{Title in Progress...}}
\author{Boris Andrews}
\affil{Mathematical Institute, University of Oxford}
\date{\today}


\begin{document}
    \pagenumbering{gobble}
    \maketitle
    
    
    \begin{abstract}
        Magnetic confinement reactors---in particular tokamaks---offer one of the most promising options for achieving practical nuclear fusion, with the potential to provide virtually limitless, clean energy. The theoretical and numerical modeling of tokamak plasmas is simultaneously an essential component of effective reactor design, and a great research barrier. Tokamak operational conditions exhibit comparatively low Knudsen numbers. Kinetic effects, including kinetic waves and instabilities, Landau damping, bump-on-tail instabilities and more, are therefore highly influential in tokamak plasma dynamics. Purely fluid models are inherently incapable of capturing these effects, whereas the high dimensionality in purely kinetic models render them practically intractable for most relevant purposes.

        We consider a $\delta\!f$ decomposition model, with a macroscopic fluid background and microscopic kinetic correction, both fully coupled to each other. A similar manner of discretization is proposed to that used in the recent \texttt{STRUPHY} code \cite{Holderied_Possanner_Wang_2021, Holderied_2022, Li_et_al_2023} with a finite-element model for the background and a pseudo-particle/PiC model for the correction.

        The fluid background satisfies the full, non-linear, resistive, compressible, Hall MHD equations. \cite{Laakmann_Hu_Farrell_2022} introduces finite-element(-in-space) implicit timesteppers for the incompressible analogue to this system with structure-preserving (SP) properties in the ideal case, alongside parameter-robust preconditioners. We show that these timesteppers can derive from a finite-element-in-time (FET) (and finite-element-in-space) interpretation. The benefits of this reformulation are discussed, including the derivation of timesteppers that are higher order in time, and the quantifiable dissipative SP properties in the non-ideal, resistive case.
        
        We discuss possible options for extending this FET approach to timesteppers for the compressible case.

        The kinetic corrections satisfy linearized Boltzmann equations. Using a Lénard--Bernstein collision operator, these take Fokker--Planck-like forms \cite{Fokker_1914, Planck_1917} wherein pseudo-particles in the numerical model obey the neoclassical transport equations, with particle-independent Brownian drift terms. This offers a rigorous methodology for incorporating collisions into the particle transport model, without coupling the equations of motions for each particle.
        
        Works by Chen, Chacón et al. \cite{Chen_Chacón_Barnes_2011, Chacón_Chen_Barnes_2013, Chen_Chacón_2014, Chen_Chacón_2015} have developed structure-preserving particle pushers for neoclassical transport in the Vlasov equations, derived from Crank--Nicolson integrators. We show these too can can derive from a FET interpretation, similarly offering potential extensions to higher-order-in-time particle pushers. The FET formulation is used also to consider how the stochastic drift terms can be incorporated into the pushers. Stochastic gyrokinetic expansions are also discussed.

        Different options for the numerical implementation of these schemes are considered.

        Due to the efficacy of FET in the development of SP timesteppers for both the fluid and kinetic component, we hope this approach will prove effective in the future for developing SP timesteppers for the full hybrid model. We hope this will give us the opportunity to incorporate previously inaccessible kinetic effects into the highly effective, modern, finite-element MHD models.
    \end{abstract}
    
    
    \newpage
    \tableofcontents
    
    
    \newpage
    \pagenumbering{arabic}
    %\linenumbers\renewcommand\thelinenumber{\color{black!50}\arabic{linenumber}}
            \input{0 - introduction/main.tex}
        \part{Research}
            \input{1 - low-noise PiC models/main.tex}
            \input{2 - kinetic component/main.tex}
            \input{3 - fluid component/main.tex}
            \input{4 - numerical implementation/main.tex}
        \part{Project Overview}
            \input{5 - research plan/main.tex}
            \input{6 - summary/main.tex}
    
    
    %\section{}
    \newpage
    \pagenumbering{gobble}
        \printbibliography


    \newpage
    \pagenumbering{roman}
    \appendix
        \part{Appendices}
            \input{8 - Hilbert complexes/main.tex}
            \input{9 - weak conservation proofs/main.tex}
\end{document}

            \documentclass[12pt, a4paper]{report}

\input{template/main.tex}

\title{\BA{Title in Progress...}}
\author{Boris Andrews}
\affil{Mathematical Institute, University of Oxford}
\date{\today}


\begin{document}
    \pagenumbering{gobble}
    \maketitle
    
    
    \begin{abstract}
        Magnetic confinement reactors---in particular tokamaks---offer one of the most promising options for achieving practical nuclear fusion, with the potential to provide virtually limitless, clean energy. The theoretical and numerical modeling of tokamak plasmas is simultaneously an essential component of effective reactor design, and a great research barrier. Tokamak operational conditions exhibit comparatively low Knudsen numbers. Kinetic effects, including kinetic waves and instabilities, Landau damping, bump-on-tail instabilities and more, are therefore highly influential in tokamak plasma dynamics. Purely fluid models are inherently incapable of capturing these effects, whereas the high dimensionality in purely kinetic models render them practically intractable for most relevant purposes.

        We consider a $\delta\!f$ decomposition model, with a macroscopic fluid background and microscopic kinetic correction, both fully coupled to each other. A similar manner of discretization is proposed to that used in the recent \texttt{STRUPHY} code \cite{Holderied_Possanner_Wang_2021, Holderied_2022, Li_et_al_2023} with a finite-element model for the background and a pseudo-particle/PiC model for the correction.

        The fluid background satisfies the full, non-linear, resistive, compressible, Hall MHD equations. \cite{Laakmann_Hu_Farrell_2022} introduces finite-element(-in-space) implicit timesteppers for the incompressible analogue to this system with structure-preserving (SP) properties in the ideal case, alongside parameter-robust preconditioners. We show that these timesteppers can derive from a finite-element-in-time (FET) (and finite-element-in-space) interpretation. The benefits of this reformulation are discussed, including the derivation of timesteppers that are higher order in time, and the quantifiable dissipative SP properties in the non-ideal, resistive case.
        
        We discuss possible options for extending this FET approach to timesteppers for the compressible case.

        The kinetic corrections satisfy linearized Boltzmann equations. Using a Lénard--Bernstein collision operator, these take Fokker--Planck-like forms \cite{Fokker_1914, Planck_1917} wherein pseudo-particles in the numerical model obey the neoclassical transport equations, with particle-independent Brownian drift terms. This offers a rigorous methodology for incorporating collisions into the particle transport model, without coupling the equations of motions for each particle.
        
        Works by Chen, Chacón et al. \cite{Chen_Chacón_Barnes_2011, Chacón_Chen_Barnes_2013, Chen_Chacón_2014, Chen_Chacón_2015} have developed structure-preserving particle pushers for neoclassical transport in the Vlasov equations, derived from Crank--Nicolson integrators. We show these too can can derive from a FET interpretation, similarly offering potential extensions to higher-order-in-time particle pushers. The FET formulation is used also to consider how the stochastic drift terms can be incorporated into the pushers. Stochastic gyrokinetic expansions are also discussed.

        Different options for the numerical implementation of these schemes are considered.

        Due to the efficacy of FET in the development of SP timesteppers for both the fluid and kinetic component, we hope this approach will prove effective in the future for developing SP timesteppers for the full hybrid model. We hope this will give us the opportunity to incorporate previously inaccessible kinetic effects into the highly effective, modern, finite-element MHD models.
    \end{abstract}
    
    
    \newpage
    \tableofcontents
    
    
    \newpage
    \pagenumbering{arabic}
    %\linenumbers\renewcommand\thelinenumber{\color{black!50}\arabic{linenumber}}
            \input{0 - introduction/main.tex}
        \part{Research}
            \input{1 - low-noise PiC models/main.tex}
            \input{2 - kinetic component/main.tex}
            \input{3 - fluid component/main.tex}
            \input{4 - numerical implementation/main.tex}
        \part{Project Overview}
            \input{5 - research plan/main.tex}
            \input{6 - summary/main.tex}
    
    
    %\section{}
    \newpage
    \pagenumbering{gobble}
        \printbibliography


    \newpage
    \pagenumbering{roman}
    \appendix
        \part{Appendices}
            \input{8 - Hilbert complexes/main.tex}
            \input{9 - weak conservation proofs/main.tex}
\end{document}

\end{document}


\title{\BA{Title in Progress...}}
\author{Boris Andrews}
\affil{Mathematical Institute, University of Oxford}
\date{\today}


\begin{document}
    \pagenumbering{gobble}
    \maketitle
    
    
    \begin{abstract}
        Magnetic confinement reactors---in particular tokamaks---offer one of the most promising options for achieving practical nuclear fusion, with the potential to provide virtually limitless, clean energy. The theoretical and numerical modeling of tokamak plasmas is simultaneously an essential component of effective reactor design, and a great research barrier. Tokamak operational conditions exhibit comparatively low Knudsen numbers. Kinetic effects, including kinetic waves and instabilities, Landau damping, bump-on-tail instabilities and more, are therefore highly influential in tokamak plasma dynamics. Purely fluid models are inherently incapable of capturing these effects, whereas the high dimensionality in purely kinetic models render them practically intractable for most relevant purposes.

        We consider a $\delta\!f$ decomposition model, with a macroscopic fluid background and microscopic kinetic correction, both fully coupled to each other. A similar manner of discretization is proposed to that used in the recent \texttt{STRUPHY} code \cite{Holderied_Possanner_Wang_2021, Holderied_2022, Li_et_al_2023} with a finite-element model for the background and a pseudo-particle/PiC model for the correction.

        The fluid background satisfies the full, non-linear, resistive, compressible, Hall MHD equations. \cite{Laakmann_Hu_Farrell_2022} introduces finite-element(-in-space) implicit timesteppers for the incompressible analogue to this system with structure-preserving (SP) properties in the ideal case, alongside parameter-robust preconditioners. We show that these timesteppers can derive from a finite-element-in-time (FET) (and finite-element-in-space) interpretation. The benefits of this reformulation are discussed, including the derivation of timesteppers that are higher order in time, and the quantifiable dissipative SP properties in the non-ideal, resistive case.
        
        We discuss possible options for extending this FET approach to timesteppers for the compressible case.

        The kinetic corrections satisfy linearized Boltzmann equations. Using a Lénard--Bernstein collision operator, these take Fokker--Planck-like forms \cite{Fokker_1914, Planck_1917} wherein pseudo-particles in the numerical model obey the neoclassical transport equations, with particle-independent Brownian drift terms. This offers a rigorous methodology for incorporating collisions into the particle transport model, without coupling the equations of motions for each particle.
        
        Works by Chen, Chacón et al. \cite{Chen_Chacón_Barnes_2011, Chacón_Chen_Barnes_2013, Chen_Chacón_2014, Chen_Chacón_2015} have developed structure-preserving particle pushers for neoclassical transport in the Vlasov equations, derived from Crank--Nicolson integrators. We show these too can can derive from a FET interpretation, similarly offering potential extensions to higher-order-in-time particle pushers. The FET formulation is used also to consider how the stochastic drift terms can be incorporated into the pushers. Stochastic gyrokinetic expansions are also discussed.

        Different options for the numerical implementation of these schemes are considered.

        Due to the efficacy of FET in the development of SP timesteppers for both the fluid and kinetic component, we hope this approach will prove effective in the future for developing SP timesteppers for the full hybrid model. We hope this will give us the opportunity to incorporate previously inaccessible kinetic effects into the highly effective, modern, finite-element MHD models.
    \end{abstract}
    
    
    \newpage
    \tableofcontents
    
    
    \newpage
    \pagenumbering{arabic}
    %\linenumbers\renewcommand\thelinenumber{\color{black!50}\arabic{linenumber}}
            \documentclass[12pt, a4paper]{report}

\documentclass[12pt, a4paper]{report}

\input{template/main.tex}

\title{\BA{Title in Progress...}}
\author{Boris Andrews}
\affil{Mathematical Institute, University of Oxford}
\date{\today}


\begin{document}
    \pagenumbering{gobble}
    \maketitle
    
    
    \begin{abstract}
        Magnetic confinement reactors---in particular tokamaks---offer one of the most promising options for achieving practical nuclear fusion, with the potential to provide virtually limitless, clean energy. The theoretical and numerical modeling of tokamak plasmas is simultaneously an essential component of effective reactor design, and a great research barrier. Tokamak operational conditions exhibit comparatively low Knudsen numbers. Kinetic effects, including kinetic waves and instabilities, Landau damping, bump-on-tail instabilities and more, are therefore highly influential in tokamak plasma dynamics. Purely fluid models are inherently incapable of capturing these effects, whereas the high dimensionality in purely kinetic models render them practically intractable for most relevant purposes.

        We consider a $\delta\!f$ decomposition model, with a macroscopic fluid background and microscopic kinetic correction, both fully coupled to each other. A similar manner of discretization is proposed to that used in the recent \texttt{STRUPHY} code \cite{Holderied_Possanner_Wang_2021, Holderied_2022, Li_et_al_2023} with a finite-element model for the background and a pseudo-particle/PiC model for the correction.

        The fluid background satisfies the full, non-linear, resistive, compressible, Hall MHD equations. \cite{Laakmann_Hu_Farrell_2022} introduces finite-element(-in-space) implicit timesteppers for the incompressible analogue to this system with structure-preserving (SP) properties in the ideal case, alongside parameter-robust preconditioners. We show that these timesteppers can derive from a finite-element-in-time (FET) (and finite-element-in-space) interpretation. The benefits of this reformulation are discussed, including the derivation of timesteppers that are higher order in time, and the quantifiable dissipative SP properties in the non-ideal, resistive case.
        
        We discuss possible options for extending this FET approach to timesteppers for the compressible case.

        The kinetic corrections satisfy linearized Boltzmann equations. Using a Lénard--Bernstein collision operator, these take Fokker--Planck-like forms \cite{Fokker_1914, Planck_1917} wherein pseudo-particles in the numerical model obey the neoclassical transport equations, with particle-independent Brownian drift terms. This offers a rigorous methodology for incorporating collisions into the particle transport model, without coupling the equations of motions for each particle.
        
        Works by Chen, Chacón et al. \cite{Chen_Chacón_Barnes_2011, Chacón_Chen_Barnes_2013, Chen_Chacón_2014, Chen_Chacón_2015} have developed structure-preserving particle pushers for neoclassical transport in the Vlasov equations, derived from Crank--Nicolson integrators. We show these too can can derive from a FET interpretation, similarly offering potential extensions to higher-order-in-time particle pushers. The FET formulation is used also to consider how the stochastic drift terms can be incorporated into the pushers. Stochastic gyrokinetic expansions are also discussed.

        Different options for the numerical implementation of these schemes are considered.

        Due to the efficacy of FET in the development of SP timesteppers for both the fluid and kinetic component, we hope this approach will prove effective in the future for developing SP timesteppers for the full hybrid model. We hope this will give us the opportunity to incorporate previously inaccessible kinetic effects into the highly effective, modern, finite-element MHD models.
    \end{abstract}
    
    
    \newpage
    \tableofcontents
    
    
    \newpage
    \pagenumbering{arabic}
    %\linenumbers\renewcommand\thelinenumber{\color{black!50}\arabic{linenumber}}
            \input{0 - introduction/main.tex}
        \part{Research}
            \input{1 - low-noise PiC models/main.tex}
            \input{2 - kinetic component/main.tex}
            \input{3 - fluid component/main.tex}
            \input{4 - numerical implementation/main.tex}
        \part{Project Overview}
            \input{5 - research plan/main.tex}
            \input{6 - summary/main.tex}
    
    
    %\section{}
    \newpage
    \pagenumbering{gobble}
        \printbibliography


    \newpage
    \pagenumbering{roman}
    \appendix
        \part{Appendices}
            \input{8 - Hilbert complexes/main.tex}
            \input{9 - weak conservation proofs/main.tex}
\end{document}


\title{\BA{Title in Progress...}}
\author{Boris Andrews}
\affil{Mathematical Institute, University of Oxford}
\date{\today}


\begin{document}
    \pagenumbering{gobble}
    \maketitle
    
    
    \begin{abstract}
        Magnetic confinement reactors---in particular tokamaks---offer one of the most promising options for achieving practical nuclear fusion, with the potential to provide virtually limitless, clean energy. The theoretical and numerical modeling of tokamak plasmas is simultaneously an essential component of effective reactor design, and a great research barrier. Tokamak operational conditions exhibit comparatively low Knudsen numbers. Kinetic effects, including kinetic waves and instabilities, Landau damping, bump-on-tail instabilities and more, are therefore highly influential in tokamak plasma dynamics. Purely fluid models are inherently incapable of capturing these effects, whereas the high dimensionality in purely kinetic models render them practically intractable for most relevant purposes.

        We consider a $\delta\!f$ decomposition model, with a macroscopic fluid background and microscopic kinetic correction, both fully coupled to each other. A similar manner of discretization is proposed to that used in the recent \texttt{STRUPHY} code \cite{Holderied_Possanner_Wang_2021, Holderied_2022, Li_et_al_2023} with a finite-element model for the background and a pseudo-particle/PiC model for the correction.

        The fluid background satisfies the full, non-linear, resistive, compressible, Hall MHD equations. \cite{Laakmann_Hu_Farrell_2022} introduces finite-element(-in-space) implicit timesteppers for the incompressible analogue to this system with structure-preserving (SP) properties in the ideal case, alongside parameter-robust preconditioners. We show that these timesteppers can derive from a finite-element-in-time (FET) (and finite-element-in-space) interpretation. The benefits of this reformulation are discussed, including the derivation of timesteppers that are higher order in time, and the quantifiable dissipative SP properties in the non-ideal, resistive case.
        
        We discuss possible options for extending this FET approach to timesteppers for the compressible case.

        The kinetic corrections satisfy linearized Boltzmann equations. Using a Lénard--Bernstein collision operator, these take Fokker--Planck-like forms \cite{Fokker_1914, Planck_1917} wherein pseudo-particles in the numerical model obey the neoclassical transport equations, with particle-independent Brownian drift terms. This offers a rigorous methodology for incorporating collisions into the particle transport model, without coupling the equations of motions for each particle.
        
        Works by Chen, Chacón et al. \cite{Chen_Chacón_Barnes_2011, Chacón_Chen_Barnes_2013, Chen_Chacón_2014, Chen_Chacón_2015} have developed structure-preserving particle pushers for neoclassical transport in the Vlasov equations, derived from Crank--Nicolson integrators. We show these too can can derive from a FET interpretation, similarly offering potential extensions to higher-order-in-time particle pushers. The FET formulation is used also to consider how the stochastic drift terms can be incorporated into the pushers. Stochastic gyrokinetic expansions are also discussed.

        Different options for the numerical implementation of these schemes are considered.

        Due to the efficacy of FET in the development of SP timesteppers for both the fluid and kinetic component, we hope this approach will prove effective in the future for developing SP timesteppers for the full hybrid model. We hope this will give us the opportunity to incorporate previously inaccessible kinetic effects into the highly effective, modern, finite-element MHD models.
    \end{abstract}
    
    
    \newpage
    \tableofcontents
    
    
    \newpage
    \pagenumbering{arabic}
    %\linenumbers\renewcommand\thelinenumber{\color{black!50}\arabic{linenumber}}
            \documentclass[12pt, a4paper]{report}

\input{template/main.tex}

\title{\BA{Title in Progress...}}
\author{Boris Andrews}
\affil{Mathematical Institute, University of Oxford}
\date{\today}


\begin{document}
    \pagenumbering{gobble}
    \maketitle
    
    
    \begin{abstract}
        Magnetic confinement reactors---in particular tokamaks---offer one of the most promising options for achieving practical nuclear fusion, with the potential to provide virtually limitless, clean energy. The theoretical and numerical modeling of tokamak plasmas is simultaneously an essential component of effective reactor design, and a great research barrier. Tokamak operational conditions exhibit comparatively low Knudsen numbers. Kinetic effects, including kinetic waves and instabilities, Landau damping, bump-on-tail instabilities and more, are therefore highly influential in tokamak plasma dynamics. Purely fluid models are inherently incapable of capturing these effects, whereas the high dimensionality in purely kinetic models render them practically intractable for most relevant purposes.

        We consider a $\delta\!f$ decomposition model, with a macroscopic fluid background and microscopic kinetic correction, both fully coupled to each other. A similar manner of discretization is proposed to that used in the recent \texttt{STRUPHY} code \cite{Holderied_Possanner_Wang_2021, Holderied_2022, Li_et_al_2023} with a finite-element model for the background and a pseudo-particle/PiC model for the correction.

        The fluid background satisfies the full, non-linear, resistive, compressible, Hall MHD equations. \cite{Laakmann_Hu_Farrell_2022} introduces finite-element(-in-space) implicit timesteppers for the incompressible analogue to this system with structure-preserving (SP) properties in the ideal case, alongside parameter-robust preconditioners. We show that these timesteppers can derive from a finite-element-in-time (FET) (and finite-element-in-space) interpretation. The benefits of this reformulation are discussed, including the derivation of timesteppers that are higher order in time, and the quantifiable dissipative SP properties in the non-ideal, resistive case.
        
        We discuss possible options for extending this FET approach to timesteppers for the compressible case.

        The kinetic corrections satisfy linearized Boltzmann equations. Using a Lénard--Bernstein collision operator, these take Fokker--Planck-like forms \cite{Fokker_1914, Planck_1917} wherein pseudo-particles in the numerical model obey the neoclassical transport equations, with particle-independent Brownian drift terms. This offers a rigorous methodology for incorporating collisions into the particle transport model, without coupling the equations of motions for each particle.
        
        Works by Chen, Chacón et al. \cite{Chen_Chacón_Barnes_2011, Chacón_Chen_Barnes_2013, Chen_Chacón_2014, Chen_Chacón_2015} have developed structure-preserving particle pushers for neoclassical transport in the Vlasov equations, derived from Crank--Nicolson integrators. We show these too can can derive from a FET interpretation, similarly offering potential extensions to higher-order-in-time particle pushers. The FET formulation is used also to consider how the stochastic drift terms can be incorporated into the pushers. Stochastic gyrokinetic expansions are also discussed.

        Different options for the numerical implementation of these schemes are considered.

        Due to the efficacy of FET in the development of SP timesteppers for both the fluid and kinetic component, we hope this approach will prove effective in the future for developing SP timesteppers for the full hybrid model. We hope this will give us the opportunity to incorporate previously inaccessible kinetic effects into the highly effective, modern, finite-element MHD models.
    \end{abstract}
    
    
    \newpage
    \tableofcontents
    
    
    \newpage
    \pagenumbering{arabic}
    %\linenumbers\renewcommand\thelinenumber{\color{black!50}\arabic{linenumber}}
            \input{0 - introduction/main.tex}
        \part{Research}
            \input{1 - low-noise PiC models/main.tex}
            \input{2 - kinetic component/main.tex}
            \input{3 - fluid component/main.tex}
            \input{4 - numerical implementation/main.tex}
        \part{Project Overview}
            \input{5 - research plan/main.tex}
            \input{6 - summary/main.tex}
    
    
    %\section{}
    \newpage
    \pagenumbering{gobble}
        \printbibliography


    \newpage
    \pagenumbering{roman}
    \appendix
        \part{Appendices}
            \input{8 - Hilbert complexes/main.tex}
            \input{9 - weak conservation proofs/main.tex}
\end{document}

        \part{Research}
            \documentclass[12pt, a4paper]{report}

\input{template/main.tex}

\title{\BA{Title in Progress...}}
\author{Boris Andrews}
\affil{Mathematical Institute, University of Oxford}
\date{\today}


\begin{document}
    \pagenumbering{gobble}
    \maketitle
    
    
    \begin{abstract}
        Magnetic confinement reactors---in particular tokamaks---offer one of the most promising options for achieving practical nuclear fusion, with the potential to provide virtually limitless, clean energy. The theoretical and numerical modeling of tokamak plasmas is simultaneously an essential component of effective reactor design, and a great research barrier. Tokamak operational conditions exhibit comparatively low Knudsen numbers. Kinetic effects, including kinetic waves and instabilities, Landau damping, bump-on-tail instabilities and more, are therefore highly influential in tokamak plasma dynamics. Purely fluid models are inherently incapable of capturing these effects, whereas the high dimensionality in purely kinetic models render them practically intractable for most relevant purposes.

        We consider a $\delta\!f$ decomposition model, with a macroscopic fluid background and microscopic kinetic correction, both fully coupled to each other. A similar manner of discretization is proposed to that used in the recent \texttt{STRUPHY} code \cite{Holderied_Possanner_Wang_2021, Holderied_2022, Li_et_al_2023} with a finite-element model for the background and a pseudo-particle/PiC model for the correction.

        The fluid background satisfies the full, non-linear, resistive, compressible, Hall MHD equations. \cite{Laakmann_Hu_Farrell_2022} introduces finite-element(-in-space) implicit timesteppers for the incompressible analogue to this system with structure-preserving (SP) properties in the ideal case, alongside parameter-robust preconditioners. We show that these timesteppers can derive from a finite-element-in-time (FET) (and finite-element-in-space) interpretation. The benefits of this reformulation are discussed, including the derivation of timesteppers that are higher order in time, and the quantifiable dissipative SP properties in the non-ideal, resistive case.
        
        We discuss possible options for extending this FET approach to timesteppers for the compressible case.

        The kinetic corrections satisfy linearized Boltzmann equations. Using a Lénard--Bernstein collision operator, these take Fokker--Planck-like forms \cite{Fokker_1914, Planck_1917} wherein pseudo-particles in the numerical model obey the neoclassical transport equations, with particle-independent Brownian drift terms. This offers a rigorous methodology for incorporating collisions into the particle transport model, without coupling the equations of motions for each particle.
        
        Works by Chen, Chacón et al. \cite{Chen_Chacón_Barnes_2011, Chacón_Chen_Barnes_2013, Chen_Chacón_2014, Chen_Chacón_2015} have developed structure-preserving particle pushers for neoclassical transport in the Vlasov equations, derived from Crank--Nicolson integrators. We show these too can can derive from a FET interpretation, similarly offering potential extensions to higher-order-in-time particle pushers. The FET formulation is used also to consider how the stochastic drift terms can be incorporated into the pushers. Stochastic gyrokinetic expansions are also discussed.

        Different options for the numerical implementation of these schemes are considered.

        Due to the efficacy of FET in the development of SP timesteppers for both the fluid and kinetic component, we hope this approach will prove effective in the future for developing SP timesteppers for the full hybrid model. We hope this will give us the opportunity to incorporate previously inaccessible kinetic effects into the highly effective, modern, finite-element MHD models.
    \end{abstract}
    
    
    \newpage
    \tableofcontents
    
    
    \newpage
    \pagenumbering{arabic}
    %\linenumbers\renewcommand\thelinenumber{\color{black!50}\arabic{linenumber}}
            \input{0 - introduction/main.tex}
        \part{Research}
            \input{1 - low-noise PiC models/main.tex}
            \input{2 - kinetic component/main.tex}
            \input{3 - fluid component/main.tex}
            \input{4 - numerical implementation/main.tex}
        \part{Project Overview}
            \input{5 - research plan/main.tex}
            \input{6 - summary/main.tex}
    
    
    %\section{}
    \newpage
    \pagenumbering{gobble}
        \printbibliography


    \newpage
    \pagenumbering{roman}
    \appendix
        \part{Appendices}
            \input{8 - Hilbert complexes/main.tex}
            \input{9 - weak conservation proofs/main.tex}
\end{document}

            \documentclass[12pt, a4paper]{report}

\input{template/main.tex}

\title{\BA{Title in Progress...}}
\author{Boris Andrews}
\affil{Mathematical Institute, University of Oxford}
\date{\today}


\begin{document}
    \pagenumbering{gobble}
    \maketitle
    
    
    \begin{abstract}
        Magnetic confinement reactors---in particular tokamaks---offer one of the most promising options for achieving practical nuclear fusion, with the potential to provide virtually limitless, clean energy. The theoretical and numerical modeling of tokamak plasmas is simultaneously an essential component of effective reactor design, and a great research barrier. Tokamak operational conditions exhibit comparatively low Knudsen numbers. Kinetic effects, including kinetic waves and instabilities, Landau damping, bump-on-tail instabilities and more, are therefore highly influential in tokamak plasma dynamics. Purely fluid models are inherently incapable of capturing these effects, whereas the high dimensionality in purely kinetic models render them practically intractable for most relevant purposes.

        We consider a $\delta\!f$ decomposition model, with a macroscopic fluid background and microscopic kinetic correction, both fully coupled to each other. A similar manner of discretization is proposed to that used in the recent \texttt{STRUPHY} code \cite{Holderied_Possanner_Wang_2021, Holderied_2022, Li_et_al_2023} with a finite-element model for the background and a pseudo-particle/PiC model for the correction.

        The fluid background satisfies the full, non-linear, resistive, compressible, Hall MHD equations. \cite{Laakmann_Hu_Farrell_2022} introduces finite-element(-in-space) implicit timesteppers for the incompressible analogue to this system with structure-preserving (SP) properties in the ideal case, alongside parameter-robust preconditioners. We show that these timesteppers can derive from a finite-element-in-time (FET) (and finite-element-in-space) interpretation. The benefits of this reformulation are discussed, including the derivation of timesteppers that are higher order in time, and the quantifiable dissipative SP properties in the non-ideal, resistive case.
        
        We discuss possible options for extending this FET approach to timesteppers for the compressible case.

        The kinetic corrections satisfy linearized Boltzmann equations. Using a Lénard--Bernstein collision operator, these take Fokker--Planck-like forms \cite{Fokker_1914, Planck_1917} wherein pseudo-particles in the numerical model obey the neoclassical transport equations, with particle-independent Brownian drift terms. This offers a rigorous methodology for incorporating collisions into the particle transport model, without coupling the equations of motions for each particle.
        
        Works by Chen, Chacón et al. \cite{Chen_Chacón_Barnes_2011, Chacón_Chen_Barnes_2013, Chen_Chacón_2014, Chen_Chacón_2015} have developed structure-preserving particle pushers for neoclassical transport in the Vlasov equations, derived from Crank--Nicolson integrators. We show these too can can derive from a FET interpretation, similarly offering potential extensions to higher-order-in-time particle pushers. The FET formulation is used also to consider how the stochastic drift terms can be incorporated into the pushers. Stochastic gyrokinetic expansions are also discussed.

        Different options for the numerical implementation of these schemes are considered.

        Due to the efficacy of FET in the development of SP timesteppers for both the fluid and kinetic component, we hope this approach will prove effective in the future for developing SP timesteppers for the full hybrid model. We hope this will give us the opportunity to incorporate previously inaccessible kinetic effects into the highly effective, modern, finite-element MHD models.
    \end{abstract}
    
    
    \newpage
    \tableofcontents
    
    
    \newpage
    \pagenumbering{arabic}
    %\linenumbers\renewcommand\thelinenumber{\color{black!50}\arabic{linenumber}}
            \input{0 - introduction/main.tex}
        \part{Research}
            \input{1 - low-noise PiC models/main.tex}
            \input{2 - kinetic component/main.tex}
            \input{3 - fluid component/main.tex}
            \input{4 - numerical implementation/main.tex}
        \part{Project Overview}
            \input{5 - research plan/main.tex}
            \input{6 - summary/main.tex}
    
    
    %\section{}
    \newpage
    \pagenumbering{gobble}
        \printbibliography


    \newpage
    \pagenumbering{roman}
    \appendix
        \part{Appendices}
            \input{8 - Hilbert complexes/main.tex}
            \input{9 - weak conservation proofs/main.tex}
\end{document}

            \documentclass[12pt, a4paper]{report}

\input{template/main.tex}

\title{\BA{Title in Progress...}}
\author{Boris Andrews}
\affil{Mathematical Institute, University of Oxford}
\date{\today}


\begin{document}
    \pagenumbering{gobble}
    \maketitle
    
    
    \begin{abstract}
        Magnetic confinement reactors---in particular tokamaks---offer one of the most promising options for achieving practical nuclear fusion, with the potential to provide virtually limitless, clean energy. The theoretical and numerical modeling of tokamak plasmas is simultaneously an essential component of effective reactor design, and a great research barrier. Tokamak operational conditions exhibit comparatively low Knudsen numbers. Kinetic effects, including kinetic waves and instabilities, Landau damping, bump-on-tail instabilities and more, are therefore highly influential in tokamak plasma dynamics. Purely fluid models are inherently incapable of capturing these effects, whereas the high dimensionality in purely kinetic models render them practically intractable for most relevant purposes.

        We consider a $\delta\!f$ decomposition model, with a macroscopic fluid background and microscopic kinetic correction, both fully coupled to each other. A similar manner of discretization is proposed to that used in the recent \texttt{STRUPHY} code \cite{Holderied_Possanner_Wang_2021, Holderied_2022, Li_et_al_2023} with a finite-element model for the background and a pseudo-particle/PiC model for the correction.

        The fluid background satisfies the full, non-linear, resistive, compressible, Hall MHD equations. \cite{Laakmann_Hu_Farrell_2022} introduces finite-element(-in-space) implicit timesteppers for the incompressible analogue to this system with structure-preserving (SP) properties in the ideal case, alongside parameter-robust preconditioners. We show that these timesteppers can derive from a finite-element-in-time (FET) (and finite-element-in-space) interpretation. The benefits of this reformulation are discussed, including the derivation of timesteppers that are higher order in time, and the quantifiable dissipative SP properties in the non-ideal, resistive case.
        
        We discuss possible options for extending this FET approach to timesteppers for the compressible case.

        The kinetic corrections satisfy linearized Boltzmann equations. Using a Lénard--Bernstein collision operator, these take Fokker--Planck-like forms \cite{Fokker_1914, Planck_1917} wherein pseudo-particles in the numerical model obey the neoclassical transport equations, with particle-independent Brownian drift terms. This offers a rigorous methodology for incorporating collisions into the particle transport model, without coupling the equations of motions for each particle.
        
        Works by Chen, Chacón et al. \cite{Chen_Chacón_Barnes_2011, Chacón_Chen_Barnes_2013, Chen_Chacón_2014, Chen_Chacón_2015} have developed structure-preserving particle pushers for neoclassical transport in the Vlasov equations, derived from Crank--Nicolson integrators. We show these too can can derive from a FET interpretation, similarly offering potential extensions to higher-order-in-time particle pushers. The FET formulation is used also to consider how the stochastic drift terms can be incorporated into the pushers. Stochastic gyrokinetic expansions are also discussed.

        Different options for the numerical implementation of these schemes are considered.

        Due to the efficacy of FET in the development of SP timesteppers for both the fluid and kinetic component, we hope this approach will prove effective in the future for developing SP timesteppers for the full hybrid model. We hope this will give us the opportunity to incorporate previously inaccessible kinetic effects into the highly effective, modern, finite-element MHD models.
    \end{abstract}
    
    
    \newpage
    \tableofcontents
    
    
    \newpage
    \pagenumbering{arabic}
    %\linenumbers\renewcommand\thelinenumber{\color{black!50}\arabic{linenumber}}
            \input{0 - introduction/main.tex}
        \part{Research}
            \input{1 - low-noise PiC models/main.tex}
            \input{2 - kinetic component/main.tex}
            \input{3 - fluid component/main.tex}
            \input{4 - numerical implementation/main.tex}
        \part{Project Overview}
            \input{5 - research plan/main.tex}
            \input{6 - summary/main.tex}
    
    
    %\section{}
    \newpage
    \pagenumbering{gobble}
        \printbibliography


    \newpage
    \pagenumbering{roman}
    \appendix
        \part{Appendices}
            \input{8 - Hilbert complexes/main.tex}
            \input{9 - weak conservation proofs/main.tex}
\end{document}

            \documentclass[12pt, a4paper]{report}

\input{template/main.tex}

\title{\BA{Title in Progress...}}
\author{Boris Andrews}
\affil{Mathematical Institute, University of Oxford}
\date{\today}


\begin{document}
    \pagenumbering{gobble}
    \maketitle
    
    
    \begin{abstract}
        Magnetic confinement reactors---in particular tokamaks---offer one of the most promising options for achieving practical nuclear fusion, with the potential to provide virtually limitless, clean energy. The theoretical and numerical modeling of tokamak plasmas is simultaneously an essential component of effective reactor design, and a great research barrier. Tokamak operational conditions exhibit comparatively low Knudsen numbers. Kinetic effects, including kinetic waves and instabilities, Landau damping, bump-on-tail instabilities and more, are therefore highly influential in tokamak plasma dynamics. Purely fluid models are inherently incapable of capturing these effects, whereas the high dimensionality in purely kinetic models render them practically intractable for most relevant purposes.

        We consider a $\delta\!f$ decomposition model, with a macroscopic fluid background and microscopic kinetic correction, both fully coupled to each other. A similar manner of discretization is proposed to that used in the recent \texttt{STRUPHY} code \cite{Holderied_Possanner_Wang_2021, Holderied_2022, Li_et_al_2023} with a finite-element model for the background and a pseudo-particle/PiC model for the correction.

        The fluid background satisfies the full, non-linear, resistive, compressible, Hall MHD equations. \cite{Laakmann_Hu_Farrell_2022} introduces finite-element(-in-space) implicit timesteppers for the incompressible analogue to this system with structure-preserving (SP) properties in the ideal case, alongside parameter-robust preconditioners. We show that these timesteppers can derive from a finite-element-in-time (FET) (and finite-element-in-space) interpretation. The benefits of this reformulation are discussed, including the derivation of timesteppers that are higher order in time, and the quantifiable dissipative SP properties in the non-ideal, resistive case.
        
        We discuss possible options for extending this FET approach to timesteppers for the compressible case.

        The kinetic corrections satisfy linearized Boltzmann equations. Using a Lénard--Bernstein collision operator, these take Fokker--Planck-like forms \cite{Fokker_1914, Planck_1917} wherein pseudo-particles in the numerical model obey the neoclassical transport equations, with particle-independent Brownian drift terms. This offers a rigorous methodology for incorporating collisions into the particle transport model, without coupling the equations of motions for each particle.
        
        Works by Chen, Chacón et al. \cite{Chen_Chacón_Barnes_2011, Chacón_Chen_Barnes_2013, Chen_Chacón_2014, Chen_Chacón_2015} have developed structure-preserving particle pushers for neoclassical transport in the Vlasov equations, derived from Crank--Nicolson integrators. We show these too can can derive from a FET interpretation, similarly offering potential extensions to higher-order-in-time particle pushers. The FET formulation is used also to consider how the stochastic drift terms can be incorporated into the pushers. Stochastic gyrokinetic expansions are also discussed.

        Different options for the numerical implementation of these schemes are considered.

        Due to the efficacy of FET in the development of SP timesteppers for both the fluid and kinetic component, we hope this approach will prove effective in the future for developing SP timesteppers for the full hybrid model. We hope this will give us the opportunity to incorporate previously inaccessible kinetic effects into the highly effective, modern, finite-element MHD models.
    \end{abstract}
    
    
    \newpage
    \tableofcontents
    
    
    \newpage
    \pagenumbering{arabic}
    %\linenumbers\renewcommand\thelinenumber{\color{black!50}\arabic{linenumber}}
            \input{0 - introduction/main.tex}
        \part{Research}
            \input{1 - low-noise PiC models/main.tex}
            \input{2 - kinetic component/main.tex}
            \input{3 - fluid component/main.tex}
            \input{4 - numerical implementation/main.tex}
        \part{Project Overview}
            \input{5 - research plan/main.tex}
            \input{6 - summary/main.tex}
    
    
    %\section{}
    \newpage
    \pagenumbering{gobble}
        \printbibliography


    \newpage
    \pagenumbering{roman}
    \appendix
        \part{Appendices}
            \input{8 - Hilbert complexes/main.tex}
            \input{9 - weak conservation proofs/main.tex}
\end{document}

        \part{Project Overview}
            \documentclass[12pt, a4paper]{report}

\input{template/main.tex}

\title{\BA{Title in Progress...}}
\author{Boris Andrews}
\affil{Mathematical Institute, University of Oxford}
\date{\today}


\begin{document}
    \pagenumbering{gobble}
    \maketitle
    
    
    \begin{abstract}
        Magnetic confinement reactors---in particular tokamaks---offer one of the most promising options for achieving practical nuclear fusion, with the potential to provide virtually limitless, clean energy. The theoretical and numerical modeling of tokamak plasmas is simultaneously an essential component of effective reactor design, and a great research barrier. Tokamak operational conditions exhibit comparatively low Knudsen numbers. Kinetic effects, including kinetic waves and instabilities, Landau damping, bump-on-tail instabilities and more, are therefore highly influential in tokamak plasma dynamics. Purely fluid models are inherently incapable of capturing these effects, whereas the high dimensionality in purely kinetic models render them practically intractable for most relevant purposes.

        We consider a $\delta\!f$ decomposition model, with a macroscopic fluid background and microscopic kinetic correction, both fully coupled to each other. A similar manner of discretization is proposed to that used in the recent \texttt{STRUPHY} code \cite{Holderied_Possanner_Wang_2021, Holderied_2022, Li_et_al_2023} with a finite-element model for the background and a pseudo-particle/PiC model for the correction.

        The fluid background satisfies the full, non-linear, resistive, compressible, Hall MHD equations. \cite{Laakmann_Hu_Farrell_2022} introduces finite-element(-in-space) implicit timesteppers for the incompressible analogue to this system with structure-preserving (SP) properties in the ideal case, alongside parameter-robust preconditioners. We show that these timesteppers can derive from a finite-element-in-time (FET) (and finite-element-in-space) interpretation. The benefits of this reformulation are discussed, including the derivation of timesteppers that are higher order in time, and the quantifiable dissipative SP properties in the non-ideal, resistive case.
        
        We discuss possible options for extending this FET approach to timesteppers for the compressible case.

        The kinetic corrections satisfy linearized Boltzmann equations. Using a Lénard--Bernstein collision operator, these take Fokker--Planck-like forms \cite{Fokker_1914, Planck_1917} wherein pseudo-particles in the numerical model obey the neoclassical transport equations, with particle-independent Brownian drift terms. This offers a rigorous methodology for incorporating collisions into the particle transport model, without coupling the equations of motions for each particle.
        
        Works by Chen, Chacón et al. \cite{Chen_Chacón_Barnes_2011, Chacón_Chen_Barnes_2013, Chen_Chacón_2014, Chen_Chacón_2015} have developed structure-preserving particle pushers for neoclassical transport in the Vlasov equations, derived from Crank--Nicolson integrators. We show these too can can derive from a FET interpretation, similarly offering potential extensions to higher-order-in-time particle pushers. The FET formulation is used also to consider how the stochastic drift terms can be incorporated into the pushers. Stochastic gyrokinetic expansions are also discussed.

        Different options for the numerical implementation of these schemes are considered.

        Due to the efficacy of FET in the development of SP timesteppers for both the fluid and kinetic component, we hope this approach will prove effective in the future for developing SP timesteppers for the full hybrid model. We hope this will give us the opportunity to incorporate previously inaccessible kinetic effects into the highly effective, modern, finite-element MHD models.
    \end{abstract}
    
    
    \newpage
    \tableofcontents
    
    
    \newpage
    \pagenumbering{arabic}
    %\linenumbers\renewcommand\thelinenumber{\color{black!50}\arabic{linenumber}}
            \input{0 - introduction/main.tex}
        \part{Research}
            \input{1 - low-noise PiC models/main.tex}
            \input{2 - kinetic component/main.tex}
            \input{3 - fluid component/main.tex}
            \input{4 - numerical implementation/main.tex}
        \part{Project Overview}
            \input{5 - research plan/main.tex}
            \input{6 - summary/main.tex}
    
    
    %\section{}
    \newpage
    \pagenumbering{gobble}
        \printbibliography


    \newpage
    \pagenumbering{roman}
    \appendix
        \part{Appendices}
            \input{8 - Hilbert complexes/main.tex}
            \input{9 - weak conservation proofs/main.tex}
\end{document}

            \documentclass[12pt, a4paper]{report}

\input{template/main.tex}

\title{\BA{Title in Progress...}}
\author{Boris Andrews}
\affil{Mathematical Institute, University of Oxford}
\date{\today}


\begin{document}
    \pagenumbering{gobble}
    \maketitle
    
    
    \begin{abstract}
        Magnetic confinement reactors---in particular tokamaks---offer one of the most promising options for achieving practical nuclear fusion, with the potential to provide virtually limitless, clean energy. The theoretical and numerical modeling of tokamak plasmas is simultaneously an essential component of effective reactor design, and a great research barrier. Tokamak operational conditions exhibit comparatively low Knudsen numbers. Kinetic effects, including kinetic waves and instabilities, Landau damping, bump-on-tail instabilities and more, are therefore highly influential in tokamak plasma dynamics. Purely fluid models are inherently incapable of capturing these effects, whereas the high dimensionality in purely kinetic models render them practically intractable for most relevant purposes.

        We consider a $\delta\!f$ decomposition model, with a macroscopic fluid background and microscopic kinetic correction, both fully coupled to each other. A similar manner of discretization is proposed to that used in the recent \texttt{STRUPHY} code \cite{Holderied_Possanner_Wang_2021, Holderied_2022, Li_et_al_2023} with a finite-element model for the background and a pseudo-particle/PiC model for the correction.

        The fluid background satisfies the full, non-linear, resistive, compressible, Hall MHD equations. \cite{Laakmann_Hu_Farrell_2022} introduces finite-element(-in-space) implicit timesteppers for the incompressible analogue to this system with structure-preserving (SP) properties in the ideal case, alongside parameter-robust preconditioners. We show that these timesteppers can derive from a finite-element-in-time (FET) (and finite-element-in-space) interpretation. The benefits of this reformulation are discussed, including the derivation of timesteppers that are higher order in time, and the quantifiable dissipative SP properties in the non-ideal, resistive case.
        
        We discuss possible options for extending this FET approach to timesteppers for the compressible case.

        The kinetic corrections satisfy linearized Boltzmann equations. Using a Lénard--Bernstein collision operator, these take Fokker--Planck-like forms \cite{Fokker_1914, Planck_1917} wherein pseudo-particles in the numerical model obey the neoclassical transport equations, with particle-independent Brownian drift terms. This offers a rigorous methodology for incorporating collisions into the particle transport model, without coupling the equations of motions for each particle.
        
        Works by Chen, Chacón et al. \cite{Chen_Chacón_Barnes_2011, Chacón_Chen_Barnes_2013, Chen_Chacón_2014, Chen_Chacón_2015} have developed structure-preserving particle pushers for neoclassical transport in the Vlasov equations, derived from Crank--Nicolson integrators. We show these too can can derive from a FET interpretation, similarly offering potential extensions to higher-order-in-time particle pushers. The FET formulation is used also to consider how the stochastic drift terms can be incorporated into the pushers. Stochastic gyrokinetic expansions are also discussed.

        Different options for the numerical implementation of these schemes are considered.

        Due to the efficacy of FET in the development of SP timesteppers for both the fluid and kinetic component, we hope this approach will prove effective in the future for developing SP timesteppers for the full hybrid model. We hope this will give us the opportunity to incorporate previously inaccessible kinetic effects into the highly effective, modern, finite-element MHD models.
    \end{abstract}
    
    
    \newpage
    \tableofcontents
    
    
    \newpage
    \pagenumbering{arabic}
    %\linenumbers\renewcommand\thelinenumber{\color{black!50}\arabic{linenumber}}
            \input{0 - introduction/main.tex}
        \part{Research}
            \input{1 - low-noise PiC models/main.tex}
            \input{2 - kinetic component/main.tex}
            \input{3 - fluid component/main.tex}
            \input{4 - numerical implementation/main.tex}
        \part{Project Overview}
            \input{5 - research plan/main.tex}
            \input{6 - summary/main.tex}
    
    
    %\section{}
    \newpage
    \pagenumbering{gobble}
        \printbibliography


    \newpage
    \pagenumbering{roman}
    \appendix
        \part{Appendices}
            \input{8 - Hilbert complexes/main.tex}
            \input{9 - weak conservation proofs/main.tex}
\end{document}

    
    
    %\section{}
    \newpage
    \pagenumbering{gobble}
        \printbibliography


    \newpage
    \pagenumbering{roman}
    \appendix
        \part{Appendices}
            \documentclass[12pt, a4paper]{report}

\input{template/main.tex}

\title{\BA{Title in Progress...}}
\author{Boris Andrews}
\affil{Mathematical Institute, University of Oxford}
\date{\today}


\begin{document}
    \pagenumbering{gobble}
    \maketitle
    
    
    \begin{abstract}
        Magnetic confinement reactors---in particular tokamaks---offer one of the most promising options for achieving practical nuclear fusion, with the potential to provide virtually limitless, clean energy. The theoretical and numerical modeling of tokamak plasmas is simultaneously an essential component of effective reactor design, and a great research barrier. Tokamak operational conditions exhibit comparatively low Knudsen numbers. Kinetic effects, including kinetic waves and instabilities, Landau damping, bump-on-tail instabilities and more, are therefore highly influential in tokamak plasma dynamics. Purely fluid models are inherently incapable of capturing these effects, whereas the high dimensionality in purely kinetic models render them practically intractable for most relevant purposes.

        We consider a $\delta\!f$ decomposition model, with a macroscopic fluid background and microscopic kinetic correction, both fully coupled to each other. A similar manner of discretization is proposed to that used in the recent \texttt{STRUPHY} code \cite{Holderied_Possanner_Wang_2021, Holderied_2022, Li_et_al_2023} with a finite-element model for the background and a pseudo-particle/PiC model for the correction.

        The fluid background satisfies the full, non-linear, resistive, compressible, Hall MHD equations. \cite{Laakmann_Hu_Farrell_2022} introduces finite-element(-in-space) implicit timesteppers for the incompressible analogue to this system with structure-preserving (SP) properties in the ideal case, alongside parameter-robust preconditioners. We show that these timesteppers can derive from a finite-element-in-time (FET) (and finite-element-in-space) interpretation. The benefits of this reformulation are discussed, including the derivation of timesteppers that are higher order in time, and the quantifiable dissipative SP properties in the non-ideal, resistive case.
        
        We discuss possible options for extending this FET approach to timesteppers for the compressible case.

        The kinetic corrections satisfy linearized Boltzmann equations. Using a Lénard--Bernstein collision operator, these take Fokker--Planck-like forms \cite{Fokker_1914, Planck_1917} wherein pseudo-particles in the numerical model obey the neoclassical transport equations, with particle-independent Brownian drift terms. This offers a rigorous methodology for incorporating collisions into the particle transport model, without coupling the equations of motions for each particle.
        
        Works by Chen, Chacón et al. \cite{Chen_Chacón_Barnes_2011, Chacón_Chen_Barnes_2013, Chen_Chacón_2014, Chen_Chacón_2015} have developed structure-preserving particle pushers for neoclassical transport in the Vlasov equations, derived from Crank--Nicolson integrators. We show these too can can derive from a FET interpretation, similarly offering potential extensions to higher-order-in-time particle pushers. The FET formulation is used also to consider how the stochastic drift terms can be incorporated into the pushers. Stochastic gyrokinetic expansions are also discussed.

        Different options for the numerical implementation of these schemes are considered.

        Due to the efficacy of FET in the development of SP timesteppers for both the fluid and kinetic component, we hope this approach will prove effective in the future for developing SP timesteppers for the full hybrid model. We hope this will give us the opportunity to incorporate previously inaccessible kinetic effects into the highly effective, modern, finite-element MHD models.
    \end{abstract}
    
    
    \newpage
    \tableofcontents
    
    
    \newpage
    \pagenumbering{arabic}
    %\linenumbers\renewcommand\thelinenumber{\color{black!50}\arabic{linenumber}}
            \input{0 - introduction/main.tex}
        \part{Research}
            \input{1 - low-noise PiC models/main.tex}
            \input{2 - kinetic component/main.tex}
            \input{3 - fluid component/main.tex}
            \input{4 - numerical implementation/main.tex}
        \part{Project Overview}
            \input{5 - research plan/main.tex}
            \input{6 - summary/main.tex}
    
    
    %\section{}
    \newpage
    \pagenumbering{gobble}
        \printbibliography


    \newpage
    \pagenumbering{roman}
    \appendix
        \part{Appendices}
            \input{8 - Hilbert complexes/main.tex}
            \input{9 - weak conservation proofs/main.tex}
\end{document}

            \documentclass[12pt, a4paper]{report}

\input{template/main.tex}

\title{\BA{Title in Progress...}}
\author{Boris Andrews}
\affil{Mathematical Institute, University of Oxford}
\date{\today}


\begin{document}
    \pagenumbering{gobble}
    \maketitle
    
    
    \begin{abstract}
        Magnetic confinement reactors---in particular tokamaks---offer one of the most promising options for achieving practical nuclear fusion, with the potential to provide virtually limitless, clean energy. The theoretical and numerical modeling of tokamak plasmas is simultaneously an essential component of effective reactor design, and a great research barrier. Tokamak operational conditions exhibit comparatively low Knudsen numbers. Kinetic effects, including kinetic waves and instabilities, Landau damping, bump-on-tail instabilities and more, are therefore highly influential in tokamak plasma dynamics. Purely fluid models are inherently incapable of capturing these effects, whereas the high dimensionality in purely kinetic models render them practically intractable for most relevant purposes.

        We consider a $\delta\!f$ decomposition model, with a macroscopic fluid background and microscopic kinetic correction, both fully coupled to each other. A similar manner of discretization is proposed to that used in the recent \texttt{STRUPHY} code \cite{Holderied_Possanner_Wang_2021, Holderied_2022, Li_et_al_2023} with a finite-element model for the background and a pseudo-particle/PiC model for the correction.

        The fluid background satisfies the full, non-linear, resistive, compressible, Hall MHD equations. \cite{Laakmann_Hu_Farrell_2022} introduces finite-element(-in-space) implicit timesteppers for the incompressible analogue to this system with structure-preserving (SP) properties in the ideal case, alongside parameter-robust preconditioners. We show that these timesteppers can derive from a finite-element-in-time (FET) (and finite-element-in-space) interpretation. The benefits of this reformulation are discussed, including the derivation of timesteppers that are higher order in time, and the quantifiable dissipative SP properties in the non-ideal, resistive case.
        
        We discuss possible options for extending this FET approach to timesteppers for the compressible case.

        The kinetic corrections satisfy linearized Boltzmann equations. Using a Lénard--Bernstein collision operator, these take Fokker--Planck-like forms \cite{Fokker_1914, Planck_1917} wherein pseudo-particles in the numerical model obey the neoclassical transport equations, with particle-independent Brownian drift terms. This offers a rigorous methodology for incorporating collisions into the particle transport model, without coupling the equations of motions for each particle.
        
        Works by Chen, Chacón et al. \cite{Chen_Chacón_Barnes_2011, Chacón_Chen_Barnes_2013, Chen_Chacón_2014, Chen_Chacón_2015} have developed structure-preserving particle pushers for neoclassical transport in the Vlasov equations, derived from Crank--Nicolson integrators. We show these too can can derive from a FET interpretation, similarly offering potential extensions to higher-order-in-time particle pushers. The FET formulation is used also to consider how the stochastic drift terms can be incorporated into the pushers. Stochastic gyrokinetic expansions are also discussed.

        Different options for the numerical implementation of these schemes are considered.

        Due to the efficacy of FET in the development of SP timesteppers for both the fluid and kinetic component, we hope this approach will prove effective in the future for developing SP timesteppers for the full hybrid model. We hope this will give us the opportunity to incorporate previously inaccessible kinetic effects into the highly effective, modern, finite-element MHD models.
    \end{abstract}
    
    
    \newpage
    \tableofcontents
    
    
    \newpage
    \pagenumbering{arabic}
    %\linenumbers\renewcommand\thelinenumber{\color{black!50}\arabic{linenumber}}
            \input{0 - introduction/main.tex}
        \part{Research}
            \input{1 - low-noise PiC models/main.tex}
            \input{2 - kinetic component/main.tex}
            \input{3 - fluid component/main.tex}
            \input{4 - numerical implementation/main.tex}
        \part{Project Overview}
            \input{5 - research plan/main.tex}
            \input{6 - summary/main.tex}
    
    
    %\section{}
    \newpage
    \pagenumbering{gobble}
        \printbibliography


    \newpage
    \pagenumbering{roman}
    \appendix
        \part{Appendices}
            \input{8 - Hilbert complexes/main.tex}
            \input{9 - weak conservation proofs/main.tex}
\end{document}

\end{document}

        \part{Research}
            \documentclass[12pt, a4paper]{report}

\documentclass[12pt, a4paper]{report}

\input{template/main.tex}

\title{\BA{Title in Progress...}}
\author{Boris Andrews}
\affil{Mathematical Institute, University of Oxford}
\date{\today}


\begin{document}
    \pagenumbering{gobble}
    \maketitle
    
    
    \begin{abstract}
        Magnetic confinement reactors---in particular tokamaks---offer one of the most promising options for achieving practical nuclear fusion, with the potential to provide virtually limitless, clean energy. The theoretical and numerical modeling of tokamak plasmas is simultaneously an essential component of effective reactor design, and a great research barrier. Tokamak operational conditions exhibit comparatively low Knudsen numbers. Kinetic effects, including kinetic waves and instabilities, Landau damping, bump-on-tail instabilities and more, are therefore highly influential in tokamak plasma dynamics. Purely fluid models are inherently incapable of capturing these effects, whereas the high dimensionality in purely kinetic models render them practically intractable for most relevant purposes.

        We consider a $\delta\!f$ decomposition model, with a macroscopic fluid background and microscopic kinetic correction, both fully coupled to each other. A similar manner of discretization is proposed to that used in the recent \texttt{STRUPHY} code \cite{Holderied_Possanner_Wang_2021, Holderied_2022, Li_et_al_2023} with a finite-element model for the background and a pseudo-particle/PiC model for the correction.

        The fluid background satisfies the full, non-linear, resistive, compressible, Hall MHD equations. \cite{Laakmann_Hu_Farrell_2022} introduces finite-element(-in-space) implicit timesteppers for the incompressible analogue to this system with structure-preserving (SP) properties in the ideal case, alongside parameter-robust preconditioners. We show that these timesteppers can derive from a finite-element-in-time (FET) (and finite-element-in-space) interpretation. The benefits of this reformulation are discussed, including the derivation of timesteppers that are higher order in time, and the quantifiable dissipative SP properties in the non-ideal, resistive case.
        
        We discuss possible options for extending this FET approach to timesteppers for the compressible case.

        The kinetic corrections satisfy linearized Boltzmann equations. Using a Lénard--Bernstein collision operator, these take Fokker--Planck-like forms \cite{Fokker_1914, Planck_1917} wherein pseudo-particles in the numerical model obey the neoclassical transport equations, with particle-independent Brownian drift terms. This offers a rigorous methodology for incorporating collisions into the particle transport model, without coupling the equations of motions for each particle.
        
        Works by Chen, Chacón et al. \cite{Chen_Chacón_Barnes_2011, Chacón_Chen_Barnes_2013, Chen_Chacón_2014, Chen_Chacón_2015} have developed structure-preserving particle pushers for neoclassical transport in the Vlasov equations, derived from Crank--Nicolson integrators. We show these too can can derive from a FET interpretation, similarly offering potential extensions to higher-order-in-time particle pushers. The FET formulation is used also to consider how the stochastic drift terms can be incorporated into the pushers. Stochastic gyrokinetic expansions are also discussed.

        Different options for the numerical implementation of these schemes are considered.

        Due to the efficacy of FET in the development of SP timesteppers for both the fluid and kinetic component, we hope this approach will prove effective in the future for developing SP timesteppers for the full hybrid model. We hope this will give us the opportunity to incorporate previously inaccessible kinetic effects into the highly effective, modern, finite-element MHD models.
    \end{abstract}
    
    
    \newpage
    \tableofcontents
    
    
    \newpage
    \pagenumbering{arabic}
    %\linenumbers\renewcommand\thelinenumber{\color{black!50}\arabic{linenumber}}
            \input{0 - introduction/main.tex}
        \part{Research}
            \input{1 - low-noise PiC models/main.tex}
            \input{2 - kinetic component/main.tex}
            \input{3 - fluid component/main.tex}
            \input{4 - numerical implementation/main.tex}
        \part{Project Overview}
            \input{5 - research plan/main.tex}
            \input{6 - summary/main.tex}
    
    
    %\section{}
    \newpage
    \pagenumbering{gobble}
        \printbibliography


    \newpage
    \pagenumbering{roman}
    \appendix
        \part{Appendices}
            \input{8 - Hilbert complexes/main.tex}
            \input{9 - weak conservation proofs/main.tex}
\end{document}


\title{\BA{Title in Progress...}}
\author{Boris Andrews}
\affil{Mathematical Institute, University of Oxford}
\date{\today}


\begin{document}
    \pagenumbering{gobble}
    \maketitle
    
    
    \begin{abstract}
        Magnetic confinement reactors---in particular tokamaks---offer one of the most promising options for achieving practical nuclear fusion, with the potential to provide virtually limitless, clean energy. The theoretical and numerical modeling of tokamak plasmas is simultaneously an essential component of effective reactor design, and a great research barrier. Tokamak operational conditions exhibit comparatively low Knudsen numbers. Kinetic effects, including kinetic waves and instabilities, Landau damping, bump-on-tail instabilities and more, are therefore highly influential in tokamak plasma dynamics. Purely fluid models are inherently incapable of capturing these effects, whereas the high dimensionality in purely kinetic models render them practically intractable for most relevant purposes.

        We consider a $\delta\!f$ decomposition model, with a macroscopic fluid background and microscopic kinetic correction, both fully coupled to each other. A similar manner of discretization is proposed to that used in the recent \texttt{STRUPHY} code \cite{Holderied_Possanner_Wang_2021, Holderied_2022, Li_et_al_2023} with a finite-element model for the background and a pseudo-particle/PiC model for the correction.

        The fluid background satisfies the full, non-linear, resistive, compressible, Hall MHD equations. \cite{Laakmann_Hu_Farrell_2022} introduces finite-element(-in-space) implicit timesteppers for the incompressible analogue to this system with structure-preserving (SP) properties in the ideal case, alongside parameter-robust preconditioners. We show that these timesteppers can derive from a finite-element-in-time (FET) (and finite-element-in-space) interpretation. The benefits of this reformulation are discussed, including the derivation of timesteppers that are higher order in time, and the quantifiable dissipative SP properties in the non-ideal, resistive case.
        
        We discuss possible options for extending this FET approach to timesteppers for the compressible case.

        The kinetic corrections satisfy linearized Boltzmann equations. Using a Lénard--Bernstein collision operator, these take Fokker--Planck-like forms \cite{Fokker_1914, Planck_1917} wherein pseudo-particles in the numerical model obey the neoclassical transport equations, with particle-independent Brownian drift terms. This offers a rigorous methodology for incorporating collisions into the particle transport model, without coupling the equations of motions for each particle.
        
        Works by Chen, Chacón et al. \cite{Chen_Chacón_Barnes_2011, Chacón_Chen_Barnes_2013, Chen_Chacón_2014, Chen_Chacón_2015} have developed structure-preserving particle pushers for neoclassical transport in the Vlasov equations, derived from Crank--Nicolson integrators. We show these too can can derive from a FET interpretation, similarly offering potential extensions to higher-order-in-time particle pushers. The FET formulation is used also to consider how the stochastic drift terms can be incorporated into the pushers. Stochastic gyrokinetic expansions are also discussed.

        Different options for the numerical implementation of these schemes are considered.

        Due to the efficacy of FET in the development of SP timesteppers for both the fluid and kinetic component, we hope this approach will prove effective in the future for developing SP timesteppers for the full hybrid model. We hope this will give us the opportunity to incorporate previously inaccessible kinetic effects into the highly effective, modern, finite-element MHD models.
    \end{abstract}
    
    
    \newpage
    \tableofcontents
    
    
    \newpage
    \pagenumbering{arabic}
    %\linenumbers\renewcommand\thelinenumber{\color{black!50}\arabic{linenumber}}
            \documentclass[12pt, a4paper]{report}

\input{template/main.tex}

\title{\BA{Title in Progress...}}
\author{Boris Andrews}
\affil{Mathematical Institute, University of Oxford}
\date{\today}


\begin{document}
    \pagenumbering{gobble}
    \maketitle
    
    
    \begin{abstract}
        Magnetic confinement reactors---in particular tokamaks---offer one of the most promising options for achieving practical nuclear fusion, with the potential to provide virtually limitless, clean energy. The theoretical and numerical modeling of tokamak plasmas is simultaneously an essential component of effective reactor design, and a great research barrier. Tokamak operational conditions exhibit comparatively low Knudsen numbers. Kinetic effects, including kinetic waves and instabilities, Landau damping, bump-on-tail instabilities and more, are therefore highly influential in tokamak plasma dynamics. Purely fluid models are inherently incapable of capturing these effects, whereas the high dimensionality in purely kinetic models render them practically intractable for most relevant purposes.

        We consider a $\delta\!f$ decomposition model, with a macroscopic fluid background and microscopic kinetic correction, both fully coupled to each other. A similar manner of discretization is proposed to that used in the recent \texttt{STRUPHY} code \cite{Holderied_Possanner_Wang_2021, Holderied_2022, Li_et_al_2023} with a finite-element model for the background and a pseudo-particle/PiC model for the correction.

        The fluid background satisfies the full, non-linear, resistive, compressible, Hall MHD equations. \cite{Laakmann_Hu_Farrell_2022} introduces finite-element(-in-space) implicit timesteppers for the incompressible analogue to this system with structure-preserving (SP) properties in the ideal case, alongside parameter-robust preconditioners. We show that these timesteppers can derive from a finite-element-in-time (FET) (and finite-element-in-space) interpretation. The benefits of this reformulation are discussed, including the derivation of timesteppers that are higher order in time, and the quantifiable dissipative SP properties in the non-ideal, resistive case.
        
        We discuss possible options for extending this FET approach to timesteppers for the compressible case.

        The kinetic corrections satisfy linearized Boltzmann equations. Using a Lénard--Bernstein collision operator, these take Fokker--Planck-like forms \cite{Fokker_1914, Planck_1917} wherein pseudo-particles in the numerical model obey the neoclassical transport equations, with particle-independent Brownian drift terms. This offers a rigorous methodology for incorporating collisions into the particle transport model, without coupling the equations of motions for each particle.
        
        Works by Chen, Chacón et al. \cite{Chen_Chacón_Barnes_2011, Chacón_Chen_Barnes_2013, Chen_Chacón_2014, Chen_Chacón_2015} have developed structure-preserving particle pushers for neoclassical transport in the Vlasov equations, derived from Crank--Nicolson integrators. We show these too can can derive from a FET interpretation, similarly offering potential extensions to higher-order-in-time particle pushers. The FET formulation is used also to consider how the stochastic drift terms can be incorporated into the pushers. Stochastic gyrokinetic expansions are also discussed.

        Different options for the numerical implementation of these schemes are considered.

        Due to the efficacy of FET in the development of SP timesteppers for both the fluid and kinetic component, we hope this approach will prove effective in the future for developing SP timesteppers for the full hybrid model. We hope this will give us the opportunity to incorporate previously inaccessible kinetic effects into the highly effective, modern, finite-element MHD models.
    \end{abstract}
    
    
    \newpage
    \tableofcontents
    
    
    \newpage
    \pagenumbering{arabic}
    %\linenumbers\renewcommand\thelinenumber{\color{black!50}\arabic{linenumber}}
            \input{0 - introduction/main.tex}
        \part{Research}
            \input{1 - low-noise PiC models/main.tex}
            \input{2 - kinetic component/main.tex}
            \input{3 - fluid component/main.tex}
            \input{4 - numerical implementation/main.tex}
        \part{Project Overview}
            \input{5 - research plan/main.tex}
            \input{6 - summary/main.tex}
    
    
    %\section{}
    \newpage
    \pagenumbering{gobble}
        \printbibliography


    \newpage
    \pagenumbering{roman}
    \appendix
        \part{Appendices}
            \input{8 - Hilbert complexes/main.tex}
            \input{9 - weak conservation proofs/main.tex}
\end{document}

        \part{Research}
            \documentclass[12pt, a4paper]{report}

\input{template/main.tex}

\title{\BA{Title in Progress...}}
\author{Boris Andrews}
\affil{Mathematical Institute, University of Oxford}
\date{\today}


\begin{document}
    \pagenumbering{gobble}
    \maketitle
    
    
    \begin{abstract}
        Magnetic confinement reactors---in particular tokamaks---offer one of the most promising options for achieving practical nuclear fusion, with the potential to provide virtually limitless, clean energy. The theoretical and numerical modeling of tokamak plasmas is simultaneously an essential component of effective reactor design, and a great research barrier. Tokamak operational conditions exhibit comparatively low Knudsen numbers. Kinetic effects, including kinetic waves and instabilities, Landau damping, bump-on-tail instabilities and more, are therefore highly influential in tokamak plasma dynamics. Purely fluid models are inherently incapable of capturing these effects, whereas the high dimensionality in purely kinetic models render them practically intractable for most relevant purposes.

        We consider a $\delta\!f$ decomposition model, with a macroscopic fluid background and microscopic kinetic correction, both fully coupled to each other. A similar manner of discretization is proposed to that used in the recent \texttt{STRUPHY} code \cite{Holderied_Possanner_Wang_2021, Holderied_2022, Li_et_al_2023} with a finite-element model for the background and a pseudo-particle/PiC model for the correction.

        The fluid background satisfies the full, non-linear, resistive, compressible, Hall MHD equations. \cite{Laakmann_Hu_Farrell_2022} introduces finite-element(-in-space) implicit timesteppers for the incompressible analogue to this system with structure-preserving (SP) properties in the ideal case, alongside parameter-robust preconditioners. We show that these timesteppers can derive from a finite-element-in-time (FET) (and finite-element-in-space) interpretation. The benefits of this reformulation are discussed, including the derivation of timesteppers that are higher order in time, and the quantifiable dissipative SP properties in the non-ideal, resistive case.
        
        We discuss possible options for extending this FET approach to timesteppers for the compressible case.

        The kinetic corrections satisfy linearized Boltzmann equations. Using a Lénard--Bernstein collision operator, these take Fokker--Planck-like forms \cite{Fokker_1914, Planck_1917} wherein pseudo-particles in the numerical model obey the neoclassical transport equations, with particle-independent Brownian drift terms. This offers a rigorous methodology for incorporating collisions into the particle transport model, without coupling the equations of motions for each particle.
        
        Works by Chen, Chacón et al. \cite{Chen_Chacón_Barnes_2011, Chacón_Chen_Barnes_2013, Chen_Chacón_2014, Chen_Chacón_2015} have developed structure-preserving particle pushers for neoclassical transport in the Vlasov equations, derived from Crank--Nicolson integrators. We show these too can can derive from a FET interpretation, similarly offering potential extensions to higher-order-in-time particle pushers. The FET formulation is used also to consider how the stochastic drift terms can be incorporated into the pushers. Stochastic gyrokinetic expansions are also discussed.

        Different options for the numerical implementation of these schemes are considered.

        Due to the efficacy of FET in the development of SP timesteppers for both the fluid and kinetic component, we hope this approach will prove effective in the future for developing SP timesteppers for the full hybrid model. We hope this will give us the opportunity to incorporate previously inaccessible kinetic effects into the highly effective, modern, finite-element MHD models.
    \end{abstract}
    
    
    \newpage
    \tableofcontents
    
    
    \newpage
    \pagenumbering{arabic}
    %\linenumbers\renewcommand\thelinenumber{\color{black!50}\arabic{linenumber}}
            \input{0 - introduction/main.tex}
        \part{Research}
            \input{1 - low-noise PiC models/main.tex}
            \input{2 - kinetic component/main.tex}
            \input{3 - fluid component/main.tex}
            \input{4 - numerical implementation/main.tex}
        \part{Project Overview}
            \input{5 - research plan/main.tex}
            \input{6 - summary/main.tex}
    
    
    %\section{}
    \newpage
    \pagenumbering{gobble}
        \printbibliography


    \newpage
    \pagenumbering{roman}
    \appendix
        \part{Appendices}
            \input{8 - Hilbert complexes/main.tex}
            \input{9 - weak conservation proofs/main.tex}
\end{document}

            \documentclass[12pt, a4paper]{report}

\input{template/main.tex}

\title{\BA{Title in Progress...}}
\author{Boris Andrews}
\affil{Mathematical Institute, University of Oxford}
\date{\today}


\begin{document}
    \pagenumbering{gobble}
    \maketitle
    
    
    \begin{abstract}
        Magnetic confinement reactors---in particular tokamaks---offer one of the most promising options for achieving practical nuclear fusion, with the potential to provide virtually limitless, clean energy. The theoretical and numerical modeling of tokamak plasmas is simultaneously an essential component of effective reactor design, and a great research barrier. Tokamak operational conditions exhibit comparatively low Knudsen numbers. Kinetic effects, including kinetic waves and instabilities, Landau damping, bump-on-tail instabilities and more, are therefore highly influential in tokamak plasma dynamics. Purely fluid models are inherently incapable of capturing these effects, whereas the high dimensionality in purely kinetic models render them practically intractable for most relevant purposes.

        We consider a $\delta\!f$ decomposition model, with a macroscopic fluid background and microscopic kinetic correction, both fully coupled to each other. A similar manner of discretization is proposed to that used in the recent \texttt{STRUPHY} code \cite{Holderied_Possanner_Wang_2021, Holderied_2022, Li_et_al_2023} with a finite-element model for the background and a pseudo-particle/PiC model for the correction.

        The fluid background satisfies the full, non-linear, resistive, compressible, Hall MHD equations. \cite{Laakmann_Hu_Farrell_2022} introduces finite-element(-in-space) implicit timesteppers for the incompressible analogue to this system with structure-preserving (SP) properties in the ideal case, alongside parameter-robust preconditioners. We show that these timesteppers can derive from a finite-element-in-time (FET) (and finite-element-in-space) interpretation. The benefits of this reformulation are discussed, including the derivation of timesteppers that are higher order in time, and the quantifiable dissipative SP properties in the non-ideal, resistive case.
        
        We discuss possible options for extending this FET approach to timesteppers for the compressible case.

        The kinetic corrections satisfy linearized Boltzmann equations. Using a Lénard--Bernstein collision operator, these take Fokker--Planck-like forms \cite{Fokker_1914, Planck_1917} wherein pseudo-particles in the numerical model obey the neoclassical transport equations, with particle-independent Brownian drift terms. This offers a rigorous methodology for incorporating collisions into the particle transport model, without coupling the equations of motions for each particle.
        
        Works by Chen, Chacón et al. \cite{Chen_Chacón_Barnes_2011, Chacón_Chen_Barnes_2013, Chen_Chacón_2014, Chen_Chacón_2015} have developed structure-preserving particle pushers for neoclassical transport in the Vlasov equations, derived from Crank--Nicolson integrators. We show these too can can derive from a FET interpretation, similarly offering potential extensions to higher-order-in-time particle pushers. The FET formulation is used also to consider how the stochastic drift terms can be incorporated into the pushers. Stochastic gyrokinetic expansions are also discussed.

        Different options for the numerical implementation of these schemes are considered.

        Due to the efficacy of FET in the development of SP timesteppers for both the fluid and kinetic component, we hope this approach will prove effective in the future for developing SP timesteppers for the full hybrid model. We hope this will give us the opportunity to incorporate previously inaccessible kinetic effects into the highly effective, modern, finite-element MHD models.
    \end{abstract}
    
    
    \newpage
    \tableofcontents
    
    
    \newpage
    \pagenumbering{arabic}
    %\linenumbers\renewcommand\thelinenumber{\color{black!50}\arabic{linenumber}}
            \input{0 - introduction/main.tex}
        \part{Research}
            \input{1 - low-noise PiC models/main.tex}
            \input{2 - kinetic component/main.tex}
            \input{3 - fluid component/main.tex}
            \input{4 - numerical implementation/main.tex}
        \part{Project Overview}
            \input{5 - research plan/main.tex}
            \input{6 - summary/main.tex}
    
    
    %\section{}
    \newpage
    \pagenumbering{gobble}
        \printbibliography


    \newpage
    \pagenumbering{roman}
    \appendix
        \part{Appendices}
            \input{8 - Hilbert complexes/main.tex}
            \input{9 - weak conservation proofs/main.tex}
\end{document}

            \documentclass[12pt, a4paper]{report}

\input{template/main.tex}

\title{\BA{Title in Progress...}}
\author{Boris Andrews}
\affil{Mathematical Institute, University of Oxford}
\date{\today}


\begin{document}
    \pagenumbering{gobble}
    \maketitle
    
    
    \begin{abstract}
        Magnetic confinement reactors---in particular tokamaks---offer one of the most promising options for achieving practical nuclear fusion, with the potential to provide virtually limitless, clean energy. The theoretical and numerical modeling of tokamak plasmas is simultaneously an essential component of effective reactor design, and a great research barrier. Tokamak operational conditions exhibit comparatively low Knudsen numbers. Kinetic effects, including kinetic waves and instabilities, Landau damping, bump-on-tail instabilities and more, are therefore highly influential in tokamak plasma dynamics. Purely fluid models are inherently incapable of capturing these effects, whereas the high dimensionality in purely kinetic models render them practically intractable for most relevant purposes.

        We consider a $\delta\!f$ decomposition model, with a macroscopic fluid background and microscopic kinetic correction, both fully coupled to each other. A similar manner of discretization is proposed to that used in the recent \texttt{STRUPHY} code \cite{Holderied_Possanner_Wang_2021, Holderied_2022, Li_et_al_2023} with a finite-element model for the background and a pseudo-particle/PiC model for the correction.

        The fluid background satisfies the full, non-linear, resistive, compressible, Hall MHD equations. \cite{Laakmann_Hu_Farrell_2022} introduces finite-element(-in-space) implicit timesteppers for the incompressible analogue to this system with structure-preserving (SP) properties in the ideal case, alongside parameter-robust preconditioners. We show that these timesteppers can derive from a finite-element-in-time (FET) (and finite-element-in-space) interpretation. The benefits of this reformulation are discussed, including the derivation of timesteppers that are higher order in time, and the quantifiable dissipative SP properties in the non-ideal, resistive case.
        
        We discuss possible options for extending this FET approach to timesteppers for the compressible case.

        The kinetic corrections satisfy linearized Boltzmann equations. Using a Lénard--Bernstein collision operator, these take Fokker--Planck-like forms \cite{Fokker_1914, Planck_1917} wherein pseudo-particles in the numerical model obey the neoclassical transport equations, with particle-independent Brownian drift terms. This offers a rigorous methodology for incorporating collisions into the particle transport model, without coupling the equations of motions for each particle.
        
        Works by Chen, Chacón et al. \cite{Chen_Chacón_Barnes_2011, Chacón_Chen_Barnes_2013, Chen_Chacón_2014, Chen_Chacón_2015} have developed structure-preserving particle pushers for neoclassical transport in the Vlasov equations, derived from Crank--Nicolson integrators. We show these too can can derive from a FET interpretation, similarly offering potential extensions to higher-order-in-time particle pushers. The FET formulation is used also to consider how the stochastic drift terms can be incorporated into the pushers. Stochastic gyrokinetic expansions are also discussed.

        Different options for the numerical implementation of these schemes are considered.

        Due to the efficacy of FET in the development of SP timesteppers for both the fluid and kinetic component, we hope this approach will prove effective in the future for developing SP timesteppers for the full hybrid model. We hope this will give us the opportunity to incorporate previously inaccessible kinetic effects into the highly effective, modern, finite-element MHD models.
    \end{abstract}
    
    
    \newpage
    \tableofcontents
    
    
    \newpage
    \pagenumbering{arabic}
    %\linenumbers\renewcommand\thelinenumber{\color{black!50}\arabic{linenumber}}
            \input{0 - introduction/main.tex}
        \part{Research}
            \input{1 - low-noise PiC models/main.tex}
            \input{2 - kinetic component/main.tex}
            \input{3 - fluid component/main.tex}
            \input{4 - numerical implementation/main.tex}
        \part{Project Overview}
            \input{5 - research plan/main.tex}
            \input{6 - summary/main.tex}
    
    
    %\section{}
    \newpage
    \pagenumbering{gobble}
        \printbibliography


    \newpage
    \pagenumbering{roman}
    \appendix
        \part{Appendices}
            \input{8 - Hilbert complexes/main.tex}
            \input{9 - weak conservation proofs/main.tex}
\end{document}

            \documentclass[12pt, a4paper]{report}

\input{template/main.tex}

\title{\BA{Title in Progress...}}
\author{Boris Andrews}
\affil{Mathematical Institute, University of Oxford}
\date{\today}


\begin{document}
    \pagenumbering{gobble}
    \maketitle
    
    
    \begin{abstract}
        Magnetic confinement reactors---in particular tokamaks---offer one of the most promising options for achieving practical nuclear fusion, with the potential to provide virtually limitless, clean energy. The theoretical and numerical modeling of tokamak plasmas is simultaneously an essential component of effective reactor design, and a great research barrier. Tokamak operational conditions exhibit comparatively low Knudsen numbers. Kinetic effects, including kinetic waves and instabilities, Landau damping, bump-on-tail instabilities and more, are therefore highly influential in tokamak plasma dynamics. Purely fluid models are inherently incapable of capturing these effects, whereas the high dimensionality in purely kinetic models render them practically intractable for most relevant purposes.

        We consider a $\delta\!f$ decomposition model, with a macroscopic fluid background and microscopic kinetic correction, both fully coupled to each other. A similar manner of discretization is proposed to that used in the recent \texttt{STRUPHY} code \cite{Holderied_Possanner_Wang_2021, Holderied_2022, Li_et_al_2023} with a finite-element model for the background and a pseudo-particle/PiC model for the correction.

        The fluid background satisfies the full, non-linear, resistive, compressible, Hall MHD equations. \cite{Laakmann_Hu_Farrell_2022} introduces finite-element(-in-space) implicit timesteppers for the incompressible analogue to this system with structure-preserving (SP) properties in the ideal case, alongside parameter-robust preconditioners. We show that these timesteppers can derive from a finite-element-in-time (FET) (and finite-element-in-space) interpretation. The benefits of this reformulation are discussed, including the derivation of timesteppers that are higher order in time, and the quantifiable dissipative SP properties in the non-ideal, resistive case.
        
        We discuss possible options for extending this FET approach to timesteppers for the compressible case.

        The kinetic corrections satisfy linearized Boltzmann equations. Using a Lénard--Bernstein collision operator, these take Fokker--Planck-like forms \cite{Fokker_1914, Planck_1917} wherein pseudo-particles in the numerical model obey the neoclassical transport equations, with particle-independent Brownian drift terms. This offers a rigorous methodology for incorporating collisions into the particle transport model, without coupling the equations of motions for each particle.
        
        Works by Chen, Chacón et al. \cite{Chen_Chacón_Barnes_2011, Chacón_Chen_Barnes_2013, Chen_Chacón_2014, Chen_Chacón_2015} have developed structure-preserving particle pushers for neoclassical transport in the Vlasov equations, derived from Crank--Nicolson integrators. We show these too can can derive from a FET interpretation, similarly offering potential extensions to higher-order-in-time particle pushers. The FET formulation is used also to consider how the stochastic drift terms can be incorporated into the pushers. Stochastic gyrokinetic expansions are also discussed.

        Different options for the numerical implementation of these schemes are considered.

        Due to the efficacy of FET in the development of SP timesteppers for both the fluid and kinetic component, we hope this approach will prove effective in the future for developing SP timesteppers for the full hybrid model. We hope this will give us the opportunity to incorporate previously inaccessible kinetic effects into the highly effective, modern, finite-element MHD models.
    \end{abstract}
    
    
    \newpage
    \tableofcontents
    
    
    \newpage
    \pagenumbering{arabic}
    %\linenumbers\renewcommand\thelinenumber{\color{black!50}\arabic{linenumber}}
            \input{0 - introduction/main.tex}
        \part{Research}
            \input{1 - low-noise PiC models/main.tex}
            \input{2 - kinetic component/main.tex}
            \input{3 - fluid component/main.tex}
            \input{4 - numerical implementation/main.tex}
        \part{Project Overview}
            \input{5 - research plan/main.tex}
            \input{6 - summary/main.tex}
    
    
    %\section{}
    \newpage
    \pagenumbering{gobble}
        \printbibliography


    \newpage
    \pagenumbering{roman}
    \appendix
        \part{Appendices}
            \input{8 - Hilbert complexes/main.tex}
            \input{9 - weak conservation proofs/main.tex}
\end{document}

        \part{Project Overview}
            \documentclass[12pt, a4paper]{report}

\input{template/main.tex}

\title{\BA{Title in Progress...}}
\author{Boris Andrews}
\affil{Mathematical Institute, University of Oxford}
\date{\today}


\begin{document}
    \pagenumbering{gobble}
    \maketitle
    
    
    \begin{abstract}
        Magnetic confinement reactors---in particular tokamaks---offer one of the most promising options for achieving practical nuclear fusion, with the potential to provide virtually limitless, clean energy. The theoretical and numerical modeling of tokamak plasmas is simultaneously an essential component of effective reactor design, and a great research barrier. Tokamak operational conditions exhibit comparatively low Knudsen numbers. Kinetic effects, including kinetic waves and instabilities, Landau damping, bump-on-tail instabilities and more, are therefore highly influential in tokamak plasma dynamics. Purely fluid models are inherently incapable of capturing these effects, whereas the high dimensionality in purely kinetic models render them practically intractable for most relevant purposes.

        We consider a $\delta\!f$ decomposition model, with a macroscopic fluid background and microscopic kinetic correction, both fully coupled to each other. A similar manner of discretization is proposed to that used in the recent \texttt{STRUPHY} code \cite{Holderied_Possanner_Wang_2021, Holderied_2022, Li_et_al_2023} with a finite-element model for the background and a pseudo-particle/PiC model for the correction.

        The fluid background satisfies the full, non-linear, resistive, compressible, Hall MHD equations. \cite{Laakmann_Hu_Farrell_2022} introduces finite-element(-in-space) implicit timesteppers for the incompressible analogue to this system with structure-preserving (SP) properties in the ideal case, alongside parameter-robust preconditioners. We show that these timesteppers can derive from a finite-element-in-time (FET) (and finite-element-in-space) interpretation. The benefits of this reformulation are discussed, including the derivation of timesteppers that are higher order in time, and the quantifiable dissipative SP properties in the non-ideal, resistive case.
        
        We discuss possible options for extending this FET approach to timesteppers for the compressible case.

        The kinetic corrections satisfy linearized Boltzmann equations. Using a Lénard--Bernstein collision operator, these take Fokker--Planck-like forms \cite{Fokker_1914, Planck_1917} wherein pseudo-particles in the numerical model obey the neoclassical transport equations, with particle-independent Brownian drift terms. This offers a rigorous methodology for incorporating collisions into the particle transport model, without coupling the equations of motions for each particle.
        
        Works by Chen, Chacón et al. \cite{Chen_Chacón_Barnes_2011, Chacón_Chen_Barnes_2013, Chen_Chacón_2014, Chen_Chacón_2015} have developed structure-preserving particle pushers for neoclassical transport in the Vlasov equations, derived from Crank--Nicolson integrators. We show these too can can derive from a FET interpretation, similarly offering potential extensions to higher-order-in-time particle pushers. The FET formulation is used also to consider how the stochastic drift terms can be incorporated into the pushers. Stochastic gyrokinetic expansions are also discussed.

        Different options for the numerical implementation of these schemes are considered.

        Due to the efficacy of FET in the development of SP timesteppers for both the fluid and kinetic component, we hope this approach will prove effective in the future for developing SP timesteppers for the full hybrid model. We hope this will give us the opportunity to incorporate previously inaccessible kinetic effects into the highly effective, modern, finite-element MHD models.
    \end{abstract}
    
    
    \newpage
    \tableofcontents
    
    
    \newpage
    \pagenumbering{arabic}
    %\linenumbers\renewcommand\thelinenumber{\color{black!50}\arabic{linenumber}}
            \input{0 - introduction/main.tex}
        \part{Research}
            \input{1 - low-noise PiC models/main.tex}
            \input{2 - kinetic component/main.tex}
            \input{3 - fluid component/main.tex}
            \input{4 - numerical implementation/main.tex}
        \part{Project Overview}
            \input{5 - research plan/main.tex}
            \input{6 - summary/main.tex}
    
    
    %\section{}
    \newpage
    \pagenumbering{gobble}
        \printbibliography


    \newpage
    \pagenumbering{roman}
    \appendix
        \part{Appendices}
            \input{8 - Hilbert complexes/main.tex}
            \input{9 - weak conservation proofs/main.tex}
\end{document}

            \documentclass[12pt, a4paper]{report}

\input{template/main.tex}

\title{\BA{Title in Progress...}}
\author{Boris Andrews}
\affil{Mathematical Institute, University of Oxford}
\date{\today}


\begin{document}
    \pagenumbering{gobble}
    \maketitle
    
    
    \begin{abstract}
        Magnetic confinement reactors---in particular tokamaks---offer one of the most promising options for achieving practical nuclear fusion, with the potential to provide virtually limitless, clean energy. The theoretical and numerical modeling of tokamak plasmas is simultaneously an essential component of effective reactor design, and a great research barrier. Tokamak operational conditions exhibit comparatively low Knudsen numbers. Kinetic effects, including kinetic waves and instabilities, Landau damping, bump-on-tail instabilities and more, are therefore highly influential in tokamak plasma dynamics. Purely fluid models are inherently incapable of capturing these effects, whereas the high dimensionality in purely kinetic models render them practically intractable for most relevant purposes.

        We consider a $\delta\!f$ decomposition model, with a macroscopic fluid background and microscopic kinetic correction, both fully coupled to each other. A similar manner of discretization is proposed to that used in the recent \texttt{STRUPHY} code \cite{Holderied_Possanner_Wang_2021, Holderied_2022, Li_et_al_2023} with a finite-element model for the background and a pseudo-particle/PiC model for the correction.

        The fluid background satisfies the full, non-linear, resistive, compressible, Hall MHD equations. \cite{Laakmann_Hu_Farrell_2022} introduces finite-element(-in-space) implicit timesteppers for the incompressible analogue to this system with structure-preserving (SP) properties in the ideal case, alongside parameter-robust preconditioners. We show that these timesteppers can derive from a finite-element-in-time (FET) (and finite-element-in-space) interpretation. The benefits of this reformulation are discussed, including the derivation of timesteppers that are higher order in time, and the quantifiable dissipative SP properties in the non-ideal, resistive case.
        
        We discuss possible options for extending this FET approach to timesteppers for the compressible case.

        The kinetic corrections satisfy linearized Boltzmann equations. Using a Lénard--Bernstein collision operator, these take Fokker--Planck-like forms \cite{Fokker_1914, Planck_1917} wherein pseudo-particles in the numerical model obey the neoclassical transport equations, with particle-independent Brownian drift terms. This offers a rigorous methodology for incorporating collisions into the particle transport model, without coupling the equations of motions for each particle.
        
        Works by Chen, Chacón et al. \cite{Chen_Chacón_Barnes_2011, Chacón_Chen_Barnes_2013, Chen_Chacón_2014, Chen_Chacón_2015} have developed structure-preserving particle pushers for neoclassical transport in the Vlasov equations, derived from Crank--Nicolson integrators. We show these too can can derive from a FET interpretation, similarly offering potential extensions to higher-order-in-time particle pushers. The FET formulation is used also to consider how the stochastic drift terms can be incorporated into the pushers. Stochastic gyrokinetic expansions are also discussed.

        Different options for the numerical implementation of these schemes are considered.

        Due to the efficacy of FET in the development of SP timesteppers for both the fluid and kinetic component, we hope this approach will prove effective in the future for developing SP timesteppers for the full hybrid model. We hope this will give us the opportunity to incorporate previously inaccessible kinetic effects into the highly effective, modern, finite-element MHD models.
    \end{abstract}
    
    
    \newpage
    \tableofcontents
    
    
    \newpage
    \pagenumbering{arabic}
    %\linenumbers\renewcommand\thelinenumber{\color{black!50}\arabic{linenumber}}
            \input{0 - introduction/main.tex}
        \part{Research}
            \input{1 - low-noise PiC models/main.tex}
            \input{2 - kinetic component/main.tex}
            \input{3 - fluid component/main.tex}
            \input{4 - numerical implementation/main.tex}
        \part{Project Overview}
            \input{5 - research plan/main.tex}
            \input{6 - summary/main.tex}
    
    
    %\section{}
    \newpage
    \pagenumbering{gobble}
        \printbibliography


    \newpage
    \pagenumbering{roman}
    \appendix
        \part{Appendices}
            \input{8 - Hilbert complexes/main.tex}
            \input{9 - weak conservation proofs/main.tex}
\end{document}

    
    
    %\section{}
    \newpage
    \pagenumbering{gobble}
        \printbibliography


    \newpage
    \pagenumbering{roman}
    \appendix
        \part{Appendices}
            \documentclass[12pt, a4paper]{report}

\input{template/main.tex}

\title{\BA{Title in Progress...}}
\author{Boris Andrews}
\affil{Mathematical Institute, University of Oxford}
\date{\today}


\begin{document}
    \pagenumbering{gobble}
    \maketitle
    
    
    \begin{abstract}
        Magnetic confinement reactors---in particular tokamaks---offer one of the most promising options for achieving practical nuclear fusion, with the potential to provide virtually limitless, clean energy. The theoretical and numerical modeling of tokamak plasmas is simultaneously an essential component of effective reactor design, and a great research barrier. Tokamak operational conditions exhibit comparatively low Knudsen numbers. Kinetic effects, including kinetic waves and instabilities, Landau damping, bump-on-tail instabilities and more, are therefore highly influential in tokamak plasma dynamics. Purely fluid models are inherently incapable of capturing these effects, whereas the high dimensionality in purely kinetic models render them practically intractable for most relevant purposes.

        We consider a $\delta\!f$ decomposition model, with a macroscopic fluid background and microscopic kinetic correction, both fully coupled to each other. A similar manner of discretization is proposed to that used in the recent \texttt{STRUPHY} code \cite{Holderied_Possanner_Wang_2021, Holderied_2022, Li_et_al_2023} with a finite-element model for the background and a pseudo-particle/PiC model for the correction.

        The fluid background satisfies the full, non-linear, resistive, compressible, Hall MHD equations. \cite{Laakmann_Hu_Farrell_2022} introduces finite-element(-in-space) implicit timesteppers for the incompressible analogue to this system with structure-preserving (SP) properties in the ideal case, alongside parameter-robust preconditioners. We show that these timesteppers can derive from a finite-element-in-time (FET) (and finite-element-in-space) interpretation. The benefits of this reformulation are discussed, including the derivation of timesteppers that are higher order in time, and the quantifiable dissipative SP properties in the non-ideal, resistive case.
        
        We discuss possible options for extending this FET approach to timesteppers for the compressible case.

        The kinetic corrections satisfy linearized Boltzmann equations. Using a Lénard--Bernstein collision operator, these take Fokker--Planck-like forms \cite{Fokker_1914, Planck_1917} wherein pseudo-particles in the numerical model obey the neoclassical transport equations, with particle-independent Brownian drift terms. This offers a rigorous methodology for incorporating collisions into the particle transport model, without coupling the equations of motions for each particle.
        
        Works by Chen, Chacón et al. \cite{Chen_Chacón_Barnes_2011, Chacón_Chen_Barnes_2013, Chen_Chacón_2014, Chen_Chacón_2015} have developed structure-preserving particle pushers for neoclassical transport in the Vlasov equations, derived from Crank--Nicolson integrators. We show these too can can derive from a FET interpretation, similarly offering potential extensions to higher-order-in-time particle pushers. The FET formulation is used also to consider how the stochastic drift terms can be incorporated into the pushers. Stochastic gyrokinetic expansions are also discussed.

        Different options for the numerical implementation of these schemes are considered.

        Due to the efficacy of FET in the development of SP timesteppers for both the fluid and kinetic component, we hope this approach will prove effective in the future for developing SP timesteppers for the full hybrid model. We hope this will give us the opportunity to incorporate previously inaccessible kinetic effects into the highly effective, modern, finite-element MHD models.
    \end{abstract}
    
    
    \newpage
    \tableofcontents
    
    
    \newpage
    \pagenumbering{arabic}
    %\linenumbers\renewcommand\thelinenumber{\color{black!50}\arabic{linenumber}}
            \input{0 - introduction/main.tex}
        \part{Research}
            \input{1 - low-noise PiC models/main.tex}
            \input{2 - kinetic component/main.tex}
            \input{3 - fluid component/main.tex}
            \input{4 - numerical implementation/main.tex}
        \part{Project Overview}
            \input{5 - research plan/main.tex}
            \input{6 - summary/main.tex}
    
    
    %\section{}
    \newpage
    \pagenumbering{gobble}
        \printbibliography


    \newpage
    \pagenumbering{roman}
    \appendix
        \part{Appendices}
            \input{8 - Hilbert complexes/main.tex}
            \input{9 - weak conservation proofs/main.tex}
\end{document}

            \documentclass[12pt, a4paper]{report}

\input{template/main.tex}

\title{\BA{Title in Progress...}}
\author{Boris Andrews}
\affil{Mathematical Institute, University of Oxford}
\date{\today}


\begin{document}
    \pagenumbering{gobble}
    \maketitle
    
    
    \begin{abstract}
        Magnetic confinement reactors---in particular tokamaks---offer one of the most promising options for achieving practical nuclear fusion, with the potential to provide virtually limitless, clean energy. The theoretical and numerical modeling of tokamak plasmas is simultaneously an essential component of effective reactor design, and a great research barrier. Tokamak operational conditions exhibit comparatively low Knudsen numbers. Kinetic effects, including kinetic waves and instabilities, Landau damping, bump-on-tail instabilities and more, are therefore highly influential in tokamak plasma dynamics. Purely fluid models are inherently incapable of capturing these effects, whereas the high dimensionality in purely kinetic models render them practically intractable for most relevant purposes.

        We consider a $\delta\!f$ decomposition model, with a macroscopic fluid background and microscopic kinetic correction, both fully coupled to each other. A similar manner of discretization is proposed to that used in the recent \texttt{STRUPHY} code \cite{Holderied_Possanner_Wang_2021, Holderied_2022, Li_et_al_2023} with a finite-element model for the background and a pseudo-particle/PiC model for the correction.

        The fluid background satisfies the full, non-linear, resistive, compressible, Hall MHD equations. \cite{Laakmann_Hu_Farrell_2022} introduces finite-element(-in-space) implicit timesteppers for the incompressible analogue to this system with structure-preserving (SP) properties in the ideal case, alongside parameter-robust preconditioners. We show that these timesteppers can derive from a finite-element-in-time (FET) (and finite-element-in-space) interpretation. The benefits of this reformulation are discussed, including the derivation of timesteppers that are higher order in time, and the quantifiable dissipative SP properties in the non-ideal, resistive case.
        
        We discuss possible options for extending this FET approach to timesteppers for the compressible case.

        The kinetic corrections satisfy linearized Boltzmann equations. Using a Lénard--Bernstein collision operator, these take Fokker--Planck-like forms \cite{Fokker_1914, Planck_1917} wherein pseudo-particles in the numerical model obey the neoclassical transport equations, with particle-independent Brownian drift terms. This offers a rigorous methodology for incorporating collisions into the particle transport model, without coupling the equations of motions for each particle.
        
        Works by Chen, Chacón et al. \cite{Chen_Chacón_Barnes_2011, Chacón_Chen_Barnes_2013, Chen_Chacón_2014, Chen_Chacón_2015} have developed structure-preserving particle pushers for neoclassical transport in the Vlasov equations, derived from Crank--Nicolson integrators. We show these too can can derive from a FET interpretation, similarly offering potential extensions to higher-order-in-time particle pushers. The FET formulation is used also to consider how the stochastic drift terms can be incorporated into the pushers. Stochastic gyrokinetic expansions are also discussed.

        Different options for the numerical implementation of these schemes are considered.

        Due to the efficacy of FET in the development of SP timesteppers for both the fluid and kinetic component, we hope this approach will prove effective in the future for developing SP timesteppers for the full hybrid model. We hope this will give us the opportunity to incorporate previously inaccessible kinetic effects into the highly effective, modern, finite-element MHD models.
    \end{abstract}
    
    
    \newpage
    \tableofcontents
    
    
    \newpage
    \pagenumbering{arabic}
    %\linenumbers\renewcommand\thelinenumber{\color{black!50}\arabic{linenumber}}
            \input{0 - introduction/main.tex}
        \part{Research}
            \input{1 - low-noise PiC models/main.tex}
            \input{2 - kinetic component/main.tex}
            \input{3 - fluid component/main.tex}
            \input{4 - numerical implementation/main.tex}
        \part{Project Overview}
            \input{5 - research plan/main.tex}
            \input{6 - summary/main.tex}
    
    
    %\section{}
    \newpage
    \pagenumbering{gobble}
        \printbibliography


    \newpage
    \pagenumbering{roman}
    \appendix
        \part{Appendices}
            \input{8 - Hilbert complexes/main.tex}
            \input{9 - weak conservation proofs/main.tex}
\end{document}

\end{document}

            \documentclass[12pt, a4paper]{report}

\documentclass[12pt, a4paper]{report}

\input{template/main.tex}

\title{\BA{Title in Progress...}}
\author{Boris Andrews}
\affil{Mathematical Institute, University of Oxford}
\date{\today}


\begin{document}
    \pagenumbering{gobble}
    \maketitle
    
    
    \begin{abstract}
        Magnetic confinement reactors---in particular tokamaks---offer one of the most promising options for achieving practical nuclear fusion, with the potential to provide virtually limitless, clean energy. The theoretical and numerical modeling of tokamak plasmas is simultaneously an essential component of effective reactor design, and a great research barrier. Tokamak operational conditions exhibit comparatively low Knudsen numbers. Kinetic effects, including kinetic waves and instabilities, Landau damping, bump-on-tail instabilities and more, are therefore highly influential in tokamak plasma dynamics. Purely fluid models are inherently incapable of capturing these effects, whereas the high dimensionality in purely kinetic models render them practically intractable for most relevant purposes.

        We consider a $\delta\!f$ decomposition model, with a macroscopic fluid background and microscopic kinetic correction, both fully coupled to each other. A similar manner of discretization is proposed to that used in the recent \texttt{STRUPHY} code \cite{Holderied_Possanner_Wang_2021, Holderied_2022, Li_et_al_2023} with a finite-element model for the background and a pseudo-particle/PiC model for the correction.

        The fluid background satisfies the full, non-linear, resistive, compressible, Hall MHD equations. \cite{Laakmann_Hu_Farrell_2022} introduces finite-element(-in-space) implicit timesteppers for the incompressible analogue to this system with structure-preserving (SP) properties in the ideal case, alongside parameter-robust preconditioners. We show that these timesteppers can derive from a finite-element-in-time (FET) (and finite-element-in-space) interpretation. The benefits of this reformulation are discussed, including the derivation of timesteppers that are higher order in time, and the quantifiable dissipative SP properties in the non-ideal, resistive case.
        
        We discuss possible options for extending this FET approach to timesteppers for the compressible case.

        The kinetic corrections satisfy linearized Boltzmann equations. Using a Lénard--Bernstein collision operator, these take Fokker--Planck-like forms \cite{Fokker_1914, Planck_1917} wherein pseudo-particles in the numerical model obey the neoclassical transport equations, with particle-independent Brownian drift terms. This offers a rigorous methodology for incorporating collisions into the particle transport model, without coupling the equations of motions for each particle.
        
        Works by Chen, Chacón et al. \cite{Chen_Chacón_Barnes_2011, Chacón_Chen_Barnes_2013, Chen_Chacón_2014, Chen_Chacón_2015} have developed structure-preserving particle pushers for neoclassical transport in the Vlasov equations, derived from Crank--Nicolson integrators. We show these too can can derive from a FET interpretation, similarly offering potential extensions to higher-order-in-time particle pushers. The FET formulation is used also to consider how the stochastic drift terms can be incorporated into the pushers. Stochastic gyrokinetic expansions are also discussed.

        Different options for the numerical implementation of these schemes are considered.

        Due to the efficacy of FET in the development of SP timesteppers for both the fluid and kinetic component, we hope this approach will prove effective in the future for developing SP timesteppers for the full hybrid model. We hope this will give us the opportunity to incorporate previously inaccessible kinetic effects into the highly effective, modern, finite-element MHD models.
    \end{abstract}
    
    
    \newpage
    \tableofcontents
    
    
    \newpage
    \pagenumbering{arabic}
    %\linenumbers\renewcommand\thelinenumber{\color{black!50}\arabic{linenumber}}
            \input{0 - introduction/main.tex}
        \part{Research}
            \input{1 - low-noise PiC models/main.tex}
            \input{2 - kinetic component/main.tex}
            \input{3 - fluid component/main.tex}
            \input{4 - numerical implementation/main.tex}
        \part{Project Overview}
            \input{5 - research plan/main.tex}
            \input{6 - summary/main.tex}
    
    
    %\section{}
    \newpage
    \pagenumbering{gobble}
        \printbibliography


    \newpage
    \pagenumbering{roman}
    \appendix
        \part{Appendices}
            \input{8 - Hilbert complexes/main.tex}
            \input{9 - weak conservation proofs/main.tex}
\end{document}


\title{\BA{Title in Progress...}}
\author{Boris Andrews}
\affil{Mathematical Institute, University of Oxford}
\date{\today}


\begin{document}
    \pagenumbering{gobble}
    \maketitle
    
    
    \begin{abstract}
        Magnetic confinement reactors---in particular tokamaks---offer one of the most promising options for achieving practical nuclear fusion, with the potential to provide virtually limitless, clean energy. The theoretical and numerical modeling of tokamak plasmas is simultaneously an essential component of effective reactor design, and a great research barrier. Tokamak operational conditions exhibit comparatively low Knudsen numbers. Kinetic effects, including kinetic waves and instabilities, Landau damping, bump-on-tail instabilities and more, are therefore highly influential in tokamak plasma dynamics. Purely fluid models are inherently incapable of capturing these effects, whereas the high dimensionality in purely kinetic models render them practically intractable for most relevant purposes.

        We consider a $\delta\!f$ decomposition model, with a macroscopic fluid background and microscopic kinetic correction, both fully coupled to each other. A similar manner of discretization is proposed to that used in the recent \texttt{STRUPHY} code \cite{Holderied_Possanner_Wang_2021, Holderied_2022, Li_et_al_2023} with a finite-element model for the background and a pseudo-particle/PiC model for the correction.

        The fluid background satisfies the full, non-linear, resistive, compressible, Hall MHD equations. \cite{Laakmann_Hu_Farrell_2022} introduces finite-element(-in-space) implicit timesteppers for the incompressible analogue to this system with structure-preserving (SP) properties in the ideal case, alongside parameter-robust preconditioners. We show that these timesteppers can derive from a finite-element-in-time (FET) (and finite-element-in-space) interpretation. The benefits of this reformulation are discussed, including the derivation of timesteppers that are higher order in time, and the quantifiable dissipative SP properties in the non-ideal, resistive case.
        
        We discuss possible options for extending this FET approach to timesteppers for the compressible case.

        The kinetic corrections satisfy linearized Boltzmann equations. Using a Lénard--Bernstein collision operator, these take Fokker--Planck-like forms \cite{Fokker_1914, Planck_1917} wherein pseudo-particles in the numerical model obey the neoclassical transport equations, with particle-independent Brownian drift terms. This offers a rigorous methodology for incorporating collisions into the particle transport model, without coupling the equations of motions for each particle.
        
        Works by Chen, Chacón et al. \cite{Chen_Chacón_Barnes_2011, Chacón_Chen_Barnes_2013, Chen_Chacón_2014, Chen_Chacón_2015} have developed structure-preserving particle pushers for neoclassical transport in the Vlasov equations, derived from Crank--Nicolson integrators. We show these too can can derive from a FET interpretation, similarly offering potential extensions to higher-order-in-time particle pushers. The FET formulation is used also to consider how the stochastic drift terms can be incorporated into the pushers. Stochastic gyrokinetic expansions are also discussed.

        Different options for the numerical implementation of these schemes are considered.

        Due to the efficacy of FET in the development of SP timesteppers for both the fluid and kinetic component, we hope this approach will prove effective in the future for developing SP timesteppers for the full hybrid model. We hope this will give us the opportunity to incorporate previously inaccessible kinetic effects into the highly effective, modern, finite-element MHD models.
    \end{abstract}
    
    
    \newpage
    \tableofcontents
    
    
    \newpage
    \pagenumbering{arabic}
    %\linenumbers\renewcommand\thelinenumber{\color{black!50}\arabic{linenumber}}
            \documentclass[12pt, a4paper]{report}

\input{template/main.tex}

\title{\BA{Title in Progress...}}
\author{Boris Andrews}
\affil{Mathematical Institute, University of Oxford}
\date{\today}


\begin{document}
    \pagenumbering{gobble}
    \maketitle
    
    
    \begin{abstract}
        Magnetic confinement reactors---in particular tokamaks---offer one of the most promising options for achieving practical nuclear fusion, with the potential to provide virtually limitless, clean energy. The theoretical and numerical modeling of tokamak plasmas is simultaneously an essential component of effective reactor design, and a great research barrier. Tokamak operational conditions exhibit comparatively low Knudsen numbers. Kinetic effects, including kinetic waves and instabilities, Landau damping, bump-on-tail instabilities and more, are therefore highly influential in tokamak plasma dynamics. Purely fluid models are inherently incapable of capturing these effects, whereas the high dimensionality in purely kinetic models render them practically intractable for most relevant purposes.

        We consider a $\delta\!f$ decomposition model, with a macroscopic fluid background and microscopic kinetic correction, both fully coupled to each other. A similar manner of discretization is proposed to that used in the recent \texttt{STRUPHY} code \cite{Holderied_Possanner_Wang_2021, Holderied_2022, Li_et_al_2023} with a finite-element model for the background and a pseudo-particle/PiC model for the correction.

        The fluid background satisfies the full, non-linear, resistive, compressible, Hall MHD equations. \cite{Laakmann_Hu_Farrell_2022} introduces finite-element(-in-space) implicit timesteppers for the incompressible analogue to this system with structure-preserving (SP) properties in the ideal case, alongside parameter-robust preconditioners. We show that these timesteppers can derive from a finite-element-in-time (FET) (and finite-element-in-space) interpretation. The benefits of this reformulation are discussed, including the derivation of timesteppers that are higher order in time, and the quantifiable dissipative SP properties in the non-ideal, resistive case.
        
        We discuss possible options for extending this FET approach to timesteppers for the compressible case.

        The kinetic corrections satisfy linearized Boltzmann equations. Using a Lénard--Bernstein collision operator, these take Fokker--Planck-like forms \cite{Fokker_1914, Planck_1917} wherein pseudo-particles in the numerical model obey the neoclassical transport equations, with particle-independent Brownian drift terms. This offers a rigorous methodology for incorporating collisions into the particle transport model, without coupling the equations of motions for each particle.
        
        Works by Chen, Chacón et al. \cite{Chen_Chacón_Barnes_2011, Chacón_Chen_Barnes_2013, Chen_Chacón_2014, Chen_Chacón_2015} have developed structure-preserving particle pushers for neoclassical transport in the Vlasov equations, derived from Crank--Nicolson integrators. We show these too can can derive from a FET interpretation, similarly offering potential extensions to higher-order-in-time particle pushers. The FET formulation is used also to consider how the stochastic drift terms can be incorporated into the pushers. Stochastic gyrokinetic expansions are also discussed.

        Different options for the numerical implementation of these schemes are considered.

        Due to the efficacy of FET in the development of SP timesteppers for both the fluid and kinetic component, we hope this approach will prove effective in the future for developing SP timesteppers for the full hybrid model. We hope this will give us the opportunity to incorporate previously inaccessible kinetic effects into the highly effective, modern, finite-element MHD models.
    \end{abstract}
    
    
    \newpage
    \tableofcontents
    
    
    \newpage
    \pagenumbering{arabic}
    %\linenumbers\renewcommand\thelinenumber{\color{black!50}\arabic{linenumber}}
            \input{0 - introduction/main.tex}
        \part{Research}
            \input{1 - low-noise PiC models/main.tex}
            \input{2 - kinetic component/main.tex}
            \input{3 - fluid component/main.tex}
            \input{4 - numerical implementation/main.tex}
        \part{Project Overview}
            \input{5 - research plan/main.tex}
            \input{6 - summary/main.tex}
    
    
    %\section{}
    \newpage
    \pagenumbering{gobble}
        \printbibliography


    \newpage
    \pagenumbering{roman}
    \appendix
        \part{Appendices}
            \input{8 - Hilbert complexes/main.tex}
            \input{9 - weak conservation proofs/main.tex}
\end{document}

        \part{Research}
            \documentclass[12pt, a4paper]{report}

\input{template/main.tex}

\title{\BA{Title in Progress...}}
\author{Boris Andrews}
\affil{Mathematical Institute, University of Oxford}
\date{\today}


\begin{document}
    \pagenumbering{gobble}
    \maketitle
    
    
    \begin{abstract}
        Magnetic confinement reactors---in particular tokamaks---offer one of the most promising options for achieving practical nuclear fusion, with the potential to provide virtually limitless, clean energy. The theoretical and numerical modeling of tokamak plasmas is simultaneously an essential component of effective reactor design, and a great research barrier. Tokamak operational conditions exhibit comparatively low Knudsen numbers. Kinetic effects, including kinetic waves and instabilities, Landau damping, bump-on-tail instabilities and more, are therefore highly influential in tokamak plasma dynamics. Purely fluid models are inherently incapable of capturing these effects, whereas the high dimensionality in purely kinetic models render them practically intractable for most relevant purposes.

        We consider a $\delta\!f$ decomposition model, with a macroscopic fluid background and microscopic kinetic correction, both fully coupled to each other. A similar manner of discretization is proposed to that used in the recent \texttt{STRUPHY} code \cite{Holderied_Possanner_Wang_2021, Holderied_2022, Li_et_al_2023} with a finite-element model for the background and a pseudo-particle/PiC model for the correction.

        The fluid background satisfies the full, non-linear, resistive, compressible, Hall MHD equations. \cite{Laakmann_Hu_Farrell_2022} introduces finite-element(-in-space) implicit timesteppers for the incompressible analogue to this system with structure-preserving (SP) properties in the ideal case, alongside parameter-robust preconditioners. We show that these timesteppers can derive from a finite-element-in-time (FET) (and finite-element-in-space) interpretation. The benefits of this reformulation are discussed, including the derivation of timesteppers that are higher order in time, and the quantifiable dissipative SP properties in the non-ideal, resistive case.
        
        We discuss possible options for extending this FET approach to timesteppers for the compressible case.

        The kinetic corrections satisfy linearized Boltzmann equations. Using a Lénard--Bernstein collision operator, these take Fokker--Planck-like forms \cite{Fokker_1914, Planck_1917} wherein pseudo-particles in the numerical model obey the neoclassical transport equations, with particle-independent Brownian drift terms. This offers a rigorous methodology for incorporating collisions into the particle transport model, without coupling the equations of motions for each particle.
        
        Works by Chen, Chacón et al. \cite{Chen_Chacón_Barnes_2011, Chacón_Chen_Barnes_2013, Chen_Chacón_2014, Chen_Chacón_2015} have developed structure-preserving particle pushers for neoclassical transport in the Vlasov equations, derived from Crank--Nicolson integrators. We show these too can can derive from a FET interpretation, similarly offering potential extensions to higher-order-in-time particle pushers. The FET formulation is used also to consider how the stochastic drift terms can be incorporated into the pushers. Stochastic gyrokinetic expansions are also discussed.

        Different options for the numerical implementation of these schemes are considered.

        Due to the efficacy of FET in the development of SP timesteppers for both the fluid and kinetic component, we hope this approach will prove effective in the future for developing SP timesteppers for the full hybrid model. We hope this will give us the opportunity to incorporate previously inaccessible kinetic effects into the highly effective, modern, finite-element MHD models.
    \end{abstract}
    
    
    \newpage
    \tableofcontents
    
    
    \newpage
    \pagenumbering{arabic}
    %\linenumbers\renewcommand\thelinenumber{\color{black!50}\arabic{linenumber}}
            \input{0 - introduction/main.tex}
        \part{Research}
            \input{1 - low-noise PiC models/main.tex}
            \input{2 - kinetic component/main.tex}
            \input{3 - fluid component/main.tex}
            \input{4 - numerical implementation/main.tex}
        \part{Project Overview}
            \input{5 - research plan/main.tex}
            \input{6 - summary/main.tex}
    
    
    %\section{}
    \newpage
    \pagenumbering{gobble}
        \printbibliography


    \newpage
    \pagenumbering{roman}
    \appendix
        \part{Appendices}
            \input{8 - Hilbert complexes/main.tex}
            \input{9 - weak conservation proofs/main.tex}
\end{document}

            \documentclass[12pt, a4paper]{report}

\input{template/main.tex}

\title{\BA{Title in Progress...}}
\author{Boris Andrews}
\affil{Mathematical Institute, University of Oxford}
\date{\today}


\begin{document}
    \pagenumbering{gobble}
    \maketitle
    
    
    \begin{abstract}
        Magnetic confinement reactors---in particular tokamaks---offer one of the most promising options for achieving practical nuclear fusion, with the potential to provide virtually limitless, clean energy. The theoretical and numerical modeling of tokamak plasmas is simultaneously an essential component of effective reactor design, and a great research barrier. Tokamak operational conditions exhibit comparatively low Knudsen numbers. Kinetic effects, including kinetic waves and instabilities, Landau damping, bump-on-tail instabilities and more, are therefore highly influential in tokamak plasma dynamics. Purely fluid models are inherently incapable of capturing these effects, whereas the high dimensionality in purely kinetic models render them practically intractable for most relevant purposes.

        We consider a $\delta\!f$ decomposition model, with a macroscopic fluid background and microscopic kinetic correction, both fully coupled to each other. A similar manner of discretization is proposed to that used in the recent \texttt{STRUPHY} code \cite{Holderied_Possanner_Wang_2021, Holderied_2022, Li_et_al_2023} with a finite-element model for the background and a pseudo-particle/PiC model for the correction.

        The fluid background satisfies the full, non-linear, resistive, compressible, Hall MHD equations. \cite{Laakmann_Hu_Farrell_2022} introduces finite-element(-in-space) implicit timesteppers for the incompressible analogue to this system with structure-preserving (SP) properties in the ideal case, alongside parameter-robust preconditioners. We show that these timesteppers can derive from a finite-element-in-time (FET) (and finite-element-in-space) interpretation. The benefits of this reformulation are discussed, including the derivation of timesteppers that are higher order in time, and the quantifiable dissipative SP properties in the non-ideal, resistive case.
        
        We discuss possible options for extending this FET approach to timesteppers for the compressible case.

        The kinetic corrections satisfy linearized Boltzmann equations. Using a Lénard--Bernstein collision operator, these take Fokker--Planck-like forms \cite{Fokker_1914, Planck_1917} wherein pseudo-particles in the numerical model obey the neoclassical transport equations, with particle-independent Brownian drift terms. This offers a rigorous methodology for incorporating collisions into the particle transport model, without coupling the equations of motions for each particle.
        
        Works by Chen, Chacón et al. \cite{Chen_Chacón_Barnes_2011, Chacón_Chen_Barnes_2013, Chen_Chacón_2014, Chen_Chacón_2015} have developed structure-preserving particle pushers for neoclassical transport in the Vlasov equations, derived from Crank--Nicolson integrators. We show these too can can derive from a FET interpretation, similarly offering potential extensions to higher-order-in-time particle pushers. The FET formulation is used also to consider how the stochastic drift terms can be incorporated into the pushers. Stochastic gyrokinetic expansions are also discussed.

        Different options for the numerical implementation of these schemes are considered.

        Due to the efficacy of FET in the development of SP timesteppers for both the fluid and kinetic component, we hope this approach will prove effective in the future for developing SP timesteppers for the full hybrid model. We hope this will give us the opportunity to incorporate previously inaccessible kinetic effects into the highly effective, modern, finite-element MHD models.
    \end{abstract}
    
    
    \newpage
    \tableofcontents
    
    
    \newpage
    \pagenumbering{arabic}
    %\linenumbers\renewcommand\thelinenumber{\color{black!50}\arabic{linenumber}}
            \input{0 - introduction/main.tex}
        \part{Research}
            \input{1 - low-noise PiC models/main.tex}
            \input{2 - kinetic component/main.tex}
            \input{3 - fluid component/main.tex}
            \input{4 - numerical implementation/main.tex}
        \part{Project Overview}
            \input{5 - research plan/main.tex}
            \input{6 - summary/main.tex}
    
    
    %\section{}
    \newpage
    \pagenumbering{gobble}
        \printbibliography


    \newpage
    \pagenumbering{roman}
    \appendix
        \part{Appendices}
            \input{8 - Hilbert complexes/main.tex}
            \input{9 - weak conservation proofs/main.tex}
\end{document}

            \documentclass[12pt, a4paper]{report}

\input{template/main.tex}

\title{\BA{Title in Progress...}}
\author{Boris Andrews}
\affil{Mathematical Institute, University of Oxford}
\date{\today}


\begin{document}
    \pagenumbering{gobble}
    \maketitle
    
    
    \begin{abstract}
        Magnetic confinement reactors---in particular tokamaks---offer one of the most promising options for achieving practical nuclear fusion, with the potential to provide virtually limitless, clean energy. The theoretical and numerical modeling of tokamak plasmas is simultaneously an essential component of effective reactor design, and a great research barrier. Tokamak operational conditions exhibit comparatively low Knudsen numbers. Kinetic effects, including kinetic waves and instabilities, Landau damping, bump-on-tail instabilities and more, are therefore highly influential in tokamak plasma dynamics. Purely fluid models are inherently incapable of capturing these effects, whereas the high dimensionality in purely kinetic models render them practically intractable for most relevant purposes.

        We consider a $\delta\!f$ decomposition model, with a macroscopic fluid background and microscopic kinetic correction, both fully coupled to each other. A similar manner of discretization is proposed to that used in the recent \texttt{STRUPHY} code \cite{Holderied_Possanner_Wang_2021, Holderied_2022, Li_et_al_2023} with a finite-element model for the background and a pseudo-particle/PiC model for the correction.

        The fluid background satisfies the full, non-linear, resistive, compressible, Hall MHD equations. \cite{Laakmann_Hu_Farrell_2022} introduces finite-element(-in-space) implicit timesteppers for the incompressible analogue to this system with structure-preserving (SP) properties in the ideal case, alongside parameter-robust preconditioners. We show that these timesteppers can derive from a finite-element-in-time (FET) (and finite-element-in-space) interpretation. The benefits of this reformulation are discussed, including the derivation of timesteppers that are higher order in time, and the quantifiable dissipative SP properties in the non-ideal, resistive case.
        
        We discuss possible options for extending this FET approach to timesteppers for the compressible case.

        The kinetic corrections satisfy linearized Boltzmann equations. Using a Lénard--Bernstein collision operator, these take Fokker--Planck-like forms \cite{Fokker_1914, Planck_1917} wherein pseudo-particles in the numerical model obey the neoclassical transport equations, with particle-independent Brownian drift terms. This offers a rigorous methodology for incorporating collisions into the particle transport model, without coupling the equations of motions for each particle.
        
        Works by Chen, Chacón et al. \cite{Chen_Chacón_Barnes_2011, Chacón_Chen_Barnes_2013, Chen_Chacón_2014, Chen_Chacón_2015} have developed structure-preserving particle pushers for neoclassical transport in the Vlasov equations, derived from Crank--Nicolson integrators. We show these too can can derive from a FET interpretation, similarly offering potential extensions to higher-order-in-time particle pushers. The FET formulation is used also to consider how the stochastic drift terms can be incorporated into the pushers. Stochastic gyrokinetic expansions are also discussed.

        Different options for the numerical implementation of these schemes are considered.

        Due to the efficacy of FET in the development of SP timesteppers for both the fluid and kinetic component, we hope this approach will prove effective in the future for developing SP timesteppers for the full hybrid model. We hope this will give us the opportunity to incorporate previously inaccessible kinetic effects into the highly effective, modern, finite-element MHD models.
    \end{abstract}
    
    
    \newpage
    \tableofcontents
    
    
    \newpage
    \pagenumbering{arabic}
    %\linenumbers\renewcommand\thelinenumber{\color{black!50}\arabic{linenumber}}
            \input{0 - introduction/main.tex}
        \part{Research}
            \input{1 - low-noise PiC models/main.tex}
            \input{2 - kinetic component/main.tex}
            \input{3 - fluid component/main.tex}
            \input{4 - numerical implementation/main.tex}
        \part{Project Overview}
            \input{5 - research plan/main.tex}
            \input{6 - summary/main.tex}
    
    
    %\section{}
    \newpage
    \pagenumbering{gobble}
        \printbibliography


    \newpage
    \pagenumbering{roman}
    \appendix
        \part{Appendices}
            \input{8 - Hilbert complexes/main.tex}
            \input{9 - weak conservation proofs/main.tex}
\end{document}

            \documentclass[12pt, a4paper]{report}

\input{template/main.tex}

\title{\BA{Title in Progress...}}
\author{Boris Andrews}
\affil{Mathematical Institute, University of Oxford}
\date{\today}


\begin{document}
    \pagenumbering{gobble}
    \maketitle
    
    
    \begin{abstract}
        Magnetic confinement reactors---in particular tokamaks---offer one of the most promising options for achieving practical nuclear fusion, with the potential to provide virtually limitless, clean energy. The theoretical and numerical modeling of tokamak plasmas is simultaneously an essential component of effective reactor design, and a great research barrier. Tokamak operational conditions exhibit comparatively low Knudsen numbers. Kinetic effects, including kinetic waves and instabilities, Landau damping, bump-on-tail instabilities and more, are therefore highly influential in tokamak plasma dynamics. Purely fluid models are inherently incapable of capturing these effects, whereas the high dimensionality in purely kinetic models render them practically intractable for most relevant purposes.

        We consider a $\delta\!f$ decomposition model, with a macroscopic fluid background and microscopic kinetic correction, both fully coupled to each other. A similar manner of discretization is proposed to that used in the recent \texttt{STRUPHY} code \cite{Holderied_Possanner_Wang_2021, Holderied_2022, Li_et_al_2023} with a finite-element model for the background and a pseudo-particle/PiC model for the correction.

        The fluid background satisfies the full, non-linear, resistive, compressible, Hall MHD equations. \cite{Laakmann_Hu_Farrell_2022} introduces finite-element(-in-space) implicit timesteppers for the incompressible analogue to this system with structure-preserving (SP) properties in the ideal case, alongside parameter-robust preconditioners. We show that these timesteppers can derive from a finite-element-in-time (FET) (and finite-element-in-space) interpretation. The benefits of this reformulation are discussed, including the derivation of timesteppers that are higher order in time, and the quantifiable dissipative SP properties in the non-ideal, resistive case.
        
        We discuss possible options for extending this FET approach to timesteppers for the compressible case.

        The kinetic corrections satisfy linearized Boltzmann equations. Using a Lénard--Bernstein collision operator, these take Fokker--Planck-like forms \cite{Fokker_1914, Planck_1917} wherein pseudo-particles in the numerical model obey the neoclassical transport equations, with particle-independent Brownian drift terms. This offers a rigorous methodology for incorporating collisions into the particle transport model, without coupling the equations of motions for each particle.
        
        Works by Chen, Chacón et al. \cite{Chen_Chacón_Barnes_2011, Chacón_Chen_Barnes_2013, Chen_Chacón_2014, Chen_Chacón_2015} have developed structure-preserving particle pushers for neoclassical transport in the Vlasov equations, derived from Crank--Nicolson integrators. We show these too can can derive from a FET interpretation, similarly offering potential extensions to higher-order-in-time particle pushers. The FET formulation is used also to consider how the stochastic drift terms can be incorporated into the pushers. Stochastic gyrokinetic expansions are also discussed.

        Different options for the numerical implementation of these schemes are considered.

        Due to the efficacy of FET in the development of SP timesteppers for both the fluid and kinetic component, we hope this approach will prove effective in the future for developing SP timesteppers for the full hybrid model. We hope this will give us the opportunity to incorporate previously inaccessible kinetic effects into the highly effective, modern, finite-element MHD models.
    \end{abstract}
    
    
    \newpage
    \tableofcontents
    
    
    \newpage
    \pagenumbering{arabic}
    %\linenumbers\renewcommand\thelinenumber{\color{black!50}\arabic{linenumber}}
            \input{0 - introduction/main.tex}
        \part{Research}
            \input{1 - low-noise PiC models/main.tex}
            \input{2 - kinetic component/main.tex}
            \input{3 - fluid component/main.tex}
            \input{4 - numerical implementation/main.tex}
        \part{Project Overview}
            \input{5 - research plan/main.tex}
            \input{6 - summary/main.tex}
    
    
    %\section{}
    \newpage
    \pagenumbering{gobble}
        \printbibliography


    \newpage
    \pagenumbering{roman}
    \appendix
        \part{Appendices}
            \input{8 - Hilbert complexes/main.tex}
            \input{9 - weak conservation proofs/main.tex}
\end{document}

        \part{Project Overview}
            \documentclass[12pt, a4paper]{report}

\input{template/main.tex}

\title{\BA{Title in Progress...}}
\author{Boris Andrews}
\affil{Mathematical Institute, University of Oxford}
\date{\today}


\begin{document}
    \pagenumbering{gobble}
    \maketitle
    
    
    \begin{abstract}
        Magnetic confinement reactors---in particular tokamaks---offer one of the most promising options for achieving practical nuclear fusion, with the potential to provide virtually limitless, clean energy. The theoretical and numerical modeling of tokamak plasmas is simultaneously an essential component of effective reactor design, and a great research barrier. Tokamak operational conditions exhibit comparatively low Knudsen numbers. Kinetic effects, including kinetic waves and instabilities, Landau damping, bump-on-tail instabilities and more, are therefore highly influential in tokamak plasma dynamics. Purely fluid models are inherently incapable of capturing these effects, whereas the high dimensionality in purely kinetic models render them practically intractable for most relevant purposes.

        We consider a $\delta\!f$ decomposition model, with a macroscopic fluid background and microscopic kinetic correction, both fully coupled to each other. A similar manner of discretization is proposed to that used in the recent \texttt{STRUPHY} code \cite{Holderied_Possanner_Wang_2021, Holderied_2022, Li_et_al_2023} with a finite-element model for the background and a pseudo-particle/PiC model for the correction.

        The fluid background satisfies the full, non-linear, resistive, compressible, Hall MHD equations. \cite{Laakmann_Hu_Farrell_2022} introduces finite-element(-in-space) implicit timesteppers for the incompressible analogue to this system with structure-preserving (SP) properties in the ideal case, alongside parameter-robust preconditioners. We show that these timesteppers can derive from a finite-element-in-time (FET) (and finite-element-in-space) interpretation. The benefits of this reformulation are discussed, including the derivation of timesteppers that are higher order in time, and the quantifiable dissipative SP properties in the non-ideal, resistive case.
        
        We discuss possible options for extending this FET approach to timesteppers for the compressible case.

        The kinetic corrections satisfy linearized Boltzmann equations. Using a Lénard--Bernstein collision operator, these take Fokker--Planck-like forms \cite{Fokker_1914, Planck_1917} wherein pseudo-particles in the numerical model obey the neoclassical transport equations, with particle-independent Brownian drift terms. This offers a rigorous methodology for incorporating collisions into the particle transport model, without coupling the equations of motions for each particle.
        
        Works by Chen, Chacón et al. \cite{Chen_Chacón_Barnes_2011, Chacón_Chen_Barnes_2013, Chen_Chacón_2014, Chen_Chacón_2015} have developed structure-preserving particle pushers for neoclassical transport in the Vlasov equations, derived from Crank--Nicolson integrators. We show these too can can derive from a FET interpretation, similarly offering potential extensions to higher-order-in-time particle pushers. The FET formulation is used also to consider how the stochastic drift terms can be incorporated into the pushers. Stochastic gyrokinetic expansions are also discussed.

        Different options for the numerical implementation of these schemes are considered.

        Due to the efficacy of FET in the development of SP timesteppers for both the fluid and kinetic component, we hope this approach will prove effective in the future for developing SP timesteppers for the full hybrid model. We hope this will give us the opportunity to incorporate previously inaccessible kinetic effects into the highly effective, modern, finite-element MHD models.
    \end{abstract}
    
    
    \newpage
    \tableofcontents
    
    
    \newpage
    \pagenumbering{arabic}
    %\linenumbers\renewcommand\thelinenumber{\color{black!50}\arabic{linenumber}}
            \input{0 - introduction/main.tex}
        \part{Research}
            \input{1 - low-noise PiC models/main.tex}
            \input{2 - kinetic component/main.tex}
            \input{3 - fluid component/main.tex}
            \input{4 - numerical implementation/main.tex}
        \part{Project Overview}
            \input{5 - research plan/main.tex}
            \input{6 - summary/main.tex}
    
    
    %\section{}
    \newpage
    \pagenumbering{gobble}
        \printbibliography


    \newpage
    \pagenumbering{roman}
    \appendix
        \part{Appendices}
            \input{8 - Hilbert complexes/main.tex}
            \input{9 - weak conservation proofs/main.tex}
\end{document}

            \documentclass[12pt, a4paper]{report}

\input{template/main.tex}

\title{\BA{Title in Progress...}}
\author{Boris Andrews}
\affil{Mathematical Institute, University of Oxford}
\date{\today}


\begin{document}
    \pagenumbering{gobble}
    \maketitle
    
    
    \begin{abstract}
        Magnetic confinement reactors---in particular tokamaks---offer one of the most promising options for achieving practical nuclear fusion, with the potential to provide virtually limitless, clean energy. The theoretical and numerical modeling of tokamak plasmas is simultaneously an essential component of effective reactor design, and a great research barrier. Tokamak operational conditions exhibit comparatively low Knudsen numbers. Kinetic effects, including kinetic waves and instabilities, Landau damping, bump-on-tail instabilities and more, are therefore highly influential in tokamak plasma dynamics. Purely fluid models are inherently incapable of capturing these effects, whereas the high dimensionality in purely kinetic models render them practically intractable for most relevant purposes.

        We consider a $\delta\!f$ decomposition model, with a macroscopic fluid background and microscopic kinetic correction, both fully coupled to each other. A similar manner of discretization is proposed to that used in the recent \texttt{STRUPHY} code \cite{Holderied_Possanner_Wang_2021, Holderied_2022, Li_et_al_2023} with a finite-element model for the background and a pseudo-particle/PiC model for the correction.

        The fluid background satisfies the full, non-linear, resistive, compressible, Hall MHD equations. \cite{Laakmann_Hu_Farrell_2022} introduces finite-element(-in-space) implicit timesteppers for the incompressible analogue to this system with structure-preserving (SP) properties in the ideal case, alongside parameter-robust preconditioners. We show that these timesteppers can derive from a finite-element-in-time (FET) (and finite-element-in-space) interpretation. The benefits of this reformulation are discussed, including the derivation of timesteppers that are higher order in time, and the quantifiable dissipative SP properties in the non-ideal, resistive case.
        
        We discuss possible options for extending this FET approach to timesteppers for the compressible case.

        The kinetic corrections satisfy linearized Boltzmann equations. Using a Lénard--Bernstein collision operator, these take Fokker--Planck-like forms \cite{Fokker_1914, Planck_1917} wherein pseudo-particles in the numerical model obey the neoclassical transport equations, with particle-independent Brownian drift terms. This offers a rigorous methodology for incorporating collisions into the particle transport model, without coupling the equations of motions for each particle.
        
        Works by Chen, Chacón et al. \cite{Chen_Chacón_Barnes_2011, Chacón_Chen_Barnes_2013, Chen_Chacón_2014, Chen_Chacón_2015} have developed structure-preserving particle pushers for neoclassical transport in the Vlasov equations, derived from Crank--Nicolson integrators. We show these too can can derive from a FET interpretation, similarly offering potential extensions to higher-order-in-time particle pushers. The FET formulation is used also to consider how the stochastic drift terms can be incorporated into the pushers. Stochastic gyrokinetic expansions are also discussed.

        Different options for the numerical implementation of these schemes are considered.

        Due to the efficacy of FET in the development of SP timesteppers for both the fluid and kinetic component, we hope this approach will prove effective in the future for developing SP timesteppers for the full hybrid model. We hope this will give us the opportunity to incorporate previously inaccessible kinetic effects into the highly effective, modern, finite-element MHD models.
    \end{abstract}
    
    
    \newpage
    \tableofcontents
    
    
    \newpage
    \pagenumbering{arabic}
    %\linenumbers\renewcommand\thelinenumber{\color{black!50}\arabic{linenumber}}
            \input{0 - introduction/main.tex}
        \part{Research}
            \input{1 - low-noise PiC models/main.tex}
            \input{2 - kinetic component/main.tex}
            \input{3 - fluid component/main.tex}
            \input{4 - numerical implementation/main.tex}
        \part{Project Overview}
            \input{5 - research plan/main.tex}
            \input{6 - summary/main.tex}
    
    
    %\section{}
    \newpage
    \pagenumbering{gobble}
        \printbibliography


    \newpage
    \pagenumbering{roman}
    \appendix
        \part{Appendices}
            \input{8 - Hilbert complexes/main.tex}
            \input{9 - weak conservation proofs/main.tex}
\end{document}

    
    
    %\section{}
    \newpage
    \pagenumbering{gobble}
        \printbibliography


    \newpage
    \pagenumbering{roman}
    \appendix
        \part{Appendices}
            \documentclass[12pt, a4paper]{report}

\input{template/main.tex}

\title{\BA{Title in Progress...}}
\author{Boris Andrews}
\affil{Mathematical Institute, University of Oxford}
\date{\today}


\begin{document}
    \pagenumbering{gobble}
    \maketitle
    
    
    \begin{abstract}
        Magnetic confinement reactors---in particular tokamaks---offer one of the most promising options for achieving practical nuclear fusion, with the potential to provide virtually limitless, clean energy. The theoretical and numerical modeling of tokamak plasmas is simultaneously an essential component of effective reactor design, and a great research barrier. Tokamak operational conditions exhibit comparatively low Knudsen numbers. Kinetic effects, including kinetic waves and instabilities, Landau damping, bump-on-tail instabilities and more, are therefore highly influential in tokamak plasma dynamics. Purely fluid models are inherently incapable of capturing these effects, whereas the high dimensionality in purely kinetic models render them practically intractable for most relevant purposes.

        We consider a $\delta\!f$ decomposition model, with a macroscopic fluid background and microscopic kinetic correction, both fully coupled to each other. A similar manner of discretization is proposed to that used in the recent \texttt{STRUPHY} code \cite{Holderied_Possanner_Wang_2021, Holderied_2022, Li_et_al_2023} with a finite-element model for the background and a pseudo-particle/PiC model for the correction.

        The fluid background satisfies the full, non-linear, resistive, compressible, Hall MHD equations. \cite{Laakmann_Hu_Farrell_2022} introduces finite-element(-in-space) implicit timesteppers for the incompressible analogue to this system with structure-preserving (SP) properties in the ideal case, alongside parameter-robust preconditioners. We show that these timesteppers can derive from a finite-element-in-time (FET) (and finite-element-in-space) interpretation. The benefits of this reformulation are discussed, including the derivation of timesteppers that are higher order in time, and the quantifiable dissipative SP properties in the non-ideal, resistive case.
        
        We discuss possible options for extending this FET approach to timesteppers for the compressible case.

        The kinetic corrections satisfy linearized Boltzmann equations. Using a Lénard--Bernstein collision operator, these take Fokker--Planck-like forms \cite{Fokker_1914, Planck_1917} wherein pseudo-particles in the numerical model obey the neoclassical transport equations, with particle-independent Brownian drift terms. This offers a rigorous methodology for incorporating collisions into the particle transport model, without coupling the equations of motions for each particle.
        
        Works by Chen, Chacón et al. \cite{Chen_Chacón_Barnes_2011, Chacón_Chen_Barnes_2013, Chen_Chacón_2014, Chen_Chacón_2015} have developed structure-preserving particle pushers for neoclassical transport in the Vlasov equations, derived from Crank--Nicolson integrators. We show these too can can derive from a FET interpretation, similarly offering potential extensions to higher-order-in-time particle pushers. The FET formulation is used also to consider how the stochastic drift terms can be incorporated into the pushers. Stochastic gyrokinetic expansions are also discussed.

        Different options for the numerical implementation of these schemes are considered.

        Due to the efficacy of FET in the development of SP timesteppers for both the fluid and kinetic component, we hope this approach will prove effective in the future for developing SP timesteppers for the full hybrid model. We hope this will give us the opportunity to incorporate previously inaccessible kinetic effects into the highly effective, modern, finite-element MHD models.
    \end{abstract}
    
    
    \newpage
    \tableofcontents
    
    
    \newpage
    \pagenumbering{arabic}
    %\linenumbers\renewcommand\thelinenumber{\color{black!50}\arabic{linenumber}}
            \input{0 - introduction/main.tex}
        \part{Research}
            \input{1 - low-noise PiC models/main.tex}
            \input{2 - kinetic component/main.tex}
            \input{3 - fluid component/main.tex}
            \input{4 - numerical implementation/main.tex}
        \part{Project Overview}
            \input{5 - research plan/main.tex}
            \input{6 - summary/main.tex}
    
    
    %\section{}
    \newpage
    \pagenumbering{gobble}
        \printbibliography


    \newpage
    \pagenumbering{roman}
    \appendix
        \part{Appendices}
            \input{8 - Hilbert complexes/main.tex}
            \input{9 - weak conservation proofs/main.tex}
\end{document}

            \documentclass[12pt, a4paper]{report}

\input{template/main.tex}

\title{\BA{Title in Progress...}}
\author{Boris Andrews}
\affil{Mathematical Institute, University of Oxford}
\date{\today}


\begin{document}
    \pagenumbering{gobble}
    \maketitle
    
    
    \begin{abstract}
        Magnetic confinement reactors---in particular tokamaks---offer one of the most promising options for achieving practical nuclear fusion, with the potential to provide virtually limitless, clean energy. The theoretical and numerical modeling of tokamak plasmas is simultaneously an essential component of effective reactor design, and a great research barrier. Tokamak operational conditions exhibit comparatively low Knudsen numbers. Kinetic effects, including kinetic waves and instabilities, Landau damping, bump-on-tail instabilities and more, are therefore highly influential in tokamak plasma dynamics. Purely fluid models are inherently incapable of capturing these effects, whereas the high dimensionality in purely kinetic models render them practically intractable for most relevant purposes.

        We consider a $\delta\!f$ decomposition model, with a macroscopic fluid background and microscopic kinetic correction, both fully coupled to each other. A similar manner of discretization is proposed to that used in the recent \texttt{STRUPHY} code \cite{Holderied_Possanner_Wang_2021, Holderied_2022, Li_et_al_2023} with a finite-element model for the background and a pseudo-particle/PiC model for the correction.

        The fluid background satisfies the full, non-linear, resistive, compressible, Hall MHD equations. \cite{Laakmann_Hu_Farrell_2022} introduces finite-element(-in-space) implicit timesteppers for the incompressible analogue to this system with structure-preserving (SP) properties in the ideal case, alongside parameter-robust preconditioners. We show that these timesteppers can derive from a finite-element-in-time (FET) (and finite-element-in-space) interpretation. The benefits of this reformulation are discussed, including the derivation of timesteppers that are higher order in time, and the quantifiable dissipative SP properties in the non-ideal, resistive case.
        
        We discuss possible options for extending this FET approach to timesteppers for the compressible case.

        The kinetic corrections satisfy linearized Boltzmann equations. Using a Lénard--Bernstein collision operator, these take Fokker--Planck-like forms \cite{Fokker_1914, Planck_1917} wherein pseudo-particles in the numerical model obey the neoclassical transport equations, with particle-independent Brownian drift terms. This offers a rigorous methodology for incorporating collisions into the particle transport model, without coupling the equations of motions for each particle.
        
        Works by Chen, Chacón et al. \cite{Chen_Chacón_Barnes_2011, Chacón_Chen_Barnes_2013, Chen_Chacón_2014, Chen_Chacón_2015} have developed structure-preserving particle pushers for neoclassical transport in the Vlasov equations, derived from Crank--Nicolson integrators. We show these too can can derive from a FET interpretation, similarly offering potential extensions to higher-order-in-time particle pushers. The FET formulation is used also to consider how the stochastic drift terms can be incorporated into the pushers. Stochastic gyrokinetic expansions are also discussed.

        Different options for the numerical implementation of these schemes are considered.

        Due to the efficacy of FET in the development of SP timesteppers for both the fluid and kinetic component, we hope this approach will prove effective in the future for developing SP timesteppers for the full hybrid model. We hope this will give us the opportunity to incorporate previously inaccessible kinetic effects into the highly effective, modern, finite-element MHD models.
    \end{abstract}
    
    
    \newpage
    \tableofcontents
    
    
    \newpage
    \pagenumbering{arabic}
    %\linenumbers\renewcommand\thelinenumber{\color{black!50}\arabic{linenumber}}
            \input{0 - introduction/main.tex}
        \part{Research}
            \input{1 - low-noise PiC models/main.tex}
            \input{2 - kinetic component/main.tex}
            \input{3 - fluid component/main.tex}
            \input{4 - numerical implementation/main.tex}
        \part{Project Overview}
            \input{5 - research plan/main.tex}
            \input{6 - summary/main.tex}
    
    
    %\section{}
    \newpage
    \pagenumbering{gobble}
        \printbibliography


    \newpage
    \pagenumbering{roman}
    \appendix
        \part{Appendices}
            \input{8 - Hilbert complexes/main.tex}
            \input{9 - weak conservation proofs/main.tex}
\end{document}

\end{document}

            \documentclass[12pt, a4paper]{report}

\documentclass[12pt, a4paper]{report}

\input{template/main.tex}

\title{\BA{Title in Progress...}}
\author{Boris Andrews}
\affil{Mathematical Institute, University of Oxford}
\date{\today}


\begin{document}
    \pagenumbering{gobble}
    \maketitle
    
    
    \begin{abstract}
        Magnetic confinement reactors---in particular tokamaks---offer one of the most promising options for achieving practical nuclear fusion, with the potential to provide virtually limitless, clean energy. The theoretical and numerical modeling of tokamak plasmas is simultaneously an essential component of effective reactor design, and a great research barrier. Tokamak operational conditions exhibit comparatively low Knudsen numbers. Kinetic effects, including kinetic waves and instabilities, Landau damping, bump-on-tail instabilities and more, are therefore highly influential in tokamak plasma dynamics. Purely fluid models are inherently incapable of capturing these effects, whereas the high dimensionality in purely kinetic models render them practically intractable for most relevant purposes.

        We consider a $\delta\!f$ decomposition model, with a macroscopic fluid background and microscopic kinetic correction, both fully coupled to each other. A similar manner of discretization is proposed to that used in the recent \texttt{STRUPHY} code \cite{Holderied_Possanner_Wang_2021, Holderied_2022, Li_et_al_2023} with a finite-element model for the background and a pseudo-particle/PiC model for the correction.

        The fluid background satisfies the full, non-linear, resistive, compressible, Hall MHD equations. \cite{Laakmann_Hu_Farrell_2022} introduces finite-element(-in-space) implicit timesteppers for the incompressible analogue to this system with structure-preserving (SP) properties in the ideal case, alongside parameter-robust preconditioners. We show that these timesteppers can derive from a finite-element-in-time (FET) (and finite-element-in-space) interpretation. The benefits of this reformulation are discussed, including the derivation of timesteppers that are higher order in time, and the quantifiable dissipative SP properties in the non-ideal, resistive case.
        
        We discuss possible options for extending this FET approach to timesteppers for the compressible case.

        The kinetic corrections satisfy linearized Boltzmann equations. Using a Lénard--Bernstein collision operator, these take Fokker--Planck-like forms \cite{Fokker_1914, Planck_1917} wherein pseudo-particles in the numerical model obey the neoclassical transport equations, with particle-independent Brownian drift terms. This offers a rigorous methodology for incorporating collisions into the particle transport model, without coupling the equations of motions for each particle.
        
        Works by Chen, Chacón et al. \cite{Chen_Chacón_Barnes_2011, Chacón_Chen_Barnes_2013, Chen_Chacón_2014, Chen_Chacón_2015} have developed structure-preserving particle pushers for neoclassical transport in the Vlasov equations, derived from Crank--Nicolson integrators. We show these too can can derive from a FET interpretation, similarly offering potential extensions to higher-order-in-time particle pushers. The FET formulation is used also to consider how the stochastic drift terms can be incorporated into the pushers. Stochastic gyrokinetic expansions are also discussed.

        Different options for the numerical implementation of these schemes are considered.

        Due to the efficacy of FET in the development of SP timesteppers for both the fluid and kinetic component, we hope this approach will prove effective in the future for developing SP timesteppers for the full hybrid model. We hope this will give us the opportunity to incorporate previously inaccessible kinetic effects into the highly effective, modern, finite-element MHD models.
    \end{abstract}
    
    
    \newpage
    \tableofcontents
    
    
    \newpage
    \pagenumbering{arabic}
    %\linenumbers\renewcommand\thelinenumber{\color{black!50}\arabic{linenumber}}
            \input{0 - introduction/main.tex}
        \part{Research}
            \input{1 - low-noise PiC models/main.tex}
            \input{2 - kinetic component/main.tex}
            \input{3 - fluid component/main.tex}
            \input{4 - numerical implementation/main.tex}
        \part{Project Overview}
            \input{5 - research plan/main.tex}
            \input{6 - summary/main.tex}
    
    
    %\section{}
    \newpage
    \pagenumbering{gobble}
        \printbibliography


    \newpage
    \pagenumbering{roman}
    \appendix
        \part{Appendices}
            \input{8 - Hilbert complexes/main.tex}
            \input{9 - weak conservation proofs/main.tex}
\end{document}


\title{\BA{Title in Progress...}}
\author{Boris Andrews}
\affil{Mathematical Institute, University of Oxford}
\date{\today}


\begin{document}
    \pagenumbering{gobble}
    \maketitle
    
    
    \begin{abstract}
        Magnetic confinement reactors---in particular tokamaks---offer one of the most promising options for achieving practical nuclear fusion, with the potential to provide virtually limitless, clean energy. The theoretical and numerical modeling of tokamak plasmas is simultaneously an essential component of effective reactor design, and a great research barrier. Tokamak operational conditions exhibit comparatively low Knudsen numbers. Kinetic effects, including kinetic waves and instabilities, Landau damping, bump-on-tail instabilities and more, are therefore highly influential in tokamak plasma dynamics. Purely fluid models are inherently incapable of capturing these effects, whereas the high dimensionality in purely kinetic models render them practically intractable for most relevant purposes.

        We consider a $\delta\!f$ decomposition model, with a macroscopic fluid background and microscopic kinetic correction, both fully coupled to each other. A similar manner of discretization is proposed to that used in the recent \texttt{STRUPHY} code \cite{Holderied_Possanner_Wang_2021, Holderied_2022, Li_et_al_2023} with a finite-element model for the background and a pseudo-particle/PiC model for the correction.

        The fluid background satisfies the full, non-linear, resistive, compressible, Hall MHD equations. \cite{Laakmann_Hu_Farrell_2022} introduces finite-element(-in-space) implicit timesteppers for the incompressible analogue to this system with structure-preserving (SP) properties in the ideal case, alongside parameter-robust preconditioners. We show that these timesteppers can derive from a finite-element-in-time (FET) (and finite-element-in-space) interpretation. The benefits of this reformulation are discussed, including the derivation of timesteppers that are higher order in time, and the quantifiable dissipative SP properties in the non-ideal, resistive case.
        
        We discuss possible options for extending this FET approach to timesteppers for the compressible case.

        The kinetic corrections satisfy linearized Boltzmann equations. Using a Lénard--Bernstein collision operator, these take Fokker--Planck-like forms \cite{Fokker_1914, Planck_1917} wherein pseudo-particles in the numerical model obey the neoclassical transport equations, with particle-independent Brownian drift terms. This offers a rigorous methodology for incorporating collisions into the particle transport model, without coupling the equations of motions for each particle.
        
        Works by Chen, Chacón et al. \cite{Chen_Chacón_Barnes_2011, Chacón_Chen_Barnes_2013, Chen_Chacón_2014, Chen_Chacón_2015} have developed structure-preserving particle pushers for neoclassical transport in the Vlasov equations, derived from Crank--Nicolson integrators. We show these too can can derive from a FET interpretation, similarly offering potential extensions to higher-order-in-time particle pushers. The FET formulation is used also to consider how the stochastic drift terms can be incorporated into the pushers. Stochastic gyrokinetic expansions are also discussed.

        Different options for the numerical implementation of these schemes are considered.

        Due to the efficacy of FET in the development of SP timesteppers for both the fluid and kinetic component, we hope this approach will prove effective in the future for developing SP timesteppers for the full hybrid model. We hope this will give us the opportunity to incorporate previously inaccessible kinetic effects into the highly effective, modern, finite-element MHD models.
    \end{abstract}
    
    
    \newpage
    \tableofcontents
    
    
    \newpage
    \pagenumbering{arabic}
    %\linenumbers\renewcommand\thelinenumber{\color{black!50}\arabic{linenumber}}
            \documentclass[12pt, a4paper]{report}

\input{template/main.tex}

\title{\BA{Title in Progress...}}
\author{Boris Andrews}
\affil{Mathematical Institute, University of Oxford}
\date{\today}


\begin{document}
    \pagenumbering{gobble}
    \maketitle
    
    
    \begin{abstract}
        Magnetic confinement reactors---in particular tokamaks---offer one of the most promising options for achieving practical nuclear fusion, with the potential to provide virtually limitless, clean energy. The theoretical and numerical modeling of tokamak plasmas is simultaneously an essential component of effective reactor design, and a great research barrier. Tokamak operational conditions exhibit comparatively low Knudsen numbers. Kinetic effects, including kinetic waves and instabilities, Landau damping, bump-on-tail instabilities and more, are therefore highly influential in tokamak plasma dynamics. Purely fluid models are inherently incapable of capturing these effects, whereas the high dimensionality in purely kinetic models render them practically intractable for most relevant purposes.

        We consider a $\delta\!f$ decomposition model, with a macroscopic fluid background and microscopic kinetic correction, both fully coupled to each other. A similar manner of discretization is proposed to that used in the recent \texttt{STRUPHY} code \cite{Holderied_Possanner_Wang_2021, Holderied_2022, Li_et_al_2023} with a finite-element model for the background and a pseudo-particle/PiC model for the correction.

        The fluid background satisfies the full, non-linear, resistive, compressible, Hall MHD equations. \cite{Laakmann_Hu_Farrell_2022} introduces finite-element(-in-space) implicit timesteppers for the incompressible analogue to this system with structure-preserving (SP) properties in the ideal case, alongside parameter-robust preconditioners. We show that these timesteppers can derive from a finite-element-in-time (FET) (and finite-element-in-space) interpretation. The benefits of this reformulation are discussed, including the derivation of timesteppers that are higher order in time, and the quantifiable dissipative SP properties in the non-ideal, resistive case.
        
        We discuss possible options for extending this FET approach to timesteppers for the compressible case.

        The kinetic corrections satisfy linearized Boltzmann equations. Using a Lénard--Bernstein collision operator, these take Fokker--Planck-like forms \cite{Fokker_1914, Planck_1917} wherein pseudo-particles in the numerical model obey the neoclassical transport equations, with particle-independent Brownian drift terms. This offers a rigorous methodology for incorporating collisions into the particle transport model, without coupling the equations of motions for each particle.
        
        Works by Chen, Chacón et al. \cite{Chen_Chacón_Barnes_2011, Chacón_Chen_Barnes_2013, Chen_Chacón_2014, Chen_Chacón_2015} have developed structure-preserving particle pushers for neoclassical transport in the Vlasov equations, derived from Crank--Nicolson integrators. We show these too can can derive from a FET interpretation, similarly offering potential extensions to higher-order-in-time particle pushers. The FET formulation is used also to consider how the stochastic drift terms can be incorporated into the pushers. Stochastic gyrokinetic expansions are also discussed.

        Different options for the numerical implementation of these schemes are considered.

        Due to the efficacy of FET in the development of SP timesteppers for both the fluid and kinetic component, we hope this approach will prove effective in the future for developing SP timesteppers for the full hybrid model. We hope this will give us the opportunity to incorporate previously inaccessible kinetic effects into the highly effective, modern, finite-element MHD models.
    \end{abstract}
    
    
    \newpage
    \tableofcontents
    
    
    \newpage
    \pagenumbering{arabic}
    %\linenumbers\renewcommand\thelinenumber{\color{black!50}\arabic{linenumber}}
            \input{0 - introduction/main.tex}
        \part{Research}
            \input{1 - low-noise PiC models/main.tex}
            \input{2 - kinetic component/main.tex}
            \input{3 - fluid component/main.tex}
            \input{4 - numerical implementation/main.tex}
        \part{Project Overview}
            \input{5 - research plan/main.tex}
            \input{6 - summary/main.tex}
    
    
    %\section{}
    \newpage
    \pagenumbering{gobble}
        \printbibliography


    \newpage
    \pagenumbering{roman}
    \appendix
        \part{Appendices}
            \input{8 - Hilbert complexes/main.tex}
            \input{9 - weak conservation proofs/main.tex}
\end{document}

        \part{Research}
            \documentclass[12pt, a4paper]{report}

\input{template/main.tex}

\title{\BA{Title in Progress...}}
\author{Boris Andrews}
\affil{Mathematical Institute, University of Oxford}
\date{\today}


\begin{document}
    \pagenumbering{gobble}
    \maketitle
    
    
    \begin{abstract}
        Magnetic confinement reactors---in particular tokamaks---offer one of the most promising options for achieving practical nuclear fusion, with the potential to provide virtually limitless, clean energy. The theoretical and numerical modeling of tokamak plasmas is simultaneously an essential component of effective reactor design, and a great research barrier. Tokamak operational conditions exhibit comparatively low Knudsen numbers. Kinetic effects, including kinetic waves and instabilities, Landau damping, bump-on-tail instabilities and more, are therefore highly influential in tokamak plasma dynamics. Purely fluid models are inherently incapable of capturing these effects, whereas the high dimensionality in purely kinetic models render them practically intractable for most relevant purposes.

        We consider a $\delta\!f$ decomposition model, with a macroscopic fluid background and microscopic kinetic correction, both fully coupled to each other. A similar manner of discretization is proposed to that used in the recent \texttt{STRUPHY} code \cite{Holderied_Possanner_Wang_2021, Holderied_2022, Li_et_al_2023} with a finite-element model for the background and a pseudo-particle/PiC model for the correction.

        The fluid background satisfies the full, non-linear, resistive, compressible, Hall MHD equations. \cite{Laakmann_Hu_Farrell_2022} introduces finite-element(-in-space) implicit timesteppers for the incompressible analogue to this system with structure-preserving (SP) properties in the ideal case, alongside parameter-robust preconditioners. We show that these timesteppers can derive from a finite-element-in-time (FET) (and finite-element-in-space) interpretation. The benefits of this reformulation are discussed, including the derivation of timesteppers that are higher order in time, and the quantifiable dissipative SP properties in the non-ideal, resistive case.
        
        We discuss possible options for extending this FET approach to timesteppers for the compressible case.

        The kinetic corrections satisfy linearized Boltzmann equations. Using a Lénard--Bernstein collision operator, these take Fokker--Planck-like forms \cite{Fokker_1914, Planck_1917} wherein pseudo-particles in the numerical model obey the neoclassical transport equations, with particle-independent Brownian drift terms. This offers a rigorous methodology for incorporating collisions into the particle transport model, without coupling the equations of motions for each particle.
        
        Works by Chen, Chacón et al. \cite{Chen_Chacón_Barnes_2011, Chacón_Chen_Barnes_2013, Chen_Chacón_2014, Chen_Chacón_2015} have developed structure-preserving particle pushers for neoclassical transport in the Vlasov equations, derived from Crank--Nicolson integrators. We show these too can can derive from a FET interpretation, similarly offering potential extensions to higher-order-in-time particle pushers. The FET formulation is used also to consider how the stochastic drift terms can be incorporated into the pushers. Stochastic gyrokinetic expansions are also discussed.

        Different options for the numerical implementation of these schemes are considered.

        Due to the efficacy of FET in the development of SP timesteppers for both the fluid and kinetic component, we hope this approach will prove effective in the future for developing SP timesteppers for the full hybrid model. We hope this will give us the opportunity to incorporate previously inaccessible kinetic effects into the highly effective, modern, finite-element MHD models.
    \end{abstract}
    
    
    \newpage
    \tableofcontents
    
    
    \newpage
    \pagenumbering{arabic}
    %\linenumbers\renewcommand\thelinenumber{\color{black!50}\arabic{linenumber}}
            \input{0 - introduction/main.tex}
        \part{Research}
            \input{1 - low-noise PiC models/main.tex}
            \input{2 - kinetic component/main.tex}
            \input{3 - fluid component/main.tex}
            \input{4 - numerical implementation/main.tex}
        \part{Project Overview}
            \input{5 - research plan/main.tex}
            \input{6 - summary/main.tex}
    
    
    %\section{}
    \newpage
    \pagenumbering{gobble}
        \printbibliography


    \newpage
    \pagenumbering{roman}
    \appendix
        \part{Appendices}
            \input{8 - Hilbert complexes/main.tex}
            \input{9 - weak conservation proofs/main.tex}
\end{document}

            \documentclass[12pt, a4paper]{report}

\input{template/main.tex}

\title{\BA{Title in Progress...}}
\author{Boris Andrews}
\affil{Mathematical Institute, University of Oxford}
\date{\today}


\begin{document}
    \pagenumbering{gobble}
    \maketitle
    
    
    \begin{abstract}
        Magnetic confinement reactors---in particular tokamaks---offer one of the most promising options for achieving practical nuclear fusion, with the potential to provide virtually limitless, clean energy. The theoretical and numerical modeling of tokamak plasmas is simultaneously an essential component of effective reactor design, and a great research barrier. Tokamak operational conditions exhibit comparatively low Knudsen numbers. Kinetic effects, including kinetic waves and instabilities, Landau damping, bump-on-tail instabilities and more, are therefore highly influential in tokamak plasma dynamics. Purely fluid models are inherently incapable of capturing these effects, whereas the high dimensionality in purely kinetic models render them practically intractable for most relevant purposes.

        We consider a $\delta\!f$ decomposition model, with a macroscopic fluid background and microscopic kinetic correction, both fully coupled to each other. A similar manner of discretization is proposed to that used in the recent \texttt{STRUPHY} code \cite{Holderied_Possanner_Wang_2021, Holderied_2022, Li_et_al_2023} with a finite-element model for the background and a pseudo-particle/PiC model for the correction.

        The fluid background satisfies the full, non-linear, resistive, compressible, Hall MHD equations. \cite{Laakmann_Hu_Farrell_2022} introduces finite-element(-in-space) implicit timesteppers for the incompressible analogue to this system with structure-preserving (SP) properties in the ideal case, alongside parameter-robust preconditioners. We show that these timesteppers can derive from a finite-element-in-time (FET) (and finite-element-in-space) interpretation. The benefits of this reformulation are discussed, including the derivation of timesteppers that are higher order in time, and the quantifiable dissipative SP properties in the non-ideal, resistive case.
        
        We discuss possible options for extending this FET approach to timesteppers for the compressible case.

        The kinetic corrections satisfy linearized Boltzmann equations. Using a Lénard--Bernstein collision operator, these take Fokker--Planck-like forms \cite{Fokker_1914, Planck_1917} wherein pseudo-particles in the numerical model obey the neoclassical transport equations, with particle-independent Brownian drift terms. This offers a rigorous methodology for incorporating collisions into the particle transport model, without coupling the equations of motions for each particle.
        
        Works by Chen, Chacón et al. \cite{Chen_Chacón_Barnes_2011, Chacón_Chen_Barnes_2013, Chen_Chacón_2014, Chen_Chacón_2015} have developed structure-preserving particle pushers for neoclassical transport in the Vlasov equations, derived from Crank--Nicolson integrators. We show these too can can derive from a FET interpretation, similarly offering potential extensions to higher-order-in-time particle pushers. The FET formulation is used also to consider how the stochastic drift terms can be incorporated into the pushers. Stochastic gyrokinetic expansions are also discussed.

        Different options for the numerical implementation of these schemes are considered.

        Due to the efficacy of FET in the development of SP timesteppers for both the fluid and kinetic component, we hope this approach will prove effective in the future for developing SP timesteppers for the full hybrid model. We hope this will give us the opportunity to incorporate previously inaccessible kinetic effects into the highly effective, modern, finite-element MHD models.
    \end{abstract}
    
    
    \newpage
    \tableofcontents
    
    
    \newpage
    \pagenumbering{arabic}
    %\linenumbers\renewcommand\thelinenumber{\color{black!50}\arabic{linenumber}}
            \input{0 - introduction/main.tex}
        \part{Research}
            \input{1 - low-noise PiC models/main.tex}
            \input{2 - kinetic component/main.tex}
            \input{3 - fluid component/main.tex}
            \input{4 - numerical implementation/main.tex}
        \part{Project Overview}
            \input{5 - research plan/main.tex}
            \input{6 - summary/main.tex}
    
    
    %\section{}
    \newpage
    \pagenumbering{gobble}
        \printbibliography


    \newpage
    \pagenumbering{roman}
    \appendix
        \part{Appendices}
            \input{8 - Hilbert complexes/main.tex}
            \input{9 - weak conservation proofs/main.tex}
\end{document}

            \documentclass[12pt, a4paper]{report}

\input{template/main.tex}

\title{\BA{Title in Progress...}}
\author{Boris Andrews}
\affil{Mathematical Institute, University of Oxford}
\date{\today}


\begin{document}
    \pagenumbering{gobble}
    \maketitle
    
    
    \begin{abstract}
        Magnetic confinement reactors---in particular tokamaks---offer one of the most promising options for achieving practical nuclear fusion, with the potential to provide virtually limitless, clean energy. The theoretical and numerical modeling of tokamak plasmas is simultaneously an essential component of effective reactor design, and a great research barrier. Tokamak operational conditions exhibit comparatively low Knudsen numbers. Kinetic effects, including kinetic waves and instabilities, Landau damping, bump-on-tail instabilities and more, are therefore highly influential in tokamak plasma dynamics. Purely fluid models are inherently incapable of capturing these effects, whereas the high dimensionality in purely kinetic models render them practically intractable for most relevant purposes.

        We consider a $\delta\!f$ decomposition model, with a macroscopic fluid background and microscopic kinetic correction, both fully coupled to each other. A similar manner of discretization is proposed to that used in the recent \texttt{STRUPHY} code \cite{Holderied_Possanner_Wang_2021, Holderied_2022, Li_et_al_2023} with a finite-element model for the background and a pseudo-particle/PiC model for the correction.

        The fluid background satisfies the full, non-linear, resistive, compressible, Hall MHD equations. \cite{Laakmann_Hu_Farrell_2022} introduces finite-element(-in-space) implicit timesteppers for the incompressible analogue to this system with structure-preserving (SP) properties in the ideal case, alongside parameter-robust preconditioners. We show that these timesteppers can derive from a finite-element-in-time (FET) (and finite-element-in-space) interpretation. The benefits of this reformulation are discussed, including the derivation of timesteppers that are higher order in time, and the quantifiable dissipative SP properties in the non-ideal, resistive case.
        
        We discuss possible options for extending this FET approach to timesteppers for the compressible case.

        The kinetic corrections satisfy linearized Boltzmann equations. Using a Lénard--Bernstein collision operator, these take Fokker--Planck-like forms \cite{Fokker_1914, Planck_1917} wherein pseudo-particles in the numerical model obey the neoclassical transport equations, with particle-independent Brownian drift terms. This offers a rigorous methodology for incorporating collisions into the particle transport model, without coupling the equations of motions for each particle.
        
        Works by Chen, Chacón et al. \cite{Chen_Chacón_Barnes_2011, Chacón_Chen_Barnes_2013, Chen_Chacón_2014, Chen_Chacón_2015} have developed structure-preserving particle pushers for neoclassical transport in the Vlasov equations, derived from Crank--Nicolson integrators. We show these too can can derive from a FET interpretation, similarly offering potential extensions to higher-order-in-time particle pushers. The FET formulation is used also to consider how the stochastic drift terms can be incorporated into the pushers. Stochastic gyrokinetic expansions are also discussed.

        Different options for the numerical implementation of these schemes are considered.

        Due to the efficacy of FET in the development of SP timesteppers for both the fluid and kinetic component, we hope this approach will prove effective in the future for developing SP timesteppers for the full hybrid model. We hope this will give us the opportunity to incorporate previously inaccessible kinetic effects into the highly effective, modern, finite-element MHD models.
    \end{abstract}
    
    
    \newpage
    \tableofcontents
    
    
    \newpage
    \pagenumbering{arabic}
    %\linenumbers\renewcommand\thelinenumber{\color{black!50}\arabic{linenumber}}
            \input{0 - introduction/main.tex}
        \part{Research}
            \input{1 - low-noise PiC models/main.tex}
            \input{2 - kinetic component/main.tex}
            \input{3 - fluid component/main.tex}
            \input{4 - numerical implementation/main.tex}
        \part{Project Overview}
            \input{5 - research plan/main.tex}
            \input{6 - summary/main.tex}
    
    
    %\section{}
    \newpage
    \pagenumbering{gobble}
        \printbibliography


    \newpage
    \pagenumbering{roman}
    \appendix
        \part{Appendices}
            \input{8 - Hilbert complexes/main.tex}
            \input{9 - weak conservation proofs/main.tex}
\end{document}

            \documentclass[12pt, a4paper]{report}

\input{template/main.tex}

\title{\BA{Title in Progress...}}
\author{Boris Andrews}
\affil{Mathematical Institute, University of Oxford}
\date{\today}


\begin{document}
    \pagenumbering{gobble}
    \maketitle
    
    
    \begin{abstract}
        Magnetic confinement reactors---in particular tokamaks---offer one of the most promising options for achieving practical nuclear fusion, with the potential to provide virtually limitless, clean energy. The theoretical and numerical modeling of tokamak plasmas is simultaneously an essential component of effective reactor design, and a great research barrier. Tokamak operational conditions exhibit comparatively low Knudsen numbers. Kinetic effects, including kinetic waves and instabilities, Landau damping, bump-on-tail instabilities and more, are therefore highly influential in tokamak plasma dynamics. Purely fluid models are inherently incapable of capturing these effects, whereas the high dimensionality in purely kinetic models render them practically intractable for most relevant purposes.

        We consider a $\delta\!f$ decomposition model, with a macroscopic fluid background and microscopic kinetic correction, both fully coupled to each other. A similar manner of discretization is proposed to that used in the recent \texttt{STRUPHY} code \cite{Holderied_Possanner_Wang_2021, Holderied_2022, Li_et_al_2023} with a finite-element model for the background and a pseudo-particle/PiC model for the correction.

        The fluid background satisfies the full, non-linear, resistive, compressible, Hall MHD equations. \cite{Laakmann_Hu_Farrell_2022} introduces finite-element(-in-space) implicit timesteppers for the incompressible analogue to this system with structure-preserving (SP) properties in the ideal case, alongside parameter-robust preconditioners. We show that these timesteppers can derive from a finite-element-in-time (FET) (and finite-element-in-space) interpretation. The benefits of this reformulation are discussed, including the derivation of timesteppers that are higher order in time, and the quantifiable dissipative SP properties in the non-ideal, resistive case.
        
        We discuss possible options for extending this FET approach to timesteppers for the compressible case.

        The kinetic corrections satisfy linearized Boltzmann equations. Using a Lénard--Bernstein collision operator, these take Fokker--Planck-like forms \cite{Fokker_1914, Planck_1917} wherein pseudo-particles in the numerical model obey the neoclassical transport equations, with particle-independent Brownian drift terms. This offers a rigorous methodology for incorporating collisions into the particle transport model, without coupling the equations of motions for each particle.
        
        Works by Chen, Chacón et al. \cite{Chen_Chacón_Barnes_2011, Chacón_Chen_Barnes_2013, Chen_Chacón_2014, Chen_Chacón_2015} have developed structure-preserving particle pushers for neoclassical transport in the Vlasov equations, derived from Crank--Nicolson integrators. We show these too can can derive from a FET interpretation, similarly offering potential extensions to higher-order-in-time particle pushers. The FET formulation is used also to consider how the stochastic drift terms can be incorporated into the pushers. Stochastic gyrokinetic expansions are also discussed.

        Different options for the numerical implementation of these schemes are considered.

        Due to the efficacy of FET in the development of SP timesteppers for both the fluid and kinetic component, we hope this approach will prove effective in the future for developing SP timesteppers for the full hybrid model. We hope this will give us the opportunity to incorporate previously inaccessible kinetic effects into the highly effective, modern, finite-element MHD models.
    \end{abstract}
    
    
    \newpage
    \tableofcontents
    
    
    \newpage
    \pagenumbering{arabic}
    %\linenumbers\renewcommand\thelinenumber{\color{black!50}\arabic{linenumber}}
            \input{0 - introduction/main.tex}
        \part{Research}
            \input{1 - low-noise PiC models/main.tex}
            \input{2 - kinetic component/main.tex}
            \input{3 - fluid component/main.tex}
            \input{4 - numerical implementation/main.tex}
        \part{Project Overview}
            \input{5 - research plan/main.tex}
            \input{6 - summary/main.tex}
    
    
    %\section{}
    \newpage
    \pagenumbering{gobble}
        \printbibliography


    \newpage
    \pagenumbering{roman}
    \appendix
        \part{Appendices}
            \input{8 - Hilbert complexes/main.tex}
            \input{9 - weak conservation proofs/main.tex}
\end{document}

        \part{Project Overview}
            \documentclass[12pt, a4paper]{report}

\input{template/main.tex}

\title{\BA{Title in Progress...}}
\author{Boris Andrews}
\affil{Mathematical Institute, University of Oxford}
\date{\today}


\begin{document}
    \pagenumbering{gobble}
    \maketitle
    
    
    \begin{abstract}
        Magnetic confinement reactors---in particular tokamaks---offer one of the most promising options for achieving practical nuclear fusion, with the potential to provide virtually limitless, clean energy. The theoretical and numerical modeling of tokamak plasmas is simultaneously an essential component of effective reactor design, and a great research barrier. Tokamak operational conditions exhibit comparatively low Knudsen numbers. Kinetic effects, including kinetic waves and instabilities, Landau damping, bump-on-tail instabilities and more, are therefore highly influential in tokamak plasma dynamics. Purely fluid models are inherently incapable of capturing these effects, whereas the high dimensionality in purely kinetic models render them practically intractable for most relevant purposes.

        We consider a $\delta\!f$ decomposition model, with a macroscopic fluid background and microscopic kinetic correction, both fully coupled to each other. A similar manner of discretization is proposed to that used in the recent \texttt{STRUPHY} code \cite{Holderied_Possanner_Wang_2021, Holderied_2022, Li_et_al_2023} with a finite-element model for the background and a pseudo-particle/PiC model for the correction.

        The fluid background satisfies the full, non-linear, resistive, compressible, Hall MHD equations. \cite{Laakmann_Hu_Farrell_2022} introduces finite-element(-in-space) implicit timesteppers for the incompressible analogue to this system with structure-preserving (SP) properties in the ideal case, alongside parameter-robust preconditioners. We show that these timesteppers can derive from a finite-element-in-time (FET) (and finite-element-in-space) interpretation. The benefits of this reformulation are discussed, including the derivation of timesteppers that are higher order in time, and the quantifiable dissipative SP properties in the non-ideal, resistive case.
        
        We discuss possible options for extending this FET approach to timesteppers for the compressible case.

        The kinetic corrections satisfy linearized Boltzmann equations. Using a Lénard--Bernstein collision operator, these take Fokker--Planck-like forms \cite{Fokker_1914, Planck_1917} wherein pseudo-particles in the numerical model obey the neoclassical transport equations, with particle-independent Brownian drift terms. This offers a rigorous methodology for incorporating collisions into the particle transport model, without coupling the equations of motions for each particle.
        
        Works by Chen, Chacón et al. \cite{Chen_Chacón_Barnes_2011, Chacón_Chen_Barnes_2013, Chen_Chacón_2014, Chen_Chacón_2015} have developed structure-preserving particle pushers for neoclassical transport in the Vlasov equations, derived from Crank--Nicolson integrators. We show these too can can derive from a FET interpretation, similarly offering potential extensions to higher-order-in-time particle pushers. The FET formulation is used also to consider how the stochastic drift terms can be incorporated into the pushers. Stochastic gyrokinetic expansions are also discussed.

        Different options for the numerical implementation of these schemes are considered.

        Due to the efficacy of FET in the development of SP timesteppers for both the fluid and kinetic component, we hope this approach will prove effective in the future for developing SP timesteppers for the full hybrid model. We hope this will give us the opportunity to incorporate previously inaccessible kinetic effects into the highly effective, modern, finite-element MHD models.
    \end{abstract}
    
    
    \newpage
    \tableofcontents
    
    
    \newpage
    \pagenumbering{arabic}
    %\linenumbers\renewcommand\thelinenumber{\color{black!50}\arabic{linenumber}}
            \input{0 - introduction/main.tex}
        \part{Research}
            \input{1 - low-noise PiC models/main.tex}
            \input{2 - kinetic component/main.tex}
            \input{3 - fluid component/main.tex}
            \input{4 - numerical implementation/main.tex}
        \part{Project Overview}
            \input{5 - research plan/main.tex}
            \input{6 - summary/main.tex}
    
    
    %\section{}
    \newpage
    \pagenumbering{gobble}
        \printbibliography


    \newpage
    \pagenumbering{roman}
    \appendix
        \part{Appendices}
            \input{8 - Hilbert complexes/main.tex}
            \input{9 - weak conservation proofs/main.tex}
\end{document}

            \documentclass[12pt, a4paper]{report}

\input{template/main.tex}

\title{\BA{Title in Progress...}}
\author{Boris Andrews}
\affil{Mathematical Institute, University of Oxford}
\date{\today}


\begin{document}
    \pagenumbering{gobble}
    \maketitle
    
    
    \begin{abstract}
        Magnetic confinement reactors---in particular tokamaks---offer one of the most promising options for achieving practical nuclear fusion, with the potential to provide virtually limitless, clean energy. The theoretical and numerical modeling of tokamak plasmas is simultaneously an essential component of effective reactor design, and a great research barrier. Tokamak operational conditions exhibit comparatively low Knudsen numbers. Kinetic effects, including kinetic waves and instabilities, Landau damping, bump-on-tail instabilities and more, are therefore highly influential in tokamak plasma dynamics. Purely fluid models are inherently incapable of capturing these effects, whereas the high dimensionality in purely kinetic models render them practically intractable for most relevant purposes.

        We consider a $\delta\!f$ decomposition model, with a macroscopic fluid background and microscopic kinetic correction, both fully coupled to each other. A similar manner of discretization is proposed to that used in the recent \texttt{STRUPHY} code \cite{Holderied_Possanner_Wang_2021, Holderied_2022, Li_et_al_2023} with a finite-element model for the background and a pseudo-particle/PiC model for the correction.

        The fluid background satisfies the full, non-linear, resistive, compressible, Hall MHD equations. \cite{Laakmann_Hu_Farrell_2022} introduces finite-element(-in-space) implicit timesteppers for the incompressible analogue to this system with structure-preserving (SP) properties in the ideal case, alongside parameter-robust preconditioners. We show that these timesteppers can derive from a finite-element-in-time (FET) (and finite-element-in-space) interpretation. The benefits of this reformulation are discussed, including the derivation of timesteppers that are higher order in time, and the quantifiable dissipative SP properties in the non-ideal, resistive case.
        
        We discuss possible options for extending this FET approach to timesteppers for the compressible case.

        The kinetic corrections satisfy linearized Boltzmann equations. Using a Lénard--Bernstein collision operator, these take Fokker--Planck-like forms \cite{Fokker_1914, Planck_1917} wherein pseudo-particles in the numerical model obey the neoclassical transport equations, with particle-independent Brownian drift terms. This offers a rigorous methodology for incorporating collisions into the particle transport model, without coupling the equations of motions for each particle.
        
        Works by Chen, Chacón et al. \cite{Chen_Chacón_Barnes_2011, Chacón_Chen_Barnes_2013, Chen_Chacón_2014, Chen_Chacón_2015} have developed structure-preserving particle pushers for neoclassical transport in the Vlasov equations, derived from Crank--Nicolson integrators. We show these too can can derive from a FET interpretation, similarly offering potential extensions to higher-order-in-time particle pushers. The FET formulation is used also to consider how the stochastic drift terms can be incorporated into the pushers. Stochastic gyrokinetic expansions are also discussed.

        Different options for the numerical implementation of these schemes are considered.

        Due to the efficacy of FET in the development of SP timesteppers for both the fluid and kinetic component, we hope this approach will prove effective in the future for developing SP timesteppers for the full hybrid model. We hope this will give us the opportunity to incorporate previously inaccessible kinetic effects into the highly effective, modern, finite-element MHD models.
    \end{abstract}
    
    
    \newpage
    \tableofcontents
    
    
    \newpage
    \pagenumbering{arabic}
    %\linenumbers\renewcommand\thelinenumber{\color{black!50}\arabic{linenumber}}
            \input{0 - introduction/main.tex}
        \part{Research}
            \input{1 - low-noise PiC models/main.tex}
            \input{2 - kinetic component/main.tex}
            \input{3 - fluid component/main.tex}
            \input{4 - numerical implementation/main.tex}
        \part{Project Overview}
            \input{5 - research plan/main.tex}
            \input{6 - summary/main.tex}
    
    
    %\section{}
    \newpage
    \pagenumbering{gobble}
        \printbibliography


    \newpage
    \pagenumbering{roman}
    \appendix
        \part{Appendices}
            \input{8 - Hilbert complexes/main.tex}
            \input{9 - weak conservation proofs/main.tex}
\end{document}

    
    
    %\section{}
    \newpage
    \pagenumbering{gobble}
        \printbibliography


    \newpage
    \pagenumbering{roman}
    \appendix
        \part{Appendices}
            \documentclass[12pt, a4paper]{report}

\input{template/main.tex}

\title{\BA{Title in Progress...}}
\author{Boris Andrews}
\affil{Mathematical Institute, University of Oxford}
\date{\today}


\begin{document}
    \pagenumbering{gobble}
    \maketitle
    
    
    \begin{abstract}
        Magnetic confinement reactors---in particular tokamaks---offer one of the most promising options for achieving practical nuclear fusion, with the potential to provide virtually limitless, clean energy. The theoretical and numerical modeling of tokamak plasmas is simultaneously an essential component of effective reactor design, and a great research barrier. Tokamak operational conditions exhibit comparatively low Knudsen numbers. Kinetic effects, including kinetic waves and instabilities, Landau damping, bump-on-tail instabilities and more, are therefore highly influential in tokamak plasma dynamics. Purely fluid models are inherently incapable of capturing these effects, whereas the high dimensionality in purely kinetic models render them practically intractable for most relevant purposes.

        We consider a $\delta\!f$ decomposition model, with a macroscopic fluid background and microscopic kinetic correction, both fully coupled to each other. A similar manner of discretization is proposed to that used in the recent \texttt{STRUPHY} code \cite{Holderied_Possanner_Wang_2021, Holderied_2022, Li_et_al_2023} with a finite-element model for the background and a pseudo-particle/PiC model for the correction.

        The fluid background satisfies the full, non-linear, resistive, compressible, Hall MHD equations. \cite{Laakmann_Hu_Farrell_2022} introduces finite-element(-in-space) implicit timesteppers for the incompressible analogue to this system with structure-preserving (SP) properties in the ideal case, alongside parameter-robust preconditioners. We show that these timesteppers can derive from a finite-element-in-time (FET) (and finite-element-in-space) interpretation. The benefits of this reformulation are discussed, including the derivation of timesteppers that are higher order in time, and the quantifiable dissipative SP properties in the non-ideal, resistive case.
        
        We discuss possible options for extending this FET approach to timesteppers for the compressible case.

        The kinetic corrections satisfy linearized Boltzmann equations. Using a Lénard--Bernstein collision operator, these take Fokker--Planck-like forms \cite{Fokker_1914, Planck_1917} wherein pseudo-particles in the numerical model obey the neoclassical transport equations, with particle-independent Brownian drift terms. This offers a rigorous methodology for incorporating collisions into the particle transport model, without coupling the equations of motions for each particle.
        
        Works by Chen, Chacón et al. \cite{Chen_Chacón_Barnes_2011, Chacón_Chen_Barnes_2013, Chen_Chacón_2014, Chen_Chacón_2015} have developed structure-preserving particle pushers for neoclassical transport in the Vlasov equations, derived from Crank--Nicolson integrators. We show these too can can derive from a FET interpretation, similarly offering potential extensions to higher-order-in-time particle pushers. The FET formulation is used also to consider how the stochastic drift terms can be incorporated into the pushers. Stochastic gyrokinetic expansions are also discussed.

        Different options for the numerical implementation of these schemes are considered.

        Due to the efficacy of FET in the development of SP timesteppers for both the fluid and kinetic component, we hope this approach will prove effective in the future for developing SP timesteppers for the full hybrid model. We hope this will give us the opportunity to incorporate previously inaccessible kinetic effects into the highly effective, modern, finite-element MHD models.
    \end{abstract}
    
    
    \newpage
    \tableofcontents
    
    
    \newpage
    \pagenumbering{arabic}
    %\linenumbers\renewcommand\thelinenumber{\color{black!50}\arabic{linenumber}}
            \input{0 - introduction/main.tex}
        \part{Research}
            \input{1 - low-noise PiC models/main.tex}
            \input{2 - kinetic component/main.tex}
            \input{3 - fluid component/main.tex}
            \input{4 - numerical implementation/main.tex}
        \part{Project Overview}
            \input{5 - research plan/main.tex}
            \input{6 - summary/main.tex}
    
    
    %\section{}
    \newpage
    \pagenumbering{gobble}
        \printbibliography


    \newpage
    \pagenumbering{roman}
    \appendix
        \part{Appendices}
            \input{8 - Hilbert complexes/main.tex}
            \input{9 - weak conservation proofs/main.tex}
\end{document}

            \documentclass[12pt, a4paper]{report}

\input{template/main.tex}

\title{\BA{Title in Progress...}}
\author{Boris Andrews}
\affil{Mathematical Institute, University of Oxford}
\date{\today}


\begin{document}
    \pagenumbering{gobble}
    \maketitle
    
    
    \begin{abstract}
        Magnetic confinement reactors---in particular tokamaks---offer one of the most promising options for achieving practical nuclear fusion, with the potential to provide virtually limitless, clean energy. The theoretical and numerical modeling of tokamak plasmas is simultaneously an essential component of effective reactor design, and a great research barrier. Tokamak operational conditions exhibit comparatively low Knudsen numbers. Kinetic effects, including kinetic waves and instabilities, Landau damping, bump-on-tail instabilities and more, are therefore highly influential in tokamak plasma dynamics. Purely fluid models are inherently incapable of capturing these effects, whereas the high dimensionality in purely kinetic models render them practically intractable for most relevant purposes.

        We consider a $\delta\!f$ decomposition model, with a macroscopic fluid background and microscopic kinetic correction, both fully coupled to each other. A similar manner of discretization is proposed to that used in the recent \texttt{STRUPHY} code \cite{Holderied_Possanner_Wang_2021, Holderied_2022, Li_et_al_2023} with a finite-element model for the background and a pseudo-particle/PiC model for the correction.

        The fluid background satisfies the full, non-linear, resistive, compressible, Hall MHD equations. \cite{Laakmann_Hu_Farrell_2022} introduces finite-element(-in-space) implicit timesteppers for the incompressible analogue to this system with structure-preserving (SP) properties in the ideal case, alongside parameter-robust preconditioners. We show that these timesteppers can derive from a finite-element-in-time (FET) (and finite-element-in-space) interpretation. The benefits of this reformulation are discussed, including the derivation of timesteppers that are higher order in time, and the quantifiable dissipative SP properties in the non-ideal, resistive case.
        
        We discuss possible options for extending this FET approach to timesteppers for the compressible case.

        The kinetic corrections satisfy linearized Boltzmann equations. Using a Lénard--Bernstein collision operator, these take Fokker--Planck-like forms \cite{Fokker_1914, Planck_1917} wherein pseudo-particles in the numerical model obey the neoclassical transport equations, with particle-independent Brownian drift terms. This offers a rigorous methodology for incorporating collisions into the particle transport model, without coupling the equations of motions for each particle.
        
        Works by Chen, Chacón et al. \cite{Chen_Chacón_Barnes_2011, Chacón_Chen_Barnes_2013, Chen_Chacón_2014, Chen_Chacón_2015} have developed structure-preserving particle pushers for neoclassical transport in the Vlasov equations, derived from Crank--Nicolson integrators. We show these too can can derive from a FET interpretation, similarly offering potential extensions to higher-order-in-time particle pushers. The FET formulation is used also to consider how the stochastic drift terms can be incorporated into the pushers. Stochastic gyrokinetic expansions are also discussed.

        Different options for the numerical implementation of these schemes are considered.

        Due to the efficacy of FET in the development of SP timesteppers for both the fluid and kinetic component, we hope this approach will prove effective in the future for developing SP timesteppers for the full hybrid model. We hope this will give us the opportunity to incorporate previously inaccessible kinetic effects into the highly effective, modern, finite-element MHD models.
    \end{abstract}
    
    
    \newpage
    \tableofcontents
    
    
    \newpage
    \pagenumbering{arabic}
    %\linenumbers\renewcommand\thelinenumber{\color{black!50}\arabic{linenumber}}
            \input{0 - introduction/main.tex}
        \part{Research}
            \input{1 - low-noise PiC models/main.tex}
            \input{2 - kinetic component/main.tex}
            \input{3 - fluid component/main.tex}
            \input{4 - numerical implementation/main.tex}
        \part{Project Overview}
            \input{5 - research plan/main.tex}
            \input{6 - summary/main.tex}
    
    
    %\section{}
    \newpage
    \pagenumbering{gobble}
        \printbibliography


    \newpage
    \pagenumbering{roman}
    \appendix
        \part{Appendices}
            \input{8 - Hilbert complexes/main.tex}
            \input{9 - weak conservation proofs/main.tex}
\end{document}

\end{document}

            \documentclass[12pt, a4paper]{report}

\documentclass[12pt, a4paper]{report}

\input{template/main.tex}

\title{\BA{Title in Progress...}}
\author{Boris Andrews}
\affil{Mathematical Institute, University of Oxford}
\date{\today}


\begin{document}
    \pagenumbering{gobble}
    \maketitle
    
    
    \begin{abstract}
        Magnetic confinement reactors---in particular tokamaks---offer one of the most promising options for achieving practical nuclear fusion, with the potential to provide virtually limitless, clean energy. The theoretical and numerical modeling of tokamak plasmas is simultaneously an essential component of effective reactor design, and a great research barrier. Tokamak operational conditions exhibit comparatively low Knudsen numbers. Kinetic effects, including kinetic waves and instabilities, Landau damping, bump-on-tail instabilities and more, are therefore highly influential in tokamak plasma dynamics. Purely fluid models are inherently incapable of capturing these effects, whereas the high dimensionality in purely kinetic models render them practically intractable for most relevant purposes.

        We consider a $\delta\!f$ decomposition model, with a macroscopic fluid background and microscopic kinetic correction, both fully coupled to each other. A similar manner of discretization is proposed to that used in the recent \texttt{STRUPHY} code \cite{Holderied_Possanner_Wang_2021, Holderied_2022, Li_et_al_2023} with a finite-element model for the background and a pseudo-particle/PiC model for the correction.

        The fluid background satisfies the full, non-linear, resistive, compressible, Hall MHD equations. \cite{Laakmann_Hu_Farrell_2022} introduces finite-element(-in-space) implicit timesteppers for the incompressible analogue to this system with structure-preserving (SP) properties in the ideal case, alongside parameter-robust preconditioners. We show that these timesteppers can derive from a finite-element-in-time (FET) (and finite-element-in-space) interpretation. The benefits of this reformulation are discussed, including the derivation of timesteppers that are higher order in time, and the quantifiable dissipative SP properties in the non-ideal, resistive case.
        
        We discuss possible options for extending this FET approach to timesteppers for the compressible case.

        The kinetic corrections satisfy linearized Boltzmann equations. Using a Lénard--Bernstein collision operator, these take Fokker--Planck-like forms \cite{Fokker_1914, Planck_1917} wherein pseudo-particles in the numerical model obey the neoclassical transport equations, with particle-independent Brownian drift terms. This offers a rigorous methodology for incorporating collisions into the particle transport model, without coupling the equations of motions for each particle.
        
        Works by Chen, Chacón et al. \cite{Chen_Chacón_Barnes_2011, Chacón_Chen_Barnes_2013, Chen_Chacón_2014, Chen_Chacón_2015} have developed structure-preserving particle pushers for neoclassical transport in the Vlasov equations, derived from Crank--Nicolson integrators. We show these too can can derive from a FET interpretation, similarly offering potential extensions to higher-order-in-time particle pushers. The FET formulation is used also to consider how the stochastic drift terms can be incorporated into the pushers. Stochastic gyrokinetic expansions are also discussed.

        Different options for the numerical implementation of these schemes are considered.

        Due to the efficacy of FET in the development of SP timesteppers for both the fluid and kinetic component, we hope this approach will prove effective in the future for developing SP timesteppers for the full hybrid model. We hope this will give us the opportunity to incorporate previously inaccessible kinetic effects into the highly effective, modern, finite-element MHD models.
    \end{abstract}
    
    
    \newpage
    \tableofcontents
    
    
    \newpage
    \pagenumbering{arabic}
    %\linenumbers\renewcommand\thelinenumber{\color{black!50}\arabic{linenumber}}
            \input{0 - introduction/main.tex}
        \part{Research}
            \input{1 - low-noise PiC models/main.tex}
            \input{2 - kinetic component/main.tex}
            \input{3 - fluid component/main.tex}
            \input{4 - numerical implementation/main.tex}
        \part{Project Overview}
            \input{5 - research plan/main.tex}
            \input{6 - summary/main.tex}
    
    
    %\section{}
    \newpage
    \pagenumbering{gobble}
        \printbibliography


    \newpage
    \pagenumbering{roman}
    \appendix
        \part{Appendices}
            \input{8 - Hilbert complexes/main.tex}
            \input{9 - weak conservation proofs/main.tex}
\end{document}


\title{\BA{Title in Progress...}}
\author{Boris Andrews}
\affil{Mathematical Institute, University of Oxford}
\date{\today}


\begin{document}
    \pagenumbering{gobble}
    \maketitle
    
    
    \begin{abstract}
        Magnetic confinement reactors---in particular tokamaks---offer one of the most promising options for achieving practical nuclear fusion, with the potential to provide virtually limitless, clean energy. The theoretical and numerical modeling of tokamak plasmas is simultaneously an essential component of effective reactor design, and a great research barrier. Tokamak operational conditions exhibit comparatively low Knudsen numbers. Kinetic effects, including kinetic waves and instabilities, Landau damping, bump-on-tail instabilities and more, are therefore highly influential in tokamak plasma dynamics. Purely fluid models are inherently incapable of capturing these effects, whereas the high dimensionality in purely kinetic models render them practically intractable for most relevant purposes.

        We consider a $\delta\!f$ decomposition model, with a macroscopic fluid background and microscopic kinetic correction, both fully coupled to each other. A similar manner of discretization is proposed to that used in the recent \texttt{STRUPHY} code \cite{Holderied_Possanner_Wang_2021, Holderied_2022, Li_et_al_2023} with a finite-element model for the background and a pseudo-particle/PiC model for the correction.

        The fluid background satisfies the full, non-linear, resistive, compressible, Hall MHD equations. \cite{Laakmann_Hu_Farrell_2022} introduces finite-element(-in-space) implicit timesteppers for the incompressible analogue to this system with structure-preserving (SP) properties in the ideal case, alongside parameter-robust preconditioners. We show that these timesteppers can derive from a finite-element-in-time (FET) (and finite-element-in-space) interpretation. The benefits of this reformulation are discussed, including the derivation of timesteppers that are higher order in time, and the quantifiable dissipative SP properties in the non-ideal, resistive case.
        
        We discuss possible options for extending this FET approach to timesteppers for the compressible case.

        The kinetic corrections satisfy linearized Boltzmann equations. Using a Lénard--Bernstein collision operator, these take Fokker--Planck-like forms \cite{Fokker_1914, Planck_1917} wherein pseudo-particles in the numerical model obey the neoclassical transport equations, with particle-independent Brownian drift terms. This offers a rigorous methodology for incorporating collisions into the particle transport model, without coupling the equations of motions for each particle.
        
        Works by Chen, Chacón et al. \cite{Chen_Chacón_Barnes_2011, Chacón_Chen_Barnes_2013, Chen_Chacón_2014, Chen_Chacón_2015} have developed structure-preserving particle pushers for neoclassical transport in the Vlasov equations, derived from Crank--Nicolson integrators. We show these too can can derive from a FET interpretation, similarly offering potential extensions to higher-order-in-time particle pushers. The FET formulation is used also to consider how the stochastic drift terms can be incorporated into the pushers. Stochastic gyrokinetic expansions are also discussed.

        Different options for the numerical implementation of these schemes are considered.

        Due to the efficacy of FET in the development of SP timesteppers for both the fluid and kinetic component, we hope this approach will prove effective in the future for developing SP timesteppers for the full hybrid model. We hope this will give us the opportunity to incorporate previously inaccessible kinetic effects into the highly effective, modern, finite-element MHD models.
    \end{abstract}
    
    
    \newpage
    \tableofcontents
    
    
    \newpage
    \pagenumbering{arabic}
    %\linenumbers\renewcommand\thelinenumber{\color{black!50}\arabic{linenumber}}
            \documentclass[12pt, a4paper]{report}

\input{template/main.tex}

\title{\BA{Title in Progress...}}
\author{Boris Andrews}
\affil{Mathematical Institute, University of Oxford}
\date{\today}


\begin{document}
    \pagenumbering{gobble}
    \maketitle
    
    
    \begin{abstract}
        Magnetic confinement reactors---in particular tokamaks---offer one of the most promising options for achieving practical nuclear fusion, with the potential to provide virtually limitless, clean energy. The theoretical and numerical modeling of tokamak plasmas is simultaneously an essential component of effective reactor design, and a great research barrier. Tokamak operational conditions exhibit comparatively low Knudsen numbers. Kinetic effects, including kinetic waves and instabilities, Landau damping, bump-on-tail instabilities and more, are therefore highly influential in tokamak plasma dynamics. Purely fluid models are inherently incapable of capturing these effects, whereas the high dimensionality in purely kinetic models render them practically intractable for most relevant purposes.

        We consider a $\delta\!f$ decomposition model, with a macroscopic fluid background and microscopic kinetic correction, both fully coupled to each other. A similar manner of discretization is proposed to that used in the recent \texttt{STRUPHY} code \cite{Holderied_Possanner_Wang_2021, Holderied_2022, Li_et_al_2023} with a finite-element model for the background and a pseudo-particle/PiC model for the correction.

        The fluid background satisfies the full, non-linear, resistive, compressible, Hall MHD equations. \cite{Laakmann_Hu_Farrell_2022} introduces finite-element(-in-space) implicit timesteppers for the incompressible analogue to this system with structure-preserving (SP) properties in the ideal case, alongside parameter-robust preconditioners. We show that these timesteppers can derive from a finite-element-in-time (FET) (and finite-element-in-space) interpretation. The benefits of this reformulation are discussed, including the derivation of timesteppers that are higher order in time, and the quantifiable dissipative SP properties in the non-ideal, resistive case.
        
        We discuss possible options for extending this FET approach to timesteppers for the compressible case.

        The kinetic corrections satisfy linearized Boltzmann equations. Using a Lénard--Bernstein collision operator, these take Fokker--Planck-like forms \cite{Fokker_1914, Planck_1917} wherein pseudo-particles in the numerical model obey the neoclassical transport equations, with particle-independent Brownian drift terms. This offers a rigorous methodology for incorporating collisions into the particle transport model, without coupling the equations of motions for each particle.
        
        Works by Chen, Chacón et al. \cite{Chen_Chacón_Barnes_2011, Chacón_Chen_Barnes_2013, Chen_Chacón_2014, Chen_Chacón_2015} have developed structure-preserving particle pushers for neoclassical transport in the Vlasov equations, derived from Crank--Nicolson integrators. We show these too can can derive from a FET interpretation, similarly offering potential extensions to higher-order-in-time particle pushers. The FET formulation is used also to consider how the stochastic drift terms can be incorporated into the pushers. Stochastic gyrokinetic expansions are also discussed.

        Different options for the numerical implementation of these schemes are considered.

        Due to the efficacy of FET in the development of SP timesteppers for both the fluid and kinetic component, we hope this approach will prove effective in the future for developing SP timesteppers for the full hybrid model. We hope this will give us the opportunity to incorporate previously inaccessible kinetic effects into the highly effective, modern, finite-element MHD models.
    \end{abstract}
    
    
    \newpage
    \tableofcontents
    
    
    \newpage
    \pagenumbering{arabic}
    %\linenumbers\renewcommand\thelinenumber{\color{black!50}\arabic{linenumber}}
            \input{0 - introduction/main.tex}
        \part{Research}
            \input{1 - low-noise PiC models/main.tex}
            \input{2 - kinetic component/main.tex}
            \input{3 - fluid component/main.tex}
            \input{4 - numerical implementation/main.tex}
        \part{Project Overview}
            \input{5 - research plan/main.tex}
            \input{6 - summary/main.tex}
    
    
    %\section{}
    \newpage
    \pagenumbering{gobble}
        \printbibliography


    \newpage
    \pagenumbering{roman}
    \appendix
        \part{Appendices}
            \input{8 - Hilbert complexes/main.tex}
            \input{9 - weak conservation proofs/main.tex}
\end{document}

        \part{Research}
            \documentclass[12pt, a4paper]{report}

\input{template/main.tex}

\title{\BA{Title in Progress...}}
\author{Boris Andrews}
\affil{Mathematical Institute, University of Oxford}
\date{\today}


\begin{document}
    \pagenumbering{gobble}
    \maketitle
    
    
    \begin{abstract}
        Magnetic confinement reactors---in particular tokamaks---offer one of the most promising options for achieving practical nuclear fusion, with the potential to provide virtually limitless, clean energy. The theoretical and numerical modeling of tokamak plasmas is simultaneously an essential component of effective reactor design, and a great research barrier. Tokamak operational conditions exhibit comparatively low Knudsen numbers. Kinetic effects, including kinetic waves and instabilities, Landau damping, bump-on-tail instabilities and more, are therefore highly influential in tokamak plasma dynamics. Purely fluid models are inherently incapable of capturing these effects, whereas the high dimensionality in purely kinetic models render them practically intractable for most relevant purposes.

        We consider a $\delta\!f$ decomposition model, with a macroscopic fluid background and microscopic kinetic correction, both fully coupled to each other. A similar manner of discretization is proposed to that used in the recent \texttt{STRUPHY} code \cite{Holderied_Possanner_Wang_2021, Holderied_2022, Li_et_al_2023} with a finite-element model for the background and a pseudo-particle/PiC model for the correction.

        The fluid background satisfies the full, non-linear, resistive, compressible, Hall MHD equations. \cite{Laakmann_Hu_Farrell_2022} introduces finite-element(-in-space) implicit timesteppers for the incompressible analogue to this system with structure-preserving (SP) properties in the ideal case, alongside parameter-robust preconditioners. We show that these timesteppers can derive from a finite-element-in-time (FET) (and finite-element-in-space) interpretation. The benefits of this reformulation are discussed, including the derivation of timesteppers that are higher order in time, and the quantifiable dissipative SP properties in the non-ideal, resistive case.
        
        We discuss possible options for extending this FET approach to timesteppers for the compressible case.

        The kinetic corrections satisfy linearized Boltzmann equations. Using a Lénard--Bernstein collision operator, these take Fokker--Planck-like forms \cite{Fokker_1914, Planck_1917} wherein pseudo-particles in the numerical model obey the neoclassical transport equations, with particle-independent Brownian drift terms. This offers a rigorous methodology for incorporating collisions into the particle transport model, without coupling the equations of motions for each particle.
        
        Works by Chen, Chacón et al. \cite{Chen_Chacón_Barnes_2011, Chacón_Chen_Barnes_2013, Chen_Chacón_2014, Chen_Chacón_2015} have developed structure-preserving particle pushers for neoclassical transport in the Vlasov equations, derived from Crank--Nicolson integrators. We show these too can can derive from a FET interpretation, similarly offering potential extensions to higher-order-in-time particle pushers. The FET formulation is used also to consider how the stochastic drift terms can be incorporated into the pushers. Stochastic gyrokinetic expansions are also discussed.

        Different options for the numerical implementation of these schemes are considered.

        Due to the efficacy of FET in the development of SP timesteppers for both the fluid and kinetic component, we hope this approach will prove effective in the future for developing SP timesteppers for the full hybrid model. We hope this will give us the opportunity to incorporate previously inaccessible kinetic effects into the highly effective, modern, finite-element MHD models.
    \end{abstract}
    
    
    \newpage
    \tableofcontents
    
    
    \newpage
    \pagenumbering{arabic}
    %\linenumbers\renewcommand\thelinenumber{\color{black!50}\arabic{linenumber}}
            \input{0 - introduction/main.tex}
        \part{Research}
            \input{1 - low-noise PiC models/main.tex}
            \input{2 - kinetic component/main.tex}
            \input{3 - fluid component/main.tex}
            \input{4 - numerical implementation/main.tex}
        \part{Project Overview}
            \input{5 - research plan/main.tex}
            \input{6 - summary/main.tex}
    
    
    %\section{}
    \newpage
    \pagenumbering{gobble}
        \printbibliography


    \newpage
    \pagenumbering{roman}
    \appendix
        \part{Appendices}
            \input{8 - Hilbert complexes/main.tex}
            \input{9 - weak conservation proofs/main.tex}
\end{document}

            \documentclass[12pt, a4paper]{report}

\input{template/main.tex}

\title{\BA{Title in Progress...}}
\author{Boris Andrews}
\affil{Mathematical Institute, University of Oxford}
\date{\today}


\begin{document}
    \pagenumbering{gobble}
    \maketitle
    
    
    \begin{abstract}
        Magnetic confinement reactors---in particular tokamaks---offer one of the most promising options for achieving practical nuclear fusion, with the potential to provide virtually limitless, clean energy. The theoretical and numerical modeling of tokamak plasmas is simultaneously an essential component of effective reactor design, and a great research barrier. Tokamak operational conditions exhibit comparatively low Knudsen numbers. Kinetic effects, including kinetic waves and instabilities, Landau damping, bump-on-tail instabilities and more, are therefore highly influential in tokamak plasma dynamics. Purely fluid models are inherently incapable of capturing these effects, whereas the high dimensionality in purely kinetic models render them practically intractable for most relevant purposes.

        We consider a $\delta\!f$ decomposition model, with a macroscopic fluid background and microscopic kinetic correction, both fully coupled to each other. A similar manner of discretization is proposed to that used in the recent \texttt{STRUPHY} code \cite{Holderied_Possanner_Wang_2021, Holderied_2022, Li_et_al_2023} with a finite-element model for the background and a pseudo-particle/PiC model for the correction.

        The fluid background satisfies the full, non-linear, resistive, compressible, Hall MHD equations. \cite{Laakmann_Hu_Farrell_2022} introduces finite-element(-in-space) implicit timesteppers for the incompressible analogue to this system with structure-preserving (SP) properties in the ideal case, alongside parameter-robust preconditioners. We show that these timesteppers can derive from a finite-element-in-time (FET) (and finite-element-in-space) interpretation. The benefits of this reformulation are discussed, including the derivation of timesteppers that are higher order in time, and the quantifiable dissipative SP properties in the non-ideal, resistive case.
        
        We discuss possible options for extending this FET approach to timesteppers for the compressible case.

        The kinetic corrections satisfy linearized Boltzmann equations. Using a Lénard--Bernstein collision operator, these take Fokker--Planck-like forms \cite{Fokker_1914, Planck_1917} wherein pseudo-particles in the numerical model obey the neoclassical transport equations, with particle-independent Brownian drift terms. This offers a rigorous methodology for incorporating collisions into the particle transport model, without coupling the equations of motions for each particle.
        
        Works by Chen, Chacón et al. \cite{Chen_Chacón_Barnes_2011, Chacón_Chen_Barnes_2013, Chen_Chacón_2014, Chen_Chacón_2015} have developed structure-preserving particle pushers for neoclassical transport in the Vlasov equations, derived from Crank--Nicolson integrators. We show these too can can derive from a FET interpretation, similarly offering potential extensions to higher-order-in-time particle pushers. The FET formulation is used also to consider how the stochastic drift terms can be incorporated into the pushers. Stochastic gyrokinetic expansions are also discussed.

        Different options for the numerical implementation of these schemes are considered.

        Due to the efficacy of FET in the development of SP timesteppers for both the fluid and kinetic component, we hope this approach will prove effective in the future for developing SP timesteppers for the full hybrid model. We hope this will give us the opportunity to incorporate previously inaccessible kinetic effects into the highly effective, modern, finite-element MHD models.
    \end{abstract}
    
    
    \newpage
    \tableofcontents
    
    
    \newpage
    \pagenumbering{arabic}
    %\linenumbers\renewcommand\thelinenumber{\color{black!50}\arabic{linenumber}}
            \input{0 - introduction/main.tex}
        \part{Research}
            \input{1 - low-noise PiC models/main.tex}
            \input{2 - kinetic component/main.tex}
            \input{3 - fluid component/main.tex}
            \input{4 - numerical implementation/main.tex}
        \part{Project Overview}
            \input{5 - research plan/main.tex}
            \input{6 - summary/main.tex}
    
    
    %\section{}
    \newpage
    \pagenumbering{gobble}
        \printbibliography


    \newpage
    \pagenumbering{roman}
    \appendix
        \part{Appendices}
            \input{8 - Hilbert complexes/main.tex}
            \input{9 - weak conservation proofs/main.tex}
\end{document}

            \documentclass[12pt, a4paper]{report}

\input{template/main.tex}

\title{\BA{Title in Progress...}}
\author{Boris Andrews}
\affil{Mathematical Institute, University of Oxford}
\date{\today}


\begin{document}
    \pagenumbering{gobble}
    \maketitle
    
    
    \begin{abstract}
        Magnetic confinement reactors---in particular tokamaks---offer one of the most promising options for achieving practical nuclear fusion, with the potential to provide virtually limitless, clean energy. The theoretical and numerical modeling of tokamak plasmas is simultaneously an essential component of effective reactor design, and a great research barrier. Tokamak operational conditions exhibit comparatively low Knudsen numbers. Kinetic effects, including kinetic waves and instabilities, Landau damping, bump-on-tail instabilities and more, are therefore highly influential in tokamak plasma dynamics. Purely fluid models are inherently incapable of capturing these effects, whereas the high dimensionality in purely kinetic models render them practically intractable for most relevant purposes.

        We consider a $\delta\!f$ decomposition model, with a macroscopic fluid background and microscopic kinetic correction, both fully coupled to each other. A similar manner of discretization is proposed to that used in the recent \texttt{STRUPHY} code \cite{Holderied_Possanner_Wang_2021, Holderied_2022, Li_et_al_2023} with a finite-element model for the background and a pseudo-particle/PiC model for the correction.

        The fluid background satisfies the full, non-linear, resistive, compressible, Hall MHD equations. \cite{Laakmann_Hu_Farrell_2022} introduces finite-element(-in-space) implicit timesteppers for the incompressible analogue to this system with structure-preserving (SP) properties in the ideal case, alongside parameter-robust preconditioners. We show that these timesteppers can derive from a finite-element-in-time (FET) (and finite-element-in-space) interpretation. The benefits of this reformulation are discussed, including the derivation of timesteppers that are higher order in time, and the quantifiable dissipative SP properties in the non-ideal, resistive case.
        
        We discuss possible options for extending this FET approach to timesteppers for the compressible case.

        The kinetic corrections satisfy linearized Boltzmann equations. Using a Lénard--Bernstein collision operator, these take Fokker--Planck-like forms \cite{Fokker_1914, Planck_1917} wherein pseudo-particles in the numerical model obey the neoclassical transport equations, with particle-independent Brownian drift terms. This offers a rigorous methodology for incorporating collisions into the particle transport model, without coupling the equations of motions for each particle.
        
        Works by Chen, Chacón et al. \cite{Chen_Chacón_Barnes_2011, Chacón_Chen_Barnes_2013, Chen_Chacón_2014, Chen_Chacón_2015} have developed structure-preserving particle pushers for neoclassical transport in the Vlasov equations, derived from Crank--Nicolson integrators. We show these too can can derive from a FET interpretation, similarly offering potential extensions to higher-order-in-time particle pushers. The FET formulation is used also to consider how the stochastic drift terms can be incorporated into the pushers. Stochastic gyrokinetic expansions are also discussed.

        Different options for the numerical implementation of these schemes are considered.

        Due to the efficacy of FET in the development of SP timesteppers for both the fluid and kinetic component, we hope this approach will prove effective in the future for developing SP timesteppers for the full hybrid model. We hope this will give us the opportunity to incorporate previously inaccessible kinetic effects into the highly effective, modern, finite-element MHD models.
    \end{abstract}
    
    
    \newpage
    \tableofcontents
    
    
    \newpage
    \pagenumbering{arabic}
    %\linenumbers\renewcommand\thelinenumber{\color{black!50}\arabic{linenumber}}
            \input{0 - introduction/main.tex}
        \part{Research}
            \input{1 - low-noise PiC models/main.tex}
            \input{2 - kinetic component/main.tex}
            \input{3 - fluid component/main.tex}
            \input{4 - numerical implementation/main.tex}
        \part{Project Overview}
            \input{5 - research plan/main.tex}
            \input{6 - summary/main.tex}
    
    
    %\section{}
    \newpage
    \pagenumbering{gobble}
        \printbibliography


    \newpage
    \pagenumbering{roman}
    \appendix
        \part{Appendices}
            \input{8 - Hilbert complexes/main.tex}
            \input{9 - weak conservation proofs/main.tex}
\end{document}

            \documentclass[12pt, a4paper]{report}

\input{template/main.tex}

\title{\BA{Title in Progress...}}
\author{Boris Andrews}
\affil{Mathematical Institute, University of Oxford}
\date{\today}


\begin{document}
    \pagenumbering{gobble}
    \maketitle
    
    
    \begin{abstract}
        Magnetic confinement reactors---in particular tokamaks---offer one of the most promising options for achieving practical nuclear fusion, with the potential to provide virtually limitless, clean energy. The theoretical and numerical modeling of tokamak plasmas is simultaneously an essential component of effective reactor design, and a great research barrier. Tokamak operational conditions exhibit comparatively low Knudsen numbers. Kinetic effects, including kinetic waves and instabilities, Landau damping, bump-on-tail instabilities and more, are therefore highly influential in tokamak plasma dynamics. Purely fluid models are inherently incapable of capturing these effects, whereas the high dimensionality in purely kinetic models render them practically intractable for most relevant purposes.

        We consider a $\delta\!f$ decomposition model, with a macroscopic fluid background and microscopic kinetic correction, both fully coupled to each other. A similar manner of discretization is proposed to that used in the recent \texttt{STRUPHY} code \cite{Holderied_Possanner_Wang_2021, Holderied_2022, Li_et_al_2023} with a finite-element model for the background and a pseudo-particle/PiC model for the correction.

        The fluid background satisfies the full, non-linear, resistive, compressible, Hall MHD equations. \cite{Laakmann_Hu_Farrell_2022} introduces finite-element(-in-space) implicit timesteppers for the incompressible analogue to this system with structure-preserving (SP) properties in the ideal case, alongside parameter-robust preconditioners. We show that these timesteppers can derive from a finite-element-in-time (FET) (and finite-element-in-space) interpretation. The benefits of this reformulation are discussed, including the derivation of timesteppers that are higher order in time, and the quantifiable dissipative SP properties in the non-ideal, resistive case.
        
        We discuss possible options for extending this FET approach to timesteppers for the compressible case.

        The kinetic corrections satisfy linearized Boltzmann equations. Using a Lénard--Bernstein collision operator, these take Fokker--Planck-like forms \cite{Fokker_1914, Planck_1917} wherein pseudo-particles in the numerical model obey the neoclassical transport equations, with particle-independent Brownian drift terms. This offers a rigorous methodology for incorporating collisions into the particle transport model, without coupling the equations of motions for each particle.
        
        Works by Chen, Chacón et al. \cite{Chen_Chacón_Barnes_2011, Chacón_Chen_Barnes_2013, Chen_Chacón_2014, Chen_Chacón_2015} have developed structure-preserving particle pushers for neoclassical transport in the Vlasov equations, derived from Crank--Nicolson integrators. We show these too can can derive from a FET interpretation, similarly offering potential extensions to higher-order-in-time particle pushers. The FET formulation is used also to consider how the stochastic drift terms can be incorporated into the pushers. Stochastic gyrokinetic expansions are also discussed.

        Different options for the numerical implementation of these schemes are considered.

        Due to the efficacy of FET in the development of SP timesteppers for both the fluid and kinetic component, we hope this approach will prove effective in the future for developing SP timesteppers for the full hybrid model. We hope this will give us the opportunity to incorporate previously inaccessible kinetic effects into the highly effective, modern, finite-element MHD models.
    \end{abstract}
    
    
    \newpage
    \tableofcontents
    
    
    \newpage
    \pagenumbering{arabic}
    %\linenumbers\renewcommand\thelinenumber{\color{black!50}\arabic{linenumber}}
            \input{0 - introduction/main.tex}
        \part{Research}
            \input{1 - low-noise PiC models/main.tex}
            \input{2 - kinetic component/main.tex}
            \input{3 - fluid component/main.tex}
            \input{4 - numerical implementation/main.tex}
        \part{Project Overview}
            \input{5 - research plan/main.tex}
            \input{6 - summary/main.tex}
    
    
    %\section{}
    \newpage
    \pagenumbering{gobble}
        \printbibliography


    \newpage
    \pagenumbering{roman}
    \appendix
        \part{Appendices}
            \input{8 - Hilbert complexes/main.tex}
            \input{9 - weak conservation proofs/main.tex}
\end{document}

        \part{Project Overview}
            \documentclass[12pt, a4paper]{report}

\input{template/main.tex}

\title{\BA{Title in Progress...}}
\author{Boris Andrews}
\affil{Mathematical Institute, University of Oxford}
\date{\today}


\begin{document}
    \pagenumbering{gobble}
    \maketitle
    
    
    \begin{abstract}
        Magnetic confinement reactors---in particular tokamaks---offer one of the most promising options for achieving practical nuclear fusion, with the potential to provide virtually limitless, clean energy. The theoretical and numerical modeling of tokamak plasmas is simultaneously an essential component of effective reactor design, and a great research barrier. Tokamak operational conditions exhibit comparatively low Knudsen numbers. Kinetic effects, including kinetic waves and instabilities, Landau damping, bump-on-tail instabilities and more, are therefore highly influential in tokamak plasma dynamics. Purely fluid models are inherently incapable of capturing these effects, whereas the high dimensionality in purely kinetic models render them practically intractable for most relevant purposes.

        We consider a $\delta\!f$ decomposition model, with a macroscopic fluid background and microscopic kinetic correction, both fully coupled to each other. A similar manner of discretization is proposed to that used in the recent \texttt{STRUPHY} code \cite{Holderied_Possanner_Wang_2021, Holderied_2022, Li_et_al_2023} with a finite-element model for the background and a pseudo-particle/PiC model for the correction.

        The fluid background satisfies the full, non-linear, resistive, compressible, Hall MHD equations. \cite{Laakmann_Hu_Farrell_2022} introduces finite-element(-in-space) implicit timesteppers for the incompressible analogue to this system with structure-preserving (SP) properties in the ideal case, alongside parameter-robust preconditioners. We show that these timesteppers can derive from a finite-element-in-time (FET) (and finite-element-in-space) interpretation. The benefits of this reformulation are discussed, including the derivation of timesteppers that are higher order in time, and the quantifiable dissipative SP properties in the non-ideal, resistive case.
        
        We discuss possible options for extending this FET approach to timesteppers for the compressible case.

        The kinetic corrections satisfy linearized Boltzmann equations. Using a Lénard--Bernstein collision operator, these take Fokker--Planck-like forms \cite{Fokker_1914, Planck_1917} wherein pseudo-particles in the numerical model obey the neoclassical transport equations, with particle-independent Brownian drift terms. This offers a rigorous methodology for incorporating collisions into the particle transport model, without coupling the equations of motions for each particle.
        
        Works by Chen, Chacón et al. \cite{Chen_Chacón_Barnes_2011, Chacón_Chen_Barnes_2013, Chen_Chacón_2014, Chen_Chacón_2015} have developed structure-preserving particle pushers for neoclassical transport in the Vlasov equations, derived from Crank--Nicolson integrators. We show these too can can derive from a FET interpretation, similarly offering potential extensions to higher-order-in-time particle pushers. The FET formulation is used also to consider how the stochastic drift terms can be incorporated into the pushers. Stochastic gyrokinetic expansions are also discussed.

        Different options for the numerical implementation of these schemes are considered.

        Due to the efficacy of FET in the development of SP timesteppers for both the fluid and kinetic component, we hope this approach will prove effective in the future for developing SP timesteppers for the full hybrid model. We hope this will give us the opportunity to incorporate previously inaccessible kinetic effects into the highly effective, modern, finite-element MHD models.
    \end{abstract}
    
    
    \newpage
    \tableofcontents
    
    
    \newpage
    \pagenumbering{arabic}
    %\linenumbers\renewcommand\thelinenumber{\color{black!50}\arabic{linenumber}}
            \input{0 - introduction/main.tex}
        \part{Research}
            \input{1 - low-noise PiC models/main.tex}
            \input{2 - kinetic component/main.tex}
            \input{3 - fluid component/main.tex}
            \input{4 - numerical implementation/main.tex}
        \part{Project Overview}
            \input{5 - research plan/main.tex}
            \input{6 - summary/main.tex}
    
    
    %\section{}
    \newpage
    \pagenumbering{gobble}
        \printbibliography


    \newpage
    \pagenumbering{roman}
    \appendix
        \part{Appendices}
            \input{8 - Hilbert complexes/main.tex}
            \input{9 - weak conservation proofs/main.tex}
\end{document}

            \documentclass[12pt, a4paper]{report}

\input{template/main.tex}

\title{\BA{Title in Progress...}}
\author{Boris Andrews}
\affil{Mathematical Institute, University of Oxford}
\date{\today}


\begin{document}
    \pagenumbering{gobble}
    \maketitle
    
    
    \begin{abstract}
        Magnetic confinement reactors---in particular tokamaks---offer one of the most promising options for achieving practical nuclear fusion, with the potential to provide virtually limitless, clean energy. The theoretical and numerical modeling of tokamak plasmas is simultaneously an essential component of effective reactor design, and a great research barrier. Tokamak operational conditions exhibit comparatively low Knudsen numbers. Kinetic effects, including kinetic waves and instabilities, Landau damping, bump-on-tail instabilities and more, are therefore highly influential in tokamak plasma dynamics. Purely fluid models are inherently incapable of capturing these effects, whereas the high dimensionality in purely kinetic models render them practically intractable for most relevant purposes.

        We consider a $\delta\!f$ decomposition model, with a macroscopic fluid background and microscopic kinetic correction, both fully coupled to each other. A similar manner of discretization is proposed to that used in the recent \texttt{STRUPHY} code \cite{Holderied_Possanner_Wang_2021, Holderied_2022, Li_et_al_2023} with a finite-element model for the background and a pseudo-particle/PiC model for the correction.

        The fluid background satisfies the full, non-linear, resistive, compressible, Hall MHD equations. \cite{Laakmann_Hu_Farrell_2022} introduces finite-element(-in-space) implicit timesteppers for the incompressible analogue to this system with structure-preserving (SP) properties in the ideal case, alongside parameter-robust preconditioners. We show that these timesteppers can derive from a finite-element-in-time (FET) (and finite-element-in-space) interpretation. The benefits of this reformulation are discussed, including the derivation of timesteppers that are higher order in time, and the quantifiable dissipative SP properties in the non-ideal, resistive case.
        
        We discuss possible options for extending this FET approach to timesteppers for the compressible case.

        The kinetic corrections satisfy linearized Boltzmann equations. Using a Lénard--Bernstein collision operator, these take Fokker--Planck-like forms \cite{Fokker_1914, Planck_1917} wherein pseudo-particles in the numerical model obey the neoclassical transport equations, with particle-independent Brownian drift terms. This offers a rigorous methodology for incorporating collisions into the particle transport model, without coupling the equations of motions for each particle.
        
        Works by Chen, Chacón et al. \cite{Chen_Chacón_Barnes_2011, Chacón_Chen_Barnes_2013, Chen_Chacón_2014, Chen_Chacón_2015} have developed structure-preserving particle pushers for neoclassical transport in the Vlasov equations, derived from Crank--Nicolson integrators. We show these too can can derive from a FET interpretation, similarly offering potential extensions to higher-order-in-time particle pushers. The FET formulation is used also to consider how the stochastic drift terms can be incorporated into the pushers. Stochastic gyrokinetic expansions are also discussed.

        Different options for the numerical implementation of these schemes are considered.

        Due to the efficacy of FET in the development of SP timesteppers for both the fluid and kinetic component, we hope this approach will prove effective in the future for developing SP timesteppers for the full hybrid model. We hope this will give us the opportunity to incorporate previously inaccessible kinetic effects into the highly effective, modern, finite-element MHD models.
    \end{abstract}
    
    
    \newpage
    \tableofcontents
    
    
    \newpage
    \pagenumbering{arabic}
    %\linenumbers\renewcommand\thelinenumber{\color{black!50}\arabic{linenumber}}
            \input{0 - introduction/main.tex}
        \part{Research}
            \input{1 - low-noise PiC models/main.tex}
            \input{2 - kinetic component/main.tex}
            \input{3 - fluid component/main.tex}
            \input{4 - numerical implementation/main.tex}
        \part{Project Overview}
            \input{5 - research plan/main.tex}
            \input{6 - summary/main.tex}
    
    
    %\section{}
    \newpage
    \pagenumbering{gobble}
        \printbibliography


    \newpage
    \pagenumbering{roman}
    \appendix
        \part{Appendices}
            \input{8 - Hilbert complexes/main.tex}
            \input{9 - weak conservation proofs/main.tex}
\end{document}

    
    
    %\section{}
    \newpage
    \pagenumbering{gobble}
        \printbibliography


    \newpage
    \pagenumbering{roman}
    \appendix
        \part{Appendices}
            \documentclass[12pt, a4paper]{report}

\input{template/main.tex}

\title{\BA{Title in Progress...}}
\author{Boris Andrews}
\affil{Mathematical Institute, University of Oxford}
\date{\today}


\begin{document}
    \pagenumbering{gobble}
    \maketitle
    
    
    \begin{abstract}
        Magnetic confinement reactors---in particular tokamaks---offer one of the most promising options for achieving practical nuclear fusion, with the potential to provide virtually limitless, clean energy. The theoretical and numerical modeling of tokamak plasmas is simultaneously an essential component of effective reactor design, and a great research barrier. Tokamak operational conditions exhibit comparatively low Knudsen numbers. Kinetic effects, including kinetic waves and instabilities, Landau damping, bump-on-tail instabilities and more, are therefore highly influential in tokamak plasma dynamics. Purely fluid models are inherently incapable of capturing these effects, whereas the high dimensionality in purely kinetic models render them practically intractable for most relevant purposes.

        We consider a $\delta\!f$ decomposition model, with a macroscopic fluid background and microscopic kinetic correction, both fully coupled to each other. A similar manner of discretization is proposed to that used in the recent \texttt{STRUPHY} code \cite{Holderied_Possanner_Wang_2021, Holderied_2022, Li_et_al_2023} with a finite-element model for the background and a pseudo-particle/PiC model for the correction.

        The fluid background satisfies the full, non-linear, resistive, compressible, Hall MHD equations. \cite{Laakmann_Hu_Farrell_2022} introduces finite-element(-in-space) implicit timesteppers for the incompressible analogue to this system with structure-preserving (SP) properties in the ideal case, alongside parameter-robust preconditioners. We show that these timesteppers can derive from a finite-element-in-time (FET) (and finite-element-in-space) interpretation. The benefits of this reformulation are discussed, including the derivation of timesteppers that are higher order in time, and the quantifiable dissipative SP properties in the non-ideal, resistive case.
        
        We discuss possible options for extending this FET approach to timesteppers for the compressible case.

        The kinetic corrections satisfy linearized Boltzmann equations. Using a Lénard--Bernstein collision operator, these take Fokker--Planck-like forms \cite{Fokker_1914, Planck_1917} wherein pseudo-particles in the numerical model obey the neoclassical transport equations, with particle-independent Brownian drift terms. This offers a rigorous methodology for incorporating collisions into the particle transport model, without coupling the equations of motions for each particle.
        
        Works by Chen, Chacón et al. \cite{Chen_Chacón_Barnes_2011, Chacón_Chen_Barnes_2013, Chen_Chacón_2014, Chen_Chacón_2015} have developed structure-preserving particle pushers for neoclassical transport in the Vlasov equations, derived from Crank--Nicolson integrators. We show these too can can derive from a FET interpretation, similarly offering potential extensions to higher-order-in-time particle pushers. The FET formulation is used also to consider how the stochastic drift terms can be incorporated into the pushers. Stochastic gyrokinetic expansions are also discussed.

        Different options for the numerical implementation of these schemes are considered.

        Due to the efficacy of FET in the development of SP timesteppers for both the fluid and kinetic component, we hope this approach will prove effective in the future for developing SP timesteppers for the full hybrid model. We hope this will give us the opportunity to incorporate previously inaccessible kinetic effects into the highly effective, modern, finite-element MHD models.
    \end{abstract}
    
    
    \newpage
    \tableofcontents
    
    
    \newpage
    \pagenumbering{arabic}
    %\linenumbers\renewcommand\thelinenumber{\color{black!50}\arabic{linenumber}}
            \input{0 - introduction/main.tex}
        \part{Research}
            \input{1 - low-noise PiC models/main.tex}
            \input{2 - kinetic component/main.tex}
            \input{3 - fluid component/main.tex}
            \input{4 - numerical implementation/main.tex}
        \part{Project Overview}
            \input{5 - research plan/main.tex}
            \input{6 - summary/main.tex}
    
    
    %\section{}
    \newpage
    \pagenumbering{gobble}
        \printbibliography


    \newpage
    \pagenumbering{roman}
    \appendix
        \part{Appendices}
            \input{8 - Hilbert complexes/main.tex}
            \input{9 - weak conservation proofs/main.tex}
\end{document}

            \documentclass[12pt, a4paper]{report}

\input{template/main.tex}

\title{\BA{Title in Progress...}}
\author{Boris Andrews}
\affil{Mathematical Institute, University of Oxford}
\date{\today}


\begin{document}
    \pagenumbering{gobble}
    \maketitle
    
    
    \begin{abstract}
        Magnetic confinement reactors---in particular tokamaks---offer one of the most promising options for achieving practical nuclear fusion, with the potential to provide virtually limitless, clean energy. The theoretical and numerical modeling of tokamak plasmas is simultaneously an essential component of effective reactor design, and a great research barrier. Tokamak operational conditions exhibit comparatively low Knudsen numbers. Kinetic effects, including kinetic waves and instabilities, Landau damping, bump-on-tail instabilities and more, are therefore highly influential in tokamak plasma dynamics. Purely fluid models are inherently incapable of capturing these effects, whereas the high dimensionality in purely kinetic models render them practically intractable for most relevant purposes.

        We consider a $\delta\!f$ decomposition model, with a macroscopic fluid background and microscopic kinetic correction, both fully coupled to each other. A similar manner of discretization is proposed to that used in the recent \texttt{STRUPHY} code \cite{Holderied_Possanner_Wang_2021, Holderied_2022, Li_et_al_2023} with a finite-element model for the background and a pseudo-particle/PiC model for the correction.

        The fluid background satisfies the full, non-linear, resistive, compressible, Hall MHD equations. \cite{Laakmann_Hu_Farrell_2022} introduces finite-element(-in-space) implicit timesteppers for the incompressible analogue to this system with structure-preserving (SP) properties in the ideal case, alongside parameter-robust preconditioners. We show that these timesteppers can derive from a finite-element-in-time (FET) (and finite-element-in-space) interpretation. The benefits of this reformulation are discussed, including the derivation of timesteppers that are higher order in time, and the quantifiable dissipative SP properties in the non-ideal, resistive case.
        
        We discuss possible options for extending this FET approach to timesteppers for the compressible case.

        The kinetic corrections satisfy linearized Boltzmann equations. Using a Lénard--Bernstein collision operator, these take Fokker--Planck-like forms \cite{Fokker_1914, Planck_1917} wherein pseudo-particles in the numerical model obey the neoclassical transport equations, with particle-independent Brownian drift terms. This offers a rigorous methodology for incorporating collisions into the particle transport model, without coupling the equations of motions for each particle.
        
        Works by Chen, Chacón et al. \cite{Chen_Chacón_Barnes_2011, Chacón_Chen_Barnes_2013, Chen_Chacón_2014, Chen_Chacón_2015} have developed structure-preserving particle pushers for neoclassical transport in the Vlasov equations, derived from Crank--Nicolson integrators. We show these too can can derive from a FET interpretation, similarly offering potential extensions to higher-order-in-time particle pushers. The FET formulation is used also to consider how the stochastic drift terms can be incorporated into the pushers. Stochastic gyrokinetic expansions are also discussed.

        Different options for the numerical implementation of these schemes are considered.

        Due to the efficacy of FET in the development of SP timesteppers for both the fluid and kinetic component, we hope this approach will prove effective in the future for developing SP timesteppers for the full hybrid model. We hope this will give us the opportunity to incorporate previously inaccessible kinetic effects into the highly effective, modern, finite-element MHD models.
    \end{abstract}
    
    
    \newpage
    \tableofcontents
    
    
    \newpage
    \pagenumbering{arabic}
    %\linenumbers\renewcommand\thelinenumber{\color{black!50}\arabic{linenumber}}
            \input{0 - introduction/main.tex}
        \part{Research}
            \input{1 - low-noise PiC models/main.tex}
            \input{2 - kinetic component/main.tex}
            \input{3 - fluid component/main.tex}
            \input{4 - numerical implementation/main.tex}
        \part{Project Overview}
            \input{5 - research plan/main.tex}
            \input{6 - summary/main.tex}
    
    
    %\section{}
    \newpage
    \pagenumbering{gobble}
        \printbibliography


    \newpage
    \pagenumbering{roman}
    \appendix
        \part{Appendices}
            \input{8 - Hilbert complexes/main.tex}
            \input{9 - weak conservation proofs/main.tex}
\end{document}

\end{document}

        \part{Project Overview}
            \documentclass[12pt, a4paper]{report}

\documentclass[12pt, a4paper]{report}

\input{template/main.tex}

\title{\BA{Title in Progress...}}
\author{Boris Andrews}
\affil{Mathematical Institute, University of Oxford}
\date{\today}


\begin{document}
    \pagenumbering{gobble}
    \maketitle
    
    
    \begin{abstract}
        Magnetic confinement reactors---in particular tokamaks---offer one of the most promising options for achieving practical nuclear fusion, with the potential to provide virtually limitless, clean energy. The theoretical and numerical modeling of tokamak plasmas is simultaneously an essential component of effective reactor design, and a great research barrier. Tokamak operational conditions exhibit comparatively low Knudsen numbers. Kinetic effects, including kinetic waves and instabilities, Landau damping, bump-on-tail instabilities and more, are therefore highly influential in tokamak plasma dynamics. Purely fluid models are inherently incapable of capturing these effects, whereas the high dimensionality in purely kinetic models render them practically intractable for most relevant purposes.

        We consider a $\delta\!f$ decomposition model, with a macroscopic fluid background and microscopic kinetic correction, both fully coupled to each other. A similar manner of discretization is proposed to that used in the recent \texttt{STRUPHY} code \cite{Holderied_Possanner_Wang_2021, Holderied_2022, Li_et_al_2023} with a finite-element model for the background and a pseudo-particle/PiC model for the correction.

        The fluid background satisfies the full, non-linear, resistive, compressible, Hall MHD equations. \cite{Laakmann_Hu_Farrell_2022} introduces finite-element(-in-space) implicit timesteppers for the incompressible analogue to this system with structure-preserving (SP) properties in the ideal case, alongside parameter-robust preconditioners. We show that these timesteppers can derive from a finite-element-in-time (FET) (and finite-element-in-space) interpretation. The benefits of this reformulation are discussed, including the derivation of timesteppers that are higher order in time, and the quantifiable dissipative SP properties in the non-ideal, resistive case.
        
        We discuss possible options for extending this FET approach to timesteppers for the compressible case.

        The kinetic corrections satisfy linearized Boltzmann equations. Using a Lénard--Bernstein collision operator, these take Fokker--Planck-like forms \cite{Fokker_1914, Planck_1917} wherein pseudo-particles in the numerical model obey the neoclassical transport equations, with particle-independent Brownian drift terms. This offers a rigorous methodology for incorporating collisions into the particle transport model, without coupling the equations of motions for each particle.
        
        Works by Chen, Chacón et al. \cite{Chen_Chacón_Barnes_2011, Chacón_Chen_Barnes_2013, Chen_Chacón_2014, Chen_Chacón_2015} have developed structure-preserving particle pushers for neoclassical transport in the Vlasov equations, derived from Crank--Nicolson integrators. We show these too can can derive from a FET interpretation, similarly offering potential extensions to higher-order-in-time particle pushers. The FET formulation is used also to consider how the stochastic drift terms can be incorporated into the pushers. Stochastic gyrokinetic expansions are also discussed.

        Different options for the numerical implementation of these schemes are considered.

        Due to the efficacy of FET in the development of SP timesteppers for both the fluid and kinetic component, we hope this approach will prove effective in the future for developing SP timesteppers for the full hybrid model. We hope this will give us the opportunity to incorporate previously inaccessible kinetic effects into the highly effective, modern, finite-element MHD models.
    \end{abstract}
    
    
    \newpage
    \tableofcontents
    
    
    \newpage
    \pagenumbering{arabic}
    %\linenumbers\renewcommand\thelinenumber{\color{black!50}\arabic{linenumber}}
            \input{0 - introduction/main.tex}
        \part{Research}
            \input{1 - low-noise PiC models/main.tex}
            \input{2 - kinetic component/main.tex}
            \input{3 - fluid component/main.tex}
            \input{4 - numerical implementation/main.tex}
        \part{Project Overview}
            \input{5 - research plan/main.tex}
            \input{6 - summary/main.tex}
    
    
    %\section{}
    \newpage
    \pagenumbering{gobble}
        \printbibliography


    \newpage
    \pagenumbering{roman}
    \appendix
        \part{Appendices}
            \input{8 - Hilbert complexes/main.tex}
            \input{9 - weak conservation proofs/main.tex}
\end{document}


\title{\BA{Title in Progress...}}
\author{Boris Andrews}
\affil{Mathematical Institute, University of Oxford}
\date{\today}


\begin{document}
    \pagenumbering{gobble}
    \maketitle
    
    
    \begin{abstract}
        Magnetic confinement reactors---in particular tokamaks---offer one of the most promising options for achieving practical nuclear fusion, with the potential to provide virtually limitless, clean energy. The theoretical and numerical modeling of tokamak plasmas is simultaneously an essential component of effective reactor design, and a great research barrier. Tokamak operational conditions exhibit comparatively low Knudsen numbers. Kinetic effects, including kinetic waves and instabilities, Landau damping, bump-on-tail instabilities and more, are therefore highly influential in tokamak plasma dynamics. Purely fluid models are inherently incapable of capturing these effects, whereas the high dimensionality in purely kinetic models render them practically intractable for most relevant purposes.

        We consider a $\delta\!f$ decomposition model, with a macroscopic fluid background and microscopic kinetic correction, both fully coupled to each other. A similar manner of discretization is proposed to that used in the recent \texttt{STRUPHY} code \cite{Holderied_Possanner_Wang_2021, Holderied_2022, Li_et_al_2023} with a finite-element model for the background and a pseudo-particle/PiC model for the correction.

        The fluid background satisfies the full, non-linear, resistive, compressible, Hall MHD equations. \cite{Laakmann_Hu_Farrell_2022} introduces finite-element(-in-space) implicit timesteppers for the incompressible analogue to this system with structure-preserving (SP) properties in the ideal case, alongside parameter-robust preconditioners. We show that these timesteppers can derive from a finite-element-in-time (FET) (and finite-element-in-space) interpretation. The benefits of this reformulation are discussed, including the derivation of timesteppers that are higher order in time, and the quantifiable dissipative SP properties in the non-ideal, resistive case.
        
        We discuss possible options for extending this FET approach to timesteppers for the compressible case.

        The kinetic corrections satisfy linearized Boltzmann equations. Using a Lénard--Bernstein collision operator, these take Fokker--Planck-like forms \cite{Fokker_1914, Planck_1917} wherein pseudo-particles in the numerical model obey the neoclassical transport equations, with particle-independent Brownian drift terms. This offers a rigorous methodology for incorporating collisions into the particle transport model, without coupling the equations of motions for each particle.
        
        Works by Chen, Chacón et al. \cite{Chen_Chacón_Barnes_2011, Chacón_Chen_Barnes_2013, Chen_Chacón_2014, Chen_Chacón_2015} have developed structure-preserving particle pushers for neoclassical transport in the Vlasov equations, derived from Crank--Nicolson integrators. We show these too can can derive from a FET interpretation, similarly offering potential extensions to higher-order-in-time particle pushers. The FET formulation is used also to consider how the stochastic drift terms can be incorporated into the pushers. Stochastic gyrokinetic expansions are also discussed.

        Different options for the numerical implementation of these schemes are considered.

        Due to the efficacy of FET in the development of SP timesteppers for both the fluid and kinetic component, we hope this approach will prove effective in the future for developing SP timesteppers for the full hybrid model. We hope this will give us the opportunity to incorporate previously inaccessible kinetic effects into the highly effective, modern, finite-element MHD models.
    \end{abstract}
    
    
    \newpage
    \tableofcontents
    
    
    \newpage
    \pagenumbering{arabic}
    %\linenumbers\renewcommand\thelinenumber{\color{black!50}\arabic{linenumber}}
            \documentclass[12pt, a4paper]{report}

\input{template/main.tex}

\title{\BA{Title in Progress...}}
\author{Boris Andrews}
\affil{Mathematical Institute, University of Oxford}
\date{\today}


\begin{document}
    \pagenumbering{gobble}
    \maketitle
    
    
    \begin{abstract}
        Magnetic confinement reactors---in particular tokamaks---offer one of the most promising options for achieving practical nuclear fusion, with the potential to provide virtually limitless, clean energy. The theoretical and numerical modeling of tokamak plasmas is simultaneously an essential component of effective reactor design, and a great research barrier. Tokamak operational conditions exhibit comparatively low Knudsen numbers. Kinetic effects, including kinetic waves and instabilities, Landau damping, bump-on-tail instabilities and more, are therefore highly influential in tokamak plasma dynamics. Purely fluid models are inherently incapable of capturing these effects, whereas the high dimensionality in purely kinetic models render them practically intractable for most relevant purposes.

        We consider a $\delta\!f$ decomposition model, with a macroscopic fluid background and microscopic kinetic correction, both fully coupled to each other. A similar manner of discretization is proposed to that used in the recent \texttt{STRUPHY} code \cite{Holderied_Possanner_Wang_2021, Holderied_2022, Li_et_al_2023} with a finite-element model for the background and a pseudo-particle/PiC model for the correction.

        The fluid background satisfies the full, non-linear, resistive, compressible, Hall MHD equations. \cite{Laakmann_Hu_Farrell_2022} introduces finite-element(-in-space) implicit timesteppers for the incompressible analogue to this system with structure-preserving (SP) properties in the ideal case, alongside parameter-robust preconditioners. We show that these timesteppers can derive from a finite-element-in-time (FET) (and finite-element-in-space) interpretation. The benefits of this reformulation are discussed, including the derivation of timesteppers that are higher order in time, and the quantifiable dissipative SP properties in the non-ideal, resistive case.
        
        We discuss possible options for extending this FET approach to timesteppers for the compressible case.

        The kinetic corrections satisfy linearized Boltzmann equations. Using a Lénard--Bernstein collision operator, these take Fokker--Planck-like forms \cite{Fokker_1914, Planck_1917} wherein pseudo-particles in the numerical model obey the neoclassical transport equations, with particle-independent Brownian drift terms. This offers a rigorous methodology for incorporating collisions into the particle transport model, without coupling the equations of motions for each particle.
        
        Works by Chen, Chacón et al. \cite{Chen_Chacón_Barnes_2011, Chacón_Chen_Barnes_2013, Chen_Chacón_2014, Chen_Chacón_2015} have developed structure-preserving particle pushers for neoclassical transport in the Vlasov equations, derived from Crank--Nicolson integrators. We show these too can can derive from a FET interpretation, similarly offering potential extensions to higher-order-in-time particle pushers. The FET formulation is used also to consider how the stochastic drift terms can be incorporated into the pushers. Stochastic gyrokinetic expansions are also discussed.

        Different options for the numerical implementation of these schemes are considered.

        Due to the efficacy of FET in the development of SP timesteppers for both the fluid and kinetic component, we hope this approach will prove effective in the future for developing SP timesteppers for the full hybrid model. We hope this will give us the opportunity to incorporate previously inaccessible kinetic effects into the highly effective, modern, finite-element MHD models.
    \end{abstract}
    
    
    \newpage
    \tableofcontents
    
    
    \newpage
    \pagenumbering{arabic}
    %\linenumbers\renewcommand\thelinenumber{\color{black!50}\arabic{linenumber}}
            \input{0 - introduction/main.tex}
        \part{Research}
            \input{1 - low-noise PiC models/main.tex}
            \input{2 - kinetic component/main.tex}
            \input{3 - fluid component/main.tex}
            \input{4 - numerical implementation/main.tex}
        \part{Project Overview}
            \input{5 - research plan/main.tex}
            \input{6 - summary/main.tex}
    
    
    %\section{}
    \newpage
    \pagenumbering{gobble}
        \printbibliography


    \newpage
    \pagenumbering{roman}
    \appendix
        \part{Appendices}
            \input{8 - Hilbert complexes/main.tex}
            \input{9 - weak conservation proofs/main.tex}
\end{document}

        \part{Research}
            \documentclass[12pt, a4paper]{report}

\input{template/main.tex}

\title{\BA{Title in Progress...}}
\author{Boris Andrews}
\affil{Mathematical Institute, University of Oxford}
\date{\today}


\begin{document}
    \pagenumbering{gobble}
    \maketitle
    
    
    \begin{abstract}
        Magnetic confinement reactors---in particular tokamaks---offer one of the most promising options for achieving practical nuclear fusion, with the potential to provide virtually limitless, clean energy. The theoretical and numerical modeling of tokamak plasmas is simultaneously an essential component of effective reactor design, and a great research barrier. Tokamak operational conditions exhibit comparatively low Knudsen numbers. Kinetic effects, including kinetic waves and instabilities, Landau damping, bump-on-tail instabilities and more, are therefore highly influential in tokamak plasma dynamics. Purely fluid models are inherently incapable of capturing these effects, whereas the high dimensionality in purely kinetic models render them practically intractable for most relevant purposes.

        We consider a $\delta\!f$ decomposition model, with a macroscopic fluid background and microscopic kinetic correction, both fully coupled to each other. A similar manner of discretization is proposed to that used in the recent \texttt{STRUPHY} code \cite{Holderied_Possanner_Wang_2021, Holderied_2022, Li_et_al_2023} with a finite-element model for the background and a pseudo-particle/PiC model for the correction.

        The fluid background satisfies the full, non-linear, resistive, compressible, Hall MHD equations. \cite{Laakmann_Hu_Farrell_2022} introduces finite-element(-in-space) implicit timesteppers for the incompressible analogue to this system with structure-preserving (SP) properties in the ideal case, alongside parameter-robust preconditioners. We show that these timesteppers can derive from a finite-element-in-time (FET) (and finite-element-in-space) interpretation. The benefits of this reformulation are discussed, including the derivation of timesteppers that are higher order in time, and the quantifiable dissipative SP properties in the non-ideal, resistive case.
        
        We discuss possible options for extending this FET approach to timesteppers for the compressible case.

        The kinetic corrections satisfy linearized Boltzmann equations. Using a Lénard--Bernstein collision operator, these take Fokker--Planck-like forms \cite{Fokker_1914, Planck_1917} wherein pseudo-particles in the numerical model obey the neoclassical transport equations, with particle-independent Brownian drift terms. This offers a rigorous methodology for incorporating collisions into the particle transport model, without coupling the equations of motions for each particle.
        
        Works by Chen, Chacón et al. \cite{Chen_Chacón_Barnes_2011, Chacón_Chen_Barnes_2013, Chen_Chacón_2014, Chen_Chacón_2015} have developed structure-preserving particle pushers for neoclassical transport in the Vlasov equations, derived from Crank--Nicolson integrators. We show these too can can derive from a FET interpretation, similarly offering potential extensions to higher-order-in-time particle pushers. The FET formulation is used also to consider how the stochastic drift terms can be incorporated into the pushers. Stochastic gyrokinetic expansions are also discussed.

        Different options for the numerical implementation of these schemes are considered.

        Due to the efficacy of FET in the development of SP timesteppers for both the fluid and kinetic component, we hope this approach will prove effective in the future for developing SP timesteppers for the full hybrid model. We hope this will give us the opportunity to incorporate previously inaccessible kinetic effects into the highly effective, modern, finite-element MHD models.
    \end{abstract}
    
    
    \newpage
    \tableofcontents
    
    
    \newpage
    \pagenumbering{arabic}
    %\linenumbers\renewcommand\thelinenumber{\color{black!50}\arabic{linenumber}}
            \input{0 - introduction/main.tex}
        \part{Research}
            \input{1 - low-noise PiC models/main.tex}
            \input{2 - kinetic component/main.tex}
            \input{3 - fluid component/main.tex}
            \input{4 - numerical implementation/main.tex}
        \part{Project Overview}
            \input{5 - research plan/main.tex}
            \input{6 - summary/main.tex}
    
    
    %\section{}
    \newpage
    \pagenumbering{gobble}
        \printbibliography


    \newpage
    \pagenumbering{roman}
    \appendix
        \part{Appendices}
            \input{8 - Hilbert complexes/main.tex}
            \input{9 - weak conservation proofs/main.tex}
\end{document}

            \documentclass[12pt, a4paper]{report}

\input{template/main.tex}

\title{\BA{Title in Progress...}}
\author{Boris Andrews}
\affil{Mathematical Institute, University of Oxford}
\date{\today}


\begin{document}
    \pagenumbering{gobble}
    \maketitle
    
    
    \begin{abstract}
        Magnetic confinement reactors---in particular tokamaks---offer one of the most promising options for achieving practical nuclear fusion, with the potential to provide virtually limitless, clean energy. The theoretical and numerical modeling of tokamak plasmas is simultaneously an essential component of effective reactor design, and a great research barrier. Tokamak operational conditions exhibit comparatively low Knudsen numbers. Kinetic effects, including kinetic waves and instabilities, Landau damping, bump-on-tail instabilities and more, are therefore highly influential in tokamak plasma dynamics. Purely fluid models are inherently incapable of capturing these effects, whereas the high dimensionality in purely kinetic models render them practically intractable for most relevant purposes.

        We consider a $\delta\!f$ decomposition model, with a macroscopic fluid background and microscopic kinetic correction, both fully coupled to each other. A similar manner of discretization is proposed to that used in the recent \texttt{STRUPHY} code \cite{Holderied_Possanner_Wang_2021, Holderied_2022, Li_et_al_2023} with a finite-element model for the background and a pseudo-particle/PiC model for the correction.

        The fluid background satisfies the full, non-linear, resistive, compressible, Hall MHD equations. \cite{Laakmann_Hu_Farrell_2022} introduces finite-element(-in-space) implicit timesteppers for the incompressible analogue to this system with structure-preserving (SP) properties in the ideal case, alongside parameter-robust preconditioners. We show that these timesteppers can derive from a finite-element-in-time (FET) (and finite-element-in-space) interpretation. The benefits of this reformulation are discussed, including the derivation of timesteppers that are higher order in time, and the quantifiable dissipative SP properties in the non-ideal, resistive case.
        
        We discuss possible options for extending this FET approach to timesteppers for the compressible case.

        The kinetic corrections satisfy linearized Boltzmann equations. Using a Lénard--Bernstein collision operator, these take Fokker--Planck-like forms \cite{Fokker_1914, Planck_1917} wherein pseudo-particles in the numerical model obey the neoclassical transport equations, with particle-independent Brownian drift terms. This offers a rigorous methodology for incorporating collisions into the particle transport model, without coupling the equations of motions for each particle.
        
        Works by Chen, Chacón et al. \cite{Chen_Chacón_Barnes_2011, Chacón_Chen_Barnes_2013, Chen_Chacón_2014, Chen_Chacón_2015} have developed structure-preserving particle pushers for neoclassical transport in the Vlasov equations, derived from Crank--Nicolson integrators. We show these too can can derive from a FET interpretation, similarly offering potential extensions to higher-order-in-time particle pushers. The FET formulation is used also to consider how the stochastic drift terms can be incorporated into the pushers. Stochastic gyrokinetic expansions are also discussed.

        Different options for the numerical implementation of these schemes are considered.

        Due to the efficacy of FET in the development of SP timesteppers for both the fluid and kinetic component, we hope this approach will prove effective in the future for developing SP timesteppers for the full hybrid model. We hope this will give us the opportunity to incorporate previously inaccessible kinetic effects into the highly effective, modern, finite-element MHD models.
    \end{abstract}
    
    
    \newpage
    \tableofcontents
    
    
    \newpage
    \pagenumbering{arabic}
    %\linenumbers\renewcommand\thelinenumber{\color{black!50}\arabic{linenumber}}
            \input{0 - introduction/main.tex}
        \part{Research}
            \input{1 - low-noise PiC models/main.tex}
            \input{2 - kinetic component/main.tex}
            \input{3 - fluid component/main.tex}
            \input{4 - numerical implementation/main.tex}
        \part{Project Overview}
            \input{5 - research plan/main.tex}
            \input{6 - summary/main.tex}
    
    
    %\section{}
    \newpage
    \pagenumbering{gobble}
        \printbibliography


    \newpage
    \pagenumbering{roman}
    \appendix
        \part{Appendices}
            \input{8 - Hilbert complexes/main.tex}
            \input{9 - weak conservation proofs/main.tex}
\end{document}

            \documentclass[12pt, a4paper]{report}

\input{template/main.tex}

\title{\BA{Title in Progress...}}
\author{Boris Andrews}
\affil{Mathematical Institute, University of Oxford}
\date{\today}


\begin{document}
    \pagenumbering{gobble}
    \maketitle
    
    
    \begin{abstract}
        Magnetic confinement reactors---in particular tokamaks---offer one of the most promising options for achieving practical nuclear fusion, with the potential to provide virtually limitless, clean energy. The theoretical and numerical modeling of tokamak plasmas is simultaneously an essential component of effective reactor design, and a great research barrier. Tokamak operational conditions exhibit comparatively low Knudsen numbers. Kinetic effects, including kinetic waves and instabilities, Landau damping, bump-on-tail instabilities and more, are therefore highly influential in tokamak plasma dynamics. Purely fluid models are inherently incapable of capturing these effects, whereas the high dimensionality in purely kinetic models render them practically intractable for most relevant purposes.

        We consider a $\delta\!f$ decomposition model, with a macroscopic fluid background and microscopic kinetic correction, both fully coupled to each other. A similar manner of discretization is proposed to that used in the recent \texttt{STRUPHY} code \cite{Holderied_Possanner_Wang_2021, Holderied_2022, Li_et_al_2023} with a finite-element model for the background and a pseudo-particle/PiC model for the correction.

        The fluid background satisfies the full, non-linear, resistive, compressible, Hall MHD equations. \cite{Laakmann_Hu_Farrell_2022} introduces finite-element(-in-space) implicit timesteppers for the incompressible analogue to this system with structure-preserving (SP) properties in the ideal case, alongside parameter-robust preconditioners. We show that these timesteppers can derive from a finite-element-in-time (FET) (and finite-element-in-space) interpretation. The benefits of this reformulation are discussed, including the derivation of timesteppers that are higher order in time, and the quantifiable dissipative SP properties in the non-ideal, resistive case.
        
        We discuss possible options for extending this FET approach to timesteppers for the compressible case.

        The kinetic corrections satisfy linearized Boltzmann equations. Using a Lénard--Bernstein collision operator, these take Fokker--Planck-like forms \cite{Fokker_1914, Planck_1917} wherein pseudo-particles in the numerical model obey the neoclassical transport equations, with particle-independent Brownian drift terms. This offers a rigorous methodology for incorporating collisions into the particle transport model, without coupling the equations of motions for each particle.
        
        Works by Chen, Chacón et al. \cite{Chen_Chacón_Barnes_2011, Chacón_Chen_Barnes_2013, Chen_Chacón_2014, Chen_Chacón_2015} have developed structure-preserving particle pushers for neoclassical transport in the Vlasov equations, derived from Crank--Nicolson integrators. We show these too can can derive from a FET interpretation, similarly offering potential extensions to higher-order-in-time particle pushers. The FET formulation is used also to consider how the stochastic drift terms can be incorporated into the pushers. Stochastic gyrokinetic expansions are also discussed.

        Different options for the numerical implementation of these schemes are considered.

        Due to the efficacy of FET in the development of SP timesteppers for both the fluid and kinetic component, we hope this approach will prove effective in the future for developing SP timesteppers for the full hybrid model. We hope this will give us the opportunity to incorporate previously inaccessible kinetic effects into the highly effective, modern, finite-element MHD models.
    \end{abstract}
    
    
    \newpage
    \tableofcontents
    
    
    \newpage
    \pagenumbering{arabic}
    %\linenumbers\renewcommand\thelinenumber{\color{black!50}\arabic{linenumber}}
            \input{0 - introduction/main.tex}
        \part{Research}
            \input{1 - low-noise PiC models/main.tex}
            \input{2 - kinetic component/main.tex}
            \input{3 - fluid component/main.tex}
            \input{4 - numerical implementation/main.tex}
        \part{Project Overview}
            \input{5 - research plan/main.tex}
            \input{6 - summary/main.tex}
    
    
    %\section{}
    \newpage
    \pagenumbering{gobble}
        \printbibliography


    \newpage
    \pagenumbering{roman}
    \appendix
        \part{Appendices}
            \input{8 - Hilbert complexes/main.tex}
            \input{9 - weak conservation proofs/main.tex}
\end{document}

            \documentclass[12pt, a4paper]{report}

\input{template/main.tex}

\title{\BA{Title in Progress...}}
\author{Boris Andrews}
\affil{Mathematical Institute, University of Oxford}
\date{\today}


\begin{document}
    \pagenumbering{gobble}
    \maketitle
    
    
    \begin{abstract}
        Magnetic confinement reactors---in particular tokamaks---offer one of the most promising options for achieving practical nuclear fusion, with the potential to provide virtually limitless, clean energy. The theoretical and numerical modeling of tokamak plasmas is simultaneously an essential component of effective reactor design, and a great research barrier. Tokamak operational conditions exhibit comparatively low Knudsen numbers. Kinetic effects, including kinetic waves and instabilities, Landau damping, bump-on-tail instabilities and more, are therefore highly influential in tokamak plasma dynamics. Purely fluid models are inherently incapable of capturing these effects, whereas the high dimensionality in purely kinetic models render them practically intractable for most relevant purposes.

        We consider a $\delta\!f$ decomposition model, with a macroscopic fluid background and microscopic kinetic correction, both fully coupled to each other. A similar manner of discretization is proposed to that used in the recent \texttt{STRUPHY} code \cite{Holderied_Possanner_Wang_2021, Holderied_2022, Li_et_al_2023} with a finite-element model for the background and a pseudo-particle/PiC model for the correction.

        The fluid background satisfies the full, non-linear, resistive, compressible, Hall MHD equations. \cite{Laakmann_Hu_Farrell_2022} introduces finite-element(-in-space) implicit timesteppers for the incompressible analogue to this system with structure-preserving (SP) properties in the ideal case, alongside parameter-robust preconditioners. We show that these timesteppers can derive from a finite-element-in-time (FET) (and finite-element-in-space) interpretation. The benefits of this reformulation are discussed, including the derivation of timesteppers that are higher order in time, and the quantifiable dissipative SP properties in the non-ideal, resistive case.
        
        We discuss possible options for extending this FET approach to timesteppers for the compressible case.

        The kinetic corrections satisfy linearized Boltzmann equations. Using a Lénard--Bernstein collision operator, these take Fokker--Planck-like forms \cite{Fokker_1914, Planck_1917} wherein pseudo-particles in the numerical model obey the neoclassical transport equations, with particle-independent Brownian drift terms. This offers a rigorous methodology for incorporating collisions into the particle transport model, without coupling the equations of motions for each particle.
        
        Works by Chen, Chacón et al. \cite{Chen_Chacón_Barnes_2011, Chacón_Chen_Barnes_2013, Chen_Chacón_2014, Chen_Chacón_2015} have developed structure-preserving particle pushers for neoclassical transport in the Vlasov equations, derived from Crank--Nicolson integrators. We show these too can can derive from a FET interpretation, similarly offering potential extensions to higher-order-in-time particle pushers. The FET formulation is used also to consider how the stochastic drift terms can be incorporated into the pushers. Stochastic gyrokinetic expansions are also discussed.

        Different options for the numerical implementation of these schemes are considered.

        Due to the efficacy of FET in the development of SP timesteppers for both the fluid and kinetic component, we hope this approach will prove effective in the future for developing SP timesteppers for the full hybrid model. We hope this will give us the opportunity to incorporate previously inaccessible kinetic effects into the highly effective, modern, finite-element MHD models.
    \end{abstract}
    
    
    \newpage
    \tableofcontents
    
    
    \newpage
    \pagenumbering{arabic}
    %\linenumbers\renewcommand\thelinenumber{\color{black!50}\arabic{linenumber}}
            \input{0 - introduction/main.tex}
        \part{Research}
            \input{1 - low-noise PiC models/main.tex}
            \input{2 - kinetic component/main.tex}
            \input{3 - fluid component/main.tex}
            \input{4 - numerical implementation/main.tex}
        \part{Project Overview}
            \input{5 - research plan/main.tex}
            \input{6 - summary/main.tex}
    
    
    %\section{}
    \newpage
    \pagenumbering{gobble}
        \printbibliography


    \newpage
    \pagenumbering{roman}
    \appendix
        \part{Appendices}
            \input{8 - Hilbert complexes/main.tex}
            \input{9 - weak conservation proofs/main.tex}
\end{document}

        \part{Project Overview}
            \documentclass[12pt, a4paper]{report}

\input{template/main.tex}

\title{\BA{Title in Progress...}}
\author{Boris Andrews}
\affil{Mathematical Institute, University of Oxford}
\date{\today}


\begin{document}
    \pagenumbering{gobble}
    \maketitle
    
    
    \begin{abstract}
        Magnetic confinement reactors---in particular tokamaks---offer one of the most promising options for achieving practical nuclear fusion, with the potential to provide virtually limitless, clean energy. The theoretical and numerical modeling of tokamak plasmas is simultaneously an essential component of effective reactor design, and a great research barrier. Tokamak operational conditions exhibit comparatively low Knudsen numbers. Kinetic effects, including kinetic waves and instabilities, Landau damping, bump-on-tail instabilities and more, are therefore highly influential in tokamak plasma dynamics. Purely fluid models are inherently incapable of capturing these effects, whereas the high dimensionality in purely kinetic models render them practically intractable for most relevant purposes.

        We consider a $\delta\!f$ decomposition model, with a macroscopic fluid background and microscopic kinetic correction, both fully coupled to each other. A similar manner of discretization is proposed to that used in the recent \texttt{STRUPHY} code \cite{Holderied_Possanner_Wang_2021, Holderied_2022, Li_et_al_2023} with a finite-element model for the background and a pseudo-particle/PiC model for the correction.

        The fluid background satisfies the full, non-linear, resistive, compressible, Hall MHD equations. \cite{Laakmann_Hu_Farrell_2022} introduces finite-element(-in-space) implicit timesteppers for the incompressible analogue to this system with structure-preserving (SP) properties in the ideal case, alongside parameter-robust preconditioners. We show that these timesteppers can derive from a finite-element-in-time (FET) (and finite-element-in-space) interpretation. The benefits of this reformulation are discussed, including the derivation of timesteppers that are higher order in time, and the quantifiable dissipative SP properties in the non-ideal, resistive case.
        
        We discuss possible options for extending this FET approach to timesteppers for the compressible case.

        The kinetic corrections satisfy linearized Boltzmann equations. Using a Lénard--Bernstein collision operator, these take Fokker--Planck-like forms \cite{Fokker_1914, Planck_1917} wherein pseudo-particles in the numerical model obey the neoclassical transport equations, with particle-independent Brownian drift terms. This offers a rigorous methodology for incorporating collisions into the particle transport model, without coupling the equations of motions for each particle.
        
        Works by Chen, Chacón et al. \cite{Chen_Chacón_Barnes_2011, Chacón_Chen_Barnes_2013, Chen_Chacón_2014, Chen_Chacón_2015} have developed structure-preserving particle pushers for neoclassical transport in the Vlasov equations, derived from Crank--Nicolson integrators. We show these too can can derive from a FET interpretation, similarly offering potential extensions to higher-order-in-time particle pushers. The FET formulation is used also to consider how the stochastic drift terms can be incorporated into the pushers. Stochastic gyrokinetic expansions are also discussed.

        Different options for the numerical implementation of these schemes are considered.

        Due to the efficacy of FET in the development of SP timesteppers for both the fluid and kinetic component, we hope this approach will prove effective in the future for developing SP timesteppers for the full hybrid model. We hope this will give us the opportunity to incorporate previously inaccessible kinetic effects into the highly effective, modern, finite-element MHD models.
    \end{abstract}
    
    
    \newpage
    \tableofcontents
    
    
    \newpage
    \pagenumbering{arabic}
    %\linenumbers\renewcommand\thelinenumber{\color{black!50}\arabic{linenumber}}
            \input{0 - introduction/main.tex}
        \part{Research}
            \input{1 - low-noise PiC models/main.tex}
            \input{2 - kinetic component/main.tex}
            \input{3 - fluid component/main.tex}
            \input{4 - numerical implementation/main.tex}
        \part{Project Overview}
            \input{5 - research plan/main.tex}
            \input{6 - summary/main.tex}
    
    
    %\section{}
    \newpage
    \pagenumbering{gobble}
        \printbibliography


    \newpage
    \pagenumbering{roman}
    \appendix
        \part{Appendices}
            \input{8 - Hilbert complexes/main.tex}
            \input{9 - weak conservation proofs/main.tex}
\end{document}

            \documentclass[12pt, a4paper]{report}

\input{template/main.tex}

\title{\BA{Title in Progress...}}
\author{Boris Andrews}
\affil{Mathematical Institute, University of Oxford}
\date{\today}


\begin{document}
    \pagenumbering{gobble}
    \maketitle
    
    
    \begin{abstract}
        Magnetic confinement reactors---in particular tokamaks---offer one of the most promising options for achieving practical nuclear fusion, with the potential to provide virtually limitless, clean energy. The theoretical and numerical modeling of tokamak plasmas is simultaneously an essential component of effective reactor design, and a great research barrier. Tokamak operational conditions exhibit comparatively low Knudsen numbers. Kinetic effects, including kinetic waves and instabilities, Landau damping, bump-on-tail instabilities and more, are therefore highly influential in tokamak plasma dynamics. Purely fluid models are inherently incapable of capturing these effects, whereas the high dimensionality in purely kinetic models render them practically intractable for most relevant purposes.

        We consider a $\delta\!f$ decomposition model, with a macroscopic fluid background and microscopic kinetic correction, both fully coupled to each other. A similar manner of discretization is proposed to that used in the recent \texttt{STRUPHY} code \cite{Holderied_Possanner_Wang_2021, Holderied_2022, Li_et_al_2023} with a finite-element model for the background and a pseudo-particle/PiC model for the correction.

        The fluid background satisfies the full, non-linear, resistive, compressible, Hall MHD equations. \cite{Laakmann_Hu_Farrell_2022} introduces finite-element(-in-space) implicit timesteppers for the incompressible analogue to this system with structure-preserving (SP) properties in the ideal case, alongside parameter-robust preconditioners. We show that these timesteppers can derive from a finite-element-in-time (FET) (and finite-element-in-space) interpretation. The benefits of this reformulation are discussed, including the derivation of timesteppers that are higher order in time, and the quantifiable dissipative SP properties in the non-ideal, resistive case.
        
        We discuss possible options for extending this FET approach to timesteppers for the compressible case.

        The kinetic corrections satisfy linearized Boltzmann equations. Using a Lénard--Bernstein collision operator, these take Fokker--Planck-like forms \cite{Fokker_1914, Planck_1917} wherein pseudo-particles in the numerical model obey the neoclassical transport equations, with particle-independent Brownian drift terms. This offers a rigorous methodology for incorporating collisions into the particle transport model, without coupling the equations of motions for each particle.
        
        Works by Chen, Chacón et al. \cite{Chen_Chacón_Barnes_2011, Chacón_Chen_Barnes_2013, Chen_Chacón_2014, Chen_Chacón_2015} have developed structure-preserving particle pushers for neoclassical transport in the Vlasov equations, derived from Crank--Nicolson integrators. We show these too can can derive from a FET interpretation, similarly offering potential extensions to higher-order-in-time particle pushers. The FET formulation is used also to consider how the stochastic drift terms can be incorporated into the pushers. Stochastic gyrokinetic expansions are also discussed.

        Different options for the numerical implementation of these schemes are considered.

        Due to the efficacy of FET in the development of SP timesteppers for both the fluid and kinetic component, we hope this approach will prove effective in the future for developing SP timesteppers for the full hybrid model. We hope this will give us the opportunity to incorporate previously inaccessible kinetic effects into the highly effective, modern, finite-element MHD models.
    \end{abstract}
    
    
    \newpage
    \tableofcontents
    
    
    \newpage
    \pagenumbering{arabic}
    %\linenumbers\renewcommand\thelinenumber{\color{black!50}\arabic{linenumber}}
            \input{0 - introduction/main.tex}
        \part{Research}
            \input{1 - low-noise PiC models/main.tex}
            \input{2 - kinetic component/main.tex}
            \input{3 - fluid component/main.tex}
            \input{4 - numerical implementation/main.tex}
        \part{Project Overview}
            \input{5 - research plan/main.tex}
            \input{6 - summary/main.tex}
    
    
    %\section{}
    \newpage
    \pagenumbering{gobble}
        \printbibliography


    \newpage
    \pagenumbering{roman}
    \appendix
        \part{Appendices}
            \input{8 - Hilbert complexes/main.tex}
            \input{9 - weak conservation proofs/main.tex}
\end{document}

    
    
    %\section{}
    \newpage
    \pagenumbering{gobble}
        \printbibliography


    \newpage
    \pagenumbering{roman}
    \appendix
        \part{Appendices}
            \documentclass[12pt, a4paper]{report}

\input{template/main.tex}

\title{\BA{Title in Progress...}}
\author{Boris Andrews}
\affil{Mathematical Institute, University of Oxford}
\date{\today}


\begin{document}
    \pagenumbering{gobble}
    \maketitle
    
    
    \begin{abstract}
        Magnetic confinement reactors---in particular tokamaks---offer one of the most promising options for achieving practical nuclear fusion, with the potential to provide virtually limitless, clean energy. The theoretical and numerical modeling of tokamak plasmas is simultaneously an essential component of effective reactor design, and a great research barrier. Tokamak operational conditions exhibit comparatively low Knudsen numbers. Kinetic effects, including kinetic waves and instabilities, Landau damping, bump-on-tail instabilities and more, are therefore highly influential in tokamak plasma dynamics. Purely fluid models are inherently incapable of capturing these effects, whereas the high dimensionality in purely kinetic models render them practically intractable for most relevant purposes.

        We consider a $\delta\!f$ decomposition model, with a macroscopic fluid background and microscopic kinetic correction, both fully coupled to each other. A similar manner of discretization is proposed to that used in the recent \texttt{STRUPHY} code \cite{Holderied_Possanner_Wang_2021, Holderied_2022, Li_et_al_2023} with a finite-element model for the background and a pseudo-particle/PiC model for the correction.

        The fluid background satisfies the full, non-linear, resistive, compressible, Hall MHD equations. \cite{Laakmann_Hu_Farrell_2022} introduces finite-element(-in-space) implicit timesteppers for the incompressible analogue to this system with structure-preserving (SP) properties in the ideal case, alongside parameter-robust preconditioners. We show that these timesteppers can derive from a finite-element-in-time (FET) (and finite-element-in-space) interpretation. The benefits of this reformulation are discussed, including the derivation of timesteppers that are higher order in time, and the quantifiable dissipative SP properties in the non-ideal, resistive case.
        
        We discuss possible options for extending this FET approach to timesteppers for the compressible case.

        The kinetic corrections satisfy linearized Boltzmann equations. Using a Lénard--Bernstein collision operator, these take Fokker--Planck-like forms \cite{Fokker_1914, Planck_1917} wherein pseudo-particles in the numerical model obey the neoclassical transport equations, with particle-independent Brownian drift terms. This offers a rigorous methodology for incorporating collisions into the particle transport model, without coupling the equations of motions for each particle.
        
        Works by Chen, Chacón et al. \cite{Chen_Chacón_Barnes_2011, Chacón_Chen_Barnes_2013, Chen_Chacón_2014, Chen_Chacón_2015} have developed structure-preserving particle pushers for neoclassical transport in the Vlasov equations, derived from Crank--Nicolson integrators. We show these too can can derive from a FET interpretation, similarly offering potential extensions to higher-order-in-time particle pushers. The FET formulation is used also to consider how the stochastic drift terms can be incorporated into the pushers. Stochastic gyrokinetic expansions are also discussed.

        Different options for the numerical implementation of these schemes are considered.

        Due to the efficacy of FET in the development of SP timesteppers for both the fluid and kinetic component, we hope this approach will prove effective in the future for developing SP timesteppers for the full hybrid model. We hope this will give us the opportunity to incorporate previously inaccessible kinetic effects into the highly effective, modern, finite-element MHD models.
    \end{abstract}
    
    
    \newpage
    \tableofcontents
    
    
    \newpage
    \pagenumbering{arabic}
    %\linenumbers\renewcommand\thelinenumber{\color{black!50}\arabic{linenumber}}
            \input{0 - introduction/main.tex}
        \part{Research}
            \input{1 - low-noise PiC models/main.tex}
            \input{2 - kinetic component/main.tex}
            \input{3 - fluid component/main.tex}
            \input{4 - numerical implementation/main.tex}
        \part{Project Overview}
            \input{5 - research plan/main.tex}
            \input{6 - summary/main.tex}
    
    
    %\section{}
    \newpage
    \pagenumbering{gobble}
        \printbibliography


    \newpage
    \pagenumbering{roman}
    \appendix
        \part{Appendices}
            \input{8 - Hilbert complexes/main.tex}
            \input{9 - weak conservation proofs/main.tex}
\end{document}

            \documentclass[12pt, a4paper]{report}

\input{template/main.tex}

\title{\BA{Title in Progress...}}
\author{Boris Andrews}
\affil{Mathematical Institute, University of Oxford}
\date{\today}


\begin{document}
    \pagenumbering{gobble}
    \maketitle
    
    
    \begin{abstract}
        Magnetic confinement reactors---in particular tokamaks---offer one of the most promising options for achieving practical nuclear fusion, with the potential to provide virtually limitless, clean energy. The theoretical and numerical modeling of tokamak plasmas is simultaneously an essential component of effective reactor design, and a great research barrier. Tokamak operational conditions exhibit comparatively low Knudsen numbers. Kinetic effects, including kinetic waves and instabilities, Landau damping, bump-on-tail instabilities and more, are therefore highly influential in tokamak plasma dynamics. Purely fluid models are inherently incapable of capturing these effects, whereas the high dimensionality in purely kinetic models render them practically intractable for most relevant purposes.

        We consider a $\delta\!f$ decomposition model, with a macroscopic fluid background and microscopic kinetic correction, both fully coupled to each other. A similar manner of discretization is proposed to that used in the recent \texttt{STRUPHY} code \cite{Holderied_Possanner_Wang_2021, Holderied_2022, Li_et_al_2023} with a finite-element model for the background and a pseudo-particle/PiC model for the correction.

        The fluid background satisfies the full, non-linear, resistive, compressible, Hall MHD equations. \cite{Laakmann_Hu_Farrell_2022} introduces finite-element(-in-space) implicit timesteppers for the incompressible analogue to this system with structure-preserving (SP) properties in the ideal case, alongside parameter-robust preconditioners. We show that these timesteppers can derive from a finite-element-in-time (FET) (and finite-element-in-space) interpretation. The benefits of this reformulation are discussed, including the derivation of timesteppers that are higher order in time, and the quantifiable dissipative SP properties in the non-ideal, resistive case.
        
        We discuss possible options for extending this FET approach to timesteppers for the compressible case.

        The kinetic corrections satisfy linearized Boltzmann equations. Using a Lénard--Bernstein collision operator, these take Fokker--Planck-like forms \cite{Fokker_1914, Planck_1917} wherein pseudo-particles in the numerical model obey the neoclassical transport equations, with particle-independent Brownian drift terms. This offers a rigorous methodology for incorporating collisions into the particle transport model, without coupling the equations of motions for each particle.
        
        Works by Chen, Chacón et al. \cite{Chen_Chacón_Barnes_2011, Chacón_Chen_Barnes_2013, Chen_Chacón_2014, Chen_Chacón_2015} have developed structure-preserving particle pushers for neoclassical transport in the Vlasov equations, derived from Crank--Nicolson integrators. We show these too can can derive from a FET interpretation, similarly offering potential extensions to higher-order-in-time particle pushers. The FET formulation is used also to consider how the stochastic drift terms can be incorporated into the pushers. Stochastic gyrokinetic expansions are also discussed.

        Different options for the numerical implementation of these schemes are considered.

        Due to the efficacy of FET in the development of SP timesteppers for both the fluid and kinetic component, we hope this approach will prove effective in the future for developing SP timesteppers for the full hybrid model. We hope this will give us the opportunity to incorporate previously inaccessible kinetic effects into the highly effective, modern, finite-element MHD models.
    \end{abstract}
    
    
    \newpage
    \tableofcontents
    
    
    \newpage
    \pagenumbering{arabic}
    %\linenumbers\renewcommand\thelinenumber{\color{black!50}\arabic{linenumber}}
            \input{0 - introduction/main.tex}
        \part{Research}
            \input{1 - low-noise PiC models/main.tex}
            \input{2 - kinetic component/main.tex}
            \input{3 - fluid component/main.tex}
            \input{4 - numerical implementation/main.tex}
        \part{Project Overview}
            \input{5 - research plan/main.tex}
            \input{6 - summary/main.tex}
    
    
    %\section{}
    \newpage
    \pagenumbering{gobble}
        \printbibliography


    \newpage
    \pagenumbering{roman}
    \appendix
        \part{Appendices}
            \input{8 - Hilbert complexes/main.tex}
            \input{9 - weak conservation proofs/main.tex}
\end{document}

\end{document}

            \documentclass[12pt, a4paper]{report}

\documentclass[12pt, a4paper]{report}

\input{template/main.tex}

\title{\BA{Title in Progress...}}
\author{Boris Andrews}
\affil{Mathematical Institute, University of Oxford}
\date{\today}


\begin{document}
    \pagenumbering{gobble}
    \maketitle
    
    
    \begin{abstract}
        Magnetic confinement reactors---in particular tokamaks---offer one of the most promising options for achieving practical nuclear fusion, with the potential to provide virtually limitless, clean energy. The theoretical and numerical modeling of tokamak plasmas is simultaneously an essential component of effective reactor design, and a great research barrier. Tokamak operational conditions exhibit comparatively low Knudsen numbers. Kinetic effects, including kinetic waves and instabilities, Landau damping, bump-on-tail instabilities and more, are therefore highly influential in tokamak plasma dynamics. Purely fluid models are inherently incapable of capturing these effects, whereas the high dimensionality in purely kinetic models render them practically intractable for most relevant purposes.

        We consider a $\delta\!f$ decomposition model, with a macroscopic fluid background and microscopic kinetic correction, both fully coupled to each other. A similar manner of discretization is proposed to that used in the recent \texttt{STRUPHY} code \cite{Holderied_Possanner_Wang_2021, Holderied_2022, Li_et_al_2023} with a finite-element model for the background and a pseudo-particle/PiC model for the correction.

        The fluid background satisfies the full, non-linear, resistive, compressible, Hall MHD equations. \cite{Laakmann_Hu_Farrell_2022} introduces finite-element(-in-space) implicit timesteppers for the incompressible analogue to this system with structure-preserving (SP) properties in the ideal case, alongside parameter-robust preconditioners. We show that these timesteppers can derive from a finite-element-in-time (FET) (and finite-element-in-space) interpretation. The benefits of this reformulation are discussed, including the derivation of timesteppers that are higher order in time, and the quantifiable dissipative SP properties in the non-ideal, resistive case.
        
        We discuss possible options for extending this FET approach to timesteppers for the compressible case.

        The kinetic corrections satisfy linearized Boltzmann equations. Using a Lénard--Bernstein collision operator, these take Fokker--Planck-like forms \cite{Fokker_1914, Planck_1917} wherein pseudo-particles in the numerical model obey the neoclassical transport equations, with particle-independent Brownian drift terms. This offers a rigorous methodology for incorporating collisions into the particle transport model, without coupling the equations of motions for each particle.
        
        Works by Chen, Chacón et al. \cite{Chen_Chacón_Barnes_2011, Chacón_Chen_Barnes_2013, Chen_Chacón_2014, Chen_Chacón_2015} have developed structure-preserving particle pushers for neoclassical transport in the Vlasov equations, derived from Crank--Nicolson integrators. We show these too can can derive from a FET interpretation, similarly offering potential extensions to higher-order-in-time particle pushers. The FET formulation is used also to consider how the stochastic drift terms can be incorporated into the pushers. Stochastic gyrokinetic expansions are also discussed.

        Different options for the numerical implementation of these schemes are considered.

        Due to the efficacy of FET in the development of SP timesteppers for both the fluid and kinetic component, we hope this approach will prove effective in the future for developing SP timesteppers for the full hybrid model. We hope this will give us the opportunity to incorporate previously inaccessible kinetic effects into the highly effective, modern, finite-element MHD models.
    \end{abstract}
    
    
    \newpage
    \tableofcontents
    
    
    \newpage
    \pagenumbering{arabic}
    %\linenumbers\renewcommand\thelinenumber{\color{black!50}\arabic{linenumber}}
            \input{0 - introduction/main.tex}
        \part{Research}
            \input{1 - low-noise PiC models/main.tex}
            \input{2 - kinetic component/main.tex}
            \input{3 - fluid component/main.tex}
            \input{4 - numerical implementation/main.tex}
        \part{Project Overview}
            \input{5 - research plan/main.tex}
            \input{6 - summary/main.tex}
    
    
    %\section{}
    \newpage
    \pagenumbering{gobble}
        \printbibliography


    \newpage
    \pagenumbering{roman}
    \appendix
        \part{Appendices}
            \input{8 - Hilbert complexes/main.tex}
            \input{9 - weak conservation proofs/main.tex}
\end{document}


\title{\BA{Title in Progress...}}
\author{Boris Andrews}
\affil{Mathematical Institute, University of Oxford}
\date{\today}


\begin{document}
    \pagenumbering{gobble}
    \maketitle
    
    
    \begin{abstract}
        Magnetic confinement reactors---in particular tokamaks---offer one of the most promising options for achieving practical nuclear fusion, with the potential to provide virtually limitless, clean energy. The theoretical and numerical modeling of tokamak plasmas is simultaneously an essential component of effective reactor design, and a great research barrier. Tokamak operational conditions exhibit comparatively low Knudsen numbers. Kinetic effects, including kinetic waves and instabilities, Landau damping, bump-on-tail instabilities and more, are therefore highly influential in tokamak plasma dynamics. Purely fluid models are inherently incapable of capturing these effects, whereas the high dimensionality in purely kinetic models render them practically intractable for most relevant purposes.

        We consider a $\delta\!f$ decomposition model, with a macroscopic fluid background and microscopic kinetic correction, both fully coupled to each other. A similar manner of discretization is proposed to that used in the recent \texttt{STRUPHY} code \cite{Holderied_Possanner_Wang_2021, Holderied_2022, Li_et_al_2023} with a finite-element model for the background and a pseudo-particle/PiC model for the correction.

        The fluid background satisfies the full, non-linear, resistive, compressible, Hall MHD equations. \cite{Laakmann_Hu_Farrell_2022} introduces finite-element(-in-space) implicit timesteppers for the incompressible analogue to this system with structure-preserving (SP) properties in the ideal case, alongside parameter-robust preconditioners. We show that these timesteppers can derive from a finite-element-in-time (FET) (and finite-element-in-space) interpretation. The benefits of this reformulation are discussed, including the derivation of timesteppers that are higher order in time, and the quantifiable dissipative SP properties in the non-ideal, resistive case.
        
        We discuss possible options for extending this FET approach to timesteppers for the compressible case.

        The kinetic corrections satisfy linearized Boltzmann equations. Using a Lénard--Bernstein collision operator, these take Fokker--Planck-like forms \cite{Fokker_1914, Planck_1917} wherein pseudo-particles in the numerical model obey the neoclassical transport equations, with particle-independent Brownian drift terms. This offers a rigorous methodology for incorporating collisions into the particle transport model, without coupling the equations of motions for each particle.
        
        Works by Chen, Chacón et al. \cite{Chen_Chacón_Barnes_2011, Chacón_Chen_Barnes_2013, Chen_Chacón_2014, Chen_Chacón_2015} have developed structure-preserving particle pushers for neoclassical transport in the Vlasov equations, derived from Crank--Nicolson integrators. We show these too can can derive from a FET interpretation, similarly offering potential extensions to higher-order-in-time particle pushers. The FET formulation is used also to consider how the stochastic drift terms can be incorporated into the pushers. Stochastic gyrokinetic expansions are also discussed.

        Different options for the numerical implementation of these schemes are considered.

        Due to the efficacy of FET in the development of SP timesteppers for both the fluid and kinetic component, we hope this approach will prove effective in the future for developing SP timesteppers for the full hybrid model. We hope this will give us the opportunity to incorporate previously inaccessible kinetic effects into the highly effective, modern, finite-element MHD models.
    \end{abstract}
    
    
    \newpage
    \tableofcontents
    
    
    \newpage
    \pagenumbering{arabic}
    %\linenumbers\renewcommand\thelinenumber{\color{black!50}\arabic{linenumber}}
            \documentclass[12pt, a4paper]{report}

\input{template/main.tex}

\title{\BA{Title in Progress...}}
\author{Boris Andrews}
\affil{Mathematical Institute, University of Oxford}
\date{\today}


\begin{document}
    \pagenumbering{gobble}
    \maketitle
    
    
    \begin{abstract}
        Magnetic confinement reactors---in particular tokamaks---offer one of the most promising options for achieving practical nuclear fusion, with the potential to provide virtually limitless, clean energy. The theoretical and numerical modeling of tokamak plasmas is simultaneously an essential component of effective reactor design, and a great research barrier. Tokamak operational conditions exhibit comparatively low Knudsen numbers. Kinetic effects, including kinetic waves and instabilities, Landau damping, bump-on-tail instabilities and more, are therefore highly influential in tokamak plasma dynamics. Purely fluid models are inherently incapable of capturing these effects, whereas the high dimensionality in purely kinetic models render them practically intractable for most relevant purposes.

        We consider a $\delta\!f$ decomposition model, with a macroscopic fluid background and microscopic kinetic correction, both fully coupled to each other. A similar manner of discretization is proposed to that used in the recent \texttt{STRUPHY} code \cite{Holderied_Possanner_Wang_2021, Holderied_2022, Li_et_al_2023} with a finite-element model for the background and a pseudo-particle/PiC model for the correction.

        The fluid background satisfies the full, non-linear, resistive, compressible, Hall MHD equations. \cite{Laakmann_Hu_Farrell_2022} introduces finite-element(-in-space) implicit timesteppers for the incompressible analogue to this system with structure-preserving (SP) properties in the ideal case, alongside parameter-robust preconditioners. We show that these timesteppers can derive from a finite-element-in-time (FET) (and finite-element-in-space) interpretation. The benefits of this reformulation are discussed, including the derivation of timesteppers that are higher order in time, and the quantifiable dissipative SP properties in the non-ideal, resistive case.
        
        We discuss possible options for extending this FET approach to timesteppers for the compressible case.

        The kinetic corrections satisfy linearized Boltzmann equations. Using a Lénard--Bernstein collision operator, these take Fokker--Planck-like forms \cite{Fokker_1914, Planck_1917} wherein pseudo-particles in the numerical model obey the neoclassical transport equations, with particle-independent Brownian drift terms. This offers a rigorous methodology for incorporating collisions into the particle transport model, without coupling the equations of motions for each particle.
        
        Works by Chen, Chacón et al. \cite{Chen_Chacón_Barnes_2011, Chacón_Chen_Barnes_2013, Chen_Chacón_2014, Chen_Chacón_2015} have developed structure-preserving particle pushers for neoclassical transport in the Vlasov equations, derived from Crank--Nicolson integrators. We show these too can can derive from a FET interpretation, similarly offering potential extensions to higher-order-in-time particle pushers. The FET formulation is used also to consider how the stochastic drift terms can be incorporated into the pushers. Stochastic gyrokinetic expansions are also discussed.

        Different options for the numerical implementation of these schemes are considered.

        Due to the efficacy of FET in the development of SP timesteppers for both the fluid and kinetic component, we hope this approach will prove effective in the future for developing SP timesteppers for the full hybrid model. We hope this will give us the opportunity to incorporate previously inaccessible kinetic effects into the highly effective, modern, finite-element MHD models.
    \end{abstract}
    
    
    \newpage
    \tableofcontents
    
    
    \newpage
    \pagenumbering{arabic}
    %\linenumbers\renewcommand\thelinenumber{\color{black!50}\arabic{linenumber}}
            \input{0 - introduction/main.tex}
        \part{Research}
            \input{1 - low-noise PiC models/main.tex}
            \input{2 - kinetic component/main.tex}
            \input{3 - fluid component/main.tex}
            \input{4 - numerical implementation/main.tex}
        \part{Project Overview}
            \input{5 - research plan/main.tex}
            \input{6 - summary/main.tex}
    
    
    %\section{}
    \newpage
    \pagenumbering{gobble}
        \printbibliography


    \newpage
    \pagenumbering{roman}
    \appendix
        \part{Appendices}
            \input{8 - Hilbert complexes/main.tex}
            \input{9 - weak conservation proofs/main.tex}
\end{document}

        \part{Research}
            \documentclass[12pt, a4paper]{report}

\input{template/main.tex}

\title{\BA{Title in Progress...}}
\author{Boris Andrews}
\affil{Mathematical Institute, University of Oxford}
\date{\today}


\begin{document}
    \pagenumbering{gobble}
    \maketitle
    
    
    \begin{abstract}
        Magnetic confinement reactors---in particular tokamaks---offer one of the most promising options for achieving practical nuclear fusion, with the potential to provide virtually limitless, clean energy. The theoretical and numerical modeling of tokamak plasmas is simultaneously an essential component of effective reactor design, and a great research barrier. Tokamak operational conditions exhibit comparatively low Knudsen numbers. Kinetic effects, including kinetic waves and instabilities, Landau damping, bump-on-tail instabilities and more, are therefore highly influential in tokamak plasma dynamics. Purely fluid models are inherently incapable of capturing these effects, whereas the high dimensionality in purely kinetic models render them practically intractable for most relevant purposes.

        We consider a $\delta\!f$ decomposition model, with a macroscopic fluid background and microscopic kinetic correction, both fully coupled to each other. A similar manner of discretization is proposed to that used in the recent \texttt{STRUPHY} code \cite{Holderied_Possanner_Wang_2021, Holderied_2022, Li_et_al_2023} with a finite-element model for the background and a pseudo-particle/PiC model for the correction.

        The fluid background satisfies the full, non-linear, resistive, compressible, Hall MHD equations. \cite{Laakmann_Hu_Farrell_2022} introduces finite-element(-in-space) implicit timesteppers for the incompressible analogue to this system with structure-preserving (SP) properties in the ideal case, alongside parameter-robust preconditioners. We show that these timesteppers can derive from a finite-element-in-time (FET) (and finite-element-in-space) interpretation. The benefits of this reformulation are discussed, including the derivation of timesteppers that are higher order in time, and the quantifiable dissipative SP properties in the non-ideal, resistive case.
        
        We discuss possible options for extending this FET approach to timesteppers for the compressible case.

        The kinetic corrections satisfy linearized Boltzmann equations. Using a Lénard--Bernstein collision operator, these take Fokker--Planck-like forms \cite{Fokker_1914, Planck_1917} wherein pseudo-particles in the numerical model obey the neoclassical transport equations, with particle-independent Brownian drift terms. This offers a rigorous methodology for incorporating collisions into the particle transport model, without coupling the equations of motions for each particle.
        
        Works by Chen, Chacón et al. \cite{Chen_Chacón_Barnes_2011, Chacón_Chen_Barnes_2013, Chen_Chacón_2014, Chen_Chacón_2015} have developed structure-preserving particle pushers for neoclassical transport in the Vlasov equations, derived from Crank--Nicolson integrators. We show these too can can derive from a FET interpretation, similarly offering potential extensions to higher-order-in-time particle pushers. The FET formulation is used also to consider how the stochastic drift terms can be incorporated into the pushers. Stochastic gyrokinetic expansions are also discussed.

        Different options for the numerical implementation of these schemes are considered.

        Due to the efficacy of FET in the development of SP timesteppers for both the fluid and kinetic component, we hope this approach will prove effective in the future for developing SP timesteppers for the full hybrid model. We hope this will give us the opportunity to incorporate previously inaccessible kinetic effects into the highly effective, modern, finite-element MHD models.
    \end{abstract}
    
    
    \newpage
    \tableofcontents
    
    
    \newpage
    \pagenumbering{arabic}
    %\linenumbers\renewcommand\thelinenumber{\color{black!50}\arabic{linenumber}}
            \input{0 - introduction/main.tex}
        \part{Research}
            \input{1 - low-noise PiC models/main.tex}
            \input{2 - kinetic component/main.tex}
            \input{3 - fluid component/main.tex}
            \input{4 - numerical implementation/main.tex}
        \part{Project Overview}
            \input{5 - research plan/main.tex}
            \input{6 - summary/main.tex}
    
    
    %\section{}
    \newpage
    \pagenumbering{gobble}
        \printbibliography


    \newpage
    \pagenumbering{roman}
    \appendix
        \part{Appendices}
            \input{8 - Hilbert complexes/main.tex}
            \input{9 - weak conservation proofs/main.tex}
\end{document}

            \documentclass[12pt, a4paper]{report}

\input{template/main.tex}

\title{\BA{Title in Progress...}}
\author{Boris Andrews}
\affil{Mathematical Institute, University of Oxford}
\date{\today}


\begin{document}
    \pagenumbering{gobble}
    \maketitle
    
    
    \begin{abstract}
        Magnetic confinement reactors---in particular tokamaks---offer one of the most promising options for achieving practical nuclear fusion, with the potential to provide virtually limitless, clean energy. The theoretical and numerical modeling of tokamak plasmas is simultaneously an essential component of effective reactor design, and a great research barrier. Tokamak operational conditions exhibit comparatively low Knudsen numbers. Kinetic effects, including kinetic waves and instabilities, Landau damping, bump-on-tail instabilities and more, are therefore highly influential in tokamak plasma dynamics. Purely fluid models are inherently incapable of capturing these effects, whereas the high dimensionality in purely kinetic models render them practically intractable for most relevant purposes.

        We consider a $\delta\!f$ decomposition model, with a macroscopic fluid background and microscopic kinetic correction, both fully coupled to each other. A similar manner of discretization is proposed to that used in the recent \texttt{STRUPHY} code \cite{Holderied_Possanner_Wang_2021, Holderied_2022, Li_et_al_2023} with a finite-element model for the background and a pseudo-particle/PiC model for the correction.

        The fluid background satisfies the full, non-linear, resistive, compressible, Hall MHD equations. \cite{Laakmann_Hu_Farrell_2022} introduces finite-element(-in-space) implicit timesteppers for the incompressible analogue to this system with structure-preserving (SP) properties in the ideal case, alongside parameter-robust preconditioners. We show that these timesteppers can derive from a finite-element-in-time (FET) (and finite-element-in-space) interpretation. The benefits of this reformulation are discussed, including the derivation of timesteppers that are higher order in time, and the quantifiable dissipative SP properties in the non-ideal, resistive case.
        
        We discuss possible options for extending this FET approach to timesteppers for the compressible case.

        The kinetic corrections satisfy linearized Boltzmann equations. Using a Lénard--Bernstein collision operator, these take Fokker--Planck-like forms \cite{Fokker_1914, Planck_1917} wherein pseudo-particles in the numerical model obey the neoclassical transport equations, with particle-independent Brownian drift terms. This offers a rigorous methodology for incorporating collisions into the particle transport model, without coupling the equations of motions for each particle.
        
        Works by Chen, Chacón et al. \cite{Chen_Chacón_Barnes_2011, Chacón_Chen_Barnes_2013, Chen_Chacón_2014, Chen_Chacón_2015} have developed structure-preserving particle pushers for neoclassical transport in the Vlasov equations, derived from Crank--Nicolson integrators. We show these too can can derive from a FET interpretation, similarly offering potential extensions to higher-order-in-time particle pushers. The FET formulation is used also to consider how the stochastic drift terms can be incorporated into the pushers. Stochastic gyrokinetic expansions are also discussed.

        Different options for the numerical implementation of these schemes are considered.

        Due to the efficacy of FET in the development of SP timesteppers for both the fluid and kinetic component, we hope this approach will prove effective in the future for developing SP timesteppers for the full hybrid model. We hope this will give us the opportunity to incorporate previously inaccessible kinetic effects into the highly effective, modern, finite-element MHD models.
    \end{abstract}
    
    
    \newpage
    \tableofcontents
    
    
    \newpage
    \pagenumbering{arabic}
    %\linenumbers\renewcommand\thelinenumber{\color{black!50}\arabic{linenumber}}
            \input{0 - introduction/main.tex}
        \part{Research}
            \input{1 - low-noise PiC models/main.tex}
            \input{2 - kinetic component/main.tex}
            \input{3 - fluid component/main.tex}
            \input{4 - numerical implementation/main.tex}
        \part{Project Overview}
            \input{5 - research plan/main.tex}
            \input{6 - summary/main.tex}
    
    
    %\section{}
    \newpage
    \pagenumbering{gobble}
        \printbibliography


    \newpage
    \pagenumbering{roman}
    \appendix
        \part{Appendices}
            \input{8 - Hilbert complexes/main.tex}
            \input{9 - weak conservation proofs/main.tex}
\end{document}

            \documentclass[12pt, a4paper]{report}

\input{template/main.tex}

\title{\BA{Title in Progress...}}
\author{Boris Andrews}
\affil{Mathematical Institute, University of Oxford}
\date{\today}


\begin{document}
    \pagenumbering{gobble}
    \maketitle
    
    
    \begin{abstract}
        Magnetic confinement reactors---in particular tokamaks---offer one of the most promising options for achieving practical nuclear fusion, with the potential to provide virtually limitless, clean energy. The theoretical and numerical modeling of tokamak plasmas is simultaneously an essential component of effective reactor design, and a great research barrier. Tokamak operational conditions exhibit comparatively low Knudsen numbers. Kinetic effects, including kinetic waves and instabilities, Landau damping, bump-on-tail instabilities and more, are therefore highly influential in tokamak plasma dynamics. Purely fluid models are inherently incapable of capturing these effects, whereas the high dimensionality in purely kinetic models render them practically intractable for most relevant purposes.

        We consider a $\delta\!f$ decomposition model, with a macroscopic fluid background and microscopic kinetic correction, both fully coupled to each other. A similar manner of discretization is proposed to that used in the recent \texttt{STRUPHY} code \cite{Holderied_Possanner_Wang_2021, Holderied_2022, Li_et_al_2023} with a finite-element model for the background and a pseudo-particle/PiC model for the correction.

        The fluid background satisfies the full, non-linear, resistive, compressible, Hall MHD equations. \cite{Laakmann_Hu_Farrell_2022} introduces finite-element(-in-space) implicit timesteppers for the incompressible analogue to this system with structure-preserving (SP) properties in the ideal case, alongside parameter-robust preconditioners. We show that these timesteppers can derive from a finite-element-in-time (FET) (and finite-element-in-space) interpretation. The benefits of this reformulation are discussed, including the derivation of timesteppers that are higher order in time, and the quantifiable dissipative SP properties in the non-ideal, resistive case.
        
        We discuss possible options for extending this FET approach to timesteppers for the compressible case.

        The kinetic corrections satisfy linearized Boltzmann equations. Using a Lénard--Bernstein collision operator, these take Fokker--Planck-like forms \cite{Fokker_1914, Planck_1917} wherein pseudo-particles in the numerical model obey the neoclassical transport equations, with particle-independent Brownian drift terms. This offers a rigorous methodology for incorporating collisions into the particle transport model, without coupling the equations of motions for each particle.
        
        Works by Chen, Chacón et al. \cite{Chen_Chacón_Barnes_2011, Chacón_Chen_Barnes_2013, Chen_Chacón_2014, Chen_Chacón_2015} have developed structure-preserving particle pushers for neoclassical transport in the Vlasov equations, derived from Crank--Nicolson integrators. We show these too can can derive from a FET interpretation, similarly offering potential extensions to higher-order-in-time particle pushers. The FET formulation is used also to consider how the stochastic drift terms can be incorporated into the pushers. Stochastic gyrokinetic expansions are also discussed.

        Different options for the numerical implementation of these schemes are considered.

        Due to the efficacy of FET in the development of SP timesteppers for both the fluid and kinetic component, we hope this approach will prove effective in the future for developing SP timesteppers for the full hybrid model. We hope this will give us the opportunity to incorporate previously inaccessible kinetic effects into the highly effective, modern, finite-element MHD models.
    \end{abstract}
    
    
    \newpage
    \tableofcontents
    
    
    \newpage
    \pagenumbering{arabic}
    %\linenumbers\renewcommand\thelinenumber{\color{black!50}\arabic{linenumber}}
            \input{0 - introduction/main.tex}
        \part{Research}
            \input{1 - low-noise PiC models/main.tex}
            \input{2 - kinetic component/main.tex}
            \input{3 - fluid component/main.tex}
            \input{4 - numerical implementation/main.tex}
        \part{Project Overview}
            \input{5 - research plan/main.tex}
            \input{6 - summary/main.tex}
    
    
    %\section{}
    \newpage
    \pagenumbering{gobble}
        \printbibliography


    \newpage
    \pagenumbering{roman}
    \appendix
        \part{Appendices}
            \input{8 - Hilbert complexes/main.tex}
            \input{9 - weak conservation proofs/main.tex}
\end{document}

            \documentclass[12pt, a4paper]{report}

\input{template/main.tex}

\title{\BA{Title in Progress...}}
\author{Boris Andrews}
\affil{Mathematical Institute, University of Oxford}
\date{\today}


\begin{document}
    \pagenumbering{gobble}
    \maketitle
    
    
    \begin{abstract}
        Magnetic confinement reactors---in particular tokamaks---offer one of the most promising options for achieving practical nuclear fusion, with the potential to provide virtually limitless, clean energy. The theoretical and numerical modeling of tokamak plasmas is simultaneously an essential component of effective reactor design, and a great research barrier. Tokamak operational conditions exhibit comparatively low Knudsen numbers. Kinetic effects, including kinetic waves and instabilities, Landau damping, bump-on-tail instabilities and more, are therefore highly influential in tokamak plasma dynamics. Purely fluid models are inherently incapable of capturing these effects, whereas the high dimensionality in purely kinetic models render them practically intractable for most relevant purposes.

        We consider a $\delta\!f$ decomposition model, with a macroscopic fluid background and microscopic kinetic correction, both fully coupled to each other. A similar manner of discretization is proposed to that used in the recent \texttt{STRUPHY} code \cite{Holderied_Possanner_Wang_2021, Holderied_2022, Li_et_al_2023} with a finite-element model for the background and a pseudo-particle/PiC model for the correction.

        The fluid background satisfies the full, non-linear, resistive, compressible, Hall MHD equations. \cite{Laakmann_Hu_Farrell_2022} introduces finite-element(-in-space) implicit timesteppers for the incompressible analogue to this system with structure-preserving (SP) properties in the ideal case, alongside parameter-robust preconditioners. We show that these timesteppers can derive from a finite-element-in-time (FET) (and finite-element-in-space) interpretation. The benefits of this reformulation are discussed, including the derivation of timesteppers that are higher order in time, and the quantifiable dissipative SP properties in the non-ideal, resistive case.
        
        We discuss possible options for extending this FET approach to timesteppers for the compressible case.

        The kinetic corrections satisfy linearized Boltzmann equations. Using a Lénard--Bernstein collision operator, these take Fokker--Planck-like forms \cite{Fokker_1914, Planck_1917} wherein pseudo-particles in the numerical model obey the neoclassical transport equations, with particle-independent Brownian drift terms. This offers a rigorous methodology for incorporating collisions into the particle transport model, without coupling the equations of motions for each particle.
        
        Works by Chen, Chacón et al. \cite{Chen_Chacón_Barnes_2011, Chacón_Chen_Barnes_2013, Chen_Chacón_2014, Chen_Chacón_2015} have developed structure-preserving particle pushers for neoclassical transport in the Vlasov equations, derived from Crank--Nicolson integrators. We show these too can can derive from a FET interpretation, similarly offering potential extensions to higher-order-in-time particle pushers. The FET formulation is used also to consider how the stochastic drift terms can be incorporated into the pushers. Stochastic gyrokinetic expansions are also discussed.

        Different options for the numerical implementation of these schemes are considered.

        Due to the efficacy of FET in the development of SP timesteppers for both the fluid and kinetic component, we hope this approach will prove effective in the future for developing SP timesteppers for the full hybrid model. We hope this will give us the opportunity to incorporate previously inaccessible kinetic effects into the highly effective, modern, finite-element MHD models.
    \end{abstract}
    
    
    \newpage
    \tableofcontents
    
    
    \newpage
    \pagenumbering{arabic}
    %\linenumbers\renewcommand\thelinenumber{\color{black!50}\arabic{linenumber}}
            \input{0 - introduction/main.tex}
        \part{Research}
            \input{1 - low-noise PiC models/main.tex}
            \input{2 - kinetic component/main.tex}
            \input{3 - fluid component/main.tex}
            \input{4 - numerical implementation/main.tex}
        \part{Project Overview}
            \input{5 - research plan/main.tex}
            \input{6 - summary/main.tex}
    
    
    %\section{}
    \newpage
    \pagenumbering{gobble}
        \printbibliography


    \newpage
    \pagenumbering{roman}
    \appendix
        \part{Appendices}
            \input{8 - Hilbert complexes/main.tex}
            \input{9 - weak conservation proofs/main.tex}
\end{document}

        \part{Project Overview}
            \documentclass[12pt, a4paper]{report}

\input{template/main.tex}

\title{\BA{Title in Progress...}}
\author{Boris Andrews}
\affil{Mathematical Institute, University of Oxford}
\date{\today}


\begin{document}
    \pagenumbering{gobble}
    \maketitle
    
    
    \begin{abstract}
        Magnetic confinement reactors---in particular tokamaks---offer one of the most promising options for achieving practical nuclear fusion, with the potential to provide virtually limitless, clean energy. The theoretical and numerical modeling of tokamak plasmas is simultaneously an essential component of effective reactor design, and a great research barrier. Tokamak operational conditions exhibit comparatively low Knudsen numbers. Kinetic effects, including kinetic waves and instabilities, Landau damping, bump-on-tail instabilities and more, are therefore highly influential in tokamak plasma dynamics. Purely fluid models are inherently incapable of capturing these effects, whereas the high dimensionality in purely kinetic models render them practically intractable for most relevant purposes.

        We consider a $\delta\!f$ decomposition model, with a macroscopic fluid background and microscopic kinetic correction, both fully coupled to each other. A similar manner of discretization is proposed to that used in the recent \texttt{STRUPHY} code \cite{Holderied_Possanner_Wang_2021, Holderied_2022, Li_et_al_2023} with a finite-element model for the background and a pseudo-particle/PiC model for the correction.

        The fluid background satisfies the full, non-linear, resistive, compressible, Hall MHD equations. \cite{Laakmann_Hu_Farrell_2022} introduces finite-element(-in-space) implicit timesteppers for the incompressible analogue to this system with structure-preserving (SP) properties in the ideal case, alongside parameter-robust preconditioners. We show that these timesteppers can derive from a finite-element-in-time (FET) (and finite-element-in-space) interpretation. The benefits of this reformulation are discussed, including the derivation of timesteppers that are higher order in time, and the quantifiable dissipative SP properties in the non-ideal, resistive case.
        
        We discuss possible options for extending this FET approach to timesteppers for the compressible case.

        The kinetic corrections satisfy linearized Boltzmann equations. Using a Lénard--Bernstein collision operator, these take Fokker--Planck-like forms \cite{Fokker_1914, Planck_1917} wherein pseudo-particles in the numerical model obey the neoclassical transport equations, with particle-independent Brownian drift terms. This offers a rigorous methodology for incorporating collisions into the particle transport model, without coupling the equations of motions for each particle.
        
        Works by Chen, Chacón et al. \cite{Chen_Chacón_Barnes_2011, Chacón_Chen_Barnes_2013, Chen_Chacón_2014, Chen_Chacón_2015} have developed structure-preserving particle pushers for neoclassical transport in the Vlasov equations, derived from Crank--Nicolson integrators. We show these too can can derive from a FET interpretation, similarly offering potential extensions to higher-order-in-time particle pushers. The FET formulation is used also to consider how the stochastic drift terms can be incorporated into the pushers. Stochastic gyrokinetic expansions are also discussed.

        Different options for the numerical implementation of these schemes are considered.

        Due to the efficacy of FET in the development of SP timesteppers for both the fluid and kinetic component, we hope this approach will prove effective in the future for developing SP timesteppers for the full hybrid model. We hope this will give us the opportunity to incorporate previously inaccessible kinetic effects into the highly effective, modern, finite-element MHD models.
    \end{abstract}
    
    
    \newpage
    \tableofcontents
    
    
    \newpage
    \pagenumbering{arabic}
    %\linenumbers\renewcommand\thelinenumber{\color{black!50}\arabic{linenumber}}
            \input{0 - introduction/main.tex}
        \part{Research}
            \input{1 - low-noise PiC models/main.tex}
            \input{2 - kinetic component/main.tex}
            \input{3 - fluid component/main.tex}
            \input{4 - numerical implementation/main.tex}
        \part{Project Overview}
            \input{5 - research plan/main.tex}
            \input{6 - summary/main.tex}
    
    
    %\section{}
    \newpage
    \pagenumbering{gobble}
        \printbibliography


    \newpage
    \pagenumbering{roman}
    \appendix
        \part{Appendices}
            \input{8 - Hilbert complexes/main.tex}
            \input{9 - weak conservation proofs/main.tex}
\end{document}

            \documentclass[12pt, a4paper]{report}

\input{template/main.tex}

\title{\BA{Title in Progress...}}
\author{Boris Andrews}
\affil{Mathematical Institute, University of Oxford}
\date{\today}


\begin{document}
    \pagenumbering{gobble}
    \maketitle
    
    
    \begin{abstract}
        Magnetic confinement reactors---in particular tokamaks---offer one of the most promising options for achieving practical nuclear fusion, with the potential to provide virtually limitless, clean energy. The theoretical and numerical modeling of tokamak plasmas is simultaneously an essential component of effective reactor design, and a great research barrier. Tokamak operational conditions exhibit comparatively low Knudsen numbers. Kinetic effects, including kinetic waves and instabilities, Landau damping, bump-on-tail instabilities and more, are therefore highly influential in tokamak plasma dynamics. Purely fluid models are inherently incapable of capturing these effects, whereas the high dimensionality in purely kinetic models render them practically intractable for most relevant purposes.

        We consider a $\delta\!f$ decomposition model, with a macroscopic fluid background and microscopic kinetic correction, both fully coupled to each other. A similar manner of discretization is proposed to that used in the recent \texttt{STRUPHY} code \cite{Holderied_Possanner_Wang_2021, Holderied_2022, Li_et_al_2023} with a finite-element model for the background and a pseudo-particle/PiC model for the correction.

        The fluid background satisfies the full, non-linear, resistive, compressible, Hall MHD equations. \cite{Laakmann_Hu_Farrell_2022} introduces finite-element(-in-space) implicit timesteppers for the incompressible analogue to this system with structure-preserving (SP) properties in the ideal case, alongside parameter-robust preconditioners. We show that these timesteppers can derive from a finite-element-in-time (FET) (and finite-element-in-space) interpretation. The benefits of this reformulation are discussed, including the derivation of timesteppers that are higher order in time, and the quantifiable dissipative SP properties in the non-ideal, resistive case.
        
        We discuss possible options for extending this FET approach to timesteppers for the compressible case.

        The kinetic corrections satisfy linearized Boltzmann equations. Using a Lénard--Bernstein collision operator, these take Fokker--Planck-like forms \cite{Fokker_1914, Planck_1917} wherein pseudo-particles in the numerical model obey the neoclassical transport equations, with particle-independent Brownian drift terms. This offers a rigorous methodology for incorporating collisions into the particle transport model, without coupling the equations of motions for each particle.
        
        Works by Chen, Chacón et al. \cite{Chen_Chacón_Barnes_2011, Chacón_Chen_Barnes_2013, Chen_Chacón_2014, Chen_Chacón_2015} have developed structure-preserving particle pushers for neoclassical transport in the Vlasov equations, derived from Crank--Nicolson integrators. We show these too can can derive from a FET interpretation, similarly offering potential extensions to higher-order-in-time particle pushers. The FET formulation is used also to consider how the stochastic drift terms can be incorporated into the pushers. Stochastic gyrokinetic expansions are also discussed.

        Different options for the numerical implementation of these schemes are considered.

        Due to the efficacy of FET in the development of SP timesteppers for both the fluid and kinetic component, we hope this approach will prove effective in the future for developing SP timesteppers for the full hybrid model. We hope this will give us the opportunity to incorporate previously inaccessible kinetic effects into the highly effective, modern, finite-element MHD models.
    \end{abstract}
    
    
    \newpage
    \tableofcontents
    
    
    \newpage
    \pagenumbering{arabic}
    %\linenumbers\renewcommand\thelinenumber{\color{black!50}\arabic{linenumber}}
            \input{0 - introduction/main.tex}
        \part{Research}
            \input{1 - low-noise PiC models/main.tex}
            \input{2 - kinetic component/main.tex}
            \input{3 - fluid component/main.tex}
            \input{4 - numerical implementation/main.tex}
        \part{Project Overview}
            \input{5 - research plan/main.tex}
            \input{6 - summary/main.tex}
    
    
    %\section{}
    \newpage
    \pagenumbering{gobble}
        \printbibliography


    \newpage
    \pagenumbering{roman}
    \appendix
        \part{Appendices}
            \input{8 - Hilbert complexes/main.tex}
            \input{9 - weak conservation proofs/main.tex}
\end{document}

    
    
    %\section{}
    \newpage
    \pagenumbering{gobble}
        \printbibliography


    \newpage
    \pagenumbering{roman}
    \appendix
        \part{Appendices}
            \documentclass[12pt, a4paper]{report}

\input{template/main.tex}

\title{\BA{Title in Progress...}}
\author{Boris Andrews}
\affil{Mathematical Institute, University of Oxford}
\date{\today}


\begin{document}
    \pagenumbering{gobble}
    \maketitle
    
    
    \begin{abstract}
        Magnetic confinement reactors---in particular tokamaks---offer one of the most promising options for achieving practical nuclear fusion, with the potential to provide virtually limitless, clean energy. The theoretical and numerical modeling of tokamak plasmas is simultaneously an essential component of effective reactor design, and a great research barrier. Tokamak operational conditions exhibit comparatively low Knudsen numbers. Kinetic effects, including kinetic waves and instabilities, Landau damping, bump-on-tail instabilities and more, are therefore highly influential in tokamak plasma dynamics. Purely fluid models are inherently incapable of capturing these effects, whereas the high dimensionality in purely kinetic models render them practically intractable for most relevant purposes.

        We consider a $\delta\!f$ decomposition model, with a macroscopic fluid background and microscopic kinetic correction, both fully coupled to each other. A similar manner of discretization is proposed to that used in the recent \texttt{STRUPHY} code \cite{Holderied_Possanner_Wang_2021, Holderied_2022, Li_et_al_2023} with a finite-element model for the background and a pseudo-particle/PiC model for the correction.

        The fluid background satisfies the full, non-linear, resistive, compressible, Hall MHD equations. \cite{Laakmann_Hu_Farrell_2022} introduces finite-element(-in-space) implicit timesteppers for the incompressible analogue to this system with structure-preserving (SP) properties in the ideal case, alongside parameter-robust preconditioners. We show that these timesteppers can derive from a finite-element-in-time (FET) (and finite-element-in-space) interpretation. The benefits of this reformulation are discussed, including the derivation of timesteppers that are higher order in time, and the quantifiable dissipative SP properties in the non-ideal, resistive case.
        
        We discuss possible options for extending this FET approach to timesteppers for the compressible case.

        The kinetic corrections satisfy linearized Boltzmann equations. Using a Lénard--Bernstein collision operator, these take Fokker--Planck-like forms \cite{Fokker_1914, Planck_1917} wherein pseudo-particles in the numerical model obey the neoclassical transport equations, with particle-independent Brownian drift terms. This offers a rigorous methodology for incorporating collisions into the particle transport model, without coupling the equations of motions for each particle.
        
        Works by Chen, Chacón et al. \cite{Chen_Chacón_Barnes_2011, Chacón_Chen_Barnes_2013, Chen_Chacón_2014, Chen_Chacón_2015} have developed structure-preserving particle pushers for neoclassical transport in the Vlasov equations, derived from Crank--Nicolson integrators. We show these too can can derive from a FET interpretation, similarly offering potential extensions to higher-order-in-time particle pushers. The FET formulation is used also to consider how the stochastic drift terms can be incorporated into the pushers. Stochastic gyrokinetic expansions are also discussed.

        Different options for the numerical implementation of these schemes are considered.

        Due to the efficacy of FET in the development of SP timesteppers for both the fluid and kinetic component, we hope this approach will prove effective in the future for developing SP timesteppers for the full hybrid model. We hope this will give us the opportunity to incorporate previously inaccessible kinetic effects into the highly effective, modern, finite-element MHD models.
    \end{abstract}
    
    
    \newpage
    \tableofcontents
    
    
    \newpage
    \pagenumbering{arabic}
    %\linenumbers\renewcommand\thelinenumber{\color{black!50}\arabic{linenumber}}
            \input{0 - introduction/main.tex}
        \part{Research}
            \input{1 - low-noise PiC models/main.tex}
            \input{2 - kinetic component/main.tex}
            \input{3 - fluid component/main.tex}
            \input{4 - numerical implementation/main.tex}
        \part{Project Overview}
            \input{5 - research plan/main.tex}
            \input{6 - summary/main.tex}
    
    
    %\section{}
    \newpage
    \pagenumbering{gobble}
        \printbibliography


    \newpage
    \pagenumbering{roman}
    \appendix
        \part{Appendices}
            \input{8 - Hilbert complexes/main.tex}
            \input{9 - weak conservation proofs/main.tex}
\end{document}

            \documentclass[12pt, a4paper]{report}

\input{template/main.tex}

\title{\BA{Title in Progress...}}
\author{Boris Andrews}
\affil{Mathematical Institute, University of Oxford}
\date{\today}


\begin{document}
    \pagenumbering{gobble}
    \maketitle
    
    
    \begin{abstract}
        Magnetic confinement reactors---in particular tokamaks---offer one of the most promising options for achieving practical nuclear fusion, with the potential to provide virtually limitless, clean energy. The theoretical and numerical modeling of tokamak plasmas is simultaneously an essential component of effective reactor design, and a great research barrier. Tokamak operational conditions exhibit comparatively low Knudsen numbers. Kinetic effects, including kinetic waves and instabilities, Landau damping, bump-on-tail instabilities and more, are therefore highly influential in tokamak plasma dynamics. Purely fluid models are inherently incapable of capturing these effects, whereas the high dimensionality in purely kinetic models render them practically intractable for most relevant purposes.

        We consider a $\delta\!f$ decomposition model, with a macroscopic fluid background and microscopic kinetic correction, both fully coupled to each other. A similar manner of discretization is proposed to that used in the recent \texttt{STRUPHY} code \cite{Holderied_Possanner_Wang_2021, Holderied_2022, Li_et_al_2023} with a finite-element model for the background and a pseudo-particle/PiC model for the correction.

        The fluid background satisfies the full, non-linear, resistive, compressible, Hall MHD equations. \cite{Laakmann_Hu_Farrell_2022} introduces finite-element(-in-space) implicit timesteppers for the incompressible analogue to this system with structure-preserving (SP) properties in the ideal case, alongside parameter-robust preconditioners. We show that these timesteppers can derive from a finite-element-in-time (FET) (and finite-element-in-space) interpretation. The benefits of this reformulation are discussed, including the derivation of timesteppers that are higher order in time, and the quantifiable dissipative SP properties in the non-ideal, resistive case.
        
        We discuss possible options for extending this FET approach to timesteppers for the compressible case.

        The kinetic corrections satisfy linearized Boltzmann equations. Using a Lénard--Bernstein collision operator, these take Fokker--Planck-like forms \cite{Fokker_1914, Planck_1917} wherein pseudo-particles in the numerical model obey the neoclassical transport equations, with particle-independent Brownian drift terms. This offers a rigorous methodology for incorporating collisions into the particle transport model, without coupling the equations of motions for each particle.
        
        Works by Chen, Chacón et al. \cite{Chen_Chacón_Barnes_2011, Chacón_Chen_Barnes_2013, Chen_Chacón_2014, Chen_Chacón_2015} have developed structure-preserving particle pushers for neoclassical transport in the Vlasov equations, derived from Crank--Nicolson integrators. We show these too can can derive from a FET interpretation, similarly offering potential extensions to higher-order-in-time particle pushers. The FET formulation is used also to consider how the stochastic drift terms can be incorporated into the pushers. Stochastic gyrokinetic expansions are also discussed.

        Different options for the numerical implementation of these schemes are considered.

        Due to the efficacy of FET in the development of SP timesteppers for both the fluid and kinetic component, we hope this approach will prove effective in the future for developing SP timesteppers for the full hybrid model. We hope this will give us the opportunity to incorporate previously inaccessible kinetic effects into the highly effective, modern, finite-element MHD models.
    \end{abstract}
    
    
    \newpage
    \tableofcontents
    
    
    \newpage
    \pagenumbering{arabic}
    %\linenumbers\renewcommand\thelinenumber{\color{black!50}\arabic{linenumber}}
            \input{0 - introduction/main.tex}
        \part{Research}
            \input{1 - low-noise PiC models/main.tex}
            \input{2 - kinetic component/main.tex}
            \input{3 - fluid component/main.tex}
            \input{4 - numerical implementation/main.tex}
        \part{Project Overview}
            \input{5 - research plan/main.tex}
            \input{6 - summary/main.tex}
    
    
    %\section{}
    \newpage
    \pagenumbering{gobble}
        \printbibliography


    \newpage
    \pagenumbering{roman}
    \appendix
        \part{Appendices}
            \input{8 - Hilbert complexes/main.tex}
            \input{9 - weak conservation proofs/main.tex}
\end{document}

\end{document}

    
    
    %\section{}
    \newpage
    \pagenumbering{gobble}
        \printbibliography


    \newpage
    \pagenumbering{roman}
    \appendix
        \part{Appendices}
            \documentclass[12pt, a4paper]{report}

\documentclass[12pt, a4paper]{report}

\input{template/main.tex}

\title{\BA{Title in Progress...}}
\author{Boris Andrews}
\affil{Mathematical Institute, University of Oxford}
\date{\today}


\begin{document}
    \pagenumbering{gobble}
    \maketitle
    
    
    \begin{abstract}
        Magnetic confinement reactors---in particular tokamaks---offer one of the most promising options for achieving practical nuclear fusion, with the potential to provide virtually limitless, clean energy. The theoretical and numerical modeling of tokamak plasmas is simultaneously an essential component of effective reactor design, and a great research barrier. Tokamak operational conditions exhibit comparatively low Knudsen numbers. Kinetic effects, including kinetic waves and instabilities, Landau damping, bump-on-tail instabilities and more, are therefore highly influential in tokamak plasma dynamics. Purely fluid models are inherently incapable of capturing these effects, whereas the high dimensionality in purely kinetic models render them practically intractable for most relevant purposes.

        We consider a $\delta\!f$ decomposition model, with a macroscopic fluid background and microscopic kinetic correction, both fully coupled to each other. A similar manner of discretization is proposed to that used in the recent \texttt{STRUPHY} code \cite{Holderied_Possanner_Wang_2021, Holderied_2022, Li_et_al_2023} with a finite-element model for the background and a pseudo-particle/PiC model for the correction.

        The fluid background satisfies the full, non-linear, resistive, compressible, Hall MHD equations. \cite{Laakmann_Hu_Farrell_2022} introduces finite-element(-in-space) implicit timesteppers for the incompressible analogue to this system with structure-preserving (SP) properties in the ideal case, alongside parameter-robust preconditioners. We show that these timesteppers can derive from a finite-element-in-time (FET) (and finite-element-in-space) interpretation. The benefits of this reformulation are discussed, including the derivation of timesteppers that are higher order in time, and the quantifiable dissipative SP properties in the non-ideal, resistive case.
        
        We discuss possible options for extending this FET approach to timesteppers for the compressible case.

        The kinetic corrections satisfy linearized Boltzmann equations. Using a Lénard--Bernstein collision operator, these take Fokker--Planck-like forms \cite{Fokker_1914, Planck_1917} wherein pseudo-particles in the numerical model obey the neoclassical transport equations, with particle-independent Brownian drift terms. This offers a rigorous methodology for incorporating collisions into the particle transport model, without coupling the equations of motions for each particle.
        
        Works by Chen, Chacón et al. \cite{Chen_Chacón_Barnes_2011, Chacón_Chen_Barnes_2013, Chen_Chacón_2014, Chen_Chacón_2015} have developed structure-preserving particle pushers for neoclassical transport in the Vlasov equations, derived from Crank--Nicolson integrators. We show these too can can derive from a FET interpretation, similarly offering potential extensions to higher-order-in-time particle pushers. The FET formulation is used also to consider how the stochastic drift terms can be incorporated into the pushers. Stochastic gyrokinetic expansions are also discussed.

        Different options for the numerical implementation of these schemes are considered.

        Due to the efficacy of FET in the development of SP timesteppers for both the fluid and kinetic component, we hope this approach will prove effective in the future for developing SP timesteppers for the full hybrid model. We hope this will give us the opportunity to incorporate previously inaccessible kinetic effects into the highly effective, modern, finite-element MHD models.
    \end{abstract}
    
    
    \newpage
    \tableofcontents
    
    
    \newpage
    \pagenumbering{arabic}
    %\linenumbers\renewcommand\thelinenumber{\color{black!50}\arabic{linenumber}}
            \input{0 - introduction/main.tex}
        \part{Research}
            \input{1 - low-noise PiC models/main.tex}
            \input{2 - kinetic component/main.tex}
            \input{3 - fluid component/main.tex}
            \input{4 - numerical implementation/main.tex}
        \part{Project Overview}
            \input{5 - research plan/main.tex}
            \input{6 - summary/main.tex}
    
    
    %\section{}
    \newpage
    \pagenumbering{gobble}
        \printbibliography


    \newpage
    \pagenumbering{roman}
    \appendix
        \part{Appendices}
            \input{8 - Hilbert complexes/main.tex}
            \input{9 - weak conservation proofs/main.tex}
\end{document}


\title{\BA{Title in Progress...}}
\author{Boris Andrews}
\affil{Mathematical Institute, University of Oxford}
\date{\today}


\begin{document}
    \pagenumbering{gobble}
    \maketitle
    
    
    \begin{abstract}
        Magnetic confinement reactors---in particular tokamaks---offer one of the most promising options for achieving practical nuclear fusion, with the potential to provide virtually limitless, clean energy. The theoretical and numerical modeling of tokamak plasmas is simultaneously an essential component of effective reactor design, and a great research barrier. Tokamak operational conditions exhibit comparatively low Knudsen numbers. Kinetic effects, including kinetic waves and instabilities, Landau damping, bump-on-tail instabilities and more, are therefore highly influential in tokamak plasma dynamics. Purely fluid models are inherently incapable of capturing these effects, whereas the high dimensionality in purely kinetic models render them practically intractable for most relevant purposes.

        We consider a $\delta\!f$ decomposition model, with a macroscopic fluid background and microscopic kinetic correction, both fully coupled to each other. A similar manner of discretization is proposed to that used in the recent \texttt{STRUPHY} code \cite{Holderied_Possanner_Wang_2021, Holderied_2022, Li_et_al_2023} with a finite-element model for the background and a pseudo-particle/PiC model for the correction.

        The fluid background satisfies the full, non-linear, resistive, compressible, Hall MHD equations. \cite{Laakmann_Hu_Farrell_2022} introduces finite-element(-in-space) implicit timesteppers for the incompressible analogue to this system with structure-preserving (SP) properties in the ideal case, alongside parameter-robust preconditioners. We show that these timesteppers can derive from a finite-element-in-time (FET) (and finite-element-in-space) interpretation. The benefits of this reformulation are discussed, including the derivation of timesteppers that are higher order in time, and the quantifiable dissipative SP properties in the non-ideal, resistive case.
        
        We discuss possible options for extending this FET approach to timesteppers for the compressible case.

        The kinetic corrections satisfy linearized Boltzmann equations. Using a Lénard--Bernstein collision operator, these take Fokker--Planck-like forms \cite{Fokker_1914, Planck_1917} wherein pseudo-particles in the numerical model obey the neoclassical transport equations, with particle-independent Brownian drift terms. This offers a rigorous methodology for incorporating collisions into the particle transport model, without coupling the equations of motions for each particle.
        
        Works by Chen, Chacón et al. \cite{Chen_Chacón_Barnes_2011, Chacón_Chen_Barnes_2013, Chen_Chacón_2014, Chen_Chacón_2015} have developed structure-preserving particle pushers for neoclassical transport in the Vlasov equations, derived from Crank--Nicolson integrators. We show these too can can derive from a FET interpretation, similarly offering potential extensions to higher-order-in-time particle pushers. The FET formulation is used also to consider how the stochastic drift terms can be incorporated into the pushers. Stochastic gyrokinetic expansions are also discussed.

        Different options for the numerical implementation of these schemes are considered.

        Due to the efficacy of FET in the development of SP timesteppers for both the fluid and kinetic component, we hope this approach will prove effective in the future for developing SP timesteppers for the full hybrid model. We hope this will give us the opportunity to incorporate previously inaccessible kinetic effects into the highly effective, modern, finite-element MHD models.
    \end{abstract}
    
    
    \newpage
    \tableofcontents
    
    
    \newpage
    \pagenumbering{arabic}
    %\linenumbers\renewcommand\thelinenumber{\color{black!50}\arabic{linenumber}}
            \documentclass[12pt, a4paper]{report}

\input{template/main.tex}

\title{\BA{Title in Progress...}}
\author{Boris Andrews}
\affil{Mathematical Institute, University of Oxford}
\date{\today}


\begin{document}
    \pagenumbering{gobble}
    \maketitle
    
    
    \begin{abstract}
        Magnetic confinement reactors---in particular tokamaks---offer one of the most promising options for achieving practical nuclear fusion, with the potential to provide virtually limitless, clean energy. The theoretical and numerical modeling of tokamak plasmas is simultaneously an essential component of effective reactor design, and a great research barrier. Tokamak operational conditions exhibit comparatively low Knudsen numbers. Kinetic effects, including kinetic waves and instabilities, Landau damping, bump-on-tail instabilities and more, are therefore highly influential in tokamak plasma dynamics. Purely fluid models are inherently incapable of capturing these effects, whereas the high dimensionality in purely kinetic models render them practically intractable for most relevant purposes.

        We consider a $\delta\!f$ decomposition model, with a macroscopic fluid background and microscopic kinetic correction, both fully coupled to each other. A similar manner of discretization is proposed to that used in the recent \texttt{STRUPHY} code \cite{Holderied_Possanner_Wang_2021, Holderied_2022, Li_et_al_2023} with a finite-element model for the background and a pseudo-particle/PiC model for the correction.

        The fluid background satisfies the full, non-linear, resistive, compressible, Hall MHD equations. \cite{Laakmann_Hu_Farrell_2022} introduces finite-element(-in-space) implicit timesteppers for the incompressible analogue to this system with structure-preserving (SP) properties in the ideal case, alongside parameter-robust preconditioners. We show that these timesteppers can derive from a finite-element-in-time (FET) (and finite-element-in-space) interpretation. The benefits of this reformulation are discussed, including the derivation of timesteppers that are higher order in time, and the quantifiable dissipative SP properties in the non-ideal, resistive case.
        
        We discuss possible options for extending this FET approach to timesteppers for the compressible case.

        The kinetic corrections satisfy linearized Boltzmann equations. Using a Lénard--Bernstein collision operator, these take Fokker--Planck-like forms \cite{Fokker_1914, Planck_1917} wherein pseudo-particles in the numerical model obey the neoclassical transport equations, with particle-independent Brownian drift terms. This offers a rigorous methodology for incorporating collisions into the particle transport model, without coupling the equations of motions for each particle.
        
        Works by Chen, Chacón et al. \cite{Chen_Chacón_Barnes_2011, Chacón_Chen_Barnes_2013, Chen_Chacón_2014, Chen_Chacón_2015} have developed structure-preserving particle pushers for neoclassical transport in the Vlasov equations, derived from Crank--Nicolson integrators. We show these too can can derive from a FET interpretation, similarly offering potential extensions to higher-order-in-time particle pushers. The FET formulation is used also to consider how the stochastic drift terms can be incorporated into the pushers. Stochastic gyrokinetic expansions are also discussed.

        Different options for the numerical implementation of these schemes are considered.

        Due to the efficacy of FET in the development of SP timesteppers for both the fluid and kinetic component, we hope this approach will prove effective in the future for developing SP timesteppers for the full hybrid model. We hope this will give us the opportunity to incorporate previously inaccessible kinetic effects into the highly effective, modern, finite-element MHD models.
    \end{abstract}
    
    
    \newpage
    \tableofcontents
    
    
    \newpage
    \pagenumbering{arabic}
    %\linenumbers\renewcommand\thelinenumber{\color{black!50}\arabic{linenumber}}
            \input{0 - introduction/main.tex}
        \part{Research}
            \input{1 - low-noise PiC models/main.tex}
            \input{2 - kinetic component/main.tex}
            \input{3 - fluid component/main.tex}
            \input{4 - numerical implementation/main.tex}
        \part{Project Overview}
            \input{5 - research plan/main.tex}
            \input{6 - summary/main.tex}
    
    
    %\section{}
    \newpage
    \pagenumbering{gobble}
        \printbibliography


    \newpage
    \pagenumbering{roman}
    \appendix
        \part{Appendices}
            \input{8 - Hilbert complexes/main.tex}
            \input{9 - weak conservation proofs/main.tex}
\end{document}

        \part{Research}
            \documentclass[12pt, a4paper]{report}

\input{template/main.tex}

\title{\BA{Title in Progress...}}
\author{Boris Andrews}
\affil{Mathematical Institute, University of Oxford}
\date{\today}


\begin{document}
    \pagenumbering{gobble}
    \maketitle
    
    
    \begin{abstract}
        Magnetic confinement reactors---in particular tokamaks---offer one of the most promising options for achieving practical nuclear fusion, with the potential to provide virtually limitless, clean energy. The theoretical and numerical modeling of tokamak plasmas is simultaneously an essential component of effective reactor design, and a great research barrier. Tokamak operational conditions exhibit comparatively low Knudsen numbers. Kinetic effects, including kinetic waves and instabilities, Landau damping, bump-on-tail instabilities and more, are therefore highly influential in tokamak plasma dynamics. Purely fluid models are inherently incapable of capturing these effects, whereas the high dimensionality in purely kinetic models render them practically intractable for most relevant purposes.

        We consider a $\delta\!f$ decomposition model, with a macroscopic fluid background and microscopic kinetic correction, both fully coupled to each other. A similar manner of discretization is proposed to that used in the recent \texttt{STRUPHY} code \cite{Holderied_Possanner_Wang_2021, Holderied_2022, Li_et_al_2023} with a finite-element model for the background and a pseudo-particle/PiC model for the correction.

        The fluid background satisfies the full, non-linear, resistive, compressible, Hall MHD equations. \cite{Laakmann_Hu_Farrell_2022} introduces finite-element(-in-space) implicit timesteppers for the incompressible analogue to this system with structure-preserving (SP) properties in the ideal case, alongside parameter-robust preconditioners. We show that these timesteppers can derive from a finite-element-in-time (FET) (and finite-element-in-space) interpretation. The benefits of this reformulation are discussed, including the derivation of timesteppers that are higher order in time, and the quantifiable dissipative SP properties in the non-ideal, resistive case.
        
        We discuss possible options for extending this FET approach to timesteppers for the compressible case.

        The kinetic corrections satisfy linearized Boltzmann equations. Using a Lénard--Bernstein collision operator, these take Fokker--Planck-like forms \cite{Fokker_1914, Planck_1917} wherein pseudo-particles in the numerical model obey the neoclassical transport equations, with particle-independent Brownian drift terms. This offers a rigorous methodology for incorporating collisions into the particle transport model, without coupling the equations of motions for each particle.
        
        Works by Chen, Chacón et al. \cite{Chen_Chacón_Barnes_2011, Chacón_Chen_Barnes_2013, Chen_Chacón_2014, Chen_Chacón_2015} have developed structure-preserving particle pushers for neoclassical transport in the Vlasov equations, derived from Crank--Nicolson integrators. We show these too can can derive from a FET interpretation, similarly offering potential extensions to higher-order-in-time particle pushers. The FET formulation is used also to consider how the stochastic drift terms can be incorporated into the pushers. Stochastic gyrokinetic expansions are also discussed.

        Different options for the numerical implementation of these schemes are considered.

        Due to the efficacy of FET in the development of SP timesteppers for both the fluid and kinetic component, we hope this approach will prove effective in the future for developing SP timesteppers for the full hybrid model. We hope this will give us the opportunity to incorporate previously inaccessible kinetic effects into the highly effective, modern, finite-element MHD models.
    \end{abstract}
    
    
    \newpage
    \tableofcontents
    
    
    \newpage
    \pagenumbering{arabic}
    %\linenumbers\renewcommand\thelinenumber{\color{black!50}\arabic{linenumber}}
            \input{0 - introduction/main.tex}
        \part{Research}
            \input{1 - low-noise PiC models/main.tex}
            \input{2 - kinetic component/main.tex}
            \input{3 - fluid component/main.tex}
            \input{4 - numerical implementation/main.tex}
        \part{Project Overview}
            \input{5 - research plan/main.tex}
            \input{6 - summary/main.tex}
    
    
    %\section{}
    \newpage
    \pagenumbering{gobble}
        \printbibliography


    \newpage
    \pagenumbering{roman}
    \appendix
        \part{Appendices}
            \input{8 - Hilbert complexes/main.tex}
            \input{9 - weak conservation proofs/main.tex}
\end{document}

            \documentclass[12pt, a4paper]{report}

\input{template/main.tex}

\title{\BA{Title in Progress...}}
\author{Boris Andrews}
\affil{Mathematical Institute, University of Oxford}
\date{\today}


\begin{document}
    \pagenumbering{gobble}
    \maketitle
    
    
    \begin{abstract}
        Magnetic confinement reactors---in particular tokamaks---offer one of the most promising options for achieving practical nuclear fusion, with the potential to provide virtually limitless, clean energy. The theoretical and numerical modeling of tokamak plasmas is simultaneously an essential component of effective reactor design, and a great research barrier. Tokamak operational conditions exhibit comparatively low Knudsen numbers. Kinetic effects, including kinetic waves and instabilities, Landau damping, bump-on-tail instabilities and more, are therefore highly influential in tokamak plasma dynamics. Purely fluid models are inherently incapable of capturing these effects, whereas the high dimensionality in purely kinetic models render them practically intractable for most relevant purposes.

        We consider a $\delta\!f$ decomposition model, with a macroscopic fluid background and microscopic kinetic correction, both fully coupled to each other. A similar manner of discretization is proposed to that used in the recent \texttt{STRUPHY} code \cite{Holderied_Possanner_Wang_2021, Holderied_2022, Li_et_al_2023} with a finite-element model for the background and a pseudo-particle/PiC model for the correction.

        The fluid background satisfies the full, non-linear, resistive, compressible, Hall MHD equations. \cite{Laakmann_Hu_Farrell_2022} introduces finite-element(-in-space) implicit timesteppers for the incompressible analogue to this system with structure-preserving (SP) properties in the ideal case, alongside parameter-robust preconditioners. We show that these timesteppers can derive from a finite-element-in-time (FET) (and finite-element-in-space) interpretation. The benefits of this reformulation are discussed, including the derivation of timesteppers that are higher order in time, and the quantifiable dissipative SP properties in the non-ideal, resistive case.
        
        We discuss possible options for extending this FET approach to timesteppers for the compressible case.

        The kinetic corrections satisfy linearized Boltzmann equations. Using a Lénard--Bernstein collision operator, these take Fokker--Planck-like forms \cite{Fokker_1914, Planck_1917} wherein pseudo-particles in the numerical model obey the neoclassical transport equations, with particle-independent Brownian drift terms. This offers a rigorous methodology for incorporating collisions into the particle transport model, without coupling the equations of motions for each particle.
        
        Works by Chen, Chacón et al. \cite{Chen_Chacón_Barnes_2011, Chacón_Chen_Barnes_2013, Chen_Chacón_2014, Chen_Chacón_2015} have developed structure-preserving particle pushers for neoclassical transport in the Vlasov equations, derived from Crank--Nicolson integrators. We show these too can can derive from a FET interpretation, similarly offering potential extensions to higher-order-in-time particle pushers. The FET formulation is used also to consider how the stochastic drift terms can be incorporated into the pushers. Stochastic gyrokinetic expansions are also discussed.

        Different options for the numerical implementation of these schemes are considered.

        Due to the efficacy of FET in the development of SP timesteppers for both the fluid and kinetic component, we hope this approach will prove effective in the future for developing SP timesteppers for the full hybrid model. We hope this will give us the opportunity to incorporate previously inaccessible kinetic effects into the highly effective, modern, finite-element MHD models.
    \end{abstract}
    
    
    \newpage
    \tableofcontents
    
    
    \newpage
    \pagenumbering{arabic}
    %\linenumbers\renewcommand\thelinenumber{\color{black!50}\arabic{linenumber}}
            \input{0 - introduction/main.tex}
        \part{Research}
            \input{1 - low-noise PiC models/main.tex}
            \input{2 - kinetic component/main.tex}
            \input{3 - fluid component/main.tex}
            \input{4 - numerical implementation/main.tex}
        \part{Project Overview}
            \input{5 - research plan/main.tex}
            \input{6 - summary/main.tex}
    
    
    %\section{}
    \newpage
    \pagenumbering{gobble}
        \printbibliography


    \newpage
    \pagenumbering{roman}
    \appendix
        \part{Appendices}
            \input{8 - Hilbert complexes/main.tex}
            \input{9 - weak conservation proofs/main.tex}
\end{document}

            \documentclass[12pt, a4paper]{report}

\input{template/main.tex}

\title{\BA{Title in Progress...}}
\author{Boris Andrews}
\affil{Mathematical Institute, University of Oxford}
\date{\today}


\begin{document}
    \pagenumbering{gobble}
    \maketitle
    
    
    \begin{abstract}
        Magnetic confinement reactors---in particular tokamaks---offer one of the most promising options for achieving practical nuclear fusion, with the potential to provide virtually limitless, clean energy. The theoretical and numerical modeling of tokamak plasmas is simultaneously an essential component of effective reactor design, and a great research barrier. Tokamak operational conditions exhibit comparatively low Knudsen numbers. Kinetic effects, including kinetic waves and instabilities, Landau damping, bump-on-tail instabilities and more, are therefore highly influential in tokamak plasma dynamics. Purely fluid models are inherently incapable of capturing these effects, whereas the high dimensionality in purely kinetic models render them practically intractable for most relevant purposes.

        We consider a $\delta\!f$ decomposition model, with a macroscopic fluid background and microscopic kinetic correction, both fully coupled to each other. A similar manner of discretization is proposed to that used in the recent \texttt{STRUPHY} code \cite{Holderied_Possanner_Wang_2021, Holderied_2022, Li_et_al_2023} with a finite-element model for the background and a pseudo-particle/PiC model for the correction.

        The fluid background satisfies the full, non-linear, resistive, compressible, Hall MHD equations. \cite{Laakmann_Hu_Farrell_2022} introduces finite-element(-in-space) implicit timesteppers for the incompressible analogue to this system with structure-preserving (SP) properties in the ideal case, alongside parameter-robust preconditioners. We show that these timesteppers can derive from a finite-element-in-time (FET) (and finite-element-in-space) interpretation. The benefits of this reformulation are discussed, including the derivation of timesteppers that are higher order in time, and the quantifiable dissipative SP properties in the non-ideal, resistive case.
        
        We discuss possible options for extending this FET approach to timesteppers for the compressible case.

        The kinetic corrections satisfy linearized Boltzmann equations. Using a Lénard--Bernstein collision operator, these take Fokker--Planck-like forms \cite{Fokker_1914, Planck_1917} wherein pseudo-particles in the numerical model obey the neoclassical transport equations, with particle-independent Brownian drift terms. This offers a rigorous methodology for incorporating collisions into the particle transport model, without coupling the equations of motions for each particle.
        
        Works by Chen, Chacón et al. \cite{Chen_Chacón_Barnes_2011, Chacón_Chen_Barnes_2013, Chen_Chacón_2014, Chen_Chacón_2015} have developed structure-preserving particle pushers for neoclassical transport in the Vlasov equations, derived from Crank--Nicolson integrators. We show these too can can derive from a FET interpretation, similarly offering potential extensions to higher-order-in-time particle pushers. The FET formulation is used also to consider how the stochastic drift terms can be incorporated into the pushers. Stochastic gyrokinetic expansions are also discussed.

        Different options for the numerical implementation of these schemes are considered.

        Due to the efficacy of FET in the development of SP timesteppers for both the fluid and kinetic component, we hope this approach will prove effective in the future for developing SP timesteppers for the full hybrid model. We hope this will give us the opportunity to incorporate previously inaccessible kinetic effects into the highly effective, modern, finite-element MHD models.
    \end{abstract}
    
    
    \newpage
    \tableofcontents
    
    
    \newpage
    \pagenumbering{arabic}
    %\linenumbers\renewcommand\thelinenumber{\color{black!50}\arabic{linenumber}}
            \input{0 - introduction/main.tex}
        \part{Research}
            \input{1 - low-noise PiC models/main.tex}
            \input{2 - kinetic component/main.tex}
            \input{3 - fluid component/main.tex}
            \input{4 - numerical implementation/main.tex}
        \part{Project Overview}
            \input{5 - research plan/main.tex}
            \input{6 - summary/main.tex}
    
    
    %\section{}
    \newpage
    \pagenumbering{gobble}
        \printbibliography


    \newpage
    \pagenumbering{roman}
    \appendix
        \part{Appendices}
            \input{8 - Hilbert complexes/main.tex}
            \input{9 - weak conservation proofs/main.tex}
\end{document}

            \documentclass[12pt, a4paper]{report}

\input{template/main.tex}

\title{\BA{Title in Progress...}}
\author{Boris Andrews}
\affil{Mathematical Institute, University of Oxford}
\date{\today}


\begin{document}
    \pagenumbering{gobble}
    \maketitle
    
    
    \begin{abstract}
        Magnetic confinement reactors---in particular tokamaks---offer one of the most promising options for achieving practical nuclear fusion, with the potential to provide virtually limitless, clean energy. The theoretical and numerical modeling of tokamak plasmas is simultaneously an essential component of effective reactor design, and a great research barrier. Tokamak operational conditions exhibit comparatively low Knudsen numbers. Kinetic effects, including kinetic waves and instabilities, Landau damping, bump-on-tail instabilities and more, are therefore highly influential in tokamak plasma dynamics. Purely fluid models are inherently incapable of capturing these effects, whereas the high dimensionality in purely kinetic models render them practically intractable for most relevant purposes.

        We consider a $\delta\!f$ decomposition model, with a macroscopic fluid background and microscopic kinetic correction, both fully coupled to each other. A similar manner of discretization is proposed to that used in the recent \texttt{STRUPHY} code \cite{Holderied_Possanner_Wang_2021, Holderied_2022, Li_et_al_2023} with a finite-element model for the background and a pseudo-particle/PiC model for the correction.

        The fluid background satisfies the full, non-linear, resistive, compressible, Hall MHD equations. \cite{Laakmann_Hu_Farrell_2022} introduces finite-element(-in-space) implicit timesteppers for the incompressible analogue to this system with structure-preserving (SP) properties in the ideal case, alongside parameter-robust preconditioners. We show that these timesteppers can derive from a finite-element-in-time (FET) (and finite-element-in-space) interpretation. The benefits of this reformulation are discussed, including the derivation of timesteppers that are higher order in time, and the quantifiable dissipative SP properties in the non-ideal, resistive case.
        
        We discuss possible options for extending this FET approach to timesteppers for the compressible case.

        The kinetic corrections satisfy linearized Boltzmann equations. Using a Lénard--Bernstein collision operator, these take Fokker--Planck-like forms \cite{Fokker_1914, Planck_1917} wherein pseudo-particles in the numerical model obey the neoclassical transport equations, with particle-independent Brownian drift terms. This offers a rigorous methodology for incorporating collisions into the particle transport model, without coupling the equations of motions for each particle.
        
        Works by Chen, Chacón et al. \cite{Chen_Chacón_Barnes_2011, Chacón_Chen_Barnes_2013, Chen_Chacón_2014, Chen_Chacón_2015} have developed structure-preserving particle pushers for neoclassical transport in the Vlasov equations, derived from Crank--Nicolson integrators. We show these too can can derive from a FET interpretation, similarly offering potential extensions to higher-order-in-time particle pushers. The FET formulation is used also to consider how the stochastic drift terms can be incorporated into the pushers. Stochastic gyrokinetic expansions are also discussed.

        Different options for the numerical implementation of these schemes are considered.

        Due to the efficacy of FET in the development of SP timesteppers for both the fluid and kinetic component, we hope this approach will prove effective in the future for developing SP timesteppers for the full hybrid model. We hope this will give us the opportunity to incorporate previously inaccessible kinetic effects into the highly effective, modern, finite-element MHD models.
    \end{abstract}
    
    
    \newpage
    \tableofcontents
    
    
    \newpage
    \pagenumbering{arabic}
    %\linenumbers\renewcommand\thelinenumber{\color{black!50}\arabic{linenumber}}
            \input{0 - introduction/main.tex}
        \part{Research}
            \input{1 - low-noise PiC models/main.tex}
            \input{2 - kinetic component/main.tex}
            \input{3 - fluid component/main.tex}
            \input{4 - numerical implementation/main.tex}
        \part{Project Overview}
            \input{5 - research plan/main.tex}
            \input{6 - summary/main.tex}
    
    
    %\section{}
    \newpage
    \pagenumbering{gobble}
        \printbibliography


    \newpage
    \pagenumbering{roman}
    \appendix
        \part{Appendices}
            \input{8 - Hilbert complexes/main.tex}
            \input{9 - weak conservation proofs/main.tex}
\end{document}

        \part{Project Overview}
            \documentclass[12pt, a4paper]{report}

\input{template/main.tex}

\title{\BA{Title in Progress...}}
\author{Boris Andrews}
\affil{Mathematical Institute, University of Oxford}
\date{\today}


\begin{document}
    \pagenumbering{gobble}
    \maketitle
    
    
    \begin{abstract}
        Magnetic confinement reactors---in particular tokamaks---offer one of the most promising options for achieving practical nuclear fusion, with the potential to provide virtually limitless, clean energy. The theoretical and numerical modeling of tokamak plasmas is simultaneously an essential component of effective reactor design, and a great research barrier. Tokamak operational conditions exhibit comparatively low Knudsen numbers. Kinetic effects, including kinetic waves and instabilities, Landau damping, bump-on-tail instabilities and more, are therefore highly influential in tokamak plasma dynamics. Purely fluid models are inherently incapable of capturing these effects, whereas the high dimensionality in purely kinetic models render them practically intractable for most relevant purposes.

        We consider a $\delta\!f$ decomposition model, with a macroscopic fluid background and microscopic kinetic correction, both fully coupled to each other. A similar manner of discretization is proposed to that used in the recent \texttt{STRUPHY} code \cite{Holderied_Possanner_Wang_2021, Holderied_2022, Li_et_al_2023} with a finite-element model for the background and a pseudo-particle/PiC model for the correction.

        The fluid background satisfies the full, non-linear, resistive, compressible, Hall MHD equations. \cite{Laakmann_Hu_Farrell_2022} introduces finite-element(-in-space) implicit timesteppers for the incompressible analogue to this system with structure-preserving (SP) properties in the ideal case, alongside parameter-robust preconditioners. We show that these timesteppers can derive from a finite-element-in-time (FET) (and finite-element-in-space) interpretation. The benefits of this reformulation are discussed, including the derivation of timesteppers that are higher order in time, and the quantifiable dissipative SP properties in the non-ideal, resistive case.
        
        We discuss possible options for extending this FET approach to timesteppers for the compressible case.

        The kinetic corrections satisfy linearized Boltzmann equations. Using a Lénard--Bernstein collision operator, these take Fokker--Planck-like forms \cite{Fokker_1914, Planck_1917} wherein pseudo-particles in the numerical model obey the neoclassical transport equations, with particle-independent Brownian drift terms. This offers a rigorous methodology for incorporating collisions into the particle transport model, without coupling the equations of motions for each particle.
        
        Works by Chen, Chacón et al. \cite{Chen_Chacón_Barnes_2011, Chacón_Chen_Barnes_2013, Chen_Chacón_2014, Chen_Chacón_2015} have developed structure-preserving particle pushers for neoclassical transport in the Vlasov equations, derived from Crank--Nicolson integrators. We show these too can can derive from a FET interpretation, similarly offering potential extensions to higher-order-in-time particle pushers. The FET formulation is used also to consider how the stochastic drift terms can be incorporated into the pushers. Stochastic gyrokinetic expansions are also discussed.

        Different options for the numerical implementation of these schemes are considered.

        Due to the efficacy of FET in the development of SP timesteppers for both the fluid and kinetic component, we hope this approach will prove effective in the future for developing SP timesteppers for the full hybrid model. We hope this will give us the opportunity to incorporate previously inaccessible kinetic effects into the highly effective, modern, finite-element MHD models.
    \end{abstract}
    
    
    \newpage
    \tableofcontents
    
    
    \newpage
    \pagenumbering{arabic}
    %\linenumbers\renewcommand\thelinenumber{\color{black!50}\arabic{linenumber}}
            \input{0 - introduction/main.tex}
        \part{Research}
            \input{1 - low-noise PiC models/main.tex}
            \input{2 - kinetic component/main.tex}
            \input{3 - fluid component/main.tex}
            \input{4 - numerical implementation/main.tex}
        \part{Project Overview}
            \input{5 - research plan/main.tex}
            \input{6 - summary/main.tex}
    
    
    %\section{}
    \newpage
    \pagenumbering{gobble}
        \printbibliography


    \newpage
    \pagenumbering{roman}
    \appendix
        \part{Appendices}
            \input{8 - Hilbert complexes/main.tex}
            \input{9 - weak conservation proofs/main.tex}
\end{document}

            \documentclass[12pt, a4paper]{report}

\input{template/main.tex}

\title{\BA{Title in Progress...}}
\author{Boris Andrews}
\affil{Mathematical Institute, University of Oxford}
\date{\today}


\begin{document}
    \pagenumbering{gobble}
    \maketitle
    
    
    \begin{abstract}
        Magnetic confinement reactors---in particular tokamaks---offer one of the most promising options for achieving practical nuclear fusion, with the potential to provide virtually limitless, clean energy. The theoretical and numerical modeling of tokamak plasmas is simultaneously an essential component of effective reactor design, and a great research barrier. Tokamak operational conditions exhibit comparatively low Knudsen numbers. Kinetic effects, including kinetic waves and instabilities, Landau damping, bump-on-tail instabilities and more, are therefore highly influential in tokamak plasma dynamics. Purely fluid models are inherently incapable of capturing these effects, whereas the high dimensionality in purely kinetic models render them practically intractable for most relevant purposes.

        We consider a $\delta\!f$ decomposition model, with a macroscopic fluid background and microscopic kinetic correction, both fully coupled to each other. A similar manner of discretization is proposed to that used in the recent \texttt{STRUPHY} code \cite{Holderied_Possanner_Wang_2021, Holderied_2022, Li_et_al_2023} with a finite-element model for the background and a pseudo-particle/PiC model for the correction.

        The fluid background satisfies the full, non-linear, resistive, compressible, Hall MHD equations. \cite{Laakmann_Hu_Farrell_2022} introduces finite-element(-in-space) implicit timesteppers for the incompressible analogue to this system with structure-preserving (SP) properties in the ideal case, alongside parameter-robust preconditioners. We show that these timesteppers can derive from a finite-element-in-time (FET) (and finite-element-in-space) interpretation. The benefits of this reformulation are discussed, including the derivation of timesteppers that are higher order in time, and the quantifiable dissipative SP properties in the non-ideal, resistive case.
        
        We discuss possible options for extending this FET approach to timesteppers for the compressible case.

        The kinetic corrections satisfy linearized Boltzmann equations. Using a Lénard--Bernstein collision operator, these take Fokker--Planck-like forms \cite{Fokker_1914, Planck_1917} wherein pseudo-particles in the numerical model obey the neoclassical transport equations, with particle-independent Brownian drift terms. This offers a rigorous methodology for incorporating collisions into the particle transport model, without coupling the equations of motions for each particle.
        
        Works by Chen, Chacón et al. \cite{Chen_Chacón_Barnes_2011, Chacón_Chen_Barnes_2013, Chen_Chacón_2014, Chen_Chacón_2015} have developed structure-preserving particle pushers for neoclassical transport in the Vlasov equations, derived from Crank--Nicolson integrators. We show these too can can derive from a FET interpretation, similarly offering potential extensions to higher-order-in-time particle pushers. The FET formulation is used also to consider how the stochastic drift terms can be incorporated into the pushers. Stochastic gyrokinetic expansions are also discussed.

        Different options for the numerical implementation of these schemes are considered.

        Due to the efficacy of FET in the development of SP timesteppers for both the fluid and kinetic component, we hope this approach will prove effective in the future for developing SP timesteppers for the full hybrid model. We hope this will give us the opportunity to incorporate previously inaccessible kinetic effects into the highly effective, modern, finite-element MHD models.
    \end{abstract}
    
    
    \newpage
    \tableofcontents
    
    
    \newpage
    \pagenumbering{arabic}
    %\linenumbers\renewcommand\thelinenumber{\color{black!50}\arabic{linenumber}}
            \input{0 - introduction/main.tex}
        \part{Research}
            \input{1 - low-noise PiC models/main.tex}
            \input{2 - kinetic component/main.tex}
            \input{3 - fluid component/main.tex}
            \input{4 - numerical implementation/main.tex}
        \part{Project Overview}
            \input{5 - research plan/main.tex}
            \input{6 - summary/main.tex}
    
    
    %\section{}
    \newpage
    \pagenumbering{gobble}
        \printbibliography


    \newpage
    \pagenumbering{roman}
    \appendix
        \part{Appendices}
            \input{8 - Hilbert complexes/main.tex}
            \input{9 - weak conservation proofs/main.tex}
\end{document}

    
    
    %\section{}
    \newpage
    \pagenumbering{gobble}
        \printbibliography


    \newpage
    \pagenumbering{roman}
    \appendix
        \part{Appendices}
            \documentclass[12pt, a4paper]{report}

\input{template/main.tex}

\title{\BA{Title in Progress...}}
\author{Boris Andrews}
\affil{Mathematical Institute, University of Oxford}
\date{\today}


\begin{document}
    \pagenumbering{gobble}
    \maketitle
    
    
    \begin{abstract}
        Magnetic confinement reactors---in particular tokamaks---offer one of the most promising options for achieving practical nuclear fusion, with the potential to provide virtually limitless, clean energy. The theoretical and numerical modeling of tokamak plasmas is simultaneously an essential component of effective reactor design, and a great research barrier. Tokamak operational conditions exhibit comparatively low Knudsen numbers. Kinetic effects, including kinetic waves and instabilities, Landau damping, bump-on-tail instabilities and more, are therefore highly influential in tokamak plasma dynamics. Purely fluid models are inherently incapable of capturing these effects, whereas the high dimensionality in purely kinetic models render them practically intractable for most relevant purposes.

        We consider a $\delta\!f$ decomposition model, with a macroscopic fluid background and microscopic kinetic correction, both fully coupled to each other. A similar manner of discretization is proposed to that used in the recent \texttt{STRUPHY} code \cite{Holderied_Possanner_Wang_2021, Holderied_2022, Li_et_al_2023} with a finite-element model for the background and a pseudo-particle/PiC model for the correction.

        The fluid background satisfies the full, non-linear, resistive, compressible, Hall MHD equations. \cite{Laakmann_Hu_Farrell_2022} introduces finite-element(-in-space) implicit timesteppers for the incompressible analogue to this system with structure-preserving (SP) properties in the ideal case, alongside parameter-robust preconditioners. We show that these timesteppers can derive from a finite-element-in-time (FET) (and finite-element-in-space) interpretation. The benefits of this reformulation are discussed, including the derivation of timesteppers that are higher order in time, and the quantifiable dissipative SP properties in the non-ideal, resistive case.
        
        We discuss possible options for extending this FET approach to timesteppers for the compressible case.

        The kinetic corrections satisfy linearized Boltzmann equations. Using a Lénard--Bernstein collision operator, these take Fokker--Planck-like forms \cite{Fokker_1914, Planck_1917} wherein pseudo-particles in the numerical model obey the neoclassical transport equations, with particle-independent Brownian drift terms. This offers a rigorous methodology for incorporating collisions into the particle transport model, without coupling the equations of motions for each particle.
        
        Works by Chen, Chacón et al. \cite{Chen_Chacón_Barnes_2011, Chacón_Chen_Barnes_2013, Chen_Chacón_2014, Chen_Chacón_2015} have developed structure-preserving particle pushers for neoclassical transport in the Vlasov equations, derived from Crank--Nicolson integrators. We show these too can can derive from a FET interpretation, similarly offering potential extensions to higher-order-in-time particle pushers. The FET formulation is used also to consider how the stochastic drift terms can be incorporated into the pushers. Stochastic gyrokinetic expansions are also discussed.

        Different options for the numerical implementation of these schemes are considered.

        Due to the efficacy of FET in the development of SP timesteppers for both the fluid and kinetic component, we hope this approach will prove effective in the future for developing SP timesteppers for the full hybrid model. We hope this will give us the opportunity to incorporate previously inaccessible kinetic effects into the highly effective, modern, finite-element MHD models.
    \end{abstract}
    
    
    \newpage
    \tableofcontents
    
    
    \newpage
    \pagenumbering{arabic}
    %\linenumbers\renewcommand\thelinenumber{\color{black!50}\arabic{linenumber}}
            \input{0 - introduction/main.tex}
        \part{Research}
            \input{1 - low-noise PiC models/main.tex}
            \input{2 - kinetic component/main.tex}
            \input{3 - fluid component/main.tex}
            \input{4 - numerical implementation/main.tex}
        \part{Project Overview}
            \input{5 - research plan/main.tex}
            \input{6 - summary/main.tex}
    
    
    %\section{}
    \newpage
    \pagenumbering{gobble}
        \printbibliography


    \newpage
    \pagenumbering{roman}
    \appendix
        \part{Appendices}
            \input{8 - Hilbert complexes/main.tex}
            \input{9 - weak conservation proofs/main.tex}
\end{document}

            \documentclass[12pt, a4paper]{report}

\input{template/main.tex}

\title{\BA{Title in Progress...}}
\author{Boris Andrews}
\affil{Mathematical Institute, University of Oxford}
\date{\today}


\begin{document}
    \pagenumbering{gobble}
    \maketitle
    
    
    \begin{abstract}
        Magnetic confinement reactors---in particular tokamaks---offer one of the most promising options for achieving practical nuclear fusion, with the potential to provide virtually limitless, clean energy. The theoretical and numerical modeling of tokamak plasmas is simultaneously an essential component of effective reactor design, and a great research barrier. Tokamak operational conditions exhibit comparatively low Knudsen numbers. Kinetic effects, including kinetic waves and instabilities, Landau damping, bump-on-tail instabilities and more, are therefore highly influential in tokamak plasma dynamics. Purely fluid models are inherently incapable of capturing these effects, whereas the high dimensionality in purely kinetic models render them practically intractable for most relevant purposes.

        We consider a $\delta\!f$ decomposition model, with a macroscopic fluid background and microscopic kinetic correction, both fully coupled to each other. A similar manner of discretization is proposed to that used in the recent \texttt{STRUPHY} code \cite{Holderied_Possanner_Wang_2021, Holderied_2022, Li_et_al_2023} with a finite-element model for the background and a pseudo-particle/PiC model for the correction.

        The fluid background satisfies the full, non-linear, resistive, compressible, Hall MHD equations. \cite{Laakmann_Hu_Farrell_2022} introduces finite-element(-in-space) implicit timesteppers for the incompressible analogue to this system with structure-preserving (SP) properties in the ideal case, alongside parameter-robust preconditioners. We show that these timesteppers can derive from a finite-element-in-time (FET) (and finite-element-in-space) interpretation. The benefits of this reformulation are discussed, including the derivation of timesteppers that are higher order in time, and the quantifiable dissipative SP properties in the non-ideal, resistive case.
        
        We discuss possible options for extending this FET approach to timesteppers for the compressible case.

        The kinetic corrections satisfy linearized Boltzmann equations. Using a Lénard--Bernstein collision operator, these take Fokker--Planck-like forms \cite{Fokker_1914, Planck_1917} wherein pseudo-particles in the numerical model obey the neoclassical transport equations, with particle-independent Brownian drift terms. This offers a rigorous methodology for incorporating collisions into the particle transport model, without coupling the equations of motions for each particle.
        
        Works by Chen, Chacón et al. \cite{Chen_Chacón_Barnes_2011, Chacón_Chen_Barnes_2013, Chen_Chacón_2014, Chen_Chacón_2015} have developed structure-preserving particle pushers for neoclassical transport in the Vlasov equations, derived from Crank--Nicolson integrators. We show these too can can derive from a FET interpretation, similarly offering potential extensions to higher-order-in-time particle pushers. The FET formulation is used also to consider how the stochastic drift terms can be incorporated into the pushers. Stochastic gyrokinetic expansions are also discussed.

        Different options for the numerical implementation of these schemes are considered.

        Due to the efficacy of FET in the development of SP timesteppers for both the fluid and kinetic component, we hope this approach will prove effective in the future for developing SP timesteppers for the full hybrid model. We hope this will give us the opportunity to incorporate previously inaccessible kinetic effects into the highly effective, modern, finite-element MHD models.
    \end{abstract}
    
    
    \newpage
    \tableofcontents
    
    
    \newpage
    \pagenumbering{arabic}
    %\linenumbers\renewcommand\thelinenumber{\color{black!50}\arabic{linenumber}}
            \input{0 - introduction/main.tex}
        \part{Research}
            \input{1 - low-noise PiC models/main.tex}
            \input{2 - kinetic component/main.tex}
            \input{3 - fluid component/main.tex}
            \input{4 - numerical implementation/main.tex}
        \part{Project Overview}
            \input{5 - research plan/main.tex}
            \input{6 - summary/main.tex}
    
    
    %\section{}
    \newpage
    \pagenumbering{gobble}
        \printbibliography


    \newpage
    \pagenumbering{roman}
    \appendix
        \part{Appendices}
            \input{8 - Hilbert complexes/main.tex}
            \input{9 - weak conservation proofs/main.tex}
\end{document}

\end{document}

            \documentclass[12pt, a4paper]{report}

\documentclass[12pt, a4paper]{report}

\input{template/main.tex}

\title{\BA{Title in Progress...}}
\author{Boris Andrews}
\affil{Mathematical Institute, University of Oxford}
\date{\today}


\begin{document}
    \pagenumbering{gobble}
    \maketitle
    
    
    \begin{abstract}
        Magnetic confinement reactors---in particular tokamaks---offer one of the most promising options for achieving practical nuclear fusion, with the potential to provide virtually limitless, clean energy. The theoretical and numerical modeling of tokamak plasmas is simultaneously an essential component of effective reactor design, and a great research barrier. Tokamak operational conditions exhibit comparatively low Knudsen numbers. Kinetic effects, including kinetic waves and instabilities, Landau damping, bump-on-tail instabilities and more, are therefore highly influential in tokamak plasma dynamics. Purely fluid models are inherently incapable of capturing these effects, whereas the high dimensionality in purely kinetic models render them practically intractable for most relevant purposes.

        We consider a $\delta\!f$ decomposition model, with a macroscopic fluid background and microscopic kinetic correction, both fully coupled to each other. A similar manner of discretization is proposed to that used in the recent \texttt{STRUPHY} code \cite{Holderied_Possanner_Wang_2021, Holderied_2022, Li_et_al_2023} with a finite-element model for the background and a pseudo-particle/PiC model for the correction.

        The fluid background satisfies the full, non-linear, resistive, compressible, Hall MHD equations. \cite{Laakmann_Hu_Farrell_2022} introduces finite-element(-in-space) implicit timesteppers for the incompressible analogue to this system with structure-preserving (SP) properties in the ideal case, alongside parameter-robust preconditioners. We show that these timesteppers can derive from a finite-element-in-time (FET) (and finite-element-in-space) interpretation. The benefits of this reformulation are discussed, including the derivation of timesteppers that are higher order in time, and the quantifiable dissipative SP properties in the non-ideal, resistive case.
        
        We discuss possible options for extending this FET approach to timesteppers for the compressible case.

        The kinetic corrections satisfy linearized Boltzmann equations. Using a Lénard--Bernstein collision operator, these take Fokker--Planck-like forms \cite{Fokker_1914, Planck_1917} wherein pseudo-particles in the numerical model obey the neoclassical transport equations, with particle-independent Brownian drift terms. This offers a rigorous methodology for incorporating collisions into the particle transport model, without coupling the equations of motions for each particle.
        
        Works by Chen, Chacón et al. \cite{Chen_Chacón_Barnes_2011, Chacón_Chen_Barnes_2013, Chen_Chacón_2014, Chen_Chacón_2015} have developed structure-preserving particle pushers for neoclassical transport in the Vlasov equations, derived from Crank--Nicolson integrators. We show these too can can derive from a FET interpretation, similarly offering potential extensions to higher-order-in-time particle pushers. The FET formulation is used also to consider how the stochastic drift terms can be incorporated into the pushers. Stochastic gyrokinetic expansions are also discussed.

        Different options for the numerical implementation of these schemes are considered.

        Due to the efficacy of FET in the development of SP timesteppers for both the fluid and kinetic component, we hope this approach will prove effective in the future for developing SP timesteppers for the full hybrid model. We hope this will give us the opportunity to incorporate previously inaccessible kinetic effects into the highly effective, modern, finite-element MHD models.
    \end{abstract}
    
    
    \newpage
    \tableofcontents
    
    
    \newpage
    \pagenumbering{arabic}
    %\linenumbers\renewcommand\thelinenumber{\color{black!50}\arabic{linenumber}}
            \input{0 - introduction/main.tex}
        \part{Research}
            \input{1 - low-noise PiC models/main.tex}
            \input{2 - kinetic component/main.tex}
            \input{3 - fluid component/main.tex}
            \input{4 - numerical implementation/main.tex}
        \part{Project Overview}
            \input{5 - research plan/main.tex}
            \input{6 - summary/main.tex}
    
    
    %\section{}
    \newpage
    \pagenumbering{gobble}
        \printbibliography


    \newpage
    \pagenumbering{roman}
    \appendix
        \part{Appendices}
            \input{8 - Hilbert complexes/main.tex}
            \input{9 - weak conservation proofs/main.tex}
\end{document}


\title{\BA{Title in Progress...}}
\author{Boris Andrews}
\affil{Mathematical Institute, University of Oxford}
\date{\today}


\begin{document}
    \pagenumbering{gobble}
    \maketitle
    
    
    \begin{abstract}
        Magnetic confinement reactors---in particular tokamaks---offer one of the most promising options for achieving practical nuclear fusion, with the potential to provide virtually limitless, clean energy. The theoretical and numerical modeling of tokamak plasmas is simultaneously an essential component of effective reactor design, and a great research barrier. Tokamak operational conditions exhibit comparatively low Knudsen numbers. Kinetic effects, including kinetic waves and instabilities, Landau damping, bump-on-tail instabilities and more, are therefore highly influential in tokamak plasma dynamics. Purely fluid models are inherently incapable of capturing these effects, whereas the high dimensionality in purely kinetic models render them practically intractable for most relevant purposes.

        We consider a $\delta\!f$ decomposition model, with a macroscopic fluid background and microscopic kinetic correction, both fully coupled to each other. A similar manner of discretization is proposed to that used in the recent \texttt{STRUPHY} code \cite{Holderied_Possanner_Wang_2021, Holderied_2022, Li_et_al_2023} with a finite-element model for the background and a pseudo-particle/PiC model for the correction.

        The fluid background satisfies the full, non-linear, resistive, compressible, Hall MHD equations. \cite{Laakmann_Hu_Farrell_2022} introduces finite-element(-in-space) implicit timesteppers for the incompressible analogue to this system with structure-preserving (SP) properties in the ideal case, alongside parameter-robust preconditioners. We show that these timesteppers can derive from a finite-element-in-time (FET) (and finite-element-in-space) interpretation. The benefits of this reformulation are discussed, including the derivation of timesteppers that are higher order in time, and the quantifiable dissipative SP properties in the non-ideal, resistive case.
        
        We discuss possible options for extending this FET approach to timesteppers for the compressible case.

        The kinetic corrections satisfy linearized Boltzmann equations. Using a Lénard--Bernstein collision operator, these take Fokker--Planck-like forms \cite{Fokker_1914, Planck_1917} wherein pseudo-particles in the numerical model obey the neoclassical transport equations, with particle-independent Brownian drift terms. This offers a rigorous methodology for incorporating collisions into the particle transport model, without coupling the equations of motions for each particle.
        
        Works by Chen, Chacón et al. \cite{Chen_Chacón_Barnes_2011, Chacón_Chen_Barnes_2013, Chen_Chacón_2014, Chen_Chacón_2015} have developed structure-preserving particle pushers for neoclassical transport in the Vlasov equations, derived from Crank--Nicolson integrators. We show these too can can derive from a FET interpretation, similarly offering potential extensions to higher-order-in-time particle pushers. The FET formulation is used also to consider how the stochastic drift terms can be incorporated into the pushers. Stochastic gyrokinetic expansions are also discussed.

        Different options for the numerical implementation of these schemes are considered.

        Due to the efficacy of FET in the development of SP timesteppers for both the fluid and kinetic component, we hope this approach will prove effective in the future for developing SP timesteppers for the full hybrid model. We hope this will give us the opportunity to incorporate previously inaccessible kinetic effects into the highly effective, modern, finite-element MHD models.
    \end{abstract}
    
    
    \newpage
    \tableofcontents
    
    
    \newpage
    \pagenumbering{arabic}
    %\linenumbers\renewcommand\thelinenumber{\color{black!50}\arabic{linenumber}}
            \documentclass[12pt, a4paper]{report}

\input{template/main.tex}

\title{\BA{Title in Progress...}}
\author{Boris Andrews}
\affil{Mathematical Institute, University of Oxford}
\date{\today}


\begin{document}
    \pagenumbering{gobble}
    \maketitle
    
    
    \begin{abstract}
        Magnetic confinement reactors---in particular tokamaks---offer one of the most promising options for achieving practical nuclear fusion, with the potential to provide virtually limitless, clean energy. The theoretical and numerical modeling of tokamak plasmas is simultaneously an essential component of effective reactor design, and a great research barrier. Tokamak operational conditions exhibit comparatively low Knudsen numbers. Kinetic effects, including kinetic waves and instabilities, Landau damping, bump-on-tail instabilities and more, are therefore highly influential in tokamak plasma dynamics. Purely fluid models are inherently incapable of capturing these effects, whereas the high dimensionality in purely kinetic models render them practically intractable for most relevant purposes.

        We consider a $\delta\!f$ decomposition model, with a macroscopic fluid background and microscopic kinetic correction, both fully coupled to each other. A similar manner of discretization is proposed to that used in the recent \texttt{STRUPHY} code \cite{Holderied_Possanner_Wang_2021, Holderied_2022, Li_et_al_2023} with a finite-element model for the background and a pseudo-particle/PiC model for the correction.

        The fluid background satisfies the full, non-linear, resistive, compressible, Hall MHD equations. \cite{Laakmann_Hu_Farrell_2022} introduces finite-element(-in-space) implicit timesteppers for the incompressible analogue to this system with structure-preserving (SP) properties in the ideal case, alongside parameter-robust preconditioners. We show that these timesteppers can derive from a finite-element-in-time (FET) (and finite-element-in-space) interpretation. The benefits of this reformulation are discussed, including the derivation of timesteppers that are higher order in time, and the quantifiable dissipative SP properties in the non-ideal, resistive case.
        
        We discuss possible options for extending this FET approach to timesteppers for the compressible case.

        The kinetic corrections satisfy linearized Boltzmann equations. Using a Lénard--Bernstein collision operator, these take Fokker--Planck-like forms \cite{Fokker_1914, Planck_1917} wherein pseudo-particles in the numerical model obey the neoclassical transport equations, with particle-independent Brownian drift terms. This offers a rigorous methodology for incorporating collisions into the particle transport model, without coupling the equations of motions for each particle.
        
        Works by Chen, Chacón et al. \cite{Chen_Chacón_Barnes_2011, Chacón_Chen_Barnes_2013, Chen_Chacón_2014, Chen_Chacón_2015} have developed structure-preserving particle pushers for neoclassical transport in the Vlasov equations, derived from Crank--Nicolson integrators. We show these too can can derive from a FET interpretation, similarly offering potential extensions to higher-order-in-time particle pushers. The FET formulation is used also to consider how the stochastic drift terms can be incorporated into the pushers. Stochastic gyrokinetic expansions are also discussed.

        Different options for the numerical implementation of these schemes are considered.

        Due to the efficacy of FET in the development of SP timesteppers for both the fluid and kinetic component, we hope this approach will prove effective in the future for developing SP timesteppers for the full hybrid model. We hope this will give us the opportunity to incorporate previously inaccessible kinetic effects into the highly effective, modern, finite-element MHD models.
    \end{abstract}
    
    
    \newpage
    \tableofcontents
    
    
    \newpage
    \pagenumbering{arabic}
    %\linenumbers\renewcommand\thelinenumber{\color{black!50}\arabic{linenumber}}
            \input{0 - introduction/main.tex}
        \part{Research}
            \input{1 - low-noise PiC models/main.tex}
            \input{2 - kinetic component/main.tex}
            \input{3 - fluid component/main.tex}
            \input{4 - numerical implementation/main.tex}
        \part{Project Overview}
            \input{5 - research plan/main.tex}
            \input{6 - summary/main.tex}
    
    
    %\section{}
    \newpage
    \pagenumbering{gobble}
        \printbibliography


    \newpage
    \pagenumbering{roman}
    \appendix
        \part{Appendices}
            \input{8 - Hilbert complexes/main.tex}
            \input{9 - weak conservation proofs/main.tex}
\end{document}

        \part{Research}
            \documentclass[12pt, a4paper]{report}

\input{template/main.tex}

\title{\BA{Title in Progress...}}
\author{Boris Andrews}
\affil{Mathematical Institute, University of Oxford}
\date{\today}


\begin{document}
    \pagenumbering{gobble}
    \maketitle
    
    
    \begin{abstract}
        Magnetic confinement reactors---in particular tokamaks---offer one of the most promising options for achieving practical nuclear fusion, with the potential to provide virtually limitless, clean energy. The theoretical and numerical modeling of tokamak plasmas is simultaneously an essential component of effective reactor design, and a great research barrier. Tokamak operational conditions exhibit comparatively low Knudsen numbers. Kinetic effects, including kinetic waves and instabilities, Landau damping, bump-on-tail instabilities and more, are therefore highly influential in tokamak plasma dynamics. Purely fluid models are inherently incapable of capturing these effects, whereas the high dimensionality in purely kinetic models render them practically intractable for most relevant purposes.

        We consider a $\delta\!f$ decomposition model, with a macroscopic fluid background and microscopic kinetic correction, both fully coupled to each other. A similar manner of discretization is proposed to that used in the recent \texttt{STRUPHY} code \cite{Holderied_Possanner_Wang_2021, Holderied_2022, Li_et_al_2023} with a finite-element model for the background and a pseudo-particle/PiC model for the correction.

        The fluid background satisfies the full, non-linear, resistive, compressible, Hall MHD equations. \cite{Laakmann_Hu_Farrell_2022} introduces finite-element(-in-space) implicit timesteppers for the incompressible analogue to this system with structure-preserving (SP) properties in the ideal case, alongside parameter-robust preconditioners. We show that these timesteppers can derive from a finite-element-in-time (FET) (and finite-element-in-space) interpretation. The benefits of this reformulation are discussed, including the derivation of timesteppers that are higher order in time, and the quantifiable dissipative SP properties in the non-ideal, resistive case.
        
        We discuss possible options for extending this FET approach to timesteppers for the compressible case.

        The kinetic corrections satisfy linearized Boltzmann equations. Using a Lénard--Bernstein collision operator, these take Fokker--Planck-like forms \cite{Fokker_1914, Planck_1917} wherein pseudo-particles in the numerical model obey the neoclassical transport equations, with particle-independent Brownian drift terms. This offers a rigorous methodology for incorporating collisions into the particle transport model, without coupling the equations of motions for each particle.
        
        Works by Chen, Chacón et al. \cite{Chen_Chacón_Barnes_2011, Chacón_Chen_Barnes_2013, Chen_Chacón_2014, Chen_Chacón_2015} have developed structure-preserving particle pushers for neoclassical transport in the Vlasov equations, derived from Crank--Nicolson integrators. We show these too can can derive from a FET interpretation, similarly offering potential extensions to higher-order-in-time particle pushers. The FET formulation is used also to consider how the stochastic drift terms can be incorporated into the pushers. Stochastic gyrokinetic expansions are also discussed.

        Different options for the numerical implementation of these schemes are considered.

        Due to the efficacy of FET in the development of SP timesteppers for both the fluid and kinetic component, we hope this approach will prove effective in the future for developing SP timesteppers for the full hybrid model. We hope this will give us the opportunity to incorporate previously inaccessible kinetic effects into the highly effective, modern, finite-element MHD models.
    \end{abstract}
    
    
    \newpage
    \tableofcontents
    
    
    \newpage
    \pagenumbering{arabic}
    %\linenumbers\renewcommand\thelinenumber{\color{black!50}\arabic{linenumber}}
            \input{0 - introduction/main.tex}
        \part{Research}
            \input{1 - low-noise PiC models/main.tex}
            \input{2 - kinetic component/main.tex}
            \input{3 - fluid component/main.tex}
            \input{4 - numerical implementation/main.tex}
        \part{Project Overview}
            \input{5 - research plan/main.tex}
            \input{6 - summary/main.tex}
    
    
    %\section{}
    \newpage
    \pagenumbering{gobble}
        \printbibliography


    \newpage
    \pagenumbering{roman}
    \appendix
        \part{Appendices}
            \input{8 - Hilbert complexes/main.tex}
            \input{9 - weak conservation proofs/main.tex}
\end{document}

            \documentclass[12pt, a4paper]{report}

\input{template/main.tex}

\title{\BA{Title in Progress...}}
\author{Boris Andrews}
\affil{Mathematical Institute, University of Oxford}
\date{\today}


\begin{document}
    \pagenumbering{gobble}
    \maketitle
    
    
    \begin{abstract}
        Magnetic confinement reactors---in particular tokamaks---offer one of the most promising options for achieving practical nuclear fusion, with the potential to provide virtually limitless, clean energy. The theoretical and numerical modeling of tokamak plasmas is simultaneously an essential component of effective reactor design, and a great research barrier. Tokamak operational conditions exhibit comparatively low Knudsen numbers. Kinetic effects, including kinetic waves and instabilities, Landau damping, bump-on-tail instabilities and more, are therefore highly influential in tokamak plasma dynamics. Purely fluid models are inherently incapable of capturing these effects, whereas the high dimensionality in purely kinetic models render them practically intractable for most relevant purposes.

        We consider a $\delta\!f$ decomposition model, with a macroscopic fluid background and microscopic kinetic correction, both fully coupled to each other. A similar manner of discretization is proposed to that used in the recent \texttt{STRUPHY} code \cite{Holderied_Possanner_Wang_2021, Holderied_2022, Li_et_al_2023} with a finite-element model for the background and a pseudo-particle/PiC model for the correction.

        The fluid background satisfies the full, non-linear, resistive, compressible, Hall MHD equations. \cite{Laakmann_Hu_Farrell_2022} introduces finite-element(-in-space) implicit timesteppers for the incompressible analogue to this system with structure-preserving (SP) properties in the ideal case, alongside parameter-robust preconditioners. We show that these timesteppers can derive from a finite-element-in-time (FET) (and finite-element-in-space) interpretation. The benefits of this reformulation are discussed, including the derivation of timesteppers that are higher order in time, and the quantifiable dissipative SP properties in the non-ideal, resistive case.
        
        We discuss possible options for extending this FET approach to timesteppers for the compressible case.

        The kinetic corrections satisfy linearized Boltzmann equations. Using a Lénard--Bernstein collision operator, these take Fokker--Planck-like forms \cite{Fokker_1914, Planck_1917} wherein pseudo-particles in the numerical model obey the neoclassical transport equations, with particle-independent Brownian drift terms. This offers a rigorous methodology for incorporating collisions into the particle transport model, without coupling the equations of motions for each particle.
        
        Works by Chen, Chacón et al. \cite{Chen_Chacón_Barnes_2011, Chacón_Chen_Barnes_2013, Chen_Chacón_2014, Chen_Chacón_2015} have developed structure-preserving particle pushers for neoclassical transport in the Vlasov equations, derived from Crank--Nicolson integrators. We show these too can can derive from a FET interpretation, similarly offering potential extensions to higher-order-in-time particle pushers. The FET formulation is used also to consider how the stochastic drift terms can be incorporated into the pushers. Stochastic gyrokinetic expansions are also discussed.

        Different options for the numerical implementation of these schemes are considered.

        Due to the efficacy of FET in the development of SP timesteppers for both the fluid and kinetic component, we hope this approach will prove effective in the future for developing SP timesteppers for the full hybrid model. We hope this will give us the opportunity to incorporate previously inaccessible kinetic effects into the highly effective, modern, finite-element MHD models.
    \end{abstract}
    
    
    \newpage
    \tableofcontents
    
    
    \newpage
    \pagenumbering{arabic}
    %\linenumbers\renewcommand\thelinenumber{\color{black!50}\arabic{linenumber}}
            \input{0 - introduction/main.tex}
        \part{Research}
            \input{1 - low-noise PiC models/main.tex}
            \input{2 - kinetic component/main.tex}
            \input{3 - fluid component/main.tex}
            \input{4 - numerical implementation/main.tex}
        \part{Project Overview}
            \input{5 - research plan/main.tex}
            \input{6 - summary/main.tex}
    
    
    %\section{}
    \newpage
    \pagenumbering{gobble}
        \printbibliography


    \newpage
    \pagenumbering{roman}
    \appendix
        \part{Appendices}
            \input{8 - Hilbert complexes/main.tex}
            \input{9 - weak conservation proofs/main.tex}
\end{document}

            \documentclass[12pt, a4paper]{report}

\input{template/main.tex}

\title{\BA{Title in Progress...}}
\author{Boris Andrews}
\affil{Mathematical Institute, University of Oxford}
\date{\today}


\begin{document}
    \pagenumbering{gobble}
    \maketitle
    
    
    \begin{abstract}
        Magnetic confinement reactors---in particular tokamaks---offer one of the most promising options for achieving practical nuclear fusion, with the potential to provide virtually limitless, clean energy. The theoretical and numerical modeling of tokamak plasmas is simultaneously an essential component of effective reactor design, and a great research barrier. Tokamak operational conditions exhibit comparatively low Knudsen numbers. Kinetic effects, including kinetic waves and instabilities, Landau damping, bump-on-tail instabilities and more, are therefore highly influential in tokamak plasma dynamics. Purely fluid models are inherently incapable of capturing these effects, whereas the high dimensionality in purely kinetic models render them practically intractable for most relevant purposes.

        We consider a $\delta\!f$ decomposition model, with a macroscopic fluid background and microscopic kinetic correction, both fully coupled to each other. A similar manner of discretization is proposed to that used in the recent \texttt{STRUPHY} code \cite{Holderied_Possanner_Wang_2021, Holderied_2022, Li_et_al_2023} with a finite-element model for the background and a pseudo-particle/PiC model for the correction.

        The fluid background satisfies the full, non-linear, resistive, compressible, Hall MHD equations. \cite{Laakmann_Hu_Farrell_2022} introduces finite-element(-in-space) implicit timesteppers for the incompressible analogue to this system with structure-preserving (SP) properties in the ideal case, alongside parameter-robust preconditioners. We show that these timesteppers can derive from a finite-element-in-time (FET) (and finite-element-in-space) interpretation. The benefits of this reformulation are discussed, including the derivation of timesteppers that are higher order in time, and the quantifiable dissipative SP properties in the non-ideal, resistive case.
        
        We discuss possible options for extending this FET approach to timesteppers for the compressible case.

        The kinetic corrections satisfy linearized Boltzmann equations. Using a Lénard--Bernstein collision operator, these take Fokker--Planck-like forms \cite{Fokker_1914, Planck_1917} wherein pseudo-particles in the numerical model obey the neoclassical transport equations, with particle-independent Brownian drift terms. This offers a rigorous methodology for incorporating collisions into the particle transport model, without coupling the equations of motions for each particle.
        
        Works by Chen, Chacón et al. \cite{Chen_Chacón_Barnes_2011, Chacón_Chen_Barnes_2013, Chen_Chacón_2014, Chen_Chacón_2015} have developed structure-preserving particle pushers for neoclassical transport in the Vlasov equations, derived from Crank--Nicolson integrators. We show these too can can derive from a FET interpretation, similarly offering potential extensions to higher-order-in-time particle pushers. The FET formulation is used also to consider how the stochastic drift terms can be incorporated into the pushers. Stochastic gyrokinetic expansions are also discussed.

        Different options for the numerical implementation of these schemes are considered.

        Due to the efficacy of FET in the development of SP timesteppers for both the fluid and kinetic component, we hope this approach will prove effective in the future for developing SP timesteppers for the full hybrid model. We hope this will give us the opportunity to incorporate previously inaccessible kinetic effects into the highly effective, modern, finite-element MHD models.
    \end{abstract}
    
    
    \newpage
    \tableofcontents
    
    
    \newpage
    \pagenumbering{arabic}
    %\linenumbers\renewcommand\thelinenumber{\color{black!50}\arabic{linenumber}}
            \input{0 - introduction/main.tex}
        \part{Research}
            \input{1 - low-noise PiC models/main.tex}
            \input{2 - kinetic component/main.tex}
            \input{3 - fluid component/main.tex}
            \input{4 - numerical implementation/main.tex}
        \part{Project Overview}
            \input{5 - research plan/main.tex}
            \input{6 - summary/main.tex}
    
    
    %\section{}
    \newpage
    \pagenumbering{gobble}
        \printbibliography


    \newpage
    \pagenumbering{roman}
    \appendix
        \part{Appendices}
            \input{8 - Hilbert complexes/main.tex}
            \input{9 - weak conservation proofs/main.tex}
\end{document}

            \documentclass[12pt, a4paper]{report}

\input{template/main.tex}

\title{\BA{Title in Progress...}}
\author{Boris Andrews}
\affil{Mathematical Institute, University of Oxford}
\date{\today}


\begin{document}
    \pagenumbering{gobble}
    \maketitle
    
    
    \begin{abstract}
        Magnetic confinement reactors---in particular tokamaks---offer one of the most promising options for achieving practical nuclear fusion, with the potential to provide virtually limitless, clean energy. The theoretical and numerical modeling of tokamak plasmas is simultaneously an essential component of effective reactor design, and a great research barrier. Tokamak operational conditions exhibit comparatively low Knudsen numbers. Kinetic effects, including kinetic waves and instabilities, Landau damping, bump-on-tail instabilities and more, are therefore highly influential in tokamak plasma dynamics. Purely fluid models are inherently incapable of capturing these effects, whereas the high dimensionality in purely kinetic models render them practically intractable for most relevant purposes.

        We consider a $\delta\!f$ decomposition model, with a macroscopic fluid background and microscopic kinetic correction, both fully coupled to each other. A similar manner of discretization is proposed to that used in the recent \texttt{STRUPHY} code \cite{Holderied_Possanner_Wang_2021, Holderied_2022, Li_et_al_2023} with a finite-element model for the background and a pseudo-particle/PiC model for the correction.

        The fluid background satisfies the full, non-linear, resistive, compressible, Hall MHD equations. \cite{Laakmann_Hu_Farrell_2022} introduces finite-element(-in-space) implicit timesteppers for the incompressible analogue to this system with structure-preserving (SP) properties in the ideal case, alongside parameter-robust preconditioners. We show that these timesteppers can derive from a finite-element-in-time (FET) (and finite-element-in-space) interpretation. The benefits of this reformulation are discussed, including the derivation of timesteppers that are higher order in time, and the quantifiable dissipative SP properties in the non-ideal, resistive case.
        
        We discuss possible options for extending this FET approach to timesteppers for the compressible case.

        The kinetic corrections satisfy linearized Boltzmann equations. Using a Lénard--Bernstein collision operator, these take Fokker--Planck-like forms \cite{Fokker_1914, Planck_1917} wherein pseudo-particles in the numerical model obey the neoclassical transport equations, with particle-independent Brownian drift terms. This offers a rigorous methodology for incorporating collisions into the particle transport model, without coupling the equations of motions for each particle.
        
        Works by Chen, Chacón et al. \cite{Chen_Chacón_Barnes_2011, Chacón_Chen_Barnes_2013, Chen_Chacón_2014, Chen_Chacón_2015} have developed structure-preserving particle pushers for neoclassical transport in the Vlasov equations, derived from Crank--Nicolson integrators. We show these too can can derive from a FET interpretation, similarly offering potential extensions to higher-order-in-time particle pushers. The FET formulation is used also to consider how the stochastic drift terms can be incorporated into the pushers. Stochastic gyrokinetic expansions are also discussed.

        Different options for the numerical implementation of these schemes are considered.

        Due to the efficacy of FET in the development of SP timesteppers for both the fluid and kinetic component, we hope this approach will prove effective in the future for developing SP timesteppers for the full hybrid model. We hope this will give us the opportunity to incorporate previously inaccessible kinetic effects into the highly effective, modern, finite-element MHD models.
    \end{abstract}
    
    
    \newpage
    \tableofcontents
    
    
    \newpage
    \pagenumbering{arabic}
    %\linenumbers\renewcommand\thelinenumber{\color{black!50}\arabic{linenumber}}
            \input{0 - introduction/main.tex}
        \part{Research}
            \input{1 - low-noise PiC models/main.tex}
            \input{2 - kinetic component/main.tex}
            \input{3 - fluid component/main.tex}
            \input{4 - numerical implementation/main.tex}
        \part{Project Overview}
            \input{5 - research plan/main.tex}
            \input{6 - summary/main.tex}
    
    
    %\section{}
    \newpage
    \pagenumbering{gobble}
        \printbibliography


    \newpage
    \pagenumbering{roman}
    \appendix
        \part{Appendices}
            \input{8 - Hilbert complexes/main.tex}
            \input{9 - weak conservation proofs/main.tex}
\end{document}

        \part{Project Overview}
            \documentclass[12pt, a4paper]{report}

\input{template/main.tex}

\title{\BA{Title in Progress...}}
\author{Boris Andrews}
\affil{Mathematical Institute, University of Oxford}
\date{\today}


\begin{document}
    \pagenumbering{gobble}
    \maketitle
    
    
    \begin{abstract}
        Magnetic confinement reactors---in particular tokamaks---offer one of the most promising options for achieving practical nuclear fusion, with the potential to provide virtually limitless, clean energy. The theoretical and numerical modeling of tokamak plasmas is simultaneously an essential component of effective reactor design, and a great research barrier. Tokamak operational conditions exhibit comparatively low Knudsen numbers. Kinetic effects, including kinetic waves and instabilities, Landau damping, bump-on-tail instabilities and more, are therefore highly influential in tokamak plasma dynamics. Purely fluid models are inherently incapable of capturing these effects, whereas the high dimensionality in purely kinetic models render them practically intractable for most relevant purposes.

        We consider a $\delta\!f$ decomposition model, with a macroscopic fluid background and microscopic kinetic correction, both fully coupled to each other. A similar manner of discretization is proposed to that used in the recent \texttt{STRUPHY} code \cite{Holderied_Possanner_Wang_2021, Holderied_2022, Li_et_al_2023} with a finite-element model for the background and a pseudo-particle/PiC model for the correction.

        The fluid background satisfies the full, non-linear, resistive, compressible, Hall MHD equations. \cite{Laakmann_Hu_Farrell_2022} introduces finite-element(-in-space) implicit timesteppers for the incompressible analogue to this system with structure-preserving (SP) properties in the ideal case, alongside parameter-robust preconditioners. We show that these timesteppers can derive from a finite-element-in-time (FET) (and finite-element-in-space) interpretation. The benefits of this reformulation are discussed, including the derivation of timesteppers that are higher order in time, and the quantifiable dissipative SP properties in the non-ideal, resistive case.
        
        We discuss possible options for extending this FET approach to timesteppers for the compressible case.

        The kinetic corrections satisfy linearized Boltzmann equations. Using a Lénard--Bernstein collision operator, these take Fokker--Planck-like forms \cite{Fokker_1914, Planck_1917} wherein pseudo-particles in the numerical model obey the neoclassical transport equations, with particle-independent Brownian drift terms. This offers a rigorous methodology for incorporating collisions into the particle transport model, without coupling the equations of motions for each particle.
        
        Works by Chen, Chacón et al. \cite{Chen_Chacón_Barnes_2011, Chacón_Chen_Barnes_2013, Chen_Chacón_2014, Chen_Chacón_2015} have developed structure-preserving particle pushers for neoclassical transport in the Vlasov equations, derived from Crank--Nicolson integrators. We show these too can can derive from a FET interpretation, similarly offering potential extensions to higher-order-in-time particle pushers. The FET formulation is used also to consider how the stochastic drift terms can be incorporated into the pushers. Stochastic gyrokinetic expansions are also discussed.

        Different options for the numerical implementation of these schemes are considered.

        Due to the efficacy of FET in the development of SP timesteppers for both the fluid and kinetic component, we hope this approach will prove effective in the future for developing SP timesteppers for the full hybrid model. We hope this will give us the opportunity to incorporate previously inaccessible kinetic effects into the highly effective, modern, finite-element MHD models.
    \end{abstract}
    
    
    \newpage
    \tableofcontents
    
    
    \newpage
    \pagenumbering{arabic}
    %\linenumbers\renewcommand\thelinenumber{\color{black!50}\arabic{linenumber}}
            \input{0 - introduction/main.tex}
        \part{Research}
            \input{1 - low-noise PiC models/main.tex}
            \input{2 - kinetic component/main.tex}
            \input{3 - fluid component/main.tex}
            \input{4 - numerical implementation/main.tex}
        \part{Project Overview}
            \input{5 - research plan/main.tex}
            \input{6 - summary/main.tex}
    
    
    %\section{}
    \newpage
    \pagenumbering{gobble}
        \printbibliography


    \newpage
    \pagenumbering{roman}
    \appendix
        \part{Appendices}
            \input{8 - Hilbert complexes/main.tex}
            \input{9 - weak conservation proofs/main.tex}
\end{document}

            \documentclass[12pt, a4paper]{report}

\input{template/main.tex}

\title{\BA{Title in Progress...}}
\author{Boris Andrews}
\affil{Mathematical Institute, University of Oxford}
\date{\today}


\begin{document}
    \pagenumbering{gobble}
    \maketitle
    
    
    \begin{abstract}
        Magnetic confinement reactors---in particular tokamaks---offer one of the most promising options for achieving practical nuclear fusion, with the potential to provide virtually limitless, clean energy. The theoretical and numerical modeling of tokamak plasmas is simultaneously an essential component of effective reactor design, and a great research barrier. Tokamak operational conditions exhibit comparatively low Knudsen numbers. Kinetic effects, including kinetic waves and instabilities, Landau damping, bump-on-tail instabilities and more, are therefore highly influential in tokamak plasma dynamics. Purely fluid models are inherently incapable of capturing these effects, whereas the high dimensionality in purely kinetic models render them practically intractable for most relevant purposes.

        We consider a $\delta\!f$ decomposition model, with a macroscopic fluid background and microscopic kinetic correction, both fully coupled to each other. A similar manner of discretization is proposed to that used in the recent \texttt{STRUPHY} code \cite{Holderied_Possanner_Wang_2021, Holderied_2022, Li_et_al_2023} with a finite-element model for the background and a pseudo-particle/PiC model for the correction.

        The fluid background satisfies the full, non-linear, resistive, compressible, Hall MHD equations. \cite{Laakmann_Hu_Farrell_2022} introduces finite-element(-in-space) implicit timesteppers for the incompressible analogue to this system with structure-preserving (SP) properties in the ideal case, alongside parameter-robust preconditioners. We show that these timesteppers can derive from a finite-element-in-time (FET) (and finite-element-in-space) interpretation. The benefits of this reformulation are discussed, including the derivation of timesteppers that are higher order in time, and the quantifiable dissipative SP properties in the non-ideal, resistive case.
        
        We discuss possible options for extending this FET approach to timesteppers for the compressible case.

        The kinetic corrections satisfy linearized Boltzmann equations. Using a Lénard--Bernstein collision operator, these take Fokker--Planck-like forms \cite{Fokker_1914, Planck_1917} wherein pseudo-particles in the numerical model obey the neoclassical transport equations, with particle-independent Brownian drift terms. This offers a rigorous methodology for incorporating collisions into the particle transport model, without coupling the equations of motions for each particle.
        
        Works by Chen, Chacón et al. \cite{Chen_Chacón_Barnes_2011, Chacón_Chen_Barnes_2013, Chen_Chacón_2014, Chen_Chacón_2015} have developed structure-preserving particle pushers for neoclassical transport in the Vlasov equations, derived from Crank--Nicolson integrators. We show these too can can derive from a FET interpretation, similarly offering potential extensions to higher-order-in-time particle pushers. The FET formulation is used also to consider how the stochastic drift terms can be incorporated into the pushers. Stochastic gyrokinetic expansions are also discussed.

        Different options for the numerical implementation of these schemes are considered.

        Due to the efficacy of FET in the development of SP timesteppers for both the fluid and kinetic component, we hope this approach will prove effective in the future for developing SP timesteppers for the full hybrid model. We hope this will give us the opportunity to incorporate previously inaccessible kinetic effects into the highly effective, modern, finite-element MHD models.
    \end{abstract}
    
    
    \newpage
    \tableofcontents
    
    
    \newpage
    \pagenumbering{arabic}
    %\linenumbers\renewcommand\thelinenumber{\color{black!50}\arabic{linenumber}}
            \input{0 - introduction/main.tex}
        \part{Research}
            \input{1 - low-noise PiC models/main.tex}
            \input{2 - kinetic component/main.tex}
            \input{3 - fluid component/main.tex}
            \input{4 - numerical implementation/main.tex}
        \part{Project Overview}
            \input{5 - research plan/main.tex}
            \input{6 - summary/main.tex}
    
    
    %\section{}
    \newpage
    \pagenumbering{gobble}
        \printbibliography


    \newpage
    \pagenumbering{roman}
    \appendix
        \part{Appendices}
            \input{8 - Hilbert complexes/main.tex}
            \input{9 - weak conservation proofs/main.tex}
\end{document}

    
    
    %\section{}
    \newpage
    \pagenumbering{gobble}
        \printbibliography


    \newpage
    \pagenumbering{roman}
    \appendix
        \part{Appendices}
            \documentclass[12pt, a4paper]{report}

\input{template/main.tex}

\title{\BA{Title in Progress...}}
\author{Boris Andrews}
\affil{Mathematical Institute, University of Oxford}
\date{\today}


\begin{document}
    \pagenumbering{gobble}
    \maketitle
    
    
    \begin{abstract}
        Magnetic confinement reactors---in particular tokamaks---offer one of the most promising options for achieving practical nuclear fusion, with the potential to provide virtually limitless, clean energy. The theoretical and numerical modeling of tokamak plasmas is simultaneously an essential component of effective reactor design, and a great research barrier. Tokamak operational conditions exhibit comparatively low Knudsen numbers. Kinetic effects, including kinetic waves and instabilities, Landau damping, bump-on-tail instabilities and more, are therefore highly influential in tokamak plasma dynamics. Purely fluid models are inherently incapable of capturing these effects, whereas the high dimensionality in purely kinetic models render them practically intractable for most relevant purposes.

        We consider a $\delta\!f$ decomposition model, with a macroscopic fluid background and microscopic kinetic correction, both fully coupled to each other. A similar manner of discretization is proposed to that used in the recent \texttt{STRUPHY} code \cite{Holderied_Possanner_Wang_2021, Holderied_2022, Li_et_al_2023} with a finite-element model for the background and a pseudo-particle/PiC model for the correction.

        The fluid background satisfies the full, non-linear, resistive, compressible, Hall MHD equations. \cite{Laakmann_Hu_Farrell_2022} introduces finite-element(-in-space) implicit timesteppers for the incompressible analogue to this system with structure-preserving (SP) properties in the ideal case, alongside parameter-robust preconditioners. We show that these timesteppers can derive from a finite-element-in-time (FET) (and finite-element-in-space) interpretation. The benefits of this reformulation are discussed, including the derivation of timesteppers that are higher order in time, and the quantifiable dissipative SP properties in the non-ideal, resistive case.
        
        We discuss possible options for extending this FET approach to timesteppers for the compressible case.

        The kinetic corrections satisfy linearized Boltzmann equations. Using a Lénard--Bernstein collision operator, these take Fokker--Planck-like forms \cite{Fokker_1914, Planck_1917} wherein pseudo-particles in the numerical model obey the neoclassical transport equations, with particle-independent Brownian drift terms. This offers a rigorous methodology for incorporating collisions into the particle transport model, without coupling the equations of motions for each particle.
        
        Works by Chen, Chacón et al. \cite{Chen_Chacón_Barnes_2011, Chacón_Chen_Barnes_2013, Chen_Chacón_2014, Chen_Chacón_2015} have developed structure-preserving particle pushers for neoclassical transport in the Vlasov equations, derived from Crank--Nicolson integrators. We show these too can can derive from a FET interpretation, similarly offering potential extensions to higher-order-in-time particle pushers. The FET formulation is used also to consider how the stochastic drift terms can be incorporated into the pushers. Stochastic gyrokinetic expansions are also discussed.

        Different options for the numerical implementation of these schemes are considered.

        Due to the efficacy of FET in the development of SP timesteppers for both the fluid and kinetic component, we hope this approach will prove effective in the future for developing SP timesteppers for the full hybrid model. We hope this will give us the opportunity to incorporate previously inaccessible kinetic effects into the highly effective, modern, finite-element MHD models.
    \end{abstract}
    
    
    \newpage
    \tableofcontents
    
    
    \newpage
    \pagenumbering{arabic}
    %\linenumbers\renewcommand\thelinenumber{\color{black!50}\arabic{linenumber}}
            \input{0 - introduction/main.tex}
        \part{Research}
            \input{1 - low-noise PiC models/main.tex}
            \input{2 - kinetic component/main.tex}
            \input{3 - fluid component/main.tex}
            \input{4 - numerical implementation/main.tex}
        \part{Project Overview}
            \input{5 - research plan/main.tex}
            \input{6 - summary/main.tex}
    
    
    %\section{}
    \newpage
    \pagenumbering{gobble}
        \printbibliography


    \newpage
    \pagenumbering{roman}
    \appendix
        \part{Appendices}
            \input{8 - Hilbert complexes/main.tex}
            \input{9 - weak conservation proofs/main.tex}
\end{document}

            \documentclass[12pt, a4paper]{report}

\input{template/main.tex}

\title{\BA{Title in Progress...}}
\author{Boris Andrews}
\affil{Mathematical Institute, University of Oxford}
\date{\today}


\begin{document}
    \pagenumbering{gobble}
    \maketitle
    
    
    \begin{abstract}
        Magnetic confinement reactors---in particular tokamaks---offer one of the most promising options for achieving practical nuclear fusion, with the potential to provide virtually limitless, clean energy. The theoretical and numerical modeling of tokamak plasmas is simultaneously an essential component of effective reactor design, and a great research barrier. Tokamak operational conditions exhibit comparatively low Knudsen numbers. Kinetic effects, including kinetic waves and instabilities, Landau damping, bump-on-tail instabilities and more, are therefore highly influential in tokamak plasma dynamics. Purely fluid models are inherently incapable of capturing these effects, whereas the high dimensionality in purely kinetic models render them practically intractable for most relevant purposes.

        We consider a $\delta\!f$ decomposition model, with a macroscopic fluid background and microscopic kinetic correction, both fully coupled to each other. A similar manner of discretization is proposed to that used in the recent \texttt{STRUPHY} code \cite{Holderied_Possanner_Wang_2021, Holderied_2022, Li_et_al_2023} with a finite-element model for the background and a pseudo-particle/PiC model for the correction.

        The fluid background satisfies the full, non-linear, resistive, compressible, Hall MHD equations. \cite{Laakmann_Hu_Farrell_2022} introduces finite-element(-in-space) implicit timesteppers for the incompressible analogue to this system with structure-preserving (SP) properties in the ideal case, alongside parameter-robust preconditioners. We show that these timesteppers can derive from a finite-element-in-time (FET) (and finite-element-in-space) interpretation. The benefits of this reformulation are discussed, including the derivation of timesteppers that are higher order in time, and the quantifiable dissipative SP properties in the non-ideal, resistive case.
        
        We discuss possible options for extending this FET approach to timesteppers for the compressible case.

        The kinetic corrections satisfy linearized Boltzmann equations. Using a Lénard--Bernstein collision operator, these take Fokker--Planck-like forms \cite{Fokker_1914, Planck_1917} wherein pseudo-particles in the numerical model obey the neoclassical transport equations, with particle-independent Brownian drift terms. This offers a rigorous methodology for incorporating collisions into the particle transport model, without coupling the equations of motions for each particle.
        
        Works by Chen, Chacón et al. \cite{Chen_Chacón_Barnes_2011, Chacón_Chen_Barnes_2013, Chen_Chacón_2014, Chen_Chacón_2015} have developed structure-preserving particle pushers for neoclassical transport in the Vlasov equations, derived from Crank--Nicolson integrators. We show these too can can derive from a FET interpretation, similarly offering potential extensions to higher-order-in-time particle pushers. The FET formulation is used also to consider how the stochastic drift terms can be incorporated into the pushers. Stochastic gyrokinetic expansions are also discussed.

        Different options for the numerical implementation of these schemes are considered.

        Due to the efficacy of FET in the development of SP timesteppers for both the fluid and kinetic component, we hope this approach will prove effective in the future for developing SP timesteppers for the full hybrid model. We hope this will give us the opportunity to incorporate previously inaccessible kinetic effects into the highly effective, modern, finite-element MHD models.
    \end{abstract}
    
    
    \newpage
    \tableofcontents
    
    
    \newpage
    \pagenumbering{arabic}
    %\linenumbers\renewcommand\thelinenumber{\color{black!50}\arabic{linenumber}}
            \input{0 - introduction/main.tex}
        \part{Research}
            \input{1 - low-noise PiC models/main.tex}
            \input{2 - kinetic component/main.tex}
            \input{3 - fluid component/main.tex}
            \input{4 - numerical implementation/main.tex}
        \part{Project Overview}
            \input{5 - research plan/main.tex}
            \input{6 - summary/main.tex}
    
    
    %\section{}
    \newpage
    \pagenumbering{gobble}
        \printbibliography


    \newpage
    \pagenumbering{roman}
    \appendix
        \part{Appendices}
            \input{8 - Hilbert complexes/main.tex}
            \input{9 - weak conservation proofs/main.tex}
\end{document}

\end{document}

\end{document}

    