\subsection*{Classical Timestepping Techniques}
    \BA{Introduction.}

    Traditional approaches for casting time-dependent PDEs into a weak form make use of a timestepping scheme, usually a Runge–Kutta (RK) method characterised—for an $s$-step method—by:
    \begin{align}
        \sfA  \in  \bbR^{s \times s},  &&
        \bfb  \in  \bbR^{s},  &&
        \bfc  \in  \bbR^{s}.
    \end{align}
    The weak form then defines function spaces $\bbU$, $\bbV$ on the spatial domain $\bfOmega$, seeking—for each timestep $*^{n}$, where $\bfu^{n}$ is a analogous to $\bfu|_{t^{n}}$—$\bfdelta\bfu^{n}_{1}, \cdots, \bfdelta\bfu^{n}_{s}  \in  \bbU$, such that $\forall  \bfv_{1}, \cdots, \bfv_{s}  \in  \bbV$,
    \begin{equation}
        0  =  \left\langle\bfv_{i}, \bfF\left[\left.\bfu^{n - 1} + \sum_{j}a_{ij}\bfdelta\bfu^{n}_{j}\right|_{*}, \cdots; \left.\frac{1}{\delta t^{n}}\bfdelta\bfu^{n}_{i}\right|_{*}\right]\left(*, t^{n - 1} + c_{i}\delta t^{n}\right)\right\rangle_{\bfOmega},
    \end{equation}
    for some chosen inner product $\langle*, *\rangle_{\bfOmega}$ on $\bfOmega$, where $\delta t^{n}  :=  t^{n} - t^{n - 1}$. The RK method then defines $\bfu^{n}  :=  \bfu^{n - 1} + \sum_{i}b_{i}\bfdelta\bfu^{n}_{i}$.
    
    \line
    \begin{example}[Gauss–Legendre]
        \BA{Bla.}
    \end{example} 

    \begin{example}[Radau IIA]
        \BA{Bla.}
    \end{example}
    \line
    
    Since such methods reduce the problem to one in space only, traditional finite-element (FE) software are well-suited for their implementation, where the timestepping can be performed through manually-implemented loops on each timestep, or through specialist packages when available, such as the \texttt{Irksome} package within \texttt{Firedrake} \BA{[Ref]}.

    Careful choice of Runge–Kutta method can give timesteppers that preserve the conservative/dissipative structures of the continuous strong form.

    \line
    
    \begin{example}[Heat equation]
        Gauss–Legendre methods are known to dissipate energy, $\rmE(t)  :=  \int_{\bfOmega}u^{2}$, in the heat equation (\ref{eqn:heat equation}) under Dirichlet/Neumann boundary conditions.
        
        At lowest order, $s  =  1$, the Crank–Nicolson/implicit midpoint rule has the exact dissipation:
        \begin{equation}
            \rmE^{n + 1} - \rmE^{n}  =  - \frac{\delta t^{n}}{\rmPe}\int_{\bfOmega}\left\|\nabla\left[\frac{1}{2}(u^{n + 1} + u^{n})\right]\right\|^{2},
        \end{equation}
        resembling the continuous result (\ref{eqn:heat equation dissipation}).
    \end{example}
    
    \begin{example}[Hamiltonian systems]
        \BA{Symplectic operators.}
    \end{example}
    \line

    \BA{Fabian uses such an approach to design structure-preserving timesteppers for incompressible Hall MHD.}

    \BA{Classical techniques using classical timesteppers (dissipative discretizations/symplectic integrators etc.).}
    