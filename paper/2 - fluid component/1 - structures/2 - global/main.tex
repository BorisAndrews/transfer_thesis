\subsection*{Global Structures}
    Define first the helicity, $\calH[\bff](t)$, of a vector field $\bff : \bfOmega\otimes\bbR \rightarrow \bbR^{3}$: \cite{Moffatt_1969, Brown_Canfield_Pevtsov_1999}
    \begin{equation}
        \calH[\bff](t)  :=  \int_{\bfOmega}\bff\cdot(\nabla\wedge\bff)|_{t}
    \end{equation}

    \cite{Laakmann_Hu_Farrell_2022} considers conservation of the following 3 quantities in the incompressible system: \BA{(What do these represent \emph{physically}? Diagrams!)}
    \begin{center}\begin{tabular}{ c | r : l }
        Properties  &  Symbol  &  Definition  \\
        \hline\hline
        Energy  &  $\rmE$  &  $\frac{1}{2}\|\bfu\|_{\bfOmega}^{2} + \frac{1}{\beta}\|\bfB\|_{\bfOmega}^{2}$  \\
        \hdashline
        Magnetic helicity  &  $\rmH_{\rmM}$  &  $\calH[\bfA]$  \\
        Hybrid helicity  &  $\rmH_{\rmH}$  &  $ab\calH[\bfA] + (a + b)\langle\bfB, \bfu\rangle_{\bfOmega} + \calH[\bfu]$
    \end{tabular}\end{center}
    where $a$, $b$ satisfy the relation $a + b  =  4/\beta\rmRH  (=  2\rmCy_{+})$, and $\bfA$ is a magnetic potential, defined such that $\bfB  =  \nabla\wedge\bfA$. By the exactness of the de Rham complex on contractible $\bfOmega$, such a potential necessarily exists, since $\bfB  \in  \ker[\rmdiv]  =  \im[\bfcurl]$.
    
    \begin{remark}
        We call this $\bfA$ \emph{a} magnetic potential since multiple such magnetic potentials can exist, up to the addition of vector field in the kernel of $\rmdiv$. It doesn't matter here however which potential $\bfA$ is chosen, the conservation/dissipation results should still hold.
    \end{remark}

    Taking the derivatives of these quantities over time (still in the incompressible system) gives \BA{(Proofs in appendix?)}
    \begin{align}
        \begin{split}
            \frac{d\rmE}{dt}  &=  \oint_{\bfGamma}\left[- \left(\frac{1}{2}\|\bfu\|^{2}\bfI + p\bfI - \frac{1}{\rmRef}\nabla\bfu^{\rmT}\right)\bfu + \frac{1}{\beta}\bfB\wedge\bfE\right]\cdot\bfn  \\
            &\;\;\;\;\;\;\;\;\;\;\;\;\;\;\;\;- \left.\left(\frac{1}{\rmRef}\|\nabla\bfu\|^{2} + \frac{2}{\beta\rmRem}\|\bfj\|^{2}\right)\right|_{\bfOmega}
        \end{split}  \\
        \frac{d\rmH_{\rmM}}{dt}  &=  \oint_{\bfGamma}[\bfA\wedge(\bfE - \nabla\varphi)]\cdot\bfn - \frac{2}{\rmRem}\calH[\bfB]  \\
        \begin{split}
            \frac{d\rmH_{\rmH}}{dt}  &=  \oint_{\bfGamma}\left[ab\bfA\wedge(\bfE - \nabla\varphi) + (a + b)\left(\bfu\wedge\bfE - p\bfB - \frac{1}{2}\|\bfu\|^{2}\bfB\right)\right.  \\
            &\;\;\;\;\;\;\;\;\;\;\;\;\;\;\;\;\;\;\;\;\;\;\;\;\;\;\;\;\;\;\;\;\left.+ a\bfA\wedge\bfu + \left(\partial_{t}\bfu\wedge\bfu - \|\bfu\|^{2}\bfomega - 2p\bfomega + \frac{1}{\rmRef}\bfB\wedge\bfomega\right)\right]\cdot\bfn  \\
            &\;\;\;\;\;\;\;\;\;\;\;\;\;\;\;\;- \left[\frac{2ab}{\rmRem}\calH[\bfB] + (a + b)\left(\frac{1}{\rmRem} + \frac{1}{\rmRef}\right)\langle\bfj, \bfomega\rangle_{\bfOmega} + \frac{2}{\rmRef}\calH[\bfomega]\right]
        \end{split}
    \end{align}
    noting $\partial_{t}\bfA  =  - \nabla\varphi - \bfE$. Thus, for boundary conditions on $\bfGamma$, \BA{(Hmmmmm, haven't defined $\varphi$ either...)}
    \begin{align}\label{eqn:incompressible canonical boundary conditions}
        \bfzero  =  \bfu,  &&
        0  =  \bfB\cdot\bfn,  &&
        \bfzero  =  \bfE\wedge\bfn \text{ or } \bfB\wedge\bfn,  &&
        \bfzero  =  (\bfE - \nabla\varphi)\wedge\bfn \text{ or } \bfA\wedge\bfn
    \end{align}
    $\frac{d\rmE}{dt}  \leq  0$, and moreover, as shown in \cite{Laakmann_Hu_Farrell_2022}, in the ideal limit for $\rmRef, \rmRem  =  \infty$,
    \begin{equation}
        \frac{d\rmE}{dt}, \frac{d\rmH_{\rmM}}{dt}, \frac{d\rmH_{\rmH}}{dt}  =  0
    \end{equation}

    Certain discretizations that preserve these properties are derived in \cite{Laakmann_Hu_Farrell_2022}.

    For the \emph{compressible} case, one can consider the energy, $\rmE$, redefined to incorporate the internal energy and variable density, and magnetic helicity, $\rmH_{\rmM}$, with definition unchanged:
    \begin{center}\begin{tabular}{ c | r l }
        Properties  &  Symbol  &  Definition  \\
        \hline\hline
        Energy  &  $\rmE$  &  $\frac{1}{2}\rho\|\bfu\|_{\bfOmega}^{2} + \frac{1}{\beta}\|\bfB\|_{\bfOmega}^{2} + \int_{\bfOmega}p$  \\
        Magnetic helicity  &  $\rmH_{\rmM}$  &  $\calH[\bfA]$
    \end{tabular}\end{center}
    \begin{remark}
        I can't find an compressible analogue for the hybrid helicity, $\rmH_{\rmH}$, due to the presence of the momentum, $\bfp$, terms. I can't find any way to make the momentum and density terms cancel in a way they do for the energy, and have no reason to expect them to do so.
    \end{remark}
    The time derivative for the energy, $\rmE$, then includes boundary contributions \emph{only}, \BA{(Proof in appendix?)}
    \begin{equation}
        \frac{d\rmE}{dt}  =  \oint_{\bfGamma}\left[- \left(\frac{1}{2}\|\bfp\|^{2}\bfI + p\bfI - \frac{1}{\rmRef}\rho\bftau\right)\bfu + \frac{1}{\beta}\bfB\wedge\bfE + \frac{1}{\rmPe}\rho\nabla\theta\right]\cdot\bfn
    \end{equation}
    with the time derivative for the magnetic helicity, $\rmH_{\rmM}$, unchanged.

    Similarly, for boundary conditions on $\bfGamma$,
    \begin{align}\label{eqn:compressible canonical boundary conditions}
        \bfzero  =  \bfp,  &&
        \bfzero  =  \bfE\wedge\bfn \text{ or } \bfB\wedge\bfn,  &&
        \bfzero  =  (\bfE - \nabla\varphi)\wedge\bfn \text{ or } \bfA\wedge\bfn
    \end{align}
    in the ideal magnetic limit, $\rmRem  =  \infty$, \BA{(The result's stronger than this for the energy of course!)}
    \begin{equation}
        \frac{d\rmH_{\rmM}}{dt}  =  0
    \end{equation}
    and further, when both:
    \begin{itemize}
        \item  The ideal kinetic limit, $\rmRef  =  \infty$, holds, or $\bftau\cdot\bfn  =  \bfzero|_{\bfGamma}$.
        \item  The ideal thermal limit, $\rmPe  =  \infty$, holds, or $\nabla\theta\cdot\bfn  =  0|_{\bfGamma}$.
    \end{itemize}
    \begin{equation}
        \frac{d\rmE}{dt}  =  0
    \end{equation}

    As detailed in the introduction, one can also interpret Gauss's law, (\ref{eqn:Gauss's law}), $\nabla\cdot\bfB  =  0$, discounted from the system at $t  >  0$ in the transient model, as a conserved property of the system, as it is enforced by Faraday's law, (\ref{eqn:Faraday's law}).

    \line

    \begin{lemma}[Continuity of helicity]\label{lem:continuity of helicity}
        For any non-empty $\bfgamma  \subseteq  \bfGamma$, there exists a Poincaré constant $C(\bfgamma)$ such that $\forall  \bff  \in  \bfH_{\bfgamma}(\bfcurl)$ (as defined in Appendix \ref{cha:Hilbert complexes}),
        \begin{equation}
            |\calH[\bff]|  \leq  C(\bfgamma)\|\bff\|_{\bfOmega}^{2}
        \end{equation}
    \end{lemma}
    \begin{proof}
        Letting $C(\bfgamma)$ be the (generalized) Poincaré constant associated with $\bfH_{\bfgamma}(\bfcurl)$, such that $\forall \bff \in \bfH_{\bfgamma}(\bfcurl), \|\bff\|_{\bfOmega}  \leq  C(\bfgamma)\|\nabla\wedge\bff\|_{\bfOmega}$,
        \begin{align}
            |\calH[\bff]|  &=  |\langle\bff, \nabla\wedge\bff\rangle_{\bfOmega}|  \\
            &\leq  \|\bff\|_{\bfOmega}\|\nabla\wedge\bff\|_{\bfOmega}  &&\impliedby  \text{Cauchy--Schwarz inequality in } L^{2}  \\
            &\leq  C(\bfgamma)\|\nabla\wedge\bff\|_{\bfOmega}  &&\impliedby  \text{(Generalized) Poincaré inequality in } \bfH_{\bfgamma}(\bfcurl)
        \end{align}
    \end{proof}

    \begin{corollary}
        When $\bfA\wedge\bfn  =  \bfzero|_{\bfgamma}$ for some non-empty $\bfgamma  \subseteq  \bfGamma$, Lemma \ref{lem:continuity of helicity} can be applied to the specific cases of the magnetic helicity, $\rmH_{\rmM}$, \BA{[Ref]} and, in the incompressible case when further $a  =  b  =  \frac{2}{\beta\rmRH}$ and $\bfu\wedge\bfn  =  \bfzero|_{\bfgamma}$, hybrid helicity, $\rmH_{\rmH}$, \cite{Laakmann_Hu_Farrell_2022} giving:
        \begin{align}
            |\rmH_{\rmM}|  &\leq  C(\bfgamma)\|\bfB\|_{\bfOmega}^{2}  \\
            |\rmH_{\rmH}|  &\leq  C(\bfgamma)\left\|\frac{2}{\beta\rmRH}\bfB + \bfomega\right\|_{\bfOmega}^{2}
        \end{align}
        This in turn gives the lower bound on the energy, $\rmE  \geq  \frac{C(\bfgamma)}{\beta}|\rmH_{\rmM}|$.
    \end{corollary}

    \begin{corollary}
        When $\bfB\wedge\bfn  =  \bfzero|_{\bfgamma}$ for some non-empty $\bfgamma  \subseteq  \bfGamma$, and either the boundary condition $\bfzero  =  (\bfE - \nabla\varphi)\wedge\bfn$ or $\bfzero  =  \bfA\wedge\bfn$ holds on all of $\bfGamma$, Lemma \ref{lem:continuity of helicity} can be applied to the time derivative of the magnetic helicity, $\rmH_{\rmM}$, giving
        \begin{equation}
            \left|\frac{d\rmH_{\rmM}}{dt}\right|  \leq  \frac{2C(\bfgamma)}{\rmRem}\|\bfj\|^{2},
        \end{equation}
        indicating large changes in the magnetic helicity are associated with large currents, $\bfj$.
    \end{corollary}

    \BA{Would like to look at how these bounds change when one switches to a non-conforming scheme.}

    \line