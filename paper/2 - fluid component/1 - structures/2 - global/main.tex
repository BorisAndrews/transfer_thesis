\subsection*{Global Structures}
    Define first the helicity, $\calH[\bff](t)$, of a vector field $\bff : \bfOmega\otimes\bbR \rightarrow \bbR^{3}$: \cite{Moffatt_1969, Brown_Canfield_Pevtsov_1999}
    \begin{equation}
        \calH[\bff](t)  :=  \int_{\bfOmega}\bff\cdot(\nabla\wedge\bff)|_{t}
    \end{equation}

    Denote a magnetic potential, $\bfA$, defined such that $\bfB  =  \nabla\wedge\bfA$. By the exactness of the $\Lambda_{0}^{\bullet}(\bfOmega)$ Hilbert complex on contractible $\bfOmega$ (See Appendix \ref{cha:Hilbert complexes}) such a potential necessarily exists, since
    \begin{equation}
        \bfB  \in  \ker[\rmdiv; \bfH_{0}(\rmdiv)]  =  \im[\bfcurl, \bfH_{0}(\bfcurl)].
    \end{equation}
    
    \begin{remark}
        We call this $\bfA$ \emph{a}, not \emph{the}, magnetic potential since multiple such magnetic potentials can exist, up to the addition of a vector field in the kernel of $\rmdiv$: a potential gradient.
        
        It is irrelevant however here which potential $\bfA$ is chosen, as the conservation/dissipation results still hold.
    \end{remark}

    Assume henceforth the boundary conditions (BCs) on $\bfGamma$:\BA{(Note on alternative BCs?)}
    \begin{align}\label{eqn:canonical boundary conditions}
        \bfzero  =  \bfu\cdot\bfn,   &&
        \bfzero  =  \bfB\cdot\bfn,   &&
        \bfzero  =  \bfE\wedge\bfn,  &&
        \bfzero  =  \bfA\wedge\bfn.
    \end{align}

    In the incompressible case, \cite{Laakmann_Hu_Farrell_2022} considers the conservation of the following 3 quantities in the incompressible system:\footnote{Note, the hybrid helicity, $\rmH_{\rmH}$, is \emph{not} (necessarily) the helicity of a vector field, except in the case of $a = b = \beta\rmRH/2 (= 1/\rmCy\!_{+})$, where
    \begin{equation}
        \rmH_{\rmH} = \calH\left[\bfA + \frac{2}{\beta\rmRH}\bfu\right].
    \end{equation}
    It is referred to here as the hybrid \emph{helicity} here only by convention.} \BA{(What do these represent \emph{physically}? Diagrams!)}
    \begin{center}\begin{tabular}{ c | r : l }
        Properties  &  Symbol  &  Definition  \\
        \hline\hline
        Energy  &  $\rmE$  &  $\frac{1}{2}\|\bfu\|_{\bfOmega}^{2} + \frac{1}{\beta}\|\bfB\|_{\bfOmega}^{2}$  \\
        \hdashline
        Magnetic helicity \cite{Blackman_2015}  &  $\rmH_{\rmM}$  &  $\calH[\bfA]$  \\
        Hybrid helicity \cite{Subramanian_Brandenburg_2006}  &  $\rmH_{\rmH}$  &  $\langle(\bfA + a\bfu), (\bfB + b\bfomega)\rangle_{\bfOmega}$
    \end{tabular}\end{center}
    where $a$, $b$ satisfy the relation
    \begin{equation}
        \frac{1}{a} + \frac{1}{b}  =  \frac{4}{\beta\rmRH}  (=  2\rmCy_{+}),
    \end{equation}
    and $\bfomega$ is the vorticity $\bfomega  :=  \nabla\wedge\bfu$.
    
    For the \emph{compressible} case, one can consider the energy, $\rmE$, redefined to incorporate the internal energy and variable density, and magnetic helicity, $\rmH_{\rmM}$, with definition unchanged:
    \begin{center}\begin{tabular}{ c | r : l }
        Properties  &  Symbol  &  Definition  \\
        \hline\hline
        Energy  &  $\rmE$  &  $\frac{1}{2}\rho\|\bfu\|_{\bfOmega}^{2} + \frac{1}{\beta}\|\bfB\|_{\bfOmega}^{2} + \int_{\bfOmega}p$  \\
        \hdashline
        Magnetic helicity  &  $\rmH_{\rmM}$  &  $\calH[\bfA]$
    \end{tabular}\end{center}

    \begin{remark}
        I can't find for the hybrid helicity, $\rmH_{\rmH}$, in the compressible case. There are claims that the hybrid helicity is conserved is ideal, barotropic MHD, but I can neither get the equations to work out, nor find a reliable source for these claims.
        
        Since on top of that, \cite{Laakmann_Hu_Farrell_2022} presents only finite-element timesteppers that preserve the conservation properties for \emph{either} $\rmH_{\rmM}$ \emph{or} $\rmH_{\rmM}$ and not both (leaving this for further work) and the magnetic helicity, $\rmH_{\rmM}$, is generally of more interest in the theory and literature, I've elected therefore only to consider $\rmH_{\rmM}$ for now. (See Chapter \ref{cha:research plan})
    \end{remark}

    Taking the derivatives of these quantities over time (in the compressible system) gives:
    \begin{align}
        \frac{d}{dt}\rmE         &=  \oint_{\bfGamma}\left[\frac{1}{\rmRef}\rho\bftau\cdot\sym[\bfu\otimes\bfn] + \frac{1}{\rmPe}\rho\nabla\theta\cdot\bfn\right],  \\
        \frac{d}{dt}\rmH_{\rmM}  &=  - \frac{2}{\rmRem}\calH[\bfB]
    \end{align}
    Notably:
    \begin{itemize}
        \item  If both:
        \begin{itemize}
            \item  One of the ideal kinetic limit $\rmRef  \rightarrow  \infty$, the stress-free BC $\bftau\cdot\bfn  =  \bfzero|_{\bfGamma}$, or the no-slip BC $\bfu\wedge\bfn  =  \bfzero|_{\bfGamma}$, holds.
            \item  Either the ideal thermal limit, $\rmPe  \rightarrow  \infty$, or the thermal insulation BC $\nabla\theta\cdot\bfn  =  0|_{\bfGamma}$, holds.
        \end{itemize}
        then
        \begin{equation}
            \frac{d\rmE}{dt}  =  0.
        \end{equation}

        \item  In the ideal magnetic limit, $\rmRem  \rightarrow  \infty$,
        \begin{equation}
            \frac{d}{dt}\rmH_{\rmM}  =  0.
        \end{equation}
      \end{itemize}

    \line

    \begin{lemma}[Arnold's inequality (Continuity of helicity)]\label{lem:continuity of helicity}
        For any non-empty $\bfgamma  \subseteq  \bfGamma$, there exists a Poincaré constant $C_{\bfgamma}$ such that $\forall  \bff  \in  \bfH_{\bfgamma}(\bfcurl)$ (as defined in Appendix \ref{cha:Hilbert complexes}),
        \begin{equation}
            |\calH[\bff]|  \leq  C_{\bfgamma}\|\bff\|_{\bfOmega}^{2}.
        \end{equation}
    \end{lemma}
    \begin{proof}
        The proof relies on the Cauchy--Schwarz inequality in $L^{2}$ and the (generalized) Poincaré inequality in $\bfH_{\bfgamma}(\bfcurl)$. See Theorem 1.5 of \cite{Arnold_Khesin_2008}.
    \end{proof}

    \begin{corollary}
        Lemma \ref{lem:continuity of helicity} can be applied to the specific cases of the magnetic helicity, $\rmH_{\rmM}$, giving
        \begin{equation}
            |\rmH_{\rmM}|  \leq  C(\bfgamma)\|\bfB\|_{\bfOmega}^{2}
        \end{equation}
        This in turn gives the lower bound on the energy, $\rmE  \geq  \frac{C(\bfgamma)}{\beta}|\rmH_{\rmM}|$.
    \end{corollary}

    \BA{Would like to look at how this bound changes when one switches to a non-conforming scheme.}

    \line