\subsection*{Periodic Case}
    \BA{Introduction.}

    \BA{If I'm going to talk about periodic problems, I need to add my ideas about how we incorporate the solve for the frequency in a separate chapter.}

    Consider now the time-periodic case, on a domain $\bfOmega\otimes T$ for some interval $T$, with the inclusion of the time derivatives.
    
    \BA{(Need some references for this paragraph. Should trace the paper trail back a little further from Giacomo's report.)} To derive weak formulations for the periodic problem, the system can be interpreted through the lens of function/finite element spaces and weak formulations in both space \emph{and} time. This can be done through the use of function/finite elements spaces that are tensor product spaces of one in space, $\bfOmega$, and time, $T$. Approaching through the lens of function/finite element spaces in time, has many benefits, notably here including facilitating the construction of structure-preserving discretizations directly from the weak formulation, using specially chosen test spaces. For a further discussion on this topic, see Appendix \BA{(Here's where I used to have a link to the FET appendix...)}. \BA{(Need to mention somewhere about how Firedrake doesn't support 4D+ simulations, and how we combat that.)}

    Casting into weak form therefore, trial and test spaces are considered to be supported on $\bfOmega\otimes T$. Proceeding directly to the compressible case, consider test functions for each equation in the test spaces as listed in Figure \ref{fig:periodic test spaces}, similarly via the $L^{2}(\bfOmega\otimes T)$ inner product. This takes the variational formulation in Figure \ref{fig:periodic weak form}.
    
    \begin{figure}
        \centering
        \begin{tabular}{ r l c | c }
            \multicolumn{2}{c}{Equation}  &  Index  &  Test space  \\
            \hline\hline
            Mass conservation  &  $- \partial_{t}\rho  =  \nabla\cdot\bfp$  &  (\ref{eqn:mass conservation})  &  $\calP$  \\
            Momentum conservation  &  $\rho\partial_{t}\bfu 
             =  - \bfp\cdot\nabla\bfu - \nabla p + \frac{2}{\beta}\bfj\wedge\bfB + \cdots$  &  (\ref{eqn:momentum conservation})  &  $\calM$  \\
            Energy conservation  &  $\partial_{t}p  =  - \nabla\cdot[p\bfu] - p\nabla\cdot\bfu + \cdots$  &  (\ref{eqn:energy conservation})  &  $\calD$  \\
            \hline
            Current identity  &  $\bfzero  =  \frac{1}{\rmRem}\bfj - (\bfE + \bfu\wedge\bfB) + \rmRH\bfj\wedge\bfB$  &  (\ref{eqn:current identity})  &  $\calJ$  \\
            \hline
            Momentum identity  &  $\bfzero  =  \bfp - \rho\bfu$  &  (\ref{eqn:velocity identity})  &  $\calU$  \\
            Temperature identity  &  $0  =  p - \rho\theta$  &  (\ref{eqn:temperature identity})  &  $\Theta$  \\
            \hline
            Ampère's law  &  $\bfzero  =  \nabla\wedge\bfB - \bfj$  &  (\ref{eqn:Ampère's law})  &  $\calE$  \\
            Faraday's law  &  $\partial_{t}\bfB  =  - \nabla\wedge\bfE$  &  (\ref{eqn:Faraday's law})  &  $\alpha\partial_{t}\calB$  \\
            Gauss's law  &  $0  =  \nabla\cdot\bfB$  &  (\ref{eqn:Gauss's law})  &  $\beta\nabla\cdot\calB$  \\
        \end{tabular}
        \caption{Test spaces for the (compressible) periodic variation formulation \BA{(I introduce ``$\alpha$'' here, but I used $\alpha$ as notation for a small thing in Subsection \ref{cha:fluid models})}}
        \label{fig:periodic test spaces}
    \end{figure}

    \begin{figure}
        \line
        \begin{align}
            \forall q \in \calP,  0  &=  \left.\left(\langle\partial_{t}\rho, q\rangle + \langle\nabla\cdot\bfp, q\rangle\tall\right)\right|_{\bfOmega\otimes T}  \\
            \begin{split}
                \forall \bfq \in \calM,  0  &=  \left.\left(- \langle\rho\partial_{t}\bfu, \bfq\rangle - \langle\bfp\cdot\nabla\bfu, \bfq\rangle + \langle p, \nabla\cdot\bfq\rangle + \frac{2}{\beta}\langle\bfj\wedge\bfB, \bfq\rangle - \frac{1}{\rmRef}\langle\rho\bftau, \nabla_{s}\bfq\rangle\right)\right|_{\bfOmega\otimes T}  \\
                &\;\;\;\;\;\;\;\;\;\;\;\;\;\;\;\;\;\;\;\;\;\;\;\;  + \left.\left(- \langle p, \bfq\cdot\bfn\rangle + \frac{1}{\rmRef}\langle\rho\bftau, \sym(\bfq\otimes\bfn)\rangle\right)\right|_{\bfGamma\otimes T}
            \end{split}  \\
            \begin{split}
                \forall \sigma \in \calD,  0  &=  \left(\langle p, \partial_{t}\sigma\rangle + \langle p\bfu, \nabla\sigma\rangle - \langle p\nabla\cdot\bfu, \sigma\rangle + \frac{1}{\rmRef}\langle\rho\bftau:\nabla_{\rms}\bfu, \sigma\rangle\right.  \\
                &\;\;\;\;\;\;\;\;\;\;\;\;\;\;\;\;\;\;\;\;\;\;\;\;\;\;\;\;\;\;\;\;\;\;\;\;\;\;\;\;\;\;\;\;\;\;\;\;\left.\left.+ \frac{2}{\beta\rmRem}\left\langle\|\bfj\|^{2}, \sigma\right\rangle - \frac{1}{\rmPe}\langle\rho\nabla\theta, \nabla\sigma\rangle\right)\right|_{\bfOmega\otimes T}  \\
                &\;\;\;\;\;\;\;\;\;\;\;\;\;\;\;\;\;\;\;\;\;\;\;\;- \langle p, \sigma\rangle_{\bfOmega\otimes\partial T} + \left.\left(- \langle p\bfu\cdot\bfn, \sigma\rangle + \frac{1}{\rmPe}\langle\rho\nabla\theta\cdot\bfn, \sigma\rangle\right)\right|_{\bfGamma\otimes T}
            \end{split}  \\
            \forall \bfk \in \calJ,  0  &=  \left.\left(\frac{1}{\rmRem}\langle\bfj, \bfk\rangle - \langle\bfE, \bfk\rangle - \langle\bfu\wedge\bfB, \bfk\rangle + \rmRH\langle\bfj\wedge\bfB, \bfk\rangle\right)\right|_{\bfOmega\otimes T}  \\
            \forall \bfv \in \calU,  0  &=  \left.\left(\langle\bfp, \bfv\rangle - \langle\rho\bfu, \bfv\rangle\tall\right)\right|_{\bfOmega\otimes T}  \\
            \forall \eta \in \Theta,  0  &=  \left.\left(\langle p, \eta\rangle - \langle\rho\theta, \eta\rangle\tall\right)\right|_{\bfOmega\otimes T}  \\
            \forall \bfF \in \calE,  0  &=  \left.\left(\langle\bfB, \nabla\wedge\bfF\rangle - \langle\bfj, \bfF\rangle\tall\right)\right|_{\bfOmega\otimes T} + \langle\bfB, \bfF\wedge\bfn\rangle_{\bfGamma\otimes T}  \\
            \forall \bfC \in \partial_{t}\calB,  0  &=  \left.\left(\langle\partial_{t}\bfB, \bfC\rangle - \langle\nabla\wedge\bfE, \bfC\rangle\tall\right)\right|_{\bfOmega\otimes T}
        \end{align}
        \line
        \caption{(Compressible) periodic weak formulation}
        \label{fig:periodic weak form}
    \end{figure}

    \BA{(What conformity does this require?)} Similarly to the stationary-state case, one can seek to enforce mass conservation (\ref{eqn:mass conservation}), Faraday's law (\ref{eqn:Faraday's law}) and Gauss's law (\ref{eqn:Gauss's law}) \emph{strongly} by considering certain test functions, and deriving conditions on the function spaces from there:
    \begin{center}\begin{tabular}{ c | c | c }
        Equation  &  Test function  &  Subspace condition  \\
        \hline\hline
        $0  =  \partial_{t}\rho + \nabla\cdot\bfp$         &  $\partial_{t}\rho + \nabla\cdot\bfp  \in  \calP$  &  $\partial_{t}\calD + \nabla\cdot\calM  \leqslant  \calP$  \\
        $\bfzero  =  \partial_{t}\bfB + \nabla\wedge\bfE$  &  ...                                               &  ...                                                       \\
        $0  =  \nabla\cdot\bfB$                            &  ...                                               &  ...                                                       \\
    \end{tabular}\end{center}

    \BA{IMPORTANT NOTE: Recently split this up into 2 distinct sections on periodic solutions and timesteppers. Since then I've had real trouble showing both Faraday's law and Gauss's are enforced in the periodic problem formulation, due to the whole nature of the redundancy/gauge invariance in Maxwell's equations. The rest is copied over from how this section previously looked but is now largely wrong. If I can't get this to work soon, I'll just leave it as a remark and move on.}
    
    Similarly restricting to Hilbert spaces, it is therefore natural to consider weak formulations where $\calM\oplus\calD \xrightarrow{\rmdiv + \partial_{t}} \calP$ and $\calE\oplus\calB \xrightarrow{\cdots} \partial_{t}\calB\oplus *$ form components of subcomplexes of $H\Lambda^{\bullet}\left(\bfOmega\otimes T\right)$, with projection/interpolation maps $\bbQ_{*}$ such that the following diagrams commute:
    \begin{center}\begin{tikzpicture}[align = center, node distance = 4cm, auto]
        \node (HL2)  at (0,   0)   {$\bfH\left(\begin{matrix} \bfcurl + \partial_{t} \\ * - \rmdiv \end{matrix}\right)$};
        \node (HL3a) at (5.5, 0)   {$\bfH(\rmdiv + \partial_{t})$};
        \node (EB)   at (0,   - 3) {$\calE\oplus\calB$};
        \node (C*)   at (5.5, - 3) {$\partial_{t}\calB\oplus*$};

        \draw[->] (HL2)  -- (HL3a) node[above, midway] {$\left(\begin{matrix} \bfcurl + \partial_{t} \\ * - \rmdiv \end{matrix}\right)$};
        \draw[->] (EB)   -- (C*)   node[above, midway] {$\left(\begin{matrix} \bfcurl + \partial_{t} \\ * - \rmdiv \end{matrix}\right)$};
        \draw[->] (HL2)  -- (EB)   node[left,  midway] {$\bbQ_{\rmE\rmB}$};
        \draw[->] (HL3a) -- (C*)   node[left,  midway] {$\bbQ_{\rmC*}$};
        
        \node (HL3b) at (9,   0)  {$\bfH(\rmdiv + \partial_{t})$};
        \node (HL4) at (12.5, 0)  {$L^{2}$};
        \node (MD)  at (9,   - 3) {$\calM\oplus\calD$};
        \node (P)  at (12.5, - 3) {$\calP$};

        \draw[->] (HL3b) -- (HL4) node[above, midway] {$\rmdiv + \partial_{t}$};
        \draw[->] (MD)   -- (P)   node[above, midway] {$\rmdiv + \partial_{t}$};
        \draw[->] (HL3b) -- (MD)  node[left,  midway] {$\bbQ_{\rmM\rmD}$};
        \draw[->] (HL4)  -- (P)   node[left,  midway] {$\bbQ_{\rmP}$};
    \end{tikzpicture}\end{center}
    Further supposing $\calE\oplus\calB \xrightarrow{\cdots} \partial_{t}\calB\oplus*$ forms part of a longer $H\Lambda^{\bullet}\left(\bfOmega\otimes T\right)$-subcomplex component $\Phi\oplus\calA \xrightarrow{\cdots} \calE\oplus\calB \xrightarrow{\cdots} \partial_{t}\calB\oplus*$ \BA{(To what extent is existence of such a $(\Phi, \calA)$ guaranteed?)},
    \begin{center}\begin{tikzpicture}[align = center, node distance = 4cm, auto]
        \node (HL1)  at (0,  0) {$\bfH\left(\begin{matrix} - \bfgrad - \partial_{t} \\ * + \bfcurl \end{matrix}\right)$};
        \node (HL2)  at (6,  0) {$\bfH\left(\begin{matrix} \bfcurl + \partial_{t} \\ * - \rmdiv \end{matrix}\right)$};
        \node (HL3)  at (12, 0) {$\bfH(\rmdiv + \partial_{t})$};
        
        \node (PhiA) at (0,  - 3) {$\Phi\oplus\calA$};
        \node (EB)   at (6,  - 3) {$\calE\oplus\calB$};
        \node (C*)   at (12, - 3) {$\partial_{t}\calB\oplus*$};
        

        \draw[->] (HL1) -- (HL2) node[above, midway] {$\left(\begin{matrix} - \bfgrad - \partial_{t} \\ * + \bfcurl \end{matrix}\right)$};
        \draw[->] (HL2) -- (HL3) node[above, midway] {$\left(\begin{matrix} \bfcurl + \partial_{t} \\ * - \rmdiv \end{matrix}\right)$};
        
        \draw[->] (HL1) -- (PhiA) node[left,  midway] {$\bbQ_{\Phi\rmA}$};
        \draw[->] (HL2) -- (EB)   node[left,  midway] {$\bbQ_{\rmE\rmB}$};
        \draw[->] (HL3) -- (C*)   node[left,  midway] {$\bbQ_{\rmC*}$};
        
        \draw[->] (PhiA) -- (EB) node[above, midway] {$\left(\begin{matrix} - \bfgrad - \partial_{t} \\ * + \bfcurl \end{matrix}\right)$};
        \draw[->] (EB)   -- (C*) node[above, midway] {$\left(\begin{matrix} \bfcurl + \partial_{t} \\ * - \rmdiv \end{matrix}\right)$};
    \end{tikzpicture}\end{center}
    provided exactness holds,
    \begin{equation}
        \ker\left[\left(\begin{matrix} \bfcurl + \partial_{t} \\ * - \rmdiv \end{matrix}\right) : \calE\oplus\calB \rightarrow \partial_{t}\calB\oplus*\right]  =  \im\left[\left(\begin{matrix} - \bfgrad - \partial_{t} \\ * + \bfcurl \end{matrix}\right) : \Phi\oplus\calA \rightarrow \calE\oplus\calB\right]
    \end{equation}
    such that, since $(\bfE, \bfB)  \in  \calE\oplus\calB$ lies in this kernel, it lies too in this image, and there exist EM potentials $\varphi  \in  \phi$, $\bfA  \in  \calS$ such that:
    \begin{align}
        \bfE  &=  - \nabla\varphi - \partial_{t}\bfA  \\
        \bfB  &=  \nabla\wedge\bfA
    \end{align}
    
    Similarly, supposing $\calM\oplus\calD  \xrightarrow{\rmdiv + \partial_{t}}  \calP$ forms part of a longer $H\Lambda^{\bullet}(\bfOmega\otimes T)$-subcomplex component $\Psi\oplus\calX  \xrightarrow{\cdots}  \calM\oplus\calD  \xrightarrow{\rmdiv + \partial_{t}}  \calP$ \BA{(To what extent is existence of such a $(\Psi, \calX)$ guaranteed?)},
    \begin{center}\begin{tikzpicture}[align = center, node distance = 4cm, auto]
        \node (HL2)  at (0,  0) {$\bfH\left(\begin{matrix} \bfcurl + \partial_{t} \\ * - \rmdiv \end{matrix}\right)$};
        \node (HL3)  at (6, 0) {$\bfH(\rmdiv + \partial_{t})$};
        \node (HL4)  at (12, 0) {$L^{2}$};
        
        \node (PsiChi) at (0,  - 3) {$\Psi\oplus\calX$};
        \node (MD)     at (6,  - 3) {$\calM\oplus\calD$};
        \node (P)      at (12, - 3) {$\calP$};
        

        \draw[->] (HL2) -- (HL3) node[above, midway] {$\left(\begin{matrix} \bfcurl + \partial_{t} \\ * - \rmdiv \end{matrix}\right)$};
        \draw[->] (HL3) -- (HL4) node[above, midway] {$\rmdiv + \partial_{t}$};
        
        \draw[->] (HL2) -- (PsiChi) node[left,  midway] {$\bbQ_{\Psi\rmX}$};
        \draw[->] (HL3) -- (MD)     node[left,  midway] {$\bbQ_{\rmM\rmD}$};
        \draw[->] (HL4) -- (P)      node[left,  midway] {$\bbQ_{\rmP}$};
        
        \draw[->] (PsiChi) -- (MD) node[above, midway] {$\left(\begin{matrix} \bfcurl + \partial_{t} \\ * - \rmdiv \end{matrix}\right)$};
        \draw[->] (MD)     -- (P)  node[above, midway] {$\rmdiv + \partial_{t}$};
    \end{tikzpicture}\end{center}
    where, provided exactness holds, there exist (generalized, compressible) streamfunctions $\bfpsi  \in  \Psi$, $\bfchi  \in  \calX$ such that
    \begin{align}
        \bfp  &=  \nabla\wedge\bfpsi + \partial_{t}\bfchi  \\
        \rho  &=  - \nabla\cdot\bfchi
    \end{align}
    
    By Corollary \ref{cor:tensor product complex inclusion}, such subcomplexes can be constructed via a tensor product construction. (Figures \ref{fig:tensor product EM subcomplexes}--\ref{fig:tensor product momentum subcomplexes})
    
    \begin{figure}[!ht]
        \centering
        {\footnotesize \begin{tikzpicture}[align = center, node distance = 4cm, auto]
            \node (HL1) at (0,  0) {$\bfH\left(\begin{matrix} - \bfgrad - \partial_{t} \\ * + \bfcurl \end{matrix}\right)$};
            \node (HL2) at (5,  0) {$\bfH\left(\begin{matrix} \bfcurl + \partial_{t} \\ * - \rmdiv \end{matrix}\right)$};
            \node (HL3) at (10, 0) {$\bfH(\rmdiv + \partial_{t})$};
            
            \node[ellipse, draw, thick, dotted, minimum width = 5cm, minimum height = 2.5cm, rotate = 30] at (2,  - 3) {};
                \node (H1-L2)    at (2 - 0.85,  - 3 - 0.85) {$H^{1}\otimes L^{2}$};
                \node (Hcurl-H1) at (2 + 0.85,  - 3 + 0.85) {$\bfH(\bfcurl)\otimes H^{1}$};
            \node[ellipse, draw, thick, dotted, minimum width = 5cm, minimum height = 2.5cm, rotate = 30] at (7,  - 3) {};
                \node (Hcurl-L2) at (7 - 0.85,  - 3 - 0.85) {$\bfH(\bfcurl)\otimes L^{2}$};
                \node (Hdiv-H1)  at (7 + 0.85,  - 3 + 0.85) {$\bfH(\rmdiv)\otimes H^{1}$};
            \node[ellipse, draw, thick, dotted, minimum width = 5cm, minimum height = 2.5cm, rotate = 30] at (12, - 3) {};
                \node (Hdiv-L2)  at (12 - 0.85, - 3 - 0.85) {$\bfH(\rmdiv)\otimes L^{2}$};
                \node (L2-H1)    at (12 + 0.85, - 3 + 0.85) {$*$};

            \node (PhiA) at (- 1, - 8) {$\Phi\oplus\calA$};
            \node (EB)   at (4,   - 8) {$\calE\oplus\calB$};
            \node (C*)   at (9,   - 8) {$\partial_{t}\calB\oplus*$};

            \node[ellipse, draw, thick, dotted, minimum width = 4cm, minimum height = 2cm, rotate = 30] at (1,  - 11) {};
                \node (Phi) at (1 - 0.85,  - 11 - 0.85) {$\Phi$};
                \node (A)   at (1 + 0.85,  - 11 + 0.85) {$\calA$};
            \node[ellipse, draw, thick, dotted, minimum width = 4cm, minimum height = 2cm, rotate = 30] at (6,  - 11) {};
                \node (E)   at (6 - 0.85,  - 11 - 0.85) {$\calE$};
                \node (B)   at (6 + 0.85,  - 11 + 0.85) {$\calB$};
            \node[ellipse, draw, thick, dotted, minimum width = 4cm, minimum height = 2cm, rotate = 30] at (11, - 11) {};
                \node (C)   at (11 - 0.85, - 11 - 0.85) {$\partial_{t}\calB$};
                \node (*)   at (11 + 0.85, - 11 + 0.85) {$*$};


            \draw[->] (HL1) -- (HL2) node[above, midway] {$\left(\begin{matrix} - \bfgrad - \partial_{t} \\ * + \bfcurl \end{matrix}\right)$};
            \draw[->] (HL2) -- (HL3) node[above, midway] {$\left(\begin{matrix} \bfcurl + \partial_{t} \\ * - \rmdiv \end{matrix}\right)$};
            
            \draw[->, dashed] (HL1) -- (2  - 0.85, - 3 + 1.15);
            \draw[->, dashed] (HL2) -- (7  - 0.85, - 3 + 1.15);
            \draw[->, dashed] (HL3) -- (12 - 0.85, - 3 + 1.15);

            \draw[->] (HL1) -- (PhiA) node[left, pos = 0.75] {$\bbQ_{\phi\rmA}$};
            \draw[->, color = white] (HL2) -- coordinate[pos = 0.1](HL2a)  coordinate[pos = 0.55](HL2b) (EB);
                \draw[-]                   (HL2)  -- (HL2a);
                \draw[-, color = black!30] (HL2a) -- (HL2b);
                \draw[->]                  (HL2b) -- (EB) node[left, pos = 0.444444] {$\bbQ_{\rmE\rmB}$};
            \draw[->, color = white] (HL3) -- coordinate[pos = 0.25](HL3a) coordinate[pos = 0.55](HL3b) (C*);
                \draw[-]                   (HL3)  -- (HL3a);
                \draw[-, color = black!30] (HL3a) -- (HL3b);
                \draw[->]                  (HL3b) -- (C*) node[left, pos = 0.444444] {$\bbQ_{\rmC*}$};
            
            \draw[->] (Hcurl-H1) -- (Hdiv-H1)  node[above, pos = 0.4]  {$\bfcurl$};
            \draw[->] (Hcurl-H1) -- (Hcurl-L2) node[above, pos = 0.55] {$\partial_{t}$};
            \draw[->] (H1-L2)    -- (Hcurl-L2) node[above, pos = 0.7]  {$\bfgrad$};
            \draw[->] (Hdiv-H1)  -- (Hdiv-L2)  node[above, pos = 0.55] {$\partial_{t}$};
            \draw[->] (Hcurl-L2) -- (Hdiv-L2)  node[above, pos = 0.65] {$\bfcurl$};

            \draw[->, color = white] (H1-L2)    -- coordinate[pos = 0.3](Phia) coordinate[pos = 0.9](Phib) (Phi);
                \draw[-]                   (H1-L2)    -- (Phia) node[left, pos = 0.666666] {$\bbQ_{\phi}$};
                \draw[-, color = black!30] (Phia)     -- (Phib);
                \draw[->]                  (Phib)     -- (Phi);
            \draw[->, color = white] (Hcurl-H1) -- coordinate[pos = 0.5](Aa)   coordinate[pos = 0.8](Ab)   (A);
                \draw[-]                   (Hcurl-H1) -- (Aa)   node[left, pos = 0.85]     {$\bbQ_{\rmA}$};
                \draw[-, color = black!30] (Aa)       -- (Ab);
                \draw[->]                  (Ab)       -- (A);
            \draw[->, color = white] (Hcurl-L2) -- coordinate[pos = 0.3](Ea)   coordinate[pos = 0.9](Eb)   (E);
                \draw[-]                   (Hcurl-L2) -- (Ea)   node[left, pos = 0.666666] {$\bbQ_{\rmE}$};
                \draw[-, color = black!30] (Ea)       -- (Eb);
                \draw[->]                  (Eb)       -- (E);
            \draw[->, color = white] (Hdiv-H1)  -- coordinate[pos = 0.5](Ba)   coordinate[pos = 0.8](Bb)   (B);
                \draw[-]                   (Hdiv-H1)  -- (Ba)   node[left, pos = 0.85]     {$\bbQ_{\rmB}$};
                \draw[-, color = black!30] (Ba)       -- (Bb);
                \draw[->]                  (Bb)       -- (B);
            \draw[->, color = white] (Hdiv-L2)  -- coordinate[pos = 0.7](Ca)   coordinate[pos = 0.9](Cb)   (C);
                \draw[-]                   (Hdiv-L2)  -- (Ca)   node[left, pos = 0.285714] {$\bbQ_{\rmC}$};
                \draw[-, color = black!30] (Ca)       -- (Cb);
                \draw[->]                  (Cb)       -- (C);
            
            \draw[->] (PhiA) -- (EB) node[above, midway] {$\left(\begin{matrix} - \bfgrad - \partial_{t} \\ * + \bfcurl \end{matrix}\right)$};
            \draw[->] (EB)   -- (C*) node[above, midway] {$\left(\begin{matrix} \bfcurl + \partial_{t} \\ * - \rmdiv \end{matrix}\right)$};
            
            \draw[->, dashed] (PhiA) -- (1  - 0.8, - 11 + 1);
            \draw[->, dashed] (EB)   -- (6  - 0.8, - 11 + 1);
            \draw[->, dashed] (C*)   -- (11 - 0.8, - 11 + 1);
            
            \draw[->] (A)   -- (B) node[above, pos = 0.4] {$\bfcurl$};
            \draw[->] (A)   -- (E) node[above, midway]    {$\partial_{t}$};
            \draw[->] (Phi) -- (E) node[above, pos = 0.6] {$\bfgrad$};
            \draw[->] (B)   -- (C) node[above, midway]    {$\partial_{t}$};
            \draw[->] (E)   -- (C) node[above, pos = 0.6] {$\bfcurl$};
        \end{tikzpicture}}
        \caption{Construction of EM subcomplexes via tensor products. \BA{(Perhaps a little explanation on what this means...)}}
        \label{fig:tensor product EM subcomplexes}
    \end{figure}
    
    \begin{figure}[!ht]
        \centering
        {\footnotesize \begin{tikzpicture}[align = center, node distance = 4cm, auto]
            \node (HL2) at (0,  0) {$\bfH\left(\begin{matrix} \bfcurl + \partial_{t} \\ * - \rmdiv \end{matrix}\right)$};
            \node (HL3) at (5,  0) {$\bfH(\rmdiv + \partial_{t})$};
            \node (HL4) at (10, 0) {$L^{2}$};
            
            \node[ellipse, draw, thick, dotted, minimum width = 5cm, minimum height = 2.5cm, rotate = 30] at (2,  - 3) {};
                \node (Hcurl-L2) at (2 - 0.85,  - 3 - 0.85) {$\bfH(\bfcurl)\otimes L^{2}$};
                \node (Hdiv-H1)  at (2 + 0.85,  - 3 + 0.85) {$\bfH(\rmdiv)\otimes H^{1}$};
            \node[ellipse, draw, thick, dotted, minimum width = 5cm, minimum height = 2.5cm, rotate = 30] at (7,  - 3) {};
                \node (Hdiv-L2)  at (7 - 0.85,  - 3 - 0.85) {$\bfH(\rmdiv)\otimes L^{2}$};
                \node (L2-H1)    at (7 + 0.85,  - 3 + 0.85) {$L^{2}\otimes H^{1}$};
            \node[ellipse, draw, thick, dotted, minimum width = 5cm, minimum height = 2.5cm, rotate = 30] at (12, - 3) {};
                \node (L2-L2)    at (12 - 0.85, - 3 - 0.85) {$L^{2}\otimes L^{2}$};

            \node (PsiChi) at (- 1, - 8) {$\Psi\oplus\calX$};
            \node (MD)     at (4,   - 8) {$\calM\oplus\calD$};
            \node (P*)     at (9,   - 8) {$\calP$};

            \node[ellipse, draw, thick, dotted, minimum width = 4cm, minimum height = 2cm, rotate = 30] at (1,  - 11) {};
                \node (Psi) at (1 - 0.85,  - 11 - 0.85) {$\Psi$};
                \node (Chi) at (1 + 0.85,  - 11 + 0.85) {$\calX$};
            \node[ellipse, draw, thick, dotted, minimum width = 4cm, minimum height = 2cm, rotate = 30] at (6,  - 11) {};
                \node (M)   at (6 - 0.85,  - 11 - 0.85) {$\calM$};
                \node (D)   at (6 + 0.85,  - 11 + 0.85) {$\calD$};
            \node[ellipse, draw, thick, dotted, minimum width = 4cm, minimum height = 2cm, rotate = 30] at (11, - 11) {};
                \node (P)   at (11 - 0.85, - 11 - 0.85) {$\calP$};


            \draw[->] (HL2) -- (HL3) node[above, midway] {$\left(\begin{matrix} \bfcurl + \partial_{t} \\ * - \rmdiv \end{matrix}\right)$};
            \draw[->] (HL3) -- (HL4) node[above, midway] {$\rmdiv + \partial_{t}$};
            
            \draw[->, dashed] (HL2) -- (2  - 0.85, - 3 + 1.15);
            \draw[->, dashed] (HL3) -- (7  - 0.85, - 3 + 1.15);
            \draw[->, dashed] (HL4) -- (12 - 0.85, - 3 + 1.15);

            \draw[->] (HL2) -- (PsiChi) node[left, pos = 0.75] {$\bbQ_{\Psi\rmX}$};
            \draw[->, color = white] (HL3) -- coordinate[pos = 0.15](HL3a) coordinate[pos = 0.55](HL3b) (MD);
                \draw[-]                   (HL3)  -- (HL3a);
                \draw[-, color = black!30] (HL3a) -- (HL3b);
                \draw[->]                  (HL3b) -- (MD) node[left, pos = 0.444444] {$\bbQ_{\rmM\rmD}$};
            \draw[->, color = white] (HL4) -- coordinate[pos = 0.25](HL4a) coordinate[pos = 0.55](HL4b) (P*);
                \draw[-]                   (HL4)  -- (HL4a);
                \draw[-, color = black!30] (HL4a) -- (HL4b);
                \draw[->]                  (HL4b) -- (P*) node[left, pos = 0.444444] {$\bbQ_{\rmP}$};
            
            \draw[->] (Hdiv-H1)  -- (L2-H1)   node[above, pos = 0.35]  {$\rmdiv$};
            \draw[->] (Hdiv-H1)  -- (Hdiv-L2) node[above, pos = 0.55] {$\partial_{t}$};
            \draw[->] (Hcurl-L2) -- (Hdiv-L2) node[above, pos = 0.65]  {$\bfcurl$};
            \draw[->] (L2-H1)    -- (L2-L2)   node[above, pos = 0.55] {$\partial_{t}$};
            \draw[->] (Hdiv-L2)  -- (L2-L2)   node[above, pos = 0.65] {$\rmdiv$};

            \draw[->, color = white] (Hcurl-L2) -- coordinate[pos = 0.3](Psia) coordinate[pos = 0.9](Psib) (Psi);
                \draw[-]                   (Hcurl-L2) -- (Psia) node[left, pos = 0.666666] {$\bbQ_{\Psi}$};
                \draw[-, color = black!30] (Psia)     -- (Psib);
                \draw[->]                  (Psib)     -- (Psi);
            \draw[->, color = white] (Hdiv-H1)  -- coordinate[pos = 0.5](Chia) coordinate[pos = 0.8](Chib) (Chi);
                \draw[-]                   (Hdiv-H1)  -- (Chia)   node[left, pos = 0.85]     {$\bbQ_{\rmX}$};
                \draw[-, color = black!30] (Chia)     -- (Chib);
                \draw[->]                  (Chib)     -- (Chi);
            \draw[->, color = white] (Hdiv-L2)  -- coordinate[pos = 0.4](Ma)   coordinate[pos = 0.9](Mb)   (M);
                \draw[-]                   (Hdiv-L2)  -- (Ma)   node[left, pos = 0.5]      {$\bbQ_{\rmM}$};
                \draw[-, color = black!30] (Ma)       -- (Mb);
                \draw[->]                  (Mb)       -- (M);
            \draw[->, color = white] (L2-H1)    -- coordinate[pos = 0.6](Da)   coordinate[pos = 0.8](Db)   (D);
                \draw[-]                   (L2-H1)    -- (Da)   node[left, pos = 0.7]      {$\bbQ_{\rmD}$};
                \draw[-, color = black!30] (Da)       -- (Db);
                \draw[->]                  (Db)       -- (D);
            \draw[->, color = white] (L2-L2)    -- coordinate[pos = 0.7](Pa)   coordinate[pos = 0.9](Pb)   (P);
                \draw[-]                   (L2-L2)    -- (Pa)   node[left, pos = 0.285714] {$\bbQ_{\rmP}$};
                \draw[-, color = black!30] (Pa)       -- (Pb);
                \draw[->]                  (Pb)       -- (P);
            
            \draw[->] (PsiChi) -- (MD) node[above, midway] {$\left(\begin{matrix} \bfcurl + \partial_{t} \\ * - \rmdiv \end{matrix}\right)$};
            \draw[->] (MD)     -- (P*) node[above, midway] {$\rmdiv + \partial_{t}$};
            
            \draw[->, dashed] (PsiChi) -- (1  - 0.8, - 11 + 1);
            \draw[->, dashed] (MD)     -- (6  - 0.8, - 11 + 1);
            \draw[->, dashed] (P*)     -- (11 - 0.8, - 11 + 1);
            
            \draw[->] (Chi) -- (D) node[above, pos = 0.4] {$\rmdiv$};
            \draw[->] (Chi) -- (M) node[above, midway]    {$\partial_{t}$};
            \draw[->] (Psi) -- (M) node[above, pos = 0.6] {$\bfcurl$};
            \draw[->] (D)   -- (P) node[above, midway]    {$\partial_{t}$};
            \draw[->] (M)   -- (P) node[above, pos = 0.6] {$\rmdiv$};
        \end{tikzpicture}}
        \caption{Construction of momentum subcomplexes via tensor products. \BA{(Perhaps a little explanation on what this means...)}}
        \label{fig:tensor product momentum subcomplexes}
    \end{figure}

    \line

    Assuming the above subspace criteria hold, the results below follow for the weak formulation, preserving some of the structure of the strong formulation:
    
    \begin{theorem}[Gauss's Law in the Weak Formulation]
        Provided Gauss's law, $\nabla\cdot\bfB  =  0$, holds strongly at $t = T$, it holds strongly at all $t$, i.e.
        \begin{equation}
            \nabla\cdot\bfB  =  0
        \end{equation}
    \end{theorem}
    \begin{proof}
        Akin to how taking the divergence of Faraday's law, $\partial_{t}\bfB  =  - \nabla\wedge\bfE$, ensures the conservation of Gauss's law in the strong formulation, the strong satisfaction of Faraday's law similarly ensures the strong satisfaction of Gauss's law in the weak formulation.
    \end{proof}
    
    \begin{theorem}[Energy Conservation in the Weak Formulation]
        Defining the energy,
        \begin{equation}
            \rmE(t)  :=  \int_{\bfOmega\otimes\{t\}}\left(\frac{1}{2}\rho\|\bfu\|^{2} + \frac{1}{\beta}\|\bfB\|^{2} + p\right)
        \end{equation}
        provided $\calU  \leqslant  \calM$, $\calE 
         \leqslant  \calE$, and $1  \in  \calD$,
        \begin{equation}
            \rmE\left(t^{k + 1}\right) - \rmE\left(T\right)  =  \oint_{\bfGamma\otimes T}\left(- \frac{1}{2}\|\bfu\|^{2}\bfp - 2p\bfu + \frac{2}{\beta}\bfB\wedge\bfE + \frac{1}{\rmRef}\rho\bftau\cdot\bfu + \frac{1}{\rmPe}\rho\nabla\theta\right)\cdot\bfn
        \end{equation}
    \end{theorem}
    \begin{proof}
        See Appendix \ref{cha:weak conservation proofs}.
    \end{proof}

    \begin{theorem}[Helicity Conservation in the Weak Formulation]
        Defining the magnetic helicity,
        \begin{equation}
            \rmH_{\rmM}(t)  :=  \int_{\bfOmega\otimes\{t\}}\bfA\cdot\bfB  \left(=  \calH(\bfA|_{t})\tall\right)
        \end{equation}
        provided $\calB  \leqslant  \calJ$ and $\calB  \leqslant  \calE$,
        \begin{align}
            \rmH_{\rmM}\left(t^{k + 1}\right) - \rmH_{\rmM}\left(T\right)  &=  - \frac{2}{\rmRem}\int_{\bfOmega\otimes\left\{T\right\}}\bfB\cdot(\nabla\wedge\bfB) + \oint_{\bfGamma\otimes\left\{T\right\}}[\bfA\wedge(\bfE - \nabla\varphi)]\cdot\bfn  \\
            \left(\tall\right.&=  \left.- \frac{2}{\rmRem}\int_{T}\calH(\bfB) + \oint_{\bfGamma\otimes\left\{T\right\}}[\bfA\wedge(\bfE - \nabla\varphi)]\cdot\bfn\tall\right)
        \end{align}
    \end{theorem}
    \begin{proof}
        See Appendix \ref{cha:weak conservation proofs}.
    \end{proof}

    \begin{corollary}
        When $\bfB\wedge\bfn  =  \bfzero|_{\bfgamma\otimes T}$ for some non-empty $\bfgamma  \subseteq  \bfGamma$, and either the boundary condition $\bfzero  =  (\bfE - \nabla\varphi)\wedge\bfn$ or $\bfzero  =  \bfA\wedge\bfn$ holds on all of $\bfGamma\otimes T$, by Lemma \ref{lem:continuity of helicity},
        \begin{equation}
            \left|\rmH_{\rmM}\left(t^{k + 1}\right) - \rmH_{\rmM}\left(T\right)\right|  \leq  \frac{2C(\bfgamma)\left(t^{k + 1} - T\right)}{\rmRem}\|\bfj\|^{2}
        \end{equation}
    \end{corollary}

    \line
    
    \BA{TO ADD:
    \begin{itemize}
        \item  Notes about what conformity I need for my function spaces/how to handle the weak formulation for non-conforming discretizations (Talk about how to modify the weak formulation for non-conforming discretization so as not to lose my structure-preserving properties (Exact mass conservation + Faraday's law/Energy conservation + helicity dissipation + bounded change in helicity))
        \item  Talk about 2.5D case
        \item  Add further work environment for referencing to the further work chapter
        \item  Change horizontal lines to ```backslash'line'' objects
        \item  Add compressible transient weak formulation \emph{B} (with the hard-to-compute division integrals)
        \item  Write up exposition on constructing timesteppers from FEs in time (using example of dissipative discretisations of the heat equation- resultant $\frac{1}{3}$-$\frac{2}{3}$--weighted Runge--Kutta scheme must've been done in the literature some time before... what if the test space is discontinuous in time, what discretization am I getting there?)
        \item  So I'm pretty convinced the subspace criteria required for energy conservation/helicity dissipation prohibit the FE-in-time formulation from giving rise to a timestepper, since e.g. we require the test space for momentum conservation to be equal to the momentum solution space, and not of lower continuity in time, as is usually the case for constructing timesteppers from FEs in time, and I can't find a way round this. I reckon, the solution is to split transient schemes into 2 separate formulations:
        \begin{itemize}
            \item  One like I have already, with energy conservation/helicity dissipation, for periodic problems- lots of interesting maths here on the topic of e.g. robust kernel-capturing multigrid subspaces for problems in space and time.  
            \item  One that \emph{doesn't} have exact energy conservation/helicity dissipation, but has something quantifiably close, that can actually be used as a timestepper.
        \end{itemize}
        \item  How do the energy conservation/helicity dissipation results change when I introduce the coupled system with the particles? Could give rise to a grounds from which to construct the particle pusher, for structure preservation.
        \item  Add {\tt subequations} environments
        \item  Improve cross-referencing of equations
        \item  Modify formulation to use the continuous projection of the electric field in the current identity, so Ampère's law can be tested against functions in $\partial_{t}\calE$ for symmetry/sparsity without losing the preserved structures
        \item  Since my formulation is best(/only) suited for periodic problem, I might re-introduce the section where I talk about symmetric group deflation for periodic problems (nice content to talk about here with the symmetry group $S^{1}\otimes\bbR$ in a tokamak/what about the symmetry group for the Boltzmann equation? I bet there's literature on that)
    \end{itemize}}
