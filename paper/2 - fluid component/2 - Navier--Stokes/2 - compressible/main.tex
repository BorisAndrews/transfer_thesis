\subsubsection*{The \emph{Compressible} Case}
    Consider now the \emph{compressible} NS equations, in the classical strong form:
    \begin{align}
                           \partial_{t}\rho  &=  - \nabla\cdot[\rho\bfu]  \label{eqn:compressible NS mass equation}  \\
                       \rho\partial_{t}\bfu  &=  - \rho\bfu\cdot\nabla\bfu - \nabla[\rho\theta] + \frac{1}{\rmRef}\nabla\cdot[\rho\bftau]  \label{eqn:compressible NS momentum equation}  \\
        \frac{3}{2}\rho\partial_{t}\theta  &=  - \frac{3}{2}\rho\bfu\cdot\nabla\theta - \rho\theta\nabla\cdot\bfu + \frac{1}{2\rmRef}\rho\|\bftau[\bfu]\|^{2} + \frac{1}{\rmPe}\nabla\cdot[\rho\nabla\theta]  \label{eqn:compressible NS energy equation}
    \end{align}
    with homogeneous BCs that are Dirichlet in $\bfu$ ($\bfzero  =  \bfu|_{\bfGamma}$) and Neumann in $\theta$ ($0  =  \nabla\theta\cdot\bfn|_{\bfGamma}$), alongside ICs on $\rho$, $\bfu$, $\theta$.
    
    $\bftau$ is the deviatoric strain as in (\ref{eqn:strain equation}).

    \shortline

    The \emph{compressible} NS system can be seen to \emph{conserve}:
    \begin{itemize}
        \item  Energy,
        \begin{equation}\label{eqn:compressible NS energy definition}
            \rmE[\rho, \bfu, p](t)  :=  \int_{\bfOmega}\left[\frac{1}{2}\rho\|\bfu\|^{2} + \frac{3}{2}\rho\theta\right],
        \end{equation}
        with
        \begin{equation}
            \frac{\rmd}{\rmd t}\rmE  =  0.
        \end{equation}
        \item  Mass,
        \begin{equation}\label{eqn:compressible NS mass definition}
            \rmM[\rho](t)  :=  \int_{\bfOmega}\rho,
        \end{equation}
        with
        \begin{equation}
            \frac{\rmd}{\rmd t}\rmM  =  0.
        \end{equation}
    \end{itemize}

    \shortline

    Again, it is advantageous for the construction of SP timesteppers to reframe the convective component of the momentum equation (\ref{eqn:compressible NS momentum equation}) in a skew-symmetric form:
    \begin{align}
        \sqrt{\rho}\partial_{t}[\sqrt{\rho}\bfu]
        &=
        \rho\partial_{t}\bfu + \frac{1}{2}\partial_{t}\rho\bfu  \\
        &=
        - \frac{1}{2}(\nabla\cdot\left[\rho\bfu\otimes\bfu\right] + \rho\bfu\cdot\nabla\bfu) - \nabla[\rho\theta] + \frac{1}{\rmRef}\nabla\cdot[\rho\bftau[\bfu]]  \label{eqn:compressible NS skew-symmetric momentum equation}
    \end{align}
    With this in mind, take:
    \begin{align}
        \rho  =  \sigma^{2}  \label{eqn:sigma definition}  \\
        \bfr  =  \sigma\bfu  \label{eqn:r definition}
    \end{align}
    Wherein the original compressible NS system (\ref{eqn:compressible NS mass equation}--\ref{eqn:compressible NS energy equation}) was posed with ICs defined on $\rho$, $\bfu$, $\theta$---(one can interpret this as a system evolving in $\rho$, $\bfu$, $\theta$)---we propose a system with ICs defined on $\sigma$, $\bfr$, $\theta$---(one can interpret this as a system evolving in $\sigma$, $\bfr$, $\theta$, coupled with $\bfu = 1/\sigma\cdot\bfr$ to keep the system multiplicative for computation)---that is identical in the continuous case:
    \begin{align}
                                   2\sigma\partial_{t}\sigma  &=  - \nabla\cdot\left[\sigma\bfr\right]  \\
                                      \sigma\partial_{t}\bfr  &=  - \frac{1}{2}\left(\nabla\cdot\left[\sigma\bfr\otimes\bfu\right] + \sigma\bfr\cdot\nabla\bfu\right) - \nabla[\sigma^{2}\theta] + \frac{1}{\rmRef}\nabla\cdot\left[\sigma^{2}\bftau[\bfu]\right]  \\
        \frac{3}{2}\partial_{t}\left[\sigma^{2}\theta\right]  &=  - \frac{3}{2}\nabla\cdot\left[\sigma\theta\bfr\right] - \sigma^{2}\theta\nabla\cdot\bfu + \frac{1}{2\rmRef}\|\sigma\bftau[\bfu]\|^{2} + \frac{1}{\rmPe}\nabla\cdot\left[\sigma^{2}\nabla\theta\right]  \label{eqn:compressible NS sqrt energy equation}
    \end{align}

    In such a reformulation, the energy, $\rmE$, and mass, $\rmM$, are redefined as:
    \begin{align}
        \rmE[\sigma, \bfr, \theta](t)  :=  \int_{\bfOmega}\left[\frac{1}{2}\|\bfr\|^{2} + \frac{3}{2}\sigma^{2}\theta\right],  &&
                      \rmM[\sigma](t)  :=  \int_{\bfOmega}\sigma^{2}.
    \end{align}

    \begin{remark}[Alternative SP weak formulations]
        While I can \emph{theoretically} construct SP timesteppers for the above system in $\sigma$, $\bfr$, $\theta$ and $\bfu$ only:
        \begin{itemize}
            \item  They have a very multiplicative order, which I suspect (although I haven't yet had the chance to run any tests) will vastly reduce the timestepper's computational efficiency.
            \item  The $\sigma^{2}\partial_{t}\theta$ term in (\ref{eqn:compressible NS sqrt energy equation}) has no nice closed form in a CG-DG-in-time discretization, such as was presented for the incompressible case above, in terms of parameters from the GL polynomials, as far as I can see.
        \end{itemize}
        It is for this reason that, instead of considering the system in the fewest variables, (\ref{eqn:r definition}--\ref{eqn:compressible NS sqrt energy equation}) in $\sigma$, $\bfr$, $\theta$ and $\bfu$, I am going to elect to consider one, for now, in more variables than is needed, by re-introducing $\rho = \sigma^{2}$. It is my hope that this will benefit, not hinder, the timestepper's computational efficiency.
    \end{remark}

    Consider a system with ICs defined on $\rho$, $\bfr$, $\theta$--(one can interpret this as a system evolving in $\rho$, $\bfr$, $\theta$, coupled with $\bfu = 1/\sigma\cdot\bfr$ \emph{and} $\sigma = \sqrt{\rho}$)---that is \emph{again} identical in the continuous case:
    \begin{align}
                         \partial_{t}\rho  &=  - \nabla\cdot\bfp  \label{eqn:compressible NS final mass equation}  \\
                   \sigma\partial_{t}\bfr  &=  - \frac{1}{2}\left(\nabla\cdot\left[\bfp\otimes\bfu\right] + \bfp\cdot\nabla\bfu\right) - \nabla[\rho\theta] + \frac{1}{\rmRef}\nabla\cdot\left[\rho\bftau[\bfu]\right]  \\
        \frac{3}{2}\rho\partial_{t}\theta  &=  - \frac{3}{2}\bfp\cdot\nabla\theta - \rho\theta\nabla\cdot\bfu + \frac{1}{2\rmRef}\rho\|\bftau[\bfu]\|^{2} + \frac{1}{\rmPe}\nabla\cdot\left[\rho\nabla\theta\right]  \label{eqn:compressible NS final energy equation}
    \end{align}
    where the momentum, $\bfp$, can be defined as either $\bfp := \sigma\bfr$ or $\bfp := \rho\bfu$.

    In such a reformulation, the energy, $\rmE$, and mass, $\rmM$, are redefined as:
    \begin{align}
        \rmE[\rho, \bfr, \theta](t)  :=  \int_{\bfOmega}\left[\frac{1}{2}\|\bfr\|^{2} + \frac{3}{2}\rho\theta\right],  &&
                      \rmM[\rho](t)  :=  \int_{\bfOmega}\rho.
    \end{align}

    \line

    We use the auxiliary function concept to convert (\ref{eqn:compressible NS final mass equation}--\ref{eqn:compressible NS final energy equation}, \ref{eqn:sigma definition}--\ref{eqn:r definition}) into a weak form in space-time that invokes an energy-conservative SP timestepper. Letting $\rho$, $\bfr$, $\theta$ each contain a lifting of the ICs, one can seek:
    \begin{equation}
        \rho_{-}    \in  \calD_{-},   \qquad
        \bfr_{-}    \in  \calR_{-},   \qquad
        \theta_{-}  \in  \Theta_{-};  \qquad
        \sigma      \in  \Sigma,      \qquad
        \bfu        \in  \calU,
    \end{equation}
    such that:
    \begin{align}
            (\partial_{t}\rho
            =)
            \bbQ_{\partial_{t}\calD}[\partial_{t}\rho]
            &=
            \bbQ_{\partial_{t}\calD}[- \nabla\cdot\bfp_{*}]  \label{eqn:compressible NS weak mass equation}  \\
            \bbQ_{\calU}[\sigma\partial_{t}\bfr]
            &=
            \bbQ_{\calU}\!\left[- \frac{1}{2}(\nabla\cdot\left[\bfp_{*}\otimes\bfu\right] + \bfp_{*}\cdot\nabla\bfu)
            - \nabla[\rho_{*}\theta_{*}]
            + \frac{1}{\rmRef}\nabla\cdot[\rho_{*}\bftau[\bfu]]\right]  \\
        \begin{split}
            \bbQ_{\partial_{t}\Theta}\left[\frac{3}{2}\bbQ_{\partial_{t}\Theta}[\rho]\partial_{t}\theta\right]
            &=
            \bbQ_{\partial_{t}\Theta}\!\left[- \frac{3}{2}\bfp_{*}\cdot\nabla\bbQ_{\partial_{t}\calD}[\theta]
            - \rho_{*}\theta_{*}\nabla\cdot\bfu\right.  \\
            &\qquad\qquad\qquad\qquad\qquad\left.+ \frac{1}{2\rmRef}\rho_{*}\|\bftau[\bfu]\|^{2}
            + \frac{1}{\rmPe}\nabla\cdot[\rho_{*}\!\nabla\theta_{*}]\right]
        \end{split}  \label{eqn:compressible NS weak energy equation}  \\
            \bbQ_{\Sigma}[\rho_{*}]
            &=
            \bbQ_{\Sigma}\!\left[\sigma^{2}\right]  \\
            \bbQ_{\partial_{t}\calR}[\bfr]
            &=
            \bbQ_{\partial_{t}\calR}[\sigma\bfu]
    \end{align}
    where:
    \begin{itemize}
        \item  $\rho_{*}$, $\theta_{*}$ can be taken as either $\rho$, $\theta$ or projections thereof
        \item  $\bfp_{*}$ represents the momentum, which one represent in any user-defined form in terms of $\sigma$, $\rho$, $\bfu$, $\bfr$ and projections thereof
    \end{itemize}
    without affecting the resultant timestepper's SP properties.

    \line

    Using the weak form (\ref{eqn:compressible NS weak mass equation}--\ref{eqn:compressible NS weak energy equation}) one can mimic the proofs of:
    \begin{itemize}
        \item  Energy conservation, provided $1 \in \partial_{t}\Theta$:
        \begin{align}
                \rmE\left(t^{n + 1}\right) - \rmE\left(t^{n}\right)
                &=  \int_{T^{n}}\frac{\rmd}{\rmd t}\rmE  \\
                &=  \int_{T^{n}}\frac{\rmd}{\rmd t}\int_{\bfOmega}\left[\frac{1}{2}\|\bfr\|^{2} + \frac{3}{2}\rho\theta\right]  \\
                &=  \int_{\bfOmega\otimes T^{n}}\left[\bfr\cdot\partial_{t}\bfr + \frac{3}{2}(\partial_{t}\rho\theta + \rho\partial_{t}\theta)\right]  \\
                &=  \int_{\bfOmega\otimes T^{n}}\left[\sigma\bfu\cdot\partial_{t}\bfr - \frac{3}{2}\theta\bbQ_{\partial_{t}\calD}[\nabla\cdot\bfp_{*}] + \frac{3}{2}\bbQ_{\partial_{\Theta}}[\rho]\partial_{t}\theta\right]  \\
            \begin{split}
                &=  \int_{\bfOmega\otimes T^{n}}\left[\bfu\cdot\left(- \frac{1}{2}(\cdots) - \mst{\nabla[\rho_{*}\theta_{*}]} + \mst{\frac{1}{\rmRef}\nabla\cdot[\rho_{*}\bftau[\bfu]]}\right)\right.  \\
                &\qquad\qquad\qquad- \mst{\frac{3}{2}\bbQ_{\partial_{t}\calD}[\theta]\nabla\cdot\bfp_{*}} + \left(- \mst{\frac{3}{2}\bfp_{*}\cdot\nabla\bbQ_{\partial_{t}\calD}[\theta]}\right.  \\
                &\qquad\qquad\left.\left.- \mst{\rho_{*}\theta_{*}\nabla\cdot\bfu} + \mst{\frac{1}{2\rmRef}\rho_{*}\|\bftau[\bfu]\|^{2}} + \frac{1}{\rmPe}\mst{\nabla\cdot[\rho_{*}\nabla\theta_{*}]}\right)\right]
            \end{split}  \\
                &=  \mst{\calC\left[\bfp_{*}; \bfu, \bfu\right]}  \\
                &=  0
        \end{align}

        \item  Mass conservation, provided $1 \in \partial_{t}\calD$:
        \begin{align}
                \rmM\left(t^{n + 1}\right) - \rmM\left(t^{n}\right)
                &=  \int_{T^{n}}\frac{\rmd}{\rmd t}\rmM  \\
                &=  \int_{T^{n}}\frac{\rmd}{\rmd t}\int_{\bfOmega}\rho  \\
                &=  \int_{\bfOmega\otimes T^{n}}\partial_{t}\rho  \\
                &=  - \int_{\bfOmega\otimes T^{n}}\mst{\nabla\cdot[\rho_{*}\bfu_{*}]}  \\
                &=  0
        \end{align}
    \end{itemize}
    Thus, the resultant timestepper preserves both the energy- and mass-conservative structures.

    \shortline

    If the following $\Lambda_{0}^{\bullet}(\bfOmega)\otimes L^{2}(T^{n})$-subcomplex diagram commutes for some momentum-like space $\calP$:
    \begin{center}\begin{tikzpicture}[align = center, node distance = 4cm, auto]
        \node (Hdiv) at (0,   0) {$\bfH(\rmdiv)\otimes L^{2}$};
        \node (L2)   at (4.5, 0) {$L^{2}\otimes L^{2}$};
        \draw[->] (Hdiv) -- (L2) node[above, midway] {$\rmdiv$};

        \node (P)  at (0,   - 2) {$\calP$};
        \node (Dt) at (4.5, - 2) {$\partial_{t}\calD$};
        \draw[->] (P) -- (Dt) node[above, midway] {$\rmdiv$};

        \draw[->] (Hdiv) -- (P)  node[left, midway] {$\bbQ_{\calP}$};
        \draw[->] (L2)   -- (Dt) node[left, midway] {$\bbQ_{\partial_{t}\calD}$};
    \end{tikzpicture}\end{center}
    then (\ref{eqn:compressible NS weak mass equation}) states in some sense a compressible analogue to the classical exact incompressibility condition:
    \begin{equation}\label{eqn:weak incompressibility}
        \partial_{t}\rho  =  - \nabla\cdot\bbQ_{\calP}[\bfp_{*}]
    \end{equation}

    \line

    \begin{example}[CG-DG-in-time conservative timestepper for compressible NS]
        Suppose the spaces take the following CG-DG-in-time tensor product forms, for some order in time, $s$:
        \begin{align}
            \calD  =  \Theta  =  \widehat{\calD}(\bfOmega)\otimes\bbP^{s}\left(T^{n}\right),  &&
                       \calR  =  \widehat{\calR}(\bfOmega)\otimes\bbP^{s}\left(T^{n}\right),
        \end{align}
        \vspace{- 10mm}
        \begin{align}
                      \Sigma  =  \widehat{\Sigma}(\bfOmega)\otimes\bbDP^{s - 1}\left(T^{n}\right),  &&
                       \calU  =  \widehat{\calU}(\bfOmega)\otimes\bbDP^{s - 1}\left(T^{n}\right).
        \end{align}

        Let $\left(\tau^{n}_{i}\right)_{i \in \{1, \cdots, s\}}$ denote the order-$s$ (shifted) GL points on $T^{n}$, and $\left(\phi^{n}_{j}\right)_{j \in \{1, \cdots, s\}}$ the basis for $\bbDP^{s - 1}\left(T^{n}\right)$ of associated GL polynomials, satisfying $\phi^{n}_{j}(\tau^{n}_{i}) = \delta_{ij}$. $\bbQ_{\partial_{t}\calD}[\rho]$, $\bbQ_{\partial_{t}\calR}[\bfr]$, $\bbQ_{\partial_{t}\Theta}[\theta]$, $\sigma$, $\bfu$ can each be represented in terms of the GL basis as:
        \begin{align}
            \bbQ_{\partial_{t}\calD}[\rho]     =  \sum_{i}\rho_{i}^{n}\phi_{i}^{n},  &&
            \bbQ_{\partial_{t}\calR}[\bfr]     =  \sum_{i}\bfr_{i}^{n}\phi_{i}^{n},  &&
            \bbQ_{\partial_{t}\Theta}[\theta]  =  \sum_{i}\theta_{i}^{n}\phi_{i}^{n},
        \end{align}
        \vspace{- 7mm}
        \begin{align}
            \sigma = \sum_{i}\sigma_{i}^{n}\phi_{i}^{n},  &&
            \bfu   = \sum_{i}\bfu_{i}^{n}\phi_{i}^{n},
        \end{align}
        where $*_{i}^{n} = *|_{\tau_{i}^{n}}$, while the derivatives $\partial_{t}\rho$, $\partial_{t}\bfr$, $\partial_{t}\theta$ can be written as:
        \begin{align}
            \partial_{t}\rho    =  \sum_{j}\delta\rho_{j}^{n}\phi_{j}^{n},  &&
            \partial_{t}\bfr    =  \sum_{j}\bfdelta\bfr_{j}^{n}\phi_{j}^{n},  &&
            \partial_{t}\theta  =  \sum_{j}\delta\theta_{j}^{n}\phi_{j}^{n},
        \end{align}
        with updates at $t^{n + 1}$ given as:
        \begin{align}
            \rho|_{t^{n + 1}}    =  \rho|_{t^{n}} + \delta t^{n}\sum_{j}b_{j}\delta\rho_{j}^{n},  &&
            \bfr|_{t^{n + 1}}    =  \rho|_{t^{n}} + \delta t^{n}\sum_{j}b_{j}\bfdelta\bfr_{j}^{n},  &&
            \theta|_{t^{n + 1}}  =  \rho|_{t^{n}} + \delta t^{n}\sum_{j}b_{j}\delta\theta_{j}^{n},
        \end{align}
        for $(b_{j})_{j}$ defined as in (\ref{eqn:b_j definition}). $\rho$, $\bfr$, $\theta$ are given at the GL points as
        \begin{align}
            \rho_{i}^{n}    =  \rho|_{t^{n}}   + \delta t^{n}\sum_{j}a_{ij}\delta\rho^{n}_{j},  &&
            \bfr_{i}^{n}    =  \bfr|_{t^{n}}   + \delta t^{n}\sum_{j}a_{ij}\bfdelta\bfr^{n}_{j},  &&
            \theta_{i}^{n}  =  \theta|_{t^{n}} + \delta t^{n}\sum_{j}a_{ij}\delta\theta^{n}_{j},
        \end{align}
        for $(a_{ij})_{ij}$ defined as in (\ref{eqn:a_ij definition}).

        Choosing $\rho_{*} = \bbQ_{\partial_{t}\calD}[\rho]$, $\theta_{*} = \bbQ_{\partial_{t}\Theta}[\theta]$ and $\bfp_{*} = \bbQ_{\partial_{t}\calD}[\rho]\bfu$, the resultant can then be stated in the weak form, $\forall i$:
        \begin{align}
                (\delta\rho_{i}^{n}
                =)
                \bbQ_{\widehat{\calD}}\!\left[\delta\rho_{i}^{n}\right]
                &=
                \bbQ_{\widehat{\calD}}\!\left[- \sum_{jk}w_{i}^{jk}\nabla\cdot\left[\rho_{j}^{n}\bfu_{k}^{n}\right]\right]  \\
            \begin{split}
                \bbQ_{\widehat{\calU}}\!\left[\sum_{jk}w_{i}^{jk}\sigma_{j}^{n}\bfdelta\bfr_{k}^{n}\right]
                &=
                \bbQ_{\widehat{\calU}}\!\left[- \frac{1}{2}\sum_{jkl}w_{i}^{jkl}\left(\nabla\cdot\left[\rho_{j}^{n}\bfu_{k}^{n}\otimes\bfu_{l}^{n}\right] + \rho_{j}^{n}\bfu_{k}^{n}\cdot\nabla\bfu_{k}^{n}\right)\right.  \\
                &\qquad\qquad\qquad\left.- \sum_{jk}w_{i}^{jk}\nabla\!\left[\rho_{j}^{n}\theta_{k}^{n}\right]
                + \frac{1}{\rmRef}\sum_{jk}w_{i}^{jk}\nabla\cdot\left[\rho_{j}^{n}\bftau\!\left[\bfu_{k}^{n}\right]\right]\right]
            \end{split}  \\
            \begin{split}
                \bbQ_{\widehat{\Theta}}\!\left[\frac{3}{2}\sum_{jk}w_{i}^{jk}\rho_{j}^{n}\delta\theta_{k}^{n}\right]
                &=
                \bbQ_{\widehat{\Theta}}\!\left[\sum_{jkl}w_{i}^{jkl}\left(- \frac{3}{2}\rho_{j}^{n}\bfu_{k}^{n}\cdot\nabla\theta_{l}^{n}
                - \rho_{j}^{n}\theta_{l}^{n}\nabla\cdot\bfu_{k}^{n}\right)\right.  \\
                &\qquad\qquad\qquad+ \frac{1}{2\rmRef}\sum_{jkl}w_{i}^{jkl}\rho_{j}^{n}\bftau\!\left[\bfu_{k}^{n}\right]:\bftau\!\left[\bfu_{l}^{n}\right]  \\
                &\qquad\qquad\qquad\qquad\qquad\qquad\left.+ \frac{1}{\rmPe}\sum_{jk}w_{i}^{jk}\nabla\cdot\left[\rho_{j}^{n}\nabla\theta_{k}^{n}\right]\right]
            \end{split}  \\
                \bbQ_{\widehat{\Sigma}}[\rho_{i}^{n}]
                &=
                \bbQ_{\widehat{\Sigma}}\!\left[\sum_{jk}w_{i}^{jk}\sigma_{j}^{n}\sigma_{k}^{n}\right]  \\
                \left(\bfr_{i}^{n}
                =\right)
                \bbQ_{\widehat{\calR}}\!\left[\bfr_{i}^{n}\right]
                &=
                \bbQ_{\widehat{\calR}}\!\left[\sum_{jk}w_{i}^{jk}\sigma_{j}^{n}\bfu_{k}^{n}\right]
        \end{align}
        where $\left(w_{i}^{jk}\right)_{i}^{jk}$ are defined as in (\ref{eqn:w_i^jk definition}) and $\left(w_{i}^{jkl}\right)_{i}^{jkl}$ are defined similarly:
        \begin{equation}
            w_{i}^{jkl}  :=  \frac{\int_{t^{n}}^{t^{n + 1}}\phi^{n}_{i}\phi^{n}_{j}\phi^{n}_{k}\phi^{n}_{l}}{\int_{t^{n}}^{t^{n + 1}}(\phi^{n}_{i})^{2}}
        \end{equation}
        When $s = 1$, $w_{i}^{jkl} = \delta_{ij}\delta_{ik}\delta_{il}$, such that this system reduces to exactly the GL method, i.e. Crank--Nicolson. For $s \geq 2$ however, $w_{i}^{jkl} \neq \delta_{ij}\delta_{ik}\delta_{il}$ implying the higher order timestepper differs from the direct, classical GL method, in the convective and viscous heating terms.
    \end{example}

    \line

    \begin{remark}[Positivity of $\rho$, $\theta$ (and $\sigma$)]
        There is another set of local structures in the compressible Navier--Stokes system that I have not touched upon here: the necessary positivity of $\rho$, $\theta$ (and $\sigma$). I am not even sure what would happen to the timestepper if they \emph{did} drop below $0$. I know this question is one that is tackled in classical timestepping schemes in compressible fluid dynamics, although I have not yet studied this area. If all other ideas fail, a timestep can always be chosen that is sufficiently short to ensure this unphysical result does not occur, although it would be very unsatisfying if that is the best solution. 
        
        This is another reasons for which I am more concerned about the compressible scheme, and a problem for future work.
    \end{remark}
    