\subsubsection*{The \emph{Incompressible} Case}
    Consider the \emph{incompressible} Navier--Stokes (NS) equations, in the classical strong form:
    \begin{align}
                       0  &=  \nabla\cdot\bfu,  \label{eqn:incompressible NS mass conservation}  \\
        \partial_{t}\bfu  &=  - \bfu\cdot\nabla\bfu - \nabla p + \frac{1}{\rmRe}\Delta\bfu,  \label{eqn:incompressible NS classical momentum conservation}
    \end{align}
    under homogeneous Dirichlet BCs in $\bfu$ ($\bfzero  =  \bfu|_{\bfGamma}$) with ICs on $\bfu$.
    
    The incompressible NS system can be seen to dissipate (kinetic) energy,
    \begin{equation}\label{eqn:incompressible NS energy definition}
        \rmE[\bfu](t)  :=  \frac{1}{2}\int_{\bfOmega}\|\bfu\|^{2},
    \end{equation}
    with
    \begin{equation}
        \frac{\rmd}{\rmd t}\rmE  =  - \frac{2}{\rmRe}\int_{\bfOmega}\|\nabla_{\rms}\bfu\|^{2}  \leq  0.
    \end{equation}

    By incompressibility, the momentum equation (\ref{eqn:incompressible NS classical momentum conservation}) can be rewritten in the following ``skew-symmetric'' form, to ease the analysis and to motivate a weak form that inherits the energy-dissipative structure,
    \begin{equation}\label{eqn:incompressible NS skew-symmetric momentum equation}
        \partial_{t}\bfu  =  - \frac{1}{2}\left(\nabla\cdot\left[\bfu^{\otimes 2}\right] + \bfu\cdot\nabla\bfu\right) - \nabla p + \frac{2}{\rmRe}\nabla\cdot\nabla_{\rms}\bfu,
    \end{equation}
    since:
    \begin{align}
        &\nabla\cdot\left[\bfu^{\otimes 2}\right]   &&=  (\mst{\nabla\cdot\bfu})\bfu + \bfu\cdot\nabla\bfu  &&=  \bfu\cdot\nabla\bfu  \\
        &\nabla\cdot\left[\nabla\bfu^{\rmT}\right]  &&=  \nabla[\mst{\nabla\cdot\bfu}]                      &&=  0
    \end{align}
    where $\nabla_{\rms}$ again denotes the symmetric gradient.

    \begin{remark}
        Alternative skew-symmetric forms for the convective term, such as those incorporating the vorticity $\bfomega  :=  \nabla\wedge\bfu$, exist. While these too induce timesteppers that preserve the energy-dissipative structure, we choose not to consider them further here, since we will be considering a similar skew-symmetric form of the convective operator for the \emph{compressible} case in the following subsection.
    \end{remark}

    \begin{remark}
        \BA{If the weak form has exact incomp., then it doesn't matter how you write the convective term (classical/skew-symmetric etc.) since it shouldn't affect the solution. Only the skew-symmetric form gives energy-conserving timesteppers even without exact incomp. A similar result holds true for the \emph{compressible} case.}
    \end{remark}

    We can again use the auxiliary function idea from the heat equation to convert (\ref{eqn:incompressible NS mass conservation}, \ref{eqn:incompressible NS skew-symmetric momentum equation}) into a weak form in space-time that invokes an energy-dissipative SP timestepper. Letting $\bfu$ similarly contain a lifting of the ICs, one can seek $\bfu_{-} \in \calU_{-}$, $p \in \calP$ such that:
    \begin{align}
            0
            &=  \bbQ_{\calP}[\nabla\cdot\bbQ_{\partial_{t}\calU}[\bfu]]  \label{eqn:incompressible NS weak form 1}  \\
        \begin{split}
            (\partial_{t}\bfu
            =)  \bbQ_{\partial_{t}\calU}[\partial_{t}\bfu]
            &=  \bbQ_{\partial_{t}\calU}\left[- \frac{1}{2}(\nabla\cdot\left[\bfu_{*}\otimes\bbQ_{\partial_{t}\calU}[\bfu]\right]
            + \bfu_{*}\cdot\nabla\bbQ_{\partial_{t}\calU}[\bfu])\right.  \\
            &\qquad\qquad\qquad\qquad\qquad\qquad\qquad\left.- \nabla p
            + \frac{2}{\rmRe}\nabla\cdot\nabla_{\rms}\bbQ_{\partial_{t}\calU}[\bfu]\right]
        \end{split}  \label{eqn:incompressible NS weak form 2}
    \end{align}
    We write $\bfu_{*}$ here to denote \emph{either} $\bfu$ \emph{or} the auxiliary projection $\bbQ_{\partial_{t}\calU}[\bfu]$. The energy-dissipative SP property of the resultant timestepper is indepedent of this choice. In variational form, one can seek $\bfu_{-} \in \calU_{-}$, $\widetilde{\bfu} \in \partial_{t}\calU$, $p \in \calP$ such that:
    \begin{align}
        &\forall                 q  \in  \calP,              &                                                            0  &=  \left\langle q, \nabla\cdot\widetilde{\bfu}\right\rangle  \\
        &\forall  \widetilde{\bfv}  \in  \partial_{t}\calU,  &  \left\langle\widetilde{\bfv}, \partial_{t}\bfu\right\rangle  &=  \calA\left[\bfu_{*}; \widetilde{\bfv}, \widetilde{\bfu}\right] + \left\langle\nabla\cdot\widetilde{\bfv}, p\right\rangle - \frac{2}{\rmRe}\left\langle\nabla_{\rms}\widetilde{\bfv}, \nabla_{\rms}\widetilde{\bfu}\right\rangle  \\
        &\forall              \bfv  \in  \partial_{t}\calU,  &                                                            0  &=  \left\langle\bfv, \widetilde{\bfu}\right\rangle - \left\langle\bfv, \bfu\right\rangle
    \end{align}
    where similarly all inner products $\langle -, -\rangle  =  \langle -, -\rangle_{\bfOmega\otimes T^{n}}$ are taken over the space-time domain, $\bfOmega\otimes T^{n}$, and $\calA[-; -, -]$ is the trilinear convective operator, defined
    \begin{align}
        \calA\left[\bfu_{*}; \widetilde{\bfv}, \widetilde{\bfu}\right]  :=  &\;\frac{1}{2}\left(\left\langle\nabla\widetilde{\bfv}, \widetilde{\bfu}\otimes\bfu_{*}\right\rangle - \left\langle\widetilde{\bfv}, \bfu_{*}\cdot\nabla\widetilde{\bfu}\right\rangle\right)  \label{eqn:convective operator definition}  \\
                                                                         =  &\;\frac{1}{2}\int_{\bfOmega\otimes T}\bfu_{*}\cdot\left(\nabla\widetilde{\bfv}\cdot\widetilde{\bfu} - \nabla\widetilde{\bfu}\cdot\widetilde{\bfv}\right)
    \end{align}
    This has the crucial skew-symmetry property $\calA\left[\bfu_{*}; \widetilde{\bfu}, \widetilde{\bfu}\right]  =  0$, $\forall \bfu_{*}, \widetilde{\bfu}$.

    \begin{remark}
        \BA{Crucial that non-$\bfgrad$-conforming discretizations preserve this skew-symmetry property for energy dissipation! (In fact, is suff. that $\calA  \leq  0$, such that various \emph{dissipative} fluxes (e.g. upwinding) can be considered.)}
    \end{remark}

    One can again use the weak form (\ref{eqn:incompressible NS weak form 1}--\ref{eqn:incompressible NS weak form 2}) directly to mimic the proof of energy dissipation:
    \begin{align}
        \rmE\left(t^{n + 1}\right) - \rmE\left(t^{n}\right)  &=  \int_{T^{n}}\frac{\rmd}{\rmd t}\rmE  \\
        &=  \int_{T^{n}}\frac{\rmd}{\rmd t}\left[\frac{1}{2}\int_{\bfOmega}\|\bfu\|^{2}\right]  \\
        &=  \int_{\bfOmega\otimes T^{n}}\bfu\cdot\partial_{t}\bfu  \\
        &=  \int_{\bfOmega\otimes T^{n}}\bfu\cdot\bbQ_{\partial_{t}\calU}\left[- \frac{1}{2}\nabla\cdot\left[\bfu_{*}\otimes\bbQ_{\partial_{t}\calU}[\bfu]\right] - \cdots + \frac{2}{\rmRe}\nabla\cdot\nabla_{\rms}\bbQ_{\partial_{t}\calU}[\bfu]\right]  \\
        &=  \int_{\bfOmega\otimes T^{n}}\left[\mst{\calA\left[\bfu_{*}; \bbQ_{\partial_{t}\calU}[\bfu], \bbQ_{\partial_{t}\calU}[\bfu]\right]} + \nabla\cdot\bbQ_{\partial_{t}\calU}[\bfu]p - \frac{2}{\rmRe}\|\nabla_{\rms}\bbQ_{\partial_{t}\calU}[\bfu]\|^{2}\right]  \\
        &=  \int_{\bfOmega\otimes T^{n}}\left[\mst{\bbQ_{\calP}[\nabla\cdot\bbQ_{\partial_{t}\calU}[\bfu]]}p - \frac{2}{\rmRe}\|\nabla_{\rms}\bbQ_{\partial_{t}\calU}[\bfu]\|^{2}\right]  \\
        &=  - \frac{2}{\rmRe}\int_{\bfOmega\otimes T^{n}}\|\nabla_{\rms}\bbQ_{\partial_{t}\calU}[\bfu]\|^{2}  \leq  0  \label{eqn:incompressible NS weak form dissipation}
    \end{align}
    Thus, the resultant timestepper preserves the energy-dissipative structure.

    Similar to the classical stationary-state case, if the following subcomplex diagram commutes,
    \begin{center}\begin{tikzpicture}[align = center, node distance = 4cm, auto]
        \node (Hdiv) at (0,   0) {$\bfH(\rmdiv)$};
        \node (L2)   at (3.5, 0) {$L^{2}$};
        \draw[->] (Hdiv) -- (L2) node[above, midway] {$\rmdiv$};

        \node (Ut) at (0,   - 2) {$\partial_{t}\calU$};
        \node (P)  at (3.5, - 2) {$\calP$};
        \draw[->] (Ut) -- (P) node[above, midway] {$\rmdiv$};

        \draw[->] (Hdiv) -- (Ut) node[left, midway] {$\bbQ_{\partial_{t}\calU}$};
        \draw[->] (L2)   -- (P)  node[left, midway] {$\bbQ_{\calP}$};
    \end{tikzpicture}\end{center}
    then (\ref{eqn:incompressible NS weak form 1}) directly gives pointwise incompressibility on the \emph{auxiliary} space, $\bbQ_{\partial_{t}\calU}[\bfu]$,
    \begin{equation}
        \nabla\cdot\bbQ_{\partial_{t}\calU}[\bfu]  =  0.
    \end{equation}

    \begin{remark}
        \emph{Arguably}, this exact incompressibility condition on the \emph{auxiliary} space makes more sense than that on $\bfu$ itself, $\nabla\cdot\bfu = 0$, since the possibility of this condition being satisfied on $\bfu$ relies on it being satisfied on the ICs of $\bfu$, $\nabla\cdot\bfu = 0|_{t = t^{n}}$, too.
        
        I suspect that, when $\calU$ is a tensor product space and the incompressibility IC $\nabla\cdot\bfu  =  0|_{t = t^{n}}$ holds pointwise, the above result, $\nabla\cdot\bbQ_{\partial_{t}\calU}[\bfu]  =  0$, is sufficient to prove pointwise incompressible on $\bfu$ itself, $\nabla\cdot\bfu = 0$. Indeed this is true simply by dimensionality when the time component of such a space-time tensor product construction for $\calU$ is finite-dimensional, as is the case after discretization. I believe it should hold in the infinite-dimensional case too, but I've yet to sort the technicalities of such a proof. Perhaps some new commutation property is necessarily satisfied in the case of tensor product $\calU$?

        The closest I've got is in noting the following:
        \begin{align}
            0  &=  \int_{\bfOmega\otimes T^{n}}(\nabla\cdot\partial_{t}\bfu)(\nabla\cdot\bbQ_{\partial_{t}\calU}[\bfu])  \\
            &=  \int_{\bfOmega\otimes T^{n}}\nabla\cdot\bbQ_{\partial_{t}\calU}[\partial_{t}\bfu](\nabla\cdot\bfu)  &&\impliedby  \rmdiv\circ\bbQ_{\partial_{t}\calU} = \bbQ_{\calP}\circ\rmdiv  \\
            &=  \int_{\bfOmega\otimes T^{n}}(\nabla\cdot\partial_{t}\bfu)(\nabla\cdot\bfu)  \\
            &=  \frac{1}{2}\int_{\bfOmega\otimes T^{n}}\partial_{t}\left[(\nabla\cdot\bfu)^{2}\right]  \\
            \|\nabla\cdot\bfu\|_{\bfOmega\otimes\left\{t^{n + 1}\right\}}^{2}  &=  
            \|\nabla\cdot\bfu\|_{\bfOmega\otimes\left\{t^{n}\right\}}^{2}
        \end{align}
        Obviously then $\nabla\cdot\bfu  =  0|_{t^{n}}  \implies  \nabla\cdot\bfu  =  0|_{t^{n + 1}}$, however it'd be nice to show this holds pointwise within the timestep interior, $\bfOmega\otimes\left(t^{n}, t^{n + 1}\right)$, too.
    \end{remark}

    When using a tensor product space for $\calU$ in the heat equation, we noted that above that the auxiliary function to be computed, $\bbQ_{\partial_{t}\calU}[u]$, could be eliminated from the weak form (\ref{eqn:heat equation auxiliary weak form}). Due to the nonlinearity of the Navier--Stokes system, this is no longer possible here. An alternative possibility however is available. Since, for tensor product spaces, $\bbQ_{\partial_{t}\calU}$ acts on $\calU$ only in the time component, $\bbQ_{\partial_{t}\calU}[\bfu]$ can be computed offline in such cases $\forall \bfu (\in \calU)$. To illustrate this, consider the following CG-DG-in-time example discretization:
    
    \BA{Can $\partial_{t}\calU$ ever be a FE space if $\calU$ is not a tensor product space?}

    \begin{example}[CG-DG-in-time dissipative timestepper for incompressible NS]
        Suppose $\calU$, $\calP$ take the following CG-DG-in-time tensor product forms:
        \begin{align}
            \calU  =  \widehat{\calU}(\bfOmega)\otimes\bbP^{s}\left(T^{n}\right),  &&
            \calP  =  \widehat{\calP}(\bfOmega)\otimes\bbDP^{s - 1}\left(T^{n}\right).
        \end{align}

        We make use of GL polynomials to write the resultant timestepper in a concise form, reminiscent of the classical GL RK method. Let $\left(\tau^{n}_{i}\right)_{i \in \{1, \cdots, s\}}$ denote the order-$s$ (shifted) GL points on $T^{n}$, and $\left(\phi^{n}_{j}\right)_{j \in \{1, \cdots, s\}}$ in $\bbDP^{s - 1}\left(T^{n}\right)$ the associated GL polynomials, satisfying $\phi^{n}_{j}(\tau^{n}_{i}) = \delta_{ij}$.

        With $\partial_{t}\bfu \in \partial_{t}\calU = \widehat{\calU}\otimes\bbDP^{s - 1}$, one can represent $\partial_{t}\bfu$ in terms of the GL polynomials as
        \begin{equation}
            \partial_{t}\bfu  =  \sum_{j}\bfdelta\bfu^{n}_{j}\phi^{n}_{j},
        \end{equation}
        such that the update $\bfu|_{t^{n + 1}}$ is given in the style of a RK method as
        \begin{equation}
            \bfu|_{t^{n + 1}}  =  \bfu|_{t^{n}} + \delta t^{n}\sum_{j}b_{j}\bfdelta\bfu^{n}_{j},
        \end{equation}
        where
        \begin{equation}
            b_{j}  :=  \frac{1}{\delta t^{n}}\int_{t^{n}}^{t^{n + 1}}\phi^{n}_{j}.
        \end{equation}

        With $\bbQ_{\partial_{t}\calU}[\bfu] \in \partial_{t}\calU = \widehat{\calU}\otimes\bbDP^{s - 1}$, one can use a similar GL representation for $\bbQ_{\partial_{t}\calU}[\bfu]$. Since projection is equivalent to GL interpolation in $\bbP^{s}$,
        \begin{equation}
            \bbQ_{\partial_{t}\calU}[\bfu]  =  \sum_{i}\bfu|_{\tau^{n}_{i}}\phi^{n}_{i}.
        \end{equation}
        For computational application, each $\bfu|_{\tau^{n}_{i}}$ can similarly be written in terms of $(\bfdelta\bfu^{n}_{j})_{j}$ as
        \begin{equation}
            \bfu|_{\tau^{n}_{i}}  =  \bfu|_{t^{n}} + \delta t^{n}\sum_{j}a_{ij}\bfdelta\bfu^{n}_{j},
        \end{equation}
        where
        \begin{equation}
            a_{ij}  :=  \frac{1}{\delta t^{n}}\int_{t^{n}}^{\tau^{n}_{i}}\phi^{n}_{j}.
        \end{equation}
        
        With $\tau^{n}_{i} = t^{n} + c_{i}\delta t^{n}$, $\sfA$, $\bfb$, $\bfc$ share definitions with those in the classical GL RK method. (See Subsection \ref{cha:RK methods})

        Similarly, write $p \in \widehat{P}\otimes\bbDP^{s - 1}$ in terms of the GL polynomial basis as
        \begin{equation}
            p  =  \sum_{i}p|_{\tau^{n}_{i}}\phi^{n}_{i}.
        \end{equation}

        Choosing $\bfu_{*} = \bbQ_{\partial_{t}\calU}[\bfu]$, the weak form (\ref{eqn:incompressible NS weak form 1}--\ref{eqn:incompressible NS weak form 2}) then states, $\forall i$:
        \begin{align}
                                                                                0  &=  \bbQ_{\widehat{\calP}}[\nabla\cdot\bfu|_{\tau^{n}_{i}}]  \\
            \begin{split}
                \bfdelta\bfu^{n}_{i}  &=  - \frac{1}{2}\sum_{j, k}\frac{w_{ijk}}{w_{ii}}\bbQ_{\widehat{\calU}}\left[\nabla\cdot\left[\bfu|_{\tau^{n}_{j}}\otimes\bfu|_{\tau^{n}_{k}}\right] + \bfu|_{\tau^{n}_{j}}\cdot\nabla\bfu|_{\tau^{n}_{k}}\right]  \\
                                      &\qquad\qquad\qquad\qquad\qquad\qquad\qquad\qquad- \bbQ_{\widehat{\calU}}\left[\nabla p|_{\tau^{n}_{i}}\right] + \frac{2}{\rmRe}\bbQ_{\widehat{\calU}}\left[\nabla\cdot\nabla_{\rms}\bfu|_{\tau^{n}_{i}}\right]
            \end{split}
        \end{align}
        where:
        \begin{align}
            w_{ii}   &:=  \int_{t^{n}}^{t^{n + 1}}(\phi^{n}_{i})^{2}  \left(=  \|\phi^{n}_{i}\|_{T^{n}}^{2}\right)  \\
            w_{ijk}  &:=  \int_{t^{n}}^{t^{n + 1}}\phi^{n}_{i}\phi^{n}_{j}\phi^{n}_{k}
        \end{align}
        Up to the convective term, this is identical to the classical GL method.
        
        At order $s = 1$ or $s = 2$, one finds in fact that $w_{ijk} = w_{ii}\delta_{ij}\delta_{ik}$, such that the timestepper simplfies to exactly the GL method:
        \begin{align}
                               0  &=  \bbQ_{\widehat{\calP}}[\nabla\cdot\bfu|_{\tau^{n}_{i}}]  \\
            \bfdelta\bfu^{n}_{i}  &=  - \frac{1}{2}\bbQ_{\widehat{\calU}}\left[\nabla\cdot\left[\bfu|_{\tau^{n}_{i}}\otimes\bfu|_{\tau^{n}_{i}}\right] + \bfu|_{\tau^{n}_{i}}\cdot\nabla\bfu|_{\tau^{n}_{i}}\right] - \bbQ_{\widehat{\calU}}\left[\nabla p|_{\tau^{n}_{i}}\right] + \frac{2}{\rmRe}\bbQ_{\widehat{\calU}}\left[\nabla\cdot\nabla_{\rms}\bfu|_{\tau^{n}_{i}}\right]
        \end{align}
        The Crank--Nicolson method, at order $s = 1$, is known classicaly to preserve the energy-dissipative structure for the incompressible Navier--Stokes equations. \BA{[Ref]} This approach offers an intuitive explanation for this, a proof that the same result holds true for the following 2-step GL method, an explanation as to why the same result does \emph{not} hold true at higher order for the classical GL methods, and a modified form of the GL method that preserves the energy dissipative structure up to arbitrary order in time, with a quantifiable energy dissipation from (\ref{eqn:incompressible NS weak form dissipation}).
        
        With $\partial_{t}\calU = \widehat{\calU}\otimes\bbDP^{s - 1}$, the $\rmdiv\circ\bbQ_{\partial_{t}\calU} = \bbQ_{\calP}\circ\rmdiv$ commutativity relation takes the following recognizable form:
        \begin{center}\begin{tikzpicture}[align = center, node distance = 4cm, auto]
            \node (Hdiv) at (0,   0) {$\bfH(\rmdiv)$};
            \node (L2)   at (3.5, 0) {$L^{2}$};
            \draw[->] (Hdiv) -- (L2) node[above, midway] {$\rmdiv$};

            \node (U) at (0,   - 2) {$\widehat{\calU}$};
            \node (P) at (3.5, - 2) {$\widehat{\calP}$};
            \draw[->] (U) -- (P) node[above, midway] {$\rmdiv$};

            \draw[->] (Hdiv) -- (U) node[left, midway] {$\bbQ_{\widehat{\calU}}$};
            \draw[->] (L2)   -- (P)  node[left, midway] {$\bbQ_{\widehat{\calP}}$};
        \end{tikzpicture}\end{center}
    \end{example}
