\subsection*{The \emph{Incompressible} Case}
    Consider first the incompressible MHD model as defined in Figure \ref{fig:incompressible strong form}.

    \line

    Timesteppers the preserve conservative energy and magnetic helicity structures for the \emph{ideal}, incompressible (Hall) MHD system were proposed by Laakmann, Hu and Farrell in \cite{Laakmann_Hu_Farrell_2022}. Consider in particular (3.18a--d) therein, adapted for the inclusion of the Hall term from the Crank--Nicolson timestepper posed by Gawlik and Gay--Balmaz in \cite{Gawlik_Gay--Balmaz_2022}. We re-define this timestepper here, over the spatial domain $\bfOmega$ and interval $T^{n}$.

    \shortline

    Let $\calP$, $\calU$, $\calB$, be function/finite-element spaces defined over $\bfOmega$, such that the following $\Lambda_{0}^{\bullet}(\bfOmega)$-subcomplex diagrams commute:
    \begin{center}\begin{tikzpicture}[align = center, node distance = 4cm, auto]
        \node (Hcurl1) at (0,   0) {$\bfH_{0}(\bfcurl)$};
        \node (Hdiv1)  at (3.5, 0) {$\bfH_{0}(\rmdiv)$};
        \draw[->] (Hcurl1) -- (Hdiv1) node[above, midway] {$\bfcurl$};

        \node (J) at (0,   - 2) {$\calG$};
        \node (B) at (3.5, - 2) {$\calB$};
        \draw[->] (J) -- (B) node[above, midway] {$\bfcurl$};

        \draw[->] (Hcurl1) -- (J) node[left, midway] {$\bbQ_{\calG}$};
        \draw[->] (Hdiv1)  -- (B) node[left, midway] {$\bbQ_{\calB}$};
        
        
        \node (Hcurl2) at (7,    0) {$\bfH_{0}(\bfcurl)$};
        \node (Hdiv2)  at (10.5, 0) {$\bfH_{0}(\rmdiv)$};
        \node (L2)     at (14,   0) {$L^{2}$};
        \draw[->] (Hcurl2) -- (Hdiv2) node[above, midway] {$\bfcurl$};
        \draw[->] (Hdiv2)  -- (L2)    node[above, midway] {$\rmdiv$};

        \node (J) at (7,    - 2) {$\calV$};
        \node (B) at (10.5, - 2) {$\calU$};
        \node (P) at (14,   - 2) {$\calP$};
        \draw[->] (J) -- (B) node[above, midway] {$\bfcurl$};
        \draw[->] (B) -- (P) node[above, midway] {$\rmdiv$};

        \draw[->] (Hcurl2) -- (J) node[left, midway] {$\bbQ_{\calV}$};
        \draw[->] (Hdiv2)  -- (B) node[left, midway] {$\bbQ_{\calU}$};
        \draw[->] (L2)     -- (P) node[left, midway] {$\bbQ_{\calP}$};
    \end{tikzpicture}\end{center}
    for some spaces $\calG$, $\calV$. Suppose $\bfu$, $\bfB$ are given at the start of the interval as:
    \begin{align}
        \bfu^{n}  =  \bfu|_{t^{n}}  \in  \calU,  &&
        \bfB^{n}  =  \bfB|_{t^{n}}  \in  \calB,
    \end{align}
    with (pointwise) $\nabla\cdot\bfu^{n} = 0$, $\nabla\cdot\bfB^{n} = 0$. One seeks:
    \begin{align}
        \bfu^{n + 1}         \in  \calU,  &&
        p^{n + \frac{1}{2}}  \in  \calP,  &&
        \bfB^{n + 1}         \in  \calB,
    \end{align}
    such that the following weak form holds:
    \begin{align}
                           0  &=  \bbQ_{\calP}\left[\nabla\cdot\bfu^{n + \frac{1}{2}}\right]  \\
        \begin{split}
            \bfdelta\bfu^{n}  &=  \bbQ_{\calU}\left[- \bbQ_{\calV}\left[\bbQ_{\calV}\left[\nabla\wedge\bfu^{n + \frac{1}{2}}\right]\wedge\bbQ_{\calV}\left[\bfu^{n + \frac{1}{2}}\right]\right] - \nabla p^{n + \frac{1}{2}}{\color{white} \frac{2}{\beta}}\right.  \\
            &\qquad\qquad\qquad\qquad\qquad\qquad\qquad\qquad\left.+ \frac{2}{\beta}\bbQ_{\calV}\left[\bbQ_{\calG}\left[\nabla\wedge\bfB^{n + \frac{1}{2}}\right]\wedge\bbQ_{\calG}\left[\bfB^{n + \frac{1}{2}}\right]\right]\right]
        \end{split}  \\
        \begin{split}
            \bfdelta\bfB^{n}  &=  \bbQ_{\calB}\left[- \nabla\wedge\left[\frac{1}{\rmRem}\nabla\wedge\bfB^{n + \frac{1}{2}} - \bbQ_{\calV}\left[\bfu^{n + \frac{1}{2}}\right]\wedge\bbQ_{\calG}\left[\bfB^{n + \frac{1}{2}}\right]\right.\right.  \\
            &\qquad\qquad\qquad\qquad\qquad\qquad\qquad\;\;\;\left.\left.{\color{white} \frac{1}{\rmRem}}+ \rmRH\bbQ_{\calG}\left[\nabla\wedge\bfB^{n + \frac{1}{2}}\right]\wedge\bbQ_{\calG}\left[\bfB^{n + \frac{1}{2}}\right]\right]\right]
        \end{split}
    \end{align}
    where:
    \begin{align}
        \bfdelta\bfu  :=  \frac{1}{\delta t^{n}}\left(\bfu^{n + 1} - \bfu^{n}\right),  &&
        \bfdelta\bfB  :=  \frac{1}{\delta t^{n}}\left(\bfB^{n + 1} - \bfB^{n}\right),
    \end{align}
    and:
    \begin{align}
        \bfu^{n + \frac{1}{2}}  :=  \frac{1}{2}\left(\bfu^{n + 1} + \bfu^{n}\right),  &&
        \bfB^{n + \frac{1}{2}}  :=  \frac{1}{2}\left(\bfB^{n + 1} + \bfB^{n}\right).
    \end{align}
    $\bfu$, $\bfB$ are then given at the end of the interval as:
    \begin{align}
        \bfu|_{t^{n + 1}} = \bfu^{n + 1},  &&
        \bfB|_{t^{n + 1}} = \bfB^{n + 1}.
    \end{align}
    
    \shortline

    Solutions to the above system:
    \begin{itemize}
        \item  Exhibit incompressibility, $\nabla\cdot\bfu|_{t^{n + 1}}  =  0$, and Gauss's law, $\nabla\cdot\bfB|_{t^{n + 1}}  =  0$, exactly.
        \item  Conserve the energy, $\rmE  :=  \int_{\bfOmega}\left[\frac{1}{2}\|\bfu\|^{2} + \frac{1}{\beta}\|\bfB\|^{2}\right]$, and magnetic helicity, $\rmH_{\rmM}  :=  \calH[\bfA]$, exactly from $t^{n}$ to $t^{n + 1}$.
    \end{itemize}

    Being a Crank--Nicolson timestepper, the NS-like component of this timestepper much resembles that which was derived from a CG-DG-in-time discretization for the incompressible NS model at lowest order $s = 1$, up to the choice of representation for the convective term. The timestepper here uses a mixed vorticity formulation, through the use of $\bbQ_{\calV}$ projections on $\bfu^{n + \frac{1}{2}}$ and $\nabla\wedge\bfu^{n + \frac{1}{2}}$. As stated above, we intend not to use a mixed formulation.

    \line

    We shall now build on the work from Section \ref{cha:Navier--Stokes}, using a FET approach to derive a class of timesteppers for the incompressible (Hall) MHD model (See Figure \ref{fig:incompressible strong form}) which:
    \begin{itemize}
        \item  Can be defined up to arbitrarily high order in time.
        \item  Incorporate viscous terms, with a quantifiable dissipation on the energy and magnetic helicity.
    \end{itemize}
    At lowest order, the resultant timestepper much resembles that from \cite{Laakmann_Hu_Farrell_2022} using different formulations for the convective terms.

    \shortline

    Suppose again we consider the problem through the lens of function/FE spaces defined on the space-time domain, $\bfOmega\otimes T^{n}$. For some $\calP$, $\calU$, $\calB$, supported on $\bfOmega\otimes T^{n}$, we seek:
    \begin{align}
        p         \in  \calP,  &&
        \bfu_{-}  \in  \calU_{-},  &&
        \bfB_{-}  \in  \calB_{-},
    \end{align}
    such that:
    \begin{align}
            0
            &=
            \bbQ_{\calP}[\nabla\cdot\bbQ_{\partial_{t}\calU}[\bfu]]  \label{eqn:incompressible MHD weak form 1}  \\
        \begin{split}
            (\partial_{t}\bfu
            =)
            \bbQ_{\partial_{t}\calU}[\partial_{t}\bfu]
            &=
            \bbQ_{\partial_{t}\calU}\left[- \frac{1}{2}(\nabla\cdot\left[\bfu_{*}\otimes\bbQ_{\partial_{t}\calU}[\bfu]\right]
            + \bfu_{*}\cdot\nabla\bbQ_{\partial_{t}\calU}[\bfu])
            - \nabla p\right.  \\
            &\qquad\qquad\qquad\qquad\left.+ \frac{2}{\beta}(\nabla\wedge\bbQ_{\partial_{t}\calB}[\bfB])\wedge\bbQ_{\calG}[\bfB]
            + \frac{2}{\rmRef}\nabla\cdot\nabla_{\rms}\bbQ_{\partial_{t}\calU}[\bfu]\right]
        \end{split}  \\
        \begin{split}
            (\partial_{t}\bfB
            =)
            \bbQ_{\partial_{t}\calB}[\partial_{t}\bfB]
            &=
            \bbQ_{\partial_{t}\calB}\left[\nabla\wedge\left[\bbQ_{\partial_{t}\calU}[\bfu]\wedge\bbQ_{\calG}[\bfB]
            - \rmRH(\nabla\wedge\bbQ_{\partial_{t}\calB}[\bfB])\wedge\bbQ_{\calG}[\bfB]\tall\right.\right.  \\
            &\qquad\qquad\qquad\qquad\qquad\qquad\qquad\qquad\qquad\qquad\left.\left.- \frac{1}{\rmRem}\nabla\wedge\bbQ_{\partial_{t}\calB}[\bfB]\right]\right]
        \end{split}
    \end{align}
    for a space $\calG$, also defined on $\bfOmega\otimes T^{n}$, defined such that the following $\Lambda_{0}^{\bullet}(\bfOmega)\otimes L^{2}$-subcomplex diagram commutes:
    \begin{center}\begin{tikzpicture}[align = center, node distance = 4cm, auto]
        \node (Hcurl) at (0,   0) {$\bfH_{0}(\bfcurl)\otimes L^{2}$};
        \node (Hdiv)  at (4.5, 0) {$\bfH_{0}(\rmdiv)\otimes L^{2}$};
        \draw[->] (Hcurl) -- (Hdiv) node[above, midway] {$\bfcurl$};

        \node (J)  at (0,   - 2) {$\calG$};
        \node (Bt) at (4.5, - 2) {$\partial_{t}\calB$};
        \draw[->] (J) -- (Bt) node[above, midway] {$\bfcurl$};

        \draw[->] (Hcurl) -- (J)  node[left, midway] {$\bbQ_{\calG}$};
        \draw[->] (Hdiv)  -- (Bt) node[left, midway] {$\bbQ_{\partial_{t}\calB}$};
    \end{tikzpicture}\end{center}
    Again, $\bfu_{*}$ can be taken as either $\bfu$ or $\bbQ_{\partial_{t}\calU}[\bfu]$, although one would typically choose the latter in practice.

    \shortline

    Let $\{-, -, -\}$ denote the antisymmetric vector triple product,
    \begin{equation}
        \{\bfa, \bfb, \bfc\}  :=  \bfa\cdot(\bfb\wedge\bfc).
    \end{equation}
    We show the resultant timestepper will preserve the dissipative structure in:
    \begin{itemize}
        \item  The energy, $\rmE  :=  \int_{\bfOmega}\left[\frac{1}{2}\|\bfu\|^{2} + \frac{1}{\beta}\|\bfB\|^{2}\right]$:
        \begin{align}
                \rmE\left(t^{n + 1}\right) - \rmE\left(t^{n}\right)
                &=  \int_{T^{n}}\frac{\rmd}{\rmd t}\rmE  \\
                &=  \int_{T^{n}}\frac{\rmd}{\rmd t}\int_{\bfOmega}\left[\frac{1}{2}\|\bfu\|^{2} + \frac{1}{\beta}\|\bfB\|^{2}\right]  \\
                &=  \int_{\bfOmega\otimes T^{n}}\left[\bfu\cdot\partial_{t}\bfu + \frac{2}{\beta}\bfB\cdot\partial_{t}\bfB\right]  \\
                &=  \int_{\bfOmega\otimes T^{n}}\left[\bfu\cdot\bbQ_{\partial_{t}\calU}\left[\cdots\tall\right] + \frac{2}{\beta}\bfB\cdot\bbQ_{\partial_{t}\calB}\left[\cdots\tall\right]\right]  \\
            \begin{split}
                &=  \mst{\calC\left[\bfu_{*}; \bbQ_{\partial_{t}\calU}[\bfu], \bbQ_{\partial_{t}\calU}[\bfu]\right]} + \int_{\bfOmega\otimes T^{n}}\left[p\mst{\nabla\cdot\bbQ_{\partial_{t}\calU}[\bfu]}\tall\right.  \\
                &\qquad\qquad+ \mst{\frac{2}{\beta}\{\bbQ_{\partial_{t}\calU}[\bfu], \nabla\wedge\bbQ_{\partial_{t}\calB}[\bfB], \bbQ_{\calG}[\bfB]\}}  \\
                &\qquad\qquad- \frac{2}{\rmRef}\|\nabla_{\rms}\bbQ_{\partial_{t}\calU}[\bfu]\|^{2}  \\
                &\qquad\qquad+ \mst{\frac{2}{\beta}\{\nabla\wedge\bbQ_{\partial_{t}\calB}[\bfB], \bbQ_{\partial_{t}\calU}[\bfu], \bbQ_{\calG}[\bfB]\}}  \\
                &\qquad\qquad+ \frac{2\rmRH}{\beta}\mst{\{\nabla\wedge\bbQ_{\partial_{t}\calB}[\bfB], \nabla\wedge\bbQ_{\partial_{t}\calB}[\bfB], \bbQ_{\calG}[\bfB]\}}  \\
                &\qquad\qquad\left.- \frac{1}{\rmRem}\|\nabla\wedge\bbQ_{\partial_{t}\calB}[\bfB]\|^{2}\right]
            \end{split}  \\
                &=  - \int_{\bfOmega}\left[\frac{2}{\rmRef}\|\nabla_{\rms}\bbQ_{\partial_{t}\calU}[\bfu]\|^{2} + \frac{1}{\rmRem}\|\nabla\wedge\bbQ_{\partial_{t}\calB}[\bfB]\|^{2}\right]  \leq  0
        \end{align}
        
        \item  The magnetic helicity, $\rmH_{\rmM}  :=  \calH[\bfA]$:
        \begin{align}
                \rmH_{\rmM}\left(t^{n + 1}\right) - \rmH_{\rmM}\left(t^{n}\right)
                &=  \int_{T^{n}}\frac{\rmd}{\rmd t}\rmH_{\rmM}  \\
                &=  \int_{T^{n}}\frac{\rmd}{\rmd t}\calH[\bfA]  \\
                &=  2\int_{\bfOmega\otimes T^{n}}\bfA\cdot\partial_{t}\bfB  \\
                &=  2\int_{\bfOmega\otimes T^{n}}\bfA\cdot\bbQ_{\partial_{t}\calB}\left[\nabla\wedge\left[\cdots\tall\right]\right]  \\
                &=  2\int_{\bfOmega\otimes T^{n}}\bfA\cdot\nabla\wedge\bbQ_{\calG}\left[\cdots\tall\right]  \impliedby  \bbQ_{\partial_{t}\calB}\circ\bfcurl = \bfcurl\circ\bbQ_{\calG}  \\
                &=  2\int_{\bfOmega\otimes T^{n}}\bbQ_{\calG}[\bfB]\cdot\left(\cdots\tall\right)  \\
            \begin{split}
                &=  2\int_{\bfOmega\otimes T^{n}}\left[\mst{\{\bbQ_{\calG}[\bfB], \bbQ_{\partial_{t}\calU}[\bfu], \bbQ_{\calG}[\bfB]\}}\tall\right.  \\
                &\qquad\qquad- \rmRH\mst{\{\bbQ_{\calG}[\bfB], \nabla\wedge\bbQ_{\partial_{t}\calB}[\bfB], \bbQ_{\calG}[\bfB]\}}  \\
                &\qquad\qquad\left.- \frac{1}{\rmRem}\bbQ_{\calG}[\bfB]\cdot(\nabla\wedge\bbQ_{\partial_{t}\calB}[\bfB])\tall\right]
            \end{split}  \\
                &=  - \frac{2}{\rmRem}\int_{\bfOmega\otimes T^{n}}(\nabla\wedge\bbQ_{\calG}[\bfB])\cdot\bbQ_{\partial_{t}\calB}[\bfB]  =  \calO\left[\frac{1}{\rmRem}\right]
        \end{align}
    \end{itemize}
    
    \shortline

    Regarding local structures:
    \begin{itemize}
        \item  Identical to the Navier--Stokes case, if the following $\Lambda_{0}^{\bullet}(\bfOmega)\otimes L^{2}(T^{n})$-subcomplex diagram commutes:
        \begin{center}\begin{tikzpicture}[align = center, node distance = 4cm, auto]
            \node (Hdiv) at (0,   0) {$\bfH(\rmdiv)\otimes L^{2}$};
            \node (L2)   at (4.5, 0) {$L^{2}\otimes L^{2}$};
            \draw[->] (Hdiv) -- (L2) node[above, midway] {$\rmdiv$};

            \node (Ut) at (0,   - 2) {$\partial_{t}\calU$};
            \node (P)  at (4.5, - 2) {$\calP$};
            \draw[->] (Ut) -- (P) node[above, midway] {$\rmdiv$};

            \draw[->] (Hdiv) -- (Ut) node[left, midway] {$\bbQ_{\partial_{t}\calU}$};
            \draw[->] (L2)   -- (P)  node[left, midway] {$\bbQ_{\calP}$};
        \end{tikzpicture}\end{center}
        then (\ref{eqn:incompressible MHD weak form 1}) directly gives pointwise incompressibility on the \emph{auxiliary} space, $\bbQ_{\partial_{t}\calU}[\bfu]$,
        \begin{equation}
            \nabla\cdot\bbQ_{\partial_{t}\calU}[\bfu]  =  0.
        \end{equation}

        \item  Defining the electric field,
        \begin{equation}
            \bfE
            :=  \bbQ_{\calG}\!\left[- \bbQ_{\partial_{t}\calU}[\bfu]\wedge\bbQ_{\calG}[\bfB]
            + \rmRH(\nabla\wedge\bbQ_{\partial_{t}\calB}[\bfB])\wedge\bbQ_{\calG}[\bfB]\tall
            + \frac{1}{\rmRem}\nabla\wedge\bbQ_{\partial_{t}\calB}[\bfB]\right],
        \end{equation}
        then by the commutativity relation, $\bbQ_{\partial_{t}\calB}\circ\bfcurl = \bfcurl\circ\bbQ_{\calG}$, Faraday's law is satisfied exactly,
        \begin{equation}
            \partial_{t}\bfB  =  - \nabla\wedge\bfE.
        \end{equation}
        Moreover, provided $\nabla\cdot\bfB|_{t^{n}} = 0$, this implies Gauss's law, $\nabla\cdot\bfB = 0$, holds on all of $\bfOmega\otimes T^{n}$. The proof is identical to that of the continuous case.
    \end{itemize}

    \line

    \begin{example}[CG-DG-in-time dissipative timestepper for incompressible MHD]
        Suppose the spaces take the following CG-DG-in-time tensor-product forms, for some order-in-time, $s$:
        \begin{align}
            \calP  =  \widehat{\calP}(\bfOmega)\otimes\bbDP^{s - 1}\left(T^{n}\right),  &&
            \calU  =  \widehat{\calU}(\bfOmega)\otimes\bbP^{s}\left(T^{n}\right),  &&
            \calB  =  \widehat{\calB}(\bfOmega)\otimes\bbP^{s}\left(T^{n}\right),
        \end{align}
        With $\partial_{t}\calB = \widehat{\calB}\otimes\bbDP^{s - 1}$, take
        \begin{equation}
            \calG  =  \widehat{\calG}(\bfOmega)\otimes\bbDP^{s - 1}\left(T^{n}\right),
        \end{equation}
        such that the commutativity relation $\bbQ_{\partial_{t}\calB}\circ\bfcurl = \bfcurl\circ\bbQ_{\calG}$ is equivalent to the requirement that the following $\Lambda_{0}^{\bullet}(\bfOmega)$-subcomplex commutes:
        \begin{center}\begin{tikzpicture}[align = center, node distance = 4cm, auto]
            \node (Hcurl1) at (0,   0) {$\bfH_{0}(\bfcurl)$};
            \node (Hdiv1)  at (3.5, 0) {$\bfH_{0}(\rmdiv)$};
            \draw[->] (Hcurl1) -- (Hdiv1) node[above, midway] {$\bfcurl$};
    
            \node (J) at (0,   - 2) {$\widehat{\calG}$};
            \node (B) at (3.5, - 2) {$\widehat{\calB}$};
            \draw[->] (J) -- (B) node[above, midway] {$\bfcurl$};
    
            \draw[->] (Hcurl1) -- (J) node[left, midway] {$\bbQ_{\widehat{\calG}}$};
            \draw[->] (Hdiv1)  -- (B) node[left, midway] {$\bbQ_{\widehat{\calB}}$};
        \end{tikzpicture}\end{center}
        Compare this with the commutativity requirement in the \cite{Laakmann_Hu_Farrell_2022} timestepper.

        Similar to the CG-DG-in-time timesteppers for the Navier--Stokes models, one can use a basis of GL functions as a basis for $\bbDP^{s - 1}$ to write the resultant timestepper in a closed form, reminiscent of the classical GL method.

        One each timestep interval $T^{n}$, one seeks, $\forall i \in \{1, \cdots, s\}$:
        \begin{align}
                       p_{i}^{n}  \in  \widehat{\calP},  &&
            \bfdelta\bfu_{i}^{n}  \in  \widehat{\calU},  &&
            \bfdelta\bfB_{i}^{n}  \in  \widehat{\calB},
        \end{align}
        such that, taking $\bfu_{*} = \bbQ_{\partial_{t}\calU}[\bfu]$, the following weak form holds, $\forall i \in \{1, \cdots, s\}$:
        \begin{align}
                0
                &=
                \bbQ_{\widehat{\calP}}\left[\nabla\cdot\bfu_{i}^{n}\right]  \\
            \begin{split}
                \left(\bfdelta\bfu_{i}^{n}
                =\right)
                \bbQ_{\widehat{\calU}}\left[\bfdelta\bfu_{i}^{n}\right]
                &=
                \bbQ_{\widehat{\calU}}\left[- \frac{1}{2}\sum_{jk}w_{i}^{jk}\left(\nabla\cdot\left[\bfu_{j}^{n}\otimes\bfu_{k}^{n}\right]
                + \bfu_{j}^{n}\cdot\nabla\bfu_{k}^{n}\right)
                - \nabla p_{i}^{n}\right.  \\
                &\qquad\qquad\qquad\;\;\;\left.+ \frac{2}{\beta}\sum_{jk}w_{i}^{jk}\left(\nabla\wedge\bfB_{j}^{n}\right)\wedge\bbQ_{\calG}\left[\bfB_{k}^{n}\right]
                + \frac{2}{\rmRef}\nabla\cdot\nabla_{\rms}\bfu_{i}^{n}\right]
            \end{split}  \\
            \begin{split}
                \left(\bfdelta\bfB_{i}^{n}
                =\right)
                \bbQ_{\widehat{\calB}}\left[\bfdelta\bfB_{i}^{n}\right]
                &=
                \bbQ_{\widehat{\calB}}\left[\nabla\wedge\left[\sum_{jk}w_{i}^{jk}\bfu_{j}^{n}\wedge\bbQ_{\calG}\left[\bfB_{k}^{n}\right]
                \right.\right.  \\
                &\qquad\qquad\qquad\left.\left.- \rmRH\sum_{jk}w_{i}^{jk}(\nabla\wedge\bfB_{j}^{n})\wedge\bbQ_{\calG}\left[\bfB_{k}^{n}\right]\tall - \frac{1}{\rmRem}\nabla\wedge\bfB_{i}^{n}\right]\right]
            \end{split}
        \end{align}
        where:
        \begin{align}
            \bfu_{i}^{n}  :=  \bfu|_{t^{n}} + \delta t^{n}\sum_{j}a_{ij}\bfdelta\bfu_{j}^{n},  &&
            \bfB_{i}^{n}  :=  \bfB|_{t^{n}} + \delta t^{n}\sum_{j}a_{ij}\bfdelta\bfB_{j}^{n},
        \end{align}
        for GL coefficients $(a_{ij})_{ij}$, defined as in (\ref{eqn:a_ij definition}), and GL triple weights $(w_{i}^{jk})_{i}^{jk}$ defined as in (\ref{eqn:w_i^jk definition}). Updates at $t^{n + 1}$ are given in terms of $(\bfdelta\bfu_{i}^{n})_{i}$, $(\bfdelta\bfB_{i}^{n})_{i}$ as:
        \begin{align}
            \bfu|_{t^{n + 1}}  :=  \bfu|_{t^{n}} + \delta t^{n}\sum_{j}b_{j}\bfdelta\bfu_{j}^{n},  &&
            \bfB|_{t^{n + 1}}  :=  \bfB|_{t^{n}} + \delta t^{n}\sum_{j}b_{j}\bfdelta\bfB_{j}^{n},
        \end{align}
        for GL coefficients $(b_{j})_{j}$, defined as in (\ref{eqn:b_j definition}).

        Similar to the incompressible NS system, this timestepper resembles a classical GL method for steps $s \leq 2$, where $w_{i}^{jk} = \delta_{ij}\delta_{ik}$, but differs at higher order for $s \geq 3$, where this no longer holds true.

        In variational form, one can seek, $\forall i \in \{1, \cdots, s\}$:
        \begin{align}
                       p_{i}^{n}  \in  \widehat{\calP},  &&
            \bfdelta\bfu_{i}^{n}  \in  \widehat{\calU},  &&
            \bfdelta\bfB_{i}^{n}  \in  \widehat{\calB},  &&
                    \bfG_{i}^{n}  \in  \widehat{\calG},
        \end{align}
        such that, $\forall i \in \{1, \cdots, s\}$:
        \begin{align}
            &\forall     q_{i}^{n}  \in  \widehat{\calP},  &
                0
                &=
                \left\langle q_{i}^{n}, \nabla\cdot\bfu_{i}^{n}\right\rangle  \\
            &\forall  \bfv_{i}^{n}  \in  \widehat{\calU},  &
            \begin{split}
                \left\langle\bfv_{i}^{n}, \bfdelta\bfu_{i}^{n}\right\rangle
                &=
                \sum_{jk}w_{i}^{jk}\calC\!\left[\bfu_{k}^{n}; \bfv_{i}^{n}, \bfu_{j}^{n}\right]
                +  \left\langle\nabla\cdot\bfv_{i}^{n}, p_{i}^{n}\right\rangle  \\
                &\qquad\qquad+ \frac{2}{\beta}\sum_{jk}w_{i}^{jk}\calE\!\left[\bfv_{i}^{n}; \bfB_{j}^{n}; \bfG_{k}^{n}\right]
                - \frac{2}{\rmRef}\left\langle\nabla_{\rms}\bfv_{i}^{n}, \nabla_{\rms}\bfu_{i}^{n}\right\rangle
            \end{split}  \\
            &\forall  \bfC_{i}^{n}  \in  \widehat{\calB},  &
            \begin{split}
                \left\langle\bfC_{i}^{n}, \bfdelta\bfB_{i}^{n}\right\rangle
                &=
                \sum_{jk}w_{i}^{jk}\calE\!\left[\bfu_{j}^{n}; \bfC_{i}^{n}; \bfG_{k}^{n}\right]  \\
                &\qquad\qquad+ \rmRH\sum_{jk}w_{i}^{jk}\calD\!\left[\bfG_{k}^{n}; \bfC_{i}^{n}, \bfB_{j}^{n}\right]
                - \frac{1}{\rmRem}\left\langle\nabla\wedge\bfC_{i}^{n}, \nabla\wedge\bfB_{i}^{n}\right\rangle
            \end{split}  \\
            &\forall  \bfH_{i}^{n}  \in  \widehat{\calG},  &
                0
                &=
                \left\langle\bfH_{i}^{n}, \bfG_{i}^{n}\right\rangle
                - \left\langle\bfH_{i}^{n}, \bfB_{i}^{n}\right\rangle
        \end{align}
        where all inner products $\langle -, -\rangle  =  \langle -, -\rangle_{\bfOmega}$ are taken over the spatial domain, $\bfOmega$, and:
        \begin{itemize}
            \item  $\calC : \widehat{\calU}^{3} \rightarrow \bbR$ is the skew-symmetric convective operator defined in (\ref{eqn:convective operator definition}).
            \item  $\calD : \widehat{\calG}\times\widehat{\calB}^{2} \rightarrow \bbR$ is defined
            \begin{equation}
                \calD\!\left[\widehat{\bfG}; \widehat{\bfB}, \widehat{\bfC}\right]
                :=
                \int_{\bfOmega}\left\{\widehat{\bfG}, \nabla\wedge\widehat{\bfB}, \nabla\wedge\widehat{\bfC}\right\},
            \end{equation}
            with the skew-symmetry property $\calD\!\left[\widehat{\bfG}; \widehat{\bfB}, \widehat{\bfB}\right]  =  0$, $\forall \widehat{\bfG} \in \widehat{\calG}$, $\widehat{\bfB} \in \widehat{\calB}$.
            \item  $\calE : \widehat{\calU}\times\widehat{\calB}\times\widehat{\calG} \rightarrow \bbR$ is defined
            \begin{equation}
                \calE\!\left[\widehat{\bfu}; \widehat{\bfB}; \widehat{\bfG}\right]
                :=
                \int_{\bfOmega}\left\{\widehat{\bfu}, \nabla\wedge\widehat{\bfB}, \widehat{\bfG}\right\}.
            \end{equation}
        \end{itemize}

        By the GL orthogonality, the dissipations in the energy, $\rmE$, and magnetic helicity, $\rmH_{\rmM}$, can be quantified concisely as:
        \begin{align}
                   \rmE|_{t^{n + 1}} -        \rmE|_{t^{n}}  &=  - \delta t^{n}\sum_{i}\int_{\bfOmega}\left[\frac{2}{\rmRef}\left\|\nabla_{\rms}\bfu_{i}^{n}\right\|^{2} + \frac{1}{\rmRem}\left\|\nabla\wedge\bfB_{i}^{n}\right\|^{2}\right]  \\
            \rmH_{\rmM}|_{t^{n + 1}} - \rmH_{\rmM}|_{t^{n}}  &=  - \delta t^{n}\frac{2}{\rmRem}\sum_{i}\int_{\bfOmega}\left(\nabla\wedge\bfG_{i}^{n}\right)\cdot\bfB_{i}^{n}
        \end{align}

        For exact incompressibility, the $\bbQ_{\calP}\circ\rmdiv = \rmdiv\circ\bbQ_{\partial_{t}\calU}$ commutativity property manifests similarly to in the incompressible NS case, in the requirement that the following diagram commutes:
        \begin{center}\begin{tikzpicture}[align = center, node distance = 4cm, auto]
            \node (Hdiv) at (0,   0) {$\bfH(\rmdiv)$};
            \node (L2)   at (3.5, 0) {$L^{2}$};
            \draw[->] (Hdiv) -- (L2) node[above, midway] {$\rmdiv$};

            \node (U) at (0,   - 2) {$\widehat{\calU}$};
            \node (P) at (3.5, - 2) {$\widehat{\calP}$};
            \draw[->] (U) -- (P) node[above, midway] {$\rmdiv$};

            \draw[->] (Hdiv) -- (U) node[left, midway] {$\bbQ_{\widehat{\calU}}$};
            \draw[->] (L2)   -- (P) node[left, midway] {$\bbQ_{\widehat{\calP}}$};
        \end{tikzpicture}\end{center}
    \end{example}
