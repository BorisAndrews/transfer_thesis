\subsubsection*{Transient Case}
    \BA{Introduction.}

    Consider now the transient case, with the inclusion of the time derivatives, and discarding Gauss's law, (\ref{eqn:Gauss's law}).
    
    To derive a transient scheme, the system can be interpreted through the lens of function spaces in time, i.e. on the domain $\bfOmega\otimes[0, T]$ for some final time, $T$. The intention here is \emph{not} to create a finite element problem that needs solving over \emph{all of $\bfOmega\otimes[0, T]$ simultaneously}, but, after discretization in time, to derive the corresponding timestepper that need solving only on $\bfOmega$ at each timestep, as is traditionally the case with a timestepping scheme. This approach allows one to more easily ensure the construction of a scheme that preserves as much structure as possible, by applying the finite element method over all of $\bfOmega\otimes[0, T]$, without the numerical difficulty of solving a 4D problem.

    We cast into weak form therefore, using test functions in the following test function spaces, via the $L^{2}$ inner product on $\bfOmega\otimes[0, T]$, where the test function spaces $\bbF$, $\bbC$ are new spaces, distinct from $\bbP$, $\cdots$, $\bbB$:
    \begin{center}\begin{tabular}{ r l c | c }
        \multicolumn{2}{c}{Equation}  &  Index  &  Test space  \\
        \hline\hline
        Mass conservation  &  $0  =  \partial_{t}\rho + \nabla\cdot\bfp$  &  (\ref{eqn:mass conservation})  &  $\bbP$  \\
        Momentum conservation  &  $0  =  - \partial_{t}\bfp - \nabla\cdot\left[\frac{1}{\rho}\bfp^{\otimes 2}\right] - \nabla p + \frac{2}{\beta}\bfj\wedge\bfB + \cdots$  &  (\ref{eqn:momentum conservation})  &  $\bbU$  \\
        Energy conservation  &  $0  =  - \partial_{t}p - \nabla\cdot\left[\frac{p}{\rho}\bfp\right] - p\nabla\cdot\left[\frac{1}{\rho}\bfp\right] + \cdots$  &  (\ref{eqn:energy conservation})  &  $\bbD$  \\
        \hline
        Current identity  &  $\bfzero  =  \frac{1}{\rmRem}\bfj - \left(\bfE + \frac{1}{\rho}\bfp\wedge\bfB\right) + \rmRH\bfj\wedge\bfB$  &  (\ref{eqn:current identity})  &  $\bbJ$  \\
        \hline
        Ampère's law  &  $\bfzero  =  \nabla\wedge\bfB - \bfj$  &  (\ref{eqn:Ampère's law})  &  $\bbF$  \\
        Faraday's law  &  $\bfzero  =  \partial_{t}\bfB + \nabla\wedge\bfE$  &  (\ref{eqn:Faraday's law})  &  $\bbC$  \\
    \end{tabular}\end{center}
    This takes the variational formulation:
    \begin{align}
        \forall q \in \bbP,  0  &=  \left.\left(\langle\partial_{t}\rho, q\rangle + \langle\nabla\cdot\bfp, q\rangle\tall\right)\right|_{\bfOmega\otimes[0, T]}  \\
        \begin{split}
            \forall \bfq \in \bbU,  0  &=  \left.\left(- \langle\partial_{t}\bfp, \bfq\rangle + \left\langle\frac{1}{\rho}\bfp^{\otimes 2}, \nabla\bfq\right\rangle + \langle p, \nabla\cdot\bfq\rangle + \frac{2}{\beta}\langle\bfj\wedge\bfB, \bfq\rangle - \frac{1}{\rmRef}\langle\rho\bftau, \nabla_{s}\bfq)\rangle\right)\right|_{\bfOmega\otimes[0, T]}  \\
            &\;\;\;\;\;\;\;\;\;\;\;\;\;\;\;\;\;\;\;\;\;\;\;\;  + \left.\left(- \left\langle\frac{1}{\rho}(\bfp\cdot\bfn)\bfp, \bfq\right\rangle - \langle p, \bfq\cdot\bfn\rangle + \frac{1}{\rmRef}\langle\rho\bftau\cdot\bfn, \bfq\rangle\right)\right|_{\bfGamma\otimes[0, T]}
        \end{split}  \\
        \begin{split}
            \forall \sigma \in \bbD,  0  &=  \left(- \langle\partial_{t}p, \sigma\rangle + \left\langle\frac{p}{\rho}\bfp, \nabla\sigma\right\rangle - \left\langle p\nabla\cdot\left[\frac{1}{\rho}{\bfp}, \sigma\right]\right\rangle + \frac{1}{\rmRef}\left\langle\rho\bftau:\nabla\left[\frac{1}{\rho}\bfp\right], \sigma\right\rangle\right.  \\
            &\;\;\;\;\;\;\;\;\;\;\;\;\;\;\;\;\;\;\;\;\;\;\;\;\;\;\;\;\;\;\;\;\;\;\;\;\;\;\;\;\;\;\;\;\;\;\;\;\left.\left.+ \frac{2}{\beta\rmRem}\left\langle\|\bfj\|^{2}, \sigma\right\rangle - \frac{1}{\rmPe}\left\langle\rho\nabla\left[\frac{p}{\rho}\right], \nabla\sigma\right\rangle\right)\right|_{\bfOmega\otimes[0, T]}  \\
            &\;\;\;\;\;\;\;\;\;\;\;\;\;\;\;\;\;\;\;\;\;\;\;\;  + \left.\left(- \left\langle\frac{p}{\rho}\bfp\cdot\bfn, \sigma\right\rangle + \frac{1}{\rmPe}\left\langle\rho\nabla\left[\frac{p}{\rho}\right]\cdot\bfn, \sigma\right\rangle\right)\right|_{\bfGamma\otimes[0, T]}
        \end{split}  \\
        \forall \bfk \in \bbJ,  0  &=  \left.\left(\frac{1}{\rmRem}\langle\bfj, \bfk\rangle - \langle\bfE, \bfk\rangle - \left\langle\frac{1}{\rho}\bfp\wedge\bfB, \bfk\right\rangle + \rmRH\langle\bfj\wedge\bfB, \bfk\rangle\right)\right|_{\bfOmega\otimes[0, T]}  \\
        \forall \bfF \in \bbF,  0  &=  \left.\left(\langle\bfB, \nabla\wedge\bfF\rangle - \langle\bfj, \bfF\rangle\tall\right)\right|_{\bfOmega} + \langle\bfB, \bfF\wedge\bfn\rangle_{\bfGamma}  \\
        \forall \bfC \in \bbC,  0  &=  \left.\left(\langle\partial_{t}\bfB, \bfC\rangle + \langle\nabla\wedge\bfE, \bfC\rangle\tall\right)\right|_{\bfOmega\otimes[0, T]}
    \end{align}
    Similarly to the stationary-state case, one can seek to enforce mass conservation (\ref{eqn:mass conservation}) and Faraday's law (\ref{eqn:Faraday's law}) \emph{strongly} by considering certain test functions, and deriving conditions on the function spaces from there:
    \begin{center}\begin{tabular}{ c | c | c }
        Equation  &  Test function  &  Subspace condition  \\
        \hline\hline
        $0  =  \partial_{t}\rho + \nabla\cdot\bfp$  &  $\partial_{t}\rho + \nabla\cdot\bfp  \in  \bbP$  &  $\partial_{t}\bbD + \nabla\cdot\bbU  \leqslant  \bbP$  \\
        $0  =  \partial_{t}\bfB + \nabla\wedge\bfE$  &  $\partial_{t}\bfB + \nabla\wedge\bfE  \in  \bbC$  &  $\partial_{t}\bbB + \nabla\wedge\bbE  \leqslant  \bbC$
    \end{tabular}\end{center}
    Similarly restricting to Hilbert spaces then, it is therefore natural to consider weak formulations where ($\bbD$, $\bbU$), $\bbP$ and ($\bbE$, $\bbB$), ($*$, $\bbC$) form subcomplexes of $H\Lambda^{\bullet}(\bfOmega\otimes[0, T])$, with projection/interpolation maps $\pi_{*}$ such that the following diagrams commute:
    \begin{center}\begin{tikzpicture}[align = center, node distance = 4cm, auto]
        \node (HL2)  at (0,   0)   {$\bfH\left(\begin{matrix} * - \rmdiv \\ \bfcurl + \partial_{t} \end{matrix}\right)$};
        \node (HL3a) at (5.5, 0)   {$\bfH(\partial_{t} + \rmdiv)$};
        \node (EB)   at (0,   - 3) {$\bbE\oplus\bbB$};
        \node (*C)   at (5.5, - 3) {$*\oplus\bbC$};

        \draw[->] (HL2)  -- (HL3a) node[above, midway] {$\left(\begin{matrix} * - \rmdiv \\ \bfcurl + \partial_{t} \end{matrix}\right)$};
        \draw[->] (EB)   -- (*C)   node[above, midway] {$\left(\begin{matrix} * - \rmdiv \\ \bfcurl + \partial_{t} \end{matrix}\right)$};
        \draw[->] (HL2)  -- (EB)   node[left,  midway] {$\pi_{\rmE\rmB}$};
        \draw[->] (HL3a) -- (*C)   node[left,  midway] {$\pi_{*\rmC}$};
        
        \node (HL3b) at (9,   0)  {$\bfH(\partial_{t} + \rmdiv)$};
        \node (HL4) at (12.5, 0)  {$L^{2}$};
        \node (DU)  at (9,   - 3) {$\bbD\oplus\bbU$};
        \node (P)  at (12.5, - 3) {$\bbP$};

        \draw[->] (HL3b) -- (HL4) node[above, midway] {$\partial_{t} + \rmdiv$};
        \draw[->] (DU)   -- (P)   node[above, midway] {$\partial_{t} + \rmdiv$};
        \draw[->] (HL3b) -- (DU)  node[left,  midway] {$\pi_{\rmD\rmU}$};
        \draw[->] (HL4)  -- (P)   node[left,  midway] {$\pi_{\rmP}$};
    \end{tikzpicture}\end{center}
    By Corollary \ref{cor:tensor product complex inclusion}, such subcomplexes can be constructed via a tensor product construction:
    \begin{center}\begin{tikzpicture}[align = center, node distance = 4cm, auto]
        \node (HL2)      at (0,      0)     {$\bfH\left(\begin{matrix} * - \rmdiv \\ \bfcurl + \partial_{t} \end{matrix}\right)$};
        \node (HL3)      at (5.5,    0)     {$\bfH(\partial_{t} + \rmdiv)$};
        \node (Hcurl-L2) at (- 0.75, - 4.5) {$\bfH(\bfcurl)\otimes L^{2}$};
        \node (Hdiv-H1)  at (0.75,   - 3)   {$\bfH(\rmdiv)\otimes H^{1}$};

        \node[ellipse, draw, dashed, rotate = 45] at (0, - 3.75) {};

        \draw[->] (HL2)  -- (HL3a) node[above, midway] {$\left(\begin{matrix} * - \rmdiv \\ \bfcurl + \partial_{t} \end{matrix}\right)$};
    \end{tikzpicture}\end{center}
            
    \BA{TO ADD:
    \begin{itemize}
        \item  Proofs of energy/helicity/Gauss's law conservation (Modify weak formulations accordingly)
        \item  Commutative diagram for tensor product complexes
        \item  Notes about what conformity I need for my function spaces/how to handle the weak formulation for non-conforming discretizations
        \item  Continuity of helicity
        \item  Talk about 2.5D case
    \end{itemize}}
