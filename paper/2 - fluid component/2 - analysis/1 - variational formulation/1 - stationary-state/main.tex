\subsubsection*{Stationary-State Case}
    Consider first the stationary-state system, as in \cite{LHF22}.\footnote{The notation therein differs slightly.} The authors consider test functions for each equation in the space as shown in Figure \ref{fig:incompressible stationary-state test spaces} via the $L^{2}(\bfOmega)$ inner product. This takes the weak formulation in Figure \ref{fig:incompressible stationary-state weak form}.
    
    \begin{figure}
        \centering
        \begin{tabular}{ r l c | c }
            \multicolumn{2}{c}{Equation}  &  Index  &  Test space  \\
            \hline\hline
            Mass conservation  &  $0  =  \nabla\cdot\bfu$  &  (\ref{eqn:incompressible mass conservation})  &  $\calP$  \\
            Momentum conservation  &  $\bfzero 
             =  - \bfu\cdot\nabla\bfu - \nabla p + \frac{2}{\beta}\bfj\wedge\bfB + \frac{1}{\rmRef}\Delta\bfu$  &  (\ref{eqn:incompressible momentum conservation})  &  $\calU$  \\
            \hline
            Current identity  &  $\bfzero  =  \frac{1}{\rmRem}\bfj - (\bfE + \bfu\wedge\bfB) + \rmRH\bfj\wedge\bfB$  &  (\ref{eqn:incompressible current identity})  &  $\calJ$  \\
            \hline
            Ampère's law  &  $\bfzero  =  \nabla\wedge\bfB - \bfj$  &  (\ref{eqn:incompressible Ampère's law})  &  $\calE$  \\
            Faraday's law  &  $\bfzero  =  \nabla\wedge\bfE$  &  (\ref{eqn:incompressible Faraday's law})  &  $\calB$  \\
            Gauss's law  &  $0  =  \nabla\cdot\bfB$  &  (\ref{eqn:incompressible Gauss's law})  &  $\nabla\cdot\calB$  \\
        \end{tabular}
        \caption{Test spaces for the incompressible stationary-state variational formulation}
        \label{fig:incompressible stationary-state test spaces}
    \end{figure}

    \begin{figure}
        \centering
        \line
        \begin{align}
            \forall q \in \calP,  0  &=  \langle\nabla\cdot\bfu, q\rangle_{\bfOmega}  \\
            \begin{split}
                \forall \bfq \in \calU,  0  &=  \left.\left(- \left\langle\bfu\cdot\nabla\bfu, \bfv\right\rangle + \langle p, \nabla\cdot\bfv\rangle + \frac{2}{\beta}\langle\bfj\wedge\bfB, \bfv\rangle - \frac{1}{\rmRef}\langle\nabla\bfu, \nabla\bfv\rangle\right)\right|_{\bfOmega}  \\
                &\;\;\;\;\;\;\;\;\;\;\;\;\;\;\;\;\;\;\;\;\;\;\;\;\;\;\;\;\;\;\;\;  + \left.\left(- \langle p, \bfv\cdot\bfn\rangle + \frac{1}{\rmRef}\langle\bfn\cdot\nabla\bfu, \bfv\rangle\right)\right|_{\bfGamma}
            \end{split}  \\
            \forall \bfk \in \calJ,  0  &=  \left.\left(\frac{1}{\rmRem}\langle\bfj, \bfk\rangle - \langle\bfE, \bfk\rangle - \langle\bfu\wedge\bfB, \bfk\rangle + \rmRH\langle\bfj\wedge\bfB, \bfk\rangle\right)\right|_{\bfOmega}  \\
            \forall \bfF \in \calE,  0  &=  \left.\left(\langle\bfB, \nabla\wedge\bfF\rangle - \langle\bfj, \bfF\rangle\tall\right)\right|_{\bfOmega} + \langle\bfB, \bfF\wedge\bfn\rangle_{\bfGamma}  \\
            \forall \bfC \in \calB,  0  &=  \left.\left(\langle\nabla\wedge\bfE, \bfC\rangle + \langle\nabla\cdot\bfB, \nabla\cdot\bfC\rangle\tall\right)\right|_{\bfOmega}
        \end{align}
        \line
        \caption{Incompressible stationary-state weak formulation}
        \label{fig:incompressible stationary-state weak form}
    \end{figure}

    \BA{(What conformity does this require?)} 4 equations in this model are linear: conservation of mass, (\ref{eqn:mass conservation}), and Maxwell's equations, (\ref{eqn:Ampère's law}-\ref{eqn:Gauss's law}). Ensuring these properties hold \emph{strongly} (i.e. pointwise) is known to lead to more physically accurate and representative numerical results. \BA{[Ref, ...]} \BA{(Definitely want more than just this to talk about the motivation, etc. Would be nice to mention e.g. GEMPIC and over evidence that exact enforcement of these conditions is a good idea.)} As is done in \cite{LFM22}, this weak formulation allows one to ensure all but Ampère's law hold strongly, provided certain subspace conditions hold on the function spaces, by considering certain test functions: \BA{(Where \emph{precisely} in \cite{LFM22}?)}
    \begin{center}\begin{tabular}{ c | c | c }
        Equation  &  Test function  &  Subspace condition  \\
        \hline\hline
        $0  =  \nabla\cdot\bfp$  &  $\nabla\cdot\bfp  \in  \calP$  &  $\nabla\cdot\calU  \leqslant  \calP$  \\
        $0  =  \nabla\wedge\bfE$  &  $\nabla\wedge\bfE  \in  \calB$  &  $\nabla\wedge\calE  \leqslant  \calB$  \\
        $0  =  \nabla\cdot\bfB$  &  $\bfB  \in  \calB$  &
    \end{tabular}\end{center}
    (Note the proof of pointwise $0  =  \nabla\cdot\bfB$ here relies on that of pointwise $0  =  \nabla\wedge\bfE$ before it.) Restricting to Hilbert spaces, it is therefore natural to consider weak formulations where $\calU \xrightarrow{\rmdiv} \calP$ and $\calE \xrightarrow{\bfcurl} \calB$ form components of subcomplexes of $H\Lambda^{\bullet}(\bfOmega)$, with projection/interpolation maps $\pi_{*}$ such that the following diagrams commute:
    \begin{center}\begin{tikzpicture}[align = center, node distance = 4cm, auto]
        \node (Hcurl) at (0,   0)   {$\bfH(\bfcurl)$};
        \node (Hdiv1) at (3.5, 0)   {$\bfH(\rmdiv)$};
        \node (E)     at (0,   - 2) {$\calE$};
        \node (B)     at (3.5, - 2) {$\calB$};

        \draw[->] (Hcurl) -- (Hdiv1) node[above, midway] {$\bfcurl$};
        \draw[->] (E)     -- (B)     node[above, midway] {$\bfcurl$};
        \draw[->] (Hcurl) -- (E)     node[left,  midway] {$\pi_{\rmE}$};
        \draw[->] (Hdiv1) -- (B)     node[left,  midway] {$\pi_{\rmB}$};
        
        \node (Hdiv2) at (7,    0)   {$\bfH(\rmdiv)$};
        \node (L2)    at (10.5, 0)   {$L^{2}$};
        \node (U)     at (7,    - 2) {$\calU$};
        \node (P)     at (10.5, - 2) {$\calP$};

        \draw[->] (Hdiv2) -- (L2) node[above, midway] {$\rmdiv$};
        \draw[->] (U)     -- (P)  node[above, midway] {$\rmdiv$};
        \draw[->] (Hdiv2) -- (U)  node[left,  midway] {$\pi_{\rmU}$};
        \draw[->] (L2)    -- (P)  node[left,  midway] {$\pi_{\rmP}$};
    \end{tikzpicture}\end{center}

    In the \emph{compressible} case, similar test spaces for the corresponding equations can be used, with the addition of the \emph{new} energy equation, naturally tested against the \emph{new} function space, $\calD$. (See Figure \ref{fig:compressible stationary-state test spaces}) The corresponding \emph{compressible} weak formulation is given in Figure \ref{fig:compressible stationary-state weak form}.

    \begin{figure}
        \centering
        \begin{tabular}{ r l c | c }
            \multicolumn{2}{c}{Equation}  &  Index  &  Test space  \\
            \hline\hline
            Mass conservation  &  $0  =  \nabla\cdot\bfp$  &  (\ref{eqn:mass conservation})  &  $\calP$  \\
            Momentum conservation  &  $\bfzero 
             =  - \bfp\cdot\nabla\bfu - \nabla p + \frac{2}{\beta}\bfj\wedge\bfB + \cdots$  &  (\ref{eqn:momentum conservation})  &  $\calM$  \\
            Energy conservation  &  $0  =  - \nabla\cdot[p\bfu] - p\nabla\cdot\bfu + \cdots$  &  (\ref{eqn:energy conservation})  &  $\calD$  \\
            \hline
            Current identity  &  $\bfzero  =  \frac{1}{\rmRem}\bfj - \left(\bfE + \bfu\wedge\bfB\right) + \rmRH\bfj\wedge\bfB$  &  (\ref{eqn:current identity})  &  $\calJ$  \\
            \hline
            Momentum identity  &  $\bfzero  =  \bfp - \rho\bfu$  &  (\ref{eqn:velocity identity})  &  $\calU$  \\
            Temperature identity  &  $0  =  p - \rho\theta$  &  (\ref{eqn:temperature identity})  &  $\Theta$  \\
            \hline
            Ampère's law  &  $\bfzero  =  \nabla\wedge\bfB - \bfj$  &  (\ref{eqn:Ampère's law})  &  $\calE$  \\
            Faraday's law  &  $\bfzero  =  \nabla\wedge\bfE$  &  (\ref{eqn:Faraday's law})  &  $\calB$  \\
            Gauss's law  &  $\bfzero  =  \nabla\cdot\bfB$  &  (\ref{eqn:Gauss's law})  &  $\nabla\cdot\calB$  \\
        \end{tabular}
        \caption{Test spaces for the compressible stationary-state variational formulation}
        \label{fig:compressible stationary-state test spaces}
    \end{figure}

    \begin{figure}
        \line
        \begin{align}
            \forall q \in \calP,  0  &=  \langle\nabla\cdot\bfp, q\rangle_{\bfOmega}  \\
            \begin{split}
                \forall \bfq \in \calM,  0  &=  \left.\left(- \langle\bfp\cdot\nabla\bfu, \bfq\rangle + \langle p, \nabla\cdot\bfq\rangle + \frac{2}{\beta}\langle\bfj\wedge\bfB, \bfq\rangle - \frac{1}{\rmRef}\langle\rho\bftau, \nabla_{s}\bfq)\rangle\right)\right|_{\bfOmega}  \\
                &\;\;\;\;\;\;\;\;\;\;\;\;\;\;\;\;\;\;\;\;\;\;\;\;  + \left.\left(- \langle p, \bfq\cdot\bfn\rangle + \frac{1}{\rmRef}\langle\rho\bftau, \sym(\bfq\otimes\bfn)\rangle\right)\right|_{\bfGamma}
            \end{split}  \\
            \begin{split}
                \forall \sigma \in \calD,  0  &=  \left(\langle p\bfu, \nabla\sigma\rangle - \langle p\nabla\cdot\bfu, \sigma\rangle + \frac{1}{\rmRef}\langle\rho\bftau:\nabla_{\rms}\bfu, \sigma\rangle\right.  \\
                &\;\;\;\;\;\;\;\;\;\;\;\;\;\;\;\;\;\;\;\;\;\;\;\;\;\;\;\;\;\;\;\;\;\;\;\;\;\;\;\;\;\;\;\;\;\;\;\;\left.\left.+ \frac{2}{\beta\rmRem}\left\langle\|\bfj\|^{2}, \sigma\right\rangle - \frac{1}{\rmPe}\langle\rho\nabla\theta, \nabla\sigma\rangle\right)\right|_{\bfOmega}  \\
                &\;\;\;\;\;\;\;\;\;\;\;\;\;\;\;\;\;\;\;\;\;\;\;\;  + \left.\left(- \langle p\bfu\cdot\bfn, \sigma\rangle + \frac{1}{\rmPe}\langle\rho\nabla\theta\cdot\bfn, \sigma\rangle\right)\right|_{\bfGamma}
            \end{split}  \\
            \forall \bfk \in \calJ,  0  &=  \left.\left(\frac{1}{\rmRem}\langle\bfj, \bfk\rangle - \langle\bfE, \bfk\rangle - \langle\bfu\wedge\bfB, \bfk\rangle + \rmRH\langle\bfj\wedge\bfB, \bfk\rangle\right)\right|_{\bfOmega}  \\
            \forall \bfv \in \calU,  0  &=  \left.\left(\langle\bfp, \bfv\rangle - \langle\rho\bfu, \bfv\rangle\tall\right)\right|_{\bfOmega}  \\
            \forall \eta \in \Theta,  0  &=  \left.\left(\langle p, \eta\rangle - \langle\rho\theta, \eta\rangle\tall\right)\right|_{\bfOmega}  \\
            \forall \bfF \in \calE,  0  &=  \left.\left(\langle\bfB, \nabla\wedge\bfF\rangle - \langle\bfj, \bfF\rangle\tall\right)\right|_{\bfOmega} + \langle\bfB, \bfF\wedge\bfn\rangle_{\bfGamma}  \\
            \forall \bfC \in \calB,  0  &=  \left.\left(\langle\nabla\wedge\bfE, \bfC\rangle + \langle\nabla\cdot\bfB, \nabla\cdot\bfC\rangle\tall\right)\right|_{\bfOmega}
        \end{align}
        \line
        \caption{Compressible stationary-state weak formulation}
        \label{fig:compressible stationary-state weak form}
    \end{figure}
    
    Assuming the above subspace criteria hold, the results below follow for the weak formulation, preserving some of the structure of the strong formulation:
    
    \begin{theorem}[Energy Balance in the Stationary-State Weak Formulation]
        Provided $\calU  \leqslant  \calM$ and $1  \in  \calD$, the boundary fluxes of energy in exact balance, i.e.
        \begin{equation}
            0  =  \oint_{\bfGamma}\left[- \frac{1}{2}\|\bfu\|^{2}\bfp - 2p\bfu + \frac{2}{\beta}\bfB\wedge\bfE + \frac{1}{\rmRef}\rho\bftau\cdot\bfu + \frac{1}{\rmPe}\rho\nabla\theta\right]\cdot\bfn.
        \end{equation}
    \end{theorem}
    \begin{proof}
        See Appendix \ref{cha:weak conservation proofs}.
    \end{proof}
    
    \begin{theorem}[Helicity Balance in the Weak Formulation]
        Provided $\calB  \leqslant  \calJ$, dissipation and boundary flux of magnetic helicity are in exact balance, i.e.
        \begin{equation}
            0  =  \frac{2}{\rmRem}\int_{\bfOmega}\bfB\cdot\bfj + 2\oint_{\bfGamma}(\bfE\wedge\bfA)\cdot\bfn,
        \end{equation}
        where $\bfA$ is a magnetic potential such that $\bfB  =  \nabla\wedge\bfA$.
    \end{theorem}
    \begin{proof}
        See Appendix \ref{cha:weak conservation proofs}.
    \end{proof}

    \begin{remark}
        $\bfu  \left(=  \frac{1}{\rho}\bfp\right)$, $\theta  \left(=  \frac{p}{\rho}\right)$ (i.e. the spaces $\calU$, $\Theta$) can be eliminated from this formulation, still leaving a weak formulation that \emph{theoretically} exhibits exact energy/helicity balance. For conciseness, I've chosen not to include such formulations here, as the non-multiplicative form will prohibit exact quadrature, which will in turn both:
        \begin{itemize}
            \item  Defeat the purpose of the exact balance from the specifically chosen weak formulation.
            \item  \emph{(I suspect)} with the high quadrature degree required will, in fact, \emph{increase} the problem's computational complexity. Whether this is true remains to be seen. \BA{(Perhaps list testing this in the further work?)}
        \end{itemize}
    \end{remark}
    