\subsection*{Discretization}
    \begin{remark}[Non-conforming discretizations]
        I would like to consider now examples of FE spaces that satisfy the commutativity properties:
        \begin{center}\begin{tikzpicture}[align = center, node distance = 4cm, auto]            
            \node (Hcurl1) at (0,   0) {$\bfH_{0}(\bfcurl)$};
            \node (Hdiv1)  at (3.5, 0) {$\bfH_{0}(\rmdiv)$};
            \draw[->] (Hcurl1) -- (Hdiv1) node[above, midway] {$\bfcurl$};
    
            \node (J) at (0,   - 2) {$\widehat{\calG}$};
            \node (B) at (3.5, - 2) {$\widehat{\calB}$};
            \draw[->] (J) -- (B) node[above, midway] {$\bfcurl$};
    
            \draw[->] (Hcurl1) -- (J) node[left, midway] {$\bbQ_{\widehat{\calG}}$};
            \draw[->] (Hdiv1)  -- (B) node[left, midway] {$\bbQ_{\widehat{\calB}}$};


            \node (Hdiv) at (7,    0) {$\bfH(\rmdiv)$};
            \node (L2)   at (10.5, 0) {$L^{2}$};
            \draw[->] (Hdiv) -- (L2) node[above, midway] {$\rmdiv$};

            \node (U) at (7,    - 2) {$\widehat{\calU}$};
            \node (P) at (10.5, - 2) {$\widehat{\calP}$};
            \draw[->] (U) -- (P) node[above, midway] {$\rmdiv$};

            \draw[->] (Hdiv) -- (U) node[left, midway] {$\bbQ_{\widehat{\calU}}$};
            \draw[->] (L2)   -- (P) node[left, midway] {$\bbQ_{\widehat{\calP}}$};
        \end{tikzpicture}\end{center}
        The vast majority of such FE spaces do not satisfy the levels of conformity the weak/variational forms require. Both $\rmdiv$- and $\bfcurl$-conformity are required in both $\widehat{\calB}$ and $\widehat{\calU}$ at least, whereas FE spaces satisfying the above commutativity properties typically each only offer $\rmdiv$-conformity.
        
        The discretizations in \cite{Laakmann_Hu_Farrell_2022} mitigate this through the use of further auxiliary spaces, for the vorticity $\nabla\wedge\bfu$ and current $\nabla\wedge\bfB$. While effective, this serves only to increase the computational complexity of the model. I would ideally like to consider non-conforming discretizations that simultaneously retain the SP properties promised by a conforming model.

        There are 5 terms in the variational formulation that are ill-defined when $\widehat{\calB}$ and $\widehat{\calU}$ satisfy $\rmdiv$-conformity only:
        \begin{align}
            \langle\nabla_{\rms}\widehat{\bfv}, \nabla_{\rms}\widehat{\bfu}\rangle_{\bfOmega},  &&
            \calC;  &&
            \langle\nabla\wedge\widehat{\bfC}, \nabla\wedge\widehat{\bfB}\rangle_{\bfOmega},  &&
            \calD;  &&
            \calE.
        \end{align}
        Considering the requirements an approximation to each of these would require in order \emph{not} to break the SP properties of the timestepper:
        \begin{itemize}
            \item  Present for non-conforming discretizations of NS, non-conforming models for the {\bf 1st}, $\langle\nabla_{\rms}\widehat{\bfv}, \nabla_{\rms}\widehat{\bfu}\rangle$, and {\bf 2nd}, $\calC$, term are ubiquitous in the literature. Notably for the 1st, $\langle\nabla_{\rms}\widehat{\bfv}, \nabla_{\rms}\widehat{\bfu}\rangle$, these include symmetric interior penalty methods, where the symmetry and non-negativity are retained. For the 2nd, $\calC$, one requires the approximation to retain the skew-symmetric property of the conforming case, or even introduce a dissipation (with $\calC[\bfv; \bfu, \bfu] \leq 0$) that contributes to the energy dissipation as is done in certain upwinding schemes.
            \item  As similar dissipative and skew-symmetric objects, it is my hope that similar techniques and requirements to those traditionally found on the 1st, $\langle\nabla_{\rms}\widehat{\bfv}, \nabla_{\rms}\widehat{\bfu}\rangle$, and 2nd, $\calC$, will be sufficient for the {\bf 3rd}, $\langle\nabla\wedge\widehat{\bfC}, \nabla\wedge\widehat{\bfB}\rangle$, and {\bf 4th}, $\calD$.
            \item  For the {\bf 5th}, $\calE$, I am not sure what the case is. I suspect that I \emph{at least} require the definition to be consistent between the momentum equation and Faraday's law, but I have yet to investigate whether this is sufficient \emph{and} what such approximations will give accurate solutions.
        \end{itemize}
        As an alternative, I have some ideas about constructing ``approximate'' $L^{2}$ projections, defined via ``approximate'' $L^{2}$ norms, on distribution-valued FE spaces that preserve these important commutative structures. If this works out, then I need not consider approximations to non-conforming variational terms on a case-by-case basis, but would have an effective over-arching solution that would allow me to freely use non-conforming discretizations without concern for the loss of the system's SP properties.

        Discussions of these questions are, I suspect, present in the literature. Investigating this is left for future work.
    \end{remark}
    