\section*{Preliminaries, Notation and Terminology}
    Since integrals over the velocity, $\bfv$, have been eliminated from the system, integrals shall henceforth be assumed to be over space, $\bfx$, denoting the integral over $\bfx  \in  \bfS$ for some domain $\bfS  \subset  \bbR^{3}$ as $\int_{\bfS}$.

    Let $\bfOmega  \subset  \bbR^{3}$ denote a sufficiently smooth, compact, contractible domain, with boundary $\bfGamma  :=  \partial\bfOmega$, on which the MHD system is considered. Denote the outward-pointing normal on $\bfGamma$ as $\bfn$.

    \begin{remark}[Non-contractability of tokamak geometries]
        Tokamak geometries, homeomorphic to a torus, and crucially \emph{not} contractible. I assume contractability here to simplify the analysis, i.e. to ensure the de Rham complex on $\bfOmega$ is exact, however with careful handling I would be very surprised if any of the results here should not carry over, albeit with slight modification, to non-contractible domains.
    \end{remark}

    Define the interval $T^{n}  :=  \left(t^{n}, t^{n + 1}\right]$ for a given $t^{n}  <  t^{n + 1}$, with duration $\delta t^{n}  :=  t^{n + 1} - t^{n}$, over which timesteppers for the MHD system are constructed. For a (bounded) interval mesh $T^{\rmh}$ of intervals $T^{n}$, and $s \in \bbZ_{+}$, denote the FE spaces on $T^{\rmh}$:
    \begin{align}
        \bbP^{s}\!\left(T^{n}\right)       &:=  \left\{f \in H^{1}\!\left(T^{\rmh}\right) : \forall n, f|_{T^{n}} \text{ is polynomial of degree} \leq s\right\}  \\
        \bbDP^{s - 1}\!\left(T^{n}\right)  &:=  \left\{f \in L^{2}\!\left(T^{\rmh}\right) : \forall n, f|_{T^{n}} \text{ is polynomial of degree} \leq s - 1\right\}
    \end{align}
    such that $\frac{\rmd}{\rmd t}\bbP^{s}  =  \bbDP^{s - 1}$.

    Denote the $L^{2}$ inner product on a domain $\bfS  \subset  \bbR^{3}$ (e.g. $\bfOmega$, $\bfGamma$) $\langle-, -\rangle_{\bfS}$ or $(\langle-, -\rangle)|_{\bfS}$,
    \begin{align}
        \langle u,    v   \rangle_{\bfS}  :=  \int_{\bfS}uv,  &&
        \text{or}  &&
        \langle \bfu, \bfv\rangle_{\bfS}  :=  \int_{\bfS}\bfu\cdot\bfv,  &&
        \text{or}  &&
        \langle \sfU, \sfV\rangle_{\bfS}  :=  \int_{\bfS}\sfU:\sfV,  &&
        \text{etc.}
    \end{align}
    with associated $L^{2}$ norm on $\bfS$, $\|-\|_{\bfS}$ or $(\|-\|)|_{\bfS}$,
    \begin{equation}
        \|u\|_{\bfS}  :=  \langle u, u\rangle_{\bfS}^{\frac{1}{2}}
    \end{equation}

    \shortline

    Let $\bbQ_{\calF} : \calF^{*} \rightarrow \calF$ denote the $L^{2}$ projection onto the space $\calF$.

    In the analysis of a PDE, we use the terms ``weak'' and ``variational'' form to represent different interpretations of the same idea. Consider by way of example Poisson's equations,
    \begin{equation}\label{eqn:Poisson's equation strong form}
        \Delta u  =  f,
    \end{equation}
    with homogeneous Dirichlet BCs, on the domain $\bfOmega$:
    \begin{itemize}
        \item  {\bf Variational form:} A typical $H^{1}$ variational form of Poisson's equation (\ref{eqn:Poisson's equation strong form}) might seek $u \in \calU$, for some solution space $\calU \leqslant H^{1}_{0}$, such that $\forall v \in \calU$,
        \begin{equation}\label{eqn:Poisson's equation variational form}
            - \langle\nabla u, \nabla v\rangle_{\bfOmega}  =  \langle v, f\rangle_{\bfOmega}.
        \end{equation}
        This is the form that would be employed in Galerkin projection and computation.
        \item  {\bf Weak form:} This variational form (\ref{eqn:Poisson's equation variational form}) can be equivalently written in terms of ($L^{2}$) projection operators, $\bbQ_{\calU}$, as
        \begin{equation}
            \bbQ_{\calU}[\Delta u]  =  \bbQ_{\calU}[f],
        \end{equation}
        where $\Delta u$ is interpreted as a distribution in $H^{- 1} (\cong (H^{1}_{0})^{*})$. The solution of this is exactly equivalent to the variational form (\ref{eqn:Poisson's equation variational form}) and often easier to work with in the analysis. Moreover, this formulation makes it clearer how the weak formulation relates to the strong form (\ref{eqn:Poisson's equation strong form}).
    \end{itemize}

    \begin{remark}[Misuse of $L^{2}$ projection commutativity results]
        Many results in this chapter rely on commutative properties of these $L^{2}$ projection operators for FE de Rham subcomplexes, i.e. it is assumed that diagrams such as the following commute:
        \begin{center}\begin{tikzpicture}[align = center, node distance = 4cm, auto]
            \node (F*) at (0, 0) {$\widetilde{\calF}$};
            \node (G*) at (3, 0) {$\widetilde{\calG}$};
            \draw[->] (F*) -- (G*) node[above, midway] {$\rmd$};
    
            \node (F) at (0, - 2) {$\calF$};
            \node (G) at (3, - 2) {$\calG$};
            \draw[->] (F)  -- (G)  node[above, midway] {$\rmd$};
    
            \draw[->] (F*) -- (F) node[left, midway] {$\bbQ_{\calF}$};
            \draw[->] (G*) -- (G) node[left, midway] {$\bbQ_{\calG}$};
        \end{tikzpicture}\end{center}
        for some spaces $\widetilde{\calF}$, $\widetilde{\calG}$, i.e. that $\bbQ_{\calG}\circ\rmd = \rmd\circ\bbQ_{\calF}$ on $\widetilde{\calF}$.

        This is (in general) not true, and was included a (quite prominent) oversight. It is only true in general if $\calF = \widetilde{\calF}$, $\calG = \widetilde{\calG}$, i.e.e that the following diagram commutes:
        \begin{center}\begin{tikzpicture}[align = center, node distance = 4cm, auto]
            \node (F*) at (0, 0) {$\calF$};
            \node (G*) at (3, 0) {$\calG$};
            \draw[->] (F*) -- (G*) node[above, midway] {$\rmd$};
    
            \node (F) at (0, - 2) {$\calF$};
            \node (G) at (3, - 2) {$\calG$};
            \draw[->] (F)  -- (G)  node[above, midway] {$\rmd$};
    
            \draw[->] (F*) -- (F) node[left, midway] {$\bbQ_{\calF}$};
            \draw[->] (G*) -- (G) node[left, midway] {$\bbQ_{\calG}$};
        \end{tikzpicture}\end{center}
        however since $\bbQ_{\calF}$, $\bbQ_{\calG}$ are the identity maps on $\calF$, $\calG$, this result is trivial. One can state from this that $\bbQ_{\calG}\circ\rmd\circ\bbQ_{\calF} = \rmd\circ\bbQ_{\calF}$ on $\calF^{*}$ (equivalently trivial, since $\rmd\circ\bbQ_{\calF}\left[\calF^{*}\right] \leqslant \calG$) which is in hindsight the commutativity result I \emph{should} have been using, if any.

        This oversight was caused by my having confused the ideas of $L^{2}$ projections and cochain projections: different projections that \emph{do} satisfy these commutativity results, often with very complex definitions \cite{Falk_Winther_2012} making their use in computation impractical. With the weak formulations as stated, certain structure preservation results in this transfer do depend on the $L^{2}$ projections being cochain projections, and are subsequently untrue. I believe I can amend the weak formulations as stated to fix this and the effects should not be lasting, and leave this for future work, however the reader should note that certain results that follow in this chapter will require modification.
    \end{remark}
    
    \begin{remark}[Application of projection operators with insufficient regularity, and non-conforming discretizations]
        These projection operators will be used extensively throughout this chapter in the representation of weak/variational forms for PDE systems. Note however the following:
        \begin{enumerate}
            \item  To preserve certain structures of the continuous model (See Section \ref{cha:structures}) there are certain commutativity relations that the projection operators $\bbQ_{\calF}$ for the spaces in the weak formulation must satisfy. Spaces with projection operators that satisfy these commutativity properties typically have relatively low regularity.
            \item  The intricacies of what regularity requirements are needed for the projections in my weak formulations to be well-posed (i.e. such that each $\bbQ_{\calF}$ acts only on object necessarily in $\calF^{*}$) are mostly left for further work (See Section \ref{cha:analysis}) however I am aware that these requirements are sufficiently strong that those spaces that have the aforementioned commutativity properties are unlikely to satisfy them.
        \end{enumerate}
        This motivates the use of non-conforming discretizations, wherein one still makes use of spaces that are insufficiently regular for the projections to be well-posed, but makes approximations to these projections when needed. (See Section \ref{cha:numerics})
        
        Assume for now therefore that if one sees $\bbQ_{\calF}$ then it is acting on an object in $\calF^{*}$. Regularity requirements and the handling of non-conforming discretizations will be left till later in this section.
    \end{remark}