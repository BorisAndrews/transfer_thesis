\subsection*{Local Structures}
    \emph{``Local''} structures here refer to those that exist pointwise in $\bfOmega$, i.e. those that are functions of $\bfx$, $t$.
    
    Arguably the most crucial local structure in the MHD system is Gauss's law, $\nabla\cdot\bfB  =  0$. Enforced on the ICs at $t  =  0$, this can be seen to be preserved for $t  >  0$ by taking the divergence of Faraday's law, $\partial_{t}\bfB  =  - \nabla\wedge\bfE$, from the complex property $\rmdiv\circ\bfcurl  =  0$: \cite{Stratton_1941, Rosen_1980, Freistühler_Warnecke_2002}
    \begin{align}
                    \partial_{t}\bfB   &=  - \nabla\wedge\bfE  \\
        \nabla\cdot[\partial_{t}\bfB]  &=  - \mst{\nabla\cdot[\nabla\wedge\bfE]}  \\
        \partial_{t}[\nabla\cdot\bfB]  &=  0  \\
                     \nabla\cdot\bfB   &=  0
    \end{align}
    To the authors' knowledge, the preservation of Gauss's law in finite-element discretizations of MHD using FEEC was first studied by Hu, Xu et al. in \cite{Hu_Xu_2015, Hu_Ma_Xu_2017}.

    \begin{remark}[Incompressibility as a local structure]
        In the incompressible system, the mass conservation/incompressibility constraint $\nabla\cdot\bfu  =  0$ is a local structure. This does not however immediately carry over to the \emph{compressible} system, instead making way for the global structure of mass conservation, so we elect not to consider it in too much detail.
    \end{remark}
