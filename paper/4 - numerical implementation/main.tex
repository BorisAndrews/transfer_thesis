\chapter{Numerical Implementation}
    \BA{Introduction.}

    \begin{figure}[!h]
        \centering
        \begin{tabular}{ c || c | c | c | c | c }
            Support for:  &  {\tt Firedrake}  &  {\tt Nektar++}  &  {\tt NGSolve}  &  {\tt deal.ii}  &  Bespoke  \\
            \hline\hline
            Open source  &  \bftick  &    &  \bftick  &    &  \bftick  \\
            \hline
            Portable  &  \bftick  &  \bftick  &  \bftick  &  \bftick  &  \bfcross  \\
            \hline\hline
            FEEC elements  &  \bftick  &  \bfcross  &  \ast\textcolor{darkyellow}{$^{\bf 5}$}  &    &  \ast\textcolor{darkyellow}{$^{\bf 1}$}  \\
            \hline
            Block/AL preconditioning  &  \bftick  &    &    &    &  \ast\textcolor{darkyellow}{$^{\bf 1}$}  \\
            \hline
            Multigrid preconditioning  &  \bftick  &    &    &    &  \ast\textcolor{darkyellow}{$^{\bf 1}$}  \\
            \hline\hline
            $> 3$ dimensions  &  \bfcross  &    &    &  \bftick  &  \ast\textcolor{darkyellow}{$^{\bf 1}$}  \\
            \hline
            Fluid/PIC interaction  &  \ast\textcolor{darkyellow}{$^{\bf 2}$}  &  \ast\textcolor{darkyellow}{$^{\bf 3}$}  &    &    &  \ast\textcolor{darkyellow}{$^{\bf 1}$}  \\
            \hline\hline
            Deflation  &  \ast\textcolor{darkyellow}{$^{\bf 4}$}  &  \bfcross  &    &    &  \ast\textcolor{darkyellow}{$^{\bf 1}$}  \\
        \end{tabular}
        \caption{\BA{Applicability of different numerical implementation frameworks.} \BA{Details about what the \bftick's, \bfcross's and \ast's mean.}}
    \end{figure}
    
    \BA{What each of the \ast's means:
    \begin{itemize}
        \item[\ast\textcolor{darkyellow}{$^{\bf 1}$}]  Obviously all of this is supported in a bespoke implementation, in as far as I would have to implement it, however I would have much more control over how it was done in this case.
        \item[\ast\textcolor{darkyellow}{$^{\bf 2}$}]  There's various possibilities here. Patrick mentioned {\tt DMSwarm} in {\tt PETSc},\footnote{Will Saunders has concerns about the way {\tt DMWarm} passed particles between MPI ranks- the particles are distributed from one rank to \emph{all possible receiving ranks} — computationally intense already! — then has no checks for existence and uniqueness for a receiving rank. Apparently solving this would be a parallelisation nightmare too.} Pablo mentioned a few other ideas and people to get in contact with about this- I'm confident that some of the ideas I'd like to try should already be supported in some form, \emph{without} me having to delve into the nitty-gritty of parallelising the PIC.
        \item[\ast\textcolor{darkyellow}{$^{\bf 3}$}]  Some of the fluid/PIC integration is implemented in \emph{certain} cases within the ExCALIBUR NEPTUNE project in {\tt NESO} (\href{https://github.com/ExCALIBUR-NEPTUNE/NESO}{link}), {\tt NESO-Particles} (\href{https://github.com/ExCALIBUR-NEPTUNE/NESO-Particles}{link}) and {\tt NESO-Spack} (\href{https://github.com/ExCALIBUR-NEPTUNE/NESO-Spack}{link}) within {\tt Nektar++}- works on GPUs and everything!
        \item[\ast\textcolor{darkyellow}{$^{\bf 4}$}]  Firedrake has DefCon! Obviously, that's going to require some tweaks — as detailed in the Bifurcation Analysis chapter — for the applications I'd like to apply it to, but that's going to be the case for any premade deflation package.
        \item[\ast\textcolor{darkyellow}{$^{\bf 5}$}]  I'm not sure if {\tt NGSolve} has support for FEEC elements on \emph{non-simplicial} domains- it's unclear from the \href{https://docu.ngsolve.org/latest/i-tutorials/unit-2.3-hcurlhdiv/hcurlhdiv.html}{documentation}.
    \end{itemize}}

    \section*{Summary}
        \BA{Summary.}