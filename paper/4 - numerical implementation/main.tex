\chapter{Numerical Implementation}
    We discuss now briefly the options available for the numerical implementation of the hybrid FEM-PiC scheme. To enhance the portability of the implementation, it is in our favor to avoid a bespoke implementation, as was done within {\tt Python} in the {\tt STRUPHY} code \cite{Holderied_Possanner_Wang_2021, Holderied_2022, Li_et_al_2023}, instead favoring an implementation within a well-maintained and open source FE package.
    
    As outlined in Chapter \ref{cha:fluid component}, the SP discretizations considered in the fluid component require the use of FE spaces whose projection operators satisfy certain commutativity relations as subcomplexes of the de Rham complex, i.e. FEEC-induced FEs. Being absolutely essential for the success of this scheme, this presents one of the greatest limitations on the options of FE packages, as support for such FE spaces are not yet wholly commonplace. {\tt Firedrake} \cite{Rathgeber_et_al_2016} and {\tt FEniCS}/{\tt FEniCSx} \cite{Logg_Wells_2010, Logg_Mardal_Wells_2012} have potentially the widest support in this area.\footnote{There is also some support for FEEC-induced FEs within {\tt NGSolve} \cite{Schöberl_2015} however the support that is offered is not as rich as within {\tt Firedrake} or {\tt FEniCS}/{\tt FEniCSx}.} Through their interaction with {\tt PETSc} \cite{Balay_et_al_1997, Balay_et_al_2023}, both also offer wide support for different preconditioners and solvers, in particular block and multigrid preconditioners as required by the robust preconditioners implementations in \cite{Laakmann_Hu_Farrell_2022}. From these 2 packages, we will elect to implement the hybrid FEM-PiC scheme within the {\tt Firedrake} package.

    \begin{remark}[Difficulties to come in implementation]
        Choosing the FE package in which we would like the scheme to be implemented represent only 2\% of the battle.
        
        One of the main difficulties to come lies in parallel data management between the fluid and kinetic component, Eulerian and Lagrangian in nature respectively. To be more precise, suppose that in parallelization the domain $\bfOmega$ is partitioned into subdomains $\bfOmega_{1}$, $\bfOmega_{2}$, $\cdots$. Let the data associated with each of these domains (FEs and pseudo-particles within each subdomain) be held within an associated MPI rank. (Building on Figure \ref{fig:PiC decoupling}, Figure \ref{fig:PiC decoupling parallel} illustrates how data is coupled within the system after parallelization.) While FEs are naturally stationary and confined to these subdomains, pseudo-particles are not, and can travel between subdomains. In such an event, the data for the pseudo-particle must be migrated between MPI ranks, which can be very computationally cumbersome if not implemented effectively.
        
        \begin{figure}[!ht]
            \centering
            \begin{tikzpicture}[align = center]%, node distance = 4cm, auto, axis equal]
                \node (1) at (      0,   2.500) {\bf Fluid};
                    \node (1p) at (      0,   5.000) {Particles};
                    \draw[<->] (1) -- (1p);
                \node (2) at (  2.165,   1.250) {\bf Fluid};
                    \node (2p) at (  4.330,   2.500) {Particles};
                    \draw[<->] (2) -- (2p);
                \node (3) at (  2.165, - 1.250) {\bf Fluid};
                    \node (3p) at (  4.330, - 2.500) {Particles};
                    \draw[<->] (3) -- (3p);
                \node (4) at (      0, - 2.500) {\bf Fluid};
                    \node (4p) at (      0, - 5.000) {Particles};
                    \draw[<->] (4) -- (4p);
                \node (5) at (- 2.165, - 1.250) {\bf Fluid};
                    \node (5p) at (- 4.330, - 2.500) {Particles};
                    \draw[<->] (5) -- (5p);
                \node (6) at (- 2.165,   1.250) {\bf Fluid};
                    \node (6p) at (- 4.330,   2.500) {Particles};
                    \draw[<->] (6) -- (6p);
                
                \draw[<->] (1) -- (2);
                \draw[<->] (1) -- (3);
                \draw[<->] (1) -- (4);
                \draw[<->] (1) -- (5);
                \draw[<->] (1) -- (6);
                
                \draw[<->] (2) -- (3);
                \draw[<->] (2) -- (4);
                \draw[<->] (2) -- (5);
                \draw[<->] (2) -- (6);
                
                \draw[<->] (3) -- (4);
                \draw[<->] (3) -- (5);
                \draw[<->] (3) -- (6);
                
                \draw[<->] (4) -- (5);
                \draw[<->] (4) -- (6);
                
                \draw[<->] (5) -- (6);

                \node[ellipse, draw, thick, dotted, minimum width = 3cm, minimum height = 4.75cm, rotate = 360] at (      0,   3.750) {};
                    \node at (      0,   6.500) {$\bfOmega_{1}$};
                \node[ellipse, draw, thick, dotted, minimum width = 3cm, minimum height = 4.75cm, rotate = 300] at (  3.248,   1.875) {};
                    \node at (  5.629,   3.250) {$\bfOmega_{2}$};
                \node[ellipse, draw, thick, dotted, minimum width = 3cm, minimum height = 4.75cm, rotate = 240] at (  3.248, - 1.875) {};
                    \node at (  5.629, - 3.250) {$\bfOmega_{3}$};
                \node[ellipse, draw, thick, dotted, minimum width = 3cm, minimum height = 4.75cm, rotate = 180] at (      0, - 3.750) {};
                    \node at (      0, - 6.500) {$\bfOmega_{4}$};
                \node[ellipse, draw, thick, dotted, minimum width = 3cm, minimum height = 4.75cm, rotate = 120] at (- 3.248, - 1.875) {};
                    \node at (- 5.629, - 3.250) {$\bfOmega_{5}$};
                \node[ellipse, draw, thick, dotted, minimum width = 3cm, minimum height = 4.75cm, rotate = 060] at (- 3.248,   1.875) {};
                    \node at (- 5.629,   3.250) {$\bfOmega_{6}$};
            \end{tikzpicture}
            \caption{Illustration of the coupling of data within the hybrid FEM-PiC model after parallelization between 6 MPI ranks.}
            \label{fig:PiC decoupling parallel}
        \end{figure}

        {\tt PETSc} offers {\tt DMSwarm}, a data management object containing the promising MPI rank migration subroutine {\tt DMSwarmMigrate} to aid in the support of particle methods, although I have not yet had the opportunity to properly investigate this. Even if I get the data management working efficiently, getting the pseudo-particles to share data with the fluid layer could still prove difficult.
    \end{remark}

    \begin{remark}[Test cases for the hybrid model]
        Once the hybrid model is implemented, I would of course like to finish off with some tokamak-relevant test simulations. In particular, I would like to consider tests for the system's SP properties, as well as its ability to model:
        \begin{itemize}
          \item  Effects absent from:
          \begin{itemize}
              \item  Ideal MHD, such as magnetic reconnection. \cite{Dungey_1961, Lockwood_2016}
              \item  Incompressible MHD, such as magnetosonic waves.
          \end{itemize}
            \item  Kinetic effects, including kinetic:
            \begin{itemize}
                \item  Waves, such as whistler modes. \cite{Chen_2015}
                \item  Instabilities, such as two-stream instabilities. \cite{ONeil_Malmberg_1968, Stix_1992}
                \item  Structures, such as beams, double layers and anisotropic pressures.
            \end{itemize}
        \end{itemize}
        I would also like to consider both global tokamak simulations and local flux tube simulations, although the latter would require extending the analysis through this thesis to a 2D setting, which may not always prove possible.
    \end{remark}
    