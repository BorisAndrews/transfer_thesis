\subsection*{Fluid Component (\ref{eqn:PiC-coupled mass conservation}--\ref{eqn:PiC-coupled Maxwell's equations steady-state})}
    The fluid component comprises a nonlinear PDE in position space $\bfx$ only. Classical techniques for the numerical solution of nonlinear PDEs include:
    \begin{itemize}
        \item  Finite element methods (FEM)
        \item  Finite difference methods (FDM)
        \item  Finite volume methods (FVM)
        \item  Spectral methods
    \end{itemize}
    In recent years, great advancements have been made on the analysis and application of FEM for MHD, including the development of provably well-posed and necessarily structure-preserving discretizations and parameter-robust preconditioners. \cite{Hu_Xu_2015, Hu_Ma_Yu_2017, Hu_Lee_Xu_2021, Green_et_al_2022, LFM22, Laakmann_Hu_Farrell_2022} Much of this work has evolved from the development of the finite element exterior calculus (FEEC) by Arnold et al, with its extensive applications in electromagnetic and hydrodynamic models. \cite{Arnold_Falk_Winther_2006, Arnold_Falk_Winther_2009, Arnold_2018} It is in part for this reason that this thesis shall consider a FEM model for the model's MHD component. \BA{(Here's where I need my FEM for MHD literature review! Need to dig up a bunch of my references, chase the paper trail back.) [Ref, Ref, ...]}

    This is much work yet to be done however in adapting these techniques to the $\delta\!f$ model in question. Much of the work to date in this area has been heavily focused on incompressible MHD models, with much of the theory and techniques leaning on the incompressibility condition, $\nabla_{\bfx}\cdot\bfu  =  0$. Those deriving from highly-kinetic tokamak plasmas however are necessarily \emph{compressible}---a model with \emph{very} different dynamical behavior---meaning there still remains a great deal of work to be done in re-adapting these ideas to the case of compressible flow. Notably also, there remains the question of what it means for an MHD model to preserve certain structures when it is coupled with a kinetic correction.

    \BA{Have done the following parts of this section in bullet points. Would be nice to restructure this subsection similarly.}
    