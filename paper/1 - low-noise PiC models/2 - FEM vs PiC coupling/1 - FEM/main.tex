\subsection*{Fluid Component (\ref{eqn:PiC-coupled mass conservation}--\ref{eqn:PiC-coupled Maxwell's equations steady-state}): FEM}
    The fluid component comprises a nonlinear PDE in position space, $\bfx$, and time, $t$, only. Classical techniques for the numerical solution of such nonlinear PDEs include:
    \begin{itemize}
        \item  Finite-element methods (FEM)
        \item  Finite-difference methods (FDM)
        \item  Finite-volume methods (FVM)
        \item  Spectral methods
    \end{itemize}
    Great advancements have been made in recent years on the analysis and application of FEM discretizations for MHD models, including the development of provably well-posed and necessarily structure-preserving (SP) discretizations alongside parameter-robust preconditioners. \cite{Hu_Xu_2015, Hu_Ma_Xu_2017, Hu_Lee_Xu_2021, Green_et_al_2022, Laakmann_Farrell_Mitchell_22, Laakmann_Hu_Farrell_2022} Much of this work has evolved from the development of the finite element exterior calculus (FEEC) by Arnold et al., with its extensive applications in electromagnetic and hydrodynamic models. \cite{Arnold_Falk_Winther_2006, Arnold_Falk_Winther_2009, Arnold_2018}
    
    \cite{Laakmann_Hu_Farrell_2022} in particular introduces such FEM discretizations for a close \emph{incompressible} analogue to the fluid model (\ref{eqn:PiC-coupled mass conservation}--\ref{eqn:PiC-coupled Maxwell's equations steady-state}). It is, in part, for this reason that this thesis shall consider a FEM discretization for the model's MHD component.
    