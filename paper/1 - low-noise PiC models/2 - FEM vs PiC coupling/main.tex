\section{FEM vs. PiC Coupling}
    \BA{Introduction.}

    \BA{Proposition: FEM on fluid background + PiC on kinetic correction.}

    \BA{Why FEM for fluid background?
    \begin{itemize}
        \item  FEM techniques for MHD very well-developed! Fabian's stuff can be used (plus I have stuff to add of course).
    \end{itemize}}

    \BA{Why PiC for correction?
    \begin{itemize}
        \item  Only need to model where the plasma is far from thermalisation.
        \item  Good choice of collision operators leaves a parabolic PDE which can be simulated using Monte Carlo-like pseudo-particle SDEs that pop into and vanish from existence as time progresses, which are (unlike traditional PiC methods that \emph{don't} use the thermalised background) entirely decoupled. (Give example of such operators - e.g. Bhatnagar–Gross–Krook (BGK) \cite{Bhatnagar_Gross_Krook_1954}/Lénard–Bernstein (should ask the Kinetic Theory lecturer for a reference here) - but wait till the kinetic component chapter before fully classifying them.) This has 2 benefits:
        \begin{itemize}
            \item  Workload goes from $\calO[n^{2}]$ to $\calO[n]$:- diagram of star-like network showing coupling of fluids with each particle individually instead of all particles together.
            \item  Can use gyroaveraging with far less concern. Very easy to gyroaverage an SDE in fact.
        \end{itemize}
    \end{itemize}}

    \BA{Why do the both work together?
    \begin{itemize}
        \item  Weak formulations from FEM/FET provide work really nice with testing against the delta-function (weak) approximations to the correction provided by PiC. Give example of resultant equations when testing against a compactly-supported smooth test function.
    \end{itemize}}

    \BA{So, idea is: compressible Hall MHD model with forcing/heating provided by particles + gyroaveraged stochastic particles that appear when kinetic effects are prominent with motion guided by the MHD model. Could do a new diagram, or make this is more clearly annotated on my star-like coupling diagram.}
    