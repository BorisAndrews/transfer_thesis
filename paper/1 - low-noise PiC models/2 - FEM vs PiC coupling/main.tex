\section{Numerical Models}\label{cha:FEM vs. PiC}
    $\delta\!f$ decompositions induce both a fluid (\ref{eqn:PiC-coupled mass conservation}--\ref{eqn:PiC-coupled Maxwell's equations steady-state}) and a kinetic component (\ref{eqn:linearized Boltzmann equation}) that require numerical simulation. When considered individually, each of these two layers offers a choice of possible discretization techniques.
    
    \line

    \section{Preserved Structures}
    \BA{Introduction.}
    
    Consider first those quantities that are conserved by the transient system, so as to seek discretisations which better represent the physical behaviour of the system by \emph{also} conserved these quantities. 
    
    \cite{LHF22} considers conservation of the following 3 quantities, which the authors define in the incompressible case as: \BA{(Oops I've never defined $\bfA$! That should probably be in the introduction...)}
    \begin{center}\begin{tabular}{ c c c }
        Properties  &  Symbol  &  Definition  \\
        \hline\hline
        Energy  &  $\rmE$  &  $\int_{\bfOmega}\left[\frac{1}{\rmEu\rho}\|\bfp\|^{2} + p + \frac{1}{\beta}\|\bfB\|^{2}\right]$  \\
        Magnetic helicity  &  $\rmH_{\rmM}$  &  $\int_{\bfOmega}\bfA\cdot\bfB$  \\
        Hybrid helicity  &  $\rmH_{\rmH}$  &  $\int_{\bfOmega}(a\bfA + \bfp)\cdot(b\bfB + \nabla\wedge\bfp)$
    \end{tabular}\end{center}
    where $a$, $b$ satisfy the relation $a + b  =  \frac{4}{\beta\rmRH}$. \BA{(What do these represent \emph{physically}? Diagrams!)} Taking the derivatives of these quantities over time (still in the incompressible system) gives
    \begin{align}
        \frac{d\rmE}{dt}  &=  \BA{\cdots}  \\
        \frac{d\rmH_{\rmM}}{dt}  &=  \int_{\bfGamma}(- \varphi\bfB + \bfA\wedge\bfE)\cdot\bfn - \frac{2}{\rmRem}\int_{\bfOmega}\bfB\cdot\bfj  \\
        \frac{d\rmH_{\rmH}}{dt}  &=  \BA{\cdots} \\
    \end{align}

    \BA{Proven that in the \emph{compressible} case, $\frac{d\rmE}{dt}$ evaluates as
    {\small \begin{equation}
        \frac{d\rmE}{dt}  =  \int_{\bfGamma}\left[- \frac{1}{2\rmEu\rho}\|\bfp\|^{2}\bfp - \frac{p}{2\rho}\bfp + \frac{1}{\rmEu\rmRe_{f}}\nabla\left[\frac{1}{\rho}\bfp\right]\cdot\frac{1}{\rho}\bfp - \frac{p}{2\rho}\bfp + \frac{1}{2\rmPe}\nabla\left[\frac{p}{\rho} + \frac{1}{\beta}\bfB\wedge\bfE\right]\right]\cdot\bfn
    \end{equation}}}
    
    \section{Preserved Structures}
    \BA{Introduction.}
    
    Consider first those quantities that are conserved by the transient system, so as to seek discretisations which better represent the physical behaviour of the system by \emph{also} conserved these quantities. 
    
    \cite{LHF22} considers conservation of the following 3 quantities, which the authors define in the incompressible case as: \BA{(Oops I've never defined $\bfA$! That should probably be in the introduction...)}
    \begin{center}\begin{tabular}{ c c c }
        Properties  &  Symbol  &  Definition  \\
        \hline\hline
        Energy  &  $\rmE$  &  $\int_{\bfOmega}\left[\frac{1}{\rmEu\rho}\|\bfp\|^{2} + p + \frac{1}{\beta}\|\bfB\|^{2}\right]$  \\
        Magnetic helicity  &  $\rmH_{\rmM}$  &  $\int_{\bfOmega}\bfA\cdot\bfB$  \\
        Hybrid helicity  &  $\rmH_{\rmH}$  &  $\int_{\bfOmega}(a\bfA + \bfp)\cdot(b\bfB + \nabla\wedge\bfp)$
    \end{tabular}\end{center}
    where $a$, $b$ satisfy the relation $a + b  =  \frac{4}{\beta\rmRH}$. \BA{(What do these represent \emph{physically}? Diagrams!)} Taking the derivatives of these quantities over time (still in the incompressible system) gives
    \begin{align}
        \frac{d\rmE}{dt}  &=  \BA{\cdots}  \\
        \frac{d\rmH_{\rmM}}{dt}  &=  \int_{\bfGamma}(- \varphi\bfB + \bfA\wedge\bfE)\cdot\bfn - \frac{2}{\rmRem}\int_{\bfOmega}\bfB\cdot\bfj  \\
        \frac{d\rmH_{\rmH}}{dt}  &=  \BA{\cdots} \\
    \end{align}

    \BA{Proven that in the \emph{compressible} case, $\frac{d\rmE}{dt}$ evaluates as
    {\small \begin{equation}
        \frac{d\rmE}{dt}  =  \int_{\bfGamma}\left[- \frac{1}{2\rmEu\rho}\|\bfp\|^{2}\bfp - \frac{p}{2\rho}\bfp + \frac{1}{\rmEu\rmRe_{f}}\nabla\left[\frac{1}{\rho}\bfp\right]\cdot\frac{1}{\rho}\bfp - \frac{p}{2\rho}\bfp + \frac{1}{2\rmPe}\nabla\left[\frac{p}{\rho} + \frac{1}{\beta}\bfB\wedge\bfE\right]\right]\cdot\bfn
    \end{equation}}}
    

    \line

    In addition to the individual motivations for a FEM model for the fluid component (\ref{eqn:PiC-coupled mass conservation}--\ref{eqn:PiC-coupled Maxwell's equations steady-state}) and a pseudo-particle method for the kinetic component (\ref{eqn:linearized Boltzmann equation}) the two techniques have notable advantages when used together for the full $\delta\!f$ decomposition:
    \begin{itemize}
        \item  {\bf Weak formulations and pseudo-particle distribution functions:} When using a pseudo-particle method for the $\delta\!f_{\pm}$, denote the pseudo-particles via indices $*_{\pm}^{(i)}$, with position, $\bfX_{\pm}^{(i)}(t)$, velocity, $\bfV_{\pm}^{(i)}(t)$, evolving according to a coupled system of stochastic differential equations (SDEs). Broadly speaking, one can interpret this as assuming that in some weak sense,
        \begin{equation}\label{eqn:delta function approximation}
            f_{\pm}(\bfx, \bfv; t)  \approx  \sum_{i}w_{\pm}^{(i)}\delta^{3}\left[\bfx - \bfX_{\pm}^{(i)}(t)\right]\delta^{3}\left[\bfv - \bfV_{\pm}^{(i)}(t)\right],
        \end{equation}
        where $\delta^{3}$ denotes the 3D $\delta$ function, and $w_{\pm}^{(i)}$ denotes the (potentially either positive or negative) weighting in $\delta\!f_{\pm}$ of the particle indexed by $*_{\pm}^{(i)}$. When incorporating this pseudo-particle approximation to the $\delta\!f_{\pm}$ corrections into the fluid model, those equations featuring $\delta\!f_{\pm}$ consequently feature $\delta$ functions in space, $\bfx$. By way of example, the charge conservation equation (\ref{eqn:PiC-coupled charge conservation}) can be written exactly as
        \begin{equation}
            \rmM^{2}\partial_{t}\rho_{\rmC} + \nabla_{\bfx}\cdot\bfj  =  - \frac{\beta}{2}\nabla_{\bfx}\cdot\left[\int_{\bfv}[\delta\!f_{+} - \delta\!f_{-}]\bfv\right].
        \end{equation}
        Making the $\delta$ function approximation (\ref{eqn:delta function approximation}),
        \begin{multline}\label{eqn:PiC-coupled charge conservation with pseudo-particles}
            \rmM^{2}\partial_{t}\rho_{\rmC} + \nabla_{\bfx}\cdot\bfj  \\
            \approx  - \frac{\beta}{2}\nabla_{\bfx}\cdot\left[\sum_{i}w_{+}^{(i)}\delta^{3}\left[\bfx - \bfX_{+}^{(i)}\right]\bfV_{+}^{(i)} - \sum_{i}w_{-}^{(i)}\delta^{3}\left[\bfx - \bfX_{-}^{(i)}\right]\bfV_{-}^{(i)}\right].
        \end{multline}        
        This is apparently problematic, as these $\delta$ functions exist only in a weak sense, as distributions in the dual space $\left(C^{\infty}_{\rmc}\right)^{*}$.
        
        FEM models however derive from weak formulations of the corresponding PDEs, with the result that the weak nature of the definition of the $\delta$ functions is not so much a problem. ``Testing'' (in $L^{2}$) against some test function $\phi(\bfx)$ in some chosen test space defined over the spatial domain for example, (\ref{eqn:PiC-coupled charge conservation with pseudo-particles}) takes the weak form
        \begin{multline}
            \int_{\bfx}\phi\partial_{t}\rho_{\rmC} + \int_{\bfx}\phi\nabla_{\bfx}\cdot\bfj  \\
            \approx  \frac{\beta}{2}\left(\sum_{i}w_{+}^{(i)}\nabla_{\bfx}\phi\left(\bfX_{+}^{(i)}\right)\cdot\bfV_{+}^{(i)} - \sum_{i}w_{-}^{(i)}\nabla_{\bfx}\phi\left(\bfX_{-}^{(i)}\right)\cdot\bfV_{-}^{(i)}\right),
        \end{multline}
        eliminating the $\delta$ functions. The other fluid equations in (\ref{eqn:PiC-coupled mass conservation}--\ref{eqn:PiC-coupled Maxwell's equations steady-state}) featuring $\delta\!f_{\pm}$ follow in a similar suit.

        This can be taken further when using finite elements in time (FET)---discussed further in Section \ref{cha:FET}---when testing (in $L^{2}$) against test functions $\varphi(\bfx; t)$ in test spaces defined over the whole space-\emph{time} domain, as
        \begin{multline}
            \int_{\bfx; t}\phi\partial_{t}\rho_{\rmC} + \int_{\bfx; t}\phi\nabla_{\bfx}\cdot\bfj \\
            \approx  \frac{\beta}{2}\int_{t}\left[\sum_{i}w_{+}^{(i)}\nabla_{\bfx}\phi\left(\bfX_{+}^{(i)}(t)\right)\cdot\bfV_{+}^{(i)}(t) - \sum_{i}w_{-}^{(i)}\nabla_{\bfx}\phi\left(\bfX_{-}^{(i)}(t)\right)\cdot\bfV_{-}^{(i)}(t)\right].
        \end{multline}
        
        \begin{remark}[Testing against stochastic processes]
            Since $\bfX_{\pm}^{(i)}(t)$ and $\bfV_{\pm}^{(i)}(t)$ are complex stochastic processes, further care must be taken regarding the continuity of the test spaces to ensure these RHS particle terms are well-defined. This is something I intend to tackle in the future, when I have investigated the coupled hybrid FEM-PiC model in more detail.
        \end{remark}

        \item  {\bf FET for quantifiable time derivatives:} It was noted in Section \ref{cha:delta f decompositions} that the RHS of the kinetic equations in the $\delta\!f_{\pm}$ corrections (\ref{eqn:linearized Boltzmann equation}) features terms corresponding to time derivatives of the fluid parameters, $\partial_{t}\rho_{\rmM}$, $\partial_{t}\rho_{\rmC}$, $\partial_{t}\bfp$, $\partial_{t}p$, that we would like not to have to substitute for their given values from the fluid equations (\ref{eqn:PiC-coupled mass conservation}--\ref{eqn:PiC-coupled pressure conservation}).
        
        Using a FET approach for the fluid equations means these time derivatives are exactly quantified at all positions in the space-time domain, avoiding this problem. 
    \end{itemize}

    We refer to this FEM--PiC discretization of the fluid (\ref{eqn:PiC-coupled mass conservation}--\ref{eqn:PiC-coupled Maxwell's equations steady-state}) and kinetic component (\ref{eqn:linearized Boltzmann equation}) of the $\delta\!f$-decomposed Boltzmann equations as a coupled hybrid FEM--PiC model.

    The recent {\tt STRUPHY} code \cite{Holderied_Possanner_Wang_2021, Holderied_2022, Li_et_al_2023} implements a similar coupled hybrid FEM--PiC model in a bespoke {\tt Python} implementation, with:
    \begin{itemize}
      \item  A reduced linear, ideal MHD background model, discretized using similar SP FEM timesteppers as those developed by Hu et al. \cite{Hu_Xu_2015, Hu_Ma_Xu_2017, Hu_Lee_Xu_2021, Green_et_al_2022, Laakmann_Farrell_Mitchell_22, Laakmann_Hu_Farrell_2022} as above, on which we would like our timesteppers to be based.
      \item  A collision-free Vlasov model for the correction.
    \end{itemize}
    The efficacy of {\tt STRUPHY} provides motivation for the efficacy of the discretization presented here.
    